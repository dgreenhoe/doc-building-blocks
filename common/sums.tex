%============================================================================
% LaTeX File
% Daniel J. Greenhoe
%============================================================================

%======================================
\chapter{Finite Sums}
%======================================

%=======================================
%\section{Combinitorial relations}
%=======================================
\qboxnps
  {
    \href{http://en.wikipedia.org/wiki/G.H._Hardy}{G.H. Hardy}
    \href{http://www-history.mcs.st-andrews.ac.uk/Timelines/TimelineG.html}{(1877--1947)}
    in his ``Presidential Address" to the
    \href{http://en.wikipedia.org/wiki/London_mathematical_society}{London Mathematical Society}
    on November 8, 1928, about a remark that he suggested was from
    \href{http://en.wikipedia.org/wiki/Harald_Bohr}{Harald Bohr}
    \href{http://www-history.mcs.st-andrews.ac.uk/Timelines/TimelineG.html}{(1887--1951)},
    \href{http://www-history.mcs.st-andrews.ac.uk/Countries/Denmark.html}{Danish}
    mathematician pictured to the left.
    \index{Bohr, Harald} \index{Hardy, G.H.}
    \index{quotes!Bohr, Harald}
    \index{quotes!Hardy, G.H.}
    \footnotemark
  }
  {../common/people/bohrh.jpg}
  {I think that it was Harald Bohr who remarked to me that
   ``all analysts spend half their time hunting through the literature
   for inequalities which they want to use and cannot prove." }
  \citetblt{
    quote: & \citerp{hardy1928}{64} \\
    image: & \url{http://www-history.mcs.st-andrews.ac.uk/PictDisplay/Bohr_Harald.html}
    }

%=======================================
\section{Summation}
%=======================================
%--------------------------------------
\begin{definition}
\footnote{\begin{tabular}{ll}
  reference:          & \citerpgc{berberian1961}{8}{0821819127}{Definition~I.3.1}\\
  ``$\sum$" notation: & \citorp{fourier1820}{280}  %{http://gallica.bnf.fr/ark:/12148/bpt6k33707/f285.image}
\end{tabular}}
\label{def:sum}
%--------------------------------------
Let $+$ be an addition operator on a tuple $\tuple{x_n}{m}{\xN}$.
\defbox{\begin{array}{M}
  The \hid{summation} of $\tuplen{x_n}$ from index $m$ to index $N$ with respect to $+$ is
  \\\qquad$\ds
  \sum_{n=m}^\xN x_n \eqd
    \brbl{\begin{array}{>{\ds}l@{\qquad}M}
      0                            & for $\xN<m$\\
     %x_\xN                          & for $N=m$ \\
    \brp{\sum_{n=m}^{\xN-1} x_n}+x_\xN & for $\xN\ge m$
    \end{array}}
  $
\end{array}}
\end{definition}

%--------------------------------------
\begin{theorem}[\thmd{Generalized associative property}]
\footnote{
  \citerppgc{berberian1961}{9}{10}{0821819127}{Theorem I.3.1}
  }
\label{thm:sum_assoc}
%--------------------------------------
Let $+$ be an addition operator on a tuple $\tuple{x_n}{m}{\xN}$.
\thmbox{\begin{array}{M}
  $+$ is \prope{associative}
  $\quad\implies$
  \\$\ds
  \mcom{\ds
    \sum_{n=m}^{\xL} x_n + \brp{\sum_{n=\xL+1}^{\xM} x_n + \sum_{n=\xM+1}^{\xN} x_n}
    =\brp{\sum_{n=m}^{\xL} x_n + \sum_{n=\xL+1}^{\xM} x_n} + \sum_{n=\xM+1}^{\xN} x_n
    \text{\qquad\scs for $m<\xL<\xM\le\xN$}
   }{$\ds\sum_{n=m}^\xN$ is \prope{associative}}
  $
\end{array}}
\end{theorem}
\begin{proof}
\begin{enumerate}
  \item Proof for $\xN<m$ case: $\ds \sum_{n=m}^\xN x_n = 0$.
  \item Proof for $N=m$ case: $\ds \sum_{n=m}^m x_n = \brp{\sum_{n=m}^{m-1} x_n}+x_m = 0+x_m = x_m$.
  \item Proof for $N=m+1$ case: \label{item:sum_assoc_m1}
        $\ds \sum_{n=m}^{m+1} x_n = \brp{\sum_{n=m}^{m} x_n}+x_{m+1} = x_{m} + x_{m+1}$

  \item Proof for $N=m+2$ case:
    \begin{align*}
      \sum_{n=m}^{m+2} x_n
        &= \brp{\sum_{n=m}^{m+1} x_n }  +  x_{m+2}
        && \text{by \prefp{def:sum}}
      \\&= \brp{x_{m} + x_{m+1}}        +  x_{m+2}
        && \text{by \pref{item:sum_assoc_m1}}
      \\&= x_{m} + \brp{x_{m+1} + x_{m+2}}
        && \text{by left hypothesis}
    \end{align*}

  \item Proof that $N$ case $\implies$ $N+1$ case:
    \begin{align*}
      \sum_{n=m}^{\xN+1} x_n
        &= \mcom{\brp{\sum_{n=m}^{\xN} x_n}}{\prope{associative}} + x_{\xN+1}
        && \text{by \prefp{def:sum}}
      \\&= \brp{\sum_{n=m}^{\xL} x_n + \brp{\sum_{n=\xL+1}^{\xM} x_n + \sum_{n=\xM+1}^{\xN} x_n}} + x_{\xN+1}
       &&= \brp{\brp{\sum_{n=m}^{\xL} x_n + \sum_{n=\xL+1}^{\xM} x_n} + \sum_{n=\xM+1}^{\xN} x_n} + x_{\xN+1}
      \\&= \brp{\sum_{n=m}^{\xL} x_n + \sum_{n=\xL+1}^{\xM} x_n} + \brp{\sum_{n=\xM+1}^{\xN} x_n + x_{\xN+1}}
       &&= \brp{\sum_{n=m}^{\xL} x_n + \sum_{n=\xL+1}^{\xM} x_n} + \brp{\sum_{n=\xM+1}^{\xN+1} x_n}
    \end{align*}
\end{enumerate}
\end{proof}

%=======================================
\section{Means}
%=======================================
%=======================================
\subsection{Weighted {\fntFreeSerif\textphi}-means}
%=======================================

%---------------------------------------
\begin{definition}
\footnote{
  \citerpg{bollobas1999}{5}{0521655773}
  }
\label{def:seq_Mphi}
%---------------------------------------
\defbox{\begin{array}{M}
  The $\tuple{\lambda_n}{1}{\xN}$ weighted \fnctd{$\fphi$-mean} of a tuple $\tuple{x_n}{1}{\xN}$ is defined as
  \\\indentx$\ds \fM_\fphi\brp{\tuplen{x_n}} \eqd \fphi^{-1}\brp{\sum_{n=1}^\xN \lambda_n \fphi\brp{x_n}}$
  \\where $\fphi$ is a \prope{continuous} and \prope{strictly monotonic} function in $\clF{\Rnn}{\R}$
  \\and   $\tuple{\lambda_n}{n=1}{\xN}$ is a sequence of weights for which $\ds\sum_{n=1}^\xN\lambda_n=1$.
  \end{array}}
\end{definition}

%---------------------------------------
\begin{lemma}
\footnote{
  \citerpg{pecaric1992}{107}{0125492502},
  \citerpg{bollobas1999}{5}{0521655773},
  \citorpg{hardy1934}{75}{0521358809}
  }
\label{lem:seq_Mphi}
%---------------------------------------
Let $\fM_\fphi\brp{\tuplen{x_n}}$ be the $\tuple{\lambda_n}{1}{\xN}$ weighted $\fphi$-mean of a tuple $\tuple{x_n}{1}{\xN}$.
\ifdochas{convex}{Let the property \prope{convex} be defined as in \prefpp{def:convexf}.}
\lembox{
  \begin{array}{MMMc>{\ds}l}
    $\fphi\fpsi^{-1}$  is \prope{convex}  &and& $\fphi$ is \prope{increasing} &\implies& \fM_\fphi\brp{\tuplen{x_n}} \ge \fM_\fpsi\brp{\tuplen{x_n}}\\
    $\fphi\fpsi^{-1}$  is \prope{convex}  &and& $\fphi$ is \prope{decreasing} &\implies& \fM_\fphi\brp{\tuplen{x_n}} \le \fM_\fpsi\brp{\tuplen{x_n}}\\
    $\fphi\fpsi^{-1}$  is \prope{concave} &and& $\fphi$ is \prope{increasing} &\implies& \fM_\fphi\brp{\tuplen{x_n}} \le \fM_\fpsi\brp{\tuplen{x_n}}\\
    $\fphi\fpsi^{-1}$  is \prope{concave} &and& $\fphi$ is \prope{decreasing} &\implies& \fM_\fphi\brp{\tuplen{x_n}} \ge \fM_\fpsi\brp{\tuplen{x_n}}\\
  \end{array}
  }
\end{lemma}
\begin{proof}
\begin{align*}
  \intertext{1. Case where $\fphi\fpsi^{-1}$  is \prope{convex} and $\fphi$ is \prope{increasing}:}
  \fM_\fphi\brp{\tuplen{x_n}}
    &\eqd \fphi^{-1}\brp{\sum_{n=1}^{\xN} \lambda_n \fphi\brp{x_n}}
    &&    \text{by definition of $\fM_\fphi$}
    &&    \text{\xref{def:seq_Mphi}}
  \\&=    \fphi^{-1}\brp{\sum_{n=1}^{\xN} \lambda_n \fphi\fpsi^{-1}\fpsi\brp{x_n}}
    &&    \text{by definition of $\fpsi^{-1}$}
  \\&\ge  \fphi^{-1}\brp{\fphi\fpsi^{-1}\sum_{n=1}^{\xN} \lambda_n \fpsi\brp{x_n}}
    &&    \text{by \ineqe{Jensen's Inequality}}
    &&    \text{\ifsxref{convex}{thm:jensen_ineq}}
  \\&=    \brp{\fpsi^{-1}\sum_{n=1}^{\xN} \lambda_n \fpsi\brp{x_n}}
    &&    \text{by definition of $\fpsi^{-1}$}
  \\&\eqd \fM_\fpsi\brp{\tuplen{x_n}}
    &&    \text{by definition of $\fM_\fpsi$}
    &&    \text{\xref{def:seq_Mphi}}
  %
  \intertext{2. Case where $\fphi\fpsi^{-1}$  is \prope{convex} and $\fphi$ is \prope{decreasing}:}
  \fM_\fphi\brp{\tuplen{x_n}}
    &\eqd \fphi^{-1}\brp{\sum_{n=1}^{\xN} \lambda_n \fphi\brp{x_n}}
    &&    \text{by definition of $\fM_\fphi$}
    &&    \text{\xref{def:seq_Mphi}}
  \\&=    \fphi^{-1}\brp{\sum_{n=1}^{\xN} \lambda_n \fphi\fpsi^{-1}\fpsi\brp{x_n}}
    &&    \text{by definition of $\fpsi^{-1}$}
  \\&\le  \fphi^{-1}\brp{\fphi\fpsi^{-1}\sum_{n=1}^{\xN} \lambda_n \fpsi\brp{x_n}}
    &&    \begin{array}{M}
            by  \ineqe{Jensen's Inequality}               \\
            and because $\fphi^{-1}$ is \prope{decreasing}
          \end{array}
    &&    \begin{array}{M}
            \ifxref{convex}{thm:jensen_ineq}\\
            (by hypothesis)
          \end{array}
  \\&=    \brp{\fpsi^{-1}\sum_{n=1}^{\xN} \lambda_n \fpsi\brp{x_n}}
    &&    \text{by definition of $\fpsi^{-1}$}
  \\&\eqd \fM_\fpsi\brp{\tuplen{x_n}}
    &&    \text{by definition of $\fM_\fpsi$}
    &&    \text{\xref{def:seq_Mphi}}
\end{align*}
\end{proof}

One of the most well known inequalities in mathematics is \ineqe{Minkowski's Inequality}
(1910, \prefp{thm:lp_minkowski}). % which demonstrates
%\\\indentx$\ds
%  \brp{\sum_{n=1}^{\xN} \abs{x_n+y_n}^p}^\frac{1}{p}
%  \le
%  \brp{\sum_{n=1}^{\xN} \abs{x_n}^p}^\frac{1}{p}
%  +
%  \brp{\sum_{n=1}^{\xN} \abs{y_n}^p}^\frac{1}{p}
%  \qquad
%  \forall \; 1<p<\infty
%$\\
In 1946, H.P. Mulholland submitted a result\footnote{\citeP{mulholland1950}}
that generalizes Minkowski's Inequality to an equal weighted \textphi-mean.
%\\\indentx$\ds
%  \fphi^{-1}\brp{\sum_{n=1}^{\xN} \fphi\brp{x_n+y_n}}
%  \le
%  \fphi^{-1}\brp{\sum_{n=1}^{\xN} \fphi\brp{x_n}}
%  +
%  \fphi^{-1}\brp{\sum_{n=1}^{\xN} \fphi\brp{y_n}}
%$\\
And Milovanovi/'c and Milovanov/'c (1979) generalized this even further to a \hie{weighted} \textphi-mean
(\pref{thm:lp_wphi}, next).

%---------------------------------------
\begin{theorem}
\label{thm:lp_wphi}
\footnote{
  \citor{milovanovic1979},
  \citerpgc{bullen2003}{306}{1402015224}{Theorem 9}
  }
%---------------------------------------
\thmboxt{
  $\ds\brb{\begin{array}{FMD | FMD}
    (1). & $\fphi$ is \prope{convex}             & and & (3). & $\fphi(0)=0$ & and \\                      
    (2). & $\fphi$ is \prope{strictly monotonic} & and & (4). & $\log\circ\fphi\circ\exp$ is \prope{convex}
  \end{array}}
  \quad\implies\quad$
  \\\qquad\qquad
  $\ds\brb{
  \fphi^{-1}\brp{\sum_{n=1}^{\xN} \lambda_n \fphi\brp{x_n+y_n}}
  \le
  \fphi^{-1}\brp{\sum_{n=1}^{\xN} \lambda_n \fphi\brp{x_n}}
  +
  \fphi^{-1}\brp{\sum_{n=1}^{\xN} \lambda_n \fphi\brp{y_n}}
  }$
}
\end{theorem}

%=======================================
\subsection{Power means}
%=======================================
%---------------------------------------
\begin{definition}
\footnote{
  \citerpg{bullen2003}{175}{1402015224},
  \citerpg{bollobas1999}{6}{0521655773}
  %\citerpg{pecaric1992}{108}{0125492502}
  }
\label{def:seq_Mr}
%---------------------------------------
Let $\fM_{\fphi(x;r)}\brp{\tuplen{x_n}}$ be the $\tuple{\lambda_n}{1}{\xN}$ weighted $\fphi$-mean of a \prope{non-negative} tuple $\tuple{x_n}{1}{\xN}$
(\prefp{def:seq_Mphi}).
%Let $\Rx$ be the set of extended real numbers $\brp{\Rx\eqd\R\setu\setn{-\infty,\,\infty}}$.%
%\footnote{
%  \citerppgc{rana2002}{385}{388}{0821829742}{Appendix A}
%  }
\defboxt{
  A mean $\fM_{\fphi(x;r)}\brp{\tuplen{x_n}}$ is a \hid{power mean} with parameter $r$ if $\fphi(x)\eqd x^r$. That is,
  \\\indentx$\ds \fM_{\fphi(x;r)}\brp{\tuplen{x_n}} = \brp{\sum_{n=1}^{\xN} \lambda_n \brp{x_n}^r}^{\frac{1}{r}}$
  }
\end{definition}

%---------------------------------------
\begin{theorem}
\footnote{
  \citerppgc{bullen2003}{175}{177}{1402015224}{see also page 203},
  \citerppg{bollobas1999}{6}{8}{0521655773},
  %\citerpg{pecaric1992}{108}{0125492502}
  \citor{besso1879},
  \citorp{bienayme1840}{68}
  }
\label{thm:seq_Mr}
%---------------------------------------
Let $\fM_{\fphi(x;r)}\brp{\tuplen{x_n}}$ be \fncte{power mean} with parameter $r$ of an $\xN$-tuple $\tuplexn{x_n}$.
Let $\Rx$ be the set of extended real numbers $\brp{\Rx\eqd\R\setu\setn{-\infty,\,\infty}}$.%
\footnote{
  \citerppgc{rana2002}{385}{388}{0821829742}{Appendix A}
  }
\thmbox{\begin{array}{>{\ds}rc>{\ds}l}
  \ds\fM_{\fphi(x;r)}\brp{\tuplen{x_n}} &\eqd& \brp{\sum_{n=1}^{\xN} \lambda_n \brp{x_n}^r}^{\frac{1}{r}}
  \text{ is \prope{continuous} and \prope{strictly increasing} in $\Rx$.}
  \\
    \ds \fM_{\fphi(x;r)}\brp{\tuplen{x_n}} &=&
    \brbl{\begin{array}{>{\ds}lM}
      \min_{n=1,2,\ldots,\xN}  \tuplen{x_n}          & for $r=-\infty$\\
      \prod_{n=1}^{\xN} x_n^{\lambda_n}  & for $r=0$\\
      \max_{n=1,2,\ldots,\xN}\tuplen{x_n}                     & for $r=+\infty$
    \end{array}}
  \end{array}}
\end{theorem}
\begin{proof}
\begin{enumerate}
  \item Proof that $M_{\fphi(x;r)}$ is \prope{strictly increasing} in $r$:
    \begin{enumerate}
      \item Let $r$ and $s$ be such that $-\infty<r<s<\infty$.
      \item Let $\fphi_r\eqd x^r$ and $\fphi_s\eqd x^s$. Then $\ds\fphi_r\fphi_s^{-1}=x^\frac{r}{s}$.
      \item The composite function $\ds\fphi_r\fphi_s^{-1}$ is \prope{convex} or \prope{concave} depending on the values of $r$ and $s$:
        \\\begin{tabular}{c||c|c}
                & $r<0$ ($\fphi_r$ decreasing) & $r>0$ ($\fphi_r$ increasing)
        \\\hline\hline
          $s<0$ & \prope{convex}               & (not possible)
        \\\hline
          $s>0$ & \prope{convex}               & \prope{concave}
        \\\hline
        \end{tabular}
      \item Therefore by \prefpp{lem:seq_Mphi},
        \\\indentx$\ds-\infty<r<s<\infty\qquad\implies\qquad\fM_{\fphi(x;r)}\brp{\tuplen{x_n}}<\fM_{\fphi(x;s)}\brp{\tuplen{x_n}}$.
    \end{enumerate}
  \item Proof that $\fM_{\fphi(x;r)}$ is continuous in $r$ for $r\in\R\setd0$:
        The sum of continuous functions is continuous.
        \\For the cases of $r\in\setn{-\infty,\,0,\,\infty}$, see the items that follow.

  \item Lemma: $\fM_{\fphi(x;-r)}\brp{\tuplen{x_n}}=\brb{\fM_{\fphi(x;r)}\brp{\tuplen{x_n^{-1}}}}^{-1}$.\label{item:Mr_lemma} Proof:
    \begin{align*}
      \brb{\fM_{\fphi(x;r)}\brp{\tuplen{x_n^{-1}}}}^{-1}
        &= \brb{\brp{\sum_{n=1}^{\xN} \lambda_n \brp{x_n^{-1}}^{r}}^{\frac{1}{r}}}^{-1}
        && \text{by definition of $\fM_\fphi$}
      \\&= \brp{\sum_{n=1}^{\xN} \lambda_n \brp{x_n}^{-r}}^{\frac{1}{-r}}
      \\&= \fM_{\fphi(x;-r)} \brp{\tuplen{x_n}}
        && \text{by definition of $\fM_\fphi$}
    \end{align*}

  \item Proof that $\ds\lim_{r\to\infty} \fM_\fphi\brp{\tuplen{x_n}} = \max_{n\in\Z} \tuplen{x_n}$: \label{item:Mr_max}
    \begin{enumerate}
      \item Let $\ds x_m\eqd \max_{n\in\Z}\tuplen{x_n}$ \label{item:Mr_xm}

      \item Note that $\ds\lim_{r\to\infty}\fM_\fphi \le \max_{n\in\Z} \tuplen{x_n}$ because
            \begin{align*}
              \lim_{r\to\infty} \fM_\fphi\brp{\tuplen{x_n}}
                &=   \lim_{r\to\infty} \brp{\sum_{n=1}^{\xN}\lambda_n x_n^r}^\frac{1}{r}
                &&   \text{by definition of $\fM_\fphi$}
              \\&\le \lim_{r\to\infty} \brp{\sum_{n=1}^{\xN}\lambda_n x_m^r}^\frac{1}{r}
                &&   \text{\parbox[t]{80mm}{by definition of $x_m$ in \pref{item:Mr_xm} and because
                          $\fphi(x)\eqd x^r$ and $\fphi^{-1}$ are both increasing or both decreasing}}
              \\&=   \lim_{r\to\infty} \brp{x_m^r\mcom{\sum_{n=1}^{\xN}\lambda_n}{$1$} }^\frac{1}{r}
                &&   \text{because $x_m$ is a constant}
              \\&=   \lim_{r\to\infty} \brp{x_m^r\cdot1 }^\frac{1}{r}
              \\&= x_m
              \\&= \max_{n\in\Z}\tuplen{x_n}
                &&   \text{by definition of $x_m$ in \pref{item:Mr_xm}}
            \end{align*}

      \item But also note that $\ds\lim_{r\to\infty}\fM_\fphi \ge \max_{n\in\Z} \tuplen{x_n}$ because
            \begin{align*}
              \lim_{r\to\infty} \fM_\fphi\brp{\tuplen{x_n}}
                &=   \lim_{r\to\infty} \brp{\sum_{n=1}^{\xN}\lambda_n x_n^r}^\frac{1}{r}
                &&   \text{by definition of $\fM_\fphi$}
              \\&\ge \lim_{r\to\infty} \brp{w_m x_m^r}^\frac{1}{r}
                &&   \text{\parbox[t]{80mm}{by definition of $x_m$ in \pref{item:Mr_xm} and because
                          $\fphi(x)\eqd x^r$ and $\fphi^{-1}$ are both increasing or both decreasing}}
              \\&=   \lim_{r\to\infty} w_m^\frac{1}{r} x_m^\frac{r}{r}
              \\&=   x_m
              \\&=   \max_{n\in\Z}\tuplen{x_n}
                &&   \text{by definition of $x_m$ in \pref{item:Mr_xm}}
            \end{align*}

      \item Combining items (b) and (c) we have
            $\ds\lim_{r\to\infty}\fM_\fphi = \max_{n\in\Z}\tuplen{x_n}$.
    \end{enumerate}

  \item Proof that $\ds\lim_{r\to-\infty} \fM_\fphi\brp{\tuplen{x_n}} = \min_{n\in\Z} \tuplen{x_n}$:
    \begin{align*}
      \lim_{r\to-\infty}\fM_{\fphi(x;r)}\brp{\tuplen{x_n}}
      &= \lim_{r\to\infty} \fM_{\fphi(x;-r)}\brp{\tuplen{x_n}}
      && \text{by change of variable $r$}
    \\&= \lim_{r\to\infty} \brb{\fM_{\fphi(x;r)}\brp{\tuplen{x_n^{-1}}}}^{-1}
      && \text{by Lemma in \prefp{item:Mr_lemma}}
    \\&= \lim_{r\to\infty} \frac{1}{\fM_{\fphi(x;r)}\brp{\tuplen{x_n^{-1}}}}
    \\&= \frac{\lim_{r\to\infty}1}{\lim_{r\to\infty}\fM_{\fphi(x;r)}\brp{\tuplen{x_n^{-1}}}}
      && \text{by property of $\lim$ \footnotemark}
    \\&= \frac{1}{\ds\max_{n\in\Z}\tuplen{x_n^{-1}}}
      && \text{by \pref{item:Mr_max}}
    \\&= \frac{1}{\ds\brp{\min_{n\in\Z}\tuplen{x_n}}^{-1}}
    \\&= \min_{n\in\Z}\tuplen{x_n}
    \end{align*}
    \citetblt{
      \citerpgc{rudinp}{85}{007054235X}{4.4 Theorem}
      }

  \item Proof that $\ds\lim_{r\to0} \fM_\fphi\brp{\tuplen{x_n}} = \prod_{n=1}^{\xN} x_n^{\lambda_n}$:
    \begin{align*}
      \lim_{r\to0} \fM_\fphi\brp{\tuplen{x_n}}
        &= \lim_{r\to0} \exp\brb{\ln\brb{\fM_\fphi\brp{\tuplen{x_n}}}}
      \\&= \lim_{r\to0} \exp\brb{\ln\brb{\brp{\sum_{n=1}^{\xN}\lambda_n\brp{x_n^r}}^\frac{1}{r}}}
        && \text{by definition of $\fM_\fphi$}
      \\&= \exp\brb{\frac{\ds\pderiv{}{r}\ln\brp{\sum_{n=1}^{\xN}\lambda_n\brp{x_n^r}}}{\ds\pderiv{}{r}r}}_{r=0}
        && \text{by \thm{l'H{/<o}pital's rule}\footnotemark}
      \\&= \exp\brb{\frac{\ds\sum_{n=1}^{\xN}\lambda_n\pderiv{}{r}\brp{x_n^r}}
                         {\ds\sum_{n=1}^{\xN}\lambda_n\brp{x_n^r}}
                   }_{r=0}
       &&= \exp\brb{\frac{\ds\sum_{n=1}^{\xN}\lambda_n\pderiv{}{r}\exp\brp{\ln\brp{x_n^r}}}
                         {\ds\sum_{n=1}^{\xN}\lambda_n}
                   }_{r=0}
      \\&= \exp\brb{\frac{\ds\sum_{n=1}^{\xN}\lambda_n\pderiv{}{r}\exp\brp{r\ln\brp{x_n}}}
                         {1}
                   }_{r=0}
       &&= \exp\brb{\sum_{n=1}^{\xN}\lambda_n\pderiv{}{r}\exp\brp{r\ln\brp{x_n}}
                   }_{r=0}
      \\&= \exp\brb{\sum_{n=1}^{\xN}\lambda_n\exp\brb{r\ln{x_n}}\ln\brp{x_n}
                   }_{r=0}
       &&= \exp\brb{\sum_{n=1}^{\xN}\lambda_n\ln\brp{x_n}}
      \\&= \exp\brb{\sum_{n=1}^{\xN}\ln\brp{x_n^{\lambda_n}}}
       &&= \exp\brb{\ln\prod_{n=1}^{\xN}x_n^{\lambda_n}}
         = \prod_{n=1}^{\xN} x_n^{\lambda_n}
    \end{align*}
    \footnotetext{
      \citerpgc{rudinp}{109}{007054235X}{5.13 Theorem}
      }
\end{enumerate}
\end{proof}

%--------------------------------------
\begin{definition}
%\footnote{
%  \citerpg{bullen2003}{71}{1402015224},
%  \citerpg{bollobas1999}{5}{0521655773},
%  \citerppc{cauchy1821}{457}{459}{Note II, theorem 17},
%  \citorp{jensen1906}{183}
%  }
\label{def:am}
\label{def:gm}
\label{def:hm}
\label{def:average}
%--------------------------------------
Let $\tuplen{x_n}_1^\xN$ be a tuple. Let $\tuplen{\lambda_n}_1^\xN$ be a tuple of weighting values. 
%such that $\ds\sum_{n=1}^\xN\lambda_n=1$.
\defbox{\begin{array}{Mrc>{\ds}lM>{\ds}l}
  The \fnctd{harmonic mean} of $\tuplen{x_n}$ is defined as 
    & \mu_h &\eqd& \brp{\sum_{n=1}^\xN \lambda_n \frac{1}{x_n}}^{-1}
    & where & \sum_{n=1}^\xN\lambda_n=1
    \\
  The \fnctd{geometric mean} of $\tuplen{x_n}$ is defined as 
    & \mu_g &\eqd& \sum_{n=1}^\xN x_n^{\lambda_n}
    & where & \sum_{n=1}^\xN\lambda_n=1
    \\
  The \fnctd{arithmetic mean} of $\tuplen{x_n}$ is defined as 
    & \mu_a &\eqd& \sum_{n=1}^\xN \lambda_n x_n
    & where & \sum_{n=1}^\xN\lambda_n=1
    \\
  The \fnctd{average} of $\tuplen{x_n}$ is defined as 
    & \mu_a &\eqd& \frac{1}{\xN}\sum_{n=1}^\xN x_n
    & %where & \sum_{n=1}^\xN\lambda_n=1
\end{array}}
\end{definition}

%=======================================
\section{Inequalities on power means}
%=======================================
%--------------------------------------
\begin{corollary}
\footnote{
  \citerpg{bullen2003}{71}{1402015224},
  \citerpg{bollobas1999}{5}{0521655773},
  \citerppc{cauchy1821}{457}{459}{Note II, theorem 17},
  \citorp{jensen1906}{183}
  }
\index{AM-GM inequality}
\index{arithmetic mean geometric mean inequality}
\index{generalized arithmetic mean geometric mean inequality}
\index{inequalities!AM-GM}
\label{cor:hm_gm_am}
%--------------------------------------
Let $\tuplen{x_n}_1^\xN$ be a tuple. Let $\tuplen{\lambda_n}_1^\xN$ be a tuple of weighting values. 
\corbox{
  \min\tuplen{x_n}
  \le
  \mcoml{\ds\brp{\sum_{n=1}^\xN \lambda_n \frac{1}{x_n}}^{-1}}{harmonic mean}
  \le
  \mcomr{\ds\prod_{n=1}^\xN x_n^{\lambda_n}}{geometric mean}
  \le
  \mcoml{\ds\sum_{n=1}^\xN \lambda_n x_n}{arithmetic mean}
  \le
  \max\tuplen{x_n}
  \qquad\text{where\quad$\ds\sum_{n=1}^\xN\lambda_n=1$}
  }
\end{corollary}
\begin{proof}
\begin{enumerate}
  \item These five means are all special cases of the \fncte{power mean} $\fM_{\fphi(x:r)}$ (\prefp{def:seq_Mr}):
    \\\begin{tabular}{ll}
        $r=\infty$:  & $\max\tuplen{x_n}$
      \\$r=1$:       & arithmetic mean
      \\$r=0$:       & geometric mean
      \\$r=-1$:      & harmonic mean
      \\$r=-\infty$: & $\min\tuplen{x_n}$
    \end{tabular}

  \item The inequalities follow directly from \prefpp{thm:seq_Mr}.

  \item \thm{Generalized AM-GM inequality}: If one is only concerned with the arithmetic mean and
        geometric mean, their relationship can be established directly using \ineqe{Jensen's Inequality}:
    \begin{align*}
      \sum_{n=1}^\xN \lambda_n x_n
        &=   b^{\log_b\left(\sum_{n=1}^\xN \lambda_n x_n\right)}
         \ge b^{\left(\sum_{n=1}^\xN \lambda_n \log_b x_n\right)}
         \quad\text{by \ineqe{Jensen's Inequality}\ifsxref{convex}{thm:jensen_ineq}}
      \\&=   \prod_{n=1}^\xN b^{\left(\lambda_n \log_b x_n\right)}
         =   \prod_{n=1}^\xN b^{\left(\log_b x_n\right)\lambda_n }
         =   \prod_{n=1}^\xN x_n^{\lambda_n}
    \end{align*}
\end{enumerate}
\end{proof}


%--------------------------------------
\begin{lemma}[\ineqd{Young's Inequality}]
\footnote{
  \citerp{carothers2000}{43},
  \citerp{tolsted1964}{5},
  \citerp{maligranda1995}{257},
  \citergc{hardy1934}{0521358809}{Theorem 24},
  \citorp{young1912}{226}
  }
\label{lem:lp_young}
\index{inequalities!Young}
%--------------------------------------
\lembox{\renewcommand{\arraystretch}{2}
  \begin{array}{rc>{\ds}l lll}
    xy  &<& \frac{x^p}{p} + \frac{y^q}{q} & \text{with $\frac{1}{p} + \frac{1}{q} = 1$} & \forall 1<p<\infty,\; x,y\ge0, & \text{but $y \ne  x^{p-1}$} \\
    xy  &=& \frac{x^p}{p} + \frac{y^q}{q} & \text{with $\frac{1}{p} + \frac{1}{q} = 1$} & \forall 1<p<\infty,\; x,y\ge0, & \text{and $y     = x^{p-1}$}
  \end{array}
  }
\end{lemma}
\begin{proof}
\begin{enumerate}
  \item Proof that $\frac{1}{p-1}=q-1$:  \label{item:lp_young_q-1}
    \begin{align*}
      \frac{1}{p} + \frac{1}{q} = 1
        &\iff \frac{q}{q} + \frac{q}{p} = q
      \\&\iff q\brp{1-\frac{1}{p}} = 1
      \\&\iff q = \frac{1}{1-\frac{1}{p}}
      \\&\iff q = \frac{p}{p-1}
      \\&\iff q-1 = \frac{p}{p-1} - \frac{p-1}{p-1}
      \\&\iff q-1 = \frac{p-(p-1)}{p-1}
      \\&\iff q-1 = \frac{1}{p-1}
    \end{align*}

  \item Proof that $v=u^{p-1}$ $\iff$ $u=v^{q-1}$:
    \begin{align*}
      u
        &= v^{\frac{1}{p-1}}
        && \text{by left hypothesis}
      \\&= v^{q-1}
        && \text{by \pref{item:lp_young_q-1}}
    \end{align*}

  \item Proof that $v=u^{p-1}$ is prope{monotonically increasing} in $u$ and
                   $u=v^{q-1}$ is prope{monotonically increasing} in $v$:
    \begin{align*}
      \deriv{v}{u}
        &= \deriv{}{u} u^{p-1}
       &&= (p-1) u^{p-2}
       &&> 0
      \\
      \deriv{u}{v}
        &= \deriv{}{v} v^{q-1}
       &&= (q-1) v^{q-2}
       &&> 0
    \end{align*}

  \item Proof that $xy \le \frac{x^p}{p} + \frac{y^q}{q}$:\\
    \begin{minipage}{8\tw/16}
      \begin{align*}
        xy
          &\le \int_0^x u^{p-1} \du + \int_0^y v^{q-1} \dv
        \\&=   \left. \frac{u^p}{p} \right|_0^x + \left. \frac{v^q}{q} \right|_0^y
        \\&=   \frac{x^p}{p}  + \frac{y^q}{q}
      \end{align*}
    \end{minipage}
    \begin{minipage}{7\tw/16}
      \begin{center}%
      \begin{fsL}%
      \setlength{\unitlength}{\tw/240}%
      \begin{picture}(140,140)(-20,-20)%
      %{\color{graphpaper}\graphpaper[10](0,0)(100,100)}%
        \thinlines%
        \color{axis}%
          \put(0,0){\line(0,1){110}}%
          \put(0,0){\line(1,0){110}}%
          \put(115,0){\makebox(0,0)[l]{$u$}}%
          \put(0,115){\makebox(0,0)[b]{$v$}}%
        \thicklines%
        \color{blue}%
          \qbezier(0,0)(80,0)(100,100)%
          \put(75,85){\makebox(0,0)[r]{$v=u^{p-1},\;u=v^{q-1}$}}%
          \put(80,85){\vector(1,0){16}}%
        \color{red}%
          \put(90,-5){\makebox(0,0)[t]{$x$}}%
          \put(-5,30){\makebox(0,0)[t]{$y$}}%
          \put(0,30){\line(1,0){90}}%
          \put(90,0){\line(0,1){65}}%
        \color{purple}%
          \put(65,65){\makebox(0,0)[br]{error}}%
          \put(60,60){\vector(1,-1){24}}%
      \end{picture}%
      \end{fsL}%
      \end{center}%
    \end{minipage}
\end{enumerate}
\end{proof}


%--------------------------------------
\begin{theorem}[\ineqd{H/:older's Inequality}]
\footnote{
  \citerpgc{bullen2003}{178}{1402015224}{2.1},
  \citerp{carothers2000}{44},
  \citerp{tolsted1964}{6},
  \citerp{maligranda1995}{257},
  \citergc{hardy1934}{0521358809}{Theorem 11},
  \citor{holder1889}
  }
\label{thm:lp_holder}
\index{inequalities!H/:older}
%--------------------------------------
Let $\tuplexn{x_n\in\C}$ and $\tuplexn{y_n\in\C}$ be complex $\xN$-tuples.
\thmbox{
  \mcom{\ds\sum_{n=1}^{\xN} \abs{x_n y_n}}{$\norm{\vx\cdot\vy}_1$} \le
  \mcom{\brp{\ds\sum_{n=1}^{\xN} \abs{x_n}^p}^\frac{1}{p}}{$\norm{\vx}_p$}
  \mcom{\brp{\ds\sum_{n=1}^{\xN} \abs{y_n}^q}^\frac{1}{q}}{$\norm{\vy}_q$}
  \qquad\text{with}\qquad
  \frac{1}{p} + \frac{1}{q} = 1
  \qquad
  \forall \; 1<p<\infty
  }
\end{theorem}
\begin{proof}
  Let $\ds \norm{x_n}_p \eqd \brp{\sum_{n=1}^{\xN} \abs{x_n}^p}^\frac{1}{p}$.
\begin{align*}
  \sum_{n=1}^{\xN} \abs{x_n y_n}
    &= \norm{\seqn{x_n}}_p\norm{\seqn{y_n}}_q
       \sum_{n=1}^{\xN} \frac{\abs{x_n y_n}}{\norm{\seqn{x_n}}_p\norm{\seqn{y_n}}_q}
  \\&= \norm{\seqn{x_n}}_p\norm{\seqn{y_n}}_q
       \sum_{n=1}^{\xN} \frac{\abs{x_n}}{\norm{\seqn{x_n}}_p} \frac{\abs{y_n}}{\norm{\seqn{y_n}}_q}
  \\&\le \norm{\seqn{x_n}}_p\norm{\seqn{y_n}}_q
       \sum_{n=1}^{\xN} \brp{\frac{1}{p}\frac{\abs{x_n}^p}{\norm{\seqn{x_n}}_p^p} +
                          \frac{1}{q}\frac{\abs{y_n}^q}{\norm{\seqn{y_n}}_q^q}}
    && \text{by \thme{Young's Inequality}}
    && \text{\xref{lem:lp_young}}
  \\&= \norm{\seqn{x_n}}_p\norm{\seqn{y_n}}_q
       \brp{
         \frac{1}{p} \cdot \frac{\ds\sum\abs{x_n}^p}{\norm{\seqn{x_n}}_p^p}
         +
         \frac{1}{q} \cdot \frac{\ds\sum\abs{y_n}^q}{\norm{\seqn{y_n}}_q^q}
         }
  \\&= \norm{\seqn{x_n}}_p \norm{\seqn{y_n}}_q
       \brp{
         \frac{1}{p} \frac{\norm{\seqn{x_n}}_p^p}{\norm{\seqn{x_n}}_p^p}
         +
         \frac{1}{q} \frac{\norm{\seqn{y_n}}_q^q}{\norm{\seqn{y_n}}_q^q}
         }
    && \text{by definition of $\normn$}
  \\&= \norm{\seqn{x_n}}_p \norm{\seqn{y_n}}_q \mcom{\brp{\frac{1}{p} + \frac{1}{q} }}{1}
  \\&= \norm{\seqn{x_n}}_p \norm{\seqn{y_n}}_q
    && \text{by $\frac{1}{p}+\frac{1}{q}=1$ constraint}
\end{align*}
\end{proof}


%--------------------------------------
\begin{theorem}[\ineqd{Minkowski's Inequality for sequences}]
\footnote{
  \citerpg{bullen2003}{179}{1402015224},
  \citerp{carothers2000}{44},
  \citerp{tolsted1964}{7},
  \citerp{maligranda1995}{258},
  \citergc{hardy1934}{0521358809}{Theorem 24},
  \citorp{minkowski1910}{115}
  }
\label{thm:lp_minkowski}
\index{inequalities!Minkowski (sequences)}
%--------------------------------------
Let $\tuplexn{x_n\in\C}$ and $\tuplexn{y_n\in\C}$ be complex $\xN$-tuples.
\thmbox{
  \brp{\sum_{n=1}^{\xN} \abs{x_n+y_n}^p}^\frac{1}{p}
  \le
  \brp{\sum_{n=1}^{\xN} \abs{x_n}^p}^\frac{1}{p}
  +
  \brp{\sum_{n=1}^{\xN} \abs{y_n}^p}^\frac{1}{p}
  \qquad
  \forall \; 1<p<\infty
  }
\end{theorem}
\begin{proof}
\mbox{}\\
\begin{enumerate}
  \item Define $\ds q\eqd\frac{p}{p-1}$ \label{idef:lp_minkowski_q}
  %\item Define $q$ in terms of $p$ such that $\frac{1}{p}+\frac{1}{q}=1$
  %\item lemma: $\ds\frac{1}{q} = \frac{p-1}{p}$. Proof:  \label{ilem:lp_minkowski_qp}
  %  \begin{align*}
  %    \frac{1}{p}+\frac{1}{q}=1
  %      &\iff \frac{1}{q} = 1-\frac{1}{p}
  %    \\&\iff \frac{1}{q} = \frac{p-1}{p}
  %  \end{align*}

  \item Define $\ds\norm{\vx}_p \eqd \brp{\sum_{n=1}^{\xN} \abs{x_n}^p}^\frac{1}{p}$.
        \label{idef:lp_minkowski_norm}

  \item Proof that $\norm{x_n+y_n}_p \le \norm{x_n}_p + \norm{y_n}_p$:
    \begin{align*}
      &\boxed{\norm{x_n+y_n}_p^p}
      \\&= \sum_{n=1}^{\xN} \abs{x_n+y_n}^p
        && \text{by definition of $\normn_p$}
        && \text{\xref{idef:lp_minkowski_norm}}
      \\&= \sum_{n=1}^{\xN} \abs{x_n+y_n} \abs{x_n+y_n}^{p-1}
        && \text{by \prope{homogeneous} property of $\absn$}
        && \text{\ifxref{numsys}{thm:C_norm}}
      \\&\boxed{\le} \sum_{n=1}^{\xN} \abs{x_n} \abs{x_n+y_n}^{p-1} +
                \sum_{n=1}^{\xN} \abs{y_n} \abs{x_n+y_n}^{p-1}
        && \text{by \prope{subadditive} property of $\absn$}
        && \text{\ifxref{numsys}{thm:C_norm}}
      \\&= \sum_{n=1}^{\xN} \abs{x_n (x_n+y_n)^{p-1}} +
                \sum_{n=1}^{\xN} \abs{y_n (x_n+y_n)^{p-1}}
        && \text{by \prope{homogeneous} property of $\absn$}
        && \text{\ifxref{numsys}{thm:C_norm}}
      \\&\boxed{\le} \norm{x_n}_p \norm{(x_n+y_n)^{p-1}}_q + \norm{y_n}_p \norm{(x_n+y_n)^{p-1}}_q
        && \text{by \ineqe{H/:older's Inequality}}
        && \text{\xref{thm:lp_holder}}
      \\&= \brp{\norm{x_n}_p+\norm{y_n}_p} \norm{(x_n+y_n)^{p-1}}_q
      \\&= \brp{\norm{x_n}_p+\norm{y_n}_p} \brp{\sum_{n=1}^{\xN}\abs{(x_n+y_n)^{p-1}}^q}^\frac{1}{q}
        && \text{by definition of $\normn_p$}
        && \text{\xref{idef:lp_minkowski_norm}}
      \\&= \brp{\norm{x_n}_p+\norm{y_n}_p} \brp{\sum_{n=1}^{\xN}\abs{(x_n+y_n)^{p-1}}^\frac{p}{p-1}}^\frac{p-1}{p}
        && \text{by \pref{idef:lp_minkowski_q}}
      \\&= \brp{\norm{x_n}_p+\norm{y_n}_p} \brp{\sum_{n=1}^{\xN}\abs{(x_n+y_n)}^p}^\frac{p-1}{p}
      \\&= \brp{\norm{x_n}_p+\norm{y_n}_p} \norm{x_n+y_n}^{p-1}
        && \text{by definition of $\normn_p$}
        && \text{\xref{idef:lp_minkowski_norm}}
      \\&\implies\quad\boxed{\norm{x_n+y_n}_p \le \norm{x_n}_p+\norm{y_n}_p}
    \end{align*}
  \end{enumerate}
\end{proof}


%---------------------------------------
\qboxnps % 1826 quote by Abel
%---------------------------------------
  {
  Niels Henrik Abel in an 1826 letter
  \index{Abel}
  \index{quotes!Abel}
  \footnotemark
  }
  {../common/people/cauchy_wkp_pdomain.jpg}
  {Cauchy is the only one occupied with pure mathematics: Poisson, Fourier, Ampere, etc.,
   busy themselves exclusively with magnetism and other physical subjects.
   Mr. Laplace writes nothing now, I believe.}
  \footnotetext{
    quote: \citerpgc{bell1986}{318}{9780671628185}{Chapter 17. ``{\scshape Genius and Poverty}" ``{\scshape Abel (1802--1829)}"},
           \citerpgc{boyer2011}{462}{0470525487}{without ``Mr. Laplace\ldots" portion}.
    image: \url{http://en.wikipedia.org/wiki/File:Augustin-Louis_Cauchy_1901.jpg}, public domain
    }

%---------------------------------------
\begin{theorem}[\ineqd{Cauchy-Schwarz Inequality for sequences}]
\footnote{
  %\citerpg{bullen2003}{183}{1402015224}\\ ???
  \citerp{ab}{278},
  \citeP{schwarz1885},
  \citeP{bouniakowsky1859},
  \citerpgc{hardy1934}{25}{0521358809}{Theorem 11},
  \citerpc{cauchy1821}{455}{???}
  }
\label{thm:seq_cs}
\index{inequalities!Cauchy-Schwarz}
\index{inequalities!Cauchy-Bunyakovsky-Schwarz}
%---------------------------------------
Let $\tuplexn{x_n\in\C}$ and $\tuplexn{y_n\in\C}$ be complex $\xN$-tuples.
 \thmbox{\begin{array}{>{\ds}rc>{\ds}l@{\qquad}C}
   \abs{\sum_{n=1}^{\xN} x_n y_n^\ast }^2 &\le& \brp{\sum_{n=1}^{\xN} \abs{x_n}^2}\;\brp{\sum_{n=1}^{\xN} \abs{y_n}^2}
     & \forall \vx,\vy\in\spX
     \\
   \abs{\sum_{n=1}^{\xN} x_n y_n^\ast }^2 &=& \brp{\sum_{n=1}^{\xN} \abs{x_n}^2}\; \brp{\sum_{n=1}^{\xN} \abs{y_n}^2}
   \quad\iff\quad \exists a\in\C \st \vy=a\vx
     & \forall \vx,\vy\in\spX
 \end{array}}
\end{theorem}
\begin{proof}
\begin{enumerate}
  \item The \ineqe{Cauchy-Schwarz Inequality for sequences} is a special case of the \ineqe{H/:older inequality} \xref{thm:lp_holder}
        for $p=q=2$.

  \item Alternatively, the \ineqe{Cauchy-Schwarz inequality for sequences} is a special case of
        the \ineqe{Cauchy-Schwarz inequality for inner-product spaces}:
    \begin{enumerate}
      \item $\inprod{x_n}{y_n}\eqd\sum_{n=1}^{\xN}x_n y_n$ is an inner-product and $\opair{\tuplen{x_n}}{\inprodn}$ is an inner-product space.
      \item By the more general \ineqe{Cauchy-Schwarz Inequality for inner-product spaces},
        \begin{align*}
          \brp{\sum_{n=1}^{\xN} a_n\lambda_n}^2
            &\eqd \inprod{a_n}{\lambda_n}^2
            && \text{by definition of $\inprod{x_n}{y_n}$}
          \\&\boxed{\leq} \norm{x_n}^2\norm{y_n}^2
            && \begin{array}{MM}
                 by & \ineqe{Cauchy-Schwarz Inequality for inner-product spaces}\\
                    & \xref{thm:cs}
               \end{array}
          \\&\eqd \brp{\sum_{n=1}^{\xN} x_n^2} \brp{\sum_{n=1}^{\xN} y_n^2}
            && \text{by definition of $\normn$}
        \end{align*}
    \end{enumerate}

  \item Not only does the \ineqe{H/:older inequality} imply the \ineqe{Cauchy-Schwarz inequality}, but somewhat surprisingly,
        the converse is also true: The Cauchy-Schwarz inequality implies the H/:older inequality.%
        \footnote{
            \citerppgc{bullen2003}{183}{185}{1402015224}{Theorem 5}
          }
\end{enumerate}
\end{proof}


%--------------------------------------
\begin{proposition}
\footnote{
  \citerp{carothers2000}{43}
  }
%--------------------------------------
\propbox{
  (x+y)^p \le 2^p(x^p + y^p)
  \qquad
  \forall x,y\ge 0,\; 1<p<\infty
  }
\end{proposition}
\begin{proof}
\begin{align*}
  (x+y)^p
    &\le \brp{2\max\setn{x,y}}^p
  \\&=   2^p \brp{\max\setn{x,y}}^p
  \\&=   2^p \brp{\max\setn{x^p,y^p}}
  \\&\le 2^p(x^p + y^p)
\end{align*}
\end{proof}

%=======================================
\section{Power Sums}
%=======================================
%--------------------------------------
\begin{theorem}[\thmd{Geometric Series}]
\footnote{
  \citerpc{hall1894}{39}{article 55}
  }
\label{thm:series_geometric}
%--------------------------------------
\thmbox{
  \sum_{k=0}^{n-1} r^k = \frac{1-r^n}{1-r}
  \qquad
  \forall r\in\C\setd0
  }
\end{theorem}
\begin{proof}
\begin{align*}
  \boxed{\sum_{k=0}^{n-1} r^k}
    &= \brp{\frac{1}{1-r}}\brs{\brp{1-r}\sum_{k=0}^{n-1} r^k}
     = \brp{\frac{1}{1-r}}\brs{\sum_{k=0}^{n-1} r^k - r\sum_{k=0}^{n-1} r^k}
     = \brp{\frac{1}{1-r}}\brs{\sum_{k=0}^{n-1} r^k - \brp{\sum_{k=0}^{n-1} r^k -1 + r^n}}
  \\&= \brp{\frac{1}{1-r}}\brs{1 - r^n}
     = \boxed{\frac{1-r^n}{1-r}}
\end{align*}
\end{proof}

%--------------------------------------
\begin{lemma}
\label{lem:series_sumT}
%--------------------------------------
Let $\ff(x)$ be a function.
\lembox{
  \fS(x)\eqd\sum_{n\in\Z} \ff(x+n\tau) = \fS(x+\tau) \qquad \text{($\fS(x)$ is \prope{periodic} with period $\tau$)}
}
\end{lemma}
\begin{proof}
\begin{align*}
  \fS(x+\tau)
    &\eqd \sum_{n\in\Z} \ff(x+\tau+n\tau)
     =    \sum_{n\in\Z} \ff(x+(n+1)\tau)
     =    \sum_{m\in\Z} \ff(x+m\tau)
     \quad\text{(where $m\eqd n+1$)}
  \\&\eqd \fS(x)
\end{align*}
\end{proof}

%--------------------------------------
\begin{proposition}[\thmd{Power Sums}]
\footnote{
  \citerppg{amann2008}{51}{57}{3764374721},
  \citerpgc{menini2004}{91}{0824709853}{Exercises 5.36--5.39}
  %http://mathworld.wolfram.com/PowerSum.html
  }
\label{prop:powersums}
%--------------------------------------
\propbox{\begin{array}{>{\ds}lc>{\ds}lC | >{\ds}lc>{\ds}lC }
    \sum_{k=1}^n k   &=& \frac{n(n+1)}{2}        &\forall n\in\Zp & \sum_{k=1}^n k^3 &=& \frac{n^2(n+1)^2}{4}  &\forall n\in\Zp
  \\\sum_{k=1}^n k^2 &=& \frac{n(n+1)(2n+1)}{6}  &\forall n\in\Zp & \sum_{k=1}^n k^4 &=& \frac{n(n+1)(2n+1)(3n^2+3n-1)}{30}  &\forall n\in\Zp
\end{array}}
\end{proposition}
\begin{proof}
\begin{align*}
\intertext{1. Proof that $\sum_{k=1}^n k=\frac{n(n+1)}{2}$: (proof by induction)}
  \sum_{k=1}^{n=1} k
    &= 1
     = \frac{1(1+1)}{2}
     = \left.\frac{n(n+1)}{2}\right|_{n=1}
  \\
  \sum_{k=1}^{n+1} k
    &= \Bigg(\sum_{k=1}^n k\Bigg) + \Bigg(n+1\Bigg)
     = \mcom{\Bigg(\frac{n(n+1)}{2}\Bigg) + \Bigg(n+1\Bigg)}
       {by left hypothesis}
     = \Bigg(n+1\Bigg)\Bigg(\frac{n}{2}+1\Bigg)
  \\&= \Bigg(n+1\Bigg)\Bigg(\frac{n+2}{2}\Bigg)
     = \frac{(n+1)(n+2)}{2}
  \\
\intertext{2. Proof that $\sum_{k=1}^n k^2 = \frac{n(n+1)(2n+1)}{6}$: (proof by induction)}
  \sum_{k=1}^{n=1} k^2
    &= 1
     = \frac{1(1+1)(2+1)}{6}
     = \left.\frac{n(n+1)(2n+1)}{6}\right|_{n=1}
  \\
  \sum_{k=1}^{n+1} k^2
    &= \Bigg(\sum_{k=1}^n k^2\Bigg) + \Bigg(n+1\Bigg)^2
     = \mcom{\Bigg(\frac{n(n+1)(2n+1)}{6}\Bigg) + \Bigg(n+1\Bigg)^2}
            {by left hypothesis}
     = \Bigg(n+1\Bigg)\Bigg(\frac{n(2n+1)+6(n+1)}{6}\Bigg)
  \\&= \Bigg(n+1\Bigg)\Bigg(\frac{2n^2+7n+6}{6}\Bigg)
     = \Bigg(n+1\Bigg)\Bigg(\frac{(n+2)(2n+3)}{6}\Bigg)
     = \frac{(n+1)[(n+1)+1][2(n+1)+1]}{6}
\end{align*}
\end{proof}


