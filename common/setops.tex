%============================================================================
% LaTeX File
% Daniel J. Greenhoe
% This is a file to be included in another .tex file in a project
% for which setstrct.tex is *not* included but for which set operations
% are referenced.
% example of use: %============================================================================
% LaTeX File
% Daniel J. Greenhoe
% This is a file to be included in another .tex file in a project
% for which setstrct.tex is *not* included but for which set operations
% are referenced.
% example of use: %============================================================================
% LaTeX File
% Daniel J. Greenhoe
% This is a file to be included in another .tex file in a project
% for which setstrct.tex is *not* included but for which set operations
% are referenced.
% example of use: %============================================================================
% LaTeX File
% Daniel J. Greenhoe
% This is a file to be included in another .tex file in a project
% for which setstrct.tex is *not* included but for which set operations
% are referenced.
% example of use: \input{../common/setops.tex}
% example of when possibly useful:
%   A document includes the concept of closure of a set,
%   and in turn includes topology.tex.
%============================================================================
%---------------------------------------
\begin{definition}
\label{def:pset}
\label{def:powerset}
\index{set!power}
%---------------------------------------
\defbox{\begin{array}{M}\indxs{\psetx}
  The \hid{power set} $\psetx$ on a set $\sid$ is defined as
    \\\indentx$\ds\psetx \eqd \set{\setA}{\setA\sorel\sid}$
      \qquad\scriptsize(the set of all subsets of $\sid$)
\end{array}}
\end{definition}

%---------------------------------------
\begin{definition}
\citetbl{
  \citerpg{molchanov2005}{389}{185233892X},
  \citerpg{pap1995}{7}{0792336585},
  \citerpg{hahn1948}{254}{111422295X}
  }
\label{def:ss}
\label{def:paving}
%---------------------------------------
Let $\psetx$ be the \structe{power set} \xref{def:pset} of a set $\setX$.
\defbox{
  \begin{array}{Ml}
  A set $\sssSx$ is a \structd{set structure} on $\sid$ if & \sssSx\sorel\psetx.\\
  A \structe{set structure} $\sssQx$ is a \structd{paving}  on $\sid$ if & \emptyset\in\sssQx.
  \end{array}
  }
\end{definition}

%---------------------------------------
\begin{definition}
\footnote{
  \citerpgc{pap1995}{8}{0792336585}{Definition 2.3: extended real-valued set function},
  \citerpgc{halmos1950}{30}{0387900888}{\textsection7. {\scshape measure on rings}},
  \citer{hahn1948},
  \citeP{choquet1954}
  }
\label{def:setf}
%---------------------------------------
Let $\sssQx$ be a \structe{paving} \xref{def:paving} on a set $\setX$.
Let $\setY$ be a set containing the element $0$.
\defboxt{
  A function $\fm\in\clF{\sssQx}{\setY}$ is a \fnctd{set function} if 
  \\\indentx$\fm(\emptyset)=0$.
  }
\end{definition}


\begin{figure}
  \centering%
  $\begin{array}{*{4}{c}}
      \includegraphics{graphics/setop_0000.pdf}%
     &\includegraphics{graphics/setop_0011.pdf}%
     &\includegraphics{graphics/setop_0100.pdf}%
     &\includegraphics{graphics/setop_0101.pdf}%
    \\%
      \emptyset
     &\cmpA
     &\setA\setd\setB
     &\cmpB
    \\%
      \includegraphics{graphics/setop_0110.pdf}%
     &\includegraphics{graphics/setop_1000.pdf}%
     &\includegraphics{graphics/setop_1110.pdf}%
     &\includegraphics{graphics/setop_1111.pdf}%
    \\%
      \setA\sets\setB
     &\setA\seti\setB
     &\setA\setu\setB
     &\setX
  \end{array}$
  \caption{Venn diagrams for standard set operations \xref{def:setops} \label{fig:setops}}
\end{figure}
%\pref{def:ss_setops} (next) introduces seven standard set operations: 
%two \prope{nullary} operations, one \prope{unary} operation, and four \structe{binary operation}s\ifsxref{relation}{def:arity}.
%---------------------------------------
\begin{definition}
\citetbl{
  \citerppg{ab}{2}{4}{0120502577}
  }
\label{def:ss_setops}
\label{def:setops}
\index{sets!operations}
%---------------------------------------
Let $\psetx$ be the \structe{power set} \xref{def:pset} on a set $\sid$.
Let $\lnot$ represent the \ope{logical not} operation,
    $\lor$  represent the \ope{logical or} operation,
    $\land$ represent the \ope{logical and} operation\ifsxref{logic}{def:logic}, and
    $\lxor$ represent the \ope{logical exclusive-or} operation\ifsxref{logic}{def:lxor}.
\defbox{%
  \begin{array}{Mcc|l@{\,}c@{\,}l   @{\;}c@{\;}  l @{\,}r@{\,}c@{\,}r |C}
    \mc{2}{N}{name/symbol} & \mc{1}{N|}{arity}   & \mc{8}{N|}{definition} & \mc{1}{N}{domain}
    \\\hline
      \opd{emptyset}             & \hxs{\szero } & 0 &        &     & \szero&\eqd& \big\{x\in\sid\big| & x\ne x           &     &                  \big\} &
    \\\opd{universal set}        & \hxs{\sid   } & 0 &        &     & \sid  &\eqd& \big\{x\in\sid\big| & x=x              &     &                  \big\} &
    \\\opd{complement}           & \hxs{\setopc} & 1 &        &     & \cmpA &\eqd& \big\{x\in\sid\big| & \lnot(x\in\setA) &     &                  \big\} & \forall \setA\in\psetx
    \\\opd{union}                & \hxs{\setu  } & 2 & \setA  &\setu& \setB &\eqd& \big\{x\in\sid\big| &      (x\in\setA) &\lor &      (x\in\setB) \big\} & \forall \setA,\setB\in\psetx
    \\\opd{intersection}         & \hxs{\seti  } & 2 & \setA  &\seti& \setB &\eqd& \big\{x\in\sid\big| &      (x\in\setA) &\land&      (x\in\setB) \big\} & \forall \setA,\setB\in\psetx
    \\\opd{difference}           & \hxs{\setd  } & 2 & \setA  &\setd& \setB &\eqd& \big\{x\in\sid\big| &      (x\in\setA) &\land& \lnot(x\in\setB) \big\} & \forall \setA,\setB\in\psetx
    \\\opd{symmetric difference} & \hxs{\sets  } & 2 & \setA  &\sets& \setB &\eqd& \big\{x\in\sid\big| &      (x\in\setA) &\lxor&      (x\in\setB) \big\} & \forall \setA,\setB\in\psetx
  \end{array}%
  }
\end{definition}

With regards to the standard seven set operations only,
\pref{thm:ss_rel_gg} (next) expresses each of the set operations
in terms of pairs of other operations.
%---------------------------------------
\begin{theorem}
\label{thm:ss_rel_gg}
%\citetbl{
%  \citerpg{vaidyanathaswamy1960}{16}{0486404560}
%  }
%---------------------------------------
%Each of the seven set operations may be expressed in terms of pairs of other set operations as follows:
\thmbox{\begin{array}{r*{6}{cl}}
  \sid&=& \cmp{\szero}
  \\
  \szero
    &=& \cmp{\sid}
     =  \cmp{\brp{\setA\setu\cmpA}}
    &=& \setA\seti\cmpA
    &=& \setA\setd\setA
    &=& \setA\sets\setA
  \\
  \sid
    &=& \setA\setu\cmpA
    &=& \cmp{\brp{\setA\seti\cmpA}}
  \\
  \cmpA
    &=& \sid\setd\setA
    &=& \sid\sets\setA
  \\
  \setA\setu\setB
    &=& \cmp{\brp{\cmpA\seti\cmpB}}
    &=& \brp{\setA\sets\setB}\sets\brp{\setA\seti\setB}
    &=& \brp{\setA\setd\setB}\sets\setB
  \\
  \setA\seti\setB
    &=& \cmp{\brp{\cmpA\setu\cmpB}}
    &=& \brp{\setA\setu\setB}\sets\setA\sets\setB
    &=& \setA\setd\brp{\setA\setd\setB}
  \\
  \setA\setd\setB
    &=& \cmp{\brp{\cmpA\setu\setB}}
    &=& \setA\seti\cmpB
    &=& \brp{\setA\setu\setB}\sets\setB
    &=& \brp{\setA\sets\setB}\seti\setA
  \\
  \setA\sets\setB
    &=& \mc{3}{l}{\brs{\cmp{\brp{\cmpA\setu\setB}}} \setu \brs{\cmp{\brp{\setA\setu\cmpB}}}}
    &=& \mc{3}{l}{\brs{\cmp{\brp{\cmpA\seti\cmpB}}} \seti \cmp{\brp{\setA\seti\setB}}}
  \\&=& \brp{\setA\setd\setB}\setu\brp{\setB\setd\setA}
\end{array}}
\end{theorem}

%---------------------------------------
\begin{definition}
\label{def:subset}
%\label{def:ss_subset}
%---------------------------------------
Let $\ssetS$ be a \structe{set structure} \xref{def:ss} on a set $\sid$.
\defbox{\begin{array}{M}
  The relation $\sorel\in\clR{\ssetS}{\ssetS}$ is defined as
  \\\indentx$ \setA \sorel \setB \qquad\text{if}\qquad x\in\setA \implies x\in\setB \qquad \forall x\in\sid$
\end{array}}
\end{definition}

%---------------------------------------
\begin{theorem}
\citetbl{
  \citerppg{dieudonne1969}{3}{4}{1406727911},
  \citerpg{copson1968}{9}{0521047226}
  }
%\label{cor:ss_fc}
\label{thm:algprop}
\index{algebra of sets}
\index{set structures!algebra of sets}
%---------------------------------------
Let $\algA$ be a \structe{set structure} \xref{def:ss} on a set $\setX$.
\thmboxt{
  %\textbf{If} $\setu$ and $\seti$ are closed in $\topT$ \textbf{then}
  $\algA$ is an \structb{algebra of sets} \quad$\implies$\quad $\forall\setA,\setB,\setC\in\algA$
  %\textbf{If} $\ssetS$ is closed under $\sor$, $\sand$, and $\snot$, \textbf{then} for all $\setA,\setB,\setC\in\ssetS$
  \\\footnotesize
  ${\begin{array}{rcl|rcl|D}
       \setA \setu \setA &=& \setA
     & \setA \seti \setA &=& \setA
     & (\prope{idempotent})
    \\ \setA \setu \setB &=& \setB \setu \setA
     & \setA \seti \setB &=& \setB \seti \setA
     & (\prope{commutative})
    \\ \setA\setu (\setB\setu\setC) &=& (\setA\setu\setB) \setu \setC
     & \setA\seti (\setB\seti\setC) &=& (\setA\seti\setB) \seti \setC
     & (\prope{associative})
    \\ \setA \setu  (\setA \seti \setB) &=& \setA
     & \setA \seti (\setA \setu  \setB) &=& \setA
     & (\prope{absorptive})
    \\ \setA\setu(\setB\seti\setC) &=& (\setA\setu\setB) \seti (\setA\setu\setC)
     & \setA\seti(\setB\setu\setC) &=& (\setA\seti\setB) \setu (\setA\seti\setC)
     & (\prope{distributive})
    \\ \setA \setu \szero      &=& \setA
     & \setA \seti \sid        &=& \setA
     & (\prope{identity})
    \\ \setA \setu \sid        &=& \sid
     & \setA \seti \szero      &=& \szero
     & (\prope{bounded})
    \\ \setA \setu \cmpA         &=& \sid
     & \setA \seti \cmpA         &=& \szero
     & (\prope{complemented})
    \\ \cmpp{\cmpA}          &=& \setA
     &                           &&
     & (\prope{uniquely complemented})
    \\ \cmpp{\setA\setu \setB} &=& \cmpA \seti \cmpB
     & \cmpp{\setA\seti \setB} &=& \cmpA \setu \cmpB
     & (\prope{de Morgan})
    \\\hline
      \mc{3}{H|}{property emphasizing $\setu$}
    & \mc{3}{H|}{dual property emphasizing $\seti$}
    & \mc{1}{H}  {property name}
  \end{array}}$
  }
\end{theorem}
\begin{proof}
\begin{enume}
  \item $\ssetS$ is an \structe{algebra of sets}\ifsxref{setstrct}{def:ss_algebra}.
  \item By the \thme{Stone Representation Theorem}\ifsxref{setstrct}{thm:lat_algebra}, 
        $\booalg{\ssetS}{\sorel}{\sor}{\sand}{\snot}{\szero}{\sid}$ is a \structe{Boolean algebra}.
  \item The properties listed are all properties of \structe{Boolean algebra}s\ifsxref{boolean}{thm:boo_prop}.
\end{enume}
\end{proof}



% example of when possibly useful:
%   A document includes the concept of closure of a set,
%   and in turn includes topology.tex.
%============================================================================
%---------------------------------------
\begin{definition}
\label{def:pset}
\label{def:powerset}
\index{set!power}
%---------------------------------------
\defbox{\begin{array}{M}\indxs{\psetx}
  The \hid{power set} $\psetx$ on a set $\sid$ is defined as
    \\\indentx$\ds\psetx \eqd \set{\setA}{\setA\sorel\sid}$
      \qquad\scriptsize(the set of all subsets of $\sid$)
\end{array}}
\end{definition}

%---------------------------------------
\begin{definition}
\citetbl{
  \citerpg{molchanov2005}{389}{185233892X},
  \citerpg{pap1995}{7}{0792336585},
  \citerpg{hahn1948}{254}{111422295X}
  }
\label{def:ss}
\label{def:paving}
%---------------------------------------
Let $\psetx$ be the \structe{power set} \xref{def:pset} of a set $\setX$.
\defbox{
  \begin{array}{Ml}
  A set $\sssSx$ is a \structd{set structure} on $\sid$ if & \sssSx\sorel\psetx.\\
  A \structe{set structure} $\sssQx$ is a \structd{paving}  on $\sid$ if & \emptyset\in\sssQx.
  \end{array}
  }
\end{definition}

%---------------------------------------
\begin{definition}
\footnote{
  \citerpgc{pap1995}{8}{0792336585}{Definition 2.3: extended real-valued set function},
  \citerpgc{halmos1950}{30}{0387900888}{\textsection7. {\scshape measure on rings}},
  \citer{hahn1948},
  \citeP{choquet1954}
  }
\label{def:setf}
%---------------------------------------
Let $\sssQx$ be a \structe{paving} \xref{def:paving} on a set $\setX$.
Let $\setY$ be a set containing the element $0$.
\defboxt{
  A function $\fm\in\clF{\sssQx}{\setY}$ is a \fnctd{set function} if 
  \\\indentx$\fm(\emptyset)=0$.
  }
\end{definition}


\begin{figure}
  \centering%
  $\begin{array}{*{4}{c}}
      \includegraphics{graphics/setop_0000.pdf}%
     &\includegraphics{graphics/setop_0011.pdf}%
     &\includegraphics{graphics/setop_0100.pdf}%
     &\includegraphics{graphics/setop_0101.pdf}%
    \\%
      \emptyset
     &\cmpA
     &\setA\setd\setB
     &\cmpB
    \\%
      \includegraphics{graphics/setop_0110.pdf}%
     &\includegraphics{graphics/setop_1000.pdf}%
     &\includegraphics{graphics/setop_1110.pdf}%
     &\includegraphics{graphics/setop_1111.pdf}%
    \\%
      \setA\sets\setB
     &\setA\seti\setB
     &\setA\setu\setB
     &\setX
  \end{array}$
  \caption{Venn diagrams for standard set operations \xref{def:setops} \label{fig:setops}}
\end{figure}
%\pref{def:ss_setops} (next) introduces seven standard set operations: 
%two \prope{nullary} operations, one \prope{unary} operation, and four \structe{binary operation}s\ifsxref{relation}{def:arity}.
%---------------------------------------
\begin{definition}
\citetbl{
  \citerppg{ab}{2}{4}{0120502577}
  }
\label{def:ss_setops}
\label{def:setops}
\index{sets!operations}
%---------------------------------------
Let $\psetx$ be the \structe{power set} \xref{def:pset} on a set $\sid$.
Let $\lnot$ represent the \ope{logical not} operation,
    $\lor$  represent the \ope{logical or} operation,
    $\land$ represent the \ope{logical and} operation\ifsxref{logic}{def:logic}, and
    $\lxor$ represent the \ope{logical exclusive-or} operation\ifsxref{logic}{def:lxor}.
\defbox{%
  \begin{array}{Mcc|l@{\,}c@{\,}l   @{\;}c@{\;}  l @{\,}r@{\,}c@{\,}r |C}
    \mc{2}{N}{name/symbol} & \mc{1}{N|}{arity}   & \mc{8}{N|}{definition} & \mc{1}{N}{domain}
    \\\hline
      \opd{emptyset}             & \hxs{\szero } & 0 &        &     & \szero&\eqd& \big\{x\in\sid\big| & x\ne x           &     &                  \big\} &
    \\\opd{universal set}        & \hxs{\sid   } & 0 &        &     & \sid  &\eqd& \big\{x\in\sid\big| & x=x              &     &                  \big\} &
    \\\opd{complement}           & \hxs{\setopc} & 1 &        &     & \cmpA &\eqd& \big\{x\in\sid\big| & \lnot(x\in\setA) &     &                  \big\} & \forall \setA\in\psetx
    \\\opd{union}                & \hxs{\setu  } & 2 & \setA  &\setu& \setB &\eqd& \big\{x\in\sid\big| &      (x\in\setA) &\lor &      (x\in\setB) \big\} & \forall \setA,\setB\in\psetx
    \\\opd{intersection}         & \hxs{\seti  } & 2 & \setA  &\seti& \setB &\eqd& \big\{x\in\sid\big| &      (x\in\setA) &\land&      (x\in\setB) \big\} & \forall \setA,\setB\in\psetx
    \\\opd{difference}           & \hxs{\setd  } & 2 & \setA  &\setd& \setB &\eqd& \big\{x\in\sid\big| &      (x\in\setA) &\land& \lnot(x\in\setB) \big\} & \forall \setA,\setB\in\psetx
    \\\opd{symmetric difference} & \hxs{\sets  } & 2 & \setA  &\sets& \setB &\eqd& \big\{x\in\sid\big| &      (x\in\setA) &\lxor&      (x\in\setB) \big\} & \forall \setA,\setB\in\psetx
  \end{array}%
  }
\end{definition}

With regards to the standard seven set operations only,
\pref{thm:ss_rel_gg} (next) expresses each of the set operations
in terms of pairs of other operations.
%---------------------------------------
\begin{theorem}
\label{thm:ss_rel_gg}
%\citetbl{
%  \citerpg{vaidyanathaswamy1960}{16}{0486404560}
%  }
%---------------------------------------
%Each of the seven set operations may be expressed in terms of pairs of other set operations as follows:
\thmbox{\begin{array}{r*{6}{cl}}
  \sid&=& \cmp{\szero}
  \\
  \szero
    &=& \cmp{\sid}
     =  \cmp{\brp{\setA\setu\cmpA}}
    &=& \setA\seti\cmpA
    &=& \setA\setd\setA
    &=& \setA\sets\setA
  \\
  \sid
    &=& \setA\setu\cmpA
    &=& \cmp{\brp{\setA\seti\cmpA}}
  \\
  \cmpA
    &=& \sid\setd\setA
    &=& \sid\sets\setA
  \\
  \setA\setu\setB
    &=& \cmp{\brp{\cmpA\seti\cmpB}}
    &=& \brp{\setA\sets\setB}\sets\brp{\setA\seti\setB}
    &=& \brp{\setA\setd\setB}\sets\setB
  \\
  \setA\seti\setB
    &=& \cmp{\brp{\cmpA\setu\cmpB}}
    &=& \brp{\setA\setu\setB}\sets\setA\sets\setB
    &=& \setA\setd\brp{\setA\setd\setB}
  \\
  \setA\setd\setB
    &=& \cmp{\brp{\cmpA\setu\setB}}
    &=& \setA\seti\cmpB
    &=& \brp{\setA\setu\setB}\sets\setB
    &=& \brp{\setA\sets\setB}\seti\setA
  \\
  \setA\sets\setB
    &=& \mc{3}{l}{\brs{\cmp{\brp{\cmpA\setu\setB}}} \setu \brs{\cmp{\brp{\setA\setu\cmpB}}}}
    &=& \mc{3}{l}{\brs{\cmp{\brp{\cmpA\seti\cmpB}}} \seti \cmp{\brp{\setA\seti\setB}}}
  \\&=& \brp{\setA\setd\setB}\setu\brp{\setB\setd\setA}
\end{array}}
\end{theorem}

%---------------------------------------
\begin{definition}
\label{def:subset}
%\label{def:ss_subset}
%---------------------------------------
Let $\ssetS$ be a \structe{set structure} \xref{def:ss} on a set $\sid$.
\defbox{\begin{array}{M}
  The relation $\sorel\in\clR{\ssetS}{\ssetS}$ is defined as
  \\\indentx$ \setA \sorel \setB \qquad\text{if}\qquad x\in\setA \implies x\in\setB \qquad \forall x\in\sid$
\end{array}}
\end{definition}

%---------------------------------------
\begin{theorem}
\citetbl{
  \citerppg{dieudonne1969}{3}{4}{1406727911},
  \citerpg{copson1968}{9}{0521047226}
  }
%\label{cor:ss_fc}
\label{thm:algprop}
\index{algebra of sets}
\index{set structures!algebra of sets}
%---------------------------------------
Let $\algA$ be a \structe{set structure} \xref{def:ss} on a set $\setX$.
\thmboxt{
  %\textbf{If} $\setu$ and $\seti$ are closed in $\topT$ \textbf{then}
  $\algA$ is an \structb{algebra of sets} \quad$\implies$\quad $\forall\setA,\setB,\setC\in\algA$
  %\textbf{If} $\ssetS$ is closed under $\sor$, $\sand$, and $\snot$, \textbf{then} for all $\setA,\setB,\setC\in\ssetS$
  \\\footnotesize
  ${\begin{array}{rcl|rcl|D}
       \setA \setu \setA &=& \setA
     & \setA \seti \setA &=& \setA
     & (\prope{idempotent})
    \\ \setA \setu \setB &=& \setB \setu \setA
     & \setA \seti \setB &=& \setB \seti \setA
     & (\prope{commutative})
    \\ \setA\setu (\setB\setu\setC) &=& (\setA\setu\setB) \setu \setC
     & \setA\seti (\setB\seti\setC) &=& (\setA\seti\setB) \seti \setC
     & (\prope{associative})
    \\ \setA \setu  (\setA \seti \setB) &=& \setA
     & \setA \seti (\setA \setu  \setB) &=& \setA
     & (\prope{absorptive})
    \\ \setA\setu(\setB\seti\setC) &=& (\setA\setu\setB) \seti (\setA\setu\setC)
     & \setA\seti(\setB\setu\setC) &=& (\setA\seti\setB) \setu (\setA\seti\setC)
     & (\prope{distributive})
    \\ \setA \setu \szero      &=& \setA
     & \setA \seti \sid        &=& \setA
     & (\prope{identity})
    \\ \setA \setu \sid        &=& \sid
     & \setA \seti \szero      &=& \szero
     & (\prope{bounded})
    \\ \setA \setu \cmpA         &=& \sid
     & \setA \seti \cmpA         &=& \szero
     & (\prope{complemented})
    \\ \cmpp{\cmpA}          &=& \setA
     &                           &&
     & (\prope{uniquely complemented})
    \\ \cmpp{\setA\setu \setB} &=& \cmpA \seti \cmpB
     & \cmpp{\setA\seti \setB} &=& \cmpA \setu \cmpB
     & (\prope{de Morgan})
    \\\hline
      \mc{3}{H|}{property emphasizing $\setu$}
    & \mc{3}{H|}{dual property emphasizing $\seti$}
    & \mc{1}{H}  {property name}
  \end{array}}$
  }
\end{theorem}
\begin{proof}
\begin{enume}
  \item $\ssetS$ is an \structe{algebra of sets}\ifsxref{setstrct}{def:ss_algebra}.
  \item By the \thme{Stone Representation Theorem}\ifsxref{setstrct}{thm:lat_algebra}, 
        $\booalg{\ssetS}{\sorel}{\sor}{\sand}{\snot}{\szero}{\sid}$ is a \structe{Boolean algebra}.
  \item The properties listed are all properties of \structe{Boolean algebra}s\ifsxref{boolean}{thm:boo_prop}.
\end{enume}
\end{proof}



% example of when possibly useful:
%   A document includes the concept of closure of a set,
%   and in turn includes topology.tex.
%============================================================================
%---------------------------------------
\begin{definition}
\label{def:pset}
\label{def:powerset}
\index{set!power}
%---------------------------------------
\defbox{\begin{array}{M}\indxs{\psetx}
  The \hid{power set} $\psetx$ on a set $\sid$ is defined as
    \\\indentx$\ds\psetx \eqd \set{\setA}{\setA\sorel\sid}$
      \qquad\scriptsize(the set of all subsets of $\sid$)
\end{array}}
\end{definition}

%---------------------------------------
\begin{definition}
\citetbl{
  \citerpg{molchanov2005}{389}{185233892X},
  \citerpg{pap1995}{7}{0792336585},
  \citerpg{hahn1948}{254}{111422295X}
  }
\label{def:ss}
\label{def:paving}
%---------------------------------------
Let $\psetx$ be the \structe{power set} \xref{def:pset} of a set $\setX$.
\defbox{
  \begin{array}{Ml}
  A set $\sssSx$ is a \structd{set structure} on $\sid$ if & \sssSx\sorel\psetx.\\
  A \structe{set structure} $\sssQx$ is a \structd{paving}  on $\sid$ if & \emptyset\in\sssQx.
  \end{array}
  }
\end{definition}

%---------------------------------------
\begin{definition}
\footnote{
  \citerpgc{pap1995}{8}{0792336585}{Definition 2.3: extended real-valued set function},
  \citerpgc{halmos1950}{30}{0387900888}{\textsection7. {\scshape measure on rings}},
  \citer{hahn1948},
  \citeP{choquet1954}
  }
\label{def:setf}
%---------------------------------------
Let $\sssQx$ be a \structe{paving} \xref{def:paving} on a set $\setX$.
Let $\setY$ be a set containing the element $0$.
\defboxt{
  A function $\fm\in\clF{\sssQx}{\setY}$ is a \fnctd{set function} if 
  \\\indentx$\fm(\emptyset)=0$.
  }
\end{definition}


\begin{figure}
  \centering%
  $\begin{array}{*{4}{c}}
      \includegraphics{graphics/setop_0000.pdf}%
     &\includegraphics{graphics/setop_0011.pdf}%
     &\includegraphics{graphics/setop_0100.pdf}%
     &\includegraphics{graphics/setop_0101.pdf}%
    \\%
      \emptyset
     &\cmpA
     &\setA\setd\setB
     &\cmpB
    \\%
      \includegraphics{graphics/setop_0110.pdf}%
     &\includegraphics{graphics/setop_1000.pdf}%
     &\includegraphics{graphics/setop_1110.pdf}%
     &\includegraphics{graphics/setop_1111.pdf}%
    \\%
      \setA\sets\setB
     &\setA\seti\setB
     &\setA\setu\setB
     &\setX
  \end{array}$
  \caption{Venn diagrams for standard set operations \xref{def:setops} \label{fig:setops}}
\end{figure}
%\pref{def:ss_setops} (next) introduces seven standard set operations: 
%two \prope{nullary} operations, one \prope{unary} operation, and four \structe{binary operation}s\ifsxref{relation}{def:arity}.
%---------------------------------------
\begin{definition}
\citetbl{
  \citerppg{ab}{2}{4}{0120502577}
  }
\label{def:ss_setops}
\label{def:setops}
\index{sets!operations}
%---------------------------------------
Let $\psetx$ be the \structe{power set} \xref{def:pset} on a set $\sid$.
Let $\lnot$ represent the \ope{logical not} operation,
    $\lor$  represent the \ope{logical or} operation,
    $\land$ represent the \ope{logical and} operation\ifsxref{logic}{def:logic}, and
    $\lxor$ represent the \ope{logical exclusive-or} operation\ifsxref{logic}{def:lxor}.
\defbox{%
  \begin{array}{Mcc|l@{\,}c@{\,}l   @{\;}c@{\;}  l @{\,}r@{\,}c@{\,}r |C}
    \mc{2}{N}{name/symbol} & \mc{1}{N|}{arity}   & \mc{8}{N|}{definition} & \mc{1}{N}{domain}
    \\\hline
      \opd{emptyset}             & \hxs{\szero } & 0 &        &     & \szero&\eqd& \big\{x\in\sid\big| & x\ne x           &     &                  \big\} &
    \\\opd{universal set}        & \hxs{\sid   } & 0 &        &     & \sid  &\eqd& \big\{x\in\sid\big| & x=x              &     &                  \big\} &
    \\\opd{complement}           & \hxs{\setopc} & 1 &        &     & \cmpA &\eqd& \big\{x\in\sid\big| & \lnot(x\in\setA) &     &                  \big\} & \forall \setA\in\psetx
    \\\opd{union}                & \hxs{\setu  } & 2 & \setA  &\setu& \setB &\eqd& \big\{x\in\sid\big| &      (x\in\setA) &\lor &      (x\in\setB) \big\} & \forall \setA,\setB\in\psetx
    \\\opd{intersection}         & \hxs{\seti  } & 2 & \setA  &\seti& \setB &\eqd& \big\{x\in\sid\big| &      (x\in\setA) &\land&      (x\in\setB) \big\} & \forall \setA,\setB\in\psetx
    \\\opd{difference}           & \hxs{\setd  } & 2 & \setA  &\setd& \setB &\eqd& \big\{x\in\sid\big| &      (x\in\setA) &\land& \lnot(x\in\setB) \big\} & \forall \setA,\setB\in\psetx
    \\\opd{symmetric difference} & \hxs{\sets  } & 2 & \setA  &\sets& \setB &\eqd& \big\{x\in\sid\big| &      (x\in\setA) &\lxor&      (x\in\setB) \big\} & \forall \setA,\setB\in\psetx
  \end{array}%
  }
\end{definition}

With regards to the standard seven set operations only,
\pref{thm:ss_rel_gg} (next) expresses each of the set operations
in terms of pairs of other operations.
%---------------------------------------
\begin{theorem}
\label{thm:ss_rel_gg}
%\citetbl{
%  \citerpg{vaidyanathaswamy1960}{16}{0486404560}
%  }
%---------------------------------------
%Each of the seven set operations may be expressed in terms of pairs of other set operations as follows:
\thmbox{\begin{array}{r*{6}{cl}}
  \sid&=& \cmp{\szero}
  \\
  \szero
    &=& \cmp{\sid}
     =  \cmp{\brp{\setA\setu\cmpA}}
    &=& \setA\seti\cmpA
    &=& \setA\setd\setA
    &=& \setA\sets\setA
  \\
  \sid
    &=& \setA\setu\cmpA
    &=& \cmp{\brp{\setA\seti\cmpA}}
  \\
  \cmpA
    &=& \sid\setd\setA
    &=& \sid\sets\setA
  \\
  \setA\setu\setB
    &=& \cmp{\brp{\cmpA\seti\cmpB}}
    &=& \brp{\setA\sets\setB}\sets\brp{\setA\seti\setB}
    &=& \brp{\setA\setd\setB}\sets\setB
  \\
  \setA\seti\setB
    &=& \cmp{\brp{\cmpA\setu\cmpB}}
    &=& \brp{\setA\setu\setB}\sets\setA\sets\setB
    &=& \setA\setd\brp{\setA\setd\setB}
  \\
  \setA\setd\setB
    &=& \cmp{\brp{\cmpA\setu\setB}}
    &=& \setA\seti\cmpB
    &=& \brp{\setA\setu\setB}\sets\setB
    &=& \brp{\setA\sets\setB}\seti\setA
  \\
  \setA\sets\setB
    &=& \mc{3}{l}{\brs{\cmp{\brp{\cmpA\setu\setB}}} \setu \brs{\cmp{\brp{\setA\setu\cmpB}}}}
    &=& \mc{3}{l}{\brs{\cmp{\brp{\cmpA\seti\cmpB}}} \seti \cmp{\brp{\setA\seti\setB}}}
  \\&=& \brp{\setA\setd\setB}\setu\brp{\setB\setd\setA}
\end{array}}
\end{theorem}

%---------------------------------------
\begin{definition}
\label{def:subset}
%\label{def:ss_subset}
%---------------------------------------
Let $\ssetS$ be a \structe{set structure} \xref{def:ss} on a set $\sid$.
\defbox{\begin{array}{M}
  The relation $\sorel\in\clR{\ssetS}{\ssetS}$ is defined as
  \\\indentx$ \setA \sorel \setB \qquad\text{if}\qquad x\in\setA \implies x\in\setB \qquad \forall x\in\sid$
\end{array}}
\end{definition}

%---------------------------------------
\begin{theorem}
\citetbl{
  \citerppg{dieudonne1969}{3}{4}{1406727911},
  \citerpg{copson1968}{9}{0521047226}
  }
%\label{cor:ss_fc}
\label{thm:algprop}
\index{algebra of sets}
\index{set structures!algebra of sets}
%---------------------------------------
Let $\algA$ be a \structe{set structure} \xref{def:ss} on a set $\setX$.
\thmboxt{
  %\textbf{If} $\setu$ and $\seti$ are closed in $\topT$ \textbf{then}
  $\algA$ is an \structb{algebra of sets} \quad$\implies$\quad $\forall\setA,\setB,\setC\in\algA$
  %\textbf{If} $\ssetS$ is closed under $\sor$, $\sand$, and $\snot$, \textbf{then} for all $\setA,\setB,\setC\in\ssetS$
  \\\footnotesize
  ${\begin{array}{rcl|rcl|D}
       \setA \setu \setA &=& \setA
     & \setA \seti \setA &=& \setA
     & (\prope{idempotent})
    \\ \setA \setu \setB &=& \setB \setu \setA
     & \setA \seti \setB &=& \setB \seti \setA
     & (\prope{commutative})
    \\ \setA\setu (\setB\setu\setC) &=& (\setA\setu\setB) \setu \setC
     & \setA\seti (\setB\seti\setC) &=& (\setA\seti\setB) \seti \setC
     & (\prope{associative})
    \\ \setA \setu  (\setA \seti \setB) &=& \setA
     & \setA \seti (\setA \setu  \setB) &=& \setA
     & (\prope{absorptive})
    \\ \setA\setu(\setB\seti\setC) &=& (\setA\setu\setB) \seti (\setA\setu\setC)
     & \setA\seti(\setB\setu\setC) &=& (\setA\seti\setB) \setu (\setA\seti\setC)
     & (\prope{distributive})
    \\ \setA \setu \szero      &=& \setA
     & \setA \seti \sid        &=& \setA
     & (\prope{identity})
    \\ \setA \setu \sid        &=& \sid
     & \setA \seti \szero      &=& \szero
     & (\prope{bounded})
    \\ \setA \setu \cmpA         &=& \sid
     & \setA \seti \cmpA         &=& \szero
     & (\prope{complemented})
    \\ \cmpp{\cmpA}          &=& \setA
     &                           &&
     & (\prope{uniquely complemented})
    \\ \cmpp{\setA\setu \setB} &=& \cmpA \seti \cmpB
     & \cmpp{\setA\seti \setB} &=& \cmpA \setu \cmpB
     & (\prope{de Morgan})
    \\\hline
      \mc{3}{H|}{property emphasizing $\setu$}
    & \mc{3}{H|}{dual property emphasizing $\seti$}
    & \mc{1}{H}  {property name}
  \end{array}}$
  }
\end{theorem}
\begin{proof}
\begin{enume}
  \item $\ssetS$ is an \structe{algebra of sets}\ifsxref{setstrct}{def:ss_algebra}.
  \item By the \thme{Stone Representation Theorem}\ifsxref{setstrct}{thm:lat_algebra}, 
        $\booalg{\ssetS}{\sorel}{\sor}{\sand}{\snot}{\szero}{\sid}$ is a \structe{Boolean algebra}.
  \item The properties listed are all properties of \structe{Boolean algebra}s\ifsxref{boolean}{thm:boo_prop}.
\end{enume}
\end{proof}



% example of when possibly useful:
%   A document includes the concept of closure of a set,
%   and in turn includes topology.tex.
%============================================================================
%---------------------------------------
\begin{definition}
\label{def:pset}
\label{def:powerset}
\index{set!power}
%---------------------------------------
\defbox{\begin{array}{M}\indxs{\psetx}
  The \hid{power set} $\psetx$ on a set $\sid$ is defined as
    \\\indentx$\ds\psetx \eqd \set{\setA}{\setA\sorel\sid}$
      \qquad\scriptsize(the set of all subsets of $\sid$)
\end{array}}
\end{definition}

%---------------------------------------
\begin{definition}
\citetbl{
  \citerpg{molchanov2005}{389}{185233892X},
  \citerpg{pap1995}{7}{0792336585},
  \citerpg{hahn1948}{254}{111422295X}
  }
\label{def:ss}
\label{def:paving}
%---------------------------------------
Let $\psetx$ be the \structe{power set} \xref{def:pset} of a set $\setX$.
\defbox{
  \begin{array}{Ml}
  A set $\sssSx$ is a \structd{set structure} on $\sid$ if & \sssSx\sorel\psetx.\\
  A \structe{set structure} $\sssQx$ is a \structd{paving}  on $\sid$ if & \emptyset\in\sssQx.
  \end{array}
  }
\end{definition}

%---------------------------------------
\begin{definition}
\footnote{
  \citerpgc{pap1995}{8}{0792336585}{Definition 2.3: extended real-valued set function},
  \citerpgc{halmos1950}{30}{0387900888}{\textsection7. {\scshape measure on rings}},
  \citer{hahn1948},
  \citeP{choquet1954}
  }
\label{def:setf}
%---------------------------------------
Let $\sssQx$ be a \structe{paving} \xref{def:paving} on a set $\setX$.
Let $\setY$ be a set containing the element $0$.
\defboxt{
  A function $\fm\in\clF{\sssQx}{\setY}$ is a \fnctd{set function} if 
  \\\indentx$\fm(\emptyset)=0$.
  }
\end{definition}


\begin{figure}
  \centering%
  $\begin{array}{*{4}{c}}
      \includegraphics{../common/math/graphics/pdfs/setop_0000.pdf}%
     &\includegraphics{../common/math/graphics/pdfs/setop_0011.pdf}%
     &\includegraphics{../common/math/graphics/pdfs/setop_0100.pdf}%
     &\includegraphics{../common/math/graphics/pdfs/setop_0101.pdf}%
    \\%
      \emptyset
     &\cmpA
     &\setA\setd\setB
     &\cmpB
    \\%
      \includegraphics{../common/math/graphics/pdfs/setop_0110.pdf}%
     &\includegraphics{../common/math/graphics/pdfs/setop_1000.pdf}%
     &\includegraphics{../common/math/graphics/pdfs/setop_1110.pdf}%
     &\includegraphics{../common/math/graphics/pdfs/setop_1111.pdf}%
    \\%
      \setA\sets\setB
     &\setA\seti\setB
     &\setA\setu\setB
     &\setX
  \end{array}$
  \caption{Venn diagrams for standard set operations \xref{def:setops} \label{fig:setops}}
\end{figure}
%\pref{def:ss_setops} (next) introduces seven standard set operations: 
%two \prope{nullary} operations, one \prope{unary} operation, and four \structe{binary operation}s\ifsxref{relation}{def:arity}.
%---------------------------------------
\begin{definition}
\citetbl{
  \citerppg{ab}{2}{4}{0120502577}
  }
\label{def:ss_setops}
\label{def:setops}
\index{sets!operations}
%---------------------------------------
Let $\psetx$ be the \structe{power set} \xref{def:pset} on a set $\sid$.
Let $\lnot$ represent the \ope{logical not} operation,
    $\lor$  represent the \ope{logical or} operation,
    $\land$ represent the \ope{logical and} operation\ifsxref{logic}{def:logic}, and
    $\lxor$ represent the \ope{logical exclusive-or} operation\ifsxref{logic}{def:lxor}.
\defbox{%
  \begin{array}{Mcc|l@{\,}c@{\,}l   @{\;}c@{\;}  l @{\,}r@{\,}c@{\,}r |C}
    \mc{2}{N}{name/symbol} & \mc{1}{N|}{arity}   & \mc{8}{N|}{definition} & \mc{1}{N}{domain}
    \\\hline
      \opd{emptyset}             & \hxs{\szero } & 0 &        &     & \szero&\eqd& \big\{x\in\sid\big| & x\ne x           &     &                  \big\} &
    \\\opd{universal set}        & \hxs{\sid   } & 0 &        &     & \sid  &\eqd& \big\{x\in\sid\big| & x=x              &     &                  \big\} &
    \\\opd{complement}           & \hxs{\setopc} & 1 &        &     & \cmpA &\eqd& \big\{x\in\sid\big| & \lnot(x\in\setA) &     &                  \big\} & \forall \setA\in\psetx
    \\\opd{union}                & \hxs{\setu  } & 2 & \setA  &\setu& \setB &\eqd& \big\{x\in\sid\big| &      (x\in\setA) &\lor &      (x\in\setB) \big\} & \forall \setA,\setB\in\psetx
    \\\opd{intersection}         & \hxs{\seti  } & 2 & \setA  &\seti& \setB &\eqd& \big\{x\in\sid\big| &      (x\in\setA) &\land&      (x\in\setB) \big\} & \forall \setA,\setB\in\psetx
    \\\opd{difference}           & \hxs{\setd  } & 2 & \setA  &\setd& \setB &\eqd& \big\{x\in\sid\big| &      (x\in\setA) &\land& \lnot(x\in\setB) \big\} & \forall \setA,\setB\in\psetx
    \\\opd{symmetric difference} & \hxs{\sets  } & 2 & \setA  &\sets& \setB &\eqd& \big\{x\in\sid\big| &      (x\in\setA) &\lxor&      (x\in\setB) \big\} & \forall \setA,\setB\in\psetx
  \end{array}%
  }
\end{definition}

With regards to the standard seven set operations only,
\pref{thm:ss_rel_gg} (next) expresses each of the set operations
in terms of pairs of other operations.
%---------------------------------------
\begin{theorem}
\label{thm:ss_rel_gg}
%\citetbl{
%  \citerpg{vaidyanathaswamy1960}{16}{0486404560}
%  }
%---------------------------------------
%Each of the seven set operations may be expressed in terms of pairs of other set operations as follows:
\thmbox{\begin{array}{r*{6}{cl}}
  \sid&=& \cmp{\szero}
  \\
  \szero
    &=& \cmp{\sid}
     =  \cmp{\brp{\setA\setu\cmpA}}
    &=& \setA\seti\cmpA
    &=& \setA\setd\setA
    &=& \setA\sets\setA
  \\
  \sid
    &=& \setA\setu\cmpA
    &=& \cmp{\brp{\setA\seti\cmpA}}
  \\
  \cmpA
    &=& \sid\setd\setA
    &=& \sid\sets\setA
  \\
  \setA\setu\setB
    &=& \cmp{\brp{\cmpA\seti\cmpB}}
    &=& \brp{\setA\sets\setB}\sets\brp{\setA\seti\setB}
    &=& \brp{\setA\setd\setB}\sets\setB
  \\
  \setA\seti\setB
    &=& \cmp{\brp{\cmpA\setu\cmpB}}
    &=& \brp{\setA\setu\setB}\sets\setA\sets\setB
    &=& \setA\setd\brp{\setA\setd\setB}
  \\
  \setA\setd\setB
    &=& \cmp{\brp{\cmpA\setu\setB}}
    &=& \setA\seti\cmpB
    &=& \brp{\setA\setu\setB}\sets\setB
    &=& \brp{\setA\sets\setB}\seti\setA
  \\
  \setA\sets\setB
    &=& \mc{3}{l}{\brs{\cmp{\brp{\cmpA\setu\setB}}} \setu \brs{\cmp{\brp{\setA\setu\cmpB}}}}
    &=& \mc{3}{l}{\brs{\cmp{\brp{\cmpA\seti\cmpB}}} \seti \cmp{\brp{\setA\seti\setB}}}
  \\&=& \brp{\setA\setd\setB}\setu\brp{\setB\setd\setA}
\end{array}}
\end{theorem}

%---------------------------------------
\begin{definition}
\label{def:subset}
%\label{def:ss_subset}
%---------------------------------------
Let $\ssetS$ be a \structe{set structure} \xref{def:ss} on a set $\sid$.
\defbox{\begin{array}{M}
  The relation $\sorel\in\clR{\ssetS}{\ssetS}$ is defined as
  \\\indentx$ \setA \sorel \setB \qquad\text{if}\qquad x\in\setA \implies x\in\setB \qquad \forall x\in\sid$
\end{array}}
\end{definition}

%---------------------------------------
\begin{theorem}
\citetbl{
  \citerppg{dieudonne1969}{3}{4}{1406727911},
  \citerpg{copson1968}{9}{0521047226}
  }
%\label{cor:ss_fc}
\label{thm:algprop}
\index{algebra of sets}
\index{set structures!algebra of sets}
%---------------------------------------
Let $\algA$ be a \structe{set structure} \xref{def:ss} on a set $\setX$.
\thmboxt{
  %\textbf{If} $\setu$ and $\seti$ are closed in $\topT$ \textbf{then}
  $\algA$ is an \structb{algebra of sets} \quad$\implies$\quad $\forall\setA,\setB,\setC\in\algA$
  %\textbf{If} $\ssetS$ is closed under $\sor$, $\sand$, and $\snot$, \textbf{then} for all $\setA,\setB,\setC\in\ssetS$
  \\\footnotesize
  ${\begin{array}{rcl|rcl|D}
       \setA \setu \setA &=& \setA
     & \setA \seti \setA &=& \setA
     & (\prope{idempotent})
    \\ \setA \setu \setB &=& \setB \setu \setA
     & \setA \seti \setB &=& \setB \seti \setA
     & (\prope{commutative})
    \\ \setA\setu (\setB\setu\setC) &=& (\setA\setu\setB) \setu \setC
     & \setA\seti (\setB\seti\setC) &=& (\setA\seti\setB) \seti \setC
     & (\prope{associative})
    \\ \setA \setu  (\setA \seti \setB) &=& \setA
     & \setA \seti (\setA \setu  \setB) &=& \setA
     & (\prope{absorptive})
    \\ \setA\setu(\setB\seti\setC) &=& (\setA\setu\setB) \seti (\setA\setu\setC)
     & \setA\seti(\setB\setu\setC) &=& (\setA\seti\setB) \setu (\setA\seti\setC)
     & (\prope{distributive})
    \\ \setA \setu \szero      &=& \setA
     & \setA \seti \sid        &=& \setA
     & (\prope{identity})
    \\ \setA \setu \sid        &=& \sid
     & \setA \seti \szero      &=& \szero
     & (\prope{bounded})
    \\ \setA \setu \cmpA         &=& \sid
     & \setA \seti \cmpA         &=& \szero
     & (\prope{complemented})
    \\ \cmpp{\cmpA}          &=& \setA
     &                           &&
     & (\prope{uniquely complemented})
    \\ \cmpp{\setA\setu \setB} &=& \cmpA \seti \cmpB
     & \cmpp{\setA\seti \setB} &=& \cmpA \setu \cmpB
     & (\prope{de Morgan})
    \\\hline
      \mc{3}{H|}{property emphasizing $\setu$}
    & \mc{3}{H|}{dual property emphasizing $\seti$}
    & \mc{1}{H}  {property name}
  \end{array}}$
  }
\end{theorem}
\begin{proof}
\begin{enume}
  \item $\ssetS$ is an \structe{algebra of sets}\ifsxref{setstrct}{def:ss_algebra}.
  \item By the \thme{Stone Representation Theorem}\ifsxref{setstrct}{thm:lat_algebra}, 
        $\booalg{\ssetS}{\sorel}{\sor}{\sand}{\snot}{\szero}{\sid}$ is a \structe{Boolean algebra}.
  \item The properties listed are all properties of \structe{Boolean algebra}s\ifsxref{boolean}{thm:boo_prop}.
\end{enume}
\end{proof}


