%============================================================================
% LaTeX File
% Daniel J. Greenhoe
%============================================================================

%======================================
\chapter{Source Code}
\label{app:src_code}
\label{app:sourcecode}
\index{source code}
%======================================
%%--------------------------------------
%\section{GNU Octave script file}
%\label{sec:src_Ry}
%%--------------------------------------
%The source code in this section for \hie{GNU Octave}.
%\footnote{ \hie{GNU Octave}: \url{http://www.octave.org/}}
%Octave is similar to \hie{MatLab} with some differences:
%\begin{enume}
%  \item GNU Octave is free.
%  \item GNU Octave is open-source.
%  \item GNU Octave uses a separate graphics engine called \hie{GNU-Plot}
%        for all graphing.
%\end{enume}
%Octave code can easily be adapted to MatLab code and vice-versa.
%
%\lstinputlisting[language=Octave]{../common/wavelets/m/Ry.m}
%
%
%%--------------------------------------
%\section{B-spline polynomial calculation and plotting}
%\label{sec:src_bspline_max}
%%--------------------------------------
%The polynomials for \fncte{B-spline}s as demonstrated in \prefpp{ex:bspline_N2}--\prefpp{ex:Nn}
%can be calculated using the free and open source software package \hie{Maxima}
%along with the following script file:
%\\
%\lstinputlisting[language=MuPAD]{../common/math/maxima/bsplines.max}
%
%\begin{minipage}{\tw-65mm}
%Once the polynomial expressions for a \fncte{B-spline} have been calculated, 
%they can be plotted within a {\LaTeX} environment using the 
%\href{http://www.ctan.org/pkg/pst-plot}{pst-plot package}
%along with a {\LaTeX} source file such as the following:\footnotemark
%\end{minipage}\hfill\tbox{\includegraphics{graphics/n3.pdf}}%
%\footnotetext{
%  For help with PostScript\puttrademark math operators, see 
%  \citerppc{adobe1999}{508}{509}{Arithmetic and Math Operators}.
%  }
%\\
%\lstinputlisting[language=TeX]{../common/math/graphics/bsplines/n3.tex}
%
%\begin{minipage}{\tw-65mm}
%Alternatively, one can plot $\fN_3(x)$ more or less directly from 
%the equation given in \prefpp{thm:Nnx}
%without first calculating the polynomial expressions:
%\end{minipage}\hfill\tbox{\includegraphics{graphics/n3x.pdf}}%
%\\
%\lstinputlisting[language=TeX]{../common/math/graphics/bsplines/n3x.tex}
%
%The {\LaTeX} source listed above can be made to output a pdf file
%with tight borders using the \href{http://www.ctan.org/pkg/preview}{preview package} and\footnote{
%\url{http://tex.stackexchange.com/questions/25400/ps2pdf-depscrop-stops-short-with-pstricks-uput}}
%\begin{lstlisting}
%\usepackage[active,tightpage]{preview}%
%\PreviewBorder=0pt%
%\PreviewEnvironment{pspicture}%
%\end{lstlisting}
%
%%--------------------------------------
%\section{B-spline wavelet polynomial calculation and plotting}
%\label{sec:src_bwav_max}
%%--------------------------------------
%The polynomials for \fncte{B-spline wavelet}s as demonstrated in \prefpp{ex:B0}--\prefpp{ex:B2}
%can be calculated using the free and open source software package \hie{Maxima}
%along with the following script file:
%\\
%\lstinputlisting[language=MuPAD]{../common/math/maxima/bwaves.max}
%
%%--------------------------------------
%\section{Auto-power spectrum plot}
%\label{sec:src_Nriesz}
%%--------------------------------------
%The plot provided in \prefpp{fig:Nriesz} was generated using C and {\LaTeX} code 
%very similar to that displayed in this section:
%\begin{enumerate}
%  \item \label{item:src_Nriesz_C}
%        C code for generating data points. This outputs to \lstinline[language=C]{stdout} 
%        (standard output terminal), but can be redirected into a file such as \lstinline{Snn_bspline1.dat},
%        \lstinline{Snn_bspline2.dat}, \lstinline{Snn_bspline3.dat}, etc.
%        In a \hie{Windows} command environment (``DOS" window), this can be performed with 
%        something like this:
%        \\\indentx\lstinline{C:\ mycprogram.exe > Snn_bspline1.dat}
%\begin{lstlisting}[language=C]
%int bspline_Sdat(void){
%  const double n=3;      /* order of B-spline                             */
%  const long M=1024;     /* approximate number of data points             */
%  const long N=1000000;  /* number of iterations per data point           */
%  double w=0,s=0;
%  long k=0;
%  printf("[\n");                               /* LaTeX \fileplot support */
%  for(w=-8.0;w<=8.0;w+=16./(double)M){
%    s=0;
%    for(k=1; k<=N; k++){
%      s += pow(1.0/(2*k-w/PI),2.*n+2.);        /* 1st summation           */
%      s += pow(1.0/(2*k+w/PI),2.*n+2.);        /* 2nd summation           */
%      }                                                                   
%    s*=pow(sin(w/2.)/(PI/2.),2.*n+2.);         /* scaling factor          */
%    if(w==0) s+=1;                             /* 1st term: w=0 case      */    
%    else     s+=pow(sin(w/2.)/(w/2.),2.*n+2.); /* 1st term: w!=0 case     */
%    printf("(%lf, %lf)\n",w,s);                /* print one data point    */
%    }
%  printf("]\n");                               /* LaTeX \fileplot support */
%  return 0;
%  }
%\end{lstlisting}
%
%  \item \label{item:src_Nriesz_pstplot} 
%        The files produced by \pref{item:src_Nriesz_C} can be plotted within {\LaTeX} 
%        using the \href{http://www.ctan.org/pkg/pst-plot}{pst-plot package}.
%\lstinputlisting[language=TeX]{../common/math/graphics/bsplines/Snn_bspline.tex}
%
%  \item \label{item:src_Nriesz_tex} 
%        The {\LaTeX} source listed in \pref{item:src_Nriesz_pstplot} can be made to output a pdf file
%        with tight borders using the \href{http://www.ctan.org/pkg/preview}{preview package} and\footnote{
%\url{http://tex.stackexchange.com/questions/25400/ps2pdf-depscrop-stops-short-with-pstricks-uput}}
%\begin{lstlisting}
%\usepackage[active,tightpage]{preview}%
%\PreviewBorder=0pt%
%\PreviewEnvironment{pspicture}%
%\end{lstlisting}
%
%\lstinputlisting[language=TeX]{graphics/shell_Snn_bspline.tex}
%
%  \item \label{item:src_Nriesz_shelltop} 
%        The source listed in \pref{item:src_Nriesz_tex} uses a file called 
%        \lstinline{shelltop.tex}:
%
%\lstinputlisting[language=TeX]{graphics/shelltop.tex}
%
%\end{enumerate}