%============================================================================
% Daniel J. Greenhoe
% LaTeX file
%============================================================================
%============================================================================
\chapter{System Identification}
%============================================================================

Let $\opS$ be a \structe{system} with \fncte{impulse response} $\fh[n]$ with
with \ope{DTFT} $\fH(\omega)$,
input $\rvx[n]$, and output $\rvy[n]$. 
Often in the field of ``digital signal processing" (DSP), $\opS$ is a ``filter" 
with known $\fh[n]$ and $\fH(\omega)$ because the filter $\opS$ was 
designed by a designer who had direct control over $\fh[n]$.

However in many other practical situations, $\opS$ is some other system 
for which $\fh[n]$ and $\fH(\omega)$ is \emph{not} known\ldots but which we may
want to \ope{estimate}. Examples of such $\opS$ is a 
device on an industrial shaker table, a communication channel, or the entire earth.

Determining $\fh[n]$ and/or $\fH(\omega)$ is part of a topic called ``\ope{system identication}".
Determining $\fH(\omega)$ in particular is referred to as
``\ope{Frequency Response Identification}"\footnote{\citerpg{shin2008}{292}{0470725648}}
or as 
``\ope{Frequency Response Function}" (``\ope{FRF}") measurement.\footnote{\citePp{cobb1988}{1}}
}
\ope{FRF} measurement is a challenging problem and one that
many people have developed much effort to. 
This chapter describes some of that effort.

In the early days, people used a rather obvious technique for determining $\fH(\omega)$---the simple
sine sweep. That is, they made the input a sine wave with slowly increasing (or decreasing) 
frequency while measuring the resulting output.
This technique, although effective, was ``very slow".\footnote{\citePpc{cobb1988}{1}{Chapter 1---Introduction}}.
\emph{And} there is another problem---we don't always have control over the input signal.
Examples of the latter include live earthquake and volcanic activity analysis.

An alternative to the sine-sweep input is \fncte{random sequence} input. 
All the techniques that follow in this chapter are of this type.
A problem with using random sequences directly for estimating $\fH(\omega)$ is that the 
estimate $\estH(\omega)$ is itself also random.
This is not what we want. We want an estimate that we can actually write down 
on paper or at least plot on paper.

A solution to this is to not use the random sequences directly to estimate $\fH(\omega)$, 
but instead to first use the \ope{expectation} operator $\pE$ \xref{def:pE}.
The expectation operator takes a quantity $\rvX$ that are inherently ``random" 
(with some probability distribution $\pdfp(x)$) and 
turns it into a deterministic ``constant" $\pE\rvX$.

The operator $\pE$ is also used by the spectral density functions 
$\Swxx$ and $\Swxy$ \xref{def:Swxy}.
And $\Swxx$ and $\Swxy$ are what are typically used to calculate $\estH$.

%=======================================
\section{Definitions}
%=======================================
Estimating $\fH$ using random sequences leads to an obvious first estimator 
often referred to as ``$\fH_1$" (next definition).
It is a natural first attempt because in the \emph{very special case} of $\opS$ being 
\prope{linear time invariant} (\prope{LTI}), the input being 
\prope{wide sense stationary}, and measurement noise equal to $0$,
then $\fH_1$ equals the true $\fH$ exactly \xref{cor:Swxy}.
%---------------------------------------
\begin{definition}[\fnctd{Least Squares Technique}]
\footnote{
  \citerpgc{shin2008}{293}{0470725648}{
    $H_1(f)=\ffrac{\Swxy(f)}{\Swxx(f)}$ (9.63);
    which differs from \pref{def:H1}, but see \prefp{rem:Rxym}
    },
  \citerpgc{bendat2011}{185}{1118210824}{$H_1(f) = \ffrac{G_{xy}(f)}{G_{xx}(f)}$ (6.37)},
  \citeP{bendat1978}{cited by Cobb(1988)---variance estimate for $\estH_1$},
  \citePc{allemang1979}{cited by Shin(2008)},
  \citePpc{leuridan1986}{910}{\fncte{Least Squares Technique}; (8) $[G_{xx}](H)=[G_{xy}]$},
  \citeP{abom1986}{cited by Cobb(1988)---variance estimate for $\estH_1$},
  \citePppc{allemang1987}{54}{55}{5.3.1 $H_1$ Technique; $[H]=[G_{XF}][G_{FF}]^{-1}$ (11)},
  \citePpc{cobb1988}{2}{$^1\hat{H}(f)=\ffrac{\hat{G}_{yx}(f)}{\hat{G}_{xx}(f)}$ (1)},
  \citePpc{goyder1984}{438}{$H(i\omega)=\ffrac{S_{qp}}{S_{pp}}$ (3)}
  }
\label{def:H1}
%---------------------------------------
\defbox{\begin{array}{Mrc>{\ds}l}
  The transfer function $\estH_1(\omega)$ is defined as & \estH_1(\omega) &\eqd& \frac{\Swyx(\omega)}{\Swxx(\omega)}
\end{array}}
\end{definition}

However, in practical situations, there is measurement error. An example may include 
``road noise" from a test being performed in a moving vehicle.
If the measurement error is at the output only (and under the assumptions of \prope{LTI} and \prope{WSS})
then $\estH_1$ is the ideal estimator in the sense that $\estH_1=\fH$ \xref{prop:estH1}.
However, in the early 1980s Mitchell pointed out that in the presence of input noise,
$\estH1$ is far from ideal in the sense that it is \prope{biased};
in fact, it \prope{under estimates} $\fH$ \xref{prop:estH2}.
Mitchell proposed a new estimator $\estH_2$ (next definition).
This estimator $\estH_2$ has the special property that when there is input noise but no 
output noise (and under \prope{LTI}, \prope{WSS}, and \prope{uncorrelated} assumptions), 
then it is ideal in the sense that $\estH_2=\fH$ \prefp{prop:estH2}.
%---------------------------------------
\begin{definition}[\fnctd{Inverse Method}]
\footnote{
  \citerpgc{shin2008}{293}{0470725648}{$H_2(f)=\ffrac{\Swyy(f)}{\Swyx(f)}$ (9.65);
    which differs from \pref{def:H2}, but see \prefp{rem:Rxym}
    },
  \citerpgc{bendat2011}{186}{1118210824}{$H_2(f) = \ffrac{G_{yy}(f)}{G_{yx}(f)}$ (6.42)},
  \citePc{mitchell1980}{cited by Cobb(1988)},
  \citePc{mitchell1982}{278}{``Define what will be called an inverse method for calculation of a FRF as\ldots"; 
    $\~H_2(f)=\ffrac{G_{yy}}{G_{yx}}$ (6); 
    Note this differs with \pref{def:H2} by a conjugate, but note that Mitchell seems to follow Bendat (see his [3] and [4]), 
    which would explain this difference \xref{rem:Rxym}},
  \citePpc{cobb1988}{3}{$^2\hat{H}(f)=\ffrac{\hat{G}_{yy}(f)}{\hat{G}_{xy}(f)}$ (1)}
  }
\label{def:H2}
%---------------------------------------
\defbox{\begin{array}{Mrc>{\ds}l}
  The transfer function $\estH_2(\omega)$ is defined as & \estH_2(\omega) &\eqd& \frac{\Swyy(\omega)}{\Swxy(\omega)}
\end{array}}
\end{definition}

Mitchell's $\estH_2$ contribution ``generated a flurry of activity"\footnote{\citePp{cobb1988}{3}}
and soon more $\fH$ estimators appeared. 
As it turns out, in the presence of \emph{both} input and output measurement noise,
the traditional estimator $\estH_1$, although ideal when input noise = 0, tends to \prope{under estimate} $\fH$ and 
Mitchell's $\estH_2$, although ideal when output noise = 0, 
tends to \prope{over estimate} $\fH$ \xxref{prop:estH1}{prop:estH2}.
What is needed is an estimator with a parameter that can be adjusted 
depending on the ratio of output and input noise.
Wicks and Vold in 1986 introduced such an estimator, often called $\estHs$, with a
parameter $s$ in the range $\intcc{0}{\infty}$.
And as it turns out, $\estH_s = \estH_1$ when $s=0$ and 
$\estH_s\to\estH_2$ as $s\to\infty$ \xref{thm:Hs}.
%---------------------------------------
\begin{definition}[\fnctd{scaling estimate}]
\footnote{
  \citePc{leclere2012}{(10) with $x$ and $y$ swapped}
  }
\label{def:Hs}
%---------------------------------------
\defboxt{
  The estimate $\estHs(\omega)$ with \vald{scaling parameter} $s\in\intco{0}{\infty}$ is defined as 
  \\\indentx$\ds
  \estHs(\omega) \eqd 
  \frac{2\Swyx(\omega)}
       {\Swxx(\omega)-s^2\Swyy(\omega) + \sqrt{\brs{\Swxx(\omega)-s^2\Swyy(\omega)}^2 + 4s^2\abs{\Swxy(\omega)}^2}}
  $
  }
\end{definition}


%---------------------------------------
\begin{definition}
\footnote{
  \citerpgc{shin2008}{293}{0470725648}{(9.67)},
  \citeP{white2006},
  \citePc{leclere2012}{(10) $\kappa(f)=1/s^2$ and $x$ and $y$ swapped}
  }
\label{def:Hw}
%---------------------------------------
\defboxt{
  The transfer function $\estHw(\omega)$ is defined as 
  \\\indentx$\ds
  \estHw(\omega) \eqd \frac{\Swyy(\omega)-\kappa(\omega)\Swxx(\omega) + \sqrt{\brs{\Swyy(\omega) - \kappa(\omega)\Swxx(\omega)}^2 + 4\kappa(\omega)\abs{\Swxy(\omega)}^2}}
                            {2\Swxy(\omega)}
  $
  }
\end{definition}

%---------------------------------------
\begin{definition}
\footnote{
  \citerpgc{shin2008}{293}{0470725648}{(9.67)},
  \citeP{white2006},
  \citePc{leclere2012}{(10) $\kappa(f)=1/s^2$ and $x$ and $y$ swapped}
  }
\label{def:Hw}
%---------------------------------------
\defboxt{
  The transfer function $\estHw(\omega)$ is defined as 
  \\\indentx$\ds
  \estHw(\omega) \eqd 
  \frac{2\kappa(\omega)\Swyx(\omega)}
       {\kappa(\omega)\Swxx(\omega)-\Swyy(\omega) + \sqrt{\brs{\kappa(\omega)\Swxx(\omega)-\Swyy(\omega)}^2 + 4\kappa(\omega)\abs{\Swxy(\omega)}^2}}
  $
  }
\end{definition}

%---------------------------------------
\begin{definition}[Three channel estimate]
\footnote{
  \citerpgc{shin2008}{297}{0470725648}{$H_3(f)=\ffrac{S_{ry}(f)}{S_{rx}(f)}$ (9.78)},
  \citePpc{cobb1988}{4}{$^c\hat{H}(f)=\ffrac{\hat{G}_{ys}(f)}{\hat{G}_{xs}(f)}$ (1.4)},
  \citePpc{goyder1984}{440}{$H(i\omega)=\ffrac{S_{qz}}{S_{pz}}$ (5)},
  \citePpc{cobb1990}{450}{(1)}
  }
\label{def:Hc}
%---------------------------------------
Let $\opS$ be a system with input $\rvx[n]$, output $\rvy[n]$, and a driver $\rvp[n]$.
\defboxt{
  The transfer function $\estHc(\omega)$ is defined as 
  \\\indentx$\ds
  \estHc(\omega) \eqd \frac{\Swxx[py](\omega)}{\Swxx[px](\omega)}
  $
  }
\end{definition}



%=======================================
\section{Properties}
%=======================================
\begin{figure}
  \centering%
  \includegraphics{graphics/xhyplusv.pdf}
  \caption{System model with additive input and output noise\label{fig:xhyplusv}}
\end{figure}
%---------------------------------------
\begin{proposition}[Output noise only case]
\footnote{
  \citerpgc{shin2008}{294}{0470725648}{$H_1(f)=H(f)$ (9.70); $H_2(f)=H(f)\brp{1+\ffrac{S_{n_yn_y}(f)}{Syy(f)}}$ (9.71)},
  \citePc{mitchell1982}{278}{$\~H_2(f)=H_0(f)\brp{1+\ffrac{G_{mm}}{G_{vv}}}$}
  }
\label{prop:estH1}
%---------------------------------------
Let $\opS$ be a \structe{system} with \fncte{impulse response} $\fh[n]$,
input $\fx[n]$, and output $\fy[n]$.
Let $\rvu[n]$ and $\rvv[n]$ be \fncte{noise sequence}.
Let $\rvx'[n]\eqd\rvx[n]+\rvu[n]$ and $\rvy'[n]\eqd\rvy[n]+\rvv[n]$.
\propbox{
  \brbr{\begin{array}{FMMD}
      (A).& The system $\fh$ is         &\prope{LTI}          & and
    \\(B).& $\rvx[n]$ and $\rvy[n]$ are &\prope{WSS}          & and
    \\(C).& $\rvx[n]$ and $\rvv[n]$ are &                     &     
        \\& \mc{2}{M}{\quad \prope{uncorrelated}}             & and
    \\(D).& $\rvu[n]=0$                 
  \end{array}}
  \implies
  \brbl{\begin{array}{F>{\ds}rc>{\ds}lDD}
      (1).&\estH_1(\omega) &=& \fH(\omega)                                              & (\prope{unbiased}) & and
    \\(2).&\estH_2(\omega) &=& \fH(\omega)\brs{1 + \frac{\Swvv(\omega)}{\Swyy(\omega)}} & \mc{2}{D}{\begin{tabular}{l}(\prope{biased})\\(\prope{over estimated})\end{tabular}}
  \end{array}}
  }
\end{proposition}
\begin{proof}
\begin{align*}
  \estH_1(\omega)
    &\eqd \frac{\Swxx[y'x](\omega)}{\Swxx(\omega)}
    && \text{by definition of $\estH_1$}
    && \text{\xref{def:H1}}
  \\&=    \frac{\Swyx(\omega)}{\Swxx(\omega)}
    && \text{by \prefp{prop:Swxyv}}
  \\&= \fH^\ast(\omega)
    && \text{by \prope{LTI} hypothesis (A)}
    && \text{and \prefp{cor:Swxy}}
  \\
  \estH_2(\omega)
    &\eqd \frac{\Swxx[y'y'](\omega)}{\Swxy(\omega)}
    && \text{by definition of $\estH_2$}
    && \text{\xref{def:H2}}
  \\&=    \frac{\Swyy(\omega) + \Swvv(\omega)}{\Swxy(\omega)}
    && \text{by \prefp{prop:Swxyv}}
  \\&= \frac{\Swyy(\omega)}{\Swxy(\omega)}
     + \frac{\Swvv(\omega)}{\Swxy(\omega)}
  \\&= \fH(\omega)
     + \frac{\Swvv(\omega)}{\frac{\Swyy(\omega)}{\fH(\omega)}}
    && \text{by \prefp{cor:Swxy}}
  \\&= \fH(\omega)\brs{1 + \frac{\Swvv(\omega)}{\Swyy(\omega)}}
\end{align*}
\end{proof}

%---------------------------------------
\begin{proposition}[Input noise only case]
\footnote{
  \citerpgc{shin2008}{294}{0470725648}{$H_1(f)=\ffrac{H(f)}{\brp{1+\ffrac{S_{n_xn_x}}{S_{xx}(f)}}}$ (9.72); $H_2(f)=H(f)$ (9.73)},
  \citePc{mitchell1982}{277}{$\~H_1(f)=\ffrac{H_0(f)}{\brp{1+\ffrac{G_{nn}}{G_{uu}}}}$}
  }
\label{prop:estH2}
%---------------------------------------
Let $\opS$ be a \structe{system} with \fncte{impulse response} $\fh[n]$,
input $\fx[n]$, and output $\fy[n]$.
Let $\rvu[n]$ and $\rvv[n]$ be \fncte{noise sequence}.
Let $\rvx'[n]\eqd\rvx[n]+\rvu[n]$ and $\rvy'[n]\eqd\rvy[n]+\rvv[n]$.
\propbox{
  \brbr{\begin{array}{FMMD}
      (A).& The system $\fh$ i s        &\prope{LTI}          & and
    \\(B).& $\rvx[n]$ and $\rvy[n]$ are &\prope{WSS}          & and
    \\(C).& $\rvy[n]$ and $\rvu[n]$ are &                     &     
        \\& \mc{2}{M}{\quad              \prope{uncorrelated}}& and
    \\(D).& $\rvv[n]=0$                 
  \end{array}}
  \implies
  \brbl{\begin{array}{F>{\ds}rc>{\ds}lDD}
      (1).&\estH_1(\omega) &=& \frac{\fH(\omega)}{1 + \frac{\ds\Swuu(\omega)}{\ds\Swxx(\omega)}} & \begin{tabular}{l}(\prope{biased})\\(\prope{under estimated})\end{tabular} &and
    \\(2).&\estH_2(\omega) &=& \fH(\omega)                                                 & (\prope{unbiased}) &
  \end{array}}
  }
\end{proposition}
\begin{proof}
\begin{align*}
  \\
  \estH_1(\omega)
    &\eqd \frac{\Swxx[yx'](\omega)}{\Swxx[x'x'](\omega)}
    && \text{by definition of $\estH_1$}
    && \text{\xref{def:H1}}
  \\&= \frac{\Swyx(\omega) + \Swyu(\omega)}
            {\Swxx(\omega) + \Swuu(\omega)}
    && \text{by \prefp{prop:Swxyv}}
  \\&= \frac{\Swyx(\omega) + \cancelto{0}{\Swyu(\omega)}}
            {\Swxx(\omega) + \Swuu(\omega)}
    && \text{because $\rvy[n]$ and $\rvu[n]$ are \prope{uncorrelated}}
    && \text{(hypothesis (C))}
  \\&= \frac{\frac{\Swyx(\omega)}{\Swxx(\omega)}}
            {\frac{\Swxx(\omega)}{\Swxx(\omega)} + \frac{\Swuu(\omega)}{\Swxx(\omega)}}
   &&= \frac{\fH(\omega)}
            {1 + \frac{\Swuu(\omega)}{\Swxx(\omega)}}
  \\
  \estH_2(\omega)
    &\eqd \frac{\Swyy(\omega)}{\Swxx[x'y](\omega)}
    && \text{by definition of $\estH_2$}
    && \text{\xref{def:H2}}
  \\&= \frac{\Swyy(\omega)}{\Swxy(\omega) + \Swuy(\omega)}
    && \text{by \prefp{prop:Swxyv}}
  \\&= \frac{\Swyy(\omega)}{\Swxy(\omega) + \cancelto{0}{\Swuy(\omega)}}
    && \text{because $\rvy[n]$ and $\rvu[n]$ are \prope{uncorrelated}}
    && \text{(hypothesis (C))}
  \\&= \fH(\omega)
    && \text{by \prefp{cor:Swxy}}
    && \text{and \prope{LTI} hypothesis (A)}
\end{align*}
\end{proof}

%=======================================
\section{Beware of estimators}
%=======================================
Estimators yield, as the name implies, estimates.
These estimates in general contain some error.

%---------------------------------------
\begin{example}[The K=1 Welch estimate of coherence]
%---------------------------------------
Suppose we have two \prope{uncorrelated} stationary sequences $\rvx[n]$ and $\rvy[n]$. Then, there
CSD $\Sxy(\omega)$ should be $0$ because
\begin{align*}
  \Sxy(\omega)
    &\eqd \opDTFT\pE\Rxy(m)
  \\&=    \opDTFT\pE\brs{x[n]y[n+m]}
  \\&=    \opDTFT\brs{\pEx[n]} \brs{\pEy[n+m]}
  \\&=    \opDTFT\brs{0} \brs{0}
  \\&=    0
\end{align*}

This will give a coherence of $0$ also:
\[ C(\omega) = \frac{\Sxy}{\sqrt{\Sxx\Syy}} = 0\]

However, the Welch estimate with $K=1$ will yield
\begin{align*}
  \abs{C(\omega)}
    &= \abs{\frac{\ds\Sxy}{\sqrt{\ds\Sxx\Syy}}}
  \\&= \abs{\frac{\ds (\opFT x)(\opFT y)^\ast}{\sqrt{\ds\abs{\opFT x}^2\abs{\opFT y}^2}}}
  \\&= 1
\end{align*}

\end{example}
