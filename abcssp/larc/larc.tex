%============================================================================
% Daniel J. Greenhoe
% LaTeX file
%============================================================================
%=======================================
%\section{Lagrange arc distance}
%=======================================
%=======================================
\section{Definition}
%=======================================
%---------------------------------------
\begin{definition} %[\exmd{Lagrange arc distance in $\R^\xN$}]
\label{def:larc}
%---------------------------------------
Let $p\eqd\oquad{x_1}{x_2}{\cdots}{x_\xN}$ 
and $q\eqd\oquad{y_1}{y_2}{\cdots}{y_\xN}$ 
be two points in the \structe{space} $\R^\xN$ with origin $(0,0,\cdots,0)$.
Let
%Let $\r_p$ be the \fncte{magnitude} of $p$,
%and $\r_q$    the \fncte{magnitude} of $q$,such that
%\\\indentx$\ds
%  r_p \eqd \brp{\sum_{n=1}^\xN x_n^2}^{\frac{1}{2}}
%  \qquad\text{and}\qquad
%  r_q \eqd \brp{\sum_{n=1}^\xN y_n^2}^{\frac{1}{2}}
%  \qquad\text{and}\qquad
%  $\\ 
\\\indentx$\begin{array}[t]{>{\ds}c@{\qquad}>{\ds}c@{\qquad}>{\ds}c}
        \underbrace{r_p    \eqd \brp{\sum_{n=1}^\xN x_n^2}^\frac{1}{2}             }   
       &\underbrace{r_q    \eqd \brp{\sum_{n=1}^\xN y_n^2}^\frac{1}{2}             }
       &\underbrace{\phi   \eqd \arccos\brp{\frac{1}{r_pr_q}\sum_{n=1}^\xN x_n y_n}}
      %\\\rho   &\eqd& r_q-r_p
      % &\gamma &\eqd& r_p\phi
      \\$\scs(magnitude of $p$)$
       &$\scs(magnitude of $q$)$
       &$\scs(angle between $p$ and $q$)$
    \end{array}$
\\
\indentx$\ds\begin{array}{>{\ds}c@{\qquad}>{\ds}c} 
  \underbrace{\fr(\theta) \eqd \brp{\frac{r_q-r_p}{\phi}}\theta + r_p} &
  \underbrace{\fR(p,q) = \int_0^\phi \sqrt{r^2(\theta) + \brp{\frac{\dr}{\dth}}^2}\dth}
  \\
  $\scs(polar interpolation polynomial)$ &
  $\scs(length of the arc $\opair{\fr(\theta)}{\theta}$ between $p$ and $q$)$
\end{array}$
\\
\defboxp{
  The \fnctd{Langrange arc distance} $\metric{p}{q}$ is defined as
  \\\indentx$
  \metric{p}{q} = \brb{\begin{array}{lM}
    \sfrac{1}{\pi}\abs{r_p-r_q}    & if $p=(0,0,\cdots,0)$ \emph{or} $q=(0,0,\cdots,0)$ \emph{or} $\phi=0$\\
    \sfrac{1}{\pi}\fR(p,q)         & otherwise
  \end{array}}%
  \indentx
  \scy\forall p,q\in\R^\xN
  $}%
\end{definition}

%=======================================
\section{Calculation}
%=======================================
The integral in \pref{def:larc} may look intimidating.
However, \prefpp{thm:Rpq} demonstrates that it has an ``easily" computable and straightforward solution 
only involving \ope{arithmatic operator}s ($+$, $-$,\ldots), 
        the \fncte{absolute value} function $\abs{x}$, 
        the \fncte{square root} function $\sqrt{x}$, and
        the \fncte{natural log} function $\ln(x)$.
But first, a lemma (next) to help with the proof of \pref{thm:Rpq}.

%---------------------------------------
\begin{lemma}
\footnote{
  \citerpgc{gradshteyn2007}{94}{0080471110}{2.25 Forms containing $\sqrt{a+bx+cx^2}$, 2.26 Forms containing $\sqrt{a+bx+cx^2}$ and integral powers of $x$},
  \citerpgc{jeffrey1995}{160}{1483295141}{4.3.4 Integrands containing $(a+bx+cx^2)^{\sfrac{1}{2}}$},
  \citerppgc{jeffrey2008}{172}{173}{0080556841}{4.3.4 Integrands containing $(a+bx+cx^2)^{\sfrac{1}{2}}$}
  %\citePc{shapiro2015}{(37)}
  }
\label{lem:abc}
%---------------------------------------
Let $\sqrt{x}\in\clFrr$ be the \fncte{square root} function,
and $\ln(x)\eqd\log_e(x)\in\clFrr$ be the \fncte{natural log} function.
Let $\varepsilon$ be any given value in $\R$.
\lembox{\begin{array}{>{\ds}l}
  \brb{2ax+b+2\sqrt{a(ax^2+bx+c)}>0}\quad\implies
  \\\brb{
  \int \sqrt{ax^2 + bx + c } \dx
      = \frac{2ax+b}{4a}\sqrt{ax^2+bx+c} 
      + \frac{4ac-b^2}{8a^{3/2}}\ln\brp{2ax+b+2\sqrt{a(ax^2+bx+c)}}
      + \varepsilon
  }
  \end{array}}
\end{lemma}
\begin{proof}
\begin{enumerate}
  \item lemma: \label{item:abc_term1} 
        (first equality by the \thme{product rule}, and the second equality by the \thme{chain rule})
    \begin{align*}
      \opddx\brs{\brp{\frac{2ax+b}{4a}}\sqrt{ax^2+bx+c}} 
        &= \brp{      \frac{2ax+b}{4a}}\brp{\opddx\sqrt{ax^2+bx+c}} +
           \brp{\opddx\frac{2ax+b}{4a}}\brp{      \sqrt{ax^2+bx+c}} 
       %&& \text{by \thme{prod. rule}}
      \\&= \brp{\frac{2ax+b}{4a}} \brp{\frac{2ax+b}{2\sqrt{ax^2+bx+c}}} 
         + \brp{\frac{2a}   {4a}} \brp{\sqrt{ax^2+bx+c}}
       %&& \text{by \thme{chain rule}}
      \\&= \frac{(2ax+b)^2 + 4a(ax^2+bx+c)}
                {8a\sqrt{ax^2+bx+c}}
      \\&= \frac{4a^2x^2 +4axb + b^2 + 4a^2x^2 + 4abx + 4ac}
                {8a\sqrt{ax^2+bx+c}}
      \\&= \frac{8a^2x^2 +8abx + b^2 + 4ac}
                {8a\sqrt{ax^2+bx+c}}
    \end{align*}

  \item lemma: \label{item:abc_lnpos} If $\brp{2ax+b+2\sqrt{a(ax^2+bx+c)}}>0$ then
    \begin{align*}
      &\opddx\brs{\frac{4ac-b^2}{8a^{3/2}}\ln\brp{2ax+b+2\sqrt{a(ax^2+bx+c)}}}
      \\&= \brp{\frac{4ac-b^2}{8a^{3/2}}}
           \brs{\opddx\ln\brp{2ax+b+2\sqrt{a(ax^2+bx+c)}}}
        && \text{by \prope{linearity} of $\opddx$}
      \\&= \brp{\frac{4ac-b^2}{8a^{3/2}}}
           \brp{\frac{1}{2ax+b+2\sqrt{a(ax^2+bx+c)}}}
           \brp{2a+\frac{2\sqrt{a}(2ax+b)}{2\sqrt{ax^2+bx+c}}}
        && \text{by \thme{chain rule}}
      \\&= \frac{(4ac-b^2)\brs{2a\sqrt{ax^2+bx+c} + \sqrt{a}(2ax+b)}}
                {8a^{3/2} \brp{2ax+b+2\sqrt{a(ax^2+bx+c)}} \sqrt{ax^2+bx+c}}
      \\&= \frac{(4ac-b^2)\brs{\sqrt{a}(2ax+b)+2a\sqrt{ax^2+bx+c}}}
                {\brp{8a\sqrt{ax^2+bx+c}}\brs{\sqrt{a}(2ax+b)+2a\sqrt{ax^2+bx+c}} }
      \\&= \frac{4ac-b^2}
                {8a\sqrt{ax^2+bx+c}}  % Praise the Lord!!!
    \end{align*}

  \item lemma: \label{item:abc_lnneg} If $\brp{2ax+b+2\sqrt{a(ax^2+bx+c)}}<0$ then
    \begin{align*}
      &\opddx\brs{\frac{4ac-b^2}{8a^{3/2}}\ln\abs{2ax+b+2\sqrt{a(ax^2+bx+c)}}}
      \\&= \brp{\frac{4ac-b^2}{8a^{3/2}}}
           \brs{\opddx\ln\brp{-2ax-b-2\sqrt{a(ax^2+bx+c)}}}
        && \text{by \prope{linearity} of $\opddx$}
      \\&= \brp{\frac{4ac-b^2}{8a^{3/2}}}
           \brp{\frac{1}{-2ax-b-2\sqrt{a(ax^2+bx+c)}}}
           \brp{2a+\frac{2\sqrt{a}(2ax+b)}{2\sqrt{ax^2+bx+c}}}
        && \text{by \thme{chain rule}}
      \\&= \frac{(4ac-b^2)\brs{2a\sqrt{ax^2+bx+c} + \sqrt{a}(2ax+b)}}
                {8a^{3/2} \brp{-2ax-b-2\sqrt{a(ax^2+bx+c)}} \sqrt{ax^2+bx+c}}
      \\&= \frac{-(4ac-b^2)\brs{\sqrt{a}(2ax+b)+2a\sqrt{ax^2+bx+c}}}
                {\brp{8a\sqrt{ax^2+bx+c}}\brs{\sqrt{a}(2ax+b)+2a\sqrt{ax^2+bx+c}} }
      \\&= \frac{-(4ac-b^2)}
                {8a\sqrt{ax^2+bx+c}} 
    \end{align*}

  \item Complete the proof: \label{item:abc_pos} If $\brp{2ax+b+2\sqrt{a(ax^2+bx+c)}}>0$ then
    \begin{align*}
        &\opddx\brs{\frac{2ax+b}{4a}\sqrt{ax^2+bx+c} 
          + \frac{4ac-b^2}{8a^{3/2}}\ln\brp{2ax+b+2\sqrt{a(ax^2+bx+c)}}}
      \\&=\opddx\brs{\brp{\frac{2ax+b}{4a}}\sqrt{ax^2+bx+c}} 
         +\opddx\brs{\brp{\frac{4ac-b^2}{8a^{3/2}}}\ln\brp{2ax+b+2\sqrt{a(ax^2+bx+c)}}}
      \\&= \frac{8a^2x^2 +8abx + b^2 + 4ac}{8a\sqrt{ax^2+bx+c}}
         + \opddx\brs{\brp{\frac{4ac-b^2}{8a^{3/2}}}\ln\brp{2ax+b+2\sqrt{a(ax^2+bx+c)}}}
        && \text{by \pref{item:abc_term1}}
      \\&= \frac{8a^2x^2 +8abx + b^2 + 4ac}{8a\sqrt{ax^2+bx+c}}
         + \frac{4ac-b^2}{8a\sqrt{ax^2+bx+c}} 
        && \text{by \pref{item:abc_lnpos}}
      \\&= \frac{8a(ax^2 + bx + c)}{8a\sqrt{ax^2+bx+c}}
         = \sqrt{ax^2+bx+c}
    \end{align*}

  \item Note that simply forcing the agruement of $\ln$ to be positive as in\footnote{
          %\citerpgc{gradshteyn2007}{94}{0080471110}{2.25 Forms containing $\sqrt{a+bx+cx^2}$, 2.26 Forms containing $\sqrt{a+bx+cx^2}$ and integral powers of $x$}
          The solution $\ln\abs{\cdots}$ is used in 
          \citerpg{jeffrey1995}{160}{1483295141} and %{4.3.4 Integrands containing $(a+bx+cx^2)^{\sfrac{1}{2}}$},
          \citerppg{jeffrey2008}{172}{173}{0080556841}. %{4.3.4 Integrands containing $(a+bx+cx^2)^{\sfrac{1}{2}}$},
          } 
      \\\indentx$\frac{2ax+b}{4a}\sqrt{ax^2+bx+c} + \frac{4ac-b^2}{8a^{3/2}}\ln\abs{2ax+b+2\sqrt{a(ax^2+bx+c)}} + \varepsilon$\\
      is \emph{not} a solution to $\int\sqrt{ax^2+bx+c}\dx$ when
      $\brp{2ax+b+2\sqrt{a(ax^2+bx+c)}}<0$:
    \begin{align*}
        &\opddx\brs{\frac{2ax+b}{4a}\sqrt{ax^2+bx+c} 
          + \frac{4ac-b^2}{8a^{3/2}}\ln\abs{2ax+b+2\sqrt{a(ax^2+bx+c)}}}
      \\&=\opddx\brs{\brp{\frac{2ax+b}{4a}}\sqrt{ax^2+bx+c}} 
         +\opddx\brs{\brp{\frac{4ac-b^2}{8a^{3/2}}}\ln\abs{2ax+b+2\sqrt{a(ax^2+bx+c)}}}
      \\&= \frac{8a^2x^2 +8abx + b^2 + 4ac}{8a\sqrt{ax^2+bx+c}}
         + \opddx\brs{\brp{\frac{4ac-b^2}{8a^{3/2}}}\ln\abs{2ax+b+2\sqrt{a(ax^2+bx+c)}}}
        && \text{by \pref{item:abc_term1}}
      \\&= \frac{8a^2x^2 +8abx + b^2 + 4ac}{8a\sqrt{ax^2+bx+c}}
         - \frac{4ac-b^2}{8a\sqrt{ax^2+bx+c}} 
        && \text{by \pref{item:abc_lnneg}}
      \\&= \frac{8a(ax^2 +bx) + 2b^2}{8a\sqrt{ax^2+bx+c}}
         = \frac{8a(ax^2 +bx + c) + 2b^2 -8ac}{8a\sqrt{ax^2+bx+c}}
      \\&= \sqrt{ax^2 +bx + c} + \frac{2b^2 -8ac}{8a\sqrt{ax^2+bx+c}}
         \neq \sqrt{ax^2+bx+c}\quad\text{for $b^2\neq 4ac$}
    \end{align*}

  \item Note further that constraining $a>0$ is also not a solution\footnote{
        The $a>0$ constraint is used in 
        \citerpg{gradshteyn2007}{94}{0080471110} %{2.25 Forms containing $\sqrt{a+bx+cx^2}$, 2.26 Forms containing $\sqrt{a+bx+cx^2}$ and integral powers of $x$}
        }
        because it does not guarantee that the argument $u$ of $\ln(u)$ will be positive. 
        %Take for example $a\to0^+$, $b=-1$, $c=1$ and $x=1$. Then
        %\\\indentx$\lim2ax+b+2\sqrt{a(ax^2+bx+c)} = 2\cdot0\cdot1 -1 + 2\sqrt{0(0\cdot1-1\cdot1+1)} = -1 < 0$ .
        Take for example $a=1$, $b=-3$, $c=3$ and $x=1$. Then
        \\\indentx$2ax+b+2\sqrt{a(ax^2+bx+c)} = 2\cdot1\cdot1 -3 + 2\sqrt{1(1\cdot1-3\cdot1+3)} = -1 < 0$ .
\end{enumerate}
\end{proof}

%---------------------------------------
\begin{theorem}
%\label{prop:larc_R}
\label{thm:Rpq}
%---------------------------------------
Let $\fR(p,q)$, $r_p$, $r_q$, and $\phi$ be as defined in \prefpp{def:larc}.
Let $\rho\eqd r_q-r_p$.\\
If $r_p\neq0$, $r_q\neq0$ and $\phi\neq0$ then 
\\\thmboxt{$\begin{array}{>{\ds}l}
    \fR(p,q)=
           \frac{r_q\sqrt{(r_q\phi)^2+\rho^2}-r_p\sqrt{(r_p\phi)^2+\rho^2}}
                {2\rho} 
           \\\qquad+\frac{\abs{\rho}}{2\phi}\ln\brp{r_q\rho\phi+\abs{\rho}\sqrt{(r_q\phi)^2+\rho^2}}
           \\\qquad-\frac{\abs{\rho}}{2\phi}\ln\brp{r_p\rho\phi+\abs{\rho}\sqrt{(r_p\phi)^2+\rho^2}}
  \end{array}$}
\end{theorem}
\begin{proof}
\begin{enumerate}
  \item Let $\gamma \eqd r_p\phi$.
  \item lemmas: \label{item:larcR_lem}
    \begin{align*}
      \rho\phi+\gamma &= (r_q-r_p)\phi + r_p\phi = r_q\phi
      \\
      \rho^2\phi^2+2\rho\gamma\phi+(\gamma^2+\rho^2)
        &= (r_q-r_p)^2\phi^2 + 2(r_q-r_p)(r_p\phi)\phi+(r_p\phi)^2+\rho^2
      \\&= (r_p^2+r_q^2-2r_pr_q)\phi^2 + 2(r_q-r_p)(r_p\phi)\phi+(r_p\phi)^2+\rho^2
      \\&= (r_p^2+r_q^2)\phi^2 - 2(r_p\phi)^2 +(r_p\phi)^2+\rho^2
      \\&= (r_q\phi)^2+\rho^2
    \end{align*}

  \item lemma: \label{item:larcR_lnarg}
      $2\rho^2\theta+2\rho\gamma+2\sqrt{\rho^2(\rho^2\theta^2+2\rho\gamma\theta+(\gamma^2+\rho^2))} > 0$. Proof:
    \begin{align*}
      &2\rho^2\theta+2\rho\gamma+2\sqrt{\rho^2(\rho^2\theta^2+2\rho\gamma\theta+(\gamma^2+\rho^2))}
      \\&\quad>   2\rho^2\theta+2\rho\gamma+2\sqrt{\rho^2\gamma^2}
        &&        \text{because $\sqrt{x}$ is \prope{strictly monotonically increasing}}
      \\&\quad=   2\rho^2\theta+2\rho\gamma+2\abs{\rho}\gamma
        &&        \text{because $\gamma>0$}
      \\&\quad=   2\rho^2\theta+2\gamma\brp{\rho+\abs{\rho}}
      \\&\quad\ge 0
    \end{align*}

  \item Completing the proof\ldots
    \begin{align*}
      &\fR(p,q)
      \\&\eqd \int_{\theta=0}^{\theta=\phi} \sqrt{r^2(\theta) + \brp{\frac{\dr}{\dth}}^2}\dth 
        && \text{by def. of $\fR(p,q)$}
      \\&= \int_{0}^{\phi} \sqrt{
              \brs{\brp{\frac{r_q-r_p}{\phi}}\theta + r_p}^2
             +\brs{\frac{r_q-r_p}{\phi}}^2
             }\dtheta
        && \text{by def. of $\fr(\theta)$}
      \\&= \int_{0}^{\phi} \sqrt{
              \brs{\frac{(r_q-r_p)\theta + r_p\phi}{\phi}}^2
             +\brs{\frac{r_q-r_p}{\phi}}^2
             }\dtheta
      \\&= \frac{1}{\phi}\int_{0}^{\phi} \sqrt{
             (r_q-r_p)^2\theta^2 + 2(r_q-r_p)(r_p\phi)\theta + (r_p\phi)^2+(r_q-r_p)^2
             }\dtheta
      \\&= \frac{1}{\phi}\int_{0}^{\phi} \sqrt{
             \underbrace{\rho^2}_a\theta^2 + 
             \underbrace{2\rho\gamma}_b\theta + 
             \underbrace{(\gamma^2+\rho^2)}_c
             }
           \dtheta
      \\&= \frac{1}{\phi}\Bigg[ 
           \frac{2\rho^2\theta+2\rho\gamma}{4\rho^2}\sqrt{\rho^2\theta^2+2\rho\gamma\theta+(\gamma^2+\rho^2)} 
        && \text{[by \pref{item:larcR_lnarg} and}
           \\&\quad+ 
           \frac{4\rho^2(\gamma^2+\rho^2)-(2\rho\gamma)^2}{8\abs{\rho}^3}\ln\brp{2\rho^2\theta+2\rho\gamma+2\sqrt{\rho^2(\rho^2\theta^2+2\rho\gamma\theta+(\gamma^2+\rho^2))}}
           \Bigg]_{\theta=0}^{\theta=\phi}
        && \text{by \pref{lem:abc}]}
      \\&= \mathrlap{
             \frac{1}{\phi}\brs{ 
             \frac{\rho\theta+\gamma}{2\rho}\sqrt{\rho^2\theta^2+2\rho\gamma\theta+(\gamma^2+\rho^2)} 
             + 
             \frac{\abs{\rho}}{2}\ln\brp{2\rho^2\theta+2\rho\gamma+2\abs{\rho}\sqrt{\rho^2\theta^2+2\rho\gamma\theta+(\gamma^2+\rho^2)}}
             }_{\theta=0}^{\theta=\phi}
           }
      \\&= \mathrlap{\brs{
           \frac{(\rho\phi+\gamma)\sqrt{\rho^2\phi^2+2\rho\gamma\phi+(\gamma^2+\rho^2)}}
                {2\rho\phi} 
           + \frac{\abs{\rho}}{2\phi} 
             \ln\brp{2\rho^2\phi+2\rho\gamma+2\abs{\rho}\sqrt{\rho^2\phi^2+2\rho\gamma\phi+(\gamma^2+\rho^2)}}
           }}
           \\&\quad- \brs{ 
           \frac{\gamma\sqrt{\gamma^2+\rho^2}}
                {2\rho\phi} 
           + \frac{\abs{\rho}}{2\phi} 
             \ln\brp{2\rho\gamma+2\abs{\rho}\sqrt{\gamma^2+\rho^2}}
           }
      \\&= \frac{r_q\phi\sqrt{(r_q\phi)^2+\rho^2}-r_p\phi\sqrt{(r_p\phi)^2+\rho^2}}
                {2\rho\phi} 
           \\&\quad+\frac{\abs{\rho}}{2\phi}\brs{ 
             \ln(2)+\ln\brp{\rho^2\phi+\rho\gamma+\abs{\rho}\sqrt{(r_q\phi)^2+\rho^2}}
           - \ln(2)-\ln\brp{\rho\gamma+\abs{\rho}\sqrt{\gamma^2+\rho^2}}
           }
        && \text{by \pref{item:larcR_lem}}
      \\&= \mathrlap{\frac{r_q\sqrt{(r_q\phi)^2+\rho^2}-r_p\sqrt{(r_p\phi)^2+\rho^2}}
                {2\rho} 
           +\frac{\abs{\rho}}{2\phi}\brs{ 
             \ln\brp{r_q\rho\phi+\abs{\rho}\sqrt{(r_q\phi)^2+\rho^2}}
           - \ln\brp{r_p\rho\phi+\abs{\rho}\sqrt{(r_p\phi)^2+\rho^2}}
           }}
    \end{align*}

%  \item 
%    \begin{align*}
%      \fR(p,q)
%        &= \int_{\theta_p}^{\theta_q} \sqrt{r^2(\theta) + \brp{\frac{\dr}{\dth}}^2}\dth 
%        && \text{by \pref{item:larcR_arclen}}
%      \\&= \int_{\theta_p}^{\theta_q} \sqrt{
%             \brs{r_p\frac{\theta-\theta_q}{\theta_p-\theta_q} + r_q\frac{\theta-\theta_p}{\theta_q-\theta_p}}^2 
%             +
%             \brs{\opddth\brp{ r_p\frac{\theta-\theta_q}{\theta_p-\theta_q} +r_q\frac{\theta-\theta_p}{\theta_q-\theta_p}}}^2 
%             }
%             \dth
%      \\&= \int_{\theta_p}^{\theta_q} \sqrt{
%             \brs{\brp{\frac{r_p-r_q}{\theta_p-\theta_q}}\theta + \brs{\frac{r_q\theta_p-r_p\theta_q}{\theta_p-\theta_q}}}^2
%            +\brp{\frac{r_p-r_q}{\theta_p-\theta_q}}^2
%             }
%             \dth
%      \\&= \abs{\frac{1}{\theta_p-\theta_q}}\int_{\theta_p}^{\theta_q} \sqrt{
%             \brs{\brp{r_p-r_q}\theta + \brs{r_q\theta_p-r_p\theta_q}}^2
%            +\brp{{r_p-r_q}}^2
%             }
%             \dth
%      \\&= \frac{1}{B}\int_{\theta_p}^{\theta_q} \sqrt{
%             \underbrace{(r_p-r_q)^2}_a\theta^2 + 
%             \underbrace{2(r_p-r_q)(r_q\theta_p-r_p\theta_q)}_b\theta + 
%             \underbrace{(r_q\theta_p-r_p\theta_q)^2+(r_p-r_q)^2}_c
%             }
%           \dth
%      \\&= \frac{1}{B}\brs{ 
%           \frac{b+2a\theta}{4a}\sqrt{a\theta^2+b\theta+c} + 
%           \frac{4ac-b^2}{8a^{3/2}}\ln\brp{2a\theta+b+2\sqrt{a(a\theta^2+b\theta+c)}}
%           }_{\theta=\theta_p}^{\theta=\theta_q}
%        && \text{by \prefpp{lem:abc}}
%      \\&= \frac{1}{\phi}\brs{ 
%           \frac{b+2a\theta}{4a}\sqrt{a\theta^2+b\theta+c} + 
%           \frac{4ac-b^2}{8a^{3/2}}\ln\brp{2a\theta+b+2\sqrt{a(a\theta^2+b\theta+c)}}
%           }_{\theta=0}^{\theta=\phi}
%        && \text{by \pref{item:larcR_theta}}
%    \end{align*}

  \end{enumerate}
\end{proof}

%=======================================
\section{Properties}
%=======================================
%=======================================
\subsection{Arc function R(p,q) properties}
%=======================================
%The next proposition gives more details of how $\fR(p,q)$ changes as $\rho$ increases.
%%---------------------------------------
%\begin{proposition}
%\label{prop:larcR_drho}
%%---------------------------------------
%Let $\fR(p,q)$ be as defined in \prefpp{def:larc}.
%\propbox{
%  \oppprho\fR(p,q) = 
%  }
%\end{proposition}
%\begin{proof}
%\begin{enumerate}
%  \item lemma: \begin{array}[t]{rclclcl}
%                 \oppprho r_p &\eqd& \oppprho\brp{r_q-\rho} &=& 
%               \end{array}
%\end{enumerate}
%\end{proof}


If we really want the \fncte{Langrange arc distance} $\metric{p}{q}$ to be an \prope{extension}
of the \fncte{spherical metric}, then $\fR(p,q)$ must equal $r_p\phi$ 
(the arc length between $p$ and $q$ on a circle centered at the origin) when $r_p=r_q$.
This is in fact the case, as demonstrated next.
%---------------------------------------
\begin{proposition}[$\fR(p,q)$ on spherical surface]
\label{prop:larcR_rho}
%---------------------------------------
Let $\fR(p,q)$, $r_p$, $r_q$, and $\phi$ be defined as in \pref{def:larc}.
%If $r_p\neq0$, $r_q\neq0$ and $\phi>0$ then 
\propbox{
  \brb{\begin{array}{rclD}
    r_p=r_q&\neq& 0 & and\\
    \phi   &\neq& 0
  \end{array}}
  \quad\implies\quad   
  \brb{\fR(p,q) = r_p\phi}
  }
\end{proposition}
\begin{proof}
\begin{enumerate}
  \item lemma: \label{item:larcR_rho_lem1}
    \begin{align*}
        \lim_{\rho\to0}
             \frac{r_q\sqrt{(r_q\phi)^2+\rho^2}-r_p\sqrt{(r_p\phi)^2+\rho^2}}
                  {2\rho} 
          &= \lim_{\rho\to0}
             \frac{r_q\sqrt{(r_q\phi)^2+0}-r_p\sqrt{(r_p\phi)^2+0}}
                  {2\rho} 
         &&= \lim_{\rho\to0}\brp{\frac{\phi}{2}}
             \frac{r_q^2-r_p^2}{r_q-r_p} 
        \\&= \lim_{\rho\to0}\brp{\frac{\phi}{2}}
             \frac{(r_q-r_p)(r_q+r_p)}{r_q-r_p} 
         &&= \lim_{\rho\to0}\brp{\frac{\phi}{2}}(r_q+r_p)
        \\&= r_p\phi
    \end{align*}

  \item lemma: \label{item:larcR_rho_lem2}
    \begin{align*}
        &\lim_{\rho\to0^+}\frac{\rho}{2\phi}\brs{ 
               \ln\brp{r_q\rho\phi+\abs{\rho}\sqrt{(r_q\phi)^2+\rho^2}}
             - \ln\brp{r_p\rho\phi+\abs{\rho}\sqrt{(r_p\phi)^2+\rho^2}}
             }
        \\&= \lim_{\rho\to0^+}
             \frac{\rho}{2\phi}\brs{ 
               \ln\brp{r_q\rho\phi+\rho\sqrt{(r_q\phi)^2}}
             - \ln\brp{r_p\rho\phi+\rho\sqrt{(r_p\phi)^2}}
             }
        \\&= \lim_{\rho\to0^+}
             \frac{\rho}{2\phi}\brs{ 
               \ln(\rho)+\ln\brp{r_q\phi+\sqrt{(r_q\phi)^2}}
             - \ln(\rho) - \ln\brp{r_p\phi+\sqrt{(r_p\phi)^2}}
             }
        \\&= \lim_{\rho\to0^+}
              \frac{\rho}{2\phi}\brs{ 
             +\ln\brp{r_q\phi+\sqrt{(r_q\phi)^2}}
             -\ln\brp{r_p\phi+\sqrt{(r_p\phi)^2}}
             }
        \\&= 0
    \end{align*}

  \item lemma: \label{item:larcR_rho_lem3}
    \begin{align*}
        &\lim_{\rho\to0^-} 
        \frac{\rho}{2\phi}\brs{ 
            \ln\brp{r_q\rho\phi+\abs{\rho}\sqrt{(r_q\phi)^2+\rho^2}}
          - \ln\brp{r_p\rho\phi+\abs{\rho}\sqrt{(r_p\phi)^2+\rho^2}}
          }
        \\&= \lim_{\rho\to0^-}
             \frac{\rho}{2\phi}\brs{ 
                 \ln\brp{-r_q\abs{\rho}\phi+\abs{\rho}\sqrt{(r_q\phi)^2}}
               - \ln\brp{-r_p\abs{\rho}\phi+\abs{\rho}\sqrt{(r_p\phi)^2}}
               }
        \\&= \lim_{\rho\to0^-}
             \frac{\rho}{2\phi}\brs{ 
                 \ln\brp{\rho}+\ln\brp{-r_q\phi+\sqrt{(r_q\phi)^2}}
               - \ln\brp{\rho}-\ln\brp{-r_p\phi+\sqrt{(r_p\phi)^2}}
               }
        \\&= \lim_{\rho\to0^-}
             \frac{\rho}{2\phi}\brs{ 
                \ln\brp{-r_q\phi+\sqrt{(r_q\phi)^2}}
               -\ln\brp{-r_p\phi+\sqrt{(r_p\phi)^2}}
               }
        \\&= 0
    \end{align*}

  \item lemma: \label{item:larcR_rho_lem4}
        By \pref{item:larcR_rho_lem2}, \pref{item:larcR_rho_lem3}, 
        and by \prope{continuity} \ldots
    \\\indentx$\ds
        \lim_{\rho\to0} 
        \frac{\rho}{2\phi}\brs{ 
            \ln\brp{r_q\rho\phi+\abs{\rho}\sqrt{(r_q\phi)^2+\rho^2}}
          - \ln\brp{r_p\rho\phi+\abs{\rho}\sqrt{(r_p\phi)^2+\rho^2}}
          }
          = 0
         $

  \item Completing the proof \ldots
    \begin{align*}
        \lim_{\rho\to0}\fR(p,q) 
          &= \lim_{\rho\to0}\Bigg[
             \frac{r_q\sqrt{(r_q\phi)^2+\rho^2}-r_p\sqrt{(r_p\phi)^2+\rho^2}}
                  {2\rho}
             \\&\qquad+
             %\lim_{\rho\to0}
             \frac{\rho}{2\phi}
               \ln\brp{r_q\rho\phi+\abs{\rho}\sqrt{(r_q\phi)^2+\rho^2}}
             - \ln\brp{r_p\rho\phi+\abs{\rho}\sqrt{(r_p\phi)^2+\rho^2}}
             \Bigg]
          && \text{by \pref{thm:Rpq}}
        \\&= 0 + 
             \lim_{\rho\to0}
             \frac{\rho}{2\phi}\brs{ 
               \ln\brp{r_q\rho\phi+\abs{\rho}\sqrt{(r_q\phi)^2+\rho^2}}
             - \ln\brp{r_p\rho\phi+\abs{\rho}\sqrt{(r_p\phi)^2+\rho^2}}
             }
          && \text{by \pref{item:larcR_rho_lem1}}
        \\&= 0 + 0
          && \text{by \pref{item:larcR_rho_lem4}}
        \\&= 0
          %&& \text{by field prop. of $\fieldR$}
          %&& \text{because $0$ is the \vale{additive identity element} in $\fieldR$.}
    \end{align*}
\end{enumerate}
\end{proof}

Later in \prefpp{thm:larc_distance}, we want to prove that the \fncte{Langrange arc distance} \xref{def:larc} $\distance{p}{q}$
is indeed, as its name suggests, a \fncte{distance function} \xref{def:distance}.
%One property of distance functions is \prope{symmetry}.
\pref{prop:larcR_sym} (\prope{symmetry}) and 
\pref{prop:larcR_pos} (\prope{positivity}) will help. %that $\distance{p}{q}$ is also \prope{symmetric}.
Meanwhile, \pref{prop:larcR_pos} will itself receive help from 
\pref{prop:larcR_mono} (\prope{monotonicity}).
%---------------------------------------
\begin{proposition}[\prope{symmetry} of $\fR$]
\label{prop:larcR_sym}
%---------------------------------------
Let $\fR(p,q)$ be defined as in \prefpp{def:larc}.
\propbox{
  \begin{array}{rcl CDD}
      \fR(p,q) &=  & \fR(q,p)        & \forall p,q   \in\R^\xN & (\prope{symmetric})      & 
  \end{array}
  }
\end{proposition}
\begin{proof}
    \begin{enumerate}
      \item dummy variable: Let $\mu\eqd\phi-\theta$ which implies  \label{item:larcR_distance_mu}
        $\theta=\phi-\mu$ and $\dtheta=-\dmu$.

      \item lemma: \label{item:larcR_distance_sub}
        \begin{align*}
          \fr(\mu;q,p)
            &\eqd \fr(\phi-\theta;q,p)
            &&    \text{by \pref{item:larcR_distance_mu}}
          \\&=    \brp{\frac{r_p-r_q}{\phi}}(\phi-\theta) + r_q
             =    \brp{\frac{r_q-r_p}{\phi}}\theta + \brp{r_p-r_q} + r_q
             =    \brp{\frac{r_q-r_p}{\phi}}\theta + r_p
          \\&=    \fr(\theta;p,q)
        \end{align*} 

      \item Completing the proof\ldots
        \begin{align*}
          \fR(p,q)
            &= \int_{\theta=0}^{\theta=\phi} \sqrt{\fr^2\brp{\theta;p,q} + \brs{\frac{\dr(\theta;p,q)}{\dtheta}}^2} \dtheta
          \\&= \int_{\phi-\mu=0}^{\phi-\mu=\phi} \sqrt{\fr^2\brp{\mu;q,p} + \brs{\frac{\dr(\mu;q,p)}{-\dmu}}^2} (-\dmu)
            &&    \text{by \pref{item:larcR_distance_mu} and \pref{item:larcR_distance_sub}}
          \\&= -\int_{\mu=\phi}^{\mu=0} \sqrt{\fr^2\brp{\mu;q,p} + \brs{\frac{\dr(\mu;q,p)}{\dmu}}^2} \dmu
          \\&= \int_{\mu=0}^{\mu=\phi} \sqrt{\fr^2\brp{\mu;q,p} + \brs{\frac{\dr(\mu;q,p)}{\dmu}}^2} \dmu
            && \text{by the \thme{Second Fundamental Theorem of Calculus}\footnotemark}
          \\&= \fR(q,p)
        \end{align*}
        \citetblt{
          \citerpgc{hijab2016}{170}{3319284002}{Theorem 4.4.3 Second Fundamental Theorem of Calculus},\\
          \citerpgc{amann2008}{31}{3764374721}{Theorem 4.13 The second fundamental theorem of calculus}
          }
    \end{enumerate}
    %\begin{enumerate}
    %  \item Proof when $\rho>0$:
    %    \begin{align*}
    %      \fR(p,q)
    %        &= \frac{r_q\sqrt{(r_q\phi)^2+\rho^2}-r_p\sqrt{(r_p\phi)^2+\rho^2}}
    %                {2\rho} 
    %           +\frac{\rho}{2\phi}\brs{ 
    %             \ln\brp{r_q\rho\phi+\abs{\rho}\sqrt{(r_q\phi)^2+\rho^2}}
    %           - \ln\brp{r_p\rho\phi+\abs{\rho}\sqrt{(r_p\phi)^2+\rho^2}}
    %           }
    %      \\&= \frac{r_q\sqrt{(r_q\phi)^2+\rho^2}-r_p\sqrt{(r_p\phi)^2+\rho^2}}
    %                {2\rho} 
    %           +\frac{\rho}{2\phi}\brs{ 
    %             \ln\brp{r_q\rho\phi+\rho\sqrt{(r_q\phi)^2+\rho^2}}
    %           - \ln\brp{r_p\rho\phi+\rho\sqrt{(r_p\phi)^2+\rho^2}}
    %           }
    %      \\&= \frac{r_q\sqrt{(r_q\phi)^2+\rho^2}-r_p\sqrt{(r_p\phi)^2+\rho^2}}
    %                {2\rho} 
    %           +\frac{\rho}{2\phi}\brs{ 
    %             \ln\brp{r_q\phi+\sqrt{(r_q\phi)^2+\rho^2}}
    %           - \ln\brp{r_p\phi+\sqrt{(r_p\phi)^2+\rho^2}}
    %           }
    %      \\&= -\frac{r_q\sqrt{(r_q\phi)^2+(-\rho)^2}+r_p\sqrt{(r_p\phi)^2+(-\rho)^2}}
    %                {2(-\rho)} 
    %           +\frac{(-\rho)}{2\phi}\brs{ 
    %           - \ln\brp{r_q\phi+\sqrt{(r_q\phi)^2+(-\rho)^2}}
    %           + \ln\brp{r_p\phi+\sqrt{(r_p\phi)^2+(-\rho)^2}}
    %           }
    %      \\&= \frac{r_p\sqrt{(r_p\phi)^2+(-\rho)^2} -r_q\sqrt{(r_q\phi)^2+(-\rho)^2}}
    %                {2(-\rho)} 
    %           +\frac{(-\rho)}{2\phi}\brs{ 
    %             \ln\brp{r_p\phi+\sqrt{(r_p\phi)^2+(-\rho)^2}}
    %           - \ln\brp{r_q\phi+\sqrt{(r_q\phi)^2+(-\rho)^2}}
    %           }
    %      \\&=\fR(q,p)
    %    \end{align*}
    %
    %  \item Proof when $\rho<0$:
    %    \begin{align*}
    %      \fR(p,q)
    %        &= \frac{r_q\sqrt{(r_q\phi)^2+\rho^2}-r_p\sqrt{(r_p\phi)^2+\rho^2}}
    %                {2\rho} 
    %           +\frac{\rho}{2\phi}\brs{ 
    %             \ln\brp{r_q\rho\phi+\abs{\rho}\sqrt{(r_q\phi)^2+\rho^2}}
    %           - \ln\brp{r_p\rho\phi+\abs{\rho}\sqrt{(r_p\phi)^2+\rho^2}}
    %           }
    %      \\&= \frac{r_q\sqrt{(r_q\phi)^2+\rho^2}-r_p\sqrt{(r_p\phi)^2+\rho^2}}
    %                {2\rho} 
    %           +\frac{\rho}{2\phi}\brs{ 
    %             \ln\brp{r_q\rho\phi-\rho\sqrt{(r_q\phi)^2+\rho^2}}
    %           - \ln\brp{r_p\rho\phi-\rho\sqrt{(r_p\phi)^2+\rho^2}}
    %           }
    %      \\&= \frac{r_q\sqrt{(r_q\phi)^2+\rho^2}-r_p\sqrt{(r_p\phi)^2+\rho^2}}
    %                {2\rho} 
    %           +\frac{\rho}{2\phi}\brs{ 
    %             \ln\brp{r_q\phi-\sqrt{(r_q\phi)^2+\rho^2}}
    %           - \ln\brp{r_p\phi-\sqrt{(r_p\phi)^2+\rho^2}}
    %           }
    %      \\&= -\frac{r_q\sqrt{(r_q\phi)^2+(-\rho)^2}+r_p\sqrt{(r_p\phi)^2+(-\rho)^2}}
    %                {2(-\rho)} 
    %           +\frac{(-\rho)}{2\phi}\brs{ 
    %           - \ln\brp{r_q\phi-\sqrt{(r_q\phi)^2+(-\rho)^2}}
    %           + \ln\brp{r_p\phi-\sqrt{(r_p\phi)^2+(-\rho)^2}}
    %           }
    %      \\&= \frac{r_p\sqrt{(r_p\phi)^2+(-\rho)^2} -r_q\sqrt{(r_q\phi)^2+(-\rho)^2}}
    %                {2(-\rho)} 
    %           +\frac{(-\rho)}{2\phi}\brs{ 
    %             \ln\brp{r_p\phi-\sqrt{(r_p\phi)^2+(-\rho)^2}}
    %           - \ln\brp{r_q\phi-\sqrt{(r_q\phi)^2+(-\rho)^2}}
    %           }
    %      \\&=\fR(q,p)
    %    \end{align*}
    %\end{enumerate}
\end{proof}


%---------------------------------------
\begin{proposition}[\prope{monotonicity} of $\fR$]
\label{prop:larcR_mono}
%---------------------------------------
Let $\fR(p,q)$ and $\phi$ be defined as in \prefpp{def:larc}.
Let $\phi_1$ be the polar angle between the point pair $\opair{p_1}{q_1}$ in $\R^\xN$ and
Let $\phi_2$    the polar angle between the point pair $\opair{p_2}{q_2}$ in $\R^\xN$.
\propbox{\begin{array}{rclCD}
  \brb{\phi_1<\phi_2} &\implies& \brb{\fR(p_1,q_1)<\fR(p_2,q_2)} & \forall \phi_1,\phi_2\in\intoc{0}{\pi}
                                                                 & (\prope{strictly monotonically increasing} in $\phi$)
\end{array}}
\end{proposition}
\begin{proof}
\begin{align*}
  \opddphi \fR(p,q)
    &\eqd \opddphi \int_0^\phi \sqrt{\fr^2(\theta) + \brs{\deriv{\fr(\theta)}{\theta}}^2} \dtheta
    && \text{by \prefpp{def:larc}}
  \\&= \opddphi \int_0^\phi \sqrt{\brs{\brp{\frac{r_q-r_p}{\phi}}\theta + r_p}^2 + \brs{\frac{r_q-r_p}{\phi}}^2} \dtheta
  \\&= \sqrt{\brs{\brp{\frac{r_q-r_p}{\phi}}\phi + r_p}^2 + \brs{\frac{r_q-r_p}{\phi}}^2}
    && \text{by the \thme{First Fundamental Theorem of Calculus}\footnotemark}
  \\&= \sqrt{r_q^2 + \brs{\frac{r_q-r_p}{\phi}}^2}
  \\&> 0
  \\&\mathrlap{\implies\text{$\fR(p,q)$ is \prope{strictly monotonically increasing} in $\phi$}}
\end{align*}
\citetblt{
  \citerpgc{schechter1996}{674}{0080532993}{25.15},
  \citerpg{haaser1991}{218}{0486665097}
  }
\end{proof}



%---------------------------------------
\begin{proposition}[\prope{positivity} of $\fR$]
\label{prop:larcR_pos}
%---------------------------------------
Let $\fR(p,q)$ and $\phi$ be defined as in \prefpp{def:larc}.
\propbox{
  \brb{\begin{array}{rcl}
    \phi &\in& \intoc{0}{\pi}\\
    (\phi&\neq& 0)
  \end{array}}
  \qquad\implies\qquad
  \brb{\begin{array}{rcl CD}
    \fR(p,q) &>& 0   & \forall p,q   \in\R^\xN & (\prope{positive})
  \end{array}}
  }
\end{proposition}
\begin{proof}
  \begin{align*}
    \opddphi \fR(p,q)
      &\eqd \int_0^\phi \sqrt{\fr^2(\theta) + \brs{\deriv{\fr(\theta)}{\theta}}^2} \dtheta
      && \text{by \prefpp{def:larc}}
    \\&= \brlr{\int \sqrt{\fr^2(\theta) + \brs{\deriv{\fr(\theta)}{\theta}}^2} \dtheta}_\phi
       - \brlr{\int \sqrt{\fr^2(\theta) + \brs{\deriv{\fr(\theta)}{\theta}}^2} \dtheta}_0
      && \begin{tabstr}{0.75}\begin{array}{M}by the Second Fundamental\\Theorem of Calculus\end{array}\end{tabstr}
    \\&> 0
      &&\text{by \prefpp{prop:larcR_mono}}
  \end{align*}
\end{proof}

\begin{minipage}{\tw-60mm}
  For the sake of \prope{continuity} at the origin of $\R^\xN$, one might hope that 
  it doesn't matter which ``direction" the points $p$ or $q$ approach the origin
  when computing the limit of $\fR(p,q)$.
  This however is \emph{not} the case, as demonstrated next and illustrated to the right and in \prefpp{ex:limrp0}.
  In fact, the limits very much depend on $\phi$\ldots resulting in a \prope{discontinuity} at the origin,
  as demonstrated in \prefpp{thm:larc_cont}.
\end{minipage}\hfill%
\begin{tabular}{c}\includegraphics{../common/math/graphics/pdfs/limrp0Rpq.pdf}\end{tabular}\\
%---------------------------------------
\begin{proposition}[limit cases of $\fR$]
\label{prop:larcR_rp0}
%---------------------------------------
Let $\fR(p,q)$, $r_p$, $r_q$, and $\phi$ be defined as in \prefpp{def:larc}.
\thmbox{\begin{array}{>{\ds}rc>{\ds}lCD}
    \lim_{r_p\to 0}\fR(p,q) 
      &=& \frac{r_q}{2}\brs{\sqrt{\phi^2+1} + \frac{\ln\brp{\phi+\sqrt{\phi^2+1}}}{\phi}}
      &   \forall p,q\in\R^\xN\setd\origin,\,\phi\neq0
      &   ($p$ approaching origin)
  \\\lim_{r_q\to 0}\fR(p,q) 
      &=& \frac{r_p}{2}\brs{\sqrt{\phi^2+1} + \frac{\ln\brp{\phi+\sqrt{\phi^2+1}}}{\phi}}
      &   \forall p,q\in\R^\xN\setd\origin,\,\phi\neq0
      &   ($q$ approaching origin)
\end{array}}
\end{proposition}
\begin{proof}
\begin{align*}
  &\lim_{r_p\to 0}\fR(p,q) 
  \\&=  \mathrlap{
        \lim_{r_p\to 0}
        \frac{r_q\sqrt{(r_q\phi)^2+\rho^2}-r_p\sqrt{(r_p\phi)^2+\rho^2}}
             {2\rho} 
        +\frac{\abs{\rho}}{2\phi}\brs{ 
          \ln\brp{r_q\rho\phi+\abs{\rho}\sqrt{(r_q\phi)^2+\rho^2}}
        - \ln\brp{r_p\rho\phi+\abs{\rho}\sqrt{(r_p\phi)^2+\rho^2}}
        }}
  \\&\qquad \text{by \prefpp{thm:Rpq}}
  \\&=  \frac{r_q\sqrt{(r_q\phi)^2+r_q^2}-0}{2r_q} 
        +\frac{\abs{r_q}}{2\phi}\brs{ 
          \ln\brp{r_q^2\phi+\abs{r_q}\sqrt{(r_q\phi)^2+r_q^2}}
        - \ln\brp{0+\abs{r_q}\sqrt{0+r_q^2}}
        }
    && \text{by $\ds\lim_{r_p\to0}$ operation}
  \\&=  \frac{r_q\sqrt{\phi^2+1}}{2} 
        +\frac{r_q}{2\phi}\brs{ 
          \ln\brp{r_q^2\phi+r_q^2\sqrt{\phi^2+1}} - \ln\brp{r_q^2}
        }
  \\&=\mathrlap{  
       \frac{r_q}{2}\brs{
            \sqrt{\phi^2+1} 
          + \frac{\ln(r_q^2) + \ln\brp{\phi+\sqrt{\phi^2+1}} - \ln\brp{r_q^2}}{\phi}
          }
     = \frac{r_q}{2}\brs{\sqrt{\phi^2+1} + \frac{\ln\brp{\phi+\sqrt{\phi^2+1}}}{\phi}}
     }
\end{align*}
\begin{align*}
  \lim_{r_q\to 0}\fR(p,q) 
    &= \lim_{r_q\to 0}\fR(q,p) 
    && \text{by \pref{prop:larcR_sym}}
   %&& \text{by \prefp{prop:larcR_sym} (\prope{symmetry})}
  \\&= \frac{r_p}{2}\brs{\sqrt{\phi^2+1} + \frac{\ln\brp{\phi+\sqrt{\phi^2+1}}}{\phi}}
    && \text{by previous result}
\end{align*}
%\begin{align*}
%  \lim_{r_p\to 0}\fR(p,q) 
%    &\eqd \lim_{r_p\to 0}\int_0^\phi \sqrt{r^2(\theta) + \brp{\frac{\dr}{\dth}}^2}\dth
%    && \text{by definition of $\fR(p,q)$ \xref{def:larc}}
%  \\&= \lim_{r_p\to 0}\int_0^\phi \sqrt{\brs{\brp{\frac{r_q-r_p}{\phi}}\theta + r_p}^2 + \brs{\frac{r_q-r_p}{\phi}}^2} \dtheta
%    && \text{by definition of $\fr(\theta)$ \xref{def:larc}}
%  \\&= \int_0^\phi \sqrt{\brs{\brp{\frac{r_q-0}{\phi}}\theta + 0}^2 + \brs{\frac{r_q-0}{\phi}}^2} \dtheta
%  \\&= \frac{r_q}{\phi}\int_0^\phi \sqrt{\theta^2 + 1} \dtheta
%  \\&= \frac{r_q}{\phi}\mathrlap{\brs{
%           \frac{2\cdot1\cdot\theta+0}{4\cdot1}\sqrt{1\theta^2+0\theta+1} 
%         + \frac{4\cdot1\cdot1-0^2}{8\cdot1^{3/2}}\ln\brp{2\cdot1\cdot \theta+0+2\sqrt{1(1\theta^2+0\theta+1)}}+\varepsilon
%       }_0^\phi
%    \quad\text{by \pref{lem:abc}}}
%   %&& \text{by \prefpp{lem:abc} for $\otriple{a}{b}{c}=\otriple{1}{0}{1}$}
%  \\&= \frac{r_q}{\phi}\brs{
%           \frac{\theta}{2}\sqrt{\theta^2+1} 
%         + \frac{1}{2}\ln\brp{2\theta+2\sqrt{\theta^2+1}}+\varepsilon
%         }_0^\phi
%  \\&= \mathrlap{\frac{r_q}{2\phi}\brs{
%           \phi\sqrt{\phi^2+1} 
%         - 0   \sqrt{0^2+1} 
%         + \ln\brp{2\phi+2\sqrt{\phi^2+1}}
%         - \ln\brp{0+2\sqrt{0+1}}
%         }}
%  \\&= \frac{r_q}{2\phi}\brs{
%         \phi\sqrt{\phi^2+1} 
%         + \ln(2) + \ln\brp{\phi+\sqrt{\phi^2+1}} -\ln(2)
%         }
%  \\&= \frac{r_q}{2}\brs{\sqrt{\phi^2+1} + \frac{\ln\brp{\phi+\sqrt{\phi^2+1}}}{\phi}}
%  \\\\
%  \lim_{r_q\to 0}\fR(p,q) 
%    &= \lim_{r_q\to 0}\fR(q,p) 
%    && \text{by \prefp{prop:larcR_sym} (\prope{symmetry})}
%  \\&= \frac{r_p}{2}\brs{\sqrt{\phi^2+1} + \frac{\ln\brp{\phi+\sqrt{\phi^2+1}}}{\phi}}
%    && \text{by previous result}
%\end{align*}
\end{proof}

%---------------------------------------
\begin{example}
\label{ex:limrp0}
%---------------------------------------
Let $\fR(p,q)$, $\phi$, and $r_q$ be defined as in \pref{def:larc}.
\\\indentx$\begin{array}{MM>{\ds}lclcl}
    If $\phi=0$     &then& \lim_{p\to0}\fR\brs{p,\opair{r_q}{0}} &=& r_q
  \\If $\phi=\pi/2$ &then& \lim_{p\to0}\fR\brs{p,\opair{r_q}{0}} &=& r_q\times(1.323652\cdots) &\eqa& 1.3r_q
  \\If $\phi=\pi$   &then& \lim_{p\to0}\fR\brs{p,\opair{r_q}{0}} &=& r_q\times(1.944847\cdots) &\eqa& 1.9r_q  %1.9448477277847245517307915691607
\end{array}$\qquad
\begin{tabular}{c}\includegraphics{../common/math/graphics/pdfs/limrp0.pdf}\end{tabular}
\end{example}
\begin{proof}
\begin{align*}
  \brlr{\fR(p,q)}_{\phi=\pi/2}
      &= \frac{r_q}{2}\brs{\sqrt{\brp{\frac{\pi}{2}}^2+1} + \frac{\ln\brp{\frac{\pi}{2}+\sqrt{\brp{\frac{\pi}{2}}^2+1}}}{\frac{\pi}{2}}}
     &&= r_q\times(1.323652\cdots) %1.3236523505130214048874285420062
    \\
  \brlr{\fR(p,q)}_{\phi=\pi}
      &= \frac{r_q}{2}\brs{\sqrt{\pi^2+1} + \frac{\ln\brp{\pi+\sqrt{\pi^2+1}}}{\pi}}
     &&= r_q\times(1.944847\cdots) %1.9448477277847245517307915691607
\end{align*}
\end{proof}


%=======================================
\subsection{Distance function d(p,q) properties}
%=======================================
The \fncte{Langrange arc distance} $\distance{p}{q}$ is defined in two parts: 
one part being the Euclidean distance $\sfrac{1}{\pi}\abs{r_q-r_p}$ and the second part the length of the arc
$\sfrac{1}{\pi}\fR(p,q)$. 
There is risk in creating a multipart definition\ldots 
with the possible consequences being \prope{discontinuity} at the boundary of the parts.
\pref{prop:larc_lim} (next) demonstrates that when $r_p\neq0$ and $r_q\neq0$, 
there is \prope{continuity} as $\phi\to0$.
However, \prefpp{thm:larc_distance} demonstrates that in general for values of $\phi>0$,
$\distance{p}{q}$ is \prope{discontinuous} at the \structe{origin}.
%---------------------------------------
\begin{proposition}
\label{prop:larc_lim}
%---------------------------------------
Let $\fR(p,q)$, $r_p$, $r_q$, $\phi$, and $\origin$ be defined as in \prefpp{def:larc}.
\propbox{\begin{array}{F>{\ds}rc>{\ds}lc>{\ds}lc>{\ds}l}
    (A).&
    \lim_{\phi\to 0}                \fR(p,q) 
     &=& \abs{r_q-r_p} 
     & &
     &=& \pi\distance{p}{q}
       \quad\text{when $\phi=0$}
  \\(B).&
     \lim_{\phi\to 0}\lim_{r_p\to 0} \fR(p,q) 
     &=& \lim_{r_p\to 0} \lim_{\phi\to 0}\fR(p,q) 
     &=& r_q           
     &=& \pi\distance{\origin}{q}
  \\(C).&
     \lim_{\phi\to 0}\lim_{r_q\to 0} \fR(p,q) 
     &=& \lim_{r_q\to 0} \lim_{\phi\to 0}\fR(p,q) 
     &=& r_p           
     &=& \pi\distance{p}{\origin}
\end{array}}
\end{proposition}
\begin{proof}
\begin{align*}
  \lim_{\phi\to 0}\fR(p,q) 
    &\eqd \lim_{\phi\to 0}\int_0^\phi \sqrt{r^2(\theta) + \brp{\frac{\dr}{\dth}}^2}\dth
    && \text{by definition of $\fR(p,q)$ \xref{def:larc}}
  \\&= \lim_{\phi\to 0}\int_0^\phi \sqrt{\brs{\brp{\frac{r_q-r_p}{\phi}}\theta + r_p}^2 + \brs{\frac{r_q-r_p}{\phi}}^2} \dtheta
    && \text{by definition of $\fr(\theta)$ \xref{def:larc}}
  \\&= \lim_{\phi\to 0}\brs{\frac{1}{\phi}\int_0^\phi \sqrt{\brs{\brp{r_q-r_p}\theta + r_p\phi}^2 + \brs{r_q-r_p}^2} \dtheta}
  \\&= \frac{\lim_{\phi\to 0}\opddphi\int_0^\phi \sqrt{\brs{\brp{r_q-r_p}\theta + r_p\phi}^2 + \brs{r_q-r_p}^2} \dtheta}{\lim_{\phi\to 0}\opddphi\phi}
    && \text{by \thme{L'H/<opital's rule}}
  \\&= \frac{\lim_{\phi\to 0}\sqrt{\brs{\brp{r_q-r_p}\phi + r_p\phi}^2 + \brs{r_q-r_p}^2}}{1}
    && \text{by \thme{First Fundamental Theorem of Calculus}}
  \\&= \sqrt{\brs{r_q-r_p}^2}
  \\&= \abs{r_q-r_p}
  \\&= \pi\brlr{\distance{p}{q}}_{\phi=0}
   %&& \text{by definition of \fncte{Lagrange arc distance} \xrefr{def:larc}}
    && \text{by \prefpp{def:larc}}
  \\
  \\
  \lim_{\phi\to 0}\lim_{r_p\to 0} \fR(p,q) 
    &= \lim_{\phi\to 0}\frac{r_q}{2}\brs{\sqrt{\phi^2+1} + \frac{\ln\brp{\phi+\sqrt{\phi^2+1}}}{\phi}}
    && \text{by \prefpp{prop:larcR_rp0}}
  \\&= \frac{r_q}{2}\brs{\sqrt{0^2+1} + \frac{\ds\lim_{\phi\to 0}\opddphi\ln\brp{\phi+\sqrt{\phi^2+1}}}{\ds\lim_{\phi\to 0}\opddphi\phi}}
    && \text{by \thme{L'H/<opital's rule}}
  \\&= \mathrlap{
       \frac{r_q}{2}\brs{1 + \lim_{\phi\to 0}\frac{\ds1+\frac{2\phi}{2\sqrt{\phi^2+1}}}{\phi+\sqrt{\phi^2+1}}}
     = \frac{r_q}{2}\brs{1 + 1/1}
     = r_q           
     = \pi\distance{\origin}{q}
     }
  \\
  \\
  \lim_{\phi\to 0}\lim_{r_q\to 0} \fR(p,q) 
    &= \lim_{\phi\to 0}\lim_{r_q\to 0} \fR(q,p)
    && \text{by \prefpp{prop:larcR_sym}}
  \\&= r_p           
     = \pi\distance{p}{\origin}
    && \text{by previous result}
  \\
  \lim_{r_p\to 0}\lim_{\phi\to 0} \fR(p,q) 
    &= \lim_{r_p\to 0}\abs{r_q-r_p}
    && \text{by (A)}
  \\&= r_q
     = \pi\distance{\origin}{q}
  \\
  \lim_{r_q\to 0}\lim_{\phi\to 0} \fR(p,q) 
    &= \lim_{r_q\to 0}\abs{r_q-r_p}
    && \text{by (A)}
  \\&= r_p
     = \pi\distance{p}{\origin}
\end{align*}
\end{proof}


%---------------------------------------
\begin{theorem}
\label{thm:larc_cont}
%---------------------------------------
Let the \fncte{Lagrange arc distance} $\distance{p}{q}$ and \structe{origin} 
be defined as in \pref{def:larc}.
\\\thmboxp{
  The function $\distance{p}{q}$ is \prope{discontinuous} at the \structe{origin} of $\R^\xN$,
  but is \prope{continuous} everywhere else in $\R^\xN$.
  }
\end{theorem}
\begin{proof}
\begin{enumerate}
  \item Proof for when $p$ and $q$ are \emph{not} at the origin and $\phi\neq0$:
    \begin{enumerate}
      \item In this case, $\distance{p}{q}=\sfrac{1}{\pi}\fR(p,q)$.
      \item $\fR(p,q)$ is continuous everywhere in its domain because its solution,
            as given by \prefpp{thm:Rpq}, consists entirely of continuous functions
            such as $\ln(x)$, $\abs{x}$, etc.
      \item Therefore, in this case, $\distance{p}{q}$ is also \prope{continuous}.
    \end{enumerate}

  \item Proof for when $p$ and $q$ are \emph{not} at the origin and $\phi=0$:
        \\This follows from (A) of \prefpp{prop:larc_lim}.

  \item Proof for \prope{discontinuity} at origin:
      This follows from \prefpp{prop:larcR_rp0}, where it is demonstrated that 
      the limit of $\fR(p,q)$ is very much dependent on the ``direction" from which $p$ or $q$ 
      approaches the origin.
      For an illustration of this concept, see \prefpp{ex:limrp0}.
\end{enumerate}
\end{proof}

%---------------------------------------
\begin{theorem}
\label{thm:larc_distance}
%---------------------------------------
Let $\distance{p}{q}$ be \fncte{Lagrange arc distance} \xref{def:larc}.
\\\thmboxp{
  The function $\distance{p}{q}$ is a \fncte{distance function} \xref{def:distance}. In particular,
  \\\indentx$\begin{array}{F rcl CDD}
        (1). & \distance{p}{q} &\ge& 0               & \forall p,q\in\R^\xN & (\prope{non-negative})   & and 
      \\(2). & \distance{p}{q} &=  & 0  \iff p=q     & \forall p,q\in\R^\xN & (\prope{nondegenerate})  & and 
      \\(3). & \distance{p}{q} &=  & \distance{q}{p} & \forall p,q\in\R^\xN & (\prope{symmetric})      & 
    \end{array}$
  }
\end{theorem}
\begin{proof}
The \fncte{Langrange arc distance} \xref{def:larc} is simply the \fncte{Euclidean metric} \xref{def:emetric}
if $p$ or $q$ is at the origin, or if $\phi=0$.
In this case, (1)--(3) are satisfied automatically because all \fncte{metric}s have these properties \xref{def:metric}.
What is left to prove is that $\fR(p,q)$ has these properties when $p$ and $q$ are not at the origin and 
$\phi\neq0$.
\begin{enumerate}
  \item Proof that $\fR(p,q)\ge0$: If $\phi=0$, then the Euclidean metric is used.\\
        For any $\phi>0$, $\fR(p,q)>0$, as demonstrated by \prefpp{prop:larcR_pos}.
  \item Proof that $p=q\implies\fR(p,q)=0$: If $p=q$, then $\phi=0$, and the Euclidean metric is used, not $\fR(p,q)$.
  \item Proof that $\fR(p,q)=0\implies p=q$:
        If $\distance{p}{q}=0$ and $\phi=0$, then the Euclidean metric is used.
        If $\distance{p}{q}=0$ and $\phi>0$, then $\fR(p,q)$ never equals $0$ anyways, as demonstrated by \prefpp{prop:larcR_pos}.
  \item Proof that $\fR(p,q)=\fR(q,p)$: This is demonstrated by \prefpp{prop:larcR_sym}.
\end{enumerate}
\end{proof}




The \fncte{Lagrange arc distance} is \emph{not} a \fncte{metric} because in general the 
\prope{triangle inequality} property does not hold (next theorem).
Furthermore, the \fncte{Lagrange arc distance} does not induce a \fncte{norm}
because it is \prope{not translation invariant}
(the \prope{translation invariant} property is a necessary condition for a \fncte{metric} to induce a \fncte{norm}, 
\xrefnp{thm:vsn_d2norm}),
and balls in a \fncte{Lagrange arc distance space} are in general \prope{not convex}
(balls are always \prope{convex} %\xrefnp{def:convex} 
in a \structe{normed linear space}, %\xrefnp{def:norm}, 
\xrefnp{thm:norm_convex}). 
For more details about \fncte{distance space}s, see \prefpp{app:dspace}.

%---------------------------------------
\begin{theorem} %[\exmd{Lagrange arc and triangle inequality}]
\label{thm:larc}
%---------------------------------------
In the \structe{Lagrange arc distance space} $\dspaceX$ over a field $\F$
\thmbox{\begin{array}{FlclCD}
    (1). & \metric{p}{r}               &\nle& \metric{p}{q} + \metric{q}{r}    & \forall p,q,r\in\setX & (\prope{triangle inequality fails})
  \\(2). & \metric{p+r}{q+r}           &\neq& \metric{p}{q}                    & \forall p,q,r\in\setX & (\prope{not translation invariant})
  \\(3). & \metric{\alpha p}{\alpha q} &=&  \abs{\alpha}\,\metric{p}{q}        & \forall p,q,r\in\setX,\,\alpha\in\R & (\prope{homogeneous})
  \\(4). & \mc{5}{M}{$\distancen$ does not induce a norm}
  \\(5). & \mc{5}{M}{balls in $\dspaceX$ are in general \prope{not convex}}
\end{array}}
\end{theorem}
\begin{proof}
\begin{enumerate}
  \item Proof that the \prope{triangle inequality} property fails to hold in $\dspaceX$:
        Consider the following case\footnote{\seessp{lab_larc_distances_R2.xlg}}\ldots
        \\\indentx
        $\begin{array}{c}%
          \includegraphics{../common/math/graphics/pdfs/larc_trieq.pdf}%
        \end{array}$%
        \hspace{10mm}%
        $\begin{array}{rclcl}%
            \metric {p}{r} &\eqd& \metric{\opair{1}{0}}{\opair{-0.5}{0}} &=& 0.767324\cdots%   0.767324312
                         \\&\nle& 0.756406\cdots &=& 0.692330\cdots + 0.064076\cdots  % 0.756406531  0.692330146      0.064076385
                         \\&=&    \metric{\opair{1}{0}}{\opair{-0.5}{0.2}} &+& \metric{\opair{-0.5}{0.2}}{\opair{-0.5}{0}}
                         \\&\eqd& \metric{p}{q} + \metric{q}{r}
          \\\implies&\mc{4}{l}{\text{\prope{triangle inequality fails} in $\dspaceX$}}
        \end{array}$%

  \item Proof that $\dspaceX$ is \prope{not translation invariant}:\label{item:larc_ti}\footnote{\seessp{lab_larc_distances_R2.xlg}}
        %Consider the following case\ldots
        Let $r\eqd\opair{\sfrac{1}{2}}{\sfrac{1}{2}}$. Then\ldots
        \\\indentx
        $\begin{array}{c}%
          \includegraphics{../common/math/graphics/pdfs/larc_ti.pdf}%
        \end{array}$%
        \hspace{10mm}%
        $\begin{array}{rcl}%
            \metric{p+r}{q+r} &\eqd& \metric{\opair{1}{\frac{1}{2}}}{\opair{\frac{1}{2}}{1}} = 0.229009\cdots \neq \frac{1}{2}
                            \\&=& \metric{\opair{\frac{1}{2}}{0}}{\opair{0}{\frac{1}{2}}}
                            \\&\eqd& \metric{p}{q}
          \\\implies&\mc{2}{l}{\text{\prope{$\dspaceX$ is not translation invariant}}}
        \end{array}$%
        %d(( 1.000, 0.500),( 0.500, 1.000))= 0.229009993  ok
        %d(( 0.500, 0.000),( 0.000, 0.500))= 0.250000000  ok

  \item Proof that $\dspaceX$ is \prope{homogeneous}: 
        \\\indentx
        $\begin{array}{c}%
          \includegraphics{../common/math/graphics/pdfs/larc_hmg.pdf}%
        \end{array}$%
        \hspace{10mm}%
        $\begin{array}{M}%
           Let $r_{\alpha p}$ be the magnitude of $\alpha p\eqd\seqn{\alpha x_1, \alpha x_2, \cdots, \alpha x_\xN}$.\\
           Let $r_{\alpha q}$ be the magnitude of $\alpha q\eqd\seqn{\alpha y_1, \alpha y_2, \cdots, \alpha y_\xN}$.\\
           Let $\phi_\alpha$ be the \fncte{polar angle} between $\alpha p$ and $\alpha q$.\\
        \end{array}$%
        \\
    \begin{enumerate}
      \item If $r_p=0$ or $r_q=0$ or $\phi=0$ then $\distance{p}{q}$ is the \fncte{Euclidean metric}, which is \prope{homogeneous}.
            %Otherwise, $\distance{p}{q}$ is calculated using $\fR(p,q)$ \ldots

      \item lemmas: \label{item:larc_hmg}
        $\begin{array}[t]{rc>{\ds}l c>{\ds}l c>{\ds}l}
           r_{\alpha p} &\eqd& \brp{\sum_1^\xN \brs{\alpha x_n}^2}^\frac{1}{2} &=& \abs{\alpha}\sum_1^\xN x_n^2 &\eqd& \abs{\alpha}r_p \\
           r_{\alpha q} &\eqd& \brp{\sum_1^\xN \brs{\alpha y_n}^2}^\frac{1}{2} &=& \abs{\alpha}\sum_1^\xN y_n^2 &\eqd& \abs{\alpha}r_q \\
           \phi_\alpha  &\eqd& \arccos\brp{\frac{1}{r_{\alpha p}r_{\alpha q}}\sum_{n=1}^\xN [\alpha x_n][\alpha y_n]}
                        &=   & \arccos\brp{\frac{1}{r_pr_q}\sum_{n=1}^\xN x_n y_n}
                        &\eqd& \phi\\
           \fr(\theta;\alpha p,\alpha q) 
                        &\eqd& \brp{\frac{r_{\alpha q}-r_{\alpha p}}{\phi_\alpha}}\theta + r_{\alpha p} 
                        &=   & \brp{\frac{\alpha r_q-\alpha r_p}{\phi}}\theta + \alpha r_p 
                        &=   & \alpha\fr(\theta;p,q)
        \end{array}$

      \item If $\distance{p}{q}$ is \emph{not} the \fncte{Euclidean metric} then \ldots
        \begin{align*}
          \pi\distance{\alpha p}{\alpha q} 
            &\eqd \fR(\alpha p,\alpha q)
            &&    \text{by definition of $\distancen$ \xref{def:larc}}
          \\&\eqd \int_0^{\phi_\alpha} \sqrt{ \brs{\fr(\theta;\alpha p,\alpha q)}^2 + \brs{\deriv{\fr(\theta;\alpha p,\alpha q)}{\theta}}^2} \dth
            &&    \text{by definition of $\fR$ \xref{def:larc}}
          \\&=    \int_0^\phi \sqrt{\brs{\alpha\fr(\theta;p,q)}^2 + \brs{\opddth \alpha\fr(\theta;p,q)}^2} \dth
            &&    \text{by \pref{item:larc_hmg}}
          \\&=    \abs{\alpha}\int_0^\phi \sqrt{\brs{\fr(\theta;p,q)}^2 + \brs{\opddth\fr(\theta;p,q)}^2} \dth
            &&    \text{by \prope{linearity} of $\int_0^\phi\dth$ operator}
          \\&\eqd \abs{\alpha}\fR(p,q)
            &&    \text{by definition of $\fR$ \xref{def:larc}}
        \end{align*}
    \end{enumerate}

  \item Proof that $\distancen$ does \emph{not} induce a norm on $\setX$:
        This follows directly from \pref{item:larc_ti} and \prefpp{thm:vsn_d2norm}.

  \item Proof that \structe{ball}s \xref{def:ball} in $\distancen$ are in general \prope{not convex} \xref{def:convex}:
        \\This is demonstrated graphically in \prefpp{fig:larc} and \prefpp{fig:larcR3}.
        \\For an algebraic demonstration, consider the following:\footnote{\seessp{lab_larc_distances_R2.xlg}}
    \begin{enumerate}
      \item Let $\ball{\opair{0}{1}}{1}$ be the \structe{unit ball} in $\dspace{\R^2}{\distancen}$ 
            centered at $\opair{0}{1}$.

      \item Let $p\eqd\opair{-0.70}{-1.12}$, $q\eqd\opair{0.70}{-1.12}$, $r\eqd\opair{0}{-1.12}$,
            and $\lambda=\sfrac{1}{2}$. \label{item:larc_defs}

      \item Then \label{item:larc_pqr}
        $\begin{array}[t]{rclclD}
            \distance{\opair{0}{1}}{p} &=& 0.959536\cdots<1 &\implies& p\in   \ball{\opair{0}{1}}{1} & and
          \\\distance{\opair{0}{1}}{q} &=& 0.959536\cdots<1 &\implies& q\in   \ball{\opair{0}{1}}{1} & \scshape but
          \\\distance{\opair{0}{1}}{r} &=& 1.060688\cdots>1 &\implies& r\notin\ball{\opair{0}{1}}{1}
        \end{array}$
        %d(( 0.000, 1.000),(-0.700,-1.120))= 0.9595367064 ~=  0.9595364335  ok
        %d((-0.700,-1.120),( 0.000, 1.000))= 0.9595367064 ~=  0.9595364335  ok
        %d(( 0.000, 1.000),( 0.700,-1.120))= 0.9595367064 ~=  0.9595364335  ok
        %d(( 0.700,-1.120),( 0.000, 1.000))= 0.9595367064 ~=  0.9595364335  ok
        %d(( 0.000, 1.000),( 0.000,-1.120))= 1.0606887311 ~=  1.0606882938  ok
        %d(( 0.000,-1.120),( 0.000, 1.000))= 1.0606887311 ~=  1.0606882938  ok
        %
        % d(( 0.000, 0.500),(-0.350,-0.560))= 0.4797683532 ~=  0.4797682168  ok
        % d((-0.350,-0.560),( 0.000, 0.500))= 0.4797683532 ~=  0.4797682168  ok
        % d(( 0.000, 0.500),( 0.350,-0.560))= 0.4797683532 ~=  0.4797682168  ok
        % d(( 0.350,-0.560),( 0.000, 0.500))= 0.4797683532 ~=  0.4797682168  ok
        % d(( 0.000, 0.500),( 0.000,-0.560))= 0.5303443655 ~=  0.5303441469  ok
        % d(( 0.000,-0.560),( 0.000, 0.500))= 0.5303443655 ~=  0.5303441469  ok
        %
        %0.959536706
        %0.959536706
        %1.060688731
         
      \item This implies that the set $\ball{\opair{0}{1}}{1}$ is \prope{not convex} because
        \begin{align*}
          \lambda p + (1-\lambda)q
            &\eqd \frac{1}{2}\opair{-0.70}{-1.12} + \brp{1-\frac{1}{2}}\opair{0.70}{-1.12}
            && \text{by \pref{item:larc_defs}}
          \\&= \opair{0}{-1.12}
           %&& \text{by \prefp{def:seqalg}}
          \\&\eqd r
            && \text{by \pref{item:larc_defs}}
          \\&\notin \ball{\opair{0}{1}}{1}
            && \text{by \pref{item:larc_pqr}}
          \\&\implies \text{the set $\ball{\opair{0}{1}}{1}$ is \prope{not convex}}
            && \text{by \prefp{def:convex}}
        \end{align*}
  \end{enumerate}
 
\end{enumerate}
\end{proof}

%---------------------------------------
\begin{remark}[\exmd{Lagrange arc distance versus Euclidean metric}]
%---------------------------------------
As is implied by the metric balls illustrated in \prefpp{fig:larc} and \prefpp{fig:larcR3},
the \fncte{Lagrange arc distance} $\metricn$ and \fncte{Euclidean metric} $\metrican$ are similar in the sense that they 
often lead to the same results\footnote{
  For empirical evidence of this, see \citeP{greenhoe2015ssp}.
  }
in determining which of the two points $q_1$ or $q_2$ is ``closer" to a point $p$.
But in some cases the two metrics lead to two different results.
One such case is illustrated as follows:\footnote{\seessp{lab_larc_distances_R2.xlg}}
  \\\indentx
  $\begin{array}{c}%
    \includegraphics{../common/math/graphics/pdfs/larc_euclid_compare.pdf}%
  \end{array}$%
  \hspace{10mm}%
  $\begin{array}{rclcl}%
      \metric {p}{q_1} &\eqd& \metric{\opair{1}{0}}{\opair{-0.5}{0}}     &=& 0.767324\cdots%
    \\\metric {p}{q_2} &\eqd& \metric{\opair{1}{0}}{\opair{-0.5}{0.75}}  &=& 0.654039\cdots%
    \\\metrica{p}{q_1} &\eqd& \metrica{\opair{1}{0}}{\opair{-0.5}{0}}    &=&    1.5%
    \\\metrica{p}{q_2} &\eqd& \metrica{\opair{1}{0}}{\opair{-0.5}{0.75}} &=& \sqrt{(1.5)^2+(0.75)^2} = 1.677050\cdots%  1.6770509831248422723068802515485
  \end{array}$%
\\\begin{tabular}{r@{\hspace{1ex}}l}
  That is,&$q_2$ is closer than $q_1$ to $p$ with respect to the \fncte{Lagrange arc distance},\\
  but     &$q_1$ is closer than $q_2$ to $p$ with respect to the \fncte{Euclidean metric}.
\end{tabular}
%Theorem 3.10: Lagrange arc distance versus Euclidean metric
%d(( 1.000, 0.000),(-0.500, 0.000))= 0.7673243120 ~=  0.7673239676  ok
%d((-0.500, 0.000),( 1.000, 0.000))= 0.7673243120 ~=  0.7673239676  ok
%d(( 1.000, 0.000),(-0.500, 0.750))= 0.6540398605 ~=  0.6540397329  ok
%d((-0.500, 0.750),( 1.000, 0.000))= 0.6540398605 ~=  0.6540397329  ok
\end{remark}

%%---------------------------------------
%\begin{remark}[\exmd{Lagrange arc distance versus elliptic arc distance}]
%%---------------------------------------
%Let $\metricn$ be the \fncte{Lagrange arc distance} \xref{def:larc} 
%and $\metrican$ be an \fncte{elliptic arc length} scaled by $\sfrac{1}{\pi}$.
%In the figure below, \structe{Lagrange arc}s are illustrated with solid blue lines,
%and \structe{elliptic arc}s are illustrated with dashed red lines.
%And as illustrated below, the two co-incide when the arcs are circular arcs, but in general are different.
%  \\\indentx
%  $\begin{array}{c}%
%    \includegraphics{../common/math/graphics/pdfs/larc_earc_compare.pdf}%
%  \end{array}$%
%  \hspace{10mm}%
%  $\begin{array}{rclcl|rcl}%
%      \metric {p_1}{q} &\eqd& \metric{\opair{0}{1}  }{\opair{1}{0}} &=& 0.500000\cdots & \metrica{p_1}{q} &=& 0.500000\cdots \\
%      \metric {p_2}{q} &\eqd& \metric{\opair{0}{0.8}}{\opair{1}{0}} &=& 0.454498\cdots & \metrica{p_2}{q} &=& 0.451389\cdots \\
%      \metric {p_3}{q} &\eqd& \metric{\opair{0}{0.2}}{\opair{1}{0}} &=& 0.401115\cdots & \metrica{p_3}{q} &=& 0.334385\cdots 
%      %\metric {p_1}{q} &=& 0.500000000   & \metrica{p_1}{q} &=& 0.500000000  \\
%      %\metric {p_2}{q} &=& 0.454498972   & \metrica{p_2}{q} &=& 0.45138996389828049785132263054847              \\
%      %\metric {p_3}{q} &=& 0.401115719   & \metrica{p_3}{q} &=& 0.33438524526711829030362231606724  \\
%  \end{array}$%
%\end{remark}

%%---------------------------------------
%\begin{remark}[\exmd{Lagrange interpolation versus Newton interpolation}]
%\label{rem:lagnew}
%%---------------------------------------
%\fncte{Lagrange arc distance} \xref{def:larc} $\metric{p}{q}$ uses the \fncte{Lagrange polynomial} \xref{def:laginterp}
%to interpolate between $p$ and $q$.
%What if the \fncte{Newton polynomial} \xref{def:newinterp} was used instead of the \fncte{Lagrange polynomial}.
%As it turns out, both yield the exact same solution.
%\end{remark}
%\begin{proof}
%  \begin{enumerate}
%    \item The \fncte{Newton polynomial} for this interpolation is $\fr(\theta) = a + b(\theta-\theta_p)$.
%    \item We must solve for $a$ and $b$:
%      \\\indentx$\begin{array}{rcl clclcl}
%        \fr(\theta_p) &\eqd& r_p &=& a + b(\theta_p-\theta_p) &=& a                         &\implies& \boxed{a = r_p} \\
%        \fr(\theta_q) &\eqd& r_q &=& a + b(\theta_q-\theta_p) &=& r_p + b(\theta_q-\theta_p)&\implies& \boxed{b = \frac{r_q - r_p}{\theta_q-\theta_p}}
%      \end{array}$
%    \item Lastly, write an expression for $\fr(\theta)$:
%      \begin{align*}
%        \fr(\theta) 
%          &= a + b(\theta-\theta_p)
%         &&= \mcom{r_p}{$a$} + \mcom{\frac{r_q - r_p}{\theta_q-\theta_p}}{$b$}(\theta-\theta_p)
%         &&= r_p\frac{\theta_q-\theta_p}{\theta_q-\theta_p} + (r_q - r_p)\frac{\theta-\theta_p}{\theta_q-\theta_p}
%        \\&= r_p\brs{\frac{\theta_q-\theta_p}{\theta_q-\theta_p}-\frac{\theta-\theta_p}{\theta_q-\theta_p}} + r_q \frac{\theta-\theta_p}{\theta_q-\theta_p}
%         &&= r_p\frac{\theta_q-\theta}{\theta_q-\theta_p} + r_q \frac{\theta-\theta_p}{\theta_q-\theta_p}
%         &&= r_p\frac{\theta-\theta_q}{\theta_p-\theta_q} + r_q \frac{\theta-\theta_p}{\theta_q-\theta_p}
%      \end{align*}
%    \item Note that this is the same as the \fncte{Lagrange polynomial} for $\fr(\theta)$.
%  \end{enumerate}
%\end{proof}


%=======================================
\section{Examples}
%=======================================
\begin{figure}
  \centering%
  \gsize%
  \includegraphics{../common/math/graphics/pdfs/larc_metex.pdf}
  \hspace{10mm}
  $\begin{array}[b]{l @{\hspace{2pt}}r@{\hspace{2pt}}r l @{\hspace{2pt}}r@{\hspace{2pt}}r cl}
    \metricn((& 0,& 1),&(& 1,& 0)) &=& \sfrac{1}{2}    \\
    \metricn((& 0,& 1),&(&-1,& 0)) &=& \sfrac{1}{2}    \\
    \metricn((& 0,& 1),&(& 0,&-1)) &=& 1               \\
    \metricn((& 1,& 0),&(& 0,&-1)) &=& \sfrac{1}{2}    \\
    \metricn((& 1,& 0),&(&-1,& 0)) &=& 1               \\
    \metricn((&-1,& 0),&(& 0,&-1)) &=& \sfrac{1}{2}    \\
    \metricn((& 0,& 1),&(& 2,& 0)) &=& 0.8167968\cdots \\
    \metricn((& 0,& 1),&(& 0,&-2)) &=& 1.5346486\cdots \\
    \metricn((& 0,& 1),&(&-2,& 1)) &=& 0.6966032\cdots
  \end{array}$
  \caption{\fncte{Lagrange arc distance} examples in $\R^2$\label{fig:larcmet}}
\end{figure}
%Example 3.12: Lagrange arc distance in R^2
%d(( 0.000, 1.000),( 1.000, 0.000))= 0.5000000000 ~=  0.4999999486  ok
%d(( 0.000, 1.000),(-1.000, 0.000))= 0.5000000000 ~=  0.4999999486  ok
%d(( 0.000, 1.000),( 0.000,-1.000))= 1.0000000000 ~=  0.9999995888  ok
%d(( 1.000, 0.000),( 0.000,-1.000))= 0.5000000000 ~=  0.4999999486  ok
%d(( 1.000, 0.000),(-1.000, 0.000))= 1.0000000000 ~=  0.9999995888  ok
%d((-1.000, 0.000),( 0.000,-1.000))= 0.5000000000 ~=  0.4999999486  ok
%d(( 0.000, 1.000),( 2.000, 0.000))= 0.8167968887 ~=  0.8167967763  ok
%d(( 0.000, 1.000),( 0.000,-2.000))= 1.5346486241 ~=  1.5346479353  ok
%d(( 0.000, 1.000),(-2.000, 1.000))= 0.6966032041 ~=  0.6966031412  ok


%---------------------------------------
\begin{example}[\exmd{Lagrange arc distance in $\R^2$}]
\label{ex:larcR2}
%---------------------------------------
\prefpp{fig:larcmet} illustrates the \fncte{Lagrange arc distance} on some pairs of points in $\R^2$.\footnote{\seessp{lab_larc_distances_R2.xlg}}
\end{example}


%\begin{figure}
%  \gsize%
%  \centering%
%  \psset{unit=10mm, plotstyle=dots, linecolor=blue, linewidth=1pt, dotsize=0.5pt}
%  \begin{tabular}{ccc}
%    \begin{pspicture}(-2.5,-2.5)(2.5,2.5)%
%      \psaxes[linecolor=axis](0,0)(-2.5,-2.5)(2.5,2.5)%
%      \pscircle[linecolor=red](0,0){1.5708}%
%      \psdot[linecolor=red,dotsize=3pt](0,0)%
%      %\psline[linecolor=red](-1,0.707)(2,0.707)%
%      %\psline[linecolor=red](0.707,-1)(0.707,2)%
%      \fileplot{../common/symseq/graphics/larc0.dat}%
%    \end{pspicture} 
%   &
%    \begin{pspicture}(-2.5,-2.5)(2.5,2.5)%
%      \psaxes[linecolor=axis](0,0)(-2.5,-2.5)(2.5,2.5)%
%      \pscircle[linecolor=red](0.707,0.707){1.5708}%
%      \psdot[linecolor=red,dotsize=3pt](0.707,0.707)%
%      \fileplot{../common/symseq/graphics/relm45.dat}%
%    \end{pspicture} 
%   &
%    \begin{pspicture}(-2.5,-2.5)(2.5,2.5)%
%      \psaxes[linecolor=axis](0,0)(-2.5,-2.5)(2.5,2.5)%
%      \pscircle[linecolor=red](0.707,0.707){1.5708}%
%      \psdot[linecolor=red,dotsize=3pt](0.707,0.707)%
%      \fileplot{../common/symseq/graphics/test.dat}%
%    \end{pspicture} 
%  \end{tabular}
%  \caption{Rotating elliptic metric unit balls ($\frac{\pi}{2}$-scaled Euclidean metric balls superimposed) \label{fig:relm}}
%\end{figure}
%%---------------------------------------
%\begin{example}[\exmd{rotating elliptic metric}]
%%---------------------------------------
%The following metric is compatible with the \structe{real die} structure.
%%Let $r_p\eqd\sqrt{x^2+y^2}$ and 
%Let $\opair{r_p}{\theta_p}$ be the polar co-ordinates of a point $p$ in $\R^2$.
%Let $R(p,q)$ be the length of the arc from a point $p$ to a point $q$ in $\R^2$
%defined by the function 
%$r(\theta) \eqd r_p\frac{\theta-\theta_q}{\theta_p-\theta_q} + r_q\frac{\theta-\theta_p}{\theta_q-\theta_p}$.
%\exbox{
%  \metric{p}{q} = \brbl{\begin{array}{lM}
%    ... & ...
%  \end{array}}
%  }
%Some unit balls in terms of this metric are illustrated in \prefp{fig:relm}.
%\end{example}


%---------------------------------------
\begin{example}[\exmd{Lagrange arc distance in $\R^3$}]
\label{ex:larcR3}
%---------------------------------------
Some examples of Lagrange arc distances in $\R^3$
are given in \prefpp{tbl:larcR3}.\footnote{\seessp{lab_larc_distances_R3.xlg}}
%In addition, some unit balls in $\R^3$ induced by the 
%Lagrange arc distance and scaled Euclidean metric are illustrated in \prefpp{fig:larcR3}.
%The value $\theta$ is the same as defined in \prefpp{ex:larcR2}.
\end{example}
\begin{table}
  \centering
   \begin{tabular}[b]{|M@{\hspace{2pt}} O@{\hspace{2pt}}O@{\hspace{2pt}}O 
                        M@{\hspace{2pt}} O@{\hspace{2pt}}O@{\hspace{2pt}}O 
                        NM |MNM|O|}
    \hline
    \metricn((& 0,& 1,&  0),&(& 1,  &  0,    & 0))                &=& \sfrac{1}{2}   & \phi &=& \sfrac{\pi}{2}  &   90^\circ\\
    \metricn((& 0,& 1,&  0),&(& 0,  &  0,    & 1))                &=& \sfrac{1}{2}   & \phi &=& \sfrac{\pi}{2}  &   90^\circ\\
    \metricn((& 0,& 1,&  0),&(& 0,  &  0,    &-1))                &=& \sfrac{1}{2}   & \phi &=& \sfrac{\pi}{2}  &   90^\circ\\
    \metricn((& 0,& 1,&  0),&(&-1,  &  0,    & 0))                &=& \sfrac{1}{2}   & \phi &=& \sfrac{\pi}{2}  &   90^\circ\\
    \metricn((& 0,& 1,&  0),&(& 0,  & -1,    & 0))                &=&        1       & \phi &=& \pi             &  180^\circ\\
    \metricn((& 1,& 0,&  0),&(& 0,  &  0,    & 1))                &=& \sfrac{1}{2}   & \phi &=& \sfrac{\pi}{2}  &   90^\circ\\
    \metricn((& 1,& 0,&  0),&(& 0,  &  0,    &-1))                &=& \sfrac{1}{2}   & \phi &=& \sfrac{\pi}{2}  &   90^\circ\\
    \metricn((& 1,& 0,&  0),&(&-1,  &  0,    & 0))                &=&        1       & \phi &=& \pi             &  180^\circ\\
    \metricn((& 1,& 0,&  0),&(& 0,  & -1,    & 0))                &=& \sfrac{1}{2}   & \phi &=& \sfrac{\pi}{2}  &   90^\circ\\
    \metricn((& 0,& 0,&  1),&(& 0,  &  0,    &-1))                &=&        1       & \phi &=& \pi             &  180^\circ\\
    \metricn((& 0,& 0,&  1),&(&-1,  &  0,    & 0))                &=& \sfrac{1}{2}   & \phi &=& \sfrac{\pi}{2}  &   90^\circ\\
    \metricn((& 0,& 0,&  1),&(& 0,  & -1,    & 0))                &=& \sfrac{1}{2}   & \phi &=& \sfrac{\pi}{2}  &   90^\circ\\
    \metricn((& 0,& 0,& -1),&(&-1,  &  0,    & 0))                &=& \sfrac{1}{2}   & \phi &=& \sfrac{\pi}{2}  &   90^\circ\\
    \metricn((& 0,& 0,& -1),&(& 0,  & -1,    & 0))                &=& \sfrac{1}{2}   & \phi &=& \sfrac{\pi}{2}  &   90^\circ\\
    \metricn((&-1,& 0,&  0),&(& 0,  & -1,    & 0))                &=& \sfrac{1}{2}   & \phi &=& \sfrac{\pi}{2}  &   90^\circ\\
    \metricn((& 0,& 1,&  0),&(& 2,  &  0,    & 0))                &=& 0.816796\cdots & \phi &=& \sfrac{\pi}{2}  &   90^\circ\\
    \metricn((& 0,& 1,&  0),&(& 0,  & -2,    & 0))                &=& 1.534648\cdots & \phi &=& \pi             &  180^\circ\\
    \metricn((& 0,& 1,&  0),&(&-2,  &  1,    & 0))                &=& 0.696603\cdots & \phi &\eqa& 1.107        &   63^\circ\\%1.107149  63.43495^\circ
    \metricn((& 0,& 1,&  0),&(&-1,  &  0,    &-1))                &=& 0.617920\cdots & \phi &=& \pi             &   90^\circ\\  %1.107149  63.43495^\circ
    \metricn((& 1,& 1,&  1),&(&-\sfrac{1}{2},&\sfrac{1}{4}, &-2)) &=& 1.366268\cdots & \phi &\eqa&2.2466        &   128.72^\circ\\  %1.107149  63.43495^\circ
    \hline
  \end{tabular}
  \caption{Some examples of Lagrange arc distances in $\R^3$ (see \prefp{ex:larcR3}) \label{tbl:larcR3}}
\end{table}

\begin{figure}
  \gsize%
  \centering%
  \includegraphics{../common/math/graphics/pdfs/larc_R2_balls.pdf}%
  \caption{\fncte{Lagrange arc distance} unit balls, and dashed \fncte{$\frac{1}{\pi}$-scaled Euclidean metric} unit balls \label{fig:larc}}
\end{figure}
%---------------------------------------
\begin{example}[\exmd{Lagrange arc distance balls in $\R^2$}]
\label{ex:larcR2balls}
%---------------------------------------
Some unit balls in $\R^2$ in with respect to the \fncte{Lagrange arc distance} are illustrated in \prefpp{fig:larc}.
%There, the quantity $\theta$ represents the minimal angular measure between the vector extending 
%from the origin to $p$ and the vector extending from the origin to $q$ in computing the 
%metric $\metric{p}{q}$.
\end{example}

\begin{figure}
  \centering%
  \begin{tabstr}{0.75}\begin{tabular}{cc}
    %\hline
        \includegraphics{../common/math/graphics/pdfs/larc_ball(0_0_0).pdf} 
      & \includegraphics{../common/math/graphics/pdfs/larc_ball(0_0_-1).pdf} 
      \\centered at $\otriple{0}{0}{0}$
      & centered at $\otriple{0}{0}{-1}$
    \\
        \includegraphics{../common/math/graphics/pdfs/larc_ball(0_0_-2).pdf} 
      & \includegraphics{../common/math/graphics/pdfs/larc_ball(0_0_-3).pdf} 
      \\centered at $\otriple{0}{0}{-2}$
      & centered at $\otriple{0}{0}{-3}$
    \\
        \includegraphics{../common/math/graphics/pdfs/larc_ball(0_0_-5).pdf} 
      & \includegraphics{../common/math/graphics/pdfs/larc_ball(0_0_-10).pdf} 
      \\centered at $\otriple{0}{0}{-5}$
      & centered at $\otriple{0}{0}{-10}$
    %\\
    %\includegraphics{../common/math/graphics/pdfs/larc_ball(0_0_-2).pdf}%
    %\\ unit Lagrange arc distance ball 
    %\includegraphics{../common/math/graphics/pdfs/larc_ball(00_00_05).pdf}%
    %&
    %\includegraphics{../common/math/graphics/pdfs/ae_ball(00_00_05).pdf}%
    %\\ unit Lagrange arc distance ball 
    % & $\sfrac{2}{\pi}$-scaled Euclidean metric ball
    %\\\hline
  \end{tabular}\end{tabstr}
  \caption{\structe{unit Lagrange arc distance ball}s in $\R^3$ \label{fig:larcR3}}
\end{figure}
%---------------------------------------
\begin{example}[\exmd{Lagrange arc distance balls in $\R^3$}]
\label{ex:larcR3balls}
%---------------------------------------
Some unit balls in $\R^3$ with respect to the \fncte{Lagrange arc distance} are illustrated in \prefpp{fig:larcR3}.
\end{example}





