%============================================================================
% Daniel J. Greenhoe
%============================================================================
%\chapter{Order and metric compatible symbolic sequence processing}\label{chp:ssp}
%\chapter{Stochastic processing using mapping to R\raisebox{0.5ex}{N}}\label{chp:ssp}
%\chapter{Stochastic processing on R\raisebox{0.5ex}{N}}\label{chp:ssp}
%\chapter[SSP on R\raisebox{0.65ex}{N}]{Symbolic sequence processing on R\raisebox{0.65ex}{N}}\label{chp:ssp}
\chapter{Symbolic sequence processing on R\raisebox{0.65ex}{N}}\label{chp:ssp}
\qboxnps
  {\href{http://en.wikipedia.org/wiki/Aristotle}{Aristotle}
   \href{http://www-history.mcs.st-andrews.ac.uk/Timelines/TimelineA.html}{384 BC -- 322 BC},
   \href{http://www-history.mcs.st-andrews.ac.uk/BirthplaceMaps/Places/Greece.html}{Greek philosopher}
   \index{Aristotle}
   \index{quotes!Aristotle}
   \footnotemark
  }
  {../common/people/small/aristotle.jpg}
  {\ldots those who assert that the mathematical sciences
     say nothing of the beautiful or the good are in error.
     For these sciences say and prove a great deal about them;
     if they do not expressly mention them, but prove attributes
     which are their results or definitions, it is not true that they tell
     us nothing about them.
     The chief forms of beauty are order and symmetry and definiteness,
     which the mathematical sciences demonstrate in a special degree.}
  \citetblt{
    %quote:  \citerc{aristotle}{paragraphs 34--35?} \\
    quote:  \citerc{aristotle_metaphysics}{Book XIII Part 3} \\
    %        \url{http://en.wikiquote.org/wiki/Aristotle} \\
    %image:  \url{http://en.wikipedia.org/wiki/Aristotle}
    image:  \url{http://upload.wikimedia.org/wikipedia/commons/9/98/Sanzio_01_Plato_Aristotle.jpg}
    }

 %%============================================================================
% Daniel J. Greenhoe
% LaTeX File
%============================================================================


%======================================
\chapter*{Introduction}
%======================================


%======================================
\section{Description}
%======================================
This text describes two independent but related operations: \emph{Detection} of objects 
and \emph{Classification} of those objects.
The scope of these objects is that they are \emph{buried}. 
This scope qualitatively suggests that the objects are hidden from human vision 
and are in intimate contact with materials typically found in the materials constituting the earth such as soil, sand, clay,
rock, and/or moisture.

%======================================
\section{Method}
%======================================
This text follows the approach of \emph{Model-Based Signal Processing}\footnote{
  \citer{candy1985}, \citer{candy1992}, \citer{candy2005}, \citer{marston2006} }
which implies three components of the detection and classification operator pair:
\begin{enume}
  \item A \fncte{cost function}---an operation by which the effectiveness of the operations can be measured.
  \item A \structe{model}---a mathematical structure representing the entire system including the objects and their surrounding channel.
  \item An \fncte{algorithm}---an operator on the model which seeks to optimize (minimize) the cost function.
\end{enume}

%======================================
\section{System}
%======================================
The system consists of a channel and an excitation source.
The excitation source is a direct-coupled "shaker" or acoustically-coupled speaker
that excites a channel that may or may not contain an object.
The excitation signal propagates through the channel in the form of four "waves":
\begin{enume}
  \item P-waves
  \item S-waves
  \item Rayleigh waves
  \item Love waves
\end{enume}
The resulting response at any given location on the surface of the channel (e.g. surface of the ground)
can be measured using an accelerometer, an ultrasonic vibrometer, and/or a laser vibrometer.

%======================================
\section{Model}
%======================================
The interaction between earth materials and a \prope{compliant} foreign object shell 
has been demonstrated to be \prope{nonlinear}; this property in turn can be and has been exploited 
for the purposes of detection and classification.\footnote{
  \cite{donskoy1998}}




  %============================================================================
% Daniel J. Greenhoe
% XeLaTeX file
%============================================================================
%=======================================
\section{Outcome subspace sequences}
%=======================================
%=======================================
%\subsection{Correlation operation on outcome subspace sequences}
%=======================================
%=======================================
\subsection{Definitions}
%=======================================
%---------------------------------------
\begin{definition}
\label{def:ocsseqmetric}
%---------------------------------------
Let $\Dom_1$ and $\Dom_2$ be \structe{convex subset}s \xref{def:convex} of $\Z$.\\
Let $\Dom\eqd\intoo{\meetop\Dom_1-\joinop\Dom_2-1}{\joinop\Dom_1-\meetop\Dom_2+1}.$
Let $\seq{x_n}{\Dom_1}$ and $\seq{y_n}{\Dom_2}$ be \structe{sequence}s over an \structe{outcome subspace} $\ocsD$.
\defboxp{
  The \fnctd{outcome subspace sequence metric} $\metrica{\seqn{x_n}}{\seqn{y_n}}$ is defined as
  \\\indentx
  $\ds\metrica{\seqn{x_n}}{\seqn{y_n}} \eqd \sum_{n\in\Dom} \ff(n)$
  \quad where \quad
  $
  \ff(n) \eqd \brb{\begin{array}{lM}
    \ocsd(x_n,y_n) & if $n\in   \Dom_1$ and $n\in   \Dom_2$\\
    1              & if $n\in   \Dom_1$ but $n\notin\Dom_2$\\
    1              & if $n\notin\Dom_1$ but $n\in   \Dom_2$\\
    0              & otherwise
  \end{array}}\qquad{\scy\forall n\in\Dom}
  $
  }
\end{definition}

%---------------------------------------
\begin{proposition}
%---------------------------------------
Let $\seq{x_n}{\Dom_1}$ and $\seq{y_n}{\Dom_2}$ be \structe{sequence}s over an \structe{outcome subspace} $\ocsD$.\\
Let $\metrican$ be the \fncte{outcome subspace sequence metric}.
\propbox{
  \Dom_1\seti\Dom_2\neq\emptyset \quad\implies\quad \text{$\metrican$ is a \fncte{metric}}
  }
\end{proposition}
\begin{proof}
This follows from the \thme{Fr{\'e}chet product metric} \xref{prop:fpm}.
In particular, $\metrican$ is a sum of metrics that include
the metrics $\ocsd(x_n,y_n)$ and the \fncte{discrete metric} \xref{def:dmetric}.
%This follows directly from either of the following two \fncte{product metric}s:
%\begin{liste}
%  \item Proof using \thme{Fr{\'e}chet product metric}:
%    \begin{align*}
%      \metrica{\seqn{x_n}}{\seqn{y_n}} 
%        &\eqd \sum_{n\in\Dom} \ocsd(x_n,y_n)
%        && \text{by \prefp{def:ocsseqmetric}}
%      \\&= \sum_{n\in\Dom} \alpha_n\ocsd(x_n,y_n)
%        && \text{for $\alpha_1=\alpha_2=\cdots=\alpha_\xN=1$}
%      \\&= \sum_{n\in\Dom} \alpha_n\ocsd_n(x_n,y_n)
%        && \text{for $\ocsd_1=\ocsd_2=\cdots=\ocsd_\xN=\ocsd$}
%      \\&\implies\text{$\metrican$ is a \fncte{metric}}
%        && \text{by \prope{Fr{\'e}chet product metric} \xref{prop:fpm}}
%    \end{align*}
%
%  \item Proof using \thme{Power mean metric}:
%    \begin{align*}
%      \metrica{\seqn{x_n}}{\seqn{y_n}} 
%        &\eqd \sum_{n\in\Dom} \ocsd(x_n,y_n)
%        && \text{by \prefp{def:ocsseqmetric}}
%      \\&= \xN\sum_{n\in\Dom} \lambda_n\ocsd(x_n,y_n)
%        && \text{for $\lambda_1=\lambda_2=\cdots=\lambda_\xN=\frac{1}{\xN}$}
%      \\&\implies\text{$\metrican$ is a \fncte{metric}}
%        && \text{by \prope{Power mean metric} \xref{prop:pmm}}
%      \\&&&\text{\qquad and \prope{$\alpha$-scaled metric} \xref{prop:asm}}
%    \end{align*}
%
%\end{liste}
\end{proof}

In standard signal processing, the \ope{autocorrelation} of a \structe{sequence} $\seqnD{x_n}$ 
is another \structe{sequence} $\seqnD{y_n}$ defined as 
$y_n\eqd\sum_{m\in\Z}x_m x_{m-n}$.
However, this definition requires that the sequence $\seqnD{x_n}$ be constructed over a \structe{field}.
In an \structe{outcome subspace sequence}, we in general do not have a \structe{field};
for example, in a \structe{die outcome subspace}, the expressions $\dieA+\dieB$ and $\dieA\times\dieB$ are undefined.
This paper offers an alternative definition (next) for \ope{autocorrelation} that uses the \fncte{distance} $\ocsd$ 
and that does not require a \structe{field}. 
%The autocorrelation function is used in \prefpp{ex:rdie_lp} and illustrated in \prefpp{fig:rdie_auto}.
%---------------------------------------
\begin{definition}
\label{def:ocsRxx}
%---------------------------------------
Let $\seqnD{x_n}$ and $\seqnD{y_n}$ be \fncte{sequence}s 
%\structe{outcome subspace sequence}s with \prope{finite support} \xref{def:support}
over the \structe{outcome subspace} $\ocsD$.
%Let $\xN$ be any whole number such that $\xN\ge\max(\seto{\support\seqn{x_n}},\seto{\support\seqn{y_n}})$, 
%where $\support\seqn{x_n}$ is the \ope{support} of $\seqn{x_n}$
%and   $\seto{\setX}$ is the \fncte{order} of a finite set $\setX$ \xref{def:seto}.
%Let $\metricbn$ be the \fncte{discrete metric} \xref{def:dmetric}.
%Let $\xN\eqd\sum_{n\in\Dom}\metricb{x_n}{\ocsz}$.
\\Let $\metrica{\seqn{x_n}}{\seqn{y_n}}$ be the \fnctd{outcome subspace sequence metric} \xref{def:ocsseqmetric}.
\\The \fnctd{cross-correlation} $\Rxy(n)$ of $\seqn{x_n}$ and $\seqn{y_n}$ and 
the \fnctd{autocorrelation}  $\Rxx(n)$ of $\seqn{x_n}$ 
are defined as
\defbox{\begin{array}{rclD}
  \Rxy(n) &\eqd& \ds - \sum_{m\in\Z}\metrica{\seqn{x_{m-n}}}{\seqn{y_m}} &\qquad (\fnctd{cross-correlation} $\Rxy(n)$ of $\seqn{x_n}$ and $\seqn{y_n}$)\\
  \Rxx(n) &\eqd& \ds - \sum_{m\in\Z}\metrica{\seqn{x_{m-n}}}{\seqn{x_m}} &\qquad (\fnctd{autocorrelation}  $\Rxx(n)$ of $\seqn{x_n}$)
\end{array}}\\
Moreover, the \opd{$\xM$-offset autocorrelation} of $\seqn{x_n}$ and $\seqn{y_n}$ is here defined 
as $\Rxx(n)+\xM$ \xref{def:axn}.
\end{definition}

%=======================================
\subsection{Examples of symbolic sequence statistics}
%=======================================
%---------------------------------------
\begin{example}[\exmd{fair die sequence}]
\label{ex:fdie_sha}
%---------------------------------------
Consider the pseudo-uniformly distributed \structe{fair die} \xref{def:fdie} 
sequence generated by the C code\footnotemark 
%\begin{lstlisting}
%\\\indentx\lstinline{#include<stdlib.h>} \ldots \lstinline!srand(0x5EED); for(n=0; n<N; n++){x[n] = 'A' + rand()%6;}!\\
%
\\\begin{minipage}{85mm}%
\begin{lstlisting}
#include<stdlib.h>
...
srand(0x5EED);
for(n=0; n<N; n++){x[n] = 'A' + rand()%6;}
\end{lstlisting}
\end{minipage}%
\hspace{10mm}%
\begin{tabular}{lclcl}
  where & \lstinline!'A'! &represents& $\dieA$ &,
     \\ & \lstinline!'B'! &represents& $\dieB$ &,
     %\\ & \lstinline!'C'! &represents& $\dieC$ &,
     %\\ & \lstinline!'D'! &represents& $\dieD$ &,
     %\\ & \lstinline!'E'! &represents& $\dieE$ &, and
     \\ & $\vdots$
     \\ & \lstinline!'F'! &represents& $\dieF$ &.
\end{tabular}
\\
%where \lstinline{'A'} represents $\dieA$, \lstinline{'B'} represents $\dieB$, etc.
The resulting sequence is partially displayed here:
   \\\includegraphics{ssp/plots/fdie_5eed_51_seq.pdf}\\
\footnotetext{For a more complete source code listing, see \prefpp{sec:src_die}}
This sequence constrained to a length of $\xN=2667\times6=16002$ elements is approximately \prope{uniformly distributed} 
and \prope{uncorrelated}, as illustrated next:
%calculated over %$(16000-16)-(16-1)+1=15970$ elements:
\\\begin{tabular}{|>{\scs}c|>{\scs}c|}
     \hline
     \includegraphics{ssp/plots/fdie_5eed_16002_histo.pdf}%
    &\includegraphics{ssp/plots/fdie_5eed_16002_auto.pdf}
    \\histogram & \ope{$2\xN$-offset autocorrelation}
   \\\hline
\end{tabular}
\end{example}

%---------------------------------------
\begin{example}[\exmd{real die sequence}]
\label{ex:rdie_sha}
%---------------------------------------
Consider the pseudo-uniformly distributed \structe{real die} \xref{def:rdie}
sequence generated as in \prefpp{ex:fdie_sha},
but with the real die metric rather than the fair die metric.
This change will not affect the distribution of the sequence, but it does affect 
the autocorrelation, as illustrated next:
\\\begin{tabular}{|>{\scs}c|>{\scs}c|}
     \hline
     \includegraphics{ssp/plots/rdie_5eed_16002_histo.pdf}%
     &\includegraphics{ssp/plots/rdie_5eed_16002_auto.pdf}
   \\\hline
\end{tabular}
\end{example}


%---------------------------------------
\begin{example}[\exmd{spinner sequence}]
\label{ex:spinner_sha}
%---------------------------------------
Consider the pseudo-uniformly distributed \structe{spinner} \xref{def:spinner} 
sequence generated by the C code\footnotemark 
%\begin{lstlisting}
%\\\indentx\lstinline{#include<stdlib.h>} \ldots \lstinline!srand(0x5EED); for(n=0; n<N; n++){x[n] = 'A' + rand()%6;}!\\
%
\\\begin{minipage}{85mm}%
\begin{lstlisting}
#include<stdlib.h>
...
srand(0x5EED);
for(n=0; n<N; n++){x[n] = 'A' + rand()%6;}
\end{lstlisting}
\end{minipage}%
\hspace{10mm}%
\begin{tabular}{lclcl}
  where & \lstinline!'A'! &represents& $\spinA$ &,
     \\ & \lstinline!'B'! &represents& $\spinB$ &,
     %\\ & \lstinline!'C'! &represents& $\spinC$ &,
     %\\ & \lstinline!'D'! &represents& $\spinD$ &,
     %\\ & \lstinline!'E'! &represents& $\spinE$ &, and
     \\ & $\vdots$
     \\ & \lstinline!'F'! &represents& $\spinF$ &.
\end{tabular}
\\
%where \lstinline{'A'} represents $\dieA$, \lstinline{'B'} represents $\dieB$, etc.
The resulting sequence is partially displayed here:
   \\\includegraphics{ssp/plots/spin_5eed_51_seq.pdf}\\
\footnotetext{For a more complete source code listing, see \prefpp{sec:src_spinner}}
This sequence is in essence identical to the fair die sequence \xref{ex:fdie_sha} and real die sequence \xref{ex:rdie_sha}
and thus yields what is essentially an identical histogram.
But because the metric is different, the autocorrelation  is also different.
In particular, because the nodes of the spinner metric are on average farther apart with respect to the spinner metric,
the sequence is less correlated (with respect to the metric), as illustrated next:
\\\begin{tabular}{|>{\scs}c|>{\scs}c|}
     \hline
     \includegraphics{ssp/plots/spin_5eed_16002_histo.pdf}%
    &\includegraphics{ssp/plots/spin_5eed_16002_auto.pdf}
   \\\hline
\end{tabular}
\end{example}


%---------------------------------------
\begin{example}[\exmd{weighted real die sequence}]
\label{ex:wrdie_sha}
%---------------------------------------
Consider the non-uniformly distributed \structe{weighted real die} \xref{def:wrdie} sequence with 
    \\\indentx$\psp(\dieE)=0.75$ and $\psp(\dieA)=\psp(\dieB)=\psp(\dieC)=\psp(\dieD)=\psp(\dieF)=0.05$,\\
    generated by the C code\footnote{For a more complete source code listing, see \prefpp{sec:src_rdie}}
\\\begin{minipage}{85mm}%
\begin{lstlisting}
srand(0x5EED);
for(n=0; n<N; n++){ u=rand()%100;
  if     (u< 5) x[n]='A'; /* 00-04 */ 
  else if(u<10) x[n]='B'; /* 05-09 */ 
  else if(u<15) x[n]='C'; /* 10-14 */ 
  else if(u<20) x[n]='D'; /* 15-19 */ 
  else if(u<95) x[n]='E'; /* 20-94 */ 
  else          x[n]='F'; /* 95-99 */ }
\end{lstlisting}
\end{minipage}%
\hspace{10mm}%
\begin{tabular}{lclcl}
  where & \lstinline!'A'! &represents& $\dieA$ &,
     \\ & \lstinline!'B'! &represents& $\dieB$ &,
     \\ & \lstinline!'C'! &represents& $\dieC$ &,
     \\ & \lstinline!'D'! &represents& $\dieD$ &,
     \\ & \lstinline!'E'! &represents& $\dieE$ &, and
     \\ & \lstinline!'F'! &represents& $\dieF$ &.
\end{tabular}
\\
     The resulting sequence is partially displayed here:
  \\\includegraphics{ssp/plots/wrdie_5eed_51_seq.pdf}\\
  Of course the resulting histogram, as illustrated below on the left, reflects the non-uniform distribution.
  Also note, as illustrated below on the right, that the weighted sequence is much more correlated 
  (as defined by \prefp{def:ocsRxx})
  as compared to the uniformly distributed 
  \structe{real die} sequence of \prefpp{ex:rdie_sha}.
  %The following two illustrations, calculated over $(160000/50)-1=3199$ elements, 
  %are based on the above sequence downsampled by a factor of 50
  %(making them better comparable to the results obtained in \pref{item:rdie_rhp_R1map} and \pref{item:rdie_hp_R3map} below.
     \\\begin{tabular}{|>{\scs}c|>{\scs}c|}
          \hline
          \includegraphics{ssp/plots/wrdie_5eed_16002_histo.pdf}%
         &\includegraphics{ssp/plots/wrdie_5eed_16002_auto.pdf}
         % \includegraphics{ssp/graphics/wdie_5eed_160000_down50_histo.pdf}%
         %&\includegraphics{ssp/graphics/wdie_5eed_160000_down50_auto.pdf}
         %\\histogram & autocorrelation
         %\\\mc{2}{|>{\scs}c|}{calculated over $(160000/50)-1=3199$ elements}
        \\\hline
     \end{tabular}
\end{example}

%---------------------------------------
\begin{example}[\exmd{weighted die sequence}]
\label{ex:wdie_sha}
%---------------------------------------
Consider the non-uniformly distributed \structe{weighted die} \xref{def:wdie} sequence 
generated as in \pref{ex:wrdie_sha}.
Of course the resulting histogram is identical to that of \pref{ex:wrdie_sha},
but because the distance function is different, the autocorrelation sequence is also different.
     \\\begin{tabular}{|>{\scs}c|>{\scs}c|}
          \hline
          \includegraphics{ssp/plots/wdie_5eed_16002_histo.pdf}%
         &\includegraphics{ssp/plots/wdie_5eed_16002_auto.pdf}
        \\\hline
     \end{tabular}
\end{example}

%---------------------------------------
\begin{example}[\exmd{weighted spinner sequence}]
\label{ex:wspin_sha}
%---------------------------------------
Consider the non-uniformly distributed die sequence with 
    \\\indentx$\psp(\spinE)=0.75$ and $\psp(\spinA)=\psp(\spinB)=\psp(\spinC)=\psp(\spinD)=\psp(\spinF)=0.05$,\\
    generated by the C code\footnote{For a more complete source code listing, see \prefpp{sec:src_spinner}}
\\\begin{minipage}{85mm}%
\begin{lstlisting}
srand(0x5EED);
for(n=0; n<N; n++){ u=rand()%100;
  if     (u< 5) x[n]='A'; /* 00-04 */ 
  else if(u<10) x[n]='B'; /* 05-09 */ 
  else if(u<15) x[n]='C'; /* 10-14 */ 
  else if(u<20) x[n]='D'; /* 15-19 */ 
  else if(u<95) x[n]='E'; /* 20-94 */ 
  else          x[n]='F'; /* 95-99 */ }
\end{lstlisting}
\end{minipage}%
\hspace{10mm}%
\begin{tabular}{lclcl}
  where & \lstinline!'A'! &represents& $\spinA$ &,
     \\ & \lstinline!'B'! &represents& $\spinB$ &,
     \\ & \lstinline!'C'! &represents& $\spinC$ &,
     \\ & \lstinline!'D'! &represents& $\spinD$ &,
     \\ & \lstinline!'E'! &represents& $\spinE$ &, and
     \\ & \lstinline!'F'! &represents& $\spinF$ &.
\end{tabular}
\\
  The resulting sequence and histogram is in essence the same as in %\prefpp{ex:wrdie_sha}.
  the \exme{weighted die sequence} example \xref{ex:wrdie_sha}.
  %\\\includegraphics{ssp/plots/wspin_5eed_51_seq.pdf}\\
  %Of course the resulting histogram, as illustrated below on the left, is in essence identical to that of the 
  %\exme{weighted die sequence} example \xref{ex:wrdie_sha}.
  But note, as illustrated below on the right, that the \exme{weighted spinner sequence} of this example is 
  significantly less correlated than the \exme{weighted die sequence} of \prefpp{ex:wrdie_sha},
  presumably due to the larger \structe{range} \xref{def:range} of the spinner metric ($\setn{0,1,2,3}$)
  as compared to the \structe{range} of the \structe{weighted real die} metric ($\setn{0,1,2}$) and 
  the \structe{weighted die} metric ($\setn{0,1}$).
     \\\begin{tabular}{|>{\scs}c|>{\scs}c|}
          \hline
          \includegraphics{ssp/plots/wspin_5eed_16002_histo.pdf}%
         &\includegraphics{ssp/plots/wspin_5eed_16002_auto.pdf}
        \\\hline
     \end{tabular}
\end{example}


%---------------------------------------
\begin{example}[\exmd{Random DNA sequence}]
\label{ex:dna_5eed}
%---------------------------------------
Consider the pseudo-uniformly distributed DNA sequence generated by the C code\footnotemark 
\\\begin{minipage}{85mm}%
\begin{lstlisting}
srand(0x5EED);
for(n=0; n<N; n++){  r=rand()%4;
  switch(r){ case 0: x[n]='A'; break;
             case 1: x[n]='T'; break;
             case 2: x[n]='C'; break;
             case 3: x[n]='G'; break; }}
\end{lstlisting}
\end{minipage}%
\hspace{10mm}%
\begin{tabular}{lclcl}
  where & \lstinline!'A'! &represents& $\symA$ &,
     \\ & \lstinline!'T'! &represents& $\symT$ &,
     \\ & \lstinline!'C'! &represents& $\symC$ &, and
     \\ & \lstinline!'G'! &represents& $\symG$ &.
\end{tabular}
\footnotetext{For a more complete source code listing, see \prefpp{sec:src_dna}}
  \\\includegraphics{ssp/plots/dna_5eed_51_seq.pdf}\\
This sequence constrained to a length of $4000\times4=16000$ elements is approximately \prope{uniformly distributed} and \prope{uncorrelated}, 
as illustrated next:
  \\\begin{tabular}{|>{\scs}c|>{\scs}c|}
       \hline
       \includegraphics{ssp/plots/dna_5eed_16000_histo.pdf}%
      &\includegraphics{ssp/plots/dna_5eed_16000_auto.pdf}%
      %\\histogram & autocorrelation
     \\\hline
  \end{tabular}
\end{example}

%---------------------------------------
\begin{example}[\exmd{SARS coronavirus DNA sequence}]
\label{ex:dna_sars}
%---------------------------------------
Consider the genome sequence (DNA sequence) for the SARS coronavirus with \hie{GenBank} accession number NC\_004718.3.\footnote{\citer{ncbiSars}} %identifier gi30271926.
This sequence is of length $\xN=29751$ and is partially displayed here, followed by its histogram and 
\ope{$2\xN$-offset autocorrelation} plots.
  \\\includegraphics{ssp/plots/dna_sars_51_seq.pdf}\\
     \\\begin{tabular}{|>{\scs}c|>{\scs}c|}
          \hline
          \includegraphics{ssp/plots/dna_sars_histo.pdf}%
         &\includegraphics{ssp/plots/dna_sars_auto.pdf}
         %\\histogram & autocorrelation
        \\\hline
     \end{tabular}
\end{example}

%---------------------------------------
\begin{example}[\exmd{Ebola virus DNA sequence}]
\label{ex:dna_ebola}
%---------------------------------------
Consider the genome sequence (DNA sequence) for the Ebola virus with GenBank accession AF086833.2.\footnote{\citer{ncbiEbola}}
This sequence is of length 18959 and is partially displayed here:
  \\\includegraphics{ssp/plots/dna_ebola_51_seq.pdf}\\
     \\\begin{tabular}{|>{\scs}c|>{\scs}c|}
          \hline
          \includegraphics{ssp/plots/dna_ebola_histo.pdf}%
         &\includegraphics{ssp/plots/dna_ebola_auto.pdf}
         %\\histogram & autocorrelation
        \\\hline
     \end{tabular}
\end{example}

%---------------------------------------
\begin{example}[\exmd{Bacterium DNA sequence}]
\label{ex:dna_mpbacterium}
%---------------------------------------
Consider the genome sequence (DNA sequence) for the bacterium 
\hie{Melissococcus plutonius strain 49.3 plasmid pMP19}
with GenBank accession NZ\_CM003360.1.\footnote{\citer{ncbiMP}}
This sequence is of length 19430 and is partially displayed here:
  \\\includegraphics{ssp/plots/dna_mpbacterium_51_seq.pdf}\\
     \\\begin{tabular}{|>{\scs}c|>{\scs}c|}
          \hline
          \includegraphics{ssp/plots/dna_mpbacterium_histo.pdf}%
         &\includegraphics{ssp/plots/dna_mpbacterium_auto.pdf}
         %\\histogram & autocorrelation
        \\\hline
     \end{tabular}
\end{example}

%---------------------------------------
\begin{example}[\exmd{Papaya DNA sequence}]
\footnote{
  \citer{ncbiPapaya}
  }
\label{ex:dna_papaya}
%---------------------------------------
Consider the genome sequence segment for the fruit
\hie{carica papaya} with GenBank accession DS982815.1.
This sequence is not a complete genome, rather it is a ``genomic scaffold" \xref{def:dnan}. 
As such, there are some elements for which the content is not known.
For these locations, the symbol $\symN$ is used.
In this particular sequence, there are 144 $\symN$ symbols.
This sequence is of length 15495 and is partially displayed here:
  \\\includegraphics{ssp/plots/dna_papaya1446_51_seq.pdf}\\
     \\\begin{tabular}{|>{\scs}c|>{\scs}c|}
          \hline
          \includegraphics{ssp/plots/dna_papaya1446_histo.pdf}%
         &\includegraphics{ssp/plots/dna_papaya1446_auto.pdf}
         %\\histogram & autocorrelation
        \\\hline
     \end{tabular}
\end{example}

%=======================================
\subsection{Extending to distance linear spaces}
%=======================================
%=======================================
\subsubsection{Motivation}
%=======================================
\prefpp{sec:ocs} reviewed how a stochastic process could be defined as an \structe{outcome subspace}
with \prope{order} and \prope{metric} structures.
\prefpp{ex:realdieXRYZ} reviewed an example of a \structe{real die outcome subspace} 
that was mapped through 4 different \fncte{random variable}s to 4 different \structe{weighted graph}s.
Two of these random variables ($\rvY$ and $\rvZ$) mapped to structures (\structe{weighted graphs}) that are very similar 
to the \structe{real die} with respect to order and metric geometry.
Two other random variables ($\rvW$ and $\rvX$) mapped to structures 
(the \structe{real line} and the \structe{integer line}) that are very dissimilar.
The implication of this example is that if we want statistics that closely model the underlying stochastic process,
then we should map to a structure that that an order structure and distance geometry similar to 
that of the underlying stochastic process, and not simply the one that is the most convenient.
Ideally, we would like to map to a structure that is \prope{isomorphic} \xref{def:isomorphic}
and \prope{isometric} \xref{def:isometric} to the structure of the stochastic process.

However, for sequence processing using very basic methods such as FIR filtering, Fourier analysis,
or wavelet analysis,
we would very much like to map into the \structe{real line} $\R^1$ or possibly some higher dimensional space
$\R^n$.
Because the real line is often very dissimilar to the stochastic process,
we are motivated to find structures in $\R^n$ that \emph{are} similar.
And that is what this section presents---mapping from a stochastic process $\ocsD$ 
into an \structe{ordered distance linear space} $\omlsRnD$ in which $\omlo$ is an extension of $\ocso$ and
$\omld$ is an extension of $\ocsd$.

Thus, for sequence processing on an \structe{outcome subspace} $\ocsD$, we would like to 
define a \fncte{random variable} $\rvX$ and an \structe{ordered distance linear space} $\omlsRnD$ 
that satisfy the following constraints: 
\begin{enumerate}
  \item The random variable maps the elements of $\ocso$ into $\R^n$ and
  \item the order relation $\omlr$ is an \prope{extension} to $\omlo$ of the order relation $\ocsr$ on $\ocso$ and
  \item the \fncte{distance} function $\omld$ is an \prope{extension} to $\omlo$ of the distance function $\ocsd$ on $\ocso$.
\end{enumerate}

%=======================================
\subsubsection{Some random variables}
%=======================================
In this section, we first define some \fncte{random variable}s \xref{def:rvocs} 
that are used later in this paper.
%---------------------------------------
\begin{definition}
\label{def:rv_dietrad}
%---------------------------------------
The \opd{traditional die random variable} $\rvX$ maps from the set %\xref{def:rvt} 
\\$\setn{\dieA,\dieB,\dieC,\dieD,\dieE,\dieF}$ into the set $\R^1$
and is defined as\footnotemark
\\\indentx
  $\rvX(\dieA)\eqd 1$, 
  $\rvX(\dieB)\eqd 2$, 
  $\rvX(\dieC)\eqd 3$, 
  $\rvX(\dieD)\eqd 4$, 
  $\rvX(\dieE)\eqd 5$, and 
  $\rvX(\dieF)\eqd 6$.
\end{definition}

%---------------------------------------
\begin{definition}
\label{def:rv_diepam}
%---------------------------------------
The \opd{PAM die random variable} $\rvX$ maps from the set $\setn{\dieA,\dieB,\dieC,\dieD,\dieE,\dieF}$ into the set $\R^1$
and is defined as\footnotemark
\\\indentx
  $\rvX(\dieA)\eqd-2.5$, 
  $\rvX(\dieB)\eqd-1.5$, 
  $\rvX(\dieC)\eqd-0.5$, 
  $\rvX(\dieD)\eqd+0.5$, 
  $\rvX(\dieE)\eqd+1.5$, and 
  $\rvX(\dieF)\eqd+2.5$.
\end{definition}
\footnotetext{PAM is an acronym for \ope{pulse amplitude modulation} and is a standard technique in 
the field of digital communications.}

%---------------------------------------
\begin{definition}
\label{def:rv_dieqpsk}
%---------------------------------------
The \opd{QPSK die random variable} $\rvX$ maps from the set $\setn{\dieA,\dieB,\dieC,\dieD,\dieE,\dieF}$ into the set $\C^1$
and is defined as\footnotemark 
\\\indentx$\begin{array}{rclrclrcl}
    \rvX(\dieA) &\eqd& \exp\brp{ 30\times\frac{\pi}{180}i}, 
   &\rvX(\dieB) &\eqd& \exp\brp{ 90\times\frac{\pi}{180}i}, 
   &\rvX(\dieC) &\eqd& \exp\brp{150\times\frac{\pi}{180}i}, 
  \\\rvX(\dieD) &\eqd& \exp\brp{210\times\frac{\pi}{180}i}, 
   &\rvX(\dieE) &\eqd& \exp\brp{270\times\frac{\pi}{180}i},
   &\rvX(\dieF) &\eqd& \exp\brp{330\times\frac{\pi}{180}i}.
\end{array}$
\end{definition}
\footnotetext{QPSK is an acronym for \ope{quadrature phase shift keying} and is a standard technique in 
the field of digital communications.}

%---------------------------------------
\begin{definition}
\label{def:rv_dieR3}
%---------------------------------------
The \opd{$\R^3$ die random variable} $\rvX$ maps from the set $\setn{\dieA,\dieB,\dieC,\dieD,\dieE,\dieF}$ into the set $\R^3$
and is defined as 
\\\indentx$\begin{array}{rclrclrcl}
    \rvX(\dieA) &\eqd& \otriple{+1}{ 0}{ 0}, 
   &\rvX(\dieB) &\eqd& \otriple{ 0}{+1}{ 0}, 
   &\rvX(\dieC) &\eqd& \otriple{ 0}{ 0}{+1}, 
  \\\rvX(\dieD) &\eqd& \otriple{ 0}{ 0}{-1}, 
   &\rvX(\dieE) &\eqd& \otriple{ 0}{-1}{ 0},
   &\rvX(\dieF) &\eqd& \otriple{-1}{ 0}{ 0}.
\end{array}$
\end{definition}

%---------------------------------------
\begin{definition}
\label{def:rv_dieR6}
%---------------------------------------
The \opd{$\R^6$ die random variable} $\rvX$ maps from the set $\setn{\dieA,\dieB,\dieC,\dieD,\dieE,\dieF}$ into the set $\R^6$
and is defined as 
\\\indentx$\begin{array}{rclrclrcl}
    \rvX(\dieA) &\eqd& \osix{1}{0}{0}{0}{0}{0}, 
   &\rvX(\dieB) &\eqd& \osix{0}{1}{0}{0}{0}{0}, 
   &\rvX(\dieC) &\eqd& \osix{0}{0}{1}{0}{0}{0}, 
  \\\rvX(\dieD) &\eqd& \osix{0}{0}{0}{1}{0}{0}, 
   &\rvX(\dieE) &\eqd& \osix{0}{0}{0}{0}{1}{0},
   &\rvX(\dieF) &\eqd& \osix{0}{0}{0}{0}{0}{1}.
\end{array}$
\end{definition}

%---------------------------------------
\begin{definition}
\label{def:rv_spinR1}
%---------------------------------------
The \opd{$\R^1$ spinner random variable} $\rvX$ maps from the set $\setn{\spinA,\spinB,\spinC,\spinD,\spinE,\spinF}$ into the set $\R^1$
and is defined as\footnotemark
\\\indentx
  $\rvX(\spinA)\eqd 1$, 
  $\rvX(\spinB)\eqd 2$, 
  $\rvX(\spinC)\eqd 3$, 
  $\rvX(\spinD)\eqd 4$, 
  $\rvX(\spinE)\eqd 5$, and 
  $\rvX(\spinF)\eqd 6$.
\end{definition}

%---------------------------------------
\begin{definition}
\label{def:rv_spinqpsk}
%---------------------------------------
The \opd{QPSK spinner random variable} $\rvX$ maps from the set $\setn{\spinA,\spinB,\spinC,\spinD,\spinE,\spinF}$ into the set $\C^1$
and is defined as 
\\\indentx$\begin{array}{rclrclrcl}
    \rvX(\spinA) &\eqd& \exp\brp{-90\times\frac{\pi}{180}i}, 
   &\rvX(\spinB) &\eqd& \exp\brp{-30\times\frac{\pi}{180}i}, 
   &\rvX(\spinC) &\eqd& \exp\brp{ 30\times\frac{\pi}{180}i}, 
  \\\rvX(\spinD) &\eqd& \exp\brp{ 90\times\frac{\pi}{180}i}, 
   &\rvX(\spinE) &\eqd& \exp\brp{150\times\frac{\pi}{180}i},
   &\rvX(\spinF) &\eqd& \exp\brp{210\times\frac{\pi}{180}i}.
\end{array}$
\end{definition}

%---------------------------------------
\begin{definition}
\footnote{
  \citePp{galleani2010}{772}
  }
\label{def:rv_dnapam}
%---------------------------------------
The \opd{PAM DNA random variable} $\rvX$ maps from the set $\setn{\dnaA,\dnaC,\dnaG,\dnaT}$ into the set $\R^1$
and is defined as\footnotemark
\\\indentx
  $\rvX(\dnaA)\eqd-1.5$, 
  $\rvX(\dnaC)\eqd-0.5$, 
  $\rvX(\dnaT)\eqd+0.5$, 
  $\rvX(\dnaG)\eqd+1.5$. 
\end{definition}

%---------------------------------------
\begin{definition}
\footnote{
  \citePp{galleani2010}{772}
  }
\label{def:rv_dnaqpsk}
%---------------------------------------
The \opd{QPSK DNA random variable} $\rvX$ maps from the set $\setn{\dnaA,\dnaC,\dnaG,\dnaT}$ into the set $\C^1$
and is defined as 
\\\indentx$\begin{array}{rclrcl}
     \rvX(\dnaA) &\eqd& \exp\brp{ 45\times\frac{\pi}{180}i}, 
    &\rvX(\dnaC) &\eqd& \exp\brp{135\times\frac{\pi}{180}i}, 
   \\\rvX(\dnaG) &\eqd& \exp\brp{225\times\frac{\pi}{180}i}, 
    &\rvX(\dnaT) &\eqd& \exp\brp{315\times\frac{\pi}{180}i}.
\end{array}$
\end{definition}
\footnotetext{QPSK is an acronym for \ope{quadrature phase shift keying} and is a standard technique in 
the field of digital communications.}

%---------------------------------------
\begin{definition}
\label{def:rv_dnaR4}
%---------------------------------------
The \opd{$\R^4$ DNA random variable} $\rvX$ maps from the set $\setn{\dnaA,\dnaC,\dnaG,\dnaT}$ into the set $\R^4$
and is defined as\footnotemark
\\\indentx$\begin{array}{rclrclrclrcl}
     \rvX(\dnaA) &\eqd& \oquad{1}{0}{0}{0}, 
    &\rvX(\dnaC) &\eqd& \oquad{0}{1}{0}{0}, 
    &\rvX(\dnaT) &\eqd& \oquad{0}{0}{1}{0}, 
    &\rvX(\dnaG) &\eqd& \oquad{0}{0}{0}{1}
\end{array}$
\end{definition}
\footnotetext{This type of mapping has previously been used by \citeP{voss1992} in calculating the \ope{Voss Spectrum}, 
  (a kind of Fourier analysis) of DNA sequences.
  See also \citePp{galleani2010}{772}.
  }


%=======================================
\subsubsection{Some ordered distance linear spaces}
%=======================================
%---------------------------------------
\begin{definition}
\label{def:dieR1oml}
%---------------------------------------
The structure $\oml{\R^1}{\omlr}{\omld}$ is the \structd{$\R^1$ die distance linear space}
if $\omlr$ is the \rele{standard ordering relation} on $\R$, 
and $\omld(x,y)\eqd\abs{x-y}$ (the \fncte{Euclidean metric} on $\R$, \xrefnp{def:emetric}).
\end{definition}

%---------------------------------------
\begin{definition}
\label{def:dieR3oml}
%---------------------------------------
%Let $\rvX$ be the \opd{$\R^3$ die random variable} \xref{def:dieR3rv}.
The structure $\oml{\R^3}{\omlr}{\omld}$ is the \structd{$\R^3$ die distance linear space}
if $\omlr=\emptyset$, and $\omld$ is the 
$2$-scaled \fncte{Lagrange arc distance} $\omld$ defined as follows:
\quad$\omld(p,q) \eqd 2\metrican(p,q)$\\
where $\metrican$ is the \fncte{Lagrange arc distance} \xref{def:larc}.
\end{definition}
%\begin{figure}[h]
%  \centering%
%  \gsize%
\mbox{}\hfill%
  \begin{tabular}{c}\includegraphics{ssp/graphics/rdie_r3met.pdf}\end{tabular}%
  \hspace{15mm}%
  $\begin{tabstr}{0.75}\begin{array}{|c|*{6}{@{\hspace{2pt}}c}|}
    \hline
    \ocsd(x,y) & \dieA &\dieB &\dieC &\dieD &\dieE &\dieF
    \\\hline
      \dieA &    0   &   1   &   1   &   1   &   1   &   2
    \\\dieB &    1   &   0   &   1   &   1   &   2   &   1
    \\\dieC &    1   &   1   &   0   &   2   &   1   &   1
    \\\dieD &    1   &   1   &   2   &   0   &   1   &   1
    \\\dieE &    1   &   2   &   1   &   1   &   0   &   1
    \\\dieF &    2   &   1   &   1   &   1   &   1   &   0
    \\\hline
  \end{array}\end{tabstr}$
\hfill\mbox{}\\
%  \caption{\structe{distance linear space} $\opair{\R^3}{\ocsd'}$ for \structe{real die}\label{fig:rdie3dmet}}
%\end{figure}
Used together with the \fncte{$\R^3$ die random variable} $\rvX$ \xref{def:rv_dieR3},
the distance $\omld$ in the \structe{$\R^3$ die distance linear space} \xrefr{def:dieR3oml}
is an extension of $\ocsd$ in the \structe{real die outcome subspace} 
$\ocsG\eqd\ocsrdie$ \xref{def:rdie}.
We can also say that $\rvX$ is an \fncte{isometry} \xref{def:isometry} 
and that the two structures are \prope{isometric}.
For example, %in $\ocsG$, $\ocsd(\dieA,\dieB)=1$ and $\ocsd(\dieA,\dieF)=2$; likewise in $\omlD$, 
\\\indentx$\begin{array}{lclclclD}
  \omld\brs{\rvX(\dieA),\rvX(\dieB)} &=& \omld\brs{\otriple{1}{0}{0},\otriple{ 0}{1}{0}} &=& 1 &=& \ocsd(\dieA,\dieB) & and\\
  \omld\brs{\rvX(\dieA),\rvX(\dieF)} &=& \omld\brs{\otriple{1}{0}{0},\otriple{-1}{0}{0}} &=& 2 &=& \ocsd(\dieA,\dieF) & .
\end{array}$
\\
As for order, the mapping $\rvX$ is also \prope{order preserving} \xref{def:opreserve},
but trivially, because the \structe{real die outcome subspace} is \prope{unordered} \xref{def:unordered}.
But if we still honor the standard ordering on each dimension $\R$ in $\R^3$,
then the two structures are \prope{not isomorphic} \xref{def:isomorphic} because\footnote{%
  Note that while $\rvX^{-1}$ \xref{def:inverse} does not exist as a \structe{function}, it does exist as a \structe{relation}.
  }
the inverse $\rvX^{-1}$ 
is \prope{not order preserving} \xref{thm:latiso}---%
for example, $\rvX(\dieD)=\otriple{0}{0}{-1}\orel\otriple{0}{0}{1}=\rvX(\dieC)$, but $\dieD$ and $\dieC$
are \prope{incomparable} \xref{def:incomparable} in $\ocsG$.

%---------------------------------------
\begin{definition}
\label{def:spinR2oml}
%---------------------------------------
The structure $\oml{\R^2}{\omlr}{\omld}$ is the \structd{$\R^2$ spinner distance linear space}
if $\omlr=\emptyset$, and $\omld$ is the 
$3$-scaled \fncte{Lagrange arc distance} $\omld$ defined as follows:
\quad$\omld(p,q) \eqd 3\metrican(p,q)$\\
where $\metrican$ is the \fncte{Lagrange arc distance} \xref{def:larc}.
\end{definition}
%\begin{figure}[h]
%  \centering%
%  \gsize%
\mbox{}\hfill%
  \begin{tabular}{c}\includegraphics{ssp/graphics/spinner_r2.pdf}\end{tabular}%
  \hspace{15mm}%
  $\begin{tabstr}{0.75}\begin{array}{|c|*{6}{@{\hspace{5pt}}c}|}
    \hline
    \ocsd(x,y)    & \spinA & \spinB & \spinC & \spinD & \spinE & \spinF
    \\\hline
      \spinA      &    0   &   1    &    2   &    3   &    2   &    1
    \\\spinB      &    1   &   0    &    1   &    2   &    3   &    2
    \\\spinC      &    2   &   1    &    0   &    1   &    2   &    3
    \\\spinD      &    3   &   2    &    1   &    0   &    1   &    2
    \\\spinE      &    2   &   3    &    2   &    1   &    0   &    1
    \\\spinF      &    1   &   2    &    3   &    2   &    1   &    0
    \\\hline
  \end{array}\end{tabstr}$
\hfill\mbox{}\\
%  \caption{\structe{distance linear space} $\opair{\R^2}{\ocsd'}$ for \structe{spinner}\label{fig:rdie3dmet}}
%\end{figure}
Used together with the \fncte{QPSK spinner random variable} $\rvX$ \xref{def:rv_spinqpsk},
the distance $\omld$ in the \structe{$\R^2$ spinner distance linear space} \xref{def:spinR2oml}
is an extension of $\ocsd$ in the \structe{spinner outcome subspace} 
$\ocsG\eqd\ocsspin$ \xref{def:spinner}.
We can again say that $\rvX$ is an \fncte{isometry} %\xref{def:isometry} 
and that the two structures are \prope{isometric}.
For example, %in $\ocsG$, $\ocsd(\spinA,\spinB)=1$, $\ocsd(\spinA,\spinC)=2$, and $\ocsd(\spinA,\spinD)=3$; likewise in $\omlD$, 
\\\indentx$\begin{array}{lclclclD}
  \omld\brs{\rvX(\spinA),\rvX(\spinB)} &=& \omld\brs{\opair{0}{-1},\opair{\sfrac{\sqrt{3}}{2}}{-\sfrac{1}{2}}} &=& 1 &=& \ocsd(\spinA,\spinB) & and\\
  \omld\brs{\rvX(\spinA),\rvX(\spinC)} &=& \omld\brs{\opair{0}{-1},\opair{\sfrac{\sqrt{3}}{2}}{+\sfrac{1}{2}}} &=& 2 &=& \ocsd(\spinA,\spinC) & and\\
  \omld\brs{\rvX(\spinA),\rvX(\spinD)} &=& \omld\brs{\opair{0}{-1},\opair{0}{1}}                               &=& 3 &=& \ocsd(\spinA,\spinD) & .
\end{array}$
\\
The mapping $\rvX$ is again trivally \prope{order preserving}.
And if we again honor the standard ordering on each dimension $\R$ in $\R^2$,
then the two structures are \prope{not isomorphic} \xref{def:isomorphic} because
the inverse $\rvX^{-1}$ 
is \prope{not order preserving}---%
for example, $\rvX(\spinA)=\opair{0}{-1}\orel\opair{0}{1}=\rvX(\spinD)$, but $\spinA$ and $\spinD$
are \prope{incomparable} in $\ocsG$.

%The next definition introduces a \structe{metric linear space} for \ope{fair die sequence processing}
%that unlike in the \structe{real die} structure, the random variable mapping 
%is fully \prope{order preserving} (trivally).
%---------------------------------------
\begin{definition}
\label{def:fdieR6oml}
%---------------------------------------
The structure $\oml{\R^6}{\omlr}{\omld}$ is the \structd{$\R^6$ die distance linear space}
if $\omlr=\emptyset$, and $\omld$ is defined as 
$\ds\omld(p,q) \eqd \sfrac{\sqrt{2}}{2}\metrica{p}{q}$,
where $\metrican$ is the \fncte{Euclidean metric} on $\R^6$ \xref{def:emetric}.
\end{definition}

Used together with the \fncte{$\R^6$ die random variable} $\rvX$ \xref{def:rv_dieR6},
the distance $\omld$ in the \structe{$\R^6$ fair die distance linear space} \xref{def:fdieR6oml}
is an extension of $\ocsd$ in the \structe{fair die outcome subspace} 
$\ocsG\eqd\ocsrdie$ \xref{def:rdie}.
We can again say that $\rvX$ is an \fncte{isometry} %\xref{def:isometry} 
and that the two structures are \prope{isometric}.
For example, %in $\ocsG$, $\ocsd(\dieA,\dieB)=1$, $\ocsd(\dieA,\dieC)=1$, and $\ocsd(\dieA,\dieF)=1$; likewise in $\omlD$, 
\\\indentx$\begin{array}{lclclclD}
  \omld\brs{\rvX(\dieA),\rvX(\dieB)} &=& \omld\brs{\osix{1}{0}{0}{0}{0}{0},\osix{0}{1}{0}{0}{0}{0}} &=& 1 &=& \ocsd(\dieA,\dieB) & and\\
  \omld\brs{\rvX(\dieA),\rvX(\dieC)} &=& \omld\brs{\osix{1}{0}{0}{0}{0}{0},\osix{0}{0}{1}{0}{0}{0}} &=& 1 &=& \ocsd(\dieA,\dieC) & and\\
  \omld\brs{\rvX(\dieA),\rvX(\dieF)} &=& \omld\brs{\osix{1}{0}{0}{0}{0}{0},\osix{0}{0}{0}{0}{0}{1}} &=& 1 &=& \ocsd(\dieA,\dieF) & .
\end{array}$
\\
The mapping $\rvX$ is again trivally \prope{order preserving},
and the inverse $\rvX^{-1}$ is trivally \prope{order preserving} as well.
And so unlike the \structe{$\R^3$ die distance linear space} \xrefr{def:dieR3oml}
and the \structe{$\R^2$ spinner distance linear space} \xrefr{def:spinR2oml},
this pair of structures is \prope{isomorphic}.




  %-------------------------------------
  \section{Symbolic sequence processing applications}
  %-------------------------------------
\qboxnpq
  {
    Joseph Louis Lagrange (1736-1813), mathematician
    \index{Lagrange, Joseph Louis}
    \index{quotes!Lagrange, Joseph Louis}
    \footnotemark
  }
  {../common/people/small/lagrange.jpg}
  {I regard as quite useless the reading of large treatises of pure analysis:
    too large a number of methods pass at once before the eyes.
    It is in the works of applications that one must study them;
    one judges their ability there and one apprises the manner of making use of them.}
  \citetblt{
    quote: &  \citerp{stopple2003}{xi} \\
          %&  \url{http://www.math.okstate.edu/~wli/teach/fmq.html} \\
          %&  \url{http://www-groups.dcs.st-and.ac.uk/~history/Quotations/Lagrange.html} \\
    image: & \url{http://en.wikipedia.org/wiki/Image:Langrange_portrait.jpg}
    }

  %============================================================================
% Daniel J. Greenhoe
% LaTeX file
%============================================================================
%=======================================
\subsection{Low pass filtering/Smoothing}
%=======================================
%=======================================
%\section{Real die sequence processing}
%\label{sec:rdie}
%=======================================

%=======================================
%\subsubsection{Low pass filtering of real die sequences}
%=======================================
%---------------------------------------
\begin{example}[\exmd{low pass filtering of real die sequence}]
\label{ex:rdie_lp}
\mbox{}\\
\addcontentsline{toc}{subsubsection}{* low pass filtering of real die sequence}
%---------------------------------------
\begin{enumerate}
  \item \label{item:rdie_lp_seq}
     Consider the pseudo-uniformly distributed die sequence presented in \prefpp{ex:rdie_sha}.
     Suppose we want to \ope{filter} this sequence with a
     \fncte{low pass sequence} in order to ``smooth out" the sequence. 
     But to perform the actual filtering, note that the die sequence
     must first be mapped into a \structe{linear space} $\R^\xN$.
  
  \item \label{item:rdie_lp_R1_rect16_euclid}
        Suppose we first use the \ope{traditional die random variable} \xref{def:rv_dietrad} 
        to map the die sequence into $\R^1$.
        \ope{Filter}ing \xref{def:filter} this $\R$-valued sequence using the 
        \fncte{length 16 rectangular low pass sequence} \xref{ex:lp_rect} 
        in the \structe{$\R^1$ die distance linear space} \xref{def:dieR1oml} 
        and then mapping the result back to a \fncte{die sequence} %\xref{def:rdie}
        using the \fncte{Euclidean metric} \xref{def:emetric}, 
        produces the result partially dispayed here:
        \\\includegraphics{ssp/plots/rdie_lp_12000_R1_rect16_euclid_seq.pdf}\\
        Note that the die sequence has indeed been smoothed out, but it's uniform distribution has been destroyed---almost all 
        of its values are around the ``expected value" 3.5, 
        as illustrated below on the left.
        Of course such filtering also introduces correlation, giving the \ope{autocorrelation} sequence 
        a slightly wider center lobe as illustrated below on the right.
        Both diagrams are calculated over a length $12000$ sequence.
     \\\begin{tabular}{|>{\scs}c|>{\scs}c|}
          \hline
          \includegraphics{ssp/plots/rdie_lp_12000_R1_rect16_euclid_histo.pdf}%
         &\includegraphics{ssp/plots/rdie_lp_12000_R1_rect16_euclid_auto.pdf}
         \\histogram & \ope{$2\xN$-offset autocorrelation} \xref{def:ocsRxx}
        \\\hline
     \end{tabular}

  \item \label{item:rdie_lp_R3_rect16_larc}
        Alternatively, suppose we next try
        using the \ope{$\R^3$ die random variable} \xref{def:rv_dieR3}
        to map the die sequence into $\R^3$.
        \ope{Filter}ing %\xref{def:filtern} 
        this new sequence using the \ope{length 16 rectangular low pass sequence}
        in the \structe{$\R^3$ distance linear space} \xref{def:dieR3oml} 
        and then mapping back to a \fncte{die sequence} %over the \structe{real die outcome subspace} 
        using the \fncte{Lagrange arc distance} yields the result partially displayed here:
        \\\includegraphics{ssp/plots/rdie_lp_12000_R3_rect16_larc_seq.pdf}\\
        Note that the \fncte{die sequence} does appear to be ``smoothed out", 
        but this time the distribution is much more uniform, as illustrated below on the left;
        and is slightly less correlated (12795 compared to 17658), as illustrated below on the right.
     \\\begin{tabular}{|>{\scs}c|>{\scs}c|}
          \hline
          \includegraphics{ssp/plots/rdie_lp_12000_R3_rect16_larc_histo.pdf}%
         &\includegraphics{ssp/plots/rdie_lp_12000_R3_rect16_larc_auto.pdf}
        \\\hline
     \end{tabular}
        

  \item \label{item:rdie_lp_R3_hann16_larc}
        Using a \ope{length 16 Hanning low pass sequence} \xref{def:lp_hann}
        rather than the \fncte{length 16 rectangular low pass sequence} as in \pref{item:rdie_lp_R3_rect16_larc}
        results in a distribution that is more uniform and in a sequence that is very slightly less correlated:
        \\\includegraphics{ssp/plots/rdie_lp_12000_R3_hann16_larc_seq.pdf}
     \\\begin{tabular}{|>{\scs}c|>{\scs}c|}
          \hline
          \includegraphics{ssp/plots/rdie_lp_12000_R3_hann16_larc_histo.pdf}%
         &\includegraphics{ssp/plots/rdie_lp_12000_R3_hann16_larc_auto.pdf}
        \\\hline
     \end{tabular}

  \item \label{item:rdie_lp_R3_hann50_larc}
        Using a \ope{length 50 Hanning low pass sequence} \xref{ex:lp_hann} rather than the 
        \fncte{length 16 Hanning low pass sequence} as in \pref{item:rdie_lp_R3_hann16_larc}
        results in about the same uniformity of distribution, about 1.8\% lower side lobes in the autocorrelation sequence
        ($\frac{12733-12505}{12733}\times100\approx1.8$),
        but a wider main lobe (presumably due to the longer filter width):
        %\\\includegraphics{ssp/plots/rdie_lp_12000_R3_hann50_larc_seq.pdf}
     \\\begin{tabular}{|>{\scs}c|>{\scs}c|}
          \hline
          \includegraphics{ssp/plots/rdie_lp_12000_R3_hann50_larc_histo.pdf}%
         &\includegraphics{ssp/plots/rdie_lp_12000_R3_hann50_larc_auto.pdf}
        \\\hline
     \end{tabular}

  \item \label{item:rdie_lp_R3_rect50_larc}
        Using a \ope{length 50 rectangular low pass sequence} rather than the 
        \fncte{length 50 Hanning low pass sequence} as in \pref{item:rdie_lp_R3_hann50_larc}
        results in a distribution that is a little less uniform and about 3.3\% more correlated
        ($\abs{\frac{12505-12916}{12505}\times100}\approx3.3$):
        %\\\includegraphics{ssp/plots/rdie_lp_12000_R3_rect50_larc_seq.pdf}
     \\\begin{tabular}{|>{\scs}c|>{\scs}c|}
          \hline
          \includegraphics{ssp/plots/rdie_lp_12000_R3_rect50_larc_histo.pdf}%
         &\includegraphics{ssp/plots/rdie_lp_12000_R3_rect50_larc_auto.pdf}
        \\\hline
     \end{tabular}

  \item \label{item:rdie_lp_R3_euclid}
        Replacing the \fncte{Lagrange arc distance} by the \fncte{Euclidean metric} in this example
        has very little effect.
        More details follow:
    \begin{enumerate}
      \item Using the \fncte{Euclidean metric} in $\R^3$ rather than the \fncte{Lagrange arc distance} in \pref{item:rdie_lp_R3_rect16_larc}
            yields sequences that are \textbf{identical}.\citetbl{\seessp{rdie_lp_12000m16.xlg}}

      \item Using the \fncte{Euclidean metric} in \pref{item:rdie_lp_R3_hann16_larc} rather than the \fncte{Lagrange arc distance} 
            yields sequences that \textbf{differ} at 6 locations out of $N+M+M-1=12000+16+16-1=12031$ locations
            (differ at approximately 0.05\% of the locations):\citetbl{\seessp{rdie_lp_12000m16.xlg}}
            \\\indentx\begin{tabstr}{0.75}\begin{tabular}{|c|c|c|}
                         \hline% 2016 June 08 Wednesday 11:03:38 PM UTC
                           n   & Euclidean & Lagrange
                         \\\hline
                             281 & $\dieF$ & $\dieD$ 
                         \\ 1630 & $\dieA$ & $\dieC$ 
                         \\11888 & $\dieF$ & $\dieB$ 
                         \\\hline
                       \end{tabular}\end{tabstr}

      \item Using the \fncte{Euclidean metric} in $\R^3$ rather than the \fncte{Lagrange arc distance} as in 
            \pref{item:rdie_lp_R3_hann50_larc} (length 50 Hanning filter)
            yields sequences that are \emph{identical}.\citetbl{\seessp{rdie_lp_12000m50.xlg}} % 2016 June 08 Wednesday 11:07:28 PM UTC

      \item Using the \fncte{Euclidean metric} in $\R^3$ rather than the \fncte{Lagrange arc distance} as 
            in \pref{item:rdie_lp_R3_rect50_larc} (length 50 rectangular filter)
            \textbf{differ} at 85 locations out of $N+M+M-1=12000+50+50-1=12099$ locations
            (differ at approximately 0.7\% of the locations).\citetbl{\seessp{rdie_lp_12000m50.xlg}} % 2016 June 08 Wednesday 11:07:28 PM UTC
    \end{enumerate}

\end{enumerate}
\end{example}


%=======================================
%\subsubsection{Low pass filtering of spinner sequences}
%=======================================
%---------------------------------------
\begin{example}[\exmd{low pass filtering of spinner sequence}]
\label{ex:spin_lp}
\addcontentsline{toc}{subsubsection}{* low pass filtering of spinner sequence}
\mbox{}\\
%---------------------------------------
\begin{enumerate}
  \item \label{item:spin_lp_seq}
     Consider the pseudo-uniformly distributed spinner sequence presented in \prefpp{ex:spinner_sha}.
     As in \prefpp{ex:rdie_lp}, suppose we want to \ope{filter} this sequence with a
     \fncte{low pass rectangular sequence} in order to ``smooth out" the sequence.
     %But to perform the actual filtering, note that the die sequence
     %must first be mapped into a \structe{linear space} $\R^\xN$.

  \item \label{item:spin_lp_R1_rect16_euclid}
        Suppose we first use the \ope{$\R^1$ spinner random variable} \xref{def:rv_spinR1} 
        to map the spinner sequence into $\R^1$.
        \ope{Filter}ing this mapped sequence using the \fncte{length 16 rectangular low pass sequence} 
        and then mapping the result back to a \fncte{spinner sequence} %over the \structe{spinner outcome subspace} \xref{def:rdie}
        using the \fncte{Euclidean metric}, produces the result partially dispayed here 
        (in essense the same as in \prefp{ex:rdie_lp}):
        \\\includegraphics{ssp/plots/spin_lp_12000_R1_rect16_euclid_seq.pdf}\\
        Again, it's uniform distribution has been essentially destroyed.
        %---almost all  of its values are around the ``expected value" 3.5.
        \\\begin{tabular}{|>{\scs}c|>{\scs}c|}
             \hline
             \includegraphics{ssp/plots/spin_lp_12000_R1_rect16_euclid_histo.pdf}%
            &\includegraphics{ssp/plots/spin_lp_12000_R1_rect16_euclid_auto.pdf}
           \\\hline
        \end{tabular}


  \item \label{item:spin_lp_R2_rect16_larc}
        Alternatively, suppose we next try
        using the \ope{QPSK spinner random variable} \xref{def:rv_spinqpsk}
        to map the spinner sequence into $\Cnum\eqd\R^2$.
        \ope{Filter}ing this new sequence using the \ope{length 16 rectangular low pass sequence}
        in the \structe{$\R^2$ spinner distance linear space} \xref{def:spinR2oml}
        and then mapping back to a \fncte{sequence} over the \structe{spinner outcome subspace} 
        using the \fncte{Lagrange arc distance} yields the result partially displayed here:
        \\\includegraphics{ssp/plots/spin_lp_12000_R2_rect16_larc_seq.pdf}\\
        Note that the sequence does appear to be ``smoothed out", 
        but this time the distribution is much more uniform and about 
        69\% less correlated than the $\R^1$ method of \pref{item:spin_lp_R1_rect16_euclid}:
        \\\begin{tabular}{|>{\scs}c|>{\scs}c|}
             \hline
             \includegraphics{ssp/plots/spin_lp_12000_R2_rect16_larc_histo.pdf}%
            &\includegraphics{ssp/plots/spin_lp_12000_R2_rect16_larc_auto.pdf}
           \\\hline
        \end{tabular}\\
        Furthermore, it is about 58\% less correlated than the $\R^3$ filtering for the die sequence
        used in \pref{item:rdie_lp_R3_rect16_larc} of \prefpp{ex:rdie_lp}.

  \item Using the \fncte{Euclidean metric} rather than the \fncte{Lagrange arc distance} as in \pref{item:spin_lp_R2_rect16_larc}
        results in a sequence that differs at 99 different locations out of $N+M+M-1=12000+16+15=12031$ locations
        (approximately 0.8\% of the locations differ).\citetbl{\seessp{spin_lp_12000m16.xlg}} % 2016 June 08 Wednesday 11:11:16 PM UTC
        \\\begin{tabular}{|>{\scs}c|>{\scs}c|}
             \hline
             \includegraphics{ssp/plots/spin_lp_12000_R2_rect16_euclid_histo.pdf}%
            &\includegraphics{ssp/plots/spin_lp_12000_R2_rect16_euclid_auto.pdf}
           \\\hline
        \end{tabular}

  \item \label{item:spin_lp_R2_hann16_larc}
        Using a \ope{length 16 Hanning low pass sequence} rather than the 
        \fncte{length 16 Rectangular low pass sequence} as in \pref{item:spin_lp_R2_rect16_larc}
        results in a distribution that is more uniform and about 5.3\% less correlated:
        \\\includegraphics{ssp/plots/spin_lp_12000_R2_hann16_larc_seq.pdf}
     \\\begin{tabular}{|>{\scs}c|>{\scs}c|}
          \hline
          \includegraphics{ssp/plots/spin_lp_12000_R2_hann16_larc_histo.pdf}%
         &\includegraphics{ssp/plots/spin_lp_12000_R2_hann16_larc_auto.pdf}
        \\\hline
     \end{tabular}

  \item \begin{minipage}[t]{\tw-70mm}%
          Using the \fncte{Euclidean metric} rather than the \fncte{Lagrange arc distance} as in \pref{item:spin_lp_R2_hann16_larc}
          results in a sequence that differs at exactly 2 locations (approximately 0.017\%) 
          out of $12031$ locations:\footnotemark
        \end{minipage}%
        \hfill\citetblt{\seessp{spin_lp_12000m16.xlg}} % 2016 June 08 Wednesday 11:11:16 PM UTC
        %\begin{minipage}{60mm}%
          \begin{tabstr}{0.75}\begin{tabular}[t]{|c|c|c|}
            \hline
              n   & Euclidean & Lagrange
            \\\hline
               4149 & $\dieD$   & $\dieE$
            \\ 5594 & $\dieB$   & $\dieA$
            \\\hline
          \end{tabular}\end{tabstr}%

  \item \label{item:spin_lp_R2_hann50_larc}
        Using a \ope{length 50 Hanning low pass sequence} rather than the 
        \fncte{length 16 Hanning low pass sequence} as in \pref{item:spin_lp_R2_hann16_larc}
        results in the following:%about the same uniformity of distribution, slightly lower side lobes in the autocorrelation function,
        %but a wider main lobe (due to the longer filter width):
        %\\\includegraphics{ssp/plots/spin_lp_12000_R2_hann50_larc_seq.pdf}
     \\\begin{tabular}{|>{\scs}c|>{\scs}c|}
          \hline
          \includegraphics{ssp/plots/spin_lp_12000_R2_hann50_larc_histo.pdf}%
         &\includegraphics{ssp/plots/spin_lp_12000_R2_hann50_larc_auto.pdf}
        \\\hline
     \end{tabular}

  \item \label{item:spin_lp_R2_rect50_larc}
        Using a \ope{length 50 Rectangular low pass sequence} rather than the 
        \fncte{length 50 Hanning low pass sequence} as in \pref{item:spin_lp_R2_hann50_larc}
        results in the following: %a little less uniform:
        %\\\includegraphics{ssp/plots/spin_lp_12000_R2_rect50_larc_seq.pdf}
     \\\begin{tabular}{|>{\scs}c|>{\scs}c|}
          \hline
          \includegraphics{ssp/plots/spin_lp_12000_R2_rect50_larc_histo.pdf}%
         &\includegraphics{ssp/plots/spin_lp_12000_R2_rect50_larc_auto.pdf}
        \\\hline
     \end{tabular}
\end{enumerate}
\end{example}



%=======================================
%\subsubsection{Low pass filtering of fair die sequences}
%=======================================
%---------------------------------------
\begin{example}[\exmd{low pass filtering of fair die sequence}]
\label{ex:fdie_lp}\mbox{}\\
\addcontentsline{toc}{subsubsection}{* low pass filtering of fair die sequence}
%---------------------------------------
\begin{enumerate}
  \item \label{item:fdie_lp_seq}
     Consider the pseudo-uniformly distributed die sequence presented in \prefpp{ex:fdie_sha}.
     Suppose we want to \ope{filter} %\xref{def:filter} 
     this sequence with a
     \fncte{low pass sequence} %\xref{ex:lp_rect} 
     in order to ``smooth out" the sequence, just as in \prefpp{ex:rdie_lp}.
     %But to perform the actual filtering, note that the die sequence
     %must first be mapped into a \structe{linear space} $\R^\xN$.

  \item \label{item:fdie_lp_R1_rect16_euclid}
        Suppose we first use the \ope{traditional die random variable} \xref{def:rv_dietrad} 
        to map the die sequence into $\R^1$.
        \ope{Filter}ing this mapped sequence using the \fncte{length 16 rectangular low pass sequence} 
        and then mapping the result back to a \fncte{die sequence} % over the \structe{real die outcome subspace} %\xref{def:rdie}
        using the \fncte{Euclidean metric}, produces a result identical to that of 
        \prefpp{item:rdie_lp_R1_rect16_euclid} of \pref{ex:rdie_lp}.
  

  \item \label{item:fdie_lp_R6_rect16_euclid}
        Alternatively, suppose we next 
        use the \ope{$\R^6$ die random variable} \xref{def:rv_dieR6}
        to map the die sequence into $\R^6$.
        \ope{Filter}ing this new sequence using the \ope{length 16 rectangular low pass sequence}
        in the \structe{$\R^6$ die distance linear space} \xref{def:fdieR6oml} 
        and then mapping back to a \fncte{die sequence} %over the \structe{fair die outcome subspace} 
        using the \fncte{Euclidean metric} yields a much more uniform distribution and 
        a sequence that is about 28\% less correlated.
        \\\includegraphics{ssp/plots/fdie_lp_12000_R6_rect16_euclid_seq.pdf}%
     \\\begin{tabular}{|>{\scs}c|>{\scs}c|}
          \hline
          \includegraphics{ssp/plots/fdie_lp_12000_R6_rect16_euclid_histo.pdf}%
         &\includegraphics{ssp/plots/fdie_lp_12000_R6_rect16_euclid_auto.pdf}
        \\\hline
     \end{tabular}\\
     Note further that this $\R^6$ technique yeilds a sequence that is about 9.9\% more correlated than 
     yielded by the $\R^3$ technique
     used in \pref{item:rdie_lp_R3_rect16_larc} of \prefpp{ex:rdie_lp}.

  \item \label{item:fdie_lp_R6_hann16_euclid}
        Using a \ope{length 16 Hanning low pass sequence} rather than the 
        \fncte{length 16 Rectangular low pass sequence} as in \pref{item:fdie_lp_R6_rect16_euclid}
        results in a distribution that is more uniform and a sequence that is about 0.12\% less correlated:
        \\\includegraphics{ssp/plots/fdie_lp_12000_R6_hann16_euclid_seq.pdf}
     \\\begin{tabular}{|>{\scs}c|>{\scs}c|}
          \hline
          \includegraphics{ssp/plots/fdie_lp_12000_R6_hann16_euclid_histo.pdf}%
         &\includegraphics{ssp/plots/fdie_lp_12000_R6_hann16_euclid_auto.pdf}
        \\\hline
     \end{tabular}\\
     Note further that this $\R^6$ technique yields a sequence that is about 10\% more correlated than 
     yielded by the $\R^3$ technique
     used in \pref{item:rdie_lp_R3_hann16_larc} of \prefpp{ex:rdie_lp}.

  \item \label{item:fdie_lp_R6_hann50_euclid}
        Using a \ope{length 50 Hanning low pass sequence} rather than the 
        \fncte{length 16 Hanning low pass sequence} as in \pref{item:fdie_lp_R6_hann16_euclid}
        results in the following: %about the same uniformity of distribution, slightly lower side lobes in the autocorrelation function,
        %but a wider main lobe (due to the longer filter width):
     \\\begin{tabular}{|>{\scs}c|>{\scs}c|}
          \hline
          \includegraphics{ssp/plots/fdie_lp_12000_R6_hann50_euclid_histo.pdf}%
         &\includegraphics{ssp/plots/fdie_lp_12000_R6_hann50_euclid_auto.pdf}
        \\\hline
     \end{tabular}\\
     Note further that this $\R^6$ technique yields a sequence that is about 11\% more correlated than 
     yielded by the $\R^3$ technique
     used in \pref{item:rdie_lp_R3_hann50_larc} of \prefpp{ex:rdie_lp}.

  \item \label{item:fdie_lp_R6_rect50_euclid}
        Using a \ope{length 50 Rectangular low pass sequence} rather than the 
        \fncte{length 50 Hanning low pass sequence} as in \pref{item:fdie_lp_R6_hann50_euclid}
        results in the following: %a distribution that is a little less uniform:
     \\\begin{tabular}{|>{\scs}c|>{\scs}c|}
          \hline
          \includegraphics{ssp/plots/fdie_lp_12000_R6_rect50_euclid_histo.pdf}%
         &\includegraphics{ssp/plots/fdie_lp_12000_R6_rect50_euclid_auto.pdf}
        \\\hline
     \end{tabular}\\
     Note further that this $\R^6$ technique yeilds a sequence that is about 8.8\% more correlated than yeilded by the $\R^3$ technique
     used in \pref{item:rdie_lp_R3_rect50_larc} of \prefpp{ex:rdie_lp}.

  \item In the \structe{fair die outcome space}, the \fncte{Lagrange arc distance} does not seem so appropriate.
        That being said however, \ldots 
    \begin{enumerate}
      \item using the \fncte{Lagrange arc distance} rather than the \fncte{Euclidean metric} in 
            \prefp{item:fdie_lp_R6_rect16_euclid} yields results that are 
            \textbf{identical}\citetbl{\seessp{fdie_lp_12000m16.xlg}} %2016 June 08 Wednesday 11:18:43 PM UTC

      \item using the \fncte{Lagrange arc distance} rather than the \fncte{Euclidean metric} in 
            \prefp{item:fdie_lp_R6_hann16_euclid} yields results that \textbf{differ} at 5 locations
            (differ at approximately 0.04\% of the total possible 
            $N+M+M-1=12000+16+16-1=12031$ locations):\citetbl{\seessp{fdie_lp_12000m16.xlg}} %2016 June 08 Wednesday 11:18:43 PM UTC
            \\\indentx\begin{tabstr}{0.75}\begin{tabular}[t]{|c|c|c|}
                         \hline
                           n   & Euclidean & Lagrange
                         \\\hline
                              430 & $\dieA$ & $\dieE$
                         \\  2181 & $\dieE$ & $\dieC$
                         \\  5055 & $\dieB$ & $\dieA$
                         \\\hline
                       \end{tabular}\end{tabstr}
              \indentx\begin{tabstr}{0.75}\begin{tabular}[t]{|c|c|c|}
                         \hline
                           n   & Euclidean & Lagrange
                         \\\hline
                             8688 & $\dieD$ & $\dieC$
                         \\ 10866 & $\dieD$ & $\dieE$
                         \\\hline
                       \end{tabular}\end{tabstr}

      \item using the \fncte{Lagrange arc distance} rather than the \fncte{Euclidean metric} in 
            \prefp{item:fdie_lp_R6_hann50_euclid} yields results that are 
            \textbf{identical}\citetbl{\seessp{fdie_lp_12000m50.xlg}} %2016 June 08 Wednesday 11:23:15 PM UTC

      \item using the \fncte{Lagrange arc distance} rather than the \fncte{Euclidean metric} in 
            \prefp{item:fdie_lp_R6_rect50_euclid} yields results that \textbf{differ} at 289 locations
            (differ at approximately 2.4\% of the total possible 
            $12031$ locations):\citetbl{\seessp{fdie_lp_12000m50.xlg}} %2016 June 08 Wednesday 11:23:15 PM UTC
    \end{enumerate}

  \item Empirical evidence observed in items
                    \ref{item:fdie_lp_R6_rect16_euclid},
                    \ref{item:fdie_lp_R6_hann16_euclid},  
                    \ref{item:fdie_lp_R6_hann50_euclid}, and
                    \ref{item:fdie_lp_R6_rect50_euclid},
        suggests that the $\R^6$ technique of this example leads to about 10\% more correlation 
        than the $\R^3$ technique of \prefpp{ex:rdie_lp}.
\end{enumerate}
\end{example}






  %============================================================================
% Daniel J. Greenhoe
% LaTeX file
%============================================================================
%=======================================
\subsection{High pass filtering}
\label{sec:hp}
%=======================================
%---------------------------------------
\begin{example}[\exmd{high pass filtering of weighted real die sequence}]
\label{ex:wrdie_hp}
\addcontentsline{toc}{subsubsection}{* high pass filtering of weighted real die sequence}
\mbox{}\\
%---------------------------------------
\begin{enumerate}
  \item \label{item:wrdie_hp_R1_rect50}
        Consider a length $50(1200+2)-(50-1)=60051$ non-uniformly distributed die sequence generated as
        described in \prefpp{ex:wrdie_sha}.
        To remove the strong $\dieE$ bais, we could map and \ope{filter} \xref{def:filter} the sequence with the 
        \ope{length 50 high pass rectangular sequence} \xref{def:hp_rect}.
        Such filtering will obviously introduce correlation into the die sequence. 
        The low pass filtering of \prefp{ex:rdie_lp} (``smoothing") also introduced correlation,
        but wanting a ``smooth" sequence informally implies a willingness to accept a highly correlated sequence.
        However in this current example, we would prefer to have an \prope{uncorrelated} sequence.
        To negate the correlation introduced by filtering, 
        we \ope{down sample} \xref{def:downsample} the filtered sequence by a factor of 50 and
        remove the first and last element, leaving a sequence of length $1200$.

  \item \label{item:wrdie_hp_R1_rect50_euclid}
        If the filtering and downsampling described in \pref{item:wrdie_hp_R1_rect50} 
        is performed in the traditional $\R^1$ space,
        then after mapping back to a \structe{die sequence} using the \fncte{Euclidean metric},
        we obtain the result partially dispayed here\ldots
          \\\includegraphics{../common/math/sspplots/wrdie_hp_1200_R1_rect50_euclid_seq.pdf}\\
        where the bias at $\dieE$ has been replaced by a new bias at $\dieA$, 
        as illustrated quantitatively below on the left, 
        calculated over $\xN=1200$ elements. 
          \\\begin{tabular}{|>{\scs}c|>{\scs}c|}
               \hline
               \includegraphics{../common/math/sspplots/wrdie_hp_1200_R1_rect50_euclid_histo.pdf}
              &\includegraphics{../common/math/sspplots/wrdie_hp_1200_R1_rect50_euclid_auto.pdf}
             \\\hline
          \end{tabular}

  \item \label{item:wrdie_hp_R3_rect50_larc}
        Alternatively, suppose we next use the \ope{$\R^3$ die random variable} \xref{def:rv_dieR3}
        to map the die sequence into $\R^3$.
        \ope{Filter}ing this new sequence using the \ope{length 50 rectangular high pass sequence}
        in the \structe{$\R^3$ distance linear space} \xref{def:dieR3oml} 
        and then mapping back to a \fncte{die sequence} %over the \structe{real die outcome subspace} 
        using the \fncte{Lagrange arc distance} yields the following results:
        \\\includegraphics{../common/math/sspplots/wrdie_hp_1200_R3_rect50_larc_seq.pdf}
        \\\begin{tabular}{|>{\scs}c|>{\scs}c|}
             \hline
             \includegraphics{../common/math/sspplots/wrdie_hp_1200_R3_rect50_larc_histo.pdf}
            &\includegraphics{../common/math/sspplots/wrdie_hp_1200_R3_rect50_larc_auto.pdf}
           \\\hline
        \end{tabular}\\
        Note that neither the $\R^1$ method of \pref{item:wrdie_hp_R1_rect50_euclid}
        nor the $\R^3$ method of \pref{item:wrdie_hp_R3_rect50_larc}
        yields a uniformly distributed sequence; 
        but the $\R^3$ method at least comes significantly closer to this end.
        Moreover, the $\R^3$ method also yields a sequence that is less correlated.

  \item \label{item:wrdie_hp_R3_hann50_larc}
        Replacing the \ope{length 50 rectangular high pass filter} in \pref{item:wrdie_hp_R3_rect50_larc}
        with the \ope{length 50 Hanning high pass filter} \xref{def:hp_hann} yields a different sequence
        with similar distribution but is slightly more correlated:
        \\\includegraphics{../common/math/sspplots/wrdie_hp_1200_R3_hann50_larc_seq.pdf}
        \\\begin{tabular}{|>{\scs}c|>{\scs}c|}
             \hline
             \includegraphics{../common/math/sspplots/wrdie_hp_1200_R3_hann50_larc_histo.pdf}
            &\includegraphics{../common/math/sspplots/wrdie_hp_1200_R3_hann50_larc_auto.pdf}
           \\\hline
        \end{tabular}

  \item \label{item:wrdie_hp_R3_euclid}
        Replacing the \fncte{Lagrange arc distance} by the \fncte{Euclidean metric} in this example
        has very little effect, even before downsampling.
        Before downsampling, the length of each sequence is $M(N+2)=50(1202)=60100$ elements.
        More details follow:
        \begin{enumerate}
          \item Using the \fncte{Euclidean metric} rather than the \fncte{Lagrange arc distance} 
                in \pref{item:wrdie_hp_R3_rect50_larc} yields results that are 
                \textbf{identical}.\footnote{\seessp{wrdie_hp_1200m50.xlg}} %2016 June 08 Wednesday 11:27:40 PM UTC

          \item \label{item:wrdie_hp_R3_hann50_euclid}
                Using the \fncte{Euclidean metric} rather than the \fncte{Lagrange arc distance} 
                in \pref{item:wrdie_hp_R3_hann50_larc}
                yields results that \textbf{differ} at 4 locations 
                (approximately 0.007\% of all the locations).\footnote{\seessp{wrdie_hp_1200m50.xlg}} %2016 June 08 Wednesday 11:27:40 PM UTC

        \end{enumerate}

  \item For the type of sequence processing described in this example, 
         \pref{item:wrdie_hp_R3_euclid} very informally \emph{suggests} the following:
          \begin{enumerate}
            \item The processing is not highly sensitive to the choice of distance function.
            \item The processing is not heavily dependent on the \prope{triangle inequality}.
            %\item The sensitivity to the choice of distance function, at least in the Hanning case, 
            %      decreases with filter length.
          \end{enumerate}
\end{enumerate}
\end{example}

%=======================================
%\subsubsection{High pass filtering of spinner sequences}
%=======================================
%---------------------------------------
\begin{example}[\exmd{high pass filtering of weighted spinner sequence}]
\label{ex:spin_rhp}
\addcontentsline{toc}{subsubsection}{* high pass filtering of weighted spinner sequence}
\mbox{}\\
%---------------------------------------
\begin{enumerate}
  \item \label{item:wspin_hp_seq} \label{item:wspin_hp_rect50}
    Consider a length $50(1200+2)-(50-1)=60051$  non-uniformly distributed \fncte{spinner sequence} 
    generated as described in \prefpp{ex:wspin_sha}.
    To remove the strong $\spinE$ bais, we could \ope{filter} the \fncte{sequence} with 
    the \fncte{length 50 high pass rectangular sequence} %\xref{def:hp_rect}. 
    and down sample the filtered sequence by a factor of 50, as described in \prefpp{ex:wrdie_hp}.

  \item \label{item:wspin_hp_rect50_R1}
        If the filtering described in \pref{item:wspin_hp_rect50} is performed in the traditional $\R^1$ space,
        then after mapping back to a \structe{spinner sequence}
        using the \fncte{Euclidean metric},
        we obtain the result partially dispayed here\ldots
          \\\includegraphics{../common/math/sspplots/wspin_hp_1200_R1_rect50_euclid_seq.pdf}\\
        where the bias at $\spinE$ has been replaced by a new bias at $\spinA$, 
        as illustrated quantitatively below on the left, 
        calculated over 1200 elements. %$\floor{(160000+50-1)/50}-1=3199$ elements.
          \\\begin{tabular}{|>{\scs}c|>{\scs}c|}
               \hline
               \includegraphics{../common/math/sspplots/wspin_hp_1200_R1_rect50_euclid_histo.pdf}
              &\includegraphics{../common/math/sspplots/wspin_hp_1200_R1_rect50_euclid_auto.pdf}
             %\\\mc{2}{|>{\scs}c|}{calculated over $\floor{(160000+50-1)/50}-1=3199$ elements}
             \\\hline
          \end{tabular}


  \item \label{item:wspin_hp_hann50}\label{item:wspin_hp_hann50_R1}
        If we replace the \ope{length 50 rectangular high pass filter} of \pref{item:wspin_hp_rect50_R1}
        with a \ope{length 50 Hanning high pass filter}
        then we obtain the result partially dispayed here\ldots
          \\\includegraphics{../common/math/sspplots/wspin_hp_1200_R1_hann50_euclid_seq.pdf}\\
        \ldots where the bias at $\spinE$ again has been replaced by a new bias at $\spinA$:
          \\\begin{tabular}{|>{\scs}c|>{\scs}c|}
               \hline
               \includegraphics{../common/math/sspplots/wspin_hp_1200_R1_hann50_euclid_histo.pdf}
              &\includegraphics{../common/math/sspplots/wspin_hp_1200_R1_hann50_euclid_auto.pdf}
             \\\hline
          \end{tabular}


  \item \label{item:wspin_hp_rect50_R2_larc}
        If the rectangular filtering in $\R^1$ of \pref{item:wspin_hp_rect50_R1}
        is instead performed in $\R^2$ 
        and mapped back to a \structe{spinner sequence} using the \fncte{Lagrange arc distance},
        then we obtain the result partially dispayed here:
        \\\includegraphics{../common/math/sspplots/wspin_hp_1200_R2_rect50_larc_seq.pdf}
        \\\begin{tabular}{|>{\scs}c|>{\scs}c|}
             \hline
             \includegraphics{../common/math/sspplots/wspin_hp_1200_R2_rect50_larc_histo.pdf}
            &\includegraphics{../common/math/sspplots/wspin_hp_1200_R2_rect50_larc_auto.pdf}
           \\\hline
        \end{tabular}\\
        %\\\includegraphics{../common/math/sspplots/wspin_hp_1200_R2_rect50_euclid_seq.pdf}
        %\\\begin{tabular}{|>{\scs}c|>{\scs}c|}
        %     \hline
        %     \includegraphics{../common/math/sspplots/wspin_hp_1200_R2_rect50_euclid_histo.pdf}
        %    &\includegraphics{../common/math/sspplots/wspin_hp_1200_R2_rect50_euclid_auto.pdf}
        %   \\\hline
        %\end{tabular}\\
        Note that neither the $\R^1$ methods 
        (described in \pref{item:wspin_hp_rect50_R1} and \pref{item:wspin_hp_hann50_R1})
        nor the $\R^2$ method (described in \pref{item:wspin_hp_rect50_R2_larc})
        yields a uniformly distributed sequence; 
        but the $\R^2$ method at least comes significantly closer to this end.

  \item \label{item:wspin_hp_rect50_R2_euclid}
        Replacing the \fncte{Lagrange arc distance} by the \fncte{Euclidean metric} as in \pref{item:wspin_hp_rect50_R2_larc}
        yields a sequence that differs at a total of 272 locations 
        (approximately 0.5\% of the locations).\footnote{\seessp{wspin_hp_1200m50.xlg}} %2016 June 08 Wednesday 11:31:09 PM UTC

  \item \label{item:wspin_hp_hann50_R2_larc}
        If instead of using the rectangular filtering (as in \pref{item:wspin_hp_rect50_R2_larc}),
        we use the Hanning filtering of \pref{item:wspin_hp_hann50}
        in $\R^2$ and map back to a \structe{spinner sequence} using the \fncte{Lagrange arc distance},
        then we obtain the result partially dispayed here:
        \\\includegraphics{../common/math/sspplots/wspin_hp_1200_R2_hann50_larc_seq.pdf}
        \\\begin{tabular}{|>{\scs}c|>{\scs}c|}
             \hline
             \includegraphics{../common/math/sspplots/wspin_hp_1200_R2_hann50_larc_histo.pdf}
            &\includegraphics{../common/math/sspplots/wspin_hp_1200_R2_hann50_larc_auto.pdf}
           \\\hline
        \end{tabular}
        %\\\includegraphics{../common/math/sspplots/wspin_hp_1200_R2_hann50_euclid_seq.pdf}
        %\\\begin{tabular}{|>{\scs}c|>{\scs}c|}
        %     \hline
        %     \includegraphics{../common/math/sspplots/wspin_hp_1200_R2_hann50_euclid_histo.pdf}
        %    &\includegraphics{../common/math/sspplots/wspin_hp_1200_R2_hann50_euclid_auto.pdf}
        %   \\\hline
        %\end{tabular}

  \item \label{item:wspin_hp_hann50_R2_euclid}
        Replacing the \fncte{Lagrange arc distance} by the \fncte{Euclidean metric} in \pref{item:wspin_hp_hann50_R2_larc}
        %yields a very similar result: % but different result, as illustrated next:
        yields a sequence that differs at a total of 3 locations 
        (approximately 0.005\% of the locations).\footnote{\seessp{wspin_hp_1200m50.xlg}} %2016 June 08 Wednesday 11:31:09 PM UTC
        %\\\begin{tabular}{|>{\scs}c|>{\scs}c|}
        %     \hline
        %     \includegraphics{../common/math/sspplots/wspin_hp_1200_R2_hann50_euclid_histo.pdf}
        %    &\includegraphics{../common/math/sspplots/wspin_hp_1200_R2_hann50_euclid_auto.pdf}
        %   \\\hline
        %\end{tabular}  
\end{enumerate}
\end{example}



%=======================================
%\subsubsection{High pass filtering of fair die sequences}
%=======================================
%---------------------------------------
\begin{example}[\exmd{high pass filtering of weighted die sequence}]
\label{ex:wdie_hp}
\addcontentsline{toc}{subsubsection}{* high pass filtering of weighted die sequence}
\mbox{}\\
%---------------------------------------
\begin{enumerate}
  \item \label{item:wdie_hp_seq}
        Consider a length $50(1200+2)-(50-1)=60051$  \fncte{weighted die sequence} generated as 
        described in \prefpp{ex:wdie_sha}.
        To remove the strong $\dieE$ bais, we could map and \ope{filter} the sequence with the 
        \ope{length 16 high pass rectangular sequence} \xref{ex:hp_rect}.
        To negate the correlation introduced by filtering, 
        we \ope{down sample} the filtered sequence by a factor of 16.

  \item \label{item:wdie_hp_R1_rect16_euclid}
        If the die sequence of \pref{item:wdie_hp_seq} is mapped into $\R^1$ using the
        \fncte{traditional die random variable} \xref{def:rv_dietrad}, \ope{filter}ed,
        \ope{down sample}d, and mapped back to a die sequence using the \fncte{Euclidean metric},
        we obtain the result partially dispayed here\ldots
          \\\includegraphics{../common/math/sspplots/wrdie_hp_1200_R1_rect50_euclid_seq.pdf}\\
        where the bias at $\dieE$ has been replaced by a new bias at $\dieA$, 
        as illustrated quantitatively below on the left, 
        calculated over 1200 elements. 
          \\\begin{tabular}{|>{\scs}c|>{\scs}c|}
               \hline
               \includegraphics{../common/math/sspplots/wdie_hp_1200_R1_rect16_euclid_histo.pdf}
              &\includegraphics{../common/math/sspplots/wdie_hp_1200_R1_rect16_euclid_auto.pdf}
             \\\hline
          \end{tabular}

  \item \label{item:wdie_hp_R6_rect16_euclid}
        But if instead of processing the die sequence in $\R^1$ as in \pref{item:wdie_hp_R1_rect16_euclid},
        processing is performed in $\R^6$ 
        and mapped back to a die sequence using the \fncte{Euclidean metric},
        then we obtain the result partially dispayed here:
        \\\includegraphics{../common/math/sspplots/wdie_hp_1200_R6_rect16_euclid_seq.pdf}
        \\\begin{tabular}{|>{\scs}c|>{\scs}c|}
             \hline
             \includegraphics{../common/math/sspplots/wdie_hp_1200_R6_rect16_euclid_histo.pdf}
            &\includegraphics{../common/math/sspplots/wdie_hp_1200_R6_rect16_euclid_auto.pdf}
           \\\hline
        \end{tabular}

  \item \label{item:wdie_hp_R6_hann16_euclid}
        Replacing the \ope{length 16 rectangular sequence} in \pref{item:wdie_hp_R6_rect16_euclid}
        with a \ope{length 16 Hanning sequence} in $\R^6$ yields the following results:
        \\\includegraphics{../common/math/sspplots/wdie_hp_1200_R6_hann16_euclid_seq.pdf}
        \\\begin{tabular}{|>{\scs}c|>{\scs}c|}
             \hline
             \includegraphics{../common/math/sspplots/wdie_hp_1200_R6_hann16_euclid_histo.pdf}
            &\includegraphics{../common/math/sspplots/wdie_hp_1200_R6_hann16_euclid_auto.pdf}
           \\\hline
        \end{tabular}

  \item \label{item:wdie_hp_R6_hann50_euclid}
        Replacing the \ope{length 16 Hanning sequence} in \pref{item:wdie_hp_R6_hann16_euclid}
        with a \ope{length 50 Hanning sequence} in $\R^6$ yields the following results:
        \\\includegraphics{../common/math/sspplots/wdie_hp_1200_R6_hann50_euclid_seq.pdf}
        \\\begin{tabular}{|>{\scs}c|>{\scs}c|}
             \hline
             \includegraphics{../common/math/sspplots/wdie_hp_1200_R6_hann50_euclid_histo.pdf}
            &\includegraphics{../common/math/sspplots/wdie_hp_1200_R6_hann50_euclid_auto.pdf}
           \\\hline
        \end{tabular}\\
     Note that this $\R^6$ technique yields a sequence that is about 8.7\% more correlated than yielded by the $\R^3$ technique
     used in \pref{item:wrdie_hp_R3_hann50_larc} of \prefpp{ex:wrdie_hp}.

  \item \label{item:wdie_hp_R6_rect50_euclid}
        Replacing the \ope{length 50 Hanning sequence} in \pref{item:wdie_hp_R6_hann50_euclid}
        with a \ope{length 50 rectangular sequence} in $\R^6$ yields the following results:
        \\\includegraphics{../common/math/sspplots/wdie_hp_1200_R6_rect50_euclid_seq.pdf}
        \\\begin{tabular}{|>{\scs}c|>{\scs}c|}
             \hline
             \includegraphics{../common/math/sspplots/wdie_hp_1200_R6_rect50_euclid_histo.pdf}
            &\includegraphics{../common/math/sspplots/wdie_hp_1200_R6_rect50_euclid_auto.pdf}
           \\\hline
        \end{tabular}\\
     Note that this $\R^6$ technique yields a sequence that is about 7.3\% more correlated than yielded by the $\R^3$ technique
     used in \pref{item:wrdie_hp_R3_rect50_larc} of \prefpp{ex:wrdie_hp}.

  \item As in \prefpp{ex:fdie_lp}, here again the \fncte{Lagrange arc distance} does not seem so appropriate.
        That again being said however, \ldots 
    \begin{enumerate}
      \item using the \fncte{Lagrange arc distance} rather than the \fncte{Euclidean metric} in 
            \prefp{item:wdie_hp_R6_rect16_euclid} yields results that are 
            \textbf{identical}.\footnote{\seessp{wdie_hp_1200m16.xlg}} %2016 June 08 Wednesday 11:34:23 PM UTC
      \item using the \fncte{Lagrange arc distance} rather than the \fncte{Euclidean metric} in 
            \prefp{item:wdie_hp_R6_hann16_euclid} yields results that \textbf{differ} at 17 locations
            (differ at approximately 0.09\% of the total possible $M(N+2)=16(1200+2)=19232$ 
            locations).\footnote{\seessp{wdie_hp_1200m16.xlg}} %2016 June 08 Wednesday 11:34:23 PM UTC
    \end{enumerate}

  \item Empirical evidence observed in items
                    \pref{item:wdie_hp_R6_hann50_euclid} and
                    \pref{item:wdie_hp_R6_rect50_euclid}
        suggests that the $\R^6$ technique of this example leads to about 8\% more correlation 
        than the $\R^3$ technique of \prefpp{ex:wrdie_hp}.

\end{enumerate}
\end{example}




  %============================================================================
% Daniel J. Greenhoe
% LaTeX file
%============================================================================
%=======================================
\subsection{Fourier Analysis}
%=======================================
%=======================================
%\subsubsection{Fourier analysis of die sequences}
%=======================================
%---------------------------------------
\begin{example}[\exmd{length 1200 non-stationary die sequence with 10Hz oscillating mean}]
\label{ex:nonstat34}
\addcontentsline{toc}{subsubsection}{* length 1200 non-stationary die sequence with 10Hz oscillating mean}
\mbox{}\\
%---------------------------------------
\begin{enumerate}
  \item \label{item:nonstat34_psp}
     Suppose we have a length $\xN\eqd1200$ die sequence $\seqn{x_n}$ with the following distribution:
     \\\indentx$\begin{array}{lDlD}
       \psp(\dieA)=\psp(\dieB)=\psp(\dieD)=\psp(\dieE)=\psp(\dieF)=0.15 &and& \psp(\dieC)=0.25            &     \\
       \mc{3}{M}{\qquad for $n\in\set{p+ (2m)\frac{\xM}{2}}{p=0,1,\ldots,\frac{\xM}{2}-1,\,m=0,1,2,\ldots,9}$} & and  \\
       \psp(\dieA)=\psp(\dieB)=\psp(\dieC)=\psp(\dieE)=\psp(\dieF)=0.15 &and& \psp(\dieD)=0.25 &                \\
       \mc{3}{M}{\qquad for $n\in\set{p+(2m+1)\frac{\xM}{2}}{p=0,1,\ldots,\frac{\xM}{2}-1,\,m=0,1,2,\ldots,9}$} & 
     \end{array}$\\
     where $\xM\eqd120$.
     That is, the distribution of the sequence oscillates every $\sfrac{\xM}{2}=60$ samples between one that favors $\dieC$ 
     and one that favors $\dieD$.
     Moreover, if we were to evaluate the sequence using a \ope{Discrete Fourier Transform} operator $\dft$ \xref{def:dft}, 
     we might expect to see a strong component at $\frac{\xN}{\xM}=10$ 
     (or 10 Hz---the distribution goes through 10 cycles during the course of the sequence).
  
  \item \label{item:nonstat34_R1pam}
        Suppose we first use the \fncte{PAM die random variable} \xref{def:rv_diepam} to map
        the sequence of \pref{item:nonstat34_psp} into $\R^1$.
        The magnitude of the $\dft:\R^1\to\C^1$ of the mapped sequence is as follows:
     \\\begin{tabular}{|>{\scs}c|}
          \hline
          \includegraphics{../common/math/sspplots/diedftR1_1525_1200m120.pdf}%
        \\\hline
     \end{tabular}\\
     Looking at the above result, it would be next to impossible to discern that the distribution had a significantly strong
     oscillation of 10 cycles.
     In fact, the magnitude of the DFT at 10Hz is only $0.895699$, or $10\log_{10}(0.895699)=-0.478377$ dB.
     There are exactly 456 out of a total $\sfrac{\xN}{2}=600$ values that are greater than 
     the DFT magnitude at 10Hz.\footnote{\seessp{diedft_1525_1200m120.xlg}}
     % 
     That is, to either a human observer or a machine algorithm, the 10Hz component is effectively lost in the noise.
     
  \item \label{item:nonstat34_C1qpsk}
    Suppose we next use the \fncte{QPSK die random variable} \xref{def:rv_dieqpsk} to map
    the sequence into the complex plane.
    The magnitude of the $\dft:\C^1\to\C^1$ operation on the mapped sequence is as follows:
    \\\begin{tabular}{|>{\scs}c|}
         \hline
         \includegraphics{../common/math/sspplots/diedftC1_1525_1200m120.pdf}%
       \\\hline
    \end{tabular}\\
    %Again, the 10Hz component is effectively lost in the noise.
    The magnitude of the DFT at 10Hz is $0.589990$, or $10\log_{10}(0.589990)=-2.291552$ dB.
    There are exactly 831 out of a total $\xN=1200$ values that are greater than 
    the DFT magnitude at 10Hz.\footnote{\seessp{diedft_1525_1200m120.xlg}}
    Again, the 10Hz component is effectively lost in the noise.
     
  \item \label{item:nonstat34_R6}
    Suppose we next use the \fncte{$\R^6$ die random variable} \xref{def:rv_dieR6} to map
    the sequence into $\R^6$.
    The magnitude of $\dft:\R^6\to\C^6$ of the mapped sequence is as follows:
    \\\begin{tabular}{|>{\scs}c|}
         \hline
         \includegraphics{../common/math/sspplots/diedftR6_1525_1200m120.pdf}%
       \\\hline
    \end{tabular}\\
    The magnitude at 10Hz is $1.556295$, or $10\log_{10}(1.556295)=1.920920$ dB.
    Besides the DC component (0Hz component), this is the uniquely greatest value of the 600 samples.
    And in fact, there are only 5 out of a total $\sfrac{\xN}{2}=600$ samples
    that are $0.90\times1.556295$ or greater.\footnote{\seessp{diedft_1525_1200m120.xlg} 
      The 5 largest values are the points
      $\opair{0}{14.251433}$, $\opair{10}{1.556295}$, $\opair{344}{1.513501}$, $\opair{456}{1.405843}$ and $\opair{557}{1.468970}$.
      } 
    %And, besides the DC component, the 10Hz component is the uniquely greatest value of the 600 samples.
    % dieC1dft_1200m120_20160312_044601.log
    Thus, using the $\R^6$ mapping technique of this example, 
    it is much simpler to detect the 10Hz oscillating distribution.
\end{enumerate}
\end{example}

%---------------------------------------
\begin{example}[\exmd{length 12000 non-stationary die sequence with 10Hz oscillating mean}]
\label{ex:nonstat34_12000m1200}
\addcontentsline{toc}{subsubsection}{* length 12000 non-stationary die sequence with 10Hz oscillating mean}
\mbox{}\\
%---------------------------------------
\begin{enumerate}
  \item \label{item:nonstat34_12000m1200_psp}
    Suppose we have a length $\xN\eqd12000$ die sequence $\seqn{x_n}$ with the following distribution:
    \\\indentx$\begin{array}{lDlD}
      \psp(\dieA)=\psp(\dieB)=\psp(\dieD)=\psp(\dieE)=\psp(\dieF)=0.16 &and& \psp(\dieC)=0.20            &     \\
      \mc{3}{M}{\qquad for $n\in\set{p+ (2m)\frac{\xM}{2}}{p=0,1,\ldots,\frac{\xM}{2}-1,\,m=0,1,2,\ldots,9}$} & and  \\
      \psp(\dieA)=\psp(\dieB)=\psp(\dieC)=\psp(\dieE)=\psp(\dieF)=0.16 &and& \psp(\dieD)=0.20 &                \\
      \mc{3}{M}{\qquad for $n\in\set{p+(2m+1)\frac{\xM}{2}}{p=0,1,\ldots,\frac{\xM}{2}-1,\,m=0,1,2,\ldots,9}$} & 
    \end{array}$\\
    where $\xM\eqd1200$.
    That is, the distribution of the sequence oscillates every $\sfrac{\xM}{2}=600$ samples between one that favors $\dieC$ 
    and one that favors $\dieD$.
    If we were to evaluate the sequence using the \ope{Discrete Fourier Transform} operator, 
    we again might expect to see a strong component at $\frac{\xN}{\xM}=10$ 
    (or 10 Hz---the distribution goes through 10 cycles during the course of the sequence).
  
  \item \label{item:nonstat34_12000m1200_R1pam}
    Suppose we first use the \fncte{PAM die random variable} \xref{def:rv_diepam} to map
    the sequence of \pref{item:nonstat34_12000m1200_psp} into $\R^1$.
    In the magnitude of $\dft:\R^1\to\C^1$ there are 1130 values out of a possible $\sfrac{\xN}{2}=6000$ 
    values greater than the value at 10Hz (that value being $2.174512$).\footnote{\seessp{diedft_1620_12000m1200.xlg}}
    As in \prefpp{ex:nonstat34}, the subtle 10Hz component is effectively lost in the noise.
     
  \item \label{item:nonstat34_12000m1200_C1qpsk}
    Suppose we next use the \fncte{QPSK die random variable} \xref{def:rv_dieqpsk} to map
    the sequence into the complex plane.
    There are exactly 1932 out of a total $\xN=12000$ values that are greater than the DFT value at 10Hz
    (that value being $1.348693$).\footnote{\seessp{diedft_1620_12000m1200.xlg}}
    As in \prefpp{ex:nonstat34}, the subtle 10Hz component is effectively lost in the noise.
     
  \item \label{item:nonstat34_12000m1200_R6}
    Suppose we next use the \fncte{$\R^6$ die random variable} \xref{def:rv_dieR6} to map
    the sequence into $\R^6$.
    The magnitude of $\dft:\R^6\to\C^6$ of the mapped sequence is as follows:
    \\\begin{tabular}{|>{\scs}c|}
         \hline
         \includegraphics{../common/math/sspplots/diedftR6_1620_12000m1200.pdf}%
       \\\hline
    \end{tabular}\\
    Besides the DC component, the value at 100Hz (that value being ($1.965018$) 
    is the uniquely greatest value of the $\sfrac{\xN}{2}=6000$ samples;
    and it is $10\log_{10}(1.965018/1.660189)=0.699\cdots$dB larger than the next largest value.\footnote{
    \seessp{diedft_1620_12000m1200.xlg}
    The 10 largest values are
    $\opair{    0}{44.763194}$, $\opair{   10}{ 1.965018}$, $\opair{   90}{ 1.602474}$, $\opair{ 1223}{ 1.660189}$,
    $\opair{ 1313}{ 1.555349}$, $\opair{ 2385}{ 1.551028}$, $\opair{ 3039}{ 1.550918}$, $\opair{ 4154}{ 1.563756}$,
    $\opair{ 4187}{ 1.586362}$, and $\opair{ 5147}{ 1.623052}$.}
    Thus, even though the oscillating distribution is very subtle
    (even more subtle than that of \prefpp{ex:nonstat34}), the $\R^6$ mapping technique and subsequent analysis
    are able to detect it.
\end{enumerate}
\end{example}

%---------------------------------------
\begin{example}[\exmd{length 12000 non-stationary die sequence with 100Hz oscillating mean}]
\label{ex:nonstat34_12000m120}
\addcontentsline{toc}{subsubsection}{* length 12000 non-stationary die sequence with 100Hz oscillating mean}
\mbox{}\\
%---------------------------------------
\begin{enumerate}
  \item \label{item:nonstat34_12000m120_psp}
     Suppose we have a length $\xN\eqd12000$ die sequence $\seqn{x_n}$ with the following distribution:
     \\\indentx$\begin{array}{lDlD}
       \psp(\dieA)=\psp(\dieB)=\psp(\dieD)=\psp(\dieE)=\psp(\dieF)=0.16 &and& \psp(\dieC)=0.20            &     \\
       \mc{3}{M}{\qquad for $n\in\set{p+ (2m)\frac{\xM}{2}}{p=0,1,\ldots,\frac{\xM}{2}-1,\,m=0,1,2,\ldots,9}$} & and  \\
       \psp(\dieA)=\psp(\dieB)=\psp(\dieC)=\psp(\dieE)=\psp(\dieF)=0.16 &and& \psp(\dieD)=0.20 &                \\
       \mc{3}{M}{\qquad for $n\in\set{p+(2m+1)\frac{\xM}{2}}{p=0,1,\ldots,\frac{\xM}{2}-1,\,m=0,1,2,\ldots,9}$} & 
     \end{array}$\\
     where $\xM\eqd120$.
     That is, the distribution of the sequence oscillates every $\sfrac{\xM}{2}=60$ samples between one that favors $\dieC$ 
     and one that favors $\dieD$.
     If we were to evaluate the sequence using the \ope{Discrete Fourier Transform} operator, 
     we might expect to see a strong component at $\frac{\xN}{\xM}=100$
     (or 100 Hz---the distribution goes through 100 cycles during the course of the sequence).
  

  \item \label{item:nonstat34_12000m120_R1pam}
    Suppose we first use the \fncte{PAM die random variable} \xref{def:rv_diepam} to map
    the sequence of \pref{item:nonstat34_12000m120_psp} into $\R^1$.
    In the magnitude $\dft:\R^1\to\C^1$ there are 1320 values out of a possible $\sfrac{\xN}{2}=6000$ 
    values greater than the value at 100Hz
    that value being $2.081469$).\footnote{\seessp{diedft_1620_12000m120.xlg}}
    The subtle 100Hz component is effectively lost in the noise.
     
  \item \label{item:nonstat34_12000m120_C1qpsk}
    Suppose we next use the \fncte{QPSK die random variable} \xref{def:rv_dieqpsk} to map
    the sequence into the complex plane.
    There are exactly 1555 out of a total $\xN12000$ values that are greater than the DFT value at 100Hz
    that value being $1.425427$).\footnote{\seessp{diedft_1620_12000m120.xlg}}
    The subtle 100Hz component is effectively lost in the noise.
     
  \item \label{item:nonstat34_12000m120_R6}
    Suppose we next use the \fncte{$\R^6$ die random variable} \xref{def:rv_dieR6} to map
    the sequence into $\R^6$.
    The magnitude of $\dft:\R^6\to\C^6$ of the mapped sequence is as follows:
    \\\begin{tabular}{|>{\scs}c|}
         \hline
         \includegraphics{../common/math/sspplots/diedftR6_1620_12000m120.pdf}%
       \\\hline
    \end{tabular}\\
     Besides the DC component, the value at 100Hz (that value being $2.256927$) 
     is the uniquely greatest value of the $\sfrac{\xN}{2}=6000$.
     and it is $10\log_{10}(2.256927/1.599335)=1.495\cdots$dB larger than the next largest value.\footnote{
     \seessp{diedft_1620_12000m120.xlg}
     The 10 largest values are
       $\opair{   0}{44.763060}$, 
       $\opair{  90}{ 1.577597}$, 
       $\opair{ 100}{ 2.256927}$, 
       $\opair{ 486}{ 1.599335}$, 
       $\opair{1223}{ 1.585154}$, 
       $\opair{1313}{ 1.547956}$,
       $\opair{3039}{ 1.553522}$, 
       $\opair{3162}{ 1.561863}$,
       $\opair{5147}{ 1.558487}$, and
       $\opair{5567}{ 1.533659}$.
     }
     Thus, even though the oscillating distribution is very subtle, the $\R^6$ mapping technique
     and subsequent analysis are able to detect it.
\end{enumerate}
\end{example}

%=======================================
%\subsubsection{Fourier analysis of DNA sequences}
%=======================================
\begin{figure}%
  \centering%
  \gsize%
  $\begin{array}{|r|rMl|c|c|c|c|}
   \hline
   \mc{1}{|M|}{cycle}   & \mc{3}{|M|}{domain} & \psp(\dnaA) & \psp(\dnaC)  & \psp(\dnaG) & \psp(\dnaT)  \\\hline\hline
     0 &     0 &--&   599 & 0.24  & \cellcolor{bgp}  0.28  & 0.24  & 0.24                  \\
       &   600 &--&  1199 & 0.24  &                  0.24  & 0.24  & \cellcolor{bgp}  0.28 \\\hline
     1 &  1200 &--&  1799 & 0.24  & \cellcolor{bgp}  0.28  & 0.24  & 0.24                  \\
       &  1800 &--&  2399 & 0.24  &                  0.24  & 0.24  & \cellcolor{bgp}  0.28 \\\hline
     2 &  2400 &--&  2999 & 0.24  & \cellcolor{bgp}  0.28  & 0.24  & 0.24                  \\
       &  3000 &--&  3599 & 0.24  &                  0.24  & 0.24  & \cellcolor{bgp}  0.28 \\\hline
     3 &  3600 &--&  4199 & 0.24  & \cellcolor{bgp}  0.28  & 0.24  & 0.24                  \\
       &  4200 &--&  4799 & 0.24  &                  0.24  & 0.24  & \cellcolor{bgp}  0.28 \\\hline
     4 &  4800 &--&  5399 & 0.24  & \cellcolor{bgp}  0.28  & 0.24  & 0.24                  \\
       &  5400 &--&  5999 & 0.24  &                  0.24  & 0.24  & \cellcolor{bgp}  0.28 \\\hline
  \end{array}$%
  \hfill%
  $\begin{array}{|r|rMl|c|c|c|c|}%
   \hline%
   \mc{1}{|M|}{cycle}   & \mc{3}{|M|}{domain} & \psp(\dnaA) & \psp(\dnaC)  & \psp(\dnaG) & \psp(\dnaT)  \\\hline\hline
     5 &  6000 &--&  6599 & 0.24  & \cellcolor{bgp}  0.28  & 0.24  & 0.24                  \\
       &  6600 &--&  7199 & 0.24  &                  0.24  & 0.24  & \cellcolor{bgp}  0.28 \\\hline
     6 &  7200 &--&  7799 & 0.24  & \cellcolor{bgp}  0.28  & 0.24  & 0.24                  \\
       &  7800 &--&  8399 & 0.24  &                  0.24  & 0.24  & \cellcolor{bgp}  0.28 \\\hline
     7 &  8400 &--&  8999 & 0.24  & \cellcolor{bgp}  0.28  & 0.24  & 0.24                  \\
       &  9000 &--&  9599 & 0.24  &                  0.24  & 0.24  & \cellcolor{bgp}  0.28 \\\hline
     8 &  9600 &--& 10199 & 0.24  & \cellcolor{bgp}  0.28  & 0.24  & 0.24                  \\
       & 10200 &--& 10799 & 0.24  &                  0.24  & 0.24  & \cellcolor{bgp}  0.28 \\\hline
     9 & 10800 &--& 11399 & 0.24  & \cellcolor{bgp}  0.28  & 0.24  & 0.24                  \\
       & 11400 &--& 11999 & 0.24  &                  0.24  & 0.24  & \cellcolor{bgp}  0.28 \\\hline
  \end{array}$
  \caption{Distribution used in \prefpp{ex:nonstatdna}\label{fig:nonstatdna}}
\end{figure}
%---------------------------------------
\begin{example}[\exmd{length 12000 non-stationary artificial DNA sequence with 10Hz oscillating mean}]
\label{ex:nonstatdna}
\addcontentsline{toc}{subsubsection}{* length 12000 non-stationary artificial dna sequence with 10Hz oscillating mean}
\mbox{}\\
%---------------------------------------
\begin{enumerate}
  \item \label{item:nonstatdna_psp}
     Suppose we have a length $\xN\eqd12000$ die sequence $\seqn{x_n}$ with the following distribution (see also \prefp{fig:nonstatdna}):
     \\\indentx$\begin{array}{lDlD}
       \psp(\dnaA)=\psp(\dnaT)=\psp(\dnaG)=0.24 &and& \psp(\dnaC)=0.28            &     \\
       \mc{3}{M}{\qquad for $n\in\set{p+ 2m   \frac{\xM}{2}}{p=0,1,\ldots,\frac{\xM}{2}-1,\,m=0,1,2,\ldots,9}$} & and  \\
       \psp(\dnaA)=\psp(\dnaC)=\psp(\dnaG)=0.24 &and& \psp(\dnaT)=0.28            &     \\
       \mc{3}{M}{\qquad for $n\in\set{p+(2m+1)\frac{\xM}{2}}{p=0,1,\ldots,\frac{\xM}{2}-1,\,m=0,1,2,\ldots,9}$} & 
     \end{array}$\\
     where $\xM\eqd1200$.
     That is, the distribution of the sequence oscillates every $\sfrac{\xM}{2}=600$ samples between one that favors $\dnaC$ 
     and one that favors $\dnaT$.
     Moreover, if we were to evaluate the sequence using a \ope{Discrete Fourier Transform} (DFT) operator, 
     we might expect to see a strong component at $\frac{\xN}{\xM}=10$ 
     (or 10 Hz---the distribution goes through 10 cycles during the course of the sequence).
  
  \item \label{item:nonstatdna_R1pam}
    Suppose we first use the \fncte{PAM DNA random variable} \xref{def:rv_dnapam} to map
    the DNA sequence into $\R^1$.
    The magnitude of $\dft:\R^1\to\C^1$ of the sequence after applying this mapping is as follows:
    \\\begin{tabular}{|>{\scs}c|}
         \hline
         \includegraphics{../common/math/sspplots/dnadftR1.pdf}%
       \\\hline
    \end{tabular}\\
    %Looking at the above result, it would be next to impossible to discern that the distribution had a significantly strong
    %oscillation of 10 cycles.
    The magnitude of the DFT at 10Hz is only $1.163575$ ($10\log_{10}(1.163575)=0.657944$ dB).
    There are exactly 2023 out of a total $\sfrac{\xN}{2}=6000$ values that are greater than the DFT value at 10Hz
    (that value being $1.163575$).\footnote{\seessp{dnadft_12000m1200.xlg}}
    Here again, the 10Hz component is effectively lost in the noise.
    
  \item \label{item:nonstatdna_C1qpsk}
    Suppose we next use the \fncte{QPSK DNA random variable} \xref{def:rv_dieqpsk} to map
    the DNA sequence into the complex plane.
    The magnitude of $\dft:\C^1\to\C^1$ of the sequence after applying this mapping is as follows:
    \\\begin{tabular}{|>{\scs}c|}
         \hline
         \includegraphics{../common/math/sspplots/dnadftC1.pdf}%
       \\\hline
    \end{tabular}\\
    The DFT at 10Hz is $1.888671$, or $10\log_{10}(1.888671)=2.761563$ dB.
    There are exactly 343 out of a total $\xN=6000$ values that are greater than the DFT value at 10Hz.
    (that value being $1.888671$).\footnote{\seessp{dnadft_12000m1200.xlg}}
    Using this mapping it would be difficult to detect the subtle but significant 10Hz component.
     
  \item \label{item:nonstatdna_R4}
    Suppose we next use the \fncte{$\R^4$ DNA random variable} \xref{def:rv_dnaR4} to map
    the sequence into $\R^4$.
    The magnitude of $\dft:\R^4\to\C^4$ of the mapped sequence is as follows:
     \\\begin{tabular}{|>{\scs}c|}
          \hline
          \includegraphics{../common/math/sspplots/dnadftR4.pdf}%
        \\\hline
     \end{tabular}\\
     The magnitude of the DFT at 10Hz is $1.932042$ ($10\log_{10}(1.932042)=2.860166$ dB).
     Besides itself and the DC component, there are only two out of a total $\sfrac{\xN}{2}=6000$ samples
     that are greater or equal to this value.\footnote{
      \seessp{dnadft_12000m1200.xlg}\\
       The 4 largest values are at
         $\opair{   0}{54.791926}$, 
         $\opair{  10}{ 1.932042}$, 
         $\opair{4187}{ 1.962836}$, and 
         $\opair{5147}{ 2.057553}$.
       } 
     Thus, using the $\R^4$ mapping technique and subsequent analysis of this example, 
     it is much simpler to detect the 10Hz oscillation.
\end{enumerate}
\end{example}

%---------------------------------------
\begin{example}[\exmd{Fourier analysis of Ebola DNA sequence}]
\label{ex:dftebola}
\addcontentsline{toc}{subsubsection}{* Fourier analysis of Ebola DNA sequence}
\mbox{}\\
%---------------------------------------
\begin{enumerate}
  \item \label{item:dftebola_psp}
     Consider the Ebola DNA sequence described in \prefpp{ex:dna_ebola}.\footnote{\citeD{ncbiEbola}}
     DNA sequences commonly exhibit a strong DFT harmonic component at $\sfrac{2\pi}{3}$ 
     radians.\footnote{\citePp{galleani2010}{771}}
  
  \item \label{item:dftebola_R1pam}
    Suppose we first use the \fncte{PAM DNA random variable} \xref{def:rv_dnapam} to map
    the DNA sequence into $\R^1$.
    The magnitude of $\dft:\R^1\to\C^1$ of the sequence after applying this mapping is as follows:
    \\\begin{tabular}{|>{\scs}c|}
         \hline
         \includegraphics{../common/math/sspplots/dna_AF086833_ebola_dftR1.pdf}%
       \\\hline
    \end{tabular}\\
    The component at $\sfrac{2\pi}{3}$ is easy to pick out with a signal to noise ratio (SNR) of\\ 
    $10\log_{10}(4.290296/1.123163)\approx5.8$ dB.\footnote{\seessp{dna_AF086833_ebola_dft.xlg}}
    Here, the noise value 1.123163 is the \fncte{RMS} (\fncte{root mean square}) of the DFT magnitude 
    sequence from $n=1$ to $n=\xN/2-1$ computed as follows:
    \\\indentx$\ds\sqrt{\frac{1}{\xN/2-1}\sum_{n=1}^{n=\xN/2-1} x_n^2}$.
     
  \item \label{item:dftebola_C1qpsk}
    Suppose we next use the \fncte{QPSK DNA random variable} \xref{def:rv_dieqpsk} to map
    the dna sequence into the complex plane.
    The magnitude of $\dft:\C^1\to\C^1$ of the sequence after applying this mapping is as follows:
     \\\begin{tabular}{|>{\scs}c|}
          \hline
          \includegraphics{../common/math/sspplots/dna_AF086833_ebola_dftC1.pdf}%
        \\\hline
     \end{tabular}\\
    The component at $\sfrac{2\pi}{3}$ is again easy to pick out with a signal to noise ratio (SNR) of\\
    $10\log_{10}(6.412578/0.998659)\approx8.1$ dB.\footnote{\seessp{dna_AF086833_ebola_dft.xlg}}
    Here, the noise value 0.998659 is the \fncte{RMS} of the DFT magnitude 
    sequence from $n=1$ to $n=\xN-1$.
    %C^1 rms = 0.998685, max=6.412578   8.0760413380372487836054657324457 dB

  \item \label{item:dftebola_R4}
    Suppose we next use the \fncte{$\R^4$ DNA random variable} \xref{def:rv_dnaR4} to map
    the sequence into $\R^4$.
    The magnitude of $\dft:\R^4\to\C^4$ of the mapped sequence is as follows:
    \\\begin{tabular}{|>{\scs}c|}
         \hline
         \includegraphics{../common/math/sspplots/dna_AF086833_ebola_dftR4.pdf}%
       \\\hline
    \end{tabular}\\
    The component at $\sfrac{2\pi}{3}$ is again easy to pick out with a signal to noise ratio (SNR) of\\ 
    $10\log_{10}(3.944811/0.860665)\approx 6.6$ dB.\footnote{\seessp{dna_AF086833_ebola_dft.xlg}}
    Here, the noise value 0.860665 is the \fncte{RMS} of the DFT magnitude 
    sequence from $n=1$ to $n=\xN/2-1$.
    %R^4 rms = 0.860710, max=3.944811    6.6116935177820583786215400221832 dB

  \item In conclusion, for this application, there is only a small advantage to using 
        the $\R^4$ mapping \xrefr{item:dftebola_R4}
        versus the $\R^1$ mapping \xrefr{item:dftebola_R1pam}, 
        and even a demonstrable disadvantage when compared to the $\C^1$ mapping \xrefr{item:dftebola_C1qpsk}.
\end{enumerate}
\end{example}


%---------------------------------------
\begin{example}[\exmd{Fourier analysis of SARS-CoV-2 DNA sequence}]
\label{ex:dftSARS-CoV-2}
\addcontentsline{toc}{subsubsection}{* Fourier analysis of SARS-CoV-2 DNA sequence}
\mbox{}\\
%---------------------------------------
\begin{enumerate}
  \item \label{item:dftSARS-CoV-2_psp}
     Consider the SARS-CoV-2 DNA sequence described in \prefpp{ex:dna_SARSCoV2}.\footnote{\citeD{ncbiSARSCoV2}}
     %DNA sequences commonly exhibit a strong DFT harmonic component at $\sfrac{2\pi}{3}$ 
     %radians.\footnote{\citePp{galleani2010}{771}}
  
  \item \label{item:dftSARS-CoV-2_R1pam}
    %Suppose we first use the \fncte{PAM DNA random variable} \xref{def:rv_dnapam} to map
    %the DNA sequence into $\R^1$.
    The magnitude of $\dft:\R^1\to\C^1$ of the sequence after applying this mapping is as follows:
    \\\begin{tabular}{|>{\scs}c|}
         \hline
         \includegraphics{../common/math/graphics/pdfs/dna_SARSCoV2_dftR1.pdf}%
       \\\hline
    \end{tabular}
    %The component at $\sfrac{2\pi}{3}$ is easy to pick out with a signal to noise ratio (SNR) of\\ 
    %$10\log_{10}(4.290296/1.123163)\approx5.8$ dB.\footnote{\seessp{dna_AF086833_SARS-CoV-2_dft.xlg}}
    %Here, the noise value 1.123163 is the \fncte{RMS} (\fncte{root mean square}) of the DFT magnitude 
    %sequence from $n=1$ to $n=\xN/2-1$ computed as follows:
    %\\\indentx$\ds\sqrt{\frac{1}{\xN/2-1}\sum_{n=1}^{n=\xN/2-1} x_n^2}$.
     
  \item \label{item:dftSARS-CoV-2_C1qpsk}
    %Suppose we next use the \fncte{QPSK DNA random variable} \xref{def:rv_dieqpsk} to map
    %the dna sequence into the complex plane.
    The magnitude of $\dft:\C^1\to\C^1$ of the sequence after applying this mapping is as follows:
     \\\begin{tabular}{|>{\scs}c|}
          \hline
          \includegraphics{../common/math/graphics/pdfs/dna_SARSCoV2_dftC1.pdf}%
        \\\hline
     \end{tabular}
    %The component at $\sfrac{2\pi}{3}$ is again easy to pick out with a signal to noise ratio (SNR) of\\
    %$10\log_{10}(6.412578/0.998659)\approx8.1$ dB.\footnote{\seessp{dna_AF086833_SARS-CoV-2_dft.xlg}}
    %Here, the noise value 0.998659 is the \fncte{RMS} of the DFT magnitude 
    %sequence from $n=1$ to $n=\xN-1$.
    %C^1 rms = 0.998685, max=6.412578   8.0760413380372487836054657324457 dB

  \item \label{item:dftSARS-CoV-2_R4}
    %Suppose we next use the \fncte{$\R^4$ DNA random variable} \xref{def:rv_dnaR4} to map
    %the sequence into $\R^4$.
    The magnitude of $\dft:\R^4\to\C^4$ of the mapped sequence is as follows:
    \\\begin{tabular}{|>{\scs}c|}
         \hline
         \includegraphics{../common/math/graphics/pdfs/dna_SARSCoV2_dftR4.pdf}%
       \\\hline
    \end{tabular}
    %The component at $\sfrac{2\pi}{3}$ is again easy to pick out with a signal to noise ratio (SNR) of\\ 
    %$10\log_{10}(3.944811/0.860665)\approx 6.6$ dB.\footnote{\seessp{dna_AF086833_SARS-CoV-2_dft.xlg}}
    %Here, the noise value 0.860665 is the \fncte{RMS} of the DFT magnitude 
    %sequence from $n=1$ to $n=\xN/2-1$.
    %R^4 rms = 0.860710, max=3.944811    6.6116935177820583786215400221832 dB

  %\item In conclusion, for this application, there is only a small advantage to using 
  %      the $\R^4$ mapping \xrefr{item:dftSARS-CoV-2_R4}
  %      versus the $\R^1$ mapping \xrefr{item:dftSARS-CoV-2_R1pam}, 
  %      and even a demonstrable disadvantage when compared to the $\C^1$ mapping \xrefr{item:dftSARS-CoV-2_C1qpsk}.
\end{enumerate}
\end{example}



  %============================================================================
% Daniel J. Greenhoe
% LaTeX file
%============================================================================
%=======================================
\subsection{Wavelet Analysis}
%=======================================
In this section, we use what is in \emph{essense} wavelet analysis,
but yet is not truly wavelet analysis in the strict sense:
\begin{enumerate}
  \item For starters, standard \fncte{wavelet}s and their associated \fncte{scaling function}s are not sequences \xref{def:sequence},
        but rather are functions with domain $\R$ (not $\Z$ or some convex subset of $\Z$).
  \item While it is true that the celebrated \ope{Fast Wavelet Transform} (FWT) does work \emph{internally} with sequences
        (using \ope{filter bank}s),\footnote{\citerppgc{greenhoe2013wsd}{360}{364}{0983801134}{J.6 Filter Banks}} 
        the FWT is actually defined to work on functions with domain $\R$;
        and so the function to be analyzed by the FWT must first be \ope{sampled} by 
        a \fncte{scaling function}, which yields a sequence that can be processed by the 
        \ope{filter bank}s.\footnote{\citerppgc{greenhoe2013wsd}{369}{372}{0983801134}{Appendix L}}
  \item Wavelet analysis is typically performed by translating the wavelet or scaling function 
        by fixed amounts depending on the ``scale" of the given wavelet.
        For example, a Haar wavelet of length 4000 would typically ``jump" in offsets of 4000:
        0, 4000, 8000, 12000, \ldots.\footnote{\citerppgc{greenhoe2013wsd}{27}{62}{0983801134}{Chapter 2. The Structure of Wavelets}}
        As one might imagine, this may be reason for concern if you are using this wavelet to 
        perform edge detection (you might jump over and miss detecting the edge).
        In this section, wavelet sequences are translated by offsets of 1, making an edge harder to miss.
\end{enumerate}
%=======================================
%\subsubsection{Wavelet analysis of die sequences}
%=======================================
%---------------------------------------
\begin{example}[\exmd{statistical edge detection using Haar wavelet on non-stationary die sequence}]
\label{ex:nonstat48}
\addcontentsline{toc}{subsubsection}{* statistical edge detection using Haar wavelet on non-stationary die sequence}
\mbox{}\\
%---------------------------------------
\begin{enumerate}
  \item \label{item:nonstat48_psp}
     Suppose we have a length $\xN\eqd12000$ die sequence $\seqn{x_n}$ with the following distribution:
     \\\indentx$\begin{array}{lD}
       \psp(\dieA)=\psp(\dieB)=\psp(\dieC)=\psp(\dieD)=\psp(\dieE)=\psp(\dieF)=\frac{1}{6}                             & for $n\in\intcc{0}{3999}\setu\intcc{8000}{11999}$\\
       \psp(\dieA)=\psp(\dieB)=\psp(\dieC)=\psp(\dieE)=\psp(\dieF)=\frac{1}{10} \;\text{and}\; \psp(\dieD)=\frac{1}{2} & for $n\in\intcc{4000}{7999}$
     \end{array}$\\
     That is, the distribution of the sequence is uniformly distributed in the first and last thirds, 
     but biased towards $\dieC$ in the middle third.
     In this example we use a simple statistical edge detector to try to find the statistical ``edges" at 4000 and 8000.
     The edge detector here is a 
     \ope{filter} operation $\opW$ \xref{def:filter} using a \fncte{length 200 Haar wavelet sequence} \xref{def:whaar}.\footnote{
     Empirical evidence due to \citeP{singh1997} suggests that the Haar wavelet performs better
     than several other common wavelets as an edge detector.}
  
  \item \label{item:nonstat48_R1pam}
    Suppose we first use the \fncte{PAM die random variable} \xref{def:rv_diepam} to map
    the sequence of \pref{item:nonstat48_psp} into $\R^1$.
    The magnitude of $\opW:\R^1\to\C^1$ of the mapped sequence is as follows:\footnote{
    Note that the plot in \pref{item:nonstat48_R1pam} 
    has been down sampled by a factor of 10 for practical reasons of displaying the very large data set.}
    \\\begin{tabular}{|>{\scs}c|}
         \hline
         \includegraphics{../common/math/sspplots/diehaarR1_12000m4000_h200_1050_D10.pdf}%
       \\\hline
    \end{tabular}\\
     We might expect to see strongest evidence of the edges at $4000+200/2=4100$ and $8100$.
     But looking at the above result, this is not apparent. 
     In fact, there are a total of 10646 values that are greater than or equal to the value at location 4100 
     (that value being $0.015$).\footnote{\seessp{diehaar_12000m4000_h200_1050.xlg}} 

  \item \label{item:nonstat48_C1qpsk}
    Suppose we next use the \fncte{QPSK die random variable} \xref{def:rv_dieqpsk} to map
    the die sequence into the complex plane.
    The magnitude of $\opW:\C^1\to\C^1$ of the mapped sequence is as follows:
    \\\begin{tabular}{|>{\scs}c|}
         \hline
         \includegraphics{../common/math/sspplots/diehaarC1_12000m4000_h200_1050_D10.pdf}%
       \\\hline
    \end{tabular}\\
     Using this method, the edges are apparent.   
     And the value of the peak at $n=4083$ (with value $0.223383$) is about
     %$10\log_{10}(0.223383/0.072097)\approx4.9$ dB above the noise floor.\footnote{\seessp{diehaar_12000m4000_h200_1050.xlg}} 
     $10\log_{10}(0.223383/0.072741)\approx4.9$ dB above the noise 
     floor.\footnote{\seessp{diehaar_12000m4000_h200_1050.xlg}} 
    Here, the noise value 0.072741 is the \fncte{RMS} (see \pref{item:dftebola_R1pam} of \prefp{ex:dftebola})
    of the DFT magnitude sequence computed over the domain $n=200\ldots\xN-1$.
     %4.9113292531261044618521061907862 dB
     %above the next highest peak, which occurs at $n=521$ (with value $0.183576$).
%max|W(C^1 sequence)| from n=4050..4150 is 0.223383:
%rms|W(C^1 sequence)| from n=200..11999 is 0.072744:
%4.8725294107136569466927663732011 dB

     
  \item \label{item:nonstat48_R4}
    Suppose we next use the \fncte{$\R^6$ die random variable} \xref{def:rv_dieR6} to map
    the sequence into $\R^6$.
    The magnitude of $\opW:\R^6\to\R^6$ of the mapped sequence is as follows:
     \\\begin{tabular}{|>{\scs}c|}
          \hline
          \includegraphics{../common/math/sspplots/diehaarR6_12000m4000_h200_1050_D10.pdf}%
        \\\hline
     \end{tabular}\\
     Using this method, the edges are also apparent.
     And the value of the peak at $n=4102$ (with value $0.209165$) is about 
     %$10\log_{10}(0.209165/0.068596)\approx4.8$ dB above the noise floor.
     $10\log_{10}(0.209165/0.065768)\approx5.0$ dB above the noise 
     floor.\footnote{Here the RMS noise value is computed over the domain $n=200\ldots\xN-1$.\\ 
       \seessp{diehaar_12000m4000_h200_1050.xlg}
       } 
     This is only a slight improvement over \pref{item:nonstat48_C1qpsk}.

%max|W(R^6 sequence)| from n=4050..4150 is 0.209165:
%rms|W(R^6 sequence)| from n=0..11999 is 0.068596:
%4.8419022323079719296397098112867 dB
%max|W(R^6 sequence)| from n=4050..4150 is 0.209165:
%rms|W(R^6 sequence)| from n=200..11999 is 0.065771:
%5.0245456986608778348245061173019 dB
\end{enumerate}
\end{example}


%=======================================
%\subsubsection{Wavelet analysis of DNA sequences}
%=======================================
%---------------------------------------
\begin{example}[\exmd{statistical edge detection using Haar wavelet on non-stationary artificial DNA sequence}]
\addcontentsline{toc}{subsubsection}{* statistical edge detection using Haar wavelet on non-stationary artificial DNA sequence}
\mbox{}\\
\label{ex:}
%---------------------------------------
\begin{enumerate}
  \item 
     Suppose we have a length $\xN\eqd12000$ dna sequence $\seqn{x_n}$ with the following distribution:
     \\\indentx$\begin{array}{lD}
       \psp(\dnaA)=\psp(\dnaC)=\psp(\dnaG)=\psp(\dnaT)=\frac{1}{4}                             & for $n\in\intcc{0}{3999}\setu\intcc{8000}{11999}$\\
       \psp(\dnaA)=\psp(\dnaG)=\psp(\dnaT)=\frac{17}{100} \;\text{and}\; \psp(\dnaC)=\frac{49}{100} & for $n\in\intcc{4000}{7999}$
     \end{array}$\\
     That is, the distribution of the sequence is uniformly distributed in the first and last thirds, 
     but biased towards $\dnaC$ in the middle third.
     Just as in \pref{ex:nonstat48}, we again use 
     a filter operation $\opW$ with \fncte{length 200 Haar wavelet sequence} as 
     a simple statistical edge detector to try to locate the statistical ``edges" at 4000 and 8000.
  
  \item \label{item:dna_haar_R1pam}
    Suppose we first use the \fncte{PAM DNA random variable} \xref{def:rv_dnapam} to map
    the DNA sequence into $\R^1$.
    The magnitude of $\opW:\R^1\to\C^1$ of the mapped sequence is as follows:\footnote{
    Note that the plot in \pref{item:dna_haar_R1pam} 
    has been down sampled by a factor of 10 for practical reasons of displaying the very large data set.}
     \\\begin{tabular}{|>{\scs}c|}
          \hline
          \includegraphics{../common/math/sspplots/dnahaarR1_12000m4000_h200_1749_D5.pdf}%
        \\\hline
     \end{tabular}\\
     We might expect to see strongest evidence of the edges at or near $4000+200/2=4100$ and $8100$.
     In fact, the sequence does have peaks at 4087 (with value 0.185) and at 8087 (with value 0.230).
     The peak at 4087 is about $10\log_{10}(0.185000/0.071355)\approx 4.1$ dB above the noise floor.
     %max|W(R^1 sequence)| from n=4050..4150 is 0.185000:
     %max|W(R^1 sequence)| from n=8050..8150 is 0.230000:
     %rms|W(R^1 sequence)| from n=200..11999 is 0.071355:
     However, there are 103 other values not around the $n=4087$ and $n=8087$ peaks that are 
  0.185 or greater.
  These 102 values represent roughly 11 other peaks, each of which could trigger a 
  ``false positive" decision.\footnote{\seessp{dnahaar_12000m4000_h200_1749.xlg}}
     
  \item \label{item:dna_haar_C1}
    Suppose we next use the \fncte{QPSK DNA random variable} \xref{def:rv_dieqpsk} to map
    the dna sequence into the complex plane.
    The magnitude of $\opW:\C^1\to\C^1$ of the mapped sequence is as follows:
     \\\begin{tabular}{|>{\scs}c|}
          \hline
          \includegraphics{../common/math/sspplots/dnahaarC1_12000m4000_h200_1749_D5.pdf}%
        \\\hline
     \end{tabular}\\
     Using this method, the edges are apparent.
     And the value of the peak at $n=4086$ (with value $0.215870$) is 
     $10\log_{10}(0.215870/0.070068)\approx 4.9$ dB above the noise floor.\footnote{\seessp{dnahaar_12000m4000_h200_1749.xlg}}
     %max|W(C^1 sequence)| from n=4050..4150 is 0.215870:
     %rms|W(C^1 sequence)| from n=200..11999 is 0.070068:
     %only 
     %$10\log_{10}(0.215870/0.190263)=0.548380$dB
     %above the next highest peak, which occurs at $n=10405$ (with value $0.190263$).\footnote{\seessp{dnahaar_12000m4000_h200_1749.xlg}}
     %However it does produce two false edges at 9815 and 10405---the value at 
     %$n=4086$ is $0.215870$.
     %$n=4100$ is $0.181108$,
     %but it is greater than or equal to this value at 9815, 10405, 10407, and 10408.\footnote{
     %The values are $\opair{9815}{0.181108}$, $\opair{10405}{0.190263}$, $\opair{10407}{0.185607}$,
     %and $\opair{10408}{0.186682}$.}
     % dnahaar_12000m4000_h200_1749_20160402_095901.log


  \item \label{item:dna_haar_R4}
    Suppose we next use the \fncte{$\R^4$ DNA random variable} \xref{def:rv_dnaR4} to map
    the sequence into $\R^4$.
    The magnitude of $\opW:\R^4\to\C^4$ of the mapped sequence is as follows:
     \\\begin{tabular}{|>{\scs}c|}
          \hline
          \includegraphics{../common/math/sspplots/dnahaarR4_12000m4000_h200_1749_D5.pdf}%
        \\\hline
     \end{tabular}\\
     Using this method, the edges are also apparent.
     And the value of the peak at $n=4096$ (with value $0.181246$) is 
     $10\log_{10}(0.181246/0.059594)\approx4.8$ dB
     above the noise floor.\footnote{\seessp{dnahaar_12000m4000_h200_1749.xlg}}
     Note that this is a slight decrease in performance as compared to \pref{item:dna_haar_C1}.

     %max|W(R^4 sequence)| from n=4050..4150 is 0.181246:
     %rms|W(R^4 sequence)| from n=200..11999 is 0.059594:
     %
     %$10\log_{10}(0.181246/0.141421)=1.077545$dB
     %above the next non-starting highest peak, which occurs at $n=9815$ (with value $0.141421$).
     %Note that this is a small improvement over \pref{item:dna_haar_C1}.
     %Unlike the complex mapping in \pref{item:dna_haar_C1},
     %it shows an edge near the start of the sequence.
     %But also unlike the method of \pref{item:dna_haar_C1}, it contains no false edges in that 
     %there are no other values greater than that around the 4100 peak, 8100 peak, and starting peak.
     %In fact, the 4100 peak is $10\log_{10}(0.181246/0.141421)=1.077545$dB above the next highest peak.\footnote{
     %Compare $\opair{4096}{0.181246}$, $\opair{4100}{0.168819}$, and $\opair{9815}{0.141421}$.
     %} 


\end{enumerate}
\end{example}

%---------------------------------------
\begin{example}[\exmd{Wavelet analysis of Phage Lambda DNA sequence}]
\label{ex:dnapl}
\addcontentsline{toc}{subsubsection}{* Wavelet analysis of Phage Lambda DNA sequence}
\mbox{}\\
%---------------------------------------
\begin{enumerate}
  \item \label{item:dnapl_psp}
     Consider the Phage Lambda DNA sequence.\footnote{\citeD{ncbiPhageLambda}}
     It has a strong $\dnaC\dnaG$ bias before $n=20000$ and 
            a strong $\dnaA\dnaT$ bias after,\footnote{\citerpg{cristianini2007}{14}{1139460153}}
     as demonstrated next by mapping 
     \\\indentx$\dnaA\to+1$\qquad$\dnaT\to+1$\qquad$\dnaC\to-1$\qquad$\dnaG\to-1$\\
     and filtering the resulting sequence in $\R^1$
     with a \fncte{length 1600 Haar scaling sequence} \xref{def:shaar}---such
     a filtering operation acts as a kind of ``sliding window" histogram of the DNA sequence.
     \\\begin{tabular}{|>{\scs}c|}
          \hline
          \includegraphics{../common/math/sspplots/dna_NC001416_phagelambdaR1b_hs1600.pdf}%
        \\\hline
     \end{tabular}\\
  
  \item \label{item:dnapl_R1pam}
    Suppose we first use the \fncte{PAM DNA random variable} \xref{def:rv_dnapam} to map
    the DNA sequence into $\R^1$.
    The magnitude of the length 1600 Haar wavelet operation on the mapped sequence is as follows:
     \\\begin{tabular}{|>{\scs}c|}
          \hline
          \includegraphics{../common/math/sspplots/dna_NC001416_phagelambdaR1_h1600.pdf}%
        \\\hline
     \end{tabular}\\
    Note that it is very difficult to pick out the edge at 20000.

  \item \label{item:dnapl_C1qpsk}
    Suppose we next use the \fncte{QPSK DNA random variable} \xref{def:rv_dnaqpsk} to map
    the DNA sequence into the complex plane.
    The length 1600 Haar wavelet operation on the mapped sequence is as follows:
     \\\begin{tabular}{|>{\scs}c|}
          \hline
          \includegraphics{../common/math/sspplots/dna_NC001416_phagelambdaC1_h1600.pdf}%
        \\\hline
     \end{tabular}\\
     If one did not know apriori that there was an edge at 20000, it would still be difficult to identify.
     
  \item \label{item:dnapl_R6}
    Suppose we next use the \fncte{$\R^4$ DNA random variable} \xref{def:rv_dnaR4} to map
    the DNA sequence into $\R^4$.
    Filtering the mapped sequence with a length 1600 Haar wavelet sequence results in the following:
     \\\begin{tabular}{|>{\scs}c|}
          \hline
          \includegraphics{../common/math/sspplots/dna_NC001416_phagelambdaR4_h1600.pdf}%
        \\\hline
     \end{tabular}\\
     Here there is a clear peak near $20000$.

  \item And here is the same analysis as used in \pref{item:dnapl_R6}, but at scale 4000
        (using a length 4000 Haar wavelet filter):
     \\\begin{tabular}{|>{\scs}c|}
          \hline
          \includegraphics{../common/math/sspplots/dna_NC001416_phagelambdaR4_h4000.pdf}%
        \\\hline
     \end{tabular}\\
     Again, the peak near 20000 is quite pronounced.
     However, at the low resolution scale (of 4000), it would be difficult to determine precisely where the
     statistical edge actually was.
\end{enumerate}
\end{example}





