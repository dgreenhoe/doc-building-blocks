%============================================================================
% XeLaTeX File
% Daniel J. Greenhoe
%============================================================================

%=======================================
\chapter{Filtering Techniques}
%=======================================

%---------------------------------------
\section{Linear operators}
%---------------------------------------
\tbox{$\ds X(z)\eqd\sum_{n\in\Z} \fx[n] z^{-n}$}
\tbox{\includegraphics{graphics/ziir2.pdf}}
\tbox{$\ds Y(z)\eqd\sum_{n\in\Z} \fy[n] z^{-n}$}

\indentx\begin{tabular}{cl}
    \imark&Is it \prope{linear}?
  \\\imark&Is it \prope{time-invariant}?
  \\\imark&Is it \prope{tractable}?
\end{tabular}

\begin{align*}
  H(z)\eqd \frac{Y(z)}{X(z)}
    &=\frac{b_0 z^2 + b_1 z + b_2}{a_0 z^2 + a_1 z + a_2}
    \qquad=\frac{z^{-2}}{z^{-2}}\times\frac{b_0 z^2 + b_1 z + b_2}{a_0 z^2 + a_1 z + a_2}
  \\&=\frac{b_0 + b_1 z^{-1} + b_2z^{-2}}{a_0 + a_1 z^{-1} + a_2z^{-2}}
  \\
  \\Y(z)\brs{a_0 + a_1 z^{-1} + a_2z^{-2}}
    &= X(z) \brs{b_0 + b_1 z^{-1} + b_2z^{-2}}
\end{align*}

If $a_0=1$ then

\begin{align*}
  Y(z) &=  b_0X(z) + b_1z^{-1}X(z)  + b_2z^{-2}X(z) - a_1 z^{-1}Y(z) + a_2z^{-2}Y(z)
\end{align*}

%---------------------------------------
\section{From z-domain back to time-domain}
%---------------------------------------
\begin{figure}
  \centering
  \includegraphics[width=\tw/2-2mm]{graphics/iir2n.pdf}
  \includegraphics[width=\tw/2-2mm]{graphics/dfI_order2_156.pdf}
  \caption{Direct form 1 order 2 IIR filters\label{fig:df1iir2}}
\end{figure}
\begin{align*}
  Y(z) &=  b_0X(z) + b_1z^{-1}X(z)  + b_2z^{-2}X(z) - a_1 z^{-1}Y(z) + a_2z^{-2}Y(z)
  \\\\
  \fy[n] &= b_0\fx[n] + b_1\fx[n-1] + b_2\fx[n-2] - a_1\fy[n-1] - a_2\fy[n-2]
\end{align*}

%---------------------------------------
\begin{example}
%---------------------------------------
See \prefpp{fig:df1iir2}

$\ds{\frac{3z^2 + 5z + 7}{2z^2 + 10z + 12}}$
=
$\ds{\frac{3z^2 + 5z + 7}{2\brp{z^2 + 5z + 6}}}$
=
$\ds{\frac{\brp{\sfrac{3}{2}z^2 + \sfrac{5}{2}z + \sfrac{7}{2}}}
               {z^2 + 5z + 6}}$
=
$\ds{\frac{\brp{\sfrac{3}{2} + \sfrac{5}{2}z^{-1} + \sfrac{7}{2}z^{-2}}}
               {1 + 5z^{-1} + 6z^{-2}}}$
\end{example}

%---------------------------------------
\section{Mirroring for real coefficients}
%---------------------------------------
\begin{figure}
  \centering
  \includegraphics{graphics/pz_minphase.pdf}%
  \includegraphics{graphics/pz_realcoefs_11.pdf}
\caption{Conjugate pair structure yielding real coefficients\label{fig:realcoefs}}
\end{figure}

If you want real coefficients, choose poles and zeros in conjugate pairs (next).
%---------------------------------------
\begin{proposition}
%---------------------------------------
\propbox{
  \brb{\begin{array}{M}
    \structe{zeros} and \structe{poles}\\
    occur in \structe{conjugate pairs} 
  \end{array}}
  \quad\implies\quad
  \brb{\begin{array}{M}
    \structe{coefficients} \\
    are \prope{real}.
  \end{array}}
  }
\end{proposition}
\begin{proof}
\begin{align*}
  \brp{z-p_1}\brp{z-p_1^*}
    &= \brs{z-\brp{a+ib}} \brs{z-\brp{a-ib}}
  \\&= z^2 +\brs{-a+ib-ib-a}z - \brs{ib}^2
  \\&= z^2 -2a z + b^2
\end{align*}
\end{proof}

%---------------------------------------
\begin{example}
%---------------------------------------
See \prefpp{fig:realcoefs}.
\begin{align*}
  H(z)   &= G\frac{\brs{z-z_1}\brs{z-z_2}}
                  {\brs{z-p_1}\brs{z-p_2}}
          = G\frac{\brs{z-\brp{1+i}}\brs{z-\brp{1-i}}}
                  {\brs{z-\brp{-\sfrac{2}{3}+i\sfrac{1}{2}}}\brs{z-\brp{-\sfrac{2}{3}-i\sfrac{1}{2}}}}
       \\&= G\frac{z^2 - z\brs{\brp{1-i}+\brp{1+i}} + \brp{1-i}\brp{1+i}}
                  {z^2 - z\brs{\brp{-\sfrac{2}{3}+i\sfrac{1}{2}}+\brp{-\sfrac{2}{3}+i\sfrac{1}{2}}} + \brp{-\sfrac{2}{3}+i\sfrac{1}{2}}\brp{-\sfrac{2}{3}+i\sfrac{1}{2}}}
       \\&= G\frac{z^2 - 2z + 2}
                  {z^2 - \sfrac{4}{3}z + \brp{\sfrac{4}{3}+\sfrac{1}{4}}}
          = G\frac{z^2 - 2z + 2}
                  {z^2 - \sfrac{4}{3}z + \sfrac{19}{12}}
\end{align*}
\end{example}

%---------------------------------------
\section{The 0Hz Gain}
%---------------------------------------
%---------------------------------------
\begin{proposition}
%---------------------------------------
\propbox{
  \brp{\ZH(z) = \frac{\ds\sum_{n=0}^{\xN}b_n z^{-n}}
                     {\ds\sum_{n=0}^{\xN}a_n z^{-n}}}
  \quad\implies\quad
  \brp{\FH(0) = \frac{\ds\sum_{n=0}^{\xN}b_n }
                          {\ds\sum_{n=0}^{\xN}a_n}}
  }
\end{proposition}
{\begin{proof}
$\ds\begin{array}{rclM}
  \ds\brb{\opDTFT \fx[n]}(\omega) &\ds\eqd& \ds\sum_{n\in\Z} \fx[n] e^{-i\omega n}  & DTFT\\
  \ds\brb{\opZ    \fx[n]}(z)      &\ds\eqd& \ds\sum_{n\in\Z} \fx[n] z^{-n}          & z-Transform
\end{array}$

\begin{align*}\
  \FH(0) 
    &= \brlr{\FH(\omega)}_{\omega=0}
  \\&= \brlr{\ZH\brp{e^{i\omega}}}_{\omega=0}
  \\&= \brlr{\ZH(z)}_{z=1}
  \\&= \brlr{\frac{\ds\sum_{n=0}^{\xN}b_n z^{-n}}
            {\ds\sum_{n=0}^{\xN}a_n z^{-n}}}_{z=1}
  \\&= \frac{\ds\sum_{n=0}^{\xN}b_n }
            {\ds\sum_{n=0}^{\xN}a_n }
\end{align*}

\end{proof}}

%---------------------------------------
\section{The Fs/2 Gain}
%---------------------------------------
\tbox{$\ds X(z)$}
\tbox{\includegraphics{graphics/ziir2.pdf}}
\tbox{$\ds Y(z)$}


\[z=e^{i\omega} \qquad z=-1 \text{ at $\omega=\pi$}\]

\begin{align*}
  \brlr{H(z)}_{z=e^{i\pi}}
    &= \brlr{\frac{b_0 z^2 + b_1 z + b_2}{a_0 z^2 + a_1 z + a_2}}_{z=e^{i\pi}=-1}%
    &= \brlr{\frac{b_0 (-1)^2 + b_1 (-1) + b_2}{a_0 (-1)^2 + a_1 (-1) + a_2}}_{z=e^{i\pi}=-1}%
  \\&= \frac{b_0     - b_1   + b_2}{a_0     - a_1   + a_2}
\end{align*}

%---------------------------------------
\section{For a stable filter, put all the poles inside the unit circle}
%---------------------------------------

\begin{tabular}{cc}
    \includegraphics{graphics/pz_unstable.pdf}
   &\includegraphics{graphics/pz_stable.pdf}%
  \\\prope{unstable}                          
   &\prope{stable}
\end{tabular}


%---------------------------------------
\section{For minimum phase, put all the zeros inside the unit circle}
%---------------------------------------
\begin{tabular}{cc}
  \includegraphics{graphics/pz_realcoefs.pdf}%
  &\includegraphics{graphics/pz_minphase.pdf}%
  \\\emph{not} minimum phase & \prope{minimum phase}
\end{tabular}


The impulse response of a minimum phase filter has most of its energy concentrated
near the beginning of its support

%--------------------------------------
\thmd{Robinson's Energy Delay Theorem}
\footnote{
  \citerpg{dumitrescu2007}{36}{1402051247},
  \citor{robinson1962},  % referenced by claerbout1976
  \citorc{robinson1966}{???},  % referenced by online thesis
  \citerpp{claerbout1976}{52}{53}
  %\citerp{os}{291}\\
  %\citerp{mallat}{253}
  }
\label{thm:ztr_redp}
\index{minimum phase!energy}
\index{energy}
%--------------------------------------
\\
%Let $\fp(z)\eqd\sum_{n=0}^\xN a_n z^{-n}$ 
%and $\fq(z)\eqd\sum_{n=0}^\xN b_n z^{-n}$ 
%be polynomials.
\thmbox{
  \brb{\begin{array}{lMD}
    \fp & is \prope{minimum phase} & and\\
    \fq & is \emph{not} minimum phase & 
  \end{array}}
  \implies
  \mcom{\sum_{n=0}^{m-1} \abs{a_n}^2}{\parbox{20mm}{``energy" of the first $m$ coefficients of $\fp(z)$}} \ge 
  \mcom{\sum_{n=0}^{m-1} \abs{b_n}^2}{\parbox{20mm}{``energy" of the first $m$ coefficients of $\fq(z)$}} 
  %\qquad \forall 0\le m\le\xN
  }

\prope{Minimum phase} filters are easy to invert: each \structe{zero} becomes a \structe{pole} 
and each \structe{pole} becomes a \structe{zero}.

$\begin{array}{ccccc}
     \tbox{\includegraphics{graphics/pz_minphase.pdf}}
    &\tbox{$\times$}
    &\tbox{\includegraphics{graphics/pz_minphase_inv.pdf}}
    &\tbox{$=$}
    &\tbox{$1$}
  \\
     \ds\frac{\brp{z-z_1}\brp{z-z_2}\brp{z-z_3}\brp{z-z_4}}
             {\brp{z-p_1}\brp{z-p_2}\brp{z-p_3}\brp{z-p_4}}
    &\times
    &\ds\frac{\brp{z-p_1}\brp{z-p_2}\brp{z-p_3}\brp{z-p_4}}
             {\brp{z-z_1}\brp{z-z_2}\brp{z-z_3}\brp{z-z_4}}
    &=
    &1
\end{array}$

%---------------------------------------
\section{For more symmetry, put some zeros inside and some outside the unit circle}
%---------------------------------------
%---------------------------------------
\begin{example}
%---------------------------------------
An example of this idea is the Daubechie's \fncte{Symlet}s.\\
\exboxt{\begin{tabular}{c|c}
  Daubechies-8 & Symlet-8
  \\\hline
  \includegraphics{graphics/D8_pz.pdf}&\includegraphics{graphics/S8_pz.pdf}\\
  \includegraphics{graphics/d8_phi_h.pdf}&\includegraphics{graphics/s8_phi_h.pdf}
\end{tabular}}
\end{example}

%---------------------------------------
\section{Inverting non-minimum phase filters}
%---------------------------------------
\begin{tabular}{ccccc}
     \tbox{\includegraphics{graphics/pz_unstable2.pdf}}
    &\tbox{$\times$}&
     \tbox{\includegraphics{graphics/pz_allpass.pdf}}
    &\tbox{$=$}&
     \tbox{\includegraphics{graphics/pz_unall.pdf}}
  \\unstable&&all-pass&&stable!
\end{tabular}

\begin{align*}
  \abs{A\brp{z}}_{z=e^{i\omega}}
    &= \frac{1}{r}\abs{\frac{z-r          e^{i\phi}}
                            {z-\frac{1}{r}e^{i\phi}}}_{z=e^{i\omega}}
   &&= \abs{\frac{ z- re^{i\phi}}
                 {rz-  e^{i\phi}}}_{z=e^{i\omega}}
  \\&= \abs{e^{i\phi}\brp{
            \frac{e^{-i\phi}z-r}
                 {rz- e^{i\phi}}}}_{z=e^{i\omega}}
   &&= \abs{z\brp{
            \frac{e^{-i\phi}-rz^{-1}}
                 {rz- e^{i\phi}}}}_{z=e^{i\omega}}
  \\&= \abs{-z\brp{
            \frac{rz^{-1}- e^{-i\phi}}
                 {rz     - e^{ i\phi}}}}_{z=e^{i\omega}}
   &&= \abs{\mcom{e^{i\pi}}{$-1$}e^{i\omega}\brp{
            \frac{re^{-i\omega} - e^{-i\phi}}
                 {re^{ i\omega} - e^{ i\phi}}}}
  \\&= \abs{\frac{1}{e^{-iv}}\brp{
            \frac{re^{-i\omega} - e^{-i\phi}}
                 {\brp{re^{ i\omega} - e^{ i\phi}}^\ast}}}
   &&= \abs{\frac{re^{-i\omega} - e^{-i\phi}}
                 {re^{-i\omega} - e^{-i\phi}}}
  \\&= 1
\end{align*}

%---------------------------------------
\section{The Reappearing Roots: Now you don't see them, now you do}
%---------------------------------------

True or False? This filter has no poles:

  $\ds H(z)= b_0 + b_1 z^{-1} + b_2 z^{-2}$

\includegraphics{graphics/dfI_order2_fir.pdf}


\begin{align*}
  H(z)
    &= b_0 + b_1 z^{-1} + b_2 z^{-2}
     = \frac{z^2}{z^2} \times \frac{b_0 + b_1 z^{-1} + b_2 z^{-2}}{1}
     = \frac{b_0 z^2 + b_1 z^{1} + b_2 }{z^2}
\end{align*}

\includegraphics{graphics/pz_pole00.pdf}

%---------------------------------------
\section{Conversion from low-pass to high-pass}
%---------------------------------------
%---------------------------------------
\begin{theorem}
%---------------------------------------
\thmbox{
  \brb{\begin{array}{M}
    $\Zh(z)$ is \propb{low-pass}
  \end{array}}
  \quad\implies\quad
  \brb{\begin{array}{M}
    $\Zh(-z)$ is \propb{high-pass}
  \end{array}}
  }
\end{theorem}
\begin{proof}
  \begin{align*}
    \abs{\Fg(\omega)}^2
      &\eqd \abs{\Zh(-z)}_{z=e^{i\omega}}
      && \text{by definition of $\Fg(\omega)$}
    \\&= \abs{\Zh(e^{-i\pi}z)}_{z=e^{i\omega}}
    \\&= \abs{\Zh(z)}_{z=e^{i\omega}e^{-i\pi}}
    \\&= \abs{\Zh(z)}_{z=e^{i(\omega-\pi)}}
    \\&\eqd \abs{\Fh(\omega-\pi)}^2
      && \text{by definition of $\Fh(\omega)$}
  \end{align*}
\end{proof}

