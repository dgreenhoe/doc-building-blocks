%============================================================================
% LaTeX File
% Daniel J. Greenhoe
%============================================================================
%======================================
\chapter{Discrete Time Fourier Transform}
%======================================
%======================================
\section{Definition}
%======================================
%--------------------------------------
\begin{definition}
\label{def:dtft}
%\footnote{
%  \structe{Laurent series}: \citerpg{aa}{49}{0821821466}
%  }
%--------------------------------------
%Let $\seq{x_n}{n\in\Z}$ be a sequence in the space $\spllR$. %over a ring $\ring$. 
\defbox{\begin{array}{M}
  The \opd{discrete-time Fourier transform} $\opDTFT$ of $\seq{x_n}{n\in\Z}$ is defined as
  \\\indentx$\ds
  \brs{\opDTFT\seqn{x_n}}(\omega) \eqd \sum_{n\in\Z} x_n e^{-i\omega n}
  \qquad{\scy\forall\seqxZ{x_n}\in\spllR}
  $
\end{array}}
\end{definition}

\begin{minipage}{\tw-35mm}
If we compare the definition of the \fncte{Discrete Time Fourier Transform} \xref{def:dtft} 
to the definition of the Z-transform\ifsxref{dsp}{def:opZ},
we see that the DTFT is just a special case of the more general Z-Transform, with $z = e^{i\omega}$.
If we imagine $z\in\C$ as a complex plane, then $e^{i\omega}$ is 
a unit circle in this plane.
The ``frequency" $\omega$ in the DTFT is the unit circle in the much larger z-plane,
as illustrated to the right.
\end{minipage}%
\hfill\tbox{\includegraphics{graphics/unitcircle.pdf}}


%======================================
\section{Properties}
%======================================
%--------------------------------------
\begin{proposition}[\thmd{DTFT periodicity}]
\label{prop:dtft_2pi}
%--------------------------------------
Let $\Dx(\omega)\eqd\opDTFT\brs{\seqn{x_n}}(\omega)$ be the \fncte{discrete-time Fourier transform} \xref{def:dtft}
of a sequence $\seqxZ{x_n}$ in $\spllR$.
\propbox{
  \mcom{\Dx(\omega) = \Dx(\omega+2\pi n)}{\prope{periodic} with period $2\pi$} \qquad \forall n\in\Z
  }
\end{proposition}
\begin{proof}
\begin{align}
   \Dx\brp{\omega+2\pi n}
     & = \sum_{m\in\Z} x_m e^{-i \left(\omega+2\pi n\right)m}
     &&= \sum_{m\in\Z} x_m e^{-i\omega m} \cancelto{1}{e^{-i2\pi nm}} \nonumber
   \\&=  \sum_{m\in\Z} x_m e^{-i\omega m}
     &&= \Dx(\omega) \label{equ:qc1}
%\\
%   \Fg\left(\omega+2\pi n\right)
%     &=  \sum_{m\in\Z} g_m e^{-i(\omega+2\pi n)m}
%     &&= \sum_{m\in\Z} g_m e^{-i\omega m} e^{-i2\pi nm}
%     &&= \sum_{m\in\Z} g_m e^{-i\omega m}  \nonumber
%     &&= \Fh(\omega) \label{equ:qc1}
%   \\&=  \Fg(\omega) \label{equ:psi_qc1}
\end{align}
\end{proof}

\begin{center}
\begin{tabular}{cc}
  \includegraphics{graphics/pounityz.pdf}&\includegraphics{graphics/Fhw.pdf}
\end{tabular}
\end{center}
%--------------------------------------
\begin{proposition}
\label{prop:tzf}
\label{prop:dsp_zone}
%--------------------------------------
Let $\Zx(z)$ be the \fncte{Z-transform}\ifsxref{dsp}{def:opZ} and 
$\Dx(\omega)$ the \fncte{discrete-time Fourier transform} \xref{def:dtft} of $\seqn{x_n}$.
\propbox{
                  \mcom{\brb{\sum_{n\in\Z}  x_n  = c}}{(1) time domain}               
  \quad\iff\quad  \mcom{\brb{\Zx(z)\Big|_{z=1}   = c}}{(2) z domain}                  
  \quad\iff\quad  \mcom{\brb{\Dx(\omega)\Big|_{\omega=0} = c}}{(3) frequency domain}
  \qquad\scy\forall\seqxZ{x_n}\in\spllR,\,c\in\R
  }
\end{proposition}
\begin{proof}
\begin{enumerate}
  \item Proof that (1) $\implies$ (2):
    \begin{align*}
      \Zx(z)\Big|_{z=1}
        &= \left.\sum_{n\in\Z} x_n z^{-n} \right|_{z=1}
        && \text{by definition of $\Zx(z)$\ifsxref{dsp}{def:opZ}}
      \\&= \sum_{n\in\Z} x_n
        && \text{because $z^n=1$ for all $n\in\Z$}
      \\&= c
        && \text{by hypothesis (1)}
    \end{align*}

  %\item Proof that (2) $\implies$ (3):
  %  \begin{align*}
  %    c
  %      &= \sum_{n\in\Z} x_n
  %      && \text{by previous result}
  %    \\&= \left.\sum_{n\in\Z} x_n e^{-i\omega n} \right|_{\omega=0}
  %      && \text{because $e^0=1$}
  %    \\&= \Dx(\omega)\Big|_{\omega=0}
  %      && \text{by definition of $\Dx(\omega)$ \xref{def:dtft}}
  %  \end{align*}

  \item Proof that (2) $\implies$ (3):
    \begin{align*}
      \Dx(\omega)\Big|_{\omega=0}
        &= \left.\sum_{n\in\Z} x_n e^{-i\omega n} \right|_{\omega=0}
        && \text{by definition of $\Dx(\omega)$ \xref{def:dtft}}
      \\&= \left.\sum_{n\in\Z} x_n z^{-n} \right|_{z=1}
      \\&= \Zx(z)\Big|_{z=1}
        && \text{by definition of $\Zx(z)$\ifsxref{dsp}{def:opZ}}
      \\&= c
        && \text{by hypothesis (2)}
    \end{align*}

  \item Proof that (3) $\implies$ (1):
    \begin{align*}
      \sum_{n\in\Z} x_n
        &= \left.\sum_{n\in\Z} x_n e^{-i\omega n} \right|_{\omega=0}
      \\&= \Dx(\omega)
        && \text{by definition of $\Dx(\omega)$ \xref{def:dtft}}
      \\&= c
        && \text{by hypothesis (3)}
    \end{align*}
\end{enumerate}
\end{proof}



%=======================================
%\section{Zeros on the unit circle}
%=======================================

%-------------------------------------
\begin{proposition}
\footnote{
  \citerpg{chui}{123}{0121745848}
  }
\label{prop:dsp_zminone}
\index{zero at $-1$}
%-------------------------------------
%Let $\seq{\alpha_n}{n\in\Z}\in\spII$
%with Fourier Transform $\Falpha(\omega)\eqd\sum_{n\in\Z} \alpha_n \fkerna{n}{\omega}$.
\propboxt{
  $\ds\begin{array}{lclcl}
    \mcom{\sum_{n\in\Z} (-1)^n x_n = c}
         {(1) in ``time"}
    &\iff&
    \mcom{\left. \Zx(z)\right|_{z=-1} = c}
         {(2) in ``$z$ domain"}
    &\iff&
    \mcom{\left. \Dx(\omega)\right|_{\omega=\pi} = c}
         {(3) in ``frequency"}
    \\&\iff&
    \mc{3}{>{\ds}l}{%
      \mcom{\opair{\sum_{n\in\Z} h_{2n}}{\sum_{n\in\Z} h_{2n+1}} = \opair{\frac{1}{2}\brp{\sum_{n\in\Z} h_n+c}}{\frac{1}{2}\brp{\sum_{n\in\Z} h_n-c}}}
           {(4) sum of even, sum of odd}
      }
  \end{array}$
  \\
  $\indentx\scy\forall c\in\R,\,\seqxZ{x_n},\,\seqxZ{y_n}\in\spllR$
  }
\end{proposition}
\begin{proof}
\begin{enumerate}
  \item Proof that (1)$\implies$(2):
    \begin{align*}
      \left.\Zx(z)\right|_{z=-1}
        &= \left.\sum_{n\in\Z} x_n z^{-n} \right|_{z=-1}
      \\&= \sum_{n\in\Z} (-1)^n x_n
      \\&= c
        && \text{by (1)}
    \end{align*}

  \item Proof that (2)$\implies$(3):
    \begin{align*}
      \brlr{\sum_{n\in\Z} x_n e^{-i\omega n}}_{\omega=\pi}
        &= \sum_{n\in\Z} (-1)^n x_n
      \\&= \sum_{n\in\Z} (-1)^{-n} x_n
        &= \brlr{\sum_{n\in\Z} z^{-n} x_n}_{z=-1}
      \\&= c
        && \text{by (2)}
    \end{align*}   

  \item Proof that (3)$\implies$(1):
    \begin{align*}
      \sum_{n\in\Z} (-1)^n x_n 
        &= \sum_{n\in\Z} (-1)^{-n} x_n 
      \\&= \brlr{\sum_{n\in\Z} e^{-i\omega n} x_n}_{\omega=\pi}
      \\&= c
        && \text{by (3)}
    \end{align*}   

  %\item Proof that (1)$\impliedby$(2):
  %  \begin{align*}
  %    c
  %      &= \left.\Zx(z)\right|_{z=-1}
  %      && \text{by (2)}
  %    \\&= \left.\sum_{n\in\Z} x_n z^{-n}\right|_{z=-1}
  %    \\&= \sum_{n\in\Z} (-1)^n x_n
  %  \end{align*}

  \item Proof that (2)$\implies$(4):
    \begin{enumerate}
      \item Define $\ds A \eqd \sum_{n\in\Z}  h_{2n} \hspace{20mm} B \eqd \sum_{n\in\Z}  h_{2n+1}$.
      
      \item Proof that $A-B=c$:
        \begin{align*}
          c
            &= \sum_{n\in\Z} (-1)^n  x_n
            && \text{by (2)}
          \\&=   \mcom{\sum_{n\in\Ze} (-1)^n  x_n}{even terms}
               + \mcom{\sum_{n\in\Zo} (-1)^n  x_n}{odd terms}
          \\&=   \sum_{n\in\Z} (-1)^{2n}    x_{2n}
               + \sum_{n\in\Z} (-1)^{2n+1}  x_{2n+1}
          \\&=   \mcom{\sum_{n\in\Z}  x_{2n}}{$A$} - \mcom{\sum_{n\in\Z}  x_{2n+1}}{$B$}
          \\&\eqd A-B
            &&  \text{by definitions of $A$ and $B$}
        \end{align*}
      
      \item Proof that $\ds A+B=\sum_{n\in\Z}x_n$:
        \begin{align*}
          \sum_{n\in\Z}  x_n
            &= \sum_{n\mathrm{\;even}}  x_n  + \sum_{n\mathrm{\;odd}}  x_n
          \\&= \mcom{\sum_{n\in\Z}  x_{2n}}{$A$} + \mcom{\sum_{n\in\Z}  x_{2n+1}}{$B$}
          \\&= A+B
            && \text{by definitions of $A$ and $B$}
        \end{align*}
      
      \item This gives two simultaneous equations:
        \begin{align*}
          A - B &= c \\
          A + B &= \sum_{n\in\Z} x_n
        \end{align*}
      
      \item Solutions to these equations give
        \begin{align*}
         \sum_{n\in\Z}  x_{2n}    &\eqd A &= \frac{1}{2}\brp{\sum_{n\in\Z} x_n + c} \\
         \sum_{n\in\Z}  x_{2n+1}  &\eqd B &= \frac{1}{2}\brp{\sum_{n\in\Z} x_n - c}
        \end{align*}
    \end{enumerate}
  
  \item Proof that (2)$\impliedby$(4):
    \begin{align*}
      \sum_{n\in\Z} (-1)^n  x_n
        &= \mcom{\sum_{n\in\Ze} (-1)^n  x_n}{even terms}
          +\mcom{\sum_{n\in\Zo} (-1)^n  x_n}{odd terms}
      \\&= \sum_{n\in\Z} (-1)^{2n  }  x_{2n} + \sum_{n\in\Z} (-1)^{2n+1}  x_{2n+1}
      \\&= \sum_{n\in\Z}  x_{2n} -\sum_{n\in\Z}  x_{2n+1}
      \\&= \frac{1}{2}\brp{\sum_{n\in\Z}x_n+c} - \frac{1}{2}\brp{\sum_{n\in\Z}x_n-c}
        && \text{by (3)}
      \\&= c
    \end{align*}

\end{enumerate}
\end{proof}




%--------------------------------------
\begin{lemma}
\label{lem:even}
\index{function!even}
%--------------------------------------
Let $\Ff(\omega)$ be the \fncte{DTFT} \xref{def:dtft} of a sequence $\seqxZ{x_n}$. 
\lembox{
  \mcom{\seqxZ{x_n\in\R}}{\prope{real-valued} sequence} 
  \qquad\implies\qquad
  \mcom{\abs{\Dx(\omega)}^2 = \abs{\Dx(-\omega)}^2}{\prope{even}}
  \qquad\scy\forall\seqxZ{x_n}\in\spllR
  }
\end{lemma}
\begin{proof}
\begin{align*}
   |\Dx(\omega)|^2
     &=  \left. |\Zx(z)|^2  \right|_{z=e^{i\omega}}
   \\&=  \left. \Zx(z)\Zx^\ast(z)  \right|_{z=e^{i\omega}}
   \\&=  \left.
         \left[ \sum_{n\in\Z}  x_n z^{-n} \right]
         \left[ \sum_{m\in\Z}  x_m z^{-n} \right]^\ast
         \right|_{z=e^{i\omega}}
   \\&=  \left.
         \left[ \sum_{n\in\Z}  x_n z^{-n} \right]
         \left[ \sum_{m\in\Z}  x^\ast_m (z^\ast)^{-m} \right]
         \right|_{z=e^{i\omega}}
   \\&=  \left.
         \sum_{n\in\Z} \sum_{m\in\Z}  x_n x^\ast_m z^{-n} (z^\ast)^{-m}
         \right|_{z=e^{i\omega}}
   \\&=  \left.\sum_{n\in\Z} \left[
            | x_n|^2 +
            \sum_{m>n}  x_n x^\ast_m z^{-n} (z^\ast)^{-m} +
            \sum_{m<n}  x_n x^\ast_m z^{-n} (z^\ast)^{-m} 
            \right]
         \right|_{z=e^{i\omega}}
   \\&=  \sum_{n\in\Z}
         \left[
            | x_n|^2 +
            \sum_{m>n}  x_n  x_m e^{i\omega (m-n)}  +
            \sum_{m<n}  x_n  x_m e^{i\omega (m-n)}
         \right]
   \\&=  \sum_{n\in\Z}
         \left[
            | x_n|^2 +
            \sum_{m>n}  x_n  x_m e^{ i\omega(m-n)}  +
            \sum_{m>n}  x_n  x_m e^{-i\omega(m-n)}
         \right]
   \\&=  \sum_{n\in\Z}
         \left[
            | x_n|^2 +
            \sum_{m>n}  x_n  x_m
            \left( e^{i\omega (m-n)}  + e^{-i\omega(m-n)}  \right)
         \right]
   \\&=  \sum_{n\in\Z}
         \left[
            | x_n|^2 + \sum_{m>n}  x_n  x_m 2\cos[\omega(m-n)]
         \right]
   \\&=  \sum_{n\in\Z} | x_n|^2 +
         2\sum_{n\in\Z} \sum_{m>n}  x_n  x_m \cos[\omega(m-n)]
\end{align*}
Since $\cos$ is real and even, then $|\Dx(\omega)|^2$
must also be real and even.
\end{proof}




%--------------------------------------
\begin{theorem}[\thmd{inverse DTFT}]
\footnote{
  \citerpgc{chitode2009}{3-95}{818431678X}{(3.6.2)}
  }
\label{thm:idtft}
%--------------------------------------
Let $\Dx(\omega)$ be the \structe{discrete-time Fourier transform} \xref{def:dtft} of a sequence $\seqxZ{x_n}\in\spllR$.
Let $\iFx$ be the inverse of $\Dx$.
\thmbox{
  \mcom{
  \brb{\Dx(\omega) \eqd %
  \sum_{n\in\Z} x_n e^{-i\omega n}}
  }{$\Dx(\omega)\eqd\opDTFT\seqn{x_n}$}
  \quad\implies\quad
  \mcom{
  \brb{
    %\brs{\opiDTFT\Dx}(n)\eqd 
    x_n = \frac{1}{2\pi}\int_{\alpha-\pi}^{\alpha+\pi} \Dx(\omega) e^{i\omega n} \dw
    \quad\scy\forall\alpha\in\R
    }
  }{$\seqn{x_n}=\opiDTFT\opDTFT\seqn{x_n}$}
  %\quad\implies\quad 
  %\brb{\opiDTFT\opDTFT\seqn{x_n} = \seqn{x_n} \quad\scy\forall \seqn{x_n}\in\spllR}
  \quad\scy\forall\seqxZ{x_n}\in\spllR
  }
\end{theorem}
\begin{proof}
\begin{align*}
  \frac{1}{2\pi}\int_{\alpha-\pi}^{\alpha+\pi} \Dx(\omega) e^{i\omega n} \dw
    &= \frac{1}{2\pi}\int_{\alpha-\pi}^{\alpha+\pi} \mcom{\brs{\sum_{m\in\Z} x_m e^{-i\omega m}}}{$\Dx(\omega)$} e^{i\omega n} \dw
    && \text{by definition of $\Dx(\omega)$}
  \\&= \frac{1}{2\pi}\int_{\alpha-\pi}^{\alpha+\pi} \sum_{m\in\Z} x_m e^{-i\omega(m-n)}\dw
  \\&= \frac{1}{2\pi}\sum_{m\in\Z} x_m \int_{\alpha-\pi}^{\alpha+\pi} e^{-i\omega(m-n)}\dw
  \\&= \frac{1}{2\pi}\sum_{m\in\Z} x_m \brs{2\pi\kdelta_{m-n}}
  \\&= x_n 
\end{align*}
\end{proof}

%--------------------------------------
\begin{theorem}[\thmd{orthonormal quadrature conditions}]
\footnote{
  \citerppgc{dau}{132}{137}{0898712742}{(5.1.20),(5.1.39)}
  }
\label{thm:oquadcon}
%--------------------------------------
Let $\Dx(\omega)$ be the \structe{discrete-time Fourier transform} \xref{def:dtft} of a sequence $\seqxZ{x_n}\in\spllR$.
Let $\kdelta_n$ be the \fncte{kronecker delta function} at $n$ \xref{def:kdelta}.
\thmbox{
  \begin{array}{>{\ds}rcl c >{\ds}rcl@{\qquad}C}
    \sum_{m\in\Z} x_m y^\ast_{m-2n} &=& 0          &\iff& \Dx(\omega)\Dy^\ast(\omega)+\Dx(\omega+\pi)\Dy^\ast(\omega+\pi) &=& 0 & \forall n\in\Z,\,\forall \seqn{x_n},\seqn{y_n}\in\spllR\\
    \sum_{m\in\Z} x_m x^\ast_{m-2n} &=& \kdelta_n  &\iff& \abs{\Dx(\omega)}^2 +\abs{\Dx(\omega +\pi)}^2                   &=& 2 & \forall n\in\Z,\,\forall \seqn{x_n},\seqn{y_n}\in\spllR%
  \end{array}
  }
\end{theorem}
\begin{proof}
Let $z\eqd e^{i\omega}$.
\begin{enumerate}
\item Proof that
  $\ds
  2\sum_{n\in\Z} \left[ \sum_{k\in\Z}x_k y^\ast_{k-2n} \right] e^{-i2\omega n}
    = \Dx(\omega) \Dy^\ast(\omega) + \Dx(\omega+\pi) \Dy^\ast(\omega+\pi)
    \label{item:quadcon}
  $:
\begin{align*}
  &2\sum_{n\in\Z} \left[ \sum_{k\in\Z}x_k y^\ast_{k-2n} \right] e^{-i2\omega n}
  \\&= 2\sum_{k\in\Z}x_k \sum_{n\in\Z} y^\ast_{k-2n} z^{-2n} %e^{-i2\omega n}
  \\&= 2\sum_{k\in\Z}x_k \sum_{n\mathrm{\;even}} y^\ast_{k-n} z^{-n} %e^{-i\omega n}
  \\&= \sum_{k\in\Z}x_k \sum_{n\in\Z} y^\ast_{k-n} z^{-n} %e^{-i\omega n}
       \left( 1 + e^{i\pi n} \right)
  \\&= \sum_{k\in\Z}x_k \sum_{n\in\Z} y^\ast_{k-n} z^{-n} %e^{-i\omega n}
     + \sum_{k\in\Z}x_k \sum_{n\in\Z} y^\ast_{k-n} z^{-n}  e^{i\pi n}
  \\&= \sum_{k\in\Z}x_k \sum_{m\in\Z} y^\ast_m z^{-(k-m)} %e^{-i\omega(k-m)}
     + \sum_{k\in\Z}x_k \sum_{m\in\Z} y^\ast_m e^{-i\left(\omega+\pi\right)(k-m)} 
     \qquad\text{where $m\eqd k-n$}% \implies n=k-m
  \\&= \sum_{k\in\Z}x_k z^{-k} \sum_{m\in\Z} y^\ast_m z^m %e^{+i\omega m}
     + \sum_{k\in\Z}x_k e^{-i\left(\omega+\pi\right)k} \sum_{m\in\Z} y^\ast_m e^{+i\left(\omega+\pi\right)m} 
  \\&= \sum_{k\in\Z}x_k e^{-i\omega k} 
       \left[\sum_{m\in\Z} y_m e^{-i\omega m} \right]^\ast
     + \sum_{k\in\Z}x_k e^{-i\left(\omega+\pi\right)k} 
       \left[\sum_{m\in\Z} y_m e^{-i\left(\omega+\pi\right)m} \right]^\ast
  \\&\eqd \Dx(\omega) \Dy^\ast(\omega)
     + \Dx(\omega+\pi) \Dy^\ast(\omega+\pi)
\end{align*}

\item Proof that
  $\sum_{m\in\Z} x_m y^\ast_{m-2n} = 0
   \implies
   \Dx(\omega) \Dy^\ast(\omega) + \Dx(\omega+\pi) \Dy^\ast(\omega+\pi)=0
  $: 
\begin{align*}
  0 
    &= 2\sum_{n\in\Z} \left[ \sum_{k\in\Z}x_k y^\ast_{k-2n} \right] e^{-i2\omega n}
    && \text{by left hypothesis}
  \\&= \Dx(\omega) \Dy^\ast(\omega)
      +\Dx(\omega+\pi) \Dy^\ast(\omega+\pi)
    && \text{by \pref{item:quadcon}}
\end{align*}


\item Proof that 
  $\sum_{m\in\Z} x_m y^\ast_{m-2n} = 0
   \impliedby
   \Dx(\omega) \Dy^\ast(\omega) + \Dx(\omega+\pi) \Dy^\ast(\omega+\pi)=0
  $: 
\begin{align*}
  2\sum_{n\in\Z} \left[ \sum_{k\in\Z}x_k y^\ast_{k-2n} \right] e^{-i2\omega n}
    &= \Dx(\omega) \Dy^\ast(\omega) + \Dx(\omega+\pi) \Dy^\ast(\omega+\pi)
    && \text{by \pref{item:quadcon}}
  \\&= 0
    && \text{by right hypothesis}
\end{align*}
Thus by the above equation, 
$\sum_{n\in\Z} \left[ \sum_{k\in\Z}x_k y^\ast_{k-2n} \right] e^{-i2\omega n}=0$.
The only way for this to be true is if 
$\sum_{k\in\Z}x_k y^\ast_{k-2n}=0$.


\item Proof that
  $\sum_{m\in\Z} x_m x^\ast_{m-2n} = \kdelta_n
   \implies
   |\Dx(\omega)|^2 + |\Dx(\omega'+\pi)|^2 = 2
  $: \\
  Let $ g_{n} \eqd x_n$. 
\begin{align*}
  2 
    &= 2\sum_{n\in\Z} \kdelta_{n\in\Z} e^{-i2\omega n}
  \\&= 2\sum_{n\in\Z} \left[ \sum_{k\in\Z}x_k y^\ast_{k-2n} \right] e^{-i2\omega n}
    && \text{by left hypothesis}
  \\&= \Dx(\omega) \Dy^\ast(\omega)
     + \Dx(\omega+\pi) \Dy^\ast(\omega+\pi)
    && \text{by \pref{item:quadcon}}
\end{align*}


\item Proof that 
  $\sum_{m\in\Z} x_m x^\ast_{m-2n} = \kdelta_n
   \impliedby
   |\Dx(\omega)|^2 + |\Dx(\omega'+\pi)|^2 = 2
  $: \\
  Let $ g_{n} \eqd x_n$. 
\begin{align*}
  2\sum_{n\in\Z} \left[ \sum_{k\in\Z}x_k y^\ast_{k-2n} \right] e^{-i2\omega n}
    &= \Dx(\omega) \Dy^\ast(\omega) + \Dx(\omega+\pi) \Dy^\ast(\omega+\pi)
    && \text{by \pref{item:quadcon}}
  \\&= 2
    && \text{by right hypothesis}
\end{align*}
Thus by the above equation, 
$\sum_{n\in\Z} \left[ \sum_{k\in\Z}x_k y^\ast_{k-2n} \right] e^{-i2\omega n}=1$.
The only way for this to be true is if 
$\sum_{k\in\Z}x_k y^\ast_{k-2n}=\kdelta_n$.

\end{enumerate}
\end{proof}

%=======================================
%\section{Basis theory properties}
%=======================================


%=======================================
\section{Derivatives}
%=======================================
%--------------------------------------
\begin{theorem}
\footnote{
  \citerpp{vidakovic}{82}{83},
  \citerpp{mallat}{241}{242}
  }
\label{thm:dtft_ddw}
\index{vanishing moments}
%--------------------------------------
Let $\Dx(\omega)$ be the \fncte{DTFT} \xref{def:dtft} of a sequence $\seqxZ{x_n}$.
\thmbox{\begin{array}{F@{\qquad}>{\ds}rcl@{\qquad}c@{\qquad}>{\ds}rcl@{\qquad}F@{\qquad}C}
  (A) & \opddwn \Dx(\omega)\Big|_{\omega=0}   &=& 0  &\iff& \sum_{k\in\Z} k^n  x_k        &=& 0  & (B) & \forall n\in\Znn\\
  (C) & \opddwn \Dx(\omega)\Big|_{\omega=\pi} &=& 0  &\iff& \sum_{k\in\Z} (-1)^k k^n  x_k &=& 0  & (D) & \forall n\in\Znn
\end{array}}
\end{theorem}
\begin{proof}
\begin{enumerate}
  \item Proof that $(A)\implies(B)$: 
    \begin{align*}
      0
        &= \left.\opddwn \Dx(\omega)\right|_{\omega=0}
        && \text{by hypothesis (A)}
      \\&= \left.\opddwn \sum_{k\in\Z} x_ke^{-i\omega k} \right|_{\omega=0}
        && \text{by definition of $\Dx(\omega)$ \xref{def:dtft}}
      \\&= \left. \sum_{k\in\Z} x_k \opddwn e^{-i\omega k} \right|_{\omega=0}
      \\&= \left. \sum_{k\in\Z} x_k\left[(-i)^n k^n e^{-i\omega k}\right] \right|_{\omega=0}
      \\&= (-i)^n \sum_{k\in\Z} k^n x_k
    \end{align*}

  \item Proof that $(A)\impliedby(B)$: 
    \begin{align*}
      \left.\opddwn \Dx(\omega)\right|_{\omega=0}
        &= \left.\opddwn \sum_{k\in\Z} x_ke^{-i\omega k} \right|_{\omega=0}
        && \text{by definition of $\Fg$}
      \\&= \left. \sum_{k\in\Z} x_k\left[\opddwn e^{-i\omega k}\right] \right|_{\omega=0}
       %&& \text{by Leibnitz GPR \prefpo{lem:LGPR}}
      \\&= \left. \sum_{k\in\Z} x_k\left[(-i)^n k^n e^{-i\omega k}\right] \right|_{\omega=0}
      \\&= (-i)^n \sum_{k\in\Z} k^n x_k
      \\&= 0
        && \text{by hypothesis (B)}
    \end{align*}

  \item Proof that $(C)\implies(D)$: 
    \begin{align*}
        0
          &= \left.\opddwn \Dx(\omega)\right|_{\omega=\pi}
          && \text{by hypothesis (C)}
        \\&= \left.\opddwn \sum_{k\in\Z} x_k e^{-i\omega k} \right|_{\omega=\pi}
          && \text{by definition of $\Dx$ \xref{def:dtft}}
        \\&= \left. \sum_{k\in\Z} x_k \opddwn e^{-i\omega k} \right|_{\omega=\pi}
        \\&= \left. \sum_{k\in\Z} x_k \left[(-i)^n k^n e^{-i\omega k}\right] \right|_{\omega=\pi}
        \\&= \sum_{k\in\Z} x_k \left[(-i)^n k^n (-1)^k\right]
        \\&= (-i)^n \sum_{k\in\Z} (-1)^k k^n x_k 
    \end{align*}

  \item Proof that $(C)\impliedby(D)$: 
    \begin{align*}
      \left.\opddwn \Dx(\omega)\right|_{\omega=\pi}
        &= \left.\opddwn \sum_{k\in\Z} x_k e^{-i\omega k} \right|_{\omega=\pi}
        && \text{by definition of $\Dx$ \xref{def:dtft}}
      \\&= \left. \sum_{k\in\Z} x_k \opddwn e^{-i\omega k} \right|_{\omega=\pi}
      \\&= \left. \sum_{k\in\Z} x_k \left[(-i)^n k^n e^{-i\omega k}\right] \right|_{\omega=\pi}
      \\&= \sum_{k\in\Z} x_k \left[(-i)^n k^n (-1)^k\right]
      \\&= (-i)^n \sum_{k\in\Z} (-1)^k k^n x_k 
      \\&= 0
        && \text{by hypothesis (D)}
    \end{align*}
\end{enumerate}
\end{proof}





%=======================================
\section{Frequency Response}
%=======================================
The pole zero locations of a digital filter determine the magnitude and 
phase frequency response of the digital filter.\citepp{cadzow}{90}{91}
This can be seen by representing the poles and zeros vectors in the complex z-plane.
Each of these vectors has a magnitude $M$ and a direction $\theta$.
Also, each factor $(z-z_i)$ and $(z-p_i)$ can be represented as vectors as well
(the difference of two vectors).
Each of these factors can be represented by a magnitude/phase factor
$M_ie^{i\theta_i}$.  The overall magnitude and phase of $H(z)$ can then 
be analyzed.

Take the following filter for example:

\begin{eqnarray*}
   H(z) &=& \frac{b_0 + b_1z^{-1} + b_2z^{-2} }
                 {1   + a_1z^{-1} + a_2z^{-2} }
   \\   &=& \frac{(z-z_1)(z-z_2)}
                 {(z-p_1)(z-p_2)}
   \\   &=& \frac{M_1e^{i\theta_1} \; M_2e^{i\theta_2} \; }
                 {M_3e^{i\theta_3} \; M_4e^{i\theta_4} \; }
   \\   &=& \left(\frac{M_1M_2}{M_3M_4} \right)
            \left(\frac{e^{i\theta_1} e^{i\theta_2} }
                       {e^{i\theta_3} e^{i\theta_4} }\right)
\end{eqnarray*}

This is illustrated in \prefpp{fig:vector}.
The unit circle represents frequency in the Fourier domain.
The frequency response of a filter is just a rotating vector on this circle.
The magnitude response of the filter is just then a {\em vector sum}.
For example, the magnitude of any $H(z)$ is as follows:

\begin{eqnarray*}
   |H(z)| &=& \frac{|(z-z_1)|\;|(z-z_2)|}
                 {|(z-p_1)|\;|(z-p_2)|}
\end{eqnarray*}




\begin{figure}[ht]
  \centering%
  \includegraphics{graphics/vecres.pdf}
  %\begin{center}
  %\begin{fsL}
  %\setlength{\unitlength}{0.3mm}
  %\begin{picture}(300,300)(-150,-130)
  %  %\graphpaper[10](0,0)(200,200)                  
  %  \thicklines
  %  \put(   0,   0){\vector( 2, 1){ 90} }
  %  %\thinlines
  %  \put(-130,   0){\line(1,0){260} }
  %  \put(   0,-130){\line(0,1){260} }
  %  \put( 140,   0){\makebox(0,0)[l]{$\Re$}}
  %  \put(   0, 140){\makebox(0,0)[b]{$\Im$}}
  %
  %  \put(- 40,  60){\makebox(0,0){$\times$}}
  %  \put(- 40,- 60){\makebox(0,0){$\times$}}
  %  \put(  80, 120){\circle{10}}
  %  \put(  80,-120){\circle{10}}
  %
  %  \put(   0,   0){\vector( 2, 3){ 80} }
  %  \put(   0,   0){\vector( 2,-3){ 80} }
  %  \put(   0,   0){\vector(-2, 3){ 40} }
  %  \put(   0,   0){\vector(-2,-3){ 40} }
  %
  %  \qbezier[30]( 89, 45)( 84.5, 82.5)( 80, 120)
  %  \qbezier[60]( 89, 45)( 84.5,-37.5)( 80,-120)
  %  \qbezier[40]( 89, 45)( 24.5,52.5)( -40, 60)
  %  \qbezier[60]( 89, 45)( 24.5,-7.5)( -40,-60)
  %
  %  \put(  50,  25){\makebox(0,0)[rb]{$z$}}
  %  \put( -20,  30){\makebox(0,0)[rt]{$p_1$}}
  %  \put( -20, -30){\makebox(0,0)[br]{$p_2$}}
  %  \put(  64,  96){\makebox(0,0)[br]{$z_1$}}
  %  \put(  40, -60){\makebox(0,0)[tr]{$z_2$}}
  %  \put(  25, -10){\makebox(0,0)[lt]{$z-p_2$}}
  %  \put( -20,  60){\makebox(0,0)[lb]{$z-p_1$}}
  %  \put(  90,  80){\makebox(0,0)[l]{$z-z_1$}}
  %  \put(  90,-100){\makebox(0,0)[lb]{$z-z_2$}}
  %
  %  %============================================================================
% NCTU - Hsinchu, Taiwan
% LaTeX File
% Daniel Greenhoe
%
% Unit circle with radius 100
%============================================================================

\qbezier( 100,   0)( 100, 41.421356)(+70.710678,+70.710678) % 0   -->1pi/4
\qbezier(   0, 100)( 41.421356, 100)(+70.710678,+70.710678) % pi/4-->2pi/4
\qbezier(   0, 100)(-41.421356, 100)(-70.710678,+70.710678) %2pi/4-->3pi/4
\qbezier(-100,   0)(-100, 41.421356)(-70.710678,+70.710678) %3pi/4--> pi 
\qbezier(-100,   0)(-100,-41.421356)(-70.710678,-70.710678) % pi  -->5pi/4
\qbezier(   0,-100)(-41.421356,-100)(-70.710678,-70.710678) %5pi/4-->6pi/4
\qbezier(   0,-100)( 41.421356,-100)( 70.710678,-70.710678) %6pi/4-->7pi/4
\qbezier( 100,   0)( 100,-41.421356)( 70.710678,-70.710678) %7pi/4-->2pi



  %\end{picture}                                   
  %\end{fsL}
  \caption{
     Vector response of digital filter
     \label{fig:vector}
     }
%\end{center}
\end{figure}

