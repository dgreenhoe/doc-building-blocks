%============================================================================
% LaTeX File
% Daniel J. Greenhoe
%============================================================================

%======================================
\chapter{Probability Density Functions}
\label{chp:pdfs}
%======================================

\qboxnpq
  {Joseph Leonard Doob (1910--2004), pioneer of and key contributor to mathematical probability\footnotemark}
  {../common/people/doobjl_dartmouthedu.jpg}
  {While writing my book I had an argument with Feller.
   He asserted that everyone said ``random variable" and I asserted that everyone said ``chance variable."
   We obviously had to use the same name in our books, so we decided the issue by a stochastic procedure.
   That is, we tossed for it and he won.}
\citetblt{
  quote: & \citerp{snell1997}{307}, \citerp{snell2005}{251}\\
  %reference: & \citerpg{suhov2008}{238}{0521847672}\\
  image: & \url{http://www.dartmouth.edu/~chance/Doob/conversation.html}
  }
%======================================
\section{Random variables}
%======================================
The concept of the \fncte{random variable} is widely used in probability and
random processes.
Before discussing what a \fncte{random variable} \emph{is}, 
note two things that a \fncte{random variable} is \emph{not} (next remark).
%---------------------------------------
\begin{remark}
\footnote{
  \citerpg{miller2006}{130}{0471458929},
  \citerpgc{feldman2010}{4}{3642051588}{``The name ``random variable" is actually a misnomer, since it is not random and not a variable.\ldots the \fncte{random variable} simply maps each point (outcome) in the sample space to a number on the real line\ldots Technically, the space into which the \fncte{random variable} maps the sample space may be more general than the real line\ldots"},
  \citerpg{curry2011}{4}{3642166180},
  \citerpgc{trivedi2016}{2.1}{1119314208}{``The term ``random variable" is actually a misnomer, since a \fncte{random variable} $\rvX$ is really a function whose domain is the sample space $S$, and whose range is the set of all real numbers, $\R$."}
  }
%---------------------------------------
As pointed out by others, the term ``random variable" is a ``misnomer":
\\\rembox{
  \begin{array}{NM}
    \imark & A \fncte{random variable} is {\bf not random}.\\
    \imark & A \fncte{random variable} is {\bf not a variable}.
  \end{array}}
\end{remark}

What is it then? It is a \structe{function} (next definition).
In particular, it is a function that maps from an underlying stochastic process into $\R$.
Any ``\prope{randomness}" (whatever that means) it may \emph{appear} to have comes from the stochastic process it
is mapping \emph{from}. But the function itself (the \fncte{random variable} itself) is very deterministic and well-defined.
What gives it the appearance of being random is that the outcome $\omega$
of the experiment appears to be random to the observer.
So the \fncte{random variable} $\rvX(\omega)$ is simply a function of an underlying
mechanism that appears to be random.
%---------------------------------------
\begin{definition}
\index{random variable}
\footnote{
  \citerp{papoulis}{63}
  }
\label{def:randvar}
\label{def:rv}
%---------------------------------------
Let $\ps$ be a \structe{probability space} \xref{def:ps}.
\defboxt{
  A \fnctd{random variable} $\rvX$ is any function in $\clFor$.
  }
\end{definition}

%=======================================
\section{Probability distributions}
%=======================================
The probability information about $\sigma$-algebra $\psE$ in a
\structe{probability space} \xref{def:ps} is completely
specified by \fncte{measure} $\psP$.
However, sometimes it is more convenient to express this same \fncte{measure}
information in terms of the \fncte{probability density function} or the
\fncte{cummulative distribution function} of the \structe{probability space}.
%---------------------------------------
\begin{definition}
\footnote{
  \citerpgc{linden2014}{93}{1107035902}{Definitions 7.1, 7.2}
  }
\label{def:pdf}
\label{def:cdf}
%---------------------------------------
Let $\rvX$ be a \fncte{random variable} \xref{def:randvar}
on a \structe{probability space} $\ps$.
%Let $\pcx(x)$ be a \fncte{function} in $\psPcl$ and $\ppx(x)$ a function in $\clF{\psE}{\R}$.
\defbox{\begin{array}{MMM}
  $\rvX$ has \fnctd{cummulative distribution function} &(\fnctd{cdf}) & $\pcx(x)$ if 
    \\\mc{3}{l}{\indentx\ds\pcx(x) \eqd \psP\set{x\in\psE}{\rvX<x}}
    \\
  $\rvX$ \fnctd{probability density function}      &(\fnctd{pdf})     & $\ppx(x)$ if 
    \\\mc{3}{l}{\indentx\ds\ppx(x) \eqd \ddx\pcx(x) \eqd \ddx\psP\set{x\in\psE}{\rvX < x }}
\end{array}}
\end{definition}

%---------------------------------------
\begin{remark}
%---------------------------------------
Suppose $\rvX$ be a \structe{random variable} on a \structe{probability space} $\ps$.
Note that 
\\\indentx\begin{tabular}{cl}
    \imark & Both $\rvX$ and $\psE$ are \fncte{function}s.
  \\\imark & But $\rvX$ is a {function} that maps from $\psO$ to $\R$,
  \\\imark & whereas $\psP$ is a {function} that maps from $\psE$ to $\R$.
\end{tabular}
\end{remark}

%---------------------------------------
\begin{definition}
\label{def:jointpdf}
\label{def:jointcdf}
%---------------------------------------
Let $\ps$ be a \structe{probability space} \xref{def:ps} and
$\rvX$ and $\rvY:\pso\to\R$ random variables. Then
a \fnctd{joint probability density function}    $\ppx:\pse\times\pso\to\intcc{0}{1}$ and
a \fnctd{joint cumulative distribution function} $\pcx:\pse\times\pso\to\intcc{0}{1}$
are defined as
\defbox{\begin{array}{rc>{\ds}lD}
  \pcxy(x,y) &\eqd& \psp\set{\rvX\le x}{\rvY\le y}
          & (\fncte{joint cumulative distribution function})
          \\
  \ppxy(x,y) &\eqd& \ddy\ddx\pcxy(x,y)
          & (\fncte{joint probability density function})
\end{array}}
\end{definition}

%---------------------------------------
\begin{definition}
\label{def:conpdf}
\label{def:concdf}
%---------------------------------------
Let $\ps$ be a \structe{probability space} \xref{def:ps} and
$\rvX$ a random variable. Then
a \fnctd{conditional probability density function}    $\ppx:\pse\times\pso\to\intcc{0}{1}$ and
a \fnctd{conditional cumulative distribution function} $\pcx:\pse\times\pso\to\intcc{0}{1}$
are defined as
\defbox{\begin{array}{rc>{\ds}lD}
  \pcx(x|y) &\eqd& \psp\brb{\rvX \le x|\rvY=y}
            & (\fncte{conditional cumulative distribution function}---\fncte{cdf})
            \\
  \ppx(x|y) &\eqd& \ddx\pcx(x|y)
  %\ppx(x|y) &\eqd& \lim_{\varepsilon\to 0} \frac{1}{\varepsilon}\psp\brb{x \le\rvX < x+\varepsilon | \rvY=y}
            & (\fncte{conditional probability density function}---\fncte{pdf})
\end{array}}
\end{definition}

%=======================================
\section{Properties}
%=======================================
\prefpp{def:pdf} defines the pdf and cdf of a \structe{probability space}
$\ps$ in terms of \fncte{measure} $\psp$.
Conversely, the probability \fncte{measure} $\psp\setn{a\le\rvX<b}$
of an event $\{a\le\rvX<b\}$ can be
expressed in terms of either the pdf or cdf.

%---------------------------------------
\begin{proposition}
\label{prop:pdfddx}
%---------------------------------------
Let $\rvX$ a \fncte{random variable} with \fncte{pdf} $\ppx$ and \fncte{cdf} $\pcx$
\xref{def:pdf} on the \structe{probability space} $\ps$ \xref{def:ps}.\\
\propboxt{
  $\brb{\begin{array}{FMD}
    (1).& $\pcx(x)$ and $\pcy(y)$ are \prope{continuous} & OR \\
    (2).& $\ppx(x)$ and $\ppy(y)$ are \prope{continuous}
  \end{array}}$
  \\\quad$\implies\quad
  \brb{\begin{array}{rc>{\ds}l}
      \ppx(x)    &=& \lim_{\varepsilon\to 0} \frac{1}{\varepsilon}\psp\setn{x \le \rvX < x+\varepsilon}
    \\\ppxy(x,y) &=& \lim_{\varepsilon\to 0} \frac{1}{\varepsilon}\psp\setn{x \le \rvX < x+\varepsilon \land y \le\rvY < y+\varepsilon}
  \end{array}}$
  }
\end{proposition}
\begin{proof}
\begin{align*}
  \ppx(x)
    &\eqd \ddx\pcx(x)
    && \text{by definition of $\ppx$}
    && \text{\xref{def:pdf}}
  \\&=    \lim_{\varepsilon\to 0} \frac{1}{\varepsilon}\psp\set{x\in\R}{x \le \rvX < x+\varepsilon}
    && \text{by definition of $\ddx$}
    && \text{\xref{def:ddx}}
\end{align*}
\end{proof}

%---------------------------------------
\begin{theorem}
\label{thm:pdfcdf}
%---------------------------------------
Let $\ps$ be a probability space,
$\rvX$ be a random variable, and $(a,b)$ a real interval.
\thmbox{
  \brb{\begin{array}{FMD}
    (1).& $\pcx(x)$ is \prope{continuous} & OR \\
    (2).& $\ppx(x)$ is \prope{continuous}
  \end{array}}
  \implies
  \brb{\begin{array}{rc>{\ds}lc>{\ds}l}
    \psp\setn{a<\rvX\le b}
      &=& \pcx(b) - \pcx(a)
      &=& \int_a^b \ppx(x) \dx
  \end{array}}
  }
\end{theorem}
\begin{proof}
\begin{align*}
  \psp\setn{a<\rvX\le b}
    &= \psp\setn{\rvX\le b} - \psp\setn{\rvX <  a}
    && \text{by \thme{sum of products}} && \text{\xref{thm:psp_sop}}
  \\&= \psp\setn{\rvX\le b} - \psp\setn{\rvX\le a}
    && \text{by \prope{continuity} hypothesis}
  \\&\eqd \pcx(b) - \pcx(a)
    && \text{by definition of $\pcx$} && \text{\xref{def:cdf}}
  \\
  \\
  \int_a^b \ppx(x) \dx
    &\eqd \int_a^b \brs{\ddx\pcx(x)} \dx
    && \text{by definition of $\ppx$} && \text{\xref{def:pdf}}
  \\&= \brlr{\pcx(x)}_{x=b} - \brlr{\pcx(x)}_{x=a}
    && \text{by \thme{Fundamental theorem of calculus}}
  \\&= \pcx(b) - \pcx(a)
\end{align*}
\end{proof}

%---------------------------------------
\begin{theorem}
%---------------------------------------
Let $\ps$ be a \structe{probability space},
$\rvX$ be a \fncte{random variable}, and $\intoo{a}{b}$ a \structe{real interval}.
\thmbox{\begin{array}{>{\ds}l}
  \psp\setn{a\le\rvX<b}
     =  \int_a^b \ppx(x) \dx
     =  \int_{-\infty}^b \pcx(x) \dx - \int_{-\infty}^a \pcx(x) \dx
\end{array}}
\end{theorem}

The properties of the pdf follow closely the properties of \fncte{measure} $\psp$.
%---------------------------------------
\begin{theorem}
\footnote{
  \citerpgc{papoulis1990}{158}{0137116985}{Auxiliary Variable},
  \citerpc{jazwinski1970}{39}{``(2.102)"},
  \citerpgc{jazwinski2007}{39}{9780486318196}{``(2.102)"}
  }
\label{thm:pdf_cond}
%---------------------------------------
\thmbox{
  \brb{\begin{array}{FMD}
    (A).& $\pcx(x)$ is \prope{continuous} & OR \\
    (B).& $\ppx(x)$ is \prope{continuous}
  \end{array}}
  \quad\implies\quad
  \brb{\begin{array}{Frc>{\ds}lD}
    (1).& \ppxcy(x|y) &=& \frac{\ppxy(x,y)}{\ppy(y)} & and \\
    (2).& \ppx(x)     &=& \int_{y\in\R}\ppxy(x,y) \dy
  \end{array}}
  }
\end{theorem}
\begin{proof}
\begin{align*}
  \ppxcy(x|y)
    &\eqd \ddx\pcxcy(x|y)
    && \text{by definition of $\pcx$}
    && \text{\xref{def:conprob}}
  \\&\eqd \lim_{\varepsilon\to0} \frac{1}{\varepsilon} \psp\set{x\le\rvX < x+\varepsilon}{\rvY=y}
    && \text{by definition of $\ddx$}
    && \text{\xref{def:ddx}}
  \\&\eqd \lim_{\varepsilon\to0}
          \frac{1}{\varepsilon}
          \frac{\ds\psp\setn{(x\le\rvX < x+\varepsilon) \land (\rvY=y)}}
               {\ds\psp\setn{\rvY=y}}
    && \text{by definition of $\psP\set{\setA}{\setB}$}
    && \text{\xref{def:conprob}}
  \\&= \lim_{\varepsilon\to0}
       \frac{1}{\varepsilon}
       \frac{\ds\psp\setn{(x\le\rvX < x+\varepsilon) \land (y\le\rvY<y+\varepsilon)}}
            {\ds\psp\setn{y\le\rvY<y+\varepsilon}}
    && \text{by \prope{continuity} hypothesis}
  \\&=  \frac{\ds\lim_{\varepsilon\to0} \frac{1}{\varepsilon}\psp\setn{(x\le\rvX < x+\varepsilon) \land (y\le\rvY<y+\varepsilon)}}
             {\ds\lim_{\varepsilon\to0}                      \psp\setn{y\le\rvY<y+\varepsilon}}
    && \text{by property of $\ds\lim_{\varepsilon\to0}$}
  \\&=    \frac{\ppxy(x,y)}{\ppy(y)}
    && \text{by \prefp{prop:pdfddx}}
\\
\\
  \int_{y\in\R}\ppxy(x,y)\dy
    &\eqd \int_{y\in\R} \brs{ \ddy\ddx\pcxy(x,y) } \dy
    && \text{by definition of $\ppx$}
    && \text{\xref{def:pdf}}
  \\&= \ddx\pcxy(x,y)
  \\&\eqd \lim_{\varepsilon\to0} \frac{1}{\varepsilon}
          \int_{y\in\R} \psp\setn{x\le\rvX<x+\varepsilon, y\le\rvY<y+\varepsilon} \dy
    && \text{by definition of $\ddx$}
    && \text{\xref{def:ddx}}
  \\&=    \lim_{\varepsilon\to0} \frac{1}{\varepsilon} \psp\setn{x\le\rvX<x+\varepsilon}
  \\&= \ppx(x)
    && \text{by \prefp{prop:pdfddx}}
\end{align*}
\end{proof}

%---------------------------------------
\begin{theorem}
\label{thm:cdf}
%---------------------------------------
\thmbox{\begin{array}{rcl}
  \pcx(\sup\R) &=& 1 \\
  \pcx(\inf\R) &=& 0
\end{array}}
\end{theorem}
\begin{proof}
\begin{align*}
  \pcx(\sup\R)
    &\eqd \psp\setn{\rvX\le\sup\R}
    && \text{by definition of $\pcx$}
    &&\text{\xref{def:cdf}}
  \\&=    1
    %&& \text{by definition of $\psp$}
    %&&\text{\xref{def:psp}}
  \\
  \pcx(\inf\R)
    &\eqd \psp\setn{\rvX\le\inf\R}
    && \text{by definition of $\pcx$} &&\text{\xref{def:cdf}}
  \\&=    0
    %&& \text{by definition of $\psp$}
    %&&\text{\xref{def:psp}}
\end{align*}
\end{proof}

The properties of the pdf follow closely the properties of measure $\psp$.
%---------------------------------------
\begin{theorem}
\label{thm:conpdf}
%---------------------------------------
\thmbox{\begin{array}{rc>{\ds}l rc>{\ds}l}
  \pcxcy(x|y) &=& \frac{\ddy\pcxy(x,y)}{\ppy(y)}  &\qquad
  \ppxcy(x|y) &=& \frac{\ppxy(x,y)}{\ppy(y)}  \\
\end{array}}
\end{theorem}
\begin{proof}
\begin{align*}
  \pcxcy(x|y)
    &\eqd \pPc{\rvX\le x}{\rvY=y}
    &&    \text{by definition of $\pcxcy$}              &&\text{\xref{def:concdf}}
  \\&\eqd \frac{\psp\set{\rvX\le x}{\rvY=y}}{\psp\setn{\rvY=y}}
    &&    \text{by definition of $\psp\setn{\rvX|\rvY}$}         &&\text{\xref{def:conP}}
  \\&=    \lim_{\varepsilon\to0}\frac{\psp\set{\rvX\le x}{y<\rvY\le y+\varepsilon}}
                                         {\psp\setn{y<\rvY\le y+\varepsilon}}
  \\&=\mathrlap{\lim_{\varepsilon\to0}\frac{\brs{\psp\set{\rvX\le x}{\rvY\le y+\varepsilon}-\psp\set{\rvX\le x}{\rvY\le y}}/\varepsilon}
                                         {\brs{\psp\setn{\rvY\le y+\varepsilon}-\psp\setn{\rvY\le y}}/\varepsilon}}
  \\&\eqd \lim_{\varepsilon\to0}\frac{\brs{\pcxy(x,y+\varepsilon)-\pcxy(x,y)}/\varepsilon}
                                         {\brs{\pcy(y+\varepsilon)-\pcy(y)}/\varepsilon}
    &&    \text{by definition of $\pcxy$}                  &&\text{\xref{def:jointcdf}}
  \\&\eqd \frac{\ddy\pcxy(x,y)}{\ddy\pcy(y)}
    &&    \text{by definition of $\ddy\ff(y)$}
  \\&\eqd \frac{\ddy\pcxy(x,y)}{\ppy(y)}
    &&    \text{by definition of $\ppy$}  &&\text{\xref{def:pdf}}
  \\&=    \frac{\ddy\pcxy(x,y)}{\ppy(y)}
    &&    \text{because $y$ is fixed}
  \\
  \\
  \ppxcy(x|y)
    &\eqd \ddx\pcxcy(x|y)                     &&\text{by definition of $\ppxcy$}                  &&\text{\xref{def:conpdf}}
  \\&=    \ddx\frac{\ddy\pcxy(x,y)}{\ppy(y)}  &&\text{by previous result}
  \\&=    \frac{\ddx\ddy\pcxy(x,y)}{\ppy(y)}  &&\text{because $\ppy(y)$ is not a function of $x$}
  \\&\eqd \frac{\ppxy(x,y)}{\ppy(y)}          &&\text{by definition of $\ppxy(x,y)$}              &&\text{\xref{def:jointpdf}}
\end{align*}

\begin{align*}
  %\ppxcy(x|y)
  %  &\eqd \lim_{h\to0} \frac{1}{h} \psp\setn{x\le\rvX < x+h | Y=y}
  %  &&    \text{by definition of $\ppxcy(x|y)$}
  %  &&    \text{\xref{def:conpdf}}
  %\\&\eqd \lim_{\varepsilon\to0} \frac{1}{\varepsilon} \frac{\psp\setn{(x\le\rvX < x+\varepsilon) \land (Y=y)}}{\psp\setn{Y=y}}
  %  &&    \text{by definition of $\psp\setn{\rvX|\rvY}$}
  %  &&    \text{\xref{def:conP}}
  %\\&=    \lim_{\varepsilon\to0} \frac{1}{\varepsilon} \frac{\psp\setn{(x\le\rvX < x+\varepsilon) \land (y\le Y<y+\varepsilon)}}{\psp\setn{y\le Y<y+\varepsilon}}
  %\\&\eqd \frac{\ppxy(x,y)}{\ppy(y)}
  %  \qquad\mathrlap{\text{by definitions of $\ppxy$ \xref{def:jointpdf} and $\ppy$ \xref{def:pdf}}}
  %\\
  %\int_y\ppxy(x,y)\dy
  %  &\eqd \lim_{\varepsilon\to0} \frac{1}{\varepsilon} \int_y \psp\setn{x\le X<x+\varepsilon \land y\le Y<y+\varepsilon} \dy
  %  &&    \text{by definition of $\ppxy$}
  %  &&    \text{\xref{def:jointpdf}}
  %\\&=    \lim_{\varepsilon\to0} \frac{1}{\varepsilon} \psp\setn{x\le X<x+\varepsilon}
  %  &&    \text{by \thme{sum of products}}
  %  &&    \text{\xref{thm:psp_sop}}
  %\\&\eqd \ppx(x)
  %  &&    \text{by definition of $\ppx$}
  %  &&    \text{\xref{def:psp}}
\end{align*}
\end{proof}

%---------------------------------------
\begin{theorem}
\label{thm:pdf1}
%---------------------------------------
Let $\ps$ be a probability space.
\thmbox{\begin{array}{>{\ds}rc>{\ds}l C| >{\ds}rc>{\ds}l}
    \int_{x\in\R} \ppx(x)    \dx &=& 1       &                  & \int_{x\in\R} \ppxcy(x|y) \dx               &=& 1\\
    \int_{y\in\R} \ppxy(x,y) \dy &=& \ppx(x) & \forall x\in\pso & \int_{x\in\R}\int_{y\in\R}\ppxy(x,y) \dy\dx &=& 1
  \end{array}}
\end{theorem}
\begin{proof}
\begin{align*}
  \int_{\R} \ppx(x) \dx
    &= \pcx(\sup\R) - \pcx(\inf\R)
    && \text{by \prefp{thm:pdfcdf}}
  \\&= 1 - 0
  \\&= 1
    && \text{because $0$ is the additive identity element in $\fieldR$}
  \\
  \int_{x\in\R} \ppxcy(x|y) \dx
    &\eqd \int_{x\in\R} \ddx\pcxcy(x|y) \dx
    &&    \text{by definition of $\ppxcy(x|y)$ \xref{def:conpdf}}
  \\&=    \pcxcy(\sup\R|y) - \pcxcy(\inf\R|y)
    &&    \text{by \thme{Fundamental theorem of calculus}}
  \\&=    1-0
  \\&=    1
    && \text{because $0$ is the additive identity element in $\fieldR$}
  \\
  \int_{y\in\R} \ppxy(x,y) \dy
    &= \int_{y\in\R} \ppyx(y,x) \dy
  \\&= \int_{y\in\R} \ppycx(y|x)\ppx(x) \dy  && \text{by \prefp{thm:conpdf}}
  \\&= \ppx(x)\int_{y\in\R} \ppycx(y|x) \dy  && \text{because $\ppx(x)$ is not a function of $y$}
  \\&= \ppx(x)\cdot1                         && \text{by previous result}
  \\&= \ppx(x)                               && \text{because $1$ is the multiplicative identity element in $\fieldR$}
\end{align*}
\begin{align*}
  \int_{x\in\R}\int_{y\in\R} \ppxy(x,y) \dy\dx
    &= \int_{x\in\R}\ppx(x)\dx               && \text{by previous result}
  \\&= 1                                     && \text{by previous result}
\end{align*}
\end{proof}

