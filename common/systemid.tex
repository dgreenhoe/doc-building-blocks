%============================================================================
% Daniel J. Greenhoe
% LaTeX file
%============================================================================
%============================================================================
\chapter{System Identification}
%============================================================================

%=======================================
\section{Definitions}
%=======================================
%---------------------------------------
\begin{definition}[\fnctd{Least Squares Technique}]
\footnote{
  \citerpgc{shin2008}{293}{0470725648}{
    $H_1(f)=\ffrac{\Swxy(f)}{\Swxx(f)}$ (9.63);
    which differs from \pref{def:H1}, but see \prefp{rem:Rxym}
    },
  \citerpgc{bendat2011}{185}{1118210824}{$H_1(f) = \ffrac{G_{xy}(f)}{G_{xx}(f)}$ (6.37)},
  \citeP{bendat1978}{cited by Cobb(1988)---variance estimate for $\estH_1$},
  \citePc{allemang1979}{cited by Shin(2008)},
  \citePpc{leuridan1986}{910}{\fncte{Least Squares Technique}; (8) $[G_{xx}](H)=[G_{xy}]$},
  \citeP{abom1986}{cited by Cobb(1988)---variance estimate for $\estH_1$},
  \citePppc{allemang1987}{54}{55}{5.3.1 $H_1$ Technique; $[H]=[G_{XF}][G_{FF}]^{-1}$ (11)},
  \citePpc{cobb1988}{2}{$^1\hat{H}(f)=\ffrac{\hat{G}_{yx}(f)}{\hat{G}_{xx}(f)}$ (1)}
  }
\label{def:H1}
%---------------------------------------
\defbox{\begin{array}{Mrc>{\ds}l}
  The transfer function $\estH_1(\omega)$ is defined as & \estH_1(\omega) &\eqd& \frac{\Swyx(\omega)}{\Swxx(\omega)}
\end{array}}
\end{definition}

%---------------------------------------
\begin{definition}
\footnote{
  \citerpgc{shin2008}{293}{0470725648}{$H_2(f)=\ffrac{\Swyy(f)}{\Swyx(f)}$ (9.65);
    which differs from \pref{def:H2}, but see \prefp{rem:Rxym}
    },
  \citerpgc{bendat2011}{186}{1118210824}{$H_2(f) = \ffrac{G_{yy}(f)}{G_{yx}(f)}$ (6.42)},
  \citePc{mitchell1980}{cited by Cobb(1988)},
  \citePc{mitchell1982}{cited by Cobb(1988)},
  \citePpc{cobb1988}{3}{$^2\hat{H}(f)=\ffrac{\hat{G}_{yy}(f)}{\hat{G}_{xy}(f)}$ (1)}
  }
\label{def:H2}
%---------------------------------------
\defbox{\begin{array}{Mrc>{\ds}l}
  The transfer function $\estH_2(\omega)$ is defined as & \estH_2(\omega) &\eqd& \frac{\Swyy(\omega)}{\Swxy(\omega)}
\end{array}}
\end{definition}

%---------------------------------------
\begin{definition}
\footnote{
  \citerpgc{shin2008}{293}{0470725648}{(9.67)},
  \citeP{white2006}
  }
\label{def:Hs}
%---------------------------------------
\defbox{\begin{array}{Mrc>{\ds}l}
  The transfer function $\estH_s(\omega)$ is defined as & 
    \estH_s(\omega) &\eqd& \frac{\Swyy(\omega)-\kappa(\omega)\Swxx(\omega) + \sqrt{\brs{\Swxx(\omega)\kappa(\omega)-\Swyy(\omega)}^2 + 4\abs{\Swxy(\omega)}\kappa(\omega)}}
                                {2\Swxy(\omega)}
\end{array}}
\end{definition}

%=======================================
\section{Properties}
%=======================================
\begin{figure}
  \centering%
  \includegraphics{graphics/xhyplusv.pdf}
  \caption{System model with additive input and output noise\label{fig:xhyplusv}}
\end{figure}
%---------------------------------------
\begin{proposition}[Output noise only case]
\footnote{
  \citerpgc{shin2008}{294}{0470725648}{$H_1(f)=H(f)$ (9.70); $H_2(f)=H(f)\brp{1+\frac{S_{n_yn_y}(f)}{Syy(f)}}$ (9.71)}
  }
%---------------------------------------
Let $\opS$ be a \structe{system} with \fncte{impulse response} $\fh[n]$,
input $\fx[n]$, and output $\fy[n]$.
Let $\rvu[n]$ and $\rvv[n]$ be \fncte{noise sequence}.
Let $\rvx'[n]\eqd\rvx[n]+\rvu[n]$ and $\rvy'[n]\eqd\rvy[n]+\rvv[n]$.
\propbox{
  \brb{\begin{array}{FMMD}
      (A).& The system $\fh$ i s        &\prope{LTI}          & and
    \\(B).& $\rvx[n]$ and $\rvy[n]$ are &\prope{WSS}          & and
    \\(C).& $\rvx[n]$ and $\rvv[n]$ are &                     &     
        \\& \mc{2}{M}{\quad              \prope{uncorrelated}}& and
    \\(D).& $\rvu[n]=0$                 
  \end{array}}
  \implies
  \brb{\begin{array}{F>{\ds}rc>{\ds}lDD}
      (1).&\estH_1(\omega) &=& \fH(\omega)                                              & (\prope{unbiased}) &and
    \\(2).&\estH_2(\omega) &=& \fH(\omega)\brs{1 + \frac{\Swvv(\omega)}{\Swyy(\omega)}} & \begin{tabular}{l}(\prope{biased})\\(\prope{over estimated})\end{tabular}
  \end{array}}
  }
\end{proposition}
\begin{proof}
\begin{align*}
  \estH_1(\omega)
    &\eqd \frac{\Swxx[y'x](\omega)}{\Swxx(\omega)}
    && \text{by definition of $\estH_1$}
    && \text{\xref{def:H1}}
  \\&=    \frac{\Swyx(\omega)}{\Swxx(\omega)}
    && \text{by \prefp{prop:Swxyv}}
  \\&= \fH^\ast(\omega)
    && \text{by \prope{LTI} hypothesis (A)}
    && \text{and \prefp{cor:Swxy}}
  \\
  \estH_2(\omega)
    &\eqd \frac{\Swxx[y'y'](\omega)}{\Swxy(\omega)}
    && \text{by definition of $\estH_2$}
    && \text{\xref{def:H2}}
  \\&=    \frac{\Swyy(\omega) + \Swvv(\omega)}{\Swxy(\omega)}
    && \text{by \prefp{prop:Swxyv}}
  \\&= \frac{\Swyy(\omega)}{\Swxy(\omega)}
     + \frac{\Swvv(\omega)}{\Swxy(\omega)}
  \\&= \fH(\omega)
     + \frac{\Swvv(\omega)}{\frac{\Swyy(\omega)}{\fH(\omega)}}
    && \text{by \prefp{cor:Swxy}}
  \\&= \fH(\omega)\brs{1 + \frac{\Swvv(\omega)}{\Swyy(\omega)}}
\end{align*}
\end{proof}

%---------------------------------------
\begin{proposition}[Input noise only case]
\footnote{
  \citerpgc{shin2008}{294}{0470725648}{$H_1(f)=\frac{H(f)}{1+S_{n_xn_x}/S_{xx}(f)}$ (9.72); $H_2(f)=H(f)$ (9.73)}
  }
%---------------------------------------
Let $\opS$ be a \structe{system} with \fncte{impulse response} $\fh[n]$,
input $\fx[n]$, and output $\fy[n]$.
Let $\rvu[n]$ and $\rvv[n]$ be \fncte{noise sequence}.
Let $\rvx'[n]\eqd\rvx[n]+\rvu[n]$ and $\rvy'[n]\eqd\rvy[n]+\rvv[n]$.
\propbox{
  \brb{\begin{array}{FMMD}
      (A).& The system $\fh$ i s        &\prope{LTI}          & and
    \\(B).& $\rvx[n]$ and $\rvy[n]$ are &\prope{WSS}          & and
    \\(C).& $\rvy[n]$ and $\rvu[n]$ are &                     &     
        \\& \mc{2}{M}{\quad              \prope{uncorrelated}}& and
    \\(D).& $\rvv[n]=0$                 
  \end{array}}
  \implies
  \brb{\begin{array}{F>{\ds}rc>{\ds}lDD}
      (1).&\estH_1(\omega) &=& \frac{\fH(\omega)}{1 + \frac{\Swuu(\omega)}{\Swxx(\omega)}} & \begin{tabular}{l}(\prope{biased})\\(\prope{under estimated})\end{tabular} &and
    \\(2).&\estH_2(\omega) &=& \fH(\omega)                                                 & (\prope{unbiased}) &
  \end{array}}
  }
\end{proposition}
\begin{proof}
\begin{align*}
  \\
  \estH_1(\omega)
    &\eqd \frac{\Swxx[yx'](\omega)}{\Swxx[x'x'](\omega)}
    && \text{by definition of $\estH_1$}
    && \text{\xref{def:H1}}
  \\&= \frac{\Swyx(\omega) + \Swyu(\omega)}
            {\Swxx(\omega) + \Swuu(\omega)}
    && \text{by \prefp{prop:Swxyv}}
  \\&= \frac{\Swyx(\omega) + \cancelto{0}{\Swyu(\omega)}}
            {\Swxx(\omega) + \Swuu(\omega)}
    && \text{by \prope{uncorrelated} hypothesis}
    && \text{(C)}
  \\&= \frac{\frac{\Swyx(\omega)}{\Swxx(\omega)}}
            {\frac{\Swxx(\omega)}{\Swxx(\omega)} + \frac{\Swuu(\omega)}{\Swxx(\omega)}}
  \\&= \frac{\fH(\omega)}
            {1 + \frac{\Swuu(\omega)}{\Swxx(\omega)}}
  \\
  \estH_2(\omega)
    &\eqd \frac{\Swyy(\omega)}{\Swxx[x'y](\omega)}
    && \text{by definition of $\estH_2$}
    && \text{\xref{def:H2}}
  \\&= \frac{\Swyy(\omega)}{\Swxy(\omega) + \Swuy(\omega)}
    && \text{by \prefp{prop:Swxyv}}
  \\&= \frac{\Swyy(\omega)}{\Swxy(\omega) + \cancelto{0}{\Swuy(\omega)}}
    && \text{by \prope{uncorrelated} hypothesis}
    && \text{(C)}
  \\&= \fH(\omega)
    && \text{by \prope{LTI} hypothesis (A)}
    && \text{and \prefp{cor:Swxy}}
\end{align*}
\end{proof}

%=======================================
\section{Beware of estimators}
%=======================================
Estimators yield, as the name implies, estimates.
These estimates in general contain some error.

%---------------------------------------
\begin{example}[The K=1 Welch estimate of coherence]
%---------------------------------------
Suppose we have two \prope{uncorrelated} stationary sequences $\rvx[n]$ and $\rvy[n]$. Then, there
CSD $\Sxy(\omega)$ should be $0$ because
\begin{align*}
  \Sxy(\omega)
    &\eqd \opDTFT\pE\Rxy(m)
  \\&=    \opDTFT\pE\brs{x[n]y[n+m]}
  \\&=    \opDTFT\brs{\pEx[n]} \brs{\pEy[n+m]}
  \\&=    \opDTFT\brs{0} \brs{0}
  \\&=    0
\end{align*}

This will give a coherence of $0$ also:
\[ C(\omega) = \frac{\Sxy}{\sqrt{\Sxx\Syy}} = 0\]

However, the Welch estimate with $K=1$ will yield
\begin{align*}
  \abs{C(\omega)}
    &= \abs{\frac{\ds\Sxy}{\sqrt{\ds\Sxx\Syy}}}
  \\&= \abs{\frac{\ds (\opFT x)(\opFT y)^\ast}{\sqrt{\ds\abs{\opFT x}^2\abs{\opFT y}^2}}}
  \\&= 1
\end{align*}

\end{example}
