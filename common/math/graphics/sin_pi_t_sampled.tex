%============================================================================
% Daniel J. Greenhoe
% LaTeX file
%
% transform of a sin(pi t) for the Haar k=0 subspace 
%
%     2
% --------- = 0.63661977236758134307553505349006
%    pi
% nominal unit = 8mm
%============================================================================
\begin{pspicture}(-3.5,-1.5)(3.5,1.5)%
  \psaxes[linecolor=axis,labels=none,ticks=y]{<->}(0,0)(-3.5,-1.5)(3.5,1.5)%
  \multirput(-2,0)(2,0){3}{\psline{-o}(0,0)(0,0.6366)}%
  \multirput(-3,0)(2,0){4}{\psline{-o}(0,0)(0,-0.6366)}%
  \uput[90](-3,0){$-3$}%
  \uput[-90](-2,0){$-2$}%
  \uput[90](-1,0){$-1$}%
  \uput[90](1,0){$1$}%
  \uput[-90](2,0){$2$}%
  \uput[90](3,0){$3$}%
  %\uput[180](0,0.6366){$\frac{2}{\pi}$}% y=1
  %\uput[0](0,-0.6366){$\frac{-2}{\pi}$}% y=1
 %\uput[0](3.5,0){$n$}%
  \psplot[plotpoints=100,linestyle=dashed,linecolor=red,linewidth=1pt]{-3}{3}{x 180 mul sin}%
  %\rput[b](17.5,10){$\inprod{\ff(t-n)}{\sin(\pi t)}$}%
\end{pspicture}
