%============================================================================
% LaTeX File
% Daniel J. Greenhoe
%============================================================================





%=======================================
\chapter{Grammar}
%=======================================
%=======================================
\section{Personal pronouns}
%=======================================
\ftbox{\tcom{\zhtssH{代}{d/`ai}{substitute} \zhtsdH{名}{詞}{m/'ing}{c/'i}{name}{word}{noun}}{pronoun}}
\quad\begin{minipage}{\tw-65mm}
In English, \hie{personal prounouns} often change substantially depending on
  \\\qquad\begin{tabular}{lll}
    \imark&\hie{gender}             & (e.g. he, she, it) \\
    \imark&\hie{grammatical number} & (e.g. he, them)    \\
    \imark&\hie{grammatical case}   & (e.g. he, his)
  \end{tabular}
\end{minipage}

In Mandarin Chinese, such modifications are simplier.
The spoken form of Mandarin does not distinguish between different genders of pronouns.
        ``He", ``she", and ``it" are all pronounced ``t/-a".
However the written form of Mandarin does distinguish gender.
        ``He" is written \zht{他}, ``she" is written \zht{她}, ``it" is written \zht{牠}.
A singular pronoun becomes plural by adding the \zht{們} (m/'en) character to the end.
A pronoun becomes possessive by adding the \zht{的} (de) character to the end.

Specific examples of pronouns are listed in \prefpp{voc:pronouns}.
\begin{tblvv}{Personal pronouns}{voc:pronouns}
  \tblx I                 & \zhtss {我}{w/>o}{I}
  \tblc we                & \zhtsds{我}{們}{w/>o}{m/'en}{I}{(plural)}{we}
  \tblh you (singular)    & \zhtss {你}{n/>i}{you}
  \tblc you (plural)      & \zhtsds{你}{們}{n/>i}{m/'en}{you}{(plural)}{you}
  \tblh he                & \zhtss {他}{t/-a}{he}
  \tblc them (masculine)  & \zhtsds{他}{們}{t/-a}{m/'en}{he}{(plural)}{them}
  \tblh she               & \zhtss {她}{t/-a}{she}
  \tblc them (femenine)   & \zhtsds{她}{們}{t/-a}{m/'en}{she}{(plural)}{them}
  \tblh it                & \zhtss {它}{t/-a}{it}
  \tblc them (neuter)     & \zhtsds{它}{們}{t/-a}{m/'en}{it}{(plural)}{them}
  \tblh my                & \zhtsds{我}{的}{w/>o}{de}{I}{('s)}{my}
  \tblc our               & \zhtsds{我}{的}{w/>o}{de}{I}{('s)}{my}  \zhtss{的}{de}{('s)}
  \tblh your (singular)   & \zhtsds{你}{的}{n/>i}{de}{you}{('s)}{your}
  \tblc your (plural)     & \zhtsds{你}{們}{n/>i}{m/'en}{you}{}{you}  \zhtss{的}{de}{('s)}
  \tblh his               & \zhtsds{他}{的}{t/-a}{de}{he}{}{his}
  \tblc their (masculine) & \zhtsds{他}{們}{t/-a}{m/'en}{he}{}{their}  \zhtss{的}{de}{('s)}
  \tblh her               & \zhtsds{她}{的}{t/-a}{de}{she}{('s)}{her}
  \tblc their (femenine)  & \zhtsds{她}{們}{t/-a}{m/'en}{she}{}{them} \zhtss{的}{de}{('s)}
  \tblh it's              & \zhtsds{它}{的}{t/-a}{de}{it}{('s)}{it's}
  \tblc their (neuter)    & \zhtsds{它}{們}{t/-a}{m/'en}{it}{}{them}  \zhtss{的}{de}{('s)}
\end{tblvv}

In English, when we want to emphasize a noun or pronoun,
we might do so by adding some word after the noun/pronoun such as in
``I myself", ``he himself", ``they themselves", ``Mary herself", ``The President himself", etc.
In Chinese, a similar effect can be achieved by the Chinese character
\zht{才} (c/'ai) or \zht{則} (z/'e).
Examples appear in \pref{ex:cai} (next).
%---------------------------------------
\begin{example}[\exmd{I myself}]
\mbox{}\\
\label{ex:cai}
%---------------------------------------
\begin{enumerate}
  \item \zhtsd{我}{才}{w/>o}{c/'ai}{I}{only}{I myself}%
        \zhtsd{討}{厭}{t/>ao}{y/`an}{to demand}{to detest}{to loathe}%
        \zhtss{你}{n/>i}{you}%
        \zhtss{呢}{ne}{(a sound)}%
        \zhtsX%
        \citetbl{\citerp{cloud_pillow}{1}}
        \engbox{I myself\\loathe you!}

  \item \zhtsd{爸}{爸}{b/`a}{ba}{dad}{dad}{dad}%
        \zhtss{則}{z/'e}{as for him}%
        \zhtsd{負}{責}{f/`u}{z/'e}{to bear}{duty}{to be in charge of}%
        \zhtss{推}{t/-ui}{to push}%
        \tcom{\zhtss{手}{sh/>ou}{hand}\zhtss{推}{t/-ui}{to push}\zhtss{車}{ch/-e}{wheeled devices}}{shopping cart}
        \zhtsP%
        \citetbl{\citerp{banban4}{8}}
        \\\engbox{Dad, as for him, was in charge of pushing the shopping cart.}
\end{enumerate}
\end{example}


%=======================================
\section{Quantifiers}
%=======================================
For every noun in Chinese, there is a \hie{quantifier} character,
or \hie{measure words} (\hie{MW}).
So when you talk about a number of something, the quantifier goes between the
number and the noun.
The most common quantifier is \zht{個} (ge).
And when in doubt, just use the the \zht{個} quantifier and you will be understood,
howbeit possibly using substandard Chinese.
\pref{voc:mw} (next) provides some quantifiers used in Chinese.
\begin{tblmw}{quantifiers for countable nouns}{voc:mw}
  \tblx \zhtss{把}{b/>a}{handle}
      & things with long handles
      & \zhtsd{鏟}{子}{ch/>an}{zi}{shovel}{son}{shovel}\zhtsC
        \zhtsd{梳}{子}{sh/-u}{zi}{comb}{son}{comb}
  \tblh \zhtss{本}{b/>en}{stem}%
      & books%
      & \zhtss{書}{sh/-u}{book}%
  \tblh \zhtss{部}{b/`u}{part}
      & vehicles
      & \zhtsd{車}{子}{ch/-e}{zi}{vehicle}{son}{vehicle}
  \tblh \zhtss{棟}{d/`ong}{main beam}
      & buildings
      & \zhtsd{木}{屋}{m/`u}{w/-u}{wood}{house}{cottage}
  \tblh \zhtss{段}{d/`uan}{paragraph}%
      & a length of things
      & \zhtsd{時}{間}{/'shi}{ji/-an}{time}{between}{time}\zhtsC
        \zhtsd{柳}{條}{li/>u}{ti/'ao}{willow}{stip}{willow twig}
  \tblh \zhtss{頓}{d/`un}{pause}
      & meals
      &
  \tblh \zhtssp{朵}{d/>uo}{(quantifier)}
      & flowers
      & \zhtss{花}{h/-ua}{flower}
  \tblh \zhtss{封}{f/-eng}{to seal}%
      & letters (of correspondence)%
      & \zhtss{信}{xi/`n}{letter}%
  \tblh \zhtss{份}{f/`en}{portion}
      & copies of documents, gifts
      & \zhtsd{報}{紙}{b/`ao}{zh/>i}{to report}{paper}{newspaper}\zhtsC
        \zhtsd{禮}{物}{l/>i}{w/`u}{ceremony}{matter}{gift}
  \tblh \zhtss{幅}{fu}{width of cloth}
      & pictures, paintings
      & \zhtss{畫}{h/`ua}{picture}
  \tblh \zhtss{個}{ge}{(quantifier)}
      & general purpose quantifier
      &
  \tblh \zhtss{根}{g/-en}{roots}
      & root like things
      & \zhtsd{樹}{支}{sh/`u}{zh/-i}{tree}{branch}{branch}\zhtsC
        \tcom{\zhtss{胡}{h/'u}{reckless}\zhtss{蘿}{l/'uo}{creeping plant}\zhtss{蔔}{bo}{edible roots}}{carrot}\zhtsC
        \zhtsd{鞭}{子}{bi/-an}{zi}{whip}{son}{whip}
  \tblh \zhtss{戶}{h/`u}{door}
      & things with doors
      & \zhtsd{人}{家}{r/'en}{ji/-a}{people}{household}{household}
  \tblh \zhtss{間}{ji/-an}{room}
      & houses and rooms
      & \zhtsd{房}{子}{f/'ang}{zi}{house}{son}{house}
  \tblh \zhtss{家}{ji/-a}{household}%
      & companies
      & \zhtsd{公}{司}{g/-ong}{s/-i}{public}{in charge of}{company}
  \tblh \zhtss{件}{ji/`an}{document}
      & matters
      & \zhtsd{事}{情}{sh/`i}{qi/'ng}{affairs}{situation}{a matter}
  \tblh \zhtss{棵}{k/-e}{(quantifier)}
      & trees, and small objects
      & \zhtss{球}{qi/'u}{ball}\zhtsC
        \zhtsd{貝}{殼}{b/`ei}{k/'e}{shellfish}{shell}{shell}\zhtsC
        \zhtsd{草}{莓}{c/>ao}{m/'ei}{straw}{berries}{strawberry}\zhtsC
        \zhtsd{番}{茄}{f/-an}{qi/'e}{a sort of}{eggplant}{tomato}\zhtsC
        \zhtss{樹}{sh/`u}{tree}
  \tblh \zhtss{口}{k/>ou}{mouth}
      & objects with openings
      & \zhtss{井}{j/>ing}{well}
  \tblh \zhtss{塊}{ku/`ai}{chunk}
      & chunks of things
      & \zhtsd{積}{木}{j/-i}{m/`u}{to accumulate}{wood}{building blocks}
  \tblh \zhtss{粒}{l/`i}{granule}
      & small items
      & \zhtsd{鈕}{扣}{ni/>u}{k/`ou}{button}{to button}{clothing button}
  \tblh \zhtssp{輛}{li/`ang}{(quantifier)}
      & vehicles
      & \zhtsd{汽}{車}{q/`i}{ch/-e}{steam}{vehicle}{car}
  \tblh \zhtss{面}{mi/`an}{face to face}%
      & things with a face
      & \zhtsd{鏡}{子}{ji/`ng}{zi}{mirror}{son}{mirror}
  \tblh \zhtss{匹}{p/-i}{equal to}
      & pack animals
      & \zhtss{馬}{m/>a}{horse}\zhtsC
        \zhtsd{斑}{馬}{b/-an}{m/>a}{spots}{horse}{zebra}%
  \tblh \zhtss{片}{pi/`an}{slice}
      & wide and flat objects
      & \zhtsd{花}{園}{h/-ua}{y/'uan}{flower}{garden}{flower garden}
  \tblh \zhtss{台}{t/'ai}{platform}
      & heavy machines and appliances
      & \zhtsd{電}{視}{di/`an}{sh/`i}{electric}{to look at}{television}\zhtsC
        \zhtsd{電}{腦}{di/`an}{n/>ao}{electric}{brain}{computer}
  \tblh \zhtss{堂}{t/'ang}{hall}
      & class session
      & \zhtss{課}{k/`e}{lesson}
  \tblh \zhtss{條}{ti/'ao}{long strip}
      & long ribbon-like objects
      & \zhtss{路}{l/`u}{road}\zhtsC
        \zhtss{魚}{y/'u}{fish} (see also \zht{尾} w/>ei)\zhtsC
        \zhtsd{毛}{巾}{m/'ao}{j/-in}{hair}{cloth}{towel}
  \tblh \zhtss{頭}{t/'ou}{head}
      & large animals
      & \zhtss{牛}{ni/'u}{cow}\zhtsC
        \zhtsd{大}{象}{d/`a}{xi/`ang}{big}{elephant}{elephant}\zhtsC
        \zhtsd{獅}{子}{sh/-i}{zi}{lion}{son}{lion}
  \tblh \zhtss{尾}{w/>ei}{tail}%
      & fish
      & \zhtss{魚}{y/'u}{fish} (more commonly \zht{條} ti/'ao)
  \tblh \zhtss{位}{w/`ei}{location}%
      & polite quantifier for people; can also use \zht{個} (ge) %, which is more informal
      & \zhtsd{女}{孩}{n/>:u}{h/'ai}{female}{child}{girl} (see also \prefp{ex:weiwei})
  \tblh \zhtss{盞}{zh/>an}{small cup}
      & lamps
      & \zhtsd{油}{燈}{y/'ou}{d/-eng}{oil}{lamp}{oil lamp}
  \tblh \zhtss{張}{zh/-ang}{to spread}
      & flat objects
      & \zhtsd{紙}{條}{zh/>i}{ti/'ao}{paper}{strip}{slip of paper}\zhtsC
        \zhtsd{書}{桌}{sh/-u}{zh/-uo}{book}{table}{desk}\zhtsC
        \zhtss{床}{ch/'uang}{bed}
  \tblh \zhtss{隻}{zh/-i}{single}
      & animals (both big and small), hand-held stick-like utensils
      & \zhtsd{大}{象}{d/`a}{xi/`ang}{big}{elephant}{elephant}\zhtsC
       %\zhtsd{小}{鴨}{xi/>ao}{y/-a}{little}{duck}{duckling}\zhtsC
        \zhtsd{蝴}{蝶}{h/'u}{di/'e}{butterfly}{butterfly}{butterfly}\zhtsC
        \zhtsd{筷}{子}{k/`uai}{zi}{chopsticks}{son}{chopsticks}\zhtsC
        \zhtsd{牙}{刷}{y/'a}{sh/-ua}{tooth}{brush}{toothbrush}
\end{tblmw}

An interesting way to make a noun plural is to double the quantifier in front of it,
as illustrated in \pref{ex:weiwei} (next).

%---------------------------------------
\begin{example}[\exmd{plural form using double \zht{位}}]
\mbox{}\\
\label{ex:weiwei}
%---------------------------------------
%這些天鵝突然變成一位位女孩。
\exboxt{
  \zhtsd{這}{些}{zh/`e}{xi/-e}{this}{some}{these}%
  \zhtsd{天}{鵝}{ti/-an}{/'e}{sky}{goose}{swan}%
  \zhtsd{突}{然}{t/'u}{r/'an}{suddenly}{just so}{suddenly}%
  \zhtsd{變}{成}{bi/`an}{ch/'eng}{become}{accomplish}{become}%
  \zhtss{一}{y/'i}{one}%
  \zhtss{位}{w/`ei}{(mw)}%
  \zhtss{位}{w/`ei}{(mw)}%
  \zhtsd{女}{孩}{n/>:u}{h/'ai}{female}{child}{girl}%
  \zhtsP\footnotemark
  \\\engbox{These swans suddenly became girls.}
  }\citetblt{\citerp{swanlake}{6}}
\end{example}

%=======================================
\section{Adjectives}
%=======================================
\ftbox{\tcom{\zhtsd{形}{容}{x/'ing}{r/'ong}{appearance}{contain}{to describe}
            \zhtss{詞}{c/'i}{word}
            }{adjective}}
\quad
\begin{minipage}{\tw-75mm}
As in English, an adjective in Chinese often comes before the noun it describes.
%Some common verbs are listed in \prefpp{voc:verbs} and some examples listed in \prefpp{sen:verbs_ex}.
\end{minipage}

%\zhtsdh{大}{小}{d/`a}{xi/>ao}{big}{small}{size}

%---------------------------------------
\begin{example}[\exmd{Common adjectives describing size}]
\mbox{}\\
\label{ex:adj_size}
%---------------------------------------
\begin{tabular}{|*{12}{c|}}
  \hline
  \tblx\zhtss{大}{d/`a}   {big}
  \tblc\zhtss{小}{x/>iao}{small}
  \tblc\zhtss{長}{ch/'ang}{long}
  \tblc\zhtss{短}{d/>uan}{short}
  \tblc\zhtss{高}{g/-ao}{tall}
  \tblc\zhtss{矮}{a/>i}  {short}
  \tblc\zhtss{寬}{k/-uan }{wide}
  \tblc\zhtss{窄}{zh/>ai}{narrow}
  \tblc\zhtss{胖}{p/`ang}{fat}
  \tblc\zhtss{瘦}{sh/`ou}{thin}
  \tblc\zhtss{重}{zh/`ong}{heavy}
  \tblc\zhtss{輕}{q/-ing}{light}
  \\\hline
\end{tabular}
\end{example}


%\begin{tabular}{l|l}
%\hline
%English 英文 $\rightarrow$ Chinese 中文  &
%Chinese 中文 $\rightarrow$ English 英文  \\
%\hline
%\hline
%\begin{tabular}{*{8}{c}}
%   Is     & it    & big     & or  & small? \\
%   是     & 它    & 大      & 還是 & 小 ? \\
%   Sh/`i  & t/-a  & d/`a    & h/'ai sh/`i & xi/>ao ?
%\end{tabular}
%&
%\begin{tabular}{*{8}{c}}
%   它    & 是    & 大      & 還是        & 小 ? \\
%   T/-a  & sh/`i & d/`a    & h/'ai sh/`i & xi/>ao ? \\
%   It    & is    & big     & or          & small?
%\end{tabular}
%\\
%\hline
%\begin{tabular}{*{8}{c}}
%   Is     & it    & long    & or   & short? \\
%   是     & 它    & 長      & 還是 & 短? \\
%   Sh/`i  & t/-a  & ch/'ang & h/'ai sh/`i & d/>uan ?
%\end{tabular}
%&
%\begin{tabular}{*{8}{c}}
%   它    & 是    & 長      & 還是        & 短 ? \\
%   T/-a  & sh/`i & ch/'ang & h/'ai sh/`i & d/>uan ? \\
%   It    & is    & long    & or          & short?
%\end{tabular}
%\\
%\hline
%\begin{tabular}{*{8}{c}}
%   It     & is    & tall.\\
%   它     & 是    & 高. \\
%   T/-a   & sh/`i & g/-ao.
%\end{tabular}
%&
%\begin{tabular}{*{8}{c}}
%   它     & 是      & 高. \\
%   T/-a   & h/>en & g/-ao. \\
%   It     & very    & tall.
%\end{tabular}
%\\
%\hline
%\begin{tabular}{*{8}{c}}
%   It     & is    & narrow.\\
%   它     & 是    & 狹窄. \\
%   T/-a   & sh/`i & x/'ia zh/>ai.
%\end{tabular}
%&
%\begin{tabular}{*{8}{c}}
%   它     & 是    & 狹窄. \\
%   T/-a   & sh/`i & x/'ia zh/>ai.\\
%   It     & is    & narrow.
%\end{tabular}
%\\
%\hline
%\end{tabular}
%


%=======================================
\section{Questions}
%=======================================
\ftbox{\zhtsdH{問}{題}{w/`en}{t/'i}{to ask}{a subject}{question}}
\quad\begin{minipage}{\tw-45mm}%
  Questions in English are often expressed with the assistance of changing the tone
  of the sentence---often using a rising tone.
  Sometimes boolean questions(where the response is expected to be either true or false)
  are expressed using the ``\hie{or not}" clause as in ``Are you coming or not?"
  In Chinese, tone is used to express the meaning of individual words and
  so tone alone is not such a good option for expressing questions in Chinese.
\end{minipage}

\begin{minipage}{\tw-30mm}%
  There are three basic forms of asking questions in Chinese:
  \begin{dingautolist}{"C0}
    \item Placing the \zht{嗎} character %(also displayed to the right)
          at the end of an interrogative sentence
          as illustrated in the left half of \prefpp{sen:boolean}.
    \item Using a question word such as
          \zht{什麼}, \zht{誰}, \zht{怎麼}, \zht{哪裡}, or \zht{哪個}
          as described in \prefpp{voc:question}.
    \item Using a boolean construction such as \hie{verb {\bf not} verb}
          as illustrated in the right half of \prefpp{sen:boolean}.
  \end{dingautolist}
\end{minipage}\quad
\ftbox{\zhtssH{嗎}{ma}{(?)}}

\begin{tblvv}{Question words}{voc:question}
  \tblx how        & \zhtsds{怎}{麼}{z/>en}{me}{who}{}{how}
  \tblc what       & \zhtsds{什}{麼}{sh/'e}{me}{what}{}{what}
  \tblh where      & \zhtsds{哪}{裡}{n/>a}{l/>i}{where}{inside}{where}
  \tblc which      & \zhtsds{哪}{個}{n/>ei}{ge}{which}{}{which}
  \tblh who        & \zhtssp{誰}{sh/'ei}{who}
  \tblc why        & \zhtss{為}{w/`ei}{for} \zhtsds{什}{麼}{sh/'e}{me}{what}{}{what}
\end{tblvv}

\begin{tblss}{Boolean question sentences}{sen:boolean}
    Do you want to eat?
    &
    \zhtss{你}{n/>i}{you} \zhtss{要}{y/`ao}{want} \zhtss{吃}{ch/-i}{eat}
    \zhtss{飯}{f/`an}{rice} \zhtss{嗎}{ma}{(?)}\zhtsQ
  &
    Do you want to eat?
    &
    \zhtss{你}{n/>i}{you}
    \zhtss{要}{y/`ao}{want} \zhtss{不}{b/'u}{not} \zhtss{要}{y/`ao}{want}
    \zhtss{吃}{ch/-i}{eat}
    \zhtss{飯}{f/`an}{rice} \zhtss{嗎}{ma}{(?)}\zhtsQ
  \tblh
    Do you have old newspapers?
    &
    \zhtss{你}{n/>i}{you}
    \zhtss{有}{y/>ou}{have}
    \zhtsd{舊}{的}{ji/`u}{de}{old}{(adj)}{old}
    \zhtsd{報}{紙}{b/`ao}{zh/>i}{inform}{paper}{newspaper}
    \zhtss{嗎}{ma}{(?)}\zhtsQ
  &
    Do you have old newspapers?
    &
    \zhtss{你}{n/>i}{you}
    \zhtss{有}{y/>ou}{have} \zhtsd{沒}{有}{m/'ei}{y/>ou}{not}{have}{not have}
    \zhtsd{舊}{的}{ji/`u}{de}{old}{(adj)}{old}
    \zhtsd{報}{紙}{b/`ao}{zh/>i}{inform}{paper}{newspaper}
    \zhtss{嗎}{ma}{(?)}\zhtsQ
  \tblh
    Is it the green one?
    &
    \zhtss{是}{sh/`i}{is}
    \zhtsd{綠}{色}{l/`:u}{s/`e}{green}{color}{green} \zhtss{的}{de}{(adj)}
    \zhtss{嗎}{ma}{(?)}\zhtsQ
  &
    Is it the green one?
    &
    \zhtss{是}{sh/`i}{is} \zhtss{不}{b/'u}{not} \zhtss{是}{sh/`i}{is}
    \zhtsd{綠}{色}{l/`:u}{s/`e}{green}{color}{green} \zhtss{的}{de}{(adj)}\zhtsQ
\end{tblss}


\begin{tblvv}{Miscellaneous Question words}{voc:question_misc}
  \tblx Excuse me, can I ask \ldots
    & \zhtsds{請}{問}{q/>ing}{w/`en}{to request}{to ask}{excuse me} \ldots
  \tblc to consult
    & \zhtsds{請}{教}{q/>ing}{ji/`ao}{to request}{to teach}{to consult}
      \citex{10crows}{59}
  \tblh is it possible that \ldots & \zhtsds{難}{道}{n/'an}{d/`ao}{difficult}{road}{is it possible that \ldots}
\end{tblvv}

\begin{tbls}{Question sentences}{sen:question_misc}
  \tblx Is it possible that someone is playing a practical joke?
      & \zhtsd{難}{道}{n/'an}{d/`ao}{difficult}{road}{is it possible that}%
        \zhtss{有}{y/>ou}{have}%
        \zhtss{人}{r/'en}{person}%
        \tcom{\zhtss{惡}{/`e}{evil}\zhtss{作}{z/`uo}{to do}\zhtss{劇}{j/`u}{a play}}{practical joke}%
        \zhtsQ%
        \citetbl{\citerp{lovemind_cat}{18}}
  \tblh Is it possible that all of us wolves have always been this bad?
      & \zhtsd{我}{們}{w/>o}{men}{I}{(s)}{we}%
        \zhtsd{野}{狼}{y/>e}{l/'ang}{wild}{wolves}{wolves}%
        \zhtsd{難}{道}{n/'an}{d/`ao}{difficult}{road}{is it possible that}%
        \zhtsd{永}{遠}{y/>ong}{y/>uan}{always}{far}{always}%
        \zhtss{都}{d/-ou}{all}%
        \zhtsd{這}{麼}{zh/`e}{me}{this}{}{like this}%
        \zhtss{壞}{hu/`ai}{bad}%
        \zhtss{嗎}{ma}{(?)}\zhtsQ%
        \citetbl{\citerp{wolfgift}{4}}%
\end{tbls}
%=======================================
\section{Conditionals}
%=======================================
\begin{tabular}{|c||c|c|}
  \hline
   \tcom{\zhtsdH{條}{件}{ti/'ao}{ji/`an}{orderliness}{document}{conditions}
        \zhtsdH{子}{句}{z/>i}{j/`u}{seed}{sentence}{clause}%
        }{conditional}
  & \zhtsdH{如}{果}{r/'u}{g/>uo}{according to}{fruit}{if}
  & \zhtsdH{只}{要}{zh/>i}{y/`ao}{only}{want}{provided that}
  \\\hline
\end{tabular}

%=======================================
\section{Conjunctions}
%=======================================
\begin{tabular}{|*{4}{c|}}
   \hline
   \tblx\zhtsd{雖}{然}{s/-ui}{r/'an}{although}{just like that}{although}
   \tblc\zhtss{和}{h/`an}{and}
   \tblc\zhtsd{其}{實}{q/'i}{sh/'i}{that}{true}{in fact}
   \tblc\zhtsd{還}{是}{h/'ai}{sh/`i}{still}{is}{or}
   \tblh\zhtss{還}{h/'ai}{still}
   \tblc\zhtsd{仍}{然}{r/'eng}{r/'an}{still}{just like that}{still}
   \tblc\zhtsd{於}{是}{y/'u}{sh/`i}{at}{that}{thus}
   \tblc\zhtss{便}{bi/`an}{thereupon}
   \tblh\zhtsd{居}{然}{j/-u}{r/'an}{to live at}{just like that}{unexpectedly}
   \tblc\zhtsd{才}{能}{c/'ai}{n/'eng}{only}{can}{so as to}
   \tblc\zhtss{還}{h/'ai}{still}\zhtss{不}{b/'u}{not}\zhtss{是}{sh/`i}{is}
       &
  \\\hline
\end{tabular}

%---------------------------------------
\begin{example}[\exmd{Conjunctions}]
\mbox{}\\
\label{ex:conj}
%---------------------------------------
\begin{enumerate}
  \item \zhtss{嗯}{/`en}{Hmmm}%
        \ldots%
        \zhtsd{怎}{樣}{z/>e}{y/`ang}{who}{shape}{how}%
        \zhtsd{才}{能}{c/'ai}{n/'eng}{only}{can}{so as to}%
        \zhtsd{吸}{引}{/-xi}{y/>in}{inhale}{pull}{attact}%
        \zhtsd{查}{裡}{ch/'a}{l/>i}{to check}{inside}{Charlie}%
        \zhtss{吃}{ch/-i}{eat}%
        \zhtsd{晚}{餐}{w/>an}{c/-an}{evening}{meal}{dinner}%
        \zhtssp{呢}{ne}{}%
        \zhtsQ%
        \cittrp{stephens1999}{6}%
        \\\engbox{Hmmm \ldots What can be done to entice Charlie to eat dinner?}

  \item %我昨天還不是被他嚇的半死﹗
        \zhtss{我}{w/>o}{I}
        \zhtsd{昨}{天}{z/'uo}{ti/-an}{past}{sky}{yesterday}
        \tcom{\zhtss{還}{h/'ai}{still}\zhtss{不}{b/'u}{not}\zhtss{是}{sh/`i}{is}}{as well}
        \zhtss{被}{b/`ei}{by}
        \zhtss{他}{t/-a}{he}
        \zhtss{嚇}{xi/`a}{to scare}
        \zhtss{的}{de}{to the extent of}
        \zhtsd{半}{死}{b/`an}{s/>i}{half}{death}{half to death}
        \zhtsX
        \citetbl{\citerp{zhang2006catbell}{4}}
        \\\engbox{Yesterday I as well was scared half to death by him!}
\end{enumerate}
\end{example}

\begin{tblvv}{Contrasting conjuctions}{voc:conj_contrast}
  \tblx but              & \zhtssp{卻}{q/`ue}{but}
  \tblc furthermore      & \zhtsds{而}{且}{/'er}{qi/>e}{and/but}{just}{furthermore}
  \tblh nevertheless     & \zhtsds{還}{是}{h/'ai}{sh/`i}{still}{that}{nevertheless} \citetbl{\citerp{fragrant_gas}{4}}
  \tblc not only         & \zhtsds{不}{但}{b/'u}{d/`an}{not}{but}{not only}
  \tblh on the contrary  & \zhtsds{相}{反}{xi/-ang}{f/>an}{mutually}{opposite}{on the contrary}\citetbl{\citerp{fragrant_gas}{10}}
  \tblc instead          & \zhtsds{反}{而}{f/>an}{/'er}{contrary}{and/but}{instead}
\end{tblvv}


%---------------------------------------
\begin{example}[\exmd{Contrasting conjuctions}]
\mbox{}\\
\label{ex:concon}
%---------------------------------------
\exboxt{
  \zhtss{口}{k/>ou}{mouth}%
  \zhtss{不}{b/`u}{not}%
  \zhtss{渴}{k/>e}{thirsty}%
  \zhtssp{了}{le}{(a change of state)}%
  \zhtsC%
  \zhtsd{肚}{子}{d/`u}{zi}{belly}{son}{stomach}%
  \zhtss{卻}{q/`ue}{however}%
  \zhtss{餓}{/`e}{hungry}%
  \zhtssp{了}{le}{(a change of state)}
  \zhtsX\footnotemark
  \engbox{The mouth is no longer thirsty,\\but the stomach however\\becomes hungry!}
  }
  \citetblt{\citerp{lovemind_octopus}{14}}%
\end{example}

Chinese is remarkably relaxed when it comes to the use of the conjuction \hie{and}.
In particular, this conjunction is very often left out altogether
when expressing two events that happen in succession,
as illustrated in \pref{ex:and} (next).

%---------------------------------------
\begin{example}[\exmd{Conjunction and}]
\mbox{}\\
\label{ex:and}
%---------------------------------------
\exboxt{\zhtsd{打}{開}{d/>a}{k/-ai}{to beat}{to open}{to open}%
        \zhtsd{大}{門}{d/`a}{m/'en}{big}{door}{front door}\zhtsC
        \zhtss{小}{xi/>ao}{little}\zhtss{茜}{qi/`an}{madder}%
        \zhtsd{發}{現}{f/-a}{xi/`an}{to send out}{current}{to discover}%
        \zhtsd{媽}{媽}{m/-a}{ma}{mother}{mother}{mother}%
        \zhtss{去}{q/`u}{to go}%
        \\\zhtsd{上}{班}{sh/`ang}{b/-an}{up}{shift}{to be at work}%
        \zhtss{還}{h/'ai}{still}\zhtss{沒}{m/'ei}{no}%
        \zhtsd{回}{家}{h/'ui}{ji/-a}{to back to}{home}{to return home}\zhtsP%
        \footnotemark
        \engbox{Opening the front door, Little Qian\\discovered that Mother had gone to work\\and had not yet returned home.}
       }\citetblt{\citerp{lovemind_rain}{15}}%
\end{example}



\ftbox{\zhtssH{又}{y/`ou}{again} \zhtssh{\blank[2ex]}{}{}
      \zhtssH{又}{y/`ou}{again} \zhtssh{\blank[2ex]}{}{}}
\quad\begin{minipage}{\tw-55mm}%
  The English sentence pattern ``\hie{both \blank  and \blank }" can be expressed
  in Chinese using the pattern illustrated to the left.
  Examples follow in \prefpp{ex:grm_both_and}.
\end{minipage}

%---------------------------------------
\begin{example}[\exmd{Both \blank and \blank}]
\mbox{}\\
\label{ex:grm_both_and}
%---------------------------------------
\begin{enumerate}
  \item \zhtss{小}{xi/>ao}{little} \zhtsd{烏}{鴉}{w/-u}{y/-a}{black}{crow}{crow}%
        \zhtss{又}{y/`ou}{again}%
        \zhtss{餓}{/`e}{hungry}%
        \zhtss{又}{y/`ou}{again}%
        \zhtss{渴}{k/>e}{thirsty}\zhtsP%
        \citetbl{\citerp{10crows}{57}}
        \engbox{The little crow was both hungry and thirsty.}

  \item \zhtsd{小}{貓}{xi/>ao}{m/-ao}{little}{cat}{kitten}%
        \zhtss{又}{y/`ou}{again}%
        \zhtss{濕}{sh/-i}{wet}%
        \zhtss{又}{y/`ou}{again}%
        \zhtss{冷}{l/>eng}{cold}%
        \zhtsP%
        \citetbl{\citerp{lovemind_cat}{10}}%
        \engbox{The kitten was both wet and cold.}
\end{enumerate}
\end{example}

\ftbox{\tcom{
    \zhtsdH{不}{是}{b/'u}{sh/`i}{not}{is}{(either)}
    \zhtssH{\blank[2ex]}{}{}
    \zhtsdH{就}{是}{ji/`u}{sh/`i}{}{is}{(or)}
    \zhtssH{\blank[2ex]}{}{}
    }{either \blank or \blank}}\hfill%
\begin{minipage}{\tw-105mm}%
  The English sentence pattern ``\hie{either \blank or \blank}" can be expressed
  in Chinese using the pattern illustrated to the left.
  Examples follow in \prefpp{ex:grm_either_or}.
\end{minipage}

%---------------------------------------
\begin{example}[\exmd{either \blank or \blank}]
\mbox{}\\
\label{ex:grm_either_or}
%---------------------------------------
\zhtsd{爸}{爸}{b/`a}{ba}{dad}{dad}{dad}
\zhtsd{平}{常}{pi/'ng}{ch/'ang}{level}{usually}{usual}
\zhtss{對}{d/`ui}{as regards}
\zhtsd{湯}{姆}{t/-ang}{m/>u}{soup}{governess}{Tom}
\zhtsd{不}{是}{b/'u}{sh/`i}{not}{is}{(either)}
\zhtss{罵}{m/`a}{to scold},
\zhtsd{就}{是}{ji/`u}{sh/`i}{}{is}{no other than}
\zhtss{打}{d/>a}{beat}\zhtsP
\citetbl{\citerp{prince_pauper}{2}}
\\\engbox{Dad's usual approach to Tom was when not scolding him, he was beating him.}
\end{example}


\ftbox{\begin{tabular}{cccc}
    \zhtsd{不}{僅}{b/`u}{j/>in}{not}{only}{not only}&
    \zhtss{\blank}{}{}\zhtsC&
    \zhtsd{而}{且}{/'er}{q/>ie}{and}{just}{but also}&
    \zhtss{\blank}{}{}
    \\
    \zhtsd{不}{但}{b/`u}{d/`an}{not}{but}{not only}&
    \zhtss{\blank}{}{}\zhtsC&
    \zhtss{還}{h/'ai}{still}\zhtss{要}{y/`ao}{want}&
    \zhtss{\blank}{}{}
  \end{tabular}}
\quad\begin{minipage}{\tw-100mm}%
  The English sentence pattern ``\hie{not only \blank, but also \blank}" can be expressed
  in Chinese using either of the patterns shown to the left.
  Examples follow in \pref{ex:notonly_butalso} (next).
\end{minipage}

%---------------------------------------
\begin{example}[\exmd{not only \blank but also \blank}]
\mbox{}\\
\label{ex:notonly_butalso}
%---------------------------------------
\begin{enumerate}
  \item \zhtss{我}{w/>o}{I}\zhtsds{不}{但}{b/'u}{d/`an}{not}{but}{not only}%
        \zhtss{想}{xi/>ang}{would like to}%
        \zhtss{見}{ji/`an}{to see}\zhtss{你}{n/>i}{you}\zhtsC
        \zhtss{還}{h/'ai}{still}\zhtss{要}{y/`ao}{want}%
        \zhtsd{好}{好}{h/>ao}{h/>ao}{good}{good}{all out}%
        \zhtsd{感}{謝}{g/>an}{xi/`e}{to feel}{to thank}{to thank}%
        \zhtss{你}{n/>i}{you}\zhtssp{呢}{ne}{}\zhtsX%
        \citetbl{\citerp{lovemind_octopus}{22}}%
        \\\engbox{I not only want to see you, I also want to all out thank you!}

  \item \zhtsd{國}{王}{g/'uo}{w/'ang}{country}{king}{king}%
        \zhtsd{不}{僅}{b/`u}{j/>in}{not}{only}{not only}%
        \zhtsd{當}{面}{d/-ang}{mi/`an}{equal}{face}{in someone's presence}%
        \zhtsd{誇}{獎}{k/-ua}{ji/>ang}{praise}{reward}{to commend}%
        \zhtsd{他}{們}{t/-a}{m/'en}{he}{s}{them}\zhtsC%
        \zhtsd{而}{且}{/'er}{q/>ie}{and}{just}{but also}%
        \zhtss{還}{h/'ai}{still}%
        \zhtsd{賞}{賜}{sh/>ang}{c/`i}{reward}{gift}{award}%
        \zhtssp{了}{le}{}%
        \zhtsd{許}{多}{x/>u}{d/-uo}{allow}{much}{much}%
        \zhtsd{今}{幣}{j/-in}{b/`i}{gold}{currency}{gold}%
        \zhtss{給}{g/>ei}{to give}%
        \zhtsd{他}{們}{t/-a}{m/'en}{he}{s}{them}\zhtsP%
        \citetbl{\citerp{3musketeers}{18}}%
        \engbox{The king not only face to face\\commended them, but also awarded\\them with much gold.}
\end{enumerate}
\end{example}

%=======================================
\section{Interjections}
%=======================================
\begin{tabular}{|c||c|c|c|}
  \hline
  \tcom{\zhtssH{感}{g/>en}{to feel}\zhtssH{歎}{t/`an}{sigh}\zhtssH{語}{y/>u}{language}}{interjection}
  &\zhtssH{嘿}{h/-ei}{hey}
  &\zhtssH{哇}{w/-a}{wow}
  &\zhtssH{嗯}{/`en}{Hmmm}
  \\\hline
\end{tabular}

%---------------------------------------
\begin{example}[\exmd{Interjections}]
\mbox{}\\
\label{ex:interjections}
%---------------------------------------
For example using ``Hmmmm", see \prefpp{ex:conj}.
\end{example}

%=======================================
\section{Prepositional phrases}
%=======================================
\ftbox{\zhtssH{來}{l/'ai}{to come}}
\hfill%
\begin{minipage}{\tw-35mm}%
  Expressing the purpose of some action can be accomplished using the \zht{來}
 character, which means ``to come".
  In this context, it can be translated as ``to", ``so that", or ``for the purpose of".
  This is illustrated in \pref{ex:purpose} (next).
\end{minipage}

%---------------------------------------
\begin{example}[\exmd{for the purpose of}]
\mbox{}\\
\label{ex:purpose}
%---------------------------------------
\begin{enumerate}
  \item \zhtss{小}{xi/>ao}{little}%
        \zhtss{狼}{l/'ang}{wolf}%
        \zhtsd{決}{定}{j/'ue}{di/`ng}{to decide}{to fix}{to resolve}%
        \zhtss{多}{d/-uo}{many}%
        \zhtss{做}{z/`uo}{to do}%
        \zhtsd{好}{事}{h/>ao}{sh/`i}{good}{affairs}{good deeds}%
        \zhtss{來}{l/'ai}{to come}%
        \zhtsd{建}{立}{ji/`an}{l/`i}{to establish}{to stand}{to build up}%
        \zhtss{野}{y/>e}{wild}%
        \zhtss{狼}{l/'ang}{wolves}%
        \zhtss{的}{de}{('s)}%
        \zhtss{好}{h/>ao}{good}%
        \zhtsd{形}{象}{xi/'ng}{xi/`ang}{appearance}{appearance}{image}%
        \zhtsP%
        \citetbl{\citerp{wolfgift}{5}}%
        \engbox{Little Wolf resolved to do many good deeds\\to build up the good image of wild wolves.}

  \item \zhtss{我}{w/>o}{I}%
        \zhtss{真}{zh/-en}{really}%
        \zhtss{不}{b/`u}{not}%
        \zhtsd{應}{該}{y/-ing}{g/-ai}{should}{should}{should}%
        \zhtss{出}{ch/-u}{to exit}%
        \zhtss{壞}{h/`uai}{bad}%
        \zhtsd{主}{意}{zh/>u}{y/`i}{owner}{meaning}{idea}%
        \zhtss{來}{l/'ai}{to come}%
        \zhtsd{捉}{弄}{zh/-uo}{n/`ong}{to catch}{}{to play a joke on}%
        \zhtss{她}{t/-a}{her}%
        \zhtsP%
        \citetbl{\citerp{fox_stork}{15}}%
        \\\engbox{I really shouldn't come up with bad ideas to play tricks on her.}

  \item \zhtsd{他}{們}{t/-a}{men}{he}{(s)}{they}%
        \zhtss{都}{d/-ou}{all}%
        \zhtss{從}{c/'ong}{from}%
        \zhtss{家}{ji/-a}{home}%
        \zhtss{裡}{l/>i}{inside}%
        \zhtsd{伸}{出}{sh/-en}{ch/-u}{to stretch}{to exit}{to stick out}%
        \zhtss{頭}{t/'ou}{head}%
        \zhtss{來}{l/'ai}{to come}%
        \zhtsd{探}{望}{t/`an}{w/`ang}{to investigate}{to look over}{to look about}%
        \zhtsP%
        \citetbl{\citerp{baobao_lion}{4}}%
        \\\engbox{They all stuck their heads out of their houses to look around.}

\end{enumerate}
\end{example}




%=======================================
\section{More}
%=======================================
\begin{minipage}{\tw-81mm}%
In English, there is a sentence pattern ``the more I \blank, the more I \blank".
This same pattern in Chinese can be accomplished using either
the \zht{越} (y/`ue, to get over) character or
the \zht{愈} (y/`u, more and more) character, as illustrated to the right.
The \zht{越} (y/`ue) pattern is the one most often used in conversation, whereas
the \zht{愈} (y/`u)  pattern is more often confined to appearances in literature.
%Examples are provided in \pref{ex:more_more} (next).
\end{minipage}\hfill%
\ftbox{\begin{tabular}{llll}
    \zhtss{越}{y/`ue}{more} &
    \charsBlank{(verb)}     &
    \zhtss{越}{y/`ue}{more} &
    \charsBlank{(verb)}
    \\
    \zhtss{愈}{y/`u}{more}  &
    \charsBlank{(verb)}     &
    \zhtss{愈}{y/`u}{more}  &
    \charsBlank{(verb)}
  \end{tabular}}

%---------------------------------------
\begin{example}[\exmd{More more}]
\mbox{}\\
\label{ex:more_more}
%---------------------------------------
\begin{enumerate}
  \item \zhtsd{他}{們}{t/-a}{men}{he}{(s)}{they}%
        \zhtsd{一}{起}{/`yi}{q/>i}{one}{to rise}{together}%
        \zhtss{盪}{d/`ang}{to swing}%
        \zhtsd{鞦}{韆}{qi/-u}{qi/-an}{swing}{swing}{swing}\zhtsC
        \zhtss{越}{y/`ue}{more}%
        \zhtss{盪}{d/`ang}{to swing}%
        \zhtss{越}{y/`ue}{more}%
        \zhtss{高}{g/-ao}{high}\zhtsP%
        \citetbl{\citerp{dream}{25}}%
        \engbox{They swang\\together; the more\\they swang, the\\higher they reached.}

  \item \zhtsd{兔}{子}{t/`u}{zi}{rabbit}{son}{rabbit}%
        \zhtss{愈}{y/`u}{more}%
        \zhtss{看}{k/`an}{to look at}%
        \zhtss{愈}{y/`u}{more}%
        \zhtss{不}{b/`u}{not}%
        \zhtsd{喜}{歡}{x/>i}{h/-uan}{happy}{joyous}{to like}%
        \zhtsd{自}{己}{/`zi}{j/>i}{self}{oneself}{one's own}%
        \zhtss{身}{sh/-en}{body}%
        \zhtss{上}{sh/`ang}{on top of}%
        \zhtss{的}{de}{('s)}%
        \zhtsd{顏}{色}{y/'an}{s/`e}{face}{color}{color}%
        \zhtsP%
        \citetbl{\citerp{coloredRabbit}{7}}%
        \\\engbox{The more the rabbit looked, the more she disliked her own coloration.}

  \item \zhtsd{公}{雞}{g/-ong}{j/-i}{male}{chicken}{rooster}%
        \zhtss{一}{y/-i}{one}\zhtss{天}{ti/-an}{day}\zhtss{比}{b/>i}{to compare}%
        \zhtss{一}{y/-i}{one}\zhtss{天}{ti/-an}{day}%
        \zhtss{老}{l/>ao}{old}\zhtsC%
        \zhtss{愈}{y/`u}{more}\zhtss{叫}{ji/`ao}{to call}
        \zhtss{愈}{y/`u}{more}%
        \zhtsd{吃}{力}{ch/-i}{l/`i}{to eat}{strength}{exhausting}\zhtsP%
        \citetbl{\citerp{cd_roosterking}{8}}%
       %\\\engbox{The rooster was day by day growing old, the more he crowed the more exhausting it became.}
        \\\engbox{Growing ever older day by day, the more Rooster crowed the more exhausting it became.}
\end{enumerate}
\end{example}


%=======================================
\section{Similes}
%=======================================
\ftbox{\zhtssH{如}{r/'u}{as}}
\hfill
\begin{minipage}{\tw-35mm}
\structe{Simile}s in English are formed using ``as" or ``like".
Examples include ``as thin as a rail", ``as flat as tin", or ``as light as air".
Similes also appear in Chinese using the character \zht{如}.
\end{minipage}

%---------------------------------------
\begin{example}[\exmd{Simile}]
\mbox{}\\
\label{ex:similes}
%---------------------------------------
        \zhtsd{一}{頭}{y/`i}{t/'ou}{one}{head}{one (mw)}
        \tcom{\zhtss{骨}{g/>u}{bone}
        \zhtss{瘦}{sh/`ou}{thin}
        \zhtss{如}{r/'u}{as}
        \zhtss{柴}{ch/'ai}{firewood}%
        \zhtss{的}{de}{('s)}%
        }{bag of bones/thin as a rail}
        \zhtsd{母}{牛}{m/>u}{ni/'u}{mother}{cow}{cow}%
        \zhtss{站}{zh/`an}{to stand}%
        \zhtss{在}{z/`ai}{at}%
        \zhtss{路}{l/`u}{road}%
        \zhtsd{中}{央}{zh/-ong}{y/-ang}{middle}{center}{middle}\zhtsP%
        \citetbl{\citerp{lovemind_water}{16}}%
        \\\engbox{A skin and bones cow, thin as a rail, stood in the center of the road.}
\end{example}

%%=======================================
%\section{Practice}
%%=======================================
%\zhtsdh{練}{習}{li/`an}{x/'i}{to practice}{to practice}{to practice}
%
%\begin{tblhhh}{Sentence structure practice}{hmw:sentence}
%           \zhtssb{我}{w/>o}{I}
%  &        \zhtssb{你}{n/>i}{you}
%  &        \zhtssb{他}{t/-a}{he}
%  \tblh \zhtssb{她}{t/-a}{she}
%  &        \zhtssb{它}{t/-a}{it}
%  &        \zhtsdb{我}{的}{w/>o}{de}{I}{('s)}{my}
%  \tblh \zhtsdb{我}{們}{w/>o}{m/'en}{I}{(plural)}{we}
%  &        \zhtssb{看}{k/`an}{see}
%  &        \zhtsdb{看}{著}{k/`an}{zhe}{see}{(adverbial particle)}{look at}
%  \tblh \zhtssb{聽}{ti/-ng}{listen}
%  &        \zhtsdb{聽}{著}{ti/-ng}{zhe}{listen}{(adverbial particle)}{listen to}
%  &        \zhtssb{聞}{w/'en}{smell}
%  \tblh \zhtssb{說}{sh/-uo}{say  }
%  &        \zhtsdb{講}{話}{ji/>ang}{h/`ua}{to speak}{talks}{talk}
%  &        \zhtsdb{討}{論}{t/>ao}{l/`un}{to demand}{to discuss}{discuss}
%  \tblh \zhtssb{坐}{z/`uo}{sit}
%  &        \zhtssb{站}{zh/`an}{stand}
%  &        \zhtssb{走}{z/>ou}{walk}
%  \tblh \zhtsdb{跑}{步}{p/>ao}{b/`u}{to run}{a step}{run}
%  &        \zhtssb{要}{y/`ao}{want}
%  &        \zhtsdb{需}{要}{x/-u}{y/`ao}{need}{want}{need}
%  \tblh \zhtsdb{喜}{歡}{x/>i}{h/-uan}{happy}{cheerful}{like}
%  &        \zhtssb{愛}{a/`i}{love}
%  &        \zhtsdb{休}{息}{xi/-u}{/'xi}{to rest}{breath}{rest}
%  \tblh \zhtsdb{睡}{覺}{sh/`ui}{ji/`ao}{to sleep}{sleep}{sleep}
%  &        \zhtssb{吃}{ch/-i}{eat}
%  &        \zhtssb{喝}{h/-e}{drink}
%  \tblh \zhtssb{爬}{p/'a}{climb}
%  &        \zhtssb{跳}{ti/`ao}{jump}
%  &        \zhtssb{停}{ti/'ng}{stop}
%  \tblh \zhtssb{去}{q/`u}{go}
%  &        \zhtssb{大}{d/`a}   {big}
%  \tblh \zhtssb{小}{x/>iao}{small}
%  &        \zhtssb{長}{ch/'ang}{long}
%  &        \zhtssb{短}{d/>uan}{short}
%  \tblh \zhtssb{高}{g/-ao}{tall}
%  &        \zhtssb{矮}{a/>i}  {short}
%  &        \zhtssb{寬}{k/-uan }{wide}
%  \tblh \zhtssb{窄}{zh/>ai}{narrow}
%  &
%  &
%\end{tblhhh}




