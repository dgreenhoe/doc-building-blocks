%============================================================================
% LaTeX File
% Daniel J. Greenhoe
%============================================================================
%=======================================
\chapter{Nonparametric Density Estimation}
\label{chp:pdfest}
%=======================================
In model-building, if a \emph{type} of parameterized
\fncte{probability density function} (\fncte{pdf}) $\ppx(x)$ \xref{def:pdf}
is known or more likely \emph{assumed}, 
then all we need to do is estimate the parameters of these functions \xref{chp:moment_estimation}.
This is called \opd{parametric density estimation}.\footnote{
  \citerpgc{gramacki2017}{2}{9783319716886}{\textsection ``1.1 Background"}
  }
However, if the density type is \emph{not} known or assumed, then we need to estimate 
the entire function.
This is called \opd{nonparametric density estimation}.

Some techniques of \ope{nonparametric density estimation} include\footnote{
  \citerg{silverman1986}{9780412246203},
  \citerppgc{tsybakov2008}{1}{27}{9780387790527}{\textsection1.1--\textsection1.3},
  \citerg{gramacki2017}{9783319716886},
  \citerppgc{vidakovic}{217}{245}{9780471293651}{Chapter 7 ``Density Estimation"}
  }
\begin{listi}
  \item Histogram %/ Smoothed Histograms
  \item Average Shifted Histogram
  \item \ope{Kernel Density Estimation} (\ope{KDE})---good for 6 dimensions or less
        (the ``curse of dimensionality"\footnotemark)
        \footnotetext{ 
        \citerp{bellman1954}{206},
        \citerpc{bellman1961}{94}{\textsection ``5.16 The Curse of Dimensionality"},
        \citerp{bellman1971}{44},
        \citerpgc{gramacki2017}{59}{9783319716886}{\textsection ``The Curse of Dimensionality"}
        }
  \item Fourier-based
  \item Wavelet-based
  \item Fast Gauss Transform
\end{listi}

One difficulty (possibly a very computationally demanding one) 
of the KDE techniques is finding a proper \vale{bandwidth}.
%\footnote{
%  \citerg{tsybakov2008}{9780387790527},
%  }

