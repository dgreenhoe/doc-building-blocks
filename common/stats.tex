%============================================================================
% LaTeX File
% Daniel J. Greenhoe
%============================================================================

%======================================
\chapter{Statistics}
\label{chp:stats}
%======================================

\qboxnpq
  {Joseph Leonard Doob (1910--2004), pioneer of and key contributor to mathematical probability\footnotemark}
  {../common/people/doobjl_dartmouthedu.jpg}
  {While writing my book I had an argument with Feller. 
   He asserted that everyone said ``random variable" and I asserted that everyone said ``chance variable." 
   We obviously had to use the same name in our books, so we decided the issue by a stochastic procedure. 
   That is, we tossed for it and he won.}
\citetblt{
  quote: & \citerp{snell1997}{307}, \citerp{snell2005}{251}\\
  %reference: & \citerpg{suhov2008}{238}{0521847672}\\
  image: & \url{http://www.dartmouth.edu/~chance/Doob/conversation.html}
  }
%=======================================
\section{Random Variables}
%=======================================
The concept of the \hie{random variable} is widely used in probability and
random processes.
Before what a random variable \emph{is}, note two things that a random variable is \emph{not} (next remark).
%---------------------------------------
\begin{remark}
\citetbl{
  \citerpg{miller2006}{130}{0471458929},
  \citerpgc{feldman2010}{4}{3642051588}{``The name ``random variable" is actually a misnomer, since it is not random and not a variable.\ldots the random variable simply maps each point (outcome) in the sample space to a number on the real line\ldots Technically, the space into which the random variable maps the sample space may be more general than the real line\ldots"},
  \citerpg{curry2011}{4}{3642166180},
  \citerpgc{trivedi2016}{2.1}{1119314208}{``The term ``random variable" is actually a misnomer, since a random variable $X$ is really a function whose domain is the sample space $S$, and whose range is the set of all real numbers, $\R$."}
  }
%---------------------------------------
As pointed out by others, the term ``random variable" is a ``misnomer":
\\\rembox{
  \begin{array}{NM}
    \imark & A random variable is {\bf not random}.\\
    \imark & A random variable is {\bf not a variable}.
  \end{array}}
\end{remark}

What is it then? It is a \structe{function} (next definition).
In particular, it is a function that maps from an underlying stochastic process into $\R$.
Any ``\prope{randomness}" (whatever that means) it may \emph{appear} to have comes from the stochastic process it 
is mapping \emph{from}. But the function itself (the random variable itself) is very deterministic and well-defined.
%
%A random variable is actually a \hie{function} $X(\omega)$, where
%$\omega\in\pso$ is an outcome of an experiment.
%More specifically, in a probability space $\ps$,
%a random variable is a mapping from the set $\pso$ of outcomes to the set of real numbers.
%What gives it the appearance of being random is that the outcome $\omega$
%of the experiment appears to be random to the observer.
%So the random variable $X(\omega)$ is simply a function of an underlying
%mechanism that appears to be random.
%---------------------------------------
\begin{definition}
\index{random variable}
\citetbl{
  \citerp{papoulis}{63}
  }
%---------------------------------------
Let $\ps$ be a \structe{probability space} \xref{def:ps}.
A \fnctd{random variable} $\rvX$ is a function $\rvX:\pso\to\R$.
\end{definition}

The probability information about $\sigma$-algebra $\pse$ is completely
specified by measure $\psp$.
However, sometimes it is more convenient to express this same measure
information in terms of the \hie{probability density function} or the
\hie{cummulative distribution function} of the probability space.
%---------------------------------------
\begin{definition}
\label{def:pdf}
\label{def:cdf}
\index{probability density function}
\index{pdf}
\index{cummulative distribution function }
\index{cdf}
%---------------------------------------
Let $\ps$ be a probability space and
$\rvX:\pso\to\R$ a random variable. Then
\begin{liste}
  \item a {\bf probability density function} (pdf)      $\ppx:\pse\to[0,1]$ and
   \item a {\bf cummulative distribution function} (cdf) $\pcx:\pse\to[0,1]$
\end{liste}
are defined as
\defbox{\begin{array}{rc>{\ds}lD}
  \ppx(x) &\eqd& \lim_{\epsilon\to 0} \frac{1}{\epsilon}\psp\{x \le X < x+\epsilon\}
          & (probability density function---pdf)
          \\
  \pcx(x) &\eqd& \psp\{X < x \}
          & (cummulative distribution function---cdf)
\end{array}}
\end{definition}

\prefpp{def:pdf} defines the pdf and cdf of a probability space
$\ps$ in terms of measure $\psp$.
Conversely, the probability measure $\psp\{a\le X<b\}$
of an event $\{a\le X<b\}$ can be
expressed in terms of either the pdf or cdf.
%---------------------------------------
\begin{theorem}
%---------------------------------------
Let $\ps$ be a probability space,
$\rvX:\pso\to\R$ be a random variable, and $(a,b)$ a real interval.
Then
\thmbox{\begin{array}{rc>{\ds}l}
  \psp\{a\le X<b\}
    &=& \int_a^b \ppx(x) \dx
  \\&=& \int_{-\infty}^b \pcx(x) \dx - \int_{-\infty}^a \pcx(x) \dx
\end{array}}
\end{theorem}

The properties of the pdf follow closely the properties of measure $\psp$.
%---------------------------------------
\begin{theorem}
%---------------------------------------
\thmbox{\begin{array}{rc>{\ds}l}
  \pp(x|y) &=& \frac{\pp(x,y)}{\pp(y)}  \\
  \ppx(x)  &=& \int_y\ppxy(x,y) \dy
\end{array}}
\end{theorem}
\begin{proof}
\begin{eqnarray*}
  \ppx[x|y](x|y)
    &\eqd& \lim_{h\to0} \frac{1}{h} \pP{x\le X < x+h | Y=y}
  \\&\eqd& \lim_{h\to0} \frac{1}{h} \frac{\pP{(x\le X < x+h) \land (Y=y)}}{\pP{Y=y}}
  \\&=&    \lim_{h\to0} \frac{1}{h} \frac{\pP{(x\le X < x+h) \land (y\le Y<y+h)}}{\pP{y\le Y<y+h}}
  \\&\eqd& \frac{\ppxy(x,y)}{\ppy(y)}
\\
\\
  \int_y\ppxy(x,y)\dy
    &\eqd& \lim_{h\to0} \frac{1}{h}
           \int_y \pP{x\le X<x+h, y\le Y<y+h} \dy
  \\&=&    \lim_{h\to0} \frac{1}{h} \pP{x\le X<x+h}
  \\&\eqd& \ppx(x)
\end{eqnarray*}
\end{proof}
%=======================================
\section{Expectation operator}
\index{expectation operator}
%=======================================
In a probability space $\ps$, all proability information
is contained in the measure $\psp$ (or equivalently in the pdf or cdf
defined in terms of $\psp$).
Often times this information is overwhelming and a simpler statistic,
which does not offer so much information, is sufficient.
Some of the most common statistics can be conveniently expressed in terms
of the \hie{expectation operator} $\pE$.
%---------------------------------------
\begin{definition}
%---------------------------------------
Let $\ps$ be a probability space and
$\rvX:\pso\to\R$ a random variable with
probability density function $\ppx:\pse\to\R$.
Then the {\bf expectation operator} on $\rvX$ is
\defbox{ \pEx X \eqd \int_x x \ppx(x) \dx. }
\end{definition}

We already said that a random variable $\rvX$ is neither random nor a variable,
but is rather a function of an underlying process that does appear to be random.
However, because it is a function of a process that does appear random,
the random variable $\rvX$ also appears to be random.
That is, if we don't know the outcome of of the underlying experimental
process, then we also don't know for sure what $\rvX$ is, and so $\rvX$ does
indeed appear to be random.
However, eventhough $\rvX$ appears to be random,
the expected value $\pEx X$  of $\rvX$ is {\bf not random}.
Rather it is a fixed value (like $0$ or $7.9$ or $-2.6$).

On the other hand, eventhough $\pE X$ is {\bf not random},
note that $\pE(X|Y)$ {\bf is random}.
This is because $\pE(X|Y)$ is a function of $Y$.
That is, once we know that $Y$ equals some fixed value $y$
(like $0$ or $2.7$ or $-5.1$) then $\pE(X|Y=y)$ is also fixed.
However, if we don't know the value of $Y$,
then $Y$ is still a random variable and the expression $\pE(X|Y)$
is also random (a function of random variable $Y$).

Two common statistics that are conveniently expressed in terms of the
expectation operator are the \hie{mean} and \hie{variance}.
The mean is an indicator of the ``middle" of a probability distribution and the
variance is an indicator of the ``spread".
%---------------------------------------
\begin{definition}
\label{def:Mx}
\index{mean}
\index{variance}
%---------------------------------------
Let $\ps$ be a probability space and $\rvX:\pso\to\R$ a random variable.
The {\bf mean} $\pmeanx$ and {\bf variance} $\pVar(X)$ of $\rvX$ are
\defbox{\begin{array}{rcl}
  \pmeanx  &\eqd& \pEx X \\
  \pVar(X) &\eqd& \pEx\left[(X-\pEx X)^2 \right]
\end{array}}
\end{definition}

The next theorem gives some useful relations for simple statistics.
%---------------------------------------
\begin{theorem}
%---------------------------------------
Let $\ps$ be a probability space with random variable $\rvX:\pso\to\R$
and let $a\in\R$.
\thmbox{\begin{array}{rcl}
  \pEx(aX)  &=& a\pEx X     \\
  \pVar(aX) &=& a^2\pVar(X) \\
  \pVar(X)  &=& \pEx X^2 - (\pEx X)^2
\end{array}}
\end{theorem}
\begin{proof}
\begin{eqnarray*}
  \pEx(aX)
    &\eqd& \int_x ax \ppx(x)  \dx
  \\&=&    a \int_x x \ppx(x)  \dx
  \\&\eqd& a\pEx X
\\
\\
  \pVar(X)
    &\eqd& \pEx\left[(X-\pEx X)^2\right]
  \\&=&    \pEx\left[X^2-2X\pEx X + (\pEx X)^2 \right]
  \\&=&    \pEx[X^2]  - \pEx 2X\pEx X  + \pEx (\pEx X)^2
  \\&=&    \pEx[X^2] - 2(\pEx X)[\pEx X] + (\pEx X)^2
  \\&=&    \pEx X^2  - (\pEx X)^2
\\
\\
  \pVar(aX)
    &=&    \pEx(aX)^2  - [\pEx(aX)]^2
  \\&=&    \pEx(a^2X^2)  - [a\pEx X]^2
  \\&=&    a^2 \pEx X^2  - a^2[\pEx X]^2
  \\&=&    a^2\left[ \pEx X^2  - (\pEx X)^2 \right]
  \\&\eqd& a^2 \pVar(X)
\end{eqnarray*}
\end{proof}


\begin{figure}[ht]
\color{figcolor}
\setlength{\unitlength}{0.3mm}
\begin{center}
\begin{picture}(200,130)(-50,-30)
  \put(  0,  0){\line(1, 0){120}}
  \put(  0,  0){\line(0, 1){120}}
  \put( 20, 80){\circle*{5}}
  \put(100, 40){\circle*{5}}
  \put( 20, 80){\line(2,-1){80}}
  \put(125,  0){\makebox(0,0)[l]{$x$}}
  \put( -5, 80){\makebox(0,0)[r]{$\ff(b)$}}
  \put( -5, 40){\makebox(0,0)[r]{$\ff(a)$}}
  \put( 20, -5){\makebox(0,0)[t]{$a$}}
  \put(100, -5){\makebox(0,0)[t]{$b$}}
  \qbezier(20,80)(70,120)(100,40)
  \qbezier(20,80)(30,0)(100,40)
  \qbezier[28](20,0)(20,40)(20,80)
  \qbezier[14](100,0)(100,20)(100,40)
  \qbezier[7](0,80)(10,80)(20,80)
  \qbezier[32](0,40)(50,40)(100,40)
  \put(120,90){\vector(-1,0){50}}
  \put(120,60){\vector(-1,0){50}}
  \put(120,30){\vector(-1,0){30}}
  \put(125,90){\makebox(0,0)[l]{concave}}
  \put(125,60){\makebox(0,0)[l]{chord $\fy(x)$}}
  \put(125,30){\makebox(0,0)[l]{convex}}
\end{picture}
\end{center}
\caption{
  Convex and concave functions
  \label{fig:convex}
  }
\end{figure}

Often a function can be proven to be \hie{convex} or \hie{concave}.
\hie{Convex} and \hie{concave} functions are
defined in \prefpp{def:convex} (next) and
illustrated in \prefpp{fig:convex}.
%---------------------------------------
\begin{definition}
\label{def:convex}
%---------------------------------------
Let
  \[ \fy(x) = \frac{\ff(b)-\ff(a)}{b-a} (x-a) + \ff(a). \]
A function $\ff:\R\to\R$ is

\defbox{
  \begin{array}{llrcll}
    \mbox{\bf convex}           \mbox{ in } (a,b) & \mbox{if} & \ff(x) &\le& \fy(x) & \mbox{for } x\in(a,b) \\
    \mbox{\bf concave}          \mbox{ in } (a,b) & \mbox{if} & \ff(x) &\ge& \fy(x) & \mbox{for } x\in(a,b) \\
    \mbox{\bf strictly convex}  \mbox{ in } (a,b) & \mbox{if} & \ff(x) &<  & \fy(x) & \mbox{for } x\in(a,b) \\
    \mbox{\bf strictly concave} \mbox{ in } (a,b) & \mbox{if} & \ff(x) &>  & \fy(x) & \mbox{for } x\in(a,b)
  \end{array}
  }
\end{definition}

The next theorem gives another form of convex functions that is a little
less intuitive but provides powerful analytic results.
%---------------------------------------
\begin{theorem}
\label{thm:convex_lambda}
%---------------------------------------
Let $\ff:\R\to\R$.
For every $x_1,x_2\in(a,b)$ and $\lambda\in[0,1]$
\formbox{
  \ff \mbox{ is convex in $(a,b)$ } \iff
        \ff(\lambda x_1+(1-\lambda)x_2) \le \lambda\ff(x_1) + (1-\lambda)\ff(x_2)
  }
\end{theorem}
\begin{proof}
\begin{enumerate}
  \item prove $\ff$ is convex $\implies$
        $\ff(\lambda x_1+(1-\lambda)x_2) \le \lambda\ff(x_1) + (1-\lambda)\ff(x_2)$:
  \begin{eqnarray*}
    \ff(\lambda x_1+(1-\lambda)x_2)
      &\le& \frac{\ff(b)-\ff(a)}{b-a}[\lambda x_1+(1-\lambda)x_2-a] + \ff(a)
    \\&=&   \frac{\ff(x_2)-\ff(x_1)}{x_2-x_1}[\lambda x_1+(1-\lambda)x_2-x_1] + \ff(x_1)
    \\&=&   \frac{\ff(x_2)-\ff(x_1)}{x_2-x_1}[(x_2-x_1)(1-\lambda)] + \ff(x_1)
    \\&=&   (1-\lambda)\ff(x_2)-(1-\lambda)\ff(x_1) + \ff(x_1)
    \\&=&   \lambda\ff(x_1) + (1-\lambda)\ff(x_2)
  \end{eqnarray*}
  \item prove $\ff$ is convex $\impliedby$
        $f(\lambda x_1+(1-\lambda)x_2) \le \lambda\ff(x_1) + (1-\lambda)\ff(x_2)$:

  Let $x=\lambda(b-a)+a$ Notice that as $\lambda$ varies from $0$ to $1$,
      $x$ varies from $b$ to $a$.
      So free variable $\lambda$ works as a change of variable for
      free variable $x$.
  \begin{eqnarray*}
    \lambda &=& \frac{x-a}{b-a} \\
\\
    \ff(x)
      &=&   \ff(\lambda(b-a)+a)
    \\&\le& \lambda\ff(b) + (1-\lambda)\ff(a)
    \\&=&   \lambda[\ff(b)-\ff(a)] + \ff(a)
    \\&=&   \frac{\ff(b)-\ff(a)}{b-a}(x-a) + \ff(a)
  \end{eqnarray*}
\end{enumerate}
\end{proof}

Taking the second derivative of a function provides a convenient test
for whether that function is convex.
%---------------------------------------
\begin{theorem}
\label{thm:convex_d2}
\citepp{cover}{24}{25}
%---------------------------------------
\formbox{
  \ff''(x)>0 \implies \ff \mbox{ is convex}
  }
\end{theorem}
\begin{proof}
\begin{eqnarray*}
  \ff(x)
    &=&   \ff(x_0) + \ff'(x_0)(x-x_0) + \ff''(c)(x-x_0)^2
  \\&\ge& \ff(x_0) + \ff'(x_0)(x-x_0)
  \\&=&   \ff(x_0) + \ff'(x_0)(x-\lambda x_1 - (1-\lambda)x_2)
\\
\\
  \ff(x_1)
    &\ge& \ff(x_0) + \ff'(x_0)(x_1-\lambda x_1 - (1-\lambda)x_2)
  \\&=&   \ff(x_0) + \ff'(x_0)(1-\lambda)(x_1-x_2)
  \\&=&   \ff(x_0) - \ff'(x_0)(1-\lambda)(x_2-x_1)
\\
\\
  \ff(x_2)
    &\ge& \ff(x_0) + \ff'(x_0)(x_2-\lambda x_1 - (1-\lambda)x_2)
  \\&=&   \ff(x_0) + \ff'(x_0)\lambda (x_2-x_1)
\\
\\
  \lambda\ff(x_1) + (1-\lambda)\ff(x_2)
    &\ge& \lambda\left[\ff(x_0) - \ff'(x_0)(1-\lambda)(x_2-x_1)\right] +
          (1-\lambda)\left[\ff(x_0) + \ff'(x_0)\lambda (x_2-x_1)\right]
  \\&=&   \lambda\left[\ff(x_0) - \ff'(x_0)(1-\lambda)(x_2-x_1)\right]
          +\left[\ff(x_0) + \ff'(x_0)\lambda (x_2-x_1)\right]
          -\lambda\left[\ff(x_0) + \ff'(x_0)\lambda (x_2-x_1)\right]
  \\&=&  \ff(x_0)
  \\&=&  \ff(\lambda x_1 + (1-\lambda)x_2)
\end{eqnarray*}

By \prefpp{thm:convex_lambda}, $\ff(x)$ is convex.
\end{proof}

\begin{figure}[ht]
\setlength{\unitlength}{0.3mm}%
\begin{center}%
\begin{picture}(200,110)(-50,-10)%
  \color{axis}%
    \put(  0,  0){\line(1, 0){120}}%
    \put(  0,  0){\line(0, 1){100}}%
    \qbezier[20](0,60)(30,60)(60,60)%
    \qbezier[20](60,0)(60,30)(60,60)%
  \color{blue}%
    \qbezier(33,100)(85,0)(110,100)%
    \put( 60, 60){\circle*{5}}%
  \color{red}%
    \put( 20,100){\line(1,-1){80}}%
  \color{label}%
  \put(125,  0){\makebox(0,0)[l]{$x$}}%
  \put( -5, 60){\makebox(0,0)[r]{$\ff(\pE X)$}}%
  \put( 60, -5){\makebox(0,0)[t]{$\pE X$}}%
  \put(130,60){\vector(-1,0){32}}%
  \put(130,40){\vector(-1,0){47}}%
  \put(135,60){\makebox(0,0)[l]{$\ff(x)$ (convex function)}}%
  \put(135,40){\makebox(0,0)[l]{$mx+c$ (support line)}}%
\end{picture}
\end{center}
\caption{
  Jensen's inequality
  \label{fig:jensen}
  }
\end{figure}

And now for an extremely useful application of convexity to the
expectation operator: \hie{Jensen's inequality}.
Jensen's inequality is stated in \prefpp{thm:jensen}
and illustrated in \prefpp{fig:jensen}.
%--------------------------------------
\begin{theorem}[Jensen's inequality]
\citep{cover}{25}
\citepp{jensen1906}{179}{180}
\index{Jensen's inequality}
\index{theorems!Jensen's inequality}
\label{thm:jensen}
%--------------------------------------
Let $\ff$ be a convex function and $\rvX$ be a random variable. Then
\formbox{
  \ff \mbox{ is convex} \implies \ff(\pE X) \le \pE\ff(X).
  }
\end{theorem}
\begin{proof}
Let $mx+c$ be a ``support line" under $\ff(x)$ such that
\[
  \begin{array}{rcll}
    mx+c &<& \ff(x) & \mbox{for } x\ne \pE X \\
    mx+c &=& \ff(x) & \mbox{for } x=\pE X.
  \end{array}
\]
Then
\begin{eqnarray*}
  \ff(\pE X)
    &=&   m[\pE X] + c
  \\&=&   \pE[mX + c]
  \\&\le& \pE\ff(X)
\end{eqnarray*}
\end{proof}


%=======================================
\section{Upper bounds on probability}
%=======================================
%---------------------------------------
\begin{theorem}[Markov's inequality]
\index{Markov's inequality}
\index{theorems!Markov's inequality}
\citetbl{
  \citerp{ross}{395}
  }
%---------------------------------------
Let $\rvX:\Omega\to[0,\infty)$ be a non-negative valued random variable and
$a\in(0,\infty)$. Then
\formbox{ \pP{X\ge a} \le \frac{1}{a} \pE X }
\end{theorem}
\begin{proof}
\begin{eqnarray*}
  I &\eqd& \left\{ \begin{array}{l@{\hspace{4ex}\mbox{for}\hspace{4ex}}l}
    1 & X\ge a \\
    0 & X < a
    \end{array}\right.
\\
  aI &\le& X           \\
   I &\le& \frac{1}{a} X \\
   \pE I &\le& \pE\left(\frac{1}{a} X\right) \\
\\
   \pP{X\ge a}
     &=& 1\cdot\pP{X\ge a} + 0\cdot\pP{X<a}
   \\&=& \pE I
   \\&\le& \pE\left(\frac{1}{a} X \right)
   \\&=&   \frac{1}{a}\pE X
\end{eqnarray*}
\end{proof}


%---------------------------------------
\begin{theorem}[Chebyshev's inequality]
\index{Chebyshev's inequality}
\index{theorems!Chebyshev's inequality}
\citetbl{
  \citerp{ross}{396}
  }
%---------------------------------------
Let $\rvX$ be a random variable with mean $\mu$ and variance $\sigma^2$.
\thmbox{ \pP{|X-\mu|\ge a} \le \frac{\sigma^2}{a^2}}
\end{theorem}
\begin{proof}
\begin{eqnarray*}
  \pP{ |X-\mu| \ge a}
    &=&   \pP{ (X-\mu)^2 \ge a^2}
  \\&\le& \frac{1}{a^2} \pE(X-\mu)^2 \hspace{4ex}\mbox{by Markov's inequality}
  \\&=&   \frac{\sigma^2}{a^2}
\end{eqnarray*}
\end{proof}




%=======================================
\section{Functions of one random variable}
%=======================================
\begin{figure}\color{figcolor}
\setlength{\unitlength}{0.4mm}
\begin{center}
\begin{footnotesize}
\begin{picture}(250,150)(-100,-20)
  \put(-100,   0){\line(1,0){200}}
  \put(   0, -20){\line(0,1){120}}
  {\color{red}
    \qbezier(-100,100)(0,-100)(100,100)
    \put( 100, 105){\makebox(0,0)[b]{$y=\ff(x)$}}
    }
  \qbezier[8](-40,0)(-40,8)(-40,16)
  \qbezier[8](40,0)(40,8)(40,16)
  \qbezier[28](-80,0)(-80,32)(-80,64)
  \qbezier[28](80,0)(80,32)(80,64)
  \qbezier[64](-80,64)(0,64)(80,64)
  \qbezier[40](-40,16)(0,16)(40,16)
  \put(   0, 110){\makebox(0,0)[r]{$y$}}
  \put( 110,   0){\makebox(0,0)[r]{$x$}}
  \put(  -5,  64){\makebox(0,0)[r]{$y+h$}}
  \put(  -5,  16){\makebox(0,0)[r]{$y$}}
  \put( -40,  -5){\makebox(0,0)[t]{$x_1=\ffi{y}$}}
  \put(  40,  -5){\makebox(0,0)[t]{$x_2=\ffi{y}$}}
  \put( -80,  -5){\makebox(0,0)[rt]{$\ffi{y+h}$}}
  \put(  80,  -5){\makebox(0,0)[lt]{$\ffi{y+h}$}}
  \put(-100,  40){\makebox(0,0)[r]{$\left.\frac{\dy}{\dx}\right|_{x=x_1}$}}
  \put( 100,  40){\makebox(0,0)[l]{$\left.\frac{\dy}{\dx}\right|_{x=x_2}$}}
  \put(  95,  40){\vector(-1,0){35}}
  \put(-100,  20){\makebox(0,0)[r]{$x_1 + \frac{1}{\ffp{x_1}}h$}}
  \put( 100,  20){\makebox(0,0)[l]{$x_2 + \Delta y\left.\frac{1}{\dy/\dx}\right|_{x=x_2} = x_1 + \frac{1}{\ffp{x_2}}h$}}
  \put( -95,  15){\vector( 1,-1){15}}
  \put(  95,  15){\vector(-1,-1){15}}
  \put(  40,  16){\line(5,6){40}}   %straight line
  \put( -40,  16){\line(-5,6){40}}   %straight line
  %{\color{green}\qbezier(40,16)(60,40)(80,64)}     %straight line
\end{picture}
\end{footnotesize}
\end{center}
\caption{
  $Y=\ff(X)$
  \label{fig:Y=f(X)}
  }
\end{figure}

%---------------------------------------
\begin{theorem}
\label{thm:Y=f(X)}
\citetbl{
  \citerp{papoulis}{93}, 
  \citerp{proakis}{30}
  }
%---------------------------------------
Let $Y=\ff(X)$ and $\set{x_n}{n=1,2,\ldots,N}$ be the roots of the equation
$y=\ff(x)$.
\formbox{
  \ppy(y) = \sum_{n=1}^N \frac{\ppx(x_n)}{|\ff'(x_n)|}
}
\end{theorem}
\begin{proof}
Let the range of $\rvX$ be partitioned into $N$ partitions
$\set{A_n}{x_n\in A_n, n=1,2,\ldots,N}$ and $h\to0$.
\begin{eqnarray*}
%  \ffi{y}
%    &=&    \sum_{n=0}^\infty \ffi^{(n)}{y_0}(y-y_0)
%  \\&\eqa& \ffi{y_0} + \ffi^{(1)}{y_0}(y-y_0)
%  \\&=&    \ffi{y_0} + \ffi^{(1)}{y_0}(y-y_0)
  \ppy(y)h
    &=& \pP{y \le Y < y+h}
  \\&=& \pP{y \le \ff(X) < y+h}
  \\&=& \sum_{n=1}^N \pPa{y \le \ff(X) < y+h}{X\in A_n}
  \\&=& \sum_{n=1}^N \pPc{\ffi{y} \le X < \ffi{y+h}}{X\in A_n}\pP{X\in A_n}
  \\&=& \left\{\begin{array}{l@{\hspace{4ex}\mbox{for}\hspace{4ex}}l}
          \sum_{n=1}^N \pPc{x_n     \le X < x_n + \frac{1}{\ffp{x_n}}h}{X\in A_n}\pP{X\in A_n} & \ffp{x_n}<0 \\
          \sum_{n=1}^N \pPc{x_n + \frac{1}{\ffp{x_n}}h \le X < x_n}{X\in A_n}\pP{X\in A_n}     & \ffp{x_n}\ge0
        \end{array}\right.
  \\&=& \sum_{n=1}^N \pPa{x_n     \le X < x_n + \frac{1}{|\ffp{x_n}|}h}{X\in A_n}
  \\&=& \sum_{n=1}^N \pP{x_n     \le X < x_n + \frac{1}{|\ffp{x_n}|}h}
  \\&=& \sum_{n=1}^N h \frac{1}{|\ffp{x_n}|} \ppx(x_n)
  \\&=& h \sum_{n=1}^N \frac{\ppx(x_n)}{|\ffp{x_n}|}
\end{eqnarray*}
\end{proof}

%---------------------------------------
\begin{proposition}
\citetbl{
  \citerp{papoulis}{95},
  \citerp{proakis}{29}
  }
%---------------------------------------
Let $a,b\in\R, a\ne 0$ and $Y=aX+b$. Then
\formbox{
  \ppy(y) =
    \frac{1}{|a|} \ppx\left(\frac{y-b}{a}\right)
  }
\end{proposition}
\begin{proof}
\begin{eqnarray*}
  \ppy(y)h
    &=&  \pP{y\le Y < y+h}
  \\&=&  \pP{y\le aX+b < y+h}
  \\&=&  \pP{y-b\le aX < y-b+h}
  \\&=&  \left\{\begin{array}{ll}
           \pP{\frac{y-b}{a}\le X < \frac{y-b}{a}+\frac{1}{a}h} &: a>0 \\
           \pP{\frac{y-b}{a}\ge X > \frac{y-b}{a}+\frac{1}{a}h} &: a<0
         \end{array}\right.
  \\&=&  \left\{\begin{array}{ll}
           \pP{\frac{y-b}{a}\le X < \frac{y-b}{a}+\frac{1}{|a|}h} &: a>0 \\
           \pP{\frac{y-b}{a}-\frac{1}{|a|}h < X \le \frac{y-b}{a}} &: a<0
         \end{array}\right.
  \\&=&  \frac{1}{|a|}h \ppx\left(\frac{y-b}{a}\right)
\\\implies
\\
  \ppy(y)
    &=&  \frac{1}{|a|} \ppx\left(\frac{y-b}{a}\right)
\end{eqnarray*}

The theorem can also be proved using \prefpp{thm:Y=f(X)}.
The only root of $y=ax+b$ is $x_1=\frac{y-b}{a}$.
\begin{eqnarray*}
  \ppy(y)
    &=& \sum_{n=1}^N \frac{\ppx(x_n)}{|\ff'(x_n)|}
  \\&=& \frac{\ppx(x_1)}{|\ff'(x_1)|}
  \\&=& \frac{\ppx(x_1)}{|a|}
  \\&=& \frac{1}{|a|}\ppx\left(\frac{y-b}{a}\right)
\end{eqnarray*}
\end{proof}





\begin{figure}\color{figcolor}
\setlength{\unitlength}{0.3mm}
\begin{center}
\begin{footnotesize}
\begin{picture}(350,200)(-100,-20)
  \put( -20,   0){\line(1,0){220}}
  \put(   0, -20){\line(0,1){120}}
  \put(   0, 110){\makebox(0,0)[r]{$y$}}
  \put( 210,   0){\makebox(0,0)[r]{$x$}}
  {\color{red}
%    \qbezier(10,140)(60,10)(160,10)
    \qbezier(20,150)(30,30)(150,20)
    \put( 35, 105){\makebox(0,0)[l]{$y=\frac{1}{x}$}}
    }
  \put(40,80){\line(1,-1){40}} %straight line
  \qbezier[28](0,80)(20,80)(40,80)
  \qbezier[50](40,0)(40,40)(40,80)
  \qbezier[40](0,40)(40,40)(80,40)
  \qbezier[20](80,0)(80,20)(80,40)
  \put(  -5,  80){\makebox(0,0)[r]{$y+h$}}
  \put(  -5,  40){\makebox(0,0)[r]{$y$}}
  \put(  40,  -5){\makebox(0,0)[t]{$\frac{1}{y+h}$}}
  \put(  80,  -5){\makebox(0,0)[t]{$\frac{1}{y}$}}
  \put( 100,  60){\makebox(0,0)[l]{$\frac{\Delta y}{\Delta x}=\left.\frac{\dy}{\dx}\right|_{x=\frac{1}{y}}=-y^2$}}
  \put(  95,  60){\vector(-1,0){35}}
  \put( 100,  40){\makebox(0,0)[l]{$\frac{1}{y}+\frac{\Delta x}{\Delta y}\Delta y = \frac{1}{y} - \frac{1}{y^2}h$}}
  \put(  95,  40){\vector(-3,-2){55}}
\end{picture}
\end{footnotesize}
\end{center}
\caption{
  $Y=\frac{1}{X}$
  \label{fig:Y=1/X}
  }
\end{figure}
%---------------------------------------
\begin{proposition}
\citetbl{
  \citerp{papoulis}{94}
  }
%---------------------------------------
Let $Y=\frac{1}{X}$. Then
\formbox{
  \ppy(y) = \left\{\begin{array}{lM}
     %                                             & for $y<0$ \\
     %0                                            & for $y=0$ \\
     \frac{1}{y^2} \ppx\left( \frac{1}{y} \right) & for $y>0$
  \end{array}\right.
  }
\end{proposition}
\begin{proof}
Let $h\to0$.
First we show a useful relation for $\frac{1}{y+h}$.
This relation is illustrated in \prefpp{fig:Y=1/X}.
\begin{eqnarray*}
  \frac{1}{y+h}
    &=&    y_1 + \frac{1}{m} \Delta y
  \\&=&    \frac{1}{y} + \left.\frac{1}{dy/dx}\right|_{x=1/y} h
  \\&=&    \frac{1}{y} - \left.x^2\right|_{x=1/y} h
  \\&=&    \frac{1}{y} - \frac{1}{y^2} h
\end{eqnarray*}

Now we prove the theorem using the above relation.
\begin{eqnarray*}
  \ppy(y)h
    &=&    \pP{y\le Y < y+h}
  \\&=&    \pP{y\le \frac{1}{X} < y+h}
  \\&=&    \pP{\frac{1}{y}\ge X > \frac{1}{y+h}}
  \\&=&    \pP{\frac{1}{y}\ge X > \frac{1}{y}-\frac{1}{y^2}h}
  \\&=&    \pP{\frac{1}{y}-\frac{1}{y^2}h <  X \le \frac{1}{y} }
  \\&=&    \frac{1}{y^2}h \ppx\left( \frac{1}{y} \right)
\\\implies
\\
  \ppy(y)
    &=&    \frac{1}{y^2} \ppx\left( \frac{1}{y} \right)
\end{eqnarray*}

The theorem can also be proved using \prefpp{thm:Y=f(X)}.
\begin{eqnarray*}
  x_1 &=& \frac{1}{y} \\
  \ff'(x) &=& -\frac{1}{x^2}    \\
  \ppy(y)
    &=& \sum_{n=1}^N \frac{\ppx(x_n)}{|\ff'(x_n)|}
  \\&=& \frac{\ppx(x_1)}{|\ff'(x_1)|}
  \\&=& \frac{\ppx(1/y)}{|\ff'(1/y)|}
  \\&=& \frac{\ppx(1/y)}{y^2}
  \\&=& \frac{1}{y^2} \ppx\left( \frac{1}{y} \right)
\end{eqnarray*}
\end{proof}





\begin{figure}\color{figcolor}
\setlength{\unitlength}{0.3mm}
\begin{center}
\begin{footnotesize}
\begin{picture}(250,150)(-100,-20)
  \put(-100,   0){\line(1,0){200}}
  \put(   0, -20){\line(0,1){120}}
  {\color{red}
    \qbezier(-100,100)(0,-100)(100,100)
    \put( 100, 105){\makebox(0,0)[b]{$y=x^2$}}
    }
  \qbezier[8](-40,0)(-40,8)(-40,16)
  \qbezier[8](40,0)(40,8)(40,16)
  \qbezier[28](-80,0)(-80,32)(-80,64)
  \qbezier[28](80,0)(80,32)(80,64)
  \qbezier[64](-80,64)(0,64)(80,64)
  \qbezier[40](-40,16)(0,16)(40,16)
  \put(   0, 110){\makebox(0,0)[r]{$y$}}
  \put( 110,   0){\makebox(0,0)[r]{$x$}}
  \put(  -5,  64){\makebox(0,0)[r]{$y+h$}}
  \put(  -5,  16){\makebox(0,0)[r]{$y$}}
  \put( -40,  -5){\makebox(0,0)[t]{$-\sqrt{y}$}}
  \put(  40,  -5){\makebox(0,0)[t]{$\sqrt{y}$}}
  \put( -80,  -5){\makebox(0,0)[t]{$-\sqrt{y+h}$}}
  \put(  80,  -5){\makebox(0,0)[t]{$\sqrt{y+h}$}}
% \put(  80,  -5){\makebox(0,0)[t]{$\sqrt{y+h}\approx \sqrt{y} + \frac{1}{2\sqrt{y}}h$}}
  \put( 100,  40){\makebox(0,0)[l]{$\frac{\Delta y}{\Delta x}=\left.\frac{\dy}{\dx}\right|_{x=\sqrt{y}}=2\sqrt{y}$}}
  \put(  95,  40){\vector(-1,0){35}}
  \put(-100,  20){\makebox(0,0)[r]{$\sqrt{y}+\frac{\Delta x}{\Delta y}\Delta y = \sqrt{y} - \frac{1}{2\sqrt{y}}h$}}
  \put( 100,  20){\makebox(0,0)[l]{$\sqrt{y}+\frac{\Delta x}{\Delta y}\Delta y = \sqrt{y} + \frac{1}{2\sqrt{y}}h$}}
  \put( -95,  15){\vector( 1,-1){15}}
  \put(  95,  15){\vector(-1,-1){15}}
  \put(  40,  16){\line(5,6){40}}   %straight line
  \put( -40,  16){\line(-5,6){40}}   %straight line
  %{\color{green}\qbezier(40,16)(60,40)(80,64)}     %straight line
\end{picture}
\end{footnotesize}
\end{center}
\caption{
  $Y=X^2$
  \label{fig:Y=X^2}
  }
\end{figure}
%---------------------------------------
\begin{proposition}
\citetbl{
  \citerp{papoulis}{95},
  \citerp{proakis}{29}
  }
\label{prop:Y=X^2}
%---------------------------------------
Let $Y=X^2$. Then
\propbox{
  \ppy(y) = \brbl{\begin{array}{lM}
     0                  & for $a<0$ \\
     \text{undefined}   & for $a=0$ \\
     \left.\left.\frac{1}{2\sqrt{y}} \right[\ppx(-\sqrt{y}) + \ppx( \sqrt{y}) \right]
                        & for $a>0$ 
  \end{array}}
  }
\end{proposition}
\begin{proof}
Let $h\to0$.
First we show a useful relation for $\sqrt{y+h}$.
This relation is illustrated in \prefpp{fig:Y=X^2}.
\begin{eqnarray*}
  \sqrt{y+h}
    &=&    y_1 + \frac{1}{m} \Delta y
  \\&=&    \sqrt{y} + \left.\frac{1}{dy/dx}\right|_{x=\sqrt{y}} h
  \\&=&    \sqrt{y} + \left.\frac{1}{2x}\right|_{x=\sqrt{y}} h
  \\&=&    \sqrt{y} + \frac{1}{2\sqrt{y}} h
\end{eqnarray*}

Now we prove the theorem using the above relation.
\begin{eqnarray*}
  \ppy(y)h
    &=&    \pP{y\le Y < y+h}
  \\&=&    \pP{y\le X^2 < y+h}
  \\&=&    \pP{(y\le X^2 < y+h) \land (X<0)}+ \pP{(y\le X^2 < y+h) \land (X\ge0)}
  \\&=&    \pP{y\le X^2 < y+h | X<0}\pP{X<0} + \pP{y\le X^2 < y+h | X\ge0}\pP{X\ge0}
  \\&=&    \pP{-\sqrt{y}\le X < -\sqrt{y+h} | X<0  } \pP{X<0} +
           \pP{+\sqrt{y}\le X < +\sqrt{y+h} | X\ge0} \pP{X\ge0}
  \\&=&    \pP{-\sqrt{y}\le X < -\left(\sqrt{y}+\frac{1}{2\sqrt{y}}h\right) \land X<0   } +
           \pP{+\sqrt{y}\le X <        \sqrt{y}+\frac{1}{2\sqrt{y}}h        \land X\ge0 }
  \\&=&    \pP{-\sqrt{y}\le X < -\left(\sqrt{y}+\frac{1}{2\sqrt{y}}h\right)  } +
           \pP{+\sqrt{y}\le X <        \sqrt{y}+\frac{1}{2\sqrt{y}}h         }
  \\&=&    \frac{1}{2\sqrt{y}}h\ppx(-\sqrt{y})  +
           \frac{1}{2\sqrt{y}}h\ppx( \sqrt{y})
\\\implies
  \ppy(y)
    &=&    \left.\left.\frac{1}{2\sqrt{y}}\right[
           \ppx(-\sqrt{y}) + \ppx( \sqrt{y}) \right]
\end{eqnarray*}

The theorem can also be proved using \prefpp{thm:Y=f(X)}.
\begin{eqnarray*}
  x_1 &=& -\sqrt{y} \\
  x_2 &=& +\sqrt{y} \\
  \ff'(x) &=& 2x    \\
  \ppy(y)
    &=& \sum_{n=1}^N \frac{\ppx(x_n)}{|\ff'(x_n)|}
  \\&=& \frac{\ppx(x_1)}{|\ff'(x_1)|} + \frac{\ppx(x_2)}{|\ff'(x_2)|}
  \\&=& \frac{\ppx(-\sqrt{y})}{|\ff'(-\sqrt{y})|} + \frac{\ppx(\sqrt{y})}{|\ff'(\sqrt{y})|}
  \\&=& \frac{\ppx(-\sqrt{y})}{2\sqrt{y}} + \frac{\ppx(\sqrt{y})}{2\sqrt{y}}
  \\&=&    \left.\left.\frac{1}{2\sqrt{y}}\right[
           \ppx(-\sqrt{y}) + \ppx( \sqrt{y}) \right]
\end{eqnarray*}
\end{proof}




\begin{figure}\color{figcolor}
\setlength{\unitlength}{0.15mm}
\begin{center}
\begin{footnotesize}
\begin{picture}(1000,220)(-500,-100)
  \put(-500,   0){\line(1,0){1000}}
  \put(   0, -100){\line(0,1){220}}
  \multiput(-400,0)(200,0){5}{
    {\color{red}
      \qbezier(0,0)(25,25)(50,37)
      \qbezier(50,37)(75,50)(95,100)
      \qbezier(0,0)(-25,-25)(-50,-37)
      \qbezier(-50,-37)(-75,-50)(-95,-100)
      }
    \qbezier[14](77,0)(77,32)(77,64)
    }
  \put( 100, 105){\makebox(0,0)[b]{$z=\atan\theta$}}
  \put(-400,-5){\makebox(0,0)[t]{$-2\pi$}}
  \put(-200,-5){\makebox(0,0)[t]{$- \pi$}}
  \put( 200,-5){\makebox(0,0)[t]{$  \pi$}}
  \put( 400,-5){\makebox(0,0)[t]{$ 2\pi$}}

  \put(-323,-5){\makebox(0,0)[t]{$\theta_{-2}$}}
  \put(-123,-5){\makebox(0,0)[t]{$\theta_{-1}$}}
  \put(  77,-5){\makebox(0,0)[t]{$\theta_{ 0}$}}
  \put( 277,-5){\makebox(0,0)[t]{$\theta_{ 1}$}}
  \put( 477,-5){\makebox(0,0)[t]{$\theta_{ 2}$}}

  \qbezier[130](-315,64)(77,64)(477,64)
  \put(   0, 110){\makebox(0,0)[r]{$z$}}
  \put( 520,   0){\makebox(0,0)[l]{$\theta$}}
\end{picture}
\end{footnotesize}
\end{center}
\caption{
  $Z=\tan\Theta$
  \label{fig:Z=tan0}
  }
\end{figure}
%---------------------------------------
\begin{proposition}
\citetbl{
  \citerpp{papoulis}{99}{100}
  }
%---------------------------------------
Let $Z=\tan\Theta$. Then
\propbox{
  \ppz(z) = \frac{1}{1+z^2}  \sum_{n\in\Z} \ppth(\atan(z)+n\pi)
  }
\end{proposition}
\begin{proof}
Let $z=\frac{y}{x}$ and $x^2 + y^2 = r^2$.
\begin{eqnarray*}
  \cos^2\atan z
    &=& \cos^2\theta
     =  \frac{x^2}{r^2}
     =  \frac{x^2}{x^2+y^2}
     =  \frac{\frac{x^2}{x^2}}{\frac{x^2}{x^2}+\frac{y^2}{x^2}}
     =  \frac{1}{1+z^2}
\end{eqnarray*}
Let $h\to0$.
\begin{eqnarray*}
  \atan{z+h}
    &=&    y_1 + \frac{1}{m} \Delta y
  \\&=&    \atan{z} + \left.\frac{1}{\dz/\dth}\right|_{\theta=\atan{z}} h
  \\&=&    \atan{z} + \left.\frac{1}{\sec^2\theta}\right|_{\theta=\atan{z}} h
  \\&=&    \atan{z} + \left.\cos^2\theta\right|_{\theta=\atan{z}} h
  \\&=&    \atan{z} + h\cos^2\atan{z}
  \\&=&    \atan{z} + h\frac{1}{1+z^2}
\end{eqnarray*}

Now we prove the theorem using the above relation.
\begin{align*}
  \ppz(z)h
    &=    \pP{z\le Z < z+h}
  \\&=    \pP{z\le \tan\Theta < z+h}
  \\&=    \sum_n \pP{z\le \tan\Theta < z+h \land
                 \pi\left(n-\frac{1}{2}\right) \le \Theta < \pi\left(n+\frac{1}{2}\right)
                 }
  \\&=    \sum_n \pP{z\le \tan\Theta < z+h \left|
                 \pi\left(n-\frac{1}{2}\right) \le \Theta < \pi\left(n+\frac{1}{2}\right)
                 \right.}
                 \pP{\pi\left(n-\frac{1}{2}\right) \le \Theta \pi\left(n+\frac{1}{2}\right)}
  \\&=    \sum_n \pP{\atan{z}+n\pi \le \Theta < \atan(z+h)+n\pi \left|
                 \pi\left(n-\frac{1}{2}\right) \le \Theta < \pi\left(n+\frac{1}{2}\right)
                 \right.}
                 \pP{\pi\left(n-\frac{1}{2}\right) \le \Theta \pi\left(n+\frac{1}{2}\right)}
  \\&=    \sum_n \pP{\atan{z}+n\pi \le \Theta < \atan(z+h)+n\pi \land
                 \pi\left(n-\frac{1}{2}\right) \le \Theta < \pi\left(n+\frac{1}{2}\right)
                 }
  \\&=    \sum_n \pP{\atan{z}+n\pi \le \Theta < \atan(z+h)+n\pi }
  \\&=    \sum_n \pP{\atan{z}+n\pi \le \Theta < +\atan(z)+n\pi + h\frac{1}{1+z^2} }
  \\&=    h\frac{1}{1+z^2} \sum_n \ppth(\atan{z}+n\pi)
\\\implies
  \ppz(z) &=  \frac{1}{1+z^2} \sum_n \ppth(\atan{z}+n\pi)
\end{align*}

The theorem can also be proved using \prefpp{thm:Y=f(X)}.
\begin{eqnarray*}
  \theta_n &=& \atan{z}+n\pi \\
  \ff'(\theta) &=& \sec^2 \theta    \\
  \ppz(z)
    &=& \sum_{n=1}^N \frac{\ppth(\theta_n)}{|\ff'(\theta_n)|}
  \\&=& \sum_n \frac{\ppth(\atan{z}+n\pi)}{|\ff'(\atan{z}+n\pi)|}
  \\&=& \sum_n \frac{\ppth(\atan{z}+n\pi)}{|\sec^2(\atan{z}+n\pi)|}
  \\&=& \sum_n \cos^2(\atan{z}+n\pi)  \ppth(\atan{z}+n\pi)
  \\&=& \cos^2(\atan{z}) \sum_n  \ppth(\atan{z}+n\pi)
  \\&=& \frac{1}{1+z^2}  \sum_n \ppth(\atan{z}+n\pi)
\end{eqnarray*}
\end{proof}




%=======================================
\section{Joint and conditional probability spaces}
%=======================================
As described in \prefpp{def:prob_space},
every probability space $\ps$ contains a probability measure $\psp:\pse\to[0,1]$.
This probability measure has some basic properties as described in
\pref{thm:P} (next).
%---------------------------------------
\begin{theorem}
\label{thm:P}
%---------------------------------------
Let $\ps$ be a probability space,
and $\set{B_n}{n=1,2,\ldots,N}$ be a partition of a set $B$.
\thmbox{\begin{array}{rc>{\ds}l@{\qquad}l}
  \psp(B)  &=&  \sum_{n=1}^N \psp(B_n)    & \forall B\in\pse\\
  \psp(AB) &=&  \sum_{n=1}^N \psp(AB_n)   & \forall A,B\in\pse
\end{array}}
\end{theorem}
\begin{proof}
This is because $\psp$ is a measure and by \prefpp{def:measure}.
\end{proof}



Sometimes the problem of finding the expected value of a random variable $\rvX$
can be simplified by ``conditioning $\rvX$ on $Y$".
%---------------------------------------
\begin{theorem}
%---------------------------------------
Let $\rvX$ and $Y$ be random variables. Then
\thmbox{\pEx{X} = \pEy\pEx[x|y](X|Y) }
\end{theorem}
\begin{proof}
\begin{eqnarray*}
   \pEy\pEx[x|y](X|Y)
     &\eqd& \pEy \left[ \int_x x \pp(X=x|Y) \dx \right]
   \\&\eqd& \int_y \left[\int_x x \pp(x|Y=y) \dx \right] \pp(y) \dy
   \\&=&    \int_y \int_x x \pp(x|y)\pp(y) \dx   \dy
   \\&=&    \int_x x \int_y \pp(x,y) \dy   \dx
   \\&=&    \int_x x \pp(x) \dx
   \\&\eqd& \pEx X
\end{eqnarray*}
\end{proof}




When possible, we like to generalize any given mathematical structure
to a more general mathematical structure and then take advantage of
the properties of that more general structure.
Such a generalization can be done with random variables.
Random variables can be viewed as vectors in a vector space.
Furthermore, the expectation of the product of two random variables
(e.g. $\pE(XY)$)
can be viewed as an innerproduct in an innerproduct space.
Since we have an inner product space,
we can then immediately use all the properties of
innerproduct spaces, normed spaces, vector spaces, metric spaces,
and topological spaces.\footnote{
  \hie{spaces:} Chapter/Appendix~\ref{chp:space} page~\pageref{chp:space}
  }




%---------------------------------------
\begin{theorem}
\citetbl{
  \citerpp{moon2000}{105}{106}
  }
\label{thm:prb_vspace}
%---------------------------------------
Let $R$ be a ring,
$\ps$ be a probability space, $\pE$ the expectation operator, and
$V=\set{X}{X:\pso\to R}$ be the set of all random vectors
in probability space $\ps$.
\thmbox{\begin{tabular}{ll}
  1. & $V$ is a vector space. \\
  2. & $\inprod{X}{Y}\eqd\pE(XY^\ast)$ is an inner product. \\
  3. & $\norm{X}\eqd\sqrt{\pE(XX^\ast)}$ is a norm. \\
  4. & $(V,\inprod{\cdot}{\cdot})$ is an innerproduct space.
\end{tabular}}
\end{theorem}
\begin{proof}
\begin{enumerate}
  \item Proof that $V$ is a vector space:
    \[\begin{array}{lll@{\hs{1cm}}D}
   1) & \forall  X, Y, Z\in V
      & ( X+ Y)+ Z =  X+( Y+ Z)
      & \text{($+$ is associative)}
      \\
   2) & \forall  X, Y\in V
      &  X+ Y =  Y+ X
      & \text{($+$ is commutative)}
      \\
   3) & \exists  0 \in V \st \forall  X\in V
      &  X+ 0 =  X
      & \text{($+$ identity)}
      \\
   4) & \forall  X \in V \exists  Y\in V \st
      &  X+ Y =  0
      & \text{($+$ inverse)}
      \\
   5) & \forall \alpha\in S \text{ and }  X, Y\in V
      & \alpha\cdot( X+ Y) = (\alpha \cdot X)+(\alpha\cdot Y)
      & \text{($\cdot$ distributes over $+$)}
      \\
   6) & \forall \alpha,\beta\in S \text{ and }  X\in V
      & (\alpha+\beta)\cdot X = (\alpha\cdot  X)+(\beta\cdot  X)
      & \text{($\cdot$ pseudo-distributes over $+$)}
      \\
   7) & \forall \alpha,\beta\in S \text{ and }  X\in V
      & \alpha(\beta\cdot X) = (\alpha\cdot\beta)\cdot X
      & \text{($\cdot$ associates with $\cdot$)}
      \\
   8) & \forall  X\in V
      & 1\cdot  X =  X
      & \text{($\cdot$ identity)}
\end{array}\]

  \item Proof that $\inprod{X}{Y}\eqd\pE(XY^\ast)$ is an inner product.
  \[\begin{array}{llllD}
   1) &  \pE(XX^\ast) &\ge 0
      &  \forall  X\in V
      &  \text{(non-negative)}
      \\
   2) &  \pE(XX^\ast) &= 0 \iff  X=0
      &  \forall  X\in V
      &  \text{(non-degenerate)}
      \\
   3) &  \pE(\alpha XY^\ast)    &= \alpha\pE(XY^\ast)
      &  \forall  X, Y\in V,\;\forall\alpha\in\C
      &  \text{(homogeneous)}
      \\
   4) &  \pE[(X+Y)Z^\ast] &= \pE(XZ^\ast) + \pE(YZ^\ast)
      &  \forall  X, Y, Z\in V
      &  \text{(additive)}
      \\
   5) &  \pE(XY^\ast) &= \pE(YX^\ast)
      &  \forall  X, Y\in V
      &  \text{(conjugate symmetric)}.
  \end{array}\]

  \item Proof that $\norm{X}\eqd\sqrt{\pE(XX^\ast)}$ is a norm:
    This norm is simply induced by the above innerproduct.
  \item Proof that $(V,\inprod{\cdot}{\cdot})$ is an innerproduct space:
    Because $V$ is a vector space and $\inprod{\cdot}{\cdot}$ is
    an innerproduct, $(V,\inprod{\cdot}{\cdot})$ is an innerproduct space.
\end{enumerate}
\end{proof}




The next theorem gives some results that follow directly from vector space
properties:
%---------------------------------------
\begin{theorem}
\index{Generalized triangle inequality}
\index{Cauchy-Schwartz inequality}
\index{Parallelogram Law}
%---------------------------------------
Let $\ps$ be a probability space with expectation functional $\pE$.
\thmbox{\begin{array}{l >{\ds}r c >{\ds}l D}
  1. & \sqrt{\pE\left(\sum_{n=1}^N X_n\right)}
     &\le& \sum_{n=1}^N \pE(X_nX_n^\ast)
     & (Generalized triangle inequality)
     \\
  2. & \left| \pE(XY^\ast)\right|^2
     &\le& \pE(XX^\ast)\:\pE(YY^\ast)
     & (Cauchy-Schwartz inequality)
     \\
  3. & 2\pE(XX^\ast) + 2\pE(YY^\ast)
     &=& \pE[(X+Y)(X+Y)^\ast] + \pE[(X-Y)(X-Y)^\ast]
     &   (Parallelogram Law)
  \end{array}}
\end{theorem}
\begin{proof}
These follow directly from vector space results in
Chapter/Appendix~\ref{chp:space}:

\begin{tabular}{llll}
  1. & Generalized triangle inequality:
     & \prefpp{thm:norm_tri}
     & page~\pageref{thm:norm_tri}
     \\
  2. & Cauchy-Schwartz inequality:
     & \prefpp{thm:cs}
     & page~\pageref{thm:cs}
     \\
  3. & Parallelogram Law:
     & \prefpp{thm:parallelogram}
     & page~\pageref{thm:parallelogram}
\end{tabular}
\end{proof}


%=======================================
%\section{Multiple random variables}
%=======================================
%---------------------------------------
\begin{definition}
\index{convolution}
%---------------------------------------
The {\bf convolution} operator $\conv$ is defined as
\defbox{ \ff(x)\conv\fg(x) \eqd \int_u \ff(u)\fg(x-u) \du }
\end{definition}

%---------------------------------------
\begin{theorem}
%---------------------------------------
Let $\rvX$ and $Y$ be independent random varibles and $Z\eqd X+Y$.
Then
\formbox{ \ppz(z) = \ppx(z)\conv\ppy(z)  }
\end{theorem}
\begin{proof}
\begin{eqnarray*}
  \ppz(z)
    &\eqd& \lim_{h\to0}\frac{1}{h} \pP{z \le Z < z+h }
  \\&=&    \lim_{h\to0}\frac{1}{h} \pP{z \le X+Y < z+h }
  \\&=&    \lim_{h\to0}\frac{1}{h} \int_y \pP{(z \le X+y < z+h) \land (y\le Y<y+h) } \dy
  \\&=&    \lim_{h\to0}\frac{1}{h} \int_y \pP{(z-y \le X < z-y+h) | (y\le Y<y+h) }\pP{y\le Y<y+h} \dy
  \\&=&    \lim_{h\to0}\frac{1}{h} \int_y \pP{(z-y \le X < z-y+h) | Y=y }\pP{y\le Y<y+h} \dy
  \\&\eqd& \int_y \ppx(z-y) \ppy(y)  \dy
  \\&=&    \ppx(z) \conv \ppy(z)
\end{eqnarray*}
\end{proof}

%---------------------------------------
\begin{theorem}
\label{thm:x1x2->y1y2}
%---------------------------------------
Let
\begin{liste}
  \item $X_1$ and $X_2$ be random variables with joint distribution
        $\ppx[X_1,X_2](x_1,x_2)$
  \item $Y_1=\ff_1(x_1,x_2)$ and $Y_2=\ff_2(x_1,x_2)$
\end{liste}
Then the joint distribution of $Y_1$ and $Y_2$ is
\thmbox{
  \ppx[Y_1,Y_2](y_1,y_2)
    = \frac{\ppx[X_1,X_2](x_1,x_2)}{|J(x_1,x_2)|}
    = \frac{\ppx[X_1,X_2](x_1,x_2)}{
        \left|\begin{array}{cc}
          \pderiv{\ff_1}{x_1} & \pderiv{\ff_1}{x_2}   \\
          \pderiv{\ff_2}{x_1} & \pderiv{\ff_2}{x_2}
        \end{array}\right|
        }
    = \frac{\ppx[X_1,X_2](x_1,x_2)}{
        \pderiv{\ff_1}{x_1}\pderiv{\ff_2}{x_2} -
        \pderiv{\ff_1}{x_2}\pderiv{\ff_2}{x_1}
        }
  }
\end{theorem}

%---------------------------------------
\begin{proposition}
\label{prop:XY->RT}
%---------------------------------------
Let $\rvX$ and $Y$ be random variables with joint distribution
$\ppxy(x,y)$ and
\[ R^2 \eqd X^2 + Y^2 \hspace{10ex} \Theta \eqd \atan\frac{Y}{X}. \]
Then
\formbox{
  \ppx[R,\Theta](r,\theta)
    =  r\;\ppxy(r\cos\theta,r\sin\theta)
  }
\end{proposition}
\begin{proof}
\begin{eqnarray*}
  \ppx[R,\Theta](r,\theta)
    &=& \frac{\ppxy(x,y)}{|J(x,y)|}
     =  \frac{\ppxy(x,y)}{
        \left|\begin{array}{cc}
          \pderiv{R}{x}      & \pderiv{R}{y}   \\
          \pderiv{\theta}{x} & \pderiv{\theta}{y}
        \end{array}\right|
        }
     =  \frac{\ppxy(x,y)}{
        \left|\begin{array}{cc}
          \frac{ x}{\sqrt{x^2+y^2}}  & \frac{y}{\sqrt{x^2+y^2}}   \\
          \frac{-y}{x^2+y^2}         & \frac{x}{x^2+y^2}
        \end{array}\right|
        }
  \\&=& \frac{\ppxy(x,y)}{
          \frac{x}{\sqrt{x^2+y^2}}\frac{x}{x^2+y^2}  -
          \frac{y}{\sqrt{x^2+y^2}}\frac{-y}{x^2+y^2}
        }
  \\&=& \frac{\ppxy(x,y)}{
          \frac{x^2+y^2}{(x^2+y^2)^{3/2}}
        }
  \\&=& \ppxy(x,y)\frac{(x^2+y^2)^{3/2}}{x^2+y^2}
  \\&=& \ppxy(r\cos\theta,r\sin\theta)\frac{r^3}{r^2}
  \\&=& r\;\ppxy(r\cos\theta,r\sin\theta)
\end{eqnarray*}
\end{proof}


%---------------------------------------
\begin{proposition}
\label{prop:XY->RT_n}
%---------------------------------------
Let $X\sim\pN{0}{\sigma^2}$ and $Y\sim\pN{0}{\sigma^2}$ be
independent random variables and
\[ R^2 \eqd X^2 + Y^2 \hspace{10ex} \Theta \eqd \atan\frac{Y}{X}. \]
Then
\formbox{\begin{array}{llrcl}
  1. & \mbox{$R$ and $\Theta$ are independent with joint distribution}
     & \ppx[R,\Theta](r,\theta) &=& \ppr(r)\ppth(\theta)
\\
  2. & \mbox{$R$ has Rayleigh distribution}
     & \ppr(r)  &=& \frac{r}{\sigma^2}\exp{\frac{r^2}{-2\sigma^2}}
\\
  3. & \mbox{$\Theta$ has uniform distribution}
     & \ppth(\theta) &=& \frac{1}{2\pi}
\end{array}}
\end{proposition}
\begin{proof}
\begin{eqnarray*}
  \ppx[R,\Theta](r,\theta)
    &=& r\;\ppxy(r\cos\theta,r\sin\theta)
        \hspace{8ex}\mbox{by Proposition~\ref{prop:XY->RT}}
  \\&=& r\;\ppx(r\cos\theta) \; \ppy(r\sin\theta)
        \hspace{8ex}\mbox{by independence hypothesis}
  \\&=& r\;
        \frac{1}{\sqrt{2\pi\sigma^2}}\exp{\frac{(r\cos\theta-0)^2}{-2\sigma^2}}
        \frac{1}{\sqrt{2\pi\sigma^2}}\exp{\frac{(r\sin\theta-0)^2}{-2\sigma^2}}
  \\&=& \frac{1}{2\pi\sigma^2}\;r\;
        \exp{\frac{r^2(\cos^2\theta + \sin^2\theta)}{-2\sigma^2}}
  \\&=& \frac{1}{2\pi\sigma^2}\;r\;
        \exp{\frac{r^2}{-2\sigma^2}}
  \\&=& \left[\frac{1}{2\pi}\right]
        \left[\frac{r}{\sigma^2}\exp{\frac{r^2}{-2\sigma^2}}\right]
\end{eqnarray*}
\end{proof}


%---------------------------------------
\begin{proposition}
%---------------------------------------
Let $X\sim\pN{\pmeanx}{\pvarx}$ and $Y\sim\pN{\pmeany}{\pvary}$ be
jointly Gaussian random variables and $\pvarxy=\cov{X}{Y}$.
Then
\formbox{
  \pP{X>Y} = \pQ\left( \frac{-\pmeanx + \pmeany}{\pvarx+\pvary-2\pvarxy}\right)
  }
\end{proposition}
\begin{proof}
Because $\rvX$ and $Y$ are jointly Gaussian,
their linear combination $Z=X-Y$ is also Gaussian.
A Gaussian distribution is completely defined by its mean and variance.
So, to determine the distribution of $Z$,
we just have to determine the mean and variance of $Z$.
\begin{eqnarray*}
  \pE Z
    &=& \pE X - \pE Y
  \\&=& \pmeanx - \pmeany
\\
\\
  \var Z
    &=& \pE Z^2 - (\pE Z)^2
  \\&=& \pE (X-Y)^2 - (\pE X - \pE Y)^2
  \\&=& \pE (X^2-2XY+Y^2) - [(\pE X)^2 -2\pE X \pE Y + (\pE Y)^2 ]
  \\&=& [\pE X^2- (\pE X)^2]  + [Y^2- (\pE Y)^2] - 2[\pE XY - \pE X \pE Y]
  \\&=& \var X + \var Y - 2\cov{X}{Y}
  \\&\eqd& \pvarx + \pvary -2\pvarxy
\\
\\
  \pP{X>Y}
    &=& \pP{X-Y>0}
  \\&=& \pP{Z>0}
  \\&=& \left.\pQ\left(\frac{z-\pE Z}{\var Z} \right)\right|_{z=0}
  \\&=& \pQ\left(\frac{0-\pmeanx+\pmeany}{\pvarx+\pvary-2\pvarxy} \right)
\end{eqnarray*}
\end{proof}
