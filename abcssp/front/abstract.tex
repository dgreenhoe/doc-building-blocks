%============================================================================
% Daniel J. Greenhoe
% XeLaTeX / LaTeX file
% abstract
%============================================================================
%=======================================
\section*{Abstract}
\addcontentsline{toc}{section}{Abstract}
%=======================================
A \fncte{real-valued random variable} $\rvX$ is a \prope{measurable} \structe{function} 
that maps \emph{from} a \structe{probability space} $\ps$
\emph{to} $\omlsRDB$ where $\borel$ is the \structe{usual Borel \txsigma-algebra}
on the ``real line" $\omlsRD$ and 
where $\R$ is the \structe{set of real numbers},
$\orel$ is the standard linear order relation on $\R$, $\metric{x}{y}\eqd\abs{x-y}$ is the 
\fncte{usual metric} on $\R$, and $\fieldR$ is the standard \structe{field} on $\R$.
This text demonstrates that this definition of random variable is   % ~53 words
often a poor choice for computing statistics
when the \structe{probability space} that $\rvX$ maps \emph{from} has structure that is dissimilar to that of the real line. %~82 words

This text proposes two alternative statistical systems that, unlike the traditional method of 
a random variable $\rvX$ mapping exclusively to the real line,
$\rvX$ instead maps more generally to (1) a \structe{weighted graph} 
(2) an \structe{ordered distance linear space} $\R^\xN$.
In each mapping method, the structure that $\rvX$ maps to  
is preferably one
that has order and metric geometry structures similar to that of the underlying stochastic process.
And ideally the structure $\rvX$ maps \emph{from} and the 
structure $\rvX$ maps \emph{to} are, with respect to each other, 
both \prope{isomorphic} and \prope{isometric}.


%=======================================
% ascii abstract
%=======================================
%A real-valued random variable X is a measurable function that maps from a probability space (S,E,P) to (R,<=,d,+,.,B) where B is the usual Borel sigma-algebra on the "real line" (R,<=,d,+,.) where R is the set of real numbers, <= is the standard linear order relation on R, d(x,y):=|x-y| is the usual metric on R, and (R,+,.,0,1) is the standard field on R. This text demonstrates that this definition of random variable is often a poor choice for computing statistics when the probability space that X maps from has structure that is dissimilar to that of the real line. 
%
%This text proposes two alternative statistical systems that, unlike the traditional method of a random variable X mapping exclusively to the real line, X instead maps more generally to (1) a weighted graph (2) an ordered distance linear space R^N. In each mapping method, the structure that X maps to is preferably one that has order and metric geometry structures similar to that of the underlying stochastic process. And ideally the structure X maps from and the structure X maps to are, with respect to each other, both isomorphic and isometric. 

%%=======================================
%\section*{Abstract in Chinese}
%\addcontentsline{toc}{section}{Abstract in Chinese}
%%=======================================
%% modified per suggestions from Po-Ning Chen in 2016 June 25 10:11am email
%{\large\makexeCJKactive%\fntWangClearZY
%%一個\fncte{傳統的隨機變量}$\rvX$是一個\emph{從}\structe{隨機過程}\emph{到}``實數線"$\omlsRD$
%一個\fncte{實值的隨機變量}$\rvX$是一個\emph{從}\structe{概率空間}$\ps$\emph{到}
%$\omlsRDB$ 
%%``實數線"$\omlsRD$
%的\prope{可量度}\structe{函數}映射,
%其中,
%$\borel$是在``實數線"$\omlsRD$上的\structe{通常博雷爾$\sigma$-代數},
%$\R$是\structe{實數集合},
%$\orel$是定義在$\R$上的標準線性順序關係,
%$\metric{x}{y}\eqd\abs{x-y}$是在$\R$上的\fncte{常規距離量度},
%而$\fieldR$則構成$\R$的場域(\structe{field})。
%本文以例論述當概率空間到實數線的 X 映射與實數線有不相類同的結構時,這個隨機變量的定義常導致不佳的統計計算量。}
%%本文以例論述當隨機過程到實數線的 X 映射與實數線有不相類同的結構時,這個隨機變量的定義常導致不佳的統計計算量。}
%
%\makexeCJKinactive
%
%{\large\makexeCJKactive%\fntWangClearZY
%本文因此提出兩種可能的替換統計系統,
%不像傳統的定義由一個隨機變量$\rvX$映射至實數線,
%$\rvX$可更普遍的映射到 (1) \structe{加權圖} (2) \structe{順序距離線性空間}$\R^\xN$。
%在以上任一映射法中,$\rvX$形成類似基礎隨機過程之有順序並有度量幾何結構映射。
%理想情況下,$\rvX$的原映射結構與$\rvX$的目標映射結構,相對於彼此,都是同構和等距的。
%}\footnote{Many many many thanks to Professor Po-Ning Chen for his help in improving an earlier version of the Chinese abstract.
%Any errors that may remain are strictly my own.}
%\makexeCJKinactive

%%=======================================
%\section*{old abstract in Chinese}
%\addcontentsline{toc}{section}{Abstract in Chinese}
%%=======================================
%{\large
%\makexeCJKactive%\fntWangClearZY
%一個\fncte{傳統的隨機變量}$\rvX$是一個\emph{從}\structe{隨機過程}\emph{到}``實直線"$\omlsRD$
%的\structe{函數}映射,
%其中,$\R$是\structe{實數集},
%$\orel$是用在$\R$標準線性順序關係上,
%$\metric{x}{y}\eqd\abs{x-y}$是用在$\R$\fncte{常規的度量}上,
%而$\fieldR$是用在$\R$\structe{實數的域}上。
%本文表明
%當X映射從隨機過程到實直線不同的結構時,
%這個隨機變量的定義常是計算統計一個不佳的選擇。
%
%本文提出兩種可供選擇的統計系統,
%不像傳統的方法由一個隨機變量$\rvX$映射專門為實直線,
%$\rvX$反而更普遍映射到(1)\structe{加權圖}
%(2)\structe{順序距離線性空間}$\R^\xN$。
%在每一個方法,$\rvX$映射是有順序並有度量幾何結構,類似底層隨機過程。
%理想情況下從$\rvX$映射結構到$\rvX$映射結構之間,
%相對於彼此,都是\prope{同構}和\prope{等距}的。
%\makexeCJKinactive
%}





