%============================================================================
% Daniel J. Greenhoe
% LaTeX file
% Lagrange arc metric examples in R^2
%         y
%         |   o p        Let (rp,tp) be the polar location of point p.
%         |  /           where rp is the Euclidean distance from (0,0) to p 
%         | /            and tp is radian measure from the x-axis to p.
%         |/tp           Let (rq,tq) be the polar location of point q.
% --------|---------- x  The "Lagrange arc" r(theta) is defined here as
%         |\tq                          theta -tq        theta -tp
%         | \            r(theta) = rp ----------- + rq -----------
%         |  o q                          tp-tq            tq-tp
%         |              
%============================================================================
\begin{pspicture}(-1.5,-0.6)(1.5,1.5)%
  %\psset{plotstyle=line, linecolor=blue, linewidth=1pt, dotsize=3pt}%
  \psset{dotsize=3pt}%
  \psaxes[linecolor=axis,labels=none]{<->}(0,0)(-1.5,-0.6)(1.5,1.5)%
  \pnode(0,1){p1}%
  \pnode(0,0.8){p2}%
  \pnode(0,0.2){p3}%
  \pnode(1,0){q}%
  %
  \psellipse[linecolor=red,linestyle=dashed](center)(1,1)%
  \psellipse[linecolor=red,linestyle=dashed](center)(1,0.8)%
  \psellipse[linecolor=red,linestyle=dashed](center)(1,0.2)%
  %
  \psplot[polarplot=true,linecolor=blue]{90}{0}{1   x 0 sub 90 0 sub div mul 1 x 90 sub 0 90 sub div mul add}%
  \psplot[polarplot=true,linecolor=blue]{90}{0}{0.8 x 0 sub 90 0 sub div mul 1 x 90 sub 0 90 sub div mul add}%
  \psplot[polarplot=true,linecolor=blue]{90}{0}{0.2 x 0 sub 90 0 sub div mul 1 x 90 sub 0 90 sub div mul add}%
  %\psplot[polarplot=true,linecolor=red]{123.690}{0}{1 x 123.690 sub 0 123.690 sub div mul 0.9014 x 0 sub 123.690 0 sub div mul add}%
  %\psplot[polarplot=true,linecolor=red]{157.380}{0}{1 x 157.380 sub 0 157.380 sub div mul 0.65 x 0 sub 157.380 0 sub div mul add}%
  \psdot[linecolor=blue](p1)%
  \psdot[linecolor=blue](p2)%
  \psdot[linecolor=blue](p3)%
  \psdot[linecolor=blue](q)%
  %
  \uput[135]{0}(p1){$p_1$}%
  \uput[225]{0}(p2){$p_2$}%
  \uput[150]{0}(p3){$p_3$}%
  \uput[-45]{0}(q){$q$}%
\end{pspicture}%
