%============================================================================
% LaTeX File
% Daniel J. Greenhoe
%============================================================================

%======================================
\chapter{ZF Set Theory}
\label{chp:zfset}
%======================================
Naive Set Theory is ``good enough" for this document
and most all other mathematics applications.
However, in general, Naive set theory is flawed as shown by
\hie{Russell's Paradox} (next theorem).



\qboxnps
  {
    \href{http://www-groups.dcs.st-and.ac.uk/~history/Biographies/Frege.html}{Friedrich Ludwig Gottlob Frege} 
    (\href{http://www-history.mcs.st-andrews.ac.uk/Timelines/TimelineF.html}{1848--1925}), 
    \href{http://www-history.mcs.st-andrews.ac.uk/BirthplaceMaps/Places/Germany.html}{German}
    mathematical logic pioneer  \\
    in his 1902 June 22 letter of reply to Bertrand Russell's 1902 June 16 letter which introduced 
    ``\hie{Russell's paradox}" showing that \prop{naive set theory} is flawed.
    \index{Frege, Friedrich Ludwig Gottlob}
    \index{Russell, Bertrand}
    \index{quotes!Frege, Friedrich Ludwig Gottlob}
    \index{quotes!Russell, Bertrand}
    \footnotemark
  }
  {../common/people/frege.jpg}
  {Your discovery of the contradiction caused me the greatest surprise and,
   I would almost say, consternation, since it has shaken the basis on which I intend
   to build arithmetic.\ldots 
   In any case your discovery is very remarkable and will perhaps result in a great
   advance in logic, unwelcomed as it may seem at first glance.}
  \citetblt{
    quote: & \citor{frege1902e}  \\
    image: & \url{http://en.wikipedia.org/wiki/Image:Frege.jpg} \\
    Russell's letter: & \citor{russell1902e} 
    }

\qboxnps
  {
    \href{http://www-groups.dcs.st-and.ac.uk/~history/Biographies/Frege.html}{Friedrich Ludwig Gottlob Frege} 
    (\href{http://www-history.mcs.st-andrews.ac.uk/Timelines/TimelineF.html}{1848--1925}), 
    \href{http://www-history.mcs.st-andrews.ac.uk/BirthplaceMaps/Places/Germany.html}{German}
    mathematical logic pioneer  \\
    in reference to a 1902 June 16 letter from Bertrand Russell showing that 
    \prop{naive set theory} is flawed.
    \index{Frege, Friedrich Ludwig Gottlob}
    \index{Russell, Bertrand}
    \index{quotes!Frege, Friedrich Ludwig Gottlob}
    \index{quotes!Russell, Bertrand}
    \footnotemark
  }
  {../common/people/frege.jpg}
  {A scientist can hardly meet with anything more undesirable than to
    have the foundation give way just as the work is finished.
    In this position I was put by a letter from Mr Bertrand Russell
    as the work was nearly through the press.}
  \citetblt{
    quote: & \citor{frege1903}  \\
           & \citor{frege1903e}  \\
           & \url{http://www-groups.dcs.st-and.ac.uk/~history/Quotations/Frege.html}\\
    image: & \url{http://en.wikipedia.org/wiki/Image:Frege.jpg} \\
    letters: & \citor{russell1902e} \\
             & \citor{frege1902e} \\
    }


%---------------------------------------
\begin{theorem}[Russell's Paradox]
\label{thm:russell}
\citetbl{
  \citerp{wolf}{137}    \\
  \citor{russell1903}  \\
  \citor{heijenoort}   \\
  }
\index{Russell's Paradox}
\index{theorems!Russell's Paradox}
%---------------------------------------
Naive set theory allows the statement
\[ A \eqd \{ B|B\nsubseteq B\}.\]
This statement produces a \emph{contradiction}
and Naive set theory is therefore \emph{inconsistent}.
\end{theorem}
\begin{proof}
\begin{align*}
   A \subseteq    A &\implies A \nsubseteq A   && \text{by definition of $A$}. \\
   A \nsubseteq A &\implies A \subseteq    A   && \text{because $A$ is a set and so $A$ is always in itself}.
\end{align*}

These two statements together produce the contradiction
   \[A \nsubseteq A \qquad\iff\qquad A \subseteq   A.\]
\end{proof}

A very famous illustration of Russell's Paradox is the
{\em Barber of Seville} (next example).


%\begin{figwindow}[4,l,\fbox{\includegraphics*[width=3\tw/16-3ex, keepaspectratio=true, clip=true]{../common/people/barber.gif}},{hello}]
%---------------------------------------
\begin{example}[Barber of Seville]
\label{ex:barber}
\footnote{``Barbershop" gif is from \url{http://www.animationfactory.com}}
\index{Barber of Seville}
%---------------------------------------
In the town of Seville, there lives a barber.
His name is \hie{the Barber of Seville}.\\
\begin{minipage}[t][3\tw/(4*4)-1em][b]{\tw/4}
  \includegraphics*[width=\tw, keepaspectratio=true, clip=true]{../common/people/barber.gif}
\end{minipage}
\hfill
\begin{minipage}[t]{3\tw/4-2ex}
The barber shaves all the men in Seville who do not shave themselves.
But there is a crises in the barber shop of the Barber of Seville:
\begin{liste}
  \item If the barber {\em does not shave himself}, then he is shaved by
        the barber (himself).
  \item And if the barber {\em does shave himself}, then he is not shaved by
        the barber (himself).
\end{liste}
\end{minipage}\\
Therefore, the crises in the barber shop of the Barber of Seville
is a crises of logical contradiction;
such that even under a well defined (Naive) set (the set of who the barber shaves),
the hypothesis that the barber does not shave himself leads to the conclusion
that the barber does shave himself (a contradiction).
The crises in Seville can also be described using set notation as follows.
Let
\[\begin{array}{rcl D}
     S &\eqd& \setn{\text{men in Seville}}    & (the set of all men in Seville)
  \\ T &\eqd& \setn{\text{men who shave themeselves}} & (the set of all men who shave themselves)
  \\ b\in S &\eqd& \text{the barber}               & (the barber who is a man in Seville)
  \\ B &\eqd& \set{x\in S}{x\notin T}        & (the set of men shaved by the barber)
\end{array}\]

Under Naive set theory, all of these sets are well defined.
But these sets lead to a contradiction:
\[ \mcom{b \notin T}{barber does {\bf not} shave himself}
   \implies
   \mcom{b \in\setB}{by definition of $B$}
   \implies
   \mcom{b \in T}{barber {\bf does} shave himself}
\]
\end{example}
%\end{figwindow}



The most commonly used replacement for the flawed Naive set theory
is \hie{Zermelo-Fraenkel (ZF) set theory}.
\footnote{\begin{tabular}[t]{l}
  \citei[pages 263--267]{zermelo1908} (7 axioms)\\
  \citei{zermelo1908e} (English translation)\\
  \citei{fraenkel1922} \\
  \citei[page 139]{wolf}
\end{tabular}}
However, ZF set theory will not be pursued here.
