%============================================================================
% Daniel J. Greenhoe
% LaTeX file
% Mercedes frame / Peace frame
% nominal unit = 10mm
%============================================================================
\begin{pspicture}(-4,-1.5)(4,1.5)%
  %-------------------------------------
  % node definitions
  %-------------------------------------
  \rput{ 90}(0,0){\pnode(1,0){p}}% p = point at distance 1 from origin and angle  90 degrees
  \rput{210}(0,0){\pnode(1,0){q}}% q = point at distance 1 from origin and angle 210 degrees
  \rput{330}(0,0){\pnode(1,0){r}}% r = point at distance 1 from origin and angle 330 degrees
  %-------------------------------------
  % axes
  %-------------------------------------
  \psaxes[linewidth=0.75pt,linecolor=axis,ticks=none,labels=none]{<->}(0,0)(-1.5,-1.5)(1.5,1.5)%
  \uput[0](1.5,0){$x$}%
  \uput[-30](0,1.5){$y$}%
  %-------------------------------------
  % annotation
  %-------------------------------------
  \uput[ 180](p){$\opair{0}{\sqrt{\frac{2}{3}}}$}%
  \uput[ 210](q){$\opair{\sqrt{\frac{2}{3}}\frac{\sqrt{3}}{2}}{-\sqrt{\frac{2}{3}}\frac{1}{2}}$}%
  \uput[ -30](r){$\opair{-\sqrt{\frac{2}{3}}\frac{\sqrt{3}}{2}}{-\sqrt{\frac{2}{3}}\frac{1}{2}}$}%
  %-------------------------------------
  % Mercedes frame / Peace frame
  %-------------------------------------
  \pscircle[linecolor=red,linestyle=dotted](0,0){1}% unit circle
  \psline[linewidth=2pt]{->}(0,0)(p)%
  \psline[linewidth=2pt]{->}(0,0)(q)%
  \psline[linewidth=2pt]{->}(0,0)(r)%
\end{pspicture}%