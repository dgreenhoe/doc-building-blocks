%============================================================================
% LaTeX File
% Daniel J. Greenhoe
%============================================================================
%=======================================
\chapter{Matched Filter Algorithm}
%=======================================
%======================================
%\paragraph{Detection methods.}
%There are basically two types of detection methods:
%\begin{enume}
%   \item signal matching
%   \item orthonormal decomposition.
%\end{enume}

Let $\setS$ be the set of transmitted waveforms and
$\setY$ be a set of orthonormal basis functions that span $\setS$.
\hie{Signal matching} computes the innerproducts of a
received signal $\rvy(t;\theta)$ with each signal from $\setS$.
\hie{Orthonormal decomposition} computes the innerproducts of
$\rvy(t;\theta)$ with each signal from the set $\setY$.

In the case where $\seto{\setS}$ is large, often $\seto{\setY}<<\seto{\setS}$
making orthonormal decomposition much easier to implement.
For example, in a QAM-64 modulation system,
signal matching requires $\seto{\setS}=64$ innerproduct calculations,
while orthonormal decomposition only requires $\seto{\setY}=2$
innerproduct calculations because all 64 signals in $\setS$ can be spanned
by just 2 orthonormal basis functions.

\paragraph{Maximizing SNR.}
\prefpp{thm:sstat} shows that the innerproducts of $\rvy(t;\theta)$ with
basis functions of $\setY$ is \prope{sufficient} for optimal detection.
\prefpp{thm:mf_maxSNR} (next) shows that a receiver can
maximize the SNR of a received signal when signal matching is used.

%--------------------------------------
\begin{theorem}
\label{thm:mf_maxSNR}
%--------------------------------------
Let $\rvx(t)$ be a transmitted signal, $\fv(t)$ noise, and $\rvy(t;\theta)$ the received signal
in an AWGN channel.
Let the \hie{signal to noise ratio} SNR be defined as
\\\indentx$\ds
      \snr[\rvy(t;\theta)] \eqd \frac{\abs{\inprod{\rvx(t)}{\rvx(t)}}^2}
                            {\pE\brs{\abs{\inprod{\fv(t)}{\rvx(t)}}^2}}.
          $
\thmboxt{
  $\ds\snr[\rvy(t;\theta)] \le \frac{2\norm{\rvx(t)}^2}{N_o }$
  \qquad
  and is maximized (equality) when $\rvx(t)=a\rvx(t)$, where $a\in\R$.
  }
\end{theorem}

\begin{proof}
\begin{align*}
   \snr[\rvy(t;\theta)]
     &\eqd \frac{\abs{\inprod{\rvx(t)}{\rvx(t)}}^2}
                {\pE\brs{\abs{\inprod{\fv(t)}{\rvx(t)}}^2}}
   \\&=    \frac{\abs{\inprod{\rvx(t)}{f(t)}}^2}
                {\pE\brs{\left[\int_{t\in\R} \fv(t)\rvx^\ast(t)\;dt\right]
                      \left[\int_{\estT} n(\estT)f^\ast(\estT)\;du\right]^\ast}
                }
   \\&=    \frac{\abs{\inprod{\rvx(t)}{\rvx(t)}}^2}
                {\pE\brs{\int_{t\in\R} \int_{\estT} \fv(t)n^\ast(\estT)\rvx^\ast(t)\rvx(\estT)\;\dt\du}}
   \\&=    \frac{\abs{\inprod{\rvx(t)}{f(t)}}^2}
                {\int_{t\in\R} \int_{\estT} \pE\brs{\fv(t)n^\ast(\estT)}\rvx^\ast(t)\rvx(\estT)\;\dt\du}
   \\&=    \frac{\abs{\inprod{\rvx(t)}{\rvx(t)}}^2}
                {\int_{t\in\R} \int_{\estT} \frac{1}{2}N_o\delta(t-\estT) \rvx^\ast(t)\rvx(\estT)\;\dt\du}
   \\&=    \frac{\abs{\inprod{\rvx(t)}{\rvx(t)}}^2}
                {\frac{1}{2}N_o \int_{t\in\R} \rvx^\ast(t)\rvx(t)\dt}
   \\&=    \frac{\abs{\inprod{\rvx(t)}{\rvx(t)}}^2}
                {\frac{1}{2}N_o \norm{\rvx(t)}^2}
   \\&\le  \frac{\abs{\norm{\rvx(t)}\;\norm{\rvx(t)}}^2}
                {\frac{1}{2}N_o \norm{\rvx(t)}^2}
     &&    \text{by \ineqe{Cauchy-Schwarz Inequality}}
     &&    \text{\ifsxref{vsinprod}{thm:cs}}
   \\&=    \frac{2\norm{\rvx(t)}^2}
                {N_o }
\end{align*}
The Cauchy-Schwarz Inequality becomes an equality
($\snr$ is maximized) when $\rvx(t)=a\rvx(t)$.
\end{proof}

\paragraph{Implementation.}
The innerproduct operations can be implemented using either
  \begin{dingautolist}{"C0}
     \item a correlator or
     \item a matched filter.
  \end{dingautolist}

A correlator is simply an integrator of the form
   $\ds\inprod{\rvy(t;\theta)}{f(t)} = \int_0^T \rvy(t;\theta)f(t)\dt.$

A matched filter introduces a function $\fh(t)$ such that
$\fh(t) =\rvx(T-t)$ (which implies $\rvx(t)=h(T-t)$) giving
  \[
    \mcom{\inprod{\rvy(t;\theta)}{\rvx(t)} = \int_0^T \rvy(t;\theta)\rvx(t)\dt }
         {correlator}
    =
    \mcom{\brlr{\int_0^\infty \rvx(\tau)h(t-\tau)\dtau}_{t=T}
            = \brlr{\rvx(t)\conv \fh(t)}_{t=T}
         }{matched filter}.
  \]

This shows that $\fh(t)$ is the impulse response of a filter operation
sampled at time $\tau$. % (see \prefpp{fig:mf}).
By \prefpp{thm:mf_maxSNR}, the optimal impulse response is
$\fh(\tau-t)=\ff(t)=\rvx(t)$.
That is, the optimal $\fh(t)$ is just a ``flipped" and shifted version of $\rvx(t)$.

