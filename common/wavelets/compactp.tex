%============================================================================
% LaTeX File
% Daniel J. Greenhoe
%============================================================================

%======================================
%\chapter{Spectral Factorization Design Technique}
%\chapter{Compact Support Constraint}
%\chapter{Vanishing Moments Constraint}
\chapter{Minimum Phase Constraint}
\label{chp:compactp}
%======================================


\qboxnpq
  {
    Joseph Louis Lagrange (1736-1813), mathematician
    \index{Lagrange, Joseph Louis}
    \index{quotes!Lagrange, Joseph Louis}
    \footnotemark
  }
  {../common/people/small/lagrange.jpg}
  {I regard as quite useless the reading of large treatises of pure analysis:
    too large a number of methods pass at once before the eyes.
    It is in the works of applications that one must study them;
    one judges their ability there and one apprises the manner of making use of them.}
  \citetblt{
    quote: &  \citerp{stopple2003}{xi} \\
          %&  \url{http://www.math.okstate.edu/~wli/teach/fmq.html} \\
          %&  \url{http://www-groups.dcs.st-and.ac.uk/~history/Quotations/Lagrange.html} \\
    image: & \scs\url{http://en.wikipedia.org/wiki/Image:Langrange_portrait.jpg}, public domain
    }

%=======================================
\section{General structure}
%=======================================
This chapter introduces a wavelet design technique that represents the 
orthogonal quadrature condition in terms of a polynomial in $z$.
It is the technique that was used by Ingrid Daubechies in her celebrated
\hie{Daubechies-$p$} class of wavelets (see \prefp{sec:dau-p}).
The technique can be summarized as follows:
\begin{dingautolist}{"AC}
  \item \textbf{Design requirements.} 
        Let $\wavsys$ be a wavelet system. %n \hib{orthonormal} wavelet system.
        We want to design this system to have two properties:
    \begin{dingautolist}{"B6}
      \item The scaling functions $\seqn{\opT^n\fphi}$ are \hib{orthonormal}
            \xref{def:ows}.
      \item The wavelet function $\fpsi$ has $p$ \prope{vanishing moments} \xref{def:vanish}.
    \end{dingautolist}

  \item \textbf{Form of solution.} 
        \prefpp{lem:van_factor} demonstrates that the presence of the 
        factor $\brp{\frac{1+z^{-1}}{2}}^p$ 
        in the z transform $\Zh(z)$ of the scaling coefficient sequence $\seqn{h_n}$
        is a sufficient condition for $\fpsi$ to have $p$ vanishing moments.
        Therefore, we constrain $\Zh(z)$ to be of the following form:
        \[
          \Zh(z) = 
          \sqrt{2}\mcom{\left(\frac{1+z^{-1}}{2}\right)^p}
                       {$\begin{array}{M}
                           provides $p$\\ 
                           vanishing moments\\
                           \xref{lem:van_factor}
                         \end{array}$}
          \mcoml{\fQ(z)}{provides orthogonality}
        \]

        So we need to compute the factor $\fQ(z)$. 
        Once we have $\fQ(z)$, we can multiply the two factors, 
        and the scaling coefficient sequence $\seqn{h_n}$ will simply be the coefficients
        of the polynomial $\Zh(z)$:
        \[ \Zh(z) = h_0 + h_1 z^{-1} + h_2 z^{-2} + h_3 z^{-3} + \cdots + h_{2p-1}z^{-2p+1}\]

  \item \textbf{Computing $\fQ(z)$.}
    \begin{dingautolist}{"B6}
      %\item The orthogonal quadrature condition together with the requirement
      %      of $p$ vanishing moments can be represented in the $z$ domain as follows:
      %      \begin{align*}
      %        2
      %          &= \abs{\Fh(\omega)}^2  + \abs{\Fh(\omega+\pi)}^2
      %        \\&= \abs{\sum_n h_n e^{-i\omega n}}^2  + \abs{\sum_n h_n e^{-i(\omega+\pi)n}}^2 
      %        \\&= \brs{\abs{\sum_n h_n z^{-n}}^2  + \abs{\sum_n h_n z^{-n}}^2}_{z\eqd e^{i\omega}} 
      %        
      %      \end{align*}

      \item The frequency representation of the orthogonal quadrature condition 
            $\abs{\Fh(\omega)}^2  + \abs{\Fh(\omega+\pi)}^2 = 2$ 
            \xref{lem:oms_quadcon} is \prope{periodic} in $\omega$ with 
            period $2\pi$ and is even about $0$.
            Because it's \prope{periodic}, the quadrature condition can be expressed
            as a \ope{Fourier Series} \xref{def:fs}.
            Because it's \prope{even}, the quadrature condition can be expressed 
            using only cosine terms.
      
      \item Any \emph{harmonic} cosine representation of a function may alternatively be 
            expressed as a \thme{trigonometric expansion} 
            (\prefp{thm:cosnx} with examples in \prefp{ex:cos5} and \prefp{ex:cosn}).
            Therefore, the frequency orthogonal quadrature condition can be represented 
            as a polynomial in $\cos\omega$.
      
      \item As indicated by the trigonometric identity 
            $\sin^2\theta\equiv\frac{1}{2}\brp{1-\cos2\theta}$
            \ifdochas{harTrig}{\xref{thm:trig_sq}},
            any polynomial in $\cos\omega$ can alternatively be represented as 
            a polynomial in $\sin^2\frac{\omega}{2}$.
            Therefore, the frequency orthogonal quadrature condition can be represented 
            as a polynomial in $\sin^2\frac{\omega}{2}$.
      
      \item With $y$ defined as $y\eqd\sin^2\frac{\omega}{2}$,
            the orthogonal quadrature condition together with the $p$ vanishing moments
            constraint can be represented as \xref{lem:quadcon_y}
            \[(1-y)^p P(y) + y^p P(1-y) = 1 \quad \text{for some polynomial $P(y)$}.\]
            
      \item The forms $(1-y)^p P(y)$ and $y^p P(1-y)$ are also used in \ope{Hermite interpolation} \xref{sec:interpo_hermite}
            with the result of the endpoints at $y=0$ and $y=1$ being increasingly ``flatter" with increasing $p$.
            In particular, the first $p-1$ derivatives at the endpoints are $0$ \xref{thm:interpo_hermite}.
      
      \item There are an infinite number of solutions for $\fP(y)$.
            These solutions are of the form \xref{lem:gcd_yp}
        \[  \fP(y) \eqd \ds \fP_m(y) + y^p \fR\brp{\frac{1}{2}-y}
            \quad\text{where}\quad
            \brb{\begin{array}{>{\ds}lM}
              \fP_m(y)     \eqd \sum_{k=0}^{p-1} {p-1+k \choose k} y^k & and
              \\
              \mcom{\fR(y) =    -\fR(-y)}{$\fR$ is any \prope{odd} polynomial}
            \end{array}}
        \]
        Furthermore, $\fP(y)$ has minimum order when $\fR(y)=0$.

      \item Examples of \structe{odd polynomials} include the 
            \fncte{Chebyshev polynomials} \xref{def:Tn} $\fT_n(x)$ with $n$ \prope{odd} \xref{thm:chebyshev_evenodd}.
      
      \item \textbf{Spectral factorization.}
            The quantity $y$ can be expressed in terms of $z$ as \xref{lem:sfact_yz}
            \[ y=\brp{\frac{1-z}{2}}\,\brp{\frac{1-z^{-1}}{2}}.\]
            This implies that for each root of a polynomial in $y$,
            there are two roots in the equivalent polynomial in $z$, 
            and that these roots occur 
            in recipricol pairs (if $z$ is a root, then $z^{-1}$ is also a root).
            Therefore, any polynomial $\fP(y)$ in $y$ can be factored into two factors
            in $z$ such that $\fP(y)=\fQ(z)\fQ(z^{-1})$.
            Once $\fP(y)$ has been computed, we can then proceed to compute $\fQ(z)$
            %and factored into $\fQ(z)$ and $\fQ(z^{-1})$,
            %we can then compute the factor $\fQ(z)$.
            %To compute $\fQ(z)$, we can factor $\fP(y)$ into $\fQ(z)\fQ(z^{-1})$ 
            %(there are in general many ways to do this) and keep the factor $\fQ(z)$ 
            with help from \prefpp{thm:cs_bezout_prop}.
  \end{dingautolist}
\end{dingautolist}

%=======================================
%\section {Structure}
%=======================================
%--------------------------------------
\begin{lemma}
\label{lem:sfact_yz}
%--------------------------------------
\lembox{
  \brbr{\begin{array}{rc>{\ds}lD}
    y &\eqd& \sin^2\brp{\frac{\omega}{2}} & and \\
    z &\eqd& e^{i\omega}
  \end{array}}
  \implies
  \brbl{\begin{array}{rc>{\ds}l}
    y
      %&=& \sin^2\brp{\frac{\omega}{2}}
      &=& \frac{2-z-z^{-1}}{4}
      = \brp{\frac{1-z}{2}}\,\brp{\frac{1-z^{-1}}{2}}
      \\
    1-y
      &=& \cos^2\brp{\frac{\omega}{2}}
      \\&=& \frac{2+z+z^{-1}}{4}
      = \left(\frac{1+z}{2}\right)\left(\frac{1+z^{-1}}{2}\right)
      \\
    \frac{1}{2}-y
      &=& \frac{1}{2}\cos\omega
      \\
    z
      &=& 1 - 2y \pm \sqrt{y(y-1)}
  \end{array}}
  }
\end{lemma}
\begin{proof}
  \begin{align*}
    y
      &\eqd \sin^2\brp{\frac{\omega}{2}}
      && \text{by definition of $y$}
    \\&= \frac{1}{2}-\frac{1}{2}\cos(\omega)
      && \text{by \thme{half-angle formulas}}
      && \ifdochas{harTrig}{\text{\xref{thm:trig_sq}}}
    \\&= \frac{1}{2}-\frac{1}{2}\frac{e^{i\omega}+e^{-i\omega}}{2}
      && \text{by \thme{Euler formulas}}
      && \ifdochas{harTrig}{\text{\xref{cor:trig_ceesee}}}
    \\&= \frac{2}{4}-\frac{z+z^{-1}}{4}
      && \text{by definition of $z$}
    \\&= \frac{2-z-z^{-1}}{4}
    \\&= \brp{\frac{1-z}{2}}\,\brp{\frac{1-z^{-1}}{2}}
  \\ \\
    1-y
      &= 1-\sin^2\brp{\frac{\omega}{2}}
      && \text{by definition of $y$}
    \\&= \cos^2\brp{\frac{\omega}{2}}
      && \text{by \thme{half-angle formulas}}
      && \ifdochas{harTrig}{\text{\xref{thm:trig_sq}}}
    \\&= \frac{1}{2}\left(1+\cos\omega\right)
      && \text{by \thme{half-angle formulas}}
      && \ifdochas{harTrig}{\text{\xref{thm:trig_sq}}}
  \\ \\
    1-y
      &= 1-\frac{2-z-z^{-1}}{4}
      && \text{by previous result}
    \\&= \frac{4-2+z+z^{-1}}{4}
    \\&= \frac{2+z+z^{-1}}{4}
    \\&= \brp{\frac{1+z}{2}}\,\brp{\frac{1+z^{-1}}{2}}
  \\ \\
    \frac{1}{2}-y
      &= \frac{1}{2}-\sin^2\brp{\frac{\omega}{2}}
      && \text{by definition of $y$}
    \\&= \frac{1}{2}-\frac{1}{2}\left(1-\cos\omega\right)
      && \text{by \thme{half-angle formulas}}
      && \ifdochas{harTrig}{\text{\xref{thm:trig_sq}}}
    \\&= \frac{1}{2}\cos\omega
  \end{align*}
\end{proof}

\prefpp{lem:oms_quadcon} demonstrates that all orthogonal scaling functions satisfy
the \thme{orthonormal quadrature condition}
$\abs{\Fh(\omega)}^2  + \abs{\Fh(\omega+\pi)}^2 = 2$.
When designing orthogonal scaling functions with compact support,
it is very useful to be able to express this quadrature condition
as a polynomial in $\sin^2\frac{\omega}{2}$.
\pref{lem:quadcon_y} (next) does this by expressing the quadrature condition
as a polynomial in $y$ where $y=\sin^2\frac{\omega}{2}$.

%--------------------------------------
\begin{lemma}
\label{lem:quadcon_y}
%--------------------------------------
Let $\wavsys$ be a wavelet system.
Let $y\eqd\sin^2\brp{\frac{\omega}{2}}$ and $P(y)$ a polynomial in $y$.
\lembox{
  \brb{\begin{array}{FlD}
    1. & \mcom{\abs{\Fh(\omega)}^2 + \abs{\Fh(\omega+\pi)}^2 = 2} 
              {\prope{quadrature condition} in ``frequency"}
       & and
       \\
    2. & \text{$\fpsi$ has $p$ \prope{vanishing moments}}
  \end{array}}
  \iff
  \brb{\begin{array}{FlD}
    1. & \text{there exists $P(y)$ such that}\\
       & \abs{\Fh(\omega)}^2 = 2\brp{1-y}^p P(y) 
       & and
    \\
    2. & \mcom{(1-y)^p P(y) + y^p P(1-y) = 1}
              {\prope{quadrature condition} as polynomial in $y$}
  \end{array}}
  }
\end{lemma}
\begin{proof}
  \begin{enumerate}
  \item Because $\abs{\fQ(\omega)}^2$ is \prope{periodic}, it has a \ope{Fourier series} expansion \xref{def:fs}.
        Because it is real, it has no imaginary part.
        Therefore, $\abs{\fQ(\omega)}^2$ can be expressed as a harmonic cosine polynomial as
        \[ \abs{\fQ(\omega)}^2 = \sum_{n=0}^\infty a_n \cos\omega n\]
        \label{item:cmp_Qcos}

  \item Proof for $\implies$ part:
    \begin{align*}
      \abs{\Fh(\omega)}^2
        &= \abs{\sqrt{2}\left(\frac{1+e^{i\omega}}{2}\right)^p \fQ(\omega) }^2
        && \text{by \prefp{lem:van_factor}}
      \\&= 2\abs{\left(\frac{1+e^{i\omega}}{2}\right)^p}^2 \abs{\fQ(\omega)}^2
        && \ifdochas{numsys}{\text{by \prefp{thm:C_norm}}}
      \\&= 2\left(\frac{1+e^{i\omega}}{2}\right)^p \left(\frac{1+e^{-i\omega}}{2}\right)^p
           \abs{\fQ(\omega)}^2
        && \ifdochas{numsys}{\text{by \prefp{thm:C_abs}}}
      \\&= 2\left(\frac{1+e^{i\omega}}{2}\right)^p \left(\frac{1+e^{-i\omega}}{2}\right)^p
           \sum_{n=0}^\infty c_n e^{-i\omega n}
        && \text{by \pref{item:cmp_Qcos}}
      \\&= 2\left(\frac{1+e^{i\omega }+e^{-i\omega}+1}{4}\right)^p
           \sum_{n=0}^\infty a_n \cos\omega n
           %\left[ \sum_{n=0}^\infty a_n \cos\omega n + \cancelto{0}{i\sum_{n=0}^\infty b_n \sin\omega n} \right]
        && \text{by \pref{item:cmp_Qcos}}
      \\&= 2\left(\frac{2+2\cos \omega }{4}\right)^p
           \sum_{n=0}^\infty a_n \cos\omega n
      \\&= 2\left(\cos^2\frac{\omega}{2}\right)^p   P'\left(\cos\omega\right)
        && \text{by \thme{trigonometric expansion}}
        && \text{\xref{thm:cosnx}}
      \\&= 2\left(1-\sin^2\frac{\omega}{2}\right)^p P'\left(1-2\sin^2\frac{\omega}{2}\right)
        && \text{by \thme{half-angle formulas}}
        && \ifdochas{harTrig}{\text{\xref{thm:trig_sq}}}
      \\&= 2\brp{1-\sin^2\frac{\omega}{2}}^p P\brp{\sin^2\frac{\omega}{2}}
      \\&= 2(1-y)^p P(y)
    \\
    \\
      \abs{\Fh(\omega+\pi)}^2
        &= 2\left(1-\sin^2\frac{\omega+\pi}{2}\right)^p P\left(\sin^2\frac{\omega+\pi}{2}\right)
        && \text{by previous result}
      \\&= 2\left(1-\cos^2\frac{\omega}{2}\right)^p P\left(\cos^2\frac{\omega}{2}\right)
        && \ifdochas{harTrig}{\text{by \prefp{thm:trig_shift}}}
      \\&= 2\left(\sin^2\frac{\omega}{2}\right)^p   P\left(1-\sin^2\frac{\omega}{2}\right)
        && \text{by \thme{half-angle formulas}}
        && \ifdochas{harTrig}{\text{\xref{thm:trig_sq}}}
      \\&= 2y^p P(1-y)
        && \text{by definition of $y$}
    \\
    \\
      2 &= \abs{\Fh(\omega)}^2 + \abs{\Fh(\omega+\pi))}^2
        && \text{by left hypothesis}
           %\abs{\Fh(\omega+\pi) }^2
      \\&= (1-y)^p P(y) + y^p P(1-y)
        && \text{by previous results}
    \end{align*}

  \item Proof for $\impliedby$ part:
    \begin{align*}
      1 &= (1-y)^p P(y) + y^p P(1-y)
        && \text{by right hypothesis}
      \\&= \abs{\Fh(\omega)}^2 + \abs{\Fh(\omega+\pi))}^2
        && \text{by previous results}
    \end{align*}
\end{enumerate}
\end{proof}


In the polynomial quadrature condition,
what are the possible solutions of $P(y)$?
\pref{lem:gcd_yp} (next) answers this question.
%--------------------------------------
\begin{lemma}
\label{lem:gcd_yp}
\index{Euclid's Algorithm (Extended)}
\index{Greatest Common Divisor} \index{$\gcd$}
\index{Binomial Theorem}
\footnote{
  \citerp{dau}{171},
  \citerpp{pinsky2002}{331}{333},
  \citerpp{chui}{175}{176}
  }
%--------------------------------------
Let $\fP$, $\fQ$, and $\fR$ be polynomials over $\R$.
% and let $\gcd(\fP,\fQ)$ be the \hie{greatest common divisor} of $\fP$ and $\fQ$.
\lemboxt{
  $\ds
    \mcom{(1-y)^p \fP(y) + y^p \fP(1-y) = 1}
         {quadrature condition as polynomial in $y$}
  \quad\iff\quad
  \left\{\begin{array}{FlD}
    1. & \fP(y) \eqd \ds \fP_m(y) + y^p \fR\brp{\frac{1}{2}-y} & where  \\
       & \fP_m(y) \eqd \ds \sum_{k=0}^{p-1} {p-1+k \choose k} y^k & and \\
    2. & \mcom{\fR(y) = -\fR(-y)}
              {$\fR$ is any {\bf odd} polynomial.}
  \end{array}\right.
  $
  \\Furthermore, $\fP(y)$ has minimum order when $\fR(y)=0$.
  }
\end{lemma}
\begin{proof}
  \begin{enumerate}
    \item Proof that 
      $(1-y)^p \fP(y) + y^p \fP(1-y) = 1 \implies \fP(y)=\fP_m(y) + y^p \fR\brp{\frac{1}{2}-y}$:
      \begin{enumerate}
        \item lemma: proof that 
          $\frac{1}{(1-y)^p}=\sum_{k=0}^{\infty} {p-1+k \choose k} y^k$:
          \label{item:gcd_yp_1yp}
          \begin{align*}
            \frac{1}{(1-y)^p}
              &= \sum_{k=0}^{\infty} \frac{1}{k!}
                 \brs{\deriv{^k}{y^k}\frac{1}{(1-y)^p}}_{y=0} \; y^k
               \qquad\text{by \thme{Maclaurin series} \xref{thm:taylor}}
            \\&= \sum_{k=0}^{\infty} \frac{1}{k!}
                 \brs{\deriv{^k}{y^k}(1-y)^{-p}}_{y=0} \; y^k
            \\&= \sum_{k=0}^{\infty} \frac{1}{k!}
                 \brs{(-p)(-1)\deriv{^{k-1}}{y^{k-1}}(1-y)^{-p-1}}_{y=0} \; y^k
            \\&= \sum_{k=0}^{\infty} \frac{1}{k!}
                 \brs{(p)(-p-1)(-1)\deriv{^{k-2}}{y^{k-2}}(1-y)^{-p-2}}_{y=0} \; y^k
            \\&= \sum_{k=0}^{\infty} \frac{1}{k!}
                 \brs{(p)(p+1)(p+2)\deriv{^{k-3}}{y^{k-3}}(1-y)^{-p-3}}_{y=0} \; y^k
            \\&\vdots
            \\&= \sum_{k=0}^{\infty} \frac{1}{k!}
                 \brs{(p)(p+1)(p+2)\cdots(p+k-2)\deriv{}{y}(1-y)^{-p-(k-1)}}_{y=0} \; y^k
            \\&= \sum_{k=0}^{\infty} \frac{1}{k!}
                 \brs{(p)(p+1)(p+2)\cdots(p+k-2)(p+k-1)(1-y)^{-p-k}}_{y=0} \; y^k
            \\&= \sum_{k=0}^{\infty} \frac{1}{k!}
                 (p)(p+1)(p+2)\cdots(p+k-2)(p+k-1) \; y^k
            \\&= \sum_{k=0}^{\infty} \frac{(p+k-1)!}{k!(p-1)!} \; y^k
            \\&= \sum_{k=0}^{\infty} \frac{(p+k-1)!}{k!(p+k-1-k)!} \; y^k
            \\&= \sum_{k=0}^{\infty} {p+k-1 \choose k} \; y^k
                 \qquad\text{by definition of ${n \choose k}$ \xref{def:bcoef}}
          \end{align*}

        \item There are an infinite number of solutions for $\fP(y)$. 
              These solutions are here expressed as
              \[ \fP(y) 
                   \eqd \mcom{\fP_m(y)}{order $p-1$} 
                   +    \mcom{y^p \fR\brp{\frac{1}{2}-y}}{order $\ge p$}
              \]
           
        \item \label{ilem:gcd_yp}
              Note that $(1-y)^p$ and $y^p$ are \hie{relatively prime} and are both of order $p$.
              Therefore by \thme{B\'ezout's Theorem} \xref{thm:bezout},
              there exists a unique solution for $\fP(y)$ of order $p-1$.
              This unique solution is here labeled $\fP_m(y)$.
              The more general solution (of order $\ge p$) is here expressed as
              $\fP_m(y) + y^p\fR\brp{\frac{1}{2}-y}$.
              \[ \mcom{(1-y)^p}{order $p$} 
                 \mcom{\fP_m(y)}{order $p-1$} + 
                 \mcom{y^p}{order $p$} 
                 \mcom{\fP_m(1-y)}{order $p-1$} 
                 = 1
              \]

        \item We can compute the unique order $p-1$ solution $\fP_m(y)$ of $\fP(y)$ by using 
              a Maclaurin expansion of $\fP(y)$. 
              Let $\opT\fP(y)$ represent a polynomial $\fP(y)$ with all terms of 
              order $\ge p$ truncated.
          \begin{align*}
            \mcom{\fP_m(y)}{order $p-1$}
              &= \frac{1}{(1-y)^p} - \frac{y^p \fP_m(1-y)}{(1-y)^p}
              %&& \text{by $(1-y)^p \fP_m(y) + y^p \fP_m(1-y) = 1$}
              && \text{by \pref{ilem:gcd_yp}}
            \\&= \opT\brb{
                   \frac{1}{(1-y)^p} -
                   \mcoml{\cancelto{0}{\frac{y^p \fP_m(1-y)}{(1-y)^p}}}{all terms order $\ge p$}
                   }
              && \text{by \thme{B\'ezout's Theorem}}
              && \text{\xref{thm:bezout}}
            \\&= \opT\brb{\frac{1}{(1-y)^p}}
              && \text{by definition of $\opTrn$}
              && \text{\xref{def:opT}}
            \\&= \opT\brb{\sum_{k=0}^{\infty} {p+k-1 \choose k} \; y^k}
              && \text{by \prefp{item:gcd_yp_1yp}}
            \\&= \sum_{k=0}^{p-1} {p+k-1 \choose k} \; y^k
              && \text{by definition of $\opTrn$}
              && \text{\xref{def:opT}}
          \end{align*}
    
        \item Alternatively, specific cases of $\fP_m(y)$ can also be calculated 
              using the \hie{Extended Euclid's Algorithm}:
          \begin{dingautolist}{"AC}
            \item $p=2$ case:  See \prefpp{ex:eea_n=2}.
            \item $p=3$ case:  See \prefpp{ex:eea_n=3}.
            \item $p=4$ case:  See \prefpp{ex:eea_n=4}.
          \end{dingautolist}
    
        \item Proof that $\fR(y)$ must be an odd polynomial:
          \begin{align*}
            &(1-y)^p \fP(y) + y^p \fP(1-y)
            \\&= (1-y)^p \left[ \fP_m(y)   + y^p     \fR\brp{\frac{1}{2}-y} \right] +
                     y^p \left[ \fP_m(1-y) + (1-y)^p \fR\brp{\frac{1}{2}-[1-y]} \right]
            \\&= (1-y)^p \fP_m(y)   + (1-y)^p y^p \fR\brp{\frac{1}{2}-y} +
                     y^p \fP_m(1-y) + y^p (1-y)^p \fR\brp{-\frac{1}{2}+y}
            \\&= \mcom{\Bigg[(1-y)^p \fP_m(y) + y^p \fP_m(1-y) \Bigg]}{$1$}  +
                 \Bigg[(1-y)^p y^p \fR\brp{\frac{1}{2}-y} +
                        y^p (1-y)^p \fR\brp{y-\frac{1}{2}}
                 \Bigg]
            \\&= 1+ (1-y)^p y^p \brs{\fR\brp{\frac{1}{2}-y} + \fR\brp{y-\frac{1}{2}}}
            \\&= 1 
            \\&\iff \brs{\fR\brp{\frac{1}{2}-y} +\fR\brp{y-\frac{1}{2}}=0} 
            \\&\iff \fR(y)=-\fR(-y) \qquad\text{($\fR(y)$ is odd)}
          \end{align*}
    \end{enumerate}

    \item Proof that 
        $(1-y)^p \fP(y) + y^p \fP(1-y) = 1 \impliedby \fP(y)=\fP_m(y) + y^p \fR\brp{\frac{1}{2}-y}$:
      \begin{enumerate}
        \item Examples of specific cases with $\fR(y)=0$:
          \begin{align*}
            \intertext{$p=1$ case:}
              &(1-y)^1\fP_m(y) + y^1\fP_m(1-y)
              \\&\quad= (1-y)^1\sum_{k=0}^{k=1-1}{1+k-1\choose k} y^k  + y^1\sum_{k=0}^{k=1-1}{1+k-1\choose k} (1-y)^k 
              \\&\quad= (1-y)^1(1) + y^1(1)
              \\&\quad= 1
            \intertext{$p=2$ case:}
              &(1-y)^2\fP_m(y) + y^2\fP_m(1-y)
              \\&\quad= (1-y)^2\sum_{k=0}^{k=2-1}{2+k-1\choose k} y^k  + y^2\sum_{k=0}^{k=2-1}{2+k-1\choose k} (1-y)^k 
              \\&\quad= (1-y)^2(1+2y) + y^2\brs{1+2(1-y)}
              \\&\quad= (1-2y+y^2)(1+2y) + y^2(3-2y)
              \\&\quad= \mcom{1-2y+y^2+2y-4y^2+2y^3}{$(1-y)^p\fP_m(y)$} 
                      + \mcom{3y^2-2y^3}            {$y^p\fP_m(1-y)  $}
              \\&\quad= \mcom{1+0y-3y^2+2y^3}       {$(1-y)^p\fP_m(y)$} 
                      + \mcom{3y^2-2y^3}            {$y^p\fP_m(1-y)  $}
              \\&\quad= 1
            \intertext{$p=3$ case:}
              &(1-y)^3\fP_m(y) + y^3\fP_m(1-y)
              \\&\quad= (1-y)^3\sum_{k=0}^{k=3-1}{3+k-1\choose k} y^k  + y^3\sum_{k=0}^{k=3-1}{3+k-1\choose k} (1-y)^k 
              \\&\quad= (1-y)^3(1+3y+6y^2) + y^3\brs{1+3(1-y)+6(1-y)^2}
              \\&\quad= \mcom{(1-3y+3y^2-y^3)(1+3y+6y^2)}{$(1-y)^p\fP_m(y)$} 
                      + \mcom{y^3\brs{4-3y+6(1-2y+y^2)}} {$y^p\fP_m(1-y)$}
              \\&\quad= \mcom{1-3y+3y^2-y^3 + 3y-9y^2+9y^3-3y^4 + 6y^2-18y^3+18y^4-6y^5}{$(1-y)^p\fP_m(y)$}
                      \\&\qquad+ \mcom{y^3\brs{10-15y+6y^2}}      {$y^p\fP_m(1-y)$}
              \\&\quad= \mcom{1 + 0y + 0y^2 -10y^3 + 15y^4 - 6y^5}{$(1-y)^p\fP_m(y)$}
                      + \mcom{10y^3 - 15y^4 + 6y^5}      {$y^p\fP_m(1-y)$}
              \\&\quad= 1
          \end{align*}

        \item Proof for $\fR(y)=0$ case (using induction):\label{item:gcd_yp_induction}
              \footnote{Many thanks to Chip Eastham
		      %\href{http://groups.google.com/groups/profile?enc_user=93XVtBIAAAAXdlHgIIrPQxc2rOftKaPs8rhlH0Pnl47z4AZhN98BFg}{Chip Eastham} 
              for his extremely valuable help with this proof.
              %\url{http://groups.google.com/group/sci.math/browse_thread/thread/2547e294b7e300de}
              }
          \begin{enumerate}
            \item Define $\fP_p(y)\eqd \sum_{k=0}^{p} {p+k \choose k}y^k$.

            \item lemma: proof that $(1-y)\fP_p(y)=\fP_{p-1}(y) + {2p-1 \choose p}y^p(1-2y)$: \label{item:gcd_yp_1-y}
              \begin{align*}
                &(1-y)\fP_p(y)
                \\&= \fP_p(y) - y\fP_p(y)
                \\&= \sum_{k=0}^{p} {p+k \choose k} \; y^k 
                   - \sum_{k=0}^{p} {p+k \choose k} \; y^{k+1}
                \\&= \mcom{1+\sum_{k=1}^{p} {p+k \choose k} \; y^k}
                          {$\sum_{k=0}^{p} {p+k \choose k} \; y^k $}
                   \mcom{- \sum_{k=1}^{p} {p+k-1 \choose k-1} \; y^{k} -{2p\choose p}y^p}
                        {$- \sum_{k=0}^{p} {p+k \choose k} \; y^{k+1}$} 
                \\&= 1+\sum_{k=1}^{p} \brs{{p+k \choose k}-{p+k-1 \choose k-1}} \; y^k
                   - {2p\choose p}y^{p+1}
                \\&= 1+\sum_{k=1}^{p} \brs{\mcom{{p+k-1 \choose k}+{p+k-1 \choose k-1}}{${p+k \choose k}$}-{p+k-1 \choose k-1}} \; y^k
                   - {2p\choose p}y^{p+1}
                  \\&\qquad \text{by Pascal's Rule \prefpo{thm:Pascals_Rule}}
                \\&= 1+\sum_{k=1}^{p} {p+k-1 \choose k} \; y^k - {2p\choose p}y^{p+1}
                \\&= \sum_{k=0}^{p} {p+k-1 \choose k} \; y^k - {2p\choose p}y^{p+1}
                \\&= \sum_{k=0}^{p-1} {p+k-1 \choose k} \; y^k 
                   + {2p-1\choose p}y^p- {2p\choose p}y^{p+1}
                \\&= \fP_{p-1}(y) + {2p-1\choose p}y^p - \frac{(2p)!}{p!p!} y^{p+1}
                  \qquad\text{by definition of $\fP_p(y)$}
                \\&= \fP_{p-1}(y) + {2p-1\choose p}y^p - \frac{2p}{p}\frac{(2p-1)!}{p!(p-1)!} y^{p+1}
                \\&= \fP_{p-1}(y) + {2p-1\choose p}y^p - 2{2p-1\choose p}y^{p+1}
                \\&= \fP_{p-1}(y) + {2p-1 \choose p}y^p(1-2y)
              \end{align*}

            \item lemma: proof that $y\fP_p(1-y)=\fP_{p-1}(1-y) - {2p-1 \choose p}(1-y)^p(1-2y)$: \label{item:gcd_yp_y}
              \begin{align*}
                y\fP_p(1-y)
                  &= \brs{(1-u)\fP_p(u)}_{u\eqd 1-y}
                \\&= \brs{(1-u)\fP_p(u)}_{u\eqd 1-y}
                \\&= \brs{\fP_{p-1}(u) + {2p-1 \choose p}u^p(1-2u)}_{u\eqd 1-y}
                  && \text{by previous lemma}
                \\&= \fP_{p-1}(1-y) + {2p-1 \choose p}(1-y)^p(-1+2y)
                \\&= \fP_{p-1}(1-y) - {2p-1 \choose p}(1-y)^p(1-2y)
              \end{align*}

            \item Proof that $p-1$ case $\implies$ $p$ case:
              \begin{align*}
                &(1-y)^{p+1} \fP_{p}(y) + y^{p+1} \fP_{p}(1-y)
                \\&= (1-y)^{p}(1-y) \fP_{p}(y) + y^{p}y \fP_{p}(1-y)
                \\&= (1-y)^{p}\mcom{\brs{\fP_{p-1}(y) + {2p-1 \choose p}y^p(1-2y)}}{by \prefp{item:gcd_yp_1-y}} 
                   +  y^{p}   \mcom{\brs{\fP_{p-1}(1-y) - {2p-1 \choose p}(1-y)^p(1-2y)}}{by \prefp{item:gcd_yp_y}}
                  %&& \text{by previous two lemmas}
                \\&= \mcom{(1-y)^{p}\fP_{p-1}(y)+y^{p}\fP_{p-1}(1-y)}{$1$}
                   + {2p-1 \choose p}(1-2y)\mcom{\brs{(1-y)^py^p -y^p(1-y)^p}}{$0$}
                \\&= 1
                  \qquad \text{by induction hypothesis}
              \end{align*}

            \item Therefore by induction, 
                  $(1-y)^{p} \fP_{p-1}(y) + y^{p} \fP_{p-1}(1-y)=1$
                  is true for all $p\in\Zp$.
          \end{enumerate}

        \item Proof for $\fR(y)\ne 0$ case:
          \begin{align*}
            &(1-y)^p \fP(y) + y^p \fP(1-y)
            \\&= (1-y)^p \left[ \fP_{p-1}(y)   + y^p     \fR\left(\frac{1}{2}-y\right) \right] +
                     y^p \left[ \fP_{p-1}(1-y) + (1-y)^p \fR\left(\frac{1}{2}-[1-y]\right) \right]
            \\&= (1-y)^p \fP_{p-1}(y)   + (1-y)^p y^p \fR\left(\frac{1}{2}-y\right) +
                     y^p \fP_{p-1}(1-y) + y^p (1-y)^p \fR\left(-\frac{1}{2}+y\right)
            \\&= \Bigg[(1-y)^p \fP_{p-1}(y) + y^p \fP_{p-1}(1-y) \Bigg]  +
                 \mcom{\Bigg[(1-y)^p y^p \fR\left(\frac{1}{2}-y\right) +
                      - y^p (1-y)^p \fR\left(y-\frac{1}{2}\right)
                 \Bigg]}{$0$ because $\fR(y)$ is an odd polynomial}
              && \text{}
            \\&= (1-y)^p \fP_{p-1}(y) + y^p \fP_{p-1}(1-y)
            \\&= 1
              \qquad \text{by \prefp{item:gcd_yp_induction}}
          \end{align*}
    \end{enumerate}
  \end{enumerate}

\end{proof}




%---------------------------------------
\begin{example}
\footnote{
  \citerp{pinsky2002}{333}
  }
\label{tbl:Pm(y)}
%---------------------------------------
Here are some examples of the minimal polynomial $\fP_m(y)$ for varying number 
of vanishing moments $p$:
\footnotesize
\exbox{
\renewcommand{\arraystretch}{1}
\renewcommand{\arraycolsep}{0.4ex}
\begin{array}{r|@{\hs{2ex}}*{20}{r}}
   p & \mc{19}{l}{\fP_m(y)= \sum_{k=0}^{p-1} {p-1+k\choose k} y^k}
\\ \hline
   1 & 1
\\ 2 & 1 &+&  2y
\\ 3 & 1 &+&  3y &+&  6y^2
\\ 4 & 1 &+&  4y &+& 10y^2 &+& 20y^3
\\ 5 & 1 &+&  5y &+& 15y^2 &+& 35y^3 &+&  70y^4
\\ 6 & 1 &+&  6y &+& 21y^2 &+& 56y^3 &+& 126y^4 &+& 252y^5
\\ 7 & 1 &+&  7y &+& 28y^2 &+& 84y^3 &+& 210y^4 &+& 462y^5 &+&  924y^6
\\ 8 & 1 &+&  8y &+& 36y^2 &+&120y^3 &+& 330y^4 &+& 792y^5 &+& 1716y^6 &+& 3432y^7
\\ 9 & 1 &+&  9y &+& 45y^2 &+&165y^3 &+& 495y^4 &+&1287y^5 &+& 3003y^6 &+& 6435y^7 &+& 12870y^8
\\10 & 1 &+& 10y &+& 55y^2 &+&220y^3 &+& 715y^4 &+&2002y^5 &+& 5005y^6 &+&11440y^7 &+& 24310y^8 &+& 48620y^9
\end{array}}
\end{example}

%---------------------------------------
\begin{example}
\label{tbl:Pyz}
%---------------------------------------
Here are some examples of the minimal polynomial $\fP_m(y)$ evaluated at
$y=[(-z+2-1/z)/4]$:
\exbox{\begin{array}{l|l}
   p & \fP_m\left(\frac{z-2+z^{-1}}{-4}\right)
\\ \hline
   1 & 1
\\ 2 & \frac{ z^2 - 4z - 1}{ -2z}
\\ 3 & \frac{ 3z^4 -18z^3 +38z^2 -18z + 3}{ 2^3z^2}
\\ 4 & \frac{ 5z^6 -40z^5 + 131z^4 -208z^3 + 131z^2 -40z + 5}{ -2^4z^3}
\\ 5 & \frac{ 35z^8 -350z^7 + 1520z^6 -3650z^5 + 5018z^4 -3650z^3 + 1520z^2 -350z + 35}{ 2^7z^4}
\\ 6 & \frac{ 63z^{10} -756z^9 +4067z^8 -12768z^7 +25374z^6 -32216z^5 +25374z^4 -12768z^3 +4067z^2 -756z +63}{ -2^8z^5}
%\\ 7 & \frac{231z^{12}-3234z^{11}+20706z^{10}-79674z^9+203161z^8-356132z^7+430908z^6-356132z^5+203161z^4-79674z^3+20706z^2-3234z+231}{2^{10}z^6}
%\\ 8 & \frac{429z^{14}-6864z^{13}+50919z^{12}-231264z^{11}+714429z^{10}-1575216z^9+2528431z^8-2963776z^7+2528431z^6-1575216z^5+714429z^4-231264z^3+50919z^2-6864z+429}{-2^{11}z^7}
\end{array}}
\end{example}




\pref{thm:cs_bezout_prop} (next)
gives a method for computing scaling functions with compact support
and $p$ vanishing moments.
This method results in an infinite number of possible solutions.
For each value $p$, there are exactly
$2^\floor{p/2}$ solutions that are of minimum order
$2p-1$.
Of these $2^\floor{p/2}$ solutions, there is one solution that
is also {\em minimum phase}.
This is the Daubechies-$p$ scaling function.
All the $2^\floor{p/2}$ solutions occur when
$\fR(y)=0$.
An infinite number of other solutions are also available when
$\fR(y)\ne 0$.

%---------------------------------------
\begin{theorem}
\label{thm:cs_bezout_prop}
\index{vanishing moments}
\index{support}
%---------------------------------------
Let $\wavsys$ be an \hib{orthonormal} wavelet system.
\thmboxp{
\begin{enumerate}
\item If $\fpsi$ has $p$ vanishing moments, then
      $\seq{h_n}{n\in\Z}$ must be of a form that satisfies
\[\begin{array}{lrclcl}
  i) & \ds \sum_{n} h_n z^{-n}
     &=& \ds \sqrt{2}\left(\frac{1+z^{-1}}{2}\right)^p \fQ(z)
  \\
  ii) & \fQ(z)\fQ(z^{-1})
    %&=& \left.P(y)\right|_{y=\frac{2-z-z^{-1}}{4}}
     &=& \ds \left.\left[\sum_{k=0}^{p-1} {{p-1+k}\choose{k}} y^k + y^p R\left(\frac{1}{2}-y\right)\right]\right|_{y=\frac{2-z-z^{-1}}{4}}
\end{array}\]

\item The support of $\seq{h_n}{}$ is minimized when $R(y)=0$.

\item If $\fpsi$ has $p$ vanishing moments, then
      the minimum support size of $\seq{h_n}{}$ is $2p-1$ ($2p$ non-zero elements).

\item If $\seq{h_n}{}$ has support size $2p-1$ ($2p$ non-zero coefficients)
      then $\fpsi$ has maximum of $p$ vanishing moments.

\end{enumerate}
}
\end{theorem}
\begin{proof}
\begin{enumerate}
\item Proof for the form of $\seq{h_n}{}$:
\begin{enumerate}
  \item For $y\eqd\sin^2\frac{\omega}{2}$
    \[
      \mcom{\abs{\Fh(\omega)}^2 + \abs{\Fh(\omega+\pi)}^2 = 2}
           {quadrature condition in frequency domain}
      \qquad\iff\qquad
      \mcom{(1-y)^p \fQdd(y) + y^p \fQdd(1-y) = 1}
           {quadrature condition as a polynomial in $y$}
    \]

\item The polynomial $\fQdd(\omega)$ is non-negative and therefore
      can be factorized into two factors:
\begin{align*}
  \left.\fQdd(y)\right|_{y=\frac{2-z-z^{-1}}{4}}
    &= \fQ\Big(z\Big) \fQ\Big(\frac{1}{z^\ast}\Big)
    && \text{by \thme{Fej\'er-Riesz factorization} \xref{thm:Fejer-Riesz}}
\end{align*}

\item
\begin{align*}
  |\Zh(z)|^2
    &= 2\left(\frac{1+z^{-1}}{2}\right)^p \fQdd(z)
    &= 2\left(\frac{1+z^{-1}}{2}\right)^p \fQ\Big(z\Big) \fQ\Big(\frac{1}{z^\ast}\Big)
  \\
  \Zh(z)
    &= \sqrt{2}\left(\frac{1+z^{-1}}{2}\right)^p \fQ(z)
\end{align*}




\end{enumerate}


\item Proof that the support of $\seq{h_n}{}$ is minimized when $R(y)=0$:\\
      This follows directly from \thme{Bezout's Theorem} \xref{lem:bezout}.
\begin{align*}
  Q\Big(z\Big)Q\Big(\frac{1}{z^\ast}\Big)
    &= \fQdd(y)\Big|_{y=\frac{2-z-z^{-1}}{4}}
    %&& \text{by \thme{Fej\'er-Riesz factorization} \xref{thm:Fejer-Riesz}}
    && \text{by \prefp{thm:Fejer-Riesz}}
  \\&= \left[
       \sum_{k=0}^{p-1} {{p-1+k}\choose{k}} y^k + y^p \cancelto{0}{R\left(\frac{1}{2}-y\right)}
       \quad\right]_{y=\frac{2-z-z^{-1}}{4}}
    && \text{by \prefp{lem:gcd_yp}}
  \\&= \sum_{k=0}^{p-1} {{p-1+k}\choose{k}} \left(\frac{2-z-z^{-1}}{4}\right)^k
    && \text{by \prefp{lem:gcd_yp}}
\end{align*}


\item Proof that the minimum support size of $\seq{h_n}{}$ is $2p-1$ ($2p$ non-zero elements).
\begin{align*}
  \mcom{\fQdd(y)\Big|_{y=\frac{2-z-z^{-1}}{4}}}{order $p-1$ in $y$}
    &= \mcom{\fQdd\left(\frac{2-z-z^{-1}}{4}\right)}{order $2p-2$ in $z$}
  \\&= \mcom{\fQ\Big(z\Big)}{order $p-1$}
       \mcom{\fQ\Big(\frac{1}{z^\ast}\Big)}{order $p-1$}
    && \text{by \thme{Fej\'er-Riesz factorization} \xref{thm:Fejer-Riesz}}
\\
  \mcom{\Zh(z)}{order $2p-1$}
    &= \mcom{\sqrt{2}\left(\frac{z+1}{2}\right)^p}{order $p$}
       \mcom{Q(z)}{order $p-1$}
\end{align*}
\end{enumerate}
\end{proof}






%=======================================
%\section{Minimum phase constraint}
%\label{sec:dau-p}
%=======================================
%=======================================
\section{Design details}
\label{sec:dau-p}
%=======================================
%---------------------------------------
\begin{definition}
\label{def:Dp}
%---------------------------------------
%Let $\spO\eqd\wavsys$ be a \structe{wavelet system} \xref{def:wavsys}.
Let $\Zh(z)$ be the \ope{z-transform} \xref{def:opZ} of a sequence $\seqnZ{h_n}$.
Let $Q(z)$ be a polynomial with real coefficients and
\begin{align*}
  y &\eqd \sin^2\brp{\frac{\omega}{2}}
  \\
  P(y)
  &\eqd \sum\limits_{k=0}^{p-1} {{p-1+k}\choose{k}} y^k
  \\
  \dot{Q}
  &\eqd \set{Q(z)}{ Q(z)Q(z^{-1})=P\left( \frac{2-z-z^{-1}}{4}} \right)
  &&    \text{(\thme{Fej\'er-Riesz spectral factorization}s of $P$)}
  \\
  \opR Q
  &\eqd \set{z_n}{\text{$z_n$ is a zero of $Q(z)$}}
  &&    \text{(roots of $Q(z)$)}
\end{align*}
\defboxt{
  $\wavsys$ is a \hid{Daubechies-p wavelet system} if
  \\\indentx$\begin{array}{FMD}
    1. & $\ds\Zh(z) = \sqrt{2}\brp{\frac{1+z^{-1}}{2}}^p Q(z)$ & and
    \\
    2. & $Q(z)$ is the polynomial in $\dot{Q}$ $\st$ $\forall z_n\in\opR Q,\; |z_n|<1$ & and 
    \\
    3. & $g_n  = (-1)^n h_{2p-1-n}$
  \end{array}$
  \\
  $\wavsys$ is also called a \hid{$D_p$ wavelet system}.
  }
\end{definition}


\setlength{\unitlength}{0.10mm}
The Daubechies-$p$ wavelets of \prefp{def:Dp} can be calculated
\footnote{For an actual implementation using \hie{Octave}, 
  see \prefpp{sec:src_Ry}.}
by the following steps:
\begin{enumerate}
  \item Compute the polynomial $P(y)$. This polynomial has $p-1$ roots in $y$.
        \[
          P(y) \eqd \sum\limits_{k=0}^{p-1} {{p-1+k}\choose{k}} y^k
         \hspace{3em}\mbox{where }
         {{n}\choose{k}}\eqd\frac{n!}{k!(n-k)!}
        \]

  \item Compute $P\left(\frac{2-z-z^{-1}}{4}\right)$.
        This polynomial has $2p-2$ roots in $z$.
        {\center %============================================================================
% LaTeX File
% Daniel J. Greenhoe
%
% Daubechies-8 Pole Zero plot
%============================================================================
\begin{picture}(500,470)(-130,-200)
  \thicklines%
  \color{axis}%
    \put(-130,   0){\line(1,0){500} }%
    \put(   0,-130){\line(0,1){260} }%
    \put( 380,   0){\makebox(0,0)[l]{$\Reb{z}$}}%
    \put(   0, 140){\makebox(0,0)[b]{$\Imb{z}$}}%
  \color{circle}%
    \qbezier( 100,   0)( 100, 41.421356)(+70.710678,+70.710678)% % 0   -->1pi/4
    \qbezier(   0, 100)( 41.421356, 100)(+70.710678,+70.710678)% % pi/4-->2pi/4
    \qbezier(   0, 100)(-41.421356, 100)(-70.710678,+70.710678)% %2pi/4-->3pi/4
    \qbezier(-100,   0)(-100, 41.421356)(-70.710678,+70.710678)% %3pi/4--> pi
    \qbezier(-100,   0)(-100,-41.421356)(-70.710678,-70.710678)% % pi  -->5pi/4
    \qbezier(   0,-100)(-41.421356,-100)(-70.710678,-70.710678)% %5pi/4-->6pi/4
    \qbezier(   0,-100)( 41.421356,-100)( 70.710678,-70.710678)% %6pi/4-->7pi/4
    \qbezier( 100,   0)( 100,-41.421356)( 70.710678,-70.710678)% %7pi/4-->2pi
    %\put( 110, 110){\makebox(0,0)[lb]{$z=e^{i\omega}$}}%
    %\put( 105, 105){\vector(-1,-1){33}}%
  \color{pzasym}%
    \qbezier(41.4214,0)(41.4214, 58.5786)(0, 100)%      % inner upper arc
    \qbezier(41.4214,0)(41.4214,-58.5786)(0,-100)%      % inner lower arc
    \qbezier(0, 100)(41.421, 141.421)(100, 141.421)%    %outer upper arc
    \qbezier(100, 141.421)(158.579,141.421)(200,100)%
    \qbezier(200,100)(241.421,58.5786)(241.421,0)%
    \qbezier(0,-100)(41.421,-141.421)(100,-141.421)%outer lower arc
    \qbezier(100,-141.421)(158.579,-141.421)(200,-100)%
    \qbezier(200,-100)(241.421,-58.5786)(241.421,0)%
  \color{pole}%
   %\put(-100,-100){\makebox(0,0)[tr]{15 poles}}%
   %\put(-100,-100){\vector(1,1){91}}%
   %\put(   0,    0){\makebox(0,0)[c]{$\times$}}%
   %\put(   0,    0){\makebox(0,0)[c]{\hspace{1em}$^{15}$}}%
  \color{zero}%
   %\put(-150, -50){\makebox(0,0)[tr]{p=8 zeros}}%
   %\put(-150, -50){\vector( 1, 1){43}}%
   %\put(-100,    0){\circle{15}$^8$}%
    \qbezier[30](0,0)(51.90, 86.52)(+103.79714430713773 ,  +173.04352168829922 )%
    \qbezier[30](0,0)(51.90,-86.52)(+103.79714430713773 ,  -173.04352168829922 )%
    \qbezier[30](0,0)(96.94, 72.79)(+193.88494686323963 ,  +145.58242201373777 )%
    \qbezier[30](0,0)(96.94,-72.79)(+193.88494686323963 ,  -145.58242201373777 )%
    \qbezier[30](0,0)(126.48, 40.99)(+252.96496127413031 , + 81.97646490830808 )%
    \qbezier[30](0,0)(126.48,-40.99)(+252.96496127413031 , - 81.97646490830808 )%
    \put( 0,- 80){\dashbox{5}(60,160){}}%
    \put(-10, 30){\makebox(0,0)[rb]{$p-1$ zeros $\rightarrow$ }}%
    \put(+36.54035130742002 ,  0                 ) {\circle{15} }%
    \put(+35.77427639711839 , +11.59310245530727 ) {\circle{15} }%
    \put(+35.77427639711839 , -11.59310245530727 ) {\circle{15} }%
    \put(+32.98169959381100 , +24.76497421117944 ) {\circle{15} }%
    \put(+32.98169959381100 , -24.76497421117944 ) {\circle{15} }%
    \put(+25.49176775413241 , +42.49813706999120 ) {\circle{15} }%
    \put(+25.49176775413241 , -42.49813706999120 ) {\circle{15} }%
  \color{red}%
    \put(80,-200){\dashbox{5}(230,400){}}%
    \put(340, 30){\makebox(0,0)[lb]{$\leftarrow$ $p-1$ zeros}}%
    \put(+273.67005631344137 ,  0                  ){ \circle{15} }%
    \put(+103.79714430713773 , +173.04352168829922 ){ \circle{15} }%
    \put(+103.79714430713773 , -173.04352168829922 ){ \circle{15} }%
    \put(+193.88494686323963 , +145.58242201373777 ){ \circle{15} }%
    \put(+193.88494686323963 , -145.58242201373777 ){ \circle{15} }%
    \put(+252.96496127413031 , +081.97646490830808 ){ \circle{15} }%
    \put(+252.96496127413031 , -081.97646490830808 ){ \circle{15} }%
\end{picture}

 }

  \begin{enumerate}
    \item Because the coefficients of
          $P\brp{\frac{2-z-z^{-1}}{4}}$
          are real, all of its roots occur in \hie{complex conjugate pairs}:
          \[ \text{root at } z_1=re^{i\theta}
             \iff
             \text{root at }
             z_1^\ast = \left(re^{i\theta}\right)^\ast
                      = re^{-i\theta}
          \]
          \begin{center}
          %\scriptsize
          \setlength{\unitlength}{0.15mm}
          \begin{picture}(300,300)(-130,-130)
            \thicklines
            \color{axis}%  
              \put(-130,   0){\line(1,0){260} }%
              \put(   0,-130){\line(0,1){260} }%
              \put( 140,   0){\makebox(0,0)[l]{$\Reb{z}$}}%
              \put(   0, 140){\makebox(0,0)[b]{$\Imb{z}$}}%
              \qbezier[30](0,0)(62.5, 62.5)(125, 125)%
              \qbezier[30](0,0)(62.5,-62.5)(125,-125)%
            \color{circle}%
              %============================================================================
% NCTU - Hsinchu, Taiwan
% LaTeX File
% Daniel Greenhoe
%
% Unit circle with radius 100
%============================================================================

\qbezier( 100,   0)( 100, 41.421356)(+70.710678,+70.710678) % 0   -->1pi/4
\qbezier(   0, 100)( 41.421356, 100)(+70.710678,+70.710678) % pi/4-->2pi/4
\qbezier(   0, 100)(-41.421356, 100)(-70.710678,+70.710678) %2pi/4-->3pi/4
\qbezier(-100,   0)(-100, 41.421356)(-70.710678,+70.710678) %3pi/4--> pi 
\qbezier(-100,   0)(-100,-41.421356)(-70.710678,-70.710678) % pi  -->5pi/4
\qbezier(   0,-100)(-41.421356,-100)(-70.710678,-70.710678) %5pi/4-->6pi/4
\qbezier(   0,-100)( 41.421356,-100)( 70.710678,-70.710678) %6pi/4-->7pi/4
\qbezier( 100,   0)( 100,-41.421356)( 70.710678,-70.710678) %7pi/4-->2pi


%
              \put( 120, 120){\makebox(0,0)[lb]{$z=e^{i\theta}$}}%
              \put( 115, 115){\vector(-1,-1){43}}%
            \color{zero}%
              \put(  56, -56){\circle{10}}%
              \put(  56,  56){\circle{10}}%
            \normalcolor
              \put(  60,  60){\makebox(0,0)[bl]{$re^{i\phi}$}}%
              \put(  60, -60){\makebox(0,0)[tl]{$re^{-i\phi}$}}%
              \put(  25, - 2){\makebox(0,0)[tl]{$\phi$}}%
          \end{picture}%
          \end{center}

    \item All of the roots of $P\left(\frac{2-z-z^{-1}}{4}\right)$
          occur in \hie{conjugate reciprocal pairs}:
          \begin{multline*}
             P\left(\frac{2-z-z^{-1}}{4}\right)
             \text{ has a root at } z=re^{i\theta}
             \qquad \iff \\
             P\left(\frac{2-z-z^{-1}}{4}\right)
             \text{ has a root at } z=\left(re^{i\theta}\right)^{-1}
                      = \frac{1}{r}e^{-i\theta}
          \end{multline*}
      \begin{center}
      \scriptsize
      \setlength{\unitlength}{0.15mm}
      \begin{picture}(300,300)(-130,-130)
        \thicklines
        \color{axis}%  
          \put(-130,   0){\line(1,0){260} }%
          \put(   0,-130){\line(0,1){260} }%
          \put( 140,   0){\makebox(0,0)[l]{$\Reb{z}$}}%
          \put(   0, 140){\makebox(0,0)[b]{$\Imb{z}$}}%
          \qbezier[30](0,0)(62.5, 62.5)(125, 125)%
          \qbezier[30](0,0)(62.5,-62.5)(125,-125)%
        \color{circle}%
          %============================================================================
% NCTU - Hsinchu, Taiwan
% LaTeX File
% Daniel Greenhoe
%
% Unit circle with radius 100
%============================================================================

\qbezier( 100,   0)( 100, 41.421356)(+70.710678,+70.710678) % 0   -->1pi/4
\qbezier(   0, 100)( 41.421356, 100)(+70.710678,+70.710678) % pi/4-->2pi/4
\qbezier(   0, 100)(-41.421356, 100)(-70.710678,+70.710678) %2pi/4-->3pi/4
\qbezier(-100,   0)(-100, 41.421356)(-70.710678,+70.710678) %3pi/4--> pi 
\qbezier(-100,   0)(-100,-41.421356)(-70.710678,-70.710678) % pi  -->5pi/4
\qbezier(   0,-100)(-41.421356,-100)(-70.710678,-70.710678) %5pi/4-->6pi/4
\qbezier(   0,-100)( 41.421356,-100)( 70.710678,-70.710678) %6pi/4-->7pi/4
\qbezier( 100,   0)( 100,-41.421356)( 70.710678,-70.710678) %7pi/4-->2pi


%
          \put( 120, 120){\makebox(0,0)[lb]{$z=e^{i\theta}$}}%
          \put( 115, 115){\vector(-1,-1){43}}%
        \color{zero}%
          \put(  56,  56){\circle{10}}%
          \put( 125,-125){\circle{10}}%
        \normalcolor
          \put(  60,  60){\makebox(0,0)[bl]{$re^{i\phi}$}}%
          \put( 130,-125){\makebox(0,0)[tl]{$\frac{1}{r}e^{-i\phi}$}}%
          \put(  25, - 2){\makebox(0,0)[tl]{$\phi$}}%
      \end{picture}%
      \end{center}

  \end{enumerate}

  \item By the \thme{Fej\'er-Riesz spectral factorization theorem} \xref{thm:Fejer-Riesz},
          \footnote{
          \thme{Fej\'er-Riesz spectral factorization} \xref{thm:Fejer-Riesz}
          }
        $P\left(\frac{2-z-z^{-1}}{4}\right)$ can be factored in the form
        \[ P\left(\frac{2-z-z^{-1}}{4}\right) = Q(z)Q(z^{-1}). \]

  \item Form $Q(z)$ from the $p-1$ zeros {\em inside} the unit circle.

  \item The scaling coefficients $\seq{h_n}{}$ are the coefficients of
        the polynomial
        \[ \Zh(z) = \sqrt{2}\mcom{\mcom{\brp{\frac{1+z^{-1}}{2}}^p}{order $p$} \mcom{Q(z)}{order $p-1$}}
                         {order $2p-1$ ($2p$ coefficients)} 
        \]
\end{enumerate}

\begin{figure}
  \centering%
  \begin{tabular*}{\textwidth-40mm}{@{\extracolsep{\fill}}cc}%
    \includegraphics{graphics/bspline_pz.pdf}&\includegraphics{graphics/Dp_pz.pdf}\\%
    Cardinal B-spline&Daubechies-p%
  \end{tabular*}%
  \caption{
     Zero locations for B-cardinal spline $N_p(x)$ (left) and
     Daubechies-p (right) scaling coefficients
     %\label{fig:vanish_spline_zero}
     \label{fig:Dp_limits}
     }
\end{figure}

%\begin{figure}[ht]
%\begin{center}
%  \setlength{\unitlength}{0.15mm}
%  %============================================================================
% Daniel J. Greenhoe
% LaTeX File
% Daubechies-p Pole Zero plot
% nominal font size = \gsize
% nominal unit = 10mm
%============================================================================
\begin{pspicture}(-1.5,-2)(4.3,2)%
  %-------------------------------------
  % settings
  %-------------------------------------
  \psset{radius=1mm,linewidth=0.75pt}%
  %-------------------------------------
  % nodes
  %-------------------------------------
  \pnode(0,0){origin}%          z-plane origin
  \pnode(-1,0){pzeroes}%        p zeroes location
  \pnode(0,0){poles}%           2p-1 poles location
  %-------------------------------------
  % asymptotic regions
  %   |z-1| = sqrt(2) (arc with radius sqrt(2) centered at z=1)
  %   |z+1| = sqrt(2) (arc with radius sqrt(2) centered at z=-1)
  % sqrt(2) ~= 1.4142135623730950488016887242097
  %-------------------------------------
  \pscustom[fillstyle=solid,fillcolor=pzasymshade,linestyle=solid,linecolor=pzasym]{%
    \psarc(-1,0){1.414214}{-45}{45}%
    \psarcn( 1,0){1.414214}{135}{-135}% 
    }%
  %\pnode(0.2,0.73){zm1P}%
  \pnode(0.2,0.73){zm1P}%
  \pnode(2,-1){zp1P}%
  \rput[tr](4.25,+1.9){\rnode[l]{zm1L}{$\scy\abs{z-1}=\sqrt{2}$}}%
  \rput[br](4.25,-1.9){\rnode[l]{zp1L}{$\scy\abs{z+1}=\sqrt{2}$}}%
  \ncline[linecolor=pzasym,linestyle=dotted,nodesepA=0pt]{->}{zm1L}{zm1P}%
  \ncline[linecolor=pzasym,linestyle=dotted,nodesepA=0pt]{->}{zp1L}{zp1P}%
  %\psline[linecolor=pzasym,linestyle=dashed](0,+1.414214)(1,+1.414214)%
  \psline[linecolor=pzasym,linestyle=dashed](0,-1.414214)(1,-1.414214)%
  %-------------------------------------
  % axes
  %-------------------------------------
  \psaxes[linecolor=axis,labels=none,ticks=none]{<->}(0,0)(-1.5,-2)(3.5,1.5)%
  \uput{1pt}[0](3.5,0){\color{axis}$\scy\Reb{z}$}%
  \uput{2pt}[210](0,1.5){\color{axis}$\scy\Imb{z}$}%
  % x ticks
  \rput(-1,0){\psline[linecolor=black,linewidth=0.5pt](0,1.4mm)(0,-1.4mm)}%
  \rput( 1,0){\psline[linecolor=black,linewidth=0.5pt](0,1.4mm)(0,-1.4mm)}%
  \rput( 2,0){\psline[linecolor=black,linewidth=0.5pt](0,1.4mm)(0,-1.4mm)}%
  %\rput(0.414214,0){\psline[linecolor=black,linewidth=0.5pt](0,1.4mm)(0,-1.4mm)}%
  \rput(2.414214,0){\psline[linecolor=black,linewidth=0.5pt](0,1.4mm)(0,-1.4mm)}%
  % y ticks
  %\rput(0,1.414214){\psline[,linecolor=black,linewidth=0.5pt](1.4mm,0)(-1.4mm,0)}%
  \rput(0,1){\psline[,linecolor=black,linewidth=0.5pt](1.4mm,0)(-1.4mm,0)}%
  \rput(0,-1){\psline[,linecolor=black,linewidth=0.5pt](1.4mm,0)(-1.4mm,0)}%
  \rput(0,-1.414214){\psline[,linecolor=black,linewidth=0.5pt](1.4mm,0)(-1.4mm,0)}%
  % x labels
  \uput[90](2,0){\scs$2$}%
  \uput[90](1,0){\scs$1$}%
  \uput[120](-1,0){\scs$-1$}%
  \uput[-90](2.414214,0){$\scy\sqrt{2}+1$}% sqrt(2)~=1.4142135623730950488016887242097
  %\uput[-90](0.414214,0){$\scy\sqrt{2}-1$}% sqrt(2)~=1.4142135623730950488016887242097
  % y labels
  \uput[180](0,1){\scs$1$}%
  \uput[180](0,-1){\scs$-1$}%
  %\uput[180](0,1.414214){$\scy\sqrt{2}$}% sqrt(2)~=1.4142135623730950488016887242097
  \uput[180](0,-1.414214){$\scy-\sqrt{2}$}% sqrt(2)~=1.4142135623730950488016887242097
  %-------------------------------------
  % unit circle
  %-------------------------------------
  \pscircle[linecolor=unitcircle](origin){1}%            unit circle
  %\rput[tl](-1.5,1.9){\rnode[b]{circleL}{$\scy z=e^{i\omega}$}}% circle label
  %\rput[bl](-1,-1.5){\rnode[t]{circleL}{$\scy\abs{z}=1$}}% circle label
  \rput[tr](-1.5mm,-1.5mm){\rnode[b]{circleL}{$\scy\abs{z}=1$}}% circle label
  \pnode(-0.707,-0.707){circleP}% circle point
  \ncline[linecolor=unitcircle,linestyle=dotted,nodesepA=0pt]{->}{circleL}{circleP}% circle pointer
  %-------------------------------------
  % B-spline zeroes (at z=-1)
  %-------------------------------------
  \Cnode[linewidth=1pt](-1,0){zp}%
  \rput[bl](-1.4,-1.9){\rnode[t]{zpL}{$p$} zeroes}%
  \ncline[linestyle=dotted,linecolor=zero,nodesepA=0pt]{->}{zpL}{zp}%
  \uput{1mm}[30](-1,0){\color{zero}$\scy{p}$}%
  %-------------------------------------
  % orthogonality zeroes
  %-------------------------------------
  \Cnode [linecolor=zero,linewidth=1pt](+0.3654035130742002,  0                 ){z1}%
  \Cnode [linecolor=zero,linewidth=1pt](+0.3577427639711839, +0.1159310245530727){z2}%
  \Cnode [linecolor=zero,linewidth=1pt](+0.3577427639711839, -0.1159310245530727){z3}%
  \Cnode [linecolor=zero,linewidth=1pt](+0.3298169959381100, +0.2476497421117944){z4}%
  \Cnode [linecolor=zero,linewidth=1pt](+0.3298169959381100, -0.2476497421117944){z5}%
  \Cnode [linecolor=zero,linewidth=1pt](+0.2549176775413241, +0.4249813706999120){z6}%
  \Cnode [linecolor=zero,linewidth=1pt](+0.2549176775413241, -0.4249813706999120){z7}%
  \Cnode*[linecolor=zero,linewidth=1pt](+2.7367005631344137,  0                 ){z8}%
  \Cnode*[linecolor=zero,linewidth=1pt](+1.0379714430713773, +1.7304352168829922){z9}%
  \Cnode*[linecolor=zero,linewidth=1pt](+1.0379714430713773, -1.7304352168829922){z10}%
  \Cnode*[linecolor=zero,linewidth=1pt](+1.9388494686323963, +1.4558242201373777){z11}%
  \Cnode*[linecolor=zero,linewidth=1pt](+1.9388494686323963, -1.4558242201373777){z12}%
  \Cnode*[linecolor=zero,linewidth=1pt](+2.5296496127413031, +0.8197646490830808){z13}%
  \Cnode*[linecolor=zero,linewidth=1pt](+2.5296496127413031, -0.8197646490830808){z14}%
  %-------------------------------------
  % p-1 radial lines
  %-------------------------------------
  \ncline[linestyle=dashed]{origin}{z8}%
  \ncline[linestyle=dashed]{origin}{z9}%
  \ncline[linestyle=dashed]{origin}{z10}%
  \ncline[linestyle=dashed]{origin}{z11}%
  \ncline[linestyle=dashed]{origin}{z12}%
  \ncline[linestyle=dashed]{origin}{z13}%
  \ncline[linestyle=dashed]{origin}{z14}%
  %-------------------------------------
  % poles
  %-------------------------------------
  %\rput(poles){\color{pole}$\mathbf\times$}%
  \rput(poles){%
    \psline[linecolor=pole,linewidth=1pt](-0.1,+0.1)(+0.1,-0.1)%
    \psline[linecolor=pole,linewidth=1pt](-0.1,-0.1)(+0.1,+0.1)%
    }%
  \rput[tl](-1.4,1.9){$2$\rnode[t]{polesL}{$p$}$-1$ poles}%
  \ncline[linecolor=pole,linestyle=dotted,nodesepA=0pt,nodesepB=5pt]{->}{polesL}{poles}%
  \uput{1mm}[30](0,0){\color{pole}$\scy(2p-1)$}%
  %-------------------------------------
  % discarded zeroes region
  %-------------------------------------
  \psframe[linestyle=dotted](0.8,-1.9)(3,1.9)%
  \rput[b]{-90}(3,0){\footnotesize\color{zero}$p-1$ discarded zeroes}%
\end{pspicture}%
%
%\begin{picture}(600,470)(-250,-200)%
%  %\graphpaper[10](0,0)(200,200)
%  \thicklines%
%  \color{axis}%
%    \put(-130,   0){\line(1,0){500} }%
%    \put(   0,-130){\line(0,1){260} }%
%    \put( 380,   0){\makebox(0,0)[l]{$\Reb{z}$}}%
%    \put(   0, 140){\makebox(0,0)[b]{$\Imb{z}$}}%
%  \color{circle}%
%    \qbezier( 100,   0)( 100, 41.421356)(+70.710678,+70.710678)% 0   -->1pi/4
%    \qbezier(   0, 100)( 41.421356, 100)(+70.710678,+70.710678)% pi/4-->2pi/4
%    \qbezier(   0, 100)(-41.421356, 100)(-70.710678,+70.710678)%2pi/4-->3pi/4
%    \qbezier(-100,   0)(-100, 41.421356)(-70.710678,+70.710678)%3pi/4--> pi
%    \qbezier(-100,   0)(-100,-41.421356)(-70.710678,-70.710678)% pi  -->5pi/4
%    \qbezier(   0,-100)(-41.421356,-100)(-70.710678,-70.710678)%5pi/4-->6pi/4
%    \qbezier(   0,-100)( 41.421356,-100)( 70.710678,-70.710678)%6pi/4-->7pi/4
%    \qbezier( 100,   0)( 100,-41.421356)( 70.710678,-70.710678)%7pi/4-->2pi
%    %\put( 110, 110){\makebox(0,0)[lb]{$z=e^{i\omega}$}}%
%    %\put( 105, 105){\vector(-1,-1){33}}%
%  \color{pzasym}%
%    \qbezier(41.4214,0)(41.4214, 58.5786)(0, 100)%      % inner upper arc
%    \qbezier(41.4214,0)(41.4214,-58.5786)(0,-100)%      % inner lower arc
%    \qbezier(0, 100)(41.421, 141.421)(100, 141.421)%    %outer upper arc
%    \qbezier(100, 141.421)(158.579,141.421)(200,100)%
%    \qbezier(200,100)(241.421,58.5786)(241.421,0)%
%    \qbezier(0,-100)(41.421,-141.421)(100,-141.421)%    %outer lower arc
%    \qbezier(100,-141.421)(158.579,-141.421)(200,-100)%
%    \qbezier(200,-100)(241.421,-58.5786)(241.421,0)%
%  \color{pole}%
%    \put(-100,-100){\makebox(0,0)[tr]{$2p-1$ poles}}%
%    \put(-100,-100){\vector(1,1){91}}%
%    \put(   0,    0){\makebox(0,0)[c]{$\times$}}%
%    %\put(   0,    0){\makebox(0,0)[c]{\hspace{1em}$^{2p-1}$}}%
%  \color{zero}%
%    \put(-150, -50){\makebox(0,0)[tr]{$p$ zeros}}%
%    \put(-150, -50){\vector( 1, 1){43}}%
%    \put(-100,    0){\circle{15}$^p$}%
%    \put(+36.54035130742002 ,  0                 ) {\circle{15} }%
%    \put(+35.77427639711839 , +11.59310245530727 ) {\circle{15} }%
%    \put(+35.77427639711839 , -11.59310245530727 ) {\circle{15} }%
%    \put(+32.98169959381100 , +24.76497421117944 ) {\circle{15} }%
%    \put(+32.98169959381100 , -24.76497421117944 ) {\circle{15} }%
%    \put(+25.49176775413241 , +42.49813706999120 ) {\circle{15} }%
%    \put(+25.49176775413241 , -42.49813706999120 ) {\circle{15} }%
%    \put(+273.67005631344137 ,  0                  ){ \circle*{15} }%
%    \put(+103.79714430713773 , +173.04352168829922 ){ \circle*{15} }%
%    \put(+103.79714430713773 , -173.04352168829922 ){ \circle*{15} }%
%    \put(+193.88494686323963 , +145.58242201373777 ){ \circle*{15} }%
%    \put(+193.88494686323963 , -145.58242201373777 ){ \circle*{15} }%
%    \put(+252.96496127413031 , +081.97646490830808 ){ \circle*{15} }%
%    \put(+252.96496127413031 , -081.97646490830808 ){ \circle*{15} }%
%    \qbezier[30](0,0)(51.90, 86.52)(+103.79714430713773 ,  +173.04352168829922 )%
%    \qbezier[30](0,0)(51.90,-86.52)(+103.79714430713773 ,  -173.04352168829922 )%
%    \qbezier[30](0,0)(96.94, 72.79)(+193.88494686323963 ,  +145.58242201373777 )%
%    \qbezier[30](0,0)(96.94,-72.79)(+193.88494686323963 ,  -145.58242201373777 )%
%    \qbezier[30](0,0)(126.48, 40.99)(+252.96496127413031 , + 81.97646490830808 )%
%    \qbezier[30](0,0)(126.48,-40.99)(+252.96496127413031 , - 81.97646490830808 )%
%    \put(90,-200){\dashbox{5}(230,400){}}%
%    \put(140,-200){\makebox(210,390)[tr]{discarded zeros}}%
%    %\put(193, 174){\makebox(0,0)[rc]{discarded zero from $\fQ(z^{-1})\;\rightarrow$}}%
%    %\put(193,-174){\makebox(0,0)[rc]{discarded zero from $\fQ(z^{-1})\;\rightarrow$}}%
%    %\put(300,-20){\makebox(0,0)[lt]{$\uparrow$ discarded zero from $\fQ(z^{-1})$}}%
%\end{picture}

%\end{center}
%\caption{
%   Asymptotic limits of Daubechies-$p$ zeros \xref{lem:QzQz_zeros}
%   \label{fig:Dp_limits}
%   }
%\end{figure}

As the number of vanishing moments approaches infinity ($p\to\infty$),
the zeros of the Daubechies-$p$ scaling coefficients approach two
asymptotic arcs as described in the next theorem and illustrated in
\prefpp{fig:Dp_limits}.
%---------------------------------------
\begin{lemma}
\footnote{
  \citerpgc{strang1996}{170}{171}{0961408871},
  \citerpgc{vidakovic}{78}{0471293652}{Fig. 3.14},
  \citorp{shen}{3823},
  \citor{kateb1997}
  }
\label{lem:QzQz_zeros}
%---------------------------------------
Let $\spO\eqd\wavsys$.
\lemboxp{
  $\spO$ is a \structe{Daubechies-$p$ wavelet system} \xref{def:Dp} $\implies$\\
  with increasing vanishing moments parameter $p$,
  the zeros of $Q(z)Q(z^{-1})$ asymptotically approach the area of the complex plain enclosed by the arcs  
  \\\indentx$\abs{z-1}=\sqrt{2} \qquad\text{and}\qquad \abs{z+1}=\sqrt{2}$\\
  }
\end{lemma}

%---------------------------------------
\begin{theorem}
%---------------------------------------
Let $\spO\eqd\wavsys$.
Let $\support\seqnZ{x_n}$ be defined as in \prefp{def:support}.
\thmboxt{
  $\brb{\text{$\spO$ is a \structe{Daubechies-$p$ wavelet system} \xref{def:Dp}}}$\qquad$\implies$
  %$\brb{\begin{array}{M}
  %  $\spO$ is a\\ 
  %  \structe{Daubechies-$p$ wavelet system}
  %\end{array}}$
  %$\implies$
  \\\indentx$\brb{\begin{array}{FMD}
     (1). & $\spO$ is an \structe{orthonormal wavelet system}.                    & and\\
     %(2). & for $p\ge2$, $\fphi$ is \prope{continuous}                            & and \\
     (2). & $\Fh(\omega)$ has \prope{minimum phase}.                              & and\\
     (3). & $\support\seqn{h_n}=2p-1$ ($\seqn{h_n}$ has $2p$ non-zero elements)   & and\\
     (4). & the support size of $\seqn{h_n}$ is the smallest possible\\
          & for wavelets with $p$ vanishing moments.
  \end{array}}$
  }
\end{theorem}
\begin{proof}
\begin{enumerate}
\item proof for (1): For $p=1$, this is the \structe{Haar wavelet system}. For $p\ge2$, this follows from \prefp{thm:ortho_sc}.
  \begin{enumerate}
    \item Proof that $\seqn{h_n}$ satisfies $\ds\sum_{n\in\Z}h_n = \sqrt{2}$:
      \begin{align*}
        \brlr{\Zh(z)}_{z=1}
          &= \brlr{\sqrt{2}\brp{\frac{1+z^{-1}}{2}}^p Q(z)}_{z=1}
          && \text{by def. of \structe{Daubechies-p wavelet system}}
          && \text{\xref{def:Dp}}
        \\&= \sqrt{2}\brp{\frac{1+1^{-1}}{2}}^p Q(1)
        \\&= \sqrt{2}Q(1)
        \\&= \sqrt{2}\sqrt{Q(1)Q^\ast(1^{-1})}
        \\&= \brlr{\sqrt{2}\sqrt{Q\brp{z}Q^\ast\brp{z^{-1}}}}_{z=1}
        \\&= \brlr{\sqrt{2}\sqrt{P\brp{\frac{2-z-z^{-1}}{4}}}}_{z=1}
        \\&= \sqrt{2}\sqrt{P\brp{0}}
        \\&= \brlr{\sqrt{2}\sqrt{\sum\limits_{k=0}^{p-1} {{p-1+k}\choose{k}} y^k}}_{y=0}
          && \text{by \thme{orthonormal quadrature conditions}}
          && \text{\xref{lem:gcd_yp}}
        \\&= \sqrt{2}\sqrt{\brp{1\cdot0^0 + 0}}
          && \text{by definition of $\bcoefnk$}
          && \text{\xref{def:bcoef}}
        \\&= \sqrt{2}\sqrt{\brp{1+0}}
        \\&= \sqrt{2}
        \\&\implies \sum_{n\in\Z}h_n = \sqrt{2}
          && \text{by \leme{orthonormal quadrature condition}s}
          && \text{\xref{lem:oms_quadcon}}
      \end{align*}

    \item Proof that $\seqn{h_n}$ satisfies $\ds\sum_{m\in\Z}h_m h^\ast_{m-2n}=\kdelta_n$:
      \begin{align*}
        &\text{$\spO$ is a \structe{$D_p$ wavelet system}}
        \\&\implies (1-y)^p P(y) + y^p P(1-y) = 1
          && \text{by \prefp{def:Dp} and \prefp{lem:gcd_yp}}
        \\&\iff \abs{\Fh(\omega)}^2 + \abs{\Fh(\omega+\pi)}^2 = 2
          && \text{by \prefp{lem:quadcon_y}}
        \\&\iff \sum_{m\in\Z}h_m h^\ast_{m-2n}=\kdelta_n
          && \text{by \leme{orthonormal quadrature condition}s \xref{lem:oms_quadcon}}
      \end{align*}

    \item Proof that $\Fphi(\omega)$ is \prope{continuous} at $0$:
      \begin{align*}
            &\Zh(z) = \sqrt{2}\brp{\frac{1+z^{-1}}{2}}^p Q(z)
            && \text{by \prefp{def:Dp}}
          \\&\implies \text{$\fpsi$ has \prope{$p$ vanishing moments}}
            && \text{by \prefp{lem:van_factor}}
          \\&\implies \text{$\Fphi(\omega)$ is \prope{continuous} at $\omega=0$}
            && \text{by \prefp{thm:vanish_continuous}}
      \end{align*}

    \item Proof that $\ds\inf_{\omega\in\intcc{-\pi/2}{\pi/2}}\abs{\Dh(\omega)}>0$:\\ 
          by \prefp{lem:QzQz_zeros}---note that the zeros are never on the unit circle in the range $\intcc{-\pi/2}{\pi/2}$.
  \end{enumerate}

%\item Proof for (2): %\citerp{ruch2011}{233}{1118165667}

\item Proof for (2): The zeros are chosen to be \emph{inside} the unit circle, thus giving $\Fh(\omega)$ \prope{minimum phase}.

\item Proof for (3): See \prefp{thm:cs_bezout_prop}.

\item Proof for (4): See \prefp{thm:cs_bezout_prop}.
\end{enumerate}
\end{proof}



%=======================================
\section{Examples}
\label{sec:examples_Dp}
\index{Daubechies wavelets}
%=======================================
%---------------------------------------
\begin{example}
\label{ex:dau-p1}
\index{Haar wavelets}
%---------------------------------------
The \structb{Daubechies-$1$} wavelet system yields the following results.
\exboxt{\begin{tabular}{cccc}
   $\begin{array}{|l|r|r|}
    \hline
    n & h_n  & g_n  
    \\\hline
    0  & \ds \cwt   & \ds \cwt   \\
    1  & \ds \cwt   & \ds -\cwt 
    \\\hline
  \end{array}$
  &\tbox{\includegraphics{graphics/D1_pz.pdf}}%
  &\tbox{\includegraphics{graphics/d1_phi_h.pdf}}%
  &\tbox{\includegraphics{graphics/d1_psi_g.pdf}}%
\end{tabular}}\\
The \structb{Daubechies-$1$} wavelet system is equivalent to the \structe{Haar wavelet system},
and it is possible to compute this wavelet system using ``time-domain" techniques as well:
\begin{dingautolist}{"AC}
  \item Admissibility condition and \hie{orthonormality}---\prefpp{ex:ortho_n=2}
  \item Admissibility condition and \hie{partition of unity}---\prefpp{ex:pun_n=2}
\end{dingautolist}
\end{example}
\begin{proof}
\begin{align*}
  \fQ(z)\fQ(z^{-1})
    &= P(y)|_{y=\frac{2-z-z^{-1}}{4}}
  \\&= \left.
       \sum\limits_{k=0}^{p-1} {{p-1+k}\choose{k}} y^k
       \right|_{y=\frac{2-z-z^{-1}}{4}}
  \\&= \left.
         \sum\limits_{k=0}^{0} {{1+k}\choose{k}} y^k
       \right|_{y=\frac{2-z-z^{-1}}{4}}
  \\&= 1
  \\&= \mcom{1}{$\fQ(z)$} \cdot \mcom{1}{$\fQ(z^{-1})$}
\\
\\
  \Zh(z)
    &= \sqrt{2}\;
       \left(\frac{1+z^{-1}}{2}\right)^p
       \cancelto{1}{\fQ(z)}
  \\&=        \mcom{\frac{1}{\sqrt{2}}}{$h_0$} +
       z^{-1} \mcom{\frac{1}{\sqrt{2}}}{$h_1$}
\end{align*}
\end{proof}

\begin{figure}
  \centering%
  \exboxt{\begin{tabular}{cc}
    $\begin{array}{|l|r|r|}
      \hline
      n & h_n  & g_n  \\
      \hline
        0   & \frac{1+\sqrt{3}}{4}  &  \frac{1-\sqrt{3}}{4}  \\
        1   & \frac{3+\sqrt{3}}{4}  & -\frac{3-\sqrt{3}}{4}  \\
        2   & \frac{3-\sqrt{3}}{4}  &  \frac{3+\sqrt{3}}{4}  \\
        3   & \frac{1-\sqrt{3}}{4}  & -\frac{1+\sqrt{3}}{4}  \\
      \hline
    \end{array}$&\tbox{\includegraphics{graphics/D2_pz.pdf}}\\%
    \includegraphics{graphics/d2_phi_h.pdf}&\includegraphics{graphics/d2_psi_g.pdf}
  \end{tabular}}
  \caption{\structe{Daubechies-$2$ wavelet system} \xref{ex:dau-p2} \label{fig:dau-p2}}
\end{figure}
%---------------------------------------
\begin{example}
\label{ex:dau-p2}
%---------------------------------------
The \structd{Daubechies-$2$ wavelet system} yields the results illustrated in \prefpp{fig:dau-p2}.
It is also possible to compute this wavelet system using a ``time-domain" technique as well---see \prefpp{ex:pun_n=2}.
\end{example}
\begin{proof}
\begin{align*}
  \fQ(z)\fQ(z^{-1})
    &= P(y)|_{y=\frac{2-z-z^{-1}}{4}}
  \\&= \left.
       \sum\limits_{k=0}^{p-1} {{p-1+k}\choose{k}} y^k
       \right|_{y=\frac{2-z-z^{-1}}{4}}
  \\&= \left.
         \sum\limits_{k=0}^{1} {{1+k}\choose{k}} y^k
       \right|_{y=\frac{2-z-z^{-1}}{4}}
  \\&= (1+2y)|_{y=\frac{2-z-z^{-1}}{4}}
  \\&= 1 + \frac{2-z-z^{-1}}{2}
  \\&= \frac{4-z-z^{-1}}{2}
  \\&= \frac{4z-z^2-1}{2z}
  \\&= \frac{z^2 - 4z +1}{-2z}
  \\&= \frac{1}{-2z}
       \left[z-(2-\sqrt{3}) \right]
       \left[z-(2+\sqrt{3}) \right]
  \\&= \frac{1}{-2}
       \left[z-(2-\sqrt{3}) \right]
       \left[1-(2+\sqrt{3})z^{-1} \right]
  \\&= \frac{1}{2}
       \left[z-(2-\sqrt{3}) \right]
       \left[(2+\sqrt{3})z^{-1} -1\right]
  \\&= \frac{2+\sqrt{3}}{2}
       \left[z-(2-\sqrt{3}) \right]
       \left[\frac{2+\sqrt{3}}{2+\sqrt{3}}z^{-1} -\frac{1}{2+\sqrt{3}}\right]
  \\&= \frac{2+\sqrt{3}}{2}
       \left[z-(2-\sqrt{3}) \right]
       \left[z^{-1} -(2-\sqrt{3}) \right]
  \\&= \mcom{\frac{1+\sqrt{3}}{\sqrt{2}}\left[z-(2-\sqrt{3}) \right]}{$\fQ(z)$}
       \mcom{\frac{1+\sqrt{3}}{\sqrt{2}}\left[z^{-1} -(2-\sqrt{3}) \right]}{$\fQ(z^{-1})$}
\\
\\
  \Zh(z)
    &= \sqrt{2}\;
       \left(\frac{1+z^{-1}}{2}\right)^p
       \fQ(z)
  \\&= \frac{\sqrt{2}}{4}\;
       \left(\frac{z+1}{z}\right)^2
       \fQ(z)
  \\&= \frac{\sqrt{2}}{4z^2}
       \frac{1+\sqrt{3}}{2}
       \left[ z-(2-\sqrt{3}) \right] \left[ z+1 \right]^2
  \\&= \frac{\sqrt{2}}{8z^2} (1+\sqrt{3})
       \left[ z-(2-\sqrt{3}) \right]
       \left[ z^2+z+1 \right]
  \\&= \frac{\sqrt{2}}{8z^2} (1+\sqrt{3})
       [z^3 + z^2(\sqrt{3}) + z(-3+2\sqrt{2}) + (-2+\sqrt{3})]
  \\&= \frac{\sqrt{2}}{8z^2}\;
       [z^3(1+\sqrt{3}) + z^2(3+\sqrt{3}) + z(3-\sqrt{3}) + (1-\sqrt{3})]
  \\&=    z     \mcom{\frac{\sqrt{2}}{8}(1+\sqrt{3})}{$h_0$} +
                \mcom{\frac{\sqrt{2}}{8}(3+\sqrt{3})}{$h_1$} +
          z^{-1}\mcom{\frac{\sqrt{2}}{8}(3-\sqrt{3})}{$h_2$} +
          z^{-2}\mcom{\frac{\sqrt{2}}{8}(1-\sqrt{3})}{$h_3$}
\end{align*}
\end{proof}

\begin{figure}
  \exboxt{\begin{tabular}{cc}
     $\begin{array}{|l|r|r|}
      \hline
      n & h_n  & g_n  \\
      \hline
        0   &  0.3326705530 &  0.0352262919 \\
        1   &  0.8068915093 &  0.0854412739 \\
        2   &  0.4598775021 & -0.1350110200 \\
        3   & -0.1350110200 & -0.4598775021 \\
        4   & -0.0854412739 &  0.8068915093 \\
        5   &  0.0352262919 & -0.3326705530 \\
      \hline
    \end{array}$&\tbox{\includegraphics{graphics/D3_pz.pdf}}\\
    \includegraphics{graphics/d3_phi_h.pdf}&\includegraphics{graphics/d3_psi_g.pdf}
  \end{tabular}}
  \caption{\structe{Daubechies-$3$ wavelet system} \xref{ex:dau-p3} \label{fig:dau-p3}}
\end{figure}
%---------------------------------------
\begin{example}
\label{ex:dau-p3}
%---------------------------------------
\setlength{\unitlength}{0.15mm}
The \structd{Daubechies-$3$ wavelet system} yields the results illustrated in \prefpp{fig:dau-p3}.
\end{example}
\begin{proof}
\begin{align*}
  \fQ(z)\fQ(z^{-1})
    &= P(y)|_{y=\frac{2-z-z^{-1}}{4}}
  \\&= \left.
       \sum\limits_{k=0}^{p-1} {{p-1+k}\choose{k}} y^k
       \right|_{y=\frac{2-z-z^{-1}}{4}}
  \\&= \left.
       \sum\limits_{k=0}^{2} {{2+k}\choose{k}} y^k
       \right|_{y=\frac{2-z-z^{-1}}{4}}
  \\&= \left.
       \frac{2!}{0!(2-0)!}y^0 + \frac{3!}{1!(3-1)!}y^1 + \frac{4!}{2!(4-2)!}y^2
       \right|_{y=\frac{2-z-z^{-1}}{4}}
  \\&= \left[1 + 3y + 6y^2\right]_{y=\frac{2-z-z^{-1}}{4}}
  \\&=  1
      + 3\left(\frac{2-z-z^{-1}}{4}\right)
      + 6\left(\frac{2-z-z^{-1}}{4}\right)^2
  \\&=  1
      + 3\left(\frac{z^2-2z+1}{-4z}\right)
      + 6\left(\frac{z^2-2z+1}{-4z}\right)^2
  \\&=  1
      + \left(\frac{-3z^2+6z-3}{4z}\right)
      + 6\left(\frac{z^4 - 4z^3 + 6z^2 -4z +1}{16z^2}\right)
  \\&=  \frac{16z^2}{16z^2}
      + \left(\frac{-12z^3+24z^2-12z}{16z^2}\right)
      + \left(\frac{6z^4 - 24z^3 + 36z^2 -24z +6}{16z^2}\right)
  \\&=  \frac{6z^4 -36z^3 + 76z^2 -36z +6}{16z^2}
  \\&=  \frac{3}{8z^2}\left( z^4 -6z^3 + \frac{38}{3}z^2 -6z +1 \right)
  \\&=  \frac{3}{8z^2}\left( z^4 -6z^3 + \frac{38}{3}z^2 -6z +1 \right)
  \\&=  \frac{3}{8z^2}(z-r_1)(z-r_1^\ast)(z-r_2)(z-r_2^\ast)
  \\&=  \frac{3}{8}(z-r_1)(z-r_1^\ast)(1-z^{-1}r_2)(1-z^{-1}r_2^\ast)
  \\&=  \frac{3r_2r_2^\ast}{8}(z-r_1)(z-r_1^\ast)(r_2^{-1}-z^{-1})(r_2^{\ast-1}-z^{-1})
  \\&=  \frac{3r_2r_2^\ast}{8}(z-r_1)(z-r_1^\ast)(\frac{1}{r_2}-z^{-1})(\frac{1}{r_2^\ast}-z^{-1})
  \\&=  \frac{3|r_2|^2}{8}(z-r_1)(z-r_1^\ast)(r_1^\ast-z^{-1})(r_1-z^{-1})
  \\&=  \frac{3|r_2|^2}{8}(z-r_1)(z-r_1^\ast)(z^{-1}-r_1^\ast)(z^{-1}-r_1)
  \\&=  \mcom{\sqrt{\frac{3|r_2|^2}{8}}(z-r_1)(z-r_1^\ast)}{$\fQ(z)$}
        \mcom{\sqrt{\frac{3|r_2|^2}{8}}(z^{-1}-r_1^\ast)(z^{-1}-r_1)}{$\fQ(z^{-1})$}
\\
  r_1 &= 0.287251+i0.152892 = \frac{1}{r_2^\ast}
\\r_2 &= 2.712749+i1.443887 = \frac{1}{r_1^\ast}
\\
\\
  \Zh(z)
    &= \sqrt{2}\; \left(\frac{1+z^{-1}}{2}\right)^p \fQ(z)
  \\&= \sqrt{2}\; \left(\frac{z+1}{2z}\right)^3     \fQ(z)
  \\&= \sqrt{2}\; \sqrt{\frac{3|r_2|^2}{8}}
       \left( \frac{z+1}{2z} \right)^3
       (z-r_1)(z-r_1^\ast)
  \\&= \left(\frac{\sqrt{3|r_2|^2}}{16}\right)
       \frac{(z+1)^3(z-r_1)(z-r_1^\ast)}{z^3}
  \\&= \left(\frac{\sqrt{3|r_2|^2}}{16}\right)
       \frac{(z^3+3z^2+3z+1)(z^2-2z\Reb{r_1} + |r_1|^2)}{z^3}
%  \\&= \left(\frac{\sqrt{3|r_2|^2}}{16}\right)
%       \frac{
%            (z^5-2z^4\Reb{r_1} + z^3|r_1|^2) +
%            (3z^4-6z^3\Reb{r_1} + 3z^2|r_1|^2) +
%            (3z^3-6z^2\Reb{r_1} + 3z|r_1|^2)   +
%            (z^2-2z\Reb{r_1} + |r_1|^2)
%            }
%            {z^3}
%  \\&= \left(\frac{\sqrt{3|r_2|^2}}{16}\right)
%       \frac{
%            z^5 + z^4(3-2\Reb{r_1}) + z^3(|r_1|^2-6\Reb{r_1}+3)
%           +z^2(3|r_1|^2-6\Reb{r_1}+1)+z(3|r_1|^2-2\Reb{r_1})
%           +|r_1|^2
%            }
%            {z^3}
  \\&= \frac{h_0z^5 + h_1z^4 + h_2z^3 + h_3z^2 + h_4z + h_5}
            {z^3}
\\
\\
  h_0 &= +0.3326705529500830 \\
  h_1 &= +0.8068915093110932 \\
  h_2 &= +0.4598775021184915 \\
  h_3 &= -0.1350110200102552 \\
  h_4 &= -0.0854412738820268 \\
  h_5 &= +0.0352262918857096 \\
\end{align*}
\end{proof}

\begin{figure}
  \exbox{\begin{tabular}{cc}
     $\begin{array}{l|r|r}
      n & h_n  & g_n  \\
      \hline
        0   &  0.2303778133 & -0.0105974018 \\
        1   &  0.7148465706 & -0.0328830117 \\
        2   &  0.6308807679 &  0.0308413818 \\
        3   & -0.0279837694 &  0.1870348117 \\
        4   & -0.1870348117 & -0.0279837694 \\
        5   &  0.0308413818 & -0.6308807679 \\
        6   &  0.0328830117 &  0.7148465706 \\
        7   & -0.0105974018 & -0.2303778133
    \end{array}$&\tbox{\includegraphics{graphics/D4_pz.pdf}}\\
    \includegraphics{graphics/d4_phi_h.pdf}&\includegraphics{graphics/d4_psi_g.pdf}
  \end{tabular}}
  \caption{\structe{Daubechies-$4$ wavelet system} \xref{ex:dau-p4} \label{fig:dau-p4}}
\end{figure}
%---------------------------------------
\begin{example}
\label{ex:dau-p4}
%---------------------------------------
The \structd{Daubechies-$4$ wavelet system} yields the results illustrated in \prefpp{fig:dau-p4}.
\end{example}
\begin{proof}
\begin{align*}
  \left.P(y)\right|_{y=\frac{2-z-z^{-1}}{4}}
    &= \left.
       \sum\limits_{k=0}^{p-1} {{p-1+k}\choose{k}} y^k
       \right|_{y=\frac{2-z-z^{-1}}{4}}
  \\&= \left.
       \sum\limits_{k=0}^{3} {{3+k}\choose{k}} y^k
       \right|_{y=\frac{2-z-z^{-1}}{4}}
  \\&= \left.
       \frac{3!}{0!(3-0)!}y^0 + \frac{4!}{1!(4-1)!}y^1 + \frac{5!}{2!(5-2)!}y^2 + \frac{6!}{3!(6-3)!}y^3
       \right|_{y=\frac{2-z-z^{-1}}{4}}
  \\&= \left[1 + 4y + 10y^2 + 20y^3 \right]_{y=\frac{2-z-z^{-1}}{4}}
  \\&= 1
     + 4\left(\frac{2-z-z^{-1}}{4}\right)
     + 10\left(\frac{2-z-z^{-1}}{4}\right)^2
     + 20\left(\frac{2-z-z^{-1}}{4}\right)^3
\\
  \\ r_1 &= +0.3288759177860292
  \\ r_2 &= +0.2840962981918215   +0.2432282259103822 i
  \\ r_3 &= +0.2840962981918215   -0.2432282259103822 i  && = r_2^\ast
  \\ r_4 &= +3.0406604616474535                          && = \frac{1}{r_1}
  \\ r_5 &= +2.0311355120914394  -1.7389508076448230 i   && = \frac{1}{r_2}
  \\ r_6 &= +2.0311355120914394   +1.7389508076448230 i  && = r_5^\ast
\\
\\
  \Zh(z)
    &= \sqrt{2}\; \left(\frac{1+z^{-1}}{2}\right)^p \fQ(z)
  \\&= \sqrt{2}\; \left(\frac{z+1}{2z}\right)^4 \fQ(z)
  \\&= A_0\;      \left[ \frac{z+1}{2} \right]^4 \frac{(z-r_1)(z-r_2)(z-r_2^\ast) }{z^4}
  \\&= \frac{h_0z^7 + h_1^6 + h_2z^5 + h_3z^4 + h_4z^3 + h_5z^2 + h_6z + h_7}
            {z^4}
\\
  \\h_0 &= +0.2303778133088954
  \\h_1 &= +0.7148465705529126
  \\h_2 &= +0.6308807679298577
  \\h_3 &= -0.0279837694168571
  \\h_4 &= -0.1870348117190903
  \\h_5 &= +0.0308413818355609
  \\h_6 &= +0.0328830116668849
  \\h_7 &= -0.0105974017850690
\end{align*}
\end{proof}

\begin{figure}
  \centering%
  \exbox{\begin{tabular}{cc}
     $\begin{array}{|r|r|r|}
      \hline
      \mc{1}{|c|}{n} & \mc{1}{c|}{h_n}  & \mc{1}{c|}{g_n}  \\
      \hline
       0 &  0.0544158422  & -0.0001174768 \\
       1 &  0.3128715909  & -0.0006754494 \\
       2 &  0.6756307363  & -0.0003917404 \\
       3 &  0.5853546837  &  0.0048703530 \\
       4 & -0.0158291053  &  0.0087460940 \\
       5 & -0.2840155430  & -0.0139810279 \\
       6 &  0.0004724846  & -0.0440882539 \\
       7 &  0.1287474266  &  0.0173693010 \\
       8 & -0.0173693010  &  0.1287474266 \\
       9 & -0.0440882539  & -0.0004724846 \\
      10 &  0.0139810279  & -0.2840155430 \\
      11 &  0.0087460940  &  0.0158291053 \\
      12 & -0.0048703530  &  0.5853546837 \\
      13 & -0.0003917404  & -0.6756307363 \\
      14 &  0.0006754494  &  0.3128715909 \\
      15 & -0.0001174768  & -0.0544158422 \\
      \hline
    \end{array}$&\tbox{\includegraphics{graphics/D8_pz.pdf}}\\  
    \includegraphics{graphics/d8_phi_h.pdf}&\includegraphics{graphics/d8_psi_g.pdf}
  \end{tabular}}
  \caption{\structe{Daubechies-$8$ wavelet system} \xref{ex:dau-p8} \label{fig:dau-p8}}
\end{figure}
%---------------------------------------
\begin{example}
\label{ex:dau-p8}
%---------------------------------------
The \structd{Daubechies-$8$ wavelet system} yields the results illustrated in \prefpp{fig:dau-p8}.
\end{example}
\begin{proof}
\begin{align*}
   \fQ(z)\f(Q^{-1})
     &= \left.
        P\left( \frac{2-z-z^{-1}}{4} \right)
        \right|_{y=\frac{2-z-z^{-1}}{4}}
   \\&= \left.
        \sum\limits_{k=0}^{p-1} {{p-1+k}\choose{k}} y^k
        \right|_{y=\frac{2-z-z^{-1}}{4}}
   \\&= \left.
        \sum\limits_{k=0}^{7} {{7+k}\choose{k}} y^k
        \right|_{y=\frac{2-z-z^{-1}}{4}}
   \\&= \left[
        1 + 8y + 36y^2 + 120y^3 + 330y^4 + 792y^5 + 1716y^6  + 3432y^7
        \right]_{y=\frac{2-z-z^{-1}}{4}}
\end{align*}
{\footnotesize$\begin{array}{|lcrr|lcrr|}
    \hline
     r_{ 1} &=&  +0.3654035130742002 &                          &  r_{ 8} &=&  +2.7367005631344137  &                      
  \\ r_{ 2} &=&  +0.3577427639711839 &  +0.1159310245530727 i   &  r_{ 9} &=&  +1.0379714430713773  & +1.7304352168829922 i
  \\ r_{ 3} &=&  +0.3577427639711839 &  -0.1159310245530727 i   &  r_{10} &=&  +1.0379714430713773  & -1.7304352168829922 i
  \\ r_{ 4} &=&  +0.3298169959381100 &  +0.2476497421117944 i   &  r_{11} &=&  +1.9388494686323963  & +1.4558242201373777 i
  \\ r_{ 5} &=&  +0.3298169959381100 &  -0.2476497421117944 i   &  r_{12} &=&  +1.9388494686323963  & -1.4558242201373777 i
  \\ r_{ 6} &=&  +0.2549176775413241 &  +0.4249813706999120 i   &  r_{13} &=&  +2.5296496127413031  & +0.8197646490830808 i
  \\ r_{ 7} &=&  +0.2549176775413241 &  -0.4249813706999120 i   &  r_{14} &=&  +2.5296496127413031  & -0.8197646490830808 i
  \\\hline
\end{array}$}
\begin{align*}
  \Zh(z)
    &= \sqrt{2}\; \left(\frac{1+z^{-1}}{2}\right)^p \fQ(z)
  \\&= \sqrt{2}\; \left(\frac{z+1}{2z}\right)^8 \fQ(z)
  \\&= A_0\;
       \frac{(z+1)^8(z-r_1)(z-r_2)(z-r_2^\ast)(z-r_4)(z-r_4^\ast)(z-r_6)(z-r_6^\ast) }{z^{8}}
  \\&= \frac{h_0z^{15} + h_1z^{14} + h_2z^{13} + \cdots + h_{13}z^2 + h_{14}z + h_{15}}
            {z^{8}}
\end{align*}
$\begin{array}{|lcr|lcr|lcr|}
  \hline
     h_{ 0} &=& +0.0544158422430650    & h_{ 6} &=& +0.0004724845739543  & h_{11} &=& +0.0087460940474007
  \\ h_{ 1} &=& +0.3128715909140905    & h_{ 7} &=& +0.1287474266204180  & h_{12} &=& -0.0048703529934458
  \\ h_{ 2} &=& +0.6756307362969000    & h_{ 8} &=& -0.0173693010017349  & h_{13} &=& -0.0003917403733767
  \\ h_{ 3} &=& +0.5853546836540295    & h_{ 9} &=& -0.0440882539307343  & h_{14} &=& +0.0006754494064499
  \\ h_{ 4} &=& -0.0158291052560285    & h_{10} &=& +0.0139810279173888  & h_{15} &=& -0.0001174767841246
  \\ h_{ 5} &=& -0.2840155429611567    &        & &                      &        & &
  \\\hline
\end{array}$
\end{proof}
\begin{table}
\scriptsize
\exbox{\renewcommand{\arraystretch}{1}
\begin{array}[t]{rrr}
   p &    n &  \mc{1}{c}{h_n}                               \\\hline
   5 &    0 &  0.1601023980                                 \\
     &    1 &  0.6038292698                                 \\
     &    2 &  0.7243085284                                 \\
     &    3 &  0.1384281459                                 \\
     &    4 & -0.2422948871                                 \\
     &    5 & -0.0322448696                                 \\
     &    6 &  0.0775714938                                 \\
     &    7 & -0.0062414902                                 \\
     &    8 & -0.0125807520                                 \\
     &    9 &  0.0033357253                                 \\
                                                            \\\hline
   6 &    0 &  0.1115407434                                 \\
     &    1 &  0.4946238904                                 \\
     &    2 &  0.7511339080                                 \\
     &    3 &  0.3152503517                                 \\
     &    4 & -0.2262646940                                 \\
     &    5 & -0.1297668676                                 \\
     &    6 &  0.0975016056                                 \\
     &    7 &  0.0275228655                                 \\
     &    8 & -0.0315820393                                 \\
     &    9 &  0.0005538422                                 \\
     &   10 &  0.0047772575                                 \\
     &   11 & -0.0010773011                                 \\
                                                            \\\hline
   7 &    0 &  0.0778520541                                 \\
     &    1 &  0.3965393195                                 \\
     &    2 &  0.7291320908                                 \\
     &    3 &  0.4697822874                                 \\
     &    4 & -0.1439060039                                 \\
     &    5 & -0.2240361850                                 \\
     &    6 &  0.0713092193                                 \\
     &    7 &  0.0806126092                                 \\
     &    8 & -0.0380299369                                 \\
     &    9 & -0.0165745416                                 \\
     &   10 &  0.0125509986                                 \\
     &   11 &  0.0004295780                                 \\
     &   12 & -0.0018016407                                 \\
     &   13 &  0.0003537138                                 \\
\end{array}
\hspace{1cm}
\begin{array}[t]{rrr}
   p &    n &  \mc{1}{c}{h_n}                               \\\hline
   8 &    0 &  0.0544158422                                 \\
     &    1 &  0.3128715909                                 \\
     &    2 &  0.6756307363                                 \\
     &    3 &  0.5853546837                                 \\
     &    4 & -0.0158291053                                 \\
     &    5 & -0.2840155430                                 \\
     &    6 &  0.0004724846                                 \\
     &    7 &  0.1287474266                                 \\
     &    8 & -0.0173693010                                 \\
     &    9 & -0.0440882539                                 \\
     &   10 &  0.0139810279                                 \\
     &   11 &  0.0087460940                                 \\
     &   12 & -0.0048703530                                 \\
     &   13 & -0.0003917404                                 \\
     &   14 &  0.0006754494                                 \\
     &   15 & -0.0001174768                                 \\
                                                            \\\hline
   9 &    0 &  0.0380779474                                 \\
     &    1 &  0.2438346746                                 \\
     &    2 &  0.6048231237                                 \\
     &    3 &  0.6572880781                                 \\
     &    4 &  0.1331973858                                 \\
     &    5 & -0.2932737833                                 \\
     &    6 & -0.0968407832                                 \\
     &    7 &  0.1485407493                                 \\
     &    8 &  0.0307256815                                 \\
     &    9 & -0.0676328291                                 \\
     &   10 &  0.0002509471                                 \\
     &   11 &  0.0223616621                                 \\
     &   12 & -0.0047232048                                 \\
     &   13 & -0.0042815037                                 \\
     &   14 &  0.0018476469                                 \\
     &   15 &  0.0002303858                                 \\
     &   16 & -0.0002519632                                 \\
     &   17 &  0.0000393473                                 \\
\end{array}
\hspace{1cm}
\begin{array}[t]{rrr}
   p &    n &  \mc{1}{c}{h_n}                               \\
   \hline
  10 &    0 &  0.0266700579                                 \\
     &    1 &  0.1881768001                                 \\
     &    2 &  0.5272011889                                 \\
     &    3 &  0.6884590395                                 \\
     &    4 &  0.2811723437                                 \\
     &    5 & -0.2498464243                                 \\
     &    6 & -0.1959462744                                 \\
     &    7 &  0.1273693403                                 \\
     &    8 &  0.0930573646                                 \\
     &    9 & -0.0713941472                                 \\
     &   10 & -0.0294575368                                 \\
     &   11 &  0.0332126741                                 \\
     &   12 &  0.0036065536                                 \\
     &   13 & -0.0107331755                                 \\
     &   14 &  0.0013953517                                 \\
     &   15 &  0.0019924053                                 \\
     &   16 & -0.0006858567                                 \\
     &   17 & -0.0001164669                                 \\
     &   18 &  0.0000935887                                 \\
     &   19 & -0.0000132642                                 \\
                                                            \\\hline
  11 &    0 &  0.0186942978                                 \\
     &    1 &  0.1440670212                                 \\
     &    2 &  0.4498997644                                 \\
     &    3 &  0.6856867749                                 \\
     &    4 &  0.4119643689                                 \\
     &    5 & -0.1622752450                                 \\
     &    6 & -0.2742308468                                 \\
     &    7 &  0.0660435882                                 \\
     &    8 &  0.1498120125                                 \\
     &    9 & -0.0464799551                                 \\
     &   10 & -0.0664387857                                 \\
     &   11 &  0.0313350902                                 \\
     &   12 &  0.0208409044                                 \\
     &   13 & -0.0153648209                                 \\
     &   14 & -0.0033408589                                 \\
     &   15 &  0.0049284177                                 \\
     &   16 & -0.0003085929                                 \\
     &   17 & -0.0008930233                                 \\
     &   18 &  0.0002491525                                 \\
     &   19 &  0.0000544391                                 \\
     &   20 & -0.0000346350                                 \\
     &   21 &  0.0000044943                                 \\
\end{array}
 %                                                          \\
\hspace{1cm}
\begin{array}[t]{rrr}
   p &    n &  \mc{1}{c}{h_n}                               \\
   \hline
   12 &    0 &  0.0131122580                                 \\
      &    1 &  0.1095662728                                 \\
      &    2 &  0.3773551352                                 \\
      &    3 &  0.6571987226                                 \\
      &    4 &  0.5158864784                                 \\
      &    5 & -0.0447638857                                 \\
      &    6 & -0.3161784538                                 \\
      &    7 & -0.0237792573                                 \\
      &    8 &  0.1824786059                                 \\
      &    9 &  0.0053595697                                 \\
      &   10 & -0.0964321201                                 \\
      &   11 &  0.0108491303                                 \\
      &   12 &  0.0415462775                                 \\
      &   13 & -0.0122186491                                 \\
      &   14 & -0.0128408252                                 \\
      &   15 &  0.0067114990                                 \\
      &   16 &  0.0022486072                                 \\
      &   17 & -0.0021795036                                 \\
      &   18 &  0.0000065451                                 \\
      &   19 &  0.0003886531                                 \\
      &   20 & -0.0000885041                                 \\
      &   21 & -0.0000242415                                 \\
      &   22 &  0.0000127770                                 \\
      &   23 & -0.0000015291                                 \\
\end{array}
}
\caption{
  Daubechies-$p$ scaling coefficients $\seq{h_n}{}$
  \label{tbl:Dp_h}
  }
\end{table}







%\begin{longtable}{cll}
%  \caption{
%    Daubechies-$p$ scaling and wavelet functions and coefficients for $p=1,2,\ldots, 12$
%    \label{fig:Dp_swhg}
%    } \\
%    p & $\fphi(t)$ and $\seqn{h_n}$ & $\fpsi(t)$ and $\seq{g_n}{}$  \\
%    \hline
%  \endfirsthead
%    p & $\fphi(t)$ and $\seq{h_n}{}$ & $\fpsi(t)$ and $\seq{g_n}{}$  \\
%    \hline
%  \endhead
%    \hline
%    \mc{3}{r}{{\emph continued on next page\ldots}}
%  \endfoot
%    \hline
%    \hline
%  \endlastfoot
%    \raisebox{15mm}{5}
%    & \psset{xunit=4mm,yunit=10mm}%============================================================================
% Daniel J. Greenhoe
% LaTeX file
% Daubechies-p5 scaling function and coefficients
% nominal xunit =  4mm
% nominal yunit = 10mm
% nominal font size = \scriptsize
%============================================================================
\begin{pspicture}(-1.3,-1.5)(9.5,1.5)%
  %-------------------------------------
  % axes
  %-------------------------------------
  \psaxes[linecolor=axis,linewidth=0.75pt,yAxis=false,labelsep=2pt,labels=none]{->}(0,0)(0,-1.5)(9.5,1.5)%
  \psaxes[linecolor=axis,linewidth=0.75pt,xAxis=false,labelsep=2pt]{<->}(0,0)(0,-1.5)(9.5,1.5)%
  %-------------------------------------
  % plot
  %-------------------------------------
  \psline[linecolor=red]{-o}(0, 0)(0,  0.1601023980)%  
  \psline[linecolor=red]{-o}(1, 0)(1,  0.6038292698)%  
  \psline[linecolor=red]{-o}(2, 0)(2,  0.7243085284)%  
  \psline[linecolor=red]{-o}(3, 0)(3,  0.1384281459)%  
  \psline[linecolor=red]{-o}(4, 0)(4, -0.2422948871)%  
  \psline[linecolor=red]{-o}(5, 0)(5, -0.0322448696)%  
  \psline[linecolor=red]{-o}(6, 0)(6,  0.0775714938)%  
  \psline[linecolor=red]{-o}(7, 0)(7, -0.0062414902)%  
  \psline[linecolor=red]{-o}(8, 0)(8, -0.0125807520)%  
  \psline[linecolor=red]{-o}(9, 0)(9,  0.0033357253)%  
  \fileplot{../../common/wavelets/graphics/d5_phi.dat}%
  %-------------------------------------
  % labels
  %-------------------------------------
  \uput{2mm}[-90](1,0){$1$}%
  \uput{2mm}[-90](2,0){$2$}%
  \uput{2mm}[-90](3,0){$3$}%
  \uput{2mm}[ 90](4,0){$4$}%
  \uput{2mm}[ 90](5,0){$5$}%
  \uput{2mm}[-90](6,0){$6$}%
  \uput{2mm}[ 90](7,0){$7$}%
  \uput{2mm}[ 90](8,0){$8$}%
  \uput{2mm}[-90](9,0){$9$}%
  \rput[tl](0,1.5){\quad$\fphi(t)$ and $\seqn{h_n}$ for $p=5$}
\end{pspicture}%
%    & \psset{xunit=4mm,yunit=10mm}%============================================================================
% Daniel J. Greenhoe
% LaTeX file
% Daubechies-p5 wavelet function and coefficients
%============================================================================
\begin{pspicture}(-1.3,-1.5)(9.5,1.5)%
  %-------------------------------------
  % axes
  %-------------------------------------
  \psaxes[linecolor=axis,linewidth=0.75pt,yAxis=false,labelsep=2pt,labels=none]{->}(0,0)(0,-1.5)(9.5,1.5)%
  \psaxes[linecolor=axis,linewidth=0.75pt,xAxis=false,labelsep=2pt]{<->}(0,0)(0,-1.5)(9.5,1.5)%
  %-------------------------------------
  % plot
  %-------------------------------------
  \psline[linecolor=red]{-o}(0, 0)(0,  0.0033357253)%  
  \psline[linecolor=red]{-o}(1, 0)(1,  0.0125807520)%  
  \psline[linecolor=red]{-o}(2, 0)(2, -0.0062414902)%  
  \psline[linecolor=red]{-o}(3, 0)(3, -0.0775714938)%  
  \psline[linecolor=red]{-o}(4, 0)(4, -0.0322448696)%  
  \psline[linecolor=red]{-o}(5, 0)(5,  0.2422948871)%  
  \psline[linecolor=red]{-o}(6, 0)(6,  0.1384281459)%  
  \psline[linecolor=red]{-o}(7, 0)(7, -0.7243085284)%  
  \psline[linecolor=red]{-o}(8, 0)(8,  0.6038292698)%  
  \psline[linecolor=red]{-o}(9, 0)(9, -0.1601023980)%  
  \fileplot{../../common/wavelets/graphics/d5_psi.dat}%
  %-------------------------------------
  % labels
  %-------------------------------------
  \uput{2mm}[-90](1,0){$1$}%
  \uput{2mm}[ 90](2,0){$2$}%
  \uput{2mm}[ 90](3,0){$3$}%
  \uput{2mm}[ 90](4,0){$4$}%
  \uput{2mm}[-90](5,0){$5$}%
  \uput{2mm}[-90](6,0){$6$}%
  \uput{2mm}[ 90](7,0){$7$}%
  \uput{2mm}[-90](8,0){$8$}%
  \uput{2mm}[ 90](9,0){$9$}%
  \rput[tl](0,1.5){\quad$\fpsi(t)$ and $\seqn{g_n}$ for $p=5$}
\end{pspicture}%
%    \\
%    \raisebox{15mm}{6}
%    & \psset{xunit=4mm,yunit=10mm}%============================================================================
% Daniel J. Greenhoe
% LaTeX file
% Daubechies-p6 scaling function and coefficients
% nominal xunit =  4mm
% nominal yunit = 10mm
% nominal font size = \scriptsize
%============================================================================
\begin{pspicture}(-1.3,-1.5)(11.5,1.5)%
  %-------------------------------------
  % axes
  %-------------------------------------
  \psaxes[linecolor=axis,linewidth=0.75pt,yAxis=false,labelsep=2pt,labels=none]{->}(0,0)(0,-1.5)(11.5,1.5)%
  \psaxes[linecolor=axis,linewidth=0.75pt,xAxis=false,labelsep=2pt]{<->}(0,0)(0,-1.5)(11.5,1.5)%
  %-------------------------------------
  % plot
  %-------------------------------------
  \psline[linecolor=red]{-o}( 0, 0)( 0,  0.1115407434)%  
  \psline[linecolor=red]{-o}( 1, 0)( 1,  0.4946238904)%  
  \psline[linecolor=red]{-o}( 2, 0)( 2,  0.7511339080)%  
  \psline[linecolor=red]{-o}( 3, 0)( 3,  0.3152503517)%  
  \psline[linecolor=red]{-o}( 4, 0)( 4, -0.2262646940)%  
  \psline[linecolor=red]{-o}( 5, 0)( 5, -0.1297668676)%  
  \psline[linecolor=red]{-o}( 6, 0)( 6,  0.0975016056)%  
  \psline[linecolor=red]{-o}( 7, 0)( 7,  0.0275228655)%  
  \psline[linecolor=red]{-o}( 8, 0)( 8, -0.0315820393)%  
  \psline[linecolor=red]{-o}( 9, 0)( 9,  0.0005538422)%  
  \psline[linecolor=red]{-o}(10, 0)(10,  0.0047772575)%  
  \psline[linecolor=red]{-o}(11, 0)(11, -0.0010773011)%  
  \fileplot{../../common/wavelets/graphics/d6_phi.dat}%
  %-------------------------------------
  % labels
  %-------------------------------------
  \uput{2mm}[-90]( 1,0){ $1$}%
  \uput{2mm}[-90]( 2,0){ $2$}%
  \uput{2mm}[-90]( 3,0){ $3$}%
  \uput{2mm}[ 90]( 4,0){ $4$}%
  \uput{2mm}[ 90]( 5,0){ $5$}%
  \uput{2mm}[-90]( 6,0){ $6$}%
  \uput{2mm}[-90]( 7,0){ $7$}%
  \uput{2mm}[-90]( 8,0){ $8$}%
  \uput{2mm}[-90]( 9,0){ $9$}%
  \uput{2mm}[-90](10,0){$10$}%
  \uput{2mm}[-90](11,0){$11$}%
  \rput[tl](0,1.5){\quad$\fphi(t)$ and $\seqn{h_n}$ for $p=6$}%
\end{pspicture}%
%    & \psset{xunit=4mm,yunit=10mm}%============================================================================
% Daniel J. Greenhoe
% LaTeX file
% Daubechies-p6 wavelet function and coefficients
% nominal xunit =  4mm
% nominal yunit = 10mm
% nominal font size = \scriptsize
%============================================================================
\begin{pspicture}(-1.3,-1.5)(11.5,1.5)%
  %-------------------------------------
  % axes
  %-------------------------------------
  \psaxes[linecolor=axis,linewidth=0.75pt,yAxis=false,labelsep=2pt,labels=none]{->}(0,0)(0,-1.5)(11.5,1.5)%
  \psaxes[linecolor=axis,linewidth=0.75pt,xAxis=false,labelsep=2pt]{<->}(0,0)(0,-1.5)(11.5,1.5)%
  %-------------------------------------
  % plot
  %-------------------------------------
  \psline[linecolor=red]{-o}( 0, 0)( 0, -0.0010773011)%  
  \psline[linecolor=red]{-o}( 1, 0)( 1, -0.0047772575)%  
  \psline[linecolor=red]{-o}( 2, 0)( 2,  0.0005538422)%  
  \psline[linecolor=red]{-o}( 3, 0)( 3,  0.0315820393)%  
  \psline[linecolor=red]{-o}( 4, 0)( 4,  0.0275228655)%  
  \psline[linecolor=red]{-o}( 5, 0)( 5, -0.0975016056)%  
  \psline[linecolor=red]{-o}( 6, 0)( 6, -0.1297668676)%  
  \psline[linecolor=red]{-o}( 7, 0)( 7,  0.2262646940)%  
  \psline[linecolor=red]{-o}( 8, 0)( 8,  0.3152503517)%  
  \psline[linecolor=red]{-o}( 9, 0)( 9, -0.7511339080)%  
  \psline[linecolor=red]{-o}(10, 0)(10,  0.4946238904)%  
  \psline[linecolor=red]{-o}(11, 0)(11, -0.1115407434)%  
  \fileplot{../../common/wavelets/graphics/d6_psi.dat}%
  %-------------------------------------
  % labels
  %-------------------------------------
  \uput{2mm}[-90]( 1,0){ $1$}%
  \uput{2mm}[-90]( 2,0){ $2$}%
  \uput{2mm}[-90]( 3,0){ $3$}%
  \uput{2mm}[-90]( 4,0){ $4$}%
  \uput{2mm}[ 90]( 5,0){ $5$}%
  \uput{2mm}[ 90]( 6,0){ $6$}%
  \uput{2mm}[-90]( 7,0){ $7$}%
  \uput{2mm}[-90]( 8,0){ $8$}%
  \uput{2mm}[ 90]( 9,0){ $9$}%
  \uput{2mm}[ 90](10,0){$10$}%
  \uput{2mm}[ 90](11,0){$11$}%
  \rput[tl](0,1.5){\quad$\fpsi(t)$ and $\seqn{g_n}$ for $p=6$}%
\end{pspicture}%
%    \\
%    \raisebox{15mm}{7}
%    & \psset{xunit=4mm,yunit=10mm}%============================================================================
% Daniel J. Greenhoe
% LaTeX file
% Daubechies-p7 scaling function and coefficients
% nominal xunit =  4mm
% nominal yunit = 10mm
% nominal font size = \scriptsize
%============================================================================
\begin{pspicture}(-1.3,-1.5)(13.5,1.5)%
  %-------------------------------------
  % axes
  %-------------------------------------
  \psaxes[linecolor=axis,linewidth=0.75pt,yAxis=false,labelsep=2pt,labels=none]{->}(0,0)(0,-1.5)(13.5,1.5)%
  \psaxes[linecolor=axis,linewidth=0.75pt,xAxis=false,labelsep=2pt]{<->}(0,0)(0,-1.5)(13.5,1.5)%
  %-------------------------------------
  % plot
  %-------------------------------------
  \psline[linecolor=red]{-o}( 0, 0)( 0, 0.0778520541)%  
  \psline[linecolor=red]{-o}( 1, 0)( 1, 0.3965393195)%  
  \psline[linecolor=red]{-o}( 2, 0)( 2, 0.7291320908)%  
  \psline[linecolor=red]{-o}( 3, 0)( 3, 0.4697822874)%  
  \psline[linecolor=red]{-o}( 4, 0)( 4,-0.1439060039)%  
  \psline[linecolor=red]{-o}( 5, 0)( 5,-0.2240361850)%  
  \psline[linecolor=red]{-o}( 6, 0)( 6, 0.0713092193)%  
  \psline[linecolor=red]{-o}( 7, 0)( 7, 0.0806126092)%  
  \psline[linecolor=red]{-o}( 8, 0)( 8,-0.0380299369)%  
  \psline[linecolor=red]{-o}( 9, 0)( 9,-0.0165745416)%  
  \psline[linecolor=red]{-o}(10, 0)(10, 0.0125509986)%  
  \psline[linecolor=red]{-o}(11, 0)(11, 0.0004295780)%  
  \psline[linecolor=red]{-o}(12, 0)(12,-0.0018016407)%  
  \psline[linecolor=red]{-o}(13, 0)(13, 0.0003537138)%  
  \fileplot{../../common/wavelets/graphics/d7_phi.dat}%
  %-------------------------------------
  % labels
  %-------------------------------------
  \uput{2mm}[-90]( 1,0){ $1$}%
  \uput{2mm}[-90]( 2,0){ $2$}%
  \uput{2mm}[-90]( 3,0){ $3$}%
  \uput{2mm}[ 90]( 4,0){ $4$}%
  \uput{2mm}[ 90]( 5,0){ $5$}%
  \uput{2mm}[-90]( 6,0){ $6$}%
  \uput{2mm}[-90]( 7,0){ $7$}%
  \uput{2mm}[-90]( 8,0){ $8$}%
  \uput{2mm}[-90]( 9,0){ $9$}%
  \uput{2mm}[-90](10,0){$10$}%
  \uput{2mm}[-90](11,0){$11$}%
  \uput{2mm}[-90](12,0){$12$}%
  \uput{2mm}[-90](13,0){$13$}%
  \rput[tl](0,1.5){\quad$\fphi(t)$ and $\seqn{h_n}$ for $p=7$}
\end{pspicture}%
%    & \psset{xunit=4mm,yunit=10mm}%============================================================================
% Daniel J. Greenhoe
% LaTeX file
% Daubechies-p7 wavelet function and coefficients
% nominal xunit =  4mm
% nominal yunit = 10mm
% nominal font size = \scriptsize
%============================================================================
\begin{pspicture}(-1.3,-1.5)(13.5,1.5)%
  %-------------------------------------
  % axes
  %-------------------------------------
  \psaxes[linecolor=axis,linewidth=0.75pt,yAxis=false,labelsep=2pt,labels=none]{->}(0,0)(0,-1.5)(13.5,1.5)%
  \psaxes[linecolor=axis,linewidth=0.75pt,xAxis=false,labelsep=2pt]{<->}(0,0)(0,-1.5)(13.5,1.5)%
  %-------------------------------------
  % plot
  %-------------------------------------
  \psline[linecolor=red]{-o}( 0, 0)( 0, 0.0003537138)%  
  \psline[linecolor=red]{-o}( 1, 0)( 1, 0.0018016407)%  
  \psline[linecolor=red]{-o}( 2, 0)( 2, 0.0004295780)%  
  \psline[linecolor=red]{-o}( 3, 0)( 3,-0.0125509986)%  
  \psline[linecolor=red]{-o}( 4, 0)( 4,-0.0165745416)%  
  \psline[linecolor=red]{-o}( 5, 0)( 5, 0.0380299369)%  
  \psline[linecolor=red]{-o}( 6, 0)( 6, 0.0806126092)%  
  \psline[linecolor=red]{-o}( 7, 0)( 7,-0.0713092193)%  
  \psline[linecolor=red]{-o}( 8, 0)( 8,-0.2240361850)%  
  \psline[linecolor=red]{-o}( 9, 0)( 9, 0.1439060039)%  
  \psline[linecolor=red]{-o}(10, 0)(10, 0.4697822874)%  
  \psline[linecolor=red]{-o}(11, 0)(11,-0.7291320908)%  
  \psline[linecolor=red]{-o}(12, 0)(12, 0.3965393195)%  
  \psline[linecolor=red]{-o}(13, 0)(13,-0.0778520541)%  
  \fileplot{../../common/wavelets/graphics/d7_psi.dat}%
  %-------------------------------------
  % labels
  %-------------------------------------
  \uput{2mm}[-90]( 1,0){ $1$}%
  \uput{2mm}[-90]( 2,0){ $2$}%
  \uput{2mm}[-90]( 3,0){ $3$}%
  \uput{2mm}[-90]( 4,0){ $4$}%
  \uput{2mm}[-90]( 5,0){ $5$}%
  \uput{2mm}[-90]( 6,0){ $6$}%
  \uput{2mm}[-90]( 7,0){ $7$}%
  \uput{2mm}[-90]( 8,0){ $8$}%
  \uput{2mm}[-90]( 9,0){ $9$}%
  \uput{2mm}[-90](10,0){$10$}%
  \uput{2mm}[-90](11,0){$11$}%
  \uput{2mm}[-90](12,0){$12$}%
  \uput{2mm}[-90](13,0){$13$}%
  \rput[tl](0,1.5){\quad$\fpsi(t)$ and $\seqn{g_n}$ for $p=7$}
\end{pspicture}%
%    \\
%    \raisebox{15mm}{8}
%    & \psset{xunit=4mm,yunit=10mm}%============================================================================
% Daniel J. Greenhoe
% LaTeX file
% Daubechies-p8 scaling function and coefficients
% nominal xunit =  4mm
% nominal yunit = 10mm
% nominal font size = \scriptsize
%============================================================================
\begin{pspicture}(-1.3,-1.5)(15.5,1.5)%
  %-------------------------------------
  % axes
  %-------------------------------------
  \psaxes[linecolor=axis,linewidth=0.75pt,yAxis=false,labelsep=2pt,labels=none]{->}(0,0)(0,-1.5)(15.5,1.5)%
  \psaxes[linecolor=axis,linewidth=0.75pt,xAxis=false,labelsep=2pt]{<->}(0,0)(0,-1.5)(15.5,1.5)%
  %-------------------------------------
  % plot
  %-------------------------------------
  \psline[linecolor=red]{-o}(0, 0)(0,  0.0544158422)%  
  \psline[linecolor=red]{-o}(1, 0)(1,  0.3128715909)%  
  \psline[linecolor=red]{-o}(2, 0)(2,  0.6756307363)%  
  \psline[linecolor=red]{-o}(3, 0)(3,  0.5853546837)%  
  \psline[linecolor=red]{-o}(4, 0)(4, -0.0158291053)%  
  \psline[linecolor=red]{-o}(5, 0)(5, -0.2840155430)%  
  \psline[linecolor=red]{-o}(6, 0)(6,  0.0004724846)%  
  \psline[linecolor=red]{-o}(7, 0)(7,  0.1287474266)%  
  \psline[linecolor=red]{-o}(8, 0)(8, -0.0173693010)%  
  \psline[linecolor=red]{-o}(9, 0)(9, -0.0440882539)%  
  \psline[linecolor=red]{-o}(10,0)(10, 0.0139810279)%  
  \psline[linecolor=red]{-o}(11,0)(11, 0.0087460940)%  
  \psline[linecolor=red]{-o}(12,0)(12,-0.0048703530)%  
  \psline[linecolor=red]{-o}(13,0)(13,-0.0003917404)%  
  \psline[linecolor=red]{-o}(14,0)(14, 0.0006754494)%  
  \psline[linecolor=red]{-o}(15,0)(15,-0.0001174768)%  
  \fileplot{../../common/wavelets/graphics/d8_phi.dat}%
  %-------------------------------------
  % labels
  %-------------------------------------
  \uput{2mm}[-90]( 1,0){ $1$}%
  \uput{2mm}[-90]( 2,0){ $2$}%
  \uput{2mm}[-90]( 3,0){ $3$}%
  \uput{2mm}[-90]( 4,0){ $4$}%
  \uput{2mm}[ 90]( 5,0){ $5$}%
  \uput{2mm}[-90]( 6,0){ $6$}%
  \uput{2mm}[-90]( 7,0){ $7$}%
  \uput{2mm}[-90]( 8,0){ $8$}%
  \uput{2mm}[-90]( 9,0){ $9$}%
  \uput{2mm}[-90](10,0){$10$}%
  \uput{2mm}[-90](11,0){$11$}%
  \uput{2mm}[-90](12,0){$12$}%
  \uput{2mm}[-90](13,0){$13$}%
  \uput{2mm}[-90](14,0){$14$}%
  \uput{2mm}[-90](15,0){$15$}%
  \rput[tl](0,1.5){\quad$\fphi(t)$ and $\seqn{h_n}$ for $p=8$}%
\end{pspicture}%
%    & \psset{xunit=4mm,yunit=10mm}%============================================================================
% Daniel J. Greenhoe
% LaTeX file
% Daubechies-p8 wavelet function and coefficients
%============================================================================
\begin{pspicture}(-1.3,-1.5)(15.5,1.5)%
  %-------------------------------------
  % axes
  %-------------------------------------
  \psaxes[linecolor=axis,linewidth=0.75pt,yAxis=false,labelsep=2pt,labels=none]{->}(0,0)(0,-1.5)(15.5,1.5)%
  \psaxes[linecolor=axis,linewidth=0.75pt,xAxis=false,labelsep=2pt]{<->}(0,0)(0,-1.5)(15.5,1.5)%
  %-------------------------------------
  % plot
  %-------------------------------------
  \psline[linecolor=red]{-o}(0, 0)(0, -0.0001174768)%  
  \psline[linecolor=red]{-o}(1, 0)(1, -0.0006754494)%  
  \psline[linecolor=red]{-o}(2, 0)(2, -0.0003917404)%  
  \psline[linecolor=red]{-o}(3, 0)(3,  0.0048703530)%  
  \psline[linecolor=red]{-o}(4, 0)(4,  0.0087460940)%  
  \psline[linecolor=red]{-o}(5, 0)(5, -0.0139810279)%  
  \psline[linecolor=red]{-o}(6, 0)(6, -0.0440882539)%  
  \psline[linecolor=red]{-o}(7, 0)(7,  0.0173693010)%  
  \psline[linecolor=red]{-o}(8, 0)(8,  0.1287474266)%  
  \psline[linecolor=red]{-o}(9, 0)(9, -0.0004724846)%  
  \psline[linecolor=red]{-o}(10,0)(10,-0.2840155430)%  
  \psline[linecolor=red]{-o}(11,0)(11, 0.0158291053)%  
  \psline[linecolor=red]{-o}(12,0)(12, 0.5853546837)%  
  \psline[linecolor=red]{-o}(13,0)(13,-0.6756307363)%  
  \psline[linecolor=red]{-o}(14,0)(14, 0.3128715909)%  
  \psline[linecolor=red]{-o}(15,0)(15,-0.0544158422)%  
  \fileplot{../../common/wavelets/graphics/d8_psi.dat}%
  %-------------------------------------
  % labels
  %-------------------------------------
  \uput{2mm}[-90]( 1,0){ $1$}%
  \uput{2mm}[-90]( 2,0){ $2$}%
  \uput{2mm}[-90]( 3,0){ $3$}%
  \uput{2mm}[-90]( 4,0){ $4$}%
  \uput{2mm}[-90]( 5,0){ $5$}%
  \uput{2mm}[-90]( 6,0){ $6$}%
  \uput{2mm}[-90]( 7,0){ $7$}%
  \uput{2mm}[-90]( 8,0){ $8$}%
  \uput{2mm}[-90]( 9,0){ $9$}%
  \uput{2mm}[ 90](10,0){$10$}%
  \uput{2mm}[-90](11,0){$11$}%
  \uput{2mm}[-90](12,0){$12$}%
  \uput{2mm}[-90](13,0){$13$}%
  \uput{2mm}[-90](14,0){$14$}%
  \uput{2mm}[-90](15,0){$15$}%
  \rput[tl](0,1.5){\quad$\fpsi(t)$ and $\seqn{g_n}$ for $p=8$}
\end{pspicture}%
%    \\
%    \raisebox{15mm}{9}
%    & \psset{xunit=3mm,yunit=10mm}%============================================================================
% Daniel J. Greenhoe
% LaTeX file
% Daubechies-p9 wavelet function and coefficients
% nominal xunit =  3mm
% nominal yunit = 10mm
% nominal font size = \scriptsize
%============================================================================
\begin{pspicture}(-1.75,-1.5)(18,1.5)%
  %-------------------------------------
  % axes
  %-------------------------------------
  \psaxes[linecolor=axis,linewidth=0.75pt,yAxis=false,labelsep=2pt,labels=none]{->}(0,0)(0,-1.5)(18,1.5)%
  \psaxes[linecolor=axis,linewidth=0.75pt,xAxis=false,labelsep=2pt]{<->}(0,0)(0,-1.5)(18,1.5)%
  %-------------------------------------
  % plot
  %-------------------------------------
  \psline[linecolor=red]{-o}( 0, 0)( 0, 0.0380779474)%  
  \psline[linecolor=red]{-o}( 1, 0)( 1, 0.2438346746)%  
  \psline[linecolor=red]{-o}( 2, 0)( 2, 0.6048231237)%  
  \psline[linecolor=red]{-o}( 3, 0)( 3, 0.6572880781)%  
  \psline[linecolor=red]{-o}( 4, 0)( 4, 0.1331973858)%  
  \psline[linecolor=red]{-o}( 5, 0)( 5,-0.2932737833)%  
  \psline[linecolor=red]{-o}( 6, 0)( 6,-0.0968407832)%  
  \psline[linecolor=red]{-o}( 7, 0)( 7, 0.1485407493)%  
  \psline[linecolor=red]{-o}( 8, 0)( 8, 0.0307256815)%  
  \psline[linecolor=red]{-o}( 9, 0)( 9,-0.0676328291)%  
  \psline[linecolor=red]{-o}(10, 0)(10, 0.0002509471)%  
  \psline[linecolor=red]{-o}(11, 0)(11, 0.0223616621)%  
  \psline[linecolor=red]{-o}(12, 0)(12,-0.0047232048)%  
  \psline[linecolor=red]{-o}(13, 0)(13,-0.0042815037)%  
  \psline[linecolor=red]{-o}(14, 0)(14, 0.0018476469)%  
  \psline[linecolor=red]{-o}(15, 0)(15, 0.0002303858)%  
  \psline[linecolor=red]{-o}(16, 0)(16,-0.0002519632)%  
  \psline[linecolor=red]{-o}(17, 0)(17, 0.0000393473)%  
  \fileplot{../../common/wavelets/graphics/d9_phi.dat}%
  %-------------------------------------
  % labels
  %-------------------------------------
  \uput{2mm}[-90]( 1,0){ $1$}%
  \uput{2mm}[-90]( 2,0){ $2$}%
  \uput{2mm}[-90]( 3,0){ $3$}%
  \uput{2mm}[-90]( 4,0){ $4$}%
  \uput{2mm}[ 90]( 5,0){ $5$}%
  \uput{2mm}[-90]( 6,0){ $6$}%
  \uput{2mm}[-90]( 7,0){ $7$}%
  \uput{2mm}[-90]( 8,0){ $8$}%
  \uput{2mm}[-90]( 9,0){ $9$}%
  \uput{2mm}[-90](10,0){$10$}%
  \uput{2mm}[-90](11,0){$11$}%
  \uput{2mm}[-90](12,0){$12$}%
  \uput{2mm}[-90](13,0){$13$}%
  \uput{2mm}[-90](14,0){$14$}%
  \uput{2mm}[-90](15,0){$15$}%
  \uput{2mm}[-90](16,0){$16$}%
  \uput{2mm}[-90](17,0){$17$}%
  \rput[tl](0,1.5){\quad$\fphi(t)$ and $\seqn{h_n}$ for $p=9$}
\end{pspicture}%
%    & \psset{xunit=3mm,yunit=10mm}%============================================================================
% Daniel J. Greenhoe
% LaTeX file
% Daubechies-p9 wavelet function and coefficients
% nominal xunit =  3mm
% nominal yunit = 10mm
% nominal font size = \scriptsize
%============================================================================
\begin{pspicture}(-1.75,-1.5)(18,1.5)%
  %-------------------------------------
  % axes
  %-------------------------------------
  \psaxes[linecolor=axis,linewidth=0.75pt,yAxis=false,labelsep=2pt,labels=none]{->}(0,0)(0,-1.5)(18,1.5)%
  \psaxes[linecolor=axis,linewidth=0.75pt,xAxis=false,labelsep=2pt]{<->}(0,0)(0,-1.5)(18,1.5)%
  %-------------------------------------
  % plot
  %-------------------------------------
  \psline[linecolor=red]{-o}( 0, 0)( 0, 0.0000393473)%  
  \psline[linecolor=red]{-o}( 1, 0)( 1, 0.0002519632)%  
  \psline[linecolor=red]{-o}( 2, 0)( 2, 0.0002303858)%  
  \psline[linecolor=red]{-o}( 3, 0)( 3,-0.0018476469)%  
  \psline[linecolor=red]{-o}( 4, 0)( 4,-0.0042815037)%  
  \psline[linecolor=red]{-o}( 5, 0)( 5, 0.0047232048)%  
  \psline[linecolor=red]{-o}( 6, 0)( 6, 0.0223616621)%  
  \psline[linecolor=red]{-o}( 7, 0)( 7,-0.0002509471)%  
  \psline[linecolor=red]{-o}( 8, 0)( 8,-0.0676328291)%  
  \psline[linecolor=red]{-o}( 9, 0)( 9,-0.0307256815)%  
  \psline[linecolor=red]{-o}(10, 0)(10, 0.1485407493)%  
  \psline[linecolor=red]{-o}(11, 0)(11, 0.0968407832)%  
  \psline[linecolor=red]{-o}(12, 0)(12,-0.2932737833)%  
  \psline[linecolor=red]{-o}(13, 0)(13,-0.1331973858)%  
  \psline[linecolor=red]{-o}(14, 0)(14, 0.6572880781)%  
  \psline[linecolor=red]{-o}(15, 0)(15,-0.6048231237)%  
  \psline[linecolor=red]{-o}(16, 0)(16, 0.2438346746)%  
  \psline[linecolor=red]{-o}(17, 0)(17,-0.0380779474)%  
  \fileplot{../../common/wavelets/graphics/d9_psi.dat}%
  %-------------------------------------
  % labels
  %-------------------------------------
  \uput{2mm}[-90]( 1,0){ $1$}%
  \uput{2mm}[-90]( 2,0){ $2$}%
  \uput{2mm}[-90]( 3,0){ $3$}%
  \uput{2mm}[-90]( 4,0){ $4$}%
  \uput{2mm}[-90]( 5,0){ $5$}%
  \uput{2mm}[-90]( 6,0){ $6$}%
  \uput{2mm}[-90]( 7,0){ $7$}%
  \uput{2mm}[-90]( 8,0){ $8$}%
  \uput{2mm}[-90]( 9,0){ $9$}%
  \uput{2mm}[-90](10,0){$10$}%
  \uput{2mm}[-90](11,0){$11$}%
  \uput{2mm}[ 90](12,0){$12$}%
  \uput{2mm}[-90](13,0){$13$}%
  \uput{2mm}[-90](14,0){$14$}%
  \uput{2mm}[-90](15,0){$15$}%
  \uput{2mm}[-90](16,0){$16$}%
  \uput{2mm}[-90](17,0){$17$}%
  \rput[tl](0,1.5){\quad$\fpsi(t)$ and $\seqn{g_n}$ for $p=9$}
  \end{pspicture}%
%    \\
%    \raisebox{15mm}{10}
%    & \psset{xunit=3mm,yunit=10mm}%============================================================================
% Daniel J. Greenhoe
% LaTeX file
% Daubechies-p10 scaling function and coefficients
% nominal xunit =  3mm
% nominal yunit = 10mm
% nominal font size = \scriptsize
%============================================================================
\begin{pspicture}(-1.75,-1.5)(20,1.5)%
  %-------------------------------------
  % axes
  %-------------------------------------
  \psaxes[linecolor=axis,linewidth=0.75pt,yAxis=false,labelsep=2pt,labels=none]{->}(0,0)(0,-1.5)(20,1.5)%
  \psaxes[linecolor=axis,linewidth=0.75pt,xAxis=false,labelsep=2pt]{<->}(0,0)(0,-1.5)(20,1.5)%
  %-------------------------------------
  % plot
  %-------------------------------------
  \psline[linecolor=red]{-o}( 0, 0)( 0, 0.0266700579)%  
  \psline[linecolor=red]{-o}( 1, 0)( 1, 0.1881768001)%  
  \psline[linecolor=red]{-o}( 2, 0)( 2, 0.5272011889)%  
  \psline[linecolor=red]{-o}( 3, 0)( 3, 0.6884590395)%  
  \psline[linecolor=red]{-o}( 4, 0)( 4, 0.2811723437)%  
  \psline[linecolor=red]{-o}( 5, 0)( 5,-0.2498464243)%  
  \psline[linecolor=red]{-o}( 6, 0)( 6,-0.1959462744)%  
  \psline[linecolor=red]{-o}( 7, 0)( 7, 0.1273693403)%  
  \psline[linecolor=red]{-o}( 8, 0)( 8, 0.0930573646)%  
  \psline[linecolor=red]{-o}( 9, 0)( 9,-0.0713941472)%  
  \psline[linecolor=red]{-o}(10, 0)(10,-0.0294575368)%  
  \psline[linecolor=red]{-o}(11, 0)(11, 0.0332126741)%  
  \psline[linecolor=red]{-o}(12, 0)(12, 0.0036065536)%  
  \psline[linecolor=red]{-o}(13, 0)(13,-0.0107331755)%  
  \psline[linecolor=red]{-o}(14, 0)(14, 0.0013953517)%  
  \psline[linecolor=red]{-o}(15, 0)(15, 0.0019924053)%  
  \psline[linecolor=red]{-o}(16, 0)(16,-0.0006858567)%  
  \psline[linecolor=red]{-o}(17, 0)(17,-0.0001164669)%  
  \psline[linecolor=red]{-o}(18, 0)(18, 0.0000935887)%  
  \psline[linecolor=red]{-o}(19, 0)(19,-0.0000132642)%  
  \fileplot{../../common/wavelets/graphics/d10_phi.dat}%
  %-------------------------------------
  % labels
  %-------------------------------------
  \uput{2mm}[-90]( 1,0){ $1$}%
  \uput{2mm}[-90]( 2,0){ $2$}%
  \uput{2mm}[-90]( 3,0){ $3$}%
  \uput{2mm}[-90]( 4,0){ $4$}%
  \uput{2mm}[ 90]( 5,0){ $5$}%
  \uput{2mm}[ 90]( 6,0){ $6$}%
  \uput{2mm}[-90]( 7,0){ $7$}%
  \uput{2mm}[-90]( 8,0){ $8$}%
  \uput{2mm}[-90]( 9,0){ $9$}%
  \uput{2mm}[-90](10,0){$10$}%
  \uput{2mm}[-90](11,0){$11$}%
  \uput{2mm}[-90](12,0){$12$}%
  \uput{2mm}[-90](13,0){$13$}%
  \uput{2mm}[-90](14,0){$14$}%
  \uput{2mm}[-90](15,0){$15$}%
  \uput{2mm}[-90](16,0){$16$}%
  \uput{2mm}[-90](17,0){$17$}%
  \uput{2mm}[-90](18,0){$18$}%
  \uput{2mm}[-90](19,0){$19$}%
  \rput[tl](0,1.5){\quad$\fphi(t)$ and $\seqn{h_n}$ for $p=10$}
\end{pspicture}%
%    & \psset{xunit=3mm,yunit=10mm}%============================================================================
% Daniel J. Greenhoe
% LaTeX file
% Daubechies-p10 wavelet function and coefficients
% nominal xunit =  3mm
% nominal yunit = 10mm
% nominal font size = \scriptsize
%============================================================================
\begin{pspicture}(-1.75,-1.5)(20,1.5)%
  %-------------------------------------
  % axes
  %-------------------------------------
  \psaxes[linecolor=axis,linewidth=0.75pt,yAxis=false,labelsep=2pt,labels=none]{->}(0,0)(0,-1.5)(20,1.5)%
  \psaxes[linecolor=axis,linewidth=0.75pt,xAxis=false,labelsep=2pt]{<->}(0,0)(0,-1.5)(20,1.5)%
  %-------------------------------------
  % plot
  %-------------------------------------
  \psline[linecolor=red]{-o}( 0, 0)( 0,-0.0000132642)%  
  \psline[linecolor=red]{-o}( 1, 0)( 1,-0.0000935887)%  
  \psline[linecolor=red]{-o}( 2, 0)( 2,-0.0001164669)%  
  \psline[linecolor=red]{-o}( 3, 0)( 3, 0.0006858567)%  
  \psline[linecolor=red]{-o}( 4, 0)( 4, 0.0019924053)%  
  \psline[linecolor=red]{-o}( 5, 0)( 5,-0.0013953517)%  
  \psline[linecolor=red]{-o}( 6, 0)( 6,-0.0107331755)%  
  \psline[linecolor=red]{-o}( 7, 0)( 7,-0.0036065536)%  
  \psline[linecolor=red]{-o}( 8, 0)( 8, 0.0332126741)%  
  \psline[linecolor=red]{-o}( 9, 0)( 9, 0.0294575368)%  
  \psline[linecolor=red]{-o}(10, 0)(10,-0.0713941472)%  
  \psline[linecolor=red]{-o}(11, 0)(11,-0.0930573646)%  
  \psline[linecolor=red]{-o}(12, 0)(12, 0.1273693403)%  
  \psline[linecolor=red]{-o}(13, 0)(13, 0.1959462744)%  
  \psline[linecolor=red]{-o}(14, 0)(14,-0.2498464243)%  
  \psline[linecolor=red]{-o}(15, 0)(15,-0.2811723437)%  
  \psline[linecolor=red]{-o}(16, 0)(16, 0.6884590395)%  
  \psline[linecolor=red]{-o}(17, 0)(17,-0.5272011889)%  
  \psline[linecolor=red]{-o}(18, 0)(18, 0.1881768001)%  
  \psline[linecolor=red]{-o}(19, 0)(19,-0.0266700579)%  
  \fileplot{../../common/wavelets/graphics/d10_psi.dat}%
  %-------------------------------------
  % labels
  %-------------------------------------
  \uput{2mm}[-90]( 1,0){ $1$}%
  \uput{2mm}[-90]( 2,0){ $2$}%
  \uput{2mm}[-90]( 3,0){ $3$}%
  \uput{2mm}[-90]( 4,0){ $4$}%
  \uput{2mm}[-90]( 5,0){ $5$}%
  \uput{2mm}[-90]( 6,0){ $6$}%
  \uput{2mm}[-90]( 7,0){ $7$}%
  \uput{2mm}[ 90]( 8,0){ $8$}%
  \uput{2mm}[-90]( 9,0){ $9$}%
  \uput{2mm}[ 90](10,0){$10$}%
  \uput{2mm}[ 90](11,0){$11$}%
  \uput{2mm}[-90](12,0){$12$}%
  \uput{2mm}[-90](13,0){$13$}%
  \uput{2mm}[ 90](14,0){$14$}%
  \uput{2mm}[ 90](15,0){$15$}%
  \uput{2mm}[-90](16,0){$16$}%
  \uput{2mm}[ 90](17,0){$17$}%
  \uput{2mm}[-90](18,0){$18$}%
  \uput{2mm}[-90](19,0){$19$}%
  \rput[tl](0,1.5){\quad$\fpsi(t)$ and $\seqn{g_n}$ for $p=10$}
\end{pspicture}%
%    \\
%    \raisebox{15mm}{11}
%    & \psset{xunit=3mm,yunit=10mm}%============================================================================
% Daniel J. Greenhoe
% LaTeX file
% Daubechies-p11 scaling function and coefficients
% nominal xunit =  3mm
% nominal yunit = 10mm
% nominal font size = \scriptsize
%============================================================================
\begin{pspicture}(-1.75,-1.5)(22,1.5)%
  %-------------------------------------
  % axes
  %-------------------------------------
  \psaxes[linecolor=axis,linewidth=0.75pt,yAxis=false,labelsep=2pt,labels=none]{->}(0,0)(0,-1.5)(22,1.5)%
  \psaxes[linecolor=axis,linewidth=0.75pt,xAxis=false,labelsep=2pt]{<->}(0,0)(0,-1.5)(22,1.5)%
  %-------------------------------------
  % plot
  %-------------------------------------
  \psline[linecolor=red]{-o}( 0, 0)( 0, 0.0186942978)%  
  \psline[linecolor=red]{-o}( 1, 0)( 1, 0.1440670212)%  
  \psline[linecolor=red]{-o}( 2, 0)( 2, 0.4498997644)%  
  \psline[linecolor=red]{-o}( 3, 0)( 3, 0.6856867749)%  
  \psline[linecolor=red]{-o}( 4, 0)( 4, 0.4119643689)%  
  \psline[linecolor=red]{-o}( 5, 0)( 5,-0.1622752450)%  
  \psline[linecolor=red]{-o}( 6, 0)( 6,-0.2742308468)%  
  \psline[linecolor=red]{-o}( 7, 0)( 7, 0.0660435882)%  
  \psline[linecolor=red]{-o}( 8, 0)( 8, 0.1498120125)%  
  \psline[linecolor=red]{-o}( 9, 0)( 9,-0.0464799551)%  
  \psline[linecolor=red]{-o}(10, 0)(10,-0.0664387857)%  
  \psline[linecolor=red]{-o}(11, 0)(11, 0.0313350902)%  
  \psline[linecolor=red]{-o}(12, 0)(12, 0.0208409044)%  
  \psline[linecolor=red]{-o}(13, 0)(13,-0.0153648209)%  
  \psline[linecolor=red]{-o}(14, 0)(14,-0.0033408589)%  
  \psline[linecolor=red]{-o}(15, 0)(15, 0.0049284177)%  
  \psline[linecolor=red]{-o}(16, 0)(16,-0.0003085929)%  
  \psline[linecolor=red]{-o}(17, 0)(17,-0.0008930233)%  
  \psline[linecolor=red]{-o}(18, 0)(18, 0.0002491525)%  
  \psline[linecolor=red]{-o}(19, 0)(19, 0.0000544391)%  
  \psline[linecolor=red]{-o}(20, 0)(20,-0.0000346350)%  
  \psline[linecolor=red]{-o}(21, 0)(21, 0.0000044943)%  
  \fileplot{../../common/wavelets/graphics/d11_phi.dat}%
  %-------------------------------------
  % labels
  %-------------------------------------
  \uput{2mm}[-90]( 1,0){ $1$}%
  \uput{2mm}[-90]( 2,0){ $2$}%
  \uput{2mm}[-90]( 3,0){ $3$}%
  \uput{2mm}[-90]( 4,0){ $4$}%
  \uput{2mm}[ 90]( 5,0){ $5$}%
  \uput{2mm}[ 90]( 6,0){ $6$}%
  \uput{2mm}[-90]( 7,0){ $7$}%
  \uput{2mm}[-90]( 8,0){ $8$}%
  \uput{2mm}[-90]( 9,0){ $9$}%
  \uput{2mm}[-90](10,0){$10$}%
  \uput{2mm}[-90](11,0){$11$}%
  \uput{2mm}[-90](12,0){$12$}%
  \uput{2mm}[-90](13,0){$13$}%
  \uput{2mm}[-90](14,0){$14$}%
  \uput{2mm}[-90](15,0){$15$}%
  \uput{2mm}[-90](16,0){$16$}%
  \uput{2mm}[-90](17,0){$17$}%
  \uput{2mm}[-90](18,0){$18$}%
  \uput{2mm}[-90](19,0){$19$}%
  \uput{2mm}[-90](20,0){$20$}%
  \uput{2mm}[-90](21,0){$21$}%
  \rput[tl](0,1.5){\quad$\fphi(t)$ and $\seqn{h_n}$ for $p=11$}
\end{pspicture}%
%    & \psset{xunit=3mm,yunit=10mm}%============================================================================
% Daniel J. Greenhoe
% LaTeX file
% Daubechies-p11 wavelet function and coefficients
% nominal xunit =  3mm
% nominal yunit = 10mm
% nominal font size = \scriptsize
%============================================================================
\begin{pspicture}(-1.75,-1.5)(22,1.5)%
  %-------------------------------------
  % axes
  %-------------------------------------
  \psaxes[linecolor=axis,linewidth=0.75pt,yAxis=false,labelsep=2pt,labels=none]{->}(0,0)(0,-1.5)(22,1.5)%
  \psaxes[linecolor=axis,linewidth=0.75pt,xAxis=false,labelsep=2pt]{<->}(0,0)(0,-1.5)(22,1.5)%
  %-------------------------------------
  % plot
  %-------------------------------------
  \psline[linecolor=red]{-o}( 0, 0)( 0, 0.0000044943)%  
  \psline[linecolor=red]{-o}( 1, 0)( 1, 0.0000346350)%  
  \psline[linecolor=red]{-o}( 2, 0)( 2, 0.0000544391)%  
  \psline[linecolor=red]{-o}( 3, 0)( 3,-0.0002491525)%  
  \psline[linecolor=red]{-o}( 4, 0)( 4,-0.0008930233)%  
  \psline[linecolor=red]{-o}( 5, 0)( 5, 0.0003085929)%  
  \psline[linecolor=red]{-o}( 6, 0)( 6, 0.0049284177)%  
  \psline[linecolor=red]{-o}( 7, 0)( 7, 0.0033408589)%  
  \psline[linecolor=red]{-o}( 8, 0)( 8,-0.0153648209)%  
  \psline[linecolor=red]{-o}( 9, 0)( 9,-0.0208409044)%  
  \psline[linecolor=red]{-o}(10, 0)(10, 0.0313350902)%  
  \psline[linecolor=red]{-o}(11, 0)(11, 0.0664387857)%  
  \psline[linecolor=red]{-o}(12, 0)(12,-0.0464799551)%  
  \psline[linecolor=red]{-o}(13, 0)(13,-0.1498120125)%  
  \psline[linecolor=red]{-o}(14, 0)(14, 0.0660435882)%  
  \psline[linecolor=red]{-o}(15, 0)(15, 0.2742308468)%  
  \psline[linecolor=red]{-o}(16, 0)(16,-0.1622752450)%  
  \psline[linecolor=red]{-o}(17, 0)(17,-0.4119643689)%  
  \psline[linecolor=red]{-o}(18, 0)(18, 0.6856867749)%  
  \psline[linecolor=red]{-o}(19, 0)(19,-0.4498997644)%  
  \psline[linecolor=red]{-o}(20, 0)(20, 0.1440670212)%  
  \psline[linecolor=red]{-o}(21, 0)(21,-0.0186942978)%  
  \fileplot{../../common/wavelets/graphics/d11_psi.dat}%
  %-------------------------------------
  % labels
  %-------------------------------------
  \uput{2mm}[-90]( 1,0){ $1$}%
  \uput{2mm}[-90]( 2,0){ $2$}%
  \uput{2mm}[-90]( 3,0){ $3$}%
  \uput{2mm}[-90]( 4,0){ $4$}%
  \uput{2mm}[-90]( 5,0){ $5$}%
  \uput{2mm}[-90]( 6,0){ $6$}%
  \uput{2mm}[-90]( 7,0){ $7$}%
  \uput{2mm}[-90]( 8,0){ $8$}%
  \uput{2mm}[-90]( 9,0){ $9$}%
  \uput{2mm}[-90](10,0){$10$}%
  \uput{2mm}[-90](11,0){$11$}%
  \uput{2mm}[ 90](12,0){$12$}%
  \uput{2mm}[ 90](13,0){$13$}%
  \uput{2mm}[-90](14,0){$14$}%
  \uput{2mm}[-90](15,0){$15$}%
  \uput{2mm}[ 90](16,0){$16$}%
  \uput{2mm}[ 90](17,0){$17$}%
  \uput{2mm}[-90](18,0){$18$}%
  \uput{2mm}[ 90](19,0){$19$}%
  \uput{2mm}[-90](20,0){$20$}%
  \uput{2mm}[-90](21,0){$21$}%
  \rput[tl](0,1.5){\quad$\fpsi(t)$ and $\seqn{g_n}$ for $p=11$}
\end{pspicture}%
%    \\
%%    \raisebox{15mm}{12}
%%    & \psset{xunit=3mm,yunit=10mm}%============================================================================
% Daniel J. Greenhoe
% LaTeX file
% Daubechies-p12 scaling function and coefficients
% nominal xunit =  3mm
% nominal yunit = 10mm
% nominal font size = \scriptsize
%============================================================================
\begin{pspicture}(-1.75,-1.5)(24,1.5)%
  %-------------------------------------
  % axes
  %-------------------------------------
  \psaxes[linecolor=axis,linewidth=0.75pt,yAxis=false,labelsep=2pt,labels=none]{->}(0,0)(0,-1.5)(24,1.5)%
  \psaxes[linecolor=axis,linewidth=0.75pt,xAxis=false,labelsep=2pt]{<->}(0,0)(0,-1.5)(24,1.5)%
  %-------------------------------------
  % plot
  %-------------------------------------
  \psline[linecolor=red]{-o}( 0, 0)( 0, 0.0131122580)%  
  \psline[linecolor=red]{-o}( 1, 0)( 1, 0.1095662728)%  
  \psline[linecolor=red]{-o}( 2, 0)( 2, 0.3773551352)%  
  \psline[linecolor=red]{-o}( 3, 0)( 3, 0.6571987226)%  
  \psline[linecolor=red]{-o}( 4, 0)( 4, 0.5158864784)%  
  \psline[linecolor=red]{-o}( 5, 0)( 5,-0.0447638856)%  
  \psline[linecolor=red]{-o}( 6, 0)( 6,-0.3161784537)%  
  \psline[linecolor=red]{-o}( 7, 0)( 7,-0.0237792573)%  
  \psline[linecolor=red]{-o}( 8, 0)( 8, 0.1824786059)%  
  \psline[linecolor=red]{-o}( 9, 0)( 9, 0.0053595697)%  
  \psline[linecolor=red]{-o}(10, 0)(10,-0.0964321201)%  
  \psline[linecolor=red]{-o}(11, 0)(11, 0.0108491303)%  
  \psline[linecolor=red]{-o}(12, 0)(12, 0.0415462775)%  
  \psline[linecolor=red]{-o}(13, 0)(13,-0.0122186491)%  
  \psline[linecolor=red]{-o}(14, 0)(14,-0.0128408252)%  
  \psline[linecolor=red]{-o}(15, 0)(15, 0.0067114990)%  
  \psline[linecolor=red]{-o}(16, 0)(16, 0.0022486072)%  
  \psline[linecolor=red]{-o}(17, 0)(17,-0.0021795036)%  
  \psline[linecolor=red]{-o}(18, 0)(18, 0.0000065451)%  
  \psline[linecolor=red]{-o}(19, 0)(19, 0.0003886531)%  
  \psline[linecolor=red]{-o}(20, 0)(20,-0.0000885041)%  
  \psline[linecolor=red]{-o}(21, 0)(21,-0.0000242415)%  
  \psline[linecolor=red]{-o}(22, 0)(22, 0.0000127770)%  
  \psline[linecolor=red]{-o}(23, 0)(23,-0.0000015291)%  
  \fileplot{../../common/wavelets/graphics/d12_phi.dat}%
  %-------------------------------------
  % labels
  %-------------------------------------
  \uput{2mm}[-90]( 1,0){ $1$}%
  \uput{2mm}[-90]( 2,0){ $2$}%
  \uput{2mm}[-90]( 3,0){ $3$}%
  \uput{2mm}[-90]( 4,0){ $4$}%
  \uput{2mm}[ 90]( 5,0){ $5$}%
  \uput{2mm}[ 90]( 6,0){ $6$}%
  \uput{2mm}[-90]( 7,0){ $7$}%
  \uput{2mm}[-90]( 8,0){ $8$}%
  \uput{2mm}[-90]( 9,0){ $9$}%
  \uput{2mm}[-90](10,0){$10$}%
  \uput{2mm}[-90](11,0){$11$}%
  \uput{2mm}[-90](12,0){$12$}%
  \uput{2mm}[-90](13,0){$13$}%
  \uput{2mm}[-90](14,0){$14$}%
  \uput{2mm}[-90](15,0){$15$}%
  \uput{2mm}[-90](16,0){$16$}%
  \uput{2mm}[-90](17,0){$17$}%
  \uput{2mm}[-90](18,0){$18$}%
  \uput{2mm}[-90](19,0){$19$}%
  \uput{2mm}[-90](20,0){$20$}%
  \uput{2mm}[-90](21,0){$21$}%
  \uput{2mm}[-90](22,0){$22$}%
  \uput{2mm}[-90](23,0){$23$}%
  \rput[tl](0,1.5){\quad$\fphi(t)$ and $\seqn{h_n}$ for $p=12$}
\end{pspicture}%
%%    & \psset{xunit=3mm,yunit=10mm}%============================================================================
% Daniel J. Greenhoe
% LaTeX file
% Daubechies-p12 wavelet function and coefficients
% nominal xunit =  3mm
% nominal yunit = 10mm
% nominal font size = \scriptsize
%============================================================================
\begin{pspicture}(-1.75,-1.5)(24,1.5)%
  %-------------------------------------
  % axes
  %-------------------------------------
  \psaxes[linecolor=axis,linewidth=0.75pt,yAxis=false,labelsep=2pt,labels=none]{->}(0,0)(0,-1.5)(24,1.5)%
  \psaxes[linecolor=axis,linewidth=0.75pt,xAxis=false,labelsep=2pt]{<->}(0,0)(0,-1.5)(24,1.5)%
  %-------------------------------------
  % plot
  %-------------------------------------
  \psline[linecolor=red]{-o}( 0, 0)( 0,-0.0000015291)%  
  \psline[linecolor=red]{-o}( 1, 0)( 1,-0.0000127770)%  
  \psline[linecolor=red]{-o}( 2, 0)( 2,-0.0000242415)%  
  \psline[linecolor=red]{-o}( 3, 0)( 3, 0.0000885041)%  
  \psline[linecolor=red]{-o}( 4, 0)( 4, 0.0003886531)%  
  \psline[linecolor=red]{-o}( 5, 0)( 5,-0.0000065451)%  
  \psline[linecolor=red]{-o}( 6, 0)( 6,-0.0021795036)%  
  \psline[linecolor=red]{-o}( 7, 0)( 7,-0.0022486072)%  
  \psline[linecolor=red]{-o}( 8, 0)( 8, 0.0067114990)%  
  \psline[linecolor=red]{-o}( 9, 0)( 9, 0.0128408252)%  
  \psline[linecolor=red]{-o}(10, 0)(10,-0.0122186491)%  
  \psline[linecolor=red]{-o}(11, 0)(11,-0.0415462775)%  
  \psline[linecolor=red]{-o}(12, 0)(12, 0.0108491303)%  
  \psline[linecolor=red]{-o}(13, 0)(13, 0.0964321201)%  
  \psline[linecolor=red]{-o}(14, 0)(14, 0.0053595697)%  
  \psline[linecolor=red]{-o}(15, 0)(15,-0.1824786059)%  
  \psline[linecolor=red]{-o}(16, 0)(16,-0.0237792573)%  
  \psline[linecolor=red]{-o}(17, 0)(17, 0.3161784537)%  
  \psline[linecolor=red]{-o}(18, 0)(18,-0.0447638856)%  
  \psline[linecolor=red]{-o}(19, 0)(19,-0.5158864784)%  
  \psline[linecolor=red]{-o}(20, 0)(20, 0.6571987226)%  
  \psline[linecolor=red]{-o}(21, 0)(21,-0.3773551352)%  
  \psline[linecolor=red]{-o}(22, 0)(22, 0.1095662728)%  
  \psline[linecolor=red]{-o}(23, 0)(23,-0.0131122580)%  
  \fileplot{../../common/wavelets/graphics/d12_psi.dat}%
  %-------------------------------------
  % labels
  %-------------------------------------
  \uput{2mm}[-90]( 1,0){ $1$}%
  \uput{2mm}[-90]( 2,0){ $2$}%
  \uput{2mm}[-90]( 3,0){ $3$}%
  \uput{2mm}[-90]( 4,0){ $4$}%
  \uput{2mm}[-90]( 5,0){ $5$}%
  \uput{2mm}[-90]( 6,0){ $6$}%
  \uput{2mm}[-90]( 7,0){ $7$}%
  \uput{2mm}[-90]( 8,0){ $8$}%
  \uput{2mm}[-90]( 9,0){ $9$}%
  \uput{2mm}[-90](10,0){$10$}%
  \uput{2mm}[-90](11,0){$11$}%
  \uput{2mm}[-90](12,0){$12$}%
  \uput{2mm}[-90](13,0){$13$}%
  \uput{2mm}[-90](14,0){$14$}%
  \uput{2mm}[-90](15,0){$15$}%
  \uput{2mm}[-90](16,0){$16$}%
  \uput{2mm}[-90](17,0){$17$}%
  \uput{2mm}[-90](18,0){$18$}%
  \uput{2mm}[-90](19,0){$19$}%
  \uput{2mm}[-90](20,0){$20$}%
  \uput{2mm}[-90](21,0){$21$}%
  \uput{2mm}[-90](22,0){$22$}%
  \uput{2mm}[-90](23,0){$23$}%
  \rput[tl](0,1.5){\quad$\fpsi(t)$ and $\seqn{g_n}$ for $p=12$}
\end{pspicture}%
%    \\\hline
%\end{longtable}

\begin{tabular}{*{2}{p{\tw/2-5mm}}}
   \includegraphics{graphics/d5_phi_h.pdf}&\includegraphics{graphics/d5_psi_g.pdf}%
\end{tabular}
\\
\begin{tabular}{*{2}{p{\tw/2-5mm}}}%
  \includegraphics{graphics/d6_phi_h.pdf}&\includegraphics{graphics/d6_psi_g.pdf}%
\end{tabular}%
\\
\begin{tabular}{*{2}{p{\tw/2-5mm}}}%
  \includegraphics{graphics/d7_phi_h.pdf}&\includegraphics{graphics/d7_psi_g.pdf}%
\end{tabular}%
\\
\begin{tabular}{*{2}{p{\tw/2-5mm}}}%
  \includegraphics{graphics/d8_phi_h.pdf}&\includegraphics{graphics/d8_psi_g.pdf}%
\end{tabular}%
\\
\begin{tabular}{*{2}{p{\tw/2-5mm}}}%
   \includegraphics{graphics/d9_phi_h.pdf}&\includegraphics{graphics/d9_psi_g.pdf}%
\end{tabular}%
\\
\begin{tabular}{*{2}{p{\tw/2-5mm}}}%
   \includegraphics{graphics/d10_phi_h.pdf}&\includegraphics{graphics/d10_psi_g.pdf}%
\end{tabular}%
\\
\begin{tabular}{*{2}{p{\tw/2-5mm}}}%
   \includegraphics{graphics/d11_phi_h.pdf}&\includegraphics{graphics/d11_psi_g.pdf}%
\end{tabular}%
\\
\begin{tabular}{*{2}{p{\tw/2-5mm}}}%
   \includegraphics{graphics/d12_phi_h.pdf}&\includegraphics{graphics/d12_psi_g.pdf}%
\end{tabular}%




\begin{figure}
  \centering%
  \begin{tabular}{cc}
    \includegraphics{graphics/D12_pz.pdf}&\includegraphics{graphics/D16_pz.pdf}%
  \end{tabular}
  \caption{
    \structe{Daubechies-$p$ wavelet system} pole zero plots
    \label{tab:Dp_zero}
    }
\end{figure}







