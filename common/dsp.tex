%============================================================================
% LaTeX File
% Daniel J. Greenhoe
%============================================================================
%======================================
\chapter{Operations on Sequences}
\label{app:dsp}
\label{app:ztrans}
%======================================
%A digital filter is an operator on a sequence
%For a wide class of digital filters, this operator is linear.
%This operation can often be more clearly
%understood by the use of a special transform called the {\em Z-transform} (\prefp{def:opZ}).
%The Z-transform represents linear filters by ratios of polynomials
%(a polynomial divided by a polynomial) in a free variable $z$.
%The roots of the numerator polynomial are called \structe{zeros}.
%The roots of the denominator polynomial are called \structe{poles}.
%The location in the $z$-plane of these poles and zeros
%determine the behavior of the filter operation.
%======================================
\section{Convolution operator}
%======================================
\ifdochasnot{seq}{
%--------------------------------------
\begin{definition}
\label{def:sequence}
\label{def:seq}
\footnote{
  \citerp{bromwich1908}{1},
  \citerpgc{thomson2008}{23}{143484367X}{Definition 2.1},
  \citerpg{joshi1997}{31}{8122408265}
  }
\label{def:tuple}
%--------------------------------------
Let $\clFyx$ be the set of all functions from a set $\setY$ to a set $\setX$.
Let $\Z$ be the set of integers.
\defbox{\begin{array}{l}
  \text{A function $\ff$ in $\clFyx$ is a \fnctd{sequence} over $\setX$ if $\setY=\Z$.}\\
  \text{A sequence may be denoted in the form $\ds\seqxZ{x_n}$ or simply as $\ds\seqn{x_n}$.}\\
  %\text{A function $\ff$ in $\clFyx$ is an \fnctd{n-tuple} over $\setX$ if $\setY=\setn{1,2,\ldots,\xN}$.}\\
  %\text{An n-tuple may be denoted in the form $\ds\tuplexn{x_n}$ or simply as $\ds\tuplen{x_n}$.}
\end{array}}
\end{definition}
  }

%--------------------------------------
\begin{definition}
\footnote{
  \citerpgc{kubrusly2011}{347}{0817649972}{Example 5.K}
  }
\label{def:spllR}
\label{def:spllF}
%--------------------------------------
Let $\fieldF$ be a \structe{field} \xref{def:field}.
\defboxt{
  The \structd{space of all absolutely square summable sequences} $\spllF$ over $\F$ is defined as
  \\\indentx$\ds\spllF\eqd\set{\seqxZ{x_n}}{\sum_{n\in\Z}\abs{x_n}^2 < \infty}$
  }
\end{definition}

The space $\spllR$ is an example of a \structe{separable Hilbert space}.
In fact, $\spllR$ is the \emph{only} separable Hilbert space in the sense that all separable Hilbert spaces
are isomorphically equivalent.
For example, $\spllR$ is isomorphic to $\spLLR$, the \structe{space of all absolutely square Lebesgue integrable functions}.
%That is, their topological structure is the same.
%Differences occur in the nature of operators on the spaces.

%--------------------------------------
\begin{definition}
\label{def:dsp_conv}
\label{def:convd}
\index{convolution}
%--------------------------------------
%Let $\seq{x_n}{n\in\Z}$ and $\seq{y_n}{n\in\Z}$ be sequences \xref{def:seq} in the space $\spllR$ \xref{def:spllR}.
\defbox{\begin{array}{M}
  The \opd{convolution} operation $\hxs{\convd}$ is defined as
  \\\indentx
  $\ds{\seqn{x_n}\convd\seqn{y_n}} \eqd \seq{\sum_{m\in\Z} x_{m} y_{n-m}}{n\in\Z}\qquad\scy\forall\seqxZ{x_n},\seqxZ{y_n}\in\spllR$
\end{array}}
\end{definition}

%--------------------------------------
\begin{proposition}
%--------------------------------------
Let $\convd$ be the \structe{convolution operator} \xref{def:dsp_conv}.
\propbox{
  \seqn{x_n}\convd\seqn{y_n} = \seqn{y_n}\convd\seqn{x_n}
  \qquad{\scy\forall\seqxZ{x_n},\seqxZ{y_n}\in\spllR}
  \qquad\text{($\convd$ is \prope{commutative})}
  }
\end{proposition}
\begin{proof}
\begin{align*}
  [x\convd y](n)
    &\eqd \sum_{m\in\Z} x_m y_{n-m}
    &&    \text{by \prefp{def:dsp_conv}}
  \\&=    \sum_{k\in\Z} x_{n-k} y(k)
    &&    \text{where $k\eqd n-m$ $\implies$ $m=n-k$}
  \\&=    \sum_{k\in\Z} x_{n-k} y(k)
    &&    \text{by \prope{commutativity} of addition}
  \\&=    \sum_{m\in\Z} x_{n-m} y_m
    &&    \text{by change of variables}
  \\&=    \sum_{m\in\Z} y_m x_{n-m}
    &&    \text{by \prop{commutative} property of the field over $\C$}
  \\&\eqd \brp{y\convd x}_n
    &&    \text{by \prefp{def:dsp_conv}}
\end{align*}
\end{proof}

%--------------------------------------
\begin{proposition}
\label{prop:conv_knk}
%--------------------------------------
Let $\convd$ be the \structe{convolution operator} \xref{def:dsp_conv}.
Let $\spllR$ be the set of \prope{absolutely summable} sequences \xref{def:spllR}.
\propbox{
  \brb{\begin{array}{F>{\ds}rc>{\ds}lD}
      (A).&\fx(n) &\in& \spllR & and
    \\(B).&\fy(n) &\in& \spllR
  \end{array}}
  \implies
  \brb{\begin{array}{>{\ds}rc>{\ds}l}
     \sum_{k\in\Z} \fx[k]\fy[n+k] &=& \fx[-n]\convd\fy(n)
  \end{array}}
  }
\end{proposition}
\begin{proof}
\begin{align*}
  \sum_{k\in\Z} \fx[k]\fy[n+k]
    &= \sum_{-p\in\Z} \fx[-p]\fy[n-p]
    && \text{where $p\eqd -k$}
    && \text{$\implies$ $k=-p$}
  \\&= \sum_{p\in\Z} \fx[-p]\fy[n-p]
    && \text{by \prope{absolutely summable} hypothesis}
    && \text{\xref{def:spllR}}
  \\&= \sum_{p\in\Z} \fx'[p]\fy[n-p]
    && \text{where $\fx'[n] \eqd\fx[-n]$}
    && \text{$\implies$ $\fx[-n]=\fx'[n]$}
  \\&\eqd \fx'[n]\convd\fy[n]
    && \text{by definition of \ope{convolution} $\convd$}
    && \text{\xref{def:convd}}
  \\&\eqd \fx[-n]\convd\fy[n]
    && \text{by definition of $\fx'[n]$}
\end{align*}
\end{proof}

%======================================
\section{Z-transform}
%======================================
%--------------------------------------
\begin{definition}
\footnote{
  \structe{Laurent series}: \citerpg{aa}{49}{0821821466}
  }
\label{def:opZ}
%--------------------------------------
%Let $\seq{x_n}{n\in\Z}$ be a sequence in the space $\spllR$. %over a ring $\ring$.
\defbox{\begin{array}{M}
  The \opd{z-transform} $\opZ$ of $\seq{x_n}{n\in\Z}$ is defined as
  \\\indentx
  $\ds\brs{\hxs{\opZ}\seqn{x_n}}(z) \eqd \mcom{\sum_{n\in\Z} x_n z^{-n}}{Laurent series}\qquad\scy\forall\seqn{x_n}\in\spllR$
\end{array}}
\end{definition}

%--------------------------------------
\begin{theorem}
\label{thm:opZ}
%--------------------------------------
Let $X(z)\eqd\opZ\fx[n]$ be the \ope{z-transform} of $\fx[n]$.
\thmbox{
  \brb{\begin{array}{rcl}
    \Zx(z) &\eqd& \opZ\seqn{\fx[n]}
  \end{array}}
  \quad\implies\quad
  \brb{\begin{array}{F>{\ds}rc>{\ds}lCD}
      (1).&\opZ\seqn{\alpha\fx[n]} &=& \alpha\Zx(z)                    & \forall\seqn{x_n}\in\spllR & and
    \\(2).&\opZ\seqn{\fx[n-k]}     &=& z^{-k}\Zx(z)                    & \forall\seqn{x_n}\in\spllR & and
    \\(3).&\opZ\seqn{\fx[-n]}      &=& \Zx\brp{\frac{1}{z}}            & \forall\seqn{x_n}\in\spllR & and
    \\(4).&\opZ\seqn{\fx^\ast[n]}  &=& \Zx^\ast\brp{z^\ast}            & \forall\seqn{x_n}\in\spllR & and
    \\(5).&\opZ\seqn{\fx^\ast[-n]} &=& \Zx^\ast\brp{\frac{1}{z^\ast}}  & \forall\seqn{x_n}\in\spllR &
  \end{array}}
  }
\end{theorem}
\begin{proof}
\begin{align*}
  \alpha\Z\Zx(z)
    &\eqd \alpha \opZ \seqn{\fx[n]}                && \text{by definition of $\Zx(z)$}
  \\&\eqd \alpha \sum_{n\in\Z} \fx[n] z^{-n}       && \text{by definition of $\opZ$ operator}
  \\&\eqd \sum_{n\in\Z} \brp{\alpha\fx[n]} z^{-n}  && \text{by \prope{distributive} property}
  \\&\eqd \opZ\seqn{\alpha\fx[n]}                  && \text{by definition of $\opZ$ operator}
  \\
  z^{-k}\Zx(z)
    &= z^{-k} \opZ\seqn{\fx[n]}
    && \text{by definition of $\Zx(z)$}
    && \text{(left hypothesis)}
  \\&\eqd z^{-k}\sum_{n=-\infty}^{n=+\infty} \fx[n] z^{-n}
    && \text{by definition of $\opZ$}
    && \text{\xref{def:opZ}}
  \\&=          \sum_{n=-\infty}^{n=+\infty} \fx[n] z^{-n-k}
  \\&=          \sum_{m-k=-\infty}^{m-k=+\infty} \fx[m-k] z^{-m}
    && \text{where $m\eqd n+k$}
    && \text{$\implies$ $n=m-k$}
  \\&=          \sum_{m=-\infty}^{m=+\infty} \fx[m-k] z^{-m}
  \\&=          \sum_{n=-\infty}^{n=+\infty} \fx[n-k] z^{-n}
    && \text{where $n\eqd m$}
  \\&\eqd \opZ\seqn{\fx[n-k]}
    && \text{by definition of $\opZ$}
    && \text{\xref{def:opZ}}
  \\
  \opZ\seqn{\fx^\ast[n]}
    &\eqd \sum_{n\in\Z}\fx^\ast[n] z^{-n}
    && \text{by definition of $\opZ$}
    && \text{\xref{def:opZ}}
  \\&\eqd \brp{\sum_{n\in\Z}\fx[n] (z^\ast)^{-n}}^\ast
    && \text{by definition of $\opZ$}
    && \text{\xref{def:opZ}}
  \\&\eqd \Zx^\ast(z^\ast)
    && \text{by definition of $\opZ$}
    && \text{\xref{def:opZ}}
  \\
  \opZ\seqn{\fx[-n]}
    &\eqd \sum_{n\in\Z}\fx[-n] z^{-n}
    && \text{by definition of $\opZ$}
    && \text{\xref{def:opZ}}
  \\&= \sum_{-m\in\Z}\fx[m] z^{m}
    && \text{where $m\eqd -n$}
    && \text{$\implies$ $n=-m$}
  \\&= \sum_{m\in\Z}\fx[m] z^{m}
    && \text{by \prope{absolutely summable} property}
    && \text{\xref{def:spllR}}
  \\&= \sum_{m\in\Z}\fx[m] \brp{\frac{1}{z}}^{-m}
    && \text{by \prope{absolutely summable} property}
    && \text{\xref{def:spllR}}
  \\&\eqd \Zx\brp{\frac{1}{z}}
    && \text{by definition of $\opZ$}
    && \text{\xref{def:opZ}}
  \\
  \opZ\seqn{\fx^\ast[-n]}
    &\eqd \sum_{n\in\Z}\fx^\ast[-n] z^{-n}
    && \text{by definition of $\opZ$}
    && \text{\xref{def:opZ}}
  \\&= \sum_{-m\in\Z}\fx^\ast[m] z^{m}
    && \text{where $m\eqd -n$}
    && \text{$\implies$ $n=-m$}
  \\&= \sum_{m\in\Z}\fx^\ast[m] z^{m}
    && \text{by \prope{absolutely summable} property}
    && \text{\xref{def:spllR}}
  \\&= \sum_{m\in\Z}\fx^\ast[m] \brp{\frac{1}{z}}^{-m}
    && \text{by \prope{absolutely summable} property}
    && \text{\xref{def:spllR}}
  \\&= \brp{\sum_{m\in\Z}\fx[m] \brp{\frac{1}{z^\ast}}^{-m}}^\ast
    && \text{by \prope{absolutely summable} property}
    && \text{\xref{def:spllR}}
  \\&\eqd \Zx^\ast\brp{\frac{1}{z^\ast}}
    && \text{by definition of $\opZ$}
    && \text{\xref{def:opZ}}
\end{align*}
\end{proof}

%--------------------------------------
\begin{theorem}[\thmd{convolution theorem}]
\label{thm:conv}
%--------------------------------------
Let $\convd$ be the convolution operator \xref{def:dsp_conv}.
\thmbox{
  \opZ\mcom{\brp{\seqn{x_n}\convd\seqn{y_n}}}{sequence convolution} = \mcom{\brp{\opZ\seqn{x_n}}\;\brp{\opZ\seqn{y_n}}}{series multiplication}
  \qquad{\scy\forall\seqxZ{x_n},\seqxZ{y_n}\in\spllR}
  }
\end{theorem}
\begin{proof}
\begin{align*}
  \brs{\opZ(x\convd y)}(z)
    &\eqd \opZ {\left(\sum_{m\in\Z} x_m y_{n-m}\right)}    && \text{by definition of $\convd$} && \text{\xref{def:dsp_conv}}
  \\&\eqd \sum_{n\in\Z} \sum_{m\in\Z} x_m y_{n-m} z^{-n}   && \text{by definition of $\opZ$}   && \text{\xref{def:opZ}}
  \\&=    \sum_{n\in\Z} \sum_{m\in\Z} x_m y_{n-m} z^{-n}
   &&=    \sum_{m\in\Z} \sum_{n\in\Z} x_m y_{n-m} z^{-n}
  \\&=    \sum_{m\in\Z} \sum_{k\in\Z} x_m y_k z^{-(m+k)}   && \text{where $k\eqd n-m$}         && \text{$\iff$ $n=m+k$}
  \\&=    \brs{\sum_{m\in\Z} x_m z^{-m}}
          \brs{\sum_{k\in\Z} y_k z^{-k}}
  \\&\eqd \brs{\opZ\seqn{x_n}}\;\brs{\opZ\seqn{y_n}}       && \text{by definition of $\opZ$}   && \text{\xref{def:opZ}}
\end{align*}
\end{proof}

%---------------------------------------
\section{From z-domain back to time-domain}
%---------------------------------------
\begin{figure}
  \centering
  \includegraphics[width=\tw/2-2mm]{graphics/iir2n.pdf}
  \includegraphics[width=\tw/2-2mm]{graphics/dfI_order2_156.pdf}
  \caption{Direct form 1 order 2 IIR filters\label{fig:df1iir2}}
\end{figure}
\begin{align*}
 \Zy(z) &=  b_0X(z) + b_1z^{-1}X(z)  + b_2z^{-2}X(z) - a_1 z^{-1}Y(z) + a_2z^{-2}Y(z)
  \\\\
  \fy[n] &= b_0\fx[n] + b_1\fx[n-1] + b_2\fx[n-2] - a_1\fy[n-1] - a_2\fy[n-2]
\end{align*}

%---------------------------------------
\begin{example}
%---------------------------------------
See \prefpp{fig:df1iir2}

$\ds{\frac{3z^2 + 5z + 7}{2z^2 + 10z + 12}}$
=
$\ds{\frac{3z^2 + 5z + 7}{2\brp{z^2 + 5z + 6}}}$
=
$\ds{\frac{\brp{\sfrac{3}{2}z^2 + \sfrac{5}{2}z + \sfrac{7}{2}}}
               {z^2 + 5z + 6}}$
=
$\ds{\frac{\brp{\sfrac{3}{2} + \sfrac{5}{2}z^{-1} + \sfrac{7}{2}z^{-2}}}
               {1 + 5z^{-1} + 6z^{-2}}}$
\end{example}

%=======================================
\section{Zero locations}
%=======================================
The system property of \prope{minimum phase} is defined in \pref{def:minphase} (next) and
illustrated in \prefpp{fig:pz_minphase}.
\begin{figure}[h]
  \centering%
  \begin{tabular}{cc}
    \includegraphics{graphics/pz_realcoefs.pdf}%
    &\includegraphics{graphics/pz_minphase.pdf}%
    \\\emph{not} minimum phase & \prope{minimum phase}
  \end{tabular}
  \caption{Minimum Phase filter\label{fig:pz_minphase}}
\end{figure}
%--------------------------------------
\begin{definition}
\footnote{
  \citerpg{farina2000}{91}{0471384569},
  \citerpg{dumitrescu2007}{36}{1402051247}
  }
\label{def:minphase}
\index{minimum phase}
%--------------------------------------
Let $\Zx(z)\eqd\opZ\seqn{x_n}$ be the \fncte{z transform} \xref{def:opZ} of a sequence $\seqxZ{x_n}$ in $\spllR$.
Let $\seqxZ{z_n}$ be the \structe{zeros} of $\Zx(z)$.
\defbox{\begin{array}{M}
  The sequence $\seqn{x_n}$ is \propd{minimum phase} if
  \\\indentx$\ds\mcom{\abs{z_n}<1\qquad \forall n\in\Z}{$\Zx(z)$ has all its \structe{zeros} inside the unit circle}$
\end{array}}
\end{definition}

The impulse response of a minimum phase filter has most of its energy concentrated
near the beginning of its support, as demonstrated next.
%--------------------------------------
\begin{theorem}[\thmd{Robinson's Energy Delay Theorem}]
\footnote{
  \citerpg{dumitrescu2007}{36}{1402051247},
  \citor{robinson1962},  % referenced by claerbout1976
  \citorc{robinson1966}{???},  % referenced by online thesis
  \citerpp{claerbout1976}{52}{53}
  %\citerp{os}{291}\\
  %\citerp{mallat}{253}
  }
\label{thm:ztr_redp}
\label{thm:redt}
\index{minimum phase!energy}
\index{energy}
%--------------------------------------
Let $\fp(z)\eqd\sum_{n=0}^\xN a_n z^{-n}$
and $\fq(z)\eqd\sum_{n=0}^\xN b_n z^{-n}$
be polynomials.
\thmbox{
  \brb{\begin{array}{lMD}
    \fp & is \prope{minimum phase} & and\\
    \fq & is \emph{not} minimum phase &
  \end{array}}
  \implies
  \mcom{\sum_{n=0}^{m-1} \abs{a_n}^2}{\parbox{20mm}{``energy" of the first $m$ coefficients of $\fp(z)$}} \ge
  \mcom{\sum_{n=0}^{m-1} \abs{b_n}^2}{\parbox{20mm}{``energy" of the first $m$ coefficients of $\fq(z)$}}
  \qquad \forall 0\le m\le\xN
  }
\end{theorem}

But for more \prope{symmetry}, put some zeros inside and some outside the unit circle \xref{fig:pz_d4}.

\begin{figure}[h]
\begin{center}
\footnotesize
\setlength{\unitlength}{0.08mm}
\begin{tabular}{|c|c|}
  \hline
  Daubechies-4 & Symlets-4
  \\\hline
  \includegraphics{graphics/D4_pz.pdf}&\includegraphics{graphics/S4_pz.pdf} \\
  \includegraphics{graphics/d4_phi_h.pdf}&\includegraphics{graphics/s4_phi_h.pdf}\\
  \hline
  \prope{minimum phase} & \emph{not} minimum phase
  \\\hline
\end{tabular}
\caption{
   Daubechies-4 and Symlet-4 scaling functions pole-zero plots
   \label{fig:pz_d4}
   }
\end{center}
\end{figure}
%--------------------------------------
\begin{example}
%--------------------------------------
An example of a minimum phase polynomial is the Daubechies-4 scaling function.
%This function is generated by a minimum phase filter.
The minimum phase polynomial causes most of the energy to be concentrated near the origin, making it very \hie{asymmetric}.
In contrast, the Symlet-4 has a design very similar to that of Daubechies-4,
but the selected zeros are not all within the unit circle in the complex $z$ plane.
This results in a scaling function that is more symmetric and less contrated near the origin.
Both scaling functions are illustrated in \prefpp{fig:pz_d4}.
\end{example}

%=======================================
\section{Pole locations}
%=======================================
%---------------------------------------
%\section{Stability and pole location}
%---------------------------------------
%--------------------------------------
\begin{definition}
\index{causal}
\label{def:causal}
%--------------------------------------
\mbox{}\\\defboxt{
  A filter (or system or operator) $\opH$ is \propd{causal}
  \\if its current output does not depend on future inputs.
  }
\end{definition}

%--------------------------------------
\begin{definition}
\index{time-invariant}
%--------------------------------------
\mbox{}\\\defboxt{
  A filter (or system or operator) $\opH$ is \propd{time-invariant}
  \\if the mapping it performs does not change with time.
  }
\end{definition}

%--------------------------------------
\begin{definition}
\index{linear}
%--------------------------------------
\mbox{}\\\defboxt{
  An operation $\opH$ is {\bf linear} if any output $y_n$ can be described
  \\as a linear combination of inputs $x_n$ as in
  \\\indentx$\ds \fy_n = \sum\limits_{m\in\Z} \fh(m) \fx(n-m)$ .
  }
\end{definition}

For a filter to be \prope{stable}, place all the poles \emph{inside} the unit circle.
%--------------------------------------
\begin{theorem}
\index{stability}
%--------------------------------------
A causal LTI filter is \propb{stable} if all of its poles are \propb{inside} the unit circle.
\end{theorem}

\begin{figure}[ht]
  \centering%
  \begin{tabular}{|c|c|}
     \hline
     \includegraphics{graphics/pz_stable.pdf}
    &\includegraphics{graphics/pz_unstable.pdf}
    \\stable & unstable
    \\\hline
\end{tabular}
  \caption{
     Pole-zero plot stable/unstable causal LTI filters \xref{ex:pz_unstable}
     \label{fig:pz_unstable}
     }
\end{figure}
%--------------------------------------
\begin{example}
\label{ex:pz_unstable}
%--------------------------------------
Stable/unstable filters are illustrated in \prefpp{fig:pz_unstable}.
\end{example}

%---------------------------------------
%\section{The Reappearing Roots: Now you don't see them, now you do}
%---------------------------------------
\begin{figure}
  \centering
  \begin{tabular}{|c|c|}
      \hline
    \\\includegraphics{graphics/dfI_order2_fir.pdf}
     &\includegraphics{graphics/pz_pole00.pdf}
    \\\hline
  \end{tabular}
  \caption{FIR filters \label{fig:fir}}
\end{figure}
True or False? This filter has no poles:

  $\ds H(z)= b_0 + b_1 z^{-1} + b_2 z^{-2}$


\begin{align*}
  \fH(z)
    &= b_0 + b_1 z^{-1} + b_2 z^{-2}
     = \frac{z^2}{z^2} \times \frac{b_0 + b_1 z^{-1} + b_2 z^{-2}}{1}
     = \frac{b_0 z^2 + b_1 z^{1} + b_2 }{z^2}
\end{align*}


%---------------------------------------
\section{Mirroring for real coefficients}
%---------------------------------------
\begin{figure}
  \centering
  \includegraphics{graphics/pz_minphase.pdf}%
  \includegraphics{graphics/pz_realcoefs_11.pdf}
\caption{Conjugate pair structure yielding real coefficients\label{fig:realcoefs}}
\end{figure}

If you want real coefficients, choose poles and zeros in conjugate pairs (next).
%---------------------------------------
\begin{proposition}
%---------------------------------------
\propbox{
  \brb{\begin{array}{M}
    \structe{zeros} and \structe{poles}\\
    occur in \structe{conjugate pairs}
  \end{array}}
  \quad\implies\quad
  \brb{\begin{array}{M}
    \structe{coefficients} \\
    are \prope{real}.
  \end{array}}
  }
\end{proposition}
\begin{proof}
\begin{align*}
  \brp{z-p_1}\brp{z-p_1^*}
    &= \brs{z-\brp{a+ib}} \brs{z-\brp{a-ib}}
  \\&= z^2 +\brs{-a+ib-ib-a}z - \brs{ib}^2
  \\&= z^2 -2a z + b^2
\end{align*}
\end{proof}

%---------------------------------------
\begin{example}
%---------------------------------------
See \prefpp{fig:realcoefs}.
\begin{align*}
  H(z)   &= G\frac{\brs{z-z_1}\brs{z-z_2}}
                  {\brs{z-p_1}\brs{z-p_2}}
          = G\frac{\brs{z-\brp{1+i}}\brs{z-\brp{1-i}}}
                  {\brs{z-\brp{-\sfrac{2}{3}+i\sfrac{1}{2}}}\brs{z-\brp{-\sfrac{2}{3}-i\sfrac{1}{2}}}}
       \\&= G\frac{z^2 - z\brs{\brp{1-i}+\brp{1+i}} + \brp{1-i}\brp{1+i}}
                  {z^2 - z\brs{\brp{-\sfrac{2}{3}+i\sfrac{1}{2}}+\brp{-\sfrac{2}{3}+i\sfrac{1}{2}}} + \brp{-\sfrac{2}{3}+i\sfrac{1}{2}}\brp{-\sfrac{2}{3}+i\sfrac{1}{2}}}
       \\&= G\frac{z^2 - 2z + 2}
                  {z^2 - \sfrac{4}{3}z + \brp{\sfrac{4}{3}+\sfrac{1}{4}}}
          = G\frac{z^2 - 2z + 2}
                  {z^2 - \sfrac{4}{3}z + \sfrac{19}{12}}
\end{align*}
\end{example}

%=======================================
\section{Rational polynomial operators}
%=======================================
A digital filter is simply an operator on $\spllR$.
If the digital filter is a causal LTI system, then it can be expressed as
a rational polynomial in $z$ as shown next.

%=======================================
\begin{lemma}
%=======================================
A causal LTI operator $\opH$ can be expressed as a rational expression $\Zh(z)$.
\begin{align*}
 \Zh(z) &= \frac{b_0 + b_1z^{-1} + b_2z^{-2} + \cdots + b_Nz^{-N}}
                {1   + a_1z^{-1} + a_2z^{-2} + \cdots + a_Nz^{-N}}
   \\   &= \frac{\sum\limits_{n=0}^{N} b_n z^{-n}}
                {1   + \sum\limits_{n=1}^{N} a_n z^{-n}}
\end{align*}
\end{lemma}

A filter operation $\Zh(z)$ can be expressed as a product of its roots (poles and zeros).
\begin{align*}
 \Zh(z) &= \frac{b_0 + b_1z^{-1} + b_2z^{-2} + \cdots + b_\xN z^{-\xN}}
                {1   + a_1z^{-1} + a_2z^{-2} + \cdots + a_\xN z^{-\xN}}
   \\   &= \alpha\frac{(z-z_1)(z-z_2)\cdots(z-z_\xN)}
                {(z-p_1)(z-p_2)\cdots(z-p_\xN)}
\end{align*}
where $\alpha$ is a constant, $z_i$ are the zeros, and $p_i$ are the poles.
The poles and zeros of such a rational expression are often plotted in the z-plane with a unit circle
about the origin (representing $z=e^{i\omega}$).
Poles are marked with $\times$ and zeros with $\bigcirc$.
An example is shown in \prefp{fig:pz}.
Notice that in this figure the zeros and poles are either real or occur in
complex conjugate pairs.

\begin{figure}[ht]
  \centering
  \includegraphics{graphics/pz_realcoefs.pdf}
  \caption{
     Pole-zero plot for rational expression with real coefficients
     \label{fig:pz}
     }
\end{figure}

%======================================
\section{Filter Banks}
%======================================
\structe{Conjugate quadrature filters} (next definition) are used in \structe{filter banks}.
If $\Zx(z)$ is a \structe{low-pass filter}, then the conjugate quadrature filter of $\Zy(z)$ is a \structe{high-pass filter}.

%--------------------------------------
\begin{definition}
\footnote{
  \citerpg{strang1996}{109}{0961408871},
  \citerppgc{haddad1992}{256}{259}{0323138365}{section 4.5},
  \citerpgc{vaidyanathan1993}{342}{0136057187}{(7.2.7), (7.2.8)},
  \citor{smith1984},
  \citor{smith1986},
  \citor{mintzer1985}
  }
\label{def:cqf}
%--------------------------------------
Let $\seqxZ{x_n}$ and $\seqxZ{y_n}$ be \structe{sequences} \xref{def:seq} in $\spllR$ \xref{def:spllR}.
\defboxt{
  The sequence $\seqn{y_n}$ is a \fnctd{conjugate quadrature filter} with shift $\xN$ with respect to $\seqn{x_n}$ if
  \\\indentx$\ds y_n = \pm(-1)^n x^\ast_{\xN-n}$\\
  A \structe{conjugate quadrature filter} is also called a \structd{CQF} or a \structd{Smith-Barnwell filter}.
  \\
  Any triple $\otriple{\seqn{x_n}}{\seqn{y_n}}{\xN}$ in this form is said to satisfy the
  \\\indentx\propd{conjugate quadrature filter condition} or
  the \propd{CQF condition}.
  }
\end{definition}

%--------------------------------------
\begin{theorem}[\thmd{CQF theorem}]
%\begin{theorem} % [\thmd{conjugate quadrature filters}/\thmd{CQF}/\thmd{Smith-Barnwell filters}]
\label{thm:cqf}
\footnote{
  %\citerpgc{dau}{135}{0898712742}{(5.1.34)},
  %\citerpgc{vidakovic}{59}{0471293652}{(3.34)},
  \citerpg{strang1996}{109}{0961408871},
  \citerppgc{mallat}{236}{238}{012466606X}{(7.58),(7.73)},
  \citerppgc{haddad1992}{256}{259}{0323138365}{section 4.5},
  \citerpgc{vaidyanathan1993}{342}{0136057187}{(7.2.7), (7.2.8)}
  }
%--------------------------------------
Let $\Dy(\omega)$ and $\Dx(\omega)$ be the \ope{DTFT}s \xref{def:dtft} of the sequences $\seqxZ{y_n}$ and $\seqxZ{x_n}$, respectively, in $\spllR$ \xref{def:spllR}.
\thmboxt{
  $\ds\begin{array}{>{\ds}r c rc>{\ds}l DD}
    \mcom{y_{n} = \pm (-1)^n x^\ast_{\xN-n}}
         {(1) \structe{CQF} in ``time"}
      &\iff& \Zy(z)      &=& \pm (-1)^\xN z^{-\xN} \Zx^\ast\brp{\frac{-1}{z^\ast}}   & (2)& \structe{CQF} in ``z-domain"
    \\&\iff& \Dy(\omega) &=& \pm (-1)^\xN e^{-i\omega\xN} \Dx^\ast(\omega+\pi)       & (3)& \structe{CQF} in ``frequency"
    \\&\iff& x_{n}       &=& \pm (-1)^\xN (-1)^n y^\ast_{\xN-n}                      & (4)& ``reversed" \structe{CQF} in ``time"
    \\&\iff& \Zx(z)      &=& \pm z^{-\xN} \Zy^\ast\brp{\frac{-1}{z^\ast}}            & (5)& ``reversed" \structe{CQF} in ``z-domain"
    \\&\iff& \Dx(\omega) &=& \pm e^{-i\omega\xN} \Dy^\ast(\omega+\pi)                & (6)& ``reversed" \structe{CQF} in ``frequency"
  \end{array}$
  \\$\indentx\scy\forall\xN\in\Z$
  }
\end{theorem}
\begin{proof}
\begin{enumerate}
  \item Proof that $(1)\implies(2)$:
    \begin{align*}
      \Zy(z)
        &= \sum_{n\in\Z}  y_{n}  z^{-n}
        && \text{by definition of \fncte{z-transform}}
        && \text{\xref{def:opZ}}
      \\&= \sum_{n\in\Z} \mcom{(\pm) (-1)^n x^\ast_{\xN-n}}{\structe{CQF}} z^{-n}
        && \text{by (1)}
      \\&= \pm\sum_{m\in\Z} (-1)^{\xN-m} x_m^\ast z^{-(\xN-m)}
        && \text{where $m\eqd\xN-n \implies$}
        && \text{$n=\xN-m$}
      \\&= \pm (-1)^\xN z^{-\xN}
           \sum_{m\in\Z}(-1)^{-m} x_m^\ast \brp{z^{-1}}^{-m}
      \\&= \pm (-1)^\xN z^{-\xN}
           \sum_{m\in\Z} x^\ast_m \brp{-\frac{1}{z}}^{-m}
      \\&= \pm (-1)^\xN z^{-\xN}
           \brs{\sum_{m\in\Z} x_m \brp{-\frac{1}{z^\ast}}^{-m}}^\ast
      \\&= \pm (-1)^\xN z^{-\xN}\Zx^\ast\brp{\frac{-1}{z^\ast}}
        && \text{by definition of \fncte{z-transform}}
        && \text{\xref{def:opZ}}
    \end{align*}

  \item Proof that $(1)\impliedby(2)$:
    \begin{align*}
      \Zy(z)
        &= \pm (-1)^\xN z^{-\xN} \Zx^\ast\brp{\frac{-1}{z^\ast}}
        && \text{by (2)}
      \\&= \pm (-1)^\xN z^{-\xN} \brs{\sum_{m\in\Z} x_m \brp{\frac{-1}{z^\ast}}^{-m}}^\ast
        && \text{by definition of \fncte{z-transform}}
        && \text{\xref{def:opZ}}
      \\&= \pm (-1)^\xN z^{-\xN} \brs{\sum_{m\in\Z} x^\ast_m \brp{-z^{-1}}^{-m}}
        && \text{by definition of \fncte{z-transform}}
        && \text{\xref{def:opZ}}
      \\&= \sum_{m\in\Z}(\pm) (-1)^{\xN-m} x^\ast_m z^{-(\xN-m)}
      \\&= \sum_{m\in\Z}(\pm) (-1)^{n} x^\ast_{\xN-n} z^{-n}
        && \text{where $n=\xN-m$ $\implies$}
        && \text{$m\eqd\xN-n$}
      \\&\implies x_n = \pm (-1)^n x^\ast_{\xN-n}
    \end{align*}

  \item Proof that $(1)\implies(3)$:
    \begin{align*}
      \Dy(\omega)
        &\eqd \Zx(z)\Big|_{z=e^{i\omega}}
        &&    \text{by definition of \fncte{DTFT} \xref{def:dtft}}
      \\&=    \brs{\pm (-1)^\xN z^{-\xN}\Zx\brp{\frac{-1}{z^\ast}}}_{z=e^{i\omega}}
        &&    \text{by (2)}
      \\&=    \pm (-1)^\xN e^{-i\omega\xN}\Zx\brp{e^{i\pi}e^{\i\omega}}
      \\&=    \pm (-1)^\xN e^{-i\omega\xN}\Zx\brp{e^{i(\omega+\pi)}}
      \\&=    \pm (-1)^\xN e^{-i\omega\xN}\Dx\brp{\omega+\pi}
        &&    \text{by definition of \fncte{DTFT} \xref{def:dtft}}
    \end{align*}

  \item Proof that $(1)\implies(6)$:
    \begin{align*}
      \Dx(\omega)
        &= \sum_{n\in\Z}  y_{n}  e^{-i\omega n}
        && \text{by definition of \fncte{DTFT}}
        && \text{\xref{def:dtft}}
      \\&= \sum_{n\in\Z} \mcom{\pm (-1)^n x_{\xN-n}^\ast}{\structe{CQF}} e^{-i\omega n}
        && \text{by (1)}
      \\&= \sum_{m\in\Z}\pm (-1)^{\xN-m} x_m^\ast e^{-i\omega (\xN-m)}
        && \text{where $m\eqd\xN-n \implies$}
        && \text{$n=\xN-m$}
      \\&= \pm (-1)^\xN e^{-i\omega\xN}
           \sum_{m\in\Z}(-1)^m x_m^\ast e^{i\omega m}
      \\&= \pm (-1)^\xN e^{-i\omega\xN}
           \sum_{m\in\Z}e^{i\pi m} x_m^\ast e^{i\omega m}
      \\&= \pm (-1)^\xN e^{-i\omega\xN}
           \sum_{m\in\Z}x_m^\ast e^{i (\omega+\pi )m}
      \\&= \pm (-1)^\xN e^{-i\omega\xN}
           \left[ \sum_{m\in\Z}x_m e^{-i(\omega+\pi )m} \right]^\ast
      \\&= \pm (-1)^\xN e^{-i\omega\xN}\Dx^\ast\left(\omega+\pi \right)
        && \text{by definition of \fncte{DTFT}}
        && \text{\xref{def:dtft}}
    \end{align*}

  \item Proof that $(1)\impliedby(3)$:
    \begin{align*}
      y_n
        &= \frac{1}{2\pi}\int_{-\pi}^{+\pi} \Dy(\omega) e^{i\omega n} \dw
        && \text{by \thme{inverse DTFT}}
        && \text{\xref{thm:idtft}}
      \\&= \frac{1}{2\pi}\int_{-\pi}^{+\pi} \mcom{\pm (-1)^\xN e^{-i\xN\omega} \Dx^\ast(\omega+\pi)}{right hypothesis} e^{i\omega n} \dw
        && \text{by right hypothesis}
      \\&= \pm (-1)^\xN\frac{1}{2\pi}\int_{-\pi}^{+\pi}  \Dx^\ast(\omega+\pi) e^{i\omega(n-\xN)} \dw
        && \text{by right hypothesis}
      \\&= \pm (-1)^\xN\frac{1}{2\pi}\int_{0}^{2\pi}  \Dx^\ast(v) e^{i(v-\pi)(n-\xN)} \dv
        && \text{where $v\eqd\omega+\pi$ $\implies$}
        && \text{$\omega=v-\pi$}
      \\&= \pm (-1)^\xN e^{-i\pi(n-\xN)} \frac{1}{2\pi}\int_{0}^{2\pi}  \Dx^\ast(v) e^{iv(n-\xN)} \dv
      \\&= \pm (-1)^\xN \mcom{(-1)^\xN}{$e^{i\pi N}$} \mcom{(-1)^n}{$e^{-i\pi n}$}
           \brs{\frac{1}{2\pi}\int_{0}^{2\pi}  \Dx(v) e^{iv(\xN-n)} \dv}^\ast
      \\&= \pm (-1)^n x_{\xN-n}^\ast
        && \text{by \thme{inverse DTFT}}
        && \text{\xref{thm:idtft}}
    \end{align*}

  \item Proof that (1)$\iff$(4): % 2013jun14fri
    \begin{align*}
      y_{n} = \pm (-1)^n x^\ast_{\xN-n}
        &\iff (\pm)(-1)^n y_{n} = (\pm)(\pm) (-1)^n (-1)^n x^\ast_{\xN-n}
      \\&\iff \pm(-1)^n y_{n} = x^\ast_{\xN-n}
      \\&\iff \brp{\pm(-1)^n y_{n}}^\ast = \brp{x^\ast_{\xN-n}}^\ast
      \\&\iff \pm(-1)^n y^\ast_{n} = x_{\xN-n}
      \\&\iff x_{\xN-n} = \pm(-1)^n y^\ast_{n}
      \\&\iff x_m = \pm(-1)^{\xN-m} y^\ast_{\xN-m}
        && \text{where $m\eqd \xN-n\implies$}
        && \text{$n=\xN-m$}
      \\&\iff x_m = \pm(-1)^{\xN-m} y^\ast_{\xN-m}
      \\&\iff x_m = \pm(-1)^\xN (-1)^m y^\ast_{\xN-m}
      \\&\iff x_n = \pm(-1)^\xN (-1)^n y^\ast_{\xN-n}
        &&    \text{by change of free variables}
    \end{align*}

  \item Proofs for (5) and (6): not included. See proofs for (2) and (3).
\end{enumerate}
\end{proof}

%--------------------------------------
\begin{theorem}
\footnote{
  \citerpp{vidakovic}{82}{83},
  \citerpp{mallat}{241}{242}
  }
\label{thm:cqf_ddw}
%--------------------------------------
Let $\Dy(\omega)$ and $\Dx(\omega)$ be the \ope{DTFT}s \xref{def:dtft} of the sequences $\seqxZ{y_n}$ and $\seqxZ{x_n}$, respectively, in $\spllR$ \xref{def:spllR}.
\thmboxt{
  Let $y_n = \pm(-1)^n x^\ast_{N-n}$ (\prope{CQF condition}, {\scs\prefp{def:cqf}}). Then\\
  $\brb{\begin{array}{D@{\qquad}>{\ds}rc>{\ds}lcl@{\qquad}D}
       (A)&\left.\opddwn  \Dy(\omega)\right|_{\omega=0} = 0
          &\iff  & \left.\opddwn  \Dx(\omega)\right|_{\omega=\pi}   &=& 0 & (B)
       \\&&\iff  & \sum_{k\in\Z} (-1)^k k^n  x_k                    &=& 0 & (C)
       \\&&\iff  & \sum_{k\in\Z} k^n  y_k                           &=& 0 & (D)
  \end{array}}
  \quad\scy\forall n\in\Znn$
  }
\end{theorem}
\begin{proof}
\begin{enumerate}
  \item Proof that (A)$\implies$(B):
    \begin{align*}
      0
        &= \opddwn\Dy(\omega)\Big|_{\omega=0}
        && \text{by (A)}
      \\&= \opddwn(\pm)(-1)^\xN e^{-i\omega\xN}\Dx^\ast(\omega+\pi)\Big|_{\omega=0}
        && \text{by \thme{CQF theorem}}
        && \text{\xref{thm:cqf}}
      \\&= \brlr{(\pm)(-1)^\xN \sum_{\ell=0}^n \bcoef{n}{\ell} \brs{\opddw}^\ell\brs{e^{-i\omega\xN}}\cdot\brs{\opddw}^{n-\ell}\brs{\Dx^\ast(\omega+\pi)}}_{\omega=0}
        && \text{by \thme{Leibnitz GPR}}
        && \text{\xref{lem:LGPR}}
      \\&= \brlr{(\pm)(-1)^\xN
                 \sum_{\ell=0}^n \bcoef{n}{\ell} {-i\xN}^\ell e^{-i\omega\xN}\brs{\opddw}^{n-\ell}\brs{\Dx^\ast(\omega+\pi)}
                }_{\omega=0}
      \\&= \brlr{(\pm)(-1)^\xN  \cancelto{1}{e^{-i0\xN}} \sum_{\ell=0}^n \bcoef{n}{\ell} {-i\xN}^\ell \brs{\opddw}^{n-\ell}\brs{\Dx^\ast(\omega+\pi)}}_{\omega=0}
    \end{align*}
  \[\begin{array}{r rcl l}
    \implies & \Dx^{(0)}(\pi) &=& 0 \\
    \implies & \Dx^{(1)}(\pi) &=& 0 \\
    \implies & \Dx^{(2)}(\pi) &=& 0 \\
    \implies & \Dx^{(3)}(\pi) &=& 0 \\
    \implies & \Dx^{(4)}(\pi) &=& 0 \\
    \vdots   & \mc{1}{c}{\vdots}    \\
    \implies & \Dx^{(n)}(\pi) &=& 0 & \text{for $n=0,1,2,\ldots$}
  \end{array}\]

  \item Proof that (A)$\impliedby$(B):
    \begin{align*}
      0
        &= \opddwn\Dx(\omega)\Big|_{\omega=\pi}
        && \text{by (B)}
      \\&= \opddwn(\pm) e^{-i\omega\xN} \Dy^\ast(\omega+\pi)\Big|_{\omega=\pi}
        && \text{by \thme{CQF theorem}}
        && \text{\xref{thm:cqf}}
      \\&= \brlr{(\pm) \sum_{\ell=0}^n \bcoef{n}{\ell} \brs{\opddw}^\ell\brs{e^{-i\omega\xN}}\cdot\brs{\opddw}^{n-\ell}\brs{\Dy^\ast(\omega+\pi)}}_{\omega=\pi}
        && \text{by \thme{Leibnitz GPR}}
        && \text{\xref{lem:LGPR}}
      \\&= \brlr{(\pm) \sum_{\ell=0}^n \bcoef{n}{\ell} (-i\xN)^\ell e^{-i\omega\xN}\brs{\opddw}^{n-\ell}\brs{\Dy^\ast(\omega+\pi)}}_{\omega=\pi}
      \\&= \brlr{(\pm) \cancelto{1}{e^{-i\pi\xN}} \sum_{\ell=0}^n \bcoef{n}{\ell} {-i\xN}^\ell \brs{\opddw}^{n-\ell}\brs{\Dy^\ast(\omega+\pi)}}_{\omega=\pi}
      \\&= \brlr{(\pm) (-1)^\xN \sum_{\ell=0}^n \bcoef{n}{\ell} {-i\xN}^\ell \brs{\opddw}^{n-\ell}\brs{\Dy^\ast(\omega+\pi)}}_{\omega=\pi}
    \end{align*}
  \[\begin{array}{r rcl l}
    \implies & \Dy^{(0)}(0) &=& 0 \\
    \implies & \Dy^{(1)}(0) &=& 0 \\
    \implies & \Dy^{(2)}(0) &=& 0 \\
    \implies & \Dy^{(3)}(0) &=& 0 \\
    \implies & \Dy^{(4)}(0) &=& 0 \\
    \vdots   & \mc{1}{c}{\vdots}    \\
    \implies & \Dy^{(n)}(0) &=& 0 \\
    \implies & \Dy^{(n)}(0) &=& 0 & \text{for $n=0,1,2,\ldots$}
  \end{array}\]

  \item Proof that (B)$\iff$(C): by \prefp{thm:dtft_ddw}
  \item Proof that (A)$\iff$(D): by \prefp{thm:dtft_ddw}
  \item Proof that (CQF)$\notimpliedby$(A): Here is a counterexample: $\Dy(\omega)=0$.

\end{enumerate}
\end{proof}

%---------------------------------------
\section{Inverting non-minimum phase filters}
%---------------------------------------
\prope{Minimum phase} filters are easy to invert: each \structe{zero} becomes a \structe{pole}
and each \structe{pole} becomes a \structe{zero}.
\begin{figure}
\centering
$\begin{array}{ccccc}
     \tbox{\includegraphics{graphics/pz_minphase.pdf}}
    &\tbox{$\times$}
    &\tbox{\includegraphics{graphics/pz_minphase_inv.pdf}}
    &\tbox{$=$}
    &\tbox{$1$}
  \\
     \ds\frac{\brp{z-z_1}\brp{z-z_2}\brp{z-z_3}\brp{z-z_4}}
             {\brp{z-p_1}\brp{z-p_2}\brp{z-p_3}\brp{z-p_4}}
    &\times
    &\ds\frac{\brp{z-p_1}\brp{z-p_2}\brp{z-p_3}\brp{z-p_4}}
             {\brp{z-z_1}\brp{z-z_2}\brp{z-z_3}\brp{z-z_4}}
    &=
    &1
\end{array}$
\end{figure}

\begin{tabular}{ccccc}
     \tbox{\includegraphics{graphics/pz_unstable2.pdf}}
    &\tbox{$\times$}&
     \tbox{\includegraphics{graphics/pz_allpass.pdf}}
    &\tbox{$=$}&
     \tbox{\includegraphics{graphics/pz_unall.pdf}}
  \\unstable&&all-pass&&stable!
\end{tabular}

\begin{align*}
  \abs{A\brp{z}}_{z=e^{i\omega}}
    &= \frac{1}{r}\abs{\frac{z-r          e^{i\phi}}
                            {z-\frac{1}{r}e^{i\phi}}}_{z=e^{i\omega}}
   &&= \abs{\frac{ z- re^{i\phi}}
                 {rz-  e^{i\phi}}}_{z=e^{i\omega}}
  \\&= \abs{e^{i\phi}\brp{
            \frac{e^{-i\phi}z-r}
                 {rz- e^{i\phi}}}}_{z=e^{i\omega}}
   &&= \abs{z\brp{
            \frac{e^{-i\phi}-rz^{-1}}
                 {rz- e^{i\phi}}}}_{z=e^{i\omega}}
  \\&= \abs{-z\brp{
            \frac{rz^{-1}- e^{-i\phi}}
                 {rz     - e^{ i\phi}}}}_{z=e^{i\omega}}
   &&= \abs{\mcom{e^{i\pi}}{$-1$}e^{i\omega}\brp{
            \frac{re^{-i\omega} - e^{-i\phi}}
                 {re^{ i\omega} - e^{ i\phi}}}}
  \\&= \abs{\frac{1}{e^{-iv}}\brp{
            \frac{re^{-i\omega} - e^{-i\phi}}
                 {\brp{re^{ i\omega} - e^{ i\phi}}^\ast}}}
   &&= \abs{\frac{re^{-i\omega} - e^{-i\phi}}
                 {re^{-i\omega} - e^{-i\phi}}}
  \\&= 1
\end{align*}
