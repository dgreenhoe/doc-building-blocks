%============================================================================
% Daniel J. Greenhoe
% LaTeX File
%============================================================================

\documentclass{article}


\usepackage{amsmath, amsthm}
\usepackage{graphicx, amssymb, array, amscd, amsfonts, color}
\usepackage{Latexsym}
\usepackage{array}                    % new tabular and array support, \newcolumntype
\usepackage{longtable}
\usepackage{calc}                     % calculation
\usepackage{cancel}                  % allows diagonal cancelation
\renewcommand{\CancelColor}{\color{red}} % color of cancellation marks
\usepackage{slashed}                 % useful for overlapping math symbols: \declareslashed
\usepackage{amsthm}
\usepackage{mathtools}
\usepackage{xfrac} 

% Theorems
%-----------------------------------------------------------------
  %\newtheorem{lemma}[theorem]{Lemma} # do this if want Theorem 1.1, Lemma 1.2
  \newtheorem{theorem}{Theorem}[section]
  \newtheorem{corollary}  [theorem]{Corollary}
  \newtheorem{lemma}      [theorem]  {Lemma}
  \newtheorem{proposition}[theorem]{Proposition}
\theoremstyle{definition}
  \newtheorem{definition} [theorem]{Definition}
  \newtheorem{example}    [theorem]{Example}
\theoremstyle{remark}
  \newtheorem{remark}     [theorem]{Remark}

%\newtheorem{thm}{Theorem}[section]
%\newtheorem{cor}[thm]{Corollary}
%\newtheorem{lem}[thm]{Lemma}
%\newtheorem{prop}[thm]{Proposition}
%\theoremstyle{definition}
%\newtheorem{definition}[thm]{Definition}
%\theoremstyle{remark}
%\newtheorem{rem}[thm]{Remark}
%  \newtheorem{remark}     {Remark}         [section]
%%\newtheorem{lemma}{Lemma}[section]
%%\newtheorem{proposition}{Proposition}[section]
%  \newtheorem{theorem}    {Theorem}        [section]
%  \newtheorem{lemma}      {Lemma}          [section]
%  \newtheorem{proposition}{Proposition}    [section]
%  \newtheorem{corollary}  {Corollary}      [section]



% Shortcuts.
% One can define new commands to shorten frequently used
% constructions. As an example, this defines the R and Z used
% for the real and integer numbers.
%-----------------------------------------------------------------
%\def\RR{\mathbb{R}}
%\def\ZZ{\mathbb{Z}}
%\def\Z{\mathbb{Z}}

% Similarly, one can define commands that take arguments. In this
% example we define a command for the absolute value.
% -----------------------------------------------------------------
\newcommand{\tw}{\textwidth}
\newcommand{\indentx}{\ensuremath{\mbox{}\qquad}}%
\newcommand{\abs}[1]{\left\vert#1\right\vert}
\newcommand{\ope}[1]{\emph{#1}}
\newcommand{\opd}[1]{\emph{#1}}
\newcommand{\fncte}[1]{\emph{#1}}
\newcommand{\fnctd}[1]{\emph{#1}}
\newcommand{\Rxx}{R_{xx}}
\newcommand{\Ryy}{R_{yy}}
\newcommand{\Rxy}{R_{xy}}
\newcommand{\Ryx}{R_{yx}}
\newcommand{\Rxt}{R_{xt}}
\newcommand{\Rtx}{R_{tx}}
\newcommand{\Ryt}{R_{yt}}
\newcommand{\Rty}{R_{ty}}
\newcommand{\Swxx}{\tilde{S}_{xx}}
\newcommand{\Swyy}{\tilde{S}_{yy}}
\newcommand{\Swxy}{\tilde{S}_{xy}}
\newcommand{\Swyx}{\tilde{S}_{yx}}
\newcommand{\Szxx}{\check{S}_{xx}}
\newcommand{\Szyy}{\check{S}_{yy}}
\newcommand{\Szxy}{\check{S}_{xy}}
\newcommand{\Szyx}{\check{S}_{yx}}
\newcommand{\Szxt}{\check{S}_{xt}}
\newcommand{\Sztx}{\check{S}_{tx}}
\newcommand{\Szyt}{\check{S}_{yt}}
\newcommand{\Szty}{\check{S}_{ty}}
\newcommand{\citerpgc}[4]{\cite[page #2]{#1}}
\newcommand{\citerpg}[3]{\cite[page #2]{#1}}
\newcommand{\xref}[1]{\ref{#1}}
\newcommand{\eqd}{=}
\newcommand{\defbox}[1]{\mbox{}\\$\displaystyle#1$\\}
\newcommand{\lembox}[1]{\mbox{}\\$\displaystyle#1$\\}
\newcommand{\propbox}[1]{\mbox{}\\$\displaystyle#1$\\}
\newcommand{\thmbox}[1]{\mbox{}\\$\displaystyle#1$\\}
\newcommand{\corbox}[1]{\mbox{}\\$\displaystyle#1$\\}
\newcommand{\rembox}[1]{\mbox{}\\$\displaystyle#1$\\}
\newcommand{\mcom}[2]{{\displaystyle\underbrace{\displaystyle#1}_{\text{\scriptsize{#2}}}}}%
\newcommand{\ocom}[2]{{\displaystyle\overbrace{\displaystyle#1}^{\text{\scriptsize{#2}}}}}%
\newcommand{\mc}[3]{\multicolumn{#1}{#2}{#3}}
\newcommand{\conj}{\ast}
\newcommand{\mathfnct}[1]{#1}
\newcommand{\rvx}[0]{\mathfnct{x}}
\newcommand{\rvy}[0]{\mathfnct{y}}
\newcommand{\rvt}[0]{\mathfnct{t}}
\newcommand{\ds}{\displaystyle}
\newcommand{\brp}[1]{{\left(#1\right)}}       % bracket with parenthesis
\newcommand{\brs}[1]{{\left[#1\right]}}       % bracket with square brackets
\newcommand{\brb}[1]{{\left\{#1\right\}}}     % bracket with curly braces
\newcommand{\brbr}[1]{\left.#1\right\}}       % bracket with curly braces on right only
\newcommand{\brbl}[1]{\left\{#1\right.}       % bracket with curly braces on left only
\newcommand{\brl}[1]{{\left|#1\right|}}       % bracket with vertical line
\newcommand{\brlr}[1]{\left.#1\right|}        % bracket with vertical line on right only
\newcommand{\brll}[1]{\left|#1\right.}        % bracket with vertical line on left only
\newcommand{\citeP}    [1]{\cite{#1}}
\newcommand{\citePp}   [2]{\cite[page~#2]{#1}}
\newcommand{\citePpc}  [2]{\cite[page~#2]{#1}}
\newcommand{\pref}     [1]{\cite{#1}}
\newcommand{\prefp}    [1]{\cite{#1}}
\newcommand{\setn}[1]{{\left\{{#1}\right\}}}
\newcommand{\dd}  [1]   {{\;\mathfnct{d#1}}}
\newcommand{\dtau}[0]   {\dd{\tau}}
\newcommand{\du}[0]   {\dd{u}}
\newcommand{\dx}[0]   {\dd{x}}
\newcommand{\dy}[0]   {\dd{y}}
\newcommand{\R}{\mathbb{R}}
\newcommand{\C}{\mathbb{C}}
\newcommand{\Z}{\mathbb{Z}}
\newcommand{\ff}     [0] {{\mathfnct{f}}}
\newcommand{\fg}     [0] {{\mathfnct{g}}}
\newcommand{\fh}     [0] {{\mathfnct{h}}}
\newcommand{\inprod} [2]{\left\langle{#1}\,|\,{#2}\right\rangle}
\newcommand{\structe}[1]{\emph{#1}}
\newcommand{\prope}  [1]{\emph{#1}}
\newcommand{\thme}   [1]{\emph{#1}}
\newcommand{\propb}  [1]{\textbf{#1}}
\newcommand{\propd}  [1]{\textbf{#1}}
\newcommand{\spllR}  [0]{\splp{2}{\R}}         % l_2(R)
\newcommand{\spllC}  [0]{\splp{2}{\C}}         % l_2(R)
\newcommand{\splp}[2]{l_{#2}^{#1}}         %\newcommand{\splp}[2]{\mbfscrl_{#2}^{\mathsfup{#1}}}         %
\newcommand{\seqn}[1]{\left(#1\right)} %!!!\newcommand{\seqn}[1]{\left\lParen#1\right\rParen}
\newcommand{\seq} [2]{\seqn{#1}_{#2}}
\newcommand{\seqxZ}[1]{\seq{#1}{n\in\Z}}
\newcommand{\seqnY}[1]{\seq{#1}{n\in\setY}}
\newcommand{\seqnD}[1]{\seq{#1}{n\in\Dom}}
\newcommand{\seqnZ}[1]{\seq{#1}{n\in\Z}}
\newcommand{\seqmZ}[1]{\seq{#1}{m\in\Z}}
\newcommand{\seqkZ}[1]{\seq{#1}{k\in\Z}}
\newcommand{\seqjZ}[1]{\seq{#1}{j\in\Z}}
\newcommand{\seqxZp}[1]{\seq{#1}{n\in\Zp}}
\newcommand{\seqnZp}[1]{\seq{#1}{n\in\Zp}}
\newcommand{\seqjZp}[1]{\seq{#1}{j\in\Zp}}
\newcommand{\seqX}{\mathbb{x}}
\newcommand{\seqY}{\mathbb{y}}
\newcommand{\seqZ}{\mathbb{z}}
\newcommand{\imark}{$\bullet$}
\newcommand{\ZT}  [1]   {{\check{{\mathfnct{#1}}}}}
\newcommand{\ZH}    [0]{{\ZT{\fH}}}
\newcommand{\fH}{\mathfnct{H}}
\newcommand{\ft}  [1]   {{\tilde{{\mathfnct{#1}}}}}
\newcommand{\FH}    [0]{{\ft{H}}}
\newcommand{\tbox}[1]{\begin{tabular}{c}#1\end{tabular}}% useful for properly centered horizontal alignment
\newcommand{\tboxt}[1]{\begin{tabular}[t]{c}#1\end{tabular}}% useful for properly top-side horizontal alignment
\newcommand{\tboxc}[1]{\begin{tabular}{@{}c@{}}#1\end{tabular}}% useful for properly centered vertical alignment
\newcommand{\conv}{\star}  % convolution
\newcommand{\convd}{\conv}  % discrete convolution
\newcommand{\invo}{\ast}   % involution

% Operators
% New operators must defined as such to have them typeset
% correctly. As an example we define the Jacobian:
% -----------------------------------------------------------------
\DeclareMathOperator{\Jac}{Jac}
\DeclareMathOperator{\opZ}{Z}
\DeclareMathOperator{\opDTFT}{\breve{F}}
\DeclareMathOperator{\pE}{E}
\DeclareMathOperator{\opH}{H}
\DeclareMathOperator{\opE}{E}

%-----------------------------------------------------------------
\title{The Effects of the assorted cross-correlation definitions}
\author{Daniel J. Greenhoe\\ %Authors\\
  %(graduated ... not currently associated with a university)\small Department  and University\\
  \small Holland Charter Township, MI 49464 USA\\ %Postal code and City and Country names\\
  \small dgreenhoe@gmail.com %E-mail(s):
}

%=======================================
% tabular support
%=======================================
\newcolumntype{R}{>{\cellcolor{red}\upshape\bfseries\color{white}}c}
\newcolumntype{G}{>{\cellcolor{tblgreen}\upshape\bfseries\color{white}}c}
\newcolumntype{B}{>{\cellcolor{blue}\upshape\bfseries\color{white}}c}
\newcolumntype{P}{>{\cellcolor{purple}\upshape\bfseries\color{white}}c}
\newcolumntype{g}{>{\cellcolor{gray}\upshape\bfseries\color{black}}c}
\newcolumntype{X}{>{\cellcolor{red}$\upshape\bfseries\color{white}}c<{$}}
\newcolumntype{Y}{>{\cellcolor{green}$\upshape\bfseries\color{white}}c<{$}}
\newcolumntype{Z}{>{\cellcolor{blue}$\upshape\bfseries\color{white}}c<{$}}

\newcolumntype{C}{>{\scriptstyle}l}
\newcolumntype{D}{>{$\scriptsize}l<{$}}
\newcolumntype{H}{>{$\scriptsize}c<{$}}
\newcolumntype{F}{>{$\scriptsize}r<{$}}
\newcolumntype{E}{>{\scriptsize}l}
\newcolumntype{L}{>{\displaystyle}l}
\newcolumntype{M}{>{$}l<{$}}
\newcolumntype{N}{>{$}c<{$}}
\newcolumntype{O}{>{$}r<{$}}
\newcolumntype{z}{>{\zhtzye}l}

\begin{document}
\maketitle
%============================================================================
% Daniel J. Greenhoe
% XeLaTeX / LaTeX file
% abstract
%============================================================================
\begin{abstract}
In the literature, there are varying and in general incompatible definitions of
\fncte{cross-correlation} functions $\Rxy(n,m)$ and its \ope{wide-sense stationary} special case $\Rxy(m)$. % \xxref{def:Rxynm}{def:Rxym}.
The choice of definitions has consequences for results involving the \fncte{cross-spectral density} function 
$\Swxy(\omega)$, %\xref{def:Swxy}.
which is defined as the \ope{Discrete-time Fourier Transform} of $\Rxy(m)$.
\end{abstract}

%\begin{asciiabstract}

%\end{asciiabstract}


%\abstract{This is a simple template for papers at NTMSCI.}

  %============================================================================
% Daniel J. Greenhoe
% LaTeX File
%============================================================================


%======================================
\chapter*{Introduction}
%======================================


%======================================
\section{Description}
%======================================
This text describes two independent but related operations: \emph{Detection} of objects 
and \emph{Classification} of those objects.
The scope of these objects is that they are \emph{buried}. 
This scope qualitatively suggests that the objects are hidden from human vision 
and are in intimate contact with materials typically found in the materials constituting the earth such as soil, sand, clay,
rock, and/or moisture.

%======================================
\section{Method}
%======================================
This text follows the approach of \emph{Model-Based Signal Processing}\footnote{
  \citer{candy1985}, \citer{candy1992}, \citer{candy2005}, \citer{marston2006} }
which implies three components of the detection and classification operator pair:
\begin{enume}
  \item A \fncte{cost function}---an operation by which the effectiveness of the operations can be measured.
  \item A \structe{model}---a mathematical structure representing the entire system including the objects and their surrounding channel.
  \item An \fncte{algorithm}---an operator on the model which seeks to optimize (minimize) the cost function.
\end{enume}

%======================================
\section{System}
%======================================
The system consists of a channel and an excitation source.
The excitation source is a direct-coupled "shaker" or acoustically-coupled speaker
that excites a channel that may or may not contain an object.
The excitation signal propagates through the channel in the form of four "waves":
\begin{enume}
  \item P-waves
  \item S-waves
  \item Rayleigh waves
  \item Love waves
\end{enume}
The resulting response at any given location on the surface of the channel (e.g. surface of the ground)
can be measured using an accelerometer, an ultrasonic vibrometer, and/or a laser vibrometer.

%======================================
\section{Model}
%======================================
The interaction between earth materials and a \prope{compliant} foreign object shell 
has been demonstrated to be \prope{nonlinear}; this property in turn can be and has been exploited 
for the purposes of detection and classification.\footnote{
  \cite{donskoy1998}}




  %============================================================================
% LaTeX File
% Daniel J. Greenhoe
%============================================================================
%======================================
\section{Definitions}
%======================================
Here is a very limited overview of the definitions of $\Rxy(m)$:
%\\\defbox{\begin{array}{lM>{\ds}rc>{\ds}l}
\begin{longtable}{ll>{$\ds}r<{$}>{$\ds}c<{$}>{$\ds}l<{$}}
  \mc{5}{l}{References that put conjugate $\conj$ on $\rvy$:}
  \\&\citerpg{papoulis1984}{263}{0070484686} & R_{xy}(m) &\eqd& E\brb{\rvx(m)\rvy^\ast(0)}
  \\&\citerpg{cadzow}{341}{0023180102}       & r_{xy}(m) &\eqd& E\brs{\rvx(m)\rvy^\ast(0)}
  \\&\citeP{matlab_cpsd},\citeP{matlab_xcorr}& R_{xy}(m) &\eqd& E\setn{x_{n+m}y_n^\ast}
 %\\&                     & R_{xy}(m) &\eqd& E\setn{x_{n+m}y_n^\ast}
  \\
  \mc{5}{l}{References that put conjugate $\conj$ on $\rvx$:}
  \\&\citerpg{kay1988}{52}{8131733564}                    & r_{xy}[m] &\eqd& \mathcal{E}\brb{x^\ast[0]y[m]}
  \\&\citerpg{weisstein2002}{594}{1420035223}\footnotemark& f\star g  &\eqd& \int_{-\infty}^{\infty}\bar{f}(\tau)g(t+\tau)\dtau
  \\&\citePpc{leuridan1986}{2}{(7)}                       & GXY_1     &\eqd& \sum X_1^\ast Y
  \\
  \mc{5}{l}{References that use no conjugate $\conj$:}
  \\&\citerpg{bendat2010}{111}{1118210824}          & R_{xy}(m)            &\eqd& E\brs{\rvx(0)\rvy(m)}
  \\&\citerpg{helstrom1991}{369}{0023535717}        & \Rxy(t_1,t_2)        &\eqd& E[\rvx(t_1)\rvy(t_2)]
  \\&\citerpg{proakis1996}{A4}{0133737624}          & \gamma_{xy}(t_1,t_2) &\eqd& E(X_{t_1}Y_{t_2})
  \\&\citerpg{shin2008}{280}{0470725648}            & R_{xy}(\tau)         &\eqd& E[x(t)y(t+\tau)]
  \\&\citerpg{bracewell1978}{46}{007007013X}\footnotemark & g^\ast\star h  &\eqd& \int_{-\infty}^{\infty}g^\ast(u)h(u+x)\du   %{Pentagram notation for cross correlation}
\end{longtable}
%\end{array}}
\addtocounter{footnote}{-1}
\footnotetext{
  Bracewell and Weisstein here use the \ope{integral operator} $\int_{\R}\!\dx$ rather than the
  \ope{expectation operator} $\pE$.
  That is, they use a \ope{time average} rather than an \ope{ensemble average}.
  But in essence, the two types of operators are ``the same" because both types represent
  \ope{inner product}s.
  That is, $\int_{x\in\R}\ff(x)\fg^\ast(x)\dx\eqd\inprod{\ff(x)}{\fg(x)}_1$ and
  $\pE\brs{\rvx(t)\rvy^\ast(t)}\eqd\inprod{\rvx(t)}{\rvy(t)}_2$
  (both are inner products, but operate in perpendicular orientations across the ensemble plane).
  }
\stepcounter{footnote}
\footnotetext{
  Note that Bracewell's ``\ope{Pentagram notation for cross correlation}"
  $g^\ast\star h =\int_{-\infty}^{\infty}g^\ast(u)h(u+x)\du$
  implies
  $g\star h =\int_{-\infty}^{\infty}g(u)h(u+x)\du$
  (and hence in the ``References that use no conjugate" category).
  }
%\end{definition}

In terms of the expectation operator $\pE$,
there are a total of eight different ways of defining the cross-correlation $\Rxy(m)$ of
\prope{complex-valued} \prope{wide-sense stationary} sequences $\rvx(n)$ and $\rvy(n)$.
There are eight because each sequence may be defined with or without the conjugate operator $\ast$,
and $\rvx$ may lead or lag $\rvy$
($2\times2\times2=8$).
Here is a formalized list of the eight definitions.

%---------------------------------------
\begin{definition}
\label{def:Rxym}
%---------------------------------------
\mbox{}\\$\begin{array}{|*{2}{FM   >{\ds}r  c     l *{4}{@{\hspace{0pt}}l}  |}}
  \hline
    (1).&\fnctd{Papoulis}:        & \Rxy(m) &\eqd& \pE[\rvx     &(m)&\rvy^\ast&(0&)]  &(5).&\fnctd{Bendat-Piersol}:\footnotemark  & \Rxy(m) &\eqd& \pE[\rvx     &(0)&\rvy     &(m&)]
  \\(2).&\fnctd{Kay}:             & \Rxy(m) &\eqd& \pE[\rvx^\ast&(0)&\rvy     &(m&)]  &(6).&\fnctd{alt-BP}:          & \Rxy(m) &\eqd& \pE[\rvx     &(m)&\rvy     &(0&)]
  \\(3).&\fnctd{y-star-m}:        & \Rxy(m) &\eqd& \pE[\rvx     &(0)&\rvy^\ast&(m&)]  &(7).&\fnctd{BP-star}:         & \Rxy(m) &\eqd& \pE[\rvx^\ast&(0)&\rvy^\ast&(m&)]
  \\(4).&\fnctd{x-star-m}:        & \Rxy(m) &\eqd& \pE[\rvx^\ast&(m)&\rvy     &(0&)]  &(8).&\fnctd{alt-BP-star}:     & \Rxy(m) &\eqd& \pE[\rvx^\ast&(m)&\rvy^\ast&(0&)]
  \\\hline
\end{array}$
\end{definition}
\footnotetext{
  Note that Bendat and Piersol are well known and highly cited for their work related to 
  random vibration testing. 
  In this field, data samples are customarily collected using an analog-to-digital converter (ADC)
  and as such, for this application, are customarily \prope{real-valued}.
  Therefore, it is very understandable that these authors would define $\Rxy(m)$ 
  \emph{without} any conjugate operator.
  }

%=======================================
\section{Results}
%=======================================
%---------------------------------------
\begin{remark}
\label{rem:results}
%---------------------------------------
The 8 different definitions of $\Rxy(m)$ listed in \pref{def:Rxym} yield \ldots
%Depending on which one of the eight definitions of $\Rxy(m)$ is selected, we get \ldots
\\\indentx$\begin{array}{|FMlM|}
  \hline
    \imark & \mc{2}{M}{only 4 cases in which $\Swxx(\omega)$ is guarenteed to be \prope{real-valued}}  & \xref{prop:Swxx_real}
  \\\imark & 2 different relations on the pair   & \opair{\Ryx(m)}{\Rxy(m)}                            & \xref{prop:Rxy}
  \\\imark & 2 different relations on the pair   & \opair{\Rxx(-m)}{\Rxx(m)}                           & \xref{prop:Rxy}
  \\\imark & 4 different relations on the triple & \otriple{\Szxy(z)     }{\ZH(z)     }{\Szxx(z)     } & \xref{prop:RxySzxy}
  \\\imark & 4 different relations on the triple & \otriple{\Szxy(z)     }{\ZH(z)     }{\Szyy(z)     } & \xref{prop:RxySzxy}
  \\\imark & 4 different relations on the triple & \otriple{\Swxy(\omega)}{\FH(\omega)}{\Swxx(\omega)} & \xref{prop:RxySwxy}
  \\\imark & 3 different relations on the triple & \otriple{\Swxy(\omega)}{\FH(\omega)}{\Swyy(\omega)} & \xref{prop:RxySwxy}
  \\\imark & with $\rvx$ and $\rvy$ \prope{real-valued}, 2 relations on & \otriple{\Swxy(\omega)}{\FH(\omega)}{\Swxx(\omega)} & \xref{prop:RxySwxy_real}
  \\\imark & with $\rvx$ and $\rvy$ \prope{real-valued}, 2 relations on & \otriple{\Swxy(\omega)}{\FH(\omega)}{\Swyy(\omega)} & \xref{prop:RxySwxy_real}
  \\\hline
\end{array}$
\end{remark}

%---------------------------------------
\begin{proposition}
\label{prop:Rxy}
%---------------------------------------
%\propbox{
Let (1)--(8) below correspond to the eight definitions of $\Rxy(m)$ in \pref{def:Rxym}.
\\
$\begin{array}{|Mc        l       c l@{\hspace{0pt}}c D   l         c l@{\hspace{0pt}}c D|}
  \hline
    (1), (2), (3), or (4) & \implies& \Ryx(m) &=&\Rxy^\ast&(-m) &and&  \Rxx(-m) &=&\Rxx^\ast&(m) & (\prope{conjugate symmetric})
  \\(5), (6), (7), or (8) & \implies& \Ryx(m) &=&\Rxy     &(-m) &and&  \Rxx(-m) &=&\Rxx     &(m) & (\prope{symmetric})
  \\\hline
\end{array}$
%}
\end{proposition}
\begin{proof}
\begin{align*}
    (1).\quad\Ryx(m)
      &\eqd \pE\brs{\rvy(m)\rvx^\ast(0)}
      && \text{by Papoulis' definition of $\Rxy(m)$}                           && \text{\xref{def:Rxym}}
    \\&= \brp{\pE\brs{\rvx(0)\rvy^\ast(m)}}^\ast
      && \text{by \prope{antiautomorphic} property of *-algebras}              && \text{\xref{def:staralg}}
    \\&= \brp{\pE\brs{\rvx(0-m)\rvy^\ast(m-m)}}^\ast
      && \text{by \prope{wide sense stationary} property}
    \\&\eqd \Rxy^\ast(-m)
      && \text{by Papoulis' definition of $\Rxy(m)$}                           && \text{\xref{def:Rxym}}
    \\
    \Rxx(-m)
      &\eqd \brlr{\Rxy(-m)}_{\rvy=\rvx}
     &&= \brlr{\Ryx^\ast(m)}_{\rvy=\rvx}
      %&& \text{by previous result}
     &&= \Rxx^\ast(m)
\end{align*}

\begin{align*}
    (2).\quad\Ryx(m)
      &\eqd \pE\brs{\rvy^\ast(0)\rvx(m)}
     &&=\brp{\pE\brs{\rvx^\ast(m)\rvy(0)}}^\ast
     &&=\brp{\pE\brs{\rvx^\ast(m-m)\rvy(0-m)}}^\ast
     &&\eqd \Rxy^\ast(-m)
    \\
    (3).\quad\Ryx(m)
      &\eqd \pE\brs{\rvy(0)\rvx^\ast(m)}
     &&=\brp{\pE\brs{\rvx(m)\rvy^\ast(0)}}^\ast
     &&=\brp{\pE\brs{\rvx(m-m)\rvy^\ast(0-m)}}^\ast
     &&\eqd \Rxy^\ast(-m)
    \\
    (4).\quad\Ryx(m)
      &\eqd \pE\brs{\rvy^\ast(m)\rvx(0)}
     &&=\brp{\pE\brs{\rvx^\ast(0)\rvy(m)}}^\ast
     &&=\brp{\pE\brs{\rvx^\ast(-m)\rvy(0)}}^\ast
     &&\eqd \Rxy^\ast(-m)
    \\
    (5).\quad\Ryx(m)
      &\eqd  \pE\brs{\rvy(0)\rvx(m)}
     &&=     \pE\brs{\rvx(m)\rvy(0)}
     &&=     \pE\brs{\rvx(m-m)\rvy(0-m)}
     &&\eqd \Rxy(-m)
    \\
    (6).\quad\Ryx(m)
      &\eqd  \pE\brs{\rvy(m)\rvx(0)}
     &&=     \pE\brs{\rvx(0)\rvy(m)}
     &&=     \pE\brs{\rvx(0-m)\rvy(m-m)}
     &&\eqd \Rxy(-m)
    \\
    (7).\quad\Ryx(m)
      &\eqd  \pE\brs{\rvy^\ast(0)\rvx^\ast(m)}
     &&=     \pE\brs{\rvx^\ast(m)\rvy^\ast(0)}
     &&=     \pE\brs{\rvx^\ast(m-m)\rvy^\ast(0-m)}
     &&\eqd \Rxy(-m)
    \\
    (8).\quad\Ryx(m)
      &\eqd  \pE\brs{\rvy^\ast(m)\rvx^\ast(0)}
     &&=     \pE\brs{\rvx^\ast(0)\rvy^\ast(m)}
     &&=     \pE\brs{\rvx^\ast(0-m)\rvy^\ast(m-m)}
     &&\eqd \Rxy(-m)
\end{align*}

\end{proof}

%---------------------------------------
\begin{proposition}
\label{prop:Szxy}
%---------------------------------------
%\propbox{
Let (1)--(8) below correspond to the eight definitions of $\Rxy(m)$ in \pref{def:Rxym}.
\\
$\begin{array}{|M           c         >{\ds}l   c >{\ds}l@{\hspace{0pt}}c            D     l         c  l@{\hspace{0pt}}l|}
  \hline
    (1), (2), (3), or (4) & \implies& \Szyx(z) &=&\Szxy^\ast&\brp{\frac{1}{z^\ast}} &and&  \Szxx(z) &=& \Szxx^\ast&\brp{\frac{1}{z^\ast}} 
  \\(5), (6), (7), or (8) & \implies& \Szyx(z) &=&\Szxy     &\brp{\frac{1}{z}}      &and&  \Szxx(z) &=& \Szxx     &\brp{\frac{1}{z}}      
  \\\hline
\end{array}$
%}
\end{proposition}
\begin{proof}
\begin{align*}
  (1)-(4):\,\Szyx(z)
     &\eqd \opZ \Ryx(m)
    && \text{by definition of $\Szxy(z)$}                          && %\text{\xref{def:csd}}
  \\&\eqd \sum_{m\in\Z}\Ryx(m) z^{-m}
    && \text{by definition of $\opZ$}                              && \text{\xref{def:opZ}}
  \\&\eqd \sum_{m\in\Z} \Rxy^\ast(-m) z^{-m}
    && \text{by \prope{conjugate symmetry} property}               && \text{\xref{prop:Rxy}}
  \\&= \brs{\sum_{m\in\Z} \Rxy(-m) (z^\ast)^{-m}}^\ast
    && \text{by \prope{antiautomorphic} property of *-algebras}    && \text{\xref{def:staralg}}
  \\&= \brs{\sum_{-p\in\Z} \Rxy(p) (z^\ast)^{p}}^\ast
    && \text{where $p\eqd -m$}                                     && \text{$\implies$ $m=-p$}
  \\&= \brs{\sum_{p\in\Z} \Rxy(p) (z^\ast)^{p}}^\ast
    && \text{by \prope{absolutely summable} property}             %&& \text{\xref{def:spllR}}
  \\&= \brs{\sum_{p\in\Z} \Rxy(p) \brp{\frac{1}{z^\ast}}^{-p}}^\ast
  \\&\eqd \Szxy^\ast\brp{\frac{1}{z^\ast}}
    && \text{by definition of $\Szxy(z)$}                          && \text{\xref{def:Szxy}}
\end{align*}
\begin{align*}
  (1)-(4):\,\Szxx^\ast(z)
    &\eqd \brs{\Szyx(z)}^\ast_{\rvy=\rvx}
   &&= \brs{\Szxy^\ast\brp{\frac{1}{z^\ast}}}^\ast_{\rvy=\rvx}
   &&= \brs{\Szxx^\ast\brp{\frac{1}{z^\ast}}}^\ast
   &&= \Szxx\brp{\frac{1}{z^\ast}}
  \\
  (5)-(8):\,\Szyx(z)
    &= \sum_{m\in\Z} \Rxy(-m) z^{-m}
   &&= \sum_{-p\in\Z} \Rxy(p) z^{p}
   &&= \sum_{p\in\Z} \Rxy(p) \brp{\frac{1}{z}}^{-p}
   &&\eqd \Szxy\brp{\frac{1}{z}}
  \\
  (5)-(8):\,\Szxx(z)
    &= \brlr{\Szyx(z)}_{\rvy=\rvx}
   &&= \brlr{\Szxy\brp{\frac{1}{z}}}_{\rvy=\rvx}
   &&= \brlr{\Szxx\brp{\frac{1}{z}}}
   &&= \Szxx\brp{\frac{1}{z}}
\end{align*}
\end{proof}

%---------------------------------------
\begin{proposition}
\label{prop:Swxx_real}
%---------------------------------------
Let (1)--(8) correspond to the eight definitions of $\Rxy(m)$ in \pref{def:Rxym}.
\\
$\begin{array}{|M           c         l               Ml|}
  \hline
    \{(1), (2), (3), or (4)\} & \implies& \{\Swxx^\ast(\omega)=\Swxx(\omega) & ($\Swxx(\omega)$ is \prope{real-valued})&\}
  \\\hline
\end{array}$
\end{proposition}
\begin{proof}
\begin{align*}
  \Swxx^\ast(\omega)
    &= \brlr{\Szxx^\ast\brp{z}}_{z=e^{i\omega}}
   &&= \brlr{\Szxx^\ast\brp{\frac{1}{z^\ast}}}_{z=e^{i\omega}}
   && \mcom{=\brlr{\Szxx\brp{z}}_{z=e^{i\omega}}}{by \pref{prop:Szxy}}
   &&= \Swxx(\omega)
\end{align*}
\end{proof}

%---------------------------------------
\begin{proposition}
\label{prop:RxySzxy}
%---------------------------------------
%\propbox{
Let (1)--(8) below correspond to the eight definitions of $\Rxy(m)$ in \pref{def:Rxym}.
\\
$\begin{array}{|Fc        l         c l       @{\hspace{0pt}}l@{\hspace{0pt}}l  D    l         c  l@{\hspace{0pt}}c@{\hspace{0pt}}l     c  l@{\hspace{0pt}}l  @{\hspace{0pt}}l        @{\hspace{0pt}}c       @{\hspace{0pt}}l|}
  \hline
    (1) &      \implies& \Szxy(z) &=&\ZH^\ast&\brp{\frac{1}{z^\ast}} &\Szxx(z) &and& \Szyy(z) &=& \ZH     &(z)                    &\Szxy(z) &=& \ZH           &(z)               &\ZH^\ast&\brp{\frac{1}{z^\ast}} &\Szxx(z)
  \\(2) &      \implies& \Szxy(z) &=&\ZH     &(z)                    &\Szxx(z) &and& \Szyy(z) &=& \ZH^\ast&\brp{\frac{1}{z^\ast}} &\Szxy(z) &=& \ZH           &(z)               &\ZH^\ast&\brp{\frac{1}{z^\ast}} &\Szxx(z)
  \\(3) &      \implies& \Szxy(z) &=&\ZH^\ast&\brp{z^\ast}           &\Szxx(z) &and& \Szyy(z) &=& \ZH     &\brp{\frac{1}{z}}      &\Szxy(z) &=& \ZH^\ast      &\brp{z^\ast}      &\ZH     &\brp{\frac{1}{z}}      &\Szxx(z)
  \\(4) &      \implies& \Szxy(z) &=&\ZH     &\brp{\frac{1}{z}}      &\Szxx(z) &and& \Szyy(z) &=& \ZH^\ast&\brp{z^\ast}           &\Szxy(z) &=& \ZH^\ast      &\brp{z^\ast}      &\ZH     &\brp{\frac{1}{z}}      &\Szxx(z)
  \\(5) &      \implies& \Szxy(z) &=&\ZH     &(z)                    &\Szxx(z) &and& \Szyy(z) &=& \ZH     &\brp{\frac{1}{z}}      &\Szxy(z) &=& \ZH           &(z)               &\ZH     &\brp{\frac{1}{z}}      &\Szxx(z)
  \\(6) &      \implies& \Szxy(z) &=&\ZH     &\brp{\frac{1}{z}}      &\Szxx(z) &and& \Szyy(z) &=& \ZH     &(z)                    &\Szxy(z) &=& \ZH           &(z)               &\ZH     &\brp{\frac{1}{z}}      &\Szxx(z)
  \\(7) &      \implies& \Szxy(z) &=&\ZH^\ast&\brp{z^\ast}           &\Szxx(z) &and& \Szyy(z) &=& \ZH^\ast&\brp{\frac{1}{z^\ast}} &\Szxy(z) &=& \ZH^\ast      &\brp{z^\ast}      &\ZH^\ast&\brp{\frac{1}{z^\ast}} &\Szxx(z)
  \\(8) &      \implies& \Szxy(z) &=&\ZH     &\brp{\frac{1}{z}}      &\Szxx(z) &and& \Szyy(z) &=& \ZH     &(z)                    &\Szxy(z) &=& \ZH           &(z)               &\ZH     &\brp{\frac{1}{z}}      &\Szxx(z)
  \\\hline
\end{array}$
%}
\end{proposition}
\begin{proof}
\begin{align*}
%\intertext{(1). If we follow Papoulis $\brp{\Rxy(m)\eqd\pE\brs{\rvx(m)\rvy^\ast(0)}}$, then\ldots}
    (1).\quad\Szyt(z)
      &\eqd \opZ\Ryt(m)
      && \text{by definition of $\Szty(z)$}                                    && \text{\xref{def:Szxy}}
    \\&\eqd \opZ\pE\brs{\rvy(m)\rvt^\ast(0)}
      && \text{by Papoulis' definition of $\Rty(m)$}                           && \text{\xref{def:Rxym}}
    \\&=    \opZ\pE\brs{\brp{\sum_{k\in\Z} \fh(k)\rvx(m-k)}\rvt^\ast(0)}
      && \mathrlap{\text{by \prope{linear time-invariant} property of $\opH$}}
    \\&=    \opZ        \sum_{k\in\Z} \fh(k) \pE\brs{\rvx(m-k)\rvt^\ast(0)}
      && \text{by \prope{linearity} of $\opE$}                                %&& \text{\xref{thm:pE_linop}}
    \\&\eqd \opZ        \sum_{k\in\Z} \fh(k) \Rxt(m-k)
      && \text{by Papoulis' definition of $\Rty(m)$}                           && \text{\xref{def:Rxym}}
    \\&\eqd \opZ\brs{\fh(m) \convd \Rxt(m)}
      && \text{by definition of \ope{convolution}}                             && \text{\xref{def:convd}}
    \\&= \brs{\opZ\fh(m)} \brs{\opZ\Rxt(m)}
      && \text{by \thme{convolution theorem}}                                  && \text{\xref{thm:conv}}
    \\&\eqd \ZH(z) \Szxt(z)
      && \text{by definitions of $\ZH(z)$ and $\Szxt(z)$}                      && \text{\xref{def:Szxy}}
    \\
    \boxed{\Szxy(z)}
      &=\Szyx^\ast\brp{\frac{1}{z^\ast}}\quad\text{(by \pref{prop:Szxy})}
     &&\eqd \brlr{\Szyt^\ast\brp{\frac{1}{z^\ast}}}_{\rvt\eqd\rvx}
       =\ZH^\ast\brp{\frac{1}{z^\ast}} \Szxx^\ast\brp{\frac{1}{z^\ast}}
     &&=\boxed{\ZH^\ast\brp{\frac{1}{z^\ast}} \Szxx(z)}
     \\
    \boxed{\Szyy(z)}
      &\eqd \brlr{\Szyt(z)}_{\rvt\eqd\rvy}
     &&=    \brlr{ \ZH(z) \Szxt\brp{z}}_{\rvt\eqd\rvy}
     =    \boxed{\ZH(z) \Szxy\brp{z}}
     &&=    \boxed{\ZH(z) \ZH^\ast\brp{\frac{1}{z^\ast}} \Szxx(z)}
\\
    (2).\quad\Szty(z)
      &\eqd \opZ\Rty(m)
     &&\eqd \opZ\pE\brs{\rvt^\ast(0)\rvy(m)}
     &&=    \opZ\pE\brs{\rvt^\ast(0)\brp{\fh(m)\convd\rvx(m)}}
    \\&=    \opZ\pE\brs{\rvt^\ast(0)\brp{\sum_{k\in\Z} \fh(k)\rvx(m-k)}}
     &&=    \opZ\brp{\sum_{k\in\Z} \fh(k)\pE\brs{\rvt^\ast(0)\rvx(m-k)}}
     &&\eqd \opZ\brp{\sum_{k\in\Z} \fh(k)\Rtx(m-k)}
    \\&\eqd \opZ\brp{\fh(m)\convd\Rtx(m)}
     &&=    \brs{\opZ\fh(m)}\brs{\opZ\Rtx(m)}
     &&\eqd \ZH(z) \Sztx(z)
    \\
    \boxed{\Szxy(z)}
      &\eqd \brlr{\Szty}_{\rvt=\rvx}
     &&=    \brlr{\ZH(z) \Sztx(z)}_{\rvt=\rvx}
     &&=    \boxed{\ZH(z) \Szxx(z)}
     \\
    \boxed{\Szyy(z)}
      &\eqd \brlr{\Szty(z)}_{\rvt\eqd\rvy}
       =    \brlr{\Szyt^\ast\brp{\frac{1}{z^\ast}}}_{\rvt\eqd\rvy}
     &&=    \brlr{\ZH(z) \Sztx(z)}_{\rvt\eqd\rvy}
       =    \ZH(z) \Szyx(z)
     &&=    \boxed{\ZH(z) \Szxy^\ast\brp{\frac{1}{z^\ast}}}
    \\&=    \ZH(z) \ZH^\ast\brp{\frac{1}{z^\ast}} \Szxx^\ast\brp{\frac{1}{z^\ast}}
     &&=    \boxed{\ZH(z) \ZH^\ast\brp{\frac{1}{z^\ast}} \Szxx(z)}
     &&     \text{by \pref{prop:Szxy}}
\\
    (3).\quad\Szty(z)
      &\eqd \opZ\Rty(m)
       \eqd \opZ\pE\brs{\rvt(0)\rvy^\ast(m)}
     &&=    \opZ\pE\brp{\rvt(0)\brs{\sum_{k\in\Z} \fh(k)\rvx(m-k)}^\ast}
    \\&=    \opZ\pE\brs{\rvt(0) \sum_{k\in\Z} \fh^\ast(k)      \rvx^\ast(m-k)}
     &&=    \opZ        \sum_{k\in\Z} \fh^\ast(k) \pE\brs{\rvt(0)\rvx^\ast(m-k)}
     &&\eqd \opZ        \sum_{k\in\Z} \fh^\ast(k) \Rtx(m-k)
    \\&\eqd \opZ\brs{\fh^\ast(m) \convd \Rtx(m)}
     &&=\mathrlap{\brs{\opZ\fh^\ast(m)} \brs{\opZ\Rtx(m)}
       = \ZH^\ast\brp{z^\ast} \Sztx\brp{z} \quad\text{by \prefp{thm:opZ}}}
    \\
    \boxed{\Szxy(z)}
      &\eqd \brlr{\Szty(z)}_{\rvt=\rvx}
     &&= \brlr{\ZH^\ast\brp{z^\ast} \Sztx(z)}_{\rvt=\rvx}
     &&= \boxed{\ZH^\ast\brp{z^\ast} \Szxx(z)}
    \\
    \boxed{\Szyy(z)}
      &\eqd \brlr{\Szty(z)}_{\rvt=\rvy}
       = \brlr{\ZH^\ast\brp{z^\ast} \Sztx\brp{z}}_{\rvt=\rvy}
     &&= \ZH^\ast\brp{z^\ast} \Szyx\brp{z}
     &&= \boxed{\ZH^\ast\brp{z^\ast} \Szxy^\ast\brp{\frac{1}{z^\ast}}}
    \\&= \ZH^\ast\brp{z^\ast} \ZH\brp{\frac{1}{z}} \Szxx^\ast\brp{\frac{1}{z^\ast}}
     &&= \boxed{\ZH^\ast\brp{z^\ast} \ZH\brp{\frac{1}{z}} \Szxx(z)}
\\
    (4).\quad\Szyt(z)
      &\eqd \opZ\Ryt(m)
     &&\eqd\mathrlap{\opZ\pE\brs{\rvy^\ast(m)\rvt(0)}
       =    \opZ\pE\brs{\brs{\sum_{k\in\Z} \fh(k)\rvx(m-k)}^\ast \rvt(0)}}
   %\\&=    \opZ\pE\brs{\brs{\sum_{k\in\Z} \fh^\ast(k)\rvx^\ast(m-k) \rvt(0)}}
    \\&=    \opZ\sum_{k\in\Z} \fh^\ast(k)\pE\brs{\rvx^\ast(m-k) \rvt(0)}
     &&=    \opZ\sum_{k\in\Z} \fh^\ast(k)\Rxt(m-k)
     &&\eqd \opZ\brs{\fh^\ast(m)\convd\Rxt(m)}
    \\&=    \brs{\opZ\fh^\ast(m)} \brs{\opZ\Rxt(m)}
     &&=    \ZH^\ast\brp{z^\ast}\Szxt(z)
     && \text{by \prefp{thm:opZ}}
    \\
    \boxed{\Szxy(z)}
      &= \Szyx^\ast\brp{\frac{1}{z^\ast}}
     &&\eqd \brlr{\Szyt^\ast\brp{\frac{1}{z^\ast}}}_{\rvt=\rvx}
     &&=  \brlr{\ZH\brp{\frac{1}{z}} \Szxt^\ast\brp{\frac{1}{z^\ast}}}_{\rvt=\rvx}
   \\&\eqd \ZH\brp{\frac{1}{z}} \Szxx^\ast\brp{\frac{1}{z^\ast}}
     &&=  \boxed{\ZH\brp{\frac{1}{z}} \Szxx(z)}
     &&   \text{by \pref{prop:Szxy}}
    \\
    \boxed{\Szyy(z)}
      &\eqd \brlr{\Szyt(z)}_{\rvt=\rvy}
       = \brlr{\ZH^\ast\brp{z^\ast}\Szxt(z)}_{\rvt=\rvy}
     &&= \boxed{\ZH^\ast\brp{z^\ast}\Szxy(z)}
     &&= \boxed{\ZH^\ast\brp{z^\ast}\ZH\brp{\frac{1}{z}} \Szxx(z)}
    \\
    (5).\quad\Szty(z)
      &\eqd \opZ\Rty(m)
     &&\eqd \opZ\pE\brs{\rvt(0)\rvy(m)}
    \\&=    \opZ\pE\brs{\rvt(0)\brp{\sum_{k\in\Z} \fh(k) \rvx(m-k)}}
     &&=    \opZ                    \sum_{k\in\Z} \fh(k) \pE\brs{\rvt(0)\rvx(m-k)}
     &&\eqd \opZ                    \sum_{k\in\Z} \fh(k) \Rtx(m-k)
    \\&\eqd \opZ\brs{\fh(m) \convd \Rtx(m)}
      &&  \brs{\opZ\fh(m)} \brs{\opZ\Rtx(m)}
      &&= \ZH(z) \Sztx(z)
    \\
    \boxed{\Szxy(z)}
      &\eqd \brlr{\Szty(z)}_{\rvt=\rvx}
     &&= \brlr{\ZH(z) \Sztx(z)}_{\rvt=\rvx}
     &&= \boxed{\ZH(z) \Szxx(z)}
    \\
    \boxed{\Szyy(z)}
      &= \Szyy\brp{\frac{1}{z}}
       \eqd \brlr{\Szty\brp{\frac{1}{z}}}_{\rvt=\rvy}
     &&= \brlr{\ZH\brp{\frac{1}{z}} \Sztx\brp{\frac{1}{z}}}_{\rvt=\rvy}
     &&= \ZH\brp{\frac{1}{z}} \Szyx\brp{\frac{1}{z}}
    \\&= \boxed{\ZH\brp{\frac{1}{z}} \Szxy(z)}
     &&= \ZH\brp{\frac{1}{z}} \ZH(z) \Szxx(z)
     &&= \boxed{ \ZH(z) \ZH\brp{\frac{1}{z}} \Szxx(z)}
\\
   (6).\quad\Szyt(z)
      &\eqd \opZ\Ryt(m)
     &&\eqd \opZ\pE\brs{\rvy(m)\rvt(0)}
       \mathrlap{\qquad
       =    \opZ\pE\brs{\brp{\sum_{k\in\Z} \fh(k) \rvx(m-k)}\rvt(0)}}
    \\&=    \opZ\sum_{k\in\Z} \fh(k) \pE\brs{\rvx(m-k)\rvt(0)}
     &&\eqd \opZ\sum_{k\in\Z} \fh(k) \Rxt(m-k)
    \\&\eqd \opZ\brs{\fh(m) \convd \Rxt(m)}
     &&= \opZ\brs{\fh(m)} \brs{\opZ\Rxt(m)}
     &&= \ZH(z) \Szxt(z)
    \\
    \boxed{\Szxy(z)}
      &= \Szyx\brp{\frac{1}{z}}
       \eqd \brlr{\Szyt\brp{\frac{1}{z}}}_{\rvt=\rvx}
     &&=\mathrlap{ \brlr{\ZH\brp{\frac{1}{z}} \Szxt\brp{\frac{1}{z}}}_{\rvt=\rvx}
       = \ZH\brp{\frac{1}{z}} \Szxx\brp{\frac{1}{z}}
      = \boxed{\ZH\brp{\frac{1}{z}} \Szxx(z)}}
    \\
    \boxed{\Szyy(z)}
      &\eqd \brlr{\Szyt(z)}_{\rvt=\rvy}
       = \brlr{\ZH(z) \Szxt(z)}_{\rvt=\rvy}
     &&= \boxed{\ZH(z) \Szxy(z)}
     &&= \boxed{\ZH(z) \ZH\brp{\frac{1}{z}} \Szxx(z)}
\\
    (7).\quad\Szty(z)
      &\eqd \opZ\Rty(m)
       \eqd \opZ\pE\brs{\rvt^\ast(0)\rvy^\ast(m)}
     &&=    \opZ\pE\brs{\rvt^\ast(0)\brp{\sum_{k\in\Z} \fh(k) \rvx(m-k)}^\ast}
     &&=    \opZ                    \sum_{k\in\Z} \fh^\ast(k) \pE\brs{\rvt^\ast(0)\rvx^\ast(m-k)}
     &&\eqd \opZ                    \sum_{k\in\Z} \fh^\ast(k) \Rtx(m-k)
    \\&\eqd \opZ\brs{\fh^\ast(m) \convd \Rtx(m)}
     &&= \brs{\opZ\fh^\ast(m)} \brs{\opZ\Rtx(m)}
     &&= \ZH^\ast\brp{z^\ast} \Sztx(z)
    \\
    \boxed{\Szxy(z)}
      &\eqd \brlr{\Szty(z)}_{\rvt=\rvx}
     &&= \brlr{\ZH^\ast\brp{z^\ast} \Sztx(z)}_{\rvt=\rvx}
     &&= \boxed{\ZH^\ast\brp{z^\ast} \Szxx(z)}
    \\
    \boxed{\Szyy(z)}
      &= \Szyy\brp{\frac{1}{z}}
      \eqd \brlr{\Szty\brp{\frac{1}{z}}}_{\rvt=\rvy}
     &&= \brlr{\ZH^\ast\brp{\frac{1}{z^\ast}} \Sztx\brp{\frac{1}{z}}}_{\rvt=\rvy}
     &&= \ZH^\ast\brp{\frac{1}{z^\ast}} \Szyx\brp{\frac{1}{z}}
    \\&= \boxed{\ZH^\ast\brp{\frac{1}{z^\ast}} \Szxy(z)}
     &&= \ZH^\ast\brp{\frac{1}{z^\ast}} \ZH^\ast\brp{z^\ast} \Szxx(z)
     &&= \boxed{ \ZH^\ast\brp{z^\ast} \ZH^\ast\brp{\frac{1}{z^\ast}}\Szxx(z)}
\\
    (8).\quad\Szyt(z)
      &\eqd \opZ\Ryt(m)
     &&\eqd \opZ\pE\brs{\rvy^\ast(m)\rvt^\ast(0)}
       \mathrlap{\qquad=    \opZ\pE\brs{\brp{\sum_{k\in\Z} \fh^\ast(k) \rvx^\ast(m-k)}\rvt^\ast(0)}}
    \\&=    \opZ\sum_{k\in\Z} \fh^\ast(k) \pE\brs{\rvx^\ast(m-k)\rvt^\ast(0)}
     &&\eqd \opZ\sum_{k\in\Z} \fh^\ast(k) \Rxt(m-k)
    \\&\eqd \opZ\brs{\fh(m) \convd \Rxt(m)}
     &&= \opZ\brs{\fh(m) \convd \Rxt(m)}
     &&= \brs{\opZ\fh(m)} \brs{\opZ\Rxt(m)}
    \\&= \ZH(z) \Szxt(z)
    \\
    \boxed{\Szxy(z)}
      &= \Szyx\brp{\frac{1}{z}}
       \eqd \brlr{\Szyt\brp{\frac{1}{z}}}_{\rvt=\rvx}
     &&= \brlr{\ZH\brp{\frac{1}{z}} \Szxt\brp{\frac{1}{z}}}_{\rvt=\rvx}
       = \ZH\brp{\frac{1}{z}} \Szxx\brp{\frac{1}{z}}
     &&= \boxed{\ZH\brp{\frac{1}{z}} \Szxx(z)}
    \\
    \boxed{\Szyy(z)}
      &\eqd \brlr{\Szyt(z)}_{\rvt=\rvy}
     &&= \brlr{\ZH(z) \Szxt(z)}_{\rvt=\rvy}
       = \boxed{\ZH(z) \Szxy(z)}
     &&= \boxed{\ZH(z) \ZH\brp{\frac{1}{z}} \Szxx(z)}
  \end{align*}
\end{proof}

%---------------------------------------
\begin{remark}
%---------------------------------------
Note that in several cases, the results listed in \pref{prop:RxySzxy} can be ``simplified"
(as measured by the number of glyphs required to render it on a page)
by the use of \pref{prop:Szxy}. For example, (1) in \pref{prop:RxySzxy} can be simplified from 
\\\indentx$\Szxy(z)=\ZH^\ast\brp{\frac{1}{z^\ast}}\Szxx(z)$\qquad{to}\qquad$\Szyx(z) = \ZH(z)\Szxx(z)$.
\\However, such simplification obfuscates the relations comparisons listed in \pref{rem:results}.
\end{remark}

%---------------------------------------
\begin{proposition}
\label{prop:Rxym}
\label{prop:RxySwxy}
%---------------------------------------
Let (1)--(8) below correspond to the eight definitions of $\Rxy(m)$ in \pref{def:Rxym}.
\\
$\begin{array}{|Fc        l              c l       *{3}{@{\hspace{0pt}}r}       D    l              c  l*{3}{@{\hspace{0pt}}l}            c  l*{5}{@{\hspace{0pt}}l}|}
  \hline
    (1) &      \implies& \Swxy(\omega) &=&\FH^\ast&(& \omega) &\Swxx(\omega)   &and& \Swyy(\omega) &=& \FH     &( &\omega)&\Swxy(\omega) &=& |&\FH     &( &\omega)|^2     &          &\Swxx(\omega)
  \\(2) &      \implies& \Swxy(\omega) &=&\FH     &(& \omega) &\Swxx(\omega)   &and& \Swyy(\omega) &=& \FH^\ast&( &\omega)&\Swxy(\omega) &=& |&\FH     &( &\omega)|^2     &          &\Swxx(\omega)
  \\(3) &      \implies& \Swxy(\omega) &=&\FH^\ast&(&-\omega) &\Swxx(\omega)   &and& \Swyy(\omega) &=& \FH     &(-&\omega)&\Swxy(\omega) &=& |&\FH     &(-&\omega)|^2     &          &\Swxx(\omega)
  \\(4) &      \implies& \Swxy(\omega) &=&\FH     &(&-\omega) &\Swxx(\omega)   &and& \Swyy(\omega) &=& \FH^\ast&(-&\omega)&\Swxy(\omega) &=& |&\FH     &(-&\omega)|^2     &          &\Swxx(\omega)
  \\(5) &      \implies& \Swxy(\omega) &=&\FH     &(& \omega) &\Swxx(\omega)   &and& \Swyy(\omega) &=& \FH     &(-&\omega)&\Swxy(\omega) &=&  &\FH     &( &\omega)\FH     &(-\omega) &\Swxx(\omega)
  \\(6) &      \implies& \Swxy(\omega) &=&\FH     &(&-\omega) &\Swxx(\omega)   &and& \Swyy(\omega) &=& \FH     &( &\omega)&\Swxy(\omega) &=&  &\FH     &( &\omega)\FH     &(-\omega) &\Swxx(\omega)
  \\(7) &      \implies& \Swxy(\omega) &=&\FH^\ast&(&-\omega) &\Swxx(\omega)   &and& \Swyy(\omega) &=& \FH^\ast&( &\omega)&\Swxy(\omega) &=&  &\FH^\ast&( &\omega)\FH^\ast&(-\omega) &\Swxx(\omega)
  \\(8) &      \implies& \Swxy(\omega) &=&\FH     &(&-\omega) &\Swxx(\omega)   &and& \Swyy(\omega) &=& \FH     &( &\omega)&\Swxy(\omega) &=&  &\FH     &( &\omega)\FH     &(-\omega) &\Swxx(\omega)
  \\\hline
\end{array}$
\end{proposition}
\begin{proof}
\begin{align*}
  (1).\quad\boxed{\Swxy(\omega)}
      &= \brlr{\Szxy(z)}_{z=e^{i\omega}}
    \\&= \brlr{\ZH^\ast\brp{\frac{1}{z^\ast}} \Szxx(z)}_{z=e^{i\omega}}
      && \text{by $\Szxy(z)$ result}         &&    \text{\xref{prop:RxySzxy}}
    \\&= \ZH^\ast\brp{e^{i\omega}} \Szxx\brp{e^{i\omega}}
      && \mathrlap{\text{(evaluation around unit circle in $z$-plane)}}
    \\&= \boxed{\FH^\ast(\omega) \Swxx(\omega)}
      && \text{by definition of \ope{DTFT}}  &&    \text{\xref{def:dtft}}
    \\
    \boxed{\Swyy(\omega)}
      &= \brlr{\Szyy(z)}_{z=e^{i\omega}}
    \\&= \brlr{\ZH(z) \Szxy(z)}_{z=e^{i\omega}}
      && \text{by $\Szxy(z)$ result}         &&    \text{\xref{prop:RxySzxy}}
    \\&= \ZH\brp{e^{i\omega}} \Szxy\brp{e^{i\omega}}
      && \mathrlap{\text{(evaluation around unit circle in $z$-plane)}}
    \\&= \boxed{\FH(\omega) \Swxy(\omega)}
      && \text{by definition of \ope{DTFT}}  &&    \text{\xref{def:dtft}}
    \\
    \boxed{\Swyy(\omega)}
      &= \brlr{\Szyy(z)}_{z=e^{i\omega}}
    \\&= \brlr{\ZH(z) \ZH^\ast\brp{\frac{1}{z^\ast}} \Szxx^\ast\brp{\frac{1}{z^\ast}}}_{z=e^{i\omega}}
      && \text{by $\Szxy(z)$ result}         &&    \text{\xref{prop:RxySzxy}}
    \\&= \ZH\brp{e^{i\omega}} \ZH^\ast\brp{\frac{1}{e^{-i\omega}}} \Szxx^\ast\brp{\frac{1}{e^{-i\omega}}}
    \\&= \ZH\brp{e^{i\omega}} \ZH^\ast\brp{e^{i\omega}} \Szxx^\ast\brp{e^{i\omega}}
     &&= \FH\brp{\omega} \FH^\ast(\omega) \Swxx^\ast(\omega)
     &&= \abs{\FH\brp{\omega}}^2 \Swxx^\ast(\omega)
    \\&= \boxed{\abs{\FH\brp{\omega}}^2 \Swxx(\omega)}
      && \text{because $\Swxx(\omega)$ is \prope{real-valued}}  
      && \text{\xref{prop:Swxx_real}}
\end{align*}

The other seven sets of proofs follow in like manner.
\end{proof}

%---------------------------------------
\begin{proposition}
\label{prop:RxySwxy_real}
%---------------------------------------
Let (1)--(8) below correspond to the eight definitions of $\Rxy(m)$ in \pref{def:Rxym}.
\\If $\rvx(n)$ and $\rvy(n)$ are \propb{real-valued}, then the following hold (!!! may not be true !!!):
\\
$\begin{array}{|Fc        l              c l       *{3}{@{\hspace{0pt}}r}       D    l              c  l*{3}{@{\hspace{0pt}}l}            c  l*{5}{@{\hspace{0pt}}l}|}
  \hline
    (1) &      \implies& \Swxy(\omega) &=&\FH&(& \omega) &\Swxx(\omega)   &and& \Swyy(\omega) &=& \FH&( &\omega)&\Swxy(\omega) &=& |&\FH&( &\omega)|^2 &          &\Swxx(\omega)
  \\(2) &      \implies& \Swxy(\omega) &=&\FH&(& \omega) &\Swxx(\omega)   &and& \Swyy(\omega) &=& \FH&( &\omega)&\Swxy(\omega) &=& |&\FH&( &\omega)|^2 &          &\Swxx(\omega)
  \\(3) &      \implies& \Swxy(\omega) &=&\FH&(&-\omega) &\Swxx(\omega)   &and& \Swyy(\omega) &=& \FH&(-&\omega)&\Swxy(\omega) &=& |&\FH&(-&\omega)|^2 &          &\Swxx(\omega)
  \\(4) &      \implies& \Swxy(\omega) &=&\FH&(&-\omega) &\Swxx(\omega)   &and& \Swyy(\omega) &=& \FH&(-&\omega)&\Swxy(\omega) &=& |&\FH&(-&\omega)|^2 &          &\Swxx(\omega)
  \\(5) &      \implies& \Swxy(\omega) &=&\FH&(& \omega) &\Swxx(\omega)   &and& \Swyy(\omega) &=& \FH&(-&\omega)&\Swxy(\omega) &=&  &\FH&( &\omega)\FH &(-\omega) &\Swxx(\omega)
  \\(6) &      \implies& \Swxy(\omega) &=&\FH&(&-\omega) &\Swxx(\omega)   &and& \Swyy(\omega) &=& \FH&( &\omega)&\Swxy(\omega) &=&  &\FH&( &\omega)\FH &(-\omega) &\Swxx(\omega)
  \\(7) &      \implies& \Swxy(\omega) &=&\FH&(&-\omega) &\Swxx(\omega)   &and& \Swyy(\omega) &=& \FH&( &\omega)&\Swxy(\omega) &=&  &\FH&( &\omega)\FH &(-\omega) &\Swxx(\omega)
  \\(8) &      \implies& \Swxy(\omega) &=&\FH&(&-\omega) &\Swxx(\omega)   &and& \Swyy(\omega) &=& \FH&( &\omega)&\Swxy(\omega) &=&  &\FH&( &\omega)\FH &(-\omega) &\Swxx(\omega)
  \\\hline
\end{array}$
\end{proposition}
\begin{proof}
These results follow directly from \pref{prop:RxySwxy} and from the fact that 
for \prope{real-valued} $\rvx(n)$ and $\rvy(n)$,
$\Swxy^\ast(\omega)=\Swxy(\omega)$.
\end{proof}

%---------------------------------------
\begin{proposition}
\label{prop:RxySwxy_real}
%---------------------------------------
Let (1)--(8) below correspond to the eight definitions of $\Rxy(m)$ in \pref{def:Rxym}.
\\If $\rvx(n)$ and $\rvy(n)$ are \propb{real-valued} and $\fh(n)$ is \propb{real-valued}, then the following hold (!!! may not be true!!!):
\\
$\begin{array}{|Fc        l              c l           @{\hspace{0pt}}r D    l              c  l*{3}{@{\hspace{0pt}}l}            c  l*{3}{@{\hspace{0pt}}l}|}
  \hline
    (1) &      \implies& \Swxy(\omega) &=&\FH(\omega) &\Swxx(\omega)   &and& \Swyy(\omega) &=& \FH&( &\omega)&\Swxy(\omega) &=& |&\FH&(\omega)|^2 &\Swxx(\omega)
  \\(2) &      \implies& \Swxy(\omega) &=&\FH(\omega) &\Swxx(\omega)   &and& \Swyy(\omega) &=& \FH&( &\omega)&\Swxy(\omega) &=& |&\FH&(\omega)|^2 &\Swxx(\omega)
  \\(3) &      \implies& \Swxy(\omega) &=&\FH(\omega) &\Swxx(\omega)   &and& \Swyy(\omega) &=& \FH&(-&\omega)&\Swxy(\omega) &=& |&\FH&(\omega)|^2 &\Swxx(\omega)
  \\(4) &      \implies& \Swxy(\omega) &=&\FH(\omega) &\Swxx(\omega)   &and& \Swyy(\omega) &=& \FH&(-&\omega)&\Swxy(\omega) &=& |&\FH&(\omega)|^2 &\Swxx(\omega)
  \\(5) &      \implies& \Swxy(\omega) &=&\FH(\omega) &\Swxx(\omega)   &and& \Swyy(\omega) &=& \FH&(-&\omega)&\Swxy(\omega) &=& |&\FH&(\omega)|^2 &\Swxx(\omega)
  \\(6) &      \implies& \Swxy(\omega) &=&\FH(\omega) &\Swxx(\omega)   &and& \Swyy(\omega) &=& \FH&( &\omega)&\Swxy(\omega) &=& |&\FH&(\omega)|^2 &\Swxx(\omega)
  \\(7) &      \implies& \Swxy(\omega) &=&\FH(\omega) &\Swxx(\omega)   &and& \Swyy(\omega) &=& \FH&( &\omega)&\Swxy(\omega) &=& |&\FH&(\omega)|^2 &\Swxx(\omega)
  \\(8) &      \implies& \Swxy(\omega) &=&\FH(\omega) &\Swxx(\omega)   &and& \Swyy(\omega) &=& \FH&( &\omega)&\Swxy(\omega) &=& |&\FH&(\omega)|^2 &\Swxx(\omega)
  \\\hline
\end{array}$
\end{proposition}
\begin{proof}
These results follow directly from \pref{prop:RxySwxy}, \pref{prop:RxySwxy_real}, and from the fact that 
for \prope{real-valued} $\fh(n)$
\begin{align*}
  \FH^\ast(\omega)
    &\eqd \brp{\sum_{n\in\Z} \fh(n) e^{-i\omega n} }^\ast
   &&=    \sum_{n\in\Z} \fh^\ast(n) e^{i\omega n}
   &&=    \sum_{n\in\Z} \fh(n) e^{-i(-\omega) n}
   &&\eqd \FH(-\omega)
\end{align*}
\end{proof}

%=======================================
\section{Which one?}
%=======================================
Which definition of $\Rxy(m)$ should we use?
Any one of them is perfectly acceptable---as long as a clear definition is provided and that definition is used consistently.
That being said, note the following:

\begin{enumerate}
\item The \ope{expectation} operator $\pE\brp{\rvX\rvY^\ast}$ is an \fncte{inner product}.
As such, it would seem the most natural to follow the convention of other inner product definitions
and thus put the conjugate $\conj$ on $\rvy$ (i.e. follow Papoulis):
\\\indentx$\begin{array}{c>{\ds}rc>{\ds}l}
    \imark & \inprod{\fx(t)}{\fy(t)} &\eqd& \int_{t\in\R} \fx(t)\fy^\conj(t) \dt
  \\\imark & \inprod{\fx(n)}{\fy(n)} &\eqd& \sum_{n\in\Z} \fx(n)\fy^\conj(n)
  \\\imark & \inprod{\rvX}{\rvY}     &\eqd& \pE\brp{\rvX\rvY^\conj}
\end{array}$

\item If we view $\Rxy(m)$ as an \ope{analysis} of $\rvy$ in terms of $\rvx$
      (or as a \ope{projection} of $\rvy$ onto $\rvx$),
      then it would seem more natural to put the conjugate on $\rvx$ (i.e. follow Kay).
      This is what is done in Fourier analysis when projecting a function $\ff(t)$ onto the
      set of basis functions $\set{e^{i\omega n}}{\omega\in\R}$, as in
      \\\begin{align*}
        \opDTFT\brs{\rvy(n)}(\omega)
          &\eqd \inprod{\rvy(n)}{e^{i\omega n}}
          && \text{(\ope{project} $\rvy(n)$ onto $e^{i\omega n}$ for some $\omega\in\R$)}
        \\&\eqd \sum_{n\in\Z} \rvy(n) \brs{e^{+i\omega n}}^\ast
        \\&\eqd \sum_{n\in\Z} \rvy(n) e^{-i\omega n}
      \end{align*}
      But arguably, a ``projection of $\rvy$ onto $\rvx$" would better be served by the use of $\Ryx(m)$ rather than $\Rxy(m)$.

\item If we follow Kay, then there is the advantage that you also end up with the Bendat-Piersol result for $\Swxy(\omega)$.
\end{enumerate}



%  %============================================================================
% LaTeX File
% Daniel J. Greenhoe
%============================================================================
%======================================
\section{Real-valued x(n) and y(n)}
%======================================
%---------------------------------------
\begin{corollary}
\label{cor:Rxy_realxy}
%---------------------------------------
Let (1)--(8) below correspond to the eight definitions of $\Rxy(m)$ in \pref{def:Rxym}.
\corboxt{%
  If $\seqn{\rvx(n)}$ and $\seqn{\rvy(n)}$ are \propb{real-valued}, then
  \\\indentx
  $\brb{\begin{array}{M     c         l             c l@{\hspace{0pt}}r D    l              c l         DD}
      each of (1)--(8) &\implies& \Rxy^\ast(m) &=&\Rxy&( m)        &and& \Rxx^\ast(m)  &=&\Rxx(m) & (\prope{real-valued}) & and
    \\each of (1)--(8) &\implies& \Ryx(m)      &=&\Rxy&(-m)        &and& \Rxx(-m)      &=&\Rxx(m) & (\prope{symmetric})   & .
  \end{array}}$
  }
\end{corollary}
\begin{proof}
This follows directly from the \prope{real-valued} hypothesis, \pref{def:Rxym}, and \pref{lem:Rxy}
\end{proof}

%---------------------------------------
\begin{proposition}
\label{prop:Szxy_realxy}
%---------------------------------------
Let (1)--(8) below correspond to the eight definitions of $\Rxy(m)$ in \pref{def:Rxym}.
\propboxt{%
  If $\seqn{\rvx(n)}$ and $\seqn{\rvy(n)}$ are \propb{real-valued}, then
  \\\indentx
  $\begin{array}{M           c         >{\ds}l   c >{\ds}l@{\hspace{0pt}}c            D     l         c  l@{\hspace{0pt}}lD}
    each of (1)--(8) & \implies& \Szxy(z) &=&\Szxy^\ast&\brp{z^\ast} = \Szyx\brp{\frac{1}{z}} &and&  \Szxx(z) &=& \Szxx^\ast&\brp{z^\ast} = \Szxx\brp{\frac{1}{z}} & .
  \end{array}$
  }
\end{proposition}
\begin{proof}
\begin{align*}
  \boxed{\Szxy^\ast\brp{z^\ast}}
    &\eqd \brs{\sum_{m\in\Z}\Rxy(m) (z^\ast)^{-m}}^\ast
    && \text{by definition of $\Szxy$}                             && \text{\xref{def:Szxy}}
  \\&= \sum_{m\in\Z}\Rxy^\ast(m) z^{-m}
    && \text{by \prope{antiautomorphic} property of *-algebras}    && \text{\xref{def:staralg}}
  \\&= \sum_{m\in\Z}\Rxy(m) z^{-m}
    && \text{by \prope{real-valued} hypothesis}
  \\&\eqd \boxed{\Szxy(z)}
    && \text{by definition of $\Szxy(z)$}                          && \text{\xref{def:Szxy}}
  \\&\eqd \sum_{m\in\Z}\Rxy(m) z^{-m}
    && \text{by definition of $\Szxy(z)$}                          && \text{\xref{def:Szxy}}
  \\&= \sum_{m\in\Z} \Ryx(-m) z^{-m}
    && \mathrlap{\text{by \prope{real-valued} hypothesis and \prefp{lem:Rxy}}}
  \\&= \sum_{-p\in\Z} \Ryx(p) z^{p}
    && \text{where $p\eqd-m$}
  \\&= \sum_{p\in\Z} \Ryx(p) z^{p}
    && \text{because $\seqn{\rvx(n)},\seqn{\rvy(n)}\in\spllC$}     && \text{\xref{def:spllC}}
  \\&= \sum_{p\in\Z} \Ryx(p) \brp{\frac{1}{z}}^{-p}
  \\&\eqd \boxed{\Szyx\brp{\frac{1}{z}}}
    && \text{by definition of $\Szyx(z)$}                          && \text{\xref{def:Szxy}}
\end{align*}
\end{proof}

%---------------------------------------
\begin{proposition}
\label{prop:RxySzxy_real}
%---------------------------------------
Let (1)--(8) below correspond to the eight definitions of $\Rxy(m)$ in \pref{def:Rxym}.
\propbox{%
\begin{array}{cM        c         l         c l       @{\hspace{0pt}}l@{\hspace{0pt}}l   D    l         c  l@{\hspace{0pt}}c@{\hspace{0pt}}l D}
  \mc{13}{M}{If $\seqn{\rvx(n)}$ and $\seqn{\rvy(n)}$ are \propb{real-valued}, then}
  \\&(1),(4),(6), or (8) &\implies& \Szxy(z) &=&\ZH     &\brp{\frac{1}{z}}      &\Szxx(z) &and& \Szyy(z) &=& \ZH     &(z)                    &\Szxy(z) & and
  \\&(2),(3),(5), or (7) &\implies& \Szxy(z) &=&\ZH     &(z)                    &\Szxx(z) &and& \Szyy(z) &=& \ZH     &\brp{\frac{1}{z     }} &\Szxy(z) & and
  \\&each of (1)--(8)    &\implies& \Szyy(z) &=&\mc{9}{l}{\ZH(z)\ZH\brp{\frac{1}{z}}\Szxx(z) = \ZH(z)\ZH^\ast\brp{\frac{1}{z^\ast}}\Szxx(z)}          & .
\end{array}}
\end{proposition}
\begin{proof}
\begin{align*}
  (1).\quad\boxed{\Szxy(z)}
    &= \Szxy^\ast\brp{z^\ast}
    && \text{by \pref{prop:Szxy_realxy}}
  \\&= \ZH\brp{\frac{1}{z}} \Szxx^\ast\brp{z^\ast}
    && \text{by \prefp{prop:RxySzxy}}
  \\&= \boxed{\ZH\brp{\frac{1}{z}} \Szxx(z)}
    && \text{by \pref{prop:Szxy_realxy}}
  \\
  \boxed{\Szyy(z)}
    &= \boxed{\ZH(z) \Szxy(z)}
    && \text{by \prefp{prop:RxySzxy}}
  \\&= \boxed{\ZH(z)\ZH\brp{\frac{1}{z}} \Szxx(z)}
    && \text{by $\Szxy(z)$ result}
  \\&= \boxed{\ZH(z)\ZH^\ast\brp{\frac{1}{z^\ast}} \Szxx(z)}
    && \text{by \prefp{lem:real_xyh}}
  \\
  (2).\quad\boxed{\Szxy(z)}
    &= \boxed{\ZH(z) \Szxx(z)}
    && \text{by \prefp{prop:RxySzxy}}
  \\
  \boxed{\Szyy(z)}
    &= \Szyy^\ast\brp{z^\ast}
    && \text{by \pref{prop:Szxy_realxy}}
  \\&= \ZH\brp{\frac{1}{z}} \Szxy^\ast\brp{z^\ast}
    && \text{by \prefp{prop:RxySzxy}}
  \\&= \boxed{\ZH\brp{\frac{1}{z}} \Szxy\brp{z}}
    && \text{by \pref{prop:Szxy_realxy}}
  \\&= \boxed{\ZH(z) \ZH\brp{\frac{1}{z}} \Szxx(z)}
    && \text{by $\Szxy(z)$ result}
  \\
  \boxed{\Szyy(z)}
    &= \Szyy^\ast\brp{z^\ast}
    && \text{by \pref{prop:Szxy_realxy}}
  \\&= \ZH^\ast\brp{z^\ast} \ZH^\ast\brp{\frac{1}{z^\ast}} \Szxx^\ast\brp{z^\ast}
    && \text{by \pref{prop:Szxy_realxy}}
  \\&= \ZH^\ast\brp{z^\ast} \ZH^\ast\brp{\frac{1}{z^\ast}} \Szxx(z)
    && \text{by \pref{prop:Szxy_realxy}}
  \\&= \boxed{\ZH(z) \ZH^\ast\brp{\frac{1}{z^\ast}} \Szxx(z)}
    && \text{by \prefp{lem:real_xyh}}
\end{align*}
\end{proof}

%---------------------------------------
\begin{corollary}
\label{cor:RxySwxy_real}
%---------------------------------------
Let (1)--(8) below correspond to the eight definitions of $\Rxy(m)$ in \pref{def:Rxym}.
\propbox{%
\begin{array}{cM        c         l         c l       @{\hspace{0pt}}l@{\hspace{0pt}}l   D    l         c  l@{\hspace{0pt}}c@{\hspace{0pt}}l  D   }
  \mc{13}{M}{If $\seqn{\rvx(n)}$ and $\seqn{\rvy(n)}$ are \propb{real-valued}, then}
  \\&(1),(4),(6), or (8) &\implies& \Swxy(\omega) &=&\FH^\ast&(\omega)               &\Swxx(\omega) &and& \Swyy(\omega) &=& \FH     &(\omega) &\Swxy(\omega) & and
  \\&(2),(3),(5), or (7) &\implies& \Swxy(\omega) &=&\FH     &(\omega)               &\Swxx(\omega) &and& \Swyy(\omega) &=& \FH^\ast&(\omega) &\Swxy(\omega) & and
  \\&each of (1)--(8)    &\implies& \Swyy(\omega) &=&\mc{9}{l}{\abs{\FH(\omega)}^2\Swxx(\omega)}                                                        & .
\end{array}}
\end{corollary}
\begin{proof}
\begin{align*}
  (1).\quad\Swxy(\omega)
    &= \brlr{\Szxy(z)}_{z=e^{i\omega}}
    && \text{by definition of \ope{DTFT}} && \text{\xref{def:dtft}}
  \\&= \brlr{\ZH\brp{\frac{1}{z}}\Szxx(z)}_{z=e^{i\omega}}
    && \text{by \pref{prop:RxySzxy_real}}
  \\&= \ZH\brp{e^{-i\omega}}\Szxx\brp{e^{i\omega}}
  \\&= \FH(-\omega)\Swxx(\omega)
    && \text{by definition of \ope{DTFT}} && \text{\xref{def:dtft}}
  \\&= \FH^\ast(\omega)\Swxx(\omega)
    && \text{by \prefp{lem:real_FH}}
\end{align*}

The remainder of the proof for \pref{cor:RxySwxy_real} follows in similar fashion.
\end{proof}


%  %============================================================================
% Daniel J. Greenhoe
% LaTeX file
%============================================================================
%============================================================================
\section{Case studies}
%============================================================================
It has been suggested by the giants that the usefulness of a mathematical idea can be measured by 
\begin{liste}
  \item how useful it is in applications\footnote{
    ``I regard as quite useless the reading of large treatises of pure analysis:
    too large a number of methods pass at once before the eyes.
    It is in the works of applications that one must study them;
    one judges their ability there and one apprises the manner of making use of them."
    ---Joseph Louis Lagrange (1736--1813).
    \citerp{stopple2003}{xi}
    }
  and
  \item how well it connects and is connected to the larger web of mathematical ideas.\footnote{
    \index{quotes!Hardy, G.H.}
    ``The ``seriousness" of a mathematical theorem lies,
    not in its practical consequences,
    which are usually negligible,
    but in the {\em significance} of the mathematical ideas which it connects.
    We may say, roughly, that a mathematical idea is ``significant" if it can be
    connected, in a natural illuminating way,
    with a large complex of other mathematical ideas."
    ---G.H. Hardy (1877--1947).
    \citerc{hardy1940}{section 11}
    }
\end{liste}

As such, this section which presents applications, 
may prove useful in gauging the usefulness of the preceding sections.
%    %============================================================================
% Daniel J. Greenhoe
% LaTeX file
%============================================================================
%============================================================================
\subsection{Case study: Additive noise}
%============================================================================
%---------------------------------------
\begin{minipage}{\tw-55mm}
\begin{proposition}
\label{prop:xvy}
%---------------------------------------
Let $\opS$ be the \structe{system} illustrated to the right,
where $\opT$ is an operator that is \prope{not necessarily linear}.
\end{proposition}
\end{minipage}
\hfill\tbox{\includegraphics{graphics/sys_xvy.pdf}}
\propbox{
  \brb{\begin{array}{@{\hspace{2pt}}FMMD}
      (A).& $\rvx(n)$ is                & \prope{WSS}          & and 
    \\(B).& $\rvx(n)$ and $\rvv(n)$ are & \prope{uncorrelated} & and
    \\(C).& $\rvv(n)$ is                & \prope{zero-mean}
  \end{array}}
  \implies
  \brb{\begin{array}{rclD}
    %  \Ryv(m) &=&           \Rvv(m)           & and
       \Rxy(m)       &=& \Rxx(m)                       & and
     \\\Szxy(z)      &=& \Szxx(z)                      & and
     \\\Swxy(\omega) &=& \Swxx(\omega)                 & and
     \\\Ryy(m)       &=& \Rxx(m)       + \Szvv(z)      & and
     \\\Szyy(z)      &=& \Szxx(z)      + \Szvv(z)      & and
     \\\Swyy(\omega) &=& \Swxx(\omega) + \Swvv(\omega) &
    %\\\Rxx(m) &=& \mc{2}{l}{\Ryy(m) + \Rvv(m) - 2\Real\Ryv(m)}
  \end{array}}
  \quad
  \begin{array}{M}
    for all\\
    (1)--(8)
  \end{array}
  }
\\
\begin{proof}
\begin{align*}
  (1).\quad\Rxy(m)
    &\eqd \pE\brs{\rvx(m)\rvy^\ast(0)}
    && \text{by (A) and Papoulis' definition of $\Rxy$}
    && \text{\xref{def:Rxym}}
  \\&\eqd \pE\brp{\rvx(m)\brs{\rvx(0)+\rvv(0)}^\ast}
    && \text{by definition of $\rvy$}
  \\&= \pE\brs{\rvx(m)\rvx^\ast(0)} + \pE\brs{\rvx(m)\rvv^\ast(0)}
    && \text{by \prope{linearity} of $\pE$}
    && \text{\xref{prop:pE_linop}}
  \\&= \pE\brs{\rvx(m)\rvx^\ast(0)} + \pE\brs{\rvx(m)} \pE\brs{\rvv^\ast(0)}
    && \text{by \prope{uncorrelated} hypothesis}
    && \text{(B)}
  \\&= \pE\brs{\rvx(m)\rvx^\ast(0)} + \pE\brs{\rvx(m)} \cancelto{0}{\pE\brs{\rvv^\ast(0)}}
    && \text{by \prope{zero-mean} hypothesis}
    && \text{(C)}
  \\&= \Rxx(m)
    && \text{by definition of $\Rxx$}
    && \text{\xref{def:Rxxm}}
\end{align*}
%
\begin{align*}
  \Ryy(m)
    &\eqd \pE\brs{\rvy(m)\rvy^\ast(0)}
    && \text{by (A) and Papoulis' definition of $\Ryy$}
    %&& \text{\xref{def:Rxym}}
  \\&\eqd \pE\brs{(\rvx(m)+\rvv(m))(\rvx(0)+\rvv(0))^\ast}
    && \text{by definition of $\rvy$}
  \\&= \pE\brs{\rvx(m)\rvx^\ast(0)}
     + \pE\brs{\rvx(m)\rvv^\ast(0)}
     + \pE\brs{\rvv(m)\rvx^\ast(0)}
     + \pE\brs{\rvv(m)\rvv^\ast(0)}
  \\&= \pE\brs{\rvx(m)\rvx^\ast(0)}
     + \pE{\rvx(m)\pE\rvv^\ast(0)}
     + \pE{\rvv(m)\pE\rvx^\ast(0)}
     + \pE\brs{\rvv(m)\rvv^\ast(0)}
    &&\text{by \prope{uncorrelated} hypothesis (B)}
  \\&= \pE\brs{\rvx(m)\rvx^\ast(0)}
     + \pE\rvx(m)\cancelto{0}{\pE\rvv^\ast(0)}
     + \cancelto{0}{\pE\rvv(m)}\pE\rvx^\ast(0)
     + \pE\brs{\rvv(m)\rvv^\ast(0)}
    &&\text{by \prope{zero-mean} hypothesis (C)}
  \\&= \Rxx(m) + \Rvv(m)
    && \text{by Papoulis' definition of $\Rxx$}
    %&& \text{\xref{def:Rxxm}}
\end{align*}

\begin{align*}
    (2).\quad\Rxy(m)
       &\eqd \pE\brs{\rvx^\ast(0)\rvy(m)}
      &&\eqd \pE\brp{\rvx^\ast(0)\brs{\rvx(m)+\rvv(m)}}
      &&=    \pE\brs{\rvx^\ast(0)\rvx(m)} + \pE\brs{\rvx^\ast(0)                       \rvv(m)}
     %&&=    \pE\brs{\rvx^\ast(0)\rvx(m)} + \pE\brs{\rvx^\ast(0)} \pE\brs{             \rvv(m)}
     %&&=    \pE\brs{\rvx^\ast(0)\rvx(m)} + \pE\brs{\rvx^\ast(0)} \cancelto{0}{\pE\brs{\rvv(m)}}
      &&=    \Rxx(m)
     \\
    (3).\quad\Rxy(m)
       &\eqd \pE\brs{\rvx(0)\rvy^\ast(m)}
      &&\eqd \pE\brp{\rvx(0)\brs{\rvx(m)+\rvv(m)}^\ast}
      &&=    \pE\brs{\rvx(0)\rvx^\ast(m)} + \pE\brs{\rvx(0)                       \rvv^\ast(m)}
     %&&=    \pE\brs{\rvx(0)\rvx^\ast(m)} + \pE\brs{\rvx(0)} \pE\brs{             \rvv^\ast(m)}
     %&&=    \pE\brs{\rvx(0)\rvx^\ast(m)} + \pE\brs{\rvx(0)} \cancelto{0}{\pE\brs{\rvv^\ast(m)}}
      &&=    \Rxx(m)
     \\
    (4).\quad\Rxy(m)
      &\eqd \pE\brs{\rvx^\ast(m)\rvy(0)}
      &&\eqd \pE\brp{\rvx^\ast(m)\brs{\rvx(0)+\rvv(0)}}
      &&=    \pE\brs{\rvx^\ast(m)\rvx(0)} + \pE\brs{\rvx^\ast(m)                       \rvv(0)}
     %&&=    \pE\brs{\rvx^\ast(m)\rvx(0)} + \pE\brs{\rvx^\ast(m)} \pE\brs{             \rvv(0)}
     %&&=    \pE\brs{\rvx^\ast(m)\rvx(0)} + \pE\brs{\rvx^\ast(m)} \cancelto{0}{\pE\brs{\rvv(0)}}
      &&=    \Rxx(m)
     \\
    (5).\quad\Rxy(m)
      &\eqd \pE\brs{\rvx(0)\rvy(m)}
      &&\eqd \pE\brp{\rvx(0)\brs{\rvx(m)+\rvv(m)}}
      &&=    \pE\brs{\rvx(0)\rvx(m)} + \pE\brs{\rvx(0)                       \rvv(m)}
     %&&=    \pE\brs{\rvx(0)\rvx(m)} + \pE\brs{\rvx(0)} \pE\brs{             \rvv(m)}
     %&&=    \pE\brs{\rvx(0)\rvx(m)} + \pE\brs{\rvx(0)} \cancelto{0}{\pE\brs{\rvv(m)}}
      &&=    \Rxx(m)
     \\
    (6).\quad\Rxy(m)
      &\eqd \pE\brs{\rvx(m)\rvy(0)}
      &&\eqd \pE\brp{\rvx(m)\brs{\rvx(0)+\rvv(0)}}
      &&=    \pE\brs{\rvx(m)\rvx(0)} + \pE\brs{\rvx(m)                       \rvv(0)}
     %&&=    \pE\brs{\rvx(m)\rvx(0)} + \pE\brs{\rvx(m)} \pE\brs{             \rvv(0)}
     %&&=    \pE\brs{\rvx(m)\rvx(0)} + \pE\brs{\rvx(m)} \cancelto{m}{\pE\brs{\rvv(0)}}
      &&=    \Rxx(0)
    \\
    (7).\quad\Rxy(m)
      &\eqd \pE\brs{\rvx^\ast(0)\rvy^\ast(m)}
      &&\eqd \pE\brp{\rvx^\ast(0)\brs{\rvx^\ast(m)+\rvv^\ast(m)}}
      &&=    \pE\brs{\rvx^\ast(0)\rvx(m)} + \pE\brs{\rvx^\ast(0)                       \rvv^\ast(m)}
     %&&=    \pE\brs{\rvx^\ast(0)\rvx(m)} + \pE\brs{\rvx^\ast(0)} \pE\brs{             \rvv^\ast(m)}
     %&&=    \pE\brs{\rvx^\ast(0)\rvx(m)} + \pE\brs{\rvx^\ast(0)} \cancelto{0}{\pE\brs{\rvv^\ast(m)}}
      &&=    \Rxx(m)
    \\
    (8).\quad\Rxy(m)
      &\eqd \pE\brs{\rvx^\ast(m)\rvy^\ast(0)}
      &&\eqd \pE\brp{\rvx^\ast(m)\brs{\rvx^\ast(0)+\rvv^\ast(0)}}
      &&=    \pE\brs{\rvx^\ast(m)\rvx(0)} + \pE\brs{\rvx^\ast(m)                       \rvv^\ast(0)}
     %&&=    \pE\brs{\rvx^\ast(m)\rvx(0)} + \pE\brs{\rvx^\ast(m)} \pE\brs{             \rvv^\ast(0)}
     %&&=    \pE\brs{\rvx^\ast(m)\rvx(0)} + \pE\brs{\rvx^\ast(m)} \cancelto{m}{\pE\brs{\rvv^\ast(0)}}
      &&=    \Rxx(m)
  \end{align*}
  \begin{align*}
    (2).\quad\Ryy(m)
        &\eqd \pE\brs{\rvy^\ast(0)\rvy(m)}
       &&\eqd \pE\brp{\brs{\rvx(0)+\rvv(0)}^\ast\brs{\rvx(m)+\rvv(m)}}
       &&= \scy \Rxx(m) + \cancelto{0}{\Rxv(m)} + \cancelto{0}{\Rvx(m)} + \Rvv(m)
       &&=    \Rxx(m)                                                 + \Rvv(m)
    \\
    (3).\quad\Ryy(m)
        &\eqd \pE\brs{\rvy(0)\rvy^\ast(m)}
       &&\eqd \pE\brp{\brs{\rvx(0)+\rvv(0)}\brs{\rvx(m)+\rvv(m)}^\ast}
       &&= \scy \Rxx(m) + \cancelto{0}{\Rxv(m)} + \cancelto{0}{\Rvx(m)} + \Rvv(m)
       &&=    \Rxx(m)                                                 + \Rvv(m)
    \\
    (4).\quad\Ryy(m)
        &\eqd \pE\brs{\rvy^\ast(m)\rvy(0)}
       &&\eqd \pE\brp{\brs{\rvx(m)+\rvv(m)}^\ast\brs{\rvx(0)+\rvv(0)}}
       &&= \scy \Rxx(m) + \cancelto{0}{\Rxv(m)} + \cancelto{0}{\Rvx(m)} + \Rvv(m)
       &&=    \Rxx(m)                                                 + \Rvv(m)
    \\
    (5).\quad\Ryy(m)
        &\eqd \pE\brs{\rvy(0)\rvy(m)}
       &&\eqd \pE\brp{\brs{\rvx(0)+\rvv(0)}\brs{\rvx(m)+\rvv(m)}}
       &&= \scy \Rxx(m) + \cancelto{0}{\Rxv(m)} + \cancelto{0}{\Rvx(m)} + \Rvv(m)
       &&=    \Rxx(m)                                                 + \Rvv(m)
    \\
    (6).\quad\Ryy(m)
        &\eqd \pE\brs{\rvy(m)\rvy(0)}
       &&\eqd \pE\brp{\brs{\rvx(m)+\rvv(m)}\brs{\rvx(0)+\rvv(0)}}
       &&= \scy \Rxx(m) + \cancelto{0}{\Rxv(m)} + \cancelto{0}{\Rvx(m)} + \Rvv(m)
       &&=    \Rxx(m)                                                 + \Rvv(m)
    \\
    (7).\quad\Ryy(m)
       &\eqd \pE\brs{\rvy^\ast(0)\rvy^\ast(m)}
       &&\eqd \pE\brp{\brs{\rvx(0)+\rvv(0)}^\ast\brs{\rvx(m)+\rvv(m)}^\ast}
       &&= \scy \Rxx(m) + \cancelto{0}{\Rxv(m)} + \cancelto{0}{\Rvx(m)} + \Rvv(m)
       &&=    \Rxx(m)                                                 + \Rvv(m)
    \\
    (8).\quad\Ryy(m)
       &\eqd \pE\brs{\rvy^\ast(m)\rvy^\ast(0)}
       &&\eqd \pE\brp{\brs{\rvx(m)+\rvv(m)}^\ast\brs{\rvx(0)+\rvv(0)}^\ast}
       &&= \scy \Rxx(m) + \cancelto{0}{\Rxv(m)} + \cancelto{0}{\Rvx(m)} + \Rvv(m)
       &&=    \Rxx(m)                                                 + \Rvv(m)
\end{align*}
\end{proof}


%    %============================================================================
% Daniel J. Greenhoe
% LaTeX file
%============================================================================
%============================================================================
\subsection{Case study: Dual additive noise}
%============================================================================
%---------------------------------------
\begin{minipage}{\tw-55mm}
\begin{proposition}
\label{prop:dual_addnoise}
%---------------------------------------
Let $\opS$ be the \structe{system} illustrated to the right.
%where $\opT$ is an operator that is \prope{not necessarily linear}.
\end{proposition}
\end{minipage}
\qquad\tbox{\includegraphics{graphics/dual_addnoise.pdf}}
\propbox{
  \brb{\begin{array}{FMD}
      (A).& $\rvx(n)$ and $\rvr(n)$ are \prope{wide sense stationary}      & and
    \\(B).& $\rvx(n)$ and $\rvw(n)$ are \prope{uncorrelated}               & and
    \\(C).& $\rvr(n)$ and $\rvv(n)$ are \prope{uncorrelated}               & and
    \\(D).& $\rvw(n)$ and $\rvv(n)$ are \prope{uncorrelated}               & and
    \\(E).& $\rvv(n)$ and $\rvw(n)$ are \prope{zero-mean}                  &
  \end{array}}
  \implies
  \brb{\begin{array}{rclD}
      \Rsy(m)       &=& \Rrx(m)       & and
    \\\Szsy(z)      &=& \Szrx(z)      & and
    \\\Swsy(\omega) &=& \Swrx(\omega) &
  \end{array}}
  \qquad
  \begin{array}{M}
    for all\\ 
    (1)--(8)
  \end{array}
  }
\\
\begin{proof}
\begin{align*}
  (1).\quad\boxed{\Rsy(m)}
    &\eqd \pE\brs{\rvs(m)\rvy^\ast(0)}
    && \text{by Papoulis' definition of $\Rxy$}
    && \text{\xref{def:Rxym}}
  \\&\eqd \pE\brp{\brs{\rvr(m)+\rvw(m)}\brs{\rvx(0)+\rvv(0)}^\ast}
    && \text{by definition of $\opS$}
  \\&= \mathrlap{
       \pE\brs{\rvr(m)\rvx^\ast(0)}
      +\pE\brs{\rvr(m)\rvv^\ast(0)}
      +\pE\brs{\rvw(m)\rvx^\ast(0)}
      +\pE\brs{\rvw(m)\rvv^\ast(0)}
      }
  \\&= \pE\brs{\rvr(m)\rvx^\ast(0)}
      +\pE\rvr(m)\pE\rvv^\ast(0)
      \\&\qquad
      +\pE\rvw(m)\pE\rvx^\ast(0)
      +\pE\rvw(m)\pE\rvv^\ast(0)
    && \text{by \prope{uncorrelated} hypotheses}
    && \text{(B), (C), and (D)}
  \\&= \pE\brs{\rvr(m)\rvx^\ast(0)}
      +\pE\rvr(m)\cancelto{0}{\pE\rvv^\ast(0)}
      \\&\qquad
      +\cancelto{0}{\pE\rvw(m)}\pE\rvx^\ast(0)
      +\cancelto{0}{\pE\rvw(m)}\pE\rvv^\ast(0)
    && \text{by \prope{zero-mean} hypothesis}
    && \text{(E)}
  \\&\eqd \boxed{\Rrx(m)}
    && \text{by definition of $\Rrx$}
    && \text{\xref{def:Rxym}}
\end{align*}
\begin{align*}
    (2).\quad\Rsy(m)
      &\eqd \pE\brs{\rvs^\ast(0)\rvy(m)}
     &&=\pE\brp{\brs{\rvr(0)+\rvw(0)}^\ast \brs{\rvx(m)+\rvv(m)}}
     &&= \scy \Rrx(m)+\cancelto{0}{\Rrv(m)}+\cancelto{0}{\Rwx(m)}+\cancelto{0}{\Rwv(m)}
     %&&=\scy\pE\brs{\rvr^\ast(0)\rvx(m)}
     %  +\cancelto{0}{\pE\brs{\rvr^\ast(0)\rvv(m)}}
     %  +\cancelto{0}{\pE\brs{\rvw^\ast(0)\rvx(m)}}
     %  +\cancelto{0}{\pE\brs{\rvw^\ast(0)\rvv(m)}}
     &&\eqd\Rrx(m)
    \\
    (3).\quad\Rsy(m)
      &\eqd \pE\brs{\rvs(0)\rvy^\ast(m)}
     &&=\pE\brp{\brs{\rvr(0)+\rvw(0)} \brs{\rvx(m)+\rvv(m)}^\ast}
     &&= \scy \Rrx(m)+\cancelto{0}{\Rrv(m)}+\cancelto{0}{\Rwx(m)}+\cancelto{0}{\Rwv(m)}
     %&&=\scy\pE\brs{\rvr(0)\rvx^\ast(m)}
     %  +\cancelto{0}{\pE\brs{\rvr(0)\rvv^\ast(m)}}
     %  +\cancelto{0}{\pE\brs{\rvw(0)\rvx^\ast(m)}}
     %  +\cancelto{0}{\pE\brs{\rvw(0)\rvv^\ast(m)}}
     &&\eqd\Rrx(m)
    \\
    (4).\quad\Rsy(m)
      &\eqd \pE\brs{\rvs^\ast(m)\rvy(0)}
     &&=\pE\brp{\brs{\rvr(m)+\rvw(m)}^\ast \brs{\rvx(0)+\rvv(0)}}
     &&= \scy \Rrx(m)+\cancelto{0}{\Rrv(m)}+\cancelto{0}{\Rwx(m)}+\cancelto{0}{\Rwv(m)}
     %&&=\scy\pE\brs{\rvr^\ast(m)\rvx(0)}
     %  +\cancelto{0}{\pE\brs{\rvr^\ast(m)\rvv(0)}}
     %  +\cancelto{0}{\pE\brs{\rvw^\ast(m)\rvx(0)}}
     %  +\cancelto{0}{\pE\brs{\rvw^\ast(m)\rvv(0)}}
     &&\eqd\Rrx(m)
    \\
    (5).\quad\Rsy(m)
      &\eqd \pE\brs{\rvs(0)\rvy(m)}
     &&=\pE\brp{\brs{\rvr(0)+\rvw(0)} \brs{\rvx(m)+\rvv(m)}}
     &&= \scy \Rrx(m)+\cancelto{0}{\Rrv(m)}+\cancelto{0}{\Rwx(m)}+\cancelto{0}{\Rwv(m)}
     %&&=\scy\pE\brs{\rvr(0)\rvx(m)}
     %  +\cancelto{0}{\pE\brs{\rvr(0)\rvv(m)}}
     %  +\cancelto{0}{\pE\brs{\rvw(0)\rvx(m)}}
     %  +\cancelto{0}{\pE\brs{\rvw(0)\rvv(m)}}
     &&\eqd\Rrx(m)
    \\
    (6).\quad\Rsy(m)
      &\eqd \pE\brs{\rvs(m)\rvy(0)}
     &&=\pE\brp{\brs{\rvr(m)+\rvw(m)} \brs{\rvx(0)+\rvv(0)}}
     &&= \scy \Rrx(m)+\cancelto{0}{\Rrv(m)}+\cancelto{0}{\Rwx(m)}+\cancelto{0}{\Rwv(m)}
     %&&=\scy\pE\brs{\rvr(m)\rvx(0)}
     %  +\cancelto{0}{\pE\brs{\rvr(m)\rvv(0)}}
     %  +\cancelto{0}{\pE\brs{\rvw(m)\rvx(0)}}
     %  +\cancelto{0}{\pE\brs{\rvw(m)\rvv(0)}}
     &&\eqd\Rrx(m)
    \\
    (7).\quad\Rsy(m)
      &\eqd \pE\brs{\rvs^\ast(0)\rvy^\ast(m)}
     &&=\pE\brp{\brs{\rvr(0)+\rvw(0)}^\ast \brs{\rvx(m)+\rvv(m)}^\ast}
     &&= \scy \Rrx(m)+\cancelto{0}{\Rrv(m)}+\cancelto{0}{\Rwx(m)}+\cancelto{0}{\Rwv(m)}
     %&&=\scy\pE\brs{\rvr(0)^\ast\rvx(m)^\ast}
     %  +\cancelto{0}{\pE\brs{\rvr^\ast(0)\rvv^\ast(m)}}
     %  +\cancelto{0}{\pE\brs{\rvw^\ast(0)\rvx^\ast(m)}}
     %  +\cancelto{0}{\pE\brs{\rvw^\ast(0)\rvv^\ast(m)}}
     &&\eqd\Rrx(m)
    \\
    (8).\quad\Rsy(m)
      &\eqd \pE\brs{\rvs^\ast(m)\rvy^\ast(0)}
     &&=\pE\brp{\brs{\rvr(m)+\rvw(m)}^\ast \brs{\rvx(0)+\rvv(0)}^\ast}
     &&= \scy \Rrx(m)+\cancelto{0}{\Rrv(m)}+\cancelto{0}{\Rwx(m)}+\cancelto{0}{\Rwv(m)}
     %&&=\scy\pE\brs{\rvr(m)^\ast\rvx(0)^\ast}
     %  +\cancelto{0}{\pE\brs{\rvr^\ast(m)\rvv^\ast(0)}}
     %  +\cancelto{0}{\pE\brs{\rvw^\ast(m)\rvx^\ast(0)}}
     %  +\cancelto{0}{\pE\brs{\rvw^\ast(m)\rvv^\ast(0)}}
     &&\eqd\Rrx(m)
\end{align*}
\end{proof}


%    %============================================================================
% Daniel J. Greenhoe
% LaTeX file
%============================================================================
%============================================================================
\subsection{Case study: Parallel operators}
%============================================================================
%---------------------------------------
\begin{minipage}{\tw-70mm}
\begin{proposition}
\label{prop:xGw_xHy}
%---------------------------------------
Let $\opS$ be the \structe{system} illustrated to the right,
where $\opT$ is \prope{not necessarily linear}.
Let 
\\\indentx$\ds\seqn{\fh(n)}\eqd\opH\kdelta(n)\eqd\sum_{m\in\Z}\fh(m)\kdelta(n-m)$
\\
be the \fncte{impulse response} of $\opH$.
\end{proposition}
\end{minipage}
\hfill\tbox{\includegraphics{graphics/xTy_xHw.pdf}}
\\
\propbox{
  \brb{\begin{array}{FMMD}
      (A).& $\rvx(n)$ is &\prope{WSS}       & and
    \\(B).& $\opH$    is &\prope{LTI}       &
  \end{array}}
  \implies
  \brb{\begin{array}{lc  >{\ds}l             @{\hspace{1pt}}l  M                     D}
      \Szwy(z)       &=& \ZH(z)              &\Szxy(z)       & for (1),(3),(5),(6) & and
    \\\Swwy(\omega)  &=& \FH(\omega)         &\Swxy(\omega)  & for (1),(3),(5),(6) & and
    \\\hline
      \Szwy(z)       &=& \ZH^\ast\brp{z^\ast}&\Szxy(z)       & for (2),(4),(7),(8) & and
    \\\Swwy(\omega)  &=& \FH^\ast(-\omega)   &\Swxy(\omega)  & for (2),(4),(7),(8) & 
  \end{array}}
  }
\\
\begin{proof}
\begin{align*}
  (1).\quad\Rwy(m)
    &\eqd \pE\brs{\rvw(m)\rvy^\ast(0)}
    && \text{by (A) and Papoulis' definition of $\Rwy$}
    && \text{\xref{def:Rxym}}
  \\&\eqd \pE\brp{\brs{\opH\rvx}(m)\rvy^\ast(0)}
    && \text{by definition of $\opS$}
  \\&= \opH\pE\brp{\rvx(m)\rvy^\ast(0)}
    && \text{by \prope{LTI} hypothesis}
    && \text{(B)}
  \\&\eqd \opH\Rxy(m)
    && \text{by Papoulis' definition of $\Rwy$}
    && \text{\xref{def:Rxym}}
  \\&= \sum_{n\in\Z}\fh(n)\Rxy(m-n)
    && \mathrlap{\text{by definition of \fncte{impulse response} $\seqn{\fh(n)}$}}
  \\&= \brs{\fh\convd\Rxy}(m)
    && \text{by definition of \ope{convolution}}
    && \text{\xref{def:dsp_conv}}
\end{align*}

\begin{align*}
    (2).\quad\Rwy(m)
      &\eqd \pE\brs{\rvw^\ast(0)\rvy(m)}
     &&\eqd \pE\brp{\brs{\opH\rvx}^\ast(0)\rvy(m)}
     &&=    \opH^\ast\brp{\pE\brs{\rvx^\ast(0)\rvy(m)}}
     &&\eqd \opH^\ast\Rxy(m)
     &&=    \brs{\fh^\ast\convd\Rxy}(m)
     \\
    (3).\quad\Rwy(m)
      &\eqd \pE\brs{\rvw(0)\rvy^\ast(m)}
     &&\eqd \pE\brp{\brs{\opH\rvx}(0)\rvy^\ast(m)}
     &&=    \opH\brp{\pE\brs{\rvx(0)\rvy^\ast(m)}}
     &&\eqd \opH\Rxy(m)
     &&=    \brs{\fh\convd\Rxy}(m)
     \\
    (4).\quad\Rwy(m)
      &\eqd \pE\brs{\rvw^\ast(m)\rvy(0)}
     &&\eqd \pE\brp{\brs{\opH\rvx}^\ast(m)\rvy(0)}
     &&=    \opH^\ast\brp{\pE\brs{\rvx^\ast(m)\rvy(0)}}
     &&\eqd \opH^\ast\Rxy(m)
     &&=    \brs{\fh^\ast\convd\Rxy}(m)
     \\
    (5).\quad\Rwy(m)
      &\eqd \pE\brs{\rvw(0)\rvy(m)}
     &&\eqd \pE\brp{\brs{\opH\rvx}(0)\rvy(m)}
     &&=    \opH\brp{\pE\brs{\rvx(0)\rvy(m)}}
     &&\eqd \opH\Rxy(m)
     &&=    \brs{\fh\convd\Rxy}(m)
     \\
    (6).\quad\Rwy(m)
      &\eqd \pE\brs{\rvw(m)\rvy(0)}
     &&\eqd \pE\brp{\brs{\opH\rvx}(m)\rvy(0)}
     &&=    \opH\brp{\pE\brs{\rvx(m)\rvy(0)}}
     &&\eqd \opH\Rxy(m)
     &&=    \brs{\fh\convd\Rxy}(m)
    \\
    (7).\quad\Rwy(m)
      &\eqd \pE\brs{\rvw^\ast(0)\rvy^\ast(m)}
     &&\eqd \pE\brp{\brs{\opH\rvx}^\ast(0)\rvy^\ast(m)}
     &&=    \opH^\ast\brp{\pE\brs{\rvx^\ast(0)\rvy^\ast(m)}}
     &&\eqd \opH^\ast\Rxy(m)
     &&=    \brs{\fh^\ast\convd\Rxy}(m)
    \\
    (8).\quad\Rwy(m)
      &\eqd \pE\brs{\rvw^\ast(m)\rvy^\ast(0)}
     &&\eqd \pE\brp{\brs{\opH\rvx}^\ast(m)\rvy^\ast(0)}
     &&=    \opH^\ast\brp{\pE\brs{\rvx^\ast(m)\rvy^\ast(0)}}
     &&\eqd \opH^\ast\Rxy(m)
     &&=    \brs{\fh^\ast\convd\Rxy}(m)
\end{align*}
\begin{align*}
  (1),(3),(5),(6).\quad\Szwy(z)
    &\eqd \opZ\Rwy(m)
   &&= \opZ\brs{\fh\convd\Rxy}(m)
   &&= \ZH(z)\Szxy(z)
  \\
  (2),(4),(7),(8).\quad\Szwy(z)
    &\eqd \opZ\Rwy(m)
   &&= \opZ\brs{\fh^\ast\convd\Rxy}(m)
   &&= \ZH^\ast\brp{z^\ast}\Szxy(z)
   && \text{by \prefp{prop:opZ}}
  \\
  (1),(3),(5),(6).\quad\Swwy(\omega)
    &\eqd \opDTFT\Rwy(m)
   &&= \opDTFT\brs{\fh\convd\Rxy}(m)
   &&= \FH(\omega)\Swxy(\omega)
  \\
  (2),(4),(7),(8).\quad\Swwy(\omega)
    &\eqd \opDTFT\Rwy(m)
   &&= \opDTFT\brs{\fh^\ast\convd\Rxy}(m)
   &&= \FH^\ast(-\omega)\Swxy(\omega)
   && \text{by \prefp{prop:opZ}}
\end{align*}
\end{proof}


%    %============================================================================
% Daniel J. Greenhoe
% LaTeX file
%============================================================================
%============================================================================
\subsection{Case study: Operator with measurement noise}
%============================================================================
%20190606%\begin{figure}[h]
%20190606%  \centering
%20190606%  \begin{tabular}{|c|c|}
%20190606%    \hline
%20190606%      \includegraphics{graphics/sysT_xyvq.pdf}
%20190606%    %&\includegraphics{graphics/sysT_xyup.pdf}
%20190606%     &\includegraphics{graphics/sysT_xypw.pdf}
%20190606%    %\\(A) $\Rxy(m)=\Rxq(m)$         &(B) $\Rpy(m)=\Rxy(m)+\Ruy(m)$     &(C) $\Rxy(m)=\Rpy(m)$
%20190606%    %\\\xref{lem:opT_addnoise_v}&    \xref{lem:opT_addnoise_u}    &    \xref{lem:opT_addnoise_w}
%20190606%    \\(A) $\Rxy(m)=\Rxq(m)$         &(B) $\Rxy(m)=\Rpy(m)$
%20190606%    \\\xref{lem:opT_addnoise_v}&    \xref{lem:opT_addnoise_w}
%20190606%    \\\hline
%20190606%  \end{tabular}
%20190606%  \caption{\label{fig:opT_addnoise}Additive noise with \prope{not necessarily linear} operator $\opT$}
%20190606%\end{figure}
%---------------------------------------
\begin{minipage}{\tw-70mm}
\begin{lemma}
\label{lem:opT_addnoise_v}
%---------------------------------------
Let $\opS$ be the \structe{system} illustrated to the right,
where $\opT$ is \prope{not necessarily linear}.
\end{lemma}
\end{minipage}
\hfill\tbox{\includegraphics{graphics/sysT_xyvq.pdf}}
\\
\lembox{
  \brb{\begin{array}{FMMD}
      (A).& $\rvx(n)$ is                & \prope{wide sense stationary} & and
    \\(B).& $\rvx(n)$ and $\rvv(n)$ are & \prope{uncorrelated}          & and
    \\(C).& $\rvv(n)$ is                & \prope{zero-mean}             &
  \end{array}}
  \implies
  \brb{\begin{array}{rclD}
      \Rxy(m)      &=& \Rxq(m)       & and
    \\\Szxy(z)     &=& \Szxq(z)      & and
    \\\Swxy(\omega)&=& \Swxq(\omega) &
  \end{array}}
  \quad
  \begin{array}{M}
    for all\\
    (1)--(8)
  \end{array}
  }
\\
\begin{proof}
\begin{align*}
  \Rxy(m)
    &\eqd \pE\brs{\rvx(m)\rvy^\ast(0)}
    && \text{by definition of $\Rxy$}
    && \text{\xref{def:Rxym}}
  \\&\eqd \pE\brs{\rvx(m)(\rvq(0)+\rvv(0))^\ast}
    && \text{by definition of $\opS$}
   %&& \text{\xref{fig:opT_addnoise} (A)}
  \\&= \pE\brs{\rvx(m)\rvq^\ast(0)+\rvp(m)\rvv^\ast(0)}
    && \text{by \prope{distributive} property of $\fieldC$}
  \\&= \pE\brs{\rvx(m)\rvq^\ast(0)} + \pE\brs{\rvx(m)\rvv^\ast(0)}
    && \text{by \prope{linearity} of $\pE$}
    && \text{\xref{prop:pE_linop}}
  \\&= \pE\brs{\rvx(m)\rvq^\ast(0)} + \brs{\pE\rvx(m)}\brs{\pE\rvv^\ast(0)}
    && \text{by \prope{uncorrelated} hypothesis}
    && \text{(B)}
  \\&= \pE\brs{\rvx(m)\rvq^\ast(0)} + \pE\brs{\rvp(m)}\cancelto{0}{\pE\brs{\rvv^\ast(0)}}
    && \text{by \prope{zero-mean} hypothesis}
    && \text{(C)}
  \\&= \Rxq(m)
    && \text{by definition of $\Rxq$}
    && \text{\xref{def:Rxym}}
  %20190605%\\
  %20190605%\Szxy(z)
  %20190605%  &\eqd \opZ\Rxy(m)
  %20190605%  && \text{by definition of $\Szxy$}
  %20190605%  && \text{\xref{def:Szxy}}
  %20190605%\\&= \opZ\Rxq(m)
  %20190605%  && \text{by previous result}
  %20190605%  && \text{(1)}
  %20190605%\\&= \Szxq(z)
  %20190605%  && \text{by definition of $\Szxq$}
  %20190605%  && \text{\xref{def:Szxy}}
  %20190605%\\
  %20190605%\Swxy(\omega)
  %20190605%  &\eqd \opDTFT\Rxy(m)
  %20190605%  && \text{by definition of $\Swxy$}
  %20190605%  && \text{\xref{def:Swxy}}
  %20190605%\\&= \opDTFT\Rxq(m)
  %20190605%  && \text{by previous result}
  %20190605%  && \text{(1)}
  %20190605%\\&= \Swxq(\omega)
  %20190605%  && \text{by definition of $\Swxq$}
  %20190605%  && \text{\xref{def:Swxy}}
\end{align*}
\begin{align*}
    (2).\quad\Rxy(m)
       &\eqd \pE\brs{\rvx^\ast(0) \rvy(m)}
      &&\eqd \pE\brp{\rvx^\ast(0) \brs{\rvq(m) + \rvv(m)}}
      &&\eqd \pE\brs{\rvx^\ast(0) \rvq(m)} + \pE\brs{\rvx^\ast(0)}\cancelto{0}{\brs{\pE{\rvv(m))}}}
      &&=    \Rxq(m)
     \\
    (3).\quad\Rxy(m)
       &\eqd \pE\brs{\rvx(0)\rvy^\ast(m)}
      &&\eqd \pE\brp{\rvx(0) \brs{\rvq(m) + \rvv(m)}^\ast}
      &&\eqd \pE\brs{\rvx(0) \rvq^\ast(m)} + \pE\brs{\rvx^\ast(0)}\cancelto{0}{\brs{\pE{\rvv^\ast(m))}}}
      &&=    \Rxq(m)
     \\
    (4).\quad\Rxy(m)
      &\eqd \pE\brs{\rvx^\ast(m)\rvy(0)}
      &&\eqd \pE\brp{\rvx^\ast(m) \brs{\rvq(0) + \rvv(0)}}
      &&\eqd \pE\brs{\rvx^\ast(m) \rvq(0)} + \pE\brs{\rvx^\ast(m)}\cancelto{0}{\brs{\pE{\rvv(0))}}}
      &&=    \Rxq(m)
     \\
    (5).\quad\Rxy(m)
      &\eqd \pE\brs{\rvx(0)\rvy(m)}
      &&\eqd \pE\brp{\rvx(0) \brs{\rvq(m) + \rvv(m)}}
      &&\eqd \pE\brs{\rvx(0) \rvq(m)} + \pE\brs{\rvx(0)}\cancelto{0}{\brs{\pE{\rvv(m))}}}
      &&=    \Rxq(m)
     \\
    (6).\quad\Rxy(m)
      &\eqd \pE\brs{\rvx(m)\rvy(0)}
      &&\eqd \pE\brp{\rvx(m) \brs{\rvq(0) + \rvv(0)}}
      &&\eqd \pE\brs{\rvx(m) \rvq(0)} + \pE\brs{\rvx(m)}\cancelto{0}{\brs{\pE{\rvv(0))}}}
      &&=    \Rxq(m)
    \\
    (7).\quad\Rxy(m)
      &\eqd \pE\brs{\rvx^\ast(0)\rvy^\ast(m)}
      &&\eqd \pE\brp{\rvx^\ast(0) \brs{\rvq(m) + \rvv(m)}^\ast}
      &&\eqd \pE\brs{\rvx^\ast(0) \rvq^\ast(m)} + \pE\brs{\rvx^\ast(0)}\cancelto{0}{\brs{\pE{\rvv^\ast(m))}}}
      &&=    \Rxq(m)
    \\
    (8).\quad\Rxy(m)
      &\eqd \pE\brs{\rvx^\ast(m)\rvy^\ast(0)}
      &&\eqd \pE\brp{\rvx^\ast(m) \brs{\rvq(0) + \rvv(0)}^\ast}
      &&\eqd \pE\brs{\rvx^\ast(m) \rvq^\ast(0)} + \pE\brs{\rvx^\ast(m)}\cancelto{0}{\brs{\pE{\rvv^\ast(0))}}}
      &&=    \Rxq(m)
  \end{align*}
\end{proof}

%20190605%%---------------------------------------
%20190605%\begin{lemma}
%20190605%\label{lem:opT_addnoise_u}
%20190605%%---------------------------------------
%20190605%Let $\opS$ be the \structe{system} illustrated in \prefpp{fig:opT_addnoise} (B).
%20190605%\lembox{
%20190605%  \brb{\begin{array}{FMMD}
%20190605%      (A).& $\rvx(n)$               is  & (\prope{WSS})          & and
%20190605%    \\(B).& $\rvu(n)$               is  & (\prope{zero-mean})    & and
%20190605%    \\(C).& $\rvx(n)$ and $\rvu(n)$ are & \mc{2}{M}{(\prope{uncorrelated})}
%20190605%  \end{array}}
%20190605%  \implies
%20190605%  \brb{\begin{array}{Frcl@{\hspace{2pt}}c@{\hspace{2pt}}lD}
%20190605%      (1).& \Rpq(m)      &=& \Rxy(m)       &+& \Ruy(m)       & and
%20190605%    \\(2).& \Szpq(z)     &=& \Szxy(z)      &+& \Szuy(z)      & and
%20190605%    \\(3).& \Swpq(\omega)&=& \Swxy(\omega) &+& \Swuy(\omega) &
%20190605%  \end{array}}
%20190605%  }
%20190605%\end{lemma}
%20190605%\begin{proof}
%20190605%\begin{align*}
%20190605%  \Rxy(m)
%20190605%    &\eqd \pE\brs{\rvx(m)\rvy^\ast(0)}
%20190605%    && \text{by definition of $\Rxy$}
%20190605%    && \text{\xref{def:Rxym}}
%20190605%  \\&\eqd \pE\brp{\brs{\rvp(m)-\rvu(m)}\rvy^\ast(0)}
%20190605%    && \text{by definition of $\opS$}
%20190605%  \\&= \pE\brs{\rvp(m)\rvy^\ast(0)-\rvu(m)\rvy^\ast(0)}
%20190605%  \\&= \pE\brs{\rvp(m)\rvy^\ast(0)} - \pE\brs{\rvu(m)\rvy^\ast(0)}
%20190605%    && \text{because $\pE$ is a \ope{linear operator}}
%20190605%    && \text{\xref{prop:pE_linop}}
%20190605%  \\&\eqd \Rpy(m) - \Ruy(m)
%20190605%    && \text{by definition of $\Rxy$}
%20190605%    && \text{\xref{def:Rxym}}
%20190605%\end{align*}
%20190605%\end{proof}

%---------------------------------------
\begin{minipage}{\tw-70mm}
\begin{lemma}
\label{lem:opT_addnoise_w}
%---------------------------------------
Let $\opS$ be the \structe{system} illustrated to the right,
where $\opT$ is \prope{not necessarily linear}.
\end{lemma}
\end{minipage}
\hfill\tbox{\includegraphics{graphics/sysT_xypw.pdf}}
\\
\lembox{
  \brb{\begin{array}{FMMD}
      (A).& $\rvx(n)$               is  & \prope{WSS}          & and
    \\(B).& $\rvu(n)$               is  & \prope{zero-mean}    & and
    \\(C).& $\rvx(n)$ and $\rvu(n)$ are & \mc{2}{M}{\prope{uncorrelated}}
  \end{array}}
  \implies
  \brb{\begin{array}{rclD}
      \Rxy(m)      &=& \Rpy(m)       & and 
    \\\Szxy(z)     &=& \Szpy(z)      & and
    \\\Swxy(\omega)&=& \Swpy(\omega) &
  \end{array}}
  \quad
  \begin{array}{M}
    for all\\
    (1)--(8)
  \end{array}
  }
\\
\begin{proof}
\begin{align*}
  \Rxy(m)
    &\eqd \pE\brs{\rvx(m)\rvy^\ast(0)}
    && \text{by definition of $\Rpy$}
    && \text{\xref{def:Rxym}}
  \\&\eqd \pE\brp{\brs{\rvp(m)+\rvu(m)}\rvy^\ast(0)}
    && \text{by definition of $\opS$}
  \\&= \pE\brs{\rvp(m)\rvy^\ast(0)+\rvu(m)\rvy^\ast(0)}
    && \text{by \prope{distributive} property of $\fieldC$}
  \\&= \pE\brs{\rvp(m)\rvy^\ast(0)} + \pE\brs{\rvu(m)\rvy^\ast(0)}
    && \text{because $\pE$ is a \ope{linear operator}}
    && \text{\xref{prop:pE_linop}}
  \\&= \pE\brs{\rvp(m)\rvy^\ast(0)} + \pE\brs{\rvu(m)}\pE\brs{\rvy^\ast(0)}
    && \text{by \prope{uncorrelated} hypothesis}
    && \text{(C)}
  \\&= \pE\brs{\rvp(m)\rvy^\ast(0)} + \pE\brs{\rvu(m)}\cancelto{0}{\pE\brs{\rvy^\ast(0)}}
    && \text{by \prope{zero-mean} hypothesis}
    && \text{(B)}
  \\&\eqd \Rpy(m)
    && \text{by definition of $\Rxy$}
    && \text{\xref{def:Rxym}}
\end{align*}
\begin{align*}
    (2).\quad\Rxy(m)
       &\eqd \pE\brs{\rvx^\ast(0) \rvy(m)}
      &&\eqd \pE\brp{\brs{\rvp(0)+\rvu(0)}^\ast\rvy(m)}
      &&=    \pE\brs{\rvp^\ast(0)\rvy(m)} + \cancelto{0}{\pE\brs{\rvu^\ast(0)}}\pE\brs{\rvy(m)}
      &&=    \Rpy(m)
     \\
    (3).\quad\Rxy(m)
       &\eqd \pE\brs{\rvx(0)\rvy^\ast(m)}
      &&\eqd \pE\brp{\brs{\rvp(0)+\rvu(0)}\rvy^\ast(m)}
      &&=    \pE\brs{\rvp(0)\rvy^\ast(m)} + \cancelto{0}{\pE\brs{\rvu^\ast(0)}}\pE\brs{\rvy^\ast(m)}
      &&=    \Rpy(m)
     \\
    (4).\quad\Rxy(m)
       &\eqd \pE\brs{\rvx^\ast(m)\rvy(0)}
      &&\eqd \pE\brp{\brs{\rvp(m)+\rvu(m)}^\ast\rvy(0)}
      &&=    \pE\brs{\rvp^\ast(m)\rvy(0)} + \cancelto{0}{\pE\brs{\rvu^\ast(m)}}\pE\brs{\rvy(0)}
      &&=    \Rpy(m)
     \\
    (5).\quad\Rxy(m)
       &\eqd \pE\brs{\rvx(0)\rvy(m)}
      &&\eqd \pE\brp{\brs{\rvp(0)+\rvu(0)}\rvy(m)}
      &&=    \pE\brs{\rvp(0)\rvy(m)} + \cancelto{0}{\pE\brs{\rvu(0)}}\pE\brs{\rvy(m)}
      &&=    \Rpy(m)
     \\
    (6).\quad\Rxy(m)
       &\eqd \pE\brs{\rvx(m)\rvy(0)}
      &&\eqd \pE\brp{\brs{\rvp(m)+\rvu(m)}\rvy(0)}
      &&=    \pE\brs{\rvp(m)\rvy(0)} + \cancelto{0}{\pE\brs{\rvu(m)}}\pE\brs{\rvy(0)}
      &&=    \Rpy(m)
    \\
    (7).\quad\Rxy(m)
       &\eqd \pE\brs{\rvx^\ast(0)\rvy^\ast(m)}
      &&\eqd \pE\brp{\brs{\rvp(0)+\rvu(0)}^\ast\rvy^\ast(m)}
      &&=    \pE\brs{\rvp^\ast(0)\rvy^\ast(m)} + \cancelto{0}{\pE\brs{\rvu^\ast(0)}}\pE\brs{\rvy^\ast(m)}
      &&=    \Rpy(m)
    \\
    (8).\quad\Rxy(m)
       &\eqd \pE\brs{\rvx^\ast(m)\rvy^\ast(0)}
      &&\eqd \pE\brp{\brs{\rvp(m)+\rvu(m)}^\ast\rvy^\ast(0)}
      &&=    \pE\brs{\rvp^\ast(m)\rvy^\ast(0)} + \cancelto{0}{\pE\brs{\rvu^\ast(m)}}\pE\brs{\rvy^\ast(0)}
      &&=    \Rpy(m)
  \end{align*}
\end{proof}

%---------------------------------------
\begin{minipage}{\tw-70mm}
\begin{proposition}[\thme{measurement additive noise cross-correlation}]
\label{prop:opT_mnoise}
%---------------------------------------
Let $\opS$ be the \structe{system} illustrated to the right,
where $\opT$ is \prope{not necessarily linear}.
\end{proposition}
\end{minipage}
\hfill\tbox{\includegraphics{graphics/opT_mnoise.pdf}}
\\
\propbox{
  \brb{\begin{array}{FMMD}
      (A).& $\rvx(n)$               is  & \prope{WSS}           & and
    \\(B).& $\rvu(n)$               is  & \prope{zero-mean}     & and
    \\(C).& $\rvv(n)$               is  & \prope{zero-mean}     & and
    \\(D).& \mc{3}{M}{$\rvx(n)$, $\rvw(n)$, $\rvv(n)$ are \prope{uncorrelated}}
  \end{array}}
  \implies
  \brb{\begin{array}{r*{3}{@{\hspace{1pt}}c@{\hspace{1pt}}l}D}
     \Rpq(m)      &=& \Rpy(m)       &=& \Rxq(m)        &=& \Rxy(m)        & and 
   \\\Szpq(z)     &=& \Szpy(z)      &=& \Szxq(z)       &=& \Szxy(z)       & and 
   \\\Swpq(\omega)&=& \Swpy(\omega) &=& \Swxq(\omega)  &=& \Swxy(\omega)  &
  \end{array}}
  \quad
  \begin{array}{M}
    for all\\
    (1)--(8)
  \end{array}
  }
\\
\begin{proof}
\begin{align*}
  \Rpq(m)
    &= \Rpy(m)
    && \text{by \prefp{lem:opT_addnoise_v}}
  \\
  \Rpq(m)
    &= \Rxq(m)
    && \text{by \prefp{lem:opT_addnoise_w}}
  \\
  \Rxy(m)
    &\eqd \pE\brs{\rvx(m)\rvy^\ast(0)}
    && \text{by definition $\Rxy$}
    && \text{\xref{def:Rxym}}
  \\&\eqd \pE\brp{\brs{\rvp(m)+\rvw(m)}\rvy^\ast(0)}
    && \text{by definition $\opS$}
   %&& \text{\xref{fig:opT_addnoise} (B)}
  \\&= \pE\brs{\rvp(m)\rvy^\ast(0)+\rvw(m)\rvy^\ast(0)}
  \\&= \pE\brs{\rvp(m)\rvy^\ast(0)} + \pE\brs{\rvw(m)\rvy^\ast(0)}
    && \text{by \prope{linearity} of $\pE$}
    && \text{\xref{prop:pE_linop}}
  \\&= \pE\brs{\rvp(m)\rvy^\ast(0)} + \cancelto{0}{\pE\brs{\rvw(m)\rvy^\ast(0)}}
    && \text{by \prope{uncorrelated} hypothesis}
    && \text{(D)}
  \\&= \Rpy(m)
    && \text{by definition of $\Rpy$}
    && \text{\xref{def:Rxym}}
\end{align*}
\begin{align*}
    (2).\quad\Rxy(m)
       &\eqd \pE\brs{\rvx^\ast(0) \rvy(m)}
      &&\eqd \pE\brp{\brs{\rvp(0)+\rvw(0)}^\ast\rvy(m)}
      &&=    \pE\brs{     \rvp^\ast(0)\rvy(m)} + \cancelto{0}{\brs{\pE\rvw^\ast(0)}} \brs{\pE\rvy(m)}
      &&= \Rpy(m)
     \\
    (3).\quad\Rxy(m)
       &\eqd \pE\brs{\rvx(0)\rvy^\ast(m)}
      &&\eqd \pE\brp{\brs{\rvp(0)+\rvw(0)}     \rvy^\ast(m)}
      &&=    \pE\brs{     \rvp(0)\rvy^\ast(m)} + \cancelto{0}{\brs{\pE\rvw(0)}} \brs{\pE\rvy^\ast(m)}
      &&= \Rpy(m)
     \\
    (4).\quad\Rxy(m)
       &\eqd \pE\brs{\rvx^\ast(m)\rvy(0)}
      &&\eqd \pE\brp{\brs{\rvp(m)+\rvw(m)}^\ast\rvy(0)}
      &&=    \pE\brs{     \rvp^\ast(m)\rvy(0)} + \cancelto{0}{\brs{\pE\rvw^\ast(m)}} \brs{\pE\rvy(0)}
      &&= \Rpy(m)
     \\
    (5).\quad\Rxy(m)
       &\eqd \pE\brs{\rvx(0)\rvy(m)}
      &&\eqd \pE\brp{\brs{\rvp(0)+\rvw(0)}     \rvy(m)}
      &&=    \pE\brs{     \rvp(0)\rvy(m)} + \cancelto{0}{\brs{\pE\rvw(0)}} \brs{\pE\rvy(m)}
      &&= \Rpy(m)
     \\
    (6).\quad\Rxy(m)
       &\eqd \pE\brs{\rvx(m)\rvy(0)}
      &&\eqd \pE\brp{\brs{\rvp(m)+\rvw(m)}\rvy(0)}
      &&=    \pE\brs{     \rvp(m)\rvy(0)} + \cancelto{0}{\brs{\pE\rvw(m)}} \brs{\pE\rvy(0)}
      &&= \Rpy(m)
    \\
    (7).\quad\Rxy(m)
       &\eqd \pE\brs{\rvx^\ast(0)\rvy^\ast(m)}
      &&\eqd \pE\brp{\brs{\rvp(0)+\rvw(0)}^\ast\rvy^\ast(m)}
      &&=    \pE\brs{     \rvp^\ast(0)\rvy^\ast(m)} + \cancelto{0}{\brs{\pE\rvw^\ast(0)}} \brs{\pE\rvy^\ast(m)}
      &&= \Rpy(m)
    \\
    (8).\quad\Rxy(m)
       &\eqd \pE\brs{\rvx^\ast(m)\rvy^\ast(0)}
      &&\eqd \pE\brp{\brs{\rvp(m)+\rvw(m)}^\ast\rvy^\ast(0)}
      &&=    \pE\brs{     \rvp^\ast(m)\rvy^\ast(0)} + \cancelto{0}{\brs{\pE\rvw^\ast(m)}} \brs{\pE\rvy^\ast(0)}
      &&= \Rpy(m)
  \end{align*}
\end{proof}

%    %============================================================================
% Daniel J. Greenhoe
% LaTeX file
%============================================================================
%============================================================================
\subsection{Case study: Parallel operators with measurement noise}
\label{sec:case_dual_mnoise}
%============================================================================
%---------------------------------------
\begin{minipage}{\tw-80mm}
\begin{proposition}
\label{prop:dual_mnoise}
%---------------------------------------
Let $\opS$ be the \structe{system} illustrated to the right,
where $\opT$ is an operator that is \prope{not necessarily linear}.
\end{proposition}
\end{minipage}
\qquad\tbox{\includegraphics{graphics/opT_opH_mnoise.pdf}}
\propbox{
  \brb{\begin{array}{FMMD}
      (A).& $\opH$  is                 &\prope{LTI}       & and
    \\(B).& $\rvx(n)$ is               &\prope{WSS}       & and
    \\(C).& $\rvu$ and $\rvv$ are      &\prope{zero-mean} & and
    \\(D).& $\rvp$, $\rvu$, $\rvv$ are &\prope{uncorrelated}
  \end{array}}
  \implies
  \brb{\begin{array}{rclMD}
      \Szsy(z)      &=& \ZH(z)     \Szxy(z)      & for (1),(3),(5),(6) & and
    \\\Swsy(\omega) &=& \FH(\omega)\Swxy(\omega) & for (1),(3),(5),(6) & and
    \\\hline
      \Szsy(z)      &=& \ZH^\ast\brp{z^\ast}\Szxy(z)      & for (2),(4),(7),(8) & and
    \\\Swsy(\omega) &=& \FH^\ast(-\omega)   \Swxy(\omega) & for (2),(4),(7),(8) & 
  \end{array}}
  }
\\
\begin{proof}
\begin{align*}
  (1),(3),(5),(6).\quad\Szsy(z)
    &= \Szrq(z)
    && \text{by \prefp{prop:dual_addnoise}}
    && \text{and (B), (C) and (D)}
  \\&= \ZH(z)\Szpq(z)
    && \text{by \prefp{prop:xGw_xHy}}
    && \text{and (A)}
  \\&= \ZH(z)\Szxq(z)
    && \text{by \prefp{lem:opT_addnoise_w}}
  \\&= \ZH(z)\Szxy(z)
    && \text{by \prefp{lem:opT_addnoise_v}}
  \\
  (1),(3),(5),(6).\quad\Swsy(\omega)
    &= \brlr{\Szsy(z)}_{z=e^{i\omega}}
    && \text{by definition of $\opZ$}
    && \text{\xref{def:opZ}}
  \\&= \brlr{\ZH(z)\Szxy(z)}_{z=e^{i\omega}}
    && \text{by previous result}
    && \text{(1)}
  \\&= \FH(\omega)\Swxy(\omega)
  \\
  (2),(4),(7),(8).\quad\Szsy(z)
    &= \Szrq(z)
    && \text{by \prefp{prop:dual_addnoise}}
    && \text{and (B), (C) and (D)}
  \\&= \ZH^\ast\brp{z^\ast}\Szpq(z)
    && \text{by \prefp{prop:xGw_xHy}}
    && \text{and (A)}
  \\&= \ZH^\ast\brp{z^\ast}\Szxq(z)
    && \text{by \prefp{lem:opT_addnoise_w}}
  \\&= \ZH^\ast\brp{z^\ast}\Szxy(z)
    && \text{by \prefp{lem:opT_addnoise_v}}
  \\
  (2),(4),(7),(8).\quad\Swsy(\omega)
    &= \brlr{\Szsy(z)}_{z=e^{i\omega}}
    && \text{by definition of $\opZ$}
    && \text{\xref{def:opZ}}
  \\&= \brlr{\ZH^\ast\brp{z^\ast}\Szxy(z)}_{z=e^{i\omega}}
    && \text{by previous result}
    && \text{(1)}
  \\&= \FH^\ast(-\omega)\Swxy(\omega)
   && \text{by \prefp{prop:opZ}}
\end{align*}
\end{proof}


%    %============================================================================
% Daniel J. Greenhoe
% LaTeX file
%============================================================================
%============================================================================
\subsection{Case study: Non-linear system identification}
\label{sec:case_systemid_nonlinear}
%============================================================================

\begin{figure}[h]
  \centering
  \begin{tabular}{|c|}
    \hline
    \tbox{\includegraphics{graphics/opT_estH_mnoise.pdf}}
    \\\hline
  \end{tabular}
  \caption{Least Square estimation \xref{prop:estHls}\label{fig:estHls}}
\end{figure}

%\begin{minipage}{\tw-50mm}
%Let $\opS$ be the \structe{system} illustrated to the right.
%\textbf{If} there is no measurement noise on the input and output and \textbf{if}
%$\opH$ is \prope{linear time invariant}, then
%$\FH = \Swxy/\Swxx$ \xref{cor:Rxyh}.
%But what if there is output measurement noise?
%And what if $\opH$ is not \prope{LTI}?
%What is the best least-squares estimate of $\FH$?
%The answer depends on how you define ``the best".
%\end{minipage}
%\hfill\tbox{\includegraphics{graphics/sysHw_xy.pdf}}

%---------------------------------------
\begin{remark}
\label{rem:fCost}
%---------------------------------------
The defintion of ``best" or ``optimal" is given by a \fncte{cost function} $\fCost(\estH)$.
There are several possible cost functions.
One possibility is to define an error $\rve(n)\eqd\rvq(n)-\rvr(n)$.
We note that if $\estH$ is closely tuned to match $\opT$, then 
not only should $\rve(n)$ be close to 0 for all $n\in\Z$, 
but the \fncte{auto-correlation} $\estRee(m)$ of $\rve(n)$ should also be close to 0 for all $m\in\Z$.
Moreover by extension, the \fncte{auto-spectral density} $\Swee(\omega)\eqd\opDTFT\estRee(m)$ 
should also be close to 0. 
As such, we can define an arguably relevant \fncte{cost function} for the system $\opS$ of
\prefpp{fig:estHls} in terms of $\Swxx$, $\Swyy$ and $\Swxy$. % as in \prefpp{def:fCost}.
In the case of Papoulis's $\Rxy(m)$, the development of such a cost function $\fCost(\estH)$ 
might look something like this:
\begin{align*}
  &(1).\quad\fCostrq\brp{\estH} %|
  \\&\eqd \opDTFT\pE\brp{\brs{\rvr(n)-\rvq(n)}\brs{\rvr(0)-\rvq(0)}^\ast}
    && \text{by definition of $\fCostrq$}
    && \text{\xref{def:fCost}}
  \\&=\opFT\brs{
            \pE\brs{\rvr(n)\rvr^\ast(0)}
           -\pE\brs{\rvr(n)\rvq^\ast(0)}
           -\pE\brs{\rvq(n)\rvr^\ast(0)}
           +\pE\brs{\rvq(n)\rvq^\ast(0)}
           }
    && \text{by \prope{linearity} of $\pE$}
    && \text{\xref{prop:pE_linop}}
  \\&\eqd \opFT\brs{\Rrr(m) - \Rrq(m) - \Rqr(m) + \Rqq(m)}
    && \text{by definition of $\Rxy$}
    && \text{\xref{def:Rxym}}
  \\&\eqd \boxed{\Swrr(\omega) - \Swrq(\omega) - \Swqr(\omega) + \Swqq(\omega)}
    && \text{by definition of $\Swxy$}
    && \text{\xref{def:Swxy}}
  %\\&= \Swrr(\omega) - \Swrq(\omega) - \Swrq^\ast(\omega) + \Swqq(\omega)
  %  && \text{by \prefp{cor:Swxy}}
  %\\&= \Swpp(\omega)\abs{\estH(\omega)}^2 - \Swrq(\omega) - \Swrq^\ast(\omega) + \Swqq(\omega)
  %  && \text{by \pref{cor:RxySwxy}}
  %\\&= \Swpp(\omega)\abs{\estH(\omega)}^2
  %   - \Swpy(\omega)\estH(\omega)
  %   - \Swpy^\ast(\omega)\estH^\ast(\omega)
  %   + \Swqq(\omega)
  %  && \text{by \pref{prop:dual_mnoise}}
\end{align*}
\end{remark}

Taking cue from the result of \pref{rem:fCost}, we arrive at \emph{a} definition of cost:

%---------------------------------------
\begin{definition}
\label{def:fCost}
%---------------------------------------
Let $\opS$ be a system defined as in \prefpp{fig:estHls}.
Define the following \fncte{cost function}s for spectral \prope{least-squares} estimates:
\defbox{\begin{array}{>{\ds}r*{2}{c>{\ds}l}}
      \fCostrq(\estH) &\eqd& 
        \Swrr(\omega) - \Swrq(\omega) - \Swqr(\omega) + \Swqq(\omega)
  \end{array}}
\end{definition}

%---------------------------------------
\begin{remark}
%---------------------------------------
Note that by \prefpp{cor:Swxx_real}, $\Swqr=\Swrq^\ast$ for (1)--(4)\ldots and thus 
the cost function $\fCost$ for (1)--(4) is \fncte{real-valued}. 
This in general is \emph{not} true for (5)--(8).
This in itself provides an argument, however weak that argument may be, 
for \emph{not} selecting any of (5)--(8) as a standard for the definition of $\Rxy(m)$.
\end{remark}

Now for each of the eight $\Rxy(m)$ definitions, we can transform the expression of $\fCostrq(\estH)$
as given by \pref{def:fCost} into expressions involving $\estH$ (next lemma). 
In doing so, one might hope to be in a good position to take partial derivatives of the real and imaginary parts of $\estH$
to find an optimal \prope{least-squares-like} solution for $\estH$.

%---------------------------------------
\begin{lemma}
\label{lem:H1LS_cost}
%---------------------------------------
Let $\fCostrq(\estH)$ be defined as in \pref{def:fCost}.
Let (1)--(8) below correspond to the eight definitions of $\Rxy(m)$ in \pref{def:Rxym}.
%\lembox{\begin{array}{>{\ds}rc>{\ds}l}
%  \fCostrq\brp{\estH}
%    &=& \Swpp     (\omega)\abs{\estH     (\omega)}^2
%      - \Swpy     (\omega)     \estH     (\omega)
%      - \Swpy^\ast(\omega)     \estH^\ast(\omega)
%      + \Swqq     (\omega)
%\end{array}}
%\end{lemma}
%\begin{proof}
\begin{small}
%\begin{align*}
%    %---------------------------
%  (1).\,\fCostrq\brp{\estH} %|
%    %---------------------------
%    &\eqd \Swrr(\omega) - \Swrq(\omega) - \Swqr(\omega) + \Swqq(\omega)
%    && \text{by definition of $\fCostrq$}
%    && \text{\xref{def:fCost}}
%  \\&= \Swrr(\omega) - \Swrq(\omega) - \Swrq^\ast(\omega) + \Swqq(\omega)
%    && \text{by \prefp{cor:Swxy}}
%
%  \\&= \Swpp(\omega)\abs{\estH(\omega)}^2 - \Swrq(\omega) - \Swrq^\ast(\omega) + \Swqq(\omega)
%    && \text{by \prefp{cor:RxySwxy}}
%
%  \\&= \Swpp(\omega)\abs{\estH(\omega)}^2
%     - \Swpy(\omega)\estH(\omega)
%     - \Swpy^\ast(\omega)\estH^\ast(\omega)
%     + \Swqq(\omega)
%    && \text{by \prefp{prop:dual_mnoise}}
%\end{align*}
\begin{align*}
  (1).\quad\fCostrq\brp{\estH}
     &= \ocom{\Swrr(\omega)- \Swrq(\omega) - \Swrq^\ast(\omega) + \Swqq(\omega)}
             {by \pref{def:fCost} and \prefp{cor:Swxy}}
    &&= \ocom{
        \Swpp(\omega)\abs{\estH(\omega)}^2
      - \Swpy(\omega)\estH(\omega)
      - \Swpy^\ast(\omega)\estH^\ast(\omega)
      + \Swqq(\omega)
      }{by \prefp{cor:RxySwxy} and \prefp{prop:dual_mnoise}}
    \\
  (2).\quad\fCostrq\brp{\estH}
     &= \Swrr(\omega)- \Swrq(\omega) - \Swrq^\ast(\omega) + \Swqq(\omega)
    &&= \Swpp(\omega)\abs{\estH(\omega)}^2
      - \Swpy(\omega)\estH^\ast(-\omega)
      - \Swpy^\ast(\omega)\estH(-\omega)
      + \Swqq(\omega)
    \\
  (3).\quad\fCostrq\brp{\estH}
     &= \Swrr(\omega)- \Swrq(\omega) - \Swrq^\ast(\omega) + \Swqq(\omega)
    &&= \Swpp(\omega)\abs{\estH(-\omega)}^2
      - \Swpy(\omega)\estH(\omega)
      - \Swpy^\ast(\omega)\estH^\ast(\omega)
      + \Swqq(\omega)
    \\
  (4).\quad\fCostrq\brp{\estH}
     &= \Swrr(\omega)- \Swrq(\omega) - \Swrq^\ast(\omega) + \Swqq(\omega)
    &&= \Swpp(\omega)\abs{\estH(-\omega)}^2
      - \Swpy(\omega)\estH^\ast(-\omega)
      - \Swpy^\ast(\omega)\estH(-\omega)
      + \Swqq(\omega)
    \\
  (5).\quad\fCostrq\brp{\estH}
     &= \Swrr(\omega)- \Swrq(\omega) - \Swrq(-\omega) + \Swqq(\omega)
    &&= \Swpp(\omega)\estH(\omega)\estH(-\omega)
      - \Swpy(\omega)\estH(\omega)
      - \Swpy(-\omega)\estH(-\omega)
      + \Swqq(\omega)
    \\
  (6).\quad\fCostrq\brp{\estH}
     &= \Swrr(\omega)- \Swrq(\omega) - \Swrq(-\omega) + \Swqq(\omega)
    &&= \Swpp(\omega)\estH(\omega)\estH(-\omega)
      - \Swpy(\omega)\estH(\omega)
      - \Swpy(-\omega)\estH(-\omega)
      + \Swqq(\omega)
    \\
  (7).\quad\fCostrq\brp{\estH}
     &= \Swrr(\omega)- \Swrq(\omega) - \Swrq(-\omega) + \Swqq(\omega)
    &&= \Swpp(\omega)\estH^\ast(\omega)\estH^\ast(-\omega)
      - \Swpy(\omega)\estH(\omega)
      - \Swpy(-\omega)\estH^\ast(\omega)
      + \Swqq(\omega)
    \\
  (8).\quad\fCostrq\brp{\estH}
     &= \Swrr(\omega)- \Swrq(\omega) - \Swrq(-\omega) + \Swqq(\omega)
    &&= \Swpp(\omega)\estH(\omega)\estH(-\omega)
      - \Swpy(\omega)\estH^\ast(-\omega)
      - \Swpy(-\omega)\estH(-\omega)
      + \Swqq(\omega)
\end{align*}
%\end{proof}
\end{small}
\end{lemma}

%%---------------------------------------
%\begin{remark}
%\label{rem:case_systemid_nonlinear}
%%---------------------------------------
For the Papoulis $\Rxy(m)$ definition (1), the $\fCostrq$ expression demonstrated in \pref{lem:H1LS_cost} is very 
useful. 
In particular, we can 
%However, it does illustrate what properties might be useful in identifying an optimal least squares solution.
%In particular, least squares methods involve 
set the partial derivatives 
$\pderiv{}{\estH_R}\estH(\omega)$ and $\pderiv{}{\estH_I}\estH(\omega)$
of the real and imaginary parts 
of $\estH(\omega)$ to zero and solve the resulting two equations to find an optimal $\estH$ (as in \prefp{prop:estHls}).

However, this becomes troublesome in the case when encountering $\estH(-\omega)$ and the 
\fncte{impulse response} of $\estH$ is \prope{complex-valued}---in which case in general $\estH(-\omega)\neq\estH^\ast(\omega)$.

Note that except for (1), \emph{all} of the expressions demonstrated in \pref{lem:H1LS_cost} contain an $\estH(-\omega)$ 
and/or $\estH^\ast(-\omega)$. 

This trouble provides an argument, however a weak one it might be, for choosing (1) as the standard definition of $\Rxy(m)$.
%\end{remark}

%---------------------------------------
\begin{proposition}
\label{prop:estHls}
%---------------------------------------
Let $\opS$ be the \structe{system} illustrated in \prefp{fig:estHls}.
\propbox{
  \brb{\begin{array}{FMMD}
      (A).& $\rvx$, $\rvu$, and $\rvv$ are & \prope{WSS}          & and
    \\(B).& $\rvx$, $\rvu$, and $\rvv$ are & \prope{uncorrelated} & and
    \\(C).& $\rvu$ and $\rvv$ are          & \prope{zero-mean}    & and
    \\(D).& $\estH$ is                     & \prope{LTI}
  \end{array}}
  \implies
  \brb{\begin{array}{>{\ds}r>{\ds}c>{\ds}l}
    \argmin_{\estH}\fCostrq(\estH) &=& \frac{\Swxy^\ast(\omega)}{\Swxx(\omega)-\Swuu(\omega)}
    \\\mc{3}{M}{for (1)}
  \end{array}}
  }
\end{proposition}
\begin{proof}
\begin{align*}
  (1).\quad 
    0 &= \pderiv{}{\estH_R}\fCostrq\brs{\estH(\omega)}
      &= 2\estH_R(\omega)\Swpp(\omega) -  \Swpy(\omega) -  \Swpy^\ast(\omega) + \cancelto{0}{\pderiv{}{\estH_R}\Swqq(\omega)}
     &&\implies \estH_R(\omega) = \frac{\Real\Swpy^\ast(\omega)}{\Swpp(\omega)}
     \\
    0 &= \pderiv{}{\estH_I}\fCostrq\brs{\estH(\omega)}
      &= 2\estH_I(\omega)\Swpp(\omega) -  i\Swpy(\omega) + i\Swpy^\ast(\omega) + \cancelto{0}{\pderiv{}{\estH_R}\Swqq(\omega)}
     &&\implies \estH_I(\omega) = \frac{\Imag\Swpy^\ast(\omega)}{\Swpp(\omega)}
\end{align*}
\begin{align*}
    \implies \estH(\omega)
      &\eqd \estH_R(\omega) + i\estH_I(\omega)
        \frac{\Real\Swpy^\ast(\omega)}{\Swpp(\omega)}
        + \frac{i\Imag\Swpy^\ast(\omega)}{\Swpp(\omega)}
    \\&= \frac{\Swpy^\ast(\omega)}{\Swpp(\omega)}
    \\&= \frac{\Swxy^\ast(\omega)}{\Swxx(\omega)-\Swuu(\omega)}
      && \text{by \prefp{prop:xvy}}
\end{align*}
\end{proof}

It follows immediately from \pref{prop:estHls} that, for (1) and in the special case
of no input noise ($\rvu(n)=0$), the standard estimate\footnote{
  \citerppgc{bendat1980}{98}{100}{0471058874}{5.1.1 Optimal Character of Calculations; note: proof minimizing $\Swvv$ but yields same result},
  \citerppgc{bendat1993}{106}{109}{0471570559}{5.1.1 Optimality of Calculations},
  \citerppgc{bendat2010}{187}{190}{1118210824}{6.1.4 Optimum Frequency Response Functions}
  }
$\estHa$ is the optimal
least-squares estimate of $\FH$ (next).

%---------------------------------------
\begin{corollary}
\label{cor:H1LSb}
%---------------------------------------
Let $\opS$ be the \structe{system} illustrated in \prefp{fig:estHls}.
\corbox{
  \brb{\begin{array}{FMD}
      (1). & hypotheses of \pref{prop:estHls} & and
    \\(2). & $\rvu(n)=0$
  \end{array}}
  \implies
  \brb{\estH(\omega) = \estHa(\omega) \eqd \frac{\Swxy^\ast(\omega)}{\Swxx(\omega)}
  }
  \qquad
  \begin{array}{M}
    for (1)
  \end{array}
  }
\end{corollary}

%   %%============================================================================
% Daniel J. Greenhoe
% LaTeX file
%============================================================================
%============================================================================
\subsection{Case study: Linear system identification}
%============================================================================
%=======================================
%\subsection{Least squares estimates of linear systems}
%=======================================
\begin{figure}
  \centering
  \tbox{\includegraphics{graphics/opH_mnoise.pdf}}
  \caption{Additive noise linear system for \pref{sec:case_estHx} Case study (System identification)\label{fig:addnoise_LTI}}
\end{figure}

% ___________________________________________________________________________________
%| |                                   ____________________________________________  |
%|D|                               \  /[                   ]2          |        |2   |
%|E|          Syy(w) - k(w)Sxx(w) + \/ [Syy(w) - k(w)Sxx(w)]   + 4 k(w)| Sxy(w) |    |
%|F| Hk(w) = --------------------------------------- ----------------- ------------- |
%| |                                       2 Sxy(w)                                  |
%|_|_________________________________________________________________________________|
\begin{definition}                                                                   %
\footnote{                                                                           %
  \citePpc{white2006}{679}{(6)},                                                     %
  \citerpgc{shin2008}{293}{0470725648}{(9.67)},                                      %
  \citePpc{wicks1986}{898}{has additional $s$ in denominator}
  }                                                                                  %
\label{def:Hkp}                                                                      %
%------------------------------------------------------------------------------------%
Let $\opS$ be the \structe{system} illustrated in \prefpp{fig:addnoise_LTI}.
\defbox{\begin{array}{l>{\ds}rc>{\ds}lC}
  \mc{5}{M}{
  The \fnctd{transfer function estimate $\estHkp(\omega; \kappa)$}
  with \vald{scaling function} $\kappa(\omega)$ is defined as
  }
  \\&
  \estHkp(\omega; \kappa) &\eqd&
      \frac{\Swyy(\omega)-\kappa(\omega)\Swxx(\omega) +
            \sqrt{\brs{\Swyy(\omega) - \kappa(\omega)\Swxx(\omega)}^2 +
                  4\kappa(\omega)\abs{\Swxy(\omega)}^2}
           }
           {2\Swxy(\omega)}
    & \forall \omega,\kappa(\omega)\in\R
  \\
  \mc{5}{M}{The \fnctd{transfer function estimate $\estHs(\omega; s)$}
  with \vald{scaling constant} $s$ is defined as}
  \\&
  \estHs(\omega; s) &\eqd& \brlr{\estHkp(\omega; \kappa)}_{\kappa=s^2}
    & \forall \omega,s\in\R
\end{array}}
\end{definition}

%---------------------------------------
\begin{lemma}
\label{lem:estHk_Suu}
%---------------------------------------
Let $\opS$ be the \structe{system} illustrated in \prefpp{fig:addnoise_LTI}.
\lemboxt{
  $\brb{\begin{array}{M}
           There exists $\kappa(\omega)$ such that
         $\Swvv(\omega)=\kappa(\omega)\Swuu(\omega)$
  \end{array}}$
  \\\indentx$\implies
  \brb{\begin{array}{>{\ds}rc>{\ds}l}
    \Swuu(\omega) = \frac{\abs{\estH(\omega)}^2\Swxx(\omega) - \estH(\omega)\Swxy(\omega) - \estH^\ast(\omega)\Swxy^\ast(\omega) + \Swyy(\omega)}
                         {\kappa(\omega)+\abs{\estH(\omega)}^2}
  \end{array}}$
  }
\end{lemma}
\begin{proof}
\begin{align*}
     (1).\quad \Swuu(\omega) &= \Swxx(\omega) - \Swpp(\omega) &&=\ocom{\Swxx(\omega) - \frac{\Swpq(\omega)}{\estH^\ast( \omega)}}{by \pref{cor:RxySwxy}} &&= \ocom{\Swxx(\omega) - \frac{\Swxy(\omega)}{\estH^\ast( \omega)}}{by \prope{uncorrelated} hypothesis}
  %\\(2).\quad \Swuu(\omega) &= \Swxx(\omega) - \Swpp(\omega) &&= \Swxx(\omega) - \ffrac{\Swpq(\omega)}{\estH     ( \omega)} &&= \Swxx(\omega) - \ffrac{\Swxy(\omega)}{\estH     ( \omega)}
  %\\(3).\quad \Swuu(\omega) &= \Swxx(\omega) - \Swpp(\omega) &&= \Swxx(\omega) - \ffrac{\Swpq(\omega)}{\estH^\ast(-\omega)} &&= \Swxx(\omega) - \ffrac{\Swxy(\omega)}{\estH^\ast(-\omega)}
  %\\(4).\quad \Swuu(\omega) &= \Swxx(\omega) - \Swpp(\omega) &&= \Swxx(\omega) - \ffrac{\Swpq(\omega)}{\estH     (-\omega)} &&= \Swxx(\omega) - \ffrac{\Swxy(\omega)}{\estH     (-\omega)}
  %\\(5).\quad \Swuu(\omega) &= \Swxx(\omega) - \Swpp(\omega) &&= \Swxx(\omega) - \ffrac{\Swpq(\omega)}{\estH     ( \omega)} &&= \Swxx(\omega) - \ffrac{\Swxy(\omega)}{\estH     ( \omega)}
  %\\(6).\quad \Swuu(\omega) &= \Swxx(\omega) - \Swpp(\omega) &&= \Swxx(\omega) - \ffrac{\Swpq(\omega)}{\estH     (-\omega)} &&= \Swxx(\omega) - \ffrac{\Swxy(\omega)}{\estH     (-\omega)}
  %\\(7).\quad \Swuu(\omega) &= \Swxx(\omega) - \Swpp(\omega) &&= \Swxx(\omega) - \ffrac{\Swpq(\omega)}{\estH^\ast(-\omega)} &&= \Swxx(\omega) - \ffrac{\Swxy(\omega)}{\estH^\ast(-\omega)}
  %\\(8).\quad \Swuu(\omega) &= \Swxx(\omega) - \Swpp(\omega) &&= \Swxx(\omega) - \ffrac{\Swpq(\omega)}{\estH     (-\omega)} &&= \Swxx(\omega) - \ffrac{\Swxy(\omega)}{\estH     (-\omega)}
  %\\
   \\(1).\quad \Swvv(\omega) &= \Swyy(\omega) - \Swqq(\omega) &&= \Swyy(\omega) - \Swpq(\omega)\estH     ( \omega) &&= \Swyy(\omega) - \Swxy(\omega)\estH     ( \omega)
  %\\(2).\quad \Swvv(\omega) &= \Swyy(\omega) - \Swqq(\omega) &&= \Swyy(\omega) - \Swpq(\omega)\estH     ( \omega) &&= \Swyy(\omega) - \Swxy(\omega)\estH     ( \omega)
  %\\(3).\quad \Swvv(\omega) &= \Swyy(\omega) - \Swqq(\omega) &&= \Swyy(\omega) - \Swpq(\omega)\estH     (-\omega) &&= \Swyy(\omega) - \Swxy(\omega)\estH     (-\omega)
  %\\(4).\quad \Swvv(\omega) &= \Swyy(\omega) - \Swqq(\omega) &&= \Swyy(\omega) - \Swpq(\omega)\estH^\ast(-\omega) &&= \Swyy(\omega) - \Swxy(\omega)\estH^\ast(-\omega)
  %\\(5).\quad \Swvv(\omega) &= \Swyy(\omega) - \Swqq(\omega) &&= \Swyy(\omega) - \Swpq(\omega)\estH     (-\omega) &&= \Swyy(\omega) - \Swxy(\omega)\estH     (-\omega)
  %\\(6).\quad \Swvv(\omega) &= \Swyy(\omega) - \Swqq(\omega) &&= \Swyy(\omega) - \Swpq(\omega)\estH     ( \omega) &&= \Swyy(\omega) - \Swxy(\omega)\estH     ( \omega)
  %\\(7).\quad \Swvv(\omega) &= \Swyy(\omega) - \Swqq(\omega) &&= \Swyy(\omega) - \Swpq(\omega)\estH     ( \omega) &&= \Swyy(\omega) - \Swxy(\omega)\estH     ( \omega)
  %\\(8).\quad \Swvv(\omega) &= \Swyy(\omega) - \Swqq(\omega) &&= \Swyy(\omega) - \Swpq(\omega)\estH     ( \omega) &&= \Swyy(\omega) - \Swxy(\omega)\estH     ( \omega)
%  \\
%  \\(1).\quad \Swuu\brs{\abs{\estH}^2+\kappa} &= \Swuu\abs{\estH}^2 + \Swvv &&= \abs{\estH}^2\brs{\Swxx - \frac{\Swxy}{\estH^\ast}} +\brs{\Swyy - \Swxy\estH}  &&= \abs{\estH}^2\Swxx - \estH^\ast\Swyx -\Swxy\estH + \Swyy
%  \\
\end{align*}

\begin{align*}
    (1).\quad \Swuu\brs{\abs{\estH}^2+\kappa}
          &= \Swuu\abs{\estH}^2 + \Swvv
         &&= \abs{\estH}^2\brs{\Swxx - \frac{\Swxy}{\estH^\ast}} +\brs{\Swyy - \Swxy\estH}
         &&= \abs{\estH}^2\Swxx - 2\estH\Swxy + \Swyy
    \\
    (2).\quad \Swuu\brs{\abs{\estH}^2+\kappa}
         &= \Swuu\abs{\estH}^2 + \Swvv
         &&= \abs{\estH}^2\brs{\Swxx - \frac{\Swxy}{\estH}} +\brs{\Swyy - \Swxy\estH}
         &&= \abs{\estH}^2\Swxx - \estH^\ast\Swxy - \estH\Swxy + \Swyy
    \\
    (3).\quad \Swuu\brs{\abs{\estH}^2+\kappa}
          &= \Swuu\abs{\estH}^2 + \Swvv
         &&= \abs{\estH}^2\brs{\Swxx - \frac{\Swxy}{\estH^\ast(-\omega)}} +\brs{\Swyy - \Swxy\estH(-\omega)}
        % &&= \abs{\estH}^2\Swxx - \estH^\ast\Swxy - \estH\Swxy + \Swyy 
\end{align*}

\begin{align*}
     \Swuu\brs{\abs{\estH}^2+\kappa}
          &= \abs{\estH}^2\Swuu +\kappa\Swuu
          && \eqd \Swuu\abs{\estH}^2 + \Swvv
          &&= \abs{\estH}^2\brs{\Swxx - \frac{\Swyx}{\estH}}
      +\brs{\Swyy - \Swxy\estH}
  \\&= \abs{\estH}^2\Swxx - \estH^\ast\Swyx
      -\Swxy\estH + \Swyy
  \\&\mathrlap{\implies
    \boxed{
      \Swuu(\omega) = \frac{\abs{\estH(\omega)}^2\Swxx(\omega) - \estH(\omega)\Swxy(\omega) - \estH^\ast(\omega)\Swxy^\ast(\omega) + \Swyy(\omega)}
                   {\kappa(\omega)+\abs{\estH(\omega)}^2}
            }}
\end{align*}
\end{proof}

%---------------------------------------
\begin{theorem}
\footnote{
  \citePpc{wicks1986}{898}{has additional $s$ in denominator},
  \citerpgc{shin2008}{293}{0470725648}{(9.67)},
  \citePpc{white2006}{679}{(6)}
  }
\label{thm:estHk}
\label{thm:estHs}
%---------------------------------------
Let $\opS$ be the \structe{system} illustrated in \prefpp{fig:addnoise_LTI}.
Let $\estHkp(\omega)$ be the transfer function estimate defined in \prefpp{def:Hkp}.
%Let \\\indentx$\norm{\Swxx(\omega)}^2\eqd\inprod{\Swxx(\omega)}{\Swxx(\omega)}\eqd\pE ...$
\thmbox{
  \brb{\begin{array}{FMD}
      (1).& There exists $\kappa(\omega)$ such that     &
        \\& $\Swvv(\omega)=\kappa(\omega)\Swuu(\omega)$ & and
    \\(2).& $\fCost(\estHls)=\Swuu(\omega)$             &
  \end{array}}
  \implies
  \brb{\begin{array}{>{\ds}rc>{\ds}l}
    \argmin_{\estH}\fCost(\estH) &=& \estHkp(\omega)
    \\\mc{3}{D}{($\estHkp$ is the ``optimal" estimator for minimizing system noise)}
  \end{array}}
  }
\end{theorem}
\begin{proof}
\begin{enumerate}
  \item Let \label{item:estHk_FG}
        $\begin{array}[t]{rclDD}
             \fF &\eqd& \abs{\estH(\omega)}^2\Swxx(\omega) - \estH(\omega)\Swxy(\omega) - \estH^\ast(\omega)\Swxy^\ast(\omega) + \Swyy(\omega)
               & (numerator in \pref{lem:estHk_Suu})
               & and
           \\\fG &\eqd& \kappa(\omega)+\abs{\estH(\omega)}^2
               & (denominator in \pref{lem:estHk_Suu})
        \end{array}$

  \item lemma $\brp{\pderiv{}{\estH_R}\Swuu}$: \label{ilem:estHk_HR}
    \begin{align*}
      \boxed{0}
        &= \frac{1}{2}\fG^20 = \frac{1}{2}\fG^2\pderiv{}{\estH_R}\Swuu
        && \text{set $\pderiv{}{\estH_R}\Swuu=0$ to find optimum $\estH_R$}
      \\&= \frac{1}{2}\fG^2\pderiv{}{\estH_R}\frac{\fF}{\fG}
        && \text{by \prefp{lem:estHk_Suu}}
      \\&= \frac{1}{2}\fG^2\frac{(\fF'\fG-\fG'\fF)}{\fG^2}
        && \text{by \thme{Quotient Rule}}
      \\&= \frac{1}{2}(\fF'\fG-\fG'\fF)
      \\&= \frac{1}{2}\brs{2\estH_R\Swxx - \Swxy - \Swxy^\ast}\fG - \frac{1}{2}2\estH_R\fF
        && \text{by definition of $\fF$, $\fG$}
        && \text{\xref{item:estHk_FG}}
      \\&= \boxed{\estH_R\Swxx \fG - \fG\Real\Swxy - \estH_R\fF}
    \end{align*}

  \item lemma $\brp{\pderiv{}{\estH_I}\Swuu}$: \label{ilem:estHk_HI}
    \begin{align*}
      \boxed{0}
        &= \frac{1}{2}\fG^20 = \frac{1}{2}\fG^2\pderiv{}{\estH_I}\Swuu
        && \text{set $\pderiv{}{\estH_I}\Swuu=0$ to find optimum $\estH_I$}
      \\&= \frac{1}{2}\fG^20 = \frac{1}{2}\fG^2\pderiv{}{\estH_I}\frac{\fF}{\fG}
        && \text{by \prefp{lem:estHk_Suu}}
      \\&= \frac{1}{2}\fG^2\frac{(\fF'\fG-\fG'\fF)}{\fG^2}
        && \text{by \thme{Quotient Rule}}
      \\&= \frac{1}{2}(\fF'\fG-\fG'\fF)
      \\&= \frac{1}{2}\brs{2\estH_I\Swxx - i\Swxy + i\Swxy^\ast}\fG - \frac{1}{2}2\estH_I\fF
        && \text{by definition of $\fF$, $\fG$}
        && \text{\xref{item:estHk_FG}}
      \\&= \boxed{\estH_I\Swxx \fG + \fG\Imag\Swxy - \estH_I\fF}
    \end{align*}

  \item Solve for $\estH$ \ldots
    \begin{align*}
      0
        &= 0 + i0
         = \frac{1}{2}\fG^20
         + \frac{1}{2}\fG^20
         = \frac{1}{2}\fG^2\pderiv{}{\estH_R}\Swuu
         +i\frac{1}{2}\fG^2\pderiv{}{\estH_I}\Swuu
      \\&= \brs{\estH_R\Swxx \fG - \fG\Real\Swxy - \estH_R\fF}
         +i\brs{\estH_I\Swxx \fG + \fG\Imag\Swxy - \estH_I\fF}
        && \text{by \pref{ilem:estHk_HR} and \pref{ilem:estHk_HI}}
      \\&= \mathrlap{
             \estH\Swxx \fG - \Swxy^\ast \fG - \estH \fF
             \quad\text{because $\Real(z) + i\Imag(z)=z$ and $\Real(z) - i\Imag(z)=z^\ast$}
             }
      \\&= \estH\Swxx \fG - \Swyx \fG - \estH \fF
        && \text{by \prefp{cor:Swxy_sym}}
      \\&= \estH\Swxx\brp{\kappa+\abs{\estH}^2} - \Swyx\brp{\kappa+\abs{\estH}^2}
         - \estH\brp{\abs{\estH}^2\Swxx - \estH\Swxy - \estH^\ast\Swxy^\ast + \Swyy}
        && \text{by $\fF$, $\fG$ defs.}
      \\&= \estH\Swxx\brp{\kappa+\cancelto{}{\abs{\estH}^2}} - \Swyx\brp{\kappa+\abs{\estH}^2}
         - \estH\brp{\cancelto{}{\abs{\estH}^2\Swxx} - \estH\Swxy - \estH^\ast\Swxy^\ast + \Swyy}
      \\&= \kappa\estH\Swxx - \Swyx\brp{\kappa+\abs{\estH}^2}
         + \brp{ \estH^2\Swxy + \abs{\estH}^2\Swxy^\ast - \estH\Swyy}
      \\&= \kappa\estH\Swxx - \kappa\Swyx - \cancelto{}{\Swyx\abs{\estH}^2}
         + \brp{ \estH^2\Swxy + \cancelto{}{\abs{\estH}^2\Swxy^\ast} - \estH\Swyy}
      \\&=  \estH^2\Swxy + \estH\brs{\kappa\Swxx - \Swyy} - \kappa\Swxy^\ast
      \\
      \implies&\boxed{
        \estH = \frac{\brp{\Swyy-\kappa\Swxx} \pm \sqrt{\brp{\Swyy-\kappa\Swxx}^2 + 4\kappa\abs{\Swxy}^2}}
                     {2\Swxy}
      }% Praise the Lord!!!     2019 March 17 Sunday
      && \text{by \thme{Quadratic Equation}}
    \end{align*}
\end{enumerate}
\end{proof}


%20190528%%============================================================================
%20190528%\subsection{Coherence}
%20190528%%============================================================================
%20190528%%=======================================
%20190528%%\subsubsection{Application}
%20190528%%=======================================
%20190528%%Coherence has two basic purposes:
%20190528%%\begin{enume}
%20190528%%  \item The \fncte{coherence} of $\rvx$ and $\rvy$ is a measure of how closely
%20190528%%        $\rvx$ and $\rvy$ are statistically related.
%20190528%%        %That is, it is an indication of how much $\rvx$ and $\rvy$
%20190528%%        %``cohere" or ``stick" together
%20190528%%  \item
%20190528%The \fncte{coherence} of $\rvx$ and $\rvy$ is a measure of how reliable
%20190528%        are the estimates $\estHa$ and $\estHb$ \xxref{def:H1}{def:H2}.
%20190528%        If the coherence is $0.70$ or above, then we can have high confidence that
%20190528%        the estimates $\estHa$ and $\estHb$ are ``good"
%20190528%        estimates.\footnote{\citerpgc{liang2015}{363}{365}{1498702341}{7.4.2\scshape Coherece Function}}
%20190528%%\end{enume}
%20190528%
%20190528%%=======================================
%20190528%%\subsubsection{Definitions}
%20190528%%=======================================
%20190528%%---------------------------------------
%20190528%\begin{definition}
%20190528%\footnote{
%20190528%  \citePp{chen2012}{4699}{(1), (2)},
%20190528%  \citerppgc{liang2015}{363}{365}{1498702341}{7.4.2 Coherence function},
%20190528%  \citerpgc{ewins1986}{131}{0863800173}{$\gamma^2=\ffrac{H_1(\omega)}{H_2(\omega)}$ (3.8)}
%20190528%  }
%20190528%\label{def:Cxy}
%20190528%\label{def:oCxy}
%20190528%%---------------------------------------
%20190528%Let $\opS$ be a \structe{system} with input $\rvx(n)$ and output $\rvy(n)$.
%20190528%\defbox{\begin{array}{MM>{\ds}rc>{\ds}l}
%20190528%  The \fnctd{complex coherence} function &is defined as&
%20190528%    \Cxy(\omega) &\eqd& \frac{\Swxy^\ast(\omega)}{\sqrt{\Swxx(\omega)\Swyy(\omega)}}
%20190528%    \\
%20190528%  The \fnctd{ordinary coherence} function &is defined as&
%20190528%    \oCxy(\omega) &\eqd& \frac{\abs{\Swxy(\omega)}^2}{\Swxx(\omega)\Swyy(\omega)}
%20190528%\end{array}}
%20190528%\end{definition}
%20190528%

%  %============================================================================
% LaTeX File
% Daniel J. Greenhoe
%============================================================================
%======================================
\section{Which one?}
%======================================
Which definition of $\Rxy(m)$ should we use?
Any one of them is perfectly acceptable---as long as a clear definition is provided and that definition is used consistently.
That being said, note the following:

\begin{enumerate}
\item The \ope{expectation} operator $\pE\brp{\rvX\rvY^\ast}$ is an \fncte{inner product}.
As such, it would seem the most natural to follow the convention of other inner product definitions
and thus put the conjugate $\conj$ on $\rvy$ (i.e. follow Papoulis):
\\\indentx$\begin{array}{c>{\ds}rc>{\ds}l}
    $\imark$ & \inprod{\fx(t)}{\fy(t)} &\eqd& \int_{t\in\R} \fx(t)\fy^\conj(t) \dt
  \\$\imark$ & \inprod{\fx(n)}{\fy(n)} &\eqd& \sum_{n\in\Z} \fx(n)\fy^\conj(n)
  \\$\imark$ & \inprod{\rvX}{\rvY}     &\eqd& \pE\brp{\rvX\rvY^\conj}
\end{array}$

\item If we view $\Rxy(m)$ as an \ope{analysis} of $\rvy$ in terms of $\rvx$
      (or as a \ope{projection} of $\rvy$ onto $\rvx$),
      then it would seem more natural to put the conjugate on $\rvx$ (i.e. follow Kay).
      This is what is done in Fourier analysis when projecting a function $\ff(t)$ onto the
      set of basis functions $\set{e^{i\omega n}}{\omega\in\R}$, as in
      \\\begin{align*}
        \opDTFT\brs{\rvy(n)}(\omega)
          &\eqd \inprod{\rvy(n)}{e^{i\omega n}}
          && \text{(\ope{project} $\rvy(n)$ onto $e^{i\omega n}$ for some $\omega\in\R$)}
        \\&\eqd \sum_{n\in\Z} \rvy(n) \brs{e^{+i\omega n}}^\ast
        \\&\eqd \sum_{n\in\Z} \rvy(n) e^{-i\omega n}
      \end{align*}
      But arguably, a ``projection of $\rvy$ onto $\rvx$" would better be served by the use of $\Ryx(m)$ rather than $\Rxy(m)$.

\item As demonstrated in \prefpp{sec:case_systemid_nonlinear}, the Papoulis definition (1)
is arguably more convenient for performing least-squares-like optimization.
\end{enumerate}

%%--------------------------------------
%\begin{appendices}
%%--------------------------------------
%%\section{Standard definitions}
%\section{Random Sequences}
%  %============================================================================
% LaTeX File
% Daniel J. Greenhoe
%============================================================================
%======================================
\chapter{Expectation operator}
\label{chp:stats}
%======================================
%=======================================
\section{Definitions}
\index{expectation operator}
%=======================================
In a \structe{probability space} $\ps$, all probability information
is contained in the \fncte{measure} $\psp$ (or equivalently in the pdf or cdf
defined in terms of $\psp$).
Often times this information is overwhelming and a simpler statistic,
which does not offer so much information, is sufficient.
Some of the most common statistics can be conveniently expressed in terms
of the \hie{expectation operator} $\pE$.
%---------------------------------------
\begin{definition}
\label{def:pE}
%---------------------------------------
Let $\ps$ be a \structe{probability space} and
$\rvX$ a \fncte{random variable} on $\ps$ with
\fncte{probability density function} $\ppx$.
\defboxt{ 
  The \opd{expectation operator} $\pEx$ on $\rvX$ is defined as
  \\\indentx$\ds\pEx\rvX \eqd \int_{x\in\F} x \ppx(x) \dx$. 
}
\end{definition}

We already said that a \fncte{random variable} $\rvX$ is neither random nor a variable,
but is rather a function of an underlying process that does appear to be random.
However, because it is a function of a process that does appear random,
the \fncte{random variable} $\rvX$ also appears to be random.
That is, if we don't know the outcome of of the underlying experimental
process, then we also don't know for sure what $\rvX$ is, and so $\rvX$ does
indeed appear to be random.
However, eventhough $\rvX$ appears to be random,
the expected value $\pEx\rvX$  of $\rvX$ is {\bf not random}.
Rather it is a fixed value (like $0$ or $7.9$ or $-2.6$).

On the other hand, eventhough $\pE\rvX$ is {\bf not random},
note that $\pE(\rvX|\rvY)$ {\bf is random}.
This is because $\pE(\rvX|\rvY)$ is a function of $\rvY$.
That is, once we know that $\rvY$ equals some fixed value $y$
(like $0$ or $2.7$ or $-5.1$) then $\pE(\rvX|\rvY=y)$ is also fixed.
However, if we don't know the value of $\rvY$,
then $\rvY$ is still a \fncte{random variable} and the expression $\pE(\rvX|\rvY)$
is also random (a function of \fncte{random variable} $\rvY$).

Two common statistics that are conveniently expressed in terms of the
expectation operator are the \hie{mean} and \hie{variance}.
The mean is an indicator of the ``middle" of a probability distribution and the
variance is an indicator of the ``spread".
%---------------------------------------
\begin{definition}
\label{def:Mx}
\label{def:pvar}
\label{def:pVar}
%---------------------------------------
Let $\rvX$ be a \fncte{random variable} on the \structe{probability space} $\ps$.
\defbox{\begin{array}{FM>{\ds}rc>{\ds}l}
  (1).&The \fnctd{mean} $\pmeanx$ of $\rvX$ is 
      & \pmeanx  &\eqd& \pEx\rvX 
  \\
  (2).&The \fnctd{variance} $\pVar(\rvX)$ or $\pvarx$ of $\rvX$ is 
      & \pVar(\rvX) &\eqd& \pEx\left[(\rvX-\pEx\rvX)^2 \right]
\end{array}}
\end{definition}

%=======================================
\section{Expectation as a linear operator}
%=======================================
The next theorem demonstrates that the operator $\pE$ is a 
\ope{linear operator} \xref{def:linop}---which in turn
makes $\pE$ part of a distinguished club of operators along with fellow member operators
differentiation $\opDif$, integration $\int\dx$,
Laplace $\opLT$, Fourier $\opFT$, z-transform $\opZ$, etc.
Because $\pE$ is a linear operator, it immediately inherits all the properties 
that it's linear operator birthright grants it \xref{cor:pE}. 
%---------------------------------------
\begin{theorem}[\thme{Linearity of $\pE$}]
\label{thm:pE}
\label{thm:pE_linop}
%---------------------------------------
Let $\rvX$ be a \fncte{random variable} on a \structe{probability space} $\ps$.
\thmbox{\begin{array}{>{\ds}rc>{\ds}lCD}
    \pEx(a\rvX+b\rvY+c)&=& \brp{a\pEx\rvX} + \brp{b\pEx\rvY} + c& \forall a,b,c\in\R & (\prope{linear})
\end{array}}
\end{theorem}
\begin{proof}
\begin{align*}
  \pExy(a\rvX+b\rvY + c)
    &\eqd \int_{x\in\R}\int_{y\in\R}\brs{ax+by+c} \pdfxy(x,y)  \dy\dx
    \qquad\text{by definition of $\pE$ \xref{def:pE}}
  \\&= \int_{x\in\R}\int_{y\in\R} ax \pdfxy(x,y)  \dy\dx
     + \int_{x\in\R}\int_{y\in\R} by \pdfxy(x,y)  \dy\dx
     + \int_{x\in\R}\int_{y\in\R} c  \pdfxy(x,y)  \dy\dx
     %&& \text{by \prope{lineararity} of $\int\int\dy\dx$}
     %&& \text{\xref{def:linop}}
  \\&=  \int_{x\in\R} ax \mcom{\int_{y\in\R}\pdfxy(x,y)  \dy}{$\pdfx(x)$}\dx
     +  \int_{y\in\R} by \mcom{\int_{x\in\R}\pdfxy(x,y)  \dx}{$\pdfy(y)$}\dy
     + c\mcom{\int_{y\in\R}\int_{x\in\R}\pdfxy(x,y)  \dx\dy}{$1$}
     %&& \text{by \prope{lineararity} of $\int\int\dy\dx$}
     %&& \text{\xref{def:linop}}
  \\&= a\mcom{\int_{x\in\R} x \pdfx(x) \dx}{$\pE\rvX$}
     + b\mcom{\int_{y\in\R} y \pdfy(y) \dy}{$\pE\rvY$}
     + c
  \\&= \brp{a\pEx\rvX} + \brp{b\pEy\rvY} + c
\end{align*}
\end{proof}

%---------------------------------------
\begin{corollary}
\label{cor:pE_linop}
%---------------------------------------
Let $\opE$ be the an operator from a linear space $\spX$ to a linear space $\spY$, both over a field $\F$.
\corbox{
  \begin{array}{F>{\ds}lc>{\ds}lD}
      (1). & \pE\vzero    &=& \vzero                                                            & and
    \\(2). & \pE(-\rvX)    &=& -(\pE\rvX)                                                       & and
    \\(3). & \pE(\rvX-\rvY) &=& \pE\rvX - \pE\rvY                                               & and
  \end{array}
  \qquad
  \begin{array}{F>{\ds}lc>{\ds}lD}
      (4). & \pE\brp{\sum_{n=1}^\xN \alpha_n\rvX_n}  &=& \sum_{n=1}^\xN \alpha_n\brp{\pE\rvX_n} &
  \end{array}
  }
\end{corollary}
\begin{proof}
These all follow immediately from the fact that $\pE$ is a \ope{linear operator} 
and from \prefpp{thm:L_prop}.
\end{proof}

%---------------------------------------
\begin{remark}
%---------------------------------------
Projecting a stochastic process onto a basis often yields valuable insights 
into the nature of the underlying data. 
Typical projection operators include the Fourier operator $\opFT$, Laplace $\opLT$,
and z-transform $\opZT$ \ldots not to mention wavelet operators.
But note that any such projection on a random sequence simply produces another random sequence.
For example, the Fourier transform $\opFT\rvx(n)$ of a random sequence $\rvx(n)$ is another random 
sequence.

One way to overcome this difficulty is to simply invoke the \ope{sampling} operator $\opS\rvx(n)$ \xref{chp:sampling},
yielding a deterministic sequence, and then take the Fourier transform of the resulting 
deterministic sequence. 
The problem here is that every time you resample the sequence, you will very likely get a 
different Fourier transform.

Arguably a better approach (and the standard one at that) 
is to first invoke the expectation operator $\pE\rvx(n)$, also yielding a deterministic sequence.

The good news here is that because $\pE$ and all the above mentioned operators are \prope{linear}, 
we can do all the standard arithmetic acrobatics associated with linear algebra operators (next corollary).
%such as $\opE\opFT\rvx(n)
\end{remark}

%---------------------------------------
\begin{corollary}
%---------------------------------------
Let $\opM$ and $\opN$ be \ope{linear operator}s \xref{def:linop}.
\corbox{\begin{array}{>{\scy}rlcl@{\qquad}C@{\qquad}D}
    1. & \pE\brp{\opM\opN}      &=& \brp{\pE\opM}\opN                          & \forall \pE\in\clLzw,\, \opM\in\clLyz,\, \opN\in\clLxy  & (\prope{associative})
  \\2. & \pE\brp{\opM\addo\opN} &=& \brp{\pE\opM}\addo\brp{\pE\opN}            & \forall \pE\in\clLyz,\, \opM\in\clLxy,\, \opN\in\clLxy  & (\prope{left distributive})
  \\3. & \brp{\pE\addo\opM}\opN &=& \brp{\pE\opN}\addo\brp{\opM\opN}           & \forall \pE\in\clLyz,\, \opM\in\clLyz,\, \opN\in\clLxy  & (\prope{right distributive})
  \\4. & \alpha\brp{\pE\opM}    &=& \brp{\alpha\pE}\opM = \pE\brp{\alpha\opM}  & \forall \pE\in\clLyz,\, \opM\in\clLxy,\, \alpha\in\F    & (\prope{homogeneous})
\end{array}}
\end{corollary}
\begin{proof}
These all follow immediately from the fact that $\pE$ is a \ope{linear operator} 
and from \prefpp{thm:L_LMN}.
\end{proof}

%---------------------------------------
\begin{corollary}
\label{cor:pVar}
%---------------------------------------
Let $\rvX$ be a \fncte{random variable} on a \structe{probability space} $\ps$.
\corbox{\begin{array}{>{\ds}rc>{\ds}lCD}
    \pVar(a\rvX+b)     &=& a^2\pVar(\rvX)                       & \forall a,b  \in\R &
  \\\pVar(\rvX)        &=& \pEx(\rvX^2) - (\pEx\rvX)^2          &                    &
\end{array}}
\end{corollary}
\begin{proof}
\begin{align*}
  \pVar(\rvX)
    &\eqd \pEx\brs{(\rvX-\pEx\rvX)^2}
    &&    \text{by definition of $\pVar$}
    &&    \text{\xref{def:pvar}}
  \\&=    \pEx\brs{\rvX^2-2\rvX\pEx\rvX + (\pEx\rvX)^2 }
    &&    \text{by \thme{Binomial Theorem}}
    &&    \text{\ifxref{polynom}{thm:binomial}}
  \\&=    \pEx\rvX^2  - \pEx\brs{2\rvX\pEx\rvX}  + \pEx (\pEx\rvX)^2
    &&    \text{by \prope{linearity} of $\pE$}
    &&    \text{\xref{thm:pE_linop}}
  \\&=    \pEx\rvX^2 - 2(\pEx\rvX)[\pEx\rvX] + (\pEx\rvX)^2
  \\&=    \pEx(\rvX^2) - (\pEx\rvX)^2
\\
  \pVar(a\rvX+b)
    &=    \pEx(a\rvX+b)^2  - [\pEx(a\rvX+b)]^2
  \\&=    \pEx(a^2\rvX^2+2ab\rvX+b^2)  - [a(\pEx\rvX)+b]^2
  \\&=    a^2 \pEx\rvX^2  +2ab\pEx\rvX + b^2 - \brs{a^2[\pEx\rvX]^2 + 2ab\pEx\rvX + b^2}
    &&    \text{by \prope{linearity} of $\pE$}
    &&    \text{\xref{thm:pE_linop}}
  \\&=    a^2\brs{ \pEx\rvX^2  - (\pEx\rvX)^2 }
  \\&\eqd a^2 \pVar(\rvX)
    &&    \text{by previous result}
\end{align*}
\end{proof}

\begin{figure}[ht]
\setlength{\unitlength}{0.3mm}%
\begin{center}%
\begin{picture}(200,110)(-50,-10)%
  \thicklines
  \color{axis}%
    \put(  0,  0){\line(1, 0){120}}%
    \put(  0,  0){\line(0, 1){100}}%
    \qbezier[20](0,60)(30,60)(60,60)%
    \qbezier[20](60,0)(60,30)(60,60)%
  \color{blue}%
    \qbezier(33,100)(85,0)(110,100)%
    \put( 60, 60){\circle*{5}}%
  \color{red}%
    \put( 20,100){\line(1,-1){80}}%
  \color{label}%
  \put(125,  0){\makebox(0,0)[l]{$x$}}%
  \put( -5, 60){\makebox(0,0)[r]{$\ff(\pE\rvX)$}}%
  \put( 60, -5){\makebox(0,0)[t]{$\pE\rvX$}}%
  \put(130,60){\vector(-1,0){32}}%
  \put(130,40){\vector(-1,0){47}}%
  \put(135,60){\makebox(0,0)[l]{$\ff(x)$ (convex function)}}%
  \put(135,40){\makebox(0,0)[l]{$mx+c$ (support line)}}%
\end{picture}
\end{center}
\caption{
  Jensen's inequality
  \label{fig:jensen}
  }
\end{figure}

\fncte{Jensen's inequality} is an extremely useful application of \prope{convex}ity \xref{def:convex} to the
\ope{expectation} operator.
Jensen's inequality is stated in \pref{cor:jensen} (next)
and illustrated in \prefpp{fig:jensen}.
%--------------------------------------
\begin{corollary}[\thmd{Jensen's inequality}]
\footnote{
  \citerp{cover}{25},
  \citerpp{jensen1906}{179}{180}
  }
\label{cor:jensen}
%--------------------------------------
Let $\ff$ be a function in $\clFrr$ and $\rvX$ be a \fncte{random variable} on $\ps$.
\corbox{
  \brb{\text{$\ff$ is \prope{convex}}} 
  \quad\implies\quad 
  \brb{\ff(\pE\rvX) \le \pE\ff(\rvX)}
  }
\end{corollary}
\begin{proof}
\begin{enumerate}
  \item Proof 1:
Let $mx+c$ be a ``support line" under $\ff(x)$ \xref{fig:jensen} such that
\[
  \begin{array}{rcll}
    mx+c &<& \ff(x) & \mbox{for } x\ne \pE\rvX \\
    mx+c &=& \ff(x) & \mbox{for } x=\pE\rvX.
  \end{array}
\]
Then
\begin{align*}
  \ff(\pE\rvX)
    &=   m[\pE\rvX] + c
  \\&=   \pE[mX + c]
  \\&\le \pE\ff(\rvX)
\end{align*}

  \item Proof 2 (alternate proof):
    \begin{align*}
      \ff\brp{\pE\rvX}
        &\eqd \ff\brp{\sum_{x\in\pse} x \psp(x)}
      \\&\le \sum_{x\in\pse} \ff(x) \psp(x)
        && \text{by \thme{Jensen's inequality} for convex sets}
        && \text{\xref{thm:jensenineq}}
    \end{align*}
\end{enumerate}
\end{proof}

%---------------------------------------
\begin{theorem}[\thmd{Law of the Unconscious Statistician}]
\footnote{
  \citerpgc{suhov2005}{145}{0521847664}{(2.69)},
  \citerpgc{allen2018}{490}{1447174208}{18.3.4 The Law of the Unconscious Statistician},
  \citerpgc{papoulis1990}{124}{0137116985}{Fundamental Theorem}
  }
%---------------------------------------
\thmbox{
  \pE\brs{\fg(\rvX)} = \int_{x\in\R} \fg(x) \pdfx(x) \dx
  }
\end{theorem}


%=======================================
\section{Expectation inequalities}
%=======================================
%---------------------------------------
\begin{theorem}[\thmd{Markov's inequality}]
\footnote{
  \citerp{ross}{395}
  }
\label{thm:markovineq}
%---------------------------------------
Let $\rvX:\Omega\to[0,\infty)$ be a non-negative valued \fncte{random variable} and
$a\in(0,\infty)$. Then
\thmbox{ \psp\setn{\rvX\ge a} \le \frac{1}{a} \pE\rvX }
\end{theorem}
\begin{proof}
\begin{align*}
  I &\eqd \left\{ \begin{array}{l@{\hspace{4ex}\mbox{for}\hspace{4ex}}l}
    1 &\rvX\ge a \\
    0 &\rvX < a
    \end{array}\right.
\\
  aI &\le\rvX           \\
   I &\le \frac{1}{a}\rvX \\
   \pE I &\le \pE\left(\frac{1}{a}\rvX\right) \\
\\
   \psp\setn{\rvX\ge a}
     &= 1\cdot\psp\setn{\rvX\ge a} + 0\cdot\psp\setn{\rvX<a}
   \\&= \pE I
   \\&\le \pE\left(\frac{1}{a}\rvX \right)
   \\&=   \frac{1}{a}\pE\rvX
\end{align*}
\end{proof}


%---------------------------------------
\begin{theorem}[Chebyshev's inequality]
\index{Chebyshev's inequality}
\index{theorems!Chebyshev's inequality}
\footnote{
  \citerp{ross}{396}
  }
%---------------------------------------
Let $\rvX$ be a \fncte{random variable} with mean $\mu$ and variance $\sigma^2$.
\thmbox{ \psp\setn{\abs{\rvX-\mu}\ge a} \le \frac{\sigma^2}{a^2}}
\end{theorem}
\begin{proof}
\begin{align*}
  \psp\setn{\abs{\rvX-\mu} \ge a}
    &=   \psp\setn{ (\rvX-\mu)^2 \ge a^2}
  \\&\le \frac{1}{a^2} \pE(\rvX-\mu)^2 
    && \text{by \thme{Markov's inequality}}
    && \text{\xref{thm:markovineq}}
  \\&=   \frac{\sigma^2}{a^2}
\end{align*}
\end{proof}

%=======================================
\section{Joint and conditional probability spaces}
%=======================================
Sometimes the problem of finding the expected value of a \fncte{random variable} $\rvX$
can be simplified by ``conditioning $\rvX$ on $\rvY$".
%---------------------------------------
\begin{theorem}
%---------------------------------------
Let $\rvX$ and $\rvY$ be \fncte{random variable}s. Then
\thmbox{\pEx{\rvX} = \pEy\pEx[x|y](\rvX|\rvY) }
\end{theorem}
\begin{proof}
\begin{align*}
   \pEy\pEx[x|y](\rvX|\rvY)
     &\eqd \pEy \left[ \int_{x\in\R}x \pp(\rvX=x|\rvY) \dx \right]
   \\&\eqd \int_{y\in\R} \left[\int_{x\in\R}x \pp(x|\rvY=y) \dx \right] \pp(y) \dy
   \\&=    \int_{y\in\R} \int_{x\in\R}x \pp(x|y)\pp(y) \dx   \dy
   \\&=    \int_{x\in\R}x \int_{y\in\R} \pp(x,y) \dy   \dx
   \\&=    \int_{x\in\R}x \pp(x) \dx
   \\&\eqd \pEx\rvX
\end{align*}
\end{proof}

%=======================================
\section{Expectation inner product space}
%=======================================
When possible, we like to generalize any given mathematical structure
to a more general mathematical structure and then take advantage of
the properties of that more general structure.
Such a generalization can be done with \fncte{random variable}s.
Random variables can be viewed as vectors in a vector space.
Furthermore, the expectation of the product of two \fncte{random variable}s
(e.g. $\pE(\rvX\rvY)$)
can be viewed as an \fncte{inner product} in an \structe{inner product space}.
Since we have an \fncte{inner product} space,
we can then immediately use all the properties of
\structe{inner product space}s, \fncte{norm}ed spaces, vector spaces, metric spaces,
and topological spaces.

%---------------------------------------
\begin{theorem}
\footnote{
  \citerppgc{lindquist2015}{25}{26}{3662457504}{2.1 Hilbert Space of Second-Order Random Variables. $\inprod{\xi}{\eta}=\pE\setn{\xi\bar{\eta}}$},
  \citerpgc{caines1988}{21}{0471081019}{$Exy=\int_\Omega x(\omega)y(\omega)dP(\omega)$},
  \citerpgc{caines2018}{21}{1611974712}{$Exy=\int_\Omega x(\omega)y(\omega)dP(\omega)$},
  \citerpp{moon2000}{105}{106}
  }
\label{thm:prb_vspace}
%---------------------------------------
Let $R$ be a ring,
$\ps$ be a \structe{probability space}, $\pE$ the expectation operator, and
$\spV=\set{\rvX}{\rvX:\pso\to R}$ be the set of all random vectors
in \structe{probability space} $\ps$.
\thmbox{\begin{array}{FlM}
  (1). & \spV\eqd\set{\rvX}{\rvX:\pso\to R}        & is a \structe{vector space}. \\
  (2). & \inprod{\rvX}{\rvY}\eqd\pE(\rvX\rvY^\ast) & is an \fncte{inner product}. \\
  (3). & \norm{\rvX}\eqd\sqrt{\pE(\rvX\rvX^\ast)}  & is a \fncte{norm}. \\
  (4). & \opair{\spV}{\inprodn}                    & is an \structe{inner product space}.
\end{array}}
\end{theorem}
\begin{proof}
\begin{enumerate}
  \item Proof that $\spV$ is a vector space:
    \[\begin{array}{lll@{\hs{1cm}}D}
   1) & \forall \rvX,\rvY, \rvZ\in\spV
      & (\rvX+\rvY)+ \rvZ = \rvX+(\rvY+ \rvZ)
      & \text{($+$ is associative)}
      \\
   2) & \forall \rvX,\rvY\in\spV
      & \rvX+\rvY = \rvY+\rvX
      & \text{($+$ is commutative)}
      \\
   3) & \exists  0 \in\spV \st \forall \rvX\in\spV
      & \rvX+ 0 = \rvX
      & \text{($+$ identity)}
      \\
   4) & \forall \rvX \in\spV \exists \rvY\in\spV \st
      & \rvX+\rvY =  0
      & \text{($+$ inverse)}
      \\
   5) & \forall \alpha\in S \text{ and } \rvX,\rvY\in\spV
      & \alpha\cdot(\rvX+\rvY) = (\alpha \cdot\rvX)+(\alpha\cdot\rvY)
      & \text{($\cdot$ distributes over $+$)}
      \\
   6) & \forall \alpha,\beta\in S \text{ and } \rvX\in\spV
      & (\alpha+\beta)\cdot\rvX = (\alpha\cdot \rvX)+(\beta\cdot \rvX)
      & \text{($\cdot$ pseudo-distributes over $+$)}
      \\
   7) & \forall \alpha,\beta\in S \text{ and } \rvX\in\spV
      & \alpha(\beta\cdot\rvX) = (\alpha\cdot\beta)\cdot\rvX
      & \text{($\cdot$ associates with $\cdot$)}
      \\
   8) & \forall \rvX\in\spV
      & 1\cdot \rvX = \rvX
      & \text{($\cdot$ identity)}
\end{array}\]

  \item Proof that $\inprod{\rvX}{\rvY}\eqd\pE(\rvX\rvY^\ast)$ is an \fncte{inner product}.
  \[\begin{array}{llllD}
   1) &  \pE(\rvX\rvX^\ast) &\ge 0
      &  \forall \rvX\in\spV
      &  \text{(non-negative)}
      \\
   2) &  \pE(\rvX\rvX^\ast) &= 0 \iff \rvX=0
      &  \forall \rvX\in\spV
      &  \text{(non-degenerate)}
      \\
   3) &  \pE(\alpha\rvX\rvY^\ast)    &= \alpha\pE(\rvX\rvY^\ast)
      &  \forall \rvX,\rvY\in\spV,\;\forall\alpha\in\C
      &  \text{(homogeneous)}
      \\
   4) &  \pE[(\rvX+\rvY)\rvZ^\ast] &= \pE(\rvX\rvZ^\ast) + \pE(Y\rvZ^\ast)
      &  \forall \rvX,\rvY, \rvZ\in\spV
      &  \text{(additive)}
      \\
   5) &  \pE(\rvX\rvY^\ast) &= \pE(\rvY\rvX^\ast)
      &  \forall \rvX,\rvY\in\spV
      &  \text{(conjugate symmetric)}.
  \end{array}\]

  \item Proof that $\norm{\rvX}\eqd\sqrt{\pE(\rvX\rvX^\ast)}$ is a \fncte{norm}:
    This \fncte{norm} is simply induced by the above \fncte{inner product}.
  \item Proof that $\opair{\spV}{\inprodn}$ is an \structe{inner product space}:
    Because $\spV$ is a vector space and $\inprodn$ is
    an \fncte{inner product}, $\opair{\spV}{\inprodn}$ is an \structe{inner product space}.
\end{enumerate}
\end{proof}

The next theorem gives some results that follow directly from vector space
properties:
%---------------------------------------
\begin{theorem}
%---------------------------------------
Let $\ps$ be a \structe{probability space} with expectation functional $\pE$.
\thmbox{\begin{array}{F >{\ds}r c >{\ds}l >{\scs}m{41mm}}
  1. & \sqrt{\pE\left(\sum_{n=1}^{\xN}\rvX_n\right)}
     &\le& \sum_{n=1}^{\xN} \pE(\rvX_nX_n^\ast)
     & (\thme{Generalized triangle inequality})
     \\
  2. & \abs{\pE(\rvX\rvY^\ast)}^2
     &\le& \pE(\rvX\rvX^\ast)\:\pE(YY^\ast)
     & (\thme{Cauchy-Schwartz inequality})
     \\
  3. & 2\pE(\rvX\rvX^\ast) + 2\pE(YY^\ast)
     &=& \pE[(\rvX+\rvY)(\rvX+\rvY)^\ast] + \pE[(\rvX-Y)(\rvX-Y)^\ast]
     &   (\thme{Parallelogram Law})
  \end{array}}
\end{theorem}
\begin{proof}
\begin{enumerate}
  \item $\opair{\clFor}{\pE(\rvx,\rvy)}$ is an \structe{inner product space}. Proof: \prefpp{thm:prb_vspace}.

  \item Because it is an \structe{inner product space}, the other properties follow:
        \\\indentx\begin{tabular}{llll}
          1. & Generalized triangle inequality:
             & \pref{thm:norm_tri}
             & \prefpo{thm:norm_tri}
             \\
          2. & Cauchy-Schwartz inequality:
             & \pref{thm:cs}
             & \prefpo{thm:cs}
             \\
          3. & Parallelogram Law:
             & \pref{thm:parallelogram}
             & \prefpo{thm:parallelogram}
        \end{tabular}
\end{enumerate}
\end{proof}


%  %============================================================================
% LaTeX File
% Daniel J. Greenhoe
%============================================================================
%======================================
%\section{Random Sequences}
%======================================
%---------------------------------------
\begin{definition}
\footnote{
  \citerpgc{papoulis1984}{220}{0070484686}{``Wide sense"},
  \citerppgc{bendat2010}{109}{111}{1118210824}{``{\scshape Chapter 5} Stationary Random Processes"\ldots``\prope{weakly stationary}"},
  \citerpgc{bendat1980}{3}{0471058874}{``1.1.1 Stationary Data"}
  }
\label{def:wss}
%---------------------------------------
Let $\seqnZ{\rvx(n)}$ be a \fncte{random sequence}.
\defboxt{
  $\seqn{\rvx(n)}$ is \propd{wide sense stationary} (\propd{WSS}) if
  \\\indentx
    $\begin{array}{FrcllD}
       (1). & \pE\rvx(n)                &=& \pE\rvx(k)                & \forall n,k  \in\Z & and
     \\(2). & \pE\brs{\rvx(n+m)\rvx(n)} &=& \pE\brs{\rvx(k+m)\rvx(k)} & \forall n,k,m\in\Z
    \end{array}$
  }
\end{definition}

%---------------------------------------
\begin{definition}
\footnote{
  \citerpgc{papoulis1984}{221}{0070484686}{``jointly WSS"}
  }
\label{def:jwss}
%---------------------------------------
Let $\seqnZ{\rvx(n)}$ and $\seqnZ{\rvy(n)}$ be \fncte{random sequence}s.
\defboxt{
  $\seqn{\rvx(n)}$ and $\seqn{\rvy(n)}$ are \propd{jointly wide sense stationary} (\propd{J-WSS}) if
  \\\indentx
    $\begin{array}{FMDD}
       (1). & $\seqn{\rvx(n)}$ is \prope{wide sense stationary}       & \pref{def:wss}       & and
     \\(2). & $\seqn{\rvy(n)}$ is \prope{wide sense stationary}       & \pref{def:wss}       & and
     \\(3). & $\pE\brs{\rvx(n+m)\rvy(n)} = \pE\brs{\rvx(k+m)\rvy(k)}$ & $\forall n,k,m\in\Z$ &
    \end{array}$
  }
\end{definition}


%  %============================================================================
% LaTeX File
% Daniel J. Greenhoe
%============================================================================

%======================================
\section{Operations on Sequences}
%\label{app:dsp}
%\label{app:ztrans}
%======================================
%======================================
%\subsection{Operations}
%======================================
%======================================
\subsection{Convolution operation}
%======================================
%--------------------------------------
\begin{definition}
\label{def:sequence}
\footnote{
  \citerp{bromwich1908}{1},
  \citerpgc{thomson2008}{23}{143484367X}{Definition 2.1},
  \citerpg{joshi1997}{31}{8122408265}
  }
\label{def:tuple}
\label{def:seq}
%--------------------------------------
Let $\clFyx$ be the set of all functions from a set $\setY$ to a set $\setX$.
Let $\Z$ be the set of integers.
\defbox{\begin{array}{l}
  \text{A function $\ff$ in $\clFyx$ is a \structd{sequence} over $\setX$ if $\setY=\Z$.}\\
  \text{A sequence may be denoted in the form $\ds\seqxZ{x_n}$ or simply as $\ds\seqn{x_n}$.}
  %\text{A function $\ff$ in $\clFyx$ is an \hid{n-tuple} over $\setX$ if $\setY=\setn{1,2,\ldots,\xN}$.}\\
  %\text{An n-tuple may be denoted in the form $\ds\tuplexn{x_n}$ or simply as $\ds\tuplen{x_n}$.}
\end{array}}
\end{definition}

%--------------------------------------
\begin{definition}
\footnote{
  \citerpgc{kubrusly2011}{347}{0817649972}{Example 5.K}
  }
\label{def:spllR}
\label{def:spllC}
\label{def:spllF}
%--------------------------------------
%Let $\fieldF$ be a \structe{field}. % \xref{def:field}.
Let $\fieldC$ be the \structe{field of complex numbers}.
\defboxt{
  The \structd{space of all absolutely square summable sequences} $\spllC$ over $\C$ is defined as
  \\\indentx$\ds\spllC\eqd\set{\seqxZ{x_n}}{\sum_{n\in\Z}\abs{x_n}^2 < \infty}$
  }
\end{definition}

%The space $\spllC$ is an example of a \structe{separable Hilbert space}.
%In fact, $\spllC$ is the \emph{only} separable Hilbert space in the sense that all separable Hilbert spaces
%are isomorphically equivalent.
%For example, $\spllC$ is isomorphic to $\spLLR$, the \structe{space of all absolutely square Lebesgue integrable functions}.
%%That is, their topological structure is the same.
%%Differences occur in the nature of operators on the spaces.

%--------------------------------------
\begin{definition}
\label{def:dsp_conv}
\label{def:convd}
\index{convolution}
%--------------------------------------
%Let $\seq{x_n}{n\in\Z}$ and $\seq{y_n}{n\in\Z}$ be sequences \xref{def:seq} in the space $\spllC$ \xref{def:spllR}.
\defbox{\begin{array}{M}
  The \opd{convolution} operation $\hxs{\convd}$ is defined as
  \\\indentx
  $\ds{\seqn{x_n}\convd\seqn{y_n}} \eqd \seq{\sum_{m\in\Z} x_{m} y_{n-m}}{n\in\Z}\qquad\scy\forall\seqxZ{x_n},\seqxZ{y_n}\in\spllC$
\end{array}}
\end{definition}

%======================================
\subsection{Z-transform}
%======================================
%--------------------------------------
\begin{definition}
\footnote{
  \structe{Laurent series}: \citerpg{aa}{49}{0821821466}
  }
\label{def:opZ}
%--------------------------------------
Let $\seqnZ{\fx(n)}$ be a sequence.
\defboxt{
%Let $\seq{x_n}{n\in\Z}$ be a sequence in the space $\spllC$. %over a ring $\ring$.
  The \opd{z-transform} $\opZ$ of $\seqn{\fx(n)}$ is defined as
  \\\indentx
  $\ds\brs{\hxs{\opZ}\seqn{\fx(n)}}(z) \eqd {\sum_{n\in\Z} \fx(n) z^{-n}}\qquad\scy\forall\seqn{\fx(n)}\in\spllC$
  }
\end{definition}

%--------------------------------------
\begin{proposition}
\label{prop:opZ}
%--------------------------------------
Let $X(z)\eqd\opZ\fx(n)$ be the \ope{z-transform} of $\fx(n)$.
\thmbox{
  \mcom{\brb{\begin{array}{rcl}
    \Zx(z) &\eqd& \opZ\seqn{\fx(n)}
  \end{array}}}{\xref{def:opZ}}
  \quad\implies\quad
  \brb{\begin{array}{F>{\ds}lc>{\ds}lCD}
      (1).&\opZ\seqn{\alpha\fx(n)} &=& \alpha\Zx(z)                    & \forall\seqn{x_n}\in\spllC & and
    \\(2).&\opZ\seqn{\fx[n-k]}     &=& z^{-k}\Zx(z)                    & \forall\seqn{x_n}\in\spllC & and
    \\(3).&\opZ\seqn{\fx(-n)}      &=& \Zx\brp{\frac{1}{z}}            & \forall\seqn{x_n}\in\spllC & and
    \\(4).&\opZ\seqn{\fx^\ast(n)}  &=& \Zx^\ast\brp{z^\ast}            & \forall\seqn{x_n}\in\spllC & and
    \\(5).&\opZ\seqn{\fx^\ast(-n)} &=& \Zx^\ast\brp{\frac{1}{z^\ast}}  & \forall\seqn{x_n}\in\spllC &
  \end{array}}
  }
\end{proposition}
\begin{proof}
\begin{align*}
  \alpha\Z\Zx(z)
    &\eqd \alpha \opZ \seqn{\fx(n)}                && \text{by definition of $\Zx(z)$}
  \\&\eqd \alpha \sum_{n\in\Z} \fx(n) z^{-n}       && \text{by definition of $\opZ$ operator}
  \\&\eqd \sum_{n\in\Z} \brp{\alpha\fx(n)} z^{-n}  && \text{by \prope{distributive} property}
  \\&\eqd \opZ\seqn{\alpha\fx(n)}                  && \text{by definition of $\opZ$ operator}
  \\
  z^{-k}\Zx(z)
    &= z^{-k} \opZ\seqn{\fx(n)}
    && \text{by definition of $\Zx(z)$}
    && \text{(left hypothesis)}
  \\&\eqd z^{-k}\sum_{n=-\infty}^{n=+\infty} \fx(n) z^{-n}
    && \text{by definition of $\opZ$}
    && \text{\xref{def:opZ}}
  \\&=          \sum_{n=-\infty}^{n=+\infty} \fx(n) z^{-n-k}
  \\&=          \sum_{m-k=-\infty}^{m-k=+\infty} \fx[m-k] z^{-m}
    && \text{where $m\eqd n+k$}
    && \text{$\implies$ $n=m-k$}
  \\&=          \sum_{m=-\infty}^{m=+\infty} \fx[m-k] z^{-m}
  \\&=          \sum_{n=-\infty}^{n=+\infty} \fx[n-k] z^{-n}
    && \text{where $n\eqd m$}
  \\&\eqd \opZ\seqn{\fx[n-k]}
    && \text{by definition of $\opZ$}
    && \text{\xref{def:opZ}}
  \\
  \opZ\seqn{\fx^\ast(n)}
    &\eqd \sum_{n\in\Z}\fx^\ast(n) z^{-n}
    && \text{by definition of $\opZ$}
    && \text{\xref{def:opZ}}
  \\&\eqd \brp{\sum_{n\in\Z}\fx(n) (z^\ast)^{-n}}^\ast
    && \text{by definition of $\opZ$}
    && \text{\xref{def:opZ}}
  \\&\eqd \Zx^\ast(z^\ast)
    && \text{by definition of $\opZ$}
    && \text{\xref{def:opZ}}
  \\
  \opZ\seqn{\fx(-n)}
    &\eqd \sum_{n\in\Z}\fx(-n) z^{-n}
    && \text{by definition of $\opZ$}
    && \text{\xref{def:opZ}}
  \\&= \sum_{-m\in\Z}\fx[m] z^{m}
    && \text{where $m\eqd -n$}
    && \text{$\implies$ $n=-m$}
  \\&= \sum_{m\in\Z}\fx[m] z^{m}
    && \text{because $\seqn{\fx(n)},\seqn{z^n}\in\spllC$}     && \text{\xref{def:spllC}}
  \\&= \sum_{m\in\Z}\fx[m] \brp{\frac{1}{z}}^{-m}
  \\&\eqd \Zx\brp{\frac{1}{z}}
    && \text{by definition of $\opZ$}
    && \text{\xref{def:opZ}}
  \\
  \opZ\seqn{\fx^\ast(-n)}
    &\eqd \sum_{n\in\Z}\fx^\ast(-n) z^{-n}
    && \text{by definition of $\opZ$}
    && \text{\xref{def:opZ}}
  \\&= \sum_{-m\in\Z}\fx^\ast[m] z^{m}
    && \text{where $m\eqd -n$}
    && \text{$\implies$ $n=-m$}
  \\&= \sum_{m\in\Z}\fx^\ast[m] z^{m}
    && \text{because $\seqn{\fx(n)},\seqn{z^n}\in\spllC$}     && \text{\xref{def:spllC}}
  \\&= \sum_{m\in\Z}\fx^\ast[m] \brp{\frac{1}{z}}^{-m}
  \\&= \brp{\sum_{m\in\Z}\fx[m] \brp{\frac{1}{z^\ast}}^{-m}}^\ast
  \\&\eqd \Zx^\ast\brp{\frac{1}{z^\ast}}
    && \text{by definition of $\opZ$}
    && \text{\xref{def:opZ}}
\end{align*}
\end{proof}

%--------------------------------------
\begin{proposition}[\thmd{Convolution Theorem}]
\label{prop:conv}
%--------------------------------------
Let $\convd$ be the convolution operator \xref{def:dsp_conv}.
%$\seq{x_n}{n\in\Z}$ and $\seq{y_n}{n\in\Z}$ be sequences in the space $\spllC$. %be sequences over a ring $\ring$.
\thmbox{
  \opZ\mcom{\brp{\seqn{x_n}\convd\seqn{y_n}}}{sequence convolution} = \mcom{\brp{\opZ\seqn{x_n}}\;\brp{\opZ\seqn{y_n}}}{series multiplication}
  \qquad{\scy\forall\seqxZ{x_n},\seqxZ{y_n}\in\spllC}
  }
\end{proposition}
\begin{proof}
\begin{align*}
  [\opZ(x\convd y)](z)
    &\eqd \opZ {\left(\sum_{m\in\Z} x_m y_{n-m}\right)}
    &&    \text{by \prefp{def:dsp_conv}}
  \\&\eqd \sum_{n\in\Z} \sum_{m\in\Z} x_m y_{n-m} z^{-n}
    &&    \text{by \prefp{def:opZ}}
  \\&=    \sum_{n\in\Z} \sum_{m\in\Z} x_m y_{n-m} z^{-n}
  \\&=    \sum_{m\in\Z} \sum_{n\in\Z} x_m y_{n-m} z^{-n}
  \\&=    \sum_{m\in\Z} \sum_{k\in\Z} x_m y_k z^{-(m+k)}
    &&    \text{where $k=n-m \iff n=m+k$}
  \\&=    {\left[\sum_{m\in\Z} x_m z^{-m}\right]}
          {\left[\sum_{k\in\Z} y_k z^{-k}\right]}
  \\&\eqd \brp{\opZ\seqn{x_n}}\;\brp{\opZ\seqn{y_n}}
    &&    \text{by \prefp{def:opZ}}
\end{align*}
\end{proof}

%---------------------------------------
\begin{lemma}
\label{lem:real_xyh}
%---------------------------------------
Let $\opH$ be a \prope{linear time-invariant} operator with \fncte{impulse response} $\seqn{\fh(n)}$.
Let $\seqn{\fy(n)}\eqd\seqn{\opH\fx(n)}$.
\lembox{
  \brb{\begin{array}{FMD}
      (A). & $\seqn{\fx(n)}$ and $\seqn{\fy(n)}$ are \prope{real-valued}  & and
    \\(B). & $\seqn{\fx(n)}$ and $\seqn{\fh(n)}$  are in $\spllC$         & and
    \\(C). & $\seqn{\fx(n)}\neq\seqn{\cdots,0,0,0,\cdots}$                & and
    \\(D). & $\seqn{\fh(n)}$ is \prope{linear time-invariant}
  \end{array}}
  \implies
  \brb{\begin{array}{FMD}
      (1). & $\seqn{\fh(n)}$ is \prope{real-valued} & and
    \\(2). & $\ZH(z) = \ZH^*\brp{z^\ast}$
  \end{array}}
  }
\end{lemma}
\begin{proof}
\begin{enumerate}
  \item Let $\fh_R(n)$ and $\fh_I(n)$ be the \structe{real-part} and \structe{imaginary-part}, respectively,
        of $\fh(n)$. \label{item:real_xyh_def}
  \item lemma: $\sum_{m\in\Z} \fh_I(m)\fx(n-m) = 0$ \label{ilem:real_xyh_lem}
    \begin{align*}
      &\sum_{m\in\Z} \fh_R(m)\fx(n-m) + i\sum_{m\in\Z} \fh_I(m)\fx(n-m)
      \\&= \sum_{m\in\Z} \fh(m)\fx(n-m)
        && \text{by definitions of $\fh_R$ and $\fh_I$}                && \text{\pref{item:real_xyh_def}}
      \\&= \fy(n)
        && \text{because $\opH$ is \prope{LTI}}                        && \text{hypothesis (D)}
      \\&= \fy^\ast(n)
        && \text{because $\fy$ is \prope{real-valued}}                 && \text{hypothesis (A)}
      \\&= \brp{\sum_{m\in\Z} \fh(m)\fx(n-m)}^\ast
        && \text{because $\opH$ is \prope{LTI}}                        && \text{hypothesis (D)}
      \\&= \sum_{m\in\Z} \fh^\ast(m)\fx^\ast(n-m)
        && \text{by \prope{antiautomorphic} property}                  && \text{\xref{def:staralg}}
       %&& \text{by \prope{antiautomorphic} property of *-algebras}    && \text{\xref{def:staralg}}
      \\&= \sum_{m\in\Z} \fh^\ast(m)\fx(n-m)
        && \text{because $\fy$ is \prope{real-valued}}                 && \text{hypothesis (A)}
      \\&= \sum_{m\in\Z} \fh_R(m)\fx(n-m) - i\sum_{m\in\Z} \fh_I(m)\fx(n-m)
        && \text{by definitions of $\fh_R$ and $\fh_I$}                && \text{\pref{item:real_xyh_def}}
      \\
      \implies&
      \boxed{\sum_{m\in\Z} \fh_I(m)\fx(n-m) = 0}
    \end{align*}

  \item Notes:
    \begin{enumerate}
      \item Without hypothesis (C), it is trivial to satisfy \pref{ilem:real_xyh_lem}.

      \item Without hypothesis (B), it is simple to satisfy \pref{ilem:real_xyh_lem} with
        \\$\fh(n)=\seqn{\cdots,0,0,0,i,-i,0,0,0,\cdots}$ and $\fx(n)=\seqn{\cdots,1,1,1,\cdots}$

      \item Without hypothesis (D), it is trivial to satisfy \pref{ilem:real_xyh_lem} with
        $\ds\opH\fx(n)\eqd\Real\brp{\sum_{m\in\Z}\fh(m)\fx(n-m)}$
    \end{enumerate}

  \item Proof that $\fh(n)$ is \prope{real-valued}:\label{item:real_xyh_realh}
    \begin{align*}
      \text{\pref{ilem:real_xyh_lem}}
        &\implies \ZH_I(z)\ZX(z) = 0
        && \text{by \thme{Convolution Theorem}}                  && \text{\xref{prop:conv}}
      \\&\implies \ZH_I(z) = 0
        && \text{because $\fx(n)\neq\seqn{\cdots,0,0,0,\cdots}$} && \text{hypothesis (C)}
      \\&\implies \fh_I(n) = \seqn{\cdots,0,0,0,\cdots}
      \\&\implies \text{$\fh(n)\eqd\fh_R(n)+i\fh_I(n)$ is \prope{real-valued}}
    \end{align*}

  \item Proof that $\ZH(z) = \ZH^*\brp{z^\ast}$:
    \begin{align*}
      \ZH(z)
        &\eqd \opZ\seqn{\fx(n)}
        && \text{by definition of $\ZH(z)$}
      \\&= \opZ\seqn{\fx^\ast(n)}
        && \text{because $\fx(n)$ is \prope{real-valued}} && \text{\xref{item:real_xyh_realh}}
      \\&= \ZH^*\brp{z^\ast}
        && \text{by \pref{prop:opZ}}
    \end{align*}
\end{enumerate}
\end{proof}

%---------------------------------------
\begin{lemma}
\label{lem:real_FH}
%---------------------------------------
Let $\FH(\omega)$ be the DTFT \xref{def:dtft} of a sequence $\fh(n)$.
\lembox{
  \brb{\begin{array}{M}
      $\fh(n)$ is \prope{real-valued}
  \end{array}}
  \implies
  \brb{\begin{array}{rclD}
      \FH(-\omega) &=& \FH^\ast(\omega) & (\prope{conjugate symmetric})
  \end{array}}
  }
\end{lemma}
\begin{proof}
\begin{align*}
  \FH(-\omega)
    &\eqd \sum_{n\in\Z} \fh(n) e^{-i(-\omega)n}
    && \text{by definition of $\FH(\omega)$}  &&\text{\xref{def:dtft}}
  \\&= \sum_{n\in\Z} \fh(n) e^{i\omega n}
  \\&= \brs{\sum_{n\in\Z} \fh^\ast(n) e^{i\omega n}}^\ast
    && \text{by \prope{antiautomorphic} property of *-algebras}    && \text{\xref{def:staralg}}
  \\&= \brs{\sum_{n\in\Z} \fh(n) e^{-i\omega n}}^\ast
    && \text{by \prope{real-valued} hypothesis}
  \\&\eqd \FH^\ast(\omega)
    && \text{by definition of $\FH(\omega)$}  &&\text{\xref{def:dtft}}
\end{align*}
\end{proof}

%  %============================================================================
% Daniel J. Greenhoe
% XeLaTeX file
%============================================================================

%=======================================
\section{Normed Algebras}
%=======================================
%=======================================
%\section{Algebras}
%=======================================

%=======================================
%\section{Star-Algebras}
%=======================================
%---------------------------------------
\begin{definition}
%\footnote{
  %\citerpg{folland1995}{1}{0849384907},
  \citerp{rickart1960}{178},
  \citeIpg{gelfand1964}{241}{0821820222}
%  }
\label{def:star_algebra}
\label{def:staralg}
\index{star-algebra}
\index{algebras!$\invo$-algebra} 
\indxs{\invo}
%---------------------------------------
Let $\algA$ be an \structe{algebra}.
\defbox{\begin{array}{FrclCDD}
  \mc{7}{M}{The pair $\opair{\algA}{\invo}$ is a \structd{$\invo$-algebra}, or ``\structd{star-algebra}", if}
    \\1.& \brp{x+y}^\invo       &=& x^\invo + y^\invo     & \forall x,y\in\algA                 & (\prope{distributive})     & and 
    \\2.& (\alpha x)^\invo      &=& \bar{\alpha}x^\invo   & \forall x\in\algA,\, \alpha\in\C    & (\prope{conjugate linear}) & and 
    \\3.& (xy)^\invo            &=& y^\invo x^\invo       & \forall x,y \in \algA               & (\prope{antiautomorphic})  & and 
    \\4.& x^{\invo\invo}        &=& x                     & \forall x \in \algA                 & (\prope{involutory}) 
  \\\mc{7}{M}{The operator $\invo$ is called an \opd{involution} on the algebra $\algA$.}
\end{array}}
\end{definition}

%---------------------------------------
\begin{definition}[Hermitian components]
\label{def:nalg_Re}
\label{def:nalg_Im}
\label{def:Re}
\label{def:Im}
%\footnote{
  \citerpg{michel1993}{430}{048667598X},
  \citerp{rickart1960}{179},
  \citeIpg{gelfand1964}{242}{0821820222}
%  }
%---------------------------------------
Let $\opair{\setX}{\normn}$ be a \structe{$\invo$-algebra} \xref{def:staralg}.
\defbox{
  \begin{array}{MMLcL}
    For $x\in\setX$, the \opd{real part}      &of $x$ is defined as & \hxs{\Real} x &\eqd& \frac{1}{2  }\Big( x+ x^\invo \Big)  \\
    For $x\in\setX$, the \opd{imaginary part} &of $x$ is defined as & \hxs{\Imag} x &\eqd& \frac{1}{2i }\Big( x- x^\invo \Big)
  \end{array}
  }
\end{definition}

%---------------------------------------
\begin{example}
\citetbl{
  \citerppg{landau1966}{106}{107}{082182693X}
  }
\label{ex:staralg_C}
%---------------------------------------
Let $\C$ be the \structe{set of complex numbers} and $\invo:\C\to\C$ the \ope{conjugate operator}.
The pair $\opair{\C}{\invo}$ is an \structe{$\invo$-algebra}. 
\end{example}




%\end{appendices}
%%--------------------------------------
%%============================================================================
% Daniel J. Greenhoe
% XeLaTeX file
%============================================================================
%---------------------------------------
% Bibliography
%---------------------------------------
\addcontentsline{toc}{section}{References}
%\bibliography{../common/bib/found,../common/bib/order,../common/bib/analysis,../common/bib/algebra,../common/bib/biology,../common/bib/mathmisc,../common/bib/mathhist,../common/bib/wavelets,../common/bib/approx,../common/bib/digcom,../common/bib/english}
\bibliography{../common/bib/analysis,../common/bib/language,../common/bib/algebra,../common/bib/biology,../common/bib/found,../common/bib/order,../common/bib/mathmisc,../common/bib/mathhist,../common/bib/probstat,../common/bib/wavelets,../common/bib/approx,../common/bib/digcom,../common/bib/english}
%%---------------------------------------
%\section*{Reference Index}
%%---------------------------------------
%\markboth{Reference Index}{Reference Index}
%\addcontentsline{toc}{section}{Reference Index}
%\begin{multicols}{3}
%\input{xref.ind}
%\end{multicols}
%%---------------------------------------
%\section*{Subject Index}
%%---------------------------------------
%\markboth{Subject Index}{Subject Index}
%\addcontentsline{toc}{section}{Subject Index}
%\begin{multicols}{3}
%  \input{xsym.ind}
%\end{multicols}
%%---------------------------------------
%\subsection*{License}
%%---------------------------------------
%\markboth{License}{License}
%\addcontentsline{toc}{section}{License}
%This document is provided under the terms of 
%the \href{https://creativecommons.org/}{Creative Commons} license \href{https://creativecommons.org/licenses/by/4.0/legalcode}{CC-BY 4.0}.
%\\For an exact statement of the license, see 
%\\\indentx\url{https://creativecommons.org/licenses/by/4.0/legalcode}
%\\
%The icon 
%\tbox{\href{https://creativecommons.org/licenses/by/4.0/legalcode}{\ccby}}
%appearing throughout this document is based on one that was once at 
%\\\indentx\url{https://creativecommons.org/}\\
%where it was stated, 
%``Except where otherwise noted, content on this site is licensed under a Creative Commons Attribution 4.0 International license."


%\section{First section}
%Texts..
%\subsection{Subsection}\label{sec:nothing}
%More text.
%\subsubsection{Subsubsection}\label{sec:nothing2}
%More ...

% Bibliography
%-----------------------------------------------------------------
\begin{thebibliography}{99}

\bibitem{1} Author, \emph{Title}, Journal/Editor, (year).

\bibitem{bendat1980} Julius S. Bendat and Allan G. Piersol, 
                     \emph{Engineering Applications of Correlation and Spectral Analysis}, 
                     ISBN 9780471058878, John Wiley \& Sons, (1980).
\bibitem{bendat2010} Julius S. Bendat and Allan G. Piersol, 
                     \emph{Random Data: Analysis and Measurement Procedures}, edition 4,
                     {Wiley Series in Probability and Statistics}, volume 729,
                     ISBN 9781118210826, John Wiley \& Sons, 640 pages, (2010).
\bibitem{kay1988}    Steven M. Kay,
                     \emph{Modern Spectral Estimation: Theory and Application},
                     Prentice-Hall signal processing series,
                     ISBN 9788131733561, Prentice Hall, 543 pages, (1988).
\bibitem{papoulis1984} {Anthanasios Papoulis},
                     \emph{Probability, Random Variables, and Stochastic Processes}, edition 2
                     {McGraw-Hill Series in Electrical Engineering},
                     ISBN {9780070484689}, {McGraw-Hill Book Company}, 576 pages, (1984).

\end{thebibliography}

\end{document}
