%============================================================================
% Daniel J. Greenhoe
% LaTeX file
%============================================================================
%=======================================
\chapter{Calculus}
%=======================================

%---------------------------------------
\begin{definition}
\label{def:spLLR}
\label{def:spLLRBu}
%---------------------------------------
Let $\R$ be the set of real numbers, $\borel$ the set of \structe{Borel sets} on $\R$, and
$\msm$ the standard \fncte{Borel measure} on $\borel$.
Let $\clFrr$ be as in \prefp{def:spXY}.
\defbox{\begin{array}{M}
  The \structd{space of Lebesgue square-integrable functions} $\spLLRBu$ (or $\spLLR$) is defined as
  \\\indentx$\ds\hxs{\spLLR}\eqd\hxs{\spLLRBu}\eqd\set{\ff\in\clFrr}{\brp{\int_\R \abs{\ff}^2}^\frac{1}{2} \dmu < \infty}$.\\
  The \hid{standard inner product} $\inprodn$ on $\spLLR$ is defined as
  \\\indentx$\ds\inprod{\ff(x)}{\fg(x)} \eqd \int_{\R}\ff(x)\fg^\ast(x)\dx$.\\
  The \hid{standard norm} $\normn$ on $\spLLR$ is defined as $\norm{\ff(x)}\eqd\inprod{\ff(x)}{\ff(x)}^\frac{1}{2}$
  %and $\opair{\spLLRBu}{\inprodn}$ is a \structe{Hilbert space}.
\end{array}}
\end{definition}

%---------------------------------------
\begin{definition}
\label{def:ddx}
%---------------------------------------
Let $\ff(x)$ be a \structe{function} in $\clFrr$.
\defbox{
  \ddx\ff(x) \eqd \ffp(x)\eqd \lim_{\varepsilon\to0}\frac{\ff(x+\varepsilon)-\ff(x)}{\varepsilon}
  }
\end{definition}

%---------------------------------------
\begin{proposition}
\label{prop:ddx_symmetry}
%---------------------------------------
\propbox{
  \brb{\begin{array}{FMD}
    (1). & $\ff(x)$ is \prope{continuous} & and\\
    (2). & $\mcom{\ff(a+x)=\ff(a-x)}{\prope{symmetric} about a point $a$}$
  \end{array}}
  \implies
  \brb{\begin{array}{FrclD}
    (1). & \ff'(a+x)&=& -\ff'(a-x) & (\prope{anti-symmetric} about $a$)\\
    (2). & \ff'(a)  &=& 0
  \end{array}}
  }
\end{proposition}
\begin{proof}
\begin{align*}
  \ff'(a+x)
    &= \lim_{\varepsilon\to0}\frac{1}{\varepsilon}\brs{\ff(a+x+\varepsilon)-\ff(a+x-\varepsilon)}
  \\&= \lim_{\varepsilon\to0}\frac{1}{\varepsilon}\brs{\ff(a-x-\varepsilon)-\ff(a-x+\varepsilon)}
    && \text{by hpothesis (2)}
  \\&= -\lim_{\varepsilon\to0}\frac{1}{\varepsilon}\brs{\ff(a-x+\varepsilon)-\ff(a-x-\varepsilon)}
  \\&= -\ff(a-x)
  \\
  \\
  \ff'(a)
    &= \frac{1}{2}\ff'(a+0) + \frac{1}{2}\ff'(a-0)
  \\&= \frac{1}{2}\brs{\ff'(a+0) - \ff'(a+0)}
    && \text{by previous result}
  \\&= 0
\end{align*}
\end{proof}

%---------------------------------------
\begin{lemma}
\label{lem:ddyffi}
%---------------------------------------
\lembox{
  \text{$\ff(x)$ is \prope{invertible}}
  \implies
  \brb{\ddy\ffi(y)=\frac{1}{\ddx\ff\brs{\ffi(y)}}}
  }
\end{lemma}
\begin{proof}
\begin{align*}
  \ddy\ffi(y)
    &\eqd \lim_{\varepsilon\to0}\frac{\ffi(y+\varepsilon)-\ffi(y)}{\varepsilon}
    &&    \text{by definition of $\ddy$} && \text{\xref{def:ddx}}
  \\&=    \lim_{\delta\to0}\brlr{\frac{1}{\ds\brs{\frac{\ff(x+\delta)-\ff(x)}{\delta}}}}_{x\eqd\ffi(y)}
    &&    \text{because in the limit, $\frac{\Delta y}{\Delta x}=\brp{\frac{\Delta x}{\Delta y}}^{-1}$}
  \\&\eqd \brlr{\frac{1}{\ddx\ff(x)}}_{x\eqd\ffi(y)}
    &&    \text{by definition of $\ddx$} && \text{\xref{def:ddx}}
  \\&=    \frac{1}{\ddx\ff\brs{\ffi(y)}}
    &&    \text{because $x\eqd\ffi(y)$}
\end{align*}
\end{proof}

%---------------------------------------
\begin{theorem}
\footnote{
  \citerpgc{chui}{86}{0121745848}{item (ii)},
  \citerppgc{prasad}{145}{146}{0849331692}{Theorem 6.2 (b)}
  }
\label{thm:int01}
%---------------------------------------
Let $\ff$ be a continuous function in $\spLLR$ and $\ff^{(n)}$ the $n$th derivative of $\ff$.
\thmbox{
  \int_{\intco{0}{1}^n} \ff^{(n)}\brp{\sum_{k=1}^n x_k} \dx_1\dx_2\cdots\dx_n = \sum_{k=0}^n (-1)^{n-k}\bcoef{n}{k}\ff(k)
  \qquad\forall n\in\Zp
  }
\end{theorem}
\begin{proof}
Proof by induction:
  \begin{enumerate}
    \item Base case \ldots proof for $n=1$ case:
      \begin{align*}
        \int_\intco{0}{1} \ff^{(1)}\brp{x} \dx
          &= \ff(1)-\ff(0)
          && \text{by \thme{Fundamental theorem of calculus}}
        \\&= (-1)^{1+1}\bcoef{1}{1}\ff(1) + (-1)^{1+0}\bcoef{1}{0}\ff(0)
        \\&= \sum_{k=0}^1 (-1)^{n-k}\bcoef{n}{k}\ff(k)
      \end{align*}

    \item Induction step \ldots proof that $n$ case $\implies$ $n+1$ case:
      \begin{align*}
          &\int_{\intco{0}{1}^{n+1}} \ff^{(n+1)}\brp{\sum_{k=1}^{n+1} x_k} \dx_1\dx_2\cdots\dx_{n+1}
        \\&= \int_{\intco{0}{1}^{n}}\brs{\int_0^1 \ff^{(n+1)}\brp{x_{n+1}+\sum_{k=1}^{n} x_k} \dx_{n+1}}\dx_1\dx_2\cdots\dx_{n}
        \\&= \mathrlap{%
             \int_{\intco{0}{1}^n}\brs{\left. \ff^{(n)}\brp{x_{n+1}+\sum_{k=1}^n x_k}\right|_{x_{n+1}=0}^{x_{n+1}=1}} \dx_1\dx_2\cdots\dx_n
             \qquad\text{by \thme{Fundamental theorem of calculus}}}
        \\&= \int_{\intco{0}{1}^n}\brs{\ff^{(n)}\brp{1+\sum_{k=1}^n x_k}-\ff^{(n)}\brp{0+\sum_{k=1}^n x_k}} \dx_1\dx_2\cdots\dx_n
        \\&= \sum_{k=0}^n (-1)^{n-k}\bcoef{n}{k}\ff(k+1) - \sum_{k=0}^n (-1)^{n-k}\bcoef{n}{k}\ff(k)
          && \text{by induction hypothesis}
        \\&= \sum_{m=1}^{m=n+1} (-1)^{n-m+1}\bcoef{n}{m-1}\ff(m) + \sum_{k=0}^n (-1)(-1)^{n-k}\bcoef{n}{k}\ff(k)
          && \text{where $m\eqd k+1\implies k=m-1$}
        \\&=\mathrlap{% 
              \brs{ \ff(n+1)+ \sum_{k=1}^{n} (-1)^{n-k+1}\bcoef{n}{k-1}\ff(k) }
            + \brs{ (-1)^{n+1} \ff(0) + \sum_{k=1}^n (-1)^{n-k+1}\bcoef{n}{k}\ff(k)}
            }
         %&& \text{by change of dummy variable ($m\rightarrow k$)}
        \\&= \ff(n+1) + (-1)^{n+1}\ff(0) 
            + \sum_{k=1}^{n} (-1)^{n-k+1}\mcom{\brs{\bcoef{n}{k-1}+\bcoef{n}{k}}}{use \thme{Stifel formula}}\ff(k) 
        \\&= (-1)^0\bcoef{n+1}{n+1}\ff(n+1) + (-1)^{n+1}\bcoef{n+1}{0}\ff(0) 
            + \sum_{k=1}^{n} (-1)^{n-k+1}\bcoef{n+1}{k}\ff(k) 
          && \text{\begin{tabular}{l}by \thme{Stifel formula}\\\ifxref{binomial}{thm:stifel}\end{tabular}}
        \\&= \sum_{k=0}^{n+1} (-1)^{n-k+1}\bcoef{n+1}{k}\ff(k) 
      \end{align*}
  \end{enumerate}
\end{proof}

Some proofs invoke differentiation multiple times.
This is simplified thanks to the \thme{Leibniz rule}, also called the 
\hie{generalized product rule} (\hie{GPR}, next lemma).
The Leibniz rule is remarkably similar in form to the \thme{binomial theorem}.
%--------------------------------------
\begin{lemma}[\thmd{Leibniz rule} / \thmd{generalized product rule}]
%\footnote{\url{http://en.wikipedia.org/wiki/Leibniz_rule_(generalized_product_rule)}}
\footnote{
  \citerpg{benisrael2002}{154}{3211829245},
  \citor{leibniz1710}
  }
\label{lem:LGPR}
%--------------------------------------
Let $\ff(x),\fg(x)\in\spLLR$ with derivatives
$\ff^{(n)}(x)\eqd\deriv{^n}{x^n}\ff(x)$ and
$\fg^{(n)}(x)\eqd\deriv{^n}{x^n}\fg(x)$ for $n=0,1,2,\ldots$,
and ${n\choose k}\eqd\frac{n!}{(n-k)!k!}$ (binomial coefficient).
Then
\lembox{
  \deriv{^n}{x^n}[\ff(x)\fg(x)] =
  \sum_{k=0}^n {n\choose k} \ff^{(k)}(x) \fg^{(n-k)}(x)
  }
\end{lemma}

%--------------------------------------
\begin{example}
\exbox{
  \deriv{^3}{x^3}\brs{\ff(x)\fg(x)} = \ff'''(x)\fg(x) + 3\ff''(x)\fg'(x) + 3\ff'(x)\fg''(x) + \ff(x)\fg'''(x)
  }
\end{example}

%---------------------------------------
\begin{theorem}[\thmd{Leibniz integration rule}]
\footnote{
  \citePpc{flanders1973}{615}{(1.1)}
  \citer{talvila2001},
  \citerpgc{knappb2005}{389}{0817632506}{Chapter VII},
  \citerpgc{protter2012}{422}{1461210860}{Leibniz Rule. Theorem 1.},
  \url{http://planetmath.org/encyclopedia/DifferentiationUnderIntegralSign.html}
  }
\label{thm:lir}
%---------------------------------------
  \thmbox{
    \ddx \int_{\fa(x)}^{\fb(x)} \fg(t) \dt  
      = \fg\brs{\fb(x)}\fb'(x) - \fg\brs{\fa(x)}\fa'(x)
    }
\end{theorem}










