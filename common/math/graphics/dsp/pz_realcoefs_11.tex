%============================================================================
% Daniel J. Greenhoe
% LaTeX File
% nominal font size = \gsize
% nominal unit = 15mm
%============================================================================
\begin{pspicture}(-1.25,-1.25)(1.5,1.25)%
  %-------------------------------------
  % node definitions
  %-------------------------------------
  \pnode(0,0){origin}%
  \uput{1.00}[-135](0,0){\pnode(0,0){circleP}}% circle point on circle for arrow
  %-------------------------------------
  % axes
  %-------------------------------------
  \psaxes[linecolor=axis,labels=none,ticks=none]{<->}(0,0)(-1.25,-1.25)(1.5,1.25)%
  \rput[tr](1.5,-1.5mm){\color{gray}$\scy\Reb{z}$}%
  \uput{3pt}[ 220](0,1.25){\color{gray}$\scy\Imb{z}$}%
  \uput{1pt}[ 45]( 0, 1){\color{gray}$\scy+1$}%
  \uput{1pt}[-45]( 0,-1){\color{gray}$\scy-1$}%
  \uput{1pt}[ 45]( 1, 0){\color{gray}$\scy+1$}%
  \uput{1pt}[135](-1, 0){\color{gray}$\scy-1$}%
  %-------------------------------------
  % unit circle
  %-------------------------------------
  \pscircle[linecolor=unitcircle](origin){1}%                 unit circle
  \rput[bl](-1.2,-1.2){\rnode[b]{circleL}{\color{gray}$\scy\abs{z}=1$}}% circle label
  \ncline[linecolor=gray,linestyle=dashed,nodesepA=0pt]
         {->}{circleL}{circleP}% circle arrow from circle label to circle point
  %-------------------------------------
  % pole and zero locations
  %-------------------------------------
  \pnode( 1,        1)  {z1}% zero 1 location
  \pnode( 1,       -1)  {z2}% zero 2 location
  \pnode(-0.666667, 0.5){p1}% pole 1 location
  \pnode(-0.666667,-0.5){p2}% pole 2 location
    %     |         |    |____label 
    %     |         |_________y coordinate
    %     |___________________x coordinate
  %-------------------------------------
  % labeling
  %-------------------------------------
  {%
  \psset{%
    linecolor=purple,
    linewidth=0.75pt,
    linestyle=dashed,
    labelsep=1pt,
    }%
  \psline( 0, 1  )(z1)\psline( 1,  0)(z1)%
  \psline( 0,-1  )(z2)\psline( 1,  0)(z2)%
  \psline( 0, 0.5)(p1)\psline(-0.666667,0)(p1)%
  \psline( 0,-0.5)(p2)\psline(-0.666667,0)(p2)%
  \uput[  0]{0}( 0,  0.5){ $\sfrac{1}{2}$}%
  \uput[  0]{0}( 0, -0.5){ $-\sfrac{1}{2}$}%
  \uput[135]{0}(-0.666667,0  ){-$\sfrac{2}{3}$}%
    %    |  |    |   |     |___label
    %    |  |    |   |_________x component of reference point
    %    |  |    |_____________y component of reference point
    %    |  |__________________text rotation about itself
    %    |_____________________rotation about reference point
  }%
  %-------------------------------------
  % pole and zero plotting
  %-------------------------------------
  \pzzero{z1}%
  \pzzero{z2}%
  \pzpole{p1}%
  \pzpole{p2}%
  %-------------------------------------
  % end
  %-------------------------------------
\end{pspicture}%
