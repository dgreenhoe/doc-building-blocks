%============================================================================
% Daniel J. Greenhoe
% XeLaTeX file
%============================================================================


%=======================================
\chapter{Boolean Lattices}
\label{chp:boolean}
%=======================================
\qboxnpq
  {\href{http://en.wikipedia.org/wiki/Augustus_De_Morgan}{Augustus de Morgan}
   \href{http://www-history.mcs.st-andrews.ac.uk/Timelines/TimelineE.html}{(1806--1871)},
   \href{http://www-history.mcs.st-andrews.ac.uk/BirthplaceMaps/Places/UK.html}{British mathematician and logician},
   \index{de Morgan, Augustus}
   \index{quotes!de Morgan, Augustus}
   \footnotemark
  }
  {../common/people/demorgan.jpg}
  {That the symbolic processes of algebra,
   invented as tools of numerical calculation,
   should be competent to express every act of thought,
   and to furnish the grammar and dictionary of an all-containing system of logic,
   would not have been believed until it was proved.\ldots
   by Mr. Boole.
   The unity of the forms of thought in all the applications of reason,
   however remotely separated,
   will one day be matter of notoriety and common wonder:
   and Boole's name will be remembered in connection with one of the most important steps
   towards the attainment of knowledge.}
  \citetblt{
    %quote: & \citerpg{demorgan1872}{80}{0836951190} \\
    quote: & \citerpu{demorgan1872}{80}{http://books.google.com/books?id=E2tC2EHgSGwC\&pg=PA80} \\
    image: & \url{http://en.wikipedia.org/wiki/Augustus_De_Morgan}
    }

%=======================================
\section{Definition and properties}
%=======================================
%%---------------------------------------
%\begin{definition}%[\thm{Huntington's \textsc{fourth set}}]
%\citetbl{
%  \citorpc{huntington1933}{280}{``4th set"}
%  }
%\label{def:booalg}
%%---------------------------------------
%Let $\setX$ be a set, $\orel$ an order relation, $\join$ and $\meet$ binary operations,
%$\bnot$ an unary operation, and $\bzero$ and $\bid$ nullary operations.
%Define $\orel$ such that
%$x\orel y \quad\iff\quad x\meet y=x \quad \forall x,y\in\setX$.
%\defboxt{
%$\latticed$ is a \hid{Boolean algebra} if
%\\$\begin{array}{@{\qquad}llcl @{\qquad}C@{\qquad}D@{\qquad}D}
%     1. & x \join  x              &=&  x                        & \forall x    \in\setX & (\prope{idempotent})  & and \\
%     2. & x \join  y              &=&  y \join  x               & \forall x,y  \in\setX & (\prope{commutative}) & and \\
%     3. & \brp{x \join  y}\join z &=&  x \join \brp{y \join z}  & \forall x,y,z\in\setX & (\prope{associative}) & and \\
%    %4. & \brp{x'\join y'}' \join \brp{x'\join y}' &=& x        & \forall x,y  \in\setX.& (\prope{Huntington's axiom})  &
%     4. & \brp{x\meet y} \join \brp{x\meet y'} &=& x            & \forall x,y  \in\setX.& (\prope{Huntington's axiom})  &
%\end{array}$}
%\end{definition}

A \structe{Boolean algebra} (next definition) is a \prope{bounded} \xref{def:latb},
\prope{distributive} \xref{def:latd}, and \prope{complemented} \xref{def:latc}, \structe{lattice} \xref{def:lattice}.
%---------------------------------------
\begin{definition}
\citetbl{
  \citerpg{maclane1999}{488}{0821816462},
  %\citorp{birkhoff1948}{149???}\\
  \citor{jevons1864}
  }
\label{def:boolean}
\label{def:booalg}
\label{def:l2n}
%---------------------------------------
%Let $\latA\eqd\booalgX$ be an \structe{algebraic structure}.
\defboxt{
  The \structe{bounded lattice} \xref{def:latb} $\latL\eqd\latbX$ is \propd{Boolean} if
  \\$\indentx\begin{array}{FMDD}
    1. & $\latL$ is \prope{complemented} & \xref{def:latc} & and\\
    2. & $\latL$ is \prope{distributive} & \xref{def:latd} & .
  \end{array}$
  \\
  A \structe{bounded lattice} $\latL$ that is \prope{Boolean} is a \structd{Boolean algebra}
  or a \structd{Boolean lattice}.\\
  A \structe{Boolean lattice} with $2^\xN$ elements is denoted $\hxs{\latL_2^\xN}$.
  }
\end{definition}

Several examples of \structe{Boolean lattice}s are illustrated in \prefpp{ex:latoc}.

%---------------------------------------
\begin{proposition}
%\citetbl{
%  \citerpg{maclane1999}{488}{0821816462}\\
%  %\citorp{birkhoff1948}{149???}\\
%  \citor{jevons1864}
%  }
\label{prop:boolean}
%---------------------------------------
%Let $\latA\eqd\booalgX$ be an \structe{algebraic structure}.
\propboxt{
  The algebraic structure $\latA\eqd\booalgX$ is a \structd{Boolean algebra} \xref{def:boolean} if
  \\$\indentx\begin{array}{FlclCDD}
    1. & \mc{5}{M}{$\latbX$ is a \structe{bounded lattice} \xref{def:latb}}                                              & and\\
    2. & x\meet\brp{y\join z}&=& \brp{x\meet y}\join\brp{x\meet z} & \forall x,y,z\in\setX & (\prope{distributive})      & and\\
    3. & x\meet x'           &=& \lzero                            & \forall x\in\setX     & (\prope{non-contradiction}) & and\\
    4. & x\join x'           &=& \lid                              & \forall x\in\setX     & (\prope{excluded middle}).  & 
  \end{array}$.
  }
\end{proposition}
\begin{proof}
This follows directly from \prefpp{def:boolean}.
\end{proof}

%Because the axioms 1--4 of \pref{def:booalg} (previous definition) are \prope{self-dual}
%and just as in the more general case of lattices (\prefp{thm:lat_duality}),
Boolean algebras support the \prope{principle of duality} (next theorem).
%---------------------------------------
\begin{theorem}[\thmd{Principle of duality}]
\label{thm:boo_duality}
\citetbl{
  \citerppgc{givant2009}{20}{22}{0387402934}{Chapter 4},
  \citerp{sikorski1969}{8}
  }
%---------------------------------------
Let $\latB\eqd\booalgd$ be a Boolean algebra.
\thmboxt{
  $\brb{\parbox{5\tw/16}{\raggedright
    $\phi$ is an identity on $\latB$ in terms of the operations\\
    $\join$, $\meet$, $\bnot$, $\bzero$, and $\bid$}}$
  $\qquad\implies\qquad$
  $\opT\phi$ is also an identity on $\latB$
  \\[1ex]
  where the operator $\opT$ performs the following mapping on the operations in $\clFxx$:
  \\\indentx $0\rightarrow1,\qquad 1\rightarrow0,\qquad \join\rightarrow\meet,\qquad \meet\rightarrow\join$
  }
\end{theorem}
\begin{proof}
For each of the identities in the definition of Boolean algebras (\prefp{prop:boo_char_H1}),
the operator $\opT$ produces another identity that is also in the definition:
\[\begin{array}{lc lcl c lcl cl}
  \opT(1a) &=& \opT[x \join  y        &=&  y \join  x                    ] &=& [x \meet  y        &=&  y \meet  x                   ] &=& (1b) \\
  \opT(1b) &=& \opT[x\meet  y         &=&  y \meet  x                    ] &=& [x\join  y         &=&  y \join  x                   ] &=& (1a) \\
  \opT(2a) &=& \opT[x\join( y\meet z) &=& ( x\join y) \meet ( x\join z)  ] &=& [x\meet( y\join z) &=& ( x\meet y) \join ( x\meet z) ] &=& (2b) \\
  \opT(2b) &=& \opT[x\meet( y\join z) &=& ( x\meet y) \join  ( x\meet z) ] &=& [x\join( y\meet z) &=& ( x\join y) \meet  ( x\join z)] &=& (2a) \\
  \opT(3a) &=& \opT[x \join \bzero    &=& x                              ] &=& [x \meet \bid      &=& x                             ] &=& (3b) \\
  \opT(3b) &=& \opT[x \meet \bid      &=& x                              ] &=& [x \join \bzero    &=& x                             ] &=& (3a) \\
  \opT(4a) &=& \opT[x \join x'        &=& \bid                           ] &=& [x \meet x'        &=& \bzero                        ] &=& (4b) \\
  \opT(4b) &=& \opT[x \meet x'        &=& \bzero                         ] &=& [x \join x'        &=& \bid                          ] &=& (4a)
\end{array}\]

Therefore, if the statement $\phi$ is consistent with regards to the Boolean algebra $\latB$,
then $\opT\phi$ is also consistent with regards to the Boolean algebra $\latB$.
\end{proof}


%=======================================
\section{Order properties}
%=======================================
The definition of Boolean algebras given by \pref{def:booalg}
is a set of postulates known as \thme{Huntington's \textsc{first set}}.
\pref{lem:boo_H1c} (next) gives a link between \thme{Huntington's \textsc{first set}}
of Boolean algebra postulates and the \thme{classic 10} set of Boolean algebra postulates
(\prefp{thm:boo_prop}).
%This lemma is used exclusivley in the proofs of
%\prefpp{prop:boo_char_H1} and \prefpp{thm:boo_prop}.
%---------------------------------------
\begin{lemma}
\label{lem:boo_H1c}
\citetbl{
  \citePppc{huntington1904}{292}{296}{``1st set"},
  \citerppg{joshi1989}{224}{227}{8122401201}
  }
%---------------------------------------
Let $\latL\eqd\booalgd$ be a bounded lattice.
\lemboxt{
\textbf{If \quad$\forall x,y,z\in\setX$}
\\\footnotesize%
$\brb{\begin{array}{l|lcl | lcl | D D}%
  \circOne
     & x \join  y         &=&  y \join  x
     &  x\meet  y         &=&  y \meet  x
     & (\prope{commutative})
     & and
  \\\circTwo
     & x\join( y\meet z) &=& ( x\join y) \meet ( x\join z)
     & x\meet( y\join z) &=& ( x\meet y) \join  ( x\meet z)
     & (\prope{distributive})
     & and
  \\\circThree
     & x \join 0           &=& x
     & x \meet 1           &=& x
     & (\prope{identity})
     & and
  \\\circFour
     & x \join x'        &=& \bid
     & x \meet x'        &=& \bzero
     & (\prope{complemented})
     &
\end{array}}$
\\
\normalsize
\textbf{then \quad$\forall x,y,z\in\setX$}
\footnotesize
\\$\brb{\begin{array}{F| lcl | lcl | DD}
   1.& x \join  x         &=&  x
     &  x\meet  x         &=&  x
     & (\prope{idempotent})
     & and
  \\
   2.& x\join ( y\join  z) &=& ( x\join   y) \join   z
     & x\meet ( y\meet  z) &=& ( x\meet   y) \meet   z
     & (\prope{associative})\footnotemark
     & and
  \\
   3.& x \join (x \meet  y) &=&  x
     & x \meet (x \join  y) &=&  x
     & (\prope{absorptive})% \prope{contractive}
     & and
  \\
   4.& x \join \bid        &=& \bid
     & x \meet \bzero      &=& \bzero
     & (\prope{bounded})
     & and
  \\
  % 5.& \brp{x'}'         &=& x
  %   &                   &&
  %   & (\prope{involutory})
  %   & and
  %\\
   5.& (x\join y)'       &=& x' \meet y'
     & (x\meet y)'       &=& x' \join y'
     & (\prope{de Morgan's laws}).
     &
\end{array}}$
}
\footnotetext{
  K.D. Joshi comments that having the \prope{associative} property as a result of an axiom
  rather than as an axiom, is a very unusual and ``remarkable property" in the world of algebras.
  \citerppg{joshi1989}{225}{226}{8122401201}
  }
\end{lemma}
\begin{proof}
For each pair of properties, it is only necessary to prove one of them,
as the other follows by the \hie{principle of duality} (\prefp{thm:boo_duality}).
Let the \ope{join} $\join$ be represented by $+$,
the operation \ope{meet} $\meet$ represented by $\cdot$ or juxtaposition,
and let $\meet$ have algebraic precedence over $\join$.
\begin{enumerate}
  \item Proof that $x + x =  x$ and $xx =  x$
        (\prope{idempotent} properties):
        \label{item:boo_H1c_idempotent}
        \begin{align*}
          x + x
            &= (x + x) \cdot \bid
            && \text{by \prope{identity} property,}
            && \text{\circThree b}
          \\&= (x + x)(x + x')
            && \text{by \prope{complemented} property,}
            && \text{\circFour a}
          \\&= x + (x x')
            && \text{by \prope{distributive} property,}
            && \text{\circTwo a}
          \\&= x + \bzero
            && \text{by \prope{complemented} property,}
            && \text{\circFour b}
          \\&= x
            && \text{by \prope{identity} property,}
            && \text{\circThree a}
        \end{align*}

  \item Proof that $x + \bid =  \bid$  and $x \cdot \bzero=\bzero$
        (\prope{bounded} properties):
        \label{item:boo_H1c_bounded}
        \begin{align*}
          x + \bid
            &= (x + \bid) \cdot \bid
            && \text{by \prope{identity} property,}
            && \text{\circThree b}
          \\&= \bid \cdot (x + \bid)
            && \text{by \prope{commutative} property,}
            && \text{\circOne b}
          \\&= (x + x') (x + \bid)
            && \text{by \prope{complemented} property,}
            && \text{\circFour a}
          \\&= x + (x' \cdot \bid)
            && \text{by \prope{distributive} property,}
            && \text{\circTwo a}
          \\&= x + x'
            && \text{by \prope{identity} property,}
            && \text{\circThree b}
          \\&= \bid
            && \text{by \prope{complemented} property,}
            && \text{\circFour a}
        \end{align*}

  \item Proof that $x + (x y) =  x$ and $x(x+ y)=x$:
        (\prope{absorptive} properties)
        \label{item:boo_H1c_absorptive}
        \begin{align*}
          x + (x\cdot y)
            &= (x \cdot \bid) + (xy)
            && \text{by \prope{identity} property,}
            && \text{\circThree b}
          \\&= x \cdot (\bid + y)
            && \text{by \prope{distributive} property,}
            && \text{\circTwo b}
          \\&= x \cdot (y + \bid)
            && \text{by \prope{commutative} property,}
            && \text{\circOne a}
          \\&= x \cdot \bid
            && \text{by \pref{item:boo_H1c_bounded}}
          \\&= x
            && \text{by \prope{identity} property,}
            && \text{\circThree b}
        \end{align*}

  \item Proof that $(x+y)+z=x+(y+z)$ and $(xy)z=x(yz)$
        (\prope{associative} properties):
        \label{item:boo_H1c_associative}\\
        Let $a\eqd x(yz)$ and $b\eqd(xy)z$.
          \begin{enumerate}
            \item Proof that $a+ x=b+ x$: \label{item:boo_H1c_axbx}
              \begin{align*}
                a + x
                  &= x(yz) + x
                  && \text{by definition of $a$}
                \\&= x(yz) + x\bid
                  && \text{by \prope{identity} property,}
                  && \text{\circThree b}
                \\&= x\brp{yz + \bid}
                  && \text{by \prope{distributive} property,}
                  && \text{\circTwo a}
                \\&= x\brp{\bid}
                  && \text{by \prope{bounded} property,}
                  && \text{\pref{item:boo_H1c_bounded}}
                \\&= x
                  && \text{by \prope{identity} property,}
                  && \text{\circThree b}
                \\&= x(x+ z)
                  && \text{by \prope{absorptive} property,}
                  && \text{\pref{item:boo_H1c_absorptive}}
                \\&= (x+ xy)(x+ z)
                  && \text{by \prope{absorptive} property,}
                  && \text{\pref{item:boo_H1c_absorptive}}
                \\&= x+(xy)z
                  && \text{by \prope{distributive} property,}
                  && \text{\circTwo b}
                \\&= (xy)z + x
                  && \text{by \prope{commutative} property,}
                  && \text{\circOne a,b}
                \\&= b + x
                  && \text{by definition of $b$}
              \end{align*}

            \item Proof that $a+x'=b+x'$: \label{item:boo_H1c_axpbxp}
              \begin{align*}
                a + x'
                  &= x(yz) + x'
                  && \text{by definition of $a$}
                \\&= x' + x(yz)
                  && \text{by \prope{commutative} property,}
                  && \text{\circOne a,b}
                \\&= (x' + x)(x'+yz)
                  && \text{by \prope{distributive} property,}
                  && \text{\circTwo b}
                \\&= \bid\cdot(x'+yz)
                  && \text{by \prope{complemented} property,}
                  && \text{\circFour a}
                \\&= x'+yz
                  && \text{by \prope{identity} property,}
                  && \text{\circThree b}
                \\&= (x'+y)(x'+z)
                  && \text{by \prope{distributive} property,}
                  && \text{\circTwo b}
                \\&= \brs{(x'+y)\cdot\bid}(x'+z)
                  && \text{by \prope{identity} property,}
                  && \text{\circThree b}
                \\&= \brs{\bid\cdot(x'+y)}(x'+z)
                  && \text{by \prope{commutative} property,}
                  && \text{\circOne b}
                \\&= \brs{(x+x')(x'+y)}(x'+z)
                  && \text{by \prope{complemented} property,}
                  && \text{\circFour a}
                \\&= (x'+xy)(x'+z)
                  && \text{by \prope{distributive} property,}
                  && \text{\circTwo b}
                \\&= x'+(xy)z
                  && \text{by \prope{distributive} property,}
                  && \text{\circTwo b}
                \\&= (xy)z+x'
                  && \text{by \prope{commutative} property,}
                  && \text{\circOne a}
                \\&= b+x'
                  && \text{by definition of $b$}
              \end{align*}

            \item Proof that $x(yz)=(xy)z$:
              \begin{align*}
                x(yz)
                  &\eqd a
                  && \text{by definition of $a$}
                \\&= a + a
                  && \text{by \prope{idempotent} property,}
                  && \text{\pref{item:boo_H1c_idempotent}}
                \\&= a + a\bid + \bzero
                  && \text{by \prope{identity} property,}
                  && \text{\circThree a,b}
                \\&= a + a(x+ x') + xx'
                  && \text{by \prope{complemented} property,}
                  && \text{\circFour a,b}
                \\&= a + ax + ax' + xx'
                  && \text{by \prope{distributive} property,}
                  && \text{\circTwo a}
                \\&= a + ax' + xa + xx'
                  && \text{by \prope{commutative} property,}
                  && \text{\circOne a,b}
                \\&= aa + ax' + xa + xx'
                  && \text{by \prope{idempotent} property,}
                  && \text{\pref{item:boo_H1c_idempotent}}
                \\&= a(a + x') + x(a + x')
                  && \text{by \prope{distributive} property,}
                  && \text{\circTwo a}
                \\&= (a + x)(a + x')
                  && \text{by \prope{distributive} property,}
                  && \text{\circTwo a}
                \\&= (b + x)(a + x')
                  && \text{by \pref{item:boo_H1c_axbx}}
                \\&= (b + x)(b + x')
                  && \text{by \pref{item:boo_H1c_axpbxp}}
                \\&= (b + x)b + (b+x)x'
                  && \text{by \prope{distributive} property,}
                  && \text{\circTwo a}
                \\&= b(b + x) + x'(b+x)
                  && \text{by \prope{commutative} property,}
                  && \text{\circOne b}
                \\&= bb + bx + x'b + x'x
                  && \text{by \prope{distributive} property,}
                  && \text{\circTwo a}
                \\&= b + bx + x'b + x'x
                  && \text{by \prope{idempotent} property,}
                  && \text{\pref{item:boo_H1c_idempotent}}
                \\&= b + bx + bx' + x'x
                  && \text{by \prope{commutative} property,}
                  && \text{\circOne b}
                \\&= b + b(x + x') + x'x
                  && \text{by \prope{distributive} property,}
                  && \text{\circTwo a}
                \\&= b + b\cdot\bid + \bzero
                  && \text{by \prope{complemented} property,}
                  && \text{\circFour a,b}
                \\&= b + b
                  && \text{by \prope{identity} property,}
                  && \text{\circThree a,b}
                \\&= b
                  && \text{by \prope{idempotent} property,}
                  && \text{\pref{item:boo_H1c_idempotent}}
                \\&\eqd (xy)z
                  && \text{by definition of $b$}
              \end{align*}
          \end{enumerate}


  \item Proof that $\brp{x + y}' = x'y'$ and $\brp{xy}' = x'+ y'$:
        (\prope{de Morgan} properties)
        \label{item:boo_H1c_demorgan}
    \begin{enumerate}
      \item Proof that $(x+y)+(x'y')=\bid$: \label{item:boo_H1c_demorgan_1}
        \begin{align*}
          &(x+ y)+ (x' y')
          \\&= \brs{(x+ y)+ x'} \brs{(x+ y)+ y'}
            && \text{by \prope{distributive} property,}
            && \text{\circTwo a}
          \\&= \brs{x'+(x+ y)} \brs{y'+(x+ y)}
            && \text{by \prope{commutative} property,}
            && \text{\circOne a}
          \\&= \brs{\brp{x'+(x+ y)}\bid} \brs{\brp{y'+(x+ y)}\bid}
            && \text{by \prope{identity} property,}
            && \text{\circThree b}
          \\&= \brs{\bid\brp{x'+(x+ y)}} \brs{\bid\brp{y'+(y+ x)}}
            && \text{by \prope{distributive} property,}
            && \text{\circTwo b}
          \\&= \brs{\brp{x'+ x}\brp{x'+(x+ y)}} \brs{\brp{y'+ y}\brp{y'+(y+ x)}}
            && \text{by \prope{complemented} property,}
            && \text{\circFour a}
          \\&= \brs{x'+\brp{x (x+ y)}} \brs{y'+\brp{y(y+ x)}}
            && \text{by \prope{distributive} property,}
            && \text{\circTwo a}
          \\&= \brs{x'+ x} \brs{y'+ y}
            && \text{by \prope{absorptive} property,}
            && \text{\pref{item:boo_H1c_absorptive}}
          \\&= \brs{\bid} \brs{\bid}
            && \text{by \prope{complemented} property,}
            && \text{\circFour a}
          \\&= \bid
            && \text{by \prope{bounded} property,}
            && \text{\pref{item:boo_H1c_bounded}}
        \end{align*}

      \item Proof that $(x+y)(x'y')=\bzero$:\label{item:boo_H1c_demorgan_0}
        \begin{align*}
          (x+ y)(x' y')
            &= \brs{x(x' y')} + \brs{y(x'y')}
            && \text{by \prope{distributive} property,}
            && \text{\circTwo b}
          \\&= \brs{\bzero+ x(x' y')} + \brs{\bzero + y(x'y')}
            && \text{by \prope{identity} property,}
            && \text{\circThree a}
          \\&= \brs{(xx')+ x(x' y')} + \brs{(yy') + y(x'y')}
            && \text{by \prope{complemented} property,}
            && \text{\circFour b}
          \\&= \brs{x\brp{x' + x' y'}} + \brs{y\brp{y' + x'y'}}
            && \text{by \prope{distributive} property,}
            && \text{\circTwo b}
          \\&= \brs{xx'} + \brs{yy'}
            && \text{by \prope{absorptive} property,}
            && \text{\pref{item:boo_H1c_absorptive}}
          \\&= \brs{\bzero} + \brs{\bzero}
            && \text{by \prope{complemented} property,}
            && \text{\circFour b}
          \\&= \bzero
            && \text{by \prope{bounded} property,}
            && \text{\pref{item:boo_H1c_bounded}}
        \end{align*}

      \item Proof that $(x+y)'=x'y'$:\\
        The quanities $(x+y)$ and $x'y'$ are \hie{complements} of each other
        as demonstrated by \pref{item:boo_H1c_demorgan_1} ($(x+y)+(x'y')=\bid$)
        and                \pref{item:boo_H1c_demorgan_0} ($(x+y)(x'y')=\bzero$).
        Therefore, $(x+y)'=x'y'$.
    \end{enumerate}
\end{enumerate}

\end{proof}

%---------------------------------------
\begin{proposition}
\label{prop:boo_poset}
\citetbl{
  \citerp{sikorski1969}{7}
  }%
%---------------------------------------
Let $\latB\eqd\booalgd$ be a Boolean algebra.
\propboxt{
  The pair $\opair{\setX}{\orel}$ is an \hie{ordered set}. In particular,
  \\\indentx$\begin{array}{F lDl l l CDD}
    1. & x \orel x  &     &           &          &           & \forall x    \in\setX   & (\prope{reflexive})       & and \\
    2. & x \orel y  & and & y \orel z & \implies & x \orel z & \forall x,y,z\in\setX   & (\prope{transitive})      & and \\
    3. & x \orel y  & and & y \orel x & \implies & x =     y & \forall x,y  \in\setX   & (\prope{anti-symmetric}). &
  \end{array}$
  }
\end{proposition}
\begin{proof}
\begin{enumerate}
  \item Proof that $\orel$ is \prope{reflexive}      in $\opair{\setX}{\orel}$:
    \begin{align*}
      x \orel x
        &\iff x \join x = x
        &&    \text{by definition of $\orel$ \xref{def:booalg}}
      \\&\iff \text{true}
        &&    \text{by \prefp{lem:boo_H1c}}
    \end{align*}

  \item Proof that $\orel$ is \prope{transitive}     in $\opair{\setX}{\orel}$:
    \begin{align*}
      \brb{\brp{x\orel y} \text{ and } \brp{y\orel z}}
        &\iff     \brb{\brp{x \join y = y} \text{ and } \brp{y \join z = z} }
        &&        \text{by definition of $\orel$ \xref{def:booalg}}
      \\&\implies \brp{x \join z}
      \\&=        x \join \brp{y \join z}
      \\&=        \brp{x\join y} \join z
        &&        \text{by \prope{associative} property of \prefp{lem:boo_H1c}}
      \\&=        y \join z
      \\&=        z
    \end{align*}

  \item Proof that $\orel$ is \prope{anti-symmetric} in $\opair{\setX}{\orel}$:
    \begin{align*}
      \brb{\brp{x\orel y} \text{ and } \brp{y\orel x}}
        &\iff     \brb{\brp{x \join y = y} \text{ and } \brp{y \join x = x} }
        &&        \text{by definition of $\orel$ \xref{def:booalg}}
      \\&\iff     \brb{\brp{x \join y = y} \text{ and } \brp{x \join y = x} }
        &&        \text{by \prope{commutative} property of \prefp{def:booalg}}
      \\&\iff     x = x \join y = y
      \\&\implies x = y
    \end{align*}

\end{enumerate}
\end{proof}

%=======================================
%\section{Lattice properties}
%=======================================
%Boolean algebras are actually equivalent to distributive complemented lattices
%(next theorem).
%Note that in the proof of the next theorem, that \prope{boundedness} is not actually
%necessary.
%---------------------------------------
\begin{proposition}
\label{prop:boo_bounds}
%---------------------------------------
Let $\booalgd$ be a Boolean algebra.
\propboxt{
  $\begin{array}{lMM}
    x \join y & is the \hie{least upper bound   } & of $x$ and $y$ in $\opair{\setX}{\orel}$.\\
    x \meet y & is the \hie{greatest lower bound} & of $x$ and $y$ in $\opair{\setX}{\orel}$.
  \end{array}$
  }
\end{proposition}



%---------------------------------------
\begin{theorem}[\thmd{classic 10 Boolean properties}]
\footnote{
  \citePppc{huntington1904}{292}{293}{``1st set"},
  \citorpc{huntington1933}{280}{``4th set"},
  \citerpg{maclane1999}{488}{0821816462},
  \citerpg{givant2009}{10}{0387402934},
  \citorpp{muller1909}{20}{21},
  \citor{schroder1890},
  \citorppu{whitehead1898}{35}{37}{http://books.google.com/books?id=P_A8AAAAIAAJ\&pg=PA35}
  }
\label{thm:boo_prop}
\label{thm:boolean}
%---------------------------------------
\thmboxt{
$\latA\eqd\booalgd$ is a \structb{Boolean algebra} $\quad\iff\quad$ $\forall x,y,z\in\setX$
%\textbf{if and only if} all of the following equations are true for all $x,y,z\in\setX$
\footnotesize
\\${\begin{array}{lcl | lcl || DD}
       x \join  x         &=&  x
     &  x\meet  x         &=&  x
     & (\prope{idempotent})
     & and
  \\
       x \join  y         &=&  y \join  x
     &  x\meet  y         &=&  y \meet  x
     & (\prope{commutative})
     & and
  \\
       x\join ( y\join  z)   &=& ( x\join   y) \join   z
     & x\meet  ( y\meet   z) &=& ( x\meet   y) \meet   z
     & (\prope{associative})
     & and
  \\
       x \join ( x\meet  y) &=&  x
     & x\meet ( x \join  y) &=&  x
     & (\prope{absorptive})% \prope{contractive}
     & and
  \\
       x\join( y\meet z) &=& ( x\join y) \meet ( x\join z)
     & x\meet( y\join z) &=& ( x\meet y) \join  ( x\meet z)
     & (\prope{distributive})
     & and
  \\
       x \join 0           &=& x
     & x \meet 1           &=& x
     & (\prope{identity})
     & and
  \\
       x \join 1           &=& 1
     & x \meet 0           &=& 0
     & (\prope{bounded})
     & and
  \\
       x \join x'        &=& 1
     & x \meet x'        &=& 0
     & (\prope{complemented})
     & and
  \\
       (x\join y)'       &=& x' \meet y'
     & (x\meet y)'       &=& x' \join y'
     & (\prope{de Morgan})
     & and
  \\
       \brp{x'}'         &=& x
     &                   &&
     & (\prope{involutory}) .
  \\\hline
       \mc{3}{H|} {property with emphasis on $\join$}
     & \mc{3}{H||}{dual property with emphasis on $\meet$}
     & \mc{2}{H}{property name}
\end{array}}$}
\end{theorem}
\begin{proof}
%By \prefpp{prop:boo_char_H1},
%  \[ \text{$\latA$ is a \prope{Boolean algebra} (\pref{def:booalg}) }
%     \qquad\iff\qquad
%     \text{$\latA$ satisfies the identities of \pref{prop:boo_char_H1}}
%  \]
%Therefore to prove \pref{thm:boo_prop} (this theorem), it is sufficient to prove
%  \[ \text{$\latA$ satisfies the identities of \pref{prop:boo_char_H1}}
%     \qquad\iff\qquad
%     \text{$\latA$ satisfies the identities of \pref{thm:boo_prop} (this theorem)}
%  \]
%
\begin{enumerate}
  \item Proof that \prefpp{prop:boo_char_H1} $\implies$ \prefpp{thm:boo_prop}:
    \\\begin{longtable}{rllll}
       1. & Proof that $\latA$ is \prope{idempotent}:              & by  1 & of \pref{lem:boo_H1c}      & \prefpo{lem:boo_H1c}      \\
       2. & Proof that $\latA$ is \prope{commutative}:             & by  1 & of \pref{prop:boo_char_H1} & \prefpo{prop:boo_char_H1} \\
       3. & Proof that $\latA$ is \prope{associative}:             & by  2 & of \pref{lem:boo_H1c}      & \prefpo{lem:boo_H1c}      \\
       4. & Proof that $\latA$ is \prope{absorptive}:              & by  3 & of \pref{lem:boo_H1c}      & \prefpo{lem:boo_H1c}      \\
       5. & Proof that $\latA$ is \prope{distributive}:            & by  2 & of \pref{prop:boo_char_H1} & \prefpo{prop:boo_char_H1} \\
       6. & Proof that $\latA$ is \prope{identity}:                & by  3 & of \pref{prop:boo_char_H1} & \prefpo{prop:boo_char_H1} \\
       7. & Proof that $\latA$ is \prope{bounded}:                 & by  4 & of \pref{lem:boo_H1c}      & \prefpo{lem:boo_H1c}      \\
       8. & Proof that $\latA$ is \prope{complemented}:            & by  4 & of \pref{prop:boo_char_H1} & \prefpo{prop:boo_char_H1} \\
       9. & Proof that $\latA$ is \prope{involutory}:     & by    &    \pref{cor:latcd_uniquecomp}      & \prefpo{cor:latcd_uniquecomp}      \\
      10. & Proof that $\latA$ is \prope{de Morgan}: & by  5 & of \pref{lem:boo_H1c}      & \prefpo{lem:boo_H1c}      \\
    \end{longtable}

  \item Proof that \prefpp{prop:boo_char_H1} $\impliedby$ \prefpp{thm:boo_prop}:
    \\\begin{tabular}{lllll}
       1. & Proof that $\latA$ is \prope{commutative}:             & by  2 & of \pref{thm:boo_prop} & \prefpo{thm:boo_prop} \\
       2. & Proof that $\latA$ is \prope{distributive}:            & by  5 & of \pref{thm:boo_prop} & \prefpo{thm:boo_prop} \\
       3. & Proof that $\latA$ is \prope{identity}:                & by  6 & of \pref{thm:boo_prop} & \prefpo{thm:boo_prop} \\
       4. & Proof that $\latA$ is \prope{complemented}:            & by  8 & of \pref{thm:boo_prop} & \prefpo{thm:boo_prop} \\
    \end{tabular}

\end{enumerate}
\end{proof}

%---------------------------------------
\begin{lemma}%[\lemd{Sasaki hook}]
\label{lem:boa_xcxy}
%---------------------------------------
\lembox{
  \brbr{\begin{array}{M}
    $\booalgd$\\
    is a \structe{Boolean algebra}
  \end{array}}
  \implies
  \brbl{\begin{array}{FrclCDD}
     1. & x' \join (x \meet y) &=& x' \join y & \forall x,y\in\setX & (\fncte{Sasaki hook}) & and\\
     2. & x  \join (x'\meet y) &=& x  \join y & \forall x,y\in\setX &                       &
  \end{array}}
  }
\end{lemma}
\begin{proof}
\begin{align*}
  x' \join (x\meet y) 
    &= \mcom{x'\join(x'\meet y)}{$x'$} \join (x\meet y) 
    && \text{by \prope{absorption} property \xref{thm:boolean}}
  \\&= x'\join \brs{(x' \join x)\meet y }
    && \text{by \prope{associative} and \prope{distributive} properties \xref{thm:boolean}}
  \\&= x'\join \brs{\lid\meet y}
    && \text{by \prope{excluded middle} property \xref{thm:boolean}}
  \\&= x' \join y 
    && \text{by definition of $\lid$ \xref{def:lub}}
  \\
  x \join (x'\meet y) 
    &= \mcom{x\join(x\meet y)}{$x$} \join (x\meet y) 
    && \text{by \prope{absorption} property \xref{thm:boolean}}
  \\&= x\join \brs{(x \join x')\meet y }
    && \text{by \prope{associative} and \prope{distributive} properties \xref{thm:boolean}}
  \\&= x\join \brs{\lid\meet y}
    && \text{by \prope{excluded middle} property \xref{thm:boolean}}
  \\&= x \join y 
    && \text{by definition of $\lid$ \xref{def:lub}}
\end{align*}
\end{proof}

%%---------------------------------------
%\begin{theorem}
%\citetbl{
%  \citerpg{joshi1989}{234}{8122401201}
%  }
%\label{thm:balg_atomic}
%%---------------------------------------
%Let $\latL\eqd\latticed$ be a lattice.
%\thmbox{
%  \text{$\latL$ is a \prope{Boolean algebra}}
%  \qquad\implies\qquad
%  \text{$\latL$ is \prope{atomic}}
%  }
%\end{theorem}

%%---------------------------------------
%\begin{theorem}[\thm{Stone Representation Theorem}]
%\citetbl{
%  \citerpg{joshi1989}{235}{8122401201}\\
%  %\citor{stone1936}
%  }
%\label{thm:srt}
%%---------------------------------------
%Let $\latL\eqd\latticed$ be a lattice.
%\thmboxt{
%  $\latL$ is \hie{Boolean}
%  $\qquad\iff\qquad$
%  $\latL$ is isomorphic to the power set of it's atoms
%  }
%\end{theorem}


%---------------------------------------
\begin{theorem}
\citetbl{
  \citerpg{joshi1989}{237}{8122401201}
  }
%---------------------------------------
Let $\seto{\setX}$ be the number of elements in a finite set $\setX$.
\thmboxt{
  $\latA$ is a \prope{Boolean algebra}
  $\qquad\implies\qquad$
  $\seto{\latA} = 2^n$ for some $n\in\Zp$.
  }
\end{theorem}

%\begin{minipage}{8\tw/16-3ex}%
%---------------------------------------
\begin{example}
%---------------------------------------
Here are some lattices that are Boolean algebras.
\exbox{\begin{tabular}{ccc}
\mc{3}{c}{Boolean algebras as algebras of sets}
  \\\hline
   \includegraphics{graphics/lat2_l2_setx.pdf}%
  &\includegraphics{graphics/lat4_m2_setxy.pdf}%
  &\includegraphics{graphics/lat8_2e3_setxyz.pdf}%
  %\\\hline
\end{tabular}}
\end{example}

%---------------------------------------
\begin{theorem}
\index{de Morgan's Laws}
\index{distributive laws}
\label{thm:cdl_seq}
%---------------------------------------
\thmboxt{
  If $\booalgd$ is a \prope{Boolean algebra} then
  \\$\ds
  \brb{\begin{array}{>{\ds}lc>{\ds}l @{\qquad\qquad} >{\ds}lc>{\ds}l @{\qquad}C}
    \mc{3}{c}{\text{tautology}} & \mc{3}{c}{\text{dual}} \\
    \hline
    \lnot \left( \meetop_{n=1}^N x_n \right)
      &=& \joinop_{n=1}^N (\lnot x_n)
      &
    \lnot \left( \joinop_{n=1}^N x_n \right)
      &=& \meetop_{n=1}^N (\lnot x_n)
    & \forall x_n\in \setX, N\in\Zp
    \\
    \left( \meetop_{n=1}^N x_n \right) \lor y
      &=& \meetop_{n=1}^N (x_n \lor y)
    &
    \left( \joinop_{n=1}^N x_n \right) \land y
      &=& \joinop_{n=1}^N (x_n \land y)
  \end{array}}$
  }
\end{theorem}
\begin{proof}
\begin{align*}
\intertext{1. Proof that
  $\lnot \left( \meetop_{n=1}^N x_n \right)
     = \joinop_{n=1}^N (\lnot x_n)
  $ (by induction):}
\intertext{Proof for $N=1$ case:}
  \lnot \left( \meetop_{n=1}^{N=1} x_n \right)
    &= \lnot x_n
    && \text{by definition of $\land$}
  \\&= \joinop_{n=1}^{N=1} (\lnot x_n)
    && \text{by definition of $\lor$}
  \\
\intertext{Proof for $N=2$ case:}
  \lnot \left( \meetop_{n=1}^{N=2} x_n \right)
    &= (\lnot x_1) \lor (\lnot x_2)
    && \text{by \prefp{thm:boo_prop} }
  \\&= \joinop_{n=1}^{N=2} (\lnot x_n)
    && \text{by definition of $\lor$}
  \\
\intertext{Proof that ($N$ case) $\implies$ ($N+1$ case):}
  \lnot \left( \meetop_{n=1}^{N+1} x_n \right)
    &= \lnot \left[ \left( \meetop_{n=1}^{N} x_n \right) \land x_N \right]
    && \text{by definition of $\land$}
  \\&= \left( \lnot \meetop_{n=1}^{N} x_n \right) \lor (\lnot x_{N+1})
    && \text{by \prefp{thm:boo_prop}}
  \\&= \left[ \joinop_{n=1}^{N}(\lnot x_n) \right] \lor (\lnot x_{N+1})
    && \text{by left hypothesis}
  \\&= \joinop_{n=1}^{N+1} (\lnot x_n)
    && \text{by definition of $\lor$}
  \\
\intertext{2. Proof that
  $\lnot \left( \joinop_{n=1}^N x_n \right)
     = \meetop_{n=1}^N (\lnot x_n)
  $:}
  \lnot \left( \joinop_{n=1}^N x_n \right)
    &= \lnot \left( \joinop_{n=1}^N (\lnot\lnot x_n) \right)
    && \text{by \prefp{thm:boo_prop}}
  \\&= \lnot \lnot \left( \meetop_{n=1}^N (\lnot x_n) \right)
    && \text{by previous result 1.}
  \\&= \meetop_{n=1}^N (\lnot x_n)
    && \text{by \prefp{thm:boo_prop}}
  \\
\intertext{3. Proof that
  $\left( \meetop_{n=1}^N x_n \right) \lor y
     = \joinop_{n=1}^N (x_n \lor y)
  $ (by induction):}
\intertext{Proof for $N=1$ case:}
  \left( \meetop_{n=1}^{N=1} x_n \right) \lor y
    &= x_1 \lor y
    && \text{by definition of $\land$}
  \\&= \meetop_{n=1}^{N=1} (x_n \lor y)
    && \text{by definition of $\land$}
  \\
\intertext{Proof for $N=2$ case:}
  \left( \meetop_{n=1}^{N=2} x_n \right) \lor y
    &= (x_1 \lor y) \land (x_2 \lor y)
    && \text{by \prefp{thm:boo_prop}}
  \\&= \meetop_{n=1}^{N=2} (x_n \lor y)
    && \text{by definition of $\land$}
  \\
\intertext{Proof that ($N$ case) $\implies$ ($N+1$ case):}
  \left( \meetop_{n=1}^{N+1} x_n \right) \lor y
    &= \left[ \left( \meetop_{n=1}^{N} x_n \right) \land x_{N+1} \right] \lor y
    && \text{by definition of $\land$}
  \\&= \left[\left(\meetop_{n=1}^{N} x_n \right) \lor y\right] \land (x_{N+1} \lor y)
    && \text{by \prefp{thm:boo_prop}}
  \\&= \left[ \meetop_{n=1}^{N}(x_n \lor y) \right] \land (x_{N+1} \lor y)
    && \text{by left hypothesis}
  \\&= \meetop_{n=1}^{N+1} (x_n \lor y)
    && \text{by definition of $\land$}
  \\
\intertext{4. Proof that
  $\left( \joinop_{n=1}^N x_n \right) \land y
     = \meetop_{n=1}^N (x_n \land y)
  $:}
  \left( \joinop_{n=1}^N x_n \right) \land y
    &= \lnot \lnot \left[ \left( \joinop_{n=1}^N x_n \right) \land y \right]
    && \text{by \prefp{thm:boo_prop}}
  \\&= \lnot \left[ \lnot \left( \joinop_{n=1}^N x_n \right) \lor (\lnot y) \right]
    && \text{by \prefp{thm:boo_prop}}
  \\&= \lnot \left[ \left( \meetop_{n=1}^N (\lnot x_n) \right) \lor (\lnot y) \right]
    && \text{by previous result 2.}
  \\&= \lnot \left( \meetop_{n=1}^N [(\lnot x_n)\lor (\lnot y)] \right)
    && \text{by previous result 3.}
  \\&= \left( \joinop_{n=1}^N \lnot [(\lnot x_n)\lor (\lnot y)] \right)
    && \text{by previous result 1.}
  \\&= \joinop_{n=1}^N (x_n \land y)
    && \text{by \prefp{thm:boo_prop}}
\end{align*}
\end{proof}



%=======================================
\section{Additional operations}
%=======================================
Propositional logic has a total of $2^4=16$ operations
in the class of functions
$\clF{\setft^2}{\setft}$ (see \prefpo{ba_log_set_cs}).
The 16 logic operations of propositional logic can all
be represented using the logic operations of \hie{disjunction} $\lor$,
\hie{conjunction} $\land$, and \hie{negation} $\lnot$.
Using these representations, all 16 operations can be generalized to
\hie{Boolean algebras} using the equivalent Boolean algebra/lattice operations of
\hie{join}, \hie{meet}, and \hie{complement}.\citepg{givant2009}{32}{0387402934}
Several of these additional operations for Boolean algebras are defined in
\pref{def:boo_rel} (next).

%---------------------------------------
\begin{definition}[additional Boolean algebra operations]
\label{def:boo_rel}
\footnote{
  \citerppg{givant2009}{32}{33}{0387402934},
  \citorpc{bernstein1934}{876}{implication $\supset$},
  \citorp{huntington1933}{276},
  \citorp{taylor1920}{243},
  \citorp{bernstein1914}{93},
  \citorpp{sheffer1913}{487}{488},
  \citorpu{peirce1902}{216}{http://www.ams.org/bull/1936-42-11/S0002-9904-1936-06439-6/S0002-9904-1936-06439-6.pdf},
  \citorppg{peirce1880}{218}{221}{0253372046}
  }
%---------------------------------------
Let $\booalgd$ be a Boolean algebra.
The following table defines additional operations in $\clFxxx$ in terms of
$\join$, $\meet$, and $\bnot$.
Let $x'\eqd \bnot x$ and $y'\eqd \bnot y$.%, and juxtaposition represent $\band$.
\\\begin{tabular}{|l|N|MNM>{\scriptstyle}M|}
  \hline
  \mc{1}{|R|}{name}        & \mc{1}{R|}{symbol} & \mc{4}{R|}{definition}
  \\\hline
  \opd{rejection}          & \indxsh{\rejection}  & x \rejection  y &\eqd& x' \meet y'                            & \forall x,y\in\setX \label{def:rejection} \\ % 0001
  \opd{exception}          & \indxsh{\exception}  & x \exception  y &\eqd& x  \meet y'                            & \forall x,y\in\setX \label{def:exception} \\ % 0100
  \opd{adjunction}         & \indxsh{\adjunction} & x \adjunction y &\eqd& x  \join y'                            & \forall x,y\in\setX \label{def:adjunction}\\ % 1101
  \opd{Sheffer stroke}     & \indxsh{\stroke}     & x \stroke     y &\eqd& x'\join y'                             & \forall x,y\in\setX \label{def:stroke}    \\ % 0111
  \opd{Boolean addition}   & \indxsh{\bxor}       & x \bxor       y &\eqd& \brp{x' \meet y} \join \brp{x \meet y'}& \forall x,y\in\setX \label{def:bxor}      \\ % 0110
  \opd{inhibit $x$}        & \indxsh{\binx}       & x \binx       y &\eqd& x' \meet y                             & \forall x,y\in\setX \label{def:binx}      \\ % 0010
  \opd{implication}        & \indxsh{\bimpl}      & x \bimpl      y &\eqd& x'\join y                              & \forall x,y\in\setX \label{def:bimpl}     \\ % 1011
  \opd{biconditional}      & \indxsh{\bequiv}     & x \bequiv     y &\eqd& \brp{x\meet y} \join \brp{x'\meet y'}  & \forall x,y\in\setX \label{def:bequiv}    \\ % 1001
  \hline
\end{tabular}
\end{definition}

%---------------------------------------
\begin{theorem}
\citetbl{
  \citerpg{givant2009}{33}{0387402934}
  }
%---------------------------------------
\thmbox{$\begin{tabular}{NllNl}
  \join       & (\op{join})             & is the dual of & \rejection  & (\op{rejection}) \\
  \meet       & (\op{meet})             & is the dual of & \stroke     & (\op{Sheffer stroke}) \\
  \bxor       & (\op{Boolean addition}) & is the dual of & \bequiv     & (\op{biconditional}) \\
  \exception  & (\op{exception})        & is the dual of & \bimpl      & (\op{implication}) \\
  \adjunction & (\op{adjunction})       & is the dual of & \binx       & (\op{inhibit $x$}) \\
\end{tabular}$}
\end{theorem}
\begin{proof}
\begin{align*}
  \text{(\ope{join})}\quad\brp{x \join y}'
    &= x' \meet y'
    && \text{by \prope{de Morgan's law} property \xref{thm:boo_prop}}
  \\&= x \rejection y \quad\text{(\ope{rejection})}
    && \text{by definition of \ope{rejection} $\rejection$ \xref{def:rejection}}
  \\
  \text{(\ope{meet})}\quad\brp{x \meet y}'
    &= x' \join y'
    && \text{by \prope{de Morgan's law} property \xref{thm:boo_prop}}
  \\&= x \stroke y \quad\text{(\ope{Sheffer stroke})}
    && \text{by definition of \ope{Sheffer stroke} $\stroke$ \xref{def:stroke}}
  \\
  \text{(\ope{Boolean addition})}\quad\brp{x \bxor y}'
    &= \brp{x'y \join xy'}'
    && \text{by def. of \ope{Boolean addition} $\bxor$ \xref{def:bxor}}
  \\&= \brp{x\join y'}\brp{x'\join y}
    && \text{by \prope{de Morgan's law} property \xref{thm:boo_prop}}
  \\&= xx' \join xy \join y'x' \join y'y
    && \text{by \prope{distributive} property \xref{thm:boo_prop}}
  \\&= xy \join x'y'
  \\&= x \bequiv y \quad\text{(\ope{biconditional})}
    && \text{by def. of \ope{biconditional} $\bequiv$ \xref{def:bequiv}}
  \\
  \text{(\ope{exception})}\quad\brp{x \exception y}'
    &= \brp{xy'}'
    && \text{by definition of \ope{exception} $\exception$ \xref{def:exception}}
  \\&= x'\join y
    && \text{by \prope{de Morgan's law} property \xref{thm:boo_prop}}
  \\&= x\bimpl y \quad\text{(\ope{implication})}
    && \text{by definition of \ope{implication} $\bimpl$ \xref{def:bimpl}}
  \\
  \text{(\ope{adjunction})}\quad\brp{x \adjunction y}'
    &= \brp{x \join y'}'
    && \text{by definition of \ope{adjunction} $\adjunction$ \xref{def:adjunction}}
  \\&= x'y
    && \text{by \prope{de Morgan's law} property \xref{thm:boo_prop}}
  \\&= x\binx y \quad\text{(\ope{inhibit $x$})}
    && \text{by definition of \ope{inhibit $x$} $\binx$ \xref{def:binx}}
  \\
  \text{(\ope{complement $x$)}}\quad \brp{x\bnotx y}'
    &= \brp{x'}'
    && \text{by definition of \ope{complement $x$} $\bnotx$}
  \\&= x
    && \text{by \prope{involutory} property \xref{thm:boo_prop}}
  \\&= x \btranx y \quad\text{(\ope{transfer $x$})}
    && \text{by definition of \ope{transfer $x$} $\btranx$}
  \\
  \text{(\ope{complement $y$)}}\quad \brp{x\bnoty y}'
    &= \brp{y'}'
    && \text{by definition of \ope{complement $y$} $\bnoty$}
  \\&= y
    && \text{by \prope{involutory} property \xref{thm:boo_prop}}
  \\&= x \btrany y \quad\text{(\ope{transfer $y$})}
    && \text{by definition of \ope{transfer $y$} $\btrany$}
\end{align*}
\end{proof}

%---------------------------------------
\begin{theorem}
\citetbl{
  \citerpg{givant2009}{39}{0387402934}
  }
%---------------------------------------
Let $\booalgd$ be a \prope{Boolean algebra}.
\thmbox{
  \begin{array}{l@{\qquad}c@{\qquad}lcl@{\qquad}C}
    x\orel y &\iff& y' &\orel     & x'   &\forall x,y\in\setX \\
    x\orel y &\iff& x  &\exception& y=0  &\forall x,y\in\setX \\
    x\orel y &\iff& x  &\bimpl    & y=1  &\forall x,y\in\setX
  \end{array}}
\end{theorem}
\begin{proof}
\begin{enumerate}
  \item Proof that $x\orel y \iff y' \orel x'$:
    \begin{align*}
      x \orel y
        &\iff x\meet y = x
        &&    \text{by definition of \ope{meet} $\meet$,}
        &&    \text{\prefp{def:meet}}
      \\&\iff \brp{x\meet y}' = x'
        &&    \text{by \prope{de Morgan's law} property,}
        &&    \text{\prefp{thm:boo_prop}}
      \\&\iff x'\join y' = x'
        &&    \text{by \prope{de Morgan's law} property,}
        &&    \text{\prefp{thm:boo_prop}}
      \\&\iff y'\orel x'
        &&    \text{by definition of \ope{join} $\join$,}
        &&    \text{\prefp{def:join}}
    \end{align*}

  \item Proof that $x\orel y \implies x\exception y=0$:
    \begin{align*}
      x\exception y
        &=     x\meet y'
        &&     \text{by definition of \ope{exception} $\exception$,}
        &&     \text{\prefp{def:exception}}
      \\&\orel y\meet y'
        &&     \text{by left hypothesis}
      \\&=     \bzero
        &&     \text{by definition of \ope{complement},}
        &&     \text{\prefp{def:latc}}
    \end{align*}

  \item Proof that $x\orel y \impliedby x\exception y=0$: \problem
    \begin{align*}
      x\exception y=0
        &\iff x\meet y' = 0
        &&     \text{by definition of \ope{exception} $\exception$,}
        &&     \text{\prefp{def:exception}}
      \\&\iff
    \end{align*}
\end{enumerate}
\end{proof}


%=======================================
\section{Representation}
%=======================================
A Boolean algebra $\booalgd$ can be represented in terms of
five operators (see \prefp{thm:boo_prop}):
\begin{liste}
  \item the binary operators join $\join$ and meet $\meet$,
  \item the unary operator complement $\bnot$, and
  \item the nullary opeartors $0$ and $1$.
\end{liste}
However, it is also possible to represent a Boolean algebra with fewer operators---
in fact, as few as one operator.
When a set of operators can completely represent all the operators of a Boolean algebra,
then that set is called \prope{functionally complete} (next definition).
%---------------------------------------
\begin{definition}
\citetbl{
  \citerpg{whitesitt1995}{69}{0486684830}
  }
\label{def:ba_fcomplete}
%---------------------------------------
Let $\latB\eqd\booalgd$ be a Boolean algebra.
\defboxt{
  A set of operators $\Phi$ is \hid{functionally complete} in $\latB$ if
  \\\indentx$\bor$, $\band$, $\bnot$, $\bzero$, and $\bid$\\ 
  can all be expressed in terms of $\Phi$.
  }
\end{definition}

Here are some examples of functionally complete sets:
\\\begin{tabular}{@{\qquad}>{\imark\quad }Mlll}
  \setn{\rejection}                  & (\hie{rejection})                                        & \pref{thm:boo_fc_rejection}       & \prefpo{thm:boo_fc_rejection}      \\
  \setn{\stroke}                     & (\hie{Sheffer stroke})                                    & \pref{thm:boo_fc_stroke}       & \prefpo{thm:boo_fc_rejection}      \\
  \setn{\adjunction,\,0}             & (\hie{adjunction} and $0$)                                  & \pref{thm:boo_fc_adjunction_0}    & \prefpo{thm:boo_fc_adjunction_0}   \\
  \setn{\exception,\,1}              & (\hie{exception} and $1$)                                   & \pref{thm:boo_fc_exception_1}     & \prefpo{thm:boo_fc_exception_1}    \\
  \setn{\join,\,\bnot}               & (\hie{join} and \hie{complement})                           & \pref{thm:boo_fc_join_not}        & \prefpo{thm:boo_fc_join_not}       \\
  \setn{\meet,\,\bnot}               & (\hie{meet} and \hie{complement})                           & \pref{thm:boo_fc_meet_not}        & \prefpo{thm:boo_fc_meet_not}       \\
  \setn{\bxor,\, \meet,\,1}          & (\hie{Boolean addition}, \hie{meet}, and $1$)                   & \pref{thm:boo_fc_xor_meet_1}      & \prefpo{thm:boo_fc_xor_meet_1}     \\
  \setn{\bxor,\, \join,\,1}          & (\hie{Boolean addition}, \hie{join}, and $1$)                   & \pref{thm:boo_fc_xor_join_1}      & \prefpo{thm:boo_fc_xor_join_1}     \\
  \setn{\bxor,\, \exception,\,\bnot} & (\hie{Boolean addition}, \hie{exception}, and \hie{complement}) & \pref{thm:boo_fc_xor_except_not}  & \prefpo{thm:boo_fc_xor_except_not}
\end{tabular}

%---------------------------------------
\begin{theorem}
\label{thm:boo_fc_join_not}
%---------------------------------------
Let $\latB\eqd\booalgd$ be a \hie{Boolean algebra}.
\thmboxt{
  The set $\setn{\join,\,\bnot}$ is \prope{functionally complete} with respect to $\latB$.
  In particular,
  \\$\begin{array}{@{\qquad}rcl@{\qquad}C}
    x \meet y &=& \brp{x' \join y}' & \forall x,y\in\setX \\
    0         &=& \brp{x \join x'}' & \forall x\in\setX \\
    1         &=& x \join x'        & \forall x\in\setX
  \end{array}$
  }
\end{theorem}
\begin{proof}
\begin{align*}
  x \meet y
    &= \brp{x \meet y}''
    && \text{by \prope{involutory} property \prefp{thm:boo_prop}}
  \\&= \brp{x' \join y'}'
    && \text{by de Morgan's Law property \prefp{thm:boo_prop}}
  \\
  1 &= x \join x'
    && \text{by complement property \prefp{thm:boo_prop}}
  \\
  0 &= 1'
  \\&= \brp{x \join x'}'
    && \text{by complement property \prefp{thm:boo_prop}}
\end{align*}
\end{proof}

%---------------------------------------
\begin{theorem}
\label{thm:boo_fc_meet_not}
%---------------------------------------
Let $\latB\eqd\booalgd$ be a \hie{Boolean algebra}.
\thmboxt{
  The set $\setn{\meet,\,\bnot}$ is \prope{functionally complete} with respect to $\latB$.
  In particular,
  \\$\begin{array}{@{\qquad}rcl@{\qquad}C}
    x \join y &=& \brp{x' \meet y}' & \forall x,y\in\setX \\
    0         &=& x \meet x'        & \forall x\in\setX \\
    1         &=& \brp{x \meet x'}' & \forall x\in\setX
  \end{array}$
  }
\end{theorem}
\begin{proof}
\begin{align*}
  x \join y
    &= \brp{x \join y}''
    && \text{by \prope{involutory} property \prefp{thm:boo_prop}}
  \\&= \brp{x' \meet y'}'
    && \text{by de Morgan's Law property \prefp{thm:boo_prop}}
  \\
  0 &= x \meet x'
    && \text{by complement property \prefp{thm:boo_prop}}
  \\
  1 &= 0'
  \\&= \brp{x \meet x'}'
    && \text{by complement property \prefp{thm:boo_prop}}
\end{align*}
\end{proof}

%%---------------------------------------
%\begin{definition}[joint denial]
%\label{def:rejection}
%\citetbl{
%  \citerpg{givant2009}{33}{0387402934}\\
%  \citerpg{shiva1998}{83}{0824700821} \\
%  \citerpg{whitesitt1995}{69}{0486684830}\\
%  \citerpg{quine1979}{45}{0674554515}\\
%  \citorp{sheffer1913}{487} \\
%  \citorppg{peirce1880}{218}{221}{0253372046}\\
%  %\url{http://www.swif.uniba.it/lei/foldop/foldoc.cgi?dagger+function}
%  }
%%---------------------------------------
%Let $\booalgd$ be a \hie{Boolean algebra}.
%\defboxt{\indxs{\rejection}
%  The \hid{rejection} operation $\rejection\in\clBx$ is defined as
%  \\\indentx
%    $\ds x\rejection y \eqd \brp{x \join y}' \qquad \forall x,y\in\setX$
%  }
%\end{definition}
%The joint denial operation is also called the \hid{rejection} operator.
%The symbol $\rejection$ itself is often referred to as the \hid{Sheffer stroke},
%sometimes denoted as $|$, or the \hid{dagger function}, denoted $\psi$.

%---------------------------------------
\begin{theorem}
\label{thm:boo_fc_rejection}
\citetbl{
  \citerpg{givant2009}{33}{0387402934}
  }
%---------------------------------------
Let $\latB\eqd\booalgd$ be a \hie{Boolean algebra}.
Let $\rejection$ represent the \hie{rejection} operator (\prefp{def:rejection}).
\thmboxt{
  The set $\setn{\rejection}$ is \prope{functionally complete} with respect to $\latB$.
  In particular,
  \\$\begin{array}{@{\qquad}rcl@{\qquad}C}
    x \join y &=& \brp{x\rejection y} \rejection \brp{x \rejection y} & \forall x,y\in\setX\\
    x \meet y &=& \brp{x\rejection x} \rejection \brp{y \rejection y} & \forall x,y\in\setX\\
    x'        &=& x \rejection x & \forall x\in\setX \\
    0         &=& x \rejection \brp{x\rejection x}  & \forall x\in\setX \\
    1         &=& \brs{x \rejection \brp{x\rejection x}} \rejection \brs{x \rejection \brp{x\rejection x}} & \forall x\in\setX
  \end{array}$
  }
\end{theorem}
\begin{proof}
\begin{align*}
  x'
    &= \brp{x \join x}'
    && \text{by \prefp{thm:boo_prop}}
  %\\&= x' \meet x'
  %  && \text{by de Morgan's Law \prefpo{thm:boo_prop}}
  \\&= x \rejection x
    && \text{by definition of $\rejection$ \prefpo{def:rejection}}
  \\
  x \join y
    &= \brp{x \join y}''
    && \text{by \prefp{thm:boo_prop}}
  %\\&= \brp{x' \meet y'}'
  %  && \text{by de Morgan's Law \prefpo{thm:boo_prop}}
  \\&= \brp{x \rejection y}'
    && \text{by definition of $\rejection$ \prefpo{def:rejection}}
  \\&= \brp{x \rejection y}\rejection\brp{x \rejection y}
    && \text{by previous result}
  \\
  x \meet y
    &= \brp{x \meet y}''
    && \text{by \prefp{thm:boo_prop}}
  \\&= \brp{x' \join y'}'
    && \text{by de Morgan's Law \prefpo{thm:boo_prop}}
  \\&= x' \rejection y'
    && \text{by definition of $\rejection$ \prefpo{def:rejection}}
  \\&= \brp{x\rejection x} \rejection \brp{y\rejection y}
    && \text{by previous result}
  \\
  0
    &= 1'
  \\&= \brp{x  \join x'}'
    && \text{by \prefp{thm:boo_prop}}
  \\&= x \rejection \brp{x'}
    && \text{by definition of $\rejection$ \prefpo{def:rejection}}
  \\&= x \rejection \brp{x\rejection x}
  \\
  1
    &= \brp{x  \join x'}
    && \text{by \prefp{thm:boo_prop}}
  \\&= \brp{x  \join x'}''
    && \text{by \prefp{thm:boo_prop}}
  \\&= \brp{x  \join x'}' \rejection \brp{x  \join x'}'
    && \text{by definition of $\rejection$ \prefpo{def:rejection}}
  \\&= \brs{x \rejection \brp{x'}} \rejection \brs{x \rejection \brp{x'}}
  \\&= \brs{x \rejection \brp{x\rejection x}} \rejection \brs{x \rejection \brp{x\rejection x}}
\end{align*}
\end{proof}


%%---------------------------------------
%\begin{definition}[Sheffer stroke]
%\label{def:stroke}
%\citetbl{
%  \citerppg{whitesitt1995}{68}{69}{0486684830}\\
%  \citerppg{quine1979}{48}{49}{0674554515}\\
%  \citorp{sheffer1913}{488}\\
%  \citorpu{peirce1902}{216}{http://www.ams.org/bull/1936-42-11/S0002-9904-1936-06439-6/S0002-9904-1936-06439-6.pdf}
%  %\url{http://www.swif.uniba.it/lei/foldop/foldoc.cgi?Sheffer+stroke}
%  }
%%---------------------------------------
%Let $\booalgd$ be a \hie{Boolean algebra}.
%\defboxt{\indxs{\stroke}
%  The \hid{Sheffer stroke} operation $\stroke\in\clBx$ is defined as
%  \\\indentx
%    $\ds x\stroke y \eqd \brp{x \meet y}' \qquad \forall x,y\in\setX$
%  }
%\end{definition}

%---------------------------------------
\begin{theorem}
\label{thm:boo_fc_stroke}
%---------------------------------------
Let $\latB\eqd\booalgd$ be a \hie{Boolean algebra}.
Let $\stroke$ represent the \hie{Sheffer stroke} operator (\prefp{def:stroke}).
\thmboxt{
  The set $\setn{\stroke}$ is \prope{functionally complete} with respect to $\latB$.
  In particular,
  \\$\begin{array}{@{\qquad}rcl@{\qquad}C}
    x \join y &=& \brp{x\stroke x} \stroke \brp{y \stroke y}  & \forall x,y\in\setX \\
    x \meet y &=& \brp{x\stroke y} \stroke \brp{x \stroke y}  & \forall x,y\in\setX \\
    x'        &=& x \stroke x  & \forall x\in\setX \\
    0         &=& \brs{x \stroke \brp{x\stroke x}} \stroke \brs{x \stroke \brp{x\stroke x}} & \forall x\in\setX \\
    1         &=& x \stroke \brp{x\stroke x}  & \forall x\in\setX
  \end{array}$
  }
\end{theorem}
\begin{proof}
\begin{align*}
  x'
    &= \brp{x \meet x}'
    && \text{by \prefp{thm:boo_prop}}
  %\\&= x' \join x'
  %  && \text{by de Morgan's Law \prefpo{thm:boo_prop}}
  \\&= x \stroke x
    && \text{by definition of $\stroke$ \prefpo{def:stroke}}
  \\
  x \join y
    &= \brp{x \join y}''
    && \text{by \prefp{thm:boo_prop}}
  \\&= \brp{x' \meet y'}'
    && \text{by de Morgan's Law \prefpo{thm:boo_prop}}
  \\&= x' \stroke y'
    && \text{by definition of $\stroke$ \prefpo{def:stroke}}
  \\&= \brp{x\stroke x} \stroke \brp{y\stroke y}
    && \text{by first result}
  \\
  x \meet y
    &= \brp{x \meet y}''
    && \text{by \prefp{thm:boo_prop}}
  %\\&= \brp{x' \join y'}'
  %  && \text{by de Morgan's Law \prefpo{thm:boo_prop}}
  \\&= \brp{x \stroke y}'
    && \text{by definition of $\stroke$ \prefpo{def:stroke}}
  \\&= \brp{x \stroke y}\stroke\brp{x \stroke y}
    && \text{by first result}
  \\
  1
    &= 0'
  \\&= \brp{x  \meet x'}'
    && \text{by \prefp{thm:boo_prop}}
  \\&= x \stroke \brp{x'}
    && \text{by definition of $\stroke$ \prefpo{def:stroke}}
  \\&= x \stroke \brp{x\stroke x}
  \\
  0
    &= \brp{x  \meet x'}
    && \text{by \prefp{thm:boo_prop}}
  \\&= \brp{x  \meet x'}''
    && \text{by \prefp{thm:boo_prop}}
  \\&= \brp{x  \meet x'}' \stroke \brp{x  \meet x'}'
    && \text{by definition of $\stroke$ \prefpo{def:stroke}}
  \\&= \brs{x \stroke \brp{x'}} \stroke \brs{x \stroke \brp{x'}}
  \\&= \brs{x \stroke \brp{x\stroke x}} \stroke \brs{x \stroke \brp{x\stroke x}}
\end{align*}
\end{proof}

Besides the \hie{rejection} singleton $\setn{\rejection}$ and
the Sheffer stroke singleton $\setn{\stroke}$,
there are no single opertor sets that are \prope{functionally complete}
(next theorem).
%---------------------------------------
\begin{theorem}
\label{thm:denial}
\citetbl{
  \citerpg{quine1979}{49}{0674554515},
  \citorpc{zylinski1925}{208}{$\rejection=\phi_{15}$, $\stroke=\phi_2$}
  %\citorpc{zylinski1925}{208}{$\rejection=\phi_{15}=$``joint falsehood, $\stroke=\phi_2=$``incompatibility}
  }
%---------------------------------------
Let $\latB\eqd\booalgd$ be a Boolean algebra.
Let $\rejection$ be the \hie{rejection} operator and
    $\stroke$ be the \hie{Sheffer stroke} operator.
\thmbox{
  \text{$\setn{+}$ is \prope{functionally complete} in $\latB$}
  \qquad\implies\qquad
  + = \rejection \quad\text{ or }\quad +=\stroke
  }
\end{theorem}

%%---------------------------------------
%\begin{definition}[adjunction]
%\label{def:adjunction}
%\citetbl{
%  \citorp{huntington1933}{276}\\
%  \citorp{bernstein1914}{93} \\
%  }
%\index{adjunction}
%%---------------------------------------
%Let $\booalgd$ be a \hie{Boolean algebra}.
%\defboxt{\indxs{\adjunction}
%  The \hid{adjunction} operation $\adjunction\in\clBx$ is defined as
%  \\\indentx
%    $\ds x\adjunction y \eqd x \join y' \qquad \forall x,y\in\setX$
%  }
%\end{definition}

%---------------------------------------
\begin{theorem}
\label{thm:boo_fc_adjunction_0}
%---------------------------------------

Let $\latB\eqd\booalgd$ be a \hie{Boolean algebra}.
Let $\adjunction$ represent the \hie{adjunction} operator (\prefp{def:adjunction}).
\thmboxt{
  The set $\setn{\adjunction,\,0}$ is \prope{functionally complete} with respect to $\latB$.
  In particular,
  \\$\begin{array}{@{\qquad}rcl@{\qquad}C}
    x \join y &=& x \adjunction \brp{0\adjunction y}                   & \forall x,y\in\setX \\
    x \meet y &=& 0\adjunction\brs{\brp{0\adjunction x} \adjunction y} & \forall x,y\in\setX \\
    x'        &=& 0 \adjunction x                                      & \forall x\in\setX \\
    1         &=& x \adjunction x                                      & \forall x\in\setX
  \end{array}$
  }
\end{theorem}
\begin{proof}
\begin{align*}
  x'
    &= 0 \join x'
    && \text{by \prefp{thm:boo_prop}}
  \\&= 0 \adjunction x
    && \text{by definition of $\adjunction$ \xref{def:adjunction}}
  \\
  x \join y
    &= x \join y''
    && \text{by \prefp{thm:boo_prop}}
  \\&= x \adjunction \brp{y'}
    && \text{by definition of $\adjunction$ \xref{def:adjunction}}
  \\&= x \adjunction \brp{0\adjunction y}
    && \text{by previous result}
  \\
  x \meet y
    &= \brp{x' \join y'}'
    && \text{by \prope{de Morgan's law} property \prefp{thm:boo_prop}}
  \\&= \brp{x' \adjunction y}'
    && \text{by definition of $\adjunction$ \xref{def:adjunction}}
  \\&= \brs{\brp{0\adjunction x} \adjunction y}'
    && \text{by previous result}
  \\&= 0\adjunction\brs{\brp{0\adjunction x} \adjunction y}
    && \text{by previous result}
  \\
  1
    &= x \join x'
    && \text{by \prope{complement} property \prefp{thm:boo_prop}}
  \\&= x \adjunction x
    && \text{by definition of $\adjunction$ \xref{def:adjunction}}
\end{align*}
\end{proof}

%%---------------------------------------
%\begin{definition}[exception]
%\label{def:exception}
%\citetbl{
%  \citerpg{givant2009}{32}{0387402934}\\
%  \citorp{huntington1933}{276}\\
%  \citorp{taylor1920}{243}\\
%  \citor{bernstein1914}
%  }
%\index{exception}\index{subtraction}
%%---------------------------------------
%Let $\booalgd$ be a \hie{Boolean algebra}.
%\defboxt{\indxs{\exception}
%  The \hid{exception} operation $\exception\in\clBx$ is defined as
%  \\\indentx
%    $\ds x\exception y \eqd x \meet y' \qquad \forall x,y\in\setX$
%  }
%\end{definition}

%---------------------------------------
\begin{theorem}
\citetbl{
  \citorpp{bernstein1914}{89}{91}
  }
\label{thm:boo_fc_exception_1}
%---------------------------------------
Let $\latB\eqd\booalgd$ be a \hie{Boolean algebra}.
Let $\exception$ represent the \hie{exception} operator (\prefp{def:exception}).
\thmboxt{
  The set $\setn{\exception,\,1}$ is \prope{functionally complete} with respect to $\latB$.
  In particular,
  \\$\begin{array}{@{\qquad}rcl@{\qquad}C}
    x \join y &=& 1\exception\brs{\brp{1\exception x} \exception y} & \forall x,y\in\setX \\
    x \meet y &=& x \exception \brp{1\exception y}                  & \forall x,y\in\setX \\
    x'        &=& 1 \exception x                                    & \forall x\in\setX \\
    0         &=& x \exception x                                    & \forall x\in\setX
  \end{array}$
  }
\end{theorem}
\begin{proof}
\begin{align*}
  x'
    &= 1 \meet x'
    && \text{by \prefp{thm:boo_prop}}
  \\&= 1 \exception x
    && \text{by definition of $\exception$ \xref{def:adjunction}}
  \\
  x \meet y
    &= x \meet y''
    && \text{by \prefp{thm:boo_prop}}
  \\&= x \exception \brp{y'}
    && \text{by definition of $\exception$ \xref{def:adjunction}}
  \\&= x \exception \brp{1\exception y}
    && \text{by previous result}
  \\
  x \join y
    &= \brp{x' \meet y'}'
    && \text{by \prope{de Morgan's law} property \prefp{thm:boo_prop}}
  \\&= \brp{x' \exception y}'
    && \text{by definition of $\exception$ \xref{def:adjunction}}
  \\&= \brs{\brp{1\exception x} \exception y}'
    && \text{by previous result}
  \\&= 1\exception\brs{\brp{1\exception x} \exception y}
    && \text{by previous result}
  \\
  0
    &= x \meet x'
    && \text{by \prope{complement} property \prefp{thm:boo_prop}}
  \\&= x \exception x
    && \text{by definition of $\exception$ \xref{def:adjunction}}
\end{align*}
\end{proof}

%---------------------------------------
\begin{theorem}
\citetbl{
  \citerpg{roth2006}{42}{0521845041}
  }
\label{thm:boo_fc_xor_meet_1}
%---------------------------------------
Let $\latB\eqd\booalgd$ be a \hie{Boolean algebra}.
\thmboxt{
  The set $\setn{\bxor,\, \meet,\,1}$ is \prope{functionally complete} with respect to $\latB$.
  In particular,
  \\$\begin{array}{@{\qquad}rcl@{\qquad}C}
    x \join y &=& xy \bxor x \bxor y & \forall x,y\in\setX \\
    x'        &=& x \bxor 1          & \forall x\in\setX \\
    0         &=& x \bxor x          & \forall x\in\setX
  \end{array}$
  }
\end{theorem}
\begin{proof}
\begin{align*}
  x'
    &= x' \join 0
    && \text{by \prefp{thm:boo_prop}}
  \\&= \brp{x'\meet 1} \join \brp{x\meet 0}
    && \text{by \prefp{thm:boo_prop}}
  \\&= \brp{x'\meet1} \join \brp{x\meet 1'}
  \\&= x\bxor 1
    && \text{by definition of $\bxor$ \xref{def:bxor}}
  \\
  0
    &= 0 \join 0
    && \text{by \prefp{thm:boo_prop}}
  \\&= \brp{x'\meet x} \join \brp{x\meet x'}
    && \text{by \prefp{thm:boo_prop}}
  \\&= x \bxor x
    && \text{by definition of $\bxor$ \xref{def:bxor}}
  \\
  xy \lxor x \lxor y
    &= (xy) \bxor (x \bxor y)
    && \text{by \prope{associative} property \prefp{thm:boo_prop}}
  \\&= (xy) \lxor \brp{x'y \join  xy'}
    && \text{by definition of $\bxor$ \xref{def:bxor}}
  \\&= (xy)'\brp{x'y \join  xy'} \join (xy)\brp{x'y \join  xy'}'
    && \text{by definition of $\bxor$ \xref{def:bxor}}
  \\&= \brp{x'\join y'}\brp{x'y \join  xy'} \join (xy)\brs{\brp{x'y}' \brp{xy'}'}
    && \text{by \prope{de Morgan's law} \prefp{thm:boo_prop}}
  \\&= \brp{x'\join y'}\brp{x'y \join  xy'} \join (xy)\brs{\brp{x''\join y'} \brp{x'\join y''}}
    && \text{by \prope{de Morgan's law} \prefp{thm:boo_prop}}
  \\&= \brp{x'\join y'}\brp{x'y \join  xy'} \join (xy)\brs{\brp{x\join y'} \brp{x'\join y}}
  \\&= \brp{x'y \join  xy'} \join (xy)\brs{xy\join x'y'}
  \\&= \brp{x'y \join  xy'} \join xy
  \\&= \brp{x'y \join  xy'} \join \brp{xy \join xy}
    && \text{by \prope{idempotent} property \pref{thm:boo_prop}}
  \\&= \brp{xy \join x'y} \join \brp{xy \join  xy'}
    %&& \text{by \prope{associative} and \prope{commutative} properties \prefpo{thm:boo_prop}}
    && \text{by \prefp{thm:boo_prop}}
  \\&= \brp{x \join x'}y \join x\brp{y \join  y'}
    && \text{by \prope{distributive} property \pref{thm:boo_prop}}
  \\&= \brp{1}y \join x\brp{1}
  \\&= x \join y
\end{align*}
\end{proof}


%---------------------------------------
\begin{theorem}
\label{thm:boo_fc_xor_join_1}
%---------------------------------------
Let $\latB\eqd\booalgd$ be a \hie{Boolean algebra}.
\thmboxt{
  The set $\setn{\bxor,\, \join,\, \,1}$ is \prope{functionally complete} with respect to $\latB$.
  In particular,
  \\$\begin{array}{@{\qquad}rcl@{\qquad}C}
    x \meet y &=& \brs{\brp{x\bxor1} \join \brp{y\bxor1}} \bxor 1 & \forall x,y\in\setX \\
    x'        &=& x \bxor 1                                       & \forall x\in\setX \\
    0         &=& x \bxor x                                       & \forall x\in\setX
  \end{array}$
  }
\end{theorem}
\begin{proof}
\begin{align*}
  0
    &= x \bxor x
  \\
  x'
    &= x \bxor 1
  \\
  x \meet y
    &= \brp{x' \join y'}'
  \\&= \brs{\brp{x\bxor1} \join \brp{y\bxor1}} \bxor 1
\end{align*}
\end{proof}

%---------------------------------------
\begin{theorem}
\label{thm:boo_fc_xor_except_not}
%---------------------------------------
Let $\latB\eqd\booalgd$ be a \hie{Boolean algebra}.
\thmboxt{
  The set $\setn{\bxor,\, \exception,\,\bnot}$ is \prope{functionally complete} with respect to $\latB$.
  In particular,
  \\$\begin{array}{@{\qquad}rcl@{\qquad}C}
    x \join y &=& \brp{x\exception y}\bxor y        & \forall x,y\in\setX \\
    x \meet y &=& x \exception \brp{x \exception y} & \forall x,y\in\setX \\
    0         &=& x \bxor x                         & \forall x\in\setX
  \end{array}$
  }
\end{theorem}
\begin{proof}
\begin{align*}
  x \join y
    &= x(y\join y') \join y
  \\&= xy \join xy' \join y
    && \text{by \prope{distributive} property \xref{thm:boo_prop}}
  \\&= (y \join xy) \join xy'
    && \text{by \prope{associative} property \xref{thm:boo_prop}}
  \\&= y \join xy'
    && \text{by \prope{absorptive} property \xref{thm:boo_prop}}
  \\&= (y \join x'y) \join xy'
    && \text{by \prope{absorptive} property \xref{thm:boo_prop}}
  \\&= (y \join x')y \join (xy')y'
    && \text{by \prope{distributive} and \prope{idempotent} properties \xref{thm:boo_prop}}
  \\&= (xy')'y \join (xy')y'
    && \text{by \prope{de Morgan's law} property \xref{thm:boo_prop}}
  \\&= (xy') \bxor y
    && \text{by definition of $\bxor$ \xref{def:bxor}}
  \\&= (x \exception y) \bxor y
    && \text{by definition of $\exception$ \xref{def:exception}}
  \\
  x\meet  y
    &= xx' \join xy
  \\&= x(x' \join y)
    && \text{by \prope{distributive} and \prope{idempotent} properties \xref{thm:boo_prop}}
  \\&= x(x''y')'
    && \text{by \prope{de Morgan's law} property \xref{thm:boo_prop}}
  \\&= x(xy')'
    && \text{by \prope{involutory} property \xref{thm:boo_prop}}
  \\&= x(x\exception y)'
    && \text{by definition of $\exception$ \xref{def:exception}}
  \\&= x \exception (x\exception y)
    && \text{by definition of $\exception$ \xref{def:exception}}
  \\
  0 &= x x'
  \\&= x \exception (x\exception x')
    && \text{by previous result}
\end{align*}
\end{proof}

%=======================================
\section{Characterizations}
%=======================================
\qboxnps
  {\href{http://en.wikipedia.org/wiki/Edward_V._Huntington}{Edward V. Huntington}
   \href{http://www-history.mcs.st-andrews.ac.uk/Timelines/TimelineF.html}{(1874--1952)},
   \href{http://www-history.mcs.st-andrews.ac.uk/BirthplaceMaps/Places/USA.html}{American} mathematician\footnotemark
  }
  {../common/people/huntington.jpg}
  {The algebra of symbolic logic\ldots
   %although originally studies merely as a means of handling certain problems in the logic of
   %classes and the logic of propositions,
   has recently assumed some importance as an independent calculus;
   it may therefore be not without interest to consider it from a purely mathematical or 
   abstract point of view\ldots%
  }
  \citetblt{
    quote: & \citePp{huntington1904}{288} \\
    image: & \url{http://en.wikipedia.org/wiki/Edward_V._Huntington}
    }


%=======================================
\subsubsection{Order characterizations}
%=======================================
An order characterization of Boolean algebras has already been given by \prefpp{def:boolean}: 
A lattice is a Boolean algebra if and only if it is 
\index{characterizations!Boolean algebra}
\prope{distributive} and \prope{complemented}.

%---------------------------------------
\begin{proposition}
\label{prop:boo_char_elkan}
\citetbl{
  \citorp{kondo2008}{1035},
  \citerpc{elkan1994}{3}{\prope{Elkan's law}}
  }
%---------------------------------------
Let $\algA\eqd\booalgd$ be a \prope{bounded} and \prope{complemented} \emph{lattice}.
\propbox{%
  \brb{\begin{array}{M}
    $\algA$ is a\\ 
    \structb{Boolean algebra}
  \end{array}}
  \quad\iff\quad
  \brb{\begin{array}{FlCDD}
     %1. & \joinop\setX,\,\meetop\setX\,\in\setX  &                     & (\prope{bounded})      \\
     %2. & x\in\setX \implies x'\in\setX          & \forall x\in\setX   & (\prope{complemented}) \\
     1. & 1' = 0                                 &                     &                       & and\\
     2. & (x\meet y')' = y \join (x' \meet y')   & \forall x,y\in\setX & (\prope{Elkan's law})
  \end{array}}
  }
\end{proposition}


%=======================================
\subsubsection{Algebraic characterizations}
%=======================================
This section presents several algebraic characterizations.
One such characterization has already been provided by \prefpp{thm:boo_prop}---
the standard properties of Boolean algebras characterized by 19 identities.
If a system satisfies these 19 identities, then that system \emph{is} a Boolean algebra.
However, the set of 19 identities is very much an \emph{over}-specification.
It is possible to characterize Boolean algebras using much fewer relationships,
from which all of the 19 identities of \pref{thm:boo_prop} can be derived.
Here are some of these reduced characterizations:
\\\begin{tabular}{cl *{4}{>{\footnotesize}l}}
 %$\imark$ &                                 &        &10 relationships, & \pref{thm:boo_char_10} & \prefpo{thm:boo_char_10} \\
  $\imark$ & \thme{Huntington's first set}:          & (1904) & 8 relationships, & \pref{prop:boo_char_H1}  & \prefpo{prop:boo_char_H1} \\
  $\imark$ & \thme{Huntington's fourth set}:         & (1933) & 4 relationships, & \pref{prop:boo_char_H4}  & \prefpo{prop:boo_char_H4} \\
  $\imark$ & \thme{Huntington's fifth set}:          & (1933) & 3 relationships, & \pref{prop:boo_char_H5}  & \prefpo{prop:boo_char_H5} \\
  $\imark$ & \thme{Stone}:                           & (1935) & 7 relationships, & \pref{prop:boo_char_stone}  & \prefpo{prop:boo_char_stone} \\
  $\imark$ & \thme{Byrne's Formulation A}:           & (1946) & 3 relationships, & \pref{prop:boo_char_Ba}  & \prefpo{prop:boo_char_Ba} \\
  $\imark$ & \thme{Byrne's Formulation B}:           & (1946) & 2 relationships, & \pref{prop:boo_char_Bb}  & \prefpo{prop:boo_char_Bb} \\
 %$\imark$ & \thme{Elkan's law on bounded lattices}: & (2008) & 2 relationships, & \pref{prop:boo_char_elkan} & \prefpo{prop:boo_char_elkan}
\end{tabular}

All of these characterizations use 3 variables.
It might be reasonable to ask if there exists a characterization that uses only two variables.
The answer is ``No", as demonstrated by the next theorem.
%---------------------------------------
\begin{theorem}
\citetbl{
  \citerp{sikorski1969}{3},
  \citorp{diamond1947}{961},
  \citerp{gerrish1978}{36}
  }
%---------------------------------------
\thmboxt{There does \emph{not} exist a characterization of Boolean algebras
  consisting of only 2 variables.}
\end{theorem}


%%---------------------------------------
%\begin{theorem}
%\label{thm:boo_char_10}
%\citetbl{
%  \citerp{sikorski1969}{1}
%  }
%%---------------------------------------
%Let $\latL\eqd\latticed$ be a bounded lattice
%with $\joinop\setX\eqd1$ (least upper bound) and $\meetop\setX\eqd0$ (greatest lower bound).
%\propboxt{
%$\latL$ is a Boolean algebra \textbf{if and only if} for all $x,y,z\in\setX$%
%\footnotesize%
%\\$\begin{array}{@{\qquad}llcl lcl D D}%
%  1. & x \join  y         &=&  y \join  x
%     &  x\meet  y         &=&  y \meet  x
%     & (\prope{commutative})
%     & and
%  \\
%  2. & x \join \brp{y \join z} &=&  \brp{x \join y} \join z
%     & x \meet \brp{y \meet z} &=&  \brp{x \meet y} \meet z
%     & (\prope{associative})
%     & and
%  \\
%  3. & x \join ( x\meet  y) &=&  x
%     & x\meet ( x \join  y) &=&  x
%     & (\prope{absorptive})
%     & and
%  \\
%  4. & x\join( y\meet z) &=& ( x\join y) \meet ( x\join z)
%     & x\meet( y\join z) &=& ( x\meet y) \join  ( x\meet z)
%     & (\prope{distributive})
%     & and
%  \\
%  5. & \brp{x \meet x'} \join y &=& y
%     & \brp{x \join x'} \meet y &=& y
%     &
%\end{array}$}
%\end{theorem}

%---------------------------------------
\begin{proposition}[\thmd{Huntington's first set}]
\footnote{
  \citerp{gerrish1978}{35},
  \citerpgc{salii1988}{33}{0821845225}{``Huntington's Theorem"},
  \citerpgc{joshi1989}{222}{8122401201}{(B1)--(B4)},
  \citePppc{huntington1904}{292}{293}{``1st set"},
  \citorpc{huntington1933}{277}{``1st set"},
  \citerpg{givant2009}{10}{0387402934}
  }
\label{prop:boo_char_H1}
%---------------------------------------
Let $\setX$ be a set, $\orel$ a relation in $\clRxx$,
$\join$ and $\meet$ binary operations in $\clFxxx$,
$\bnot$ an unary operation in $\clFxx$,
and $\bzero$ and $\bid$ nullary operations on $\setX$.
\propboxt{
$\booalgd$ is a \structd{Boolean algebra} if for all $x,y,z\in\setX$%
\footnotesize%
\\$\begin{array}{Flcl lcl D}%
  1. & x \join  y         &=&  y \join  x
     &  x\meet  y         &=&  y \meet  x
     & (\prope{commutative})
  \\
  2. & x\join( y\meet z) &=& ( x\join y) \meet ( x\join z)
     & x\meet( y\join z) &=& ( x\meet y) \join  ( x\meet z)
     & (\prope{distributive})
  \\
  3. & x \join \bzero      &=& x
     & x \meet \bid        &=& x
     & (\prope{identity})
  \\
  4. & x \join x'        &=& \bid
     & x \meet x'        &=& \bzero
     & (\prope{complemented})
\end{array}$
\\
and where the relation $\orel$ is defined as
$\begin{array}{lclC}
  x \orel y &\iff& x \join y = y & \forall x,y\in\setX.
\end{array}$
}

The property $x\join x'=\bid$ is referred to as ``the law of the \prope{excluded middle}".
The property $x\meet x'=\bzero$ is referred to as ``the law of \prope{non-contradiction}".
\end{proposition}
\begin{proof}
\begin{enumerate}
  \item Proof that $\latA$ is a Boolean algebra $\implies$   $\latA$ is a \prope{distributive} \prope{complemented} \hie{lattice}:
    \begin{enumerate}
      \item Proof that $\latA$ is \prope{distributive}: by \prefp{def:booalg}
      \item Proof that $\latA$ is \prope{complemented}: by \prefp{def:booalg}
      \item Proof that $\latA$ is \prope{bounded}:      by \prefp{lem:boo_H1c}
      \item Proof that $\latA$ is a \hie{lattice}:
        \begin{enumerate}
          \item Proof that $\latA$ is \prope{idempotent}:   by \prefp{lem:boo_H1c}
          \item Proof that $\latA$ is \prope{commutative}:  by \prefp{def:booalg}
          \item Proof that $\latA$ is \prope{associative}:  by \prefp{lem:boo_H1c}
          \item Proof that $\latA$ is \prope{absorptive}:   by \prefp{lem:boo_H1c}
          \item Therefore, by \prefpp{thm:lattice}, $\latA$ is a \hie{lattice}
        \end{enumerate}
    \end{enumerate}

  \item Proof that $\latA$ is a Boolean algebra $\impliedby$ $\latA$ is a \prope{distributive} \prope{complemented} \hie{lattice}:
    \begin{enumerate}
      \item Proof that $\latA$ is \prope{commutative}:  by property of lattices, \prefp{thm:lattice}
      \item Proof that $\latA$ is \prope{distributive}: by right hypothesis
      \item Proof that $\latA$ has \prope{identity}:
        \begin{align*}
          x \join \bzero
            &= x \join \brp{x \meet x'}
            && \text{by \prope{complemented} property in right hypothesis}
          \\&= x
            && \text{by \prope{absorptive} property of lattices \prefp{thm:lattice}}
          \\
          x \meet \bid
            &= x \meet \brp{x \join x'}
            && \text{by \prope{complemented} property in right hypothesis}
          \\&= x
            && \text{by \prope{absorptive} property of lattices \prefp{thm:lattice}}
        \end{align*}
      \item Proof that $\latA$ is \prope{complemented}: by right hypothesis
    \end{enumerate}
\end{enumerate}
\end{proof}





\thme{Huntington's fourth set} (next)
characterizes Boolean algebras in terms of the standard properties of
\prope{idempotent}, \prope{commutative}, and \prope{associative} (see \prefp{thm:boo_prop}),
and also in terms of an additional property called \prope{Huntington's axiom},%
\citetbl{
  \citerpgc{givant2009}{13}{0387402934}{problem 7}
  }
or (in terms of $x$ and $y$), \prope{$x$ commutes $y$}.
Huntington's axiom is significant in the context of \prope{orthomodular} lattices
in that an orthomodular lattice that satisfies Huntington's axiom is
a Boolean algebra.%
\citetbl{
  \citerpc{renedo2003}{72}{Definition 3},
  \citerpg{beran1985}{52}{902771715X},
  \citor{beran1982}
  }
%---------------------------------------
\begin{proposition}[\thmd{Huntington's fourth set}]
\label{prop:boo_char_H4}
\citetbl{
  %\citerpg{givant2009}{13}{0387402934} \\
  \citorpc{huntington1933}{280}{``4th set"}
  }
%---------------------------------------
Let $\algA\eqd\latticed$ be an \emph{algebraic structure}.
\propboxt{
$\algA$ is a \structb{Boolean algebra} \quad$\iff$
\\\indentx$\brb{\begin{array}{Flcl @{\qquad}C@{\qquad}D@{\qquad}D}
     1. & x \join  x              &=&  x                        & \forall x    \in\setX & (\prope{idempotent})  & and \\
     2. & x \join  y              &=&  y \join  x               & \forall x,y  \in\setX & (\prope{commutative}) & and \\
     3. & \brp{x \join  y}\join z &=&  x \join \brp{y \join z}  & \forall x,y,z\in\setX & (\prope{associative}) & and \\
     4. & \brp{x'\join y'}' \join \brp{x'\join y}' &=& x        & \forall x,y  \in\setX.& (\prope{Huntington's axiom})  &
    %4. & \brp{x\meet y} \join \brp{x\meet y'} &=& x            & \forall x,y  \in\setX.& (\prope{Huntington's axiom})  &
\end{array}}$}
\end{proposition}
\begin{proof}
\begin{enumerate}
  \item Proof that [$\latA$ is a \prope{Boolean algebra}] $\implies$ [$\latA$ satisfies the 4 pairs of properties]:
    \begin{enumerate}
      \item Proof that $x \join  x = x$ (\prope{idempotent} property with respect to $\join$):
            \\by 1a of \prefpp{lem:boo_H1c}.
      \item Proof that $x \join  y = y \join  x$ (\prope{commutative} property with respect to $\join$):
            \\by 1a of this proposition.
      \item Proof that $\brp{x \join  y}\join z = x \join \brp{y \join z}$ (\prope{associative} property with respect to $\join$):
            \\by 2a of \prefpp{lem:boo_H1c}.
      \item Proof that $\brp{x\meet y} \join \brp{x\meet y'} = x$ (\prope{Huntington's axiom}):
        \begin{align*}
          \brp{x\meet y} \join \brp{x\meet y'}
            &= x \meet \brp{y \join y'}
            && \text{by 2a}
            && \text{(\prope{distributive} property wrt $\join$)}
          \\&= x \meet \bid
            && \text{by 3a}
            && \text{(\prope{complemented} property wrt $\join$)}
          \\&= x
            && \text{by 4b}
            && \text{(\prope{identity} property wrt $\meet$)}
        \end{align*}
    \end{enumerate}

  \item Proof that [$\latA$ is a \prope{Boolean algebra}] $\impliedby$ [$\latA$ satisfies the 4 pairs of properties]:
    \begin{enumerate}
      \item Proof that $x \join  y =  y \join  x$: by 2 of \prefp{def:booalg}.
      \item Proof that $x \meet  y =  y \meet  x$:
      \item Proof that $x\join( y\meet z) = ( x\join y) \meet ( x\join z)$:
      \item Proof that $x\meet( y\join z) = ( x\meet y) \join  ( x\meet z)$:
      \item Proof that $x \join x' = \bid$:
      \item Proof that $x \meet x' = \bzero$:
      \item Proof that $x \join \bzero= x$:
      \item Proof that $x \meet \bid = x$:
    \end{enumerate}

\end{enumerate}
\end{proof}

%---------------------------------------
\begin{proposition}[\thmd{Huntington's fifth set}]
\label{prop:boo_char_H5}
\citetbl{
  \citerpg{givant2009}{13}{0387402934},
  \citorpc{huntington1933}{286}{``5th set"}
  }
%---------------------------------------
Let $\algA\eqd\latticed$ be an \emph{algebraic structure}.
\propboxt{
$\algA$ is a \structb{Boolean algebra} \quad$\iff$
\\\indentx$\brb{\begin{array}{Flcl @{\qquad}C@{\qquad}D@{\qquad}D}
     1. & x''                      &=& x                                            & \forall x,y,z\in\setX & & and \\
     2. & x \join \brp{y\join y'}' &=& x                                            & \forall x,y  \in\setX & & and \\
     3. & x \join \brp{y\join z}'  &=& \brs{\brp{y'\join x}'\join\brp{z'\join x}'}' & \forall x,y,z\in\setX.& &
\end{array}}$}
\end{proposition}

%---------------------------------------
\begin{proposition}[Stone]
\label{prop:boo_char_stone}
\citetbl{
  \citorp{stone1935}{705}
  }
%---------------------------------------
Let $\algA\eqd\latticed$ be an \emph{algebraic structure}.
\propboxt{
$\algA$ is a \structb{Boolean algebra} \quad$\iff$
\\$\brb{\begin{array}{Flcl @{\qquad}C@{\qquad}D@{\quad}D}
     1. & x \join y         &=& y \join x                 & \forall x,y\in\setX   & (\prope{join commutative})   & and \\
     2. & x \meet(y\join z) &=& (x\meet y)\join(x\meet z) & \forall x,y,z\in\setX & (\prope{left distributive})  & and \\
     3. & (x\join y)\meet z &=& (x\meet z)\join(y\meet z) & \forall x,y,z\in\setX & (\prope{right distributive}) & and \\
     4. & x \join \bzero    &=& x                         & \forall x\in\setX     & (\prope{join identity})      & and \\
     5. & \exists x' \st x \join x'  &=& \bid \text{ and } x\meet x'=\bzero    & \forall x\in\setX     & (\prope{complemented})    & and \\
     6. & x \join x         &=& x                         & \forall x\in\setX & (\prope{idempotent}) & and \\
     7. & x \meet x         &=& x                         & \forall x\in\setX &   &
\end{array}}$}
\end{proposition}


%---------------------------------------
\begin{proposition}[\thm{Byrne's \textsc{Formulation A}}]
\label{prop:boo_char_Ba}
\citetbl{
  \citerpg{givant2009}{13}{0387402934},
  \citorpc{byrne1946}{270}{\textsc{``Formulation A"}}
  }
%---------------------------------------
Let $\algA\eqd\latticed$ be an \emph{algebraic structure}.
\propboxt{
  $\algA$ is a \structb{Boolean algebra} \quad$\iff$
  \\\indentx$\brb{\begin{array}{Flcl @{\qquad}C@{\qquad}D@{\qquad}D}
     1. & x \join  y              &=&  y \join  x               & \forall x,y  \in\setX & (\prope{commutative}) & and \\
     2. & \brp{x \join  y}\join z &=&  x \join \brp{y \join z}  & \forall x,y,z\in\setX & (\prope{associative}) & and \\
     3. & x\join y' = z\join z' &\iff& x\join y=x               & \forall x,y,z\in\setX.&                      &
  \end{array}}$
  }
\end{proposition}
\begin{proof}
\begin{enumerate}
  \item Proof that $\algA$ is a Boolean algebra $\implies$ 3 identities:
    \begin{enumerate}
      \item \prope{commutative} property: By \prefpp{thm:boo_prop}, all Boolean algebras are \prope{commutative}.
      \item \prope{associative} property: By \prefpp{thm:boo_prop}, all Boolean algebras are \prope{associative}.
      \item Proof that $x\join y' = y\join y' \implies x\join y=x$:
        \begin{align*}
          x \join y
            &= y \join x
            && \text{by Boolean hypothesis and \prefp{thm:boo_prop}}
          \\&= y \join \brp{x'}'
            && \text{by Boolean hypothesis and \prefp{thm:boo_prop}}
          \\&= y \join \brp{x'}'
            && \text{by Boolean hypothesis and \prefp{thm:boo_prop}}
          \\&= x' \join \brp{x'}'
            && \text{by $x\join y' = y\join y'$ hypothesis}
          \\&= x' \join x
            && \text{by Boolean hypothesis and \prefp{thm:boo_prop}}
          \\&= x
            && \text{by Boolean hypothesis and \prefp{thm:boo_prop}}
        \end{align*}

      \item Proof that $x\join y' = y\join y' \impliedby x\join y=x$:
        \begin{align*}
          x\join y'
            &= \brp{x \join y} \join y'
            && \text{by $x\join y=x$ hypothesis}
          \\&= x \join \brp{y \join y'}
            && \text{by Boolean hypothesis and \prefp{thm:boo_prop}}
          \\&= x \join 1
            && \text{by Boolean hypothesis and \prefp{thm:boo_prop}}
          \\&= x
            && \text{by Boolean hypothesis and \prefp{thm:boo_prop}}
        \end{align*}
    \end{enumerate}

  \item Proof that $\algA$ is a Boolean algebra $\impliedby$ 3 identities:
    \begin{enumerate}
      \item Proof that $x\join x = x$ (\prope{idempotent} property): \label{item:boo_char_Ba_idempotent}
            because $x\join x'=x \join x'$ and by identity 3

      \item Proof that $x\join x' = y\join y'$: \label{item:boo_char_Ba_xxyy}
            by \pref{item:boo_char_Ba_idempotent} and identity 3

      \item Proof that $x\join y=x$ and $y\join z=y$ $\implies$ $x\join z=x$: \label{item:boo_char_Ba_xzx}
        \begin{align*}
          x \join z
            &= (x\join y) \join z
            && \text{by $x\join y=x$ hypothesis}
          \\&= x\join (y \join z)
            && \text{by identity 2 (\prope{associative} property)}
          \\&= x\join y
            && \text{by $y\join z=y$ hypothesis}
          \\&= x
            && \text{by $x\join y=x$ hypothesis}
        \end{align*}

      \item Proof that $x''=x$ (\prope{involutory} property): \label{item:boo_char_Ba_double_complement}
        \begin{align}
          x''\join x' \label{equ:boo_char_Ba_complement_21zz}
            &= x'\join x''
            && \text{by identity 1 (\prope{commutative} property)}
          \\&= z\join z'
            && \text{by \pref{item:boo_char_Ba_xxyy}}\notag
          \\
          x''\join x \label{equ:boo_char_Ba_complement_202}
            &= x''
            && \text{by \pref{equ:boo_char_Ba_complement_21zz} and identity 3}
          \\
          x'''\join x' \label{equ:boo_char_Ba_complement_313}
            &= x'''
            && \text{by \pref{equ:boo_char_Ba_complement_202}}
          \\
          x''''\join x'' \label{equ:boo_char_Ba_complement_424}
            &= x''''
            && \text{by \pref{equ:boo_char_Ba_complement_202}}
          \\
          x''''\join x \label{equ:boo_char_Ba_complement_404}
            &= x''''
            && \text{by \pref{equ:boo_char_Ba_complement_424}, \pref{equ:boo_char_Ba_complement_404}, and \pref{item:boo_char_Ba_xzx}}
          \\
          x''''\join x' \label{equ:boo_char_Ba_complement_41zz}
            &= z\join z'
            && \text{by \pref{equ:boo_char_Ba_complement_404} and identity 3}
          \\
          x'\join x''' \label{equ:boo_char_Ba_complement_131}
            &= x'
            && \text{by \pref{equ:boo_char_Ba_complement_41zz} and identity 3}
          \\
          x''' \label{equ:boo_char_Ba_complement_31}
            &= x'''\join x'
            && \text{by \pref{equ:boo_char_Ba_complement_313}}
          \\&= x'\join x'''
            && \text{by identity 1 (\prope{commutative} property)}\notag
          \\&= x'
            && \text{by \pref{equ:boo_char_Ba_complement_131}}\notag
          \\
          x\join x''' \label{equ:boo_char_Ba_complement_13zz}
            &= x\join x'
            && \text{by \pref{equ:boo_char_Ba_complement_31}}
          \\&= z \join z'
            && \text{by \pref{item:boo_char_Ba_xxyy}}\notag
          \\
          x\join x'' \label{equ:boo_char_Ba_complement_020}
            &= x
            && \text{by \pref{equ:boo_char_Ba_complement_13zz} and identity 3}
          \\
          x''
            &= x'' \join x
            && \text{by \pref{equ:boo_char_Ba_complement_202}}\notag
          \\&= x \join x''
            && \text{by identity 1 (\prope{commutative} property)}\notag
          \\&= x
            && \text{by \pref{equ:boo_char_Ba_complement_020}}\notag
        \end{align}

      \item Proof that $x\join(x'\join y)''=z\join z'$: \label{item:boo_char_Ba_xxyzz}
        \begin{align*}
          x\join (x'\join y)''
            &= x\join (x'\join y)
            && \text{by \pref{item:boo_char_Ba_double_complement} (\prope{involutory} property)}
          \\&= (x\join x')\join y
            && \text{by identity 2 (\prope{associative} property)}
          \\&= y\join(x\join x')
            && \text{by identity 1 (\prope{commutative} property)}
          \\&= y\join (y\join y')
            && \text{by \pref{item:boo_char_Ba_xxyy}}
          \\&= (y\join y)\join y'
            && \text{by identity 2 (\prope{associative} property)}
          \\&= y\join y'
            && \text{by \pref{item:boo_char_Ba_idempotent}}
          \\&= z\join z'
            && \text{by \pref{item:boo_char_Ba_xxyy}}
        \end{align*}

      \item Proof that $x\join(x'\join y)'=x$: \label{item:boo_char_Ba_xxyx}
            by \pref{item:boo_char_Ba_xxyzz} and identity 3

      \item Proof that $x\join y''\join(x\join y)'=z \join z'$: \label{item:boo_char_Ba_xyxyzz}
        \begin{align*}
          x\join y''\join(x\join y)'
            &= x\join y\join(x\join y)'
            && \text{by \pref{item:boo_char_Ba_double_complement}}
          \\&= z\join z'
            && \text{by \pref{item:boo_char_Ba_xxyy}}
        \end{align*}

      \item Proof that $x\join(x\join y)'=x\join y'$: \label{item:boo_char_Ba_xxyxy}
        \begin{align*}
          x\join(x\join y)'
            &= x \join (x\join y)' \join y'
            && \text{by \pref{item:boo_char_Ba_xyxyzz} and identity 3}
          \\&= x\join y' \join (x\join y)'
            && \text{by identity 1 (\prope{commutative} property)}
          \\&= x\join y' \join \brs{(x\join y')'z}
            && \text{by \pref{item:boo_char_Ba_xxyx}}
          \\&= x\join y'
            && \text{by \pref{item:boo_char_Ba_xxyx}}
        \end{align*}

      \item Proof that $\brs{\brp{x'\join y'}'\join\brp{x'\join y}'}\join x'=z\join z'$: \label{item:boo_char_Ba_zz}
        \begin{align*}
          \brs{\brp{x'\join y'}' \join\brp{x'\join y}'}\join x'
            &= x' \join \brs{\brp{x'\join y'}' \join\brp{x'\join y}'}
            && \text{by identity 1 (\prope{commutative} property)}
          \\&= \brs{x' \join \brp{x'\join y'}'} \join\brp{x'\join y}'
            && \text{by identity 2 (\prope{associative} property)}
          \\&= \brp{x'\join y''}\join\brp{x'\join y}'
            && \text{by \pref{item:boo_char_Ba_xxyxy}}
          \\&= \brp{x'\join y}\join\brp{x'\join y}'
            && \text{by \pref{item:boo_char_Ba_double_complement} (\prope{involutory})}
          \\&= z\join z'
            && \text{by \pref{item:boo_char_Ba_xxyy}}
        \end{align*}

      \item Proof that $\brp{x'\join y'}'\join\brp{x'\join y}'=x$ (\prope{Huntington's axiom}): \label{item:boo_char_Ba_ha}
        \begin{align*}
          \mcom{\brp{x'\join y'}'\join\brp{x'\join y}'}{``$x$" in identity 3}
            &= \mcom{(x'\join y')' \join (x'\join y)'}{``$x$" in identity 3} \join \mcom{x}{``$y$"}
            && \text{by \pref{item:boo_char_Ba_zz} and identity 3}
          %\\&= \mcom{x}{``$y$"} \join \mcom{(x'\join y')' \join (x'\join y)'}{``$x$" in identity 3}
          %  && \text{by \pref{item:boo_char_Ba_zz} and identity 3}
          \\&= \mcom{x \join (x'\join y)'}{$x$ by \pref{item:boo_char_Ba_xxyx}} \join (x'\join y')'
            && \text{by identity 1 (\prope{commutative} property)}
          \\&= \mcom{x \join (x'\join y')'}{$x$ by \pref{item:boo_char_Ba_xxyx}}
            && \text{by \pref{item:boo_char_Ba_xxyx}}
          \\&= x
            && \text{by \pref{item:boo_char_Ba_xxyx}}
        \end{align*}

      \item The three identities therefore imply that $\algA$
        \begin{enumerate}
          \item is \prope{idempotent}  (\pref{item:boo_char_Ba_idempotent}),
          \item is \prope{commutative} (identity 1),
          \item is \prope{associative} (identity 2), and
          \item satisfies \prope{Huntington's axiom} (\pref{item:boo_char_Ba_ha}).
        \end{enumerate}
        Therefore, by \prefp{prop:boo_char_H4} (\thme{Huntington's Fourth Set}),
        $\algA$ is a \prope{Boolean algebra}.

    \end{enumerate}
\end{enumerate}
\end{proof}


%---------------------------------------
\begin{proposition}[\thm{Byrne's \textsc{Formulation B}}]
\label{prop:boo_char_Bb}
\citetbl{
  \citorpc{byrne1946}{271}{\textsc{``Formulation B"}}
  }
%---------------------------------------
Let $\algA\eqd\latticed$ be an \emph{algebraic structure}.
\propboxt{
$\algA$ is a \structb{Boolean algebra} \quad$\iff$
\\\indentx$\brb{\begin{array}{Flcl @{\qquad}C@{\qquad}D@{\qquad}D}
     1. & x \join  y' = z\join z' &\iff&  x\join y = x           & \forall x,y,z\in\setX &  & and \\
     2. & \brp{x\join y}\join z   &=&     \brp{y\join z}\join x  & \forall x,y,z\in\setX.&  &
\end{array}}$}
\end{proposition}

%---------------------------------------
\begin{theorem}
\citetbl{
  \citorppcu{sholander1951}{28}{29}{$P1,\, P2,\, P3^\ast$}{http://books.google.com/books?hl=en\&lr=\&id=dKDdYkMCfAIC\&pg=PA28}
  }
\label{thm:boo_char_s1951}
%---------------------------------------
Let $\latA\eqd\latticed$ be an \hie{algebraic structure}.
\thmboxt{
  $\latA$ is a \structb{Boolean algebra} \quad$\iff$
  \\\indentx$\brb{\begin{array}{FlclCD}
       1. & x \meet (x\join y)  &=& x                             & \forall x,y\in\setX   & and \\
       2. & x \meet (y \join z) &=& (z \meet x) \join (y \meet x) & \forall x,y,z\in\setX & and \\
       3. & \exists y' \st x \meet (y \join y')&=& x \join (y\meet y')  & \forall x,y\in\setX.
     \end{array}}$
  }
\end{theorem}
\begin{proof}
\begin{enumerate}
  \item Proof that $\latA$ is a \prope{distributive} \emph{lattice}:\label{item:boo_char_s1951_latd}
        by 1 and 2 and by \prefpp{thm:lat_char_s1951}.

  \item Define $\bzero\eqd x\meet x'$ and $\bid\eqd x\join x'$. \label{item:boo_char_s1951_01}

  \item Proof that $\bzero$ is the \prope{join-identity} element 
        and   that $\bid$   is the \prope{meet-identity} element:\label{item:boo_char_s1951_id}
    \begin{align*}
      x \join \bzero
        &= x \join \brp{y\meet y'}
        && \text{by definition of $\bzero$ \xref{item:boo_char_s1951_01}}
      \\&= \brp{x\join x} \join \brp{y\meet y'}
        && \text{by \prope{idempotent} property of lattices \xref{thm:lattice}}
      \\&= x\join \brs{x \join \brp{y\meet y'}}
        && \text{by \prope{associative} property of lattices \xref{thm:lattice}}
      \\&= x\join \brs{x \meet \brp{y\join y'}}
        && \text{by 3}
      \\&= x
        && \text{by \prope{absorptive} property of lattices \xref{thm:lattice}}
      \\
      \\
      x \meet \bid
        &= x \meet \brp{y\join y'}
        && \text{by definition of $\bid$ \xref{item:boo_char_s1951_01}}
      \\&= \brp{x\meet x} \meet \brp{y\join y'}
        && \text{by \prope{idempotent} property of lattices \xref{thm:lattice}}
      \\&= x\meet \brs{x \meet \brp{y\join y'}}
        && \text{by \prope{associative} property of lattices \xref{thm:lattice}}
      \\&= x\meet \brs{x \join \brp{y\meet y'}}
        && \text{by 3}
      \\&= x
        && \text{by \prope{absorptive} property of lattices \xref{thm:lattice}}
    \end{align*}

  \item Proof that $\latA$ is \prope{bounded} with 
        $\bzero$ being the \hie{greatest lower bound} and 
        $\bid$ being the \hie{least upper bound}:\label{item:boo_char_s1951_bnd}
    \begin{align*}
      x \meet \bzero
        &= \brp{x \join \bzero} \meet \bzero
        && \text{by \prope{identity} property \xref{item:boo_char_s1951_id}}
      \\&= \bzero \meet \brp{\bzero \join x}
        && \text{by \prope{commutative} property of lattices \xref{thm:lattice}}
      \\&= \bzero
        && \text{by \prope{absorptive} property of lattices \xref{thm:lattice}}
      \\
      \\
      x \join \bid
        &= \brp{x \meet \bid} \join \bid
        && \text{by \prope{identity} property \xref{item:boo_char_s1951_id}}
      \\&= \bid \join \brp{\bid \meet x}
        && \text{by \prope{commutative} property of lattices \xref{thm:lattice}}
      \\&= \bid
        && \text{by \prope{absorptive} property of lattices \xref{thm:lattice}}
    \end{align*}

  \item Proof that $\latA$ is \prope{complemented}:\label{item:boo_char_s1951_latc}
    Because $\latA$ is \prope{bounded} with greatest lower bound $\bzero$ and 
    least upper bound $\bid$ (\pref{item:boo_char_s1951_bnd}) and because 
    $x\meet x'=\bzero$ and $x\join x'=\bid$ 
    (definition of $\bzero$ and $\bid$ \xref{item:boo_char_s1951_01}).

  \item Proof that $\latA$ is a \prope{Boolean} \emph{algebra}:
    Because $\latA$ is \prope{distributive} (\pref{item:boo_char_s1951_latd}) and 
    \prope{complemented} (\pref{item:boo_char_s1951_latc}), 
    and by \prefpp{def:boolean}.
\end{enumerate}
\end{proof}



%=======================================
\section{Literature}
%=======================================
\begin{survey}
\begin{enumerate}
  \item General information about Boolean algebras:
    \\\citer{sikorski1969}
    \\\citer{dwinger1971}
      \\\indentdr\citer{dwinger1961}
    \\\citer{halmos1972}
    \\\citer{monk1989}
    \\\citer{givant2009}

  \item Characterizations:
    \begin{enumerate}
      \item Survey of characterizations:
        \\\citer{padmanabhan2008}

      \item Characterizations in terms of traditional \prope{binary} operations
            \ope{join} $\join$, \ope{meet} $\meet$, and \ope{complement} $\bnot$:
        \\\citePc{huntington1904}
        \\\citorc{huntington1933}
        \\\citor {diamond1933}
        \\\citor {diamond1934}
        \\\citor {stone1935}
        \\\citor {hoberman1937}
        \\\citorc{frink1941}{4 identities involving $\join$, $\meet$, $\bnot$}
        \\\citor {newman1941}
        \\\citor {braithwaite1942}
        \\\citorc{byrne1946}{Form. A and B}
        \\\citerc{gerrish1978}{independence of Huntington's characterizations}
      
      \item Characterizations in terms of non-traditional \prope{binary} operations:
        \\\citorc{sheffer1913}{rejection $\rejection$}
        \\\citorc{bernstein1914}{exception $\exception$}
        \\\citorc{bernstein1916}{rejection $\rejection$}
        \\\citorc{bernstein1933}{rejection $\rejection$}
        \\\citorc{bernstein1934}{implication $\bimpl$}
        \\\citorc{bernstein1936}{complete disjuction $\bxor$}
        \\\citorc{byrne1948}{inclusion}
        \\\citorc{byrne1951}{ring operations}
        \\\citorc{miller1952}{ring operations}
      
      \item Characterizations in terms of \prope{ternary} operations:
        \\\citor {whiteman1937}{\ope{ternary rejection}}
      
      \item Characterizations involving \prope{Elkan's law}:
        \\\citorc{kondo2008}{for bounded lattices}
        \\\citorc{renedo2003}{for orthomodular lattices}
        \\\citorc{trillas2004}{for orthocomplemented lattices}

    \end{enumerate}

  \item Analytic properties:
    \\\citer{vladimirov2002}

  \item Miscellaneous:
    \\\citer {montague1954}
    \\\citerc{rudeanu1961}{referenced by \citer{sikorski1969}}

  \item Actually, ``Boolean algebras" are not really ``algebras". Rather, they are
    ``a commutative ring with unit, without nilpotents, and having idempotents which stood for classes"
    \\\citerp{hailperin1981}{184}

  \item Pioneering works related to Boolean algebras:
    \\\citor{boole1847}
    \\\citor{boole1854}
    \\\citorc{jevons1864}{join and meet operations}
    \\\citorc{peirce1870}{order concepts}
    \\\citePc{huntington1904}{axiomization}

  \item History of development of Boolean algebra:
    \\\citer{burris2000lbt}

\end{enumerate}
\end{survey}



