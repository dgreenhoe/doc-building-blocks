%============================================================================
% LaTeX File
% Daniel J. Greenhoe
%============================================================================

%======================================
\chapter{The Structure of Transforms}
%======================================


\paragraph{Homogeneous subspace topologies.}
Normally, the subspace topology is \hie{homogeneous}.
That is, each subspace in a space has the same topology.
Each subspace inherits the topological properties of the transform domain space.
Thus, in a normed linear space $\spX\eqd(\setX,\norm{\cdot})$, each
subspace $\spV_n$ inherits the topology generating norm $\norm{\cdot}$;
that is, $\spV_n\eqd(\setV_n,\norm{\cdot})$.
Thus, in each subspace, we can still characterize elements with respect to
basic properties such as proximity, convergence, and connectivity.


%---------------------------------------
% Euclidean 3-space partitioned by progressive lattice
%---------------------------------------
\begin{minipage}[c]{\tw/3}
%\begin{figure}[th]
\begin{center}
\footnotesize
\setlength{\unitlength}{\tw/400}%
\begin{picture}(400,520)(-200,0)%
  \thicklines
  %{\color{graphpaper}\graphpaper[50](-200,0)(400,520)}%
  {\color{black}%
    \put( -50,400){\framebox(100,100){$(\spX,\,\inprod{\cdot}{\cdot})$}}%
    \put(-125,250){\framebox(100,100){$(\spX,\,\norm{\cdot})$}}%
    \put(-200,100){\framebox(100,100){$(\spX,\,\fd(\cdot,\cdot))$}}%
    \put( -50,100){\framebox(100,100){$\spW_{\norm{\cdot}}$}}%
    \put( 100,100){\framebox(100,100){$\spW_{\inprod{\cdot}{\cdot}}$}}%
    \put( -25,  0){\framebox(50,50){$\opZero$}}%
    }%
  {\color{blue}%
    \put(   0,400){\line( 3,-4){150}}%
    \put(   0,400){\line(-3,-2){ 75}}%
    \put( -75,250){\line( 3,-2){ 75}}%
    \put( -75,250){\line(-3,-2){ 75}}%
    \put(   0, 50){\line(-3, 1){150}}%
    \put(   0, 50){\line( 0, 1){ 50}}%
    \put(   0, 50){\line( 3, 1){150}}%
    }%
\end{picture}
\end{center}
%\caption{
%  Euclidean 3-dimensional space partitioned as a progressive lattice
%  \label{fig:lat_E3d_power_progressive}
%  }
%\end{figure}
\end{minipage}
%\hfill
\begin{minipage}[c]{2\tw/3}
  %---------------------------------------
  \begin{example}
  \label{ex:lat_space_op}
  %---------------------------------------
  The figure to the left illustrates a decomposition of the powerful inner-product operator
  into less and less powerful operations and their complements.
\exbox{
  \left.
  \parbox[c][][c]{\textwidth/8}{\scriptsize\raggedright
    $(\spV,\inprod{\cdot}{\cdot})$
    is an inner product space
    }
  \right\}
  \iff
  \mcom{2\norm{\vx}^2 + 2\norm{\vy}^2 = \norm{\vx+\vy}^2 + \norm{\vx-\vy}^2}
       {parallelogram law \footnotemark}
  %\qquad\scriptstyle
  %\forall \vx,\vy \in \spV
  }
\end{example}
\end{minipage}
\footnotetext{\hie{parallelogram law}: \prefpp{thm:parallelogram}}









