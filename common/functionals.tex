%============================================================================
% LaTeX File
% Daniel J. Greenhoe
%============================================================================
%======================================
\chapter{Linear Functionals}
%=======================================
\begin{figure}[th]
  \centering
  \psset{xunit=\latunit,yunit=1.5\latunit}%
  %============================================================================
% Daniel J. Greenhoe
% LaTeX file
% nominal xunit = 10mm
% nominal yunit = 15mm
%============================================================================
\begin{pspicture}(-7.5,1.5)(7.5,8.5)%
  %-------------------------------------
  % settings
  %-------------------------------------
   \footnotesize
   \psset{%
       linecolor=blue,
      %linewidth=1pt,
       gridcolor=graph,
       subgriddiv=1,
       cornersize=relative,
       framearc=0.25,
       gridcolor=graph,
       subgriddiv=1,
       gridlabels=4pt,
       gridwidth=0.2pt,
     }%
  %-------------------------------------
  % nodes
  %-------------------------------------
  \begin{tabstr}{0.75}%
   \rput(  0, 8){\rnode{relations}     {\psframebox{\begin{tabular}{c}relations\\\ifxref{relation}{def:relation}\end{tabular}}}}%
   \rput(  0, 7){\rnode{functions}     {\psframebox{\begin{tabular}{c}functions\\\ifxref{relation}{def:functions}\end{tabular}}}}%
   \rput(  0, 6){\rnode{operators}     {\psframebox{\begin{tabular}{c}operators\\\xref{def:operator}\end{tabular}}}}%
   \rput(  0, 5){\rnode{functionals}   {\psframebox{\begin{tabular}{c}functionals\\\xref{def:functional}\end{tabular}}}}%
  %\rput(-8, 4){\rnode{nlforms}       {\psframebox{\begin{tabular}{c}nonlinear functionals\end{tabular}}}}%
   \rput(-5, 4){\rnode{lforms}        {\psframebox{\begin{tabular}{c}linear functionals\\\xref{def:functional_prop}\end{tabular}}}}%
   \rput(  0, 4){\rnode{sforms}        {\psframebox{\begin{tabular}{c}sesquilinear functionals\\\xref{def:sform}\end{tabular}}}}%
   \rput( 5, 4){\rnode{bforms}        {\psframebox{\begin{tabular}{c}bilinear functionals\\\xref{def:bform}\end{tabular}}}}%
   \rput(-2, 3){\rnode{hforms}        {\psframebox{\begin{tabular}{c}hermetian forms\\\xref{def:bform_prop}\end{tabular}}}}        
   \rput( 2, 3){\rnode{qforms}        {\psframebox{\begin{tabular}{c}quadratic forms\\\xref{def:qform}\end{tabular}}}}%
   \rput(-2, 2){\rnode{inprod}        {\psframebox{\begin{tabular}{c}inner-products\\\ifxref{vsinprod}{def:inprod}\end{tabular}}}}         
   \rput( 2, 2){\rnode{norms}         {\psframebox{\begin{tabular}{c}norms\\\ifxref{vsnorm}{def:norm}\end{tabular}}}}%
  %\rput(  0, 1){\rnode{zero}          {\psframebox{\begin{tabular}{c}$0$\end{tabular}}}}  
  \end{tabstr}%
   %
  %-------------------------------------
  % connecting lines
  %-------------------------------------
   \ncline{relations}    {functions}
   \ncline{functions}    {operators}
   \ncline{operators}    {functionals}
   \ncline{functionals}  {nlforms}   
   \ncline{functionals}  {lforms}   
   \ncline{functionals}  {sforms}   
   \ncline{functionals}  {bforms}   
   %
   \ncline{sforms}{hforms}          \ncline{sforms}       {qforms}
   \ncline{hforms}{inprod}          \ncline{qforms}       {norms}
   %\ncline{inprod}{zero}            \ncline{norms}        {zero}%
   %\ncline{lforms}{zero}            \ncline{bforms}        {zero}%
   %
  %-------------------------------------
  % annotation
  %-------------------------------------
   %\psccurve[linestyle=dashed,linecolor=red,fillstyle=none]%
   %  (0,-5)(20,4)(70,60)(20,60)(15,55)(-5,38)(-20,35)(-10,20)(-25,5)%
   %\psline[linecolor=red]{->}(60,75)(60,68)%
   %\uput[135](60,75){complete spaces}%
   %\psline[linecolor=red]{->}(26,80)(15,74)%
   %\psline[linecolor=red]{->}(30,80)(30,62.5)%
   %\uput[90](28,80){analytic spaces}%
   %
   %\psgrid[unit=10\psunit](-11,-1)(8,9)%
\end{pspicture}%
  \caption{Relations\label{fig:vector_functionals}}
\end{figure}%

%=======================================
\section{Definitions}
%=======================================
%--------------------------------------
\begin{definition}
\footnote{
  \citerpg{bachman1966}{5}{0486402517},
  \citerppgc{michel1993}{109}{114}{048667598X}{Definitions 3.5.1, 3.6.1}
  }
\label{def:functional}
%--------------------------------------
Let $\spO\eqd\linearspaceX$ be a \structe{linear space} \xref{def:vspace}.
Let $\clFxf$ be the set of all functions from the set $\setX$ into the \structe{field} \ifxref{algebra}{def:field} $\F$.
\defbox{\begin{array}{>{\qquad}FrclCD}
  \mc{5}{M}{A function $\ff$ is a \hid{functional} on $\spO$ if\quad $\ff$ is in $\clFxf$.}\\
\end{array}}
\end{definition}

%--------------------------------------
\begin{definition}
\footnote{
  \citerppgc{michel1993}{109}{114}{048667598X}{Definitions 3.5.1, 3.6.1},
  \citerppgc{nikolskij1992}{109}{110}{3540505849}{subadditive property}
  }
\label{def:functional_prop}
%--------------------------------------
Let $\ff\in\clFxf$ be a \structe{functional} on a \structe{linear space} $\spO\eqd\linearspaceX$.
\defbox{\begin{array}{MrclC}
  $\ff$ is \hid{linear} if             & \ff(\alpha\vx+\beta\vy)   &=& \alpha\ff(\vx) + \beta\ff(\vy)                       & \forall \vx,\vy\in\setX,\, \forall\alpha,\beta\in\F .
  \\
  $\ff$ is a \hid{conjugate linear} if & \ff(\alpha\vx+\beta\vy)   &=& \overline{\alpha}\ff(\vx) + \overline{\beta}\ff(\vy) & \forall \vx,\vy\in\setX,\, \forall\alpha,\beta\in\F .
  \\
  $\ff$ is \hid{subaddtive} if         & \ff(\vx+\vy)              &\le&\ff(\vx)+\ff(\vy)                                   &\forall\vx,\vy\in\setX .
\end{array}}
\end{definition}

%--------------------------------------
\begin{definition}
\label{def:bform}
\label{def:sform}
\footnote{
  \citerpg{kubrusly2001}{312}{0817641742},
  \citerpg{brown1995}{58}{9780387943695}
  %\citerpgc{michel1993}{114}{048667598X}{Definition 3.6.4}
  }
%--------------------------------------
Let $\spO\eqd\linearspaceX$ be a \structe{linear space} \xref{def:vspace}.
%\defbox{\begin{array}{>{\qquad}FMrclMD}
\defbox{\begin{array}{>{\qquad}FrclCD}
  \mc{6}{M}{A function $\fb\in\clFxxf$ is a \hid{bilinear functional} or \hid{bilinear form} on $\clFxxf$ if }\\
      1. & \fb(\alpha\vx+\beta\vy,\vu) &=& \alpha\fb(\vx,\vu) + \beta\fb(\vy,\vu) & \forall \vx,\vy,\vu\in\setX,\,\alpha,\beta\in\F & (\prope{linear} in first variable)  and
    \\2. & \fb(\vu,\alpha\vx+\beta\vy) &=& \alpha\fb(\vu,\vx) + \beta\fb(\vu,\vy) & \forall \vx,\vy,\vu\in\setX,\,\alpha,\beta\in\F & (\prope{linear} in second variable).
   %\\3. & for each fixed $\vy$, & \ff(\vx)&\eqd&\fb(\vx,\vy)& is a \prope{linear} functional & and
   %\\4. & for each fixed $\vx$, & \fg(\vy)&\eqd&\fb(\vx,\vy)& is a \prope{linear} functional.
  \\
  \mc{6}{M}{A function $\fs\in\clFxxf$ is a \hid{sesquilinear functional} or \hid{sesquilinear form} on $\clFxxf$ if \footnotemark}\\
      1. & \fb(\alpha\vx+\beta\vy,\vu) &=& \alpha\fb(\vu,\vx)  + \beta\fb(\vu,\vy)  & \forall \vx,\vy,\vu\in\setX,\,\alpha,\beta\in\F & (\prope{linear} in first variable)  and
    \\2. & \fb(\vu,\alpha\vx+\beta\vy) &=& \alphac\fb(\vu,\vx) + \betac\fb(\vu,\vy) & \forall \vx,\vy,\vu\in\setX,\,\alpha,\beta\in\F & (\prope{conjugate linear} in second variable).
   %  1. & for each fixed $\vy$, & \ff(\vx)&\eqd&\fs(\vx,\vy)& is a \prope{linear}    functional & and
   %\\2. & for each fixed $\vx$, & \fg(\vy)&\eqd&\fs(\vx,\vy)& is a \prope{conjugate linear} functional.
\end{array}}
\end{definition}
\footnotetext{The prefix \hie{sesqui--} is Latin for ``one and a half".
Reference: \citerc{merriam}{\url{http://www.merriam-webster.com/dictionary/sesqui-}}
}

%%--------------------------------------
%\begin{proposition}
%\label{prop:bform}
%\footnote{
%  \citerpgc{michel1993}{114}{048667598X}{Definition 3.6.4}\\
%  \citerpgc{debnath2005}{151}{0080455921}{Theorem 4.3.1}
%  }
%%--------------------------------------
%Let $\spO\eqd\linearspaceX$ be a \structe{linear space} \xref{def:vspace}.
%\propbox{\begin{array}{>{\ds}l}
%  \text{$\fb\in\clFxxf$ is a \hid{bilinear functional}}
%  \\\qquad\iff
%  \brbl{\begin{array}{FrclCD}
%    \cnto & \fb(\alpha\vx+\beta\vy,\vz) &=& \alpha\fb(\vx,\vz) + \beta\fb(\vy,\vz) & \forall \vx,\vy,\vz\in\setX,\, \alpha,\beta\in\F & and
%    \cntn & \fb(\vx,\alpha\vy+\beta\vz) &=& \overline{\alpha}\fb(\vx,\vy) + \overline{\beta}\fb(\vx,\vz) & \forall \vx,\vy,\vz\in\setX,\, \alpha,\beta\in\F
%  \end{array}}
%\end{array}}
%\end{proposition}

%--------------------------------------
\begin{definition}
\label{def:bform_prop}
\footnote{
  \citerpg{kubrusly2001}{312}{0817641742},
  \citerpg{brown1995}{58}{9780387943695},
  \citerpgc{michel1993}{115}{048667598X}{Definition 3.6.10},
  \citerpgc{debnath2005}{152}{0080455921}{Theorem 4.3.5},
  \citerpgc{grove2002}{85}{0821820192}{item 4}
  }
%--------------------------------------
Let $\fphi\in\clFxxf$ be a \structe{bilinear functional} or a \fncte{sesquilinear functional} \xref{def:sform} on
a \structe{linear space} $\spO\eqd\linearspaceX$.
\defbox{\begin{array}{FMlC}
    1. & $\fphi$ is \hid{symmetric}           if & \fphi(\vx,\vy)=\fphi(\vy,\vx)             & \forall \vx,\vy\in\setX.
  \\2. & $\fphi$ is \hid{Hermitian symmetric} if & \fphi(\vx,\vy)=\overline{\fphi(\vy,\vx)}  & \forall \vx,\vy\in\setX.
  \\3. & $\fphi$ is \hid{nonnegative}         if & \fphi(\vx,\vy)\ge0                        & \forall \vx,\vy\in\setX.
  \\4. & $\fphi$ is \hid{positive}            if & \fphi(\vx,\vx)>0                          & \forall \vx\in\setX\setd0.
  \\5. & \mc{3}{M}{If $\fphi$ is \prope{Hermitian symmetric}, then $\fphi$ is a \hid{hermitian form}.}
\end{array}}
\end{definition}

%--------------------------------------
\begin{definition}
\label{def:qform}
\footnote{
  \citerpg{kubrusly2001}{312}{0817641742},
  \citerpgc{michel1993}{115}{048667598X}{Definition 3.6.11},
  \citerpgc{debnath2005}{152}{0080455921}{Theorem 4.3.6}
  }
%--------------------------------------
Let $\linearspaceX$ be a \structe{linear space} \xref{def:vspace}.
%Let $\fs$ be a \structe{sesquilinear functional} in $\clFxxf$.
%\defbox{\begin{array}{>{\qquad\scy}FMD}
\defbox{\begin{array}{>{\qquad}FMD}
  \mc{3}{M}{A function $\fq\in\clFxf$ is the \hid{quadratic form} induced (or generated) by $\fs\in\clFxxf$ if}\\
      1. & $\fs$ is a \structe{sesquilinear functional} & and
    \\2. & $\ds\fq(\vx)\eqd\fs(\vx,\vx) \qquad\scy \forall \vx\in\setX $
\end{array}}
\end{definition}

%=======================================
\section{Basic results}
%=======================================
%--------------------------------------
\begin{lemma}
\label{lem:qform}
%--------------------------------------
Let $\linearspaceX$ be a \structe{linear space} \xref{def:vspace}.
\lembox{
  \text{$\fq\in\clFxf$ is a \fncte{quadratic form}}
  \qquad\implies\qquad
  \brb{\begin{array}{FrclCD}
      1. & \fq(-\vx) &=& \fq(\vx) & \forall \vx\in\setX & and
    \\2. & \fq(i\vx) &=& \fq(\vx) & \forall \vx\in\setX.
  \end{array}}}
\end{lemma}
\begin{proof}
Let $\fs\in\clFxxf$ be the \structe{sesquilinear functional} in $\clFxxf$ that induces $\fq$.
\begin{align*}
  \fq(\vx)
    &= \fs(\vx,\vx)
  \\&= (-1)(-1)\fs(\vx,\vx)
  \\&= \fs((-1)\vx,\overline{(-1)}\vx)
  \\&= \fs(-\vx,-\vx)
  \\&= \fq(-\vx)
  \\
  \fq(\vx)
    &= \fs(\vx,\vx)
  \\&= (i)(-i)\fs(\vx,\vx)
  \\&= \fs(i\vx,\overline{(-i)}\vx)
  \\&= \fs(i\vx,i\vx)
  \\&= \fq(i\vx)
\end{align*}
\end{proof}



%--------------------------------------
\begin{theorem}
\label{thm:bform_bipolar}
\footnote{
  \citerppgc{michel1993}{115}{116}{048667598X}{Theorem 3.6.12}
  }
%--------------------------------------
Let $\spO\eqd\linearspaceX$ be a \structe{linear space} \xref{def:vspace}.
Let $\fq\in\clFxf$ be the \fncte{quadratic form} \xref{def:qform} induced by a {function} $\fs\in\clFxxf$.
\thmbox{\begin{array}{>{\qquad}rclC}
  \mc{4}{M}{$\fs$ is a \structe{sesquilinear functional} $\quad\implies$}\\
  2\fs(\vx,\vy)+2\fs(\vy,\vx) &=& \fq(\vx+\vy)-\fq(\vx-\vy) & \forall \vx,\vy\in\setX
\end{array}}
\end{theorem}
\begin{proof}
  %\item Proof that $4\fs(\vx,\vy) = \fq(\vx+\vy)-\fq(\vx-\vy)+i\fq(\vx+i\vy)-i\fq(\vx-i\vy)$:
    \begin{align*}
      \fq(\vx+\vy)-\fq(\vx-\vy)
        &= \fs(\vx+\vy,\vx+\vy)-\fs(\vx-\vy,\vx-\vy)
      \\&=   \mcom{\brb{\fs(\vx,\vx)+\fs(\vx,\vy)+\fs(\vy,\vx)+\fs(\vy,\vy)}}         {$\fs(\vx+ \vy,\vx+ \vy)$}
           - \mcom{\brb{\fs(\vx,\vx)-\fs(\vx,\vy)-\fs(\vy,\vx)+\fs(\vy,\vy)}}         {$\fs(\vx- \vy,\vx- \vy)$}
      \\&= \brb{\fs(\vx,\vy)+\fs(\vy,\vx)} - \brb{-\fs(\vx,\vy)-\fs(\vy,\vx)}
      \\&= 2\fs(\vx,\vy)+2\fs(\vy,\vx)
    \end{align*}
\end{proof}




%--------------------------------------
\begin{theorem}[\thmd{polarization identities}]
\label{thm:bform_polar}
\footnote{
  \citerppgc{brown1995}{62}{63}{9780387943695}{Problem P(ii)},
  \citerppgc{halmos1998}{13}{14}{0821813781}{Theorem 1},
  \citerpgc{michel1993}{116}{048667598X}{Theorem 3.6.13},
  \citerppgc{debnath2005}{152}{153}{0080455921}{Theorem 4.3.7}
  }
%--------------------------------------
Let $\linearspaceX$ be a \structe{linear space} \xref{def:vspace}.
Let $\fq\in\clFxf$ be the \fncte{quadratic form} \xref{def:qform} induced by a {function} $\fs\in\clFxxf$.
\thmbox{\begin{array}{F>{\qquad}rclC}
     1. & \mc{4}{M}{$\fs$ is a \structe{sesquilinear functional} $\quad\implies$}\\
        & 4\fs(\vx,\vy) &=& \fq(\vx+\vy)-\fq(\vx-\vy)+i\fq(\vx+i\vy)-i\fq(\vx-i\vy) & \forall \vx,\vy\in\setX
  %
   \\2. & \mc{4}{M}{$\fs$ is a \prope{Hermitian symmetric} {sesquilinear functional} and $\F=\R\quad\implies$}\\
        & 4\fs(\vx,\vy) &=& \fq(\vx+\vy)-\fq(\vx-\vy) & \forall \vx,\vy\in\setX
\end{array}}
\end{theorem}
\begin{proof}
\begin{enumerate}
  \item Proof that $4\fs(\vx,\vy) = \fq(\vx+\vy)-\fq(\vx-\vy)+i\fq(\vx+i\vy)-i\fq(\vx-i\vy)$:
    \begin{align*}
      &\fq(\vx+\vy)-\fq(\vx-\vy)+i\fq(\vx+i\vy)-i\fq(\vx-i\vy)
      \\&= \fs(\vx+\vy,\vx+\vy)-\fs(\vx-\vy,\vx-\vy)+i\fs(\vx+i\vy,\vx+i\vy)-i\fs(\vx-i\vy,\vx-i\vy)
      \\&=       \mcom{\brb{\fs(\vx,\vx)+\fs(\vx,\vy)+\fs(\vy,\vx)+\fs(\vy,\vy)}}         {$\fs(\vx+ \vy,\vx+ \vy)$}
               - \mcom{\brb{\fs(\vx,\vx)-\fs(\vx,\vy)-\fs(\vy,\vx)+\fs(\vy,\vy)}}         {$\fs(\vx- \vy,\vx- \vy)$}
      \\&\qquad+i\mcom{\brb{\fs(\vx,\vx)-i\fs(\vx,\vy)+i\fs(\vy,\vx)-i^2\fs(\vy,\vy)}}    {$\fs(\vx+i\vy,\vx+i\vy)$}
               -i\mcom{\brb{\fs(\vx,\vx)+i\fs(\vx,\vy)-i\fs(\vy,\vx)-i^2\fs(\vy,\vy)}}    {$\fs(\vx-i\vy,\vx-i\vy)$}
      \\&=       \brb{\fs(\vx,\vy)+\fs(\vy,\vx)}
               - \brb{-\fs(\vx,\vy)-\fs(\vy,\vx)}
               +i\brb{-i\fs(\vx,\vy)+i\fs(\vy,\vx)}
               -i\brb{+i\fs(\vx,\vy)-i\fs(\vy,\vx)}
      \\&=       \brb{\fs(\vx,\vy)+\fs(\vy,\vx)}
               + \brb{\fs(\vx,\vy)+\fs(\vy,\vx)}
               + \brb{\fs(\vx,\vy)- \fs(\vy,\vx)}
               + \brb{\fs(\vx,\vy) -\fs(\vy,\vx)}
      \\&=4\fs(\vx,\vy)
    \end{align*}

  \item Proof that $\fs$ is a \prope{Hermitian symmetric} {sesquilinear functional} and $\F=\R$ $\implies$ $4\fs(\vx,\vy) = \fq(\vx+\vy)-\fq(\vx-\vy)$:
    \begin{align*}
      \fq(\vx+\vy)-\fq(\vx-\vy)
        &= 2\fs(\vx,\vy)+2\fs(\vy,\vx)
        && \text{by \prefp{thm:bform_bipolar}}
      \\&= 2\fs(\vx,\vy)+2\overline{\fs(\vx,\vy)}
        && \text{by \prope{symmetric} hypothesis}
      \\&= 2\fs(\vx,\vy)+2\fs(\vx,\vy)
        && \text{by \prope{real} hypothesis}
      \\&= 4\fs(\vx,\vy)
    \end{align*}
\end{enumerate}
\end{proof}

%--------------------------------------
\begin{theorem}
%\label{thm:bform_polar}
\footnote{
  \citerpgc{michel1993}{117}{048667598X}{Theorem 3.6.15},
  \citerpgc{debnath2005}{153}{0080455921}{Corollary 4.3.8}
  }
%--------------------------------------
Let $\linearspaceX$ be a \structe{linear space}.
Let $\fq_1\in\clFxf$ be the \fncte{quadratic form} \xref{def:qform} induced by a {sesquilinear functional} \xref{def:sform} $\fs_1\in\clFxxf$
and $\fq_2\in\clFxf$ be the \fncte{quadratic form} induced by a {sesquilinear functional} $\fs_2\in\clFxxf$.
\thmbox{
  \brb{\fq_1(\vx)=\fq_2(\vx)\quad\scy\forall\vx\in\setX}
  \qquad\iff\qquad
  \brb{\fs_1(\vx,\vy)=\fs_2(\vx,\vy)\quad\scy\forall\vx,\vy\in\setX}
  }
\end{theorem}
\begin{proof}
\begin{enumerate}
  \item Proof that $\fq_1(\vx)=\fq_2(\vx)\implies\fs_1(\vx,\vy)=\fs_2(\vx,\vy)$:
    \begin{align*}
      \fs_1(\vx,\vy)
        &= \fq_1(\vx+\vy)-\fq_1(\vx-\vy)+i\fq_1(\vx+i\vy)-i\fq_1(\vx-i\vy)
        && \text{by \thme{polarization identity} \xref{thm:bform_polar}}
      \\&= \fq_2(\vx+\vy)-\fq_2(\vx-\vy)+i\fq_2(\vx+i\vy)-i\fq_2(\vx-i\vy)
        && \text{by left hypothesis}
      \\&= \fs_2(\vx,\vy)
    \end{align*}

  \item Proof that $\fq_1(\vx)=\fq_2(\vx)\implies\fs_1(\vx,\vy)=\fs_2(\vx,\vy)$:
    \begin{align*}
      \fq_1(\vx)
        &= \fs_1(\vx,\vx)
        && \text{by \prefp{def:qform}}
      \\&= \fs_2(\vx,\vx)
        && \text{by right hypothesis}
      \\&= \fq_2(\vx)
        && \text{by \prefp{def:qform}}
    \end{align*}
\end{enumerate}
\end{proof}

%--------------------------------------
\begin{theorem}
%\label{thm:bform_polar}
\footnote{
  \citerpg{kubrusly2001}{312}{0817641742},
  \citerppgc{michel1993}{115}{116}{048667598X}{Theorem 3.6.12},
  \citerpgc{debnath2005}{153}{0080455921}{Theorem 4.3.9}
  }
%--------------------------------------
Let $\spO\eqd\linearspaceX$ be a \structe{linear space}.
Let $\fq\in\clFxf$ be the \fncte{quadratic form} \xref{def:qform} induced by a \fncte{sesquilinear functional} \xref{def:sform} $\fs\in\clFxxf$.
\thmbox{
  \mcom{\fq\in\R}{$\fq$ is \prope{real}}
  \qquad\iff\qquad
  \mcom{\fs(\vx,\vy)=\overline{\fs(\vy,\vx)}}{$\fs$ is \prope{Hermitian symmetric}}
  }
\end{theorem}
\begin{proof}
\begin{enumerate}
  \item Proof that $\fq$ is \prope{real} $\implies$ $\fs$ is \prope{hermitian}:
  %\item Proof that $4\fs(\vx,\vy) = \fq(\vx+\vy)-\fq(\vx-\vy)+i\fq(\vx+i\vy)-i\fq(\vx-i\vy)$:
    \begin{align*}
      \overline{4\fs(\vy,\vx)}
        &= \overline{\fq(\vy+\vx)} - \overline{\fq(\vy-\vx)} + \overline{i\fq(\vy+i\vx)} - \overline{i\vq(\vy-i\vx)}
        && \text{by \thme{polarization id.} \xref{thm:bform_polar}}
      \\&= \fq(\vy+\vx) - \fq(\vy-\vx) - i\fq(\vy+i\vx) + i\vq(\vy-i\vx)
        && \text{by left hypothesis}
      \\&= \fq(\vy+\vx) - \fq(-\vy+\vx) - i\fq(-i\vy-i^2\vx) + i\vq(i\vy-i^2\vx)
        && \text{by \prefp{lem:qform}}
      \\&= \fq(\vx+\vy) - \fq(\vx-\vy) - i\fq(\vx-i\vy) + i\vq(\vx+i\vy)
      \\&= 4\fs(\vx,\vy)
        && \text{by \thme{polarization id.} \xref{thm:bform_polar}}
    \end{align*}

  \item Proof that $\fq$ is \prope{real} $\impliedby$ $\fs$ is \prope{Hermitian symmetric}:
    \begin{align*}
      \overline{\fq(\vx)}
        &= \overline{\fs(\vx,\vx)}
        && \text{by definition of $\fq$ \xref{def:qform}}
      \\&= \overline{\overline{\fs(\vx,\vx)}}
        && \text{by right hypothesis}
      \\&= \fs(\vx,\vx)
      \\&= \fq(\vx)
        && \text{by definition of $\fq$ \xref{def:qform}}
      \\&\implies\quad\fq\in\R
    \end{align*}
\end{enumerate}
\end{proof}


%%--------------------------------------
%\begin{theorem}[\thmd{Riesz Representation Theorem}]
%\footnote{
%  \citerpg{yosida1980}{90}{3540586547}\\
%  \citerppgc{schechter2002}{29}{30}{0821828959}{Theorem 2.1}\\
%  \citerpgc{kubrusly2001}{374}{0817641742}{Proposition 5.62}
%  }
%%--------------------------------------
%Let $\spH\eqd\HspaceX$ be a \structe{Hilbert space}
%and $\ff$ a \structe{bounded linear functional} in $\clBhf$.
%\thmbox{\begin{array}{>{\qquad}rclC}
%  \mc{4}{M}{For every $\ff\in\clBxf$, there exists a \prope{unique} $\vy$ such that}
%  \\\ff(\vx) &=& \inprod{\vx}{\vy} & \forall\vx\in\setX;
%  \\
%  \text{Moreover}\quad\normop{\ff}&=& \norm{\vy}
%\end{array}}
%\end{theorem}
%\begin{proof}
%\begin{enumerate}
%  \item Proof that $\vy$ is unique:
%    \begin{align*}
%                 && \ff(\vx) &= \inprod{\vx}{\vy}=\inprod{\vx}{\vz}
%      \\\implies &&    0     &= \inprod{\vx}{\vy}-\inprod{\vx}{\vz}
%      \\         &&          &= \inprod{\vx}{\vy-\vz}                  && \text{by \prefp{def:inprod}}
%      \\\implies &&    \vy   &= \vz                                    && \text{by \pref{thm:inprod_prop}}
%    \end{align*}
%
%  \item Proof that $\exists\vy\st\ff(\vx) = \inprod{\vx}{\vy}$:
%  \item Proof that $\normop{\ff}= \norm{\vy}$:
%    \begin{align*}
%      \normop{\ff}
%        &\eqd \sup_{\norm{\vx}\le1}\abs{\ff(\vx)}
%      \\&\ge  \abs{\ff\brp{\frac{\vy}{\norm{\vy}}}}
%      \\&=    \inprod{\frac{\vy}{\norm{\vy}}}{\vy}
%      \\&=    \frac{1}{\norm{\vy}}\:\inprod{\vy}{\vy}
%      \\&=    \norm{\vy}
%      \\
%      \normop{\ff}
%        &\eqd \sup_{\norm{\vx}\le1}\abs{\ff(\vx)}
%      \\&=    \sup_{\norm{\vx}\le1}\abs{\inprod{\vx}{\vy}}
%      \\&\le  \sup_{\norm{\vx}\le1} \,\norm{\vx}\,\norm{\vy}
%      \\&=    \norm{\vy}
%    \end{align*}
%\end{enumerate}
%\end{proof}


%--------------------------------------
\begin{theorem}[\thmd{Riesz Representation Theorem}]
\footnote{
  \citerpg{yosida1980}{90}{3540586547},
  \citerppgc{schechter2002}{29}{30}{0821828959}{Theorem 2.1},
  \citerpgc{kubrusly2001}{374}{0817641742}{Proposition 5.62}
  }
%--------------------------------------
Let $\spH\eqd\HspaceX$ be a \structe{Hilbert space} \xref{def:hilbert}
and $\ff$ a \fncte{bounded linear functional} in $\clBhf$.
\thmbox{\begin{array}{>{\qquad}rclC}
  \mc{4}{M}{For every $\ff\in\clBxf$, there exists a \prope{unique} $\vy$ such that}
  \\\ff(\vx) &=& \inprod{\vx}{\vy} & \forall\vx\in\setX;
  \\
  \text{Moreover}\quad\normop{\ff}&=& \norm{\vy}
\end{array}}
\end{theorem}
\begin{proof}
\begin{enumerate}
  \item Proof that $\vy$ is unique:
    \begin{align*}
                 && \ff(\vx) &= \inprod{\vx}{\vy}=\inprod{\vx}{\vz}
      \\\implies &&    0     &= \inprod{\vx}{\vy}-\inprod{\vx}{\vz}
      \\         &&          &= \inprod{\vx}{\vy-\vz}                  && \text{by \prefp{def:inprod}}
      \\\implies &&    \vy   &= \vz                                    && \text{by \pref{thm:inprod_prop}}
    \end{align*}

  \item Proof that $\exists\vy\st\ff(\vx) = \inprod{\vx}{\vy}$:
  \item Proof that $\normop{\ff}= \norm{\vy}$:
    \begin{align*}
      \normop{\ff}
        &\eqd \sup_{\norm{\vx}\le1}\abs{\ff(\vx)}
      \\&\ge  \abs{\ff\brp{\frac{\vy}{\norm{\vy}}}}
      \\&=    \inprod{\frac{\vy}{\norm{\vy}}}{\vy}
      \\&=    \frac{1}{\norm{\vy}}\:\inprod{\vy}{\vy}
      \\&=    \norm{\vy}
      \\
      \normop{\ff}
        &\eqd \sup_{\norm{\vx}\le1}\abs{\ff(\vx)}
      \\&=    \sup_{\norm{\vx}\le1}\abs{\inprod{\vx}{\vy}}
      \\&\le  \sup_{\norm{\vx}\le1} \,\norm{\vx}\,\norm{\vy}
      \\&=    \norm{\vy}
    \end{align*}
\end{enumerate}
\end{proof}

