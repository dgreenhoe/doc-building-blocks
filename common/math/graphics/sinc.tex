%============================================================================
% Daniel J. Greenhoe
% LaTeX file
% Dirichlet function D_1(x), tau=1 (period 1)
% nominal unit = 10mm
%            sin[(pi/tau)x]     sind[(360/2pi)(pi/tau)x]     sind[(180/tau)x]  
% sinc(x) = ---------------- = -------------------------  = -----------------
%                (pi/tau)x          (pi/tau)x                    (pi/tau)x 
% where sin(x)  is the sine function with argument in radians
% and   sind(x) is the sine function with argument in degrees.
%============================================================================
\begin{pspicture}(-3.5,-0.5)(3.5,1.5)% N=1
  %-------------------------------------
  % definitions
  %-------------------------------------
  %\newcommand{\pi}{3.1415926535897932384626433832795}
  %-------------------------------------
  % axes
  %-------------------------------------
  \psaxes[linewidth=0.75pt,linecolor=axis,labels=y]{<->}(0,0)(-3.5,-0.5)(3.5,1.5)%
  %-------------------------------------
  % annotation
  %-------------------------------------
  \rput[b]( 1,-4.3mm){$\tau$}%
  \rput[b]( 2,-4.3mm){$2\tau$}%
  \rput[b]( 3,-4.3mm){$3\tau$}%
  \rput[b](-1,-4.3mm){$-\tau$}%
  \rput[b](-2,-4.3mm){$-2\tau$}%
  \rput[b](-3,-4.3mm){$-3\tau$}%
  %-------------------------------------
  % sinc(x)
  %-------------------------------------
  \psplot[plotpoints=1024]{-3.25}{3.25}{x 3.1416 mul RadtoDeg sin x 3.1416 mul div}%
  \psplot[plotpoints=64,linestyle=dotted]{-3.25}{-3.5}{x 3.1416 mul RadtoDeg sin x 3.1416 mul div}%
  \psplot[plotpoints=64,linestyle=dotted]{3.25}{3.5}{x 3.1416 mul RadtoDeg sin x 3.1416 mul div}%
\end{pspicture}%
