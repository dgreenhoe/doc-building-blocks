%============================================================================
% LaTeX File
% Daniel J. Greenhoe
%============================================================================

%=======================================
\chapter*{Preface}
\addcontentsline{toc}{section}{Preface}
\markboth{Preface}{Preface}
%=======================================

\begin{personal}

%=======================================
\section*{Writing for understanding}
%=======================================
\qboxnps
  {
    \href{http://en.wikipedia.org/wiki/Igor_Stravinsky}{Igor Fyodorovich Stravinsky}
    (1882--1971), Russian-born composer
    \index{Stravinsky, Igor}
    \index{quotes!Stravinsky, Igor}
    \footnotemark
  }
  {../common/people/small/stravinsky.jpg}
  {The uninitiated imagine that one must await inspiration in order to create.
   That is a mistake.
   I am far from saying that there is no such thing as inspiration;
   quite the opposite.
   It is found as a driving force in every kind of human activity, 
   and is in no wise peculiar to artists.
   But that force is brought into action by an effort, 
   and that effort is work.
   Just as appetite comes by eating so work brings inspiration, 
   if inspiration is not discernible at the beginning.
   %But it is not simply inspiration that counts:
   %it is the result of inspiration---
  }
  \citetblt{
    quote: \citerp{ewen1961}{408}, \citer{ewen1950}.
    %      & \citerp{deonaraine}{11} \\
    %      & \url{http://www3.thinkexist.com/quotes/igor_stravinsky/} \\
    image: \url{http://en.wikipedia.org/wiki/Image:Igor_Stravinsky.jpg}
  }

\qboxnps
  {
    \href{http://en.wikipedia.org/wiki/Blaise_Pascal}{Blaise Pascal} 
    \href{http://www-history.mcs.st-andrews.ac.uk/Timelines/TimelineC.html}{(1623--1662)}, 
    \href{http://www-history.mcs.st-andrews.ac.uk/Countries/France.html}{French} mathematician
    \index{Pascal, Blaise}
    \index{quotes!Pascal, Blaise}
    \footnotemark
  }
  {../common/people/small/pascal.jpg}
  {The last thing one settles in writing a book is what one should put in first.}
  \citetblt{
    quote: \citerc{pascal1670}{chapter I, note 19}.
          %& \url{http://www-groups.dcs.st-and.ac.uk/~history/Quotations/Pascal.html} \\
    image: \url{http://en.wikipedia.org/wiki/Image:Blaise_pascal.jpg}
    }


  I began this text while a Ph.D. student at 
  National Chiao-Tung University in the Telecommunications Department.
  I have found that two of the best ways to really understand a subject
  is to teach it to someone else or to write about it. 
  One reason for writing this text is to help myself more deeply understand
  mathematical structure.
  Sometimes I will think that I understand something deeply 
  enough until I start to try to write about it. 
  But it is the writing process that exposes what I don't understand 
  and forces me to really make sure I understand in a rigorous and complete way.
  
  
%=======================================
\section*{Writing for sharing}
%=======================================
\qboxnpqt
  {  
    \href{http://en.wikipedia.org/wiki/Housman}{Alfred Edward Housman}, English poet (1859--1936)
    \index{Housman, Alfred Edward}
    \index{quotes!Housman, Alfred Edward}
    \footnotemark
  }
  {../common/people/small/housman.jpg}
  {Here, on the level sand, \\
    Between the sea and land, \\
    What shall I build or write \\
    Against the fall of night?
  }
  {Tell me of runes to grave \\
    That hold the bursting wave, \\
    Or bastions to design \\
    For longer date than mine.
  }
  \citetblt{
    quote: \citerpc{housman1936}{64}{``Smooth Between Sea and Land"}, \citerc{hardy1940}{section 7}.
    image: \url{http://en.wikipedia.org/wiki/Image:Housman.jpg}
    }

\qboxnpq
  { \href{http://en.wikipedia.org/wiki/Bertrand_Russell}{Bertrand Russell}
    \href{http://www-history.mcs.st-andrews.ac.uk/Timelines/TimelineF.html}{(1872--1970)},
    \href{http://www-history.mcs.st-andrews.ac.uk/BirthplaceMaps/Places/UK.html}{British mathematician},
    in a 1962 November 23 letter to Dr. van Heijenoort.
    \index{Russull, Bertrand}
    \index{quotes!Russull, Bertrand}
    \footnotemark
  }
  {../common/people/small/russell.jpg}
  {As I think about acts of integrity and grace,
   I realise that there is nothing in my knowledge to compare with Frege's dedication to truth.
   His entire life's work was on the verge of completion,
   much of his work had been ignored to the benefit of men infinitely less capable,
   his second volume was about to be published, and upon finding that his fundamental assumption
   was in error, 
   he responded with intellectual pleasure clearly submerging any feelings of personal disappointment.
   It was almost superhuman and a telling indication of that of which men are capable if their
   dedication is to creative work and knowledge instead of cruder efforts to dominate and be 
   known.}
  \citetblt{
    quote: \citerp{heijenoort}{127}.
    image: \url{http://www-history.mcs.st-andrews.ac.uk/PictDisplay/Russell.html}.
    }

  However, I hope that this text does more than just benefit my own 
  understanding.
  I think that part of being human and the human experience is to share
  what we have with others just as we have benefited from what 
  others have shared---
  those whose dedication has been 
  ``to creative work and knowledge instead of cruder efforts to dominate and be 
   known."
  And to ``build or write" something that will last ``for longer date than mine."

%%=======================================
%\section*{Writing freely}
%%=======================================
%\qboxnps
%  {
%    \href{http://en.wikipedia.org/wiki/Georg_Cantor}{Georg Cantor} 
%    \href{http://www-history.mcs.st-andrews.ac.uk/Timelines/TimelineF.html}{(1845--1918)}, 
%    \href{http://www-history.mcs.st-andrews.ac.uk/BirthplaceMaps/Places/Germany.html}{German set theory pioneer}   
%    \index{Cantor, Georg}
%    \index{quotes!Cantor, Georg}
%    \footnotemark
%  }
%  {../common/people/small/cantor.jpg}
%  {The essence of mathematics lies entirely in its freedom.}
%  \footnotetext{\begin{tabular}[t]{ll}
%    quote: & \url{http://en.wikiquote.org/wiki/Georg_Cantor} \\
%    image: & \url{http://en.wikipedia.org/wiki/Image:Georg_Cantor.jpg}
%  \end{tabular}}
%
%
%
%  I also hope that this text can always be freely available in all parts of the world.
%  I am not against authors selling their works to publishers---
%  authors have families that they are responsible to provide for.
%  However in the traditional author/publisher relationship,
%  the author sells his ownership of his work to the publisher---
%  that is, the author no longer owns his own work, rather the publisher does.
%  And hence the work is no longer free to all but restricted to those who either have 
%  access to a library possessing the book---of which very few locations 
%  enjoy\footnote{As of December 2006, the international library cataloging system \hie{Worldcat}
%  could not find the classic text \citei{rudinp} in the countries of 
%  Cambodia, Vietnam, Philippines, Thailand, India, Pakistan, Romania, Poland, 
%  Ghana, or Kenya.
%  }---
%  or financial ability to purchase the book---
%  at ever increasing 
%  prices.\footnote{As of 2006 December, the list price of the classic text
%  \citei{rudinp} was US\$151.56 (\url{http://www.amazon.com/dp/007054235X}).
%  This is more than the \hie{International Monetary Fund}'s estimated annual (an entire year!) 2006
%  income of the average person in Burundi, which is US\$124.933;
%  and it is not much below the incomes of those in other countries as well:\\
%  \begin{tabular}[t]{ll|}
%    Malawi                   & 164.452  \\
%    Ethiopia                 & 176.686  \\
%    Guinea-Bissau            & 191.564  \\
%    Liberia                  & 197.375  \\
%    Sierre Leone             & 241.022  \\
%    Rwanda                   & 259.903  \\
%  \end{tabular}
%  \begin{tabular}[t]{ll|}
%    Niger                    & 263.754  \\
%    Madagascar               & 294.484  \\
%    Uganda                   & 314.984  \\
%    Gambia                   & 326.569  \\
%    Mozambique               & 337.702  \\
%    Guinea                   & 346.954  \\
%  \end{tabular}
%  \begin{tabular}[t]{ll}
%    Central African Republic & 352.228  \\
%    Afghanistan              & 354.008  \\
%    Bangladesh               & 406.965  \\
%    Cambodia                 & 459.284  \\
%    Burkina Faso             & 461.370  \\
%    Benin                    & 587.648  \\
%  \end{tabular}\\
%  (\url{http://www.imf.org/external/pubs/ft/weo/2006/02/data/index.aspx})
%  }
%  That is, the ability to aquire knowlege has become somewhat an aristocratic affair---
%  limited in some sense to those with ample fanancial resources or living in locations
%  of ample financial resources.
%  In addition to these concerns, there is the issue that the publisher often stops publishing 
%  the work after some time 
%  and it soon becomes very difficult for an interested reader
%  to purchase a copy.
%  I realize that internet access is currently also limited in the world;
%  but it is perhaps currently the best option.
%  It is my hope that through the use of the internet, 
%  this text can be freely available for many years to come.
%
%\qboxnpq
%  {
%    \href{http://en.wikipedia.org/wiki/Carl_Friedrich_Gauss}{Karl Friedrich Gauss}
%    \href{http://www-history.mcs.st-andrews.ac.uk/Timelines/TimelineE.html}{(1777--1855)},
%    \href{http://www-history.mcs.st-andrews.ac.uk/Countries/Germany.html}{German} 
%    mathematician and scientist  
%    in a September 2, 1808 letter to 
%    \href{http://en.wikipedia.org/wiki/Farkas_Bolyai}{Farkas Bolyai Bolyai}
%    \index{Gauss, Karl Friedrich}
%    \index{quotes!Gauss, Karl Friedrich}
%    \index{Bolyai, Farkas Bolyai}
%    \index{quotes!Bolyai, Farkas Bolyai}
%    \footnotemark
%  }
%  {../common/people/small/gauss.jpg}
%  {It is not knowledge, but the act of learning, 
%     not possession but the act of getting there, 
%     which grants the greatest enjoyment. 
%     When I have clarified and exhausted a subject, 
%     then I turn away from it, in order to go into darkness again. 
%     The never-satisfied man is so strange---
%     if he has completed a structure, 
%     then it is not in order to dwell in it peacefully, 
%     but in order to begin another. 
%     I imagine the world conqueror must feel thus, who, 
%     after one kingdom is scarcely conquered, 
%     stretches out his arms again for others.}
%  \citetblt{
%    quote \citerp{dunnington2004}{416}.
%    %quote: & \url{http://en.wikiquote.org/wiki/Karl_Friedrich_Gauss} \\
%    image: \url{http://en.wikipedia.org/wiki/Karl_Friedrich_Gauss}.
%    }
%
%%=======================================
%\section*{Typesetting Tools}
%%=======================================
%  This text was typeset using Xe\LaTeX.
%  
  
  
%=======================================
\section*{Writing with hope}
%=======================================
\parbox{2\tw/16}{%
  \includegraphics*[width=2\tw/16-2ex]{../common/people/small/dan.jpg}
  }%
\hfill
\parbox{13\tw/16}{%
  It is my hope that this text will be useful to at least a few people
  in their struggle to understand the structure of mathematics;
  and that these can in turn use their knowledge for the benefit of others.

  \textcolor{signature}{Daniel J. Greenhoe \zhthw{(柯晨光)}}
  }
 


\end{personal}