%============================================================================
% Daniel J. Greenhoe
% LaTeX file
%============================================================================
%=======================================
\subsection{Wavelet Analysis}
%=======================================
In this section, we use what is in \emph{essense} wavelet analysis,
but yet is not truly wavelet analysis in the strict sense:
\begin{enumerate}
  \item For starters, standard \fncte{wavelet}s and their associated \fncte{scaling function}s are not sequences \xref{def:sequence},
        but rather are functions with domain $\R$ (not $\Z$ or some convex subset of $\Z$).
  \item While it is true that the celebrated \ope{Fast Wavelet Transform} (FWT) does work \emph{internally} with sequences
        (using \ope{filter bank}s),\footnote{\citerppgc{greenhoe2013wsd}{360}{364}{0983801134}{J.6 Filter Banks}} 
        the FWT is actually defined to work on functions with domain $\R$;
        and so the function to be analyzed by the FWT must first be \ope{sampled} by 
        a \fncte{scaling function}, which yields a sequence that can be processed by the 
        \ope{filter bank}s.\footnote{\citerppgc{greenhoe2013wsd}{369}{372}{0983801134}{Appendix L}}
  \item Wavelet analysis is typically performed by translating the wavelet or scaling function 
        by fixed amounts depending on the ``scale" of the given wavelet.
        For example, a Haar wavelet of length 4000 would typically ``jump" in offsets of 4000:
        0, 4000, 8000, 12000, \ldots.\footnote{\citerppgc{greenhoe2013wsd}{27}{62}{0983801134}{Chapter 2. The Structure of Wavelets}}
        As one might imagine, this may be reason for concern if you are using this wavelet to 
        perform edge detection (you might jump over and miss detecting the edge).
        In this section, wavelet sequences are translated by offsets of 1, making an edge harder to miss.
\end{enumerate}
%=======================================
%\subsubsection{Wavelet analysis of die sequences}
%=======================================
%---------------------------------------
\begin{example}[\exmd{statistical edge detection using Haar wavelet on non-stationary die sequence}]
\label{ex:nonstat48}
\addcontentsline{toc}{subsubsection}{* statistical edge detection using Haar wavelet on non-stationary die sequence}
\mbox{}\\
%---------------------------------------
\begin{enumerate}
  \item \label{item:nonstat48_psp}
     Suppose we have a length $\xN\eqd12000$ die sequence $\seqn{x_n}$ with the following distribution:
     \\\indentx$\begin{array}{lD}
       \psp(\dieA)=\psp(\dieB)=\psp(\dieC)=\psp(\dieD)=\psp(\dieE)=\psp(\dieF)=\frac{1}{6}                             & for $n\in\intcc{0}{3999}\setu\intcc{8000}{11999}$\\
       \psp(\dieA)=\psp(\dieB)=\psp(\dieC)=\psp(\dieE)=\psp(\dieF)=\frac{1}{10} \;\text{and}\; \psp(\dieD)=\frac{1}{2} & for $n\in\intcc{4000}{7999}$
     \end{array}$\\
     That is, the distribution of the sequence is uniformly distributed in the first and last thirds, 
     but biased towards $\dieC$ in the middle third.
     In this example we use a simple statistical edge detector to try to find the statistical ``edges" at 4000 and 8000.
     The edge detector here is a 
     \ope{filter} operation $\opW$ \xref{def:filter} using a \fncte{length 200 Haar wavelet sequence} \xref{def:whaar}.\footnote{
     Empirical evidence due to \citeP{singh1997} suggests that the Haar wavelet performs better
     than several other common wavelets as an edge detector.}
  
  \item \label{item:nonstat48_R1pam}
    Suppose we first use the \fncte{PAM die random variable} \xref{def:rv_diepam} to map
    the sequence of \pref{item:nonstat48_psp} into $\R^1$.
    The magnitude of $\opW:\R^1\to\C^1$ of the mapped sequence is as follows:\footnote{
    Note that the plot in \pref{item:nonstat48_R1pam} 
    has been down sampled by a factor of 10 for practical reasons of displaying the very large data set.}
    \\\begin{tabular}{|>{\scs}c|}
         \hline
         \includegraphics{../common/math/sspplots/diehaarR1_12000m4000_h200_1050_D10.pdf}%
       \\\hline
    \end{tabular}\\
     We might expect to see strongest evidence of the edges at $4000+200/2=4100$ and $8100$.
     But looking at the above result, this is not apparent. 
     In fact, there are a total of 10646 values that are greater than or equal to the value at location 4100 
     (that value being $0.015$).\footnote{\seessp{diehaar_12000m4000_h200_1050.xlg}} 

  \item \label{item:nonstat48_C1qpsk}
    Suppose we next use the \fncte{QPSK die random variable} \xref{def:rv_dieqpsk} to map
    the die sequence into the complex plane.
    The magnitude of $\opW:\C^1\to\C^1$ of the mapped sequence is as follows:
    \\\begin{tabular}{|>{\scs}c|}
         \hline
         \includegraphics{../common/math/sspplots/diehaarC1_12000m4000_h200_1050_D10.pdf}%
       \\\hline
    \end{tabular}\\
     Using this method, the edges are apparent.   
     And the value of the peak at $n=4083$ (with value $0.223383$) is about
     %$10\log_{10}(0.223383/0.072097)\approx4.9$ dB above the noise floor.\footnote{\seessp{diehaar_12000m4000_h200_1050.xlg}} 
     $10\log_{10}(0.223383/0.072741)\approx4.9$ dB above the noise 
     floor.\footnote{\seessp{diehaar_12000m4000_h200_1050.xlg}} 
    Here, the noise value 0.072741 is the \fncte{RMS} (see \pref{item:dftebola_R1pam} of \prefp{ex:dftebola})
    of the DFT magnitude sequence computed over the domain $n=200\ldots\xN-1$.
     %4.9113292531261044618521061907862 dB
     %above the next highest peak, which occurs at $n=521$ (with value $0.183576$).
%max|W(C^1 sequence)| from n=4050..4150 is 0.223383:
%rms|W(C^1 sequence)| from n=200..11999 is 0.072744:
%4.8725294107136569466927663732011 dB

     
  \item \label{item:nonstat48_R4}
    Suppose we next use the \fncte{$\R^6$ die random variable} \xref{def:rv_dieR6} to map
    the sequence into $\R^6$.
    The magnitude of $\opW:\R^6\to\R^6$ of the mapped sequence is as follows:
     \\\begin{tabular}{|>{\scs}c|}
          \hline
          \includegraphics{../common/math/sspplots/diehaarR6_12000m4000_h200_1050_D10.pdf}%
        \\\hline
     \end{tabular}\\
     Using this method, the edges are also apparent.
     And the value of the peak at $n=4102$ (with value $0.209165$) is about 
     %$10\log_{10}(0.209165/0.068596)\approx4.8$ dB above the noise floor.
     $10\log_{10}(0.209165/0.065768)\approx5.0$ dB above the noise 
     floor.\footnote{Here the RMS noise value is computed over the domain $n=200\ldots\xN-1$.\\ 
       \seessp{diehaar_12000m4000_h200_1050.xlg}
       } 
     This is only a slight improvement over \pref{item:nonstat48_C1qpsk}.

%max|W(R^6 sequence)| from n=4050..4150 is 0.209165:
%rms|W(R^6 sequence)| from n=0..11999 is 0.068596:
%4.8419022323079719296397098112867 dB
%max|W(R^6 sequence)| from n=4050..4150 is 0.209165:
%rms|W(R^6 sequence)| from n=200..11999 is 0.065771:
%5.0245456986608778348245061173019 dB
\end{enumerate}
\end{example}


%=======================================
%\subsubsection{Wavelet analysis of DNA sequences}
%=======================================
%---------------------------------------
\begin{example}[\exmd{statistical edge detection using Haar wavelet on non-stationary artificial DNA sequence}]
\addcontentsline{toc}{subsubsection}{* statistical edge detection using Haar wavelet on non-stationary artificial DNA sequence}
\mbox{}\\
\label{ex:}
%---------------------------------------
\begin{enumerate}
  \item 
     Suppose we have a length $\xN\eqd12000$ dna sequence $\seqn{x_n}$ with the following distribution:
     \\\indentx$\begin{array}{lD}
       \psp(\dnaA)=\psp(\dnaC)=\psp(\dnaG)=\psp(\dnaT)=\frac{1}{4}                             & for $n\in\intcc{0}{3999}\setu\intcc{8000}{11999}$\\
       \psp(\dnaA)=\psp(\dnaG)=\psp(\dnaT)=\frac{17}{100} \;\text{and}\; \psp(\dnaC)=\frac{49}{100} & for $n\in\intcc{4000}{7999}$
     \end{array}$\\
     That is, the distribution of the sequence is uniformly distributed in the first and last thirds, 
     but biased towards $\dnaC$ in the middle third.
     Just as in \pref{ex:nonstat48}, we again use 
     a filter operation $\opW$ with \fncte{length 200 Haar wavelet sequence} as 
     a simple statistical edge detector to try to locate the statistical ``edges" at 4000 and 8000.
  
  \item \label{item:dna_haar_R1pam}
    Suppose we first use the \fncte{PAM DNA random variable} \xref{def:rv_dnapam} to map
    the DNA sequence into $\R^1$.
    The magnitude of $\opW:\R^1\to\C^1$ of the mapped sequence is as follows:\footnote{
    Note that the plot in \pref{item:dna_haar_R1pam} 
    has been down sampled by a factor of 10 for practical reasons of displaying the very large data set.}
     \\\begin{tabular}{|>{\scs}c|}
          \hline
          \includegraphics{../common/math/sspplots/dnahaarR1_12000m4000_h200_1749_D5.pdf}%
        \\\hline
     \end{tabular}\\
     We might expect to see strongest evidence of the edges at or near $4000+200/2=4100$ and $8100$.
     In fact, the sequence does have peaks at 4087 (with value 0.185) and at 8087 (with value 0.230).
     The peak at 4087 is about $10\log_{10}(0.185000/0.071355)\approx 4.1$ dB above the noise floor.
     %max|W(R^1 sequence)| from n=4050..4150 is 0.185000:
     %max|W(R^1 sequence)| from n=8050..8150 is 0.230000:
     %rms|W(R^1 sequence)| from n=200..11999 is 0.071355:
     However, there are 103 other values not around the $n=4087$ and $n=8087$ peaks that are 
  0.185 or greater.
  These 102 values represent roughly 11 other peaks, each of which could trigger a 
  ``false positive" decision.\footnote{\seessp{dnahaar_12000m4000_h200_1749.xlg}}
     
  \item \label{item:dna_haar_C1}
    Suppose we next use the \fncte{QPSK DNA random variable} \xref{def:rv_dieqpsk} to map
    the dna sequence into the complex plane.
    The magnitude of $\opW:\C^1\to\C^1$ of the mapped sequence is as follows:
     \\\begin{tabular}{|>{\scs}c|}
          \hline
          \includegraphics{../common/math/sspplots/dnahaarC1_12000m4000_h200_1749_D5.pdf}%
        \\\hline
     \end{tabular}\\
     Using this method, the edges are apparent.
     And the value of the peak at $n=4086$ (with value $0.215870$) is 
     $10\log_{10}(0.215870/0.070068)\approx 4.9$ dB above the noise floor.\footnote{\seessp{dnahaar_12000m4000_h200_1749.xlg}}
     %max|W(C^1 sequence)| from n=4050..4150 is 0.215870:
     %rms|W(C^1 sequence)| from n=200..11999 is 0.070068:
     %only 
     %$10\log_{10}(0.215870/0.190263)=0.548380$dB
     %above the next highest peak, which occurs at $n=10405$ (with value $0.190263$).\footnote{\seessp{dnahaar_12000m4000_h200_1749.xlg}}
     %However it does produce two false edges at 9815 and 10405---the value at 
     %$n=4086$ is $0.215870$.
     %$n=4100$ is $0.181108$,
     %but it is greater than or equal to this value at 9815, 10405, 10407, and 10408.\footnote{
     %The values are $\opair{9815}{0.181108}$, $\opair{10405}{0.190263}$, $\opair{10407}{0.185607}$,
     %and $\opair{10408}{0.186682}$.}
     % dnahaar_12000m4000_h200_1749_20160402_095901.log


  \item \label{item:dna_haar_R4}
    Suppose we next use the \fncte{$\R^4$ DNA random variable} \xref{def:rv_dnaR4} to map
    the sequence into $\R^4$.
    The magnitude of $\opW:\R^4\to\C^4$ of the mapped sequence is as follows:
     \\\begin{tabular}{|>{\scs}c|}
          \hline
          \includegraphics{../common/math/sspplots/dnahaarR4_12000m4000_h200_1749_D5.pdf}%
        \\\hline
     \end{tabular}\\
     Using this method, the edges are also apparent.
     And the value of the peak at $n=4096$ (with value $0.181246$) is 
     $10\log_{10}(0.181246/0.059594)\approx4.8$ dB
     above the noise floor.\footnote{\seessp{dnahaar_12000m4000_h200_1749.xlg}}
     Note that this is a slight decrease in performance as compared to \pref{item:dna_haar_C1}.

     %max|W(R^4 sequence)| from n=4050..4150 is 0.181246:
     %rms|W(R^4 sequence)| from n=200..11999 is 0.059594:
     %
     %$10\log_{10}(0.181246/0.141421)=1.077545$dB
     %above the next non-starting highest peak, which occurs at $n=9815$ (with value $0.141421$).
     %Note that this is a small improvement over \pref{item:dna_haar_C1}.
     %Unlike the complex mapping in \pref{item:dna_haar_C1},
     %it shows an edge near the start of the sequence.
     %But also unlike the method of \pref{item:dna_haar_C1}, it contains no false edges in that 
     %there are no other values greater than that around the 4100 peak, 8100 peak, and starting peak.
     %In fact, the 4100 peak is $10\log_{10}(0.181246/0.141421)=1.077545$dB above the next highest peak.\footnote{
     %Compare $\opair{4096}{0.181246}$, $\opair{4100}{0.168819}$, and $\opair{9815}{0.141421}$.
     %} 


\end{enumerate}
\end{example}

%---------------------------------------
\begin{example}[\exmd{Wavelet analysis of Phage Lambda DNA sequence}]
\label{ex:dnapl}
\addcontentsline{toc}{subsubsection}{* Wavelet analysis of Phage Lambda DNA sequence}
\mbox{}\\
%---------------------------------------
\begin{enumerate}
  \item \label{item:dnapl_psp}
     Consider the Phage Lambda DNA sequence.\footnote{\citeD{ncbiPhageLambda}}
     It has a strong $\dnaC\dnaG$ bias before $n=20000$ and 
            a strong $\dnaA\dnaT$ bias after,\footnote{\citerpg{cristianini2007}{14}{1139460153}}
     as demonstrated next by mapping 
     \\\indentx$\dnaA\to+1$\qquad$\dnaT\to+1$\qquad$\dnaC\to-1$\qquad$\dnaG\to-1$\\
     and filtering the resulting sequence in $\R^1$
     with a \fncte{length 1600 Haar scaling sequence} \xref{def:shaar}---such
     a filtering operation acts as a kind of ``sliding window" histogram of the DNA sequence.
     \\\begin{tabular}{|>{\scs}c|}
          \hline
          \includegraphics{../common/math/sspplots/dna_NC001416_phagelambdaR1b_hs1600.pdf}%
        \\\hline
     \end{tabular}\\
  
  \item \label{item:dnapl_R1pam}
    Suppose we first use the \fncte{PAM DNA random variable} \xref{def:rv_dnapam} to map
    the DNA sequence into $\R^1$.
    The magnitude of the length 1600 Haar wavelet operation on the mapped sequence is as follows:
     \\\begin{tabular}{|>{\scs}c|}
          \hline
          \includegraphics{../common/math/sspplots/dna_NC001416_phagelambdaR1_h1600.pdf}%
        \\\hline
     \end{tabular}\\
    Note that it is very difficult to pick out the edge at 20000.

  \item \label{item:dnapl_C1qpsk}
    Suppose we next use the \fncte{QPSK DNA random variable} \xref{def:rv_dnaqpsk} to map
    the DNA sequence into the complex plane.
    The length 1600 Haar wavelet operation on the mapped sequence is as follows:
     \\\begin{tabular}{|>{\scs}c|}
          \hline
          \includegraphics{../common/math/sspplots/dna_NC001416_phagelambdaC1_h1600.pdf}%
        \\\hline
     \end{tabular}\\
     If one did not know apriori that there was an edge at 20000, it would still be difficult to identify.
     
  \item \label{item:dnapl_R6}
    Suppose we next use the \fncte{$\R^4$ DNA random variable} \xref{def:rv_dnaR4} to map
    the DNA sequence into $\R^4$.
    Filtering the mapped sequence with a length 1600 Haar wavelet sequence results in the following:
     \\\begin{tabular}{|>{\scs}c|}
          \hline
          \includegraphics{../common/math/sspplots/dna_NC001416_phagelambdaR4_h1600.pdf}%
        \\\hline
     \end{tabular}\\
     Here there is a clear peak near $20000$.

  \item And here is the same analysis as used in \pref{item:dnapl_R6}, but at scale 4000
        (using a length 4000 Haar wavelet filter):
     \\\begin{tabular}{|>{\scs}c|}
          \hline
          \includegraphics{../common/math/sspplots/dna_NC001416_phagelambdaR4_h4000.pdf}%
        \\\hline
     \end{tabular}\\
     Again, the peak near 20000 is quite pronounced.
     However, at the low resolution scale (of 4000), it would be difficult to determine precisely where the
     statistical edge actually was.
\end{enumerate}
\end{example}



