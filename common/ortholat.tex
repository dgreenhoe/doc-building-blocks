%============================================================================
% Daniel J. Greenhoe
% XeLaTeX file
% Ortho lattice structures
%============================================================================


%=======================================
\chapter{Orthocomplemented Lattices}
%=======================================
\structe{Orthocomplemented lattice}s \xref{def:latoc} are a kind of generalization of \structe{Boolean algebra}s.
The relationship between lattices of several types, including orthocomplemented and Boolean lattices,
is stated in \prefpp{thm:latoc_types} and illustrated in \prefpp{fig:latortholat}.

\begin{figure}[th]
  \begin{center}
    {\psset{yunit=15mm}%============================================================================
% Daniel J. Greenhoe
% LaTeX file
% lattice ({factors of 30}, |)
% nominal unit = 15mm
%============================================================================
  \begin{pspicture}(-6,-0.3)(6,5.5)%
     \footnotesize
     \psset{%
       cornersize=relative,
       framearc=0.25,
       subgriddiv=1,
       gridlabels=4pt,
       gridwidth=0.2pt,
       }%
     \begin{tabstr}{0.75}
     \rput(3,5){\rnode{bounded}  {\psframebox{\begin{tabular}{c}\prop{bounded}\ifnxref{lattice}{def:latb}\end{tabular}}}}%
     %
     \rput(-3,3){\rnode{modular}  {\psframebox{\begin{tabular}{c}\prop{modular}\ifnxref{latm}{def:latm}\end{tabular}}}}%
     \rput(-3,2){\rnode{distributive}{\psframebox{\begin{tabular}{c}\prop{distributive}\ifnxref{latd}{def:latd}\end{tabular}}}}%
     %
     \rput(3,4){\rnode{complemented}  {\psframebox{\begin{tabular}{c}\prop{complemented}\ifnxref{latc}{def:latc}\end{tabular}}}}%
     \rput(3,3){\rnode{ortholat}  {\psframebox{\begin{tabular}{c}\prop{orthocomplemented}\ifnxref{ortholat}{def:latoc}\end{tabular}}}}%
     \rput(3,2){\rnode{orthomod}  {\psframebox{\begin{tabular}{c}\prop{orthomodular}\ifnxref{ortholat}{def:latoc_omod}\end{tabular}}}}%
     \rput(3,1){\rnode{modortho}  {\psframebox{\begin{tabular}{c}\prop{modular orthocomplemented}\ifnxref{ortholat}{def:latmoc}\end{tabular}}}}%
     %
     \rput(0,0){\rnode{boolean}   {\psframebox{\begin{tabular}{c}\prop{boolean}\ifnxref{boolean}{def:boolean}\end{tabular}}}}%
     \end{tabstr}
     %
     \psset{doubleline=true}%
     %\ncline{<-}{bounded}{modular}%
     \ncline{<-}{bounded}{complemented}%
     \ncline{<-}{complemented}{ortholat}%
     \ncline{<-}{modular}{distributive}%
     \ncline{<-}{distributive}{boolean}%
     \ncline{<-}{ortholat}{orthomod}%
     \ncline{<-}{modular}{modortho}%
     \ncline{<-}{orthomod}{modortho}%
     \ncline{<-}{modortho}{boolean}%
     %
     %\psgrid[unit=10mm](-8,-1)(8,9)%
  \end{pspicture}%}
  \end{center}
  \caption{lattice of orthocomplemented lattices\label{fig:latortholat}}
\end{figure}
%=======================================
\section{Orthocomplemented Lattices}
%=======================================
%=======================================
\subsection{Definition}
%=======================================
%---------------------------------------
\begin{definition}
\footnote{
  \citerpg{stern1999}{11}{0521461057},
  \citerpg{beran1985}{28}{902771715X},
  \citerpg{kalmbach1983}{16}{0123945801},
  \citerpg{gudder1988}{76}{0123053404},
  %\citerpg{lidl1998}{90}{0387982906}\\
  \citorp{loomis1955}{3},
  \citePpc{birkhoffjvn1936}{830}{L71--L73}
  }
\label{def:latoc}
\label{def:ocop}
%---------------------------------------
Let $\latL\eqd\latbd$ be a \structe{bounded lattice} \xref{def:latb}.
\defboxt{
  %An \structd{orthocomplemented lattice} $\latocd$ is a \structe{bounded lattice} $\latL$\\
  %with an \fncte{ortho negation} $^\ocop$ \xref{def:latn} on $\latL$.
  An element $x^\ocop\in\setX$ is an \vald{orthocomplement} of an element $x\in\setX$ if 
  %$\latL\eqd\latocd$ is an \structd{orthocomplemented lattice} if\\
  %$\ocop$ is an \fncte{ortho negation} \xref{def:latn} on $\latL$.
  \\\indentx$\begin{array}{FlclCDD}
    %1. & x \join x^\ocop      &=& \bid            & \forall x\in\setX   & (\prope{excluded middle})         & and \\
    %2. & x\meet x^\ocop       &=& \bzero          & \forall x\in\setX   & (\prope{non-contradiction})       & and \\
   %1. & \lzero^\ocop         &=&        \lid     &                     & (\prope{boundary condition})      & and \\
   %2. & \lid^\ocop           &=&        \lzero   &                     & (\prope{boundary condition})      & and\\
    1. & x^{\ocop\ocop}       &=& x               &                     & (\prope{involutory})              & and \\
    2. & x\meet x^\ocop       &=& \lzero          &                     & (\prope{non-contradiction})       & and \\
    3. & x\orel y &\implies& y^\ocop\orel x^\ocop & \forall   y\in\setX & (\prope{antitone}).               & 
  \end{array}$
  \\
  The \structe{lattice} $\latL$ is \propd{orthocomplemented} ($\latL$ is an \structd{orthocomplemented lattice})\\
  if every element $x$ in $\setX$ has an \vale{orthocomplement} $x^\orthog$ in $\setX$.
  }
\end{definition}


\begin{minipage}{13\tw/16}%
%---------------------------------------
\begin{definition}
\label{def:o6}
\footnotemark
%---------------------------------------
\mbox{}\\\defboxp{$\begin{array}{M}
  The \structd{O$_6$ lattice} is the ordered set $\opair{\setn{0,p,q,p^\ocop,q^\ocop,1}}{\orel}$
  with cover relation
  \\\indentx
    $\coverrel = \setn{\opair{0}{p},\, \opair{0}{q},\, \opair{p}{q^\ocop},\, \opair{q}{p^\ocop},\, \opair{p^\ocop}{1},\, \opair{q^\ocop}{1}}$.
\end{array}$}
The $O_6$ lattice is illustrated by the Hasse diagram to the right.
\end{definition}
\end{minipage}%
\footnotetext{%
  \citerpg{kalmbach1983}{22}{0123945801},
  \citeIp{holland1970}{50},
  \citerpg{beran1985}{33}{902771715X},
  \citerpg{stern1999}{12}{0521461057},
  The \structe{O$_6$ lattice} is also called the \hid{Benzene ring} or the \hid{hexagon}.
  }%
\qquad\tbox{\includegraphics{graphics/lat6_o6_latoc_pq.pdf}}%
%\begin{minipage}[c]{3\tw/16}%
%  \psset{unit=5mm}%
%  %\centering%============================================================================
% Daniel J. Greenhoe
% LaTeX file
% lattice (2^{x,y,z}, subseteq)
%============================================================================
%{%\psset{unit=0.075mm}
\begin{pspicture}(-1.6,-0.5)(1.6,3.5)
  %---------------------------------
  % settings
  %---------------------------------
  \psset{%
    labelsep=1.5mm,
    }%
  %---------------------------------
  % nodes
  %---------------------------------
  \Cnode(0,3){t}%
  \Cnode(1,2){xc}\Cnode(1,1){yc}%
  \Cnode(-1,1){x}\Cnode(-1,2){y}%
  \Cnode(0,0){b}%
  %---------------------------------
  % node connections
  %---------------------------------
  \ncline{t}{y}\ncline{t}{xc}%
  \ncline{x}{y}\ncline{yc}{xc}%
  \ncline{b}{x}\ncline{b}{yc}%
  %---------------------------------
  % node labels
  %---------------------------------
  \uput[0](t){$\bid$}%
  \uput[0](yc){$y^\ocop$}\uput[0](xc){$x^\ocop$}%
  \uput[180](x){$x$}\uput[180](y){$y$}
  \uput[0](b){$\bzero$}
\end{pspicture}%
%    }
%
%  \centering\psset{unit=5mm}{%============================================================================
% Daniel J. Greenhoe
% LaTeX file
% lattice O6
% nominal unit = 10mm
%============================================================================
\begin{pspicture}(-1.9,-\latbot)(1.9,3.4)%
  %---------------------------------
  % nodes
  %---------------------------------
  \Cnode(0,3){t}%
  \Cnode(-1,2){c}\Cnode(1,2){d}%
  \Cnode(-1,1){x}\Cnode(1,1){y}%
  \Cnode(0,0){b}%
  %---------------------------------
  % node connections
  %---------------------------------
  \ncline{t}{c}\ncline{t}{d}%
  \ncline{c}{x}\ncline{d}{y}%
  \ncline{b}{x}\ncline{b}{y}%
  %---------------------------------
  % node labels
  %---------------------------------
  \uput[0](t){$1$}%
  \uput[45](d){$p^\ocop$}%
  \uput{1pt}[135](c){$q^\ocop$}%
  \uput[-45](y){$q$}%
  \uput[-135](x){$p$}%
  \uput[0](b){$0$}%
\end{pspicture}%}% 2014sep01mon  2013dec10tues
%\end{minipage}%


%---------------------------------------
\begin{example}
\footnote{
  \citerp{holland1963}{50}
  }
%---------------------------------------
\exboxt{
  The \structb{$O_6$ lattice} \xref{def:o6} is an\\
  \structb{orthocomplemented lattice} \xref{def:latoc}.
  }
\end{example}

%%---------------------------------------
%\begin{example}
%\footnote{
%  \citePpp{maeda1966}{250}{251}\\
%  \citerpg{beran1985}{33}{902771715X}
%  }%
%\label{ex:latoc_n5n5}
%%---------------------------------------
%\exbox{%
%  \begin{tabular}{cm{34mm}}%
%    \begin{tabular}{l}%
%      The lattice illustrated to the right is \propb{orthocomplemented}. 
%      %but \prope{non-orthomodular}\\% 
%     %(and hence, \prope{non-modular} and \prope{non-Boolean}).%
%    \end{tabular}%
%    &\centering\psset{unit=8mm}%============================================================================
% Daniel J. Greenhoe
% LaTeX file
%============================================================================
%{%\psset{unit=0.075mm}
\begin{pspicture}(-1.6,-0.5)(1.6,3.2)
  %---------------------------------
  % settings
  %---------------------------------
  %\psset{%
  %  labelsep=1.5mm,
  %  }%
  %---------------------------------
  % nodes
  %---------------------------------
  \Cnode(0,3){t}%
  \Cnode(1,2.5){pc}%
  \Cnode(0,1){y}%
  \Cnode(1,1.5){xc}%
  \Cnode(-1,2){zc}%
  %
  \Cnode(0,2){yc}%
  \Cnode(1,1){z}%
  \Cnode(-1,1.5){x}%
  \Cnode(-1,0.5){p}%
  \Cnode(0,0){b}%
  %---------------------------------
  % node connections
  %---------------------------------
  \ncline{t}{zc}\ncline{t}{yc}\ncline{t}{pc}%
  \ncline{xc}{pc}%
  \ncline{z}{yc}\ncline{z}{xc}%
  \ncline{y}{zc}\ncline{y}{xc}%
  \ncline{x}{zc}\ncline{x}{yc}%
  \ncline{p}{x}%
  \ncline{b}{p}\ncline{b}{y}\ncline{b}{z}%
  %---------------------------------
  % node labels
  %---------------------------------
  \uput[45](t){$\bid$}%
  \uput[0](pc){$p^\ocop$}%
  \uput[180](zc){$z^\ocop$}%
  \uput[45](yc){$y^\ocop$}%
  \uput[0](xc){$x^\ocop$}%
  \uput[0](z){$z$}%
  \uput[-45](y){$y$}%
  \uput[180](x){$x$}%
  \uput[180](p){$p$}%
  \uput[-45](b){$\bzero$}
\end{pspicture}%
%    }
%
%  \end{tabular}%
%    }%
%\end{example}%

%---------------------------------------
\begin{example}
\label{ex:latoc}
\footnote{
  \citerppg{beran1985}{33}{42}{902771715X},
  \citePp{maeda1966}{250},
  \citerpgc{kalmbach1983}{24}{0123945801}{Figure 3.2},
  \citerpg{stern1999}{12}{0521461057},
  \citerp{holland1970}{50}
  }%
%---------------------------------------
There are a total of 10 \structb{orthocomplemented lattices} with 8 elements or less.
These 10, along with 3 other orthocomplemented lattices with 10 elements, are illustrated next:
\begin{longtable}{|cccc|}
  \hline
  \mc{4}{|m{\tw-8mm}|}{\ragr Lattices that are \propb{orthocomplemented} but \prope{non-orthomodular} and hence also \prope{not modular orthocomplemented} and \prope{non-Boolean}:}
  \\
     \includegraphics{graphics/lat6_o6_xy.pdf}%
    &\includegraphics{graphics/lat8_o8_xyp.pdf}%
    &\includegraphics{graphics/lat8_l3l4l4l3_xypoc.pdf}%
    &\includegraphics{graphics/lat8_2e3m2_xyzoc.pdf}%
    \\\cntos\quad\structe{$O_6$ lattice}
     &\cntxs\quad\structe{$O_8$ lattice}
     &\cntxs\quad%\structe{$O_8$ lattice}
     &\cntxs\quad%\structe{$O_8$ lattice}
  \\
     \includegraphics{graphics/lat10_n5n5oc_xyp.pdf}%
    &\includegraphics{graphics/lat10_m4m4oc_wxyz.pdf}%
    &\includegraphics{graphics/lat10_o8m2_xypqoc.pdf}%
    &%\psset{unit=10mm}%============================================================================
% Daniel J. Greenhoe
% LaTeX file
% lattice (2^{w,x,y,z}, ortholat)
%============================================================================
{%\psset{unit=0.5\psunit}%
\begin{pspicture}(-3,-\latbot)(3,3.25)
  %---------------------------------
  % nodes
  %---------------------------------
  \Cnode(0,3){lub}
  \Cnode(-0.5,2.5){yc}\Cnode(0.5,2.5){wc}%
  \Cnode(-2.5,2){jc}\Cnode(-1.5,2){hc}\Cnode(-0.5,2){fc}\Cnode(0.5,2){dc}\Cnode(1.5,2){cc}\Cnode(2.5,2){ac}%
  \Cnode(-2.5,1){a}\Cnode(-1.5,1){c}\Cnode(-0.5,1){d}\Cnode(0.5,1){f}\Cnode(1.5,1){h}\Cnode(2.5,1){j}%
  \Cnode(-0.5,0.5){w}\Cnode(0.5,0.5){y}%
  \Cnode(0,0){glb}%
  %---------------------------------
  % node connections
  %---------------------------------
  \ncline{lub}{wc}\ncline{lub}{yc}%
  \ncline{wc}{ac}\ncline{wc}{dc}\ncline{wc}{hc}\ncline{yc}{cc}\ncline{yc}{fc}\ncline{yc}{jc}%
  \ncline{j}{dc}\ncline{j}{ac}%
  \ncline{h}{fc}\ncline{h}{cc}%
  \ncline{f}{hc}\ncline{f}{ac}%
  \ncline{d}{jc}\ncline{d}{cc}%
  \ncline{c}{hc}\ncline{c}{dc}%
  \ncline{a}{jc}\ncline{a}{fc}%
  \ncline{w}{a}\ncline{w}{d}\ncline{w}{h}\ncline{y}{c}\ncline{y}{f}\ncline{y}{j}%
  \ncline{glb}{w}\ncline{glb}{y}%
  %
  %---------------------------------
  % node labels
  %---------------------------------
  \uput[45](lub){$\bid$}%
  %
  \uput[135](yc){$y^\ocop$}%
  \uput[45](wc){$w^\ocop$}%
  %
  \uput[135](jc){$j^\ocop$}%     
  \uput[135](hc){$h^\ocop$}%
  \uput[135](fc){$f^\ocop$}%
  \uput[45](dc){$d^\ocop$}%     
  \uput[45](cc){$c^\ocop$}%
  \uput[45](ac){$a^\ocop$}%
  \uput[-45](j){$j$}%     
  \uput[-45](h){$h$}%
  \uput[-45](f){$f$}%
  \uput[-135](d){$d$}%     
  \uput[-135](c){$c$}%
  \uput[-135](a){$a$}%
  %
  \uput[-45](y){$y$}%
  \uput[-135](w){$w$}%
  %
  \uput[-45](glb) {$\bzero$}%
\end{pspicture}
}%
  \\\cntxs 
    &\cntxs 
    &\cntxs
    &%\cntxs\quad\structe{$\latL_2^5\setd\latL_2^4\setu\setn{\bzero,\bid}$ lattice}
  %\\
  %    \mc{4}{|c|}{\psset{unit=10mm}%============================================================================
% Daniel J. Greenhoe
% LaTeX file
% lattice (2^{w,x,y,z}, ortholat)
%============================================================================
{%\psset{unit=0.5\psunit}%
\begin{pspicture}(-3,-\latbot)(3,3.25)
  %---------------------------------
  % nodes
  %---------------------------------
  \Cnode(0,3){lub}
  \Cnode(-0.5,2.5){yc}\Cnode(0.5,2.5){wc}%
  \Cnode(-2.5,2){jc}\Cnode(-1.5,2){hc}\Cnode(-0.5,2){fc}\Cnode(0.5,2){dc}\Cnode(1.5,2){cc}\Cnode(2.5,2){ac}%
  \Cnode(-2.5,1){a}\Cnode(-1.5,1){c}\Cnode(-0.5,1){d}\Cnode(0.5,1){f}\Cnode(1.5,1){h}\Cnode(2.5,1){j}%
  \Cnode(-0.5,0.5){w}\Cnode(0.5,0.5){y}%
  \Cnode(0,0){glb}%
  %---------------------------------
  % node connections
  %---------------------------------
  \ncline{lub}{wc}\ncline{lub}{yc}%
  \ncline{wc}{ac}\ncline{wc}{dc}\ncline{wc}{hc}\ncline{yc}{cc}\ncline{yc}{fc}\ncline{yc}{jc}%
  \ncline{j}{dc}\ncline{j}{ac}%
  \ncline{h}{fc}\ncline{h}{cc}%
  \ncline{f}{hc}\ncline{f}{ac}%
  \ncline{d}{jc}\ncline{d}{cc}%
  \ncline{c}{hc}\ncline{c}{dc}%
  \ncline{a}{jc}\ncline{a}{fc}%
  \ncline{w}{a}\ncline{w}{d}\ncline{w}{h}\ncline{y}{c}\ncline{y}{f}\ncline{y}{j}%
  \ncline{glb}{w}\ncline{glb}{y}%
  %
  %---------------------------------
  % node labels
  %---------------------------------
  \uput[45](lub){$\bid$}%
  %
  \uput[135](yc){$y^\ocop$}%
  \uput[45](wc){$w^\ocop$}%
  %
  \uput[135](jc){$j^\ocop$}%     
  \uput[135](hc){$h^\ocop$}%
  \uput[135](fc){$f^\ocop$}%
  \uput[45](dc){$d^\ocop$}%     
  \uput[45](cc){$c^\ocop$}%
  \uput[45](ac){$a^\ocop$}%
  \uput[-45](j){$j$}%     
  \uput[-45](h){$h$}%
  \uput[-45](f){$f$}%
  \uput[-135](d){$d$}%     
  \uput[-135](c){$c$}%
  \uput[-135](a){$a$}%
  %
  \uput[-45](y){$y$}%
  \uput[-135](w){$w$}%
  %
  \uput[-45](glb) {$\bzero$}%
\end{pspicture}
}%}
  %\\\mc{4}{|c|}{\cntxs\quad \structe{$\latL_2^5\setd\latL_2^4\setu\setn{\bzero,\bid}$ lattice}}
  \\\hline
  \mc{4}{|m{\tw-8mm}|}{\ragr Lattices that are \propb{orthocomplemented} and \propb{orthomodular} but \prope{not modular orthocomplemented} and hence also \prope{non-Boolean}:}
  \\
      \includegraphics{graphics/lat10_hsum_pxyz.pdf}
     &\mc{3}{c|}{\includegraphics{graphics/latocdil7_abxyzcd.pdf}}%
  \\\cntxs 
   &\mc{3}{c|}{\cntxs}
  \\\hline
  \mc{4}{|m{\tw-8mm}|}{\ragr Lattices that are \propb{orthocomplemented}, \propb{orthomodular}, and \propb{modular orthocomplemented} but \prope{non-Boolean}:}
  \\
    \includegraphics{graphics/lat6_m4_wxyz.pdf}%
   &\includegraphics{graphics/lat8_m6_xyzoc.pdf}%
   &
   &
  %&\psset{unit=7mm}%============================================================================
% Daniel J. Greenhoe
% LaTeX file
% lattice M4
%============================================================================
{%\psset{unit=0.5\psunit}%
\begin{pspicture}(-3.7,-\latbot)(3.7,2.2)
  %---------------------------------
  % nodes
  %---------------------------------
  \Cnode(0,2){t}%
  \Cnode(-3.5,1){p}\Cnode(-2.5,1){q}\Cnode(-1.5,1){u}\Cnode(-0.5,1){v}\Cnode(0.5,1){w}\Cnode(1.5,1){x}\Cnode(2.5,1){y}\Cnode(3.5,1){z}%
  \Cnode(0,0){b}%
  %---------------------------------
  % node connections
  %---------------------------------
  \ncline{t}{p}\ncline{t}{q}\ncline{t}{u}\ncline{t}{v}\ncline{t}{w}\ncline{t}{x}\ncline{t}{y}\ncline{t}{z}%
  \ncline{b}{p}\ncline{b}{q}\ncline{b}{u}\ncline{b}{v}\ncline{b}{w}\ncline{b}{x}\ncline{b}{y}\ncline{b}{z}%
  %---------------------------------
  % node labels
  %---------------------------------
  %\uput[ 90](t) {$\setn{x,y,z}$}%
\end{pspicture}
}%
  \\\cntxs\quad\structe{$M_4$ lattice}
   &\cntxs\quad\structe{$M_6$ lattice}
   &
   &
  \\\hline
  \mc{4}{|m{\tw-8mm}|}{\ragr Lattices that are \propb{orthocomplemented}, \propb{orthomodular}, \propb{modular orthocomplemented} and \propb{Boolean}:}
  \\
      \includegraphics{graphics/lat1_01.pdf}%
     &\includegraphics{graphics/lat2_l2_01.pdf}%
     &\includegraphics{graphics/lat4_m2_latoc_p.pdf}%
     &\includegraphics{graphics/lat8_l2e3_latoc_pqr.pdf}%
  \\\cntxs\quad \structe{$L_1$ lattice}
   &\cntxs\quad \structe{$L_2$ lattice}
   &\cntxs\quad \structe{$L_2^2$ lattice}
   &\cntxs\quad \structe{$L_2^3$ lattice}
  \\\hline
      \mc{4}{|c|}{\includegraphics{graphics/lat16_l2e4_latoc_pqrs.pdf}}%
    \\\mc{4}{|c|}{\cntxs\structe{$L_2^4$ lattice}}%
  \\\hline
      \mc{4}{|c|}{\includegraphics{graphics/lat32_l2e5_latoc_pqrst.pdf}}%
    \\\mc{4}{|c|}{\cntxs\structe{$L_2^5$ lattice}}%
  \\\hline
\end{longtable}
%\exbox{
%  \begin{tabular}{cm{44mm}}
%    \begin{tabular}{l}
%    The lattice illustrated to the right is \prope{orthocomplemented}.
%    %but \prope{non-orthomodular} and \prope{non-modular}\\
%    %(and hence also \prope{non-Boolean}).
%    \end{tabular}
%    &\centering\psset{unit=7mm}%============================================================================
% Daniel J. Greenhoe
% LaTeX file
% lattice (2^{x,y,z}, subseteq)
%============================================================================
%{%\psset{unit=0.075mm}
\begin{pspicture}(-2.2,-0.5)(2.2,3.5)
  %---------------------------------
  % settings
  %---------------------------------
  \psset{%
    labelsep=1.5mm,
    }%
  %---------------------------------
  % nodes
  %---------------------------------
  \Cnode(0,3){t}
  \Cnode(-1.5,2){zc} \Cnode(-0.5,2){yc} \Cnode(0.5,2){xc} \Cnode(1.5,2){wc}
  \Cnode(-1.5,1){w}  \Cnode(-0.5,1){x}  \Cnode(0.5,1){y}  \Cnode(1.5,1){z}
  \Cnode(0,0){b}
  %---------------------------------
  % node connections
  %---------------------------------
  \ncline{t}{zc}\ncline{t}{yc}\ncline{t}{xc}\ncline{t}{wc}
  \ncline{w}{zc}
  \ncline{x}{zc}\ncline{x}{yc}
  \ncline{y}{zc}\ncline{y}{xc}
  \ncline{z}{yc}\ncline{z}{xc}\ncline{z}{wc}
  \ncline{b}{w}\ncline{b}{x}\ncline{b}{y}\ncline{b}{z}
  %---------------------------------
  % node labels
  %---------------------------------
  \uput[90](t){$\bid$}
  \uput[180](zc){$z^\ocop$} \uput[158](yc){$y^\ocop$} \uput[180](xc){$x^\ocop$} \uput[0](wc){$w^\ocop$}
  \uput[180](w){$w$} \uput[0](x){$x$} \uput[0](y){$y$} \uput[0](z){$z$}
  \uput[-90](b){$\bzero$}
\end{pspicture}%
%    }

%  \end{tabular}
%  }
\end{example}


%---------------------------------------
\begin{example}
\label{ex:latoc_r3}
%---------------------------------------
\exbox{\begin{tabular}{m{\tw-63mm}|m{44mm}}
    \begin{tabular}{cl}
      \mc{2}{l}{The structure $\ds\latoc{\pset{\R^\xN}}{\subseteq}{+}{\seti}{^\ocop}{\emptyset}{\spH}$}\\
      \mc{2}{l}{is an \structb{orthocomplemented lattice} where}
      \\\hspace{2ex}\imark&$\R^\xN$ is an \structb{Euclidean space} with dimension $\xN$
      \\\hspace{2ex}\imark&$\pset{\R^\xN}$ is the set of all subspaces of $\R^{\xN}$
      \\\hspace{2ex}\imark&$\spV+\spW$ is the \ope{Minkowski sum} of subspaces $\spV$ and $\spW$
      \\\hspace{2ex}\imark&$\spV\seti\spW$ is the \ope{intersection} of subspaces $\spV$ and $\spW$
    \end{tabular}
  &\centering\includegraphics{graphics/r3subspaces_xyzxcyczc.pdf}
  \end{tabular}}
\end{example}

%---------------------------------------
\begin{example}
%---------------------------------------
\exboxt{
  The structure $\ds\latoc{\pset{\spH}}{\subseteq}{\oplus}{\seti}{^\ocop}{\emptyset}{\spH}$
  is an \structb{orthocomplemented lattice} where
  \\\begin{tabular}{cl}
      \hspace{2ex}\imark&$\spH$ is a \structb{Hilbert space}
    \\\hspace{2ex}\imark&$\pset{\spH}$ is the set of all closed subspaces of $\spH$
    \\\hspace{2ex}\imark&$\spX+\spY$ is the \ope{Minkowski sum} of subspaces $\spX$ and $\spY$
    \\\hspace{2ex}\imark&$\spX\oplus\spY\eqd\cls{\brp{\spX+\spY}}$ is the \ope{closure} of $\spX+\spY$
    \\\hspace{2ex}\imark&$\spX\seti\spY$ is the \ope{intersection} of subspaces $\spX$ and $\spY$
  \end{tabular}
  }
\end{example}

%=======================================
\subsection{Properties}
%=======================================
%---------------------------------------
\begin{theorem}
\footnote{
  \citerppg{beran1985}{30}{31}{902771715X},
  \citePpc{birkhoffjvn1936}{830}{L74},
  \citerpgc{cohen1989}{37}{1461388430}{3B.13. Theorem}
  }
\label{thm:latoc_prop}
%---------------------------------------
Let $\latL\eqd\latocd$ be a \structe{bounded lattice}.
\thmbox{
  \brb{\begin{array}{M}
    $\latL$ is\\
    \prope{orthocomplemented}\\
    \xref{def:latoc}
  \end{array}}
  \implies
  \brbl{\begin{array}{FrclCDD}
    (1).& \lzero^\ocop     &=& \lid                  &                     & (\prope{boundary condition})    & and \\
    (2).& \lid^\ocop       &=& \lzero                &                     & (\prope{boundary condition})    & and \\
    (3).& (x\join y)^\ocop &=& x^\ocop \meet y^\ocop & \forall x,y\in\setX & (\prope{disjunctive de morgan}) & and \\
    (4).& (x\meet y)^\ocop &=& x^\ocop \join y^\ocop & \forall x,y\in\setX & (\prope{conjunctive de morgan}) & and \\
    (5).& x \join x^\ocop  &=& \lid                  & \forall x\in\setX   & (\prope{excluded middle}).      &  
   %(5).& \mc{6}{M}{$\ocop$ is an \fncte{ortho negation} \xref{def:latn}.}
  \end{array}}
  }
\end{theorem}
\begin{proof}
Let $x^\ocop\eqd\negat{x}$, where $\negat$ is an \fncte{ortho negation} function \xref{def:negor}.
Then, this theorem follows directly from \prefpp{thm:latn_ortho}.
\end{proof}


%---------------------------------------
\begin{corollary}
%---------------------------------------
Let $\latL\eqd\latocd$ be a \structe{bounded lattice} \xref{def:latb}.
\corbox{
  \brb{\begin{array}{M}
    $\latL$ is \propb{orthocomplemented}\\ 
    \xref{def:latoc}
  \end{array}}
  \implies
  \brb{\begin{array}{M}
    $\latL$ is \propb{complemented}\\ 
    \ifxref{latc}{def:latc}
  \end{array}}
  }
\end{corollary}
\begin{proof}
This follows directly from the definition of \structe{orthocomplemented lattice}s \xref{def:latoc} and
\structe{complemented lattice}s\ifsxref{latc}{def:latc}.
\end{proof}

%---------------------------------------
\begin{example}
%---------------------------------------
\exbox{\begin{tabular}{m{18mm}m{\tw-59mm}m{18mm}}
  \centering\includegraphics{graphics/lat6_o6_pqab01.pdf}
  & The \structe{$O_6$ lattice} \xref{def:o6} illustrated to the left is both
    \propb{orthocomplemented} \xref{def:latoc} and
    \propb{multiply complemented}\ifsxref{latc}{def:latc}.
    The lattice illustrated to the right is \propb{multiply complemented},
    but is \propb{non-orthocomplemented}.
  &\centering\includegraphics{graphics/lat6_primo_pqra01.pdf}%
\end{tabular}}
\end{example}
\begin{proof}
\begin{enumerate}
  \item Proof that \structe{$O_6$ lattice} is multiply complemented: $b$ and $q$ are both \structe{complements} of $p$.
  \item Proof that the right side lattice is multiply complemented: $a$, $p$, and $q$ are all \structe{complements} of $r$.
\end{enumerate}
\end{proof}

%%---------------------------------------
%\begin{theorem}[\thmd{Principle of duality}]
%\label{thm:latoc_duality}
%\footnote{
%  \citerppg{beran1985}{29}{30}{902771715X}
%  }
%%---------------------------------------
%Let $\latL\eqd\latticed$ be an \structe{orthocomplemented lattice}.
%\thmbox{\begin{array}{M}
%  $\brb{\parbox{6\tw/16}{\raggedright
%    $\phi$ is an \structe{identity} on $\latL$ in terms of the operations
%    $\join$ and $\meet$}}$
%  $\qquad\implies\qquad$
%  $\opT\phi$ is also an \structe{identity} on $\latL$
%  \\[1ex]
%  where the operator $\opT$ performs the following mapping on the operations of $\phi$:
%  \\\indentx $\join\rightarrow\meet,\qquad \meet\rightarrow\join,\qquad\bid\rightarrow\bzero,\qquad\bzero\rightarrow\bid$
%\end{array}}
%\end{theorem}
%\begin{proof}
%\begin{align*}
%  \opT\brb{\begin{array}{llclC}
%    1. & x \join x^\ocop      &=& \bid            &\forall x,y\in\latL \\
%    2. & x\meet x^\ocop       &=& \bzero          &\forall x,y\in\latL \\
%    3. & \brp{x^\ocop}^\ocop  &=& x               &\forall x  \in\latL \\
%    4. & x\orel y &\implies& y^\ocop\orel x^\ocop &\forall x,y\in\latL 
%  \end{array}}
%  &\implies
%  \brb{\begin{array}{llclC}
%    1. & x \meet x^\ocop      &=& \bzero            &\forall x,y\in\latL \\
%    2. & x\join x^\ocop       &=& \bid              &\forall x,y\in\latL \\
%    3. & \brp{x^\ocop}^\ocop  &=& x                 &\forall x  \in\latL \\
%    4. & x\oreld y &\implies& y^\ocop\oreld x^\ocop &\forall x,y\in\latL 
%  \end{array}}
%  \\&\implies
%  \brb{\begin{array}{llclC}
%    1. & x \join x^\ocop      &=& \bid             &\forall x,y\in\latL \\
%    2. & x\meet x^\ocop       &=& \bzero           &\forall x,y\in\latL \\
%    3. & \brp{x^\ocop}^\ocop  &=& x                &\forall x  \in\latL \\
%    4. & x\orel y &\implies& y^\ocop\orel x^\ocop  &\forall x,y\in\latL
%  \end{array}}
%\end{align*}
%\end{proof}



\pref{lem:latoc_demorgan} (next) is useful in proving that 
\prope{de Morgan}'s laws \xref{thm:demorgan} hold in orthocomplemented lattices \xref{thm:latoc_prop} 
and in proving the characterization of \prefpp{thm:latoc_char3}.
%---------------------------------------
\begin{lemma}
\footnote{
  \citerppg{beran1985}{30}{31}{902771715X},
  \citePc{fay1967}{cf Beran 1985 page 30},
  \citerc{nakano1971}{cf Beran 1985} %{cf Beran 1985 page 30}
  }
\label{lem:latoc_demorgan}
%---------------------------------------
Let $\latL\eqd\latocd$ be an \structe{orthocomplemented lattice} \xref{def:latoc}. 
\lembox{
  \mcom{x\orel y \implies y^\ocop\orel x^\ocop}{\prope{antitone}}
  \qquad\iff\qquad
  \mcom{\brbl{\begin{array}{rclCD}
    (x\join y)^\ocop &=& x^\ocop \meet y^\ocop & x,y\in\setX & and \\
    (x\meet y)^\ocop &=& x^\ocop \join y^\ocop & x,y\in\setX
  \end{array}}}{\prope{de Morgan}}
  }
\end{lemma}
\begin{proof}
This follows directly from \prefpp{lem:latn_demorgan}.
%\begin{enumerate}
%  \item Proof that \prope{antitone} $\implies$ \prope{de Morgan}:
%    \begin{enumerate}
%      \item Proof that $(x\meet y)^\ocop \ge x^\ocop \join y^\ocop$: \label{item:latoc_demorgan_xmeety}
%        \begin{align*}
%          &\text{$x\meet y\le x$ and $x\meet y\le y$}
%            && \text{by definition of $\meet$}
%          \\&\implies \text{$(x\meet y)^\ocop\ge x^\ocop$ and $(x\meet y)^\ocop\ge y^\ocop$}
%            && \text{by \prope{antitone}}
%          \\&\implies (x\meet y)^\ocop\ge x^\ocop \join y^\ocop
%            && \text{by definition of $\join$}
%        \end{align*}
%    
%      \item Proof that $(x\join y)^\ocop \le x^\ocop \meet y^\ocop$: \label{item:latoc_demorgan_xjoiny}
%        \begin{align*}
%          &\text{$x\le x\join y$ and $y\le x\join y$}
%            && \text{by definition of $\join$}
%          \\&\implies \text{$x^\ocop\ge(x\join y)^\ocop $ and $y^\ocop\ge(x\join y)^\ocop$}
%            && \text{by \prope{antitone}}
%          \\&\implies x^\ocop \meet y^\ocop \ge (x\join y)^\ocop
%            && \text{by definition of $\meet$}
%          \\&\implies (x\join y)^\ocop \le x^\ocop \meet y^\ocop
%        \end{align*}
%    
%      \item Proof that $(x^\ocop\meet y^\ocop)^\ocop \ge x \join y$: \label{item:latoc_demorgan_xomeetyo}
%        \begin{align*}
%          (x^\ocop\meet y^\ocop)^\ocop
%            &\ge \brp{x^\ocop}^\ocop \join \brp{y^\ocop}^\ocop
%            &&   \text{by \pref{item:latoc_demorgan_xmeety}}
%          \\&=   x \join y
%            &&   \text{by \prope{involutory} property \xref{def:latoc}}
%        \end{align*}
%    
%      \item Proof that $(x^\ocop\join y^\ocop)^\ocop \le x \meet y$: \label{item:latoc_demorgan_xojoinyo}
%        \begin{align*}
%          (x^\ocop\join y^\ocop)^\ocop
%            &\le \brp{x^\ocop}^\ocop \meet \brp{y^\ocop}^\ocop
%            &&   \text{by \pref{item:latoc_demorgan_xjoiny}}
%          \\&=   x \meet y
%            &&   \text{by \prope{involutory} property \xref{def:latoc}}
%        \end{align*}
%    
%      \item Proof that $(x\meet y)^\ocop = x^\ocop \join y^\ocop$:
%        \begin{align*}
%          (x\meet y)^\ocop
%            &\ge x^\ocop \join y^\ocop
%            &&   \text{by \pref{item:latoc_demorgan_xmeety}}
%            \\
%          (x\meet y)^\ocop
%            &=   \brs{\brp{x^\ocop}^\ocop \meet \brp{y^\ocop}^\ocop}^\ocop
%            &&   \text{by \prope{involutory} property \xref{def:latoc}}
%          \\&\le x^\ocop \join y^\ocop
%            &&   \text{by \pref{item:latoc_demorgan_xojoinyo}}
%        \end{align*}
%    
%      \item Proof that $(x\join y)^\ocop = x^\ocop \meet y^\ocop$:
%        \begin{align*}
%          (x\join y)^\ocop
%            &\ge x^\ocop \meet y^\ocop
%            &&   \text{by \pref{item:latoc_demorgan_xjoiny}}
%            \\
%          (x\join y)^\ocop
%            &=   \brs{\brp{x^\ocop}^\ocop \join \brp{y^\ocop}^\ocop}^\ocop
%            &&   \text{by \prope{involutory} property \xref{def:latoc}}
%          \\&\le x^\ocop \meet y^\ocop
%            &&   \text{by \pref{item:latoc_demorgan_xomeetyo}}
%        \end{align*}
%    
%    \end{enumerate}
%
%  \item Proof that \prope{antitone} $\impliedby$ \prope{de Morgan}:
%    \begin{align*}
%      x\le y \implies y^\ocop
%        &= \brp{x \join y}^\ocop
%        && \text{because $x\le y$}
%      \\&= x^\ocop \meet y^\ocop
%        && \text{by \prope{de Morgan}}
%      \\&\le x^\ocop
%        && \text{by definition of $\meet$}
%    \end{align*}
%\end{enumerate}
\end{proof}


%---------------------------------------
\begin{lemma}
\label{lem:latoc_distrib}
%---------------------------------------
Let $\latL\eqd\latocd$ be an \structe{orthocomplemented lattice} \xref{def:latoc}. 
\lemboxt{
  The set $\setn{0,x,x^\orthog}$ is \prope{distributive} \xref{def:Drel}
  for all $x\in\setX$.
  }
\end{lemma}
\begin{proof}
  \begin{align*}
    0         \meet (x \join x^\orthog) 
      &= 0                                             
      && \text{by \prope{lower bounded} property}
      && \text{\xref{prop:latb_prop}}
    \\&= 0 \join 0
      && \text{by \prope{join identity}}
      && \text{\xref{prop:latb_prop}}
    \\&= (0\meet x) \join (0\meet x^\orthog)
      && \text{by \prope{lower bounded} property}
      && \text{\xref{prop:latb_prop}}
    \\
    0         \meet (x^\orthog \join x) 
      &= 0                                             
      && \text{by \prope{lower bounded} property}
      && \text{\xref{prop:latb_prop}}
    \\&= 0 \join 0
      && \text{by \prope{join identity}}
      && \text{\xref{prop:latb_prop}}
    \\&= (0\meet x^\orthog) \join (0\meet x)
      && \text{by \prope{lower bounded} property}
      && \text{\xref{prop:latb_prop}}
    \\
   %0         \meet (x^\orthog \join x) 
   %  &= 0 \meet (x \join x^\orthog) 
   %  && \text{by \prope{commutative} property of \structe{lattice}s} 
   %  && \text{\xref{thm:lattice}}
   %\\&= (0\meet x) \join (0\meet x^\orthog)
   %  && \text{by previous result}
   %\\&= (0\meet x^\orthog) \join (0\meet x)
   %  && \text{by \prope{commutative} property of \structe{lattice}s} 
   %  && \text{\xref{thm:lattice}}
   %\\
    x         \meet (x^\orthog \join 0        ) 
      &= x         \meet   x^\orthog
      && \text{by \prope{join identity}}
      && \text{\xref{prop:latb_prop}}
    \\&= 0
      && \text{by \prope{non-contradiction} property}
      && \text{\xref{def:latoc}}
    \\&= 0 \join 0
      && \text{by \prope{join identity}}
      && \text{\xref{prop:latb_prop}}
    \\&= (x\meet x^\orthog) \join 0
      && \text{by \prope{non-contradiction} property}
      && \text{\xref{def:latoc}}
    \\&= (x\meet x^\orthog) \join (x\meet 0)
      && \text{by \prope{lower bounded} property}
      && \text{\xref{prop:latb_prop}}
    \\
    x \meet (0 \join x^\orthog) 
      &= x \meet (x^\orthog \join 0)
      && \text{by \prope{commutative} property of \structe{lattice}s} 
      && \text{\xref{thm:lattice}}
    \\&= (x\meet x^\orthog) \join (x\meet 0)
      && \text{by previous result}
    \\&= (x\meet 0) \join (x\meet x^\orthog) 
      && \text{by \prope{commutative} property of \structe{lattice}s} 
      && \text{\xref{thm:lattice}}
    \\
    x^\orthog         \meet (x \join 0        ) 
      &= (x^\orthog\meet x) \join (x^\orthog\meet 0)
      && \text{by $x\meet (x^\orthog \join 0)$ result}
    \\
    x^\orthog         \meet (0 \join x        ) 
      &= (x^\orthog\meet 0) \join (x^\orthog\meet x)
      && \text{by $x\meet (0\join x^\orthog)$ result}
  \end{align*}
\end{proof}

%=======================================
\subsection{Characterization}
%=======================================
%---------------------------------------
\begin{theorem}
\footnote{
  \citerppg{beran1985}{31}{33}{902771715X},
  \citePpp{beran1976}{251}{252} %{http://projecteuclid.org/euclid.ndjfl/1093887530}
  }  
\label{thm:latoc_char3}
%---------------------------------------
Let $\latL\eqd\latocd$ be an algebraic structure.
\thmbox{
  \brbr{\begin{array}{M}
    $\latL$ is an\\
    \structb{orthocomplemented lattice}
  \end{array}}
  \iff
  \brbl{\begin{array}{FlclCD}
    1. & (z^\ocop \meet y^\ocop)^\ocop \join x &=& (x \join y) \join z & \forall x,y,z\in\setX & and \\
    2. & x \meet (x \join y)                   &=& x                   & \forall x,y\in\setX   & and \\
    3. & x \join (y\meet y^\ocop)              &=& x                   & \forall x,y\in\setX.
  \end{array}}
  }
\end{theorem}
\begin{proof}
\begin{enumerate}
  \item Proof that orthocomplemented lattice $\implies$ 3 properties:
    \begin{align*}
      \brp{z^\ocop \meet y^\ocop}^\ocop \join x 
        &= \brs{\brp{z^\ocop}^\ocop \join \brp{y^\ocop}^\ocop} \join x 
        && \text{by \prope{de Morgan} property \xref{thm:latoc_prop}}
      \\&= \brp{z \join y} \join x 
        && \text{by \prope{involutory} property \xref{def:latoc}}
      \\&= x \join \brp{z \join y}
        && \text{by \prope{commutative} property \ifxref{lattice}{thm:lattice}}
      \\&= x \join \brp{y \join z}
        && \text{by \prope{commutative} property \ifxref{lattice}{thm:lattice}}
      \\&= \brp{x \join y} \join z
        && \text{by \prope{associative} property \ifxref{lattice}{thm:lattice}}
      \\\\
      x \meet (x \join y)
        &= x
        && \text{by \prope{absorptive} property \ifxref{lattice}{thm:lattice}}
      \\\\
      x \join (y\meet y^\ocop)
        &= x \join \bzero
        && \text{by \prope{complemented} property \xref{def:latoc}}
      \\&= x
    \end{align*}

  \item Proof that orthocomplemented lattice $\impliedby$ 3 properties:
    \begin{enumerate}
      \item Proof that $\latL$ is \prope{meet-idempotent}:  \label{item:latoc_char3_idempotent}
        \begin{align*}
          x \meet x
            &= x \meet \brs{x\join\brp{y\meet y^\ocop}}
            && \text{by (3)}
          \\&= x \meet \brs{x\join\brp{y\meet y^\ocop}}
            && \text{by (3)}
          \\&= x 
            && \text{by (2)}
        \end{align*}

      \item Define $\bzero\eqd xx^\ocop$ for some $x\in\setX$. 
            Proof that $\bzero$  is the 
            \hie{greatest lower bound} of $\latL$:\label{item:latoc_char3_0}
            The element $\bzero$ is the greatest lower bound if and only if 
            $xx^\ocop=yy^\ocop\quad\forall x,y\in\setX$\ldots
        \begin{enumerate}
          \item Proof that $(xx^\ocop)^{\ocop\ocop}=(xx^\ocop)\quad\forall x\in\setX$: \label{item:latoc_char3_xxooo}
            \begin{align*}
              (xx^\ocop)^{\ocop\ocop}
                &= (xx^\ocop)^{\ocop\ocop} + (xx^\ocop)
                && \text{by (3)}
              \\&= \brs{(xx^\ocop)^\ocop(xx^\ocop)^\ocop}^\ocop + (xx^\ocop)
                && \text{by \pref{item:latoc_char3_idempotent}}
              \\&= \brs{(xx^\ocop) + (xx^\ocop)} + (xx^\ocop)
                && \text{by (1)}
              \\&= \brs{(xx^\ocop)} + (xx^\ocop)
                && \text{by (3)}
              \\&= (xx^\ocop)
                && \text{by (3)}
            \end{align*}

          \item Proof that $a=(xx^\ocop) + a\quad\forall a,x\in\setX$: \label{item:latoc_char3_axxa}
            \begin{align*}
              a &= a + (xx^\ocop)
                && \text{by (3)}
              \\&= \brs{a + (xx^\ocop)} + (xx^\ocop)
                && \text{by (3)}
              \\&= \brs{(xx^\ocop)^\ocop (xx^\ocop)^\ocop}^\ocop + a
                && \text{by (1)}
              \\&= \brs{(xx^\ocop)^\ocop}^\ocop + a
                && \text{by \pref{item:latoc_char3_idempotent}}
              \\&= (xx^\ocop) + a
                && \text{by \pref{item:latoc_char3_xxooo}}
            \end{align*}

          \item Proof that $(xx^\ocop)=(yy^\ocop)\quad\forall x,y\in\setX$: \label{item:latoc_char3_xxyy}
            \begin{align*}
              (xx^\ocop) 
                &= (xx^\ocop) + (yy^\ocop)
                && \text{by (3)}
              \\&= (yy^\ocop)
                && \text{by \pref{item:latoc_char3_axxa}}
            \end{align*}
        \end{enumerate}

      \item Proof that $x +\bzero=\bzero + x=x \quad\forall x\in\setX$ 
            (\prope{join identity}): \label{item:latoc_char3_joinid}
        \begin{align*}
          x + \bzero
            &= x + (yy^\ocop)
            && \text{by \pref{item:latoc_char3_xxyy}}
          \\&= x
            && \text{by (3)}
            \\
          \bzero + x
            &= (uu^\ocop) + x
            && \text{by \pref{item:latoc_char3_xxyy}}
          \\&= x
            && \text{by \pref{item:latoc_char3_axxa}}
        \end{align*}

      \item Proof that $x + y = (y^\ocop x^\ocop)^\ocop \quad\forall x,y\in\setX$: \label{item:latoc_char3_pdemorgan}
        \begin{align*}
          (y^\ocop x^\ocop)^\ocop
            &= (y^\ocop x^\ocop)^\ocop + \bzero
            && \text{by \pref{item:latoc_char3_joinid}}
          \\&= (\bzero + x) + y
            && \text{by (1)}
          \\&= x + y
            && \text{by \pref{item:latoc_char3_joinid}}
        \end{align*}
     
      \item Proof that $x+x=x^{\ocop\ocop}\quad\forall x\in\setX$: \label{item:latoc_char3_xxxoo}
        \begin{align*}
          x + x
            &= (x^\ocop x^\ocop)^\ocop
            && \text{by \pref{item:latoc_char3_pdemorgan}}
          \\&= (x^\ocop)^\ocop
            && \text{by \pref{item:latoc_char3_idempotent}}
        \end{align*}

      \item Proof that $x+y=y+x\quad\forall x,y\in\setX$ 
            \quad(\prope{join-commutative}): \label{item:latoc_char3_jcom}
        \begin{align*}
          x + y
            &= (x+\bzero)+y
            && \text{by \pref{item:latoc_char3_joinid}}
          \\&= (\bzero^\ocop x^\ocop)^\ocop+y
            && \text{by \pref{item:latoc_char3_pdemorgan}}
          \\&= (y+x)+\bzero
            && \text{by (1)}
          \\&= y+x
            && \text{by \pref{item:latoc_char3_joinid}}
        \end{align*}

      \item Proof that $(x+y)+z=x+(y+z)\quad\forall x,y,z\in\setX$ 
            \quad(\prope{join-associative}): \label{item:latoc_char3_jasc}
        \begin{align*}
          (x + y)+z
            &= (z^\ocop y^\ocop)^\ocop + x
            && \text{by (1)}
          \\&= (y + z) + x
            && \text{by \pref{item:latoc_char3_pdemorgan}}
          \\&= x + (y + z)
            && \text{by \pref{item:latoc_char3_jcom}}
        \end{align*}

      \item Proof that $x^{\ocop\ocop}=x\quad\forall x\in\setX$ 
            \quad(\prope{involutory}): \label{item:latoc_char3_dc}
        \begin{align*}
          x^{\ocop\ocop}
            &= \brp{x^\ocop}\ocop
            && \text{by definition of $x^{\ocop\ocop}$}
          \\&= \brs{x^\ocop\brp{x^\ocop+x}}\ocop
            && \text{by (2)}
          \\&= \brs{x^\ocop\brp{x^\ocop x^{\ocop\ocop}}^\ocop}\ocop
            && \text{by \pref{item:latoc_char3_pdemorgan}}
          \\&= \brp{x^\ocop x^{\ocop\ocop}} + x
            && \text{by \pref{item:latoc_char3_pdemorgan}}
          \\&= \brp{\bzero} + x
            && \text{by \pref{item:latoc_char3_0}}
          \\&= x
            && \text{by \pref{item:latoc_char3_joinid}}
        \end{align*}

      \item Proof of \prope{de Morgan}'s laws:\label{item:latoc_char3_demorgan}
        \begin{align*}
          (x+y)^\ocop
            &= (y+x)^\ocop
            && \text{by \pref{item:latoc_char3_jasc}}
          \\&= \brs{\brp{x^\ocop y^\ocop}^\ocop}^\ocop
            && \text{by \pref{item:latoc_char3_pdemorgan}}
          \\&= x^\ocop y^\ocop
            && \text{by \pref{item:latoc_char3_dc}}
          \\\\
          \brp{xy}^\ocop
            &= \brp{x^{\ocop\ocop}y^{\ocop\ocop}}^\ocop
            && \text{by \pref{item:latoc_char3_dc}}
          \\&= y^\ocop + x^\ocop
            && \text{by \pref{item:latoc_char3_pdemorgan}}
          \\&= x^\ocop + y^\ocop
            && \text{by \pref{item:latoc_char3_jasc}}
        \end{align*}

      \item Proof that $(xy)z=x(yz)\quad\forall x,y,z\in\setX$ 
            \quad(\prope{meet-commutative}): \label{item:latoc_char3_mcom}
        \begin{align*}
          xy
            &= \brp{xy}^{\ocop\ocop}
            && \text{by \pref{item:latoc_char3_dc}}
          \\&= \brp{x^\ocop+y^\ocop}^\ocop
            && \text{by \pref{item:latoc_char3_demorgan}}
          \\&= \brp{y^\ocop+x^\ocop}^\ocop
            && \text{by \pref{item:latoc_char3_jasc}}
          \\&= y^{\ocop\ocop} x^{\ocop\ocop}
            && \text{by \pref{item:latoc_char3_demorgan}}
          \\&= y x
            && \text{by \pref{item:latoc_char3_demorgan}}
        \end{align*}

      \item Proof that $(xy)z=x(yz)\quad\forall x,y,z\in\setX$ 
            \quad(\prope{meet-associative}): \label{item:latoc_char3_masc}
        \begin{align*}
          (xy)z
            &= \brs{(xy)z}{^\ocop\ocop}
            && \text{by \pref{item:latoc_char3_dc}}
          \\&= \brs{(xy)^\ocop+z^\ocop}^\ocop
            && \text{by \pref{item:latoc_char3_demorgan}}
          \\&= \brs{\brp{x^\ocop+y^\ocop}+z^\ocop}^\ocop
            && \text{by \pref{item:latoc_char3_demorgan}}
          \\&= \brs{x^\ocop+\brp{y^\ocop+z^\ocop}}^\ocop
            && \text{by \pref{item:latoc_char3_jasc}}
          \\&= x^{\ocop\ocop} \brp{y^\ocop+z^\ocop}^\ocop
            && \text{by \pref{item:latoc_char3_demorgan}}
          \\&= x^{\ocop\ocop} \brp{y^{\ocop\ocop}z^{\ocop\ocop}}
            && \text{by \pref{item:latoc_char3_demorgan}}
          \\&= x \brp{yz}
            && \text{by \pref{item:latoc_char3_dc}}
        \end{align*}

      \item Proof that $x+(xz)=x$ (\prope{join-meet-absorptive}):  \label{item:latoc_char3_jmab}
        \begin{align*}
          x\join(xz)
            &= \brs{x+(xz)}^{\ocop\ocop}
            && \text{by \pref{item:latoc_char3_dc}}
          \\&= \brs{x^\ocop(xz)^\ocop}^\ocop
            && \text{by \pref{item:latoc_char3_demorgan}}
          \\&= \brs{x^\ocop \brp{x^\ocop+z^\ocop}}^\ocop
            && \text{by \pref{item:latoc_char3_demorgan}}
          \\&= \brs{x^\ocop}^\ocop
            && \text{by (2)}
          \\&= x
            && \text{by \pref{item:latoc_char3_dc}}
        \end{align*}

      \item Because $\latL$ is   \label{item:latoc_char3_lattice}
              \prope{commutative} (\pref{item:latoc_char3_jcom} and \pref{item:latoc_char3_mcom}),
              \prope{associative} (\pref{item:latoc_char3_jasc} and \pref{item:latoc_char3_masc}), and
              \prope{absorptive}  ((2) and \pref{item:latoc_char3_jmab})\ifdochas{lattice}{, and by \prefpp{thm:lat_char_6e3v}},
            $\latL$ is a \emph{lattice}.

      \item Define $\bid\eqd x+x^\ocop$ for some $x\in\setX$. 
            Proof that $\bid$  is the 
            \hie{least upper bound} of $\latL$:\label{item:latoc_char3_1}
            The element $\bid$ is the least upper bound if and only if 
            $x+x^\ocop=y+y^\ocop\quad\forall x,y\in\setX$\ldots
        \begin{align*}
          \bid
            &= \brp{x+x^\ocop}
            && \text{by definition of $\bid$}
          \\&= \brp{x+x^\ocop}^{\ocop\ocop}
            && \text{by \pref{item:latoc_char3_dc}}
          \\&= \brp{x^\ocop x}^\ocop
            && \text{by \pref{item:latoc_char3_dc}}
          \\&= \brp{xx^\ocop}^\ocop
            && \text{by \pref{item:latoc_char3_mcom}}
          \\&= \brp{yy^\ocop}^\ocop
            && \text{by \pref{item:latoc_char3_xxyy}}
          \\&= y^\ocop+y^{\ocop\ocop}
            && \text{by \pref{item:latoc_char3_demorgan}}
          \\&= y^\ocop+y
            && \text{by \pref{item:latoc_char3_dc}}
          \\&= y+y^\ocop
            && \text{by \pref{item:latoc_char3_jcom}}
        \end{align*}

      \item Proof that $\latL$ is \prope{antitone}:\label{item:latoc_char3_cp}
            by \prefpp{thm:latn_demorgan}.

      \item Proof that $\latL$ is \prope{complemented}:\label{item:latoc_char3_complemented}
            by \pref{item:latoc_char3_xxyy} and \pref{item:latoc_char3_1}.

      \item Because $\latL$ is a \prope{bounded} 
            (\pref{item:latoc_char3_0} and \pref{item:latoc_char3_1})
            lattice (\pref{item:latoc_char3_lattice}),
            and because $\latL$ 
              is \prope{complemented}         (\pref{item:latoc_char3_complemented}),
              is \prope{involutory}  (\pref{item:latoc_char3_dc}), and
              is \prope{antitone}       (\pref{item:latoc_char3_cp}),
            and by \prefpp{def:latoc},
            $\latL$ is an \prope{orthocomplemented} \emph{lattice}.
    \end{enumerate}
\end{enumerate}
\end{proof}



%2014jun15sun moved to relneg.tex%%=======================================
%2014jun15sun moved to relneg.tex%\subsection{Orthogonality}
%2014jun15sun moved to relneg.tex%%=======================================
%2014jun15sun moved to relneg.tex%%---------------------------------------
%2014jun15sun moved to relneg.tex%\begin{proposition}
%2014jun15sun moved to relneg.tex%\label{prop:latoc_x_orel_y}
%2014jun15sun moved to relneg.tex%%---------------------------------------
%2014jun15sun moved to relneg.tex%Let $\latocd$ be an \structe{orthocomplemented lattice} \xref{def:latoc}.
%2014jun15sun moved to relneg.tex%\propbox{
%2014jun15sun moved to relneg.tex%  x \orel y
%2014jun15sun moved to relneg.tex%  \qquad\implies\qquad 
%2014jun15sun moved to relneg.tex%  \brb{\begin{array}{rclccD}
%2014jun15sun moved to relneg.tex%    x^\ocop &\join& y       &=& \lid   & and \\
%2014jun15sun moved to relneg.tex%    x       &\meet& y^\ocop &=& \lzero &  \\
%2014jun15sun moved to relneg.tex%  \end{array}}
%2014jun15sun moved to relneg.tex%  \qquad\scy\forall x,y\in\setX
%2014jun15sun moved to relneg.tex%  }
%2014jun15sun moved to relneg.tex%\end{proposition}
%2014jun15sun moved to relneg.tex%\begin{proof}
%2014jun15sun moved to relneg.tex%\begin{align*}
%2014jun15sun moved to relneg.tex%  x \orel y
%2014jun15sun moved to relneg.tex%    &\implies x\join x^\ocop \orel y\join x^\ocop
%2014jun15sun moved to relneg.tex%    && \text{by \prope{monotone} property of \structe{lattice}s \xref{prop:latmono}}
%2014jun15sun moved to relneg.tex%  \\&\implies \lid \orel y\join x^\ocop
%2014jun15sun moved to relneg.tex%    && \text{by \prope{excluded middle} property of \structe{ortho lattice}s \xref{def:latoc}}
%2014jun15sun moved to relneg.tex%  \\&\implies x^\ocop\join y = \lid
%2014jun15sun moved to relneg.tex%    && \text{by \prope{upper bounded} property of \structe{bounded lattices} \xref{def:latb}}
%2014jun15sun moved to relneg.tex%  \\
%2014jun15sun moved to relneg.tex%  x \orel y
%2014jun15sun moved to relneg.tex%    &\implies x\meet y^\ocop \orel y\meet y^\ocop
%2014jun15sun moved to relneg.tex%    && \text{by \prope{monotone} property of \structe{lattice}s \xref{prop:latmono}}
%2014jun15sun moved to relneg.tex%  \\&\implies x\meet y^\ocop \orel \lzero
%2014jun15sun moved to relneg.tex%    && \text{by \prope{non-contradiction} property of \structe{ortho lattice}s \xref{def:latoc}}
%2014jun15sun moved to relneg.tex%  \\&\implies x\meet y^\ocop = \lzero
%2014jun15sun moved to relneg.tex%    && \text{by \prope{lower bounded} property of \structe{bounded lattices} \xref{def:latb}}
%2014jun15sun moved to relneg.tex%\end{align*}
%2014jun15sun moved to relneg.tex%\end{proof}
%2014jun15sun moved to relneg.tex%
%2014jun15sun moved to relneg.tex%%---------------------------------------
%2014jun15sun moved to relneg.tex%\begin{definition}
%2014jun15sun moved to relneg.tex%\footnote{
%2014jun15sun moved to relneg.tex%  \citerpg{stern1999}{12}{0521461057}\\
%2014jun15sun moved to relneg.tex%  %\citerp{holland1963}{67}\\
%2014jun15sun moved to relneg.tex%  \citorp{loomis1955}{3}
%2014jun15sun moved to relneg.tex%  }
%2014jun15sun moved to relneg.tex%\label{def:latoc_orthog}
%2014jun15sun moved to relneg.tex%%---------------------------------------
%2014jun15sun moved to relneg.tex%Let $\latocd$ be an \structe{orthocomplemented lattice} \xref{def:latoc}.
%2014jun15sun moved to relneg.tex%\defbox{\begin{array}{M}%
%2014jun15sun moved to relneg.tex%  The \hid{orthogonality} relation $\symxd{\orthog}\in\clRxx$ is defined as
%2014jun15sun moved to relneg.tex%  \\\indentx$
%2014jun15sun moved to relneg.tex%      x\orthog y 
%2014jun15sun moved to relneg.tex%      \qquad\iffdef\qquad
%2014jun15sun moved to relneg.tex%      x \orel y^\orthog
%2014jun15sun moved to relneg.tex%    $
%2014jun15sun moved to relneg.tex%  \\
%2014jun15sun moved to relneg.tex%  If $x\orthog y$, we say that $x$ is \hid{orthogonal} to $y$.
%2014jun15sun moved to relneg.tex%\end{array}}
%2014jun15sun moved to relneg.tex%\end{definition}
%2014jun15sun moved to relneg.tex%
%2014jun15sun moved to relneg.tex%%---------------------------------------
%2014jun15sun moved to relneg.tex%\begin{lemma}
%2014jun15sun moved to relneg.tex%\footnote{
%2014jun15sun moved to relneg.tex%  \citerp{holland1963}{67}
%2014jun15sun moved to relneg.tex%  }
%2014jun15sun moved to relneg.tex%\label{lem:latoc_orthog}
%2014jun15sun moved to relneg.tex%%---------------------------------------
%2014jun15sun moved to relneg.tex%Let $\latocd$ be an \structe{orthocomplemented lattice} \xref{def:latoc}.
%2014jun15sun moved to relneg.tex%\lembox{
%2014jun15sun moved to relneg.tex%  %\brb{\begin{array}{M}
%2014jun15sun moved to relneg.tex%    \mcom{x \orthog y}{\prope{orthogonal} \xref{def:latoc_orthog}}
%2014jun15sun moved to relneg.tex%  %\end{array}}
%2014jun15sun moved to relneg.tex%  \qquad\implies\qquad 
%2014jun15sun moved to relneg.tex%  \brb{\begin{array}{FlclDD}
%2014jun15sun moved to relneg.tex%    1. & y \orthog x           & &         & (\prope{symmetric}) & and\\
%2014jun15sun moved to relneg.tex%    2. & x \meet y             &=& \lzero  &                     & and\\
%2014jun15sun moved to relneg.tex%    3. & x^\ocop \join y^\ocop &=& \lid
%2014jun15sun moved to relneg.tex%  \end{array}}
%2014jun15sun moved to relneg.tex%  }
%2014jun15sun moved to relneg.tex%\end{lemma}
%2014jun15sun moved to relneg.tex%\begin{proof}
%2014jun15sun moved to relneg.tex%\begin{align*}
%2014jun15sun moved to relneg.tex%  x \orthog y
%2014jun15sun moved to relneg.tex%    &\implies x \orel y^\ocop
%2014jun15sun moved to relneg.tex%    && \text{by definition of $\orthog$ \xref{def:latoc_orthog}}
%2014jun15sun moved to relneg.tex%  \\&\implies \brp{y^\ocop}^\ocop \orel x^\ocop
%2014jun15sun moved to relneg.tex%    && \text{by \prope{antitone} property \xref{def:latoc}}
%2014jun15sun moved to relneg.tex%  \\&\implies y \orel x^\ocop
%2014jun15sun moved to relneg.tex%    && \text{by \prope{involutory} property \xref{def:latoc}}
%2014jun15sun moved to relneg.tex%  \\&\implies y \orthog x
%2014jun15sun moved to relneg.tex%    && \text{by definition of $\orthog$ \xref{def:latoc_orthog}}
%2014jun15sun moved to relneg.tex%  \\
%2014jun15sun moved to relneg.tex%  \\
%2014jun15sun moved to relneg.tex%  x \orthog y
%2014jun15sun moved to relneg.tex%    &\implies x \orel y^\ocop
%2014jun15sun moved to relneg.tex%    && \text{by definition of $\orthog$ \xref{def:latoc_orthog}}
%2014jun15sun moved to relneg.tex%  \\&\implies x \meet y \orel y^\ocop \meet y
%2014jun15sun moved to relneg.tex%    && \text{by \prope{monotone} property of \structe{lattice}s \xref{prop:latmono}}
%2014jun15sun moved to relneg.tex%  \\&\implies x \meet y \orel y \meet y^\ocop 
%2014jun15sun moved to relneg.tex%    && \text{by \prope{commutative} property of lattices \ifxref{lattice}{thm:lattice}}
%2014jun15sun moved to relneg.tex%  \\&= \bzero
%2014jun15sun moved to relneg.tex%    && \text{by definition of orthocomplement \xref{def:latoc}}
%2014jun15sun moved to relneg.tex%  \\
%2014jun15sun moved to relneg.tex%  \\
%2014jun15sun moved to relneg.tex%  x \orthog y
%2014jun15sun moved to relneg.tex%    &\implies x \orel y^\ocop
%2014jun15sun moved to relneg.tex%    && \text{by definition of $\orthog$ \xref{def:latoc_orthog}}
%2014jun15sun moved to relneg.tex%  \\&\implies x^\ocop \join x  \orel x^\ocop \join y^\ocop
%2014jun15sun moved to relneg.tex%    && \text{by \prope{monotone} property of \structe{lattice}s \xref{prop:latmono}}
%2014jun15sun moved to relneg.tex%  \\&\implies x \join x^\ocop  \orel x^\ocop \join y^\ocop
%2014jun15sun moved to relneg.tex%    && \text{by \prope{commutative} property of \structe{lattices} \xref{thm:lattice}}
%2014jun15sun moved to relneg.tex%  \\&\implies \lid \orel x^\ocop \join y^\ocop 
%2014jun15sun moved to relneg.tex%    && \text{by \prope{excluded middle} property of \structe{ortho lattice}s \xref{thm:latoc_prop}}
%2014jun15sun moved to relneg.tex%  \\&\implies x^\ocop \join y^\ocop
%2014jun15sun moved to relneg.tex%    && \text{by \prope{upper bound} property of \structe{bounded lattice}s \xref{def:latb}}
%2014jun15sun moved to relneg.tex%\end{align*}
%2014jun15sun moved to relneg.tex%\end{proof}
%2014jun15sun moved to relneg.tex%
%2014jun15sun moved to relneg.tex%%---------------------------------------
%2014jun15sun moved to relneg.tex%\begin{example}
%2014jun15sun moved to relneg.tex%%---------------------------------------
%2014jun15sun moved to relneg.tex%\exboxp{
%2014jun15sun moved to relneg.tex%In the \structe{$O_6$ lattice} \xref{def:o6}, there are a total of $\bcoef{6}{2}=\frac{6!}{(6-2)!2!}=\frac{6\times5}{2}=15$ 
%2014jun15sun moved to relneg.tex%distinct unordered (the $\orthog$ relation is \prope{symmetric} by \prefp{lem:latoc_orthog} so the order doesn't matter) pairs of elements.
%2014jun15sun moved to relneg.tex%\\\begin{tabular}{m{86mm}l}
%2014jun15sun moved to relneg.tex%  Of these 15 pairs, 8 are orthogonal to each other, and $\bzero$ is orthogonal to itself, making a total of 9 orthogonal pairs: &
%2014jun15sun moved to relneg.tex%  $\begin{array}{|lcl|lcl|lcl|}
%2014jun15sun moved to relneg.tex%    \hline
%2014jun15sun moved to relneg.tex%    x &\orthog& y          & x &\orthog& \bzero        & y^\ocop &\orthog& \bzero\\
%2014jun15sun moved to relneg.tex%    x &\orthog& x^\ocop    & y &\orthog& \bzero        & \bid&\orthog&\bzero\\
%2014jun15sun moved to relneg.tex%    y &\orthog& y^\ocop    & x^\ocop &\orthog& \bzero  & \bzero&\orthog&\bzero\\
%2014jun15sun moved to relneg.tex%    \hline
%2014jun15sun moved to relneg.tex%  \end{array}$
%2014jun15sun moved to relneg.tex%\end{tabular}
%2014jun15sun moved to relneg.tex%  }
%2014jun15sun moved to relneg.tex%\end{example}
%2014jun15sun moved to relneg.tex%
%2014jun15sun moved to relneg.tex%%---------------------------------------
%2014jun15sun moved to relneg.tex%\begin{example}
%2014jun15sun moved to relneg.tex%%---------------------------------------
%2014jun15sun moved to relneg.tex%\exboxp{
%2014jun15sun moved to relneg.tex%In lattice 5 of \prefpp{ex:latoc}, 
%2014jun15sun moved to relneg.tex%there are a total of $\bcoef{10}{2}=\frac{10!}{(10-2)!2!}=\frac{10\times9}{2}=45$ 
%2014jun15sun moved to relneg.tex%distinct unordered pairs of elements.
%2014jun15sun moved to relneg.tex%\\\begin{tabular}{m{66mm}l}
%2014jun15sun moved to relneg.tex%  Of these 45 pairs, 18 are orthogonal to each other, and $\bzero$ is orthogonal to itself, making a total of 19 orthogonal pairs: &
%2014jun15sun moved to relneg.tex%  $\begin{array}{|lcl|lcl|lcl|lcl|}
%2014jun15sun moved to relneg.tex%    \hline
%2014jun15sun moved to relneg.tex%    p &\orthog& p^\ocop    & x &\orthog& x^\ocop    & y       &\orthog& z          & x^\ocop &\orthog& \bzero\\
%2014jun15sun moved to relneg.tex%    p &\orthog& x^\ocop    & x &\orthog& y          & y       &\orthog& \bzero     & y^\ocop &\orthog& \bzero\\
%2014jun15sun moved to relneg.tex%    p &\orthog& y          & x &\orthog& z          & z       &\orthog& z^\ocop    & z^\ocop &\orthog& \bzero\\
%2014jun15sun moved to relneg.tex%    p &\orthog& z          & x &\orthog& \bzero     & z       &\orthog& \bzero     & \bzero  &\orthog& \bzero\\
%2014jun15sun moved to relneg.tex%    p &\orthog& \bzero     & y &\orthog& y^\ocop    & p^\ocop &\orthog& \bzero     &         &       &       \\
%2014jun15sun moved to relneg.tex%    \hline
%2014jun15sun moved to relneg.tex%  \end{array}$
%2014jun15sun moved to relneg.tex%\end{tabular}
%2014jun15sun moved to relneg.tex%  }
%2014jun15sun moved to relneg.tex%\end{example}
%2014jun15sun moved to relneg.tex%
%2014jun15sun moved to relneg.tex%%---------------------------------------
%2014jun15sun moved to relneg.tex%\begin{example}
%2014jun15sun moved to relneg.tex%%---------------------------------------
%2014jun15sun moved to relneg.tex%\exboxt{
%2014jun15sun moved to relneg.tex%In the \structb{$\R^3$ Euclidean space} illustrated in \prefpp{ex:latoc_r3},
%2014jun15sun moved to relneg.tex%\\\indentx$\begin{array}{lcl@{\qquad}lcl}
%2014jun15sun moved to relneg.tex%    \spX\subseteq\spY^\ocop &\implies& \spX\orthog\spY & \spY\subseteq\spX^\ocop &\implies& \spY\orthog\spX \\
%2014jun15sun moved to relneg.tex%    \spX\subseteq\spZ^\ocop &\implies& \spX\orthog\spZ & \spY\subseteq\spZ^\ocop &\implies& \spY\orthog\spZ \\
%2014jun15sun moved to relneg.tex%    \mc{6}{l}{\spX\meet\spY=\spX\meet\spZ=\spY\meet\spZ=\spZero}
%2014jun15sun moved to relneg.tex%  \end{array}$
%2014jun15sun moved to relneg.tex%}
%2014jun15sun moved to relneg.tex%\end{example}%


%2014jun15sun moved to negation.tex% %=======================================
%2014jun15sun moved to negation.tex% \subsection{Projections}
%2014jun15sun moved to negation.tex% %=======================================
%2014jun15sun moved to negation.tex% %---------------------------------------
%2014jun15sun moved to negation.tex% \begin{definition}
%2014jun15sun moved to negation.tex% \footnote{
%2014jun15sun moved to negation.tex%   \citorppc{nakamura1957}{158}{159}{equation (S)} \\
%2014jun15sun moved to negation.tex%   \citorpc{sasaki1954}{300}{Def.5.1, cf Foulis 1962}\\
%2014jun15sun moved to negation.tex%   \citerpg{kalmbach1983}{117}{0123945801}\\
%2014jun15sun moved to negation.tex%   }
%2014jun15sun moved to negation.tex% \label{def:sasaki}
%2014jun15sun moved to negation.tex% %---------------------------------------
%2014jun15sun moved to negation.tex% Let $\latL\eqd\latocd$ be an \structe{orthocomplemented lattice} \xref{def:latoc}. % \xref{def:latoc}.
%2014jun15sun moved to negation.tex% \defbox{\begin{array}{M}
%2014jun15sun moved to negation.tex%   A function $\sasakin{x}\in\clFxx$ is a \hid{Sasaki projection} on $x\in\setX$ if 
%2014jun15sun moved to negation.tex%   \\\indentx$ \sasaki{x}{y} = \brp{y \join x^\ocop} \meet x$.
%2014jun15sun moved to negation.tex%   \\
%2014jun15sun moved to negation.tex%   The \ope{Sasaki projection}s $\sasakin{x}$ and $\sasakin{y}$ are \hid{permutable} if 
%2014jun15sun moved to negation.tex%   \\\indentx$\sasakin{x}\circ\sasakin{y}(u)=\sasakin{y}\circ\sasakin{x}(u) \quad\forall u\in\setX$.
%2014jun15sun moved to negation.tex% \end{array}}
%2014jun15sun moved to negation.tex% \end{definition}
%2014jun15sun moved to negation.tex% 
%2014jun15sun moved to negation.tex% %---------------------------------------
%2014jun15sun moved to negation.tex% \begin{example}
%2014jun15sun moved to negation.tex% %---------------------------------------
%2014jun15sun moved to negation.tex% \exboxp{
%2014jun15sun moved to negation.tex% Here are some examples of projections in the \structe{$O_6$ lattice} \xref{def:o6} onto the element $x$:
%2014jun15sun moved to negation.tex% \\\indentx$\begin{array}{*{6}{lc}lD}
%2014jun15sun moved to negation.tex%     \sasaki{x}{y}       &\eqd& (y       &\join& x^\ocop) &\meet& x &=& x^\ocop &\meet& x &=& \bzero & (because $x \orthog y$)\\
%2014jun15sun moved to negation.tex%     \sasaki{x}{x^\ocop} &\eqd& (x^\ocop &\join& x^\ocop) &\meet& x &=& x^\ocop &\meet& x &=& \bzero & (because $x \orthog x^\ocop$)\\
%2014jun15sun moved to negation.tex%     \sasaki{x}{y^\ocop} &\eqd& (y^\ocop &\join& x^\ocop) &\meet& x &=& \bid    &\meet& x &=& x      & (because $x \le y^\ocop$)\\
%2014jun15sun moved to negation.tex%     \sasaki{x}{\bid}    &\eqd& (\bid    &\join& x^\ocop) &\meet& x &=& \bid    &\meet& x &=& x      & (because $x \le \bid$)\\
%2014jun15sun moved to negation.tex%     \sasaki{x}{\bzero}  &\eqd& (\bzero  &\join& x^\ocop) &\meet& x &=& x^\ocop &\meet& x &=& \bzero & (because $x \orthog \bzero$)
%2014jun15sun moved to negation.tex%   \end{array}$
%2014jun15sun moved to negation.tex%   }
%2014jun15sun moved to negation.tex% \end{example}
%2014jun15sun moved to negation.tex% 
%2014jun15sun moved to negation.tex% %---------------------------------------
%2014jun15sun moved to negation.tex% \begin{example}
%2014jun15sun moved to negation.tex% %---------------------------------------
%2014jun15sun moved to negation.tex% \exboxp{
%2014jun15sun moved to negation.tex% Here are some examples of projections in lattice 5 of \prefpp{ex:latoc}:
%2014jun15sun moved to negation.tex% \\\indentx$\begin{array}{*{6}{lc}lD}
%2014jun15sun moved to negation.tex%     \sasaki{x}{p}       &\eqd& (p       &\join& x^\ocop) &\meet& x &=& \bid    &\meet& x &=& x      & \\
%2014jun15sun moved to negation.tex%     \sasaki{x}{y}       &\eqd& (y       &\join& x^\ocop) &\meet& x &=& x^\ocop &\meet& x &=& \bzero & (because $x \orthog y$)\\
%2014jun15sun moved to negation.tex%     \sasaki{x}{z}       &\eqd& (z       &\join& x^\ocop) &\meet& x &=& x^\ocop &\meet& x &=& \bzero & (because $x \orthog z$)\\
%2014jun15sun moved to negation.tex%     \sasaki{x}{p^\ocop} &\eqd& (p^\ocop &\join& x^\ocop) &\meet& x &=& p^\ocop &\meet& x &=& \bzero & \\
%2014jun15sun moved to negation.tex%     \sasaki{x}{x^\ocop} &\eqd& (x^\ocop &\join& x^\ocop) &\meet& x &=& x^\ocop &\meet& x &=& \bzero & (because $x \orthog x^\ocop$)\\
%2014jun15sun moved to negation.tex%     \sasaki{x}{y^\ocop} &\eqd& (y^\ocop &\join& x^\ocop) &\meet& x &=& \bid    &\meet& x &=& x      & (because $x \le y^\ocop$)\\
%2014jun15sun moved to negation.tex%     \sasaki{x}{z^\ocop} &\eqd& (z^\ocop &\join& x^\ocop) &\meet& x &=& \bid    &\meet& x &=& x      & (because $x \le z^\ocop$)\\
%2014jun15sun moved to negation.tex%     \sasaki{x}{\bid}    &\eqd& (\bid    &\join& x^\ocop) &\meet& x &=& \bid    &\meet& x &=& x      & (because $x \le \bid$)\\
%2014jun15sun moved to negation.tex%     \sasaki{x}{\bzero}  &\eqd& (\bzero  &\join& x^\ocop) &\meet& x &=& x^\ocop &\meet& x &=& \bzero & (because $x \orthog \bzero$)
%2014jun15sun moved to negation.tex%   \end{array}$
%2014jun15sun moved to negation.tex%   }
%2014jun15sun moved to negation.tex% \end{example}
%2014jun15sun moved to negation.tex% 
%2014jun15sun moved to negation.tex% 
%2014jun15sun moved to negation.tex% %---------------------------------------
%2014jun15sun moved to negation.tex% \begin{example}
%2014jun15sun moved to negation.tex% \label{ex:latoc_r3_projection}
%2014jun15sun moved to negation.tex% %---------------------------------------
%2014jun15sun moved to negation.tex% %\exbox{\begin{tabular}{m{\tw-73mm}|m{54mm}}
%2014jun15sun moved to negation.tex% \exbox{\begin{tabular}{lm{54mm}}
%2014jun15sun moved to negation.tex%   \parbox{\tw-73mm}{%
%2014jun15sun moved to negation.tex%   Let $\R^3$ be the \structe{3-dimensional Euclidean space} \xref{ex:latoc_r3}
%2014jun15sun moved to negation.tex%   with subspaces $\spZ$ and $\spV$.
%2014jun15sun moved to negation.tex%   Then the projection operator $P_{\spZ^\ocop}$ onto $\spZ^\ocop$ is a \ope{sasaki projection} $\sasakin{\spZ^\ocop}$.
%2014jun15sun moved to negation.tex%   In particular
%2014jun15sun moved to negation.tex%     \\\indentx$\begin{array}{rcl}
%2014jun15sun moved to negation.tex%       \opP_{\spZ^\ocop}\spV &\eqd& \sasaki{\spZ^\ocop}{\spV} 
%2014jun15sun moved to negation.tex%                           \\&\eqd& \brp{\spV + \spZ^{\ocop\ocop}}\seti\spZ^\ocop
%2014jun15sun moved to negation.tex%                           \\&=&    \brp{\spV + \spZ}\seti\spZ^\ocop
%2014jun15sun moved to negation.tex%     \end{array}$\\
%2014jun15sun moved to negation.tex%   as illustrated to the right.}
%2014jun15sun moved to negation.tex%   &\centering\psset{unit=8mm}\footnotesize%============================================================================
% Daniel J. Greenhoe
% LaTeX file
%============================================================================
{%
\begin{pspicture}(-3.6,-2.1)(3.6,2.5)%
  \psset{%
    fillstyle=solid,%
    linewidth=1pt,
    }%
  %------------------------------------
  % subspaces
  %------------------------------------
  %
  %\pspolygon[linecolor=red](-1,-1)(-1,-3.5)(1,-1.5)(1,1)% lower half subspace Y'
  \psline[linecolor=purple,linewidth=2pt]{->}(0,0)(0,-1.75)% lower subspace Z
  \pspolygon[linecolor=red](-3.5,-1)(-3.5,-2)(3.5,0)(3.5,1)% lower V+Z
  \psline[linewidth=2pt]{->}(0,0)(-1.75,-1.25)% left subspace V
  \pspolygon[linecolor=blue](-3.5,-1)(-1.5,1)(3.5,1)(1.5,-1)% subspace Z' (xy plane)
  \pspolygon[linecolor=red](-3.5,-1)(-3.5,0)(3.5,2)(3.5,1)% upper V+Z
  %
  \psline[linewidth=2pt,linestyle=dotted](0,0)(-1.38,-0.99)% left subspace V hidden by Z'
  %\psline[linewidth=2pt,linestyle=dotted]{->}(0,0)(-1,-0.5)% left projection onto Z' hidden by Z' plane
  %\psline[linecolor=purple,linewidth=2pt]{->}(0,0)(-1.5,0)% left subspace Y
  %\pspolygon[linecolor=red](-1,-1)(-1,1.5)(1,3.5)(1,1)% upper half subspace Y' 
  %\psline[linecolor=purple,linewidth=2pt]{<->}(-0.75,-0.75)(0.75,0.75)% subspace X
  \psline[linewidth=2pt]{->}(0,0)(0,1.5)% upper subspace Z
  %\psline[linecolor=purple,linewidth=2pt]{->}(0,0)(1.5,0)% right subspace Y
  \psline[linewidth=2pt,linestyle=dotted](0,0)(0,-1)% lower subspace Z hidden by Z'
  %\psline[linecolor=purple,linewidth=2pt,linestyle=dotted](0,0)(-1,0)% left subspace Y
  %
  \psline[linewidth=2pt]{->}(0,0)(1.75,1.25)% right subspace V
  \psline[linewidth=2pt]{<->}(-1.75,-0.5)(1.75,0.5)% right projection onto Z'
  %------------------------------------
  % nodes
  %------------------------------------
  \pnode(-0.5,0.5){Zcpnt}%
  \pnode(0.5,1.5){Ycpnt}%
  \pnode(0.5,0.5){Xpntzzz}%
  %------------------------------------
  % labels
  %------------------------------------
  %\rput[lt](-2,3){\rnode{Zclab}{subspace $\spZ^\ocop$}}%
  %\rput[rt](3,3){\rnode{Yclab}{subspace $\spY^\ocop$}}%
  %\rput[rb](3,-2){\rnode{Xlabzzz}{subspace $\spX$}}%
  %\uput[-90](-0.5,-0.5){$\spX$}%
  %\uput[0](1.5,0){$\spY$}%
  \uput[90](0,1.5){$\spZ$}%
  \rput[tr](3.5,2.4){$\spZ+\spV$}%
  \uput[0](1.75,1.25){$\spV$}%
  \rput[tl](1.75,0.5){$\opP_{\spZ^\ocop}\spV$}%
  %\uput[0](0.5,1.5){subspace $\spX^\ocop$}
  %\uput[0](1,3){$\spY^\ocop$}
  \uput[0](1.5,-1){$\spZ^\ocop$}
  %------------------------------------
  % arrows
  %------------------------------------
  %\ncdiag[linewidth=1pt,fillstyle=none,angleA=-90,angleB=90,arm=0.5,linearc=0.75,linecolor=black]{->}{Zclab}{Zcpnt}%
  %\ncangle[linewidth=1pt,linearc=0.5,fillstyle=none,angleA=-90,angleB=0,linecolor=black]{->}{Yclab}{Ycpnt}%
  %\ncangle[linewidth=1pt,linearc=0.5,fillstyle=none,angleA=90,angleB=0,linecolor=black]{->}{Xlabzzz}{Xpntzzz}%  some bug???
  %\ncline[linewidth=1pt,linearc=0.5,fillstyle=none,angleA=90,angleB=0,linecolor=black]{->}{Xlabzzz}{Xpntzzz}%
\end{pspicture}
}%
%2014jun15sun moved to negation.tex%   \end{tabular}}
%2014jun15sun moved to negation.tex% \end{example}
%2014jun15sun moved to negation.tex% 
%2014jun15sun moved to negation.tex% %2014jun14sat% %---------------------------------------
%2014jun15sun moved to negation.tex% %2014jun14sat% \begin{lemma}
%2014jun15sun moved to negation.tex% %2014jun14sat% \label{lem:numDL}
%2014jun15sun moved to negation.tex% %2014jun14sat% %---------------------------------------
%2014jun15sun moved to negation.tex% %2014jun14sat% Let $\latL_2^\xN$ be a \structe{Boolean lattice} of order $\xN$.
%2014jun15sun moved to negation.tex% %2014jun14sat% \\\lemboxt{
%2014jun15sun moved to negation.tex% %2014jun14sat%   The number of pairwise orthogonal elements in $\latL_2^\xN$ is
%2014jun15sun moved to negation.tex% %2014jun14sat%   \\\indentx
%2014jun15sun moved to negation.tex% %2014jun14sat%     $\brbl{\begin{array}{lD}
%2014jun15sun moved to negation.tex% %2014jun14sat%       0                        & for $\xN<2$\\
%2014jun15sun moved to negation.tex% %2014jun14sat%       1                        & for $\xN=2$\\
%2014jun15sun moved to negation.tex% %2014jun14sat%       3                        & for $\xN=3$\\
%2014jun15sun moved to negation.tex% %2014jun14sat%       \xN + \bcoef{\xN}{\xN-2} & for $\xN>3$
%2014jun15sun moved to negation.tex% %2014jun14sat%     \end{array}}$
%2014jun15sun moved to negation.tex% %2014jun14sat%   }
%2014jun15sun moved to negation.tex% %2014jun14sat% \end{lemma}
%2014jun15sun moved to negation.tex% %2014jun14sat% \begin{proof}
%2014jun15sun moved to negation.tex% %2014jun14sat% 
%2014jun15sun moved to negation.tex% %2014jun14sat% \end{proof}



%2014jun15sun moved to relneg.tex% %=======================================
%2014jun15sun moved to relneg.tex% \subsection{Commutativity}
%2014jun15sun moved to relneg.tex% %=======================================
%2014jun15sun moved to relneg.tex% The \relxe{commutes} relation is defined next.
%2014jun15sun moved to relneg.tex% Motivation for the name ``commutes" is provided by \prefpp{prop:latoc_pxypyx}
%2014jun15sun moved to relneg.tex% which shows that if $x$ commutes with $y$ in a lattice $\latL$, 
%2014jun15sun moved to relneg.tex% then $x$ and $y$ commute in the \ope{Sasaki projection} $\sasaki{x}{y}$ on $\latL$.
%2014jun15sun moved to relneg.tex% %---------------------------------------
%2014jun15sun moved to relneg.tex% \begin{definition}
%2014jun15sun moved to relneg.tex% \footnote{
%2014jun15sun moved to relneg.tex%   %\citorp{nakamura1957}{158}\\
%2014jun15sun moved to relneg.tex%   \citerpg{kalmbach1983}{20}{0123945801}\\
%2014jun15sun moved to relneg.tex%   \citerpc{holland1970}{79}{A. Commutativity}\\
%2014jun15sun moved to relneg.tex%   \citerpc{maeda1958}{227}{Hilfssatz (Lemma) XII.1.2}\\
%2014jun15sun moved to relneg.tex%   \citePpc{sasaki1954}{301}{Def.5.2, cf Foulis 1962}\\ % page 66
%2014jun15sun moved to relneg.tex%   \citePpc{birkhoff1936oct}{833}{``$a=(a\seti x)\setu(a\seti x')$"} % holland p.79
%2014jun15sun moved to relneg.tex%   }
%2014jun15sun moved to relneg.tex% \label{def:latoc_commutes}
%2014jun15sun moved to relneg.tex% %---------------------------------------
%2014jun15sun moved to relneg.tex% Let $\latL\eqd\latocd$ be an \structe{orthocomplemented lattice} \xref{def:latoc}.
%2014jun15sun moved to relneg.tex% \defboxt{
%2014jun15sun moved to relneg.tex%   %For any two elements $x,y\in\setX$, 
%2014jun15sun moved to relneg.tex%   The \reld{commutes} relation $\hxs{\commutes}$ is defined as 
%2014jun15sun moved to relneg.tex%     \\\indentx$x\commutes y \qquad\iffdef\qquad x = \brp{x \meet y} \join \brp{x \meet y^\ocop}\qquad{\scy\forall x,y\in\setX}$,\\
%2014jun15sun moved to relneg.tex%   in which case we say, ``$x$ \hid{commutes} with $y$ in $\latL$".\\
%2014jun15sun moved to relneg.tex%   That is, $\commutes$ is a relation in $\clRxx$ such that
%2014jun15sun moved to relneg.tex%   \\\indentx$
%2014jun15sun moved to relneg.tex%       \commutes \eqd \set{\opair{x}{y}\in\setX^2}
%2014jun15sun moved to relneg.tex%                          {x = \brp{x \meet y} \join \brp{x \meet y^\ocop}} 
%2014jun15sun moved to relneg.tex%     $
%2014jun15sun moved to relneg.tex% % \\A pair $\opair{x}{y}\in\commutes$  is alternatively denoted as either $\opair{x}{y}\commutes$ or $x\commutes y$.
%2014jun15sun moved to relneg.tex%   %and is called a \propd{modular} pair.
%2014jun15sun moved to relneg.tex%   }
%2014jun15sun moved to relneg.tex% \end{definition}
%2014jun15sun moved to relneg.tex% %That is, $x\commutes y \quad\iffdef\quad x = \brp{x \meet y} \join \brp{x \meet y^\ocop}$.
%2014jun15sun moved to relneg.tex% 
%2014jun15sun moved to relneg.tex% %---------------------------------------
%2014jun15sun moved to relneg.tex% \begin{proposition}
%2014jun15sun moved to relneg.tex% \footnote{
%2014jun15sun moved to relneg.tex%   \citerp{holland1963}{67}
%2014jun15sun moved to relneg.tex%   }
%2014jun15sun moved to relneg.tex% \label{prop:commutes}
%2014jun15sun moved to relneg.tex% %---------------------------------------
%2014jun15sun moved to relneg.tex% Let $\latL\eqd\latocd$ be an \structe{orthocomplemented lattice} \xref{def:latoc}.
%2014jun15sun moved to relneg.tex% \propbox{
%2014jun15sun moved to relneg.tex%   \begin{array}{lclC}
%2014jun15sun moved to relneg.tex%     x \commutes \bzero &    &                         & \forall x   \in\latL\\
%2014jun15sun moved to relneg.tex%     \bzero \commutes x &    &                         & \forall x   \in\latL\\
%2014jun15sun moved to relneg.tex%     x \commutes \bid   &    &                         & \forall x   \in\latL\\
%2014jun15sun moved to relneg.tex%     \bid \commutes x   &    &                         & \forall x   \in\latL\\
%2014jun15sun moved to relneg.tex%     x \commutes x      &    &                         & \forall x   \in\latL\\
%2014jun15sun moved to relneg.tex%     x \commutes y      &\iff&     x \commutes y^\ocop & \forall x,y \in\latL\\
%2014jun15sun moved to relneg.tex%     x \orel y          &\implies& x \commutes y       & \forall x,y \in\latL\\
%2014jun15sun moved to relneg.tex%     x \orthog y        &\implies& x \commutes y       & \forall x,y \in\latL
%2014jun15sun moved to relneg.tex%   \end{array}
%2014jun15sun moved to relneg.tex%   }
%2014jun15sun moved to relneg.tex% \end{proposition}
%2014jun15sun moved to relneg.tex% \begin{proof}
%2014jun15sun moved to relneg.tex% \begin{enumerate}
%2014jun15sun moved to relneg.tex%   \item Proof that $x \commutes \bzero$:
%2014jun15sun moved to relneg.tex%     \begin{align*}
%2014jun15sun moved to relneg.tex%       \brp{x\meet \bzero} \join \brp{x \meet \bzero^\ocop}
%2014jun15sun moved to relneg.tex%         &= \brp{\bzero} \join \brp{x \meet \bzero^\ocop}
%2014jun15sun moved to relneg.tex%         && \text{by definition of $\meet$ \xref{def:meet}}
%2014jun15sun moved to relneg.tex%       \\&= \brp{\bzero} \join \brp{x \meet \bid}
%2014jun15sun moved to relneg.tex%         && \text{by definition of $\ocop$ \xref{def:ocop}}
%2014jun15sun moved to relneg.tex%       \\&= \brp{\bzero} \join \brp{x}
%2014jun15sun moved to relneg.tex%         && \text{by definition of $\meet$ \xref{def:meet}}
%2014jun15sun moved to relneg.tex%       \\&= x
%2014jun15sun moved to relneg.tex%         && \text{by definition of $\join$ \xref{def:join}}
%2014jun15sun moved to relneg.tex%       \\\implies& x \commutes \bzero
%2014jun15sun moved to relneg.tex%         && \text{by definition of $\commutes$ relation \xref{def:latoc_commutes}}
%2014jun15sun moved to relneg.tex%     \end{align*}
%2014jun15sun moved to relneg.tex% 
%2014jun15sun moved to relneg.tex%   \item Proof that $\bzero \commutes x$:
%2014jun15sun moved to relneg.tex%     \begin{align*}
%2014jun15sun moved to relneg.tex%       \brp{\bzero \meet x} \join \brp{\bzero \meet x^\ocop}
%2014jun15sun moved to relneg.tex%         &= \brp{\bzero} \join \brp{\bzero}
%2014jun15sun moved to relneg.tex%         && \text{by definition of $\meet$ \xref{def:meet}}
%2014jun15sun moved to relneg.tex%       \\&= \bzero
%2014jun15sun moved to relneg.tex%         && \text{by definition of $\join$ \xref{def:join}}
%2014jun15sun moved to relneg.tex%       \\\implies& \bzero \commutes x
%2014jun15sun moved to relneg.tex%         && \text{by definition of $\commutes$ relation \xref{def:latoc_commutes}}
%2014jun15sun moved to relneg.tex%     \end{align*}
%2014jun15sun moved to relneg.tex% 
%2014jun15sun moved to relneg.tex%   \item Proof that $x \commutes \bid$:
%2014jun15sun moved to relneg.tex%     \begin{align*}
%2014jun15sun moved to relneg.tex%       \brp{x\meet \bid} \join \brp{x \meet \bid^\ocop}
%2014jun15sun moved to relneg.tex%         &= \brp{x} \join \brp{x \meet \bid^\ocop}
%2014jun15sun moved to relneg.tex%         && \text{by definition of $\meet$ \xref{def:meet}}
%2014jun15sun moved to relneg.tex%       \\&= \brp{x} \join \brp{x \meet \bzero}
%2014jun15sun moved to relneg.tex%         && \text{by definition of $\ocop$ \xref{def:ocop}}
%2014jun15sun moved to relneg.tex%       \\&= \brp{x} \join \brp{\bzero}
%2014jun15sun moved to relneg.tex%         && \text{by definition of $\meet$ \xref{def:meet}}
%2014jun15sun moved to relneg.tex%       \\&= x
%2014jun15sun moved to relneg.tex%         && \text{by definition of $\join$ \xref{def:join}}
%2014jun15sun moved to relneg.tex%       \\\implies& x \commutes \bid
%2014jun15sun moved to relneg.tex%         && \text{by definition of $\commutes$ relation \xref{def:latoc_commutes}}
%2014jun15sun moved to relneg.tex%     \end{align*}
%2014jun15sun moved to relneg.tex% 
%2014jun15sun moved to relneg.tex%   \item Proof that $\bid \commutes x$:
%2014jun15sun moved to relneg.tex%     \begin{align*}
%2014jun15sun moved to relneg.tex%       \brp{\bid \meet x} \join \brp{\bid \meet x^\ocop}
%2014jun15sun moved to relneg.tex%         &= \brp{x} \join \brp{x^\ocop}
%2014jun15sun moved to relneg.tex%         && \text{by definition of $\meet$ \xref{def:meet}}
%2014jun15sun moved to relneg.tex%       \\&= \bid
%2014jun15sun moved to relneg.tex%         && \text{by definition of $\ocop$ \xref{def:ocop}}
%2014jun15sun moved to relneg.tex%       \\\implies&\bid \commutes x
%2014jun15sun moved to relneg.tex%         && \text{by definition of $\commutes$ relation \xref{def:latoc_commutes}}
%2014jun15sun moved to relneg.tex%     \end{align*}
%2014jun15sun moved to relneg.tex% 
%2014jun15sun moved to relneg.tex%   \item Proof that $x \commutes x$:
%2014jun15sun moved to relneg.tex%     \begin{align*}
%2014jun15sun moved to relneg.tex%       \brp{x\meet x} \join \brp{x \meet x^\ocop}
%2014jun15sun moved to relneg.tex%         &= \brp{x} \join \brp{x \meet x^\ocop}
%2014jun15sun moved to relneg.tex%         && \text{by definition of $\meet$ \xref{def:meet}}
%2014jun15sun moved to relneg.tex%       \\&= \brp{x} \join \brp{\bzero}
%2014jun15sun moved to relneg.tex%         && \text{by definition of $\ocop$ \xref{def:latoc}}
%2014jun15sun moved to relneg.tex%       \\&= x
%2014jun15sun moved to relneg.tex%         && \text{by definition of $\join$ \xref{def:join}}
%2014jun15sun moved to relneg.tex%       \\\implies &x \commutes x
%2014jun15sun moved to relneg.tex%         && \text{by definition of $\commutes$ relation \xref{def:latoc_commutes}}
%2014jun15sun moved to relneg.tex%     \end{align*}
%2014jun15sun moved to relneg.tex%     
%2014jun15sun moved to relneg.tex%   \item Proof that $x \commutes y \implies x \commutes y^\ocop$:
%2014jun15sun moved to relneg.tex%     \begin{align*}
%2014jun15sun moved to relneg.tex%       \brp{x\meet y^\ocop} \join \brp{x \meet y^{\ocop\ocop}}
%2014jun15sun moved to relneg.tex%         &= \brp{x\meet y^\ocop} \join \brp{x \meet y}
%2014jun15sun moved to relneg.tex%         && \text{by \prope{involution} property of $\ocop$ \xref{def:latoc}}
%2014jun15sun moved to relneg.tex%       \\&= \brp{x\meet y} \join \brp{x \meet y^\ocop}
%2014jun15sun moved to relneg.tex%         && \text{by \prope{commutative} property of \structe{lattice}s \xref{def:lattice}}
%2014jun15sun moved to relneg.tex%       \\&= x
%2014jun15sun moved to relneg.tex%         && \text{by $x \commutes y$ hypothesis and \prefp{def:latoc_commutes}}
%2014jun15sun moved to relneg.tex%       \\\implies& x\commutes y^\ocop
%2014jun15sun moved to relneg.tex%         && \text{by definition of $\commutes$ relation \xref{def:latoc_commutes}}
%2014jun15sun moved to relneg.tex%     \end{align*}
%2014jun15sun moved to relneg.tex% 
%2014jun15sun moved to relneg.tex%   \item Proof that $x \commutes y \impliedby x \commutes y^\ocop$:
%2014jun15sun moved to relneg.tex%     \begin{align*}
%2014jun15sun moved to relneg.tex%       \brp{x\meet y} \join \brp{x \meet y^{\ocop}}
%2014jun15sun moved to relneg.tex%         &= \brp{x\meet y^{\ocop\ocop}} \join \brp{x \meet y^\ocop}
%2014jun15sun moved to relneg.tex%         && \text{by \prope{involution} property of $\ocop$ \xref{def:latoc}}
%2014jun15sun moved to relneg.tex%       \\&= \brp{x\meet y^\ocop} \join \brp{x \meet y^{\ocop\ocop}}
%2014jun15sun moved to relneg.tex%         && \text{by \prope{commutative} property of \structe{lattice}s \xref{def:lattice}}
%2014jun15sun moved to relneg.tex%       \\&= x
%2014jun15sun moved to relneg.tex%         && \text{by $x \commutes y^\ocop$ hypothesis and \prefp{def:latoc_commutes}}
%2014jun15sun moved to relneg.tex%       \\\implies& x\commutes y
%2014jun15sun moved to relneg.tex%         && \text{by definition of $\commutes$ relation \xref{def:latoc_commutes}}
%2014jun15sun moved to relneg.tex%     \end{align*}
%2014jun15sun moved to relneg.tex% 
%2014jun15sun moved to relneg.tex%   \item Proof that $x \orel y     \implies x \commutes y$:
%2014jun15sun moved to relneg.tex%     \begin{align*}
%2014jun15sun moved to relneg.tex%       \brp{x \meet y} \join \brp{x \meet y^\ocop}
%2014jun15sun moved to relneg.tex%         &=        x \join \brp{x \meet y^\ocop}
%2014jun15sun moved to relneg.tex%         &&        \text{by $x\orel y$ hypothesis}
%2014jun15sun moved to relneg.tex%       \\&=        x
%2014jun15sun moved to relneg.tex%         &&        \text{by \prope{absorptive} property \ifxref{lattice}{thm:lattice}}
%2014jun15sun moved to relneg.tex%       \\&\implies x\commutes y
%2014jun15sun moved to relneg.tex%         &&        \text{by definition of $\commutes$ \xref{def:latoc_commutes}}
%2014jun15sun moved to relneg.tex%     \end{align*}
%2014jun15sun moved to relneg.tex% 
%2014jun15sun moved to relneg.tex%   \item Proof that $x \orthog y   \implies x \commutes y$:
%2014jun15sun moved to relneg.tex%     \begin{align*}
%2014jun15sun moved to relneg.tex%       \brp{x \meet y} \join \brp{x \meet y^\ocop}
%2014jun15sun moved to relneg.tex%         &= \bzero \join \brp{x \meet y^\ocop}
%2014jun15sun moved to relneg.tex%         && \text{by \prefp{lem:latoc_orthog}}
%2014jun15sun moved to relneg.tex%       \\&= \bzero \join x
%2014jun15sun moved to relneg.tex%         && \text{by $x\orthog y$ hypothesis ($x\orthog y\implies x\orel y^\ocop$)} % \xref{def:latoc_orthog}
%2014jun15sun moved to relneg.tex%       \\&= x \join \bzero
%2014jun15sun moved to relneg.tex%         && \text{by \prope{commutative} property \ifxref{lattice}{thm:lattice}}
%2014jun15sun moved to relneg.tex%       \\&= x
%2014jun15sun moved to relneg.tex%         && \text{by \prope{identity} property of \structe{bounded lattice}s\ifsxref{latvar}{prop:latb_prop}}
%2014jun15sun moved to relneg.tex%       \\&\implies x\commutes y
%2014jun15sun moved to relneg.tex%         &&        \text{by definition of $\commutes$ \xref{def:latoc_commutes}}
%2014jun15sun moved to relneg.tex%     \end{align*}
%2014jun15sun moved to relneg.tex% 
%2014jun15sun moved to relneg.tex%   \end{enumerate}
%2014jun15sun moved to relneg.tex% \end{proof}
%2014jun15sun moved to relneg.tex% 
%2014jun15sun moved to relneg.tex% %%=======================================
%2014jun15sun moved to relneg.tex% %\subsection{Symmetry}
%2014jun15sun moved to relneg.tex% %%=======================================
%2014jun15sun moved to relneg.tex% %---------------------------------------
%2014jun15sun moved to relneg.tex% \begin{definition}
%2014jun15sun moved to relneg.tex% %---------------------------------------
%2014jun15sun moved to relneg.tex% Let $\latL\eqd\latocd$ be an \structe{orthocomplemented lattice} \xref{def:latoc}.
%2014jun15sun moved to relneg.tex% \defboxt{
%2014jun15sun moved to relneg.tex%   A lattice $\latL$ is \propd{symmetric} if
%2014jun15sun moved to relneg.tex%   \\\indentx$x \commutes y\quad\implies\quad y \commutes x \qquad{\scy\forall x,y\in\latL}$
%2014jun15sun moved to relneg.tex%   }
%2014jun15sun moved to relneg.tex% \end{definition}
%2014jun15sun moved to relneg.tex% 
%2014jun15sun moved to relneg.tex% In general, the commutes relation is not \prope{symmetric}.
%2014jun15sun moved to relneg.tex% But \pref{prop:latoc_symmetry} (next) describes some conditions under which it 
%2014jun15sun moved to relneg.tex% is symmetric.
%2014jun15sun moved to relneg.tex% %---------------------------------------
%2014jun15sun moved to relneg.tex% \begin{proposition}
%2014jun15sun moved to relneg.tex% \footnote{
%2014jun15sun moved to relneg.tex%   \citorp{holland1963}{68}\\
%2014jun15sun moved to relneg.tex%   \citePp{nakamura1957}{158}
%2014jun15sun moved to relneg.tex%   }
%2014jun15sun moved to relneg.tex% \label{prop:latoc_symmetry}
%2014jun15sun moved to relneg.tex% %---------------------------------------
%2014jun15sun moved to relneg.tex% Let $\latocd$ be an \structe{orthocomplemented lattice} \xref{def:latoc}.
%2014jun15sun moved to relneg.tex% \propbox{
%2014jun15sun moved to relneg.tex%   \mcom{\brb{x\commutes y \implies y\commutes x}}{$\commutes$ is \prope{symmetric}}
%2014jun15sun moved to relneg.tex%   \begin{array}[t]{@{\quad}l@{\quad}lD}
%2014jun15sun moved to relneg.tex%     \iff & \brb{x \orel y \implies y = x \join \brp{x^\ocop \meet y}} & (\prope{orthomodular identity})\\
%2014jun15sun moved to relneg.tex%     \iff & \brb{x \orel y \implies x = y \meet \brp{x \join y^\ocop}} & ($x=\sasaki{y}{x}$ (\ope{Sasaki projection}) )\\
%2014jun15sun moved to relneg.tex%     \iff & \brb{y = \brp{x\meet y} \join \brs{y\meet\brp{x\meet y}^\ocop}}\\
%2014jun15sun moved to relneg.tex%     \iff & \brb{x = \brp{x\join y} \meet \brs{x\join\brp{x\join y}^\ocop}}
%2014jun15sun moved to relneg.tex%   \end{array}
%2014jun15sun moved to relneg.tex%   }
%2014jun15sun moved to relneg.tex% \end{proposition}
%2014jun15sun moved to relneg.tex% \begin{proof}
%2014jun15sun moved to relneg.tex% Let $+$ represent $\join$ and juxtaposition represent $\meet$, with 
%2014jun15sun moved to relneg.tex% $\meet$ taking algebraic precedence over $\join$.
%2014jun15sun moved to relneg.tex% \begin{enumerate}
%2014jun15sun moved to relneg.tex%   \item Proof that 
%2014jun15sun moved to relneg.tex%       $\brb{x \orel y \implies y = x \join \brp{x^\ocop \meet y}}$
%2014jun15sun moved to relneg.tex%       $\iff$
%2014jun15sun moved to relneg.tex%       $\brb{x \orel y \implies x = y \meet \brp{x \join y^\ocop}}$:
%2014jun15sun moved to relneg.tex%     \begin{align*}
%2014jun15sun moved to relneg.tex%       x\orel y
%2014jun15sun moved to relneg.tex%         &\implies y^\ocop \orel x^\ocop
%2014jun15sun moved to relneg.tex%         && \text{by \prope{antitone} property \xref{def:latoc}}
%2014jun15sun moved to relneg.tex%       \\&\implies x^\ocop = y^\ocop \join \brp{y^{\ocop\ocop} \meet x^\ocop}
%2014jun15sun moved to relneg.tex%         && \text{by left hypothesis}
%2014jun15sun moved to relneg.tex%       \\&\implies \brp{x^\ocop}^\ocop = \brs{y^\ocop \join \brp{y^{\ocop\ocop} \meet x^\ocop}}^\ocop
%2014jun15sun moved to relneg.tex%       \\&\implies x = \brs{y^\ocop \join \brp{y^{\ocop\ocop} \meet x^\ocop}}^\ocop
%2014jun15sun moved to relneg.tex%         && \text{by \prope{involutory} property \xref{def:latoc}}
%2014jun15sun moved to relneg.tex%       \\&\qquad= y^{\ocop\ocop} \meet \brp{y^{\ocop\ocop} \meet x^\ocop}^\ocop
%2014jun15sun moved to relneg.tex%         && \text{by \prope{de Morgan} property \xref{thm:latoc_prop}}
%2014jun15sun moved to relneg.tex%       \\&\qquad= y \meet \brp{y \meet x^\ocop}^\ocop
%2014jun15sun moved to relneg.tex%         && \text{by \prope{involutory} property \xref{def:latoc}}
%2014jun15sun moved to relneg.tex%       \\&\qquad= y \meet \brp{y^\ocop \join x^{\ocop\ocop}}
%2014jun15sun moved to relneg.tex%         && \text{by \prope{de Morgan} property \xref{thm:latoc_prop}}
%2014jun15sun moved to relneg.tex%       \\&\qquad= y \meet \brp{y^\ocop \join x}
%2014jun15sun moved to relneg.tex%         && \text{by \prope{involutory} property \xref{def:latoc}}
%2014jun15sun moved to relneg.tex%       \\&\qquad= y \meet \brp{x \join y^\ocop}
%2014jun15sun moved to relneg.tex%         && \text{by \prope{commutative} property \ifxref{lattice}{thm:lattice}}
%2014jun15sun moved to relneg.tex%       \\
%2014jun15sun moved to relneg.tex%       \\
%2014jun15sun moved to relneg.tex%       x\orel y
%2014jun15sun moved to relneg.tex%         &\implies y^\ocop \orel x^\ocop
%2014jun15sun moved to relneg.tex%         && \text{by \prope{antitone} property \xref{def:latoc}}
%2014jun15sun moved to relneg.tex%       \\&\implies y^\ocop = x^\ocop \meet \brp{y^\ocop \join x^{\ocop\ocop}}
%2014jun15sun moved to relneg.tex%         && \text{by right hypothesis}
%2014jun15sun moved to relneg.tex%       \\&\implies \brp{y^\ocop}^\ocop = \brs{x^\ocop \meet \brp{y^\ocop \join x^{\ocop\ocop}}}^\ocop
%2014jun15sun moved to relneg.tex%       \\&\implies y = \brs{x^\ocop \meet \brp{y^\ocop \join x^{\ocop\ocop}}}^\ocop
%2014jun15sun moved to relneg.tex%         && \text{by \prope{involutory} property \xref{def:latoc}}
%2014jun15sun moved to relneg.tex%       \\&\qquad= x^{\ocop\ocop} \join \brp{y^\ocop\join x^{\ocop\ocop}}^\ocop
%2014jun15sun moved to relneg.tex%         && \text{by \prope{de Morgan} property \xref{thm:latoc_prop}}
%2014jun15sun moved to relneg.tex%       \\&\qquad= x \join \brp{y^\ocop\join x}^\ocop
%2014jun15sun moved to relneg.tex%         && \text{by \prope{involutory} property \xref{def:latoc}}
%2014jun15sun moved to relneg.tex%       \\&\qquad= x \join \brp{y^{\ocop\ocop}\meet x^\ocop}
%2014jun15sun moved to relneg.tex%         && \text{by \prope{de Morgan} property \xref{thm:latoc_prop}}
%2014jun15sun moved to relneg.tex%       \\&\qquad= x \join \brp{y \meet x^\ocop}
%2014jun15sun moved to relneg.tex%         && \text{by \prope{involutory} property \xref{def:latoc}}
%2014jun15sun moved to relneg.tex%       \\&\qquad= x \join \brp{x^\ocop\meet y}
%2014jun15sun moved to relneg.tex%         && \text{by \prope{commutative} property \ifxref{lattice}{thm:lattice}}
%2014jun15sun moved to relneg.tex%     \end{align*}
%2014jun15sun moved to relneg.tex% 
%2014jun15sun moved to relneg.tex%   \item Proof that 
%2014jun15sun moved to relneg.tex%         $ \brb{x \orel y \implies y = x \join \brp{x^\ocop \meet y}}
%2014jun15sun moved to relneg.tex%           \iff
%2014jun15sun moved to relneg.tex%           \brb{y = \brp{x\meet y} \join \brs{y\meet\brp{x\meet y}^\ocop}}
%2014jun15sun moved to relneg.tex%         $:\label{item:latoc_symmetry_iffeq}
%2014jun15sun moved to relneg.tex%     \begin{align*}
%2014jun15sun moved to relneg.tex%       \brp{xy} + \brs{y\brp{xy}^\ocop}
%2014jun15sun moved to relneg.tex%         &= u + \brs{yu^\ocop}
%2014jun15sun moved to relneg.tex%         && \text{where $u\eqd xy \orel y$}
%2014jun15sun moved to relneg.tex%       \\&= u + \brs{u^\ocop y}
%2014jun15sun moved to relneg.tex%         && \text{by \prope{commutative} property of lattices \ifxref{lattice}{thm:lattice}}
%2014jun15sun moved to relneg.tex%       \\&= y
%2014jun15sun moved to relneg.tex%         && \text{by left hypothesis}
%2014jun15sun moved to relneg.tex%       \\
%2014jun15sun moved to relneg.tex%       \\
%2014jun15sun moved to relneg.tex%       x \orel y \implies x + \brp{x^\ocop y} 
%2014jun15sun moved to relneg.tex%         &= xy + \brs{(xy)^\ocop y}
%2014jun15sun moved to relneg.tex%         && \text{by $x\orel y$ hypothesis}
%2014jun15sun moved to relneg.tex%       \\&= xy + \brs{y(xy)^\ocop}
%2014jun15sun moved to relneg.tex%         && \text{by \prope{commutative} property of lattices \ifxref{lattice}{thm:lattice}}
%2014jun15sun moved to relneg.tex%       \\&= y
%2014jun15sun moved to relneg.tex%         && \text{by right hypothesis}
%2014jun15sun moved to relneg.tex%     \end{align*}
%2014jun15sun moved to relneg.tex% 
%2014jun15sun moved to relneg.tex%   \item Proof that 
%2014jun15sun moved to relneg.tex%         $ \brb{x \orel y \implies x = y \meet \brp{x \join y^\ocop}}
%2014jun15sun moved to relneg.tex%           \iff
%2014jun15sun moved to relneg.tex%           \brb{x = \brp{x\join y} \meet \brs{x\join\brp{x\join y}^\ocop}}
%2014jun15sun moved to relneg.tex%         $:\label{item:latoc_symmetry_iffeqd}
%2014jun15sun moved to relneg.tex%     \begin{align*}
%2014jun15sun moved to relneg.tex%       \brp{x+y}\brs{x+\brp{x+y}^\ocop}
%2014jun15sun moved to relneg.tex%         &= u\brs{x+u^\ocop}
%2014jun15sun moved to relneg.tex%         && \text{where $x\orel u\eqd x+y$}
%2014jun15sun moved to relneg.tex%       \\&= x
%2014jun15sun moved to relneg.tex%         && \text{by left hypothesis}
%2014jun15sun moved to relneg.tex%       \\
%2014jun15sun moved to relneg.tex%       \\
%2014jun15sun moved to relneg.tex%       x \orel y \implies y\brp{x + y^\ocop} 
%2014jun15sun moved to relneg.tex%         &= \brp{x+y} \brs{x + \brp{x+y}^\ocop}
%2014jun15sun moved to relneg.tex%         && \text{by $x\orel y$ hypothesis}
%2014jun15sun moved to relneg.tex%       \\&= x
%2014jun15sun moved to relneg.tex%         && \text{by right hypothesis}
%2014jun15sun moved to relneg.tex%     \end{align*}
%2014jun15sun moved to relneg.tex% 
%2014jun15sun moved to relneg.tex%   \item Proof that \prope{symmetric} property $\implies \brb{x \orel y \implies y = x \join \brp{x^\ocop \meet y}}$:\label{item:latoc_symmetry_iffineq}
%2014jun15sun moved to relneg.tex%     \begin{align*}
%2014jun15sun moved to relneg.tex%       x \orel y
%2014jun15sun moved to relneg.tex%         &\implies x\commutes y
%2014jun15sun moved to relneg.tex%         &&        \text{by \prefp{prop:commutes}}
%2014jun15sun moved to relneg.tex%       \\&\implies y\commutes x
%2014jun15sun moved to relneg.tex%         &&        \text{by \prope{symmetry} hypothesis (left hypothesis)}
%2014jun15sun moved to relneg.tex%       \\&\implies y = \brp{y \meet x} \join \brp{y \meet x^\ocop}
%2014jun15sun moved to relneg.tex%         &&        \text{by definition of $\commutes$ \xref{def:latoc_commutes}}
%2014jun15sun moved to relneg.tex%       \\&\implies y = x \join \brp{y \meet x^\ocop}
%2014jun15sun moved to relneg.tex%         &&        \text{by $x\orel y$ hypothesis}
%2014jun15sun moved to relneg.tex%       \\&\implies y = x \join \brp{x^\ocop \meet y}
%2014jun15sun moved to relneg.tex%         &&        \text{by \prope{commutative} property of lattices \ifxref{lattice}{thm:lattice}}
%2014jun15sun moved to relneg.tex%     \end{align*}
%2014jun15sun moved to relneg.tex% 
%2014jun15sun moved to relneg.tex%   \item Proof that $\brb{y = \brp{x\meet y} \join \brs{y\meet\brp{x\meet y}^\ocop}}$
%2014jun15sun moved to relneg.tex%         $\implies$ \prope{symmetric} property:
%2014jun15sun moved to relneg.tex%     \begin{enumerate}
%2014jun15sun moved to relneg.tex%       \item Proof that $x \commutes y \implies x^\ocop y = \brp{xy}^\ocop y$: \label{item:latoc_symmetry_xyxyy}
%2014jun15sun moved to relneg.tex%         \begin{align*}
%2014jun15sun moved to relneg.tex%           x \commutes y \implies
%2014jun15sun moved to relneg.tex%           x^\ocop y
%2014jun15sun moved to relneg.tex%             &= \brp{xy + xy^\ocop}^\ocop y
%2014jun15sun moved to relneg.tex%             && \text{by definition of $\commutes$ \xref{def:latoc_commutes}}
%2014jun15sun moved to relneg.tex%           \\&= \brp{xy}^\ocop \brp{xy^\ocop}^\ocop y
%2014jun15sun moved to relneg.tex%             && \text{by \prope{de Morgan}'s law \xref{thm:latn_demorgan}}
%2014jun15sun moved to relneg.tex%           \\&= \brp{xy}^\ocop \brs{\brp{x^\ocop+y^{\ocop^\ocop}} y}
%2014jun15sun moved to relneg.tex%             && \text{by \prope{de Morgan}'s law \xref{thm:latn_demorgan}}
%2014jun15sun moved to relneg.tex%           \\&= \brp{xy}^\ocop \brs{\brp{x^\ocop+y} y}
%2014jun15sun moved to relneg.tex%             &&        \text{by \prope{involutory}'s property \xref{def:latoc}}
%2014jun15sun moved to relneg.tex%           \\&= \brp{xy}^\ocop y
%2014jun15sun moved to relneg.tex%             &&        \text{by \prope{absorptive} property of lattices \ifxref{lattice}{thm:lattice}}
%2014jun15sun moved to relneg.tex%         \end{align*}
%2014jun15sun moved to relneg.tex% 
%2014jun15sun moved to relneg.tex%       \item Proof that $x\commutes y\implies y\commutes x$:
%2014jun15sun moved to relneg.tex%         \begin{align*}
%2014jun15sun moved to relneg.tex%           x\commutes y\implies
%2014jun15sun moved to relneg.tex%           xy + y\brp{xy}^\ocop
%2014jun15sun moved to relneg.tex%             &= xy + \brp{xy}^\ocop y
%2014jun15sun moved to relneg.tex%             && \text{by \prope{commutative} property \ifxref{lattice}{thm:lattice}}
%2014jun15sun moved to relneg.tex%           \\&= xy + x^\ocop y
%2014jun15sun moved to relneg.tex%             && \text{by $x\commutes y$ hypothesis and \pref{item:latoc_symmetry_xyxyy}}
%2014jun15sun moved to relneg.tex%           \\&= (yx) + \brs{y x^\ocop }
%2014jun15sun moved to relneg.tex%             && \text{by \prope{commutative} property \ifxref{lattice}{thm:lattice}}
%2014jun15sun moved to relneg.tex%           \\&\implies y \commutes x
%2014jun15sun moved to relneg.tex%             && \text{by definition of $\commutes$ \xref{def:latoc_commutes}}
%2014jun15sun moved to relneg.tex%       \end{align*}
%2014jun15sun moved to relneg.tex%   \end{enumerate}
%2014jun15sun moved to relneg.tex% \end{enumerate}
%2014jun15sun moved to relneg.tex% \end{proof}
%2014jun15sun moved to relneg.tex% 
%2014jun15sun moved to relneg.tex% 
%2014jun15sun moved to relneg.tex% %---------------------------------------
%2014jun15sun moved to relneg.tex% \begin{theorem}
%2014jun15sun moved to relneg.tex% \footnote{
%2014jun15sun moved to relneg.tex%   \citerpg{kalmbach1983}{20}{0123945801}\\
%2014jun15sun moved to relneg.tex%   \citeP{maclaren1964}
%2014jun15sun moved to relneg.tex%   }
%2014jun15sun moved to relneg.tex% %---------------------------------------
%2014jun15sun moved to relneg.tex% Let $\latL\eqd\latocd$ be an \structe{orthocomplemented lattice} \xref{def:latoc}.
%2014jun15sun moved to relneg.tex% \thmbox{\begin{array}{M}
%2014jun15sun moved to relneg.tex%   $\brb{x\commutes c \quad \forall x\in\setX}$
%2014jun15sun moved to relneg.tex%   $\qquad\iff\qquad$
%2014jun15sun moved to relneg.tex%   $\brb{\text{$\latL$ is \emph{isomorphic} to $\intcc{\bzero}{c}\times\intcc{\bzero}{c^\ocop}$}}$
%2014jun15sun moved to relneg.tex%   \\\indentx with isomorphism 
%2014jun15sun moved to relneg.tex%     $\ftheta(x) \eqd \opair{\intcc{\bzero}{c}}{\intcc{\bzero}{c^\ocop}}$.
%2014jun15sun moved to relneg.tex% \end{array}}
%2014jun15sun moved to relneg.tex% \end{theorem}
%2014jun15sun moved to relneg.tex% 
%2014jun15sun moved to relneg.tex% 
%2014jun15sun moved to relneg.tex% %---------------------------------------
%2014jun15sun moved to relneg.tex% \begin{proposition}
%2014jun15sun moved to relneg.tex% \footnote{
%2014jun15sun moved to relneg.tex%   \citePp{foulis1962}{66}\\
%2014jun15sun moved to relneg.tex%   \citePc{sasaki1954}{cf Foulis 1962}
%2014jun15sun moved to relneg.tex%   }
%2014jun15sun moved to relneg.tex% \label{prop:latoc_pxypyx}
%2014jun15sun moved to relneg.tex% %---------------------------------------
%2014jun15sun moved to relneg.tex% Let $\latM$ be an \prope{orthomodular} lattice.
%2014jun15sun moved to relneg.tex% \propbox{
%2014jun15sun moved to relneg.tex%   x\commutes y 
%2014jun15sun moved to relneg.tex%   \qquad\iff\qquad
%2014jun15sun moved to relneg.tex%   \sasaki{x}{y} = \sasaki{y}{x} = x \meet y
%2014jun15sun moved to relneg.tex%   }
%2014jun15sun moved to relneg.tex% \end{proposition}
%2014jun15sun moved to relneg.tex% %\begin{proof}
%2014jun15sun moved to relneg.tex% %\begin{enumerate}
%2014jun15sun moved to relneg.tex% %  \item Proof that $x\commutes y \implies   \sasaki{x}{y} = \sasaki{y}{x} = x \meet y$:
%2014jun15sun moved to relneg.tex% %  \item Proof that $x\commutes y \impliedby \sasaki{x}{y} = \sasaki{y}{x} = x \meet y$:
%2014jun15sun moved to relneg.tex% %\end{enumerate}
%2014jun15sun moved to relneg.tex% %\end{proof}
%2014jun15sun moved to relneg.tex% 
%2014jun15sun moved to relneg.tex% %=======================================
%2014jun15sun moved to relneg.tex% \subsection{Center}
%2014jun15sun moved to relneg.tex% %=======================================
%2014jun15sun moved to relneg.tex% An element in an \structe{orthocomplemented lattice} \xref{def:latoc} is in the \structe{center} of the lattice
%2014jun15sun moved to relneg.tex% if it \rele{commutes} \xref{def:latoc_commutes} with every other element in the lattice (next definition).
%2014jun15sun moved to relneg.tex% \emph{All} the elements of an \structe{orthocomplemented lattice} are in the \structe{center} 
%2014jun15sun moved to relneg.tex% if and only if that lattice is \prope{Boolean} \xref{prop:latoc_center_boolean}.
%2014jun15sun moved to relneg.tex% %---------------------------------------
%2014jun15sun moved to relneg.tex% \begin{definition}
%2014jun15sun moved to relneg.tex% \footnote{
%2014jun15sun moved to relneg.tex%   \citerp{holland1970}{80}
%2014jun15sun moved to relneg.tex%   }
%2014jun15sun moved to relneg.tex% \label{def:center}
%2014jun15sun moved to relneg.tex% %---------------------------------------
%2014jun15sun moved to relneg.tex% Let $\latL\eqd\latocd$ be an \structe{orthocomplemented lattice} \xref{def:latoc}.
%2014jun15sun moved to relneg.tex% \defboxt{
%2014jun15sun moved to relneg.tex%   The \structd{center} of $\latL$ is defined as
%2014jun15sun moved to relneg.tex%   \\\indentx$\ds\set{x\in\setX}{x\commutes y\quad\forall y\in\setX}$
%2014jun15sun moved to relneg.tex%   %An element $x\in\setX$ is in the \structd{center} of $\latL$ if
%2014jun15sun moved to relneg.tex%   %\\\indentx$x\commutes y$ for every $y\in\setX$
%2014jun15sun moved to relneg.tex%   }
%2014jun15sun moved to relneg.tex% \end{definition}
%2014jun15sun moved to relneg.tex% 
%2014jun15sun moved to relneg.tex% %---------------------------------------
%2014jun15sun moved to relneg.tex% \begin{proposition}
%2014jun15sun moved to relneg.tex% \label{prop:center}
%2014jun15sun moved to relneg.tex% %---------------------------------------
%2014jun15sun moved to relneg.tex% Let $\latL\eqd\latocd$ be a n \structe{orthocomplemented lattice} \xref{def:latoc}.
%2014jun15sun moved to relneg.tex% \propboxt{
%2014jun15sun moved to relneg.tex%   $\bzero$ and $\bid$ are in the \structb{center} of $\latL$.
%2014jun15sun moved to relneg.tex%   }
%2014jun15sun moved to relneg.tex% \end{proposition}
%2014jun15sun moved to relneg.tex% 
%2014jun15sun moved to relneg.tex% %---------------------------------------
%2014jun15sun moved to relneg.tex% \begin{theorem}
%2014jun15sun moved to relneg.tex% \footnote{
%2014jun15sun moved to relneg.tex%   \citePpc{jeffcott1972}{645}{\textsection5. Main theorem}
%2014jun15sun moved to relneg.tex%   }
%2014jun15sun moved to relneg.tex% \label{thm:center_boolean}
%2014jun15sun moved to relneg.tex% %---------------------------------------
%2014jun15sun moved to relneg.tex% Let $\latL\eqd\latocd$ be a n \structe{orthocomplemented lattice} \xref{def:latoc}.
%2014jun15sun moved to relneg.tex% \thmboxt{
%2014jun15sun moved to relneg.tex%   The \structe{center} of $\latL$ is \prope{boolean}.
%2014jun15sun moved to relneg.tex%   }
%2014jun15sun moved to relneg.tex% \end{theorem}
%2014jun15sun moved to relneg.tex% 
%2014jun15sun moved to relneg.tex% %---------------------------------------
%2014jun15sun moved to relneg.tex% \begin{example}
%2014jun15sun moved to relneg.tex% %---------------------------------------
%2014jun15sun moved to relneg.tex% \exbox{\begin{tabular}{m{\tw-53mm}m{34mm}}
%2014jun15sun moved to relneg.tex%   The \structb{center} of the \structb{$O_6$ lattice} \xref{def:o6} is the set $\setn{0,\,x,\,z,\,1}$.
%2014jun15sun moved to relneg.tex%   The elements $x^\ocop$ and $z^\ocop$ are \textbf{not} in the center of $\latL$.
%2014jun15sun moved to relneg.tex%   The $O_6$ lattice is illustrated to the right, with the center elements as solid dots.
%2014jun15sun moved to relneg.tex%   Note that the center is the \prope{Boolean} lattice $\latL_2^2$ \xref{prop:latoc_center_boolean}.
%2014jun15sun moved to relneg.tex%   & \centering\psset{unit=7.5mm}%============================================================================
% Daniel J. Greenhoe
% LaTeX file
% lattice O6
%============================================================================
{%\psset{unit=0.5\psunit}%
\begin{pspicture}(-1.1,-0.1)(1.1,3.2)
  %---------------------------------
  % nodes
  %---------------------------------
  \Cnode*(0,3){t}
  \Cnode(-1,2){c}\Cnode(1,2){d}%
  \Cnode*(-1,1){x}\Cnode*(1,1){y}%
  \Cnode*(0,0){b}
  %---------------------------------
  % node connections
  %---------------------------------
  \ncline{t}{c}\ncline{t}{d}%
  \ncline{c}{x}\ncline{d}{y}%
  \ncline{b}{x}\ncline{b}{y}%
  %---------------------------------
  % node labels
  %---------------------------------
  \uput[0](t) {$1$}%     
  \uput[0](d) {$x^\ocop$}%     
  \uput[180](c) {$z^\ocop$}%     
  \uput[0](y) {$z$}%     
  \uput[180](x) {$x$}%     
  \uput[0](b) {$0$}%     
\end{pspicture}
}%%
%2014jun15sun moved to relneg.tex% \end{tabular}}
%2014jun15sun moved to relneg.tex% \end{example}
%2014jun15sun moved to relneg.tex% \begin{proof}
%2014jun15sun moved to relneg.tex% \begin{enumerate}
%2014jun15sun moved to relneg.tex%   \item Proof that $\bzero$ and $\bid$ are in the \structe{center} of $\latL$: by \prefpp{prop:center}.
%2014jun15sun moved to relneg.tex% 
%2014jun15sun moved to relneg.tex%   \item Proof that $x$ is in the \structe{center} of $\latL$: 
%2014jun15sun moved to relneg.tex%     \begin{align*}
%2014jun15sun moved to relneg.tex%       (x \meet x)       \join (x \meet x^\ocop)      &= x      \join \bzero &&= x  &&\implies x \commutes x       \\
%2014jun15sun moved to relneg.tex%       (x \meet z)       \join (x \meet z^\ocop)      &= \bzero \join x      &&= x  &&\implies x \commutes z       \\
%2014jun15sun moved to relneg.tex%     \end{align*}
%2014jun15sun moved to relneg.tex%     $x\commutes x$, $x\commutes x^\ocop$, $x \commutes z^\ocop$, $x\commutes\bzero$, and $x\commutes\bid$ by \prefpp{prop:commutes}.
%2014jun15sun moved to relneg.tex% 
%2014jun15sun moved to relneg.tex%   \item Proof that $z$ is in the \structe{center} of $\latL$: 
%2014jun15sun moved to relneg.tex%     \begin{align*}
%2014jun15sun moved to relneg.tex%       (z \meet z)       \join (z \meet z^\ocop)      &= z\join \bzero       &&= z &&\implies z \commutes z       \\
%2014jun15sun moved to relneg.tex%       (z \meet x)       \join (z \meet x^\ocop)      &= \bzero \join z      &&= z &&\implies z \commutes x       \\
%2014jun15sun moved to relneg.tex%     \end{align*}
%2014jun15sun moved to relneg.tex%     $z\commutes z$, $z\commutes x^\ocop$, $z \commutes z^\ocop$, $z\commutes\bzero$, and $z\commutes\bid$ by \prefpp{prop:commutes}.
%2014jun15sun moved to relneg.tex% 
%2014jun15sun moved to relneg.tex%   \item Proof that $x^\ocop$ and $z^\ocop$ are \emph{not} in the \structe{center} of $\latL$:
%2014jun15sun moved to relneg.tex%     \begin{align*}
%2014jun15sun moved to relneg.tex%       (x^\ocop \meet y) \join (x^\ocop \meet y^\ocop)  &= y \join \bzero &&= y &&\implies x^\ocop \notcommutes y   \\
%2014jun15sun moved to relneg.tex%       (z^\ocop \meet x) \join (z^\ocop \meet x^\ocop)  &= x \join \bzero &&= x &&\implies z^\ocop \notcommutes x   
%2014jun15sun moved to relneg.tex%     \end{align*}
%2014jun15sun moved to relneg.tex% 
%2014jun15sun moved to relneg.tex% \end{enumerate}
%2014jun15sun moved to relneg.tex% \end{proof}
%2014jun15sun moved to relneg.tex% 
%2014jun15sun moved to relneg.tex% 
%2014jun15sun moved to relneg.tex% %---------------------------------------
%2014jun15sun moved to relneg.tex% \begin{example}
%2014jun15sun moved to relneg.tex% %---------------------------------------
%2014jun15sun moved to relneg.tex% \exbox{\begin{tabular}{m{\tw-50mm}m{30mm}}
%2014jun15sun moved to relneg.tex%   The \structb{center} the lattice illustrated to the right \xref{ex:latoc}, 
%2014jun15sun moved to relneg.tex%   with center elements as solid dots, is the set $\setn{\bzero,\bid,p,y,z,x^\ocop,y^\ocop,z^\ocop,}$.
%2014jun15sun moved to relneg.tex%   The elements $x$ and $p^\ocop$ are \emph{not} in the \structe{center} of $\latL$.
%2014jun15sun moved to relneg.tex%   Note that the center is the \prope{Boolean} lattice $\latL_2^3$ \xref{prop:latoc_center_boolean}.
%2014jun15sun moved to relneg.tex%   & \centering\psset{unit=7.5mm}%============================================================================
% Daniel J. Greenhoe
% LaTeX file
%============================================================================
%{%\psset{unit=0.075mm}
\begin{pspicture}(-1.6,-0.2)(1.6,3.2)
  %---------------------------------
  % settings
  %---------------------------------
  %\psset{%
  %  labelsep=1.5mm,
  %  }%
  %---------------------------------
  % nodes
  %---------------------------------
  \Cnode*(0,3){t}%
  \Cnode*(-1,2){zc}\Cnode*(0,2){yc}\Cnode*(1,2){xc}%
  \Cnode(1,1.5){pc}%
  \Cnode(-1,1.5){p}%
  \Cnode*(-1,1){x}\Cnode*(0,1){y}\Cnode*(1,1){z}%
  \Cnode*(0,0){b}%
  %---------------------------------
  % node connections
  %---------------------------------
  \ncline{t}{zc}\ncline{t}{yc}\ncline{t}{xc}%
  \ncline{pc}{xc}%
  \ncline{z}{yc}\ncline{z}{pc}%
  \ncline{y}{zc}\ncline{y}{pc}%
  \ncline{p}{zc}\ncline{p}{yc}%
  \ncline{x}{p}%
  \ncline{b}{x}\ncline{b}{y}\ncline{b}{z}%
  %---------------------------------
  % node labels
  %---------------------------------
  \uput[0](t){$\bid$}%
  \uput[45](xc){$x^\ocop$}%
  \uput[135](zc){$z^\ocop$}%
  \uput[0](yc){$y^\ocop$}%
  \uput[0](pc){$p^\ocop$}%
  \uput[-45](z){$z$}%
  \uput[-45](y){$y$}%
  \uput[180](p){$p$}%
  \uput[225](x){$x$}%
  \uput[0](b){$\bzero$}
\end{pspicture}%
%    }
%
%2014jun15sun moved to relneg.tex% \end{tabular}}
%2014jun15sun moved to relneg.tex% \end{example}
%2014jun15sun moved to relneg.tex% \begin{proof}
%2014jun15sun moved to relneg.tex% \begin{enumerate}
%2014jun15sun moved to relneg.tex%   \item Proof that $\bzero$ and $\bid$ are in the \structe{center} of $\latL$: by \prefpp{prop:center}.
%2014jun15sun moved to relneg.tex% 
%2014jun15sun moved to relneg.tex%   \item Proof that $x$ is in the \structe{center} of $\latL$:
%2014jun15sun moved to relneg.tex%     \begin{align*}
%2014jun15sun moved to relneg.tex%       (x \meet p)       \join (x \meet p^\ocop)      &= x      \join \bzero &&= x  &&\implies x \commutes p       \\
%2014jun15sun moved to relneg.tex%       (x \meet y)       \join (x \meet y^\ocop)      &= \bzero \join x      &&= x  &&\implies x \commutes y       \\
%2014jun15sun moved to relneg.tex%       (x \meet z)       \join (x \meet z^\ocop)      &= \bzero \join x      &&= x  &&\implies x \commutes z       
%2014jun15sun moved to relneg.tex%     \end{align*}
%2014jun15sun moved to relneg.tex%     $x\commutes x$, $x\commutes x^\ocop$, $x\commutes p^\ocop$, $x\commutes y^\ocop$, $x \commutes z^\ocop$, $x\commutes\bzero$, and $x\commutes\bid$ by \prefpp{prop:commutes}.
%2014jun15sun moved to relneg.tex% 
%2014jun15sun moved to relneg.tex%   \item Proof that $y$ is in the \structe{center} of $\latL$:
%2014jun15sun moved to relneg.tex%     \begin{align*}
%2014jun15sun moved to relneg.tex%       (y \meet x)       \join (y \meet x^\ocop)      &= \bzero \join y      &&= y  &&\implies y \commutes x       \\
%2014jun15sun moved to relneg.tex%       (y \meet p)       \join (y \meet p^\ocop)      &= \bzero \join y      &&= y  &&\implies y \commutes p       \\
%2014jun15sun moved to relneg.tex%       (y \meet z)       \join (y \meet z^\ocop)      &= \bzero \join y      &&= y  &&\implies y \commutes z       
%2014jun15sun moved to relneg.tex%     \end{align*}
%2014jun15sun moved to relneg.tex%     $y\commutes y$, $y\commutes x^\ocop$, $y\commutes p^\ocop$, $y\commutes y^\ocop$, $y \commutes z^\ocop$, $y\commutes\bzero$, and $y\commutes\bid$ by \prefpp{prop:commutes}.
%2014jun15sun moved to relneg.tex% 
%2014jun15sun moved to relneg.tex%   \item Proof that $z$ is in the \structe{center} of $\latL$:
%2014jun15sun moved to relneg.tex%     \begin{align*}
%2014jun15sun moved to relneg.tex%       (z \meet x)       \join (z \meet x^\ocop)      &= \bzero \join z      &&= z  &&\implies z \commutes x       \\
%2014jun15sun moved to relneg.tex%       (z \meet p)       \join (z \meet p^\ocop)      &= \bzero \join z      &&= z  &&\implies z \commutes p       \\
%2014jun15sun moved to relneg.tex%       (z \meet y)       \join (z \meet y^\ocop)      &= \bzero \join z      &&= z  &&\implies z \commutes y       
%2014jun15sun moved to relneg.tex%     \end{align*}
%2014jun15sun moved to relneg.tex%     $z\commutes z$, $z\commutes x^\ocop$, $z\commutes p^\ocop$, $z\commutes y^\ocop$, $z \commutes z^\ocop$, $z\commutes\bzero$, and $z\commutes\bid$ by \prefpp{prop:commutes}.
%2014jun15sun moved to relneg.tex% 
%2014jun15sun moved to relneg.tex%   \item Proof that $x^\ocop$ is in the \structe{center} of $\latL$:
%2014jun15sun moved to relneg.tex%     \begin{align*}
%2014jun15sun moved to relneg.tex%       (p^\ocop \meet x)   \join (p^\ocop \meet x^\ocop)      &= \bzero \join p^\ocop      &&= p^\ocop  &&\implies p^\ocop \commutes x       \\
%2014jun15sun moved to relneg.tex%       (p^\ocop \meet y)   \join (p^\ocop \meet y^\ocop)      &= y \join z                 &&= p^\ocop  &&\implies p^\ocop \commutes y       \\
%2014jun15sun moved to relneg.tex%       (p^\ocop \meet z)   \join (p^\ocop \meet z^\ocop)      &= z \join y                 &&= p^\ocop  &&\implies p^\ocop \commutes z       \\
%2014jun15sun moved to relneg.tex%     \end{align*}
%2014jun15sun moved to relneg.tex%     $p^\ocop\commutes x^\ocop$, $p^\ocop\commutes p^\ocop$, $p^\ocop\commutes y^\ocop$, $p^\ocop\commutes z^\ocop$, $p^\ocop \commutes \bzero$, and $p^\ocop\commutes\bid$ by \prefpp{prop:commutes}.
%2014jun15sun moved to relneg.tex% 
%2014jun15sun moved to relneg.tex%   \item Proof that $y^\ocop$ is in the \structe{center} of $\latL$:
%2014jun15sun moved to relneg.tex%     \begin{align*}
%2014jun15sun moved to relneg.tex%       (y^\ocop \meet x)   \join (y^\ocop \meet x^\ocop)      &= x \join z  &&= y^\ocop  &&\implies y^\ocop \commutes x       \\
%2014jun15sun moved to relneg.tex%       (y^\ocop \meet p)   \join (y^\ocop \meet p^\ocop)      &= p \join z  &&= y^\ocop  &&\implies y^\ocop \notcommutes p       \\
%2014jun15sun moved to relneg.tex%       (y^\ocop \meet z)   \join (y^\ocop \meet z^\ocop)      &= z \join p  &&= y^\ocop  &&\implies y^\ocop \commutes z       \\
%2014jun15sun moved to relneg.tex%     \end{align*}
%2014jun15sun moved to relneg.tex%     $p^\ocop\commutes x^\ocop$, $p^\ocop\commutes p^\ocop$, $p^\ocop\commutes y^\ocop$, $p^\ocop\commutes z^\ocop$, $p^\ocop \commutes \bzero$, and $p^\ocop\commutes\bid$ by \prefpp{prop:commutes}.
%2014jun15sun moved to relneg.tex% 
%2014jun15sun moved to relneg.tex%   \item Proof that $z^\ocop$ is in the \structe{center} of $\latL$:
%2014jun15sun moved to relneg.tex%     \begin{align*}
%2014jun15sun moved to relneg.tex%       (z^\ocop \meet x)   \join (z^\ocop \meet x^\ocop)      &= x \join y  &&= z^\ocop  &&\implies z^\ocop \commutes x       \\
%2014jun15sun moved to relneg.tex%       (z^\ocop \meet p)   \join (z^\ocop \meet p^\ocop)      &= p \join y  &&= z^\ocop  &&\implies y^\ocop \notcommutes p       \\
%2014jun15sun moved to relneg.tex%       (z^\ocop \meet y)   \join (z^\ocop \meet y^\ocop)      &= z \join p  &&= z^\ocop  &&\implies y^\ocop \commutes z       \\
%2014jun15sun moved to relneg.tex%     \end{align*}
%2014jun15sun moved to relneg.tex%     $z^\ocop\commutes x^\ocop$, $z^\ocop\commutes p^\ocop$, $z^\ocop\commutes y^\ocop$, $z^\ocop\commutes z^\ocop$, $z^\ocop \commutes \bzero$, and $z^\ocop\commutes\bid$ by \prefpp{prop:commutes}.
%2014jun15sun moved to relneg.tex% 
%2014jun15sun moved to relneg.tex%   \item Proof that $p$ and $x^\ocop$ are \emph{not} in the \structe{center} of $\latL$:
%2014jun15sun moved to relneg.tex%     \begin{align*}
%2014jun15sun moved to relneg.tex%       (p \meet x)       \join (p \meet x^\ocop)        &= x      \join \bzero   &&= x        &&\implies p \notcommutes x         \\
%2014jun15sun moved to relneg.tex%       (x^\ocop \meet p) \join (x^\ocop \meet p^\ocop)  &= \bzero \join p^\ocop  &&= p^\ocop  &&\implies x^\ocop \notcommutes p       \\
%2014jun15sun moved to relneg.tex%     \end{align*}
%2014jun15sun moved to relneg.tex% \end{enumerate}
%2014jun15sun moved to relneg.tex% \end{proof}
%2014jun15sun moved to relneg.tex% 
%2014jun15sun moved to relneg.tex% 
%2014jun15sun moved to relneg.tex% %---------------------------------------
%2014jun15sun moved to relneg.tex% \begin{example}
%2014jun15sun moved to relneg.tex% %---------------------------------------
%2014jun15sun moved to relneg.tex% \exbox{\begin{tabular}{m{\tw-53mm}m{34mm}}
%2014jun15sun moved to relneg.tex%   The \structb{center} of the lattice illustrated to the right is illustrated with solid dots.
%2014jun15sun moved to relneg.tex%   Note that the center is the \prope{Boolean} lattice $\latL_2^2$ \xref{prop:latoc_center_boolean}.
%2014jun15sun moved to relneg.tex%   & \centering\psset{unit=5.25mm}%============================================================================
% Daniel J. Greenhoe
% LaTeX file
% nominal unit = 7.5mm
%============================================================================
\begin{pspicture}(-3,-\latbot)(3,4.5)
  %---------------------------------
  % nodes
  %---------------------------------
  \Cnode*(0,4){t}
  \Cnode(-1,3){yc}\Cnode(1,3){xc}%
  \Cnode(-2,2){q}\Cnode(-1,2){p}\Cnode(1,2){pc}\Cnode(2,2){qc}%
  \Cnode*(-1,1){x}\Cnode*(1,1){y}%
  \Cnode*(0,0){b}
  %---------------------------------
  % node connections
  %---------------------------------
  \ncline{t}{yc}\ncline{t}{xc}%
  \ncline{p}{yc}\ncline{pc}{xc}%
  \ncline{yc}{q}\ncline{xc}{qc}%
  \ncline{x}{p}\ncline{y}{pc}%
  \ncline{x}{q}\ncline{y}{qc}%
  \ncline{b}{x}\ncline{b}{y}%
  %---------------------------------
  % node labels
  %---------------------------------
  \uput[0](t){$\bid$}%     
  \uput[45](xc){$x^\ocop$}%     
  \uput[135](yc){$y^\ocop$}%     
  \uput[0](pc){$p^\ocop$}%     
  \uput[180](p){$p$}%     
  \uput[-180](q){$q$}%     
  \uput[0](qc){$q^\ocop$}%     
  \uput[-45](y){$y$}%     
  \uput[225](x){$x$}%     
  \uput[0](b) {$\bzero$}%     
\end{pspicture}%%
%2014jun15sun moved to relneg.tex% \end{tabular}}
%2014jun15sun moved to relneg.tex% \end{example}
%2014jun15sun moved to relneg.tex% 
%2014jun15sun moved to relneg.tex% 
%2014jun15sun moved to relneg.tex% 
%2014jun15sun moved to relneg.tex% %---------------------------------------
%2014jun15sun moved to relneg.tex% \begin{example}
%2014jun15sun moved to relneg.tex% %---------------------------------------
%2014jun15sun moved to relneg.tex% \exbox{\begin{tabular}{m{\tw-53mm}m{34mm}}
%2014jun15sun moved to relneg.tex%   In a \prope{Boolean} lattice, such as the one illustrated to the right,
%2014jun15sun moved to relneg.tex%   every element is in the center \xref{prop:latoc_center_boolean}.
%2014jun15sun moved to relneg.tex%   & \centering\psset{unit=7.5mm}%============================================================================
% Daniel J. Greenhoe
% LaTeX file
% lattice (2^{x,y,z}, subseteq)
% nominal unit = 7.5mm
%============================================================================
\begin{pspicture}(-1.7,-\latbot)(1.7,3.4)
  %---------------------------------
  % nodes
  %---------------------------------
  \Cnode*(0,3){t}%
  \Cnode*(-1,2){zc}\Cnode*(0,2){yc}\Cnode*(1,2){xc}%
  \Cnode*(-1,1){x}\Cnode*(0,1){y}\Cnode*(1,1){z}%
  \Cnode*(0,0){b}%
  %---------------------------------
  % node connections
  %---------------------------------
  \ncline{t}{zc}\ncline{t}{yc}\ncline{t}{xc}%
  \ncline{x}{zc}\ncline{x}{yc}%
  \ncline{y}{zc}\ncline{y}{xc}%
  \ncline{z}{yc}\ncline{z}{xc}%
  \ncline{b}{x} \ncline{b}{y} \ncline{b}{z}%
  %---------------------------------
  % node labels
  %---------------------------------
  \uput[10](t) {$\bid$}%
  \uput{1pt}[150](zc){$z^\ocop$}%   
  \uput{1pt}[-30](xc){$x^\ocop$}%
  \uput[150](x) {$x$}%     
  \uput[-30](z) {$z$}%
  \uput[-10](b) {$\bzero$}%
  %
  \uput[-45](y){$y$}   
  %\uput[0](1,0){\rnode{ylabel}{$y$}}%
  %\ncline[linestyle=dotted,nodesep=1pt]{->}{ylabel}{y}%
  %
  %\uput[45](yc){$y^\ocop$}
  \pnode(1,3){yclabelp}%
  \uput{1pt}[-30](yclabelp){\rnode{yclabel}{$y^\ocop$}}% 
  \ncline[linestyle=dotted,linecolor=red,nodesep=1pt]{->}{yclabel}{yc}%
\end{pspicture}%
%
%2014jun15sun moved to relneg.tex% \end{tabular}}
%2014jun15sun moved to relneg.tex% \end{example}

%=======================================
\subsection{Restrictions resulting in Boolean algebras}
%=======================================
%---------------------------------------
\begin{proposition}
\footnote{
  \citerpg{kalmbach1983}{22}{0123945801}
  }
\label{prop:latoc_distrib_boolean}
%---------------------------------------
Let $\latL=\latocd$ be a \structe{lattice} \xref{def:lattice}.
\propbox{
  \brb{\begin{array}{FMDD}
    1. & $\latL$ is \propb{orthocomplemented} & \xref{def:latoc} & and\\ % \xref{def:latoc}& and  \\
    2. & $\latL$ is \propb{distributive}      & \ifxref{latd}{def:latd}  &
  \end{array}}
  \quad\implies\quad
  \brb{\begin{array}{M}
    $\latL$ is \propb{Boolean}
    \ifnxref{boolean}{def:booalg}
  \end{array}}
  }
%2014jun08sun% \propbox{
%2014jun08sun%   \brbr{\begin{array}{FMDCDD}
%2014jun08sun%     1. & $\latL$ is \propb{orthocomplemented} & and\\ % \xref{def:latoc}& and  \\
%2014jun08sun%     2. & $\mcom{x\join\brp{y\meet z} = \brp{x\join y} \meet \brp{x\join z}}{\propb{dual distributive} \xref{def:Drel}}$ & $\forall x,y,z\in\setX$
%2014jun08sun%   \end{array}}
%2014jun08sun%   \quad\implies\quad
%2014jun08sun%   \brb{\begin{array}{M}
%2014jun08sun%     $\latL$ is \\ 
%2014jun08sun%     \propb{Boolean}
%2014jun08sun%     \ifdochas{boolean}{\\\xref{def:booalg}}
%2014jun08sun%   \end{array}}
%2014jun08sun%   }
\end{proposition}
\begin{proof}
To be a \structe{Boolean algebra}, $\latL$ must satisfy the 8 requirements of \structe{boolean algebra}s\ifsxref{boolean}{def:booalg}:
\begin{enumerate}
  \item Proof for \prope{commutative} properties: These are true for \emph{all} lattices\ifsxref{lattice}{def:lattice}.
  \item Proof for \prope{join-distributive} property: by hypothesis (2).
  \item Proof for \prope{meet-distributive} property: by \prope{join-distributive} property and 
        the \thme{Principle of duality}\ifsxref{lattice}{thm:duality} for lattices.
  \item Proof for \prope{identity} properties: because $\latL$ is a \structe{bounded lattice}\ifsxref{latvar}{def:latb}
        and by definitions of $1$ (\structe{least upper bound}), $0$ (\structe{greatest lower bound}), $\join$, and $\meet$.
  \item Proof for \prope{complemented} properties: by hypothesis (1) and definition of \structe{orthocomplemented lattice}s \xref{def:latoc}.
\end{enumerate}
\end{proof}

%---------------------------------------
\begin{proposition} % 2013dec11wed 6:33am
%\footnote{
%  }
\label{prop:latoc_center_boolean}
%---------------------------------------
Let $\latL=\latocd$ be a \structe{lattice} \xref{def:lattice}.
\propbox{
  \brb{\begin{array}{FMDD}
    1. & $\latL$ is \propb{orthocomplemented}                     &\xref{def:latoc}& and\\
    2. & Every $x\in\latL$ is in the \structb{center} of $\latL$  &\xref{def:center}
  \end{array}}
  \quad\iff\quad
  \brb{\begin{array}{M}
    $\latL$ is \\ 
    \propb{Boolean}
  \end{array}}
  }
\end{proposition}
\begin{proof}
\begin{enumerate}
  \item Proof that (1,2) $\implies$ \prope{Boolean}:
    $\latL$ is \prope{Boolean} because it satisfies \thme{Huntington's Fourth Set}\ifsxref{boolean}{prop:boo_char_H4}, as demonstrated by the following \ldots
    \begin{enumerate}
      \item Proof that $x\join x=x$ (\prope{idempotent}):         $\latL$ is a \structe{lattice} (by definition of $\latL$), and all lattices are \prope{idempotent} \xref{def:lattice}.
      \item Proof that $x\join y=y\join x$ (\prope{commutative}): $\latL$ is a \structe{lattice} (by definition of $\latL$), and all lattices are \prope{commutative} \xref{def:lattice}.
      \item Proof that $(x\join y)\join z=x\join(y\join z)$ (\prope{associative}): $\latL$ is a \structe{lattice} (by definition of $\latL$), and all lattices are \prope{associative} \xref{def:lattice}.
      \item Proof that $(x^\ocop\join y^\ocop)^\ocop \join (x^\ocop\join y)^\ocop = x$ (\prope{Huntington's axiom}):
        \begin{align*}
          (x^\ocop\join y^\ocop)^\ocop \join (x^\ocop\join y)^\ocop
            &= (x^\ocop\ocop\meet y^\ocop\ocop) \join (x^\ocop\ocop\meet y^\ocop)
            && \text{by \prope{de Morgan} property \xref{thm:latoc_prop}}
          \\&= (x\meet y) \join (x\meet y^\ocop)
            && \text{by \prope{involution} property \xref{def:latoc}}
          \\&= x
            && \text{by definition of \structe{center} \xref{def:center}}
        \end{align*}
    \end{enumerate}

  \item Proof that (1) $\impliedby$ \prope{Boolean}:
    \begin{enumerate}
      \item Proof that $x\join x^\ocop=\bid$: by definition of \structe{Boolean algebra}s\ifsxref{boolean}{def:booalg}.
      \item Proof that $x\meet x^\ocop=\bzero$: by definition of \structe{Boolean algebra}s\ifsxref{boolean}{def:booalg}.
      \item Proof that $x^{\ocop\ocop}=x$: by \prope{involutory} property of \structe{Boolean algebra}\ifsxref{boolean}{thm:boo_prop}.
      \item Proof that $x\le y\implies y^\ocop\le x^\ocop$: % 2013dec11wed 1:16pm
        \begin{align*}
          y^\ocop \le x^\ocop
            &\iff y^\ocop &&= y^\ocop \meet x^\ocop
            && \text{by \prefp{lem:le_meet}}
          \\&\iff y^{\ocop\ocop} &&= \brp{y^\ocop \meet x^\ocop}^\ocop
          \\&\iff y^{\ocop\ocop} &&= y^{\ocop\ocop} \join x^{\ocop\ocop}
            && \text{by \prope{de Morgan} property\ifsxref{boolean}{thm:boo_prop}}
          \\&\iff y &&= y \join x
            && \text{by \prope{involutory} property\ifsxref{boolean}{thm:boo_prop}}
          \\&\iff y &&= y
            && \text{by $x\le y$ hypothesis}
        \end{align*}
    \end{enumerate}

  \item Proof that (2) $\impliedby$ \prope{Boolean}: for all $x,y\in\latL$
    \begin{align*}
      \brp{x\meet y}\join\brp{x\meet y^\ocop}
        &= \brs{\brp{x\meet y}\join x} \meet \brs{\brp{x\meet y}\join y^\ocop}
        && \text{by \prope{distributive} property\ifsxref{boolean}{thm:boo_prop}}% of \structe{Boolean algebra}
      \\&= x \meet \brs{\brp{x\meet y}\join y^\ocop}
        && \text{by \prope{absorptive} property\ifsxref{boolean}{thm:boo_prop}}% of \structe{Boolean algebra}
      \\&= x \meet \brs{\brp{x\join y^\ocop} \meet \brp{y\join y^\ocop}}
        && \text{by \prope{distributive} property\ifsxref{boolean}{thm:boo_prop}}% of \structe{Boolean algebra}
      \\&= x \meet \brp{x\join y^\ocop} \meet \bid
        && \text{by \prope{complement} property\ifsxref{boolean}{thm:boo_prop}}% of \structe{Boolean algebra}
      \\&= x 
        && \text{by \prope{absorptive} property\ifsxref{boolean}{thm:boo_prop}}% of \structe{Boolean algebra}
      \\&\implies x\commutes y \quad\forall x,y\in\latL
        && \text{by \prefp{def:latoc_commutes}}
      \\&\implies \text{$x$ is in the \structe{center} of $\latL$ for all $x\in\latL$}
        && \text{by \prefp{def:center}}
    \end{align*}

\end{enumerate}
\end{proof}

%---------------------------------------
\begin{example}
\label{ex:latoc_latoc_d}
%---------------------------------------
\exbox{%
  \begin{tabular}{m{20mm}m{\tw-63mm}m{20mm}}%
    \centering\includegraphics{graphics/lat6_o6.pdf}%
    & The \structe{$O_6$ lattice} \xref{def:o6} illustrated to the left is
      \propb{orthocomplemented} \xref{def:latoc} but
      \propb{non-join-distributive}\ifsxref{latd}{def:latd},and hence \prope{non-Boolean}.
      The lattice illustrated to the right is
      \propb{orthocomplemented} \emph{and} \propb{distributive}
      and hence also \propb{Boolean} \xref{prop:latoc_distrib_boolean}.
      Alternatively, the right side lattice is 
      \propb{orthocomplemented} \emph{and} every element is in the \structe{center},
      and hence also \propb{Boolean} \xref{prop:latoc_center_boolean}.
    &\centering\includegraphics{graphics/lat8_l2e3.pdf}%
  \end{tabular}%
  }%
\ifexclude{mssa}{\\
Note that of the 5 lattices on 5 element sets \xref{ex:lat_set5}, the 15 lattices on 6 element sets \xref{ex:lat_set6},
and 53 lattices on 7 element sets \xref{ex:lat_set7}, \textbf{none} are \propb{uniquely complemented}.}
\end{example}
\begin{proof}
\begin{enumerate}
  \item Proof that the \structe{$O_6$ lattice} is \prope{non-join-distributive}:
    \begin{align*}
      x \join (x^\ocop \meet z^\ocop)
        &=    x \join \bzero
      \\&=    x
      \\&\neq z^\ocop
      \\&=    \bid \meet z^\ocop
      \\&=    (x \join x^\ocop) \meet (x \join z^\ocop)
    \end{align*}

  \item Proof that the \structe{$O_6$ lattice} is also \prope{non-meet-distributive}:
    \begin{align*}
      z^\ocop \meet (x \join z)
        &=    z^\ocop \meet \bid
      \\&=    z^\ocop
      \\&\neq x
      \\&=    x \join \bid
      \\&=    (z^\ocop \meet x) \join (z^\ocop \meet z)
    \end{align*}
\end{enumerate}
\end{proof}



%=======================================
\section{Orthomodular lattices}
%=======================================
%=======================================
\subsection{Properties}
%=======================================
%---------------------------------------
\begin{definition}
\footnote{
  %\citerpg{stern1999}{?}{0521461057}\\
  %\citerpg{beran1985}{?}{902771715X}\\
  \citerpg{kalmbach1983}{22}{0123945801},
  \citerpg{lidl1998}{90}{0387982906},
  \citeP{husimi1937}
  }
\label{def:latoc_omod}
\label{def:latom}
%---------------------------------------
Let $\latL\eqd\latocd$ be an algebraic structure. % \structe{orthocomplemented lattice}. % \xref{def:latoc}.
\defboxt{
  $\latL$ is an \structd{orthomodular lattice} if
  \\\indentx$\begin{array}{FMCD}
    1. & $\latL$ is an \structe{orthocomplemented lattice}                &                    & and\\
    2. & $x \orel y \quad\implies\quad x \join \brp{x^\ocop \meet y} = y$ &\forall x,y\in\setX & (\prope{orthomodular identity})
  \end{array}$
  }
\end{definition}

%%---------------------------------------
%\begin{example}
%\label{ex:latoc_hsum}
%\footnote{
%  \citerpgc{kalmbach1983}{24}{0123945801}{Figure 3.2}\\
%  \citerpg{beran1985}{42}{902771715X}\\
%  \citerpg{stern1999}{12}{0521461057}
%  }%
%%---------------------------------------
%\exbox{
%  \begin{tabular}{cm{44mm}}
%    \begin{tabular}{l}
%    The lattice illustrated to the right is \prope{orthomodular},\\
%    (and hence also \prope{orthocomplemented}).
%    %but also \prope{non-modular} (and hence also \prope{non-Boolean}).
%    \end{tabular}
%    &\centering\psset{unit=8mm}%============================================================================
% Daniel J. Greenhoe
% LaTeX file
% lattice (2^{x,y,z}, subseteq)
%============================================================================
%{%\psset{unit=0.075mm}
\begin{pspicture}(-2.2,-0.2)(2.2,3.2)
  %---------------------------------
  % settings
  %---------------------------------
  \psset{%
    labelsep=1.5mm,
    }%
  %---------------------------------
  % nodes
  %---------------------------------
  \Cnode(0,3){t}
  \Cnode(1,2){xc} \Cnode(0,2){yc} \Cnode(-1,2){zc}
  \Cnode(-2,1.5){a} \Cnode(2,1.5){ac}
  \Cnode(-1,1){x}  \Cnode(0,1){y}  \Cnode(1,1){z}
  \Cnode(0,0){b}
  %---------------------------------
  % node connections
  %---------------------------------
  \ncline{t}{a}\ncline{t}{xc}\ncline{t}{yc}\ncline{t}{zc}\ncline{t}{ac}%
  \ncline{x}{zc}\ncline{x}{yc}%
  \ncline{y}{xc}\ncline{y}{zc}%
  \ncline{z}{xc}\ncline{z}{yc}%
  \ncline{b}{a}\ncline{b}{x}\ncline{b}{y}\ncline{b}{z}\ncline{b}{ac}%
  %---------------------------------
  % node labels
  %---------------------------------
  \uput[0](t){$\bid$}%
  \uput[22](xc){$x^\ocop$} \uput[45](yc){$y^\ocop$} \uput[-45](zc){$z^\ocop$}%
  \uput[0](a){$a$} \uput[0](ac){$a^\ocop$}%
  \uput[-22](x){$x$} \uput[-22](y){$y$} \uput[0](z){$z$}%
  \uput[0](b){$\bzero$}%
\end{pspicture}%
%}

%  \end{tabular}
%  }
%\end{example}

%---------------------------------------
\begin{example}
%\footnote{
%  }%
\label{ex:o6_nonmod}
%---------------------------------------
\exboxt{%
    The \structe{$O_6$ lattice} \xref{def:o6} is \prope{orthocomplemented}, but \prope{non-orthomodular}\\% 
    (and hence, \prope{non-modular} and \prope{non-Boolean}).%
    }%
\end{example}%

%%---------------------------------------
%\begin{example}
%\footnote{
%  \citerp{maeda1966}{250}\\
%  \citerpg{beran1985}{33}{902771715X}
%  }%
%\label{ex:latoc_n5n5_nonmod}
%%---------------------------------------
%\exboxt{%
%  %\begin{tabular}{cm{34mm}}%
%    %\begin{tabular}{l}%
%      The lattice illustrated in \prefp{ex:latoc_n5n5} is \prope{orthocomplemented}, but \prope{non-orthomodular}\\% 
%     (and hence, \prope{non-modular} and \prope{non-Boolean}).%
%    %\end{tabular}%
%    %&\centering%============================================================================
% Daniel J. Greenhoe
% LaTeX file
%============================================================================
%{%\psset{unit=0.075mm}
\begin{pspicture}(-1.6,-0.5)(1.6,3.2)
  %---------------------------------
  % settings
  %---------------------------------
  %\psset{%
  %  labelsep=1.5mm,
  %  }%
  %---------------------------------
  % nodes
  %---------------------------------
  \Cnode(0,3){t}%
  \Cnode(1,2.5){pc}%
  \Cnode(0,1){y}%
  \Cnode(1,1.5){xc}%
  \Cnode(-1,2){zc}%
  %
  \Cnode(0,2){yc}%
  \Cnode(1,1){z}%
  \Cnode(-1,1.5){x}%
  \Cnode(-1,0.5){p}%
  \Cnode(0,0){b}%
  %---------------------------------
  % node connections
  %---------------------------------
  \ncline{t}{zc}\ncline{t}{yc}\ncline{t}{pc}%
  \ncline{xc}{pc}%
  \ncline{z}{yc}\ncline{z}{xc}%
  \ncline{y}{zc}\ncline{y}{xc}%
  \ncline{x}{zc}\ncline{x}{yc}%
  \ncline{p}{x}%
  \ncline{b}{p}\ncline{b}{y}\ncline{b}{z}%
  %---------------------------------
  % node labels
  %---------------------------------
  \uput[45](t){$\bid$}%
  \uput[0](pc){$p^\ocop$}%
  \uput[180](zc){$z^\ocop$}%
  \uput[45](yc){$y^\ocop$}%
  \uput[0](xc){$x^\ocop$}%
  \uput[0](z){$z$}%
  \uput[-45](y){$y$}%
  \uput[180](x){$x$}%
  \uput[180](p){$p$}%
  \uput[-45](b){$\bzero$}
\end{pspicture}%
%    }
%
%  %\end{tabular}%
%    }%
%\end{example}%



\begin{minipage}{11\tw/16}%
%---------------------------------------
\begin{example}
\footnotemark
%---------------------------------------
Let $\spH$ be a Hilbert space and $\pset{\spH}$ the set of closed linear subspaces of $\spH$.
\exbox{\begin{array}{M}
  $\latoc{\pset{\spH}}{\subseteq}{\oplus}{\seti}{^\ocop}{\emptyset}{\spH}$\\
  is an orthomodular lattice.
\end{array}}
\\
This concept is illustrated to the right where $\spX,\spY\in\pset{\spH}$ 
are linear subspaces of the linear space $\spH$ and
\\\indentx$ \spX \subseteq \spY \quad\implies\quad \spY=\spX \oplus \brp{\spX^\ocop \seti \setY}$.
\end{example}%
\end{minipage}%
\citetblt{%
  \citerppg{iturrioz1985}{56}{57}{0821850768}
  }%
\begin{minipage}{5\tw/16}%
  \begin{center}
  \footnotesize
  \setlength{\unitlength}{\tw/250}%
  \begin{picture}(250,250)(-100,-100)%
    \thicklines%
    %{\color{graphpaper}\graphpaper[10](-100,-100)(250,250)}%
    \color{blue}%
      \put(  25,  25){\vector(1, 0){75}}%
      \put( 105,  25){\makebox(0,0)[l]{$\spX$}}%
    \color{green}%
      \put(   0,-100){\line( 0, 1){200} }%
      \put(   0, 100){\line( 1, 1){ 50} }%
      \put(   0,-100){\line( 1, 1){ 50} }%
      \put(  50,  50){\line( 0, 1){100} }%
      \put(  50, -50){\line( 0, 1){ 50} }%
      \put(  70,  90){\makebox(0,0)[l]{$\spX^\ocop$}}%
      \put(  60,  90){\vector(-1,0){35}}%
    \color{red}%
      \put(-100,   0){\line( 1, 0){200} }%
      \put(-100,   0){\line( 1, 1){ 50} }%
      \put( 100,   0){\line( 1, 1){ 50} }%
      \put( -50,  50){\line( 1, 0){ 50} }%
      \put(  50,  50){\line( 1, 0){100} }%
      \put( -50, 110){\makebox(0,0)[bl]{$\spY=\spX \oplus \brp{\spX^\ocop \seti \setY}$}}%
      \put( -30, 100){\vector(0,-1){75}}%
    \color{black}%
      \put(   0,   0){\line( 1, 1){ 50} }%
      \put(  70, -30){\makebox(0,0)[tl]{$\spX^\ocop\seti\spY$}}%
      \put(  65, -35){\vector(-1,1){50}}%
  \end{picture}%
  \end{center}
\end{minipage}%

%---------------------------------------
\begin{theorem}
\footnote{
  \citerpg{kalmbach1983}{25}{0123945801},
  \citePppc{holland1963}{69}{70}{{\scshape Theorem 3}}, %{cf kalmbach 1983}
  \citePpc{foulis1962}{68}{{\scshape Theorem 5}} %{cf kalmbach 1983}
  }
%---------------------------------------
Let $\latL\eqd\latocd$ be a lattice.
\thmbox{
  \brbr{\begin{array}{FMD}
    1. & $\latL$ is \prope{orthomodular} & and \\
    2. & $y\commutes x$ and $z\commutes x$
  \end{array}}
  \qquad\implies\qquad
  \otriple{x}{y}{z}\in\distrib
  }
\end{theorem}

%=======================================
\subsection{Characterizations}
%=======================================

%---------------------------------------
\begin{theorem}
\footnote{
  \citerpg{kalmbach1983}{22}{0123945801},
  \citerpg{stern1999}{12}{0521461057},
  \citor{nakamura1957},
  \citor{holland1963},
  \citeP{foulis1962},
  \citerpc{maeda1970}{132}{Theorem 29.13}  %, cf Stern 1999 p.12
  %\citerpg{chiara2004}{14}{1402019785}\\
  %\citerpg{iturrioz1985}{57}{0821850768}
  %\citer{chiara2001}\\
  %\citer{chiara2002}
  }
\label{thm:latoc_omod}
%---------------------------------------
Let $\latL\eqd\latocd$ be an \structe{orthocomplemented lattice} \xref{def:latoc}.
Let $\modular$ and $\modulard$ be the \relx{modularity} relation 
and \relx{dual modularity} relation, respectively\ifsxref{latm}{def:lat_Mrel},
$\orthog$ the \relx{orthogonality} relation (\prefp{def:latoc_orthog}), and
$\commutes$ the \relx{commutes} relation (\prefp{def:latoc_commutes}).
\thmboxt{
  The following statements are \emph{equivalent}:
  \\\indentx$\begin{array}{cFM}
         & 1. & $\latL$ is \prope{orthomodular} \\
    \iff & 2. & $x\orel y$ and $y\meet x^\ocop=\bzero$ $\implies$ $x=y$ \\
    \iff & 3. & $\latL$ does \emph{not} contain the $O_6$ lattice\\
    \iff & 4. & $x\commutes y$ $\iff$ $y\commutes x$ ($\commutes$ is \prope{symmetric})\\
    \iff & 5. & $x \modular  x^\ocop \qquad \forall x\in\setX$ \\
    \iff & 6. & $x \modulard x^\ocop \qquad \forall x\in\setX$ \\
    \iff & 7. & $x \join \brs{x^\ocop \meet \brp{x\join y}} = x \join y \qquad\forall x,y\in\setX$\\
    \iff & 8. & $x\orel y \qquad\implies\qquad \exists p\in\setX \st x\orthog p \text{ and } x\join p=y$
  \end{array}$
  }
\end{theorem}
\begin{proof}
\begin{enumerate}
  \item Proof that \prope{orthomodular} $\iff$ \prope{symmetric}:
    by \prefpp{prop:latoc_symmetry}.
\end{enumerate}
\end{proof}

%=======================================
\subsection{Restrictions resulting in Boolean algebras}
%=======================================
%---------------------------------------
\begin{theorem}
\footnote{
  \citePp{renedo2003}{72}
  }
%---------------------------------------
Let $\latL=\latocd$ be an algebraic structure.
\thmbox{
  \brb{\begin{array}{MD}
    $\latL$ is an \structb{orthomodular lattice} & and\\ %\xref{def:latom}& and\\
    $\mcom{\brp{x\meet y^\ocop}^\ocop = y \join \brp{x^\ocop\meet y^\ocop}}{\prope{Elkan's law}}$ & $\forall x,y\in\setX$
  \end{array}}
  \quad\implies\quad
  \brb{\begin{array}{M}
    $\latL$ is a\\ 
    \structb{Boolean algebra}
    \ifdochas{boolean}{\\\xref{def:booalg}}
  \end{array}}
  }
\end{theorem}


%---------------------------------------
\begin{definition}
\footnote{
  \citerpg{iturrioz1985}{57}{0821850768},
  \citerppgc{davey2002}{18}{19}{0521784514}{1.25 Products}
  }
\label{def:latoc_cl}
\label{def:latoc_mo2}
\label{def:mo2}
%\footnotemark
%---------------------------------------
\defboxt{
  The \structd{MO$_2$ lattice} is the ordered set $\opair{\setn{0,x,y,x^\ocop,y^\ocop,1}}{\orel}$
  with cover relation
  \\\indentx
    $\coverrel = \setn{\opair{0}{x},\, \opair{0}{y},\, \opair{0}{x^\ocop},\, \opair{0}{y^\ocop},\, \opair{x}{1},\, \opair{y}{1},\, \opair{x^\ocop}{1},\, \opair{y^\ocop}{1}}$
  \\This lattice is also called the \hid{Chinese lantern}.
  }
\end{definition}

\begin{tabular}{m{33mm}m{33mm}m{33mm}}%
   \centering\includegraphics{graphics/lat6_mo2_xy.pdf}%
  &\centering\includegraphics{graphics/lat2_l2.pdf}%
  &\centering\includegraphics{graphics/lat12_mo2x2.pdf}%
  \tabularnewline%
   \centering MO$_2$%
  &\centering L$_2$%
  &\centering MO$_2\times$L$_2$%
\end{tabular}


%---------------------------------------
\begin{theorem}
\footnote{
  \citerpg{iturrioz1985}{57}{0821850768},
  \citePc{carrega1982}{cf Iturrioz 1985 page 57}
  }
%---------------------------------------
Let $\latM=\latocd$ be an \prope{orthomodular} lattice.
\thmbox{
  \brb{\begin{array}{M}
    $\latM$ is\\ 
    \prope{Boolean}
  \end{array}}
  \iff
  \brb{\begin{array}{FMD}
    1. & $\latM$ does not contain the MO$_2$ lattice \xref{def:mo2} & and \\
    2. & $\latM$ does not contain the MO$_2\times$L$_2$ lattice.
  \end{array}}
  }
\end{theorem}

%=======================================
\section{Modular orthocomplemented lattices}
%=======================================
%---------------------------------------
\begin{definition}
\label{def:latmoc}
%---------------------------------------
Let $\latL\eqd\latbd$ be a \structe{bounded lattice} \xref{def:latb}.
\defboxt{
  $\latL$ is a \structd{modular orthocomplemeted lattice} if
  \\\indentx$\begin{array}{FMMD}
    1. & $\latL$ is \propb{orthocomplemented} & \xref{def:latoc} & and \\
    2. & $\latL$ is \propb{modular}           & \ifxref{latm}{def:latm}
  \end{array}$
  }
\end{definition}

%%---------------------------------------
%\begin{example}
%\label{ex:latoc_hsum}
%\footnote{
%  \citerpgc{kalmbach1983}{24}{0123945801}{Figure 3.2}\\
%  \citerpg{beran1985}{42}{902771715X}\\
%  \citerpg{stern1999}{12}{0521461057}
%  }%
%%---------------------------------------
%\exbox{
%  \begin{tabular}{cm{44mm}}
%    \begin{tabular}{l}
%    The lattice illustrated to the right is \prope{orthomodular},\\
%    (and hence also \prope{orthocomplemented}),\\
%    but also \prope{non-modular} (and hence also \prope{non-Boolean}).
%    \end{tabular}
%    &
%    {%============================================================================
% Daniel J. Greenhoe
% LaTeX file
% lattice (2^{x,y,z}, subseteq)
%============================================================================
%{%\psset{unit=0.075mm}
\begin{pspicture}(-2.2,-0.2)(2.2,3.2)
  %---------------------------------
  % settings
  %---------------------------------
  \psset{%
    labelsep=1.5mm,
    }%
  %---------------------------------
  % nodes
  %---------------------------------
  \Cnode(0,3){t}
  \Cnode(1,2){xc} \Cnode(0,2){yc} \Cnode(-1,2){zc}
  \Cnode(-2,1.5){a} \Cnode(2,1.5){ac}
  \Cnode(-1,1){x}  \Cnode(0,1){y}  \Cnode(1,1){z}
  \Cnode(0,0){b}
  %---------------------------------
  % node connections
  %---------------------------------
  \ncline{t}{a}\ncline{t}{xc}\ncline{t}{yc}\ncline{t}{zc}\ncline{t}{ac}%
  \ncline{x}{zc}\ncline{x}{yc}%
  \ncline{y}{xc}\ncline{y}{zc}%
  \ncline{z}{xc}\ncline{z}{yc}%
  \ncline{b}{a}\ncline{b}{x}\ncline{b}{y}\ncline{b}{z}\ncline{b}{ac}%
  %---------------------------------
  % node labels
  %---------------------------------
  \uput[0](t){$\bid$}%
  \uput[22](xc){$x^\ocop$} \uput[45](yc){$y^\ocop$} \uput[-45](zc){$z^\ocop$}%
  \uput[0](a){$a$} \uput[0](ac){$a^\ocop$}%
  \uput[-22](x){$x$} \uput[-22](y){$y$} \uput[0](z){$z$}%
  \uput[0](b){$\bzero$}%
\end{pspicture}%
%}
}
%  \end{tabular}
%  }
%\end{example}

%%---------------------------------------
%\begin{example}
%%\footnote{
%%  }%
%\label{ex:o6_nonmod}
%%---------------------------------------
%\exboxt{%
%    The \structe{$O_6$ lattice} \xref{def:o6} is \prope{orthocomplemented}, but \prope{non-orthomodular}\\% 
%    (and hence, \prope{non-modular} and \prope{non-Boolean}).%
%    }%
%\end{example}%

%%---------------------------------------
%\begin{example}
%\footnote{
%  \citerp{maeda1966}{250}\\
%  \citerpg{beran1985}{33}{902771715X}
%  }%
%\label{ex:latoc_n5n5_nonmod}
%%---------------------------------------
%\exboxt{%
%  %\begin{tabular}{cm{34mm}}%
%    %\begin{tabular}{l}%
%      The lattice illustrated in \prefp{ex:latoc_n5n5} is \prope{orthocomplemented}, but \prope{non-orthomodular}\\% 
%     (and hence, \prope{non-modular} and \prope{non-Boolean}).%
%    %\end{tabular}%
%    %&\centering%============================================================================
% Daniel J. Greenhoe
% LaTeX file
%============================================================================
%{%\psset{unit=0.075mm}
\begin{pspicture}(-1.6,-0.5)(1.6,3.2)
  %---------------------------------
  % settings
  %---------------------------------
  %\psset{%
  %  labelsep=1.5mm,
  %  }%
  %---------------------------------
  % nodes
  %---------------------------------
  \Cnode(0,3){t}%
  \Cnode(1,2.5){pc}%
  \Cnode(0,1){y}%
  \Cnode(1,1.5){xc}%
  \Cnode(-1,2){zc}%
  %
  \Cnode(0,2){yc}%
  \Cnode(1,1){z}%
  \Cnode(-1,1.5){x}%
  \Cnode(-1,0.5){p}%
  \Cnode(0,0){b}%
  %---------------------------------
  % node connections
  %---------------------------------
  \ncline{t}{zc}\ncline{t}{yc}\ncline{t}{pc}%
  \ncline{xc}{pc}%
  \ncline{z}{yc}\ncline{z}{xc}%
  \ncline{y}{zc}\ncline{y}{xc}%
  \ncline{x}{zc}\ncline{x}{yc}%
  \ncline{p}{x}%
  \ncline{b}{p}\ncline{b}{y}\ncline{b}{z}%
  %---------------------------------
  % node labels
  %---------------------------------
  \uput[45](t){$\bid$}%
  \uput[0](pc){$p^\ocop$}%
  \uput[180](zc){$z^\ocop$}%
  \uput[45](yc){$y^\ocop$}%
  \uput[0](xc){$x^\ocop$}%
  \uput[0](z){$z$}%
  \uput[-45](y){$y$}%
  \uput[180](x){$x$}%
  \uput[180](p){$p$}%
  \uput[-45](b){$\bzero$}
\end{pspicture}%
%    }
%
%  %\end{tabular}%
%    }%
%\end{example}%


%=======================================
%\section{Boolean lattices}
\section{Relationships between orthocomplemented lattices}
%=======================================
%---------------------------------------
\begin{theorem}
\footnote{
  \citerpgc{kalmbach1983}{32}{0123945801}{20.},
  \citerpg{iturrioz1985}{57}{0821850768}
  }
\label{thm:latoc_types}
%---------------------------------------
Let $\latL$ be a lattice.
\thmbox{
  \brb{\parbox{3\tw/32}{\ragr$\latL$ is \prope{Boolean}}}
  \implies
  \brb{\parbox{4\tw/32}{\ragr$\latL$ is \prope{modular} \prope{orthocomplemented}}}
  \implies
  \brb{\parbox{4\tw/32}{\ragr$\latL$ is \prope{orthomodular}}}
  \implies
  \brb{\parbox{4\tw/32}{\ragr$\latL$ is \prope{orthocomplemented}}}
  }
\end{theorem}


%=======================================
%\section{More examples}
%=======================================
%2013dec27 %---------------------------------------
%2013dec27 \begin{example}
%2013dec27 \label{ex:latoc_benzene}
%2013dec27 \footnote{%
%2013dec27   \citerp{maeda1966}{250}\\
%2013dec27   \citerp{holland1970}{50}\\
%2013dec27   \citerppg{beran1985}{33}{35}{902771715X}\\
%2013dec27   \citerpg{stern1999}{12}{0521461057}
%2013dec27   }%
%2013dec27 %---------------------------------------
%2013dec27 \exbox{%
%2013dec27   \begin{tabular}{cm{34mm}}%
%2013dec27     \begin{tabular}{l}%
%2013dec27       The \hie{Benzene ring} \xref{def:latoc_benzene}, illustrated to the right,\\%
%2013dec27       is \prope{orthocomplemented}, but \prope{non-orthomodular}\\% 
%2013dec27       (and hence also \prope{non-modular}, and \prope{non-Boolean}).%
%2013dec27     \end{tabular}%
%2013dec27     &\centering\psset{unit=5mm}%============================================================================
% Daniel J. Greenhoe
% LaTeX file
% lattice (2^{x,y,z}, subseteq)
%============================================================================
%{%\psset{unit=0.075mm}
\begin{pspicture}(-1.6,-0.5)(1.6,3.5)
  %---------------------------------
  % settings
  %---------------------------------
  \psset{%
    labelsep=1.5mm,
    }%
  %---------------------------------
  % nodes
  %---------------------------------
  \Cnode(0,3){t}%
  \Cnode(1,2){xc}\Cnode(1,1){yc}%
  \Cnode(-1,1){x}\Cnode(-1,2){y}%
  \Cnode(0,0){b}%
  %---------------------------------
  % node connections
  %---------------------------------
  \ncline{t}{y}\ncline{t}{xc}%
  \ncline{x}{y}\ncline{yc}{xc}%
  \ncline{b}{x}\ncline{b}{yc}%
  %---------------------------------
  % node labels
  %---------------------------------
  \uput[0](t){$\bid$}%
  \uput[0](yc){$y^\ocop$}\uput[0](xc){$x^\ocop$}%
  \uput[180](x){$x$}\uput[180](y){$y$}
  \uput[0](b){$\bzero$}
\end{pspicture}%
%    }
%
%2013dec27   \end{tabular}%
%2013dec27     }%
%2013dec27 \end{example}


%%---------------------------------------
%\begin{example}
%\footnote{
%  \citerp{maeda1966}{250}\\
%  \citerpg{beran1985}{33}{902771715X}
%  }%
%\label{ex:latoc_n5n5}
%%---------------------------------------
%\exbox{%
%  \begin{tabular}{cm{34mm}}%
%    \begin{tabular}{l}%
%      The lattice illustrated to the right is \prope{orthocomplemented},\\% 
%      but \prope{non-orthomodular}\\% 
%     (and hence, \prope{non-modular} and \prope{non-Boolean}).%
%    \end{tabular}%
%    &\centering%============================================================================
% Daniel J. Greenhoe
% LaTeX file
%============================================================================
%{%\psset{unit=0.075mm}
\begin{pspicture}(-1.6,-0.5)(1.6,3.2)
  %---------------------------------
  % settings
  %---------------------------------
  %\psset{%
  %  labelsep=1.5mm,
  %  }%
  %---------------------------------
  % nodes
  %---------------------------------
  \Cnode(0,3){t}%
  \Cnode(1,2.5){pc}%
  \Cnode(0,1){y}%
  \Cnode(1,1.5){xc}%
  \Cnode(-1,2){zc}%
  %
  \Cnode(0,2){yc}%
  \Cnode(1,1){z}%
  \Cnode(-1,1.5){x}%
  \Cnode(-1,0.5){p}%
  \Cnode(0,0){b}%
  %---------------------------------
  % node connections
  %---------------------------------
  \ncline{t}{zc}\ncline{t}{yc}\ncline{t}{pc}%
  \ncline{xc}{pc}%
  \ncline{z}{yc}\ncline{z}{xc}%
  \ncline{y}{zc}\ncline{y}{xc}%
  \ncline{x}{zc}\ncline{x}{yc}%
  \ncline{p}{x}%
  \ncline{b}{p}\ncline{b}{y}\ncline{b}{z}%
  %---------------------------------
  % node labels
  %---------------------------------
  \uput[45](t){$\bid$}%
  \uput[0](pc){$p^\ocop$}%
  \uput[180](zc){$z^\ocop$}%
  \uput[45](yc){$y^\ocop$}%
  \uput[0](xc){$x^\ocop$}%
  \uput[0](z){$z$}%
  \uput[-45](y){$y$}%
  \uput[180](x){$x$}%
  \uput[180](p){$p$}%
  \uput[-45](b){$\bzero$}
\end{pspicture}%
%    }
%
%  \end{tabular}%
%    }%
%\end{example}%



%%---------------------------------------
%\begin{example}
%\label{ex:latocm_oc10}
%\footnote{
%  \citerp{maeda1966}{250}\\
%  \citerpg{beran1985}{34}{902771715X}
%  }%
%%---------------------------------------
%\exbox{
%  \begin{tabular}{cm{44mm}}
%    \begin{tabular}{l}
%    The lattice illustrated to the right
%    %in \prefpp{ex:latoc_oc10} 
%    is \prope{orthocomplemented},\\
%    but \prope{non-orthomodular} and \prope{non-modular}\\
%    (and hence also \prope{non-Boolean}).
%    \end{tabular}
%    &
%    {%============================================================================
% Daniel J. Greenhoe
% LaTeX file
% lattice (2^{x,y,z}, subseteq)
%============================================================================
%{%\psset{unit=0.075mm}
\begin{pspicture}(-2.2,-0.5)(2.2,3.5)
  %---------------------------------
  % settings
  %---------------------------------
  \psset{%
    labelsep=1.5mm,
    }%
  %---------------------------------
  % nodes
  %---------------------------------
  \Cnode(0,3){t}
  \Cnode(-1.5,2){zc} \Cnode(-0.5,2){yc} \Cnode(0.5,2){xc} \Cnode(1.5,2){wc}
  \Cnode(-1.5,1){w}  \Cnode(-0.5,1){x}  \Cnode(0.5,1){y}  \Cnode(1.5,1){z}
  \Cnode(0,0){b}
  %---------------------------------
  % node connections
  %---------------------------------
  \ncline{t}{zc}\ncline{t}{yc}\ncline{t}{xc}\ncline{t}{wc}
  \ncline{w}{zc}
  \ncline{x}{zc}\ncline{x}{yc}
  \ncline{y}{zc}\ncline{y}{xc}
  \ncline{z}{yc}\ncline{z}{xc}\ncline{z}{wc}
  \ncline{b}{w}\ncline{b}{x}\ncline{b}{y}\ncline{b}{z}
  %---------------------------------
  % node labels
  %---------------------------------
  \uput[90](t){$\bid$}
  \uput[180](zc){$z^\ocop$} \uput[158](yc){$y^\ocop$} \uput[180](xc){$x^\ocop$} \uput[0](wc){$w^\ocop$}
  \uput[180](w){$w$} \uput[0](x){$x$} \uput[0](y){$y$} \uput[0](z){$z$}
  \uput[-90](b){$\bzero$}
\end{pspicture}%
%    }
}
%  \end{tabular}
%  }
%\end{example}
%\begin{proof}
%\begin{enumerate}
%  \item Proof that $\latL$ is \prope{non-orthomodular}:
%    \begin{enumerate}
%      \item The nodes $\bid$, $y^\ocop$, $x$, $\bzero$, $y$ and $x^\ocop$ form the 
%            $O_6$ lattice \xref{def:o6}.
%      \item Therefore, by \prefpp{thm:latoc_omod}, $\latL$ is \prope{non-orthomodular}.
%    \end{enumerate}
%
%  \item Proof that $\latL$ is \prope{non-modular}:
%    \begin{enumerate}
%      \item The nodes $\bid$, $y^\ocop$, $x$, $\bzero$, and $y$form the 
%            $N5$ lattice\ifsxref{latm}{def:n5}.
%      \item Therefore, $\latL$ is \prope{modular}\ifsxref{latm}{thm:latm_n5}.
%    \end{enumerate}
%\end{enumerate}
%\end{proof}

%\hrule
%
%%---------------------------------------
%\begin{example}
%\label{ex:latoc_hsum}
%\footnote{
%  \citerpgc{kalmbach1983}{24}{0123945801}{Figure 3.2}\\
%  \citerpg{beran1985}{42}{902771715X}\\
%  \citerpg{stern1999}{12}{0521461057}
%  }%
%%---------------------------------------
%\exbox{
%  \begin{tabular}{cm{44mm}}
%    \begin{tabular}{l}
%    The lattice illustrated to the right is \prope{orthomodular},\\
%    (and hence also \prope{orthocomplemented}),\\
%    but also \prope{non-modular} (and hence also \prope{non-Boolean}).
%    \end{tabular}
%    &
%    {%============================================================================
% Daniel J. Greenhoe
% LaTeX file
% lattice (2^{x,y,z}, subseteq)
%============================================================================
%{%\psset{unit=0.075mm}
\begin{pspicture}(-2.2,-0.2)(2.2,3.2)
  %---------------------------------
  % settings
  %---------------------------------
  \psset{%
    labelsep=1.5mm,
    }%
  %---------------------------------
  % nodes
  %---------------------------------
  \Cnode(0,3){t}
  \Cnode(1,2){xc} \Cnode(0,2){yc} \Cnode(-1,2){zc}
  \Cnode(-2,1.5){a} \Cnode(2,1.5){ac}
  \Cnode(-1,1){x}  \Cnode(0,1){y}  \Cnode(1,1){z}
  \Cnode(0,0){b}
  %---------------------------------
  % node connections
  %---------------------------------
  \ncline{t}{a}\ncline{t}{xc}\ncline{t}{yc}\ncline{t}{zc}\ncline{t}{ac}%
  \ncline{x}{zc}\ncline{x}{yc}%
  \ncline{y}{xc}\ncline{y}{zc}%
  \ncline{z}{xc}\ncline{z}{yc}%
  \ncline{b}{a}\ncline{b}{x}\ncline{b}{y}\ncline{b}{z}\ncline{b}{ac}%
  %---------------------------------
  % node labels
  %---------------------------------
  \uput[0](t){$\bid$}%
  \uput[22](xc){$x^\ocop$} \uput[45](yc){$y^\ocop$} \uput[-45](zc){$z^\ocop$}%
  \uput[0](a){$a$} \uput[0](ac){$a^\ocop$}%
  \uput[-22](x){$x$} \uput[-22](y){$y$} \uput[0](z){$z$}%
  \uput[0](b){$\bzero$}%
\end{pspicture}%
%}
}
%  \end{tabular}
%  }
%\end{example}
%\begin{proof}
%\begin{enumerate}
%  \item Proof that $\latL$ is \prope{non-modular}:
%    The nodes $\bid$, $x^\ocop$, $x$, $\bzero$, and $a$ form an $N5$ lattice\ifsxref{latm}{def:n5}.
%    Therefore,\ifdochas{latm}{ by \prefpp{thm:latm_n5},} $\latL$ is \prope{non-modular}.
%\end{enumerate}
%\end{proof}


%\begin{minipage}[c]{8\tw/16}%
%%---------------------------------------
%\begin{example}
%\label{ex:latoc_dilworth7}
%%\footnotemark
%\footnote{
%  \citer{dilworth1940}\\
%  \citer{dilworth1940r}\\
%  \citerpg{kalmbach1983}{9}{0123945801}
%  }%
%%---------------------------------------
%\exboxp{
%  \begin{tabular}{m{\tw-106mm-10mm}m{106mm}}%
%    %\begin{tabular}{l}
%      \raggedright%
%      The lattice illustrated to the right is \prope{orthomodular}
%      (and hence also \prope{orthocomplemented}).
%    %\end{tabular}
%    &%\centering%============================================================================
% Daniel J. Greenhoe
% LaTeX file
% lattice (2^{x,y,z}, subseteq)
% recommended unit = 10mm
%============================================================================
\begin{pspicture}(-3.3,-\latbot)(3.3,3.3)%
  %---------------------------------
  % settings
  %---------------------------------
  %\psset{%
  %  labelsep=1.5mm,%
  %  }%
  %---------------------------------
  % nodes
  %---------------------------------
  \Cnode(0,3){t}%
  \Cnode(-3,2){dc}\Cnode(-2,2){cc}\Cnode(-1,2){zc}\Cnode(0,2){yc}\Cnode(1,2){xc}\Cnode(2,2){bc}\Cnode(3,2){ac}%
  \Cnode(-3,1){a}\Cnode(-2,1){b}\Cnode(-1,1){x}\Cnode(0,1){y}\Cnode(1,1){z}\Cnode(2,1){c}\Cnode(3,1){d}%
  \Cnode(0,0){glb}%
  %---------------------------------
  % node connections
  %---------------------------------
  {\psset{linecolor=red}%
    \ncline{t}{yc}\ncline{t}{xc}\ncline{t}{zc}%
    \ncline{x}{zc}\ncline{x}{yc}%
    \ncline{y}{zc}\ncline{y}{xc}%
    \ncline{z}{yc}\ncline{z}{xc}%
    \ncline{glb}{x}\ncline{glb}{y}\ncline{glb}{z}%
  }%
  \ncline{t}{bc}%
  \ncline{t}{dc}\ncline{t}{cc}%
  \ncline{t}{ac}%
  \ncline{a}{cc}\ncline{a}{zc}%
  \ncline{b}{dc}\ncline{b}{xc}%
  \ncline{x}{dc}\ncline{x}{bc}%
  \ncline{z}{cc}\ncline{z}{ac}%
  \ncline{c}{zc}\ncline{d}{bc}%
  \ncline{c}{ac}%
  \ncline{d}{xc}\ncline{d}{bc}%
  \ncline{glb}{a}\ncline{glb}{b}%
  \ncline{glb}{c}\ncline{glb}{d}%
  %---------------------------------
  % node labels
  %---------------------------------
  \uput[30](t){$\bid$}%
  \uput[45](xc){$x^\ocop$}\uput[30](yc){$y^\ocop$}\uput{1pt}[135](zc){$z^\ocop$}%
  \uput[210](x){$x$}\uput[-45](y){$y$}\uput[-30](z){$z$}%
  \uput[90](ac){$a^\ocop$}\uput[22.5](bc){$b^\ocop$}\uput[180](cc){$c^\ocop$}\uput[90](dc){$d^\ocop$}%
  \uput[-90](a){$a$}\uput[180](b){$b$}\uput[0](c){$c$}\uput[-90](d){$d$}%
  \uput{3mm}[-5](glb){$\bzero$}%
\end{pspicture}%
%
%  \end{tabular}}
%  \\%
%  The above lattice was originally introduced by Dilworth as a counterexample to \hie{Husimi's conjecture} (1937).
%  Kalmbach(1983) points out that this lattice was the first example of a \prope{finite orthomodular} lattice.
%\end{example}

%---------------------------------------
\begin{remark} % 2013dec27fri
%\label{rem:latoc_dilworth7}
\footnote{
  \citeP{dilworth1940},
  \citeP{dilworth1940r},
  \citerpg{kalmbach1983}{9}{0123945801}
  }%
%---------------------------------------
  Lattice number 8 in \prefpp{ex:latoc} was originally introduced by Dilworth as a counterexample to \hie{Husimi's conjecture} (1937).
  Kalmbach(1983) points out that this lattice was the first example of a \prope{finite orthomodular} lattice.
\end{remark}

%2013dec27 %---------------------------------------
%2013dec27 \begin{example}
%2013dec27 \footnote{%
%2013dec27   \citerpg{kalmbach1983}{32}{0123945801}
%2013dec27   }%
%2013dec27 %---------------------------------------
%2013dec27 \exbox{%
%2013dec27   \begin{tabular}{cm{33mm}}%
%2013dec27     \begin{tabular}{l}%
%2013dec27       The MO$_2$ lattice \xref{def:mo2}, illustrated to the right,\\
%2013dec27       is \prope{modular orthocomplemented}\\
%2013dec27       (and hence also \prope{orthomodular} and \prope{orthocomplemented}),\\
%2013dec27       but \prope{non-Boolean}.
%2013dec27     \end{tabular}%
%2013dec27     &\centering\psset{unit=7.5mm}%============================================================================
% Daniel J. Greenhoe
% LaTeX file
% lattice (2^{x,y,z}, subseteq)
%============================================================================
\begin{pspicture}(-1.9,-\latbot)(1.9,2.3)%
  %---------------------------------
  % settings
  %---------------------------------
  %\psset{%
  %  labelsep=1.5mm,%
  %  }%
  %---------------------------------
  % nodes
  %---------------------------------
  \Cnode(0,2){t}%
  \Cnode(-1.5,1){x}\Cnode(-0.5,1){y}%
  \Cnode(0.5,1){yc}\Cnode(1.5,1){xc}%
  \Cnode(0,0){b}%
  %---------------------------------
  % node connections
  %---------------------------------
  \ncline{t}{x}\ncline{t}{y}\ncline{t}{yc}\ncline{t}{xc}%
  \ncline{b}{x}\ncline{b}{y}\ncline{b}{yc}\ncline{b}{xc}%
  %---------------------------------
  % node labels
  %---------------------------------
  \uput[20](t) {$\bid$}%
  \uput[-90](xc) {$x^\ocop$}%     
  \uput[180](yc){$y^\ocop$}%   
  \uput[-90](x) {$x$}%     
  \uput[180](y){$y$}%   
  \uput[-10](b) {$\bzero$}%
\end{pspicture}%%
%2013dec27   \end{tabular}%
%2013dec27   }%
%2013dec27 \end{example}

%%---------------------------------------
%\begin{example}
%\label{ex:latoc_aos}
%%---------------------------------------
%\exbox{%
%%  \begin{tabular}{cm{29mm}}%
%  \begin{tabular}{cm{20mm}}%
%    \begin{tabular}{l}%
%      The lattice illustrated to the right is \prope{Boolean}\\
%      (and hence also \prope{modular}, \prope{orthomodular}, and \prope{orthocomplemented}).
%    \end{tabular}%
%    &\centering\psset{unit=7.5mm}%============================================================================
% Daniel J. Greenhoe
% LaTeX file
% lattice (2^{x,y,z}, subseteq)
%============================================================================
%\latmatl{4}{%
%                 & [name=1]\null                   \\
%  [name=zc]\null & [name=yc]\null & [name=xc]\null \\
%  [name=x]\null  & [name=y] \null & [name=z ]\null \\
%                 & [name=0]\null
%  }{%
%  \ncline{1}{zc}\ncline{1}{yc}\ncline{1}{xc}
%  \ncline{x}{zc}\ncline{x}{yc}
%  \ncline{y}{zc}\ncline{y}{xc}
%  \ncline{z}{yc}\ncline{z}{xc}
%  \ncline{0}{x} \ncline{0}{y} \ncline{0}{z}
%  }{%
%  \nput{ 90}{1} {$\setn{x,y,z}$}
%  \nput{135}{zc}{$\setn{x,y}$} \nput{ 45}{yc}{$\setn{x,z}$} \nput{0}{xc}{$\setn{y,z}$}
%  \nput{180}{x} {$\setn{x}$}   \nput{-45}{y} {$\setn{y}$}   \nput{0}{z}{$\setn{z}$}
%  \nput{-90}{0} {$\szero$}
%  }%
%\fbox{
%{\psset{unit=0.075mm}
\begin{pspicture}(-1.5,-0.5)(1.5,3.5)
  %---------------------------------
  % settings
  %---------------------------------
  \psset{%
    labelsep=1.5mm,
    }%
  %---------------------------------
  % nodes
  %---------------------------------
  \Cnode(0,3){t}%
  \Cnode(-1,2){zc}\Cnode(0,2){yc}\Cnode(1,2){xc}%
  \Cnode(-1,1){x}\Cnode(0,1){y}\Cnode(1,1){z}%
  \Cnode(0,0){b}%
  %---------------------------------
  % node connections
  %---------------------------------
  \ncline{t}{zc}\ncline{t}{yc}\ncline{t}{xc}%
  \ncline{x}{zc}\ncline{x}{yc}%
  \ncline{y}{zc}\ncline{y}{xc}%
  \ncline{z}{yc}\ncline{z}{xc}%
  \ncline{b}{x} \ncline{b}{y} \ncline{b}{z}%
  %---------------------------------
  % node labels
  %---------------------------------
  \uput[ 90](t) {$\bid$}%
  \uput[180](zc){$z^\ocop$}%   
  \uput[45](yc){$y^\ocop$}
  \uput[0](xc){$x^\ocop$}%
  \uput[180](x) {$x$}%     
  \uput[-45](y){$y$}   
  \uput[0](z) {$z$}%
  \uput[-90](b) {$\bzero$}%
  %\uput[0](100,300){\rnode{yclabel}{$y^\ocop$}}% 
  %\uput[0](100,  0){\rnode{ylabel}{$y$}}%
  %\ncline[linestyle=dotted,nodesep=1pt]{->}{yclabel}{yc}%
  %\ncline[linestyle=dotted,nodesep=1pt]{->}{ylabel}{y}%
\end{pspicture}%
%}%
%}%%
%  \end{tabular}%
%  }%
%\end{example}


%2013dec27 %---------------------------------------
%2013dec27 \begin{example}[orthocomplemented lattices on 1--3 element sets]
%2013dec27 %---------------------------------------
%2013dec27 \exbox{
%2013dec27 \begin{minipage}{\tw-60mm}
%2013dec27 There is only one unlabeled lattice for a finite set
%2013dec27 with 3 or fewer elements\ifsxref{lattice}{prop:num_lattices},
%2013dec27 as illustrated to the right.
%2013dec27 The lattices on sets of one and two elements are \prope{Boolean}
%2013dec27 (and hence also \prope{modular-orthoocomplemented}, \prope{orthomodular}, and \prope{orthocomplemented}).
%2013dec27 The lattice on a three element set is \prope{non-orthocomplemented}
%2013dec27 (and hence also \prope{non-orthomodular} and \prope{non-Boolean}).
%2013dec27 \end{minipage}%
%2013dec27 \hspace{5mm}%
%2013dec27 \begin{minipage}{45mm}%
%2013dec27 \center
%2013dec27 \begin{tabular}{ccc}
%2013dec27   \latmat{3}{
%2013dec27     [name=0]\null
%2013dec27     }
%2013dec27   &
%2013dec27   \latmat{3}{
%2013dec27     [name=1]\null \\
%2013dec27     [name=0]\null
%2013dec27     }{
%2013dec27     \ncline{0}{1}
%2013dec27     }
%2013dec27   &
%2013dec27   \latmat{3}{
%2013dec27     [name=1]\null \\
%2013dec27     [name=x]\null \\
%2013dec27     [name=0]\null
%2013dec27     }{
%2013dec27     \ncline{0}{x}
%2013dec27     \ncline{x}{1}
%2013dec27     }
%2013dec27 \end{tabular}
%2013dec27 \end{minipage}
%2013dec27   }
%2013dec27 \end{example}%

%2013dec27 %---------------------------------------
%2013dec27 \begin{example}[orthocomplemented lattices on 4 element sets]
%2013dec27 \label{ex:latoc_set4}
%2013dec27 %---------------------------------------
%2013dec27 \exbox{%
%2013dec27 \begin{minipage}{23mm}\centering%
%2013dec27 \latmatls{3}{0.75}{0.75}
%2013dec27   {              & [name=1]\null         \\
%2013dec27    [name=x]\null &               & [name=y]\null \\
%2013dec27                  & [name=0]\null
%2013dec27   }
%2013dec27   {\ncline{1}{x}\ncline{1}{y}
%2013dec27    \ncline{0}{x}\ncline{0}{y}
%2013dec27   }{
%2013dec27   \nput{ 90}{1}{$\bid$}
%2013dec27   \nput{180}{x}{$x$}
%2013dec27   \nput{  0}{y}{$x^\ocop$}
%2013dec27   \nput{-90}{0}{$\bzero$}
%2013dec27   }
%2013dec27 \end{minipage}%
%2013dec27 \hspace{5mm}%
%2013dec27 \begin{minipage}{\tw-70mm}%
%2013dec27 There are 2 unlabeled lattices on a 4 element set\ifsxref{lattice}{prop:num_lattices},
%2013dec27 as illustrated to the left and right.
%2013dec27 Of these two, the diamond lattice on the left is 
%2013dec27 \prope{Boolean} (and hence also \prope{modular}, \prope{orthomodular}, and \prope{orthocomplemented}).
%2013dec27 The linearly ordered lattice on the right is 
%2013dec27 \prope{non-orthocomplemented} (and hence also \prope{non-orthomodular} and \prope{non-Boolean}).
%2013dec27 \end{minipage}%
%2013dec27 \hspace{5mm}%
%2013dec27 \begin{minipage}{24mm}\centering%
%2013dec27 \latmatls{4}{0.75}{0.75}
%2013dec27   {[name=1]\null \\
%2013dec27    [name=y]\null \\
%2013dec27    [name=x]\null \\
%2013dec27    [name=0]\null
%2013dec27   }
%2013dec27   {\ncline{0}{x}
%2013dec27    \ncline{x}{y}
%2013dec27    \ncline{y}{1}
%2013dec27   }{
%2013dec27   \nput{ 90}{1}{$\bid$}
%2013dec27   \nput{  0}{y}{$y\ne x^\ocop$}
%2013dec27   \nput{  0}{x}{$x$}
%2013dec27   \nput{-90}{0}{$\bzero$}
%2013dec27   }
%2013dec27   \hspace{5mm}
%2013dec27 \end{minipage}
%2013dec27 }
%2013dec27 \end{example}

%2013dec27 %---------------------------------------
%2013dec27 \begin{example}[orthocomplemented lattices on 5 element sets]
%2013dec27 \label{ex:latoc_set5}
%2013dec27 %---------------------------------------
%2013dec27 There are 5 unlabeled lattices on a 5 element 
%2013dec27 set\ifdochas{lattice}{, as stated in \prefpp{prop:num_lattices} and illustrated in \prefpp{ex:lat_set5}}.
%2013dec27 All of these are \prope{non-orthocomplemented}.
%2013dec27 \end{example}


%2013dec27 %---------------------------------------
%2013dec27 \begin{example}[orthocomplemented lattices on 6 element sets]
%2013dec27 \footnote{
%2013dec27   \citerpg{kalmbach1983}{32}{0123945801}
%2013dec27   }
%2013dec27 \label{ex:latoc_set6}
%2013dec27 %---------------------------------------
%2013dec27 There are 15 unlabeled lattices on a 6 element 
%2013dec27 set\ifdochas{lattice}{, as stated in \prefpp{prop:num_lattices} and illustrated in \prefpp{ex:lat_set6}}.
%2013dec27 Of these 15, two are \prope{orthocomplemented}:
%2013dec27 The MO$_2$ lattice \xref{def:mo2} and the $O_6$ lattice \xref{def:o6}.
%2013dec27 \exbox{%
%2013dec27 \begin{tabular}{m{34mm}m{\tw-82mm}m{22mm}}%
%2013dec27   \centering\psset{unit=7.5mm}{%============================================================================
% Daniel J. Greenhoe
% LaTeX file
% lattice (2^{x,y,z}, subseteq)
%============================================================================
\begin{pspicture}(-1.9,-\latbot)(1.9,2.3)%
  %---------------------------------
  % settings
  %---------------------------------
  %\psset{%
  %  labelsep=1.5mm,%
  %  }%
  %---------------------------------
  % nodes
  %---------------------------------
  \Cnode(0,2){t}%
  \Cnode(-1.5,1){x}\Cnode(-0.5,1){y}%
  \Cnode(0.5,1){yc}\Cnode(1.5,1){xc}%
  \Cnode(0,0){b}%
  %---------------------------------
  % node connections
  %---------------------------------
  \ncline{t}{x}\ncline{t}{y}\ncline{t}{yc}\ncline{t}{xc}%
  \ncline{b}{x}\ncline{b}{y}\ncline{b}{yc}\ncline{b}{xc}%
  %---------------------------------
  % node labels
  %---------------------------------
  \uput[20](t) {$\bid$}%
  \uput[-90](xc) {$x^\ocop$}%     
  \uput[180](yc){$y^\ocop$}%   
  \uput[-90](x) {$x$}%     
  \uput[180](y){$y$}%   
  \uput[-10](b) {$\bzero$}%
\end{pspicture}%}%
%2013dec27   &%
%2013dec27   The MO$_2$ lattice is \prope{modular} 
%2013dec27   (and hence \prope{modular orthocomplemented} and \prope{orthomodular}),
%2013dec27   but \prope{non-Boolean}.\footnotemark\hspace{1ex}
%2013dec27   The $O_6$ lattice is \prope{non-orthomodular} 
%2013dec27   (and hence \prope{non-modular} and \prope{non-Boolean}).%
%2013dec27   &%
%2013dec27   %\centering\psset{unit=5mm}{%============================================================================
% Daniel J. Greenhoe
% LaTeX file
% lattice (2^{x,y,z}, subseteq)
%============================================================================
%{%\psset{unit=0.075mm}
\begin{pspicture}(-1.6,-0.5)(1.6,3.5)
  %---------------------------------
  % settings
  %---------------------------------
  \psset{%
    labelsep=1.5mm,
    }%
  %---------------------------------
  % nodes
  %---------------------------------
  \Cnode(0,3){t}%
  \Cnode(1,2){xc}\Cnode(1,1){yc}%
  \Cnode(-1,1){x}\Cnode(-1,2){y}%
  \Cnode(0,0){b}%
  %---------------------------------
  % node connections
  %---------------------------------
  \ncline{t}{y}\ncline{t}{xc}%
  \ncline{x}{y}\ncline{yc}{xc}%
  \ncline{b}{x}\ncline{b}{yc}%
  %---------------------------------
  % node labels
  %---------------------------------
  \uput[0](t){$\bid$}%
  \uput[0](yc){$y^\ocop$}\uput[0](xc){$x^\ocop$}%
  \uput[180](x){$x$}\uput[180](y){$y$}
  \uput[0](b){$\bzero$}
\end{pspicture}%
%    }
}%
%2013dec27   \centering\psset{unit=5mm}{%============================================================================
% Daniel J. Greenhoe
% LaTeX file
% nominal unit = 10mm
%============================================================================
\begin{pspicture}(-1.5,-\latbot)(1.5,3.5)
  %---------------------------------
  % nodes
  %---------------------------------
  \Cnode(0,3){t}%
  \Cnode(-1,2){c}\Cnode(1,2){d}%
  \Cnode(-1,1){x}\Cnode(1,1){y}%
  \Cnode(0,0){b}%
  %---------------------------------
  % node connections
  %---------------------------------
  \ncline{t}{c}\ncline{t}{d}%
  \ncline{c}{x}\ncline{d}{y}%
  \ncline{b}{x}\ncline{b}{y}%
  %---------------------------------
  % node labels
  %---------------------------------
  \uput[0](t) {$\lid$}%
  \uput[80](d) {$x^\ocop$}%
  \uput[90](c) {$y^\ocop$}%
  \uput[-90](y) {$y$}%
  \uput[-90](x) {$x$}%
  \uput[0](b) {$\lzero$}%
\end{pspicture}%}%
%2013dec27 \end{tabular}%
%2013dec27 }
%2013dec27 \end{example}


%%=======================================
%\section{Literature}
%%=======================================
%\begin{survey}
%\begin{enumerate}
%  \item Orthocomplemented lattices
%    \\\citer{birkhoffjvn1936}
%    \\\citer{loomis1955}
%    \\\citerpg{dunn2001}{86}{0198531923}
%    \\\citerpg{foulis2006}{469}{3527607250}
%
%  \item Orthomodular lattices:
%    \\\citer{husimi1937}
%    \\\citer{foulis1962}
%    \\\citer{maeda1966}
%    \\\citer{holland1970}
%    \\\citer{iturrioz1985}    %\\NTHU Math QA9.A1 L349 1985
%    \\\citer{kalmbach1983}
%    \\\citer{beran1985}
%
%  \item Complemented:
%    \\\citer{nakamura1957}
%
%  \item Orthoposet:
%    \\\citerppg{kalmbach1983}{16}{19}{0123945801}
%
%  \item Orthocomplemented lattices in quantum physics:
%    \\\citer{chiara2001}
%    \\\citer{chiara2002}
%
%  \item Varieties of orthomodular lattices:
%    \\\citer{mayet1985}
%    \\\citer{mayet1986}
%
%  \item References I don't yet have:
%    \begin{enumerate}
%       \item \fullcite{husimi1937}
%       \item \fullcite{sasaki1954}
%       \item \fullcite{holland1970}
%       \item \fullcite{carrega1982}
%       \item \fullcite{kalmbach1983} \\NTHU Math
%    \end{enumerate}
%  
%\end{enumerate}
%\end{survey}
