%============================================================================
% Daniel J. Greenhoe
% LaTeX file
% Lagrange arc metric examples in R^2
%         y
%         |   o p        Let (rp,tp) be the polar location of point p.
%         |  /           where rp is the Euclidean distance from (0,0) to p 
%         | /            and tp is radian measure from the x-axis to p.
%         |/tp           Let (rq,tq) be the polar location of point q.
% --------|---------- x  The "Lagrange arc" r(theta) is defined here as
%         |\tq                          theta -tq        theta -tp
%         | \            r(theta) = rp ----------- + rq -----------
%         |  o q                          tp-tq            tq-tp
%         |              
%============================================================================
\begin{pspicture}(-1.5,-0.6)(1.5,1.5)%
  %\psset{plotstyle=line, linecolor=blue, linewidth=1pt, dotsize=3pt}%
  \psset{dotsize=3pt}%
  \psaxes[linecolor=axis]{<->}(0,0)(-1.5,-0.6)(1.5,1.5)%
  \pnode(0.5,0){p}%
  \pnode(0,0.5){q}%
  \pnode(1,0.5){pp}%
  \pnode(0.5,1){qq}%
  %
  \psline[linestyle=dotted,linecolor=green](p)(pp)%
  \psline[linestyle=dotted,linecolor=green](q)(qq)%
  %
  \psplot[polarplot=true,linecolor=blue]{90}{0}{0.5 x 180 sub 0 180 sub div mul 0.5 x 0 sub 180 0 sub div mul add}%
  \psplot[polarplot=true,linecolor=red]{26.565}{63.435}{1.118 x 63.435 sub 26.565 63.435 sub div mul 1.118 x 26.565 sub 63.435 26.565 sub div mul add}%
  %                                      tp      tq     rp                                           rq
  \psdot[linecolor=blue](p)%
  \psdot[linecolor=blue](q)%
  \psdot[linecolor=red](pp)%
  \psdot[linecolor=red](qq)%
  %
  \uput[-90]{0}(p){$p$}%
  \uput[180]{0}(q){$q$}%
  \uput[-90]{0}(pp){$p+r$}%
  \uput[60]{0}(qq){$q+r$}%
\end{pspicture}%
