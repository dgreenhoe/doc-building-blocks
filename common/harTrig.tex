%============================================================================
% LaTeX File
% Daniel J. Greenhoe
%============================================================================

%======================================
\chapter{Trigonometric Functions}
%\label{app:trig}
%\index{trigonometry}
%======================================

%=======================================
\section{Definition Candidates}
%=======================================
\ifdochasnot{normalg}{
%---------------------------------------
\begin{definition}[Hermitian components]
\label{def:nalg_Re}
\label{def:nalg_Im}
\label{def:Re}
\label{def:Im}
\footnote{
  \citerpg{michel1993}{430}{048667598X},
  \citerpu{rickart1960}{179}{http://books.google.com/books?id=xYCPHgAACAAJ\&pg=PA179},
  \citorpg{gelfand1964}{242}{0821820222}
  }
%---------------------------------------
Let $\opair{\F}{\invo}$ be a $\invo$-algebra a (\structe{star algebra}).
\defbox{
  \begin{array}{MLcLC}
    The \hid{real part}      of $x$ is defined as & \hxs{\Re} x &\eqd& \frac{1}{2  }\Big( x+ x^\invo \Big) & \forall x\in\F \\
    The \hid{imaginary part} of $x$ is defined as & \hxs{\Im} x &\eqd& \frac{1}{2i }\Big( x- x^\invo \Big) & \forall x\in\F
  \end{array}
  }
\end{definition}
}

There are several ways of defining the sine and cosine functions, including the following:\footnote{
  %The terms \hie{sine} and \hie{cosine} are mistranslated versions of the 
  %the words \hie{jya} and \hie{kojya} originally appearing in the 
  %\href{http://www-history.mcs.st-andrews.ac.uk/BirthplaceMaps/Places/India.html}{Indian} 
  %manuscript \citer{aryabhata}.
  The term \hie{sine} originally came from the Hindu word \hie{jiva} and
  later adapted to the Arabic word \hie{jiba}. 
  Abrabic-Latin translator 
  \href{http://en.wikipedia.org/wiki/Robert_of_Chester}{Robert of Chester} 
  apparently confused this word  with the Arabic word \hie{jaib},
  which means ``bay" or ``inlet"---
  thus resulting in the Latin translation \hie{sinus}, which also means ``bay" or ``inlet".
  %All six trigonometric functions first appeared in the European manuscript \citer{rheticus}. 
  Reference: %\url{http://en.wikipedia.org/wiki/History_of_trigonometric_functions}
  \citerpg{boyer1991}{252}{0471543977}
  %\citep{boyer}{278}
  }






\begin{dingautolist}{"AC}
  \item {\bf Planar geometry:}
    Trigonometric functions are traditionally introduced as they have come to us 
    historically---that is, as related to the parameters of triangles.\footnote{ 
      \citerp{as}{78}
      %\url{http://www.convertit.com/Go/ConvertIt/Reference/AMS55.ASP?Page=78} 
      }\\
    \begin{minipage}{3\tw/16}
      \setlength{\unitlength}{\tw/300}
      \footnotesize
      \begin{picture}(200,200)(-100,-100)
        %\color{graphpaper}\graphpaper[10](-100,-100)(200,200)%
        \thicklines%
        \color{axis}%
          \put(-100 ,   0 ){\line(1,0){200} }%
          \put(   0 ,-100 ){\line(0,1){200} }%
          \qbezier[16](70.7, 0)(70.7,40)(70.7,70.7)%
          \qbezier[16]( 0,70.7)(40,70.7)(70.7,70.7)%
        \color{circle}%
          %============================================================================
% NCTU - Hsinchu, Taiwan
% LaTeX File
% Daniel Greenhoe
%
% Unit circle with radius 100
%============================================================================

\qbezier( 100,   0)( 100, 41.421356)(+70.710678,+70.710678) % 0   -->1pi/4
\qbezier(   0, 100)( 41.421356, 100)(+70.710678,+70.710678) % pi/4-->2pi/4
\qbezier(   0, 100)(-41.421356, 100)(-70.710678,+70.710678) %2pi/4-->3pi/4
\qbezier(-100,   0)(-100, 41.421356)(-70.710678,+70.710678) %3pi/4--> pi 
\qbezier(-100,   0)(-100,-41.421356)(-70.710678,-70.710678) % pi  -->5pi/4
\qbezier(   0,-100)(-41.421356,-100)(-70.710678,-70.710678) %5pi/4-->6pi/4
\qbezier(   0,-100)( 41.421356,-100)( 70.710678,-70.710678) %6pi/4-->7pi/4
\qbezier( 100,   0)( 100,-41.421356)( 70.710678,-70.710678) %7pi/4-->2pi


%
        \color{black}%
          \put(   0 ,   0 ){\vector(1,1){70.7} }%
          \put( 70.7,-10){\makebox(0,0)[t ]{$x$}}%
          \put(-10, 70.7){\makebox(0,0)[r ]{$y$}}%
          \put( 40, 40){\makebox(0,0)[br]{$r$}}%
          \put( 25,  5){\makebox(0,0)[bl]{$\fvx$}}%
          \qbezier(20,0)(20,10)(10,10)%
      \end{picture}
    \end{minipage}%
    \hfill%
    \begin{minipage}{12\tw/16}
      $\begin{array}{rcl}
        \cos\fvx &\eqd& \frac{x}{r}  \\
        \sin\fvx &\eqd& \frac{y}{r}  
      \end{array}$
    \end{minipage}

  \item {\bf Complex exponential:} The cosine and sine functions are 
    the real and imaginary parts of the complex exponential such that\footnote{\citer{euler1748v1}}
    \\\indentx$\ds
      \cos\fvx \eqd \Real{e^{i\fvx}}  \qquad\qquad  \sin\fvx \eqd \Imag\brp{e^{i\fvx}}
      %\cos\fvx \eqd \frac{1}{2}\brp{e^{i\fvx}+e^{-i\fvx}}  
      %\qquad\qquad  
      %\sin\fvx \eqd \frac{1}{2i}\brp{e^{i\fvx}-e^{-i\fvx}}
    $

  \item {\bf Polynomial:}
    Let $\ds\sum_{n=0}^\infty x_n \eqd \lim_{\xN\to\infty}\sum_{n=0}^\xN x_n$ in some topological space\ifsxref{series}{def:suminf}.
    The sine and cosine functions can be defined in terms of \fncte{Taylor expansion}s such 
    that\footnote{
      \citerp{rosenlicht}{157},
      \citerp{as}{74}
      %\url{http://www.convertit.com/Go/ConvertIt/Reference/AMS55.ASP?Page=74} \\
      }
      \[\begin{array}{rc>{\ds}l c>{\ds}l}
        \cos(x)
          &\eqd& \sum_{n=0}^\infty \frac{(-1)^n }{(2n)!} \; x^{2n}
          &=& 1 - \frac{x^2}{2} + \frac{x^4}{4!} - \frac{x^6}{6!} + \cdots
        \\
        \sin(x)
          &\eqd& \sum_{n=0}^\infty  \frac{(-1)^{n}}{(2n+1)!} \; x^{2n+1}
          &=& x - \frac{x^3}{3!} + \frac{x^5}{5!} - \frac{x^7}{7!} + \cdots
      \end{array}\]


  \item {\bf Product of factors:}
    Let $\ds\prod_{n=0}^\infty x_n \eqd \lim_{\xN\to\infty}\prod_{n=0}^\xN x_n$ in some topological space\ifsxref{series}{def:suminf}.
    The sine and cosine functions can be defined in terms of a product of factors such
    that\footnote{
      \citerp{as}{75}
      %\url{http://www.convertit.com/Go/ConvertIt/Reference/AMS55.ASP?Page=75} \\
      }
      \[
        \cos(x) \eqd \prod_{n=1}^\infty \brs{1 - \brp{\frac{x}{(2n-1)\frac{\pi}{2}}}^2 }
        \qquad\qquad
        \sin(x) \eqd x\;\prod_{n=1}^\infty \brs{1 - \brp{\frac{x}{n\pi}}^2 } 
      \]


  \item {\bf Partial fraction expansion:}
    The sine function can be defined in terms of a partial fraction expansion such
    that\footnote{
      \citerp{as}{75}
      %\url{http://www.convertit.com/Go/ConvertIt/Reference/AMS55.ASP?Page=75} \\
      }
      \[
        \sin(x) \eqd 
          \frac{1}{\ds
                   \frac{1}{x} +
                   2x\; \sum_{n=1}^\infty \frac{(-1)^n}{x^2-(n\pi)^2}
                  }
        \qquad\qquad
          \cos(x) \eqd
            \mcom{\brp{\frac{1}{x} + 2x\; \sum_{n=1}^\infty \frac{1}{x^2-(n\pi)^2}}}
                 {$\cot(x)$}
            \sin(x)
      \]


  \item {\bf Differential operator:} The sine and cosine functions can be defined
    as solutions to differential equations expressed in terms of the differential operator 
    $\opDif$ such that
      \[\begin{array}{rcl @{\qquad}c@{\qquad} lll}
        \cos(x) &\eqd& \ff(x)
          & \text{where} 
          & \mcom{\brn{\opDiff\ff}+ \ff=0}{differential equation}  
          & \mcom{\ff(0)=1           }{1st initial condition}
          & \mcom{\brs{\opDif\ff}(0)=0 }{2nd initial condition}
        \\
        \sin(x) &\eqd& \fg(x)
          & \text{where} 
          & \mcom{\brn{\opDiff\fg}+ \fg=0}{differential equation}  
          & \mcom{\fg(0)=0           }{1st initial condition}
          & \mcom{\brs{\opDif\fg}(0)=1 }{2nd initial condition}
      \end{array}\]

  \item {\bf Integral operator:}
    The sine and cosine functions can be defined
    as inverses of integrals of square roots of rational functions such 
    that\footnote{
        \citerp{as}{79}
        %\url{http://www.convertit.com/Go/ConvertIt/Reference/AMS55.ASP?Page=79} \\
         %G. A. Edgar, \url{http://groups.google.com/group/sci.math/msg/d40e5e5c56ab85af}
      }\\
      \[\begin{array}{rclcrc>{\ds}l}
        \cos(x) &\eqd& \ff^{-1}(x)
          & \text{where} 
          & \ff(x) &\eqd& 
            \mcom{\int_x^1 \sqrt{\frac{1}{1-y^2}} \dy}{$\arccos(x)$}
          \\
        \sin(x) &\eqd& \fg^{-1}(x)
          & \text{where} 
          & \fg(x) &\eqd& 
            \mcom{\int_0^x \sqrt{\frac{1}{1-y^2}} \dy}{$\arcsin(x)$}
      \end{array}\]
\end{dingautolist}


%And in planar geometry, this is arguably the most natural approach.
%But for purposes of analysis, there is arguably a more natural approach
For purposes of analysis, it can be argued that the more natural approach
for defining harmonic functions is in terms of the 
differentiation operator $\opDif$ \xref{def:cos}.
Support for such an approach includes the following:
\begin{liste}
  \item Both sine and cosine are very easily 
        represented analytically as polynomials with coefficients involving
        the operator $\opDif$ \xref{thm:cos_taylor}.
  \item All solutions of homogeneous second order differential equations
        are linear combinations of sine and cosine \xref{thm:D2f_cos_sin}.
  \item Sine and cosine themselves are related to each other in terms of 
        the differentiation operator \xref{thm:Dcos}.
  \item The complex exponential function is a solution of a second order
        homogeneous differential equation \xref{def:exp}.
  \item Sine and cosine are orthogonal with respect to an innerproduct
        generated by an integral operator---which is a kind of inverse 
        differential operator \xref{sec:har_plane}.
\end{liste}


%======================================
\section{Definitions}
%======================================
%--------------------------------------
\begin{definition}
\footnote{
  \citerpg{rosenlicht}{157}{0486650383},
  \citerppg{flanigan1983}{228}{229}{0486613887}
  }
\label{def:cos}
%--------------------------------------
Let $\spC$ be the \structe{space of all continuously differentiable real functions}
and $\opDif\in\clOcc$ the differentiation operator.
\defboxt{%
  The function $\ff\in\clFcc$ is the \fnctd{cosine} function $\hxs{\cos(x)}\eqd\ff(x)$ if
  \\\indentx$\begin{array}{FrclDD}
      1. & \brn{\opDiff\ff}+\ff &=&0  & (second order homogeneous differential equation) & and \\
      2. & \ff(0)               &=&1  & (first initial condition)                        & and \\
      3. & \brs{\opDif\ff}(0)   &=&0  & (second initial condition).                      & 
  \end{array}$}
\end{definition}

%--------------------------------------
\begin{definition}
\footnote{
  \citerpg{rosenlicht}{157}{0486650383},
  \citerppg{flanigan1983}{228}{229}{0486613887}
  }
\label{def:sin}
%--------------------------------------
Let $\spC$ and $\opDif\in\clOcc$ be defined as in definition of $\cos(x)$ \xref{def:cos}.
\defboxt{%
  The function $\ff\in\clFcc$ is the \fnctd{sine} function $\hxs{\sin(x)}\eqd\ff(x)$ if
  \\\indentx$\begin{array}{FrclDD}
      1. & \brn{\opDiff\ff}+\ff &=&0  & (second order homogeneous differential equation) & and \\
      2. & \ff(0)               &=&0  & (first initial condition)                        & and \\
      3. & \brs{\opDif\ff}(0)   &=&1  & (second initial condition).                      & 
  \end{array}$}
\end{definition}

%\defbox{\indxs{\cos}\indxs{\sin}%
%  \begin{array}{l@{\qquad}lll}
%    \mc{4}{l}{\text{The \hid{cosine} function \text{$\cos(x)$ is the function $\ff\in\spC$ that satisfies the following conditions:}}}\\
%      & \mcom{\brn{\opDiff\ff}+ \ff=0}{2nd order homogeneous differential equation}  
%      & \mcom{\ff(0)=1           }{1st initial condition}
%      & \mcom{\brs{\opDif\ff}(0)=0 }{2nd initial condition}
%    \\
%    \mc{4}{l}{\text{The \hid{sine} function \text{$\sin(x)$ is the function $\fg\in\spC$ that satisfies the following conditions:}}}\\
%      & \mcom{\brn{\opDiff\fg}+ \fg=0}{2nd order homogeneous differential equation}  
%      & \mcom{\fg(0)=0           }{1st initial condition}
%      & \mcom{\brs{\opDif\fg}(0)=1 }{2nd initial condition}
%  \end{array}}
%\end{definition}

%--------------------------------------
\begin{definition}
\footnote{
  \citerpg{rosenlicht}{158}{0486650383}
  }
\label{def:pi}
%--------------------------------------
\defboxt{
  Let $\pi$ (``pi") be defined as the element in $\R$ such that
  \\\indentx$\begin{array}{DrclD}
    (1). & \ds\cos\brp{\frac{\pi}{2}} &=& 0 & and \\
    (2). & \pi &>& 0 & and \\
    (3). & \mc{4}{M}{$\pi$ is the \textbf{smallest} of all elements in $\R$ that satisfies (1) and (2).}
  \end{array}$
  }
\end{definition}

%=======================================
\section{Basic properties}
%=======================================
%--------------------------------------
\begin{lemma}
\footnote{
  \citerp{rosenlicht}{156},
  \citor{liouville1839}
  }
\label{lem:D2f-f}
%--------------------------------------
Let $\spC$ be the \structe{space of all continuously differentiable real functions}
and $\opDif\in\clOcc$ the differentiation operator.
\lemboxt{
  $\ds\brb{\brn{\opDiff\ff} +  \ff = 0}$ 
  $\ds\qquad\iff$
  \\\indentx
  $\ds\left\{\begin{array}{rc>{\ds}l}
  \ff(x)
    &=& \mcom{\brs{ \ff}(0)\:\sum_{n=0}^\infty (-1)^n \frac{x^{2n}}{(2n)!}     }{even terms}
      + \mcom{\brs{\opDif\ff}(0)\:\sum_{n=0}^\infty (-1)^n \frac{x^{2n+1}}{(2n+1)!} }{odd terms}
    \\
    &=& \Bigg(\ff(0) + \brs{\opDif\ff}(0)x\Bigg) 
       -\brp{ \frac{\ff(0)}{2!}x^2 + \frac{\brs{\opDif\ff}(0)}{3!}x^3 }
       + \brp{\frac{\ff(0)}{4!}x^4 + \frac{\brs{\opDif\ff}(0)}{5!}x^5 }
       \cdots
  \end{array}\right\}$
  }
\end{lemma}
\begin{proof}
Let $\ff^\prime(x)\eqd\opDif\ff(x)$.
\begin{align*}
  \ff^{\prime\prime\prime}(x) &= -\brs{\opDif\ff}(x)  \\
  \ff^{(4)}(x) &= -\brs{\opDif\ff}(x) &= -\brs{\opDiff\ff}(x) &= \ff(x)
\end{align*}
\begin{align*}
  \intertext{1. Proof that 
    $\brs{\opDiff\ff}(x) + \ff(x) = 0 \implies
     \ff(x)=\sum_{n=0}^\infty (-1)^n \brs{\frac{\ff(0)}{(2n)!}x^{2n} + \frac{\brs{\opDif\ff}(0)}{(2n+1)!}x^{2n+1}}$:
    }
  \ff(x)
    &= \sum_{n=0}^\infty \frac{\ff^{(n)}(0)}{n!} x^n
       \qquad\text{by Taylor expansion\ifsxref{polynom}{thm:taylor}}
  \\&= \ff(0) + \brs{\opDif\ff}(0)x - \frac{\brs{\opDiff\ff}(0)}{2!}x^2 - \frac{\ff^3(0)}{3!}x^3 
    + \frac{\ff^4(0)}{4!}x^4 + \frac{\ff^5(0)}{5!}x^5 - \cdots
  \\&= \ff(0) + \brs{\opDif\ff}(0)x - \frac{\ff(0)}{2!}x^2 - \frac{\brs{\opDif\ff}(0)}{3!}x^3 
    + \frac{\ff(0)}{4!}x^4 + \frac{\brs{\opDif\ff}(0)}{5!}x^5 - \cdots
  \\&=\ff(x)=\sum_{n=0}^\infty (-1)^n \brs{\frac{\ff(0)}{(2n)!}x^{2n} + \frac{\brs{\opDif\ff}(0)}{(2n+1)!}x^{2n+1}}
  %
  \intertext{2. Proof that 
    $\brs{\opDiff\ff}(x) + \ff(x) = 0 \impliedby
     \ff(x)=\sum_{n=0}^\infty (-1)^n \brs{\frac{\ff(0)}{(2n)!}x^{2n} + \frac{\brs{\opDif\ff}(0)}{(2n+1)!}x^{2n+1}}$:
    }
  \brs{\opDiff\ff}(x)
    &= \opDif\opDif \brs{\ff(x)}
  \\&= \opDif\opDif {\sum_{n=0}^\infty (-1)^n \brs{\frac{\ff(0)}{(2n)!}x^{2n} + \frac{\brs{\opDif\ff}(0)}{(2n+1)!}x^{2n+1}} } 
    && \text{by right hypothesis}
  \\&= \sum_{n=1}^\infty (-1)^n \brs{\frac{(2n)(2n-1)\ff(0)}{(2n)!}x^{2n-2} + \frac{(2n+1)(2n)\brs{\opDif\ff}(0)}{(2n+1)!}x^{2n-1}}
  \\&= \sum_{n=1}^\infty (-1)^n \brs{\frac{\ff(0)}{(2n-2)!}x^{2n-2} + \frac{\brs{\opDif\ff}(0)}{(2n-1)!}x^{2n-1}}
  \\&= \sum_{n=0}^\infty (-1)^{n+1} \brs{\frac{\ff(0)}{(2n)!}x^{2n} + \frac{\brs{\opDif\ff}(0)}{(2n+1)!}x^{2n+1}}
  \\&= - \ff(x)
    && \text{by right hypothesis}
\end{align*}
\end{proof}


%--------------------------------------
\begin{theorem}[\thmd{Taylor series for cosine/sine}]
\footnote{
  \citerp{rosenlicht}{157}
  }
\label{thm:cos_taylor}
\label{thm:sin_taylor}
%--------------------------------------
\thmbox{\begin{array}{rc>{\ds}l c>{\ds}l@{\qquad}C}
  \cos(x)
    &=& \sum_{n=0}^\infty (-1)^n \frac{x^{2n}}{(2n)!} 
    &=& 1 - \frac{x^2}{2} + \frac{x^4}{4!} - \frac{x^6}{6!} + \cdots
    & \forall x\in\R
  \\
  \sin(x)
    &=& \sum_{n=0}^\infty  (-1)^{n}\frac{x^{2n+1}}{(2n+1)!} 
    &=& x - \frac{x^3}{3!} + \frac{x^5}{5!} - \frac{x^7}{7!} + \cdots
    & \forall x\in\R
\end{array}}
\end{theorem}
\begin{proof}
\begin{align*}
  \cos(x)
    &= \mcom{\ff(0)\:\sum_{n=0}^\infty (-1)^n \frac{x^{2n}}{(2n)!}     }{even terms}
     + \mcom{\brs{\opDif\ff}(0)\:\sum_{n=0}^\infty (-1)^n \frac{x^{2n+1}}{(2n+1)!} }{odd terms}
    && \text{by \prefp{lem:D2f-f}}
  \\&= 1\:\sum_{n=0}^\infty (-1)^n \frac{x^{2n}}{(2n)!}
     + 0\:\sum_{n=0}^\infty (-1)^n \frac{x^{2n+1}}{(2n+1)!}
    && \text{by $\cos$ initial conditions \xref{def:cos}}
  \\&= \sum_{n=0}^\infty (-1)^n \frac{x^{2n}}{(2n)!} 
  \\
  \sin(x)
    &= \mcom{\ff(0)\:\sum_{n=0}^\infty (-1)^n \frac{x^{2n}}{(2n)!}     }{even terms}
     + \mcom{\brs{\opDif\ff}(0)\:\sum_{n=0}^\infty (-1)^n \frac{x^{2n+1}}{(2n+1)!} }{odd terms}
    && \text{by \prefp{lem:D2f-f}}
  \\&= 0\:\sum_{n=0}^\infty (-1)^n \frac{x^{2n}}{(2n)!}
     + 1\:\sum_{n=0}^\infty (-1)^n \frac{x^{2n+1}}{(2n+1)!}
    && \text{by $\sin$ initial conditions \xref{def:sin}}
  \\&= \sum_{n=0}^\infty (-1)^n \frac{x^{2n+1}}{(2n+1)!}
\end{align*}
\end{proof}


%--------------------------------------
\begin{theorem}
\footnote{
  \citerp{rosenlicht}{157}
  }
\label{thm:cos0}
\label{thm:sin0}
\label{thm:sin-x}
\label{thm:cos-x}
%--------------------------------------
\thmbox{\begin{array}{rcc|rcrC}
  \cos(0) &=& 1  & \cos(-x) &=&  \cos(x) & \forall x\in\R\\
  \sin(0) &=& 0  & \sin(-x) &=& -\sin(x) & \forall x\in\R
\end{array}}
\end{theorem}
\begin{proof}
\begin{align*}
  \cos(0)
    &= \brlr{1 - \frac{x^2}{2} + \frac{x^4}{4!} - \frac{x^6}{6!} + \cdots}_{x=0}
    && \text{by \thme{Taylor series for cosine}}
    && \text{\xref{thm:cos_taylor}}
  \\&= 1
  \\
  \sin(0)
    &= \brlr{x - \frac{x^3}{3!} + \frac{x^5}{5!} - \frac{x^7}{7!} + \cdots}_{x=0}
    && \text{by \thme{Taylor series for sine}}
    && \text{\xref{thm:sin_taylor}}
  \\&= 0
  \\
  \cos(-x)
    &= 1 - \frac{(-x)^2}{2} + \frac{(-x)^4}{4!} - \frac{(-x)^6}{6!} + \cdots
    && \text{by \thme{Taylor series for cosine}}
    && \text{\xref{thm:cos_taylor}}
  \\&= 1 - \frac{x^2}{2} + \frac{x^4}{4!} - \frac{x^6}{6!} + \cdots
  \\&= \cos(x)
    && \text{by \thme{Taylor series for cosine}}
    && \text{\xref{thm:cos_taylor}}
  \\
  \sin(-x)
    &= (-x) - \frac{(-x)^3}{3!} + \frac{(-x)^5}{5!} - \frac{(-x)^7}{7!} + \cdots
    && \text{by \thme{Taylor series for sine}}
    && \text{\xref{thm:sin_taylor}}
  \\&= -\brs{x - \frac{x^3}{3!} + \frac{x^5}{5!} - \frac{x^7}{7!} + \cdots}
  \\&= \sin(x)
    && \text{by \thme{Taylor series for sine}}
    && \text{\xref{thm:sin_taylor}}
\end{align*}
\end{proof}


%--------------------------------------
\begin{lemma}
\footnote{
  \citerp{rosenlicht}{158}
  }
\label{lem:cos12}
\label{lem:sin02}
%--------------------------------------
\lembox{\begin{array}{rcl | lcl}
  \cos(1) &>& 0  & x\in\intoo{0}{2} &\implies& \sin(x) > 0\\
  \cos(2) &<& 0  & 
\end{array}}
\end{lemma}
\begin{proof}
\begin{align*}
  \cos(1)
    &= \brlr{1 - \frac{x^2}{2} + \frac{x^4}{4!} - \frac{x^6}{6!} + \cdots}_{x=1}
    && \text{by \thme{Taylor series for cosine}}
    && \text{\xref{thm:cos_taylor}}
  \\&= 1 - \frac{1}{2} + \frac{1}{4!} - \frac{1}{6!} + \cdots
  \\&> 0
  \\
  \cos(2)
    &= \brlr{1 - \frac{x^2}{2} + \frac{x^4}{4!} - \frac{x^6}{6!} + \cdots}_{x=2}
    && \text{by \thme{Taylor series for cosine}}
    && \text{\xref{thm:cos_taylor}}
  \\&= 1 - \frac{4}{2} + \frac{16}{24} - \frac{64}{720} + \cdots
  \\&< 0
\end{align*}
\begin{align*}
  x\in\intoo{0}{2}
    &\implies \text{each term in the sequence 
    $\seqn{\brp{x - \frac{x^3}{3!}},\, 
          \brp{\frac{x^5}{5!} - \frac{x^7}{7!}},\,
          \brp{\frac{x^9}{9!} - \frac{x^{11}}{11!}},\,
          \ldots
         }$
     is $>0$}
  \\&\implies \quad\sin(x)>0
\end{align*}
\end{proof}

%--------------------------------------
\begin{proposition}
%--------------------------------------
Let $\pi$ be defined as in \prefpp{def:pi}.
\propbox{\begin{array}{DM}
  (A). & The value $\pi$ \textbf{exists} in $\R$.\\
  (B). & $2<\pi<4$.
\end{array}}
\end{proposition}
\begin{proof}
\begin{align*}
  \cos(1)
    &> 0
    && \text{by \prefp{lem:cos12}}
  \\
  \cos(2)
    &< 0
    && \text{by \prefp{lem:cos12}}
  \\
    \implies& 1<\frac{\pi}{2}<2
  \\\implies& 2<\pi<4
\end{align*}
\end{proof}

%--------------------------------------
\begin{theorem}
\footnote{
  \citerp{rosenlicht}{157}.
  The general solution for the \prope{non-homogeneous} equation
  $\opDiff\ff(x) + \ff(x) = \fg(x)$ 
  with initial conditions $\ff(a)=1$ and $\ff^\prime(a)=\rho$ is
  \indentx$\ff(x) = \cos(x) + \rho\sin(x) + \int_a^x \fg(y) \sin(x-y) \dy$.
  This type of equation is called a \fncte{Volterra integral equation of the second type}.
  References:  
    \citerp{folland}{371},
    \citer{liouville1839}.
    %\citerp{carothers???}{3}
  Volterra equation references:
    \citerp{pedersen2000}{99},
    %\citerp{michel1993}{97},
    \citer{lalescu1908},
    \citer{lalescu1911}
  }
\label{thm:D2f_cos_sin}
%--------------------------------------
Let $\spC$ be the space of all continuously differentiable real functions
and $\opDif\in\clOcc$ the differentiation operator.
Let $\ff^\prime(0) \eqd \brs{\opDif\ff}(0)$.
\thmbox{
  \brb{\opDiff\ff + \ff=0}
  %     {\fncte{2nd order homogeneous differential equation}}
  \quad\iff\quad
  \brb{\ff(x) = \ff(0)\,\cos(x) + \ff^\prime(0)\,\sin(x)}
  \qquad\msizes\forall\ff\in\spC,\,\forall x\in\R
}
\end{theorem}
\begin{proof}
\begin{align*}
  \intertext{1. Proof that 
    $\brs{\opDiff\ff}(x)=-\ff(x) \implies \ff(x) = \ff(0)\cos(x) + \brs{\opDif\ff}(0)\sin(x)$
    :}
  \ff(x)
    &= \ff(0) \mcom{\sum_{n=0}^\infty (-1)^n \frac{x^{2n}}{(2n)!}}{$\cos(x)$}
     + \brs{\opDif\ff}(0) \mcom{\sum_{n=0}^\infty (-1)^n \frac{x^{2n+1}}{(2n+1)!}}{$\sin(x)$}
    && \text{by left hypothesis and \prefp{lem:D2f-f}}
  \\&= \mathrlap{\ff(0)\cos x + \brs{\opDif\ff}(0)\sin x
    \qquad \text{by definitions of $\cos$ and $\sin$ \xxref{def:cos}{def:sin}}}
  %
  \intertext{2. Proof that 
    $\brn{\opDiff\ff}=-\ff \impliedby \ff(x) = \ff(0)\cos(x) + \brs{\opDif\ff}(0)\sin(x)$
    :}
  \ff(x)
    &= \ff(0) \cos x + \brs{\opDif\ff}(0)\sin x 
    && \text{by right hypothesis}
  \\&= \ff(0) \mcom{\sum_{n=0}^\infty (-1)^n \frac{x^{2n}}{(2n)!}}{$\cos(x)$}
     + \brs{\opDif\ff}(0) \mcom{\sum_{n=0}^\infty (-1)^n \frac{x^{2n+1}}{(2n+1)!}}{$\sin(x)$}
  \\\implies
    &\quad \brn{\opDiff\ff} +\ff = 0 
    && \text{by \prefp{lem:D2f-f}}
\end{align*}
\end{proof}



%--------------------------------------
\begin{theorem}
\footnote{
  \citerp{rosenlicht}{157}
  }
\label{thm:Dcos}
\label{thm:Dsin}
\label{thm:ccss1}
%--------------------------------------
Let $\opDif\in\clOcc$ be the differentiation operator.
\thmbox{\begin{array}{lcrC | lcrC | lcrC}
  \opDif\cos(x) &=& -\sin(x) & \forall x\in\R &
  \opDif\sin(x) &=&  \cos(x) & \forall x\in\R &
  \cos^2(x)+\sin^2(x) &=& 1 & \forall x\in\R
\end{array}}
\end{theorem}
\begin{proof}
\begin{align*}
  \opDif\cos(x)
    &= \opDif \sum_{n=0}^\infty (-1)^n \frac{x^{2n}}{(2n)!} 
    && \text{by \thme{Taylor series}}
    && \text{\xref{thm:cos_taylor}}
  \\&= \sum_{n=1}^\infty (-1)^n \frac{2n x^{2n-1}}{(2n)!} 
     = \sum_{n=1}^\infty (-1)^n \frac{x^{2n-1}}{(2n-1)!} 
     = \sum_{n=0}^\infty (-1)^{n+1} \frac{x^{2n}}{(2n)!} 
  \\&= -\sin(x)
    && \text{by \thme{Taylor series}}
    && \text{\xref{thm:cos_taylor}}
  \\
  \opDif\sin(x)
    &= \opDif \sum_{n=0}^\infty  (-1)^{n}\frac{x^{2n+1}}{(2n+1)!}
    && \text{by \thme{Taylor series}}
    && \text{\xref{thm:sin_taylor}}
  \\&= \sum_{n=0}^\infty  (-1)^{n}\frac{(2n+1)x^{2n}}{(2n+1)!}
     = \sum_{n=0}^\infty  (-1)^{n}\frac{x^{2n}}{(2n)!}
  \\&= \cos(x)
    && \text{by \thme{Taylor series}}
    && \text{\xref{thm:sin_taylor}}
\end{align*}
\begin{align*}
  \opDif\brs{\cos^2(x)+\sin^2(x)} 
    &= -2\cos(x)\sin(x) + 2\sin(x)\cos(x)
  \\&= 0
  \\&\implies \text{$\cos^2(x)+\sin^2(x)$ is \prope{constant}}
  \\&\implies \cos^2(x)+\sin^2(x)
  \\&= \cos^2(0)+\sin^2(0)
  \\&= 1+0 = 1
    && \text{by \prefp{thm:cos0}}
\end{align*}
\end{proof}

%--------------------------------------
\begin{proposition}
\label{prop:sinpi2}
%--------------------------------------
\propbox{
  \sin\brp{\frac{\pi}{2}} = 1
  }
\end{proposition}
\begin{proof}
\begin{align*}
  \sin\brp{\sfrac{\pi}{2}}
    &= \pm\sqrt{\sin^2\brp{\sfrac{\pi}{2}} + 0}
  \\&= \pm\sqrt{\sin^2\brp{\sfrac{\pi}{2}} + \cos^2\brp{\sfrac{\pi}{2}}}
    && \text{by definition of $\pi$}&&\text{\xref{def:pi}}
  \\&= \pm\sqrt{1}
    && \text{by \prefp{thm:ccss1}}
  \\&= \pm1
  \\&= 1
    && \text{by \prefp{lem:sin02}}
\end{align*}
\end{proof}

%=======================================
\section{The complex exponential}
%=======================================
%%--------------------------------------
%\begin{lemma}
%\label{lem:exp_taylor}
%%--------------------------------------
%\lembox{
%   e^x
%     = \sum_{n=0}^\infty \frac{x^n}{n!}
%     = 1 + x + \frac{x^2}{2} + \frac{x^3}{3!} + \frac{x^4}{4!} + \frac{x^5}{5!} + \cdots
%%\\
%%   \tanx
%%     &=& x + 2\frac{x^3}{3!} + 16\frac{x^5}{5!} + 272\frac{x^7}{7!} + \cdots \\
%%\end{array}
%}
%\end{lemma}
%\begin{proof}
%\begin{eqnarray*}
%  e^x
%    &=& \left.\sum_{n=0}^\infty \frac{e^{(n)}(a)}{n!}(x-a)^n\right|_{a=0}
%  \\&=& \frac{e^0}{0!}x^0
%       +\frac{e^0}{1!}x^1
%       +\frac{e^0}{2!}x^2
%       +\frac{e^0}{3!}x^3
%       +\frac{e^0}{4!}x^4
%       +\frac{e^0}{5!}x^5
%       +\frac{e^0}{6!}x^6
%       +\frac{e^0}{7!}x^7
%       + \cdots
%  \\&=& 1
%       +x
%       +\frac{x^2}{2!}
%       +\frac{x^3}{3!}
%       +\frac{x^4}{4!}
%       +\frac{x^5}{5!}
%       +\frac{x^6}{6!}
%       +\frac{x^7}{7!}
%       + \cdots
%  \\&=& \sum_{n=0}^\infty \frac{x^n}{n!}
%\end{eqnarray*}
%\end{proof}

%--------------------------------------
\begin{definition}
\label{def:exp}
%--------------------------------------
\defboxt{%
  The function $\ff\in\clFcc$ is the \hid{exponential function} $\hxs{\exp(ix)}\eqd\ff(x)$ if
  \\\indentx$\begin{array}{FrclDD}
      1. & \brn{\opDiff\ff}+\ff &=&0  & (second order homogeneous differential equation) & and \\
      2. & \ff(0)               &=&1  & (first initial condition)                        & and \\
      3. & \brs{\opDif\ff}(0)   &=&i  & (second initial condition).                      & 
  \end{array}$}
\end{definition}
%\begin{proof}
%\begin{align*}
%  \opDiff \exp(ix)
%    &= \opDif\brs{i\exp(ix)}
%    &&= -\exp(ix)
%  \\
%  \left.\opI \exp(ix)\right|_{x=0}
%    &= \left. \exp(ix) \right|_{x=0}
%    &&= 1
%  \\
%  \left.\opDif \exp(ix)\right|_{x=0}
%    &= \left. i\exp(ix) \right|_{x=0}
%    &&= i
%\end{align*}
%\end{proof}


%--------------------------------------
\begin{theorem}[\thmd{Euler's identity}]
\footnote{
  \citor{euler1748v1},
  \citerp{bottazzini}{12}
  }
\label{thm:eid}
%--------------------------------------
\thmbox{
  e^{ix} = \cos(x) + i\sin(x) \qquad\scy \forall x\in\R
}
\end{theorem}
\begin{proof}
  \begin{align*}
    \exp(ix)
      &= \ff(0)\;\cos(x) + \brs{\opDif\ff}(0)\;\sin(x)
      && \text{by \prefp{thm:D2f_cos_sin}}
    \\&= \cos(x) + i\sin(x)
      && \text{by \prefp{def:exp}}
  \end{align*}
\end{proof}


%--------------------------------------
\begin{proposition}
\label{prop:eipi2}
%--------------------------------------
\propbox{\begin{array}{>{\ds}rcl | >{\ds}rcl}
  %e^{-i0}               &=&  1  & e^{i0}               &=&  1\\
  e^{-i\sfrac{\pi}{2}}  &=& -i  & e^{i\sfrac{\pi}{2}}  &=&  i\\
  %e^{-i\pi}             &=& -1  & e^{i\pi}             &=& -1\\
  %e^{-i\sfrac{3\pi}{2}} &=& -i  & e^{i\sfrac{3\pi}{2}} &=&  i\\
  %e^{-i2\pi}            &=&  1  & e^{i2\pi}            &=&  1
\end{array}}
\end{proposition}
\begin{proof}
\begin{align*}
  e^{i\sfrac{\pi}{2}} 
    &= \cos\brp{\sfrac{\pi}{2}} + i\sin\brp{\sfrac{\pi}{2}}
    && \text{by \thme{Euler's identity} \xref{thm:eid}}
  \\&= 0 + i
    && \text{by \prefpp{thm:cos0} and \prefpp{prop:sinpi2}}
  \\
  e^{-i\sfrac{\pi}{2}} 
    &= \cos\brp{-\sfrac{\pi}{2}} + i\sin\brp{-\sfrac{\pi}{2}}
    && \text{by \thme{Euler's identity} \xref{thm:eid}}
  \\&= \cos\brp{\sfrac{\pi}{2}} - i\sin\brp{\sfrac{\pi}{2}}
    && \text{by \prefp{thm:cos-x}}
  \\&= 0 - i
    && \text{by \prefpp{thm:cos0} and \prefpp{prop:sinpi2}}
\end{align*}
\end{proof}

%--------------------------------------
\begin{corollary}
\label{cor:eid}
%--------------------------------------
\corbox{
  e^{ix} = \sum_{n\in\Znn} \frac{(ix)^n}{n!} \qquad\scy\forall x\in\R
}
\end{corollary}
\begin{proof}
\begin{align*}
  \boxed{e^{ix}}
    &=  \cos(x) +i\sin(x)
    &&  \text{by \thme{Euler's identity}}&&\text{\xref{thm:eid}}
  \\&=  \mcom{\sum_{n\in\Znn} \frac{(-1)^nx^{2n}}{(2n)!}}{$\cos(x)$}
      +i\mcom{\sum_{n\in\Znn} \frac{(-1)^nx^{2n+1}}{(2n+1)!}}{$\sin(x)$}
    &&  \text{by \thme{Taylor series}}&&  \text{\xref{thm:cos_taylor}}
  \\&=  \sum_{n\in\Znn} \frac{(i^2)^nx^{2n}}{(2n)!}
       +\sum_{n\in\Znn} \frac{i(i^2)^nx^{2n+1}}{(2n+1)!}
    &&=\sum_{n\in\Znn} \frac{(ix)^{2n}}{(2n)!}
       +\sum_{n\in\Znn} \frac{(ix)^{2n+1}}{(2n+1)!}
      =  \sum_{n\in\Znn\seti\Ze} \frac{(ix)^n}{n!}
       +\sum_{n\in\Znn\seti\Zo} \frac{(ix)^n}{n!}
      = \boxed{\sum_{n\in\Znn} \frac{(ix)^n}{n!}}
\end{align*}
\end{proof}

%--------------------------------------
\begin{corollary}[\thmd{Euler formulas}]
\footnote{
  \citor{euler1748v1},
  \citerp{bottazzini}{12}
  }
\label{cor:trig_ceesee}
\label{cor:eform}
%--------------------------------------
\corbox{\begin{array}{rc>{\ds}lC | rc>{\ds}lC}
   \cos(x) &=& \Real\Big(e^{ix}\Big) = \frac{e^{ix} + e^{-ix}}{2 } & \forall x\in\R &
   \sin(x) &=& \Imag\Big(e^{ix}\Big) = \frac{e^{ix} - e^{-ix}}{2i} & \forall x\in\R
  \end{array}}
\end{corollary}
\begin{proof}
\begin{align*}
  \boxed{\Real\Big(e^{ix}\Big)}
    &\eqd \frac{e^{ix} + \brp{e^{ix}}^\invo}{2}
     = \frac{e^{ix} + e^{-ix}}{2}
    && \text{by definition of $\Re$}
    && \text{\ifdochas{normalg}{\xref{def:nalg_Re}}}
  \\&= \frac{\cos( x) + i\sin( x)}{2} + \frac{\cos(-x) + i\sin(-x)}{2}
    && \text{by \thme{Euler's identity}}
    && \text{\xref{thm:eid}}
  \\&= \frac{\cos( x) + i\sin( x)}{2} + \frac{\cos( x) - i\sin( x)}{2}
   &&= \frac{\cos( x)}{2} + \frac{\cos( x)}{2}
   &&= \boxed{\cos(x)}
  \\
  \boxed{\Imag\Big(e^{ix}\Big)}
    &\eqd \frac{e^{ix} - \brp{e^{ix}}^\invo}{2i}
     = \frac{e^{ix} - e^{-ix}}{2i}
    && \text{by definition of $\Im$}
    && \text{\ifdochas{normalg}{\xref{def:nalg_Re}}}
  \\&= \frac{\cos( x) + i\sin( x)}{2i} - \frac{\cos(-x) + i\sin(-x)}{2i}
    && \text{by \thme{Euler's identity}}
    && \text{\xref{thm:eid}}
  \\&= \frac{\cos( x) + i\sin( x)}{2i}
      -\frac{\cos( x) - i\sin( x)}{2i}
   &&= \frac{i\sin( x)}{2i}
      +\frac{i\sin( x)}{2i}
   &&= \boxed{\sin(x)}
\end{align*}
\end{proof}



%--------------------------------------
\begin{theorem}
\footnote{
  \citerpg{rudinr}{1}{0070542341}
  }
\label{thm:trig_eab}
%--------------------------------------
\thmbox{ 
  e^{(\alpha+\beta)} = e^{\alpha}\:e^{\beta} 
  \qquad\scy \forall \alpha,\beta\in\C
  }
\end{theorem}
\begin{proof}
\begin{align*}
  e^{\alpha}\:e^{\beta}
    &= \brp{\sum_{n\in\Znn} \frac{\alpha^n}{n!}} \: 
       \brp{\sum_{m\in\Znn} \frac{\beta^m}{m!} }
    && \text{by \prefp{cor:eid}}
  \\&= \sum_{n\in\Znn} \sum_{k=0}^n
       \frac{\alpha^k}{k!} \: 
       \frac{\beta^{n-k}}{(n-k)!} 
    && \ifdochas{series}{\text{by \prefp{cor:seq_mult}}}
  \\&= \sum_{n\in\Znn}  \sum_{k=0}^n
       \frac{n!}{n!} 
       \frac{\alpha^k}{k!} \: 
       \frac{\beta^{n-k}}{(n-k)!} 
  \\&= \sum_{n\in\Znn} \frac{1}{n!} \sum_{k=0}^n
       \frac{n!}{k!\,(n-k)!}\:
       \alpha^k \: \beta^{n-k}
  \\&= \sum_{n\in\Znn} \frac{1}{n!} \sum_{k=0}^n
       {n \choose k}
       \alpha^k \: \beta^{n-k}
  \\&= \sum_{n\in\Znn} \frac{\brp{\alpha+\beta}^n}{n!} 
    && \text{by the \thme{Binomial Theorem}}
    && \text{\ifsxref{polynom}{thm:binomial}}
  \\&= e^{\alpha+\beta}
    && \text{by \prefp{cor:eid}}
\end{align*}
\end{proof}


%=======================================
\section{Trigonometric Identities}
%=======================================
%--------------------------------------
\begin{theorem}[\thmd{shift identities}]
\label{thm:trig_shift}
%--------------------------------------
\thmbox{\begin{array}{rcr@{\qquad}C@{\qquad}|@{\qquad}rcr@{\qquad}C}
  \cos\brp{ x+\frac{\pi}{2}} &=& -\sin x & \forall  x\in\R & \sin\brp{ x+\frac{\pi}{2}} &=&  \cos x & \forall  x\in\R \\
  \cos\brp{ x-\frac{\pi}{2}} &=&  \sin x & \forall  x\in\R & \sin\brp{ x-\frac{\pi}{2}} &=& -\cos x & \forall  x\in\R 
\end{array}}
\end{theorem}
\begin{proof}
\begin{align*}
  \cos\brp{ x+\frac{\pi}{2}} 
    &= \frac{e^{i\brp{ x+\frac{\pi}{2}}} + e^{ -i\brp{ x+\frac{\pi}{2}} } 
            }{2}
    && \text{by \thme{Euler formulas}}
    && \text{\xref{cor:trig_ceesee}}
  \\&= \frac{e^{i x} e^{i\frac{\pi}{2}} + 
             e^{-i x} e^{-i\frac{\pi}{2}} 
            }{2}
    && \text{by $e^{\alpha\beta}=e^\alpha e^\beta$ result}&&\text{\xref{thm:trig_eab}}
  \\&= \frac{e^{i x}(i) + e^{-i x}(-i)}{2}
    && \text{by \prefp{prop:eipi2}}
  \\&= \frac{e^{i x} - e^{-i x}}{-2i}
  \\&= -\sin x 
    && \text{by \thme{Euler formulas}}
    && \text{\xref{cor:trig_ceesee}}
  \\
  \cos\brp{ x-\frac{\pi}{2}}
    &= \frac{e^{i\brp{ x-\frac{\pi}{2}}} + e^{ -i\brp{ x-\frac{\pi}{2}} } 
            }{2}
    && \text{by \thme{Euler formulas}}
    && \text{\xref{cor:trig_ceesee}}
  \\&= \frac{e^{i x} e^{-i\frac{\pi}{2}} + 
             e^{-i x} e^{+i\frac{\pi}{2}} 
            }{2}
    && \text{by $e^{\alpha\beta}=e^\alpha e^\beta$ result}&&\text{\xref{thm:trig_eab}}
  \\&= \frac{e^{i x}(-i) + e^{-i x}(i)}{2}
    && \text{by \prefp{prop:eipi2}}
  \\&= \frac{e^{i x} - e^{-i x}}{2i}
  \\&= \sin x 
    && \text{by \thme{Euler formulas}}
    && \text{\xref{cor:trig_ceesee}}
  \\
  \sin\brp{x+\frac{\pi}{2}}
    &= \cos\brp{\brs{x+\frac{\pi}{2}}-\frac{\pi}{2}}
    && \text{by previous result}
  \\&= \cos\brp{x}
  \\
  \sin\brp{x-\frac{\pi}{2}}
    &= -\cos\brp{\brs{x-\frac{\pi}{2}}+\frac{\pi}{2}}
    && \text{by previous result}
  \\&= -\cos\brp{x}
\end{align*}
\end{proof}

%--------------------------------------
\begin{theorem}[\thmd{product identities}]
\label{thm:trig_cs}
%--------------------------------------
\thmbox{\begin{array}{D lcc@{\;}lcl C}
  (A).& \cos x \cos y &=& &\sfrac{1}{2}\cos(x-y) &+& \sfrac{1}{2}\cos(x+y) & \forall x,y\in\R \\
  (B).& \cos x \sin y &=&-&\sfrac{1}{2}\sin(x-y) &+& \sfrac{1}{2}\sin(x+y) & \forall x,y\in\R \\
  (C).& \sin x \cos y &=& &\sfrac{1}{2}\sin(x-y) &+& \sfrac{1}{2}\sin(x+y) & \forall x,y\in\R \\
  (D).& \sin x \sin y &=& &\sfrac{1}{2}\cos(x-y) &-& \sfrac{1}{2}\cos(x+y) & \forall x,y\in\R
\end{array}}
\end{theorem}
\begin{proof}
\begin{enumerate}
  \item Proof for (A) using \thme{Euler formulas} \xref{cor:trig_ceesee}\\
        (algebraic method requiring \structe{complex number system} $\C$):
    \begin{align*}
      \cos x \cos y
           &= \left(\frac{e^{ix} + e^{-ix}}{2}\right)\left(\frac{e^{iy} + e^{-iy}}{2} \right)
           && \text{by \thme{Euler formulas}}
           && \text{\xref{cor:trig_ceesee}}
         \\&= \frac{e^{i(x+y)} + e^{i(x-y)} + e^{i(-x+y)} + e^{i(-x-y)} }{4}
         \\&= \frac{e^{i(x+y)} + e^{-i(x+y)}}{4} + \frac{e^{i(x-y)} + e^{-i(x-y)} }{4}
         \\&= \frac{2\cos(x+y)}{4} + \frac{2\cos(x-y)}{4}
           && \text{by \thme{Euler formulas}}
           && \text{\xref{cor:trig_ceesee}}
         \\&= \frac{1}{2}\cos(x+y) + \frac{1}{2}\cos(x-y)
    \end{align*}

  \item Proof for (A) using \fncte{Volterra integral equation} \xref{thm:D2f_cos_sin}\\
        (differential equation method requiring only \structe{real number system} $\R$):
    {\begin{align*}
      &\ff(x)\eqd\sfrac{1}{2}\cos(x-y)+\sfrac{1}{2}\cos(x+y) 
      \\&\quad\implies \opDif\ff(x)  = -\sfrac{1}{2}\sin(x-y)-\sfrac{1}{2}\sin(x+y)   && \text{by \prefp{thm:Dcos}}
      \\&\quad\implies \opDiff\ff(x) = -\sfrac{1}{2}\cos(x-y)-\sfrac{1}{2}\cos(x+y)   && \text{by \prefp{thm:Dcos}}
      \\&\quad\implies \opDiff\ff(x)+\ff(x) = 0                                       && \text{by \prope{additive inverse} property\hspace{23mm}\mbox{}}
      \\&\quad\implies\mathrlap{%
          \mcom{\sfrac{1}{2}\cos(x-y)+\sfrac{1}{2}\cos(x+y)}{$\ff(x)$}
        = \mcom{\brs{\sfrac{1}{2}\cos(0-y)+\sfrac{1}{2}\cos(0+y)}}{$\ff''(0)$}\cos(x)
        + \mcom{\brs{-\sfrac{1}{2}\sin(0-y)-\sfrac{1}{2}\sin(0+y)}}{$\ff'(0)$}\sin(x)
        }
      \\&\quad\implies \sfrac{1}{2}\cos(x-y)+\sfrac{1}{2}\cos(x+y) = \cos{y}\cos{x} + 0\sin(x)
      \\&\quad\implies \cos{x}\cos{y} = \sfrac{1}{2}\cos(x-y)+\sfrac{1}{2}\cos(x+y)
    \end{align*}}

  \item Proof for (B) using \thme{Euler formulas} \xref{cor:trig_ceesee}:
      %\begin{align*}
      %  \sin x \sin y
      %       &= \cos\brp{x-\frac{\pi}{2}}\cos\brp{y-\frac{\pi}{2}}
      %       && \text{by \thme{shift identities}}
      %       && \text{\xref{thm:trig_shift}}
      %     \\&= \frac{1}{2}\cos\brp{x-\frac{\pi}{2}+y-\frac{\pi}{2}} + \frac{1}{2}\cos\brp{x-\frac{\pi}{2}-y+\frac{\pi}{2}}
      %       && \text{by (A)}
      %     \\&= \frac{1}{2}\cos\brp{x+y-\pi} + \frac{1}{2}\cos\brp{x-y}
      %     \\&= \frac{1}{2}\cos(x-y) - \frac{1}{2}\cos(x+y)
      %       && \text{by \prefp{thm:trig_sign}}
      %\end{align*}}
    \begin{align*}
      \sin x \sin y
        &= \left(\frac{e^{ix} - e^{-ix}}{2i} \right)\left(\frac{e^{iy} - e^{-iy}}{2i} \right)
        && \text{by \prefp{cor:trig_ceesee}}
      \\&= \frac{e^{i(x+y)} - e^{i(x-y)} - e^{i(-x+y)} + e^{i(-x-y)} }{-4}
      \\&= \frac{e^{i(x+y)} + e^{-i(x+y)} - e^{i(x-y)} - e^{-i(x-y)} }{-4}
      \\&= \frac{e^{i(x+y)} + e^{-i(x+y)}}{-4} - \frac{e^{i(x-y)} + e^{-i(x-y)} }{-4}
      \\&= \frac{2\cos(x-y)}{4} - \frac{2\cos(x+y)}{4}
        && \text{by \prefp{cor:trig_ceesee}}
      \\&= \frac{1}{2}\cos(x-y) - \frac{1}{2}\cos(x+y)
    \end{align*}

  \item Proofs for (C) and (D) using (A) and (B):
    \begin{align*}
       \cos x \sin y
            &= \cos\brp{x}\cos\brp{y-\frac{\pi}{2}}
           && \text{by \thme{shift identities}}
           && \text{\xref{thm:trig_shift}}
          \\&= \frac{1}{2}\cos\brp{x+y-\frac{\pi}{2}} + \frac{1}{2}\cos\brp{x-y+\frac{\pi}{2}}
           && \text{by (A)}
          \\&= \frac{1}{2}\sin\brp{x+y} - \frac{1}{2}\sin\brp{x-y}
           && \text{by \thme{shift identities}}
           && \text{\xref{thm:trig_shift}}
         \\
%       \cos x \sin y
%            &= \left(\frac{e^{ix} + e^{-ix}}{2}\right)\left(   \frac{e^{iy} - e^{-iy}}{2i} \right)
%            && \text{by \prefp{cor:trig_ceesee}}
%          \\&= \frac{e^{i(x+y)} - e^{ i(x-y)} + e^{i(-x+y)} - e^{i(-x-y)} }{4i}
%          \\&= \frac{e^{i(x+y)} - e^{-i(x+y)} -\left[ e^{ i(x-y)} - e^{-i(x-y)} \right] }{4i}
%          \\&= \frac{2i\sin(x+y) -2i\sin(x-y) }{4i}
%            && \text{by \prefp{cor:trig_ceesee}}
%          \\&= \frac{1}{2}\sin(x+y) - \frac{1}{2}\sin(x-y)
%\    \ \\
       \sin x \cos y
            &= \cos y \sin x 
          \\&= \frac{1}{2}\sin\brp{y+x} - \frac{1}{2}\sin\brp{y-x}
            && \text{by (B)}
          \\&= \frac{1}{2}\sin\brp{x+y} + \frac{1}{2}\sin\brp{x-y}
            && \text{by \prefp{thm:sin-x}}
%   \sin x \cos y
%        &= \left(\frac{e^{ix} - e^{-ix}}{2} \right)\left(  \frac{e^{iy} + e^{-iy}}{2i} \right)
%        && \text{by \prefp{cor:trig_ceesee}}
%      \\&= \frac{e^{i(x+y)} + e^{ i(x-y)} - e^{i(-x+y)} - e^{i(-x-y)} }{4i}
%      \\&= \frac{e^{i(x+y)} - e^{-i(x+y)} +\left[ e^{ i(x-y)} - e^{-i(x-y)} \right] }{4i}
%      \\&= \frac{2i\sin(x+y) +2i\sin(x-y) }{4i}
%        && \text{by \prefp{cor:trig_ceesee}}
%      \\&= \frac{1}{2}\sin(x+y) + \frac{1}{2}\sin(x-y)
  \end{align*}
\end{enumerate}
\end{proof}

%--------------------------------------
\begin{proposition}
\label{prop:cospi}
\label{prop:sinpi}
\label{prop:cos2pi}
\label{prop:sin2pi}
\label{prop:expipi}
\label{prop:expi2pi}
%--------------------------------------
\propbox{\begin{array}{Drcr | Drcr | Drcr}
  (A).& \cos(\pi) &=& -1 & (C). & \cos(2\pi) &=&  1 & (E). & e^{i \pi} &=& -1\\
  (B).& \sin(\pi) &=&  0 & (D). & \sin(2\pi) &=&  0 & (F). & e^{i2\pi} &=&  0
\end{array}}
\end{proposition}
\begin{proof}
\begin{align*}
  \cos\brp{\pi} 
    &= -1 + 1 + \cos\brp{\pi} 
  \\&= -1 + 2\brs{
       \sfrac{1}{2}\cos\brp{\sfrac{\pi}{2}-\sfrac{\pi}{2}} 
     + \sfrac{1}{2}\cos\brp{\sfrac{\pi}{2}+\sfrac{\pi}{2}} }
    && \text{by $\cos(0)=1$ result}&&\text{\xref{thm:cos0}}
  \\&= -1 + 2\cos\brp{\sfrac{\pi}{2}} \cos\brp{\sfrac{\pi}{2}}
    && \text{by \thme{product identities}}&&\text{\xref{thm:trig_cs}}
  \\&= -1 + 2(0)(0)
    && \text{by definition of $\pi$}&&\text{\xref{def:pi}}
  \\&= -1
  \\
  \sin\brp{\pi} 
    &= 0 + \sin\brp{\pi} 
  \\&= 2\brs{
       -\sfrac{1}{2}\sin\brp{\sfrac{\pi}{2}-\sfrac{\pi}{2}} 
     + \sfrac{1}{2}\sin\brp{\sfrac{\pi}{2}+\sfrac{\pi}{2}} }
    && \text{by $\sin(0)=0$ result}&&\text{\xref{thm:sin0}}
  \\&= 2\cos\brp{\sfrac{\pi}{2}} \sin\brp{\sfrac{\pi}{2}}
    && \text{by \thme{product identities}}&&\text{\xref{thm:trig_cs}}
  \\&= 2(0)\sin\brp{\sfrac{\pi}{2}}
    && \text{by definition of $\pi$}&&\text{\xref{def:pi}}
  \\&= 0
  \\
  \cos\brp{2\pi}
    &= 1 + \cos(2\pi) -1
  \\&= 2\brs{\sfrac{1}{2}\cos\brp{\pi-\pi} + \sfrac{1}{2}\cos\brp{\pi+\pi}} -1
    && \text{by $\cos(0)=1$ result}&&\text{\xref{thm:cos0}}
  \\&= 2\cos(\pi) \cos(\pi) -1
    && \text{by \thme{product identities}}&&\text{\xref{thm:trig_cs}}
  \\&= 2(-1)(-1) - 1
    && \text{by (A)}
  \\&= 1
  \\
  \sin(2\pi)
    &= 0 + \sin(2\pi)
  \\&= 2\brs{\sfrac{1}{2}\sin(\pi-\pi)+\sfrac{1}{2}\sin(\pi+\pi)}
    && \text{by $\sin(0)=0$ result}&&\text{\xref{thm:sin0}}
  \\&= 2\sin(\pi)\cos(\pi)
    && \text{by \thme{product identities}}&&\text{\xref{thm:trig_cs}}
  \\&= 2(0)(-1)
    && \text{by (A) and (B)}
  \\&= 0
  \\
  e^{i\pi}
    &= \cos(\pi) + i\sin(\pi)
    &&  \text{by \thme{Euler's identity}}&&\text{\xref{thm:eid}}
  \\&= -1 + 0
    && \text{by (A) and (B)}
  \\&= -1
  \\
  e^{i2\pi}
    &= \cos(2\pi) + i\sin(2\pi)
    &&  \text{by \thme{Euler's identity}}&&\text{\xref{thm:eid}}
  \\&= 1 + 0
    && \text{by (C) and (D)}
  \\&= 1
\end{align*}
\end{proof}

%--------------------------------------
\begin{theorem}[\thmd{double angle formulas}]
\footnote{Expressions for $\cos(\alpha+\beta)$, $\sin(\alpha+\beta)$, and $\sin^2\fvx$
  appear in works as early as \citer{almagest}.
  Reference: \url{http://en.wikipedia.org/wiki/History_of_trigonometric_functions}
  }
\label{thm:trig_a+b}
\label{thm:trig_double}
%--------------------------------------
\thmbox{
\begin{array}{Drcl@{\qquad}C}
   (A).& \hxs{\cos}(x+y)    &=& \ds \cos x \cos y - \sin x \sin y             & \forall x,y\in\R \\
   (B).& \hxs{\sin}(x+y)    &=& \ds \sin x \cos y + \cos x \sin y             & \forall x,y\in\R \\
   (C).& \hxs{\tan}(x+y)    &=& \ds \frac{\tan x + \tan y}{1 - \tan x \tan y} & \forall x,y\in\R 
\end{array}
}
\end{theorem}
\begin{proof}
\begin{enumerate}
  \item Proof for (A) using \thme{product identities} \xref{thm:trig_cs}.
    \begin{align*}
      \cos\brp{x+y}
        &= \mcom{\frac{1}{2}\cos(x+y) + \frac{1}{2}\cos(x+y)}{$\cos(x+y)$}
         + \mcom{\frac{1}{2}\cos(x-y) - \frac{1}{2}\cos(x-y)}{0}
      \\&=  \brs{\frac{1}{2}\cos(x-y) + \frac{1}{2}\cos(x+y)}
           -\brs{\frac{1}{2}\cos(x-y) - \frac{1}{2}\cos(x+y)}
      \\&=  \cos x \cos y - \sin x \sin y
        &&  \text{by \prefp{thm:trig_cs}}
    \end{align*}

  \item Proof for (A) using \fncte{Volterra integral equation} \xref{thm:D2f_cos_sin}:
    \begin{align*}
      \ff(x)\eqd\cos(x+y) &\implies \opDif\ff(x)  = -\sin(x+y)   && \text{by \prefp{thm:Dcos}}
                        \\&\implies \opDiff\ff(x) = -\cos(x+y)   && \text{by \prefp{thm:Dcos}}
                        \\&\implies \opDiff\ff(x)+\ff(x) = 0     && \text{by \prope{additive inverse} property}
                        \\&\implies \cos(x+y) = \cos{y}\cos{x}-\sin{y}\sin{x} && \text{by \prefp{thm:D2f_cos_sin}}
                        \\&\implies \cos(x+y) = \cos{x}\cos{y}-\sin{x}\sin{y} && \text{by \prope{commutative} property}
    \end{align*}

  \item Proof for (B) and (C) using (A):
    \begin{align*}
     \sin\brp{x+y}
       &= \cos\brp{x-\frac{\pi}{2}+y}
          && \text{by \thme{shift identities} \xref{thm:trig_shift}}
     \\&=  \cos\brp{x-\frac{\pi}{2}}\cos\brp{y} - \sin\brp{x-\frac{\pi}{2}} \sin\brp{y}
       && \text{by (A)}
     \\&=  \sin\brp{x}\cos\brp{y} + \cos\brp{x}\sin\brp{y}
          && \text{by \thme{shift identities} \xref{thm:trig_shift}}
  \\ \\
  %   \sin(x+y)
  %     &= \brs{\frac{1}{2}\sin(x+y) + \frac{1}{2}\sin(x+y)} 
  %      + \brs{\frac{1}{2}\sin(x-y) - \frac{1}{2}\sin(x-y)}
  %   \\&=  \brs{\frac{1}{2}\sin(x-y) + \frac{1}{2}\sin(x+y) }
  %        +\brs{-\frac{1}{2}\sin(x-y) + \frac{1}{2}\sin(x+y)}
  %   \\&=  \sin x \cos y + \cos x \sin y
  %     &&  \text{by \prefp{thm:trig_cs}}
  %\\ \\
     \tan(x+y)
       &= \frac{\sin(x+y)}{\cos(x+y)}
     \\&= \frac{\sin x \cos y + \cos x \sin y}{\cos x \cos y - \sin x \sin y}
       && \text{by (A)}
     \\&= \brp{\frac{\sin x \cos y + \cos x \sin y}{\cos x \cos y - \sin x \sin y}}
          \brp{\frac{ \cos x \cos y } {\cos x \cos y}}
     \\&=\mathrlap{
          \frac{\frac{\sin x \cos y}{\cos x \cos y}   + \frac{\cos x \sin y}{\cos x \cos y}}
               {\frac{\cos x \cos y}{ \cos x \cos y } - \frac{\sin x \sin y}{ \cos x \cos y }}
        = \frac{\frac{\sin x}{\cos x} + \frac{\sin y}{\cos y}}
               {1 - \frac{\sin x}{\cos x} \frac{\sin y}{\cos y}}
        = \frac{ \tan x  +  \tan y}{1 - \tan x \tan y}
        }
  \end{align*}
\end{enumerate}

\end{proof}

%--------------------------------------
\begin{theorem}[\thmd{trigonometric periodicity}]
\label{thm:cosxpi}
\label{thm:sinxpi}
\label{thm:trig_periodic}
%--------------------------------------
\thmbox{\begin{array}{DrclCC | DrclCC}
   (A).& \ds \cos(x+\xM\pi)  &=& \ds (-1)^\xM\cos(x) & \forall x\in\R, & \xM\in\Z & (D).& \ds \cos(x+2\xM\pi)  &=& \ds \cos(x) & \forall x\in\R, & \xM\in\Z \\
   (B).& \ds \sin(x+\xM\pi)  &=& \ds (-1)^\xM\sin(x) & \forall x\in\R, & \xM\in\Z & (E).& \ds \sin(x+2\xM\pi)  &=& \ds \sin(x) & \forall x\in\R, & \xM\in\Z \\
   (C).& \ds e^{i(x+\xM\pi)} &=& \ds (-1)^\xM e^{ix} & \forall x\in\R, & \xM\in\Z & (F).& \ds e^{i(x+2\xM\pi)} &=& \ds  e^{ix} & \forall x\in\R, & \xM\in\Z   
\end{array}}
\end{theorem}
\begin{proof}
\begin{enumerate}
  \item Proof for (A):
    \begin{enumerate}
      \item $\xM=0$ case:
        $\begin{array}[t]{rclcl}
          \cos(x+0\pi) &=& \cos(x) &=& (-1)^0\cos(x)\\
        \end{array}$
        
      \item Proof for $\xM>0$ cases (by induction): \label{item:cosxMpiA_gt0}
        \begin{enumerate}
          \item Base case $\xM=1$: \label{item:cosxMpiA_base}
            \begin{align*}
              \cos(x+\pi)
                &= \cos x \cos \pi - \sin x \sin \pi
                && \text{by \thme{double angle formulas}}&&\text{\xref{thm:trig_double}}
              \\&= \cos x (-1) - \sin x (0)
                && \text{by $\cos\pi=-1$ result}&&\text{\xref{prop:cospi}}
              \\&= (-1)^1\cos x
            \end{align*}
        
          \item Inductive step\ldots Proof that $\xM$ case $\implies$ $\xM+1$ case:
            \begin{align*}
              \cos\brp{x+[\xM+1]\pi}
                &= \cos\brp{[x+\pi] + \xM\pi}
              \\&= (-1)^\xM \cos\brp{x+\pi}
                && \text{by induction hypothesis ($\xM$ case)}
              \\&= (-1)^\xM(-1)\cos\brp{x}
                && \text{by base case \xref{item:cosxMpiA_base}}
              \\&= (-1)^{\xM+1}\cos\brp{x}
              \\&\implies \text{$\xM+1$ case}
            \end{align*}
        \end{enumerate}

      \item Proof for $\xM<0$ cases: Let $\xN\eqd-\xM$ \ldots $\implies \xN>0$.
        \begin{align*}
          \cos\brp{x+\xM\pi}
            &\eqd \cos\brp{x-\xN\pi}
            && \text{by definition of $\xN$}
          \\&= \cos(x)\cos(-\xN\pi)-\sin(x)\sin(-\xN\pi)
            && \text{by \thme{double angle formulas}}&&\text{\xref{thm:trig_double}}
          \\&= \cos(x)\cos(\xN\pi)+\sin(x)\sin(\xN\pi)
            && \text{by \prefp{thm:cos-x}}
          \\&= \cos(x)\cos(0+\xN\pi)+\sin(x)\sin(0+\xN\pi)
          \\&= \cos(x)(-1)^\xN\cos(0)+\sin(x)(-1)^\xN\sin(0)
            && \text{by $\xM\ge0$ results}&&\text{\xref{item:cosxMpiA_gt0}}
          \\&= (-1)^\xN\cos(x)
            && \text{by $\scy\cos(0)=1$, $\scy\sin(0)=0$ results}&&\text{\xref{thm:sin0}}
          \\&\eqd (-1)^{-\xM}\cos(x)
            && \text{by definition of $\xN$}
          \\&= (-1)^{\xM}\cos(x)
        \end{align*}

      \item Proof using complex exponential:
        \begin{align*}
          \cos\brp{ x+\xM\pi} 
            &= \frac{e^{i\brp{ x+\xM\pi}} + e^{ -i\brp{ x+\xM\pi} }}{2}
            && \text{by \thme{Euler formulas}}
            && \text{\xref{cor:trig_ceesee}}
          \\&= e^{i\xM\pi} \brs{\frac{e^{ix} + e^{ -ix}}{2}}
            && \text{by $e^{\alpha\beta}=e^\alpha e^\beta$ result}&&\text{\xref{thm:trig_eab}}
          \\&= \brp{e^{i\pi}}^\xM \cos x 
            && \text{by \thme{Euler formulas}}
            && \text{\xref{cor:trig_ceesee}}
          \\&= (-1)^\xM \cos x 
            && \text{by $e^{i\pi}=-1$ result}&&\text{\xref{prop:expipi}}
        \end{align*}
    \end{enumerate}

  \item Proof for (B):
    \begin{enumerate}
      \item $\xM=0$ case:
        $\begin{array}[t]{rclcl}
          \sin(x+0\pi) &=& \sin(x) &=& (-1)^0\sin(x)
        \end{array}$
        
      \item Proof for $\xM>0$ cases (by induction): \label{item:cosxMpiB_gt0}
        \begin{enumerate}
          \item Base case $\xM=1$: \label{item:cosxMpiB_base}
            \begin{align*}
              \sin(x+\pi) 
                &= \sin x \cos \pi + \cos x \sin \pi
                && \text{by \thme{double angle formulas}}&&\text{\xref{thm:trig_double}}
              \\&= \sin x (-1) - \cos x (0)
                && \text{by $\sin\pi=0$ results}&&\text{\xref{prop:cospi}}
              \\&= (-1)^1\sin x
            \end{align*}
        
          \item Inductive step\ldots Proof that $\xM$ case $\implies$ $\xM+1$ case:
            \begin{align*}
              \sin\brp{x+[\xM+1]\pi}
                &= \sin\brp{[x+\pi] + \xM\pi}
              \\&= (-1)^\xM \sin\brp{x+\pi}
                && \text{by induction hypothesis ($\xM$ case)}
              \\&= (-1)^\xM(-1)\sin\brp{x}
                && \text{by base case \xref{item:cosxMpiB_base}}
              \\&= (-1)^{\xM+1}\sin\brp{x}
              \\&\implies \text{$\xM+1$ case}
            \end{align*}
        \end{enumerate}

      \item Proof for $\xM<0$ cases: Let $\xN\eqd-\xM$ \ldots $\implies \xN>0$.
        \begin{align*}
          \sin\brp{x+\xM\pi}
            &\eqd \sin\brp{x-\xN\pi}
            && \text{by definition of $\xN$}
          \\&= \sin(x)\sin(-\xN\pi)-\sin(x)\sin(-\xN\pi)
            && \text{by \thme{double angle formulas}}&&\text{\xref{thm:trig_double}}
          \\&= \sin(x)\sin(\xN\pi)+\sin(x)\sin(\xN\pi)
            && \text{by \prefp{thm:cos-x}}
          \\&= \sin(x)\sin(0+\xN\pi)+\sin(x)\sin(0+\xN\pi)
          \\&= \sin(x)(-1)^\xN\sin(0)+\sin(x)(-1)^\xN\sin(0)
            && \text{by $\xM\ge0$ results}&&\text{\xref{item:cosxMpiB_gt0}}
          \\&= (-1)^\xN\sin(x)
            && \text{by $\scy\sin(0)=1$, $\scy\sin(0)=0$ results}&&\text{\xref{thm:sin0}}
          \\&\eqd (-1)^{-\xM}\sin(x)
            && \text{by definition of $\xN$}
          \\&= (-1)^{\xM}\sin(x)
        \end{align*}

      \item Proof using complex exponential:
        \begin{align*}
          \sin\brp{ x+\xM\pi} 
            &= \frac{e^{i\brp{ x+\xM\pi}} - e^{ -i\brp{ x+\xM\pi} }}{2i}
            && \text{by \thme{Euler formulas}}
            && \text{\xref{cor:trig_ceesee}}
          \\&= e^{i\xM\pi} \brs{\frac{e^{ix} - e^{ -ix}}{2i}}
            && \text{by $e^{\alpha\beta}=e^\alpha e^\beta$ result}&&\text{\xref{thm:trig_eab}}
          \\&= \brp{e^{i\pi}}^\xM \sin x 
            && \text{by \thme{Euler formulas}}&&\text{\xref{cor:trig_ceesee}}
          \\&= (-1)^\xM \sin x 
            && \text{by $e^{i\pi}=-1$ result}&&\text{\xref{prop:expipi}}
        \end{align*}
    \end{enumerate}

  \item Proof for (C):
    \begin{align*}
      e^{i\brp{ x+\xM\pi}}
        &= e^{i\xM\pi}e^{ix}
        && \text{by $e^{\alpha\beta}=e^\alpha e^\beta$ result}&&\text{\xref{thm:trig_eab}}
      \\&= \brp{e^{i\pi}}^\xM \brp{e^{ix}}
      \\&= (-1)^\xM e^{ix}
        && \text{by $e^{i\pi}=-1$ result}&&\text{\xref{prop:expipi}}
    \end{align*}

  \item Proofs for (D), (E), and (F):
    $\ds\begin{array}[t]{lclclM}
      \cos\brp{i\brp{ x+2\xM\pi}} &=& (-1)^{2\xM}\cos\brp{ix} &=& \cos\brp{ix} & by (A)\\ 
      \sin\brp{i\brp{ x+2\xM\pi}} &=& (-1)^{2\xM}\sin\brp{ix} &=& \sin\brp{ix} & by (B)\\ 
      e^{i\brp{ x+2\xM\pi}}       &=& (-1)^{2\xM}e^{ix}       &=& e^{ix}       & by (C) 
    \end{array}$
\end{enumerate}
\end{proof}

%--------------------------------------
\begin{theorem}[\thmd{half-angle formulas}/\thmd{squared identities}]
\label{thm:trig_half}
\label{thm:trig_sq}
%--------------------------------------
\thmbox{\begin{array}{DrclC | DrclC}
   (A).& \cos^2x         &=& \sfrac{1}{2} \brp{1 + \cos2x} & \forall x\in\R &
   (C).& \cos^2x+\sin^2x &=& 1                             & \forall x\in\R \\
   (B).& \sin^2x         &=& \sfrac{1}{2} \brp{1 - \cos2x} & \forall x\in\R &
\end{array}}
\end{theorem}
\begin{proof}
\begin{align*}
   \cos^2x
     &\eqd (\cos x)(\cos x)
      = \sfrac{1}{2}\cos(x-x) + \sfrac{1}{2}\cos(x+x)
     && \text{by \thme{product identities}}
     && \text{\xref{thm:trig_cs}}
   \\&= \sfrac{1}{2}\brs{1 + \cos(2x)}
     && \text{by $\cos(0)=1$ result}&&\text{\xref{thm:cos0}}
   \\
   \sin^2x
     &= (\sin x)(\sin x)
      = \sfrac{1}{2}\cos(x-x) - \sfrac{1}{2}\cos(x+x)
     && \text{by \thme{product identities}}
     && \text{\xref{thm:trig_cs}}
   \\&= \sfrac{1}{2}\brs{1 - \cos(2x)}
     && \text{by $\cos(0)=1$ result}&&\text{\xref{thm:cos0}}
   \\
   \cos^2x + \sin^2x
     &= \sfrac{1}{2}\brs{1 + \cos(2x)} + \sfrac{1}{2}\brs{1 - \cos(2x)}
      = 1
     && \text{by (A) and (B)}
   \\&&&\text{note: see also}&&\text{\prefp{thm:ccss1}}
\end{align*}
\end{proof}


%=======================================
\section{Planar Geometry}
\label{sec:har_plane}
%=======================================
The harmonic functions $\cos(x)$ and $\sin(x)$ are \hie{orthogonal} to each other
in the sense
\begin{align*}
  \inprod{\cos(x)}{\sin(x)}
    &= \int_{-\pi}^{+\pi} \cos(x) \sin(x) \dx
  \\&= \frac{1}{2}\int_{-\pi}^{+\pi} \sin(x-x) \dx
     + \frac{1}{2}\int_{-\pi}^{+\pi} \sin(x+x) \dx
    && \text{by \prefp{thm:trig_cs}}
  \\&= \frac{1}{2}\cancelto{0}{\int_{-\pi}^{+\pi} \sin(0) \dx}
      + \frac{1}{2}\int_{-\pi}^{+\pi} \sin(2x) \dx
  \\&= \left.-\frac{1}{2}\cdot\frac{1}{2}\cos(2x)\right|_{-\pi}^{+\pi} \cos(2x)
  \\&= -\frac{1}{4}\brs{ \cos(2\pi) - \cos(-2\pi)}
  \\&= 0
\end{align*}

Because $\cos(x)$ are $\sin(x)$ are orthogonal, they can be conveniently 
represented by the $x$ and $y$ axes in a plane---
because perpendicular axes in a plane are also orthogonal.
Vectors in the plane can be represented by linear combinations of 
$\cos x$ and $\sin x$.
Let $\tan x$ be defined as
  \[ \tan x \eqd \frac{\sin x}{\cos x}.\]
We can also define a value $\theta$ to represent the angle between such a vector 
and the $x$-axis such that
  \[ \theta = \tan^{-1}\brp{\frac{\sin\theta}{\cos\theta}} \] % = \tan^{-1}\brp{\frac{y}{x}}. \]

%--------------------------------------
%\begin{definition}[Basic trigonometric functions]
%\label{def:trig}
%--------------------------------------
%Let $\cos:\R\to[-1,\,+1]$, 
%    $\sin:\R\to[-1,\,+1]$, 
%    $\tan:\R\to\R$, 
%    $\sec:\R\to\R$, 
%    $\csc:\R\to\R$, and 
%    $\cot:\R\to\R$
%be defined as follows and with respect to the figure below and to the left:\\
\begin{minipage}{3\tw/16}
  \setlength{\unitlength}{\tw/200}
  \begin{picture}(200,200)(-100,-100)
    %\color{graphpaper}\graphpaper[10](-100,-100)(200,200)%
    \thicklines%
    \color{axis}%
      \put(-100 ,   0 ){\line(1,0){200} }%
      \put(   0 ,-100 ){\line(0,1){200} }%
      \qbezier[16](70.7, 0)(70.7,40)(70.7,70.7)%
      \qbezier[16]( 0,70.7)(40,70.7)(70.7,70.7)%
    \color{circle}%
      %============================================================================
% NCTU - Hsinchu, Taiwan
% LaTeX File
% Daniel Greenhoe
%
% Unit circle with radius 100
%============================================================================

\qbezier( 100,   0)( 100, 41.421356)(+70.710678,+70.710678) % 0   -->1pi/4
\qbezier(   0, 100)( 41.421356, 100)(+70.710678,+70.710678) % pi/4-->2pi/4
\qbezier(   0, 100)(-41.421356, 100)(-70.710678,+70.710678) %2pi/4-->3pi/4
\qbezier(-100,   0)(-100, 41.421356)(-70.710678,+70.710678) %3pi/4--> pi 
\qbezier(-100,   0)(-100,-41.421356)(-70.710678,-70.710678) % pi  -->5pi/4
\qbezier(   0,-100)(-41.421356,-100)(-70.710678,-70.710678) %5pi/4-->6pi/4
\qbezier(   0,-100)( 41.421356,-100)( 70.710678,-70.710678) %6pi/4-->7pi/4
\qbezier( 100,   0)( 100,-41.421356)( 70.710678,-70.710678) %7pi/4-->2pi


%
    \color{black}%
      \put(   0 ,   0 ){\vector(1,1){70.7} }%
      \put( 70.7,-10){\makebox(0,0)[t ]{$x$}}%
      \put(-10, 70.7){\makebox(0,0)[r ]{$y$}}%
      \put( 40, 40){\makebox(0,0)[br]{$r$}}%
      \put( 25,  5){\makebox(0,0)[bl]{$\theta$}}%
      \qbezier(20,0)(20,10)(10,10)%
  \end{picture}
\end{minipage}%
\hfill%
\begin{minipage}{12\tw/16}
  \[\begin{array}{rcl@{\qquad}rcl}
    \cos\theta &\eqd& \frac{x}{r} & \sec\theta &\eqd& \frac{r}{x} \\
    \sin\theta &\eqd& \frac{y}{r} & \csc\theta &\eqd& \frac{r}{y} \\
    \tan\theta &\eqd& \frac{y}{x} & \cot\theta &\eqd& \frac{x}{y}
  \end{array}\]
\end{minipage}


%%=======================================
%\section{Small angle formulas}
%\index{trigonometry!small angle formulas}
%\label{sec:trig_small}
%%=======================================
%\begin{theorem}
%For ``small" $\fvx$,
%\thmbox{\begin{array}{rcl}
%   \cos\fvx &\eqa& 1  \\
%   \sin\fvx &\eqa& \fvx  \\
%   \tan\fvx &\eqa& \fvx \\
%   \tan(x+\fvx) &\eqa& \tan x + \fvx
%\end{array}}
%\end{theorem}
%\begin{proof}
%\begin{align*}
%   \cos\fvx &= 1 - \frac{\fvx^2}{2} + \frac{\fvx^4}{4!} - \frac{\fvx^6}{6!} + \cdots
%              &\eqa& 1
%   \\
%   \sin\fvx &= \fvx - \frac{\fvx^3}{3!} + \frac{\fvx^5}{5!} - \frac{\fvx^7}{7!} + \cdots
%              &\eqa& \fvx
%   \\
%   \tan\fvx &= \fvx + 2\frac{\fvx^3}{3!} + 16\frac{\fvx^5}{5!} + 272\frac{\fvx^7}{7!} + \cdots
%              &\eqa& \fvx
%   \\
%   \tan(x+\fvx)
%     &=    \frac{\tan x + \tan\fvx}{1 - \tan x \tan\fvx}
%   \\&\eqa \frac{\tan x + \fvx}{1 - \fvx\tan x }
%   \\&\eqa \frac{\tan x + \fvx}{1 - 0 }
%   \\&=    \tan x + \fvx
%\end{align*}
%\end{proof}
%


%=======================================
\section{The power of the exponential}
%=======================================
\qboxnps
  { \href{http://www-history.mcs.st-andrews.ac.uk/Biographies/Peirce_Benjamin.html}{Benjamin Peirce} 
    \href{http://www-history.mcs.st-andrews.ac.uk/Timelines/TimelineF.html}{(1809--1880)}, 
    \href{http://www-history.mcs.st-andrews.ac.uk/BirthplaceMaps/Places/USA.html}{American Harvard University mathematician}
    after proving $e^{i\pi}=-1$ in a lecture.
    \index{Peirce, Benjamin}
    \index{quotes!Peirce, Benjamin}
    \footnotemark
  }
  {../common/people/peirce.jpg}
  {Gentlemen, that is surely true, it is absolutely paradoxical; 
   we cannot understand it, and we don't know what it means. 
   But we have proved it, and therefore we know it must be the truth.}
  \citetblt{
    quote: & \citerp{kasner1940}{104} \\
    %      & \url{http://www-gap.dcs.st-and.ac.uk/~history/Quotations/Peirce_Benjamin.html}\\
    image: & \scs\url{ http://www-history.mcs.st-andrews.ac.uk/history/PictDisplay/Peirce_Benjamin.html} 
    }
  
\qboxnps
  { \href{http://en.wikipedia.org/wiki/John_von_Neumann}{John von Neumann} 
    \href{http://www-history.mcs.st-andrews.ac.uk/Timelines/TimelineG.html}{(1903--1957)}, 
    \href{http://www-history.mcs.st-andrews.ac.uk/BirthplaceMaps/Places/Russia.html}{Hungarian-American mathematician},
    as allegedly told to Gary Zukav by Felix T. Smith, Head of Molecular Physics at Stanford Research Institute,
    about a ``physicist friend".
    %in response to the statement ``I'm afraid I don't understand the method of characteristics."
    \index{von Neumann, John}
    \index{quotes!von Neumann, John}
    \footnotemark
  }
  {../common/people/neumann.jpg}
  {Young man, in mathematics you don't understand things. You just get used to them.}
  \footnotetext{
    \begin{tabular}{ll}
      quote: & \citerp{zukav1980}{208} \\
            %& \url{http://en.wikiquote.org/wiki/John_von_Neumann} \\
      image: & \scs\url{http://en.wikipedia.org/wiki/John_von_Neumann} 
    \end{tabular}\\
    The quote appears in a footnote in Zukav (1980) that reads like this: \scs
    Dr. Felix Smith, Head of Molecular Physics, Stanford Research Institute, 
    once related to me the true story of a physicist friend who worked at Los Alamos after World War II. 
    Seeking help on a difficult problem, 
    he went to the great Hungarian mathematician, John von Neumann, who was at Los Alamos as a consultant.
    ``Simple," said von Neumann. ``This can be solved by using the method of characteristics."
    After the explanation the physicist said, ``I'm afraid I don't understand the method of characteristics."
    ``Young man," said von Neumann, ``in mathematics you don't understand things, you just get used to them."
    }
  

The following corollary presents one of the most amazing relations in all of mathematics.
It shows a simple and compact relationship between the transcendental numbers
$\pi$ and $e$, the imaginary number $i$, and the additive and multiplicative identity
elements $0$ and $1$. 
The fact that there is any relationship at all is somewhat amazing;
but for there to be such an elegant one is truly one of the wonders of the 
world of numbers.
%--------------------------------------
\begin{corollary}
\footnote{
  \citor{euler1748v1},
  \citerc{euler1748e}{chapter 8?},
  \url{http://www.daviddarling.info/encyclopedia/E/Eulers_formula.html}
  }
%--------------------------------------
\corbox{ e^{i\pi} + 1 = 0 }
\end{corollary}
\begin{proof}
\begin{align*}
  \left.e^{i\fvx}\right|_{\fvx=\pi}
    &= \brs{\cos\fvx + i\sin\fvx}_{\fvx=\pi}
    && \text{by Euler's identity \xref{thm:eid}}
  \\&= -1 + i\cdot0
    && \text{by \prefp{prop:sinpi}}
  \\&= -1 
\end{align*}
\end{proof}

There are many transforms available, several of them integral transforms
$\brs{\opA\ff}(s)\eqd\int_t \ff(s) \kappa(t,s) \dds$
using different kernels $\kappa(t,s)$. 
But of all of them, two of the most often used themselves use an exponential kernel:
  \\\qquad
  \begin{tabular}{lll}
    \circOne & The \hie{Laplace Transform} & with kernel $\lkern{t}{s}\eqd \lkerne{t}{s}$  \\
    \circTwo & The \hie{Fourier Transform} & with kernel $\fkern{t}{\omega}\eqd \fkerne{t}{\omega}$.
  \end{tabular}\\
Of course, the Fourier kernel is just a special case of the Laplace kernel with $s=i\omega$
($i\omega$ is a unit circle in $s$ if $s$ is depicted as a plane with real and imaginary axes).
What is so special about exponential kernels?
Is it just that they were discovered sooner than other kernels with other transforms? 
The answer in general is ``no". 
The exponential has two properties that makes it extremely special:
  \begin{liste}
    \item The exponential is an eigenvalue of any \prope{linear time invariant} (LTI) operator \xref{thm:Le=he}.
    \item The exponential generates a \fncte{continuous point spectrum} for the \ope{differential operator}. 
  \end{liste}

%--------------------------------------
\begin{theorem}
\footnote{
  \citerp{mallat}{2},
  \ldots page 2 online: \url{http://www.cmap.polytechnique.fr/~mallat/WTintro.pdf}
  }
\label{thm:Le=he}
%--------------------------------------
Let $\opL$ be an operator with kernel $\fh(t,\omega)$ and 
%\[ \Fh(\omega)\eqd \inprod{\fh(t,\omega)}{\fkerne{t}{\omega}}
%   \qquad\scriptstyle
%   \text{(\hie{Fourier transform}).}
%\]
\\\indentx$\Lh(s)\eqd \inprod{\fh(t,\omega)}{\lkerne{t}{s}}
   \qquad\scriptstyle
   \text{(\ope{Laplace transform}).}
$\\
\thmbox{
  \brb{\begin{array}{Fl}
    1. & \text{$\opL$ is \prope{linear} and} \\
    2. & \text{$\opL$ is \prope{time-invariant}}
  \end{array}}
  \qquad\implies\qquad\brb{
    \opL \lkerne{t}{s} 
    = 
    \mcomr{\Lh^\ast(-s)}{eigenvalue} \mcoml{\lkerne{t}{s}}{eigenvector}
    }}
\end{theorem}
\begin{proof}
%\begin{align*}
%  \opL \fkerne{t}{\omega}
%    &= \inprod{\fkerne{u}{\omega}}{\fh((t;u),\omega)}
%    && \text{by linear hypothesis}
%  \\&= \inprod{\fkerne{u}{\omega}}{\fh((t-u),\omega)}
%    && \text{by time-invariance hypothesis}
%  \\&= \inprod{\fkerne{(t-v)}{\omega}}{\fh(v,\omega)}
%    && \text{let $v=t-u \implies u-t-v$}
%  \\&= \fkerne{t}{\omega}\; \inprod{\fkernea{v}{\omega}}{\fh(v,\omega)}
%    && \text{by additivity of $\inprodn$ (\prefp{def:inprod})}
%  \\&= \inprod{\fh(v,\omega)}{\fkernea{v}{\omega}}^\ast \; \fkerne{t}{\omega}
%    && \text{by conjugate symmetry of $\inprodn$ (\prefp{def:inprod})}
%  \\&= \inprod{\fh(v,\omega)}{\fkerne{v}{(-\omega)}}^\ast \, \fkerne{t}{\omega}
%  \\&= \Fh^\ast(-\omega)\, \fkerne{t}{\omega}
%    && \text{by definition of $\Fh(\omega)$}
%\end{align*}

\begin{align*}
  \brs{\opL \lkerne{t}{ s}}(s)
    &= \inprod{\lkerne{u}{ s}}{\fh((t;u), s)}
    && \text{by linear hypothesis}
  \\&= \inprod{\lkerne{u}{ s}}{\fh((t-u), s)}
    && \text{by time-invariance hypothesis}
  \\&= \inprod{\lkerne{(t-v)}{ s}}{\fh(v, s)}
    && \text{let $v=t-u \implies u=t-v$}
  \\&= \lkerne{t}{ s}\; \inprod{\lkerne{v}{-s}}{\fh(v, s)}
    && \text{by additivity of $\inprodn$ \ifdochas{vsinprod}{(\prefp{def:inprod})}}
  \\&= \inprod{\fh(v, s)}{\lkerne{v}{-s}}^\ast \; \lkerne{t}{ s}
    && \text{by conjugate symmetry of $\inprodn$ \ifdochas{vsinprod}{(\prefp{def:inprod})}}
  \\&= \inprod{\fh(v, s)}{\lkerne{v}{(- s)}}^\ast \, \lkerne{t}{ s}
  \\&= \Lh^\ast(- s)\, \lkerne{t}{ s}
    && \text{by definition of $\Lh( s)$}
\end{align*}
\end{proof}

%%--------------------------------------
%\begin{theorem}
%\label{thm:spec_D}
%\citep{pedersen2000}{79}
%%--------------------------------------
%Let $\opDif$ be the differential operator.
%\thmbox{\begin{array}{rcl@{\qquad}D}
%    \mc{4}{l}{\text{The set $\set{e^{\lambda x}}{\lambda\in\C}$ are the eigenvectors of $\opDif$. }} \\
%    \oppSpecp(\opDif) &=& \oppSpec(\opDif)  & (the spectrum of $\opDif$ is all eigenvalues) \\
%    \oppRes(\opDif)   &=& \emptyset       & ($\opDif$ has no non-spectral points whatsoever) \\
%    \oppSpecc(\opDif) &=& \emptyset       & ($\opDif$ has no continuous spectrum) \\
%    \oppSpecr(\opDif) &=& \emptyset       & ($\opDif$ has no resolvent spectrum)
%\end{array}}
%\end{theorem}
%\begin{proof}
%\begin{align*}
%  (\opDif-\lambda \opI) e^{\lambda x} 
%    &= \opDif e^{\lambda x}  - \lambda \opI e^{\lambda x}
%  \\&= \lambda e^{\lambda x}  - \lambda e^{\lambda x}
%  \\&= 0 
%  && \forall \lambda\in\C
%\end{align*}
%This theorem and proof needs more work and investigation 
%to prove/disprove its claims.\problem
%\end{proof}


%========================================================
%========================================================
%========================================================
%========================================================
%========================================================
\if 0

  \cos\fvx
    &=& \left.\sum_{n=0}^\infty \frac{\cos^{(n)}(a)}{n!}(x-a)^n\right|_{a=0}
  \\&=& \frac{\cos(0)}{0!}\fvx^0
       -\cancelto{0}{\frac{\sin(0)}{1!}\fvx^1}
       -\frac{\cos(0)}{2!}\fvx^2
       +\cancelto{0}{\frac{\sin(0)}{3!}\fvx^3}
       +\frac{\cos(0)}{4!}\fvx^4
       -\cancelto{0}{\frac{\sin(0)}{5!}\fvx^5}
       -\frac{\cos(0)}{6!}\fvx^6
       + \cdots
  \\&=& 1
       -\frac{\fvx^2}{2!}
       +\frac{\fvx^4}{4!}
       -\frac{\fvx^6}{6!}
       + \cdots
  \\&=& \sum_{n=0}^\infty \frac{(-1)^n}{(2n)!}\fvx^{2n}
\\ \\
  \sin\fvx
    &=& \left.\sum_{n=0}^\infty \frac{\sin^{(n)}(a)}{n!}(x-a)^n\right|_{a=0}
  \\&=& \cancelto{0}{\frac{\sin(0)}{0!}\fvx^0}
       +\frac{\cos(0)}{1!}\fvx^1
       -\cancelto{0}{\frac{\sin(0)}{2!}\fvx^2}
       -\frac{\cos(0)}{3!}\fvx^3
       +\cancelto{0}{\frac{\sin(0)}{4!}\fvx^4}
       +\frac{\cos(0)}{5!}\fvx^5
       -\cancelto{0}{\frac{\sin(0)}{6!}\fvx^6}
       -\frac{\cos(0)}{7!}\fvx^7
       + \cdots
  \\&=&
       \fvx
       -\frac{\fvx^3}{3!}
       +\frac{\fvx^5}{5!}
       -\frac{\fvx^7}{7!}
       + \cdots
  \\&=& \sum_{n=0}^\infty \frac{(-1)^n}{(2n+1)!}\fvx^{2n+1}
\\ \\

\begin{minipage}{3\tw/16}
  \setlength{\unitlength}{\tw/200}
  \begin{picture}(260,260)(-120,-120)
    %\color{graphpaper}\graphpaper[10](-120,-120)(260,260)%
    \thicklines%
    \color{axis}%
      \put(-120 ,   0 ){\line(1,0){240} }%
      \put(   0 ,-120 ){\line(0,1){240} }%
      \qbezier[16](70.7, 0)(70.7,40)(70.7,70.7)%
      \qbezier[16]( 0,70.7)(40,70.7)(70.7,70.7)%
    \color{circle}%
      %============================================================================
% NCTU - Hsinchu, Taiwan
% LaTeX File
% Daniel Greenhoe
%
% Unit circle with radius 100
%============================================================================

\qbezier( 100,   0)( 100, 41.421356)(+70.710678,+70.710678) % 0   -->1pi/4
\qbezier(   0, 100)( 41.421356, 100)(+70.710678,+70.710678) % pi/4-->2pi/4
\qbezier(   0, 100)(-41.421356, 100)(-70.710678,+70.710678) %2pi/4-->3pi/4
\qbezier(-100,   0)(-100, 41.421356)(-70.710678,+70.710678) %3pi/4--> pi 
\qbezier(-100,   0)(-100,-41.421356)(-70.710678,-70.710678) % pi  -->5pi/4
\qbezier(   0,-100)(-41.421356,-100)(-70.710678,-70.710678) %5pi/4-->6pi/4
\qbezier(   0,-100)( 41.421356,-100)( 70.710678,-70.710678) %6pi/4-->7pi/4
\qbezier( 100,   0)( 100,-41.421356)( 70.710678,-70.710678) %7pi/4-->2pi


%
    \color{red}%
      \put(   0 ,   0 ){\vector(1,1){70.7} }%
    \color{black}%
      \put(130,   0){\makebox(0,0)[l ]{$\Re$}}%
      \put(  0, 130){\makebox(0,0)[b ]{$\Im$}}%
      \put( 70.7,-10){\makebox(0,0)[t ]{$\cos\fvx$}}%
      \put(-10, 70.7){\makebox(0,0)[r ]{$i\sin\fvx$}}%
      \put( 80, 80){\makebox(0,0)[bl]{$e^{i\fvx}$}}%
      \put( 40, 40){\makebox(0,0)[br]{$1$}}%
      \put( 30,  5){\makebox(0,0)[bl]{$\fvx$}}%
  \end{picture}
\end{minipage}%
\hfill%
\begin{minipage}{12\tw/16}
\end{minipage}



\fi


