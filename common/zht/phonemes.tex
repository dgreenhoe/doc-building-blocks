%============================================================================
% Daniel J. Greenhoe
% XeLaTeX file
%============================================================================
%=======================================
\chapter{Phonemes of Mandarin Chinese}
%=======================================
%=======================================
\section{Languages}
%=======================================
\tbox{\includegraphics[height=30mm]{../common/english/language/ethnologue2009.jpg}}
\begin{minipage}{\tw-75mm}%
  As of 2009, there were a total of about {6909} ``living languages" in the world.
  Given that the world currently has very sophisticated travel and communication systems,
  one might suppose that every spoken language in the world is known and has been documented by 
  linguists. However, it is possible that there are still some living languages in the world 
  unknown to the world community at large.
  In fact, in 2008, a new language called \hie{Koro} was found in {India}.\footnotemark
\end{minipage}
\tbox{\includegraphics[height=25mm]{../common/english/language/koro_nytimes2010oct11.jpg}}
\citetblt{
  \citer{ethnologue2009}\\
  \citer{wilford2010}
  }
  
One use for studying languages is to help discover the origins of remote social groups.
For example, people on the islands in the Pacific ocean speak about {1200} different languages.
       %Scientists at Ackland University have used computer analysis on 400 of these languages and say many of these people
       %%including Somoans, 
       %originally came from 
       %{Taiwan} thousands of years ago.
       %\footnote{
       %\url{http://www.samoaobserver.ws/en/22_05_2017/?view=article&catid=1\%3Alatest-news&id=3573\%3Asamoans-may-hail&format=pdf&option=com_content&tpl=107}
       %%\url{http://www.samoaobserver.ws/index.php?option=com_content&view=article&id=3573\%3Asamoans-may-hail&Itemid=1}
       %%http://www.dailymail.co.uk/sciencetech/article-3819651/The-great-seafarers-DNA-ancient-skeletons-reveals-Polynesians-come-Taiwan-5-000-years-ago.html
       %%http://www.stuff.co.nz/travel/destinations/asia/67390585/New-Zealands-long-lost-Taiwanese-cuzzies
       %%https://en.wikipedia.org/wiki/Polynesian_languages
       %}
       %People on the islands in the Pacific ocean speak about {1200} different languages.
       Scientists at Ackland University have used computer analysis on 400 of these languages and say many of these people
       %including Somoans, 
       originally came from 
       {Taiwan} thousands of years ago (see \prefpp{fig:twlang}).\footnote{\url{http://language.psy.auckland.ac.nz/austronesian/research.php}}
       %\\\includegraphics[height=70mm]{../common/english/language/twlangspread_sciencedaily.jpg}% http://www.sciencedaily.com/releases/2009/01/090122141146.htm
       \begin{figure}
       \centering
        \includegraphics[width=\tw-10mm]{../common/english/language/twlangspread_map_aukland.jpg}% http://language.psy.auckland.ac.nz/austronesian/research.php
        \\\includegraphics[width=\tw-10mm]{../common/english/language/twlangspread_graph_aukland.jpg}% http://language.psy.auckland.ac.nz/austronesian/research.php
       \caption{Pacific language migration\label{fig:twlang}}
       \end{figure}
       %\\\begin{tabular}{cc}
       %    \includegraphics[height=40mm]{../common/english/language/twlangspread_map_aukland.jpg}% http://language.psy.auckland.ac.nz/austronesian/research.php
       %   &\includegraphics[height=40mm]{../common/english/language/twlangspread_graph_aukland.jpg}% http://language.psy.auckland.ac.nz/austronesian/research.php
       %  \end{tabular}
       %\\You can read more about it here:
       %\footnote{\url{http://language.psy.auckland.ac.nz/austronesian/research.php}}
       %\\\url{http://www.sciencedaily.com/releases/2009/01/090122141146.htm}

%       {\scs \url{http://www.samoaobserver.ws/index.php?option=com_content&view=article&id=3573\%3Asamoans-may-hail&Itemid=1}}


%=======================================
\section{Consonants and Vowels}
%=======================================
A \hib{phoneme} is a basic sound in a language.
Phonemes are combined together to form \hib{phones}, or syllable sounds \xref{chp:phones}.
English has about {39} \hib{phonemes};
Of these, {24} are \hib{consonants} and {15} are \hib{vowels}.
\citetbl{
  \citerppgc{ipa1999}{41}{43}{0521637511}{American English}\\
  \citerpg{prasad2008}{35}{8120334299}
  }
Of the vowel sounds, there are about {11} {monophthongs} %\zcomi{monophthongs}{單元音}
and about {4} {dipthongs}.  %\zcomi{dipthongs}{雙元音}: 

The phonemes of English are listed in \prefpp{tbl:eng_phonemes}.
There, the phonemes are transcribed using IPA, KK,\citetbl{
  \citer{kk1949}\\
  \citer{kk1953}
  }
PinYin, and ZhuYin.
For help pronouncing the IPA transcription, see
\\\indentx\url{http://www.internationalphoneticalphabet.org/ipa-sounds/ipa-chart-with-sounds/}
\\\indentx\url{http://www.ipachart.com/}
\\
%  {ㄅ} & b  & {ㄋ} & n &  {ㄑ} & q   &  {ㄗ} & z   & {ㄝ} & ye  & {ㄣ} & en   &{ㄩ} & \"u  \\
%  {ㄆ} & p  & {ㄌ} & l &  {ㄒ} & xi  &  {ㄘ} & c   & {ㄞ} & ai  & {ㄤ} & ang  &   &           \\
%  {ㄇ} & m  & {ㄍ} & g &  {ㄓ} & zhi &  {ㄙ} & s   & {ㄟ} & ei  & {ㄥ} & eng  &   &           \\
%  {ㄈ} & f  & {ㄎ} & k &  {ㄔ} & chi &  {ㄚ} & a   & {ㄠ} & ao  & {ㄦ} & er   &   &           \\
%  {ㄉ} & d  & {ㄏ} & h &  {ㄕ} & shi &  {ㄛ} & o   & {ㄡ} & ou  & {ㄧ} & yi   &   &           \\
%  {ㄊ} & t  & {ㄐ} & j &  {ㄖ} & re  &  {ㄜ} & e   & {ㄢ} & an  & {ㄨ} & u    &   &           \\
\begin{table}
\centering
\caption{Phonemes of English\label{tbl:eng_phonemes}}
\begin{tabular}[t]{|>{\scs}r|%
                    c|%                      Latin
                    >{\fntipa}c|%            IPA
                    >{\fntipa}c|%            KK
                    >{\sffamily\scshape}c|%  CMU
                    >{\fntipa}c|%            PinYin
                    >{\fntzh}c|%             ZhuYin
                    l|%                      example
                    N|}%                     voiced?
  \hline
  \mc{2}{|B|}{Latin}&\mc{1}{B|}{IPA}
                    &\mc{1}{B|}{KK}
                    &\mc{1}{B|}{CMU}
                    &\mc{1}{B|}{PinYin} 
                    &\mc{1}{B|}{ZhuYin}
                    &\mc{1}{B}{example}
                    &\mc{1}{B|}{voiced?}
  \\\hline
  \cnto & p  & p   & p    &  p   & p    & ㄆ    & as in \textbf{p}ear     & 
  \cntn & b  & b   & b    &  b   & b    & ㄅ    & as in \textbf{b}ear     & \checkmark
  \cntn & t  & t   & t    &  t   & t    & ㄊ    & as in \textbf{t}op      & 
  \cntn & d  & d   & d    &  d   & d    & ㄉ    & as in \textbf{d}oor     & \checkmark
  \cntn & k  & k   & k    &  k   & k    & ㄎ    & as in \textbf{k}ite     & 
  \cntn & g  & g   & g    &  g   & g    & ㄍ    & as in \textbf{g}irl     & \checkmark
  \cntn & ch & /tS & /tS  &  ch  & q    & ㄑ    & as in \textbf{ch}eese   & 
  \cntn & j  & /dj & /dj  &  jh  & j    & ㄐ    & as in \textbf{j}ump     & \checkmark
  \cntn & m  & m   & m    &  m   & m    & ㄇ    & as in \textbf{mouse}    & \checkmark
  \cntn & n  & n   & n    &  n   & n    & ㄋ    & as in \textbf{n}et      & \checkmark
  \cntn & ng & N   & N    &  ng  & ng   & ㄙ    & as in si\textbf{ng}     & \checkmark
  \cntn & f  & f   & f    &  f   & f    & ㄈ    & as in \textbf{f}ish     & 
  \cntn & v  & v   & v    &  v   &      &       & as in \textbf{v}et      & \checkmark
  \cntn & th & T   & T    &  th  &      &       & as in ma\textbf{th}     & 
  \cntn & th & D   & D    &  dh  &      &       & as in \textbf{th}       & \checkmark
  \cntn & s  & s   & s    &  s   & s    & ㄙ    & as in \textbf{s}ocks    & 
  \cntn & z  & z   & z    &  z   &      &       & as in \textbf{z}ero     & \checkmark
  \cntn & sh & S   & S    &  sh  & sh   & ㄒ    & as in \textbf{sh}oe     & 
  \cntn & z  & Z   & Z    &  zh  &      &       & as in a\textbf{z}ure    & \checkmark
  \cntn & h  & h   & h    &  hh  & h    & ㄏ    & as in \textbf{h}at      & 
  \cntn & r  & /*r & r    &  r   &      &       & as in \textbf{r}ug      & \checkmark
  \cntn & y  & j   & j    &  y   & y    & ㄧ    & as in \textbf{y}ou      & \checkmark
  \cntn & w  & w   & w    &  w   & w    & ㄨ    & as in \textbf{w}ait     & \checkmark
  \cntn & l  & l   & l    &  l   & l    & ㄌ    & as in \textbf{l}ate     & \checkmark
  \cnth & ee & i/: & i    &  iy  & i    & ㄧ    &  as in h\textbf{e}      & \checkmark
  \cntn & e  & E   & E    &  eh  &      &       &  as in r\textbf{e}d     & \checkmark
  \cntn & a  & /ae & /ae  &  ae  &      &       &  as in c\textbf{a}t     & \checkmark
  \cntn & i  & I   & I    &  ih  &      &       &  as in l\textbf{i}d     & \checkmark
  \cntn & ir & /@r & r    &  er  & er   & ㄦ    &  as in b\textbf{ir}d    & \checkmark
  \cntn & oo & U   & {\footnotesize/U} & uh & & &  as in g\textbf{oo}d    & \checkmark
  \cntn & ue & u   & u    &  uw  & u    & ㄨ    &  as in gl\textbf{ue}    & \checkmark
  \cntn & u  & 2   & 2    &  ah  &      &       &  as in b\textbf{u}d     & \checkmark
  \cntn & a  & a   & a    &  aa  & A    & ㄚ    &  as in f\textbf{a}ther  & \checkmark
  \cntn & a  & @   & @    &  ao  &      &       &  as in \textbf{a}bove   & \checkmark
  \cnth & e  & e   & e    &  ey  & ei   & ㄟ    &  as in b\textbf{a}y     & \checkmark
  \cntn & i  & aI  & aI   &  ai  & ay   & ㄞ    &  as in k\textbf{i}te    & \checkmark
  \cntn & oa & o   & o    &  ow  & ou   & ㄡ    &  as in g\textbf{oa}t    & \checkmark
  \cntn & ow & aU  & aU   &  aw  & ao   & ㄠ    &  as in n\textbf{ow}     & \checkmark
  \cntn & oy & OI  & OI   &  oy  &      &       &  as in b\textbf{oy}     & \checkmark
  \\\hline
\end{tabular}
\end{table}
%        P 	pee	P IY
%        B 	be	B IY
%        T 	tea	T IY
%        D 	dee	D IY
%        K 	key	K IY
%        G 	green	G R IY N
%        CH	cheese	CH IY Z
%        JH	gee	JH IY
%        M 	me	M IY
%        N 	knee	N IY
%        NG	ping	P IH NG
%        F 	fee	F IY
%        V 	vee	V IY
%        TH	theta	TH EY T AH
%        DH	thee	DH IY
%        S 	sea	S IY
%        Z 	zee	Z IY
%        SH	she	SH IY
%        ZH	seizure	S IY ZH ER
%        HH	he	HH IY
%        R 	read	R IY D
%        Y 	yield	Y IY L D
%        W 	we	W IY
%        L 	lee	L IY
%        IY	eat	IY T
%        EH	Ed	EH D
%        AE	at	AE T
%        IH	it	IH T
%        ER	hurt	HH ER T
%        UH	hood	HH UH D
%        UW	two	T UW
%        AH	hut	HH AH T
%        AA	odd     AA D
%        AO	ought	AO T
%        EY	ate	EY T
%        AY	hide	HH AY D
%        OW	oat	OW T
%        AW	cow	K AW
%        OY	toy	T OY




%\begin{tabular}{|p{10mm}| *{12}{>{\fntipa}c>{\fntipa}c|}}
%  \hline
%  &\multicolumn{2}{|p{\textwidth/11-8mm}|}{\scriptsize{bilabial}} 
%  &\multicolumn{2}{|p{\textwidth/11-8mm}|}{\scriptsize{labio-dental}} 
%  &\multicolumn{2}{|p{\textwidth/11-8mm}|}{\scriptsize{dental}} 
%  &\multicolumn{2}{|p{\textwidth/11-8mm}|}{\scriptsize{alveolar}} 
%  &\multicolumn{2}{|p{\textwidth/11-8mm}|}{\scriptsize{postal-veolar}} 
%  &\multicolumn{2}{|p{\textwidth/11-8mm}|}{\scriptsize{retroflex}}
%  &\multicolumn{2}{|p{\textwidth/11-8mm}|}{\scriptsize{palatal}} 
%  &\multicolumn{2}{|p{\textwidth/11-8mm}|}{\scriptsize{velar}} 
%  &\multicolumn{2}{|p{\textwidth/11-8mm}|}{\scriptsize{uvular}} 
%  &\multicolumn{2}{|p{\textwidth/11-8mm}|}{\scriptsize{pharyn-geal}} 
%  &\multicolumn{2}{|p{\textwidth/11-8mm}|}{\scriptsize{glottal}}
%  &\multicolumn{2}{|p{\textwidth/11-8mm}|}{\scriptsize{epi-glottal}}
%  \\\hline
%    \hline\scriptsize plosive             &p&b  &&   &&   &t&d      &&   &        &      &k&g  &         &&     &&  &&   
% %\\\hline\scriptsize implosive           &/!b& &&   &&   &&/!d     &&   &&       &&/!j  &&/!g &/;q&/!G  &&     &&   &&
%  \\\hline\scriptsize affricate           &&    &&   &&   &&        &/ch&/dz &&   &&     &&    &&        &&     &&   &&
%  \\\hline\scriptsize nasal               &&m   &&   &&   &&n       &&   &&       &&     &&N   &&        &&     &&   &&
% %\\\hline\scriptsize trill               &&/;B &&   &&   &&r       &&   &&       &&     &&    &&/;R     &&     &&   &&
% %\\\hline\scriptsize tap or flap         &&    &&   &&   &&R       &&   &&/:r    &&     &&    &&        &&     &&   &&
%  \\\hline\scriptsize fricative           &&    &f&v &T&D &s&z      &S&Z &&       &&     &&    &&        &&     &h&  &&
% %\\\hline\scriptsize laminal fricative   &&    &&   &&   &&        &&   &&       &C&    &&    &&        &&     &&   &&   
% %\\\hline\scriptsize lateral fricative   &&    &&   &&   &/-l&/;l  &&   &&       &&     &&    &&        &&     &&   &&
%  \\\hline\scriptsize approximate         &&    &&   &&   &&/*r     &&   &&       &&j    &&/*m &&        &&     &&   &&
%  \\\hline\scriptsize lateral approximate &&    &&   &&   &&l       &&   &&       &&     &&    &&        &&     &&   &&
%  \\\hline
%\end{tabular}


%\begin{tabular}{|p{10mm}| *{5}{>{\fntipa}c>{\fntipa}c|}}
%  \hline
%  &\multicolumn{2}{|p{\textwidth/11-2mm}|}{\scriptsize{front}} 
%  &\multicolumn{2}{|p{\textwidth/11-2mm}|}{\scriptsize{}} 
%  &\multicolumn{2}{|p{\textwidth/11-2mm}|}{\scriptsize{central}} 
%  &\multicolumn{2}{|p{\textwidth/11-2mm}|}{\scriptsize{}} 
%  &\multicolumn{2}{|p{\textwidth/11-2mm}|}{\scriptsize{back}}
%  \\\hline
%    \hline\scriptsize close      &i&    &&   &&   &&  &&u
%  \\\hline\scriptsize semi-high  &&     &I&  &&   &&U &&     
%  \\\hline\scriptsize close-mid  &e&    &&   &&   &&  &&o
%  \\\hline\scriptsize            &&     &&   &&/@r  &&  &&
%  \\\hline\scriptsize open-mid   &E&    &&   && && &2&
%  \\\hline\scriptsize near-open  &/ae&  &&   &&   && &&
%  \\\hline\scriptsize open       &&     &&   &&   && &A&
%  \\\hline
%\end{tabular}
%\\
%{{\fntipa /aI/, /aU/, /OI/, /@/}}.
%}




Chinese has about {29} \hib{phonemes} \xref{tbl:zht_phonemes}
Of these, about 23 are \hib{consonants} and about {11} are \hib{vowels}.
Of the vowel phonemes, there are about {9} \zcomi{monophthongs}{單元音}
and about {2} \zcomi{dipthongs}{雙元音}.

%  {ㄅ} & b  & {ㄋ} & n &  {ㄑ} & q   &  {ㄗ} & z   & {ㄝ} & ye  & {ㄣ} & en   &{ㄩ} & \"u  \\
%  {ㄆ} & p  & {ㄌ} & l &  {ㄒ} & xi  &  {ㄘ} & c   & {ㄞ} & ai  & {ㄤ} & ang  &   &           \\
%  {ㄇ} & m  & {ㄍ} & g &  {ㄓ} & zhi &  {ㄙ} & s   & {ㄟ} & ei  & {ㄥ} & eng  &   &           \\
%  {ㄈ} & f  & {ㄎ} & k &  {ㄔ} & chi &  {ㄚ} & a   & {ㄠ} & ao  & {ㄦ} & er   &   &           \\
%  {ㄉ} & d  & {ㄏ} & h &  {ㄕ} & shi &  {ㄛ} & o   & {ㄡ} & ou  & {ㄧ} & yi   &   &           \\
%  {ㄊ} & t  & {ㄐ} & j &  {ㄖ} & re  &  {ㄜ} & e   & {ㄢ} & an  & {ㄨ} & u    &   &           \\
\begin{table}
\centering
\caption{Phonemes of Mandarin Chinese\label{tbl:zht_phonemes}}
\begin{tabular}{|>{\scs}r|%                       number
                        c|%                       PinYin
                        >{\fntzh}c|%              ZhuYin
                        >{\fntipa}c|%             IPA
                        >{\fntipa}c|%             IPA-TW
                        >{\sffamily\scshape}c|%   CMU
                        l|%                       description
                        }
  \hline
  \mc{2}{|B|}{PinYin}
    &\mc{1}{B|}{ZhuYin}
    &\mc{1}{|B|}{IPA}
    &\mc{1}{|B|}{IPA-TW}
    &\mc{1}{|B|}{CMU}
    &\mc{1}{|B|}{description}
  \\\hline
  \hline
  \cnto & b   & ㄅ & b     & b     & b   & as the {\bf b} in {\bf b}oy                                                                  %& \textsubring{b}   
  \cntn & c   & ㄘ & /ts   & /ts   & ts  & as the \textbf{ts} in ca\textbf{ts} and ha\textbf{ts}                                        %& /ts/>h            
  \cntn & d   & ㄉ & d     & d     & d   & as the \textbf{d} in \textbf{d}ay                                                            %& \textsubring{d}   
  \cntn & f   & ㄈ & f     & f     & f   & as the \textbf{f} in \textbf{f}ade, \textbf{f}in, and \textbf{f}un                           %& f                 
  \cntn & g   & ㄍ & g     & g     & g   & the hard \textbf{g} sound in \textbf{g}ather, \textbf{g}et, \textbf{g}ive, and \textbf{g}un  %& \r{g}             
  \cntn & h   & ㄏ & x     & x     & hh  & the \textbf{h} sound in \textbf{h}a \textbf{h}a or \textbf{h}appy                            %& x                 
  \cntn & j   & ㄐ & /dj   & /dj   & jh  & the \textbf{j} sound in \textbf{j}ay or the \textbf{g} sound in \textbf{g}ym                 %& \textsubring{/;dz}
  \cntn & k   & ㄎ & k     & k     & k   & the \textbf{k} sound in \textbf{k}angaroo or the \textbf{c} sound in \textbf{c}ap            %& k/>h              
  \cntn & l   & ㄌ & l     & l     & l   & the \textbf{l} sound in \textbf{l}abel, \textbf{l}ime, and \textbf{l}ow                      %& l                 
  \cntn & m   & ㄇ & m     & m     & m   & the \textbf{m} sound in \textbf{m}ayor, \textbf{m}eet, \textbf{m}ow, and \textbf{m}ute       %& m                 
  \cntn & n   & ㄋ & n     & n     & n   & the \textbf{n} sound in \textbf{n}ame, \textbf{n}ew, \textbf{n}ine, and \textbf{n}o          %& n                 
  \cntn & ng  &    & N     & N     & ng  & the \textbf{ng} sound in ri\textbf{ng}
  \cntn & p   & ㄆ & p     & p     & p   & the \textbf{p} sound in \textbf{p}ast, \textbf{p}en, \textbf{p}ine, and \text{p}rune         %& p                 
  \cntn & q   & ㄑ & /ch   & /ch   & ch  & the \textbf{ch} sound in \textbf{ch}ew                                                       %& /ch               
  \cntn & s   & ㄙ & s     & s     & s   & the \textbf{s} sound in \textbf{s}and, \textbf{s}end, \textbf{s}ince, and \textbf{s}un       %& s                 
  \cntn & t   & ㄊ & t     & t     & t   & the \textbf{t} sound in \textbf{t}alk, \textbf{t}ent, \textbf{t}ick, and \textbf{t}ent       %& t/>h              
  \cntn & w   & ㄨ & w     & w     & w   & the \textbf{w} sound in \textbf{w}alk, \textbf{w}ent, and \textbf{w}in                       %& w                 
  \cntn & x   & ㄒ & C     & S     &     & a semi-shrill form of \textbf{sh} %(try it smiling)                                          %& C                  
  \cntn & y   & ㄝ & j     & j     & y   & the \textbf{y} sound in \textbf{y}ellow                                                      %& j                 
  \cntn & z   & ㄗ & /dj   & /dj   & z   & the \textbf{z} sound in \textbf{z}ebra                                                       %& /;dz              
  \\\hline
  \cntx & ch  & ㄔ & /:tS  & /tS   & ch  & retroflex sounds (no English equivalent)                                                     %& /:tS              
  \cntn & sh  & ㄕ & /:s   & S     & sh  & retroflex sounds (no English equivalent)                                                     %& /:s               
  \cntn & zh  & ㄓ & /:z   & /dj   & jh  & retroflex sounds (no English equivalent)                                                     %& /:z               
  \cntn & r   & ㄖ & /:R   & /*r   & r   & retroflex sounds (no English equivalent)                                                     %& /:R               
  \\\hline
  \cntx & a   & ㄚ & a     & a     & aa  & as the {\bf a} in f{\bf a}ther                                                               %& A                 
  \cntn & e   & ㄜ & @     & @     & eh  & the \hie{uh} sound as the {\bf ai} in s{\bf ai}d                                             %& E                 
  \cntn & er  & ㄦ & /@r   & /@r   & er  & the \hie{ir} sound as in b\textbf{ir}d                                                                                             %&                   
  \cntn & i   & ㄧ & i/:   & i     & iy  & the \hie{ee} sound as the {\bf ee} in s{\bf ee}                                              %& i                 
  \cntn & o   & ㄛ & o     & o     & uh  & the short \hie{oh} sound as the {\bf o} in n{\bf o}pe                                        %& O                 
  \cntn & u   & ㄨ & u     & u     & uw  & as the {\bf o} sound in d{\bf o}                                                             %& U                 
  \cntn & /:u & ㄩ & y     & y     &     & the \textbf{ew} sound, with lips rounded,                                                    %& u                 
      \\&     &    &       &       &     & similar to the \textbf{ew} sound in \textbf{ewe} (female sheep).                             %&                   
  \\\hline
  \cntx & ai  & ㄞ & aI    & aI    & ay  & as the {\bf i} in k\textbf{i}te                                                              %& aI                
  \cntn & ao  & ㄠ & aU    & aU    & aw  &  the \textbf{ow} sound as in c{\bf ow} and n{\bf ow}.                                        %& aU                
  \cntn & ei  & ㄟ & eI    & eI    & ey  & as the {\bf ay} in s{\bf a}y                                                                 %& e                 
  \cntn & o   & ㄡ & oU    & oU    & ow  & the long \hie{oh} sound as the {\bf o} in m{\bf o}w                                          %& o                 
  \\\hline
\end{tabular}
\end{table}

  %{ㄣ} & en                                           
  %{ㄤ} & ang                                          
  %{ㄥ} & eng                                          
  %{ㄢ} & an                                           


%        B 	be	B IY
%        D 	dee	D IY
%        F 	fee	F IY
%        G 	green	G R IY N
%        HH	he	HH IY
%        JH	gee	JH IY
%        K 	key	K IY
%        L 	lee	L IY
%        M 	me	M IY
%        N 	knee	N IY
%        NG	ping	P IH NG
%        P 	pee	P IY
%        R 	read	R IY D
%        CH	cheese	CH IY Z
%        S 	sea	S IY
%        T 	tea	T IY
%        W 	we	W IY
%        Y 	yield	Y IY L D
%        Z 	zee	Z IY
%        ZH	seizure	S IY ZH ER

%        SH	she	SH IY

%        AA	odd     AA D
%        EH	Ed	EH D
%        ER	hurt	HH ER T
%        IY	eat	IY T
%        UH	hood	HH UH D
%        UW	two	T UW
%        AE	at	AE T
%        AH	hut	HH AH T
%        AO	ought	AO T
%        AY	hide	HH AY D
%        AW	cow	K AW
%        EY	ate	EY T
%        OW	oat	OW T
%        DH	thee	DH IY
%        IH	it	IH T
%        OY	toy	T OY
%        TH	theta	TH EY T AH
%        V 	vee	V IY


%{%\fntipa  %\fntFreeSans\addfontfeatures{Mapping=ascii-uipa}

%\begin{tabular}{|p{10mm}| *{12}{>{\fntipa}c>{\fntipa}c|}}
%  \hline
%  &\multicolumn{2}{|p{\textwidth/11-8mm}|}{\scriptsize{bilabial}} 
%  &\multicolumn{2}{|p{\textwidth/11-8mm}|}{\scriptsize{labio-dental}} 
%  &\multicolumn{2}{|p{\textwidth/11-8mm}|}{\scriptsize{dental}} 
%  &\multicolumn{2}{|p{\textwidth/11-8mm}|}{\scriptsize{alveolar}} 
%  &\multicolumn{2}{|p{\textwidth/11-8mm}|}{\scriptsize{postal-veolar}} 
%  &\multicolumn{2}{|p{\textwidth/11-8mm}|}{\scriptsize{retroflex}}
%  &\multicolumn{2}{|p{\textwidth/11-8mm}|}{\scriptsize{palatal}} 
%  &\multicolumn{2}{|p{\textwidth/11-8mm}|}{\scriptsize{velar}} 
%  &\multicolumn{2}{|p{\textwidth/11-8mm}|}{\scriptsize{uvular}} 
%  &\multicolumn{2}{|p{\textwidth/11-8mm}|}{\scriptsize{pharyn-geal}} 
%  &\multicolumn{2}{|p{\textwidth/11-8mm}|}{\scriptsize{glottal}}
%  &\multicolumn{2}{|p{\textwidth/11-8mm}|}{\scriptsize{epi-glottal}}
%  \\\hline
%    \hline\scriptsize plosive             &p&b  &&   &&   &t&d      &&   && && &k&g  &&    &&     &&  &&
% %\\\hline\scriptsize implosive           &/!b& &&   &&   &&/!d     &&   &&       &&/!j  &&/!g &/;q&/!G  &&     &&   &&
%  \\\hline\scriptsize affricate           &&    &&   &&   &&        &/ch& &&   &&/;dz &&    &&        &&     &&   &&
%  \\\hline\scriptsize nasal               &&m   &&   &&   &&n       &&   &&       &&     &&N   &&        &&     &&   &&
% %\\\hline\scriptsize trill               &&/;B &&   &&   &&r       &&   &&       &&     &&    &&/;R     &&     &&   &&
% %\\\hline\scriptsize tap or flap         &&    &&   &&   &&R       &&   &&/:r    &&     &&    &&        &&     &&   &&
%  \\\hline\scriptsize fricative           &&  &f& && &s&      &S& &/:s&/:z && &x&  &&      && && &&
% %\\\hline\scriptsize laminal fricative   &&    &&   &&   &&        &&   &&       &C&    &&    &&        &&     &&   &&   
% %\\\hline\scriptsize lateral fricative   &&    &&   &&   &/-l&/;l  &&   &&       &&     &&    &&        &&     &&   &&
%  \\\hline\scriptsize approximate         &&    &&  &&   &&     &&   &&/:R    &&    && &&        &&     &&   &&
% %\\\hline\scriptsize lateral approximate &&    &&   &&   &&l       &&   &&/:l    &&L    &&/;L &&        &&     &&   &&
%  \\\hline
%\end{tabular}
%
%\begin{tabular}{|p{10mm}| *{5}{>{\fntipa}c>{\fntipa}c|}}
%  \hline
%  &\multicolumn{2}{|p{\textwidth/11-2mm}|}{\scriptsize{front}} 
%  &\multicolumn{2}{|p{\textwidth/11-2mm}|}{\scriptsize{}} 
%  &\multicolumn{2}{|p{\textwidth/11-2mm}|}{\scriptsize{central}} 
%  &\multicolumn{2}{|p{\textwidth/11-2mm}|}{\scriptsize{}} 
%  &\multicolumn{2}{|p{\textwidth/11-2mm}|}{\scriptsize{back}}
%  \\\hline
%    \hline\scriptsize close      &i&   &&   && &&  &&u
%  %\\\hline\scriptsize semi-high  &&     &I&Y &&   &&U &&     
%  \\\hline\scriptsize close-mid  &e&  &&   && &&  &7&
%  %\\\hline\scriptsize            &&     &&   &&@  &&  &&
%  \\\hline\scriptsize open-mid   &E& &&   && && &2&O
%  \\\hline\scriptsize near-open  &&  &&   &a&   && &&
%  \\\hline\scriptsize open       && && &&    && &A&
%  \\\hline
%\end{tabular}
%}

%=======================================
\section{Phonetic Systems}
%=======================================
The challenge with all language scripts,
is with what sound a grapheme (a ``letter" or ``character") represents.
This is a particular problem with logograms, such as the set of Chinese characters.
And this is not just a problem faced by non-Chinese persons trying to learn Chinese,
but for native Chinese speaking children who are learning to read and write their own language.
For this purpose there are several \hie{transcription}s that provide phonetic support for Chinese characters---%
that is, phonetic systems that let a reader know how he or she should pronounce a given character.
Of these systems, the two most common standard systems are\citetbl{\citer{iso70981991}}
\\\begin{tabular}{lll}
  \circOne & (based on the Latin alphabet abc\ldots)
           & \prefp{sec:pinyin}
           \\
  \circTwo & (based on other symbols \zht{ㄅㄆㄇㄈㄉㄊㄋ}, etc)
           & \prefp{sec:ZhuYin}
\end{tabular}\\
Of these two systems, PinYin tends to be more convenient for Westerners who tend
in turn to be more familiar with the Latin alphabet.

These two transcriptions support the sounds of Mandarin Chinese, but not all the sounds of
all the known languages in the world. In fact, they do not even support all the phonemes of English.
A much more general transcription is the 
\hie{International  Phonetic Alphabet} (\hie{IPA}).\citetbl{\citer{ipa1999}}
Much of this transcription is similar to the Latin alphabet, but uses some non-Latin graphemes as well.
IPA transcription can be used to represent the phonemes of Chinese, English, and every other known language
that can be made audible by the human vocal system (tongue, teeth, lungs, nasal passages, etc.).

If concern is restricted to English only, then supporting transcriptions include\citetbl{
  \citer{cmudict07b}\\
  \citer{beep}\\
  \citer{kk1949}\\
  \citer{kk1953}
  }
\\\begin{tabular}{lll}
  \circThree & CMU & (based on the \hie{Carnegie Mellon University Pronouncing Dictionary})\\
  \circFour  & BEEP& (``British English prounciations")\\
  \circFive  & KK  & (based on Kenyon and Knott's \hie{A Pronouncing Dictionary of American English}).
\end{tabular}\\
The transcription KK, it seems, is currently mostly limited in use to Taiwan.
The CMU and BEEP transcriptions use only the basic Latin alphabet (A,B,C,\ldots)
and therefore are convenient for processing using computers.


%=======================================
\subsection{PinYin}
\label{sec:pinyin}
%=======================================
\begin{minipage}{\tw-85mm}
The phonemes of PinYin are listed under ``PinYin" in \prefpp{tbl:zht_phonemes}.
Chinese/English dictionaries and phrase books for English speaking people
most often use Hanyu PinYin to specify the phonetic pronunciation
of Chinese characters.
\end{minipage}%
\hfill\ftbox{\zhtsd{漢}{語}{h/`an}{y{/>u}}{Han dynasty}{language}{Chinese}
             \zhtsd{拼}{音}{p/-in}{y/-in}{piece together}{sound}{pinyin}%
            }\\
Hanyu PinYin has been adopted by mainland China
as the standard phonetic system for Mandarin Chinese.
It has been standardized by the \hie{International Organization for Standardization}
as specified in the 1991 standard \emph{ISO 7098:1991}.\citetbl{\citer{iso70981991}}


%%\begin{tblvv}{PinYin: consonants}{voc:pinyin_consonants}
%\begin{tabular}{|>{\scs}r|c|>{\fntzh}c|>{\fntipa}c|l|}
%  \hline
%        & pinyin & \mc{1}{c|}{ZhuYin} & \mc{1}{|c|}{IPA} & description
%  \\\hline
%%              {} & n &  {} & q   &  {} & z   & {ㄝ} & ye  & {ㄣ} & en   &{} & \"u  \\
%%  {} & p  & {} & l &  {ㄒ} & xi  &  {} & c   & {ㄞ} & ai  & {ㄤ} & ang  &   &           \\
%%  {} & m  & {} & g &  {ㄓ} & zhi &  {} & s   & {ㄟ} & ei  & {ㄥ} & eng  &   &           \\
%%  {} & f  & {} & k &  {ㄔ} & chi &  {} & a   & {ㄠ} & ao  & {ㄦ} & er   &   &           \\
%%  {} & d  & {ㄏ} & h &  {ㄕ} & shi &  {ㄛ} & o   & {} & ou  & {} & yi   &   &           \\
%%  {} & t  & {ㄐ} & j &  {ㄖ} & re  &  {} & e   & {ㄢ} & an  & {} & u    &   &           \\
%  \hline
%  \cnto & b & ㄅ & \textsubring{b}            & as the {\bf b} in {\bf b}oy                                                                  
%  \cntn & c & ㄘ & /ts/>h         & as the \textbf{ts} in ca\textbf{ts} and ha\textbf{ts}                                        
%  \cntn & d & ㄉ & \textsubring{d}            & as the \textbf{d} in \textbf{d}ay                                                            
%  \cntn & f & ㄈ & f            & as the \textbf{f} in \textbf{f}ade, \textbf{f}in, and \textbf{f}un                           
%  \cntn & g & ㄍ & \r{g}            & the hard \textbf{g} sound in \textbf{g}ather, \textbf{g}et, \textbf{g}ive, and \textbf{g}un  
%  \cntn & h & ㄏ & x            & the \textbf{h} sound in \textbf{h}a \textbf{h}a or \textbf{h}appy                            
%  \cntn & j & ㄐ & \textsubring{/;dz}          & the \textbf{j} sound in \textbf{j}ay or the \textbf{g} sound in \textbf{g}ym                 
%  \cntn & k & ㄎ & k/>h            & the \textbf{k} sound in \textbf{k}angaroo or the \textbf{c} sound in \textbf{c}ap            
%  \cntn & l & ㄌ & l            & the \textbf{l} sound in \textbf{l}abel, \textbf{l}ime, and \textbf{l}ow                      
%  \cntn & m & ㄇ & m            & the \textbf{m} sound in \textbf{m}ayor, \textbf{m}eet, \textbf{m}ow, and \textbf{m}ute       
%  \cntn & n & ㄋ & n            & the \textbf{n} sound in \textbf{n}ame, \textbf{n}ew, \textbf{n}ine, and \textbf{n}o          
%  \cntn & p & ㄆ & p            & the \textbf{p} sound in \textbf{p}ast, \textbf{p}en, \textbf{p}ine, and \text{p}rune         
%  \cntn & q & ㄑ & /ch          & the \textbf{ch} sound in \textbf{ch}ew                                                       
%  \cntn & r & ㄖ & /:R          & the \textbf{r} sound in \textbf{r}an, \textbf{r}ed, and \textbf{r}ipe                        
%  \cntn & s & ㄙ & s            & the \textbf{s} sound in \textbf{s}and, \textbf{s}end, \textbf{s}ince, and \textbf{s}un       
%  \cntn & t & ㄊ & t/>h         & the \textbf{t} sound in \textbf{t}alk, \textbf{t}ent, \textbf{t}ick, and \textbf{t}ent       
%  \cntn & w & ㄨ & w            & the \textbf{w} sound in \textbf{w}alk, \textbf{w}ent, and \textbf{w}in                       
%  \cntn & x & ㄒ & C            & a semi-shrill form of \textbf{sh} (try it smiling)                                           
%  \cntn & y &    & j            & the \textbf{y} sound in \textbf{y}ellow                                                      
%  \cntn & z & ㄗ & /;dz         & the \textbf{z} sound in \textbf{z}ebra
%  \\\hline
%\end{tabular}
%
%%\begin{tblvv}{PinYin: vowels}{voc:pinyin_vowels}
%\begin{tabular}{|>{\scs}r|c|>{\fntzh}c|>{\fntipa}c|p{\tw-56mm}|}
%  \hline
%        & pinyin & \mc{1}{c|}{ZhuYin} & \mc{1}{|c|}{IPA} & description
%  \\\hline
%  \hline
%  \cnto & a  & ㄚ & A          & as the {\bf a} in f{\bf a}ther                        
%  \cntn & e  & ㄜ & E          & the \hie{uh} sound as the {\bf ai} in s{\bf ai}d      
%  \cntn & i  & ㄧ & i          & the \hie{ee} sound as the {\bf ee} in s{\bf ee}       
%  \cntn & o  & ㄡ & o          & the short \hie{oh} sound as the {\bf o} in m{\bf o}w  
%  \cntn & u  & ㄨ & U          & as the {\bf o} sound in d{\bf o}                      
%  \cntn & v  & ㄩ & u          & the \textbf{ew} sound, with lips rounded, similar to the \textbf{ew} sound in \textbf{ewe} (female sheep).
%  \\\hline
%\end{tabular}
%%\end{tblvv}



%=======================================
\subsection{ZhuYin}
\label{sec:ZhuYin}
%=======================================\
\begin{minipage}{\tw-75mm}%
  An alternative phonetic system to pinyin is \zht{注音符號} (zh/`u y/-in f/'u h/`ao),
  or here referred to simply as \hie{ZhuYin}.
  The phonemes of ZhuYin are described in \prefpp{tbl:zht_phonemes}.
  Alternative names for ZhuYin include the following:
\end{minipage}\hfill%
\ftbox{\zhtsd{注}{音}{zh/`u}{y/-in}{to annotate}{sound}{ZhuYin}
       \zhtsd{符}{號}{f/'u}{h/`ao}{symbol}{mark}{symbol}}
\\\begin{tabular}{lll}
  \circOne & \hie{BoPoMoFo}
           & based on the first four characters of ZhuYin (\zht{ㄅㄆㄇㄈ})
           \\
  \circTwo & \hie{Mandarin Phonetic Symbols I}
           & (\zhtzy{國語注音符號第一式})
           \\
  \circThree & MPS I
           & (\zhtzy{注音一式}) --- an abbreviation of \circTwo
           \\
  \circFour & \hie{Mandarin Phonetic Symbols}
           & (\zhtzy{國語注音符號}) --- another abbreviation of \circTwo
\end{tabular}

If you are interested in learning traditional Chinese characters,
then you may find learning ZhuYin very much worth your time.
There is a huge number of children's books available written using
traditional Chinese characters with ZhuYin beside each character.
So even if you do not know many Chinese characters,
you will still be able to pronounce accurately each character,
will be able to look up the meaning of each character in a dictionary,
and can have access to a wealth of language learning material.
Many such children's books also come with an audio CD, allowing you
to also improve your listening comprehension.

Here is a table showing the relationship between ZhuYin and PinYin.
Once you know these relationships, you can easily convert the
ZhuYin you see in a book to pinyin that is often used in
Chinese/English dictionaries.

%\begin{tabular}{|*{7}{>{\fontsize{35mm}{35mm}\usefont{OT1}{put}{b}{n}}ll|}}
\begin{tabular}{|*{7}{>{\fntzh}ll|}}
  \hline
  {ㄅ} & b  & {ㄋ} & n &  {ㄑ} & q   &  {ㄗ} & z   & {ㄝ} & ye  & {ㄣ} & en   &{ㄩ} & \"u  \\
  {ㄆ} & p  & {ㄌ} & l &  {ㄒ} & xi  &  {ㄘ} & c   & {ㄞ} & ai  & {ㄤ} & ang  &   &           \\
  {ㄇ} & m  & {ㄍ} & g &  {ㄓ} & zhi &  {ㄙ} & s   & {ㄟ} & ei  & {ㄥ} & eng  &   &           \\
  {ㄈ} & f  & {ㄎ} & k &  {ㄔ} & chi &  {ㄚ} & a   & {ㄠ} & ao  & {ㄦ} & er   &   &           \\
  {ㄉ} & d  & {ㄏ} & h &  {ㄕ} & shi &  {ㄛ} & o   & {ㄡ} & ou  & {ㄧ} & yi   &   &           \\
  {ㄊ} & t  & {ㄐ} & j &  {ㄖ} & re  &  {ㄜ} & e   & {ㄢ} & an  & {ㄨ} & u    &   &           \\
  \hline
\end{tabular}

%%%In addition to these sounds, tones are usually represented by marks next to the words,
%%%as described next:
%%%% (with accompanying Chinese characters to the left):
%%%\\\begin{tabular}{l|ll|>{\Large}c|*{3}{>{\Large}l}}
%%%             & \mc{2}{c|}{tone}                          & \mc{1}{c}{marking} & \mc{3}{c}{examples}        \\
%%%  \hline
%%%  \circOne   & \hi{first  tone} & (\hi{high    tone}) &   {\hspace{1ex}}   & \fntzhr{媽}, & \fntzhr{八} & \fntzhr{歌} \\
%%%  \circTwo   & \hi{second tone} & (\hi{rising  tone}) & \'{\hspace{1ex}}   & \fntzhr{麻}, & \fntzhr{拔} & \fntzhr{格} \\
%%%  \circThree & \hi{third  tone} & (\hi{low     tone}) & \v{\hspace{1ex}}   & \fntzhr{馬}, & \fntzhr{把} & \fntzhr{舸} \\
%%%  \circFour  & \hi{fourth tone} & (\hi{falling tone}) & \`{\hspace{1ex}}   & \fntzhr{罵}, & \fntzhr{爸} & \fntzhr{各} \\
%%%  \circFive  & \hi{no     tone} & (\hi{neutral tone}) & \.{\hspace{1ex}}   & \fntzhr{麼}, & \fntzhr{吧} & \fntzhr{個}
%%%\end{tabular}




Similar to English speaking children's custom of singing
the ``ABC" song to learn the English alphabet,
Chinese speaking children learning ZhuYin often chant the ZhuYin alphabet
starting in the upper left corner, going down, and proceeding
column by column like this:

\begin{tabular}{r *{6}{>{\color{blue}$\rightarrow$ }lr} >{\color{blue}$\curvearrowright$}l }
  (start) && be && pe  && me  && fe  && de  && te  & \\
          && ne && le  && ge  && ke  && he  && ji  & \\
          && qi && xi  && zhi && chi && shi && re  & \\
          && zi && ci  && si  && a   && o   && e   & \\
          && yi && ai  && ei  && ao  && ou  && an  & \\
          && en && ang && eng && er  && yi  && u   & \\
          && \"u
\end{tabular}


Here is an example of a very old and very famous Chinese poem
(and more recently, a very famous song as well, \xrefnp{sec:brightmoon})
using traditional Chinese characters with ZhuYin phonetic symbols:

\begin{large}\fntzhr
%\begin{zhtzye}
明月幾時有,把酒問青天。
\\不知天上宮闕,今夕是何年。
\\我欲乘風歸去,又恐瓊樓玉宇,高處不勝寒。
\\起舞弄清影,何似在人間。
\\轉朱閣,低綺戶,照無眠。
\\不應有恨,何事長向別時圓。
\\人有悲歡離合,月有陰晴圓缺,此事古難全。
\\但願人長久,千里共嬋娟。
%\end{zhtzye}
\end{large}

%=======================================
\section{Caveats}
%=======================================
Note that both PinYin and ZhuYin does have some curiosities and accompanying caveats:
\begin{enumerate}
  \item One might suppose that the ZhuYin ``\zht{ㄥ}" and PinYin ``eng" are always pronounced
        as the IPA ``{\fntipa 2N}" (English ``ung" as in ``rung") sound. 
        Indeed, it does have this sound in almost every instance when it immediately follows a consonant
        as in the following examples:
        \\\indentx\begin{tabular}{ccclll}
            \imark&\zht{ㄘㄥ}&as in& \zht{三層}   & s/-an c/'eng    & (3 floors) 
          \\\imark&\zht{ㄔㄥ}&as in& \zht{成人}   & ch/'eng r/'en   & (an adult) 
          \\\imark&\zht{ㄉㄥ}&as in& \zht{等一下} & d/>eng yi xi/`a & (wait a moment) 
          \\\imark&\zht{ㄍㄥ}&as in& \zht{更多}   & g/`eng d/-uo    & (even more) 
          \\\imark&\zht{ㄏㄥ}&as in& \zht{橫的}   & h/'eng de       & (horizontal) 
          \\\imark&\zht{ㄎㄥ}&as in& \zht{坑道}   & k/-eng d/`ao    & (a tunnel) 
          \\\imark&\zht{ㄌㄥ}&as in& \zht{冷凍}   & l/>eng d/`ong   & (freezing) 
          \\\imark&\zht{ㄆㄥ}&as in& \zht{碰見}   & p/`eng ji/`an   & (to meet unexpectedly) 
          \\\imark&\zht{ㄓㄥ}&as in& \zht{正方}   & zh/`eng f/-ang  & (a square) 
        \end{tabular}\\
        \ftbox{\zhtsdH{風}{箏}{f/-eng}{zh/-eng}{wind}{zither}{kite}}
        \begin{minipage}{\tw-50mm}
        However, curiously, when it immediately follows the ZhuYin \zht{ㄈ} or PinYin ``f",
        \zht{ㄥ} takes on the IPA ``{\fntipa uN}" (ZhuYin \zht{ㄨㄥ}) sound as in 
        \zhtzy{風箏} (see illustration to the left).
        The character pair \zhtzy{風箏} is especially curious because in the first character \zht{風}
        the \zht{ㄥ} is pronounced with the IPA {\fntipa uN} (similar to the English ``ong" as in ``song");
        but in the second character \zhtzy{箏}, the \zht{ㄥ} is pronounced with the IPA {\fntipa 2N} 
        (English ``eng").
        \end{minipage}

  \item One might suppose that the ZhuYin ``\zht{ㄡ}" and PinYin ``o" are always pronounced
        as the IPA ``{\fntipa o}" (English ``o" as in ``go") sound. 
        Indeed, it does have this sound in almost every instance when it immediately follows a consonant
        as in the following examples:
        \\\indentx\begin{tabular}{ccclll}
            \imark&\zht{ㄔㄡ}&as in& \zhtzy{小丑} & xi/>ao ch/>ou   & (a clown) 
          \\\imark&\zht{ㄉㄡ}&as in& \zhtzy{綠豆} & l/`:u d/`ou     & (green bean) 
          \\\imark&\zht{ㄍㄡ}&as in& \zhtzy{小狗} & xi/>ao g/>ou    & (a puppy) 
        \end{tabular}\\
       \ftbox{\zhtsdH{外}{婆}{w/`ai}{p/'o}{outside}{old woman}{maternal grandmother}}
       \begin{minipage}{\tw-103mm}
       However, curiously, when it immediately follows the ZhuYin \zht{ㄆ} or PinYin ``p",
       \zht{ㄆ} takes on the IPA ``{\fntipa uo}" (ZhuYin \zht{ㄨㄛ}) sound as in 
       \zhtzy{外婆} or \zhtzy{巫婆}.
       \end{minipage}
       \ftbox{\zhtsdH{巫}{婆}{w/-u}{p/'o}{sorcery}{old woman}{witch}}
\end{enumerate}

%  ㄅㄆㄇㄈ ㄉㄊㄋㄌㄍㄎㄏ ㄐㄑㄒ ㄓㄔㄕㄖ ㄗㄘㄙ ㄚㄛㄜㄝㄞㄟㄠㄡㄢㄣㄤㄥㄦ ㄧㄨㄩ
%                    ㄚㄟㄞㄢㄤㄠ ㄜㄟㄣㄥㄝ  ㄧ ㄛㄡㄨㄩ
%                    ㄅㄆㄇㄈ ㄉㄊㄋㄌㄍㄣ

%=======================================
\section{Some vocabulary}
%=======================================
\begin{tegzkD}{8}
   \busegzkD{language}      {n}  {語言}      {}{"l{/ae}NgwI/dz}    {}{}   \busegzkD{tone}          {n}  {聲}        {}{ton}                {}{}  
   \busegzkD{phone}         {n}  {聲音}      {}{fon}               {}{}   \busegzkD{tonal}         {adj}{有聲的}    {}{"tonl}              {}{}  
   \busegzkD{sound}         {n}  {聲音}      {}{saUnd}             {}{}   \busegzkD{nontonal}      {adj}{無聲的}    {}{"nAn""tonl}         {}{}  
   \busegzkD{consonant}     {n}  {子音}      {}{"kAns@n@nt}        {}{}   \busegzkD{alphabet}      {n}  {字母}      {}{{"/ae}lf@""bEt}     {}{}  
   \busegzkD{vowel}         {n}  {母音}      {}{"vaU@l}            {}{}   \busegzkD{script}        {n}  {字母}      {}{skrIpt}             {}{}  
   \busegzkD{voiced}        {adj}{濁音的}    {}{vOIst}             {}{}   \busegzkD{grapheme}      {n}  {字位}      {}{"gr{/ae}fim}        {}{}  
   \busegzkD{voiceless}     {adj}{清音的}    {}{"vOIslIs}          {}{}   \busegzkD{character}     {n}  {字位}      {}{"k{/ae}rIkt{/@r}}   {}{}  
   \busegzkD{syllable}      {n}  {音節}      {}{"sIl@bl}           {}{}   \busegzkD{glyph}         {n}  {字形}      {}{glIf}               {}{}  
\end{tegzkD}%
\\
\begin{tegzk}[-1]
  \busegzk{phonetics}     {n}  {語音系}    {}{fo"nEtIcs}         {}{}   
  \busegzk{phoneme}       {n}  {音素/音位} {}{"fonim}            {}{}   
  \busegzk{ideograph}     {n}  {形意文字}  {}{I"dI@""gr{/ae}f}    {}{}  
  \busegzk{pictograph}    {n}  {象形文字}  {}{"pIkt@""gr{/ae}f}   {}{}  
\end{tegzk}


  % \busegzkD{language}      {n}  {語言}      {}{"l{/ae}NgwI/dz}      {}{}
  % \busegzkD{phone}         {n}  {聲音}      {}{fon}                {}{}
  % \busegzkD{sound}         {n}  {聲音}      {}{saUnd}              {}{}
  % \busegzkD{phoneme}       {n}  {音素/音位} {}{"fonim}             {}{}
  % \busegzkD{consonant}     {n}  {子音}      {}{"kAns@n@nt}         {}{}
  % \busegzkD{vowel}         {n}  {母音}      {}{"vaU@l}     {}{}
  % \busegzkD{voiced}        {adj}{濁音的}   {}{vOIst}             {}{}  
  % \busegzkD{voiceless}     {adj}{清音的}   {}{"vOIslIs}          {}{}  
  % \busegzkD{syllable}      {n}  {音節}      {}{"sIl@bl}            {}{}
  % \busegzkD{phonetics}     {n}  {語音系}    {}{fo"nEtIcs}          {}{}
  % \busegzkD{tone}          {n}  {聲}        {}{ton}                {}{} 
  % \busegzkD{tonal}         {adj}{有聲的}    {}{"tonl}              {}{} 
  % \busegzkD{nontonal}      {adj}{無聲的}    {}{"nAn""tonl}         {}{} 
  % \busegzkD{alphabet}      {n}  {字母}      {}{{"/ae}lf@""bEt}     {}{}
  % \busegzkD{script}        {n}  {字母}      {}{skrIpt}             {}{}  
  % \busegzkD{grapheme}      {n}  {字位}      {}{"gr{/ae}fim}        {}{}
  % \busegzkD{character}     {n}  {字位}      {}{"k{/ae}rIkt{/@r}}   {}{}
  % \busegzkD{glyph}         {n}  {字形}      {}{glIf}               {}{}
  % \busegzkD{ideograph}     {n}  {形意文字}  {}{I"dI@""gr{/ae}f}    {}{}
  % \busegzkD{pictograph}    {n}  {象形文字}  {}{"pIkt@""gr{/ae}f}   {}{}



%   \busegzkD{linguist}      {n}  {語言學者}  {}{"lINgwIst}          {}{}  
%   \busegzkD{linguistics}   {n}  {語言學}    {}{liN"gwistIks}       {}{}  
  %\busegzkD{phonetics}     {n}  {全部語言的音系學} {}{fo"nEtIcs}     {}{}
  %\busegzkD{phonology}     {n}  {一個語言的音系學} {}{fo"nAl@{/dz}I} {}{}
  %\busegzkD{pronunciation} {n}  {發音}    {}{pr@""n2nsI"eS@n}    {}{}  
  %\busegzkD{accent}        {n}  {口音}    {}{"{/ae}ksEnt}         {}{}  
  %\busegzkD{translate}     {v}  {翻譯}    {}{tr{/ae}ns"let}       {}{}  
  %\busegzkD{translation}   {n}  {翻譯}    {}{tr{/ae}ns"leS@n}     {}{}  
  %\busegzkD{translator}    {n}  {翻譯家}   {}{tr{/ae}ns"let/@r}   {}{} 
  %\busegzkD{interpreter}   {n}  {譯員}     {}{In"t{/3r}prIt/@r}    {}{} 
  %\busegzkD{voice box}     {n}  {喉頭}     {}{vOIs bAks}         {}{}  
  %\busegzkD{vibrate}       {v}  {顫動}     {}{"vaIbret}          {}{}  
  %\busegzkD{nasal}         {adj}{鼻音的}   {}{"nez.l}            {}{}  
  %\busegzkD{frictive}      {adj}{摩擦的}   {}{}                  {}{}  

%{{\scs References:
%\fullcite{prasad2008}, pages 34f, Chapter 3
%}}


