%============================================================================
% XeLaTeX File
% Daniel J. Greenhoe
%============================================================================

%=======================================
\chapter{Magnitude characteristics of z-filters}
%=======================================
%=======================================
\section{Pole and zero location analysis}
%=======================================
Note the following:
\\\indentx\begin{tabular}{cll}
    \imark & The frequency response of $\Zh(z)$ \propb{repeats} every $2\pi$.
           & \prefp{prop:dtft_periodic}
  \\\imark & If the coefficients are \propb{real}, &
         \\& then the magnitude response is \propb{symmetric}
           & \prefp{prop:dtft_real}
  \\\imark & Moments and derivatives are related:
           & \prefp{thm:dtft_ddw}
\end{tabular}

The pole zero locations of a digital filter determine the magnitude and 
phase frequency response of the digital filter.\footnote{\citerpp{cadzow}{90}{91}}
This can be seen by representing the pole and zero vectors in the complex z-plane.
Each of these vectors has a magnitude $M$ and a direction $\theta$.
Also, each factor $(z-z_i)$ and $(z-p_i)$ can be represented as vectors as well
(the difference of two vectors).
Each of these factors can be represented by a magnitude/phase factor
$M_ie^{i\theta_i}$.  The overall magnitude and phase of $H(z)$ can then 
be analyzed.

\begin{figure}[ht]
  \centering%
  \includegraphics[height=80mm]{graphics/vecres.pdf}
  \caption{
     Vector response of digital filter \xref{ex:zchar_vector}
     \label{fig:zchar_vector}
     }
\end{figure}
%---------------------------------------
\begin{example}
%---------------------------------------
Take the following filter for example.
\begin{align*}
   H(z) &= \frac{b_0 + b_1z^{-1} + b_2z^{-2} }
                {1   + a_1z^{-1} + a_2z^{-2} }
   \\   &= \frac{(z-z_1)(z-z_2)}
                {(z-p_1)(z-p_2)}
   \\   &= \frac{M_1e^{i\theta_1} \; M_2e^{i\theta_2} \; }
                {M_3e^{i\theta_3} \; M_4e^{i\theta_4} \; }
   \\   &= \left(\frac{M_1M_2}{M_3M_4} \right)
           \left(\frac{e^{i\theta_1} e^{i\theta_2} }
                      {e^{i\theta_3} e^{i\theta_4} }\right)
\end{align*}

This is illustrated in \prefpp{fig:zchar_vector}.
The unit circle represents frequency in the Fourier domain.
The frequency response of a filter is just a rotating vector on this circle.
The magnitude response of the filter is just then a {\em vector sum}.
For example, the magnitude of any $H(z)$ is as follows:

\begin{align*}
   \abs{H(z)} &=& \frac{|(z-z_1)|\;|(z-z_2)|}
                 {|(z-p_1)|\;|(z-p_2)|}
\end{align*}
\end{example}

%=======================================
\section{Coefficient analysis}
%=======================================
%---------------------------------------
\begin{lemma}
\label{lem:aacos}
%---------------------------------------
\lembox{
  \sum_{n=0}^{\xN}\ds\sum_{m=0}^{\xN} a_n a_m e^{-i\omega(n-m)}
    =\sum_{n=0}^{\xN}\abs{a_n}^2 + 2\sum_{n=0}^{\xN}\ds\sum_{m=n+1}^{\xN} \Re\brs{a_n a_m^\ast} \cos\brs{\omega(n-m)}
  }
\end{lemma}

%---------------------------------------
\begin{example}
This example graphically illustrates \prefpp{lem:aacos} for the case $\xN=4$.
%---------------------------------------
\begin{align*}
  \sum_{n=0}^{4}\ds\sum_{m=0}^{4} a_n a_m e^{-i\omega(n-m)}
    &= \brp{\begin{array}{|c||*{5}{>{\ds}c|}}
           \hline          &{\ccg}$m=0$                 &{\ccg}$m=1$                 &{\ccg}$m=2$                 &{\ccg}$m=3$                 &{\ccg}$m=4$
         \\\hline
           \hline\ccg$n=0$ &{\ccc}a_0 a_0^\ast               &      a_0 a_1^\ast e^{+i \omega} &      a_0 a_2^\ast e^{+i2\omega} &      a_0 a_3^\ast e^{+i3\omega} &      a_0 a_4^\ast e^{+i4\omega} 
         \\\hline\ccg$n=1$ &      a_1 a_0^\ast e^{-i \omega} &{\ccc}a_1 a_1^\ast               &      a_1 a_2^\ast e^{+i \omega} &      a_1 a_3^\ast e^{+i2\omega} &      a_1 a_4^\ast e^{+i3\omega} 
         \\\hline\ccg$n=2$ &      a_2 a_0^\ast e^{-i2\omega} &      a_2 a_1^\ast e^{-i \omega} &{\ccc}a_2 a_2^\ast               &      a_2 a_3^\ast e^{+i \omega} &      a_2 a_4^\ast e^{+i2\omega} 
         \\\hline\ccg$n=3$ &      a_3 a_0^\ast e^{-i3\omega} &      a_3 a_1^\ast e^{-i2\omega} &      a_3 a_2^\ast e^{-i \omega} &{\ccc}a_3 a_3^\ast               &      a_3 a_4^\ast e^{+i \omega} 
         \\\hline\ccg$n=4$ &      a_4 a_0^\ast e^{-i4\omega} &      a_4 a_1^\ast e^{-i3\omega} &      a_4 a_2^\ast e^{-i2\omega} &      a_4 a_3^\ast e^{-i \omega} &{\ccc}a_4 a_4^\ast                  
         \\\hline
       \end{array}}
  \\&=  \sum_{n=0}^{4}a_n a_n^\ast
     + 2\sum_{n=0}^{4}\ds\sum_{m=n+1}^{4} \brs{\brp{a_n a_m^\ast e^{i\omega}} + \brp{a_n^\ast a_m e^{-i\omega}}}
  \\&=  \sum_{n=0}^{4}a_n a_n^\ast
     + 2\sum_{n=0}^{4}\ds\sum_{m=n+1}^{4} \brs{\brp{a_n a_m^\ast e^{i\omega}} + \brp{a_n a_m^\ast e^{i\omega}}^\ast}
  \\&=  \sum_{n=0}^{4}a_n a_n^\ast
     +  \sum_{n=0}^{4}\ds\sum_{m=n+1}^{4} 2\Re\brs{\brp{a_n a_m^\ast e^{i\omega}}}
  \\&=  \sum_{n=0}^{4}\abs{a_n}^2 
     + 2\sum_{n=0}^{4}\ds\sum_{m=n+1}^{4}  \Re\brs{a_n a_m^\ast} \Re\brs{e^{i\omega(n-m)}}
  \\&=  \sum_{n=0}^{4}\abs{a_n}^2 
     + 2\sum_{n=0}^{4}\ds\sum_{m=n+1}^{4}  \Re\brs{a_n a_m^\ast} \cos\brs{\omega(n-m)}
\end{align*}
\end{example}

%---------------------------------------
\begin{lemma}
\label{lem:qzcos}
%---------------------------------------
\lembox{
  \brb{\Zq(z) \eqd \sum_{n=0}^{\xN} a_n z^{-n}}
  \quad\implies\quad
  \brb{\abs{\Fq(\omega)}^2
    = \sum_{n=0}^{\xN}\abs{a_n}^2 + 2\sum_{n=0}^{\xN} \sum_{m=n+1}^{\xN} a_n a_m^\ast \cos\brs{\omega(n-m)}
    }
  }
\end{lemma}
\begin{proof}
\begin{align*}
  \boxed{\abs{\Fq(\omega)}^2}
    &= \brs{\abs{\Zq(z)}^2}_{z=e^{i\omega}}
  \\&= \brs{\Zq(z)\Zq^\ast(z)}_{z=e^{i\omega}}
  \\&= \brs{\brp{\sum_{n=0}^{\xN} a_n      z^{-n}}
            \brp{\sum_{n=0}^{\xN} a_n^\ast z^{ n}}}_{z=e^{i\omega}}
  \\&= \brs{\sum_{m=0}^{\xN}\sum_{n=0}^{\xN} a_m a_n^\ast z^{n-m}}_{z=e^{i\omega}}
  \\&= \sum_{m=0}^{\xN}\sum_{n=0}^{\xN} a_m a_n^\ast e^{i\omega(n-m)}
  \\&= \sum_{n=0}^{\xN}a_n^2 + 2\sum_{n=0}^{\xN}\ds\sum_{m=n+1}^{\xN} a_n a_m \cos\brs{\omega(n-m)}
    && \text{by \prefp{lem:aacos}}
\end{align*}
\end{proof}

%---------------------------------------
\begin{theorem}
%---------------------------------------
\thmbox{
  \abs{\Fh(\omega)}^2
    =\frac{\ds\sum_{n=0}^{\xN}b_n^2 + 2\sum_{n=0}^{\xN}\ds\sum_{m=n+1}^{\xN} b_n b_m \cos\brs{\omega(n-m)}}
          {\ds\sum_{n=0}^{\xN}a_n^2 + 2\sum_{n=0}^{\xN}\ds\sum_{m=n+1}^{\xN} a_n a_m \cos\brs{\omega(n-m)}}
  }
\end{theorem}
\begin{proof}
\begin{align*}
  \boxed{\abs{\Fh(\omega)}^2}
    &= \abs{\Zh(z)}^2_{z=e^{i\omega}}
  \\&= \brs{\Zh(z)\Zh^\ast(z)}_{z=e^{i\omega}}
  \\&= \abs{\frac{\ds\sum_{n=0}^{\xN}b_n z^{-n}}{\ds\sum_{n=0}^{\xN}a_n z^{-n}}}^2_{z=e^{i\omega}}  
  \\&=\boxed{\frac{\ds\sum_{n=0}^{\xN}b_n^2 + 2\sum_{n=0}^{\xN}\ds\sum_{m=n+1}^{\xN} b_n b_m \cos\brs{\omega(n-m)}}
                  {\ds\sum_{n=0}^{\xN}a_n^2 + 2\sum_{n=0}^{\xN}\ds\sum_{m=n+1}^{\xN} a_n a_m \cos\brs{\omega(n-m)}}}
    && \text{by \prefp{lem:qzcos}}
\end{align*}
\end{proof}


%---------------------------------------
\begin{theorem}
%---------------------------------------
\thmbox{\ddw\abs{\Fh(\omega)}^2_{\omega=0} = 0}
\end{theorem}
\begin{proof}
\begin{align*}
  \ddw\abs{\Zh(z)}^2_{z=e^{i\omega},\omega=0}  
    &= \ddw\brs{\Zh(z)\Zh^\ast(z)}_{z=e^{i\omega},\omega=0}
  \\&= \ddw\brs{\frac{\ds\sum_{n=0}^{\xN}b_n^2 + 2\sum_{n=0}^{\xN}\ds\sum_{m=n}^{\xN} b_n b_m \cos\brs{\omega(n-m)}}
                 {\ds\sum_{n=0}^{\xN}a_n^2 + 2\sum_{n=0}^{\xN}\ds\sum_{m=n}^{\xN} a_n a_m \cos\brs{\omega(n-m)}}}_{\omega=0}
  \\&\eqd \ddw\brs{\frac{\ff(\omega)}{\fg(\omega)}}_{\omega=0}
  \\&= \brs{\frac{\ff'(\omega)\fg(\omega) - \ff(\omega)\fg'(\omega)}{\fg^2(\omega)}}_{\omega=0}
    \qquad\text{by the \thmb{Quotient Rule}}
  \\&= 0
    \qquad\brp{\begin{array}[t]{MMM}
      because & $\ddw$constant$=0$                   & and \\
              & $\ddw\cos(k\omega)=-\sin(k\omega)=0$ & at $\omega=0,\,\pi$
    \end{array}}
\end{align*}
\end{proof}

