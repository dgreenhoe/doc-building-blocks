%============================================================================
% Daniel J. Greenhoe
% LaTeX file
% Symmlet-p16 wavlet function and coefficients
% nominal xunit = 10mm
% nominal yunit = 10mm
% nominal font size = \scriptsize
%============================================================================
\psset{xunit=0.10\psxunit}%
\psset{yunit=1.75\psyunit}%
\begin{pspicture}(1860,-1)(2035,1.45)%
  %-------------------------------------
  % grid
  %-------------------------------------
  \rput(1870,0){%
    \psset{linecolor=grid,linewidth=0.5pt}%
    \multirput{0}(0, -0.4)(0,0.2){9}{\psline[linestyle=dashed](0,0)(160,0)}% horizontal
    \multirput{0}(10,-0.6)(10,0){15}{\psline[linestyle=dotted](0,0)( 0,2)}% vertical
    \psline(0,0)(160,0)% horizontal x axis
  }%
  %-------------------------------------
  % axes
  %-------------------------------------
  {%
    \psset{gridcolor=grid, subgridcolor=subgrid,subgriddiv=0}%
    \psset{linecolor=axis,linewidth=0.75pt}%
    \psset{showorigin=true,axesstyle=frame,tickstyle=top}%
    \psset{ticks=all, labels=y,   Dx=10,Dy=0.2}%
         %       |           |    |     |________ y-labels increment
         %       |           |    |______________ x-labels increment
         %       |           |___________________ ticks  x/y/all/none
         %       |_______________________________ labels x/y/all/none
   %\psgrid(0,0)(1870,-0.6)(2020,1.4)%
    \psaxes[yAxis=false](0,-0.6)(1870,-0.6)(2030,1.4)% x-axis
    \psaxes[xAxis=false](1870,0)(1870,-0.6)(2030,1.4)% y-axis
  }%                             |_ shift right by 1870 years
  \multido{\n=1870+10}{17}{\rput[t](\n,-0.7){\n}}% x-axis labels
         %       |  |   |               |________ vertical placement
         %       |  |   |________________________ number of labels
         %       |  |____________________________ step size
         %       |_______________________________ starting value
  %-------------------------------------
  % annotation
  % R lubridate::decimal_date(lubridate::ymd("1956-10-17", tz="UTC"))=1956.792
  %-------------------------------------
  {%
  \psset{linewidth=1pt, linestyle=dashed, linecolor=red, showpoints=true, dotstyle=triangle}%
  \psline(1870,-0.6)(1870,0.2)%
  \rput[bl](1870,0.3){Second Industrial Revolution}%
  \psline(1956.792,-0.6)(1956.792,1.1)%
  \rput[bl](1956.792,1.1){First ``full scale" nuclear power station}%
  }%
  %-------------------------------------
  % plot data
  %-------------------------------------
  \psset{linewidth=0.75pt, dotstyle=diamond, showpoints=true}%
  \readdata{\mydata}{../../common/math/graphics/data/global-temp.dat}%
  \dataplot{\mydata}%
 %\listplot{\mydata}%
 %\fileplot{../../common/math/graphics/data/global-temp.dat}%
  %-------------------------------------
  % plot best fit polynomial
  %-------------------------------------
  \psset{linewidth=0.75pt, plotpoints=64, showpoints=false, linecolor=red}%
  \psplot{1870}{2030}{0.00000034336 x 1880 sub 3 exp mul
                      0.000016571   x 1880 sub 2 exp mul add
                     -0.00076398    x 1880 sub       mul add
                     -0.24606                            add
                     }%
\end{pspicture}%