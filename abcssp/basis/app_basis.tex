%============================================================================
% Daniel J. Greenhoe
%============================================================================
\chapter{MRA-wavelet structures}\label{app:mrawav}

\qboxnpqt
  {
    A competition awards committee consisting of the mathematical and scientific giants \citeauthor*{mdf1812jan6} 
    %(at least three of whom are now considered to have been mathematical giants)
    %the mathematical giants 
    %\href{http://en.wikipedia.org/wiki/Joseph_Louis_Lagrange}{Lagrange}, 
    %\href{http://en.wikipedia.org/wiki/Laplace}{Laplace}, 
    %\href{http://en.wikipedia.org/wiki/Legendre}{Legendre}, and others,
    %commenting on 
    %\href{http://en.wikipedia.org/wiki/Joseph_Fourier}{Fourier's} 
    %\href{http://en.wikipedia.org/wiki/1807}{1807} landmark paper
    commenting on \citeauthor{fourier1807}'s landmark \citeyear{fourier1807} paper
    \href{http://gallica.bnf.fr/ark:/12148/bpt6k33707/f220n7}{\hie{Dissertation on the propagation of heat in solid bodies}}
    that introduced the \hie{Fourier Series}.
    As the quote demonstrates, Fourier's essay was quite controversial and Lagrange was 
    successful in effectively blocking its publication.\footnotemark  % see Hawking page 524
  }{../common/people/fourier1807z.jpg}
  %{ \ldots et la nouveaut\'e du sujet, jointe \`a son importance, 
  { \ldots et la nouveaut\'e de l'objet, jointe \`a son importance, 
    a d\'etermin\'e la classe \`a couronner cet ouvrage, 
    en observant cependant que la mani\`ere dont l`auteur parvient 
    \`a ses \'equations n`est pas exempte de difficult\'es, et que son analyse, 
    pour les int\'egrer, laisse encore quelque chose \`a d\'esirer, 
    soit relativement \`a la g\'en\'eralit\'e, soit m\^eme du cot\'e de la rigueur.}
  { \ldots and the innovation of the subject, 
    together with its importance, 
    convinced the committee to crown this work.
    %By observing however that the way in which the author [Fourier] arrives at his equations 
    By observing however that the way in which the author arrives at his equations 
    is not free from difficulties, and the analysis of which, 
    to integrate them, still leaves something to be desired, 
    either relative to generality, or even on the side of rigour.}
  \footnotetext{
%   quote:       & \url{http://www.todayinsci.com/F/Fourier_JBJ/fourglos.htm#commenx} \\
%    quote:       & \url{http://www.todayinsci.com/F/Fourier_JBJ/fourglos.htm} \\
    quote: \citePp{mdf1812jan6}{374},
           \citePp{edf1812jan6}{112},
           \citerpg{kahane2008}{199}{0821844245}.
    translation of quote: assisted by \href{http://translate.google.com/}{Google Translate}, \citerc{castanedo2005}{chapter 2 footnote 5}.
%                 & \url{http://www.theses.ulaval.ca/2005/23016/ch03.html#ftn.N11235} \\
    history: \citeIg{hawking2007fourier}{0762430044}.
  %\begin{tabular}[t]{lp{\tw-31mm}}
%    image: & \url{http://en.wikipedia.org/wiki/Joseph_Fourier}
  paper: \citeP{fourier1807}.
  }


\qboxnpq{
  Joseph Fourier (1768--1830)
  \index{Fourier, Joseph}
  \index{quotes!Fourier, Joseph}
  \footnotemark
  }
  {../common/people/fourier_bunzil_wkp_pdomain_gray.jpg}
  %{Mathematics compares the most diverse phenomena and discovers the secret analogies that unite them.}
  {%
  The analytical equations
  \ldots
  %, unknown to the ancient geometers, which Descartes was the first to introduce into the study of curves and surfaces, 
  %are not restricted to the properties of figures, and to 
  %those properties which are the object of rational mechanics ; they 
  extend to all general phenomena. 
  There cannot be a language more universal and more simple, 
  more free from errors and from obscurities, 
  \ldots
  %that is to say more worthy to express the invariable relations of natural things. 
  
  %Considered from this point of view, 
  mathematical analysis is as extensive as nature itself; 
  \ldots
  %it defines all perceptible relations, measures times, spaces, forces, temperatures ; 
  %this difficult science is formed slowly, but 
  it preserves every principle which it has once acquired; 
  it grows and strengthens itself incessantly in the midst of the many variations and errors of the human mind. 
  Its chief attribute is clearness; it has no marks to express confused notions. 
  It brings together phenomena the most diverse, and discovers the hidden analogies which unite them.
  }
  \footnotetext{
    quote: \citerppc{fourier1822e}{7}{8}{Preliminary Discourse}.
    image: \scs\url{http://en.wikipedia.org/wiki/File:Fourier2.jpg}, public domain.
    }
%
  %============================================================================
% Daniel J. Greenhoe
% LaTeX File
%============================================================================


%======================================
\section{Introduction}
%======================================

%\qboxnpqt
%  { Jules Henri Poincar\'e (1854-1912), physicist and mathematician
%    \index{Poincar\'e, Jules Henri}
%    \index{quotes!Poincar\'e, Jules Henri}
%    \footnotemark
%  }
%  {../common/people/small/poincare.jpg}
%  {\ldots on fait la science avec des faits comme une maison avec des pierres ; 
%   mais une accumulation de faits n'est pas plus une science qu'un tas de 
%   pierres n'est une maison.}
%  {Science is built up of facts, as a house is built of stones;
%   but an accumulation of facts is no more a science than a heap of stones is a house.}
%  \footnotetext{
%    quote:       & \citerc{poincare_sah}{Chapter IX, paragraph 7} \\
%    translation: & \citerp{poincare_sah_eng}{141} \\
%    image:       & \url{http://www-groups.dcs.st-and.ac.uk/~history/PictDisplay/Poincare.html}
%    }
%
%\qboxnps
%  {
%    Freeman Dyson (1923--), physicist and mathematician  %(January 1994)
%    \index{Dyson, Freeman}
%    \index{quotes!Dyson, Freeman}
%    \footnotemark
%  }
%  %{../common/people/dyson_.flickr8168451.jpg}
%  {../common/people/dyson_isepp-org_95-96.jpg}  %http://www.isepp.org/Media/Speaker%20Images/95-96%20Images/dyson.jpg
%  %{../common/people/small/dyson.jpg}
%  {The bottom line for mathematicians is that the architecture has to be right.
%    In all the mathematics that I did, the essential point was to find
%    the right architecture.
%    It's like building a bridge.
%    Once the main lines of the structure are right,
%    then the details miraculously fit.
%    The problem is the overall design.}
%  \footnotetext{
%    quote: & \citerp{dyson1994}{20}  \\
%    %image: & \url{http://www.flickr.com/photos/russnelson/8168451/}
%    image: & \scs\url{http://www.isepp.org/Media/Speaker\%20Images/95-96\%20Images/dyson.jpg}
%    }

%======================================
\subsection{Foundation for bases and transforms}
%======================================
\paragraph{Hilbert spaces.}
A \structe{basis} in a \structe{linear space} $\linearspaceX$ is a 
sequence of \structe{vectors} $\tuplexN{\vpsi_n\in\setX}$ 
such that for any vector $\vx\in\setX$, there exists 
a sequence $\tuplexN{\alpha_n\in\addf}$ such that 
$\vx = \sum_{n=1}^\xN \alpha_n\vpsi_n$.
If the linear space is finite, then any basis in the space is called a \structe{Hamel basis}.
However, finite linear spaces are often insufficient in theory and impractical in practice,
and we therefore would like to find bases in infinite linear spaces. 
In an infinite linear space, a basis will likely be countably or uncountably infinite, so we need a way to 
be able to take sums of infinite numbers of basis vectors.
However an infinite sum such as $\ff(x)=\sum_{n=1}^\infty\alpha_n\psi_n(x)$ has absolutely no meaning 
without the concept of a \ope{limit} because, by definition, 
\quad$\sum_{n=1}^\infty\vx_n \eqd \lim_{\xN\to\infty}\sum_{n=1}^\xN\vx_n$.\quad
And \ope{limit} has no meaning without a topology. %in general has no meaning in an arbitrary linear space. 
%For limit (and convergence) to have meaning, the linear space must also have a topology.
One of the most common methods of inducing a topology on a linear space is to define a norm on that space, 
yielding a \structe{normed linear space}.
A normed linear space in which all \structe{Cauchy sequences} converge is called a \structe{Banach space}.
If the norm in a Banach space induces an inner product, then that space is called a \structe{Hilbert space}.
One very commonly used \structe{Hilber space} is the set $\spLLR$, 
the space of all square integrable functions. 


\begin{figure}[h]
  \centering%
  \gsize%
  \includegraphics{../common/math/graphics/pdfs/opTrn.pdf}\qquad\qquad
  \includegraphics{../common/math/graphics/pdfs/opDil.pdf}
  %%============================================================================
% Daniel J. Greenhoe
% LaTeX file
% Translation operator
%============================================================================
\begin{pspicture}(-2.5,-0.6)(3,2)%
  %-------------------------------------
  % axes
  %-------------------------------------
  \psaxes[linecolor=axis,labels=x,xAxis=false]{->}(0,0)(-2.5,0)(2.5,2)% y axis
  \psaxes[linecolor=axis,labels=x,yAxis=false]{<->}(0,0)(-2.5,0)(2.5,2)% x axis
  \uput[0](2.5,0){$x$}%
  %-------------------------------------
  % functions
  %-------------------------------------
  \psline[linecolor=purple](-2,0)(-1,1)(0,0)% T^{-1}x
  \psline[linecolor=blue]  (-1,0)( 0,1)(1,0)% x
  \psline[linecolor=red]   ( 0,0)( 1,1)(2,0)% T^1 x
  %-------------------------------------
  % labeling
  %-------------------------------------
  \psset{linecolor=black,linewidth=0.5pt,labelsep=1pt}%
  \pnode(0.25,0.75){pfx}\pnode(0.5,1.5){Lfx}%         labeling locations for f(x)
  \pnode(1.25,0.75){pTfx}\pnode(1.75,1.5){LTfx}%       labeling locations for Tf(x)
  \pnode(-1.25,0.75){pTifx}\pnode(-1.75,1.5){LTifx}%  labeling locations for T^{-1}f(x)
  \uput[90](Lfx){$\ff(x)$}%
  \uput[90](LTfx){$\opTrn\ff(x)$}%
  \uput[90](LTifx){$\opTrni\ff(x)$}%
  \psline{->}(Lfx)(pfx)%
  \psline{->}(LTfx)(pTfx)%
  \psline{->}(LTifx)(pTifx)%
\end{pspicture}%
\qquad\qquad
  %%============================================================================
% Daniel J. Greenhoe
% LaTeX file
% Dilation operator
%============================================================================
\begin{pspicture}(-2.5,-0.6)(3,2)%
  %-------------------------------------
  % axes
  %-------------------------------------
  \psset{linewidth=1pt}%
  \psaxes[linecolor=axis,labels=x,xAxis=false]{->}(0,0)(-2.5,0)(2.5,2)% y axis
  \psaxes[linecolor=axis,labels=x,yAxis=false]{<->}(0,0)(-2.5,0)(2.5,2)% x axis
  \uput[0](2.5,0){$x$}%
  %-------------------------------------
  % functions
  %-------------------------------------
  \psline[linecolor=purple](-2,  0)( 0,0.707)(2,  0)% D^{-1}x
  \psline[linecolor=blue]  (-1,  0)( 0,1    )(1,  0)% x
  \psline[linecolor=red]   (-0.5,0)( 0,1.414)(0.5,0)% D x
  %-------------------------------------
  % labeling
  %-------------------------------------
  \psset{linecolor=black,linewidth=0.5pt,labelsep=1pt}%
  \pnode(0.1,1.15){pDfx}\pnode(1.5,1.5){LDfx}%       labeling locations for Df(x)
  \pnode(0.35,0.65){pfx}\pnode(1.5,1){Lfx}%         labeling locations for f(x)
  \pnode(1,0.3535){pDifx}\pnode(1.5,0.5){LDifx}%  labeling locations for D^{-1}f(x)
  \uput[0](Lfx){$\ff(x)$}%
  \uput[0](LDfx){$\opDil\ff(x)$}%
  \uput[0](LDifx){$\opDili\ff(x)$}%
  \psline{->}(Lfx)(pfx)%
  \psline{->}(LDfx)(pDfx)%
  \psline{->}(LDifx)(pDifx)%
\end{pspicture}%

  \caption{\ope{translation operator} $\opTrn$ and \ope{dilation operator} $\opDil$ \label{fig:opTD}}
\end{figure}
\paragraph{Translation and dilation.}
A basis can often be conveniently expressed in terms of a \ope{translation operator} $\opTrn$ and a 
\ope{dilation operator} $\opDil$ as illustrated in \pref{fig:opTD} and defined as follows:\footnote{
  \citerppgc{greenhoe2013wsd}{1}{26}{0983801134}{Chapter 1. Transversal Operators},
  \citerppgc{walnut2002}{79}{80}{0817639624}{Definition 3.39},
  \citerppg{christensen2003}{41}{42}{0817642951},
  \citerpgc{wojtaszczyk1997}{18}{0521578949}{Definitions 2.3,2.4},
  \citerpg{kammler2008}{A-21}{0521883407},
  \citerpg{bachman2000}{473}{0387988998},
  \citerpg{packer2004}{260}{0821834029}, %{section 3.1},
  \citerpg{zayed1996}{2}{0849378516},
  %\citerpg{zayed2004}{}{0817643044},
  \citerpgc{heil2011}{250}{0817646868}{Notation 9.4},
  \citerpg{casazza1998}{74}{0817639594},
  \citerp{goodman1993}{639},
  \citerp{dai1996}{81},
  \citerpg{dai1998}{2}{0821808001}
  %\citerpg{dai1998}{21}{0821808001}
  }
%\label{def:wav_opT}
%\label{def:wav_opD}
%\label{def:wavstrct_TD}
%\label{def:opT}
%\label{def:opD}
%---------------------------------------
\indentx$\begin{array}{Frc>{\ds}lMlC}
    1. & \mc{6}{M}{$\hxs{\opTrn}$ is the \hid{translation operator} on $\spCC$ defined as}\\
       & \indentx\hxs{\opTrn_\tau}  \ff(x) &\eqd&         \ff(x-\tau) &and& \hxs{\opTrn}\eqd\opTrn_1 & \forall \ff\in\spCC
       \\
    2. & \mc{6}{M}{$\hxs{\opDil}$ is the \hid{dilation operator} on $\spCC$ defined as}\\
       & \indentx\hxs{\opDil_\alpha} \ff(x) &\eqd& \ff(\alpha x) &and& \opDil\eqd\sqrt{2}\opDil_2 & \forall \ff\in\spCC
\end{array}$



\paragraph{Analyses and transforms.}
An \ope{analysis}\footnotemark of a vector $\vx$ in a linear space with respect to a basis is the coefficients
of the linear combination of the basis vectors.
The coefficients can reveal much information regarding the characteristics of a vector (e.g. function).
What characteristics are revealed depends on the selected basis.
Therefore, basis design can be both a science and an art.
An \fncte{operator} that computes the analysis of a vector $\vx$ with respect to a basis is called a \ope{transform}.
In a \structe{Hilbert space} in which the basis is \prope{orthogonal}, such a transform is simply the \fncte{inner product} $\inprodn$ of the Hilbert space.
%Therefore in basis theory, ``analysis" and ``transform" mean essentially the same thing.
The partial or complete reconstruction of $\vx$ from the analysis is a \ope{synthesis}.%
        \footnotetext{%
          The word \hie{analysis} comes from the Greek word
          {\fntagreek{>av'alusis}},
          meaning ``dissolution" (\citerpc{perschbacher1990}{23}{entry 359}),
          which in turn means
          ``the resolution or separation into component parts"
          (\citer{collins2009}, \scs\url{http://dictionary.reference.com/browse/dissolution})
          }

\paragraph{Subspace structures.}
All the linear combinations $\Span\setY$ of any set $\setY$ of vectors in a linear space forms another linear space 
called a \structe{subspace}.
And that subspace is said to be the \ope{span} of $\setY$. 
\\
\begin{minipage}{\tw-65mm}%
  Subspaces are ordered with respect to the set relation $\subseteq$ to form a \structe{lattice of subspaces}.
  Therefore, a transform induces such a subspace lattice, 
  and this lattice characterizes the transform.
  Most transforms induce a very simple M-$n$ order structure,
  as illustrated to the right.
  The M-$n$ lattices for $n\ge3$ are \prope{modular} but not \prope{distributive}.
  Analyses typically have one subspace that is a \hie{scaling} subspace;
  and this subspace is often simply a family of constants
  (as is the case with \hi{Fourier Analysis}).\footnotemark
\end{minipage}%
\hfill%
{\begin{minipage}{60mm}%
  %%============================================================================
% Daniel J. Greenhoe
% LaTeX File
%============================================================================
\begin{pspicture}(-3.1,-\latbot)(3.1,2.65)%
  %---------------------------------
  % settings
  %---------------------------------
  \psset{labelsep=5pt}
  %---------------------------------
  % nodes
  %---------------------------------
                        \Cnode( 0,2){X}%    1
  \Cnode(-2,1){V0}\Cnode(-1,1){V1}\Cnode( 0,1){V2}\Cnode( 2,1){Vn1}%
                        \Cnode( 0,0){Z}%    0
  \rput{ 0}(1,1){{\color{blue}\Large$\cdots$}}%
  %---------------------------------
  % node connections
  %---------------------------------
  \ncline{X}{V0}\ncline{X}{V1}\ncline{X}{V2}\ncline{X}{Vn1}%
  \ncline{Z}{V0}\ncline{Z}{V1}\ncline{Z}{V2}\ncline{Z}{Vn1}%
  %---------------------------------
  % node labels
  %---------------------------------
  \uput[ 15](X)  {$\spX$}%
  \uput[180](V0) {$\spV_{0}$}%
  \uput[180](V1) {$\spV_{1}$}%
  \uput[  0](V2) {$\spV_{2}$}%
  \uput[  0](Vn1){$\spV_{n-1}$}%
  \uput[-10](Z)  {$\spZero$}%
  %---------------------------------
  % other labels
  %---------------------------------
  \pnode(2,1.36){analysisP}%
  \rput[ 0] (V2){\psellipse[fillstyle=none,linestyle=dashed,linecolor=red](0,0)(3,0.5)}%
  \rput[tl](-3,2.6){\rnode[b]{scalingL}{scaling subspace}}%
  \rput[tr]( 3,2.6){\rnode[b]{analysisL}{analysis of $\spX$}}%
  \ncline[linecolor=red,linestyle=dotted]{->}{analysisL}{analysisP}%
  \ncline[linecolor=blue,linestyle=dotted]{->}{scalingL}{V0}%
\end{pspicture}%%
  \includegraphics{../common/math/graphics/pdfs/latmn.pdf}%
\end{minipage}}%
\footnotetext{
  \citerpgc{greenhoe2013wsd}{29}{0983801134}{2.2 Analyses}
  }


%======================================
\subsection{Bases for functions on the real line}
%======================================
%---------------------------------------
% Fourier Transform basis
%---------------------------------------
\paragraph{The Fourier Transform.}
In \citeyear{fourier1822}, \citeauthor{fourier1822} introduced the first basis for the Hilbert space $\spLLR$:
%This basis is the \prope{uncountable} set $\set{e^{i\omega}}{\omega\in\R}$.
%The basis is attractive because it's vectors are the \fncte{eigenvectors} for any linear operator on the Hilbert space:
%The \structe{analysis}, or \ope{Fourier Transform}, of a function $\ff(x)\in\spLLR$ is 
%the \ope{projections} of $\ff(x)$ onto this basis as follows:
%  \\\indentx$\ds%
%      \brs{\opFT\ff}(\omega) 
%        \eqd \inprod{\ff(x)}{\opDil_\omega e^{ix}}
%        \eqd\mcom{\frac{1}{\sqrt{2\pi}} \int_\R \ff(x) e^{-i\omega x} \dx}
%                 {\ope{Fourier transform} of $\ff(x)$}
%  $ .\\
\formbox{\begin{array}{Frc>{\ds}lc>{\ds}lCD}
  1. & \mc{6}{M}{$\ds\set{\opDil_\omega e^{ix}}{\scy\omega\in\R}$ is an \prope{orthogonal} and \prope{uncountable} \structe{basis} for $\spLLR$} & (Fourier transform basis)\\
  2. & \ff(x)     &=&    \cft \int_\R \Ff(\omega) \opDil_x e^{i\omega} \dw  & &                                          & \forall\ff\in\spLLR   & (synthesis) where \\
  3. & \Ff(\omega)&\eqd& \cft\inprod{\ff(x)}{\opDil_\omega e^{ix}} &\eqd& \cft \int_\R \ff(x) \opDil_\omega e^{-ix} \dx  & \forall\ff\in\spLLR   & (\ope{Fourier transform}/analysis)
\end{array}}\\
This basis has some properties that makes it extremely special:
  \begin{liste}
    %\item The basis is \prope{orthogonal}.
    \item The exponential is an eigenvalue of any \ope{linear time invariant operator}. % (\prefp{thm:Le=he}).
    \item The exponential generates a continuous point spectrum for the differential operator.    %(\prefp{thm:spec_D})
  \end{liste}
There are also some other properties that make it arguably unattractive:
  \begin{liste}
    \item It's analysis coefficients and basis are \prope{uncountable} rather than \prope{countable} so we can't perfom \ope{reconstruction} of
          a function $\ff(x)$ over a basis sequence $\seqxZp{\psi_n(x)}$ using
          a summation as in $\ff(x)=\sum_{n=1}^\infty \alpha_n\psi_n(x)$.
    \item The basis functions do not have \prope{compact support}, so it is not very efficient to use them to represent
          signals that do have compact support. Compact support basically means that a function ``sits" on a finite
          interval (e.g. $\intcc{0}{2}$), and is $0$ everywhere else.
  \end{liste}
The Fourier transform induces the $M-n$ lattice subspace structure.
Examples of subspaces in a Fourier analysis include $\spV_1=\Span\setn{e^{ix}}$, 
  $\spV_{2.3}=\Span\setn{e^{i2.3x}}$, $\spV_{\sqrt{2}}=\Span\setn{e^{i\sqrt{2}x}}$, etc.

%---------------------------------------
% Haar basis
%---------------------------------------
\begin{figure}[h]
  \gsize%
  \centering%
  %%============================================================================
% Daniel J. Greenhoe
% XeLaTeX file
% nominal unit = 10mm
%============================================================================
\begin{pspicture}(-3.8,-0.5)(3.8,3.5)
  \psset{linewidth=1pt,linecolor=latline,radius=0.75\psunit}%
  %-------------------------------------
  % nodes
  %-------------------------------------
  \rput(0,3){\ovalnode{lub}{$\spX=\oppS\set{\fh(nx)}{n=0,1,2,3}$}}%
  \cnodeput( 0,0){glb}{$\spZero$}%
  \rput[bl](-3.5,-0.3){\rnode[tl]{slabel}{scaling subspace}}%
  \Cnode(-3,1.5){cos0hz}%
  \Cnode(-1,1.5){cos1hz}%
  \Cnode( 1,1.5){cos2hz}%
  \Cnode( 3,1.5){cos3hz}%
  %-------------------------------------
  % connecting lines
  %-------------------------------------
  \ncline{cos0hz}{lub}%
  \ncline{cos1hz}{lub}%
  \ncline{cos2hz}{lub}%
  \ncline{cos3hz}{lub}%
  \ncline{cos0hz}{glb}%
  \ncline{cos1hz}{glb}%
  \ncline{cos2hz}{glb}%
  \ncline{cos3hz}{glb}%
  \ncline{->}{Xlabel}{lub}%
  \ncline{->}{slabel}{cos0hz}%
  %\ncarc[arcangle=30]{->}{slabel}{cos0hz}
  %-------------------------------------
  % plots
  %-------------------------------------
  \rput(cos0hz){\rput(-0.625,0){%
    \psline[linecolor=axis]{->}(0,0)(1.4,0)%
    \psline[linecolor=axis]{<->}(0,-0.7)(0,0.7)%
    \psline[linecolor=blue](0,0)(0,0.212)(1.2,0.212)(1.2,0)%
    }}%
  \rput(cos1hz){\rput(-0.5,0){%
    \psline[linecolor=axis]{->}(0,0)(1.4,0)%
    \psline[linecolor=axis]{<->}(0,-0.7)(0,0.7)%
    \psline[linecolor=red](0,0)(0,0.30)(0.6,0.30)(0.6,-0.30)(1.2,-0.30)(1.2,0)%
    }}%
  \rput(cos2hz){\rput(-0.125,0){%
    \psline[linecolor=axis]{->}(0,0)(0.7,0)%
    \psline[linecolor=axis]{<->}(0,-0.7)(0,0.7)%
    \psline[linecolor=red](0,0)(0,0.42)(0.3,0.42)(0.3,-0.42)(0.6,-0.42)(0.6,0)%
    }}%
  \rput(cos3hz){\rput(-0.125,0){%
    \psline[linecolor=axis]{->}(0,0)(0.7,0)%
    \psline[linecolor=axis]{<->}(0,-0.7)(0,0.7)%
    \psline[linecolor=red](0,0)(0,0.6)(0.15,0.6)(0.15,-0.6)(0.30,-0.6)(0.30,0)%
    }}%
  %-------------------------------------
  % development support
  %-------------------------------------
  %\psgrid(-4,-1.8)(4,1.8)%
\end{pspicture}%%
  \includegraphics{../common/math/graphics/pdfs/baslat_haar.pdf}%
  \caption{Fourier subspace structure \label{fig:baslat_haar}}
\end{figure}
\paragraph{The Haar basis (1909).}
Perhaps the first \prope{countable} basis for $\spLLR$ was introduced by \citeauthor{haar1910} in 
his Ph.D. dissertation (under Hilbert) in July \citeyear{haar1909} and later in a journal article in 1910.
Although not exactly the way Haar expressed it, Haar in essence constructed a basis for $\spLLR$ as follows:
Let\\\indentx$\begin{array}{lMl} 
  \chi_0(x) \eqd \brb{\begin{array}{rM}
                    1  & for $x\in\intco{0}{1}$ \\
                    0  & otherwise.\end{array}}
  &and&
  \chi_1(x) \eqd \brb{\begin{array}{rM}
                    1  & for $x\in\intco{0}{\sfrac{1}{2}}$ \\
                   -1  & for $x\in\intco{\sfrac{1}{2}}{1}$ \\
                    0  & otherwise.\end{array}}
\end{array}$.\\
Then the set 
\quad$\set{\chi_0(x-m)}{m\in\Z}\setu\set{{2}^{(n/2)}\chi_1(2^n x-m)}{m\in\Z,\,n=1,2,3,\ldots}$\\
is a \prope{countable} basis for 
$\spLLR$.\footnote{
  \citer{haar1909},
  \citePpc{haar1910}{361}{Kapitel III. \textsection1. Das orthogonale Funktionensystem $\chi.$},
  \citeIpc{haar1910e}{179}{Chapter III. \textsection1. The Orthogonal Function System $\chi$}
  }

%---------------------------------------
% The Cardinal Series
%---------------------------------------
\paragraph{The Cardinal Series.}
\citefullauthor{whittaker1915} in \citeyear{whittaker1915} introduced the \fncte{cardinal function}
$\frac{\sin\pi x}{\pi x}$.
\citeauthor{hardy1941} in \citeyear{hardy1941} showed that the \prope{countable} set
$\quad\setbigleft{\frac{\sin\left[(\pi/T)(x-nT)\right]}{(\pi/T)(x-nT)}}{n\in\Z,\,\frac{1}{T}\le2\sigma}\quad$
is an \prope{orthonormal} \structe{basis} for the \structe{Paley-Wiener class of functions} $\spPW_\sigma^2$,
where $\spPW_\sigma^2$ is an extremely large class of functions on $\spLLR$ and includes
all the functions $\ff(x)\in\spLLR$ such that for some $C\in\Rp$, $\abs{\ff(x)}\le Ce^{\sigma\abs{x}}$
(note however that this does not include periodic functions).\footnotemark
Thus, the \structe{Cardinal Series} provides a \prope{countable} alternative to the \prope{uncountable} basis of the 
\ope{Fourier Transform}.
%
%Let $\opTrn$ be the \structe{translation operator} \xrefP{def:opT}.
%The \prope{Paley-Wiener} class of functions $\hxs{\spPW_\sigma^2}$ \ifxref{frames}{def:PW}
%are those functions which are ``\hie{bandlimited}"
%with respect to their Fourier transform\ifsxref{harFour}{def:ft}.
%The cardinal series forms an orthogonal basis for such a space\ifsxref{frames}{thm:cardinalSeries}.
The \fncte{Fourier coefficients} \xref{def:fcoef} for a projection of a function $\ff$ onto the Cardinal series basis elements is particularly
simple---these coefficients are samples of $\ff(x)$ taken at regular intervals\ifsxref{frames}{thm:t_sampling}.
In fact, one could represent the coefficients using inner product notation with the
\structe{Dirac delta distribution} $\delta$ \ifxref{relation}{def:dirac} as
follows: %\footnote{see \prefp{sec:cardinal} for more details}
\formbox{\begin{array}{FMCD}
  1. & $\ds\setxZp{\opTrn^n\frac{\sin\brp{\pi x}}{\pi x}}$ is a \structe{basis} for $\spPW_\sigma^2$                  &               & (\structe{Cardinal series basis})\\
  2. & $\ds\ff(x) = \sum_{n=1}^\infty \Ff(n) \opTrn^n\frac{\sin\brp{\pi x}}{\pi x}$ & \forall \ff\in\spPW_\sigma^2,\,\sigma\le \half  & (synthesis) where\\
  3. & $\Ff(n)\eqd\inprod{\ff(x)}{\opTrn^n\delta(x)} \eqd \int_{\R} \ff(x)\delta(x-n) \dx \eqd \ff(n)$                &               & (\ope{Cardinal series transform}/analysis)
\end{array}}
%\citeauthor{hardy1941} gave two proofs.
%The \seqe{coefficients} $\seqxZ{\alpha_n}$ for the expansion of a function $\ff(x)\in\spPW_\sigma^2$ 
%in terms of a \structe{Cardinal series basis}
%was given by \citeauthor{shannon1949} in \citeyear{shannon1948} and \citeyear{shannon1949} and is now
%known as the \thme{Sampling Theorem}. The \thme{Sampling Theorem} says that 
%\\\indentx$\ff(x)=\sum_{n\in\Z} \alpha_n \frac{\sin\left[(\pi/T)(x-nT)\right]}{(\pi/T)(x-nT)}$\\
%where $\seqn{\alpha_n}$ (the samples of $\ff(x)$) are given by projections of $\ff(x)$ onto the sequence
%$\seqxZ{\delta(x-n/T)}$ 
%\\\indentx$\alpha_n \eqd \inprod{\ff(x)}{\delta(x-n/T)} \eqd \int_\R\ff(x)\delta(x-n/T)\dx \eqd \ff(n/T)$.%
\footnotetext{
  \citer{whittaker1915},
  \citerc{hardy1941}{\prope{orthonormality}},
  \citorc{shannon1948}{Theorem 13},
  \citorp{shannon1949}{11},
  \citerpc{higgins1985}{56}{H1.; historical notes},
  \citerpgc{higgins1996}{52}{0198596995}{Definition 6.15}
  }

\paragraph{Short-time Fourier transform.}
\citeauthor{gabor1946} in \citeyear{gabor1946} introduced the \ope{Short-time Fourier transform}.
The \ope{Short-time Fourier transform} of a function $\ff(x)$ is simply the \ope{Fourier transform} 
of $\ff(x)\fw(x)$, where $\fw(x)$ is a \fncte{window function}.
%\paragraph{Gabor Transform}
\footnote{
  \citeP{gabor1946},
  \citergc{qian1996}{0132543842}{Chapter 3},
  \citerpgc{forster2009}{32}{0817648909}{Definition 1.69}
  }
%\begin{array}{Frc>{\ds}lCD}
%  1. & \mc{4}{M}{$\ds\setbigleft{\brp{\opTrn_\tau e^{-\pi x^2}}\brp{\opDil_\omega e^{ix}}}{\scy\tau,\omega\in\R}$ is a \structe{basis} for $\spLLR$}         & and\\
%  2. & \ff(x)                  &=&    \int_\R \fG\opair{\tau}{\omega} \opDil_x e^{i\omega} \dw & \forall x\in\R,\,\ff\in\spLLR   & where\\
%  3. & \fG\opair{\tau}{\omega} &\eqd& \int_\R \ff(x) \brp{\opTrn_\tau e^{-\pi x^2}}\brp{\opDil_\omega e^{-ix}} \dx & \forall x\in\R,\,\ff\in\spLLR
%\end{array}



\begin{figure}[h]
  \gsize%
  \centering%
  %%============================================================================
% Daniel J. Greenhoe
% LaTeX file
% Morlet real-valued wavelet
%============================================================================
\begin{pspicture}(-4.5,-1.5)(4.5,1.5)%
  \psset{%
    linecolor=blue,%
    dotsize=5pt,%
    }%
  \psset{linecolor=blue,linewidth=1pt}%
  \psaxes[linecolor=axis]{<->}(0,0)(-4.5,-1.5)(4.5,1.5)%
  %-------------------------------------
  % envelope factor: e^{-x^2/2}
  %-------------------------------------
  \psplot[plotpoints=100,linecolor=red,linestyle=dotted]{-4.25}{4.25}{2.71828 x x mul neg 2 div exp}% e^{-x^2/2}: exponential envelope
  \psplot[plotpoints=100,linecolor=red,linestyle=dotted]{-4.25}{4.25}{2.71828 x x mul neg 2 div exp neg}% -e^{-x^2/2}: exponential envelope
  %-------------------------------------
  % Morlet real valued wavelet: cos(5x) e^{-x^2/2}
  %-------------------------------------
  \psplot[plotpoints=512]{-4.25}{4.25}{x 360 mul 2 div 3.14159 div 5 mul cos 2.71828 x x mul neg 2 div exp mul}% Morlet real-valued wavelet
  %                                    |-------------------------|           |---------------------------|
  %                                    convert radians to degrees                    e^{-x^2/2}
  %                                    |---------------------------------------------------------------------|
  %                                                            cos(5x) e^{-x^2/2}
  %-------------------------------------
  % labeling
  %-------------------------------------
  \pnode(0.75,0.7548){Penv}% single point on envelope (for labeling)
  \pnode(2,1.2){Lenv}% label for envelope
  \psline[linecolor=red,linewidth=0.75pt]{->}(Lenv)(Penv)%
  \uput{1pt}[30]{0}(Lenv){$\ds e^{-\sfrac{{x^2}}{2}}$}% exponential envelope
  %
  \pnode(1.2566,-0.4540){Pnenv}% single point on envelope (for labeling)
  \pnode(2,-1){Lnenv}% label for envelope
  \psline[linecolor=red,linewidth=0.75pt]{->}(Lnenv)(Pnenv)%
  \uput{1pt}[-45]{0}(Lnenv){$\ds -e^{-\sfrac{{x^2}}{2}}$}% exponential envelope
\end{pspicture}%
  \includegraphics{../common/math/graphics/pdfs/mexhatwavelet.pdf}
  \caption{Ricker wavelet / Marr's wavelet / Mexican Hat wavelet : $\fpsi(x)\eqd (1-x^2)e^{-x^2/2}$ \label{fig:morletwavelet}}
\end{figure}
\paragraph{Ricker wavelet (1953).}
Norman H. \citeauthor{ricker1953} in \citeyear{ricker1953} introduced what today is sometimes referred to as the \fncte{Ricker wavelet} for seismic prospecting.
The \fncte{Ricker wavelet} is also called \fncte{Marr's wavelet} and the \fncte{Mexican Hat wavelet}.\footnotemark
\footnotetext{
  \citeP{ricker1953}
  %\citeP{ricker1952}
  }

\begin{figure}[h]
  \gsize%
  \centering%
  %%============================================================================
% Daniel J. Greenhoe
% LaTeX file
% Morlet real-valued wavelet
%============================================================================
\begin{pspicture}(-4.5,-1.5)(4.5,1.5)%
  \psset{%
    linecolor=blue,%
    dotsize=5pt,%
    }%
  \psset{linecolor=blue,linewidth=1pt}%
  \psaxes[linecolor=axis]{<->}(0,0)(-4.5,-1.5)(4.5,1.5)%
  %-------------------------------------
  % envelope factor: e^{-x^2/2}
  %-------------------------------------
  \psplot[plotpoints=100,linecolor=red,linestyle=dotted]{-4.25}{4.25}{2.71828 x x mul neg 2 div exp}% e^{-x^2/2}: exponential envelope
  \psplot[plotpoints=100,linecolor=red,linestyle=dotted]{-4.25}{4.25}{2.71828 x x mul neg 2 div exp neg}% -e^{-x^2/2}: exponential envelope
  %-------------------------------------
  % Morlet real valued wavelet: cos(5x) e^{-x^2/2}
  %-------------------------------------
  \psplot[plotpoints=512]{-4.25}{4.25}{x 360 mul 2 div 3.14159 div 5 mul cos 2.71828 x x mul neg 2 div exp mul}% Morlet real-valued wavelet
  %                                    |-------------------------|           |---------------------------|
  %                                    convert radians to degrees                    e^{-x^2/2}
  %                                    |---------------------------------------------------------------------|
  %                                                            cos(5x) e^{-x^2/2}
  %-------------------------------------
  % labeling
  %-------------------------------------
  \pnode(0.75,0.7548){Penv}% single point on envelope (for labeling)
  \pnode(2,1.2){Lenv}% label for envelope
  \psline[linecolor=red,linewidth=0.75pt]{->}(Lenv)(Penv)%
  \uput{1pt}[30]{0}(Lenv){$\ds e^{-\sfrac{{x^2}}{2}}$}% exponential envelope
  %
  \pnode(1.2566,-0.4540){Pnenv}% single point on envelope (for labeling)
  \pnode(2,-1){Lnenv}% label for envelope
  \psline[linecolor=red,linewidth=0.75pt]{->}(Lnenv)(Pnenv)%
  \uput{1pt}[-45]{0}(Lnenv){$\ds -e^{-\sfrac{{x^2}}{2}}$}% exponential envelope
\end{pspicture}%
  \includegraphics{../common/math/graphics/pdfs/morletwavelet.pdf}
  \caption{Morlet real-valued wavelet: $\fpsi(x)\eqd \cos(5x) e^{-x^2/2}$ \label{fig:morletwavelet}}
\end{figure}
\paragraph{Morlet real-valued wavelet (1981).}
\citeauthor{morlet1981} in \citeyear{morlet1981} and \citeyear{morlet1983} introduced 
what is now called the \fncte{Morlet wavelet}.
\citefullauthor{grossmann1984} in \citeyear{grossman1984}
demonstrated that the \prope{uncountable} set of dilations and translations of this wavelet form a  basis for $\spLLR$.\footnotemark
\formbox{\begin{array}{Frc>{\ds}lc>{\ds}lCD}
  1. & \mc{6}{M}{$\ds\set{\opDil_\omega\opTrn_t \fpsi(x)}{\scy\omega,t\in\R}$ is an \prope{uncountable} \structe{basis} for $\spLLR$} & (Morlet wavelet basis) where\\
  2. & \fpsi(x) &\eqd& \cos(5x)e^{-x^2/2} & & & \forall x\in\R & (\fncte{Morlet real-valued wavelet})
  %2. & \ff(x)     &=&    c \int_\R \int_\R \Fpsi(t,\omega)\Ff(\omega) \opDil_x e^{i\omega} \dw  & &                                          & \forall\ff\in\spLLR   & (synthesis) where \\
  %3. & \Ff(\omega)&\eqd& \cft\inprod{\ff(x)}{\opDil_\omega e^{ix}} &\eqd& \cft \int_\R \ff(x) \opDil_\omega e^{-ix} \dx  & \forall\ff\in\spLLR   & (\ope{Fourier transform}/analysis)
\end{array}}\\
However there are some properties that make the \structe{Morlet wavelet basis} unattractive:
\begin{liste}
  \item The basis is \prope{uncountable}.
  \item The basis is \prope{non-orthogonal}.
\end{liste}
\footnotetext{
  \citeP{morlet1981},
  \citeP{morlet1983},
  \citerpgc{gao2011h}{27}{1441915451}{``\ldots A Historical Perspective"}
  }

\paragraph{Str/:omberg wavelets (1982).}
\citeauthor{stromberg1982} in \citeyear{stromberg1982} introduced what are now known as \fncte{Str/:omberg wavelets}.\footnotemark
\footnotetext{
  \citeP{stromberg1982},
  \citeI{stromberg1982r}
  }

\paragraph{Continuous Wavelet Transform (1984).}
\citefullauthor{grossmann1984} in \citeyear{grossman1984}
demonstrated that the \prope{uncountable} set of dilations and translations of this wavelet form a  basis for $\spLLR$.
Furthermore, \citefullauthor{grossmann1984} introduced what is now known as the 
\ope{Continuous Wavelet Transform} (\ope{CWT}) which provides an \ope{analysis} of a function $\ff(x)$
in terms of the \fncte{Morlet wavelet} or any other continuous wavelet.
Despite the earlier contributions of \citeauthor{haar1910} \citeyearpar{haar1910}
and \citeauthor{ricker1953} \citeyearpar{ricker1953} of what are now also known as wavelets,
\citefullauthor{grossman1984} introduced the term ``wavelet" to mathematics, and 
their work essentially marks the beginning of \hie{wavelet analysis}.\footnotemark

\paragraph{Dyadic wavelet bases.}
In Wavelet analysis, for some \fncte{mother wavelet} \xref{def:wavelet} $\fpsi(x)$,
  \\\indentx$\ds\spLLR=\linspan\set{\opDil_\omega\opTrn_\tau \fpsi(x)}{\omega,\tau\in\R}$.
However, the ranges of parameters $\omega$ and $\tau$ can be much reduced to the countable set $\Z$ resulting in
a \prope{dyadic} wavelet basis such that for some mother wavelet $\fpsi(x)$,
  \\\indentx$\ds\spLLR=\linspan\set{\opDil^j\opTrn^n \fpsi(x)}{j,n\in\Z}$.
%This text deals almost exclusively with dyadic wavelets. 
%Wavelets that are both \prope{dyadic} and \prope{compactly supported} have the attractive feature 
%that they can be easily implemented in hardware or software by use of the 
%\structe{Fast Wavelet Transform} \xref{fig:fwt}.

\begin{minipage}{\tw-65mm}
  \paragraph{MRA-wavelet bases.}
  In 1989, St{/'e}phane G. Mallat introduced the \structe{Multiresolution Analysis} (MRA, \prefp{def:mra})
  method for wavelet construction. The MRA has become the dominate wavelet construction method.
  This text uses the MRA method extensively, 
  and combines the MRA ``scaling subspaces" \xref{def:mra} with ``wavelet subspaces" \xref{def:seqWn} 
  to form a subspace structure as represented by the 
  \structe{Hasse diagram} to the right.
  The \structe{Fast Wavelet Transform} combines both sets of subspaces as well, 
  providing the results of projections onto both wavelet and MRA subspaces.
  %The MRA is not the only method of wavelet construction,
  Moreover, P.G. Lemari{/'e} has proved that all wavelets with \prope{compact support} are generated by an MRA.\footnotemark
\end{minipage}\hfill%
\begin{minipage}{60mm}
  %%============================================================================
% Daniel J. Greenhoe
% LaTeX file
% wavelet subspace lattice
%============================================================================
\begin{pspicture}(-2.7,-0.4)(2.7,3.6)%
  \fns%
  \psset{
    boxsize=0.40\psunit,
    linearc=0.40\psunit,
    %unit=0.1mm,
    %fillstyle=none,
    % cornersize=relative,
    %framearc=0.5,
    %gridcolor=graph,
    %linewidth=1pt,
    %radius=1.25mm,
    %dotsep=1pt,
    %labelsep=1pt,
    %linecolor=latline,
    }%
  %---------------------------------
  % nodes
  %---------------------------------
  \Cnode( 0,0){Z}%    0
  \Cnode( 0,3){X}%    1
  \Cnode(-1,1){W0}% 
  \Cnode(-2,1){V0}%
  \Cnode( 0,1){W1}%
  \Cnode( 2,1){Wn1}%
  \Cnode(-1.50,1.5){V1}% V1
  \Cnode(-1,2){V2}% V2
  %---------------------------------
  % node labels
  %---------------------------------
  \uput{1.5mm}[ 90](X)  {$\spLLR$}%
  \uput{1.5mm}[135](V2) {$\spV_{2}$}%
  \uput{1.5mm}[135](V1) {$\spV_{1}$}%
  \uput{1.5mm}[180](V0) {$\spV_{0}$}%
  \uput{1.5mm}[180](W0) {$\spW_{0}$}%
  \uput{1.5mm}[  0](W1) {$\spW_{1}$}%
  %\uput{1.5mm}[  0](Wn1){$\spW_{n-1}$}%
  \uput{1.5mm}[-90](Z)  {$\spZero$}%
  %---------------------------------
  % node connections
  %---------------------------------
  \ncline{Z}  {V0}%  0    --> V0
  \ncline{Z}  {W0}%  0    --> W0
  \ncline{Z}  {W1}%  0    --> W1
  \ncline{Z}  {Wn1}% 0    --> W_{n-1}
  \ncline{Wn1}{X}%   Wn-1 --> 1
  \ncline{V0} {V1}%  V0   --> V1
  \ncline{W0} {V1}%  W0   --> V1
  \ncline{V1} {V2}%  V1   --> V2
  \rput{45}(-0.5,2.50){{\color{blue}\Large$\cdots$}}%
  \rput[c]{ 0}(1,1){{\color{blue}\Large$\cdots$}}%
  \ncline{W1}{V2}%   W1 --> V2
  %---------------------------------
  % discriptions
  %---------------------------------
  \ncbox[nodesep=0.25\psunit,linestyle=dotted,linecolor=red]{W0}{Wn1}%
  \ncbox[nodesep=0.25\psunit,linestyle=dotted,linecolor=red]{V0}{X}%
  %\ncbox[nodesep=7pt,linestyle=dotted,linecolor=red]{W0}{Wn1}%
  %\ncbox[nodesep=7pt,linestyle=dotted,linecolor=red]{V0}{X}%
  %\rnode{wavsubbox}  {\ncbox[nodesep=50\psunit,linestyle=dotted,linecolor=red]{W0}{Wn1}}%
  %\rnode{scalesubbox}{\ncbox[nodesep=50\psunit,linestyle=dotted,linecolor=red]{V0}{X}}%
  \pnode[0,-0.40](Wn1){wavsubbox}%  
  \pnode[0,0.60](V2){scalesubbox}%
  %\rput[ 0](  5,10){\psellipse[fillstyle=none,linestyle=dashed,linecolor=red](0,0)(20,5)}%
  %\rput{45}(-12,22){\psellipse[fillstyle=none,linestyle=dashed,linecolor=red](0,0)(25,6)}%
  \rput[br](2.5,1.5){\rnode{wavsublabel}{wavelet subspaces}}%
  %\psline[linecolor=red]{->}(24,24)(20,14)%
  \rput[bl]{45}(-2.4,1.6){\rnode{scalesublabel}{scaling subspaces}}%
 %\psline[linecolor=red]{->}(-15,36)(-15,26)%
  %\ncline[linecolor=red]{->}{wavsublabel}{wavsubbox}%
  %\ncline[linecolor=red]{->}{scalesublabel}{scalesubbox}%
  %---------------------------------
  % debug support
  %---------------------------------
%  \psgrid[unit=\psunit](-30,-10)(30,40)%
  %\psgrid[unit=10\psunit](-3,-1)(3,4)%
\end{pspicture}%%
  \includegraphics{../common/math/graphics/pdfs/latwav.pdf}%
\end{minipage}
\footnotetext{
  \citor{lemarie1990},
  \citerpg{mallat}{240}{012466606X}
  }



%======================================
\subsection{Basis for functions on the unit interval}
%======================================
There are several bases available for functions on the unit interval $\intcc{0}{1}$.
Any such basis can be extended to any subset of the real line by translations and/or dilations.

%---------------------------------------
% Fourier Series basis
%---------------------------------------
\begin{figure}[h]
  \gsize%
  \centering%
  %\input{../common/math/graphics/baslat_cosh.tex}%
  \includegraphics{../common/math/graphics/pdfs/baslat_cosh.pdf}%
  \caption{Fourier subspace structure \label{fig:baslat_cosh}}
\end{figure}
\paragraph{The Fourier Series.}
In \citeyear{fourier1807}, more than 10 years before the \ope{Fourier transform},
\citeauthor{fourier1807} introduced what is now called the \ope{Fourier Series}.
The \ope{Fourier Series} is often used as an orthogonal basis for periodic functions,
but it can just as easily be used as a \prope{countable} \prope{orthogonal} basis for the interval $\intcc{0}{1}$,
The Fourier Series basis is the set
\quad$\set{\funit(x)\cos 2\pi nx,\, \funit(x)\sin 2\pi nx}{n=0,1,2,\ldots}$\quad
where 
\quad$\funit(x)\eqd\brb{\begin{array}{lM}
  1 & for $x\in\intcc{0}{1}$\\
  0 & otherwise
\end{array}}$.\quad
The Fourier Series basis is illustrated in \pref{fig:baslat_cosh}.

The proof of the pointwise convergence of the Fourier Series is notoriously difficult.
Not all functions have a Fourier series that converges to that function.
However \citeauthor{dirichlet1829} in 1829 published some sufficient conditions for convergence
of the continuous portions of a function $\ff:\intcc{a}{b}\to\R$.
\begin{enume}
  \item $\ff$ is \prope{bounded} on $\intcc{a}{b}$.
  \item $\ff$ has only a finite number of maxima and minima on $\intcc{a}{b}$.
  \item $\ff$ has only a finite number of finite discontinuities on $\intcc{a}{b}$.
  \item $\ff$ does not have any infinite discontinuities on $\intcc{a}{b}$.
\end{enume}
It was conjectured in \citeyear{luzin1913} by \citeauthor{luzin1913} that the Fourier Series 
for all square summable periodic functions are pointwise convergent.
This is more general than the Dirichlet conditions.
Fifty-three years later \citeyearpar{carleson1966} at a conference in Moscow, Lennart Axel Edvard \citeauthor{carleson1966} 
presented one of the most spectacular 
results ever in mathematics; he demonstrated that the Luzin conjecture is indeed correct.\footnote{
  Interestingly enough, Carleson started out trying to disprove Luzin's conjecture.
  Carleson said this in an interview published in \citeyear{carleson2001}:
  ``\ldots the problem of course presents itself already when you are a student and I was thinking about the problem on and off, 
  but the situation was more interesting than that. 
  The great authority in those days was Zygmund and he was completely convinced that what one should produce was not a proof 
  but a counter-example. 
  When I was a young student in the United States, 
  I met Zygmund and I had an idea how to produce some very complicated functions for a counter-example 
  and Zygmund encouraged me very much to do so. 
  I was thinking about it for about 15 years on and off, on how to make these counter-examples work 
  and the interesting thing that happened was that I realised why there should be a counter-example and how you should produce it. 
  I thought I really understood what was the background and then to my amazement I could prove that 
  this ``correct" counter-example couldn't exist and I suddenly realised that what you should try to do was the opposite, 
  you should try to prove what was not fashionable, namely to prove convergence. 
  The most important aspect in solving a mathematical problem is the conviction of what is the true result. 
  Then it took 2 or 3 years using the techniques that had been developed during the past 20 years or so."
  References: \citer{carleson2001}, \url{http://www.gap-system.org/~history/Biographies/Carleson.html}.
  }
Carleson formally published his result that same year.
Two years later in \citeyear{hunt1968}, \citeauthor{hunt1968} proved convergence almost everywhere of the Fourier Series
in the spaces $L^p\opair{\intcc{0}{1}}{\C}$, $p>1$.
%In \citeyear{carleson1966}, \citeauthor{carleson1966} proved the following:
%\begin{enume}
%  \item For all $\ff\in C\opair{\intcc{0}{1}}{\R}$, the Fourier Series of $\ff$ converges.
%  \item For all $\fg\in L^2\opair{\intcc{0}{1}}{\R}$, the Fourier Series of $\ff$ converges almost everywhere.
%\end{enume}
\footnote{
  \cite{fourier1807},
  \cite{dirichlet1829},
  \cite{dirichlet1829b},
  \cite{dirichlet1837},
  \cite{dirichlet1837w},
  \cite{luzin1913},
  \cite{carleson1966},
  \cite{hunt1968}.
  Carleson's proof is expounded upon in \citeauthor{reyna2002}'s \citeyear{reyna2002} 175 page book.
  Historical notes about the Dirichlet Conditions are in \citerpg{Bressoud2007}{218}{0883857472}.
  \citerpg{gelbaum1990}{158}{1461209935}.
  }

%---------------------------------------
% Chebyshev polynomial basis
%---------------------------------------
\begin{figure}[h]
  \gsize%
  \centering%
  %%============================================================================
% Daniel J. Greenhoe
% XeLaTeX file
% nominal unit = 10mm
%============================================================================
\begin{pspicture}(-3.8,-0.5)(3.8,3.5)
  %-------------------------------------
  % settings
  %-------------------------------------
  \psset{linewidth=1pt,linecolor=latline,radius=0.75\psunit}%
  %-------------------------------------
  % nodes
  %-------------------------------------
  \rput(0,3){\ovalnode{lub}{$\spX=\oppS\set{T^n(x)}{n=0,1,2,3}$}}%
  %\cnodeput(0,2){lub}{$\spX$}
  %\rput[l](1,2){\rnode{Xlabel}{$\spX=\oppS\set{\cos^n(2\pi x)}{n=0,1,2,3}$}}%
  \rput[bl](-3.5,-0.3){\rnode[tl]{slabel}{scaling subspace}}%
  \Cnode(-3,1.5){cos0hz}%
  \Cnode(-1,1.5){cos1hz}%
  \Cnode( 1,1.5){cos2hz}%
  \Cnode( 3,1.5){cos3hz}%
  \cnodeput( 0,0){glb}{$\spZero$}%
  %-------------------------------------
  % connecting lines
  %-------------------------------------
  \ncline{cos0hz}{lub}%
  \ncline{cos1hz}{lub}%
  \ncline{cos2hz}{lub}%
  \ncline{cos3hz}{lub}%
  \ncline{cos0hz}{glb}%
  \ncline{cos1hz}{glb}%
  \ncline{cos2hz}{glb}%
  \ncline{cos3hz}{glb}%
  \ncline{->}{Xlabel}{lub}
  \ncline{->}{slabel}{cos0hz}
  %\ncarc[arcangle=30]{->}{slabel}{cos0hz}
  %\ncdiag[angleA=120,angleB=-120]{->}{slabel}{cos0hz}
  %-------------------------------------
  % plots
  %-------------------------------------
  \rput(cos0hz){% %(-3, 0){% cos(0x)
    \psline[linecolor=axis]{<->}(-0.7,0)(0.7,0)%
    \psline[linecolor=axis]{<->}(0,-0.7)(0,0.7)%
    \psplot[linecolor=blue,plotpoints=64]{-0.4}{0.4}{1 0.5 mul}
    \psplot[linecolor=blue,plotpoints=32,linestyle=dotted,dotsep=0.05]{0.4}{0.6}{1 0.5 mul}
    \psplot[linecolor=blue,plotpoints=32,linestyle=dotted,dotsep=0.05]{-0.4}{-0.6}{1 0.5 mul}
    }
  \rput(cos1hz){% (-1, 0){% cos(x)
    \psline[linecolor=axis]{<->}(-0.7,0)(0.7,0)%
    \psline[linecolor=axis]{<->}(0,-0.7)(0,0.7)%
    \psplot[linecolor=red,plotpoints=64]{-0.4}{0.4}{x}
    \psplot[linecolor=red,plotpoints=32,linestyle=dotted,dotsep=0.05]{0.4}{0.6}{x}
    \psplot[linecolor=red,plotpoints=32,linestyle=dotted,dotsep=0.05]{-0.4}{-0.6}{x}
    }
  \rput(cos2hz){% ( 1, 0){% cos(2x)
    \psline[linecolor=axis]{<->}(-0.7,0)(0.7,0)%
    \psline[linecolor=axis]{<->}(0,-0.7)(0,0.7)%
    \psplot[linecolor=red,plotpoints=64]{-0.45}{0.45}{2 x 2 mul 2 exp mul 1 sub 0.5 mul}
    \psplot[linecolor=red,plotpoints=32,linestyle=dotted,dotsep=0.05]{0.45}{0.55}{2 x 2 mul 2 exp mul 1 sub 0.5 mul}
    \psplot[linecolor=red,plotpoints=32,linestyle=dotted,dotsep=0.05]{-0.45}{-0.55}{2 x 2 mul 2 exp mul 1 sub 0.5 mul}
    }
  \rput(cos3hz){% ( 3, 0){% cos(3x)
    \psline[linecolor=axis]{<->}(-0.7,0)(0.7,0)%
    \psline[linecolor=axis]{<->}(0,-0.7)(0,0.7)%
    \psplot[linecolor=red,plotpoints=64]{-0.45}{0.45}{4 x 2 mul 3 exp mul 3 x 2 mul mul sub 0.5 mul}
    \psplot[linecolor=red,plotpoints=32,linestyle=dotted,dotsep=0.05]{0.45}{0.5}{4 x 2 mul 3 exp mul 3 x 2 mul mul sub 0.5 mul}
    \psplot[linecolor=red,plotpoints=32,linestyle=dotted,dotsep=0.05]{-0.45}{-0.5}{4 x 2 mul 3 exp mul 3 x 2 mul mul sub 0.5 mul}
    }
\end{pspicture}%
%
  \includegraphics{../common/math/graphics/pdfs/baslat_cheby.pdf}%
  \caption{Chebyshev polynomial subspace structure \label{fig:baslat_cheby}}
\end{figure}
\paragraph{The Chebyshev polynomial basis.}
\citeauthor{chebyshev1854} in \citeyear{chebyshev1854} introduced a polynomial basis for the interval $\intcc{0}{1}$.\footnotemark
By translation and/or dilation, this basis can be extended to the entire real line.
The Chebyshev polynomials induce an $M-n$ subspace lattice structure, as illustrated in \pref{fig:baslat_cheby}.
\footnotetext{
  \citeI{chebyshev1854}, 
  \cittrp{rivlin1974}{4}
  }

%---------------------------------------
% Sinusoidal polynomial basis
%---------------------------------------
\begin{figure}[h]
  \gsize%
  \centering%
  %\input{../common/math/graphics/baslat_cose.tex}%
  \includegraphics{../common/math/graphics/pdfs/baslat_cose.pdf}%
  \caption{Fourier subspace structure \label{fig:baslat_cose}}
\end{figure}
\paragraph{The Cosine polynomials.} ...


%\paragraph{Discrete-Time Fourier Analysis on $\spllR$.}
%\citeauthor{shannon

%%=======================================
%\subsection{Linear Spaces}
%%=======================================
%
%%=======================================
%\subsection{What are wavelets?}
%%=======================================
%In Fourier analysis, \prope{continuous} {dilations} \xref{def:opD} of the \fncte{complex exponential} \xref{def:exp}
%form a  \structe{basis} \xref{def:basis_schauder} for the \structe{space of square integrable functions} $\spLLR$ \xref{def:spLLR} 
%such that
%  \\\indentx$\ds\spLLR=\linspan\set{\opDil_\omega e^{ix}}{\scy\omega\in\R}$.
%
%In Fourier series analysis \xref{thm:opFSi}, \prope{discrete} dilations of the complex exponential 
%form a  basis for $\spLL{\intoo{0}{2\pi}}$ such that
%  \\\indentx$\ds\spLL{\intoo{0}{2\pi}}=\linspan\setjZ{\opDil_j e^{ix}}$.


\if 0
%=======================================
\subsection{Analyses}
%=======================================
The MRA is an \hib{analysis} of the linear space $\spLLR$.
An analysis of a linear space $\spX$ is any sequence $\seq{\spV_j}{j\in\Z}$ of linear subspaces of $\spX$.
%A sequence $\seq{\spV_j}{j\in\Z}$ of linear subspaces of a linear space $\spX$
%        is an \hib{analysis} of $\spX$.
        %if  $\seq{\spV_j}{j\in\Z}$ is a partition of $\spX$.
        The partial or complete reconstruction of $\spX$ from $\seq{\spV_j}{j\in\Z}$ is a \hib{synthesis}.%
        \footnote{%
          The word \hie{analysis} comes from the Greek word
          {\fntagreek{>av'alusis}},
          meaning ``dissolution" (\citerpc{perschbacher1990}{23}{entry 359}),
          which in turn means
          ``the resolution or separation into component parts"
          (\citer{collins2009}, \scs\url{http://dictionary.reference.com/browse/dissolution})
          }
  An analysis is sometimes completely \hie{characterized} by a \hie{transform}.
  For example, a Fourier analysis is a sequence of subspaces with sinusoidal bases.
  Examples of subspaces in a Fourier analysis include $\spV_1=\Span\setn{e^{ix}}$, 
  $\spV_{2.3}=\Span\setn{e^{i2.3x}}$, $\spV_{\sqrt{2}}=\Span\setn{e^{i\sqrt{2}x}}$, etc.
  A \hib{transform} is loosely defined as a function that maps a family of functions
  into an analysis.
  A very useful transform (a ``\hie{Fourier transform}") for Fourier Analysis is \xref{def:opFT}
  \\\indentx$\ds\brs{\opFT\ff}(\omega) \eqd \frac{1}{\sqrt{2\pi}} \int_\R \ff(x) e^{-i\omega x} \dx$


%  \item A sequence $\opT$ in $\clFxy$ is a \hib{transform} \label{item:wavstrct_T}
%        if each element in the sequence is a projection operator in $\clFxy$.
%        An example of a transform is the \hib{cosine transform} $\opT$ in $\clFrr$ such that
%        \begin{align*}
%          \opT\fx(t) &\eqd \seq{\opP_j}{j\in\Z}
%             \\&\eqd \seq{\int_{t\in\R} \fx(t)\,\mcom{\cos(nt)}{kernel} \dt}{n\in\Z}
%             \\&\eqd \seqn{\cdots,\,
%                           %\int_{t\in\R} \cos\brs{(-2)t}\,\fx(t) \dt,\,
%                           \int_{t\in\R} \fx(t)\,\cos\brs{(-1)t} \dt,\,
%                           \int_{t\in\R} \fx(t)\,                \dt,\,
%                           \int_{t\in\R} \fx(t)\,\cos\brs{( 1)t} \dt,\,
%                           %\int_{t\in\R} \cos\brs{(-2)t}\,\fx(t) \dt,\,
%                           \cdots
%                          }
%        \end{align*}
%        Further examples of transforms include the \hie{Fourier Transform} and various \hie{Wavelet Transforms}.
% it is a \hib{sequence} of projection operators on $\A function $\opT$ in $\clFxy$ is a \hib{transform} if with domain $\clFxy$ and range $\clF{\setA}{\setB}$ if

    \begin{minipage}{\tw-65mm}%
      An analysis can be partially characterized by its order structure with respect
      to an order relation such as the set inclusion relation $\subseteq$.
      Most transforms have a very simple M-$n$ order structure,
      as illustrated to the right.
      The M-$n$ lattices for $n\ge3$ are \prope{modular} but not \prope{distributive}.
      Analyses typically have one subspace that is a \hie{scaling} subspace;
      and this subspace is often simply a family of constants
      (as is the case with \hi{Fourier Analysis}).
    \end{minipage}%
    \hfill%
    {\begin{minipage}{60mm}%
    %  \mbox{}\\% force (just above?) top of graphic to be the top of the minipage
      %============================================================================
% Daniel J. Greenhoe
% LaTeX File
%============================================================================
\begin{pspicture}(-3.1,-\latbot)(3.1,2.65)%
  %---------------------------------
  % settings
  %---------------------------------
  \psset{labelsep=5pt}
  %---------------------------------
  % nodes
  %---------------------------------
                        \Cnode( 0,2){X}%    1
  \Cnode(-2,1){V0}\Cnode(-1,1){V1}\Cnode( 0,1){V2}\Cnode( 2,1){Vn1}%
                        \Cnode( 0,0){Z}%    0
  \rput{ 0}(1,1){{\color{blue}\Large$\cdots$}}%
  %---------------------------------
  % node connections
  %---------------------------------
  \ncline{X}{V0}\ncline{X}{V1}\ncline{X}{V2}\ncline{X}{Vn1}%
  \ncline{Z}{V0}\ncline{Z}{V1}\ncline{Z}{V2}\ncline{Z}{Vn1}%
  %---------------------------------
  % node labels
  %---------------------------------
  \uput[ 15](X)  {$\spX$}%
  \uput[180](V0) {$\spV_{0}$}%
  \uput[180](V1) {$\spV_{1}$}%
  \uput[  0](V2) {$\spV_{2}$}%
  \uput[  0](Vn1){$\spV_{n-1}$}%
  \uput[-10](Z)  {$\spZero$}%
  %---------------------------------
  % other labels
  %---------------------------------
  \pnode(2,1.36){analysisP}%
  \rput[ 0] (V2){\psellipse[fillstyle=none,linestyle=dashed,linecolor=red](0,0)(3,0.5)}%
  \rput[tl](-3,2.6){\rnode[b]{scalingL}{scaling subspace}}%
  \rput[tr]( 3,2.6){\rnode[b]{analysisL}{analysis of $\spX$}}%
  \ncline[linecolor=red,linestyle=dotted]{->}{analysisL}{analysisP}%
  \ncline[linecolor=blue,linestyle=dotted]{->}{scalingL}{V0}%
\end{pspicture}%%
    \end{minipage}}%

    \begin{minipage}{\tw-55mm}%
      A special characteristic of wavelet analysis is that there is not just one
      scaling subspace,
      %(as is with the case of Fourier and other analyses),
      but an entire sequence of scaling subspaces.
      These scaling subspaces are \prope{linearly ordered} with respect to the
      ordering relation $\subseteq$. In wavelet theory, this structure is called a \structe{multiresolution analysis},
      or \structe{MRA} \xref{def:mra}.

     The MRA was introduced by St{/'e}phane G. Mallat in 1989.
     The concept of a scaling space was perhaps first introduced by Taizo Iijima in 1959 in Japan,
    and later as the \structe{Gaussian Pyramid} by Burt and Adelson in the 1980s in the West.\footnotemark
    \end{minipage}%
    \footnotetext{%
      \citorp{mallat89}{70}\\
      \citor{iijima1959}\\
      \citor{burt1983}\\
      \citor{adelson1981}\\
      \citer{lindeberg1993}\\
      \citer{alvertez1993}\\
      \citer{guichard2012}\\
      \citerc{weickert1999}{historical survey}
      }
    \hfill%
    {\begin{minipage}{50mm}%
      %\mbox{}\\% force (just above?) top of graphic to be the top of the minipage
      \fns%
      \psset{yunit=0.5\psunit}%
      %============================================================================
% Daniel J. Greenhoe
% LaTeX File
%============================================================================
\begin{pspicture}(-0.75,-0.5)(4,7.5)%
  \psset{%
    labelsep=7pt,
    }
  %---------------------------------
  % nodes
  %---------------------------------
  \rput(0,6){{\large$\vdots$}}% 
  \Cnode*(  0,7){X}%    1
  \Cnode (  0,5){V2}%    V_2
  \Cnode (  0,4){V1}%    V_1
  \Cnode (  0,3){V0}%    V_0
  \Cnode (  0,2){Vn1}%    V_{n-1}
  \Cnode (  0,0) {Z}%    0
  \rput(0,1){{\color{blue}\large$\vdots$}}% 
  %---------------------------------
  % node connections
  %---------------------------------
  \ncline{Vn1}{V0}%
  \ncline{V0}{V1}%
  \ncline{V1}{V2}%
  %---------------------------------
  % node labels
  %---------------------------------
  \uput[180](X){$\spLLR$}%
  \uput[180](V2){$\spV_{2}$}%
  \uput[180](V1){$\spV_{1}$}%
  \uput[180](V0){$\spV_{0}$}%
  \uput[180](Vn1){$\spV_{-1}$}%
  \uput[180](Z){$\spZero$}%
  %---------------------------------
  % other labels
  %---------------------------------
  \rput[l](1.10,7){\rnode{labelentire}{entire linear space}}%
  \rput[c](2,4){\rnode{labellarger}{larger subspaces}}%
  \rput[c](2,3){\rnode{labelsmaller}{smaller subspaces}}%
  \rput[l](1.10,0) {\rnode{labelsmallest}{smallest subspace}}
  \ncline[labelsep=2pt,linecolor=red]{->}{labelentire}  {X}%  
  \psline[linecolor=red]{->}(2,4.20)(2,6.20)%
  \psline[linecolor=red]{->}(2,2.80)(2,0.80)%  
  \ncline[labelsep=2pt,linecolor=red]{->}{labelsmallest}{Z}%  
  %---------------------------------
  % design support
  %---------------------------------
  %\psgrid[unit=100\psunit](-1,-1)(5,8)%
\end{pspicture}%
%
    \end{minipage}}%

    \begin{minipage}{\tw-65mm}%
      A second special characteristic of wavelet analysis is that it's order structure
      with respect to the $\subseteq$ relation is not a simple M-$n$ lattice 
     (as is with the case of Fourier and other analyses).
      Rather, it is a lattice of the form illustrated to the right.
      This lattice is \prope{non-complemented}, \prope{non-distributive},
      \prope{non-modular}, and \prope{non-Boolean} \xref{prop:order_wavstrct}.
    \end{minipage}%
    \hfill%
    {\begin{minipage}{60mm}%
      %\mbox{}\\% force (just above?) top of graphic to be the top of the minipage
      %============================================================================
% Daniel J. Greenhoe
% LaTeX file
% wavelet subspace lattice
%============================================================================
\begin{pspicture}(-2.7,-0.4)(2.7,3.6)%
  \fns%
  \psset{
    boxsize=0.40\psunit,
    linearc=0.40\psunit,
    %unit=0.1mm,
    %fillstyle=none,
    % cornersize=relative,
    %framearc=0.5,
    %gridcolor=graph,
    %linewidth=1pt,
    %radius=1.25mm,
    %dotsep=1pt,
    %labelsep=1pt,
    %linecolor=latline,
    }%
  %---------------------------------
  % nodes
  %---------------------------------
  \Cnode( 0,0){Z}%    0
  \Cnode( 0,3){X}%    1
  \Cnode(-1,1){W0}% 
  \Cnode(-2,1){V0}%
  \Cnode( 0,1){W1}%
  \Cnode( 2,1){Wn1}%
  \Cnode(-1.50,1.5){V1}% V1
  \Cnode(-1,2){V2}% V2
  %---------------------------------
  % node labels
  %---------------------------------
  \uput{1.5mm}[ 90](X)  {$\spLLR$}%
  \uput{1.5mm}[135](V2) {$\spV_{2}$}%
  \uput{1.5mm}[135](V1) {$\spV_{1}$}%
  \uput{1.5mm}[180](V0) {$\spV_{0}$}%
  \uput{1.5mm}[180](W0) {$\spW_{0}$}%
  \uput{1.5mm}[  0](W1) {$\spW_{1}$}%
  %\uput{1.5mm}[  0](Wn1){$\spW_{n-1}$}%
  \uput{1.5mm}[-90](Z)  {$\spZero$}%
  %---------------------------------
  % node connections
  %---------------------------------
  \ncline{Z}  {V0}%  0    --> V0
  \ncline{Z}  {W0}%  0    --> W0
  \ncline{Z}  {W1}%  0    --> W1
  \ncline{Z}  {Wn1}% 0    --> W_{n-1}
  \ncline{Wn1}{X}%   Wn-1 --> 1
  \ncline{V0} {V1}%  V0   --> V1
  \ncline{W0} {V1}%  W0   --> V1
  \ncline{V1} {V2}%  V1   --> V2
  \rput{45}(-0.5,2.50){{\color{blue}\Large$\cdots$}}%
  \rput[c]{ 0}(1,1){{\color{blue}\Large$\cdots$}}%
  \ncline{W1}{V2}%   W1 --> V2
  %---------------------------------
  % discriptions
  %---------------------------------
  \ncbox[nodesep=0.25\psunit,linestyle=dotted,linecolor=red]{W0}{Wn1}%
  \ncbox[nodesep=0.25\psunit,linestyle=dotted,linecolor=red]{V0}{X}%
  %\ncbox[nodesep=7pt,linestyle=dotted,linecolor=red]{W0}{Wn1}%
  %\ncbox[nodesep=7pt,linestyle=dotted,linecolor=red]{V0}{X}%
  %\rnode{wavsubbox}  {\ncbox[nodesep=50\psunit,linestyle=dotted,linecolor=red]{W0}{Wn1}}%
  %\rnode{scalesubbox}{\ncbox[nodesep=50\psunit,linestyle=dotted,linecolor=red]{V0}{X}}%
  \pnode[0,-0.40](Wn1){wavsubbox}%  
  \pnode[0,0.60](V2){scalesubbox}%
  %\rput[ 0](  5,10){\psellipse[fillstyle=none,linestyle=dashed,linecolor=red](0,0)(20,5)}%
  %\rput{45}(-12,22){\psellipse[fillstyle=none,linestyle=dashed,linecolor=red](0,0)(25,6)}%
  \rput[br](2.5,1.5){\rnode{wavsublabel}{wavelet subspaces}}%
  %\psline[linecolor=red]{->}(24,24)(20,14)%
  \rput[bl]{45}(-2.4,1.6){\rnode{scalesublabel}{scaling subspaces}}%
 %\psline[linecolor=red]{->}(-15,36)(-15,26)%
  %\ncline[linecolor=red]{->}{wavsublabel}{wavsubbox}%
  %\ncline[linecolor=red]{->}{scalesublabel}{scalesubbox}%
  %---------------------------------
  % debug support
  %---------------------------------
%  \psgrid[unit=\psunit](-30,-10)(30,40)%
  %\psgrid[unit=10\psunit](-3,-1)(3,4)%
\end{pspicture}%%
    \end{minipage}}%

    \begin{minipage}{\tw-65mm}%
      The wavelet subspace structure is similar in form to that of the \structe{Primorial numbers},\footnotemark
      illustrated to the right by a \hie{Hasse diagram}.
      %In the world of mathematical structures,
      %there is circumstantial evidence that the order structure of wavelet analyses is quite rare,
      %if not outright unique.
      %For example, suppose we replace the wavelet subspaces with prime numbers
      %and the scaling subspaces with their products as illustrated to the right.
      %The resulting sequence $\seqn{1,\,2,\,6,\,30,\,210}$ as of 2011 July 30
      %has no matches in Neil J.A. Sloane's  \emph{Online Encyclopedia of Integer Sequences}
      %(hosted by \emph{AT\&T Research}).\footnotemark
    \end{minipage}%
    \footnotetext{%
      \citeoeis{A002110}%
      }%
    \hfill%
    {\begin{minipage}{60mm}%
      %\mbox{}\\% force (just above?) top of graphic to be the top of the minipage
      \input{../common/math/graphics/latp_1235711.tex}%
    \end{minipage}}%

  An analysis can be represented using three different structures:
    %\paragraph{Equivalence of lattice representations.}
    %So far we have discussed representing a wavelet analysis using three different structures:
\\\begin{tabular}{@{\qquad}ll}
  \circOne    & sequence of subspaces             \\
  \circTwo    & sequence of basis vectors         \\
  \circThree  & sequence of basis coefficients
\end{tabular}\\
These structures are isomorphic to each other, and can therefore be used interchangeably.
%(see \prefp{thm:VPb_isomorphic}).
%(see \prefp{fig:wav_VPb_isomorphic}).
%That is, a ``\hie{wavelet analysis}" can be described using any of these structures.
%However, sometimes when introducing theorems about wavelets,
%it is convenient to use elements from not just one, but from multiple lattices;
%and so it is convenient to have a ``collection" of wavelet analysis elements
%all assembled together into one formally defined tuple.
%\pref{def:wavsys} (next) does just that---it defines a \hie{wavelet analysis} in terms of a tuple with elements
%extracted from the four wavelet structures.


%---------------------------------------
% isomorphic lattices
%---------------------------------------
%\begin{figure}[th]
  \begin{center}%
  \begin{fsL}%
  \begin{minipage}[c]{12\tw/16}
  \begin{minipage}[c]{4\tw/16}%
  \center
  \latmatlw{4}{0.5}
    {
           &       & \null                 \\
           & \null                         \\
     \null &       & \null &       & \null \\
           &       & \null
    }
    {\ncline{1,3}{2,2}\ncline{2,2}{3,1}
     \ncline{1,3}{3,5}
     \ncline{2,2}{3,3}
     \ncline{4,3}{3,1}\ncline{4,3}{3,3}\ncline{4,3}{3,5}
    }
    {\nput{ 90}{1,3}{$\spV_3$}
     \nput{135}{2,2}{$\spV_2$}
     \nput{180}{3,1}{$\spV_1$}
     \nput{ 67}{3,3}{$\spW_1$}
     \nput{  0}{3,5}{$\spW_2$}
     \nput{-90}{4,3}{$\spZero$}
    }
  \end{minipage}
  \hfill{\Large$\thickapprox$}\hfill
  \begin{minipage}[c]{4\tw/16}%
  \center
  \latmatlw{4}{0.5}
    {
           &       & \null                 \\
           & \null                         \\
     \null &       & \null &       & \null \\
           &       & \null
    }
    {\ncline{1,3}{2,2}\ncline{2,2}{3,1}
     \ncline{1,3}{3,5}
     \ncline{2,2}{3,3}
     \ncline{4,3}{3,1}\ncline{4,3}{3,3}\ncline{4,3}{3,5}
    }
    {\nput{ 90}{1,3}{$\seqn{h_n}_3$}
     \nput{135}{2,2}{$\seqn{h_n}_2$}
     \nput{180}{3,1}{$\seqn{h_n}_1$}
     \nput{ 67}{3,3}{$\seqn{g_n}_1$}
     \nput{  0}{3,5}{$\seqn{g_n}_2$}
     \nput{-90}{4,3}{$\opZero$}
    }
  \end{minipage}
  \hfill{\Large$\thickapprox$}\hfill
  \begin{minipage}[c]{4\tw/16}%
  \center
  \latmatlw{4}{0.5}
    {
           &       & \null                 \\
           & \null                         \\
     \null &       & \null &       & \null \\
           &       & \null
    }
    {\ncline{1,3}{2,2}\ncline{2,2}{3,1}
     \ncline{1,3}{3,5}
     \ncline{2,2}{3,3}
     \ncline{4,3}{3,1}\ncline{4,3}{3,3}\ncline{4,3}{3,5}
    }
    {\nput{ 90}{1,3}{$\seqn{\phi_{3,m}}$}
     \nput{135}{2,2}{$\seqn{\phi_{2,m}}$}
     \nput{180}{3,1}{$\seqn{\phi_{1,m}}$}
     \nput{ 67}{3,3}{$\seqn{\psi_{1,m}}$}
     \nput{  0}{3,5}{$\seqn{\psi_{2,m}}$}
     \nput{-90}{4,3}{$\opZero$}
    }
  \end{minipage}
\end{minipage}
\end{fsL}
\end{center}
%\caption{
%  Subspace, coefficient, and basis lattice isomorphisms
%  \label{fig:wav_VPb_isomorphic}
%  }
%\end{figure}

  Here are some examples of the order structures of some analyses,
        including two wavelet analyses:



{\begin{center}%
  \begin{fsL}%
\psset{unit=8mm}%
\begin{longtable}{|c|c|}%
\hline%
\mc{1}{B}{Cosine analysis  (even Fourier series)} & \mc{1}{B}{Cosine polynomial analysis}%
\\%
  \input{../common/math/graphics/baslat_cosh.tex}%
&%
  \input{../common/math/graphics/baslat_cose.tex}%
\\\hline%
\mc{1}{|B|}{Chebyshev polynomial analysis\cittrp{rivlin1974}{4}}&\mc{1}{|B|}{Hadamard-3 analysis}%
\\%
  %============================================================================
% Daniel J. Greenhoe
% XeLaTeX file
% nominal unit = 10mm
%============================================================================
\begin{pspicture}(-3.8,-0.5)(3.8,3.5)
  %-------------------------------------
  % settings
  %-------------------------------------
  \psset{linewidth=1pt,linecolor=latline,radius=0.75\psunit}%
  %-------------------------------------
  % nodes
  %-------------------------------------
  \rput(0,3){\ovalnode{lub}{$\spX=\oppS\set{T^n(x)}{n=0,1,2,3}$}}%
  %\cnodeput(0,2){lub}{$\spX$}
  %\rput[l](1,2){\rnode{Xlabel}{$\spX=\oppS\set{\cos^n(2\pi x)}{n=0,1,2,3}$}}%
  \rput[bl](-3.5,-0.3){\rnode[tl]{slabel}{scaling subspace}}%
  \Cnode(-3,1.5){cos0hz}%
  \Cnode(-1,1.5){cos1hz}%
  \Cnode( 1,1.5){cos2hz}%
  \Cnode( 3,1.5){cos3hz}%
  \cnodeput( 0,0){glb}{$\spZero$}%
  %-------------------------------------
  % connecting lines
  %-------------------------------------
  \ncline{cos0hz}{lub}%
  \ncline{cos1hz}{lub}%
  \ncline{cos2hz}{lub}%
  \ncline{cos3hz}{lub}%
  \ncline{cos0hz}{glb}%
  \ncline{cos1hz}{glb}%
  \ncline{cos2hz}{glb}%
  \ncline{cos3hz}{glb}%
  \ncline{->}{Xlabel}{lub}
  \ncline{->}{slabel}{cos0hz}
  %\ncarc[arcangle=30]{->}{slabel}{cos0hz}
  %\ncdiag[angleA=120,angleB=-120]{->}{slabel}{cos0hz}
  %-------------------------------------
  % plots
  %-------------------------------------
  \rput(cos0hz){% %(-3, 0){% cos(0x)
    \psline[linecolor=axis]{<->}(-0.7,0)(0.7,0)%
    \psline[linecolor=axis]{<->}(0,-0.7)(0,0.7)%
    \psplot[linecolor=blue,plotpoints=64]{-0.4}{0.4}{1 0.5 mul}
    \psplot[linecolor=blue,plotpoints=32,linestyle=dotted,dotsep=0.05]{0.4}{0.6}{1 0.5 mul}
    \psplot[linecolor=blue,plotpoints=32,linestyle=dotted,dotsep=0.05]{-0.4}{-0.6}{1 0.5 mul}
    }
  \rput(cos1hz){% (-1, 0){% cos(x)
    \psline[linecolor=axis]{<->}(-0.7,0)(0.7,0)%
    \psline[linecolor=axis]{<->}(0,-0.7)(0,0.7)%
    \psplot[linecolor=red,plotpoints=64]{-0.4}{0.4}{x}
    \psplot[linecolor=red,plotpoints=32,linestyle=dotted,dotsep=0.05]{0.4}{0.6}{x}
    \psplot[linecolor=red,plotpoints=32,linestyle=dotted,dotsep=0.05]{-0.4}{-0.6}{x}
    }
  \rput(cos2hz){% ( 1, 0){% cos(2x)
    \psline[linecolor=axis]{<->}(-0.7,0)(0.7,0)%
    \psline[linecolor=axis]{<->}(0,-0.7)(0,0.7)%
    \psplot[linecolor=red,plotpoints=64]{-0.45}{0.45}{2 x 2 mul 2 exp mul 1 sub 0.5 mul}
    \psplot[linecolor=red,plotpoints=32,linestyle=dotted,dotsep=0.05]{0.45}{0.55}{2 x 2 mul 2 exp mul 1 sub 0.5 mul}
    \psplot[linecolor=red,plotpoints=32,linestyle=dotted,dotsep=0.05]{-0.45}{-0.55}{2 x 2 mul 2 exp mul 1 sub 0.5 mul}
    }
  \rput(cos3hz){% ( 3, 0){% cos(3x)
    \psline[linecolor=axis]{<->}(-0.7,0)(0.7,0)%
    \psline[linecolor=axis]{<->}(0,-0.7)(0,0.7)%
    \psplot[linecolor=red,plotpoints=64]{-0.45}{0.45}{4 x 2 mul 3 exp mul 3 x 2 mul mul sub 0.5 mul}
    \psplot[linecolor=red,plotpoints=32,linestyle=dotted,dotsep=0.05]{0.45}{0.5}{4 x 2 mul 3 exp mul 3 x 2 mul mul sub 0.5 mul}
    \psplot[linecolor=red,plotpoints=32,linestyle=dotted,dotsep=0.05]{-0.45}{-0.5}{4 x 2 mul 3 exp mul 3 x 2 mul mul sub 0.5 mul}
    }
\end{pspicture}%
%
&%
  %============================================================================
% Daniel J. Greenhoe
% XeLaTeX file
% Hadamard matrix
% 
% H1 = [1]
% 
% H2 = |H1 H1 | = |1  1|
%      |H1 H1^|   |1 -1|
%
%                 |1  1   1   1|
% H3 = |H2 H2 | = |1 -1   1  -1|
%      |H2 H2^|   |1  1  -1  -1|
%                 |1 -1  -1   1|
%============================================================================
\begin{pspicture}(-4,-2)(4,2)%
  \psset{%
    linewidth=1pt,%
    linecolor=latline,%
    radius=0.75\psunit,%
    dotsize=5pt,%
    }%
  %-------------------------------------
  % nodes
  %-------------------------------------
  \rput(0,1.5){\ovalnode{lub}{$\spX=\oppS H_3$}}%
  %\cnodeput(0,2){lub}{$\spX$}
  %\rput[l](1,2){\rnode{Xlabel}{\footnotesize$\spX=\oppS\set{\cos(2\pi n x)}{n=0,1,2,3}$}}%
  \cnodeput( 0,-1.5){glb}{$\spZero$}%
  \rput[bl](-4,-1.8){\rnode[t]{slabel}{scaling subspace}}%
  \Cnode(-3, 0){h1111}%
  \Cnode(-1, 0){h11nn}%
  \Cnode( 1, 0){h1nn1}%
  \Cnode( 3, 0){h1n1n}%
  %-------------------------------------
  % connecting lines
  %-------------------------------------
  \ncline{h1111}{lub}%
  \ncline{h1n1n}{lub}%
  \ncline{h11nn}{lub}%
  \ncline{h1nn1}{lub}%
  \ncline{h1111}{glb}%
  \ncline{h1n1n}{glb}%
  \ncline{h11nn}{glb}%
  \ncline{h1nn1}{glb}%
  \ncline{->}{Xlabel}{lub}%
  \ncline{->}{slabel}{h1111}%
  %\ncarc[arcangle=30]{->}{slabel}{h1111}
  %-------------------------------------
  % plots
  %-------------------------------------
  \psset{yunit=3.5mm,xunit=2.75mm}%
  \rput(h1111){\begin{pspicture}(-1,-2)(4,2)%
      \psset{linecolor=blue}%
      \psline[linecolor=axis]{->}(0,0)(4,0)%
      \psline[linecolor=axis]{<->}(0,-1.75)(0,1.75)%
      \psline{-o}(0,0)(0, 1)%  1
      \psline{-o}(1,0)(1, 1)%  1
      \psline{-o}(2,0)(2, 1)%  1
      \psline{-o}(3,0)(3, 1)%  1
    \end{pspicture}}%
  \rput(h1n1n){\begin{pspicture}(-1,-2)(4,2)%
      \psset{linecolor=red}%
      \psline[linecolor=axis]{->}(0,0)(4,0)%
      \psline[linecolor=axis]{<->}(0,-1.75)(0,1.75)%
      \psline{-o}(0,0)(0, 1)%  1
      \psline{-o}(1,0)(1,-1)% -1
      \psline{-o}(2,0)(2, 1)%  1
      \psline{-o}(3,0)(3,-1)% -1
    \end{pspicture}}%
  \rput(h11nn){\begin{pspicture}(-1,-2)(4,2)%
      \psset{linecolor=red}%
      \psline[linecolor=axis]{->}(0,0)(4,0)%
      \psline[linecolor=axis]{<->}(0,-1.75)(0,1.75)%
      \psline{-o}(0,0)(0, 1)%  1
      \psline{-o}(1,0)(1, 1)%  1
      \psline{-o}(2,0)(2,-1)% -1
      \psline{-o}(3,0)(3,-1)% -1
    \end{pspicture}}%
  \rput(h1nn1){\begin{pspicture}(-1,-2)(4,2)%
      \psset{linecolor=red}%
      \psline[linecolor=axis]{->}(0,0)(4,0)%
      \psline[linecolor=axis]{<->}(0,-1.75)(0,1.75)%
      \psline{-o}(0,0)(0, 1)%  1
      \psline{-o}(1,0)(1,-1)% -1
      \psline{-o}(2,0)(2,-1)% -1
      \psline{-o}(3,0)(3, 1)%  1
      %\psgrid(-4,-1.8)(4,1.8)%
    \end{pspicture}}%
\end{pspicture}%%
\\\hline%
\mc{1}{|B|}{Haar/Daubechies-$p1$ wavelet analysis} & \mc{1}{B|}{Daubechies-$p2$ wavelet analysis}%
\\%
  \input{../common/math/graphics/baslat_d1.tex}%
&%
  \input{../common/math/graphics/baslat_d2.tex}%
\\\hline%
\end{longtable}%
  \end{fsL}%
\end{center}}%


%=======================================
\subsection{Multiresolution analysis}
%=======================================
%=======================================
\subsubsection{Definition}
%=======================================
A multiresolution analysis provides ``coarse" approximations of a function in a linear space $\spLLR$ at multiple
``scales" or ``resolutions".
%\paragraph{Scaling function.}
Key to this process is a sequence of \hie{scaling functions}.
Most traditional transforms feature a single \hie{scaling function} $\fphi(x)$
set equal to one ($\fphi(x)=1$).
This allows for convenient representation of the most basic functions, such as constants.\citep{jawerth}{8}
A multiresolution system, on the other hand, uses a generalized form of the scaling concept:
\begin{dingautolist}{"AC}
  \item Instead of the scaling function simply being set \emph{equal to unity} ($\fphi(x)=1$),
        a multiresolution system \xref{def:mrasys} is often constructed in such a way that the scaling function 
        $\fphi(x)$ forms a \hie{partition of unity} \xref{def:pun} such that
        $\sum_{n\in\Z} \opTrn^n\fphi(x) = 1$.
  \item Instead of there being \emph{just one} scaling function, there
        is an entire sequence of scaling functions $\seqjZ{\opDil^j\fphi(x)}$, 
        each corresponding to a different ``\hie{resolution}".
\end{dingautolist}

%--------------------------------------
\begin{definition}% [multiresolution system]
\label{def:seqVn}
\label{def:mra}
\label{def:wavstrct_phi}
\footnote{
  \citerpg{hernandez1996}{44}{0849382742}\\
  \citerpgc{mallat}{221}{012466606X}{Definition 7.1} \\
  \citorp{mallat89}{70}\\
  \citorpgc{meyer1992}{21}{0521458692}{Definition 2.2.1}\\
  \citerpgc{christensen2003}{284}{0817642951}{Definition 13.1.1}\\
  \citerppgc{bachman2000}{451}{452}{0387988998}{Definition 7.7.6}\\
  \citerppgc{walnut2002}{300}{301}{0817639624}{Definition 10.16}\\
 %\citerppgc{vidakovic}{51}{52}{0471293652}{Riesz basis: footnote on page 52}\\
  \citerppgc{dau}{129}{140}{0898712742}{Riesz basis: page 139}\\
  %\citerppgc{christensen2003}{73}{74}{0817642951}{Definition 3.8.2}\\
  %\citerpgc{heil2011}{371}{0817646868}{Definition 12.8}\\
  %\citerpgc{walter}{38}{1584882271}{3.1 Multiresolution Analysis}
  }
%--------------------------------------
%Let $\spLLR$ be the space of all \structe{square Lebesgue integrable functions} \xref{def:spLLR}.
Let $\seqjZ{\spV_j}$ be a sequence of subspaces on $\spLLR$ \xref{def:spLLR}.  %be a \prope{separable} \structe{Hilbert space}.
Let $\clsA$ be the \structe{closure} of a set $\setA$.
\\\defboxt{
  The sequence $\seqjZ{\spV_j}$ is a \hid{multiresolution analysis} on $\spLLR$ if
  \\
  $\begin{array}{@{\qquad}F>{\ds}lCDD}
   %1. & \spV_j \text{ is a linear subspace of $\spX$}\qquad \forall \spV_j\in\seqjZ{\spV_j}
    %\cnto & \mc{2}{M}{$\spLLR$ is \prope{complete}}                       & ($\spLLR$ is a \structe{Hilbert Space})         & and
    %\cntn & \mc{2}{M}{$\spLLR$ is \prope{separable}}                      &                                               & and
    \cnto & \spV_j          = \cls{\spV_j}              & \forall j\in\Z                 & (\prope{closed})                              & and 
    \cntn & \spV_j          \subset \spV_{j+1}          & \forall j\in\Z                 & (\prope{linearly ordered})                    & and 
    \cntn & \clsp{\Setu_{j\in\Z} \spV_j} = \spLLR         &                                & (\prope{dense} in $\spLLR$)    & and 
   %\cntn & \Seti_{j\in\Z} \spV_j = \setn{\vzero}       &                                & (\structe{greatest lower bound} is $\spZero$) & and 
    \cntn & \ff\in\spV_j \iff    \opDil\ff\in\spV_{j+1} & \forall j\in\Z,\,\ff\in\spLLR  & (\prope{self-similar})                        & and
   %\cntn & \ff\in\spV_j \iff    \opTrn\ff\in\spV_j     & \forall n\in\Z,\,\ff\in\spLLR  & (\prope{translation invariant})               & and 
    \cntn & \mc{3}{l}{\ds\exists \fphi \st \setxZ{\opTrn^n\fphi} \text{ is a \structe{Riesz basis} for $\spV_0$.}}                       & 
  \end{array}$
  \\
  A \structe{multiresolution analysis} is also called an \hid{MRA}.\\
  An element $\spV_j$ of $\seqjZ{\spV_j}$ is a \hid{scaling subspace} of the space $\spLLR$.\\
  The pair $\hxs{\MRAspaceLLRV}$ is a \hid{multiresolution analysis space}, or \hid{MRA space}.\\
  The function $\hxs{\fphi}$ is the \hid{scaling function} of the \structe{MRA space}.
  }
\end{definition}

The traditional definition of the \structe{MRA} also includes the following:
  \\\indentx$\begin{array}{F>{\ds}lCD}
      \cntn & \ff\in\spV_j \iff    \opTrn^n\ff\in\spV_j     & \forall n,j\in\Z,\,\ff\in\spLLR  & (\prope{translation invariant})
      \cntn & \Seti_{j\in\Z} \spV_j = \setn{\vzero}         &                                  & (\structe{greatest lower bound} is $\spZero$)
  \end{array}$\\
However, \pref{prop:mra_transinvar} (next) and \prefpp{prop:mra_glb} demonstrate that
these follow from the \structe{MRA} as defined in \pref{def:mra}.

%--------------------------------------
\begin{proposition}
\footnote{
  \citerpgc{hernandez1996}{45}{0849382742}{Theorem 1.6}
  }
\label{prop:mra_transinvar}
%--------------------------------------
Let \structe{MRA} be defined as in \prefp{def:mra}.
\propbox{
  \brb{\text{$\seqjZ{\spV_j}$ is an \structe{MRA}}}
  \qquad\implies\qquad
  \mcom{\brb{\begin{array}{>{\ds}lC}
    \ff\in\spV_j \iff    \opTrn^n\ff\in\spV_j     & \forall n,j\in\Z,\,\ff\in\spLLR  %& (\prope{translation invariant})
  \end{array}}}{\prope{translation invariant}}
  }
\end{proposition}
\begin{proof}
\begin{align*}
  \opTrn^n\ff\in\spV_j
    &\iff \opTrn^n\ff\in\linspan\set{\opDil^j\opTrn^m\fphi}{\scy m\in\Z}
          &&
          && \text{by definition of $\setn{\fphi}$ \xref{def:mra}}
  \\&\iff \exists \seqxZ{\alpha_n} \st \opTrn^n\ff(x)
          &&= \sum_{k\in\Z}\alpha_k\opDil^j\opTrn^k\fphi(x)
          && \text{by definition of $\setn{\fphi}$ \xref{def:mra}}
  \\&\iff \exists \seqxZ{\alpha_n} \st \ff(x)
          &&= \opTrn^{-n}\sum_{k\in\Z}\alpha_k\opDil^j\opTrn^k\fphi(x)
          && \text{by definition of $\opTrn$ \xref{def:opT}}
  \\&     &&= \sum_{k\in\Z}\alpha_k\opTrn^{-n}\opDil^j\opTrn^k\fphi(x)
          %&& \text{by \prefp{prop:opTD_sum}}
  \\&     &&= \sum_{k\in\Z}\alpha_k\opDil^j\opTrn^{k-2n}\fphi(x)
          && \text{by \thme{commutator relation} \xref{prop:DTTD}}
  \\&     &&= \sum_{\ell\in\Z}\alpha_{\ell+2n}\opDil^j\opTrn^{\ell}\fphi(x)
          && \text{where $\ell\eqd k-2n\implies k=\ell+2n$}
  \\&     &&= \sum_{\ell\in\Z}\beta_{\ell}\opDil^j\opTrn^{\ell}\fphi(x)
          && \text{where $\beta_{\ell}\eqd\alpha_{\ell+2n}$}
  \\&     &&\iff \quad\ff\in\spV_j
          && \text{by definition of $\setn{\opTrn^n\fphi}$ \xref{def:mra}}
\end{align*}
\end{proof}

%--------------------------------------
\begin{proposition}
\footnote{
  \citerppgc{wojtaszczyk1997}{19}{28}{0521578949}{Proposition 2.14}\\
  \citerpgc{hernandez1996}{45}{0849382742}{Theorem 1.6}\\
  \citerppgc{pinsky2002}{313}{314}{0534376606}{Lemma 6.4.28}
  }
\label{prop:mra_glb}
%--------------------------------------
Let \structe{MRA} be defined as in \prefp{def:mra}.
\propbox{
  \brb{\text{$\seqjZ{\spV_j}$ is an \structe{MRA}}}
  \qquad\implies\qquad
  \brb{\begin{array}{>{\ds}lCDD}
     \Seti_{j\in\Z} \spV_j = \setn{\vzero}       &                                & (\structe{greatest lower bound} is $\spZero$) &  \\
    %2. & \ff\in\spV_j \iff    \opTrn\ff\in\spV_j     & \forall j\in\Z,\,\ff\in\spLLR  & (\prope{translation invariant})               &
  \end{array}}
  }
\end{proposition}
\begin{proof}
\begin{enumerate}
  \item Let $\opP_j$ be the \ope{projection operator} that generates the scaling subspace $\spV_j$ such that \label{idef:mra_glb_Pj}
    \\\indentx$\ds\opV_j=\set{\opP_j\ff}{\ff\in\spLLR}$

  \item lemma: Functions with \prope{compact support} are \prope{dense} in $\spLLR$.
        \label{ilem:mra_glb_dense}
        Therefore, we only need to prove that the proposition is true for functions with support in $\intcc{-R}{R}$, for all $R>0$.

  \item For some function $\ff\in\spLLR$, let $\seqxZ{\ff_n}$ be a sequence of functions in $\spLLR$ 
        with \prope{compact support} such that
        \\\indentx
        $\support\ff_n\subseteq\intcc{-R}{R}$ for some $R>0$
        \quad and\quad
        $\ds\ff(x)=\lim_{n\to\infty}\seqn{\ff_n(x)}$.
        \label{idef:mra_glb_ffn}

  \item lemma: $\ds\Seti\spV_j=\setn{\vzero}\quad\iff\quad\lim_{j\to-\infty}\norm{\opP_j\ff}=0\quad{\scy\forall\ff\in\spLLR}$. Proof:  \label{ilem:mra_glb_VjPj}
    \begin{align*}
      \Seti_{j\in\Z}\spV_j 
        &= \Seti_{j\in\Z}\set{\opP_j\ff}{\ff\in\spLLR}
        && \text{by definition of $\spV_j$ \xref{idef:mra_glb_Pj}}
      \\&= \lim_{j\to-\infty}\set{\opP_j\ff}{\ff\in\spLLR}
        && \text{by definition of $\seti$}
      \\&= \vzero
      \iff \lim_{j\to-\infty}\norm{\opP_j\ff}=0
        && \text{by \prope{nondegenerate} property of $\normn$ \xref{def:norm}}
    \end{align*}
  
  \item lemma: $\ds\lim_{j\to-\infty}\norm{\opP_j\ff}=0\quad{\scy\forall\ff\in\spLLR}$. Proof:\\
        Let $\setindAx$ be the \fncte{set indicator function} \xref{def:setind} \label{ilem:mra_glb_norm}
    \begin{align*}
      &\lim_{j\to-\infty}\norm{\opP_j\ff}^2
      \\&=   \lim_{j\to-\infty}\norm{\opP_j\lim_{n\to\infty}\seqn{\ff_n}}^2
        &&   \text{by \prefp{idef:mra_glb_ffn}}
      \\&\le \lim_{j\to-\infty}B\sum_{n\in\Z}\abs{\inprod{\opP_j\lim_{n\to\infty}\seqn{\ff_n}}{\opDil^j\opTrn^n\fphi}}^2
        &&   \text{by \prope{frame property} \xref{prop:rbasis_frame}}
      \\&=   \lim_{j\to-\infty}B\sum_{n\in\Z}\abs{\inprod{\lim_{n\to\infty}\seqn{\ff_n}}{\opDil^j\opTrn^n\fphi}}^2
        &&   \text{by definition of $\opP_j$ \xref{idef:mra_glb_Pj}}
      \\&=   \lim_{j\to-\infty}B\sum_{n\in\Z}\abs{\inprod{\setind_\intcc{-R}{R}(x)\lim_{n\to\infty}\seqn{\ff_n}}{\opDil^j\opTrn^n\fphi(x)}}^2
        &&   \text{by definition of $\seqn{\ff_n}$ \xref{idef:mra_glb_ffn}}
      \\&=   \lim_{j\to-\infty}B\sum_{n\in\Z}\abs{\inprod{\lim_{n\to\infty}\seqn{\ff_n}}{\setind_\intcc{-R}{R}(x)\opDil^j\opTrn^n\fphi(x)}}^2
        &&   \text{prop. of $\inprodn$ in $\spLLR$ \xref{def:spLLR}}
      \\&\le \lim_{j\to-\infty}B\sum_{n\in\Z}\norm{\lim_{n\to\infty}\seqn{\ff_n}}^2\norm{\setind_\intcc{-R}{R}(x)\opDil^j\opTrn^n\fphi(x)}^2
        &&   \text{by \thme{Cauchy-Schwarz Inequality}\ifsxref{vsinprod}{thm:cs}}
      \\&=   \lim_{j\to-\infty}B\sum_{n\in\Z}\norm{\ff}^2\norm{\setind_\intcc{-R}{R}(x)\opDil^j\opTrn^n\fphi(x)}^2
        &&   \text{by definition of $\seqn{\ff_n}$ \xref{idef:mra_glb_ffn}}
      \\&=   \lim_{j\to-\infty}B\sum_{n\in\Z}\norm{\ff}^2\norm{\brs{\mcom{\opDil^j\opDil^{-j}}{$\opI$}\setind_\intcc{-R}{R}(x)}\brs{\opDil^j\opTrn^n\fphi(x)}}^2
        &&   \text{by \prefp{prop:opDi}}
      \\&=   \lim_{j\to-\infty}B\sum_{n\in\Z}\norm{\ff}^2\norm{2^{j/2}\opDil^j\brb{\brs{\opDil^{-j}\setind_\intcc{-R}{R}(x)}\brs{\opTrn^n\fphi(x)}}}^2
        &&   \text{by \prefp{prop:DjTnfg}}
      \\&=   \lim_{j\to-\infty}B\sum_{n\in\Z}\norm{\ff}^2\norm{\opDil^j\brb{2^{j/2}2^{-j/2}\setind_\intcc{-R}{R}(2^{-j}x)\brs{\opTrn^n\fphi(x)}}}^2
        &&   \text{by \prefp{prop:opDi}}
      \\&=   \lim_{j\to-\infty}B\sum_{n\in\Z}\norm{\ff}^2\norm{\opDil^j\brb{\brs{\mcom{\opTrn^n\opTrn^{-n}}{$\opI$}\setind_\intcc{-R}{R}(2^{-j}x)}\brs{\opTrn^n\fphi(x)}}}^2
        &&   \text{by \prefp{prop:opTi}}
      \\&=   \lim_{j\to-\infty}B\sum_{n\in\Z}\norm{\ff}^2\norm{\opDil^j\brb{\brs{\opTrn^n\setind_\intcc{-R}{R}(2^{-j}x+n)}\brs{\opTrn^n\fphi(x)}}}^2
        &&   \text{by \prefp{prop:opTi}}
      \\&=   \lim_{j\to-\infty}B\sum_{n\in\Z}\norm{\ff}^2\norm{\opDil^j\opTrn^n\brb{\setind_\intcc{-R}{R}(2^{-j}x+n)\fphi(x)}}^2
        &&   \text{by \prefp{prop:opDi}}
      \\&=   \lim_{j\to-\infty}B\sum_{n\in\Z}\norm{\ff}^2\norm{\setind_\intcc{-R}{R}(2^{-j}x+n)\fphi(x)}^2
        &&   \text{by \prefp{thm:TD_unitary}}
      \\&=   B\norm{\ff}^2\sum_{n\in\Z}\lim_{j\to-\infty}\norm{\setind_\intcc{-2^jR+n}{2^jR+n}(u)\fphi(2^{-j}(u-n))}^2
        &&   \text{$u\eqd 2^jx+n\implies x=2^{-j}(u-n)$}
      \\&=   B\norm{\ff}^2\sum_{n\in\Z}\lim_{j\to-\infty}\int_{-2^jR+n}^{2^jR+n}\abs{\fphi(2^{-j}(u-n))}^2\du
      \\&=   B\norm{\ff}^2\sum_{n\in\Z}\int_{n}^{n}\abs{\fphi(0)}^2\du
      \\&=   0
    \end{align*}

  \item Final step in proof that $\ds\Seti\spV_j=\setn{\vzero}$: by \prefp{ilem:mra_glb_VjPj} and \prefp{ilem:mra_glb_norm}
\end{enumerate}
\end{proof}



%---------------------------------------
\begin{proposition}
\footnote{
  \citerppgc{wojtaszczyk1997}{28}{31}{0521578949}{Proposition 2.15}
  }
\label{prop:mra_UVj}
%--------------------------------------
%Let $\spO\eqd\mrasys$.
Let a \structe{Riesz sequence} be defined as in \prefp{def:rieszseq}.
\propbox{
  \brb{\begin{array}{FMD}
    (1). & $\seqn{\opTrn^n\fphi}$ is a \structe{Riesz sequence} & and \\
    (2). & $\Fphi(\omega)$ is \prope{continuous} at $0$ & and \\
    (3). & $\Fphi(0)\neq0$
  \end{array}}
  \implies
  \brb{\begin{array}{>{\ds}lD}
     \cls{\brp{\Setu_{j\in\Z} \spV_j}} = \spLLR  & (\prope{dense} in $\spLLR$) 
  \end{array}}
  }
\end{proposition}
\begin{proofns}
\begin{enumerate}
  \item Let $\opP_j$ be the \ope{projection operator} that generates the scaling subspace $\spV_j$ such that \label{item:mra_UVj_Pj}
    \\\indentx$\ds\opV_j=\set{\opP_j\ff}{\ff\in\spH}$

  \item definition: Choose $\ff\in\spLLR$ such that $\ff\orthog\Setu_{j\in\Z}\spV_j$.
        Let $\Ff(\omega)$ be the \ope{Fourier Transform} \xref{def:opFT} of $\ff(x)$.
        \label{idef:mra_UVj_f}

  \item lemma: The function $\ff$ \xref{idef:mra_UVj_f} \emph{exists} because the set of functions that 
        can be chosen to be $\ff$ at least contains $0$ (it is not the emptyset). Proof:
        \label{ilem:mra_UVj_fexists}
        \begin{align*}
          \ff(x)=0
            &\implies \inprodr{\ff}{\set{\fh\in\spLLR}{\fh\in\Setu_{j\in\Z}\spV_j}}
          \\&= \inprodr{0}{\set{\fh\in\spLLR}{\fh\in\Setu_{j\in\Z}\spV_j}}
          \\&= 0
          \\&\implies\quad \ff\orthog\Setu_{j\in\Z}\spV_j
          \\&\implies\quad \text{$\ff$ exists}
        \end{align*}

  \item lemma: $\norm{\opP_j\ff}=0\quad{\scy\forall j\in\Z}$. Proof:
        \label{ilem:mra_UVj_Pf}
    \begin{align*}
      \norm{\opP_j\ff}
        &= \norm{0}
        && \text{by definition of $\ff$ \xref{idef:mra_UVj_f}}
      \\&= 0
        && \text{by \prope{nondegenerate} property of $\normn$}
    \end{align*}

  \item definition: Choose some function $\fg\in\spLLR$ such that $\Fg(\omega)=\Ff(\omega)\setind_\intcc{-R}{R}$ \xref{def:setind} 
        for some $R>0$ 
        and such that $\norm{\ff-\fg}<\varepsilon$.
        Let $\Fg(\omega)$ be the \ope{Fourier Transform} \xref{def:opFT} of $\fg(x)$.
        \label{idef:mra_UVj_g}

  \item lemma: The function $\fg$ \xref{idef:mra_UVj_g} \emph{exists}. Proof: For some (possibly very large) $R$,
        \label{idef:mra_UVj_gexists}
    \begin{align*}
      \varepsilon
        &> \norm{\Ff(\omega)-\Fg(\omega)}
        && \text{by definition of $\fg$ \xref{idef:mra_UVj_g}}
      \\&= \norm{\opFT\ff(x)-\opFT\fg(x)}
        && \text{by definition of $\Ff$ and $\Fg$ \xref{idef:mra_UVj_f}, \xref{idef:mra_UVj_g}}
      \\&= \norm{\opFT\brs{\ff(x)-\fg(x)}}
        && \text{by \prope{linearity} of $\opFT$}
      \\&= \norm{\ff(x)-\fg(x)}
        && \text{by \prope{unitary} property of $\opFT$ \xref{thm:ft_unitary}}
      \\&\implies\quad\text{$\fg$ exists}
        && \text{because it's possible to satisfy \prefp{idef:mra_UVj_g}}
    \end{align*}

  \item lemma: $\norm{\opP_j\fg}<\varepsilon\quad{\scy\forall j\in\Z}$ for sufficiently large $R$. Proof:
        \label{ilem:mra_UVj_ge}
    \begin{align*}
      \varepsilon
        &>   \norm{\ff-\fg}
        &&   \text{by definition of $\fg$ \xref{idef:mra_UVj_g}}
      \\&\ge \norm{\opP_j\brs{\ff-\fg}}
        &&   \text{by property of \ope{projection operator}s}
      \\&=   \norm{\opP_j\ff-\opP_j\fg}
        &&   \text{by \prope{additive} property of $\opP_j$ \xref{def:linop}}
      \\&\ge \abs{\norm{\opP_j\ff}-\norm{\opP_j\fg}}
        &&   \text{by \thme{Reverse Triangle Inequality}\ifsxref{vsnorm}{thm:rti}}
      \\&=   \abs{0-\norm{\opP_j\fg}}
        &&   \text{by \xref{ilem:mra_UVj_Pf}}
      \\&=   \norm{\opP_j\fg}
        &&   \text{by \prope{strictly positive} property of $\normn$ \xref{def:norm}}
    \end{align*}

  \item  lemma: $\fg=0$. Proof: \label{ilem:mra_UVj_g0}
    \begin{align*}
       0
        &=   \lim_{j\to\infty}\norm{\opP_j\fg}^2
        &&   \text{by \prefp{ilem:mra_UVj_ge}}
      \\&\ge \lim_{j\to\infty}A\sum_{n\in\Z}\abs{\inprod{\opP_j\fg}{\opDil^j\opTrn^n\fphi}}^2
        &&   \text{by \prope{frame property} \xref{prop:rbasis_frame}}
      \\&=   \lim_{j\to\infty}A\sum_{n\in\Z}\abs{\inprod{\fg}{\opDil^j\opTrn^n\fphi}}^2
        &&   \text{by definition of $\opP_j$ \xref{item:mra_UVj_Pj}}
      \\&=   \lim_{j\to\infty}A\sum_{n\in\Z}\abs{\inprod{\opFT\fg}{\opFT\opDil^j\opTrn^n\fphi}}^2
        &&   \text{by \prope{unitary} property of $\opFT$ \xref{thm:ft_unitary}}
      \\&=   \lim_{j\to\infty}A\sum_{n\in\Z}\abs{\inprod{\Fg(\omega)}{2^{-j/2}e^{-i2^{-j}\omega n}\Fphi(2^{-j}\omega)}}^2
        &&   \text{by \prefp{prop:FTDf}}
      \\&=   \lim_{j\to\infty}A\sum_{n\in\Z}\abs{\inprod{\Fg(\omega)\Fphi^\ast(2^{-j}\omega)}{2^{-j/2}e^{-i2^{-j}\omega n}}}^2
        &&   \text{by property of $\inprodn$ in $\spLLR$}
      \\&=   \lim_{j\to\infty}A\norm{\Fg(\omega)\Fphi^\ast(2^{-j}\omega)}^2
        &&   \text{by \thme{Parseval's Identity} \xref{thm:fst}}
      \\&=   A\norm{\Fg(\omega)\Fphi^\ast(0)}^2
        &&   \text{by left hypothesis (2)}
      \\&=   A\abs{\Fphi^\ast(0)}^2\,\norm{\Fg(\omega)}^2
        &&   \text{by \prope{homogeneous} property of $\normn$\ifsxref{vsnorm}{def:norm}}
      \\&=   A\abs{\Fphi(0)}^2\,\norm{\fg}^2
        &&   \text{by \prope{unitary} property of $\opFT$ \xref{thm:ft_unitary}}
      \\&\implies \norm{\fg}=0
        &&   \text{by left hypothesis (3)}
      \\&\iff     \fg=0
        &&   \text{by \prope{nondegenerate} property of $\normn$\ifsxref{vsnorm}{def:norm}}
      %\\&\implies  \cls{\brp{\Setu_{j\in\Z}\spV_j}}=\spLLR
    \end{align*}

  \item Final step in proof that $\ds\cls{\brp{\Setu_{j\in\Z} \spV_j}} = \spLLR$:
    \begin{align*}
      \fg
        &=0
        && \text{by \prefp{ilem:mra_UVj_g0}}
      \\&\implies\ff=0
        && \text{by definition of $\fg$ \xref{idef:mra_UVj_g}}
      \\&\implies \cls{\brp{\Setu_{j\in\Z} \spV_j}} = \spLLR
    \end{align*}
\end{enumerate}
\end{proofns}

%=======================================
%\subsubsection{Separable Hilbert Space}
%=======================================
\pref{def:mra} defines an MRA on the space $\spLLR$, which is a special case of a \structe{separable Hilbert space}.
A Hilbert space\ifsxrefs{seq}{def:hilbert}is a \structe{linear space}\ifsxrefs{vector}{def:vspace}that is 
equipped with an \structe{inner product}\ifsxref{vsinprod}{def:inprod},
is \prope{complete}\ifsxrefs{seq}{def:complete}with respect to the 
\structe{metric}\ifsxrefs{metric}{def:metric}induced by the inner product,
and contains a subset that is \prope{dense}\ifsxrefs{topology}{def:dense}in $\spLLR$.

An \structe{inner product} on a linear space endows the linear space with a \structe{topology}\ifsxref{topology}{def:topology}.
The sum such as $\sum_{n=1}^\xN \alpha_n \ff_n$ is finite and thus suitable for a finite linear space only.
An infinite space requires an infinite sum $\sum_{n=1}^\infty \alpha_n \fphi_n$, and an infinite sum is defined
in terms of a limit \xref{def:suminf}.
%  \\\indentx$\ds\sum_{n=1}^\infty \alpha_n \fphi_n \eqd \lim_{\xN\to\infty}\mcom{\ds\sum_{n=1}^\xN \alpha_n \fphi_n}{partial sum}$.\\
The limit, in turn, is defined in terms of a \structe{topology}\ifsxref{topology}{def:topology}.
The \structe{inner product}\ifsxrefs{vsinprod}{def:inprod} induces a \structe{norm} \xref{def:norm} which induces a 
\structe{metric}\ifsxrefs{metric}{def:metric} which induces a topology\ifsxref{metric}{thm:(X,d)->(X,t)}.

%A common example of a separable Hilbert space is the space of square integrable functions, $\spLLR$.
%And in fact, for the design examples in this book, the reader may simply set $\spLLR=\spLLR$.

%%---------------------------------------
%\begin{proposition}
%\label{prop:Vn_separable}
%%---------------------------------------
%Let $\MRAspaceLLRV$ be an \structe{MRA space}.
%\propbox{
%  \text{$\spV_j$ is \prope{separable}}\qquad\scy\forall j\in\Z
%  }
%\end{proposition}
%\begin{proof}
%\begin{enume}
%  \item By \pref{def:mra}, $\spLLR$ is \prope{separable}.
%  \item So by \prefp{thm:XdYd_separable}, each $\spV_j$ is \prope{separable} as well.
%\end{enume}
%\end{proof}
%


%=======================================
%\subsubsection{Closure properties}
%=======================================
\pref{def:mra} defines each subspace $\spV_j$ to be \prope{closed} ($\spV_j=\cls{\spV_j}$) in $\spLLR$.
As one might imagine, the properties of \prope{completeness}\ifsxrefs{seq}{def:complete}and 
\prope{closure}\ifsxrefs{topology}{def:clsA}%, \prefp{def:subspace_closed}
are closely related. % (see next proposition).
Moreover, Every \prope{complete} sequence is also \prope{bounded}\ifsxref{metric}{def:bounded},
and so each subspace $\spV_j$ is \prope{bounded} as well. % (see \prefp{prop:Vn_bounded}).
%Both are topological properties. Completeness is defined on sequences \xrefP{def:sequence}; %closure is defined on sets.



%%---------------------------------------
%\begin{proposition}
%%---------------------------------------
%Let $\MRAspaceLLRV$ be an \structe{MRA space}.
%\propbox{
%  \mcom{\spLLR=\cls\spLLR}{$\spLLR$ is \prope{closed}.}
%  }
%\begin{proof}
%        The limit of an expansion (if the limit exists) may be inside the linear space or outside. \label{item:mra_Hcomplete}
%        We would like it to be inside. That is, we would like the space $\spLLR$ to contain all its 
%        \structe{limit points} \xrefP{def:limitpnt}.
%        The space $\spLLR$ does contain all its limit points because by definition, it is \propb{complete} \xrefP{def:complete}.
%        Any metric space (which includes all inner product spaces) that is \prope{complete} is also \prope{closed}
%        \xrefP{thm:comcls}.
%        And a metric space is \prope{closed} if and only if it contains all its limit points \xrefP{thm:cst}.
%        An inner product space that is \prope{complete} is called a \structe{Hilbert space} \xrefP{def:hilbert}.
%\end{proof}

%---------------------------------------
\begin{proposition}
\label{prop:Vn_complete}
%---------------------------------------
Let $\MRAspaceLLRV$ be an \structe{MRA space}.
\propbox{
  \text{Each subspace $\spV_j$ is \prope{complete}.}
  }
\end{proposition}
\begin{proof}
\begin{enume}
  \item By definition \pref{def:mra}, $\spLLR$ is \prope{complete}. 
  \item In any metric space, (which includes all inner product spaces such as $\spLLR$),
        a \prope{closed} subspace of a \prope{complete} metric space is itself also \prope{complete}\ifsxref{seq}{thm:comcls}.
  \item In any \prope{complete} metric space $\spX$ (which includes all Hilbert spaces such as $\spLLR$), 
        the two properties coincide---that is, a subspace is complete \emph{if and only if} 
        it is closed in the space $\spX$\ifsxref{seq}{cor:comcomcls}.
  \item So because $\spLLR$ is \prope{complete} and each $\spV_j$ is \prope{closed}, then each $\spV_j$ is also \prope{complete}.
\end{enume}
\end{proof}

%%---------------------------------------
%\begin{proposition}
%\label{prop:Vn_bounded}
%%---------------------------------------
%Let $\MRAspaceLLRV$ be an \structe{MRA space}.
%\propbox{\begin{array}{MMC}
%  $\spLLR$   & is \prope{bounded}.\\
%  $\spV_j$ & is \prope{bounded} & \forall n\in\Z .
%\end{array}}
%\end{proposition}
%\begin{proof}
%\begin{enume}
%  \item Every \prope{complete} metric space is \prope{bounded}\ifsxref{seq}{thm:convergent==>cauchy}.
%  \item $\spLLR$ is \prope{complete}, so it is also \prope{bounded} \xref{def:mra}.
%  \item Each $\spV_j$ is \prope{complete}, so each $\spV_j$ is also \prope{bounded} \xref{prop:Vn_complete}.
%\end{enume}
%\end{proof}

%=======================================
\subsubsection{Order structure}
%=======================================

\begin{minipage}[t]{\tw-58mm}%
  A \structe{multiresolution analysis} \xref{def:mra} together with the set inclusion relation $\subseteq$
  forms the \hie{linearly ordered set} \ifdochas{order}{\xrefP{def:toset}}
  $\hxs{\opair{\seqn{\spV_j}}{\subseteq}}$, illustrated to the right by a \structe{Hasse diagram}\ifsxref{order}{def:hasse}.
  Subspaces $\spV_j$ increase in ``size" with increasing $j$.
  That is, they contain more and more vectors (functions) for larger and larger $j$---%
  with the upper limit of this sequence being $\spLLR$.
  %and the subspace $\spZero$ (smallest $n$) containing only the $\vzero$ vector.
  Alternatively, we can say that approximation within a subspace $\spV_j$ 
  yields greater ``\hie{resolution}" for increasing $j$.
  %In general, the number of subspaces in such a sequence can be countably infinite (e.g. $n\in\Z$).
\end{minipage}%
\hfill%
\begin{minipage}[t]{55mm}%
  \mbox{}\\% force (just above?) top of graphic to be the top of the minipage
  \psset{yunit=0.5\psunit}%
  %============================================================================
% Daniel J. Greenhoe
% LaTeX File
%============================================================================
\begin{pspicture}(-0.75,-0.5)(4,7.5)%
  \psset{%
    labelsep=7pt,
    }
  %---------------------------------
  % nodes
  %---------------------------------
  \rput(0,6){{\large$\vdots$}}% 
  \Cnode*(  0,7){X}%    1
  \Cnode (  0,5){V2}%    V_2
  \Cnode (  0,4){V1}%    V_1
  \Cnode (  0,3){V0}%    V_0
  \Cnode (  0,2){Vn1}%    V_{n-1}
  \Cnode (  0,0) {Z}%    0
  \rput(0,1){{\color{blue}\large$\vdots$}}% 
  %---------------------------------
  % node connections
  %---------------------------------
  \ncline{Vn1}{V0}%
  \ncline{V0}{V1}%
  \ncline{V1}{V2}%
  %---------------------------------
  % node labels
  %---------------------------------
  \uput[180](X){$\spLLR$}%
  \uput[180](V2){$\spV_{2}$}%
  \uput[180](V1){$\spV_{1}$}%
  \uput[180](V0){$\spV_{0}$}%
  \uput[180](Vn1){$\spV_{-1}$}%
  \uput[180](Z){$\spZero$}%
  %---------------------------------
  % other labels
  %---------------------------------
  \rput[l](1.10,7){\rnode{labelentire}{entire linear space}}%
  \rput[c](2,4){\rnode{labellarger}{larger subspaces}}%
  \rput[c](2,3){\rnode{labelsmaller}{smaller subspaces}}%
  \rput[l](1.10,0) {\rnode{labelsmallest}{smallest subspace}}
  \ncline[labelsep=2pt,linecolor=red]{->}{labelentire}  {X}%  
  \psline[linecolor=red]{->}(2,4.20)(2,6.20)%
  \psline[linecolor=red]{->}(2,2.80)(2,0.80)%  
  \ncline[labelsep=2pt,linecolor=red]{->}{labelsmallest}{Z}%  
  %---------------------------------
  % design support
  %---------------------------------
  %\psgrid[unit=100\psunit](-1,-1)(5,8)%
\end{pspicture}%
%
\end{minipage}%

The \structe{least upper bound} (\structe{l.u.b.}) of the linearly ordered set $\opair{\seqn{\spV_j}}{\subseteq}$ is $\spLLR$ \xref{def:mra}:
  \\\indentx
   $\ds\clsp{\Setu_{j\in\Z} \spV_j} = \spLLR$.
   %$\ds\lim_{\xN\to\infty}\spV_j \eqd \clsp{\Setu_{j\in\Z} \spV_j} = \spLLR$.
  \\

%      \propb{upper bounded}:
%  Furthermore, the property $\clsp{\Setu_{j\in\Z} \spV_j} = \spLLR$
%  demonstrates that the sequence of scaling subspaces $\seqn{\spV_j}$ is \prope{upper bounded} by $\spLLR$. % \xrefP{def:complete_set}.
%  Because the subspaces are nested (or linearly ordered with respect to $\subset$) such that $\spV_j\subset\spV_{j+1}$,
%  we could define the least upper bound (or the limit) of such a sequence 
%  as\footnote{Many thanks to William Elliot, David C. Ullrich, and Seymour J. Shmuel Metz for help with this topic.
%             %\url{https://groups.google.com/forum/\#!topic/sci.math/YD4N58JH5to}
%            }

The \structe{greatest lower bound} (\structe{g.l.b.}) of the linearly ordered set $\opair{\seqn{\spV_j}}{\subseteq}$ is $\spZero$ \xref{prop:mra_glb}:
  \\\indentx
    $\ds\Seti_{j\in\Z}\spV_j = \spZero$.
  \\

All linear subspaces contain the zero vector\ifsxref{subspace}{prop:subspace_prop}.
So the intersection of any two subspaces must at least contain $\vzero$.
If the intersection of any two linear subspaces $\spX$ and $\spY$ is exactly $\setn{\vzero}$, 
then for any vector in
the sum of those subspaces ($\vu\in\spX\adds\spY$) there are \propb{unique} vectors $\ff\in\spX$ and 
$\fg\in\spY$ such that $\vu=\ff+\fg$.
This is \emph{not} necessarily true if the intersection contains more than just $\setn{\vzero}$
\ifsxref{subspace}{thm:XY0_unique}.


%%=======================================
%\subsubsection{Bases for wavelet system}
%%=======================================
%%%A linear space is a separable Hilbert space if and only if it has a complete basis.
%%Note that \pref{def:mra} does not require $\lim_{\xN\to\infty}\spV_j$ to be equal to $\spLLR$, 
%%it is only requires it to be \prope{dense} in $\spLLR$
%%(just as the rationals are dense in the real numbers).
%%In the set of real numbers, a countable union of closed sets is called a $\symx{\setFsigma}$ set
%%($\setF$ stands for the French word \hie{ferm/'e} or \prope{closed}, and $\sigma$ stands for the French word \hie{somme} or sum).\footnote{
%%  \citerpg{carothers2000}{130}{0521497566}\\
%%  \citerpg{givant2009}{270}{0387402934}
%%  }                                                                             
%
%\prefp{def:mra} defines an MRA on the space $\spLLR$.
%The space $\spLLR$ is an example of a Hilbert space. % $\spLLR$. 
%A Hilbert space is a linear space equipped with an inner product 
%and that is complete with respect to the topology induced by the inner product.
%
%\begin{enumerate}
%  \item A \structb{linear space} \ifsxrefs{vector}{def:vspace} supports the expansion of a vector $\ff$
%        (e.g. a function) in terms of a set of \structe{coordinates} $\setxn{\alpha_n}$ and a 
%        \structe{Hamel basis} $\setxn{\fphi_n}$\ifsxrefs{frames}{def:hamel}such that \label{item:mra_hamel}
%          \\\indentx$\ds \ff(x)=\sum_{n=1}^\xN \alpha_n \fphi_n(x)$.\\
%        If such coordinates exist for a vector $\ff$ and basis $\setxn{\fphi_n}$, 
%        then those coordinates are \prope{unique}\ifsxref{frames}{thm:hamel_unique}.
%
%  \item The Hamel basis described in \pref{item:mra_hamel} provides sufficient support for expansion in finite linear spaces, 
%        but is problematic in infinite spaces.                           
%        In an infinite linear space with a topology (such as a Banach space or a Hilbert space),
%        a \structe{Schauder basis}\ifsxrefs{frames}{def:schauder}is often used.
%        The Schauder basis is defined in terms of a special type of convergence called \prope{strong convergence}\ifsxref{seq}{def:strong_converge}.
%        Strong convergence is defined in terms of the norm induced by the inner product: \label{item:mra_strong}
%        \\\indentx$\ds  
%          \ff \eqs \sum_{n=1}^\infty\alpha_n\fphi_n
%              \eqd \lim_{\xN\to\infty}\sum_{n=1}^\xN\alpha_n\fphi_n
%              \implies
%              \mcom{\ds\lim_{\xN\to\infty}\norm{\ff-\sum_{n=1}^\xN\alpha_n\fphi_n}=0}{\prope{strong convergence}}
%        $.\\
%        %\\\indentx$\ds  
%        %  \ff \eqs \sum_{n=1}^\infty\inprod{\ff}{\fphi_n}\fphi_n
%        %      \eqd \lim_{\xN\to\infty}\sum_{n=1}^\xN\inprod{\ff}{\fphi_n}\fphi_n
%        %      \implies
%        %      \mcom{\ds\lim_{\xN\to\infty}\norm{\ff-\sum_{n=1}^\xN\inprod{\ff}{\fphi_n}\fphi_n}=0}{\prope{strong convergence}}
%        %$.\\
%        That is, the sum $\sum_{n=1}^\infty\alpha_n\fphi_n$ is by definition  
%        the limit of the partial sums
%        $\sum_{n=1}^\xN\alpha_n\fphi_n$\ifsxref{series}{def:suminf},
%        and that these sums \prope{converge strongly} (``$\eqs$",\ifsxref{seq}{def:strong_converge}) to a limit $\ff$
%        with respect to the topology induced by the norm $\normn$, which in turn is 
%        induced by the inner product $\inprodn$. % \xrefP{def:norm=inprod}.
%        The completeness property ensures that all of these limits $\ff$ are also in the space $\spLLR$.
% 
%  \item In an MRA space $\MRAspaceLLRV$, the space $\spLLR$ is separable \xrefP{def:mra}, and the subspaces $\spV_j$ are
%        separable as well. % \xrefP{prop:Vn_separable}.
%        The property of a space being separable is very important in analysis:
%    \begin{enumerate}
%      \item Every Banach space (which includes all Hilbert spaces such as $\spLLR$ and each $\spV_j$) with 
%            a Schauder basis is \prope{separable}\ifsxref{frames}{thm:Bschauder==>separable}.\label{item:mra_Bschauder_separable}
%      \item The converse is \emph{not} true---not every separable Banach space has a basis\ifsxrefpo{frames}{BasisProblem}.
%    \end{enumerate}
%
%  \item Besides providing a topology, the \structe{inner product} also supports 
%        the notion of a subspace geometry, 
%        including the property of \prope{orthogonality}\ifsxref{vsinprod}{def:orthog}.
%    \begin{enumerate}
%      \item Orthogonality supports the \structe{Fourier expansion}\ifsxrefs{frames}{def:hspace_fex}of a vector $\ff$ over an 
%        \structe{orthornormal basis} $\setxZp{\fphi_n}$ in the form \label{item:mra_inprod}
%        \\\indentx$\ds  
%          \ff \eqs \sum_{n=1}^\infty \mcom{\inprod{\ff}{\fphi_n}}{\structe{Fourier coefficient}}\fphi_n
%          $
%      \item In contrast to \pref{item:mra_Bschauder_separable}, life in Hilbert spaces is much simpler. 
%            A Hilbert space has a Schauder basis \emph{if and only if} it is 
%            separable \ifsxrefs{frames}{thm:schauder<==>separable}. 
%            And so $\spLLR$ and each $\spV_j$ \emph{have} Schauder bases\ifsxref{frames}{thm:schauder<==>separable}.
%      \item A special case of a Schauder basis is an orthonormal basis\ifsxref{frames}{def:basis_ortho}.
%            A Hilbert space has an orthonormal basis if and only if it is separable\ifsxref{frames}{thm:ortho<==>separable}.
%            And so $\spLLR$ and each $\spV_j$ \emph{have} orthonormal bases as well% 
%            \ifdochas{frames}{ (\xref{thm:schauder<==>separable}, \xrefp{thm:HV_orthobasis})}.
%      \item It is always possible to construct an orthonormal basis for a separable Hilbert space using the 
%            \thme{Graham Schmidt orthogonalization} procedure.
%    \end{enumerate} 
%
%
%
%\item 
%  %Note that the intersection of any two orthogonal subspaces contains the zero vector only \xrefP{thm:YoZ==>YZ0}.
%  A basis\ifsxrefs{frames}{sec:hspace_bases} $\setxn{\fphi_n}$ that is \prope{orthonormal}  possesses a number of useful properties
%  including the following:
%    \begin{enumerate}
%      \item The \thmb{Pathogorean Theorem} holds such that  $\norm{\sum_{n=1}^\xN \fphi_{n}}^2 = \sum_{n=1}^\xN \norm{\fphi_n}^2$\ifsxref{frames}{thm:pythag}.
%      \item The sequence $\setn{\fphi_n}$ is \prope{linearly independent}\ifsxref{frames}{thm:orthog==>linin}.
%      %\item \thmb{Bessel's equality} holds such that 
%      %      $\ds\norm{\ff-\sum_{i=n}^\xN \inprod{\ff}{\fphi_n} \fphi_n }^2 = \norm{\ff}^2 - \sum_{i=1}^\xN |\inprod{\ff}{\fphi_n}|^2$
%      \item \thmb{Bessel's inequality} holds such that 
%            $\ds\sum_{n=1}^\infty \abs{\inprod{\ff}{\fphi_n}}^2 \le \norm{\ff}^2$\ifsxref{frames}{thm:bessel_ineq}.
%      \item Every vector $\ff$ in $\spLLR$ has a \structe{Fourier expansion}\ifsxrefs{frames}{def:hspace_fex}such that \\
%            $\ds\ff\eqs\sum_{n=1}^\infty \inprod{\ff}{\fphi_n} \fphi_n$\ifsxref{frames}{thm:hspace_fex}.
%      \item \thmb{Parseval's identity} holds \textbf{if and only if} $\setn{\fphi_n}$ is an orthonormal basis:\\
%            $\ds\norm{\ff}^2 \eqs\sum_{n=1}^\infty \abs{\inprod{\ff}{\fphi_n}}^2  \quad\scy\forall\ff\in\setX$%
%            \ifsxref{frames}{thm:parsevalid}.
%      \item The \structe{Fourier expansion} of a vector $\ff$ in a Hilbert space $\spLLR$
%            on an orthonormal basis $\setn{\fphi_n}$ 
%            that spans a subspace $\spY\subseteq\spLLR$ is the best approximation of $\ff$ in $\spY$ with respect to
%            the metric induced by the inner product (\thme{Best Approximation Theorem}\ifsxref{frames}{thm:bat}).
%    \end{enumerate}
%
%
%  \item \structb{Riesz basis}: \pref{def:mra} does not explicitly require an orthonormal basis.
%         Instead, it only specifies the weaker (more general) constraint of a Riesz basis.
%         This constraint implies simply that there is a linear mapping between the Riesz basis and an orthonormal basis.
%         In particular, an orthonormal basis can be constructed from the Riesz basis.
%\end{enumerate}

%%--------------------------------------
%\begin{proposition}
%\label{thm:HV_orthobasis}
%%--------------------------------------
%Let $\MRAspaceLLRV$ be an MRA space.
%\propbox{\begin{array}{MMC}
%  $\spLLR$   & has an \structe{orthonormal basis}\\
%  $\spV_j$ & has an \structe{orthonormal basis} & \forall j\in\Z
%\end{array}}
%\end{proposition}
%\begin{proofns}
%\begin{enume}
%  %\item By \prefp{def:mra}, $\spLLR$ is \prope{separable}.
%  \item $\spLLR$ is \prope{separable}.
%  \item Therefore $\spLLR$ has an orthonormal basis\ifsxref{frames}{thm:ortho<==>separable}.
%  \item Therefore each $\spV_j$ is \prope{separable}\ifsxref{subspace}{prop:Vn_separable}.
%  \item So each $\spV_j$ has an orthonormal basis\ifsxref{frames}{thm:ortho<==>separable}.
%\end{enume}
%\end{proofns}

%=======================================
\subsubsection{Dilation equation}
%=======================================
Several functions in mathematics exhibit a kind of \prope{self-similar} or \prope{recursive} property:
\begin{listi}
  \item If a function $\ff(x)$ is \prope{linear}, then \xref{ex:TD_flinear}
        \\\indentx$\ds\ff(x) = \ff(1)x - \ff(0)\opTrn x$.   %{$\setn{x,\,\opTrn x}$ is a \structe{basis} for $\clLcc$}$.
  \item If a function $\ff(x)$ is sufficiently \prope{bandlimited}, then the \structe{Cardinal series} \xref{ex:TD_cardinalseries} demonstrates
        \\\indentx$\ds\ff(x) = \sum_{n=1}^\infty \ff(n) \opTrn^n\frac{\sin\brs{\pi(x)}}{\pi(x)}$.
  \item \fncte{B-splines} \xref{thm:bspline_recursion} are another example:
        \\\indentx$\ds\fN_n(x)   = \frac{1}{n}x\fN_{n-1}(x) - \frac{1}{n}x\opTrn\fN_{n-1}(x) + \frac{n+1}{n}\opTrn\fN_{n-1}(x)  \qquad\scy\forall n\in\Znn\setd\setn{1},\,  \forall x\in\R$.
\end{listi}

The scaling function $\fphi(x)$ \xref{def:mra} also exhibits a kind of \prope{self-similar} property.
By \prefp{def:mra}, the dilation $\opDil\ff$ of each vector $\ff$ in $\spV_0$ is in $\spV_1$.
If $\setxZ{\opTrn^n\fphi}$ is a basis for $\spV_0$,
then $\setxZ{\opDil\opTrn^n\fphi}$ is a basis for $\spV_1$,
$\setxZ{\opDil^2\opTrn^n\fphi}$ is a basis for $\spV_2$, \ldots;
and in general $\set{\opDil^j\opTrn^m\fphi}{j\in\Z}$ is a basis for $\spV_j$.
Also, if $\fphi$ is in $\spV_0$, then it is also in $\spV_1$ (because $\spV_0\subset\spV_1$).
And because $\fphi$ is in $\spV_1$ and because $\setxZ{\opDil\opTrn^n\fphi}$ is a basis for $\spV_1$,
$\fphi$ is a linear combination of the elements in $\setxZ{\opDil\opTrn^n\fphi}$.
That is, $\fphi$ can be represented as a linear combination of translated and dilated versions of itself.
The resulting equation is called the \hie{dilation equation} (\pref{def:dilation_eq}, next).\footnote{%
The property of \prope{translation invariance} is of particular significance in the theory of 
\structe{normed linear spaces} (a Hilbert space is a complete normed linear space equipped with an inner product)%
\ifdochas{vsnorm}{---see \prefp{lem:vsn_ti} and \prefp{thm:vsn_d2norm}}.
}

%--------------------------------------
\begin{definition}
\footnote{
  \citerp{jawerth}{7}
  }
\label{def:dilation_eq}
%--------------------------------------
Let $\MRAspaceLLRV$ be a \structe{multiresolution analysis space} with scaling function $\fphi$ \xref{def:mra}.
Let $\seqxZ{h_n}$ be a \structe{sequence} \xref{def:seq} in $\spllR$ \xref{def:spllR}.
\defboxp{
  The equation 
    \\\indentx$\ds\fphi(x)=\sum_{n\in\Z}  h_n \opDil \opTrn^n \fphi(x)\qquad\scy\forall x\in\R$\\
  is called the \equd{dilation equation}.
  It is also called the \equd{refinement equation},
  \equd{two-scale difference equation}, and \equd{two-scale relation}.
  }
\end{definition}

%--------------------------------------
\begin{theorem}[\thmd{dilation equation}]
\label{thm:h->phi}
\label{thm:dilation_eq}
%--------------------------------------
%Let $\MRAspaceLLRV$ be a \structe{multiresolution analysis space} with scaling function $\fphi$ \xref{def:mra}.
%Let $\mrasys$ be an \structe{MRA system} \xref{def:mrasys}.
Let an \structe{MRA space} and \fncte{scaling function} be as defined in \prefp{def:mra}.
%Let $\Fphi(\omega)$ be the \fncte{Fourier transform}\ifsxrefs{harFour}{def:ft}of $\fphi(x)$.
%Let $\Dh(\omega)$ be the \fncte{Discrete time Fourier transform}\ifsxref{dsp}{def:dtft} of $\seqn{h_n}$.
%\\Let $\ds\prod_{n=1}^\infty x_n \eqd \lim_{\xN\to\infty}\prod_{n=1}^\xN x_n$.
\thmbox{
  \brb{\begin{array}{M}
    $\MRAspaceLLRV$ is an \structe{MRA space}\\ 
    with \structe{scaling function} $\fphi$
  \end{array}}
  \quad\implies\quad
  \mcom{\brb{\begin{array}{>{\ds}l}
    \scy\exists \seqxZ{h_n} \st\\
    \fphi(x)=\sum_{n\in\Z}  h_n \opDil \opTrn^n \fphi(x)\qquad
    \scy\forall x\in\R
  \end{array}}}{\prope{dilation equation in ``time"}}
  }
\end{theorem}
\begin{proof}
    \begin{align*}
      \fphi &\in \spV_0
            && \text{by \prefp{def:mra}}
          \\&\subseteq \spV_1
            && \text{by \prefp{def:mra}}
          \\&= \Span\setxZ{\opDil\opTrn^n\fphi(x)}
          \\&\implies 
             \exists \seqxZ{h_n} \st
                \fphi = \sum_{n\in\Z} h_n \opDil \opTrn^n \fphi
    \end{align*}
\end{proof}

%--------------------------------------
\begin{lemma}
\footnote{
  \citerpg{mallat}{228}{012466606X}
  }
\label{lem:Fphi}
%--------------------------------------
Let $\fphi(x)$ be a function in $\spLLR$ \xref{def:spLLR}.
Let $\Fphi(\omega)$ be the \fncte{Fourier transform}\ifsxrefs{harFour}{def:ft}of $\fphi(x)$.
Let $\Dh(\omega)$ be the \fncte{Discrete time Fourier transform}\ifsxref{dsp}{def:dtft} of a sequence $\seqnZ{h_n}$.
\lembox{
  \begin{array}{>{\ds}lc rc>{\ds}lCD}
  {\scy(A)}\quad\fphi(x)=\sum_{n\in\Z}  h_n \opDil \opTrn^n \fphi(x)\quad{\scy \forall x\in\R} %{\prope{dilation equation in ``time" (A)}}
    &\iff&
    \Fphi\brp{\omega} &=& \cwt \: \Dh\brp{\frac{\omega}{2}}\: \Fphi\brp{\frac{\omega}{2}}
                        & \forall \omega\in\R
                        & (1)
  \\&\iff&
    \Fphi\brp{\omega} &=& \Fphi\brp{\frac{\omega}{2^\xN}} \prod_{n=1}^\xN \cwt\:\Dh\brp{\frac{\omega}{2^n}}
                        & \forall n\in\Zp,\,\omega\in\R
                        & (2)
  \end{array} 
  }
\end{lemma}
\begin{proof}
\begin{enumerate}
  \item Proof that (A)$\implies$(1): \label{item:Fphi_A1}
    \begin{align*}
      \Fphi\brp{\omega}
        &\eqd \opFT\fphi
      \\&= \opFT\sum_{n\in\Z} h_n \opDil \opTrn^n \fphi(x)
        && \text{by (A)}
      \\&= \sum_{n\in\Z} h_n \opFT\opDil\opTrn^n \fphi(x)
      \\&= \sum_{n\in\Z} h_n \mcom{\cwt e^{-i\frac{\omega}{2} n}\fphi\brp{\frac{\omega}{2}}}{$\opFT\opDil\opTrn^n \fphi(x)$}
        && \text{by \prefp{prop:FTDf}}
      \\&= \cwt \mcom{\brs{\sum_{n\in\Z} h_n e^{-i\frac{\omega}{2} n}}}{$\Dh(\omega/2)$} \Fphi\brp{\frac{\omega}{2}}
      \\&= \cwt\:\Dh\brp{\frac{\omega}{2}}\: \Fphi\brp{\frac{\omega}{2}}
        && \text{by definition of \ope{DTFT} \xref{def:dtft}}
    \end{align*}

  \item Proof that (A)$\impliedby$(1):
    \begin{align*}
      \fphi(x) 
        &= \opFTi\Fphi(\omega)
        && \text{by definition of $\Fphi(\omega)$}
      \\&= \opFTi\cwt\:\Dh\brp{\frac{\omega}{2}}\: \Fphi\brp{\frac{\omega}{2}}
        && \text{by (1)}
      \\&= \opFTi\cwt\:\sum_{n\in\Z}h_n e^{-i\frac{\omega}{2} n}\: \Fphi\brp{\frac{\omega}{2}}
        && \text{by definition of \ope{DTFT} \xref{def:dtft}}
      \\&= \cwt\:\sum_{n\in\Z}h_n \opFTi e^{-i\frac{\omega}{2} n}\: \Fphi\brp{\frac{\omega}{2}}
        && \text{by property of linear operators}
      \\&= \cwt\:\sum_{n\in\Z}h_n \opFTi \opFT\opDil\opTrn^n\fphi
        && \text{by \prefp{prop:FTDf}}
      \\&= \sum_{n\in\Z} h_n \opDil\opTrn^n \fphi(x)
    \end{align*}

  \item Proof that (1)$\implies$(2):
    \begin{enumerate}
      \item Proof for $\xN=1$ case:
        \begin{align*}
          \brlr{\Fphi\brp{\frac{\omega}{2^\xN}}\:\prod_{n=1}^\xN \cwt \Dh\brp{\frac{\omega}{2^n}}}_{\xN=1}
            &= \cwt\:\Dh\brp{\frac{\omega}{2}}\Fphi\brp{\frac{\omega}{2}}
          \\&= \Fphi(\omega)
            && \text{by (1)}
        \end{align*}

      \item Proof that [$\xN$ case]$\implies$[$\xN+1$ case]:
        \begin{align*}
          \Fphi\brp{\frac{\omega}{2^{\xN+1}}}\:\prod_{n=1}^{\xN+1} \cwt \Dh\brp{\frac{\omega}{2^n}}
            &= \brs{\prod_{n=1}^{\xN} \cwt \Dh\brp{\frac{\omega}{2^n}}}
               \mcom{\cwt \Dh\brp{\frac{\omega}{2^{N+1}}}\Fphi\brp{\frac{\omega}{2^{\xN+1}}}}{$\Fphi(\omega/2^\xN)$}
          \\&= \Fphi(\omega/2^\xN) \prod_{n=1}^{\xN} \cwt \Dh\brp{\frac{\omega}{2^n}}
          \\&= \Fphi(\omega)
            && \text{by [$\xN$ case] hypothesis}
        \end{align*}
    \end{enumerate}

  \item Proof that (1)$\impliedby$(2):
    \begin{align*}
      \Fphi(\omega)
        &= \brlr{\Fphi\brp{\frac{\omega}{2^{\xN}}}\:\prod_{n=1}^{\xN} \cwt \Dh\brp{\frac{\omega}{2^n}}}_{\xN=1}
        && \text{by (2)}
      \\&= \Fphi\brp{\frac{\omega}{2}}\cwt \Dh\brp{\frac{\omega}{2}}
      \\&= \cwt\Dh\brp{\frac{\omega}{2}}\Fphi\brp{\frac{\omega}{2}} 
    \end{align*}

\end{enumerate}
\end{proof}

%--------------------------------------
\begin{lemma}
\label{lem:Fphi_infty}
% 2013 August 09 Friday
% 2013 August 20 Monday: modified \implies relation
%--------------------------------------
Let $\fphi(x)$ be a function in $\spLLR$ \xref{def:spLLR}.
Let $\Fphi(\omega)$ be the \fncte{Fourier transform}\ifsxrefs{harFour}{def:ft}of $\fphi(x)$.
Let $\Dh(\omega)$ be the \fncte{Discrete time Fourier transform}\ifsxref{dsp}{def:dtft} of $\seqn{h_n}$.
Let $\ds\prod_{n=1}^\infty x_n \eqd \lim_{\xN\to\infty}\prod_{n=1}^\xN x_n$, with respect to the standard norm in $\spLLR$.

\lembox{
  \begin{array}{>{\ds}l c rc>{\ds}lCD}
    \brb{\begin{array}{M}
      $\Fphi\brp{\omega} = C\prod_{n=1}^\infty \cwt\:\Dh\brp{\frac{\omega}{2^n}}$\\
      $\scy\forall C>0,\,\omega\in\R$\qquad\qquad\scs(A)
    \end{array}}
      &\implies& \fphi(x)          &=& \sum_{n\in\Z}  h_n \opDil \opTrn^n \fphi(x)
                                     & \forall x\in\R 
                                     & (1) %{\prope{dilation equation in ``time" (A)}}
    \\&\iff&     \Fphi\brp{\omega} &=& \cwt \: \Dh\brp{\frac{\omega}{2}}\: \Fphi\brp{\frac{\omega}{2}}
                                     & \forall \omega\in\R
                                     & (2)
  \\
      &\iff&     \Fphi\brp{\omega} &=& \Fphi\brp{\frac{\omega}{2^\xN}} \prod_{n=1}^\xN \cwt\:\Dh\brp{\frac{\omega}{2^n}}
                                     & \forall n\in\Zp,\,\omega\in\R
                                     & (3)
  \end{array}
  }
\end{lemma}
\begin{proof}
  \begin{enumerate}
    \item Proof that (1)$\iff$(2)$\iff$(3): by \prefp{lem:Fphi}
    \item Proof that (A)$\implies$(2):
      \begin{align*}
        \Fphi(\omega)
          &= C\:\prod_{n=1}^{\infty} \cwt \Dh\brp{\frac{\omega}{2^n}}
          && \text{by left hypothesis}
        \\&= C\:\cwt \Dh\brp{\frac{\omega}{2}} \prod_{n=1}^{\infty} \cwt \Dh\brp{\frac{\omega}{2^{n+1}}}
        \\&= C\:\cwt \Dh\brp{\frac{\omega}{2}} \prod_{n=1}^{\infty} \cwt \Dh\brp{\frac{\omega/2}{2^{n}}}
        \\&= \cwt \Dh\brp{\frac{\omega}{2}}\brs{C\prod_{n=1}^{\infty} \cwt \Dh\brp{\frac{\omega/2}{2^{n}}}}
        \\&= \cwt \Dh\brp{\frac{\omega}{2}} \Fphi\brp{\frac{\omega}{2}}
          && \text{by left hypothesis}
      \end{align*}
  \end{enumerate}
\end{proof}


%--------------------------------------
\begin{proposition}
\label{prop:Fphi}
%--------------------------------------
Let $\fphi(x)$ be a function in $\spLLR$ \xref{def:spLLR}.
Let $\Fphi(\omega)$ be the \fncte{Fourier transform}\ifsxrefs{harFour}{def:ft}of $\fphi(x)$.
Let $\Dh(\omega)$ be the \fncte{Discrete time Fourier transform}\ifsxref{dsp}{def:dtft} of $\seqn{h_n}$.
Let $\ds\prod_{n=1}^\infty x_n \eqd \lim_{\xN\to\infty}\prod_{n=1}^\xN x_n$, with respect to the standard norm in $\spLLR$.
\propbox{
  \brb{\begin{array}{M}
    $\Fphi(\omega)$ is\\
    \prope{continuous}\\ 
    at $\,\omega=0$
  \end{array}}
  \quad\implies\quad
  \brb{\begin{array}{c rc>{\ds}lCD}
        &
    \fphi(x)          &=& \sum_{n\in\Z}  h_n \opDil \opTrn^n \fphi(x)
                        & \forall x\in\R 
                        & (1) %{\prope{dilation equation in ``time" (A)}}
  \\\iff&
    \Fphi\brp{\omega} &=& \cwt \: \Dh\brp{\frac{\omega}{2}}\: \Fphi\brp{\frac{\omega}{2}}
                        & \forall \omega\in\R
                        & (2)
  \\\iff&
    \Fphi\brp{\omega} &=& \Fphi\brp{\frac{\omega}{2^\xN}} \prod_{n=1}^\xN \cwt\:\Dh\brp{\frac{\omega}{2^n}}
                        & \forall n\in\Zp,\,\omega\in\R
                        & (3)
  \\\iff&
    \Fphi\brp{\omega} &=& \Fphi\brp{0} \prod_{n=1}^\infty \cwt\:\Dh\brp{\frac{\omega}{2^n}} 
                        & \omega\in\R
                        & (4)
  \end{array}}
  }
\end{proposition}
%\lembox{
%  \brb{\begin{array}{FMCD}
%    (A) & $\ds\fphi(x)=\sum_{n\in\Z}  h_n \opDil \opTrn^n \fphi(x)$ & \forall x\in\R & and\\ % & {\prope{dilation equation}}\\
%    (B) & \mc{3}{M}{$\Fphi(\omega)$ is \prope{continuous} at $\omega=0$}
%  \end{array}}
%  \implies
%  \brb{\begin{array}{>{\ds}lCD}
%    \Fphi\brp{\omega} = \Fphi\brp{0} \prod_{n=1}^\infty \cwt\:\Dh\brp{\frac{\omega}{2^n}} & \forall \omega\in\R
%  \end{array}} 
%  }
%\end{lemma}
\begin{proof}
  \begin{enumerate}
    \item Proof that (1)$\iff$(2)$\iff$(3): by \prefp{lem:Fphi}
    \item Proof that (3)$\implies$(4):
      \begin{align*}
        \Fphi\brp{0}\:\prod_{n=1}^{\infty} \cwt \Dh\brp{\frac{\omega}{2^n}}
          &= \lim_{\xN\to\infty}\Fphi\brp{\frac{\omega}{2^{\xN}}}\:\prod_{n=1}^{\xN} \cwt \Dh\brp{\frac{\omega}{2^n}}
          && \text{by \prope{continuity} and definition of $\prod_{n=1}^\infty x_n$}
        \\&= \Fphi(\omega)
          && \text{by (3) and \prefp{lem:Fphi}}
      \end{align*}
    \item Proof that (2)$\impliedby$(4): by \prefp{lem:Fphi_infty}
      %\begin{align*}
      %  \Fphi(\omega)
      %    &= \Fphi\brp{0}\:\prod_{n=1}^{\infty} \cwt \Dh\brp{\frac{\omega}{2^n}}
      %    && \text{by (4)}
      %  \\&= \Fphi\brp{0}\:\cwt \Dh\brp{\frac{\omega}{2}} \prod_{n=1}^{\infty} \cwt \Dh\brp{\frac{\omega}{2^{n+1}}}
      %  \\&= \Fphi\brp{0}\:\cwt \Dh\brp{\frac{\omega}{2}} \prod_{n=1}^{\infty} \cwt \Dh\brp{\frac{\omega/2}{2^{n}}}
      %  \\&= \cwt \Dh\brp{\frac{\omega}{2}}\brs{ \Fphi\brp{0}\prod_{n=1}^{\infty} \cwt \Dh\brp{\frac{\omega/2}{2^{n}}}}
      %  \\&= \cwt \Dh\brp{\frac{\omega}{2}} \Fphi\brp{\frac{\omega}{2}}
      %    && \text{by (4)}
      %\end{align*}
  \end{enumerate}
\end{proof}



%\if 0


\pref{def:hn} (next) formally defines the coefficients that appear in \prefpp{thm:dilation_eq}.
%--------------------------------------
\begin{definition}%[subspace coefficients]
\label{def:hn}
%\label{def:gn}
\label{def:mrasys}
%\raggedright
%--------------------------------------
Let $\MRAspaceLLRV$ be a multiresolution analysis space with scaling function $\fphi$.
Let $\seqxZ{h_n}$ be a sequence of coefficients such that $\fphi = \sum_{n\in\Z}  h_n \opDil\opTrn^n \fphi$.
\defboxp{
  A \hid{multiresolution system} is the tuple $\mrasys$.
  The sequence $\seqxZ{h_n}$ is the \hid{scaling coefficient sequence}.
  A multiresolution system is also called an \hid{MRA system}.
  An \structe{MRA system} is an \hid{orthonormal MRA system} if $\setnZ{\opTrn^n\fphi}$ is \prope{orthonormal}.
  }
\end{definition}

%Examples of \hi{multiresolution analyses} are provided in
%\pref{ex:wavstrct_haar_sin} (next)
%-- \prefp{ex:sw_gh_bspline}.

%%--------------------------------------
%\begin{definition}
%\label{def:wavstrct_normcoef}
%%--------------------------------------
%Let $\mrasys$ be a {multiresolution system}, and $\opDil$ the dilation operator.
%\defboxp{
%  The \hid{normalization coefficient at resolution $n$} is the quantity 
%  \\\indentx$\norm{\opDil^j\fphi}$.
%  }
%\end{definition}


%--------------------------------------
\begin{theorem}
\label{thm:V0Vn}
%--------------------------------------
Let $\mrasys$ be an \structe{MRA system} \xref{def:mrasys}.
Let $\linspan\setA$ be the \structe{linear span} \xref{def:span} of a set $\setA$.
\thmbox{
  \mcom{\linspan\setxZ{\opTrn^n\fphi}=\spV_0}{$\setxZ{\opTrn^n\fphi}$ is a \structe{basis} for $\spV_0$}
  \qquad\implies\qquad
  \mcom{\linspan\setxZ{\opDil^j\opTrn^n\fphi}=\spV_j\quad{\scy\forall j\in\Znn}}{$\setxZ{\opDil^j\opTrn^n\fphi}$ is a \structe{basis} for $\spV_j$}
  }
\end{theorem}
\begin{proof} Proof is by induction:\footnote{\citerpg{smith2011}{4}{1420063723}}
\begin{enumerate}
\item induction basis (proof for $j=0$ case):
      %$\setxZ{\opDil^0\opTrn^n\fphi}$ is a basis for $\spV_0$
      %\\\indentx$\ds \spV_0 = \set{\ff(x)}{\ff(x) = \sum_{n\in\Z} \fc_{n} \fphi(x-n)}.$
  \begin{align*}
    \brlr{\linspan\setxZ{\opDil^j\opTrn^n\fphi}}_{j=0}
      &= \linspan\setxZ{\opTrn^n\fphi}
    \\&= \spV_0
      && \text{by left hypothesis}
  \end{align*}

\item induction step (proof that $j$ case $\implies$ $j+1$ case):
      %$\setxZ{\opDil^j\opTrn^n\fphi}$ is a basis for $\spV_j$
      %$\implies$ $\setxZ{\opDil^{j+1}\opTrn^n\fphi}$ is a basis for $\spV_{j+1}$:
  \begin{align*}
    &\linspan\setxZ{\opDil^{j+1}\opTrn^n\fphi}
    \\&= \set{\ff\in\spLLR}{\exists \seqn{\alpha_n} \st \ff(x)=\sum_{n\in\Z}\alpha_n \opDil^{j+1}\opTrn^n\fphi}
      && \text{by definition of $\linspan$ \xref{def:span}}
    \\&= \set{\ff\in\spLLR}{\exists \seqn{\alpha_n} \st \ff(x)=\opDil\sum_{n\in\Z}\alpha_n \opDil^{j}\opTrn^n\fphi}
    \\&= \set{\ff\in\spLLR}{\exists \seqn{\alpha_n} \st \opDili\ff(x)=\sum_{n\in\Z}\alpha_n \opDil^{j}\opTrn^n\fphi}
    \\&= \set{\brs{\opDil\ff}\in\spLLR}{\exists \seqn{\alpha_n} \st \opDili\brs{\opDil\ff(x)}=\sum_{n\in\Z}\alpha_n \opDil^{j}\opTrn^n\fphi}
    \\&= \opDil\set{\ff\in\spLLR}{\exists \seqn{\alpha_n} \st \ff(x)=\sum_{n\in\Z}\alpha_n \opDil^{j}\opTrn^n\fphi}
    \\&= \opDil\linspan\setxZ{\opDil^j\opTrn^n\fphi}
      && \text{by definition of $\linspan$ \xref{def:span}}
    \\&= \opDil\spV_j
      && \text{by induction hypothesis}
    \\&= \spV_{j+1}
      && \text{by \prope{self-similar} prop. \xref{def:mra}}
  \end{align*}
\end{enumerate}
\end{proof}

%--------------------------------------
\begin{example}
\label{ex:wavstrct_haar_sin}
\exmx{Haar scaling function}
%--------------------------------------
\exbox{\begin{array}{rclm{54mm}} 
  \mc{4}{M}{In the \hie{Haar} MRA, the scaling function $\fphi(x)$ is the \hie{pulse function}}
  \\
  \fphi(x) &=& \brbl{\begin{array}{lM}
                       1  & for $x\in\intco{0}{1}$ \\
                       0  & otherwise.
                     \end{array}}
  &
  %============================================================================
% Daniel J. Greenhoe
% LaTeX file
%============================================================================
{\begin{pspicture}(-2,-0.5)(3,1.2)%
  \scs
  %\psset{
    %linecolor=blue,
    %linewidth=1pt,
    %dotsize=5pt,
    %dotsep=1pt,
    %}%
  \psaxes[linecolor=axis,yAxis=false,linewidth=0.75pt]{<->}(0,0)(-2,0)(3,1.2)%
  \psline[linestyle=dotted](0,0)(0,1)%
  \psline[linestyle=dotted](1,1)(1,0)%
  \psline{-o}(-1.2, 0)( 0, 0)% left horizontal
  \psline{*-o}(0,1)(1,1)% middle horizontal
  \psline{*-}(1,0)(2.2,0)% right horizontal
  \psline[linestyle=dotted](2.2,0)(2.75,0)%
  \psline[linestyle=dotted](-1.2,0)(-1.75,0)%
  \uput{3.5pt}[180](0,1){$1$}%
\end{pspicture}}

  \\
  \mc{4}{M}{In the subspace $\spV_j$ ($j\in\Z$) the scaling functions are}
  \\
  \opDil^j\fphi(x) &=& \brbl{\begin{array}{lM}
                               \brp{2}^{j/2}   & for $x\in\intco{0}{\brp{2^{-j}}}$ \\
                               0                  & otherwise.
                              \end{array}}
  &
  %============================================================================
% Daniel J. Greenhoe
% LaTeX file
% lattice ({factors of 30}, |)
%============================================================================
\begin{pspicture}(-1.7,-0.5)(3,1.5)%
  \footnotesize
  %\psset{linecolor=blue}%
  \psaxes[linecolor=axis,labels=none,linewidth=0.75pt]{<->}(0,0)(-1.5,0)(2.9,1.5)%
  \psline[linestyle=dotted](0,0)(0,1)%
  \psline[linestyle=dotted](1,1)(1,0)%
  \psline{-o}(-1, 0)( 0, 0)% left horizontal
  \psline{*-o}(  0,1)(1,1)% middle horizontal
  \psline{*-}( 1, 0)(2, 0)% right horizontal
  \psline[linestyle=dotted](2.2,0)(2.7,0)%
  \psline[linestyle=dotted](-1.2,0)(-1.7,0)%
  %-------------------------------------
  % labels
  %-------------------------------------
  \uput[-90](1,0){$2^{-j}$}%
  \uput{8pt}[180](0,1){$2^{j/2}$}%
  %\uput[0](2.5,0){$x$}%
  %\uput[0](10,5){$\fp(t)$}%
\end{pspicture}
%
\end{array}}

The scaling subspace $\spV_0$ is the span $\spV_0\eqd\Span\setxZ{\opTrn^n\fphi}$.
The scaling subspace $\spV_j$ is the span $\spV_j\eqd\Span\set{\opDil^j\opTrn^n\fphi}{n\in\Z}$.
  %$\opDil^j\fphi$ such that
Note that $\norm{\opDil^j\opTrn^n\fphi}$ for each resolution $j$ and shift $n$ is unity:
  \begin{align*}
    \norm{\opDil^j\opTrn^n\fphi}^2
      &= \norm{\fphi}^2  
      && \text{by \prefp{thm:TD_unitary}}
   %\\&= \int_{\intco{0}{\brp{2^{-j}}} \abs{\brp{\sqrt{2}}^j}^2 \dx
    \\&= \int_0^1 \abs{1}^2 \dx
      && \text{by definition of $\normn$ on $\spLLR$ \xref{def:spLLR}}
    %\\&= \brp{2^{-j}\brp{2^j}
    \\&= 1
  \end{align*}

\begin{minipage}{\tw-68mm}
Let $\ff(x)=\sin(\pi x)$.
Suppose we want to project $\ff(x)$ onto the subspaces $\spV_0$, $\spV_1$, $\spV_2$, \ldots.
\end{minipage}%
\hfill%
\begin{minipage}{64mm}%
  \mbox{}\\%
  \psset{unit=0.8mm}%
  %============================================================================
% Daniel J. Greenhoe
% LaTeX file
% sin(t)
% nominal unit = 8mm
%============================================================================
\begin{pspicture}(-3.5,-1.5)(3.5,1.5)%
  \psaxes[linecolor=axis]{<->}(0,0)(-3.5,-1.5)(3.5,1.5)%
  \psplot[plotpoints=100]{-3}{3}{x 180 mul sin}%
  \psplot[plotpoints=10,linestyle=dotted,linewidth=2pt]{3}{3.5}{x 180 mul sin}%
  \psplot[plotpoints=10,linestyle=dotted,linewidth=2pt]{-3.5}{-3}{x 180 mul sin}%
  %\rput[b](17.5,11){$\ff(t)\eqd\sin(\pi t)$}%
  %\rput[0](35,0){$t$}%
\end{pspicture}%%
\end{minipage}%
\\
\begin{minipage}{\tw-68mm}
\ragr
The values of the transform coefficients for the subspace $\spV_j$ are %illustrated in \prefp{fig:wavstrct_Haar_sin}
given by
\end{minipage}%
\hfill%
\begin{minipage}{64mm}
%\begin{figure}[t]
  %\mbox{}\\%
  \psset{unit=8mm}%
  %%============================================================================
% Daniel J. Greenhoe
% LaTeX file
% sin(t)
%============================================================================
%  \psset{unit=1mm}
\begin{pspicture}(-40,-15)(40,15)%
  \footnotesize
  \psset{linecolor=blue}%
  %\rput(0,0){% axis
  %  \psset{linecolor=axis}
  %  \multirput(-30,0)(10,0){7}{\psline(0,-1)(0,1)}% markers on x axis
  %  \psline{<->}(-35,0)(35,0)% x axis
  %  \psline{<->}(0,-15)(0,15)%    y axis
  %  \psline(-1,10)(1,10)%
  %  \psline(-1,-10)(1,-10)%
  %  \uput[180](0,10){$\frac{1}{\pi}$}% y=1
  %  \uput[0](0,-10){$\frac{-1}{\pi}$}% y=1
  %  \multido{\ival=-3+1,\ipos=-30+10}{7}{%
  %    \uput[-90](\ipos,0){$\ival$}% x=
  %    }%
  %  \uput[0](40,0){$t$}%
  %  }%
  \psaxes[linecolor=axis,unit=10,labels=x]{<->}(0,0)(-3.5,-1.5)(3.5,1.5)%
  \multirput(-20,0)(20,0){3}{\psline{-o}(0,0)(0,10)}%
  \multirput(-30,0)(20,0){4}{\psline{-o}(0,0)(0,-10)}%
  \uput[180](0,10){$\frac{2}{\pi}$}% y=1
  \uput[0](0,-10){$\frac{-2}{\pi}$}% y=1
  \uput[0](35,0){$t$}%
  \rput[b](17.5,10){$\inprod{\ff(t-n)}{\sin(\pi t)}$}%
\end{pspicture}
%
  %============================================================================
% Daniel J. Greenhoe
% LaTeX file
%
% transform of a sin(pi t) for the Haar k=0 subspace 
%
%     2
% --------- = 0.63661977236758134307553505349006
%    pi
% nominal unit = 8mm
%============================================================================
\begin{pspicture}(-3.5,-1.5)(4,1.5)%
  \psaxes[linecolor=axis,labels=none,ticks=y]{<->}(0,0)(-3.5,-1.5)(3.5,1.5)%
  \multirput(-2,0)(2,0){3}{\psline{-o}(0,0)(0,0.6366)}%
  \multirput(-3,0)(2,0){4}{\psline{-o}(0,0)(0,-0.6366)}%
  \uput[90](-3,0){$-3$}%
  \uput[-90](-2,0){$-2$}%
  \uput[90](-1,0){$-1$}%
  \uput[90](1,0){$1$}%
  \uput[-90](2,0){$2$}%
  \uput[90](3,0){$3$}%
  \uput[180](0,0.6366){$\frac{2}{\pi}$}% y=1
  \uput[0](0,-0.6366){$\frac{-2}{\pi}$}% y=1
  \uput[0](3.5,0){$n$}%
  \psplot[plotpoints=100,linestyle=dashed,linecolor=red,linewidth=1pt]{-3}{3}{x 180 mul sin}%
  %\rput[b](17.5,10){$\inprod{\ff(t-n)}{\sin(\pi t)}$}%
\end{pspicture}

%  \caption{Approximation of a sinusoid using the Haar MRA\label{fig:wavstrct_Haar_sin}}
%\end{figure}
\end{minipage}
\\
  \begin{align*}
    \brs{\opR_j\ff(x)}(n) 
      &=    \frac{1}{\norm{\opDil^j\opTrn^n\fphi}^2}\inprod{\ff(x)}{\opDil^j\opTrn^n\fphi} 
    %\\&=    \frac{1}{\cancelto{1}{\norm{\fphi}^2}}
    %        \inprod{\ff(x)}{\opDil^j\fphi\brp{x-n}}
    %  &&    \text{by definition of $\opTrn$ \xrefP{def:wav_opT}} 
    \\&=    \frac{1}{\cancelto{1}{\norm{\fphi}^2}}\inprod{\ff(x)}{2^{j/2}\fphi\brp{2^j x-n}} 
      &&    \text{by \prefp{prop:DjTn}}
    \\&=    2^{j/2} \inprod{\ff(x)}{\fphi\brp{2^j x-n}} 
    \\&=    2^{j/2} 
            \int_{2^{-j}n}^{2^{-j}(n+1)} \ff(x) \dx
    \\&=    2^{j/2} 
            \int_{2^{-j}n}^{2^{-j}(n+1)} \sin(\pi x) \dx
    \\&=    2^{j/2}
            \left. \brp{-\frac{1}{\pi}}\cos\brp{\pi x} \right|_{2^{-j} n}^{2^{-j}(n+1)}
    \\&=    \frac{2^{j/2}}{\pi}
            \brs{
              \cos\brp{{2^{-j}n\pi}} -
              \cos\brp{{2^{-j}(n+1)\pi}}  
              }
  \end{align*}



And the projection $\opA_n\ff(x)$ of the function $\ff(x)$ onto the subspace $\spV_j$ is
%(alternatively, the \hie{projection} of $\ff(x)$ \emph{onto} the space $\spV_j$ is) %\\
%\begin{minipage}{\tw-95mm}
  \begin{align*}
    \opA_j\ff(x) 
      &= \sum_{n\in\Z} \inprod{\ff(x)}{\opDil^j\opTrn^n\fphi} \opDil^j\opTrn^n\fphi 
    \\&= \frac{2^{j/2}}{\pi}
         \sum_{n\in\Z}
         \brs{
           \cos\brp{2^{-j}n\pi} -
           \cos\brp{2^{-j}(n+1)\pi}  
           } 2^{j/2}\fphi\brp{2^j x-n}
    \\&= \frac{2^j}{\pi}
         \sum_{n\in\Z}
         \brs{
           \cos\brp{2^{-j} n\pi} -
           \cos\brp{2^{-j}(n+1)\pi}  
           } \fphi\brp{2^j x-n}
  \end{align*}
%\end{minipage}%
%\hfill%
%\begin{minipage}{90mm}%
%  \mbox{}\\%
%  \psset{unit=8mm}%
%  %\input{../common/math/graphics/square_123_pi.tex}
%  \input{../common/wavelets/graphics/haar0_sin_a.tex}
%\end{minipage}

The transforms into the subspaces $\spV_0$, $\spV_1$, and $\spV_2$,
as well as the approximations in those subspaces are as illustrated next: %in \prefp{fig:wavstrct_psin}.

{\psset{unit=8mm}
\begin{longtable}{|l|l|l|}
  \hline
  \mc{1}{|c|}{subspace}&\mc{1}{c|}{transform}&\mc{1}{c|}{approximation}
  \\\hline\hline
  $\spV_0$
  & %============================================================================
% Daniel J. Greenhoe
% LaTeX file
%
% transform of a sin(pi t) for the Haar k=0 subspace 
%
%     2
% --------- = 0.63661977236758134307553505349006
%    pi
% nominal unit = 8mm
%============================================================================
\begin{pspicture}(-3.5,-1.5)(4,1.5)%
  \psaxes[linecolor=axis,labels=none,ticks=y]{<->}(0,0)(-3.5,-1.5)(3.5,1.5)%
  \multirput(-2,0)(2,0){3}{\psline{-o}(0,0)(0,0.6366)}%
  \multirput(-3,0)(2,0){4}{\psline{-o}(0,0)(0,-0.6366)}%
  \uput[90](-3,0){$-3$}%
  \uput[-90](-2,0){$-2$}%
  \uput[90](-1,0){$-1$}%
  \uput[90](1,0){$1$}%
  \uput[-90](2,0){$2$}%
  \uput[90](3,0){$3$}%
  \uput[180](0,0.6366){$\frac{2}{\pi}$}% y=1
  \uput[0](0,-0.6366){$\frac{-2}{\pi}$}% y=1
  \uput[0](3.5,0){$n$}%
  \psplot[plotpoints=100,linestyle=dashed,linecolor=red,linewidth=1pt]{-3}{3}{x 180 mul sin}%
  %\rput[b](17.5,10){$\inprod{\ff(t-n)}{\sin(\pi t)}$}%
\end{pspicture}

  & \input{../common/wavelets/graphics/haar0_sin_a.tex}
  \\\hline
  $\spV_1$
  & \psset{xunit=4mm}%============================================================================
% Daniel J. Greenhoe
% LaTeX file
%
% transform of a sin(pi t) for the Haar k=1 subspace 
%
%  sqrt(2)
% --------- = 0.45015815807855303477759959550337
%    pi
% nominal xunit = 4mm
% nominal yunit = 8mm
%============================================================================
\begin{pspicture}(-7,-1.5)(8,1.5)%
  \psaxes[linecolor=axis,labels=none,ticks=y]{<->}(0,0)(-7,-1.5)(7,1.5)%
  \multirput(-4,0)(4,0){3}{%
    \psline{-o}( 0,0)( 0,0.45016)
    \psline{-o}(1,0)(1,0.45016)
    }%
  \multirput(-6,0)(4,0){3}{%
    \psline{-o}(0,0)(0,-0.45016)
    \psline{-o}(1,0)(1,-0.45016)
    }%
  \uput[180](0,0.45016){$\frac{\sqrt{2}}{\pi}$}% y=1
  \uput[0](0,-0.45016){$\frac{-\sqrt{2}}{\pi}$}% y=1
  \uput[0](7,0){$n$}%
  \psplot[plotpoints=100,linestyle=dashed,linecolor=red,linewidth=1pt]{-6}{6}{x 90 mul sin}%
  %\rput[b](17.5,10){$\inprod{\ff(t-n)}{\sin(\pi t)}$}%
\end{pspicture}

  & \psset{xunit=4mm}%============================================================================
% Daniel J. Greenhoe
% LaTeX file
% 
%
% approximation of a sin(pi t) in the Haar k=1 subspace 
%
%  sqrt(2)
% --------- * sqrt(2)^k = 0.63661977236758134307553505349006
%    pi
% nominal xunit = 4mm
% nominal yunit = 8mm
%============================================================================
\begin{pspicture}(-7,-1.5)(8,1.5)%
  \psaxes[linecolor=axis,labels=none,ticks=all]{<->}(0,0)(-7,-1.5)(7,1.5)%
  \multirput(-4,0)(2,0){5}{% dotted connecting segments
    \psline[linestyle=dotted](0,-0.6366)(0,0.6366)%
    }%
  \multirput(-4,0)(4,0){3}{%
    \psline{*-o}(0,0.6366)(1,0.6366)
    \psline{*-o}(1,0.6366)(2,0.6366)
    }%
  \multirput(-6,0)(4,0){3}{%
    \psline{*-o}(0,-0.6366)(1,-0.6366)
    \psline{*-o}(1,-0.6366)(2,-0.6366)
    }%
  \uput[180](0,0.6366){$\frac{\sqrt{2}}{\pi}$}% y=1
  \uput[0](0,-0.6366){$\frac{-\sqrt{2}}{\pi}$}% y=1
  \uput[0](7,0){$x$}%
  \psplot[plotpoints=100,linestyle=dashed,linecolor=red,linewidth=1pt]{-6}{6}{x 90 mul sin}%
  %\rput[b](17.5,10){$\inprod{\ff(t-n)}{\sin(\pi t)}$}%
\end{pspicture}

  \\\hline
  $\spV_2$
  & \psset{xunit=2mm}%============================================================================
% Daniel J. Greenhoe
% LaTeX file
%
% transform of a sin(pi t) for the Haar k=2 subspace 
%
% 2-sqrt(2)
% --------- = 0.18646161428902830829793545798669
%    pi
%
%  sqrt(2)
% --------- = 0.45015815807855303477759959550337
%    pi
%
% nominal xunit = 2mm
% nominal yunit = 8mm
%============================================================================
\begin{pspicture}(-14,-1.5)(16,1.5)%
  \psaxes[linecolor=axis,labels=none,ticks=y]{<->}(0,0)(-14,-1.5)(14,1.5)%
  %\multirput(-4,0)(4,0){3}{%
  \multirput(-8,0)(8,0){2}{%
    \psline{-o}( 0,0)( 0, 0.1865)%
    \psline{-o}( 1,0)( 1, 0.4502)%
    \psline{-o}( 2,0)( 2, 0.4502)%
    \psline{-o}( 3,0)( 3, 0.1865)%
    \psline{-o}( 4,0)( 4,-0.1865)%
    \psline{-o}( 5,0)( 5,-0.4502)%
    \psline{-o}( 6,0)( 6,-0.4502)%
    \psline{-o}( 7,0)( 7,-0.1865)%
    }%
  \rput(8,0){%
    \psline{-o}( 0,0)( 0, 0.1865)%
    \psline{-o}( 1,0)( 1, 0.4502)%
    \psline{-o}( 2,0)( 2, 0.4502)%
    \psline{-o}( 3,0)( 3, 0.1865)%
    \psline{-o}( 4,0)( 4,-0.1865)%
    %\psline{-o}( 5,0)( 5, 0.4502)%
    %\psline{-o}( 6,0)( 6, 0.4502)%
    %\psline{-o}( 7,0)( 7,-0.1865)%
    }%
  \rput(-12,0){%
    \psline{-o}( 0,0)( 0,-0.1865)%
    \psline{-o}( 1,0)( 1,-0.4502)%
    \psline{-o}( 2,0)( 2,-0.4502)%
    \psline{-o}( 3,0)( 3,-0.1865)%
    \psline{-o}( 4,0)( 4, 0.1865)%
    %\psline{-o}( 5,0)( 5,-0.4502)%
    %\psline{-o}( 6,0)( 6,-0.4502)%
    %\psline{-o}( 7,0)( 7, 0.1865)%
    }%
 %\uput[180](0,0.45016){$\frac{\sqrt{2}}{\pi}$}% y=1
 %\uput[0](0,-0.45016){$\frac{-\sqrt{2}}{\pi}$}% y=1
  \uput[0](14,0){$n$}%
  \psplot[plotpoints=100,linestyle=dashed,linecolor=red,linewidth=0.5pt]{-12}{12}{x 45 mul sin}%
\end{pspicture}

  & \psset{xunit=2mm}%============================================================================
% Daniel J. Greenhoe
% LaTeX file
%
% approximation of a sin(pi t) in the Haar k=2 subspace 
% 2-sqrt(2)
% --------- * sqrt(2)^k = 0.37292322857805661659587091597337
%    pi
%
%  sqrt(2)
% --------- * sqrt(2)^k = 0.90031631615710606955519919100674
%    pi
%
% nominal xunit = 2mm
% nominal yunit = 8mm
%============================================================================
\begin{pspicture}(-14,-1.5)(16,1.5)%
  \psaxes[linecolor=axis,labels=none,ticks=all]{<->}(0,0)(-14,-1.5)(14,1.5)%
  %\multirput(-4,0)(4,0){3}{%
  \multirput(-8,0)(8,0){2}{%
    \psline[linestyle=dotted]( 0,0)( 0, 0.3729)\psline{*-o}( 0, 0.3729)( 1, 0.3729)%
    \psline[linestyle=dotted]( 1, 0.3729)( 1, 0.9003)\psline{*-o}( 1, 0.9003)( 2, 0.9003)%
    \psline[linestyle=dotted]( 2, 0.9003)( 2, 0.9003)\psline{*-o}( 2, 0.9003)( 3, 0.9003)%
    \psline[linestyle=dotted]( 3, 0.9003)( 3, 0.3729)\psline{*-o}( 3, 0.3729)( 4, 0.3729)%
    \psline[linestyle=dotted]( 4, 0.3729)( 4,-0.3729)\psline{*-o}( 4,-0.3729)( 5,-0.3729)%
    \psline[linestyle=dotted]( 5,-0.3729)( 5,-0.9003)\psline{*-o}( 5,-0.9003)( 6,-0.9003)%
    \psline[linestyle=dotted]( 6,-0.9003)( 6,-0.9003)\psline{*-o}( 6,-0.9003)( 7,-0.9003)%
    \psline[linestyle=dotted]( 7,-0.9003)( 7,-0.3729)\psline{*-o}( 7,-0.3729)( 8,-0.3729)%
    }%                       
  \rput(8,0){%
    \psline[linestyle=dotted]( 0,-0.3729)( 0, 0.3729)\psline{*-o}( 0, 0.3729)( 1, 0.3729)%
    \psline[linestyle=dotted]( 1, 0.3729)( 1, 0.9003)\psline{*-o}( 1, 0.9003)( 2, 0.9003)%
    \psline[linestyle=dotted]( 2, 0.9003)( 2, 0.9003)\psline{*-o}( 2, 0.9003)( 3, 0.9003)%
    \psline[linestyle=dotted]( 3, 0.9003)( 3, 0.3729)\psline{*-o}( 3, 0.3729)( 4, 0.3729)%
    \psline[linestyle=dotted]( 4, 0.3729)( 4,-0.3729)\psline{*-o}( 4,-0.3729)( 5,-0.3729)%
    }%                       ( 4,-0.3729)
  \rput(-12,0){%
    \psline[linestyle=dotted]( 0,0)( 0,-0.3729)\psline{*-o}( 0,-0.3729)( 1,-0.3729)%
    \psline[linestyle=dotted]( 1,-0.3729)( 1,-0.9003)\psline{*-o}( 1,-0.9003)( 2,-0.9003)%
    \psline[linestyle=dotted]( 2,-0.9003)( 2,-0.9003)\psline{*-o}( 2,-0.9003)( 3,-0.9003)%
    \psline[linestyle=dotted]( 3,-0.9003)( 3,-0.3729)\psline{*-o}( 3,-0.3729)( 4,-0.3729)%
    \psline[linestyle=dotted]( 4,-0.3729)( 4, 0.3729)\psline{*-o}( 4, 0.3729)( 5, 0.3729)%
    }%                             ( 4, 0.3729)
 %\uput[180](0,0.45016){$\frac{\sqrt{2}}{\pi}$}% y=1
 %\uput[0](0,-0.45016){$\frac{-\sqrt{2}}{\pi}$}% y=1
  \uput[0](14,0){$x$}%
  \psplot[plotpoints=100,linestyle=dashed,linecolor=red,linewidth=0.5pt]{-12}{12}{x 45 mul sin}%
\end{pspicture}

  \\\hline
\end{longtable}}

\end{example}




%=======================================
\subsubsection{Necessary Conditions}
%=======================================
%Next we look at  two necessary conditions in the ``time domain" for scaling coefficient design.
%%They can be used in generating simultaneous equations for wavelet system design.
%\\\indentx
%  \begin{tabular}{@{\qquad}clll}
%    \imark & \hie{admissibility condition}: & \pref{thm:admiss}        & \xrefP{thm:admiss} \\
%    \imark & \hie{quadrature condition}:    & \pref{thm:wav_quadcon}   & \xrefP{thm:wav_quadcon}
%  \end{tabular}

%--------------------------------------
\begin{theorem}[\thmd{admissibility condition}]
\label{thm:admiss}
%--------------------------------------
%Let $\mrasys$ be a multiresolution system.
Let $\Zh(z)$ be the \fncte{Z-transform} \xref{def:opZ} and 
$\Dh(\omega)$ the \fncte{discrete-time Fourier transform} \xref{def:dtft} of a sequence $\seqxZ{h_n}$.
\thmbox{\begin{array}{M}
  %\brb{\begin{array}{M}$\mrasys$\\is an \structe{MRA system}\end{array}} &\impnotimpby&
  $\brb{\text{$\mrasys$ is an \structe{MRA system} \xref{def:mrasys}}}$
  \\$\ds\quad
  \impnotimpby \mcom{\brb{\sum_{n\in\Z}  h_n  = \sqrt{2}}}{(1) \prope{admissibility} in ``time"}               
  \iff         \mcom{\brb{\Zh(z)\Big|_{z=1}   = \sqrt{2}}}{(2) \prope{admissibility} in ``z domain"}                  
  \iff         \mcom{\brb{\Dh(\omega)\Big|_{\omega=0} = \sqrt{2}}}{(3) \prope{admissibility} in ``frequency"}
  $
\end{array}}
\end{theorem}
\begin{proof}
\begin{enumerate}
  \item Proof that MRA system $\implies$ (1):
    \begin{align*}
      \sum_{n\in\Z} h_n
        &= \frac{\int_\R \fphi(x) \dx}{\int_\R \fphi(x) \dx} \sum_{n\in\Z} h_n
      \\&= \frac{1}{\int_\R \fphi(x) \dx} \int_\R \sum_{n\in\Z} h_n \fphi(x) \dx
      \\&= \frac{1}{\int_\R \fphi(x) \dx} \int_\R \sum_{n\in\Z} h_n \frac{\sqrt{2}}{\sqrt{2}}\fphi(2y-n) 2\dy
        && \text{let $y\eqd \frac{x+n}{2}\implies x=2y-n \implies \dx=2\dy$}
      \\&= \frac{2}{\sqrt{2}}\frac{1}{\int_\R \fphi(x) \dx} \int_\R \sum_{n\in\Z} h_n \opDil\opTrn^n\fphi(y)\dy
        && \text{by definitions of $\opTrn$ and $\opDil$ \xref{def:opT}}
      \\&= \sqrt{2} \frac{1}{\int_\R \fphi(x) \dx} \int_\R \fphi(y) \dy
        && \text{by \thme{dilation equation} \xref{thm:dilation_eq}}
      \\&= \sqrt{2}
    \end{align*}

  \item Alternate proof that MRA system $\implies$ (1):\\
    %Select a vector $\ff$ such that $\inprod{\fphi}{\ff}\ne 0$.
    Let $\ff(x)\eqd 1\quad\forall x\in\R$. % be a constant vector (e.g. $\ff(x)=1$). Then \ldots
    \begin{align*}
      \inprod{\fphi}{\ff}
        &= \inprod{\sum_{n\in\Z} h_n \opDil\opTrn^n \fphi}{\ff}
        && \text{by dilation equation \xrefP{thm:dilation_eq}}
      \\&= \sum_{n\in\Z} h_n \inprod{\opDil\opTrn^n \fphi}{\ff}
        && \text{by \prop{linearity} of $\inprodn$ \ifxref{vsinprod}{def:inprod}}
      \\&= \sum_{n\in\Z} h_n \inprod{ \fphi}{(\opDil\opTrn^n)^\ast\ff}
        && \text{by definition of operator adjoint \ifxref{operator}{thm:op_star}}
      \\&= \sum_{n\in\Z} h_n \inprod{ \fphi}{(\opTrna)^n \opDila \ff}
        && \text{by property of operator adjoint \ifxref{operator}{thm:op_star}}
      \\&= \sum_{n\in\Z} h_n \inprod{ \fphi}{(\opTrni)^n \opDili \ff}
        && \text{by unitary property of $\opTrn$ and $\opDil$ \xrefP{prop:TD_unitary}}
      \\&= \sum_{n\in\Z} h_n \inprod{ \fphi}{(\opTrni)^n \cwt  \ff}
        && \text{because $\ff$ is a constant hypothesis and by \prefp{prop:opDi}}
      \\&= \sum_{n\in\Z} h_n \inprod{ \fphi}{ \cwt \ff}
        && \text{by $\ff(x)=1$ definition}
      \\&= \sum_{n\in\Z} h_n \cwt \inprod{ \fphi}{ \ff}
        && \text{by property of $\inprodn$ \ifxref{vsinprod}{def:inprod}}
      \\&= \cwt \; \inprod{\fphi}{\ff}\; \sum_{n\in\Z} h_n
      \\&\implies \sum_{n\in\Z} h_n = \sqrt{2}
    \end{align*}

  \item Proof that (1) $\iff$ (2) $\iff$ (3): by \prefp{prop:tzf}.

  \item Proof for $\notimpliedby$ part: by \prefp{cnt:admiss}.
\end{enumerate}
\end{proof}

%--------------------------------------
\begin{counterex}
\label{cnt:admiss}
%--------------------------------------
Let $\mrasys$ be an \structe{MRA system} \xref{def:mrasys}.
\cntbox{\begin{array}{M}
  $\brb{\begin{array}{lm{28mm}}
     \seqn{h_n} \eqd \sqrt{2}\kdelta_{n-1} \eqd
       \brbl{\begin{array}{lD}
         \sqrt{2}     & for $n=1$  \\
         0            & otherwise.
       \end{array}}
    &%============================================================================
% Daniel J. Greenhoe
% LaTeX file
% nominal unit = 8mm
%============================================================================
\begin{pspicture}(-0.5,-0.875)(2.5,0.875)%\scs
  \psset{%
    linecolor=blue,%
    labelsep=7pt,%
    }%
  \psaxes[linecolor=axis,yAxis=false,linewidth=0.5pt](0,0)(-0.5,0)(2.5,1.5)%
  %\psline{-o}(0,0)(0,0.707)%
  \psline{-o}(1,0)(1,0.707)%
  %\uput[225](0,1){$\frac{\sqrt{2}}{2}$}
  \uput[-45](1,1){$\sqrt{2}$}%
\end{pspicture}%
 
  \end{array}}
  \quad\implies\quad
  \brb{\fphi(x)=0}$
  \\
  which means
  \\
  $\ds\brb{\sum_{n\in\Z} h_n = \sqrt{2}} \quad\notimplies\quad \brb{\text{$\mrasys$ is an MRA system for $\spLLR$.}}$
\end{array}}
\end{counterex}
\begin{proof}
\begin{align*}
  \fphi(x)
    &= \sum_{n\in\Z} h_n \opDil\opTrn^n\fphi(x)
    && \text{by \thme{dilation equation} \xref{thm:dilation_eq}}
  \\&= \sum_{n\in\Z} h_n \fphi(2x-n)
    && \text{by definitions of $\opDil$ and $\opTrn$ \xref{def:opT}}
  \\&= \sum_{n\in\Z} \mcom{\sqrt{2}\kdelta_{n-1}}{$\seqn{h_n}$} \fphi(2x-n)
    && \text{by definitions of $\seqn{h_n}$}
  \\&= \sqrt{2}\fphi(2x-1)
    && \text{by definition of $\fphi(x)$}
  \\\implies
  \fphi(x) &= 0
\end{align*}
This implies $\fphi(x)=0$, which implies that $\mrasys$ is \emph{not} an \structe{MRA system} for $\spLLR$ because
  \\\indentx$\ds \clsp{\Setu_{j\in\Z} \spV_j} = \clsp{\Setu_{j\in\Z} \linspan\set{\opDil^j\opTrn^n\fphi}{\scy n\in\Z}} \neq \spLLR$\\
(the \structe{least upper bound} is \emph{not} $\spLLR$).
\end{proof}




%--------------------------------------
\begin{theorem}[\thmd{Quadrature condition} in ``time"]
\label{thm:wav_quadcon}
\label{thm:wav_hh}
%--------------------------------------
Let $\mrasys$ be an \structe{MRA system} \xref{def:mrasys}.
\thmbox{
  \sum_{m\in\Z} h_m \sum_{k\in\Z} h_k^\ast \inprod{\fphi}{\opTrn^{2n-m+k} \fphi}
  =\inprod{\fphi}{\opTrn^n \fphi}
  \qquad\scy\forall n\in\Z
  }
\end{theorem}
\begin{proof}
\begin{align*}
  \inprod{\fphi}{\opTrn^n \fphi}
    &= \inprod{\sum_{m\in\Z} h_m \opDil \opTrn^m \fphi }{\opTrn^n \sum_{k\in\Z} h_k \opDil \opTrn^k \fphi}
    && \text{by dilation equation \xrefP{thm:dilation_eq}}
  \\&= \sum_{m\in\Z} h_m \sum_{k\in\Z} h_k^\ast \inprod{\opDil \opTrn^m \fphi }{\opTrn^n \opDil \opTrn^k \fphi}
    && \text{by properties of $\inprodn$ \ifxref{vsinprod}{def:inprod}}
  \\&= \sum_{m\in\Z} h_m \sum_{k\in\Z} h_k^\ast \inprod{\fphi }{\left(\opDil \opTrn^m \right)^\ast \opTrn^n \opDil \opTrn^k \fphi}
    && \text{by definition of operator adjoint \ifxref{operator}{prop:op_adjoint}}
  \\&= \sum_{m\in\Z} h_m \sum_{k\in\Z} h_k^\ast \inprod{\fphi }{\left(\opDil \opTrn^m \right)^\ast \opDil \opTrn^{2n} \opTrn^k \fphi}
    && \text{by \prefp{prop:DTTD}}
  \\&= \sum_{m\in\Z} h_m \sum_{k\in\Z} h_k^\ast \inprod{\fphi }{\opTrna^m \opDila \opDil \opTrn^{2n} \opTrn^k \fphi}
    && \text{by operator star-algebra properties \ifxref{operator}{thm:op_star}}
  \\&= \sum_{m\in\Z} h_m \sum_{k\in\Z} h_k^\ast \inprod{\fphi }{\opTrn^{-m} \opDil^{-1} \opDil \opTrn^{2n} \opTrn^k \fphi}
    && \text{by \prefp{prop:TD_unitary}}
  \\&= \sum_{m\in\Z} h_m \sum_{k\in\Z} h_k^\ast \inprod{\fphi }{\opTrn^{2n-m+k} \fphi}
\end{align*}
\end{proof}

%%---------------------------------------
%\begin{theorem}[Neumann Expansion Theorem]
%\index{Neumann Expansion Theorem}
%\thmx{Neumann Expansion Theorem}
%\label{thm:op_net2}
%\citep{michel1993}{415}
%%---------------------------------------
%Let $\opA\in\clFxx$ be an operator on a linear space $\spX$.
%Let $\opA^0\eqd \opI$.
%\thmbox{\begin{array}{ll}
%  \left.\begin{array}{lrclD}
%    1. & \opA          &\in& \oppB(\spX,\spX) & ($\opA$ is bounded) \\
%    2. & \normop{\opA} &<&   1
%  \end{array}\right\}
%  \implies
%  \left\{\begin{array}{lrc>{\ds}l}
%    1. & (\opI-\opA)^{-1} &&\text{ exists} \\
%    2. & \normop{(\opI-\opA)^{-1}} &\le& \frac{1}{1-\normop{\opA}} \\
%    3. & (\opI-\opA)^{-1} &=& \sum_{n=0}^\infty \opA^\xN  \\
%       & \mc{3}{c}{\text{ with uniform convergence}}
%  \end{array}\right.
%\end{array}}
%\end{theorem}
%
%
%
%
%
%
%
%%--------------------------------------
%\begin{theorem}
%\label{thm:wav_net}
%\thmx{$\sum_{n\in\Z} \abs{h_n} \ge 1$}
%%--------------------------------------
%Let $\wavsys$ be a \hi{wavelet system}.
%\thmbox{
%  \sum_{n\in\Z} \abs{h_n} \ge 1
%  }
%\end{theorem}
%\begin{proof}
%\begin{align*}
%  &&
%  \fphi &= \sum_{n\in\Z} h_n \opTrn^n \opDil \fphi
%  \\\implies&&
%  \left(\opI - \sum_{n\in\Z} h_n \opTrn^n \opDil \right)\fphi &= \vzero
%  \\\implies&&
%  \left(\opI - \sum_{n\in\Z} h_n \opTrn^n \opDil \right)^{-1} & \text{must not exist}
%  \\\implies&&
%  \normop{\sum_{n\in\Z} h_n \opTrn^n \opDil} & \ge 1
%    && \text{by Neumann Expansion Theorem \xrefP{thm:op_net2}}
%  \\\implies&&
%  1
%      &\le \normop{\sum_{n\in\Z} h_n \opTrn^n \opDil}
%     &&    %\text{by Neumann Expansion Theorem \xrefP{thm:op_net2}}
%  \\&&&\le \sum_{n\in\Z}  \normop{h_n \opTrn^n \opDil}
%     &&    \text{by generalized triangle inequality \ifdochas{vsnorm}{\xrefP{thm:norm_tri}}}
%  \\&&&=   \sum_{n\in\Z}  \abs{h_n}\; \cancelto{1}{\normop{ \opTrn^n \opDil}}
%     &&    \text{by homogeneous property of norm \ifdochas{vsnorm}{\xrefP{def:norm}}}
%  \\&&&=   \sum_{n\in\Z}  \abs{h_n}
%     &&    \ifdochas{operator}{\text{by \prefp{prop:op_unitary_UV} and \prefp{thm:unitary_prop}}}
%\end{align*}
%\end{proof}




\pref{thm:gen_quadcon} (next) presents the \structe{quadrature} necessary conditions of a \hi{wavelet system}.
These relations simplify dramatically in the special case of an
\structe{orthonormal wavelet system} \xrefP{thm:oquadcon}.
%--------------------------------------
\begin{theorem}[\thmd{Quadrature condition} in ``frequency"]
\footnote{
  \citerp{chui}{135}       \\
  \citerp{goswami}{110}
  }
\label{thm:Sphi}
%--------------------------------------
Let $\mrasys$ be an \structe{MRA system} \xref{def:mrasys}.
Let $\Fx(\omega)$ be the \fncte{discrete time Fourier transform}\ifsxrefs{dsp}{def:dtft}for a sequence $\seqxZ{x_n}$ in $\spllR$.
Let $\Swphi(\omega)$ be the \fncte{auto-power spectrum} \xref{def:Swfg} of $\fphi$.
\thmbox{\begin{array}{>{\ds}lc>{\ds}l}
   \abs{\Dh\left(\omega     \right)}^2 \rnode[b]{noteSphi1}{\Sphi}(\omega) + \abs{\Dh\brp{\omega+\pi }}^2 \rnode[b]{noteSphi2}{\Sphi}(\omega+\pi) &=& 2\rnode[b]{noteSphi3}{\Sphi}(2\omega)
\end{array}}
\hfill
\rnode[bl]{noteSphi}{\footnotesize$\brp{\begin{array}{M}
  Note: $\Sphi(\omega)=1$\\% $\iff$ $\setxZ{\opTrn^n\fphi}$\\
  for \prope{orthonormal} MRA
\end{array}}$}
\ncarc[arcangle=30,linewidth=0.5pt]{->}{noteSphi}{noteSphi1}
\ncarc[arcangle=30,linewidth=0.5pt]{->}{noteSphi}{noteSphi2}
\ncarc[arcangle=30,linewidth=0.5pt]{->}{noteSphi}{noteSphi3}
\end{theorem}
\begin{proof}
%\begin{enumerate}
%  \item First note that $\Dh(\omega)$ and $\Dg(\omega)$ are periodic with period $2\pi$ such that\ifsxrefs{dsp}{prop:dtft_2pi} 
%  \label{item:qc1}
%    \\\indentx$\begin{array}{rclC}
%         \Dh(\omega+2\pi n) &=& \Dh(\omega) & \forall n\in\Z   \\
%         \Dg(\omega+2\pi n) &=& \Dg(\omega) & \forall n\in\Z   
%       \end{array}$
%
%\item Proof for theorem statement:
\begin{align*}
   &2\Sphi(2\omega)
   \\&= 2\brp{2\pi}\sum_{n\in\Z} \left|\Fphi(2\omega+2\pi n)\right|^2
         \indentx\text{by \prefp{thm:Swfg}}
   \\&= 2\brp{2\pi}\sum_{n\in\Z} \left|\cwt \Dh\left(\frac{2\omega+2\pi n}{2}\right)\Fphi\left(\frac{2\omega+2\pi n}{2}\right)\right|^2
        \indentx\text{by \prefp{lem:Fphi}}
   \\&= {2\pi}\sum_{n\in\Ze} \left|\Dh\left(\frac{2\omega+2\pi n}{2}\right)\right|^2\left|\Fphi\left(\frac{2\omega+2\pi n}{2}\right)\right|^2 +
        {2\pi}\sum_{n\in\Zo} \left|\Dh\left(\frac{2\omega+2\pi n}{2}\right)\right|^2\left|\Fphi\left(\frac{2\omega+2\pi n}{2}\right)\right|^2
   \\&= 2\pi\sum_{n\in\Z} \left|\Dh\left(\omega+2\pi n\right)\right|^2\left|\Fphi\left(\omega+2\pi n\right)\right|^2 +
        2\pi\sum_{n\in\Z} \left|\Dh\left(\omega+2\pi n+\pi \right)\right|^2\left|\Fphi\left(\omega+2\pi n+ \pi \right)\right|^2
   \\&= 2\pi\sum_{n\in\Z} \left|\Dh\left(\omega\right)\right|^2\left|\Fphi\left(\omega+2\pi n\right)\right|^2 +
        2\pi\sum_{n\in\Z} \left|\Dh\left(\omega+\pi \right)\right|^2\left|\Fphi\left(\omega+2\pi n+ \pi \right)\right|^2
        \indentx\text{by \prefp{prop:dtft_2pi}}
        %\indentx\text{by (\prefp{item:qc1})}
   \\&= \left|\Dh\left(\omega     \right)\right|^2 \brp{2\pi\sum_{n\in\Z} \left|\Fphi\left(\omega    +2\pi n\right)\right|^2} +
        \left|\Dh\left(\omega+\pi \right)\right|^2 \brp{2\pi\sum_{n\in\Z} \left|\Fphi\left(\omega+\pi+2\pi n\right)\right|^2}
   \\&= \left|\Dh\left(\omega     \right)\right|^2 \Sphi(\omega) +
        \left|\Dh\left(\omega+\pi \right)\right|^2 \Sphi(\omega+\pi)
        \indentx\text{by \prefp{thm:Swfg}}
\end{align*}

%\end{enumerate}
\end{proof}


%=======================================
\subsubsection{Sufficient conditions}
%=======================================
\pref{thm:mra_rdc} (next) gives a set of \emph{sufficient} conditions on the \fncte{scaling function} \xref{def:mra} 
$\fphi$ to generate an \structe{MRA}.
\prefpp{thm:h_ns} provides a set of sufficient conditions on the \fncte{scaling coefficients} \xref{def:hn} $\seqnZ{h_n}$ 
to generate an \structe{MRA}; howbeit, this set results in the more restrictive \prope{orthonormal} MRA.
%---------------------------------------
\begin{theorem}
\footnote{
  \citerpgc{wojtaszczyk1997}{28}{0521578949}{Theorem 2.13}\\
  \citerpgc{pinsky2002}{313}{0534376606}{Theorem 6.4.27}
  }
\label{thm:mra_rdc}
\label{thm:mra_sufficient_phi}
%--------------------------------------
Let an \structe{MRA} be defined as in \prefp{def:mra}.
Let a \structe{Riesz sequence} be defined as in \prefp{def:rieszseq}.
Let $\spV_j\eqd\linspan\setxZ{\opTrn\fphi(x)}$.
\thmbox{
  \brb{\begin{array}{FMD}
    (1). & $\seqn{\opTrn^n\fphi}$ is a \structe{Riesz sequence}                        & and \\
    (2). & $\ds\exists \seqn{h_n} \st \fphi(x)=\sum_{n\in\Z}h_n\opDil\opTrn^n\fphi(x)$ & and \\
    (3). & $\Fphi(\omega)$ is \prope{continuous} at $0$                                & and \\
    (4). & $\Fphi(0)\neq0$
  \end{array}}
  \qquad\implies\qquad
  \brb{\text{$\seqjZ{\spV_j}$ is an \structe{MRA}}}
  }
\end{theorem}
\begin{proof}
For this to be true, each of the conditions in the definition of an \structe{MRA} \xref{def:mra} must be satisfied:
\begin{enumerate}
  \item Proof that each $\spV_j$ is \prope{closed}: by definition of $\linspan$

  \item Proof that $\seqn{\spV_j}$ is \prope{linearly ordered}: 
    \begin{align*}
      \spV_j\subseteq\spV_{j+1}
        &\iff \linspan\setn{\opDil^j\opTrn^n\fphi}\subseteq\linspan\setn{\opDil^{j+1}\opTrn^n\fphi}
        &\iff (2)
    \end{align*}

  \item Proof that $\ds\Setu_{j\in\Z}\spV_j$ is \prope{dense} in $\spLLR$: by \prefp{prop:mra_UVj}
  
  \item Proof of \prope{self-similar} property:
    \begin{align*}
      \brb{\ff\in\spV_j\iff\opDil\ff\in\spV_{j+1}}
        &\iff \ff\in\linspan\setn{\opTrn^n\fphi}\iff\opDil\ff\in\linspan\setn{\opDil\opTrn^n\fphi}
        &\iff (2)
    \end{align*}

  \item Proof for \structe{Riesz basis}: by (1) and \prefp{prop:mra_UVj}.
\end{enumerate}
\end{proof}



%=======================================
\subsection{Wavelet analysis}
%=======================================
%=======================================
\subsubsection{Definition}
%=======================================
The term ``wavelet" comes from the French word ``\hie{ondelette}", meaning ``small wave". 
And in essence, wavelets are ``small waves" (as opposed to the ``long waves" of Fourier analysis) 
that form a basis for the Hilbert space $\spLLR$.\footnote{
  \citerpg{strang1996}{ix}{0961408871}\\
  \citerpg{atkinson2009}{191}{1441904581}
  }
%---------------------------------------
\begin{definition}
\footnote{
  \citerpgc{wojtaszczyk1997}{17}{0521578949}{Definition 2.1}
  }
\label{def:wavelet}
\label{def:seqWn}
\label{def:wavstrct_psi}
%---------------------------------------
%Let $\MRAspaceLLRV$ be an \structe{multiresolution space} \xref{def:mra}.
Let $\opTrn$ and $\opDil$ be as defined in \prefp{def:opT}.
\defboxp{
  A function $\fpsi(x)$ in $\spLLR$ is a \hid{wavelet function} for $\spLLR$ if
  \\\indentx$\set{\opDil^j\opTrn^n\fpsi}{\scy j,n\in\Z}$ is a \structe{Riesz basis} for $\spLLR$.\\
  In this case, $\fpsi$ is also called the \hid{mother wavelet} of the basis $\set{\opDil^j\opTrn^n\fpsi}{\scy j,n\in\Z}$.
  The sequence of subspaces $\seqjZ{\spW_j}$ is the \hid{wavelet analysis} induced by $\fpsi$, 
  where each subspace $\spW_j$ is defined as
  \\\indentx$\spW_j\eqd\linspan\setnZ{\opDil^j\opTrn^n\fpsi}$ .
  }
\end{definition}

%---------------------------------------
%\begin{definition}
%---------------------------------------
%Let $\MRAspaceLLRV$ be an \structe{multiresolution space} \xref{def:mra}.
%Let the operation $\adds$ represent \hie{Minkowski addition} on $\spLLR$\ifsxref{subspace}{def:sub_add}.%
%\defboxt{
%  %The \hid{wavelet subspace} $\spW_j$ is the orthogonal complement of $\spV_j$ in $\spV_{j+1}$ such that
%  %  \\\indentx$\spV_j \adds \spW_j = \spV_{j+1}$
%  %  \\
%  The sequence $\seqjZ{\spW_j}$ is a \hid{wavelet analysis} on $\MRAspaceLLRV$ if
%  \\\indentx$\begin{array}{FMCD}
%    1. & $\spV_{j+1} = \mcom{\spV_j \adds \spW_j}{\hi{Minkowski addition}}$
%       & \forall \spW_j \in \seqxZ{\spW_j}\text{ and }\spV_j\in\seqjZ{\spV_j}
%       & and
%     %\qquad\text{\scriptsize ($\spW_j$ is the complement of $\spV_j$ in $\spV_{j+1}$)}
%    \\
%    2. & \mc{2}{M}{There exists $\fpsi\in\spLLR$ such that $\seqxZ{\opTrn^n\fpsi}$ is a \structe{Riesz basis} for $\spW_0$.}
%  \end{array}$
%  }
%\end{definition}

%%---------------------------------------
%\begin{proposition}[complemented subspaces]
%%---------------------------------------
%Let $\MRAspaceLLRV$ be an \structe{MRA space}.
%Let $\spV_j^\orthog$ be the \structe{orthogonal complement} \xrefP{def:sub_ocomp} of $\spV_j$.
%\propbox{
%  \spV_j^{\orthog\orthog}=\spV_j \qquad\scy\forall n\in\Z \qquad\scs\text{\prope{involutory}}
%  }
%\end{proposition}
%\begin{proof}
%\begin{enume}
%  \item By \pref{def:mra}, $\spV_j$ is \prope{closed} in $\spLLR$ ($\spV_j=\cls\spV_j$).
%  \item By \prefpp{thm:inprod_orthog}, $\spV_j^{\orthog\orthog}=\spV_j$.
%\end{enume}
%\end{proof}

A \structe{wavelet analysis} $\seqn{\spW_j}$ is often constructed from a \structe{multiresolution anaysis} \xref{def:mra}
$\seqn{\spV_j}$ under the relationship
\\\indentx$\ds\spV_{j+1} = \spV_j \adds \spW_j$,\qquad where $\adds$ is subspace addition (\ope{Minkowski addition}).\\
By this relationship alone, $\seqn{\spW_j}$ is in no way uniquely defined 
in terms of a multiresolution analysis $\seqn{\spV_j}$.
In general there are many possible complements of a subspace $\spV_j$.
To uniquely define such a wavelet subspace, one or more additional constraints are required.
One of the most common additional constraints is \hie{orthogonality}, such that
$\spV_j$ and $\spW_j$ are orthogonal to each other (see \prefp{chp:ortho}).




%=======================================
\subsubsection{Dilation equation}
%=======================================
Suppose $\seqxZ{\opTrn^n\fpsi}$ is a basis for $\spW_0$.
By \prefp{def:seqWn}, the wavelet subspace $\spW_0$ is contained in the 
scaling subspace $\spV_1$.
By \prefp{def:mra}, the sequence $\seqxZ{\opDil\opTrn^n\fphi}$ is a basis for $\spV_1$.
Because $\spW_0$ is contained in $\spV_1$,
the sequence $\seqxZ{\opDil\opTrn^n\fphi}$ is also a basis for $\spW_0$.

%--------------------------------------
\begin{theorem}
\label{thm:g->psi}
%--------------------------------------
Let $\mrasys$ be a multiresolution system
and $\seqjZ{\spW_j}$ a wavelet analysis with respect to $\mrasys$ and with wavelet function $\fpsi$.
\thmbox{
  \begin{array}{l rc>{\ds}l @{\qquad}D}
    \exists \seqxZ{g_n} \st
      & \fpsi &=& \sum_{n\in\Z}  g_n \opDil \opTrn^n \fphi
      & 
  \end{array}}
\end{theorem}
\begin{proof}
\begin{align*}
  \fpsi &\in \spW_0
        && \text{by \prefp{def:seqWn}}
      \\&\subseteq \spV_1
        && \text{by \prefp{def:seqWn}}
      \\&= \Span\seqxZ{\opDil\opTrn^n\fphi(x)}
        && \text{by \prefp{def:mra} (MRA)}
      \\&\implies 
         \exists \seqxZ{g_n} \st
            \fpsi = \sum_{n\in\Z}  g_n \opDil \opTrn^n \fphi
\end{align*}

%\item Proof that $\fpsi(x) = \sqrt{2} \sum_{n\in\Z}  g_n  \fphi(2x-n)$:
%\begin{align*}
%              &  \set{\fphi(2x-n)}{n\in\Z} \text{ spans } \spV_1
%              && \text{by (1)}
%  \\
%              &  \set{\fpsi(x-n)}{n\in\Z} \text{ spans } \spW_0
%  \\ \implies & \fpsi(x)\in \spW_0 \subset \spV_1
%  \\ \implies & \text{$\fpsi(x)$ can be represented as a linear combination of $\fphi(2x-n)$}.
%\end{align*}
%\end{enumerate}
\end{proof}

A \structe{wavelet system} (next definition) consists of two subspace sequences: 
\begin{liste}
  \item A \hib{multiresolution analysis} $\seqn{\spV_j}$ \xrefP{def:mra}
     provides ``coarse" approximations of a function in $\spLLR$ at different ``scales" or resolutions.
  \item A \hib{wavelet analysis} $\seqn{\spW_j}$
     provides the ``detail" of the function missing from the approximation provided by a given scaling subspace
     \xrefP{def:seqWn}.
\end{liste}

%--------------------------------------
\begin{definition}
\label{def:wavsys}
\label{def:gn}
%--------------------------------------
Let $\mrasys$ be a multiresolution system \xrefP{def:mra}
and $\seqjZ{\spW_j}$ a wavelet analysis \xrefP{def:seqWn}
with respect to $\seqjZ{\spV_j}$.
Let $\seqxZ{g_n}$ be a sequence of coefficients such that 
$\fpsi = \sum_{n\in\Z}  g_n \opDil \opTrn^n \fphi$.
\defbox{\begin{array}{M}
  A \hid{wavelet system} is the tuple
    \\\qquad$\ds\wavsys$\\
  and the sequence $\seqxZ{g_n}$ is the \hid{wavelet coefficient sequence}.
\end{array}}
\end{definition}

%--------------------------------------
\begin{remark}
%--------------------------------------
The pair of coefficient sequences $\opair{\seqn{h_n}}{\seqn{g_n}}$ generates 
the scaling function $\fphi(x)$ \xrefP{def:wavstrct_phi} 
and the wavelet function $\fpsi(x)$ \xrefP{def:wavstrct_psi}.
These functions in turn generate 
the multiresolution analysis $\seqn{\spV_j}$ \xrefP{def:seqVn}
and the wavelet analysis $\seqn{\spW_j}$ \xrefP{def:seqWn}.
Therefore, the coefficient sequence pair $\opair{\seqn{h_n}}{\seqn{g_n}}$ 
totally defines a wavelet system $\wavsys$ \xref{def:wavsys}.

Furthermore, especially in the case of orthonormal wavelets, the wavelet coefficient
sequence $\seqxZ{g_n}$ is often defined in terms of the 
scaling coefficient sequence $\seqxZ{h_n}$
in a very simple and straightforward manner.
Therefore, in the case of an orthonormal wavelet system, the coefficient
scaling sequence $\seqxZ{h_n}$ often totally defines the entire wavelet system.
And in this case, designing a wavelet system is only a matter of finding a handful of
scaling coefficients $\seqn{h_1,\,h_2,\,\ldots,\,h_n}$\ldots because once you have these,
you can generate everything else.
\end{remark}


%%---------------------------------------
%\begin{definition}
%\label{def:wav_lat_coef}
%%---------------------------------------
%Let $\seqxZ{h_n}$ be a sequence of scaling coefficients and
%    $\seqxZ{g_n}$ be the associated sequence of wavelet coefficients.
%%    $\subseteq$ the set inclusion relation,
%%    $\setu$ the set union operation, and
%%    $\seti$ the set intersection operation.
%\defbox{\begin{array}{l}
%  \text{The tupple }
%  \qquad \wavlatcoef \\
%  \text{is called the \hid{lattice of wavelet bases coefficients}.}
%\end{array}}
%\end{definition}



%\begin{figure}[t]
%\setlength{\unitlength}{8mm}
%\begin{center}
%\begin{tabular}{cc}
%   \includegraphics*[width=6\tw/16, height=6\tw/16, clip=true]{../common/wavelets/haar_sj.eps} &
%   \includegraphics*[width=6\tw/16, height=6\tw/16, clip=true]{../common/wavelets/haar_sk.eps} \\
%   \footnotesize varying dilation, constant translation   &
%   \footnotesize varying translation, constant dilation
%\end{tabular}
%\caption{
%   \label{fig:haar-jn}
%   Haar scaling function at varying dilations and translations.
%   }
%\end{center}
%\end{figure}



%=======================================
\subsubsection{Order structure}
%=======================================
%The axioms of \prefp{def:mra} generate a subspace architecture.
%These transform representation sequences may be \hie{ordered} with \hie{ordering relations}
%as follows:%
%\footnote{\hie{projection operator ordering}: \prefp{def:operator_lattice}}

\begin{minipage}{\tw-70mm}\raggedright
The \structe{wavelet system} $\wavsys$ \xref{def:wavsys} together with the 
set inclusion relation $\subseteq$ 
forms an \structe{ordered set}\ifsxref{order}{def:poset}, 
illustrated to the right by a \hie{Hasse diagram}\ifsxref{order}{def:hasse}.%\ifdochas{order}{\footnotemark}
%Under these three order relations, wavelet system generate three isomorphic lattices such 
%as are illustrated in \prefp{fig:wav_VPb_isomorphic} 
%and in the figure to the right.
\end{minipage}%
\hfill%
\begin{minipage}{60mm}%
  %\mbox{}\\% force (just above?) top of graphic to be the top of the minipage
  %============================================================================
% Daniel J. Greenhoe
% LaTeX file
% wavelet subspace lattice
%============================================================================
\begin{pspicture}(-2.7,-0.4)(2.7,3.6)%
  \fns%
  \psset{
    boxsize=0.40\psunit,
    linearc=0.40\psunit,
    %unit=0.1mm,
    %fillstyle=none,
    % cornersize=relative,
    %framearc=0.5,
    %gridcolor=graph,
    %linewidth=1pt,
    %radius=1.25mm,
    %dotsep=1pt,
    %labelsep=1pt,
    %linecolor=latline,
    }%
  %---------------------------------
  % nodes
  %---------------------------------
  \Cnode( 0,0){Z}%    0
  \Cnode( 0,3){X}%    1
  \Cnode(-1,1){W0}% 
  \Cnode(-2,1){V0}%
  \Cnode( 0,1){W1}%
  \Cnode( 2,1){Wn1}%
  \Cnode(-1.50,1.5){V1}% V1
  \Cnode(-1,2){V2}% V2
  %---------------------------------
  % node labels
  %---------------------------------
  \uput{1.5mm}[ 90](X)  {$\spLLR$}%
  \uput{1.5mm}[135](V2) {$\spV_{2}$}%
  \uput{1.5mm}[135](V1) {$\spV_{1}$}%
  \uput{1.5mm}[180](V0) {$\spV_{0}$}%
  \uput{1.5mm}[180](W0) {$\spW_{0}$}%
  \uput{1.5mm}[  0](W1) {$\spW_{1}$}%
  %\uput{1.5mm}[  0](Wn1){$\spW_{n-1}$}%
  \uput{1.5mm}[-90](Z)  {$\spZero$}%
  %---------------------------------
  % node connections
  %---------------------------------
  \ncline{Z}  {V0}%  0    --> V0
  \ncline{Z}  {W0}%  0    --> W0
  \ncline{Z}  {W1}%  0    --> W1
  \ncline{Z}  {Wn1}% 0    --> W_{n-1}
  \ncline{Wn1}{X}%   Wn-1 --> 1
  \ncline{V0} {V1}%  V0   --> V1
  \ncline{W0} {V1}%  W0   --> V1
  \ncline{V1} {V2}%  V1   --> V2
  \rput{45}(-0.5,2.50){{\color{blue}\Large$\cdots$}}%
  \rput[c]{ 0}(1,1){{\color{blue}\Large$\cdots$}}%
  \ncline{W1}{V2}%   W1 --> V2
  %---------------------------------
  % discriptions
  %---------------------------------
  \ncbox[nodesep=0.25\psunit,linestyle=dotted,linecolor=red]{W0}{Wn1}%
  \ncbox[nodesep=0.25\psunit,linestyle=dotted,linecolor=red]{V0}{X}%
  %\ncbox[nodesep=7pt,linestyle=dotted,linecolor=red]{W0}{Wn1}%
  %\ncbox[nodesep=7pt,linestyle=dotted,linecolor=red]{V0}{X}%
  %\rnode{wavsubbox}  {\ncbox[nodesep=50\psunit,linestyle=dotted,linecolor=red]{W0}{Wn1}}%
  %\rnode{scalesubbox}{\ncbox[nodesep=50\psunit,linestyle=dotted,linecolor=red]{V0}{X}}%
  \pnode[0,-0.40](Wn1){wavsubbox}%  
  \pnode[0,0.60](V2){scalesubbox}%
  %\rput[ 0](  5,10){\psellipse[fillstyle=none,linestyle=dashed,linecolor=red](0,0)(20,5)}%
  %\rput{45}(-12,22){\psellipse[fillstyle=none,linestyle=dashed,linecolor=red](0,0)(25,6)}%
  \rput[br](2.5,1.5){\rnode{wavsublabel}{wavelet subspaces}}%
  %\psline[linecolor=red]{->}(24,24)(20,14)%
  \rput[bl]{45}(-2.4,1.6){\rnode{scalesublabel}{scaling subspaces}}%
 %\psline[linecolor=red]{->}(-15,36)(-15,26)%
  %\ncline[linecolor=red]{->}{wavsublabel}{wavsubbox}%
  %\ncline[linecolor=red]{->}{scalesublabel}{scalesubbox}%
  %---------------------------------
  % debug support
  %---------------------------------
%  \psgrid[unit=\psunit](-30,-10)(30,40)%
  %\psgrid[unit=10\psunit](-3,-1)(3,4)%
\end{pspicture}%%
\end{minipage}%


%---------------------------------------
\begin{proposition}
\label{prop:order_wavstrct}
%---------------------------------------
Let $\wavsys$ be a wavelet system with order relation $\subseteq$.
The lattice $\latL\eqd\lattice{\seqn{\spV_j},\seqn{\spW_j}}{\subseteq}{\join}{\meet}$ has 
the following properties:
\propbox{\begin{array}{FM}
    \cnto & $\latL$ is \prope{nondistributive}.
    \cntn & $\latL$ is \prope{nonmodular}.
    %\cntn & $\latL$ is \prope{complemented}.
    %\cntn & $\latL$ is \prope{not uniquely complemented}.
    %\cntn & $\latL$ is \prope{nonorthocomplemented}.
    \cntn & $\latL$ is \prope{noncomplemented}.
    \cntn & $\latL$ is \prope{nonBoolean}.
\end{array}}
\end{proposition}
\begin{proof}
\mbox{}\hspace{20mm}
  \latmatlw{4}{0.5}
    {
           &       & \null                 \\  
           & \null                         \\  
     \null &       & \null &       & \null \\  
           &       & \null                   
    }
    {\ncline{1,3}{2,2}\ncline{2,2}{3,1}
     \ncline{1,3}{3,5}
     \ncline{2,2}{3,3}
     \ncline{4,3}{3,1}\ncline{4,3}{3,3}\ncline{4,3}{3,5}
    }
    {\nput{ 90}{1,3}{$1$}
     \nput{135}{2,2}{$v$}
     \nput{0}{3,1}{$x$}
     \nput{ 67}{3,3}{$y$}
     \nput{  0}{3,5}{$z$}
     \nput{-90}{4,3}{$0$}
    }

\begin{enumerate}
  \item Proof that $\latL$ is \prope{nondistributive}: \label{item:wavprop_nondistrib}
    \begin{enumerate}
      \item $\latL$ contains the $N5$ lattice\ifsxref{latm}{def:lat_N5}.
      \item Because $\latL$ contains the $N5$ lattice, $\latL$ is \prope{nondistributive}\ifdochas{latm}{ by \prefp{thm:latd_char_n5m3}}.
    \end{enumerate}

  \item Proof that $\latL$ is \prope{nonmodular} and \prope{nondistributive}: 
    \begin{enumerate}
      \item $\latL$ contains the $N5$ lattice\ifsxref{latm}{def:lat_N5}.
      \item Because $\latL$ contains the $N5$ lattice, $\latL$ is \prope{nonmodular}\ifdochas{latm}{ by \prefp{thm:lat_mod_iff_N5}}.
    \end{enumerate}

  \item Proof that $\latL$ is \prope{noncomplemented}:
    \begin{align*}
        x' &= y' = v' = z
      \\z' &= \setn{x,y,v}
      \\x''&= \brp{x'}'
         \\&= z'
         \\&= \setn{x,y,v}
         \\&\ne  x
    \end{align*}

  %\item Proof that $\latL$ is \prope{not uniquely complemented}:\\
  %   For example, subspace $\spW_2$ in \prefp{fig:wav_VPb_isomorphic} is complemented
  %   by $\spV_1$, $\spV_2$, and $\spW_1$.
  %\item Proof that $\latL$ is \prope{orthomodular}:
  %  \begin{enumerate}
  %    \item $\latL$ does \emph{not} contain the $O_6$ lattice\ifdochas{ortholat}{ \xrefP{def:latoc_omod}}.
  %    \item Because $\latL$ does not contain the $O_6$ lattice, $\latL$ is \prope{orthomodular}\ifdochas{ortholat}{ by \prefp{thm:latoc_omod}}.
  %  \end{enumerate}

  \item Proof that $\latL$ is \prope{nonBoolean}:
    \begin{enumerate}
      \item $\latL$ is \prope{nondistributive} (\pref{item:wavprop_nondistrib}).
      \item Because $\latL$ is \prope{nondistributive}, it is \prope{nonBoolean}\ifdochas{boolean}{ by \prefp{def:booalg}}.
    \end{enumerate}
\end{enumerate}
\end{proof}




%=======================================
\subsubsection{Subspace algebraic structure}
%=======================================
%--------------------------------------
\begin{theorem}
\label{thm:mra_subalg}
%--------------------------------------
Let $\wavsys$ be a \structe{wavelet system} \xref{def:wavsys}.
Let $\spV_1 \adds \spV_2$ represent \fncte{Minkowski addition} of two subspaces $\spV_1$ and $\spV_2$ of a Hilbert space $\spH$.
\thmbox{\begin{array}{rc>{\ds}l D}
    \spLLR &=& \lim_{j\to\infty}\spV_j                 
             & ($\spLLR$ is equivalent to one very large scaling subspace)\\
           &=& \spV_j \adds \spW_j \adds \spW_{j+1} \adds \spW_{j+2} \adds\, \cdots 
             & $\brp{\begin{array}{D}$\spLLR$ is equivalent to one scaling space\\
                                       and a sequence of wavelet subspaces\end{array}} $\\
           &=& \cdots\,\adds \spW_{-2} \adds \spW_{-1} \adds \spW_0 \adds \spW_1 \adds \spW_2 \adds\,\cdots        
             & ($\spLLR$ is equivalent to a sequence of wavelet subspaces)
  \end{array}}
\end{theorem}
\begin{proof}
\begin{enumerate}
  \item Proof for (1):
    \begin{align*}
      \spLLR 
        &= \lim_{j\to\infty}\spV_j                 
        && \text{by \prefp{def:mra}}
    \end{align*}

  \item Proof for (2):
    \begin{align*}
      \mcom{\spV_j \adds \spW_j}{$\spV_{j+1}$} \adds \spW_{j+1} \adds \spW_{j+2} \adds \cdots
        &= \mcom{\spV_{j+1} \adds \spW_{j+1}}{$\spV_{j+2}$} \adds \spW_{j+2} \adds \spW_{j+3} \adds \cdots
      \\&= \mcom{\spV_{j+2} \adds \spW_{j+2}}{$\spV_{j+3}$} \adds \spW_{j+3} \adds \spW_{j+4} \adds \cdots
      \\&= \mcom{\spV_{j+3} \adds \spW_{j+3}}{$\spV_{j+4}$} \adds \spW_{j+4} \adds \spW_{j+5} \adds \cdots
      \\&= \mcom{\spV_{j+5} \adds \spW_{j+5}}{$\spV_{j+5}$} \adds \spW_{j+6} \adds \spW_{j+6} \adds \cdots
      \\&= \lim_{j\to\infty}\spV_{j+5} \adds \spW_{j+5} \adds \spW_{j+6} \adds \spW_{j+6} \adds \cdots
      \\&= \spLLR
    \end{align*}

  \item Proof for (3):
    \begin{align*}
      \spLLR &= \mcom{\spV_0}{$\spV_{-1}\adds\spW_{-1}$} \adds \spW_0 \adds \spW_1 \adds \spW_2 \adds \spW_3 \adds \cdots
             && \text{by (2)}
           \\&= \mcom{\spV_{-1}}{$\spV_{-2}\adds\spW_{-2}$} \spW_{-1} \adds \spW_0 \adds \spW_1 \adds \spW_2 \adds \spW_3 \adds \cdots
           \\&= \mcom{\spV_{-2}}{$\spV_{-3}\adds\spW_{-3}$} \spW_{-2} \adds \spW_{-1} \adds \spW_0 \adds \spW_1 \adds \spW_2 \adds \spW_3 \adds \cdots
           \\&= \mcom{\spV_{-3}}{$\spV_{-4}\adds\spW_{-4}$} \spW_{-3} \adds \spW_{-2} \adds \spW_{-1} \adds \spW_0 \adds \spW_1 \adds \spW_2 \adds \spW_3 \adds \cdots
           \\&\vdots
           \\&= \cdots \adds \spW_{-3} \adds \spW_{-2} \adds \spW_{-1} \adds \spW_0 \adds \spW_1 \adds \spW_2 \adds \spW_3 \adds \cdots
    \end{align*}
\end{enumerate}
\end{proof}

%--------------------------------------
\begin{remark}
%--------------------------------------
In the special case that two subspaces $\spW_1$ and $\spW_2$ are \prope{orthogonal} to each other, then 
the \fncte{subspace addition} operation $\spW_1\adds\spW_2$ is frequently expressed as
$\spW_1\oplus\spW_2$.
In the case of an \structe{orthonormal wavelet system} \xref{def:ows}, 
the expressions in \prefpp{thm:mra_subalg} could be expressed as
\\\indentx$\begin{array}{rc>{\ds}l}
    \spLLR &=& \lim_{j\to\infty}\spV_j                 \\
           &=& \spV_j \oplus \spW_j \oplus \spW_{j+1} \oplus \spW_{j+2} \oplus\, \cdots \\
           &=& \cdots\,\oplus \spW_{-2} \oplus \spW_{-1} \oplus \spW_0 \oplus \spW_1 \oplus \spW_2 \oplus\,\cdots .       
  \end{array}$
\end{remark}.


%=======================================
\subsubsection{Necessary conditions}
%=======================================

%--------------------------------------
\begin{theorem}[\thmd{quadrature condition}s in ``time"]
\label{thm:wavsys_quadcon}
%--------------------------------------
Let $\wavsys$ be a wavelet system \xref{def:wavsys}.
\thmbox{\begin{array}{F>{\ds}rc>{\ds}lC}
  1. & \sum_{m\in\Z} h_m \sum_{k\in\Z} h_k^\ast \inprod{\fphi}{\opTrn^{2n-m+k} \fphi} &=& \inprod{\fphi}{\opTrn^n \fphi} & \forall n\in\Z\\
  2. & \sum_{m\in\Z} g_m \sum_{k\in\Z} g_k^\ast \inprod{\fphi}{\opTrn^{2n-m+k} \fphi} &=& \inprod{\fpsi}{\opTrn^n \fpsi} & \forall n\in\Z\\
  3. & \sum_{m\in\Z} h_m \sum_{k\in\Z} g_k^\ast \inprod{\fphi}{\opTrn^{2n-m+k} \fphi} &=& \inprod{\fphi}{\opTrn^n \fpsi} & \forall n\in\Z
\end{array}}
\end{theorem}
\begin{proof}
\begin{enumerate}
  \item Proof for (1): by \prefp{thm:wav_quadcon}.
  \item Proof for (2): 
    \begin{align*}
      &\inprod{\fpsi}{\opTrn^n \fpsi}
      \\&= \inprod{\sum_{m\in\Z} g_m \opDil \opTrn^m \fphi }{\opTrn^n \sum_{k\in\Z} g_k \opDil \opTrn^k \fphi}
        && \text{by \prefp{thm:g->psi}}
      \\&= \sum_{m\in\Z} g_m \sum_{k\in\Z} g_k^\ast \inprod{\opDil \opTrn^m \fphi }{\opTrn^n \opDil \opTrn^k \fphi}
        && \text{by properties of $\inprodn$ \ifxref{vsinprod}{def:inprod}}
      \\&= \sum_{m\in\Z} g_m \sum_{k\in\Z} g_k^\ast \inprod{\fphi }{\left(\opDil \opTrn^m \right)^\ast \opTrn^n \opDil \opTrn^k \fphi}
        && \text{by definition of operator adjoint \ifxref{operator}{prop:op_adjoint}}
      \\&= \sum_{m\in\Z} g_m \sum_{k\in\Z} g_k^\ast \inprod{\fphi }{\left(\opDil \opTrn^m \right)^\ast \opDil \opTrn^{2n} \opTrn^k \fphi}
        && \text{by \prefp{prop:DTTD}}
      \\&= \sum_{m\in\Z} g_m \sum_{k\in\Z} g_k^\ast \inprod{\fphi }{\opTrna^m \opDila \opDil \opTrn^{2n} \opTrn^k \fphi}
        && \text{by operator star-algebra properties \ifxref{operator}{thm:op_star}}
      \\&= \sum_{m\in\Z} g_m \sum_{k\in\Z} g_k^\ast \inprod{\fphi }{\opTrn^{-m} \opDil^{-1} \opDil \opTrn^{2n} \opTrn^k \fphi}
        && \text{by \prefp{prop:TD_unitary}}
      \\&= \sum_{m\in\Z} g_m \sum_{k\in\Z} g_k^\ast \inprod{\fphi }{\opTrn^{2n-m+k} \fphi}
    \end{align*}

  \item Proof for (3): 
    \begin{align*}
      &\inprod{\fphi}{\opTrn^n \fpsi}
      \\&= \inprod{\sum_{m\in\Z} h_m \opDil \opTrn^m \fphi }{\opTrn^n \sum_{k\in\Z} g_k \opDil \opTrn^k \fphi}
        && \text{by \prefp{thm:dilation_eq} and \prefp{thm:g->psi}}
      \\&= \sum_{m\in\Z} h_m \sum_{k\in\Z} g_k^\ast \inprod{\opDil \opTrn^m \fphi }{\opTrn^n \opDil \opTrn^k \fphi}
        && \text{by properties of $\inprodn$ \ifxref{vsinprod}{def:inprod}}
      \\&= \sum_{m\in\Z} h_m \sum_{k\in\Z} g_k^\ast \inprod{\fphi }{\left(\opDil \opTrn^m \right)^\ast \opTrn^n \opDil \opTrn^k \fphi}
        && \text{by definition of operator adjoint \ifxref{operator}{prop:op_adjoint}}
      \\&= \sum_{m\in\Z} h_m \sum_{k\in\Z} g_k^\ast \inprod{\fphi }{\left(\opDil \opTrn^m \right)^\ast \opDil \opTrn^{2n} \opTrn^k \fphi}
        && \text{by \prefp{prop:DTTD}}
      \\&= \sum_{m\in\Z} h_m \sum_{k\in\Z} g_k^\ast \inprod{\fphi }{\opTrna^m \opDila \opDil \opTrn^{2n} \opTrn^k \fphi}
        && \text{by operator star-algebra properties \ifxref{operator}{thm:op_star}}
      \\&= \sum_{m\in\Z} h_m \sum_{k\in\Z} g_k^\ast \inprod{\fphi }{\opTrn^{-m} \opDil^{-1} \opDil \opTrn^{2n} \opTrn^k \fphi}
        && \text{by \prefp{prop:TD_unitary}}
      \\&= \sum_{m\in\Z} h_m \sum_{k\in\Z} g_k^\ast \inprod{\fphi }{\opTrn^{2n-m+k} \fphi}
    \end{align*}
\end{enumerate}
\end{proof}



%=======================================
%\subsubsection{Fourier properties}
%=======================================

%--------------------------------------
\begin{proposition}
\label{prop:vsmra_real_Fpsi}
\label{prop:psi_g_phi}
%--------------------------------------
Let $\wavsys$ be a wavelet system.
Let $\Fphi(\omega)$ and $\Fpsi(\omega)$ be the \fncte{Fourier transform}s\ifsxrefs{harFour}{def:ft}of $\fphi(x)$ and $\fpsi(x)$, respectively.
Let $\Dg(\omega)$ be the \fncte{Discrete time Fourier transform}\ifsxrefs{dsp}{def:dtft}of $\seqn{g_n}$.
%  $\begin{array}[t]{rc>{\ds}l c>{\ds}l D}
%    \Fpsi\brp{\omega}
%      &\eqd& \opFT\fpsi
%      &\eqd& \frac{1}{\sqrt{2\pi}}\int_t \fpsi(x) e^{-i\omega t} \dx
%      &      (\structe{Fourier transform}, \prefp{def:ft})
%      \\
%    \Dg(\omega)
%      &\eqd& \opDTFT\seqn{g_n}
%      &\eqd& \sum_{n\in\Z} g_n e^{-i\omega n}
%      &      (\structe{Discrete-time Fourier Transform}).
%  \end{array}$
\propbox{
  \Fpsi\brp{\omega}
    %\eqd
    %\mcom{\opFT\fpsi = \brp{\opDili \opDTFT\seqn{g_n}} \; \brp{\opDili \opFT\fphi}}
    %     {operator notation}
    =
    {\cwt \: \Dg\brp{\frac{\omega}{2}}\: \Fphi\brp{\frac{\omega}{2}}}
    %     {traditional notation}
  }
\end{proposition}
\begin{proof}
\begin{align*}
  \Fpsi\brp{\omega}
    &\eqd \opFT\fpsi
  \\&= \opFT\sum_{n\in\Z} g_n \opDil \opTrn^n \fphi
    && \text{by \prefp{thm:g->psi}}
  \\&= \sum_{n\in\Z} g_n \opFT\opDil \opTrn^n \fphi
  \\&= \sum_{n\in\Z} g_n \opDili \opFT\opTrn^n \fphi
    && \text{by \prefp{cor:wavstrct_FTD}}
  \\&= \sum_{n\in\Z} g_n \opDili e^{-i\omega n} \opFT\fphi
    && \text{by \prefp{cor:wavstrct_FTD}}
  \\&= \sum_{n\in\Z} g_n \sqrt{2}\brp{\opDili e^{-i\omega n}} \brp{\opDili\opFT\fphi}
    && \text{by \prefp{prop:DjTnfg}}
  \\&= \sqrt{2}\brp{\opDili \sum_{n\in\Z} g_n e^{-i\omega n}} \; \brp{\opDili \opFT\fphi}
  \\&= \sqrt{2}\brp{\opDili \opDTFT\seqn{g_n}} \; \brp{\opDili \opFT\fphi}
    && \text{by definition of $\opDTFT$}
  \\&= \sqrt{2}\cwt \: \Dg\brp{\frac{\omega}{2}}\: \cwt \Fphi\brp{\frac{\omega}{2}}
    && \text{by \prefp{prop:opDi}}
  \\&= \cwt \: \Dg\brp{\frac{\omega}{2}}\: \Fphi\brp{\frac{\omega}{2}}
\end{align*}
%
%\begin{align*}
%  \Fpsi\brp{\omega}
%    &= \opF\fpsi
%  \\&= \opFT\sum_{n\in\Z} g_n \opDil \opTrn^n \fphi
%    && %\text{by dilation equation \xrefP{thm:dilation_eq}}
%  \\&= \sum_{n\in\Z} g_n \opFT\opDil \opTrn^n \fphi
%  \\&= \sum_{n\in\Z} g_n \opDili \opFT\opTrn^n \fphi
%    && \text{by \prefp{prop:vsmra_real_FD}}
%  \\&= \sum_{n\in\Z} g_n \opDili e^{-i\omega n} \opFT\fphi
%    && \text{by \prefp{prop:vsmra_real_FT}}
%  \\&= \brp{\opDili \sum_{n\in\Z} g_n e^{-i\omega n}} \; \brp{\opDili \opFT\fphi}
%  \\&= \brp{\opDili \opDTFT\seqn{g_n}} \; \brp{\opDili \opFT\fphi}
%  \\&= \fscale \: \Dg\brp{\frac{\omega}{2}}\: \fscale \Fphi\brp{\frac{\omega}{2}}
%    && \text{by \prefp{prop:opDi}}
%  \\&= \frac{1}{2}\: \Dh\brp{\frac{\omega}{2}}\: \Fphi\brp{\frac{\omega}{2}}
%\end{align*}
\end{proof}

%=======================================
%\subsection{Immediate results}
%=======================================



%=======================================
%\subsubsection{Power Spectrum}
%=======================================
%%--------------------------------------
%\begin{definition}
%\citep{chui}{134}
%\label{def:wav_S}
%\index{scaling power spectrum function         }
%\index{wavelet power spectrum function         }
%\index{scaling wavelet power spectrum function }
%\index{Laurent polynomial}
%%--------------------------------------
%Let $\wavsys$ be a \hi{wavelet system}.
%Let $\Szfg(z)$ be the \fncte{complex cross-power spectrum} of $\ff$ and $\fg$ \xref{def:Szfg} in $\spLLR$
%and $\Swfg(\omega)$ be the \fncte{cross-power spectrum} of $\ff$ and $\fg$ \xref{def:Swfg} in $\spLLR$.
%\defbox{\begin{array}{>{\ds}rc>{\ds}lM}
%    \Szphi(z)  &\eqd&  \left.\Szfg(z)\right|_{\ff=\fg=\fphi} &is the \hid{scaling power spectrum function}.
%    \Szpsi(z)  &\eqd&  \left.\Szfg(z)\right|_{\ff=\fg=\fpsi} &is the \hid{wavelet power spectrum function}.
%    \Szpsi(z)  &\eqd&  \Szfg(z) &is the \hid{scaling power spectrum function}.
%    \\
%    \Spsi(\omega) &\eqd&  \sum_{n\in\Z} \Rpsi(n) \fkernea{n}{\omega}
%    &is the \hid{wavelet power spectrum function}.
%    \\
%    \Shs (\omega) &\eqd&  \sum_{n\in\Z} \Rhs (n) \fkernea{n}{\omega}
%    &is the \hid{scaling wavelet power spectrum function}.
%\end{array}}
%%\\The Laurent polynomial $\Sphi(\omega)$ is also called the \hid{Euler-Frobenius polynomial}.
%\end{definition}


%In this chapter, we don't assume the special case of orthonormality.
%But good things happen if we do happen to have orthonormality.
%One of them is that the power spectrum equations in \prefp{lem:SSS}
%simplify to constants \xrefP{lem:SSSo}.

\pref{thm:gen_quadcon} (next) presents the \structe{quadrature} necessary conditions of a \hi{wavelet system}.
These relations simplify dramatically in the special case of an
\structe{orthonormal wavelet system} \xrefP{thm:oquadcon}.
%--------------------------------------
\begin{theorem}[\thmd{Quadrature conditions} in ``frequency"]
\footnote{
  \citerp{chui}{135}       \\
  \citerp{goswami}{110}
  }
\label{thm:gen_quadcon}
%--------------------------------------
Let $\wavsys$ be a \hi{wavelet system}.
Let $\Fx(\omega)$ be the \fncte{discrete time Fourier transform}\ifsxrefs{dsp}{def:dtft}for a sequence $\seqxZ{x_n}$ in $\spllR$.
Let $\Swphi(\omega)$ be the \fncte{auto-power spectrum} \xref{def:Swfg} of $\fphi$,
    $\Swpsi(\omega)$ be the \fncte{auto-power spectrum} of $\fpsi$,
and $\Shs(\omega)$ be the \fncte{cross-power spectrum} of $\fphi$ and $\fpsi$.
\thmbox{\begin{array}{F>{\ds}lc>{\ds}l}
   1. & \abs{\Dh\left(\omega     \right)}^2 \Sphi(\omega) + \abs{\Dh\brp{\omega+\pi }}^2 \Sphi(\omega+\pi) &=& 2\Sphi(2\omega)
\\ 2. & \abs{\Dg\left(\omega     \right)}^2 \Sphi(\omega) + \abs{\Dg\brp{\omega+\pi }}^2 \Sphi(\omega+\pi) &=& 2\Spsi(2\omega)
\\ 3. & \Dh(\omega)\Dg^\ast(\omega)         \Sphi(\omega) + \Dh(\omega +\pi)\Dg^\ast(\omega +\pi)\Sphi(\omega+\pi) &=& 2\Shs(2\omega)
\end{array}}
\end{theorem}
\begin{proof}
\begin{enumerate}
  %\item First note that $\Dh(\omega)$ and $\Dg(\omega)$ are periodic with period $2\pi$ such that\ifsxrefs{dsp}{prop:dtft_2pi}
  %  \\$\begin{array}{rclC}
  %       \Dh(\omega+2\pi n) &=& \Dh(\omega) & \forall n\in\Z   \\
  %       \Dg(\omega+2\pi n) &=& \Dg(\omega) & \forall n\in\Z   
  %     \end{array}$

\item Proof for (1): by \prefp{thm:Sphi}.

\item Proof for (2):
\begin{align*}
   2\Spsi(2\omega)
     &\eqd 2\brp{2\pi}\sum_{n\in\Z} \left|\Fpsi(2\omega+2\pi n)\right|^2
   \\&= 2\brp{2\pi}\sum_{n\in\Z} \left|\cwt \Dg\left(\frac{2\omega+2\pi n}{2}\right)\Fphi\left(\frac{2\omega+2\pi n}{2}\right)\right|^2
        \indentx\text{by \prefp{lem:Fphi}}
   \\&= 2\pi
        \sum_{n\in\Ze} \left|\Dg\left(\frac{2\omega+2\pi n}{2}\right)\right|^2\left|\Fphi\left(\frac{2\omega+2\pi n}{2}\right)\right|^2 +
     \\&\qquad 2\pi
        \sum_{n\in\Zo} \left|\Dg\left(\frac{2\omega+2\pi n}{2}\right)\right|^2\left|\Fphi\left(\frac{2\omega+2\pi n}{2}\right)\right|^2
   \\&= 2\pi\sum_{n\in\Z} \abs{\Dg\brp{\omega+2\pi n     }}^2 \abs{\Fphi\brp{\omega+2\pi n       }}^2 +
        2\pi\sum_{n\in\Z} \abs{\Dg\brp{\omega+2\pi n+\pi }}^2 \abs{\Fphi\brp{\omega+2\pi n + \pi }}^2  
   \\&= 2\pi\sum_{n\in\Z} \abs{\Dg\brp{\omega            }}^2 \abs{\Fphi\brp{\omega+2\pi n       }}^2 +
        2\pi\sum_{n\in\Z} \abs{\Dg\brp{\omega+\pi        }}^2 \abs{\Fphi\brp{\omega+2\pi n + \pi }}^2  
   \\&= \abs{\Dg\brp{\omega     }}^2 \brp{2\pi\sum_{n\in\Z} \abs{\Fphi\brp{\omega+2\pi n       }}^2 +}
        \abs{\Dg\brp{\omega+\pi }}^2 \brp{2\pi\sum_{n\in\Z} \abs{\Fphi\brp{\omega+\pi+2\pi n   }}^2  }
   \\&= \abs{\Dg\brp{\omega     }}^2\Sphi(\omega) +
        \abs{\Dg\brp{\omega+\pi }}^2 \Sphi(\omega+\pi)
        \indentx\text{by \prefp{thm:Swfg}}
\end{align*}


\item Proof for (3):
\begin{align*}
  2\Shs(2\omega)
    &=  2\brp{2\pi}\sum_{n\in\Z} \Fphi(2\omega+2\pi n) \Fpsi^\ast(2\omega+2\pi n)
  \\&=  2\brp{2\pi}\sum_{n\in\Z}
        \cwt 
        \Dh  \left(\omega +\pi n \right)
        \Fphi\left(\omega +\pi n \right)
        \cwt 
        \Dg^\ast  \left(\omega +\pi n \right)
        \Fphi^\ast\left(\omega +\pi n \right)
        \quad\text{by \prefp{lem:Fphi}}
  \\&=  2\pi
        \sum_{n\in\Z}
        \Dh  \left(\omega +\pi n \right)
        \Dg^\ast  \left(\omega +\pi n \right)
        \left| \Fphi\left(\omega +\pi n \right) \right|^2
  \\&=  2\pi
        \sum_{n\in\Zo}
        \Dh  \left(\omega +\pi n \right)
        \Dg^\ast  \left(\omega +\pi n \right)
        \left| \Fphi\left(\omega +\pi n \right) \right|^2
      \\&\qquad+ 2\pi\sum_{n\in\Ze}
        \Dh  \left(\omega +\pi n \right)
        \Dg^\ast  \left(\omega +\pi n \right)
        \left| \Fphi\left(\omega +\pi n \right) \right|^2
  \\&=  2\pi\sum_{n\in\Z}
        \Dh  \left(\omega +2\pi n+\pi \right)
        \Dg^\ast  \left(\omega +2\pi n+\pi \right)
        \left| \Fphi\left(\omega +2\pi n+\pi \right) \right|^2
      \\&\qquad+ 2\pi\sum_{n\in\Z}
        \Dh  \left(\omega +2\pi n\right)
        \Dg^\ast  \left(\omega +2\pi n\right)
        \left| \Fphi\left(\omega +2\pi n\right) \right|^2
  \\&=  2\pi
        \sum_{n\in\Z}
        \Dh  \left(\omega +\pi \right)
        \Dg^\ast  \left(\omega +\pi \right)
        \left| \Fphi\left(\omega +2\pi n+\pi \right) \right|^2
      + 2\pi\sum_{n\in\Z}
        \Dh  \left(\omega \right)
        \Dg^\ast  \left(\omega \right)
        \left| \Fphi\left(\omega +2\pi n\right) \right|^2
  \\&=  \Dh  \left(\omega \right)
        \Dg^\ast  \left(\omega \right)
        \brp{2\pi\sum_{n\in\Z} \left| \Fphi\left(\omega +2\pi n\right) \right|^2}
      \\&\qquad+ \Dh  \left(\omega +\pi \right)
        \Dg^\ast  \left(\omega +\pi \right)
        \brp{2\pi\sum_{n\in\Z}\left| \Fphi\left(\omega +\pi+2\pi n\right) \right|^2}
  \\&=  \Dh(\omega)
        \Dg^\ast(\omega)
        \brp{2\pi\sum_{n\in\Z} \left| \Fphi(\omega +2\pi n) \right|^2}
      + \Dh  (\omega +\pi)
        \Dg^\ast (\omega +\pi)
        \brp{2\pi\sum_{n\in\Z}\left| \Fphi(\omega +\pi+2\pi n) \right|^2}
  \\&=  \Dh(\omega     ) \Dg^\ast(\omega     )\Sphi(\omega)
      + \Dh(\omega +\pi) \Dg^\ast(\omega +\pi)\Sphi(\omega+\pi)
        \indentx\text{by \prefp{thm:Swfg}}
\end{align*}

\end{enumerate}
\end{proof}

%%=======================================
%\begin{definition}
%%=======================================
%Let $\wavsys$ be a wavelet system.
%Let $\oppS\seqn{\fphi}_m$ be the \hid{span} of the basis vectors $\seqn{\fphi}_m$.
%We define the following order relations. 
%\defbox{\begin{array}{rcll}
%%  \opP_m         &\orel&  \opP_n         & \quad\text{if}\quad \opP_m\opP_n=\opP_n\opP_m=\opP_m \\
%  \spV_m         &\orela& \spV_j         & \quad\text{if}\quad \spV_m \subseteq \spV_j \\
%  \seqn{\fphi}_m &\orelb& \seqn{\fphi}_n & \quad\text{if}\quad \oppS\seqn{\fphi}_m \subseteq \oppS\seqn{\fphi}_n %\qquad\text{where $\oppS$ is the span}
%\end{array}}
%\end{definition}
%



%The sequences of subspaces discussed in this section together with
%set relations $\subseteq$, $\setu$, and $\seti$,
%for a lattice.
%This ``\hie{lattice of wavelet subspaces}" is defined next.
%%---------------------------------------
%\begin{definition}
%\label{def:wav_lat_subspace}
%%---------------------------------------
%Let $\seqjZ{\spV_j}$ be a sequence of scaling subspaces and
%    $\seqjZ{\spW_j}$ be a sequence of wavelet subspaces.
%%    $\subseteq$ the set inclusion relation,
%%    $\odot$ the set union operation, and
%%    $\seti$ the set intersection operation.
%\defbox{\begin{array}{l}
%  \text{The tupple }
%  \qquad \wavlatsubs \\
%  \text{is called the \hid{lattice of wavelet subspaces}.}
%\end{array}}
%\end{definition}


%=======================================
\subsubsection{Sufficient condition}
%=======================================
In this text, an often used sufficient condition for designing the \structe{wavelet coefficient sequence} 
$\seqn{g_n}$ \xref{def:gn} is the \prope{conjugate quadrature filter condition} \xref{def:cqf}. 
It expresses the sequence $\seqn{g_n}$ in terms of the \structe{scaling coefficient sequence} \xref{def:hn}
and a ``shift" integer $\xN$ as $g_n = \pm(-1)^n h^\ast_{\xN-n}$.
The \structe{CQF condition} has the following ``nice" properties:
\\\indentx\begin{tabular}{>{\scs}rp{\tw-30mm}}
    1. & Given a \structe{scaling coefficient sequence} $\seqn{h_n}$ \xref{def:hn}, 
         it is extremely simple to compute the \structe{wavelet coefficient sequence} $\seqn{g_n}$ \xref{def:gn}.
  \\2. & If $\setn{\opTrn\fphi}$ of a \structe{wavelet system} $\wavsys$ \xref{def:wavsys} is \prope{orthonormal} and 
         $\otriple{\seqn{g_n}}{\seqn{h_n}}{\xN}$ satisfies the \prope{CQF condition}, 
         then $\setn{\opTrn^n\fpsi}$ is also \prope{orthnormal} \xref{thm:ortho_qmr}.
  \\3. & If $\setn{\opTrn\fphi}$ of a \structe{wavelet system} $\wavsys$ \xref{def:wavsys} is \prope{orthonormal} and 
         $\otriple{\seqn{g_n}}{\seqn{h_n}}{\xN}$ satisfies the \prope{CQF condition}, 
         then the \structe{wavelet subspace} $\spW_0$ is 
         \prope{orthnormal} to the \structe{scaling subspace} $\spV_0$ ($\spW_0\orthog\spV_0$) \xref{thm:ortho_qmr}.
\end{tabular}

%---------------------------------------
\begin{theorem}
\label{thm:cqf}
%---------------------------------------
Let $\wavsys$ be a \structe{wavelet system} \xref{def:wavsys}.
Let $\Dg(\omega)$ be the \ope{DTFT} \xref{def:dtft} and $\Zg(z)$ the \ope{Z-transform} \xref{def:opZ} of $\seqn{g_n}$.
\thmbox{
  \begin{array}{>{\ds}rc >{\ds}rcl @{\qquad}D}
  \mcom{g_n = \pm(-1)^n h^\ast_{\xN-n},\,{\scy\xN\in\Z}}{\structe{conjugate quadrature filter}}
      &\iff&     \Dg(\omega)                   &=& \pm (-1)^\xN e^{-i\omega\xN} \Dh^\ast(\omega+\pi)\Big|_{\omega=\pi}   & (1)
    \\&\implies& \sum_{n\in\Z} (-1)^n g_n      &=& \sqrt{2}                                                              & (2)
    \\&\iff&     \Zg(z)\Big|_{z=-1}            &=& \sqrt{2}                                                              & (3)
    \\&\iff&     \Dg(\omega)\Big|_{\omega=\pi} &=& \sqrt{2}                                                              & (4)
  \end{array}}
\end{theorem}
\begin{proof}
  \begin{enumerate}
    \item Proof that CQF$\iff$(1): by \prefp{thm:cqf}

    \item Proof that CQF$\implies$(4):
      \begin{align*}
        \Dg(\pi)
          &= \Dg(\omega)\Big|_{\omega=\pi}
        \\&= \pm (-1)^\xN e^{-i\omega\xN} \Dh^\ast(\omega+\pi)\Big|_{\omega=\pi}
          && \text{by \prefp{thm:cqf}}
        \\&= \pm (-1)^\xN e^{-i\pi\xN} \Dh^\ast(2\pi)
        \\&= \pm (-1)^\xN (-1)^\xN \Dh^\ast(0)
          && \text{by \prefp{prop:dtft_2pi}}
        \\&= \sqrt{2}
          && \text{by \thme{admissibility condition} \xref{thm:admiss}}
      \end{align*}

    \item Proof that (2)$\iff$(3)$\iff$(4): by \prefp{prop:dsp_zminone}
  \end{enumerate}
\end{proof}

%=======================================
\subsection{Support size}
%=======================================
The \hie{support} of a function is what it's non-zero part ``sits" on.
If the support of the scaling coefficients $\seqn{h_n}$ goes from say
$[0,3]$ in $\Z$, what is the support of the scaling function $\fphi(x)$?
The answer is $[0,3]$ in $\R$---essentially the same
as the support of $\seqn{h_n}$ except that the two functions have different
domains ($\Z$ versus $\R$).
This concept is defined in \pref{def:support} (next definition),
proven in \pref{thm:support} (next theorem),
and illustrated in and \prefpp{sec:examples_pounity} and \prefpp{sec:examples_Dp}.
%\pref{ex:sw_gh_d2} -- \pref{ex:sw_gh_bspline} (\prefpo{ex:sw_gh_d2} -- \prefpo{ex:sw_gh_bspline}).

%--------------------------------------
\begin{definition}
\index{support}
\label{def:support}
%--------------------------------------
Let $\wavsys$ be a wavelet system.
Let $\cls{\setX}$ represent the closure of a set $\setX$ in $\spLLR$, 
$\hxs{\join}\setX$ the \hie{least upper bound} of an ordered set $\opair{\setX}{\orel}$,
$\hxs{\meet}\setX$ the \hie{greatest lower bound} of an ordered set $\opair{\setX}{\orel}$,
and
\\\indentx$\begin{array}{lclCD}
  \hxs{\floor{x}} &\eqd& \joinop\set{n\in\Z}{n\le x} & \forall x\in\R & (\hie{floor} of $x$)\\
  \hxs{\ceil{x}}  &\eqd& \meetop\set{n\in\Z}{n\ge x} & \forall x\in\R & (\hie{ceiling} of $x$).
\end{array}$
%$\delta(x)
%   =\brbl{\begin{array}{ll}
%      1 & \text{for } x\ne 0 \\
%      0 & \text{otherwise}.
%     \end{array}}$
\defboxt{
  The \hid{support} $\opS\ff$ of a function $\ff\in\clFxy$ is defined as
  \\\indentx$\ds
    \opS\ff \eqd
    \brbl{\begin{array}{>{\ds}llD}
      \cls{\set{x\in\R}{\ff(x)\ne 0}}
        & \text{for } \setX=\R 
        & (domain of $\ff$ is $\R$)
      \\
      \cls{\set{x\in\R}{\ff\brp{\floor{x}}\ne0 \text{ and } \ff\brp{\ceil{x}}\ne0}}
        & \text{for } \setX=\Z
        & (domain of $\ff$ is $\Z$)
    \end{array}}
  $
  }
\end{definition}

%--------------------------------------
\begin{theorem}[\thmd{support size}]
\footnote{
  \citerppg{mallat}{243}{244}{012466606X}
  }
\label{thm:support}
\thmx{support size}
%--------------------------------------
Let $\wavsys$ be a wavelet system.
Let $\support\ff$ be the support of a function $\ff$ \xref{def:support}.
\thmbox{
  %\begin{array}{>{\ds}rc>{\ds}l}
  \support\fphi = \support\fh
  %\\
  %N\in\Zo \text{ and }  g_n  = \pm(-1)^\xN\fh(N-n) \quad\implies\quad
  %\support\fpsi &=& \left[ \frac{N-(n_2-n_1)}{2}, \frac{N+(n_2-n_1)}{2} \right]
  %\end{array}$
  }
\end{theorem}
\begin{proof}
%\begin{enumerate}
%\item Proof that $\support\fphi = \support\fh$:
\begin{enumerate}

  \item Definitions: \label{item:wavstrct_support_def}
    \\$\begin{array}{rcl}
      \support\fphi &\eqd& [a,b] \\
      \support\fh   &\eqd& [k,m].
     \end{array}$

  \item Lemma: \label{item:wavstrct_support_lemma}
    \\$\support\fphi(x)=\brs{a,b} \quad\iff\quad \support\fphi(2x)=\brs{\frac{a}{2},\frac{b}{2}}$

  \item Proof that $k=a$:
    \begin{align*}
      a
        &= \meetop\support\fphi(x)
        && \text{by definition of $a$ (\pref{item:wavstrct_support_def})}
      \\&\eqd \meetop\support \brs{ \sum_{n\in\Z} h_n \opDil\opTrn^n\fphi(x) }
        && \text{by dilation equation \xref{thm:dilation_eq}}
      \\&= \meetop\support \brs{\sqrt{2}\sum_{n\in\Z}  h_n \fphi(2x-n)}
        && \text{by definition of $\opTrn$ and $\opDil$ \xrefP{def:wavstrct_TD}}
      \\&= \meetop\support \brs{\sum_{n\in\Z}  h_n \fphi(2x-n)}
        && \text{because $\sqrt{2}$ has no effect on $\support$}
      \\&= \meetop\support \brs{h_{k} \fphi\brp{2x-k}}
        && \text{because $n={k}$ is the least value of $n$ for which $h_n\neq0$}
      \\&= \meetop\support \brs{\fphi\brp{2x-k}}
        && \text{because the non-zero constant $h_{k}$ has no effect on $\support$}
      \\&= \meetop\support \brs{\fphi\brp{2\brs{x-\frac{k}{2}}}}
      \\&= \meetop\set{t}{\fphi\brp{2\brs{x-\frac{k}{2}}}\neq0}
        && \text{by definition of $\support$ \xrefP{def:support}}
      \\&= t \st  x-\frac{k}{2} = \frac{a}{2}
        && \text{by Lemma in \pref{item:wavstrct_support_lemma}}
      \\&= \frac{k}{2} + \frac{a}{2}
      \\\implies &\qquad\frac{k}{2} = a - \frac{a}{2}
      \\\iff &\qquad k = a
    \end{align*}

  \item Proof that $m=b$:
    \begin{align*}
      b
        &= \joinop\support\fphi(x)
        && \text{by definition of $a$ (\pref{item:wavstrct_support_def})}
      \\&\eqd \joinop\support \brs{ \sum_{n\in\Z} h_n \opDil\opTrn^n\fphi(x) }
        && \text{by dilation equation \xrefP{thm:dilation_eq}}
      \\&= \joinop\support \brs{\sqrt{2}\sum_{n\in\Z}  h_n \fphi(2x-n)}
        && \text{by definition of $\opTrn$ and $\opDil$ \xrefP{def:wavstrct_TD}}
      \\&= \joinop\support \brs{\sum_{n\in\Z}  h_n \fphi(2x-n)}
        && \text{because $\sqrt{2}$ has no effect on $\support$}
      \\&= \joinop\support \brs{h_{m} \fphi\brp{2x-m}}
        && \text{because $n={m}$ is the greatest value of $n$ for which $h_n\neq0$}
      \\&= \joinop\support \brs{\fphi\brp{2x-m}}
        && \text{because the non-zero constant $h_{m}$ has no effect on $\support$}
      \\&= \joinop\support \brs{\fphi\brp{2\brs{x-\frac{m}{2}}}}
      \\&= \joinop\set{t}{\fphi\brp{2\brs{x-\frac{m}{2}}}\neq0}
        && \text{by definition of $\support$ \xrefP{def:support}}
      \\&= t \st  x-\frac{m}{2} = \frac{b}{2}
        && \text{by Lemma in \pref{item:wavstrct_support_lemma}}
      \\&= \frac{m}{2} + \frac{b}{2}
      \\\implies &\qquad\frac{m}{2} = b - \frac{b}{2}
      \\\iff &\qquad m = b
    \end{align*}

\end{enumerate}
%\item $\support \psi$:
%\begin{align*}
%  \support\fpsi(x)
%    &= \support \brs{ \sum_{n\in\Z} g_n \opDil\opTrn^n\fphi(x) }
%  \\&= \support \left[ \sqrt{2}\sum_{n\in\Z}  g_n \fphi(2x-n)\right]
%  \\&= \support \left[ \sqrt{2}\sum_{n\in\Z} \pm (-1)^\xN \fh(N-n)\fphi(2x-n)\right]
%  \\&= \support \left[ \sum_{n\in\Z} \fh(N-n)\fphi(2x-n)\right]
%  \\&= \cls{\set{x\in\R}{\sum_{n\in\Z} \fh(N-n)\fphi(2x-n)\ne 0}}
%  \\&= \left[ \frac{n_1}{2}+\frac{N-n_2}{2}, \frac{n_2}{2}+\frac{N-n_1}{2}\right]
%  \\&= \left[ \frac{N-(n_2-n_1)}{2}, \frac{N+(n_2-n_1)}{2}\right]
%\end{align*}
%\end{enumerate}
\end{proof}




%=======================================
\subsection{Examples}
%=======================================
%Under the very general constraints of this chapter, I know of no wavelet examples.
No examples of wavelets are presented in this section. 
Examples begin in the next chapter which is about a property called the \prope{partition of unity}.
%The very minimal of requirements, it seems is a \prope{partition of unity} \xref{chp:pounity}.
Other design constraints leading to wavelets with more ``powerful" properties include 
\prope{vanishing moments} \xref{chp:vanish}, \prope{orthonormality} \xref{chp:ortho},
\prope{compact support} \xref{chp:compactp}, and \prope{minimum phase} \xref{def:ztr_minphase}.

Here are some examples of \structe{wavelet systems} \xref{def:wavsys} found later in this text:
\begin{longtable}{|l||ll||c|c|c|c|c|}
  \hline
  Name   & \mc{2}{||c|}{Reference} & \rotatebox{75}{\prope{partition of unity}} 
                                   & \rotatebox{75}{\prope{vanishing moments}} 
                                   & \rotatebox{75}{\prope{orthonormality}} 
                                   & \rotatebox{75}{\prope{compact support}}
                                   & \rotatebox{75}{\prope{minimum phase}}
%         &          &       & \xref{chp:pounity}         & \xref{chp:vanish}         & \xref{chp:ortho}       & \xref{chp:compactp}
  \\\hline
  Haar             & \scs\pref{ex:pun_n=2}       & \scs\prefpo{ex:pun_n=2}       & $\checkmark$ & 1 & $\checkmark$ & $\checkmark$ & $\checkmark$ \\
  order 1 B-spline & \scs\pref{ex:sw_gh_tent}    & \scs\prefpo{ex:sw_gh_tent}    & $\checkmark$ & 2 & $          $ & $          $ & $          $ \\
  order 3 B-spline & \scs\pref{ex:sw_gh_bspline} & \scs\prefpo{ex:sw_gh_bspline} & $\checkmark$ & 4 & $          $ & $          $ & $          $ \\
  Daubechies-p2    & \scs\pref{ex:dau-p2}        & \scs\prefpo{ex:dau-p2}        & $\checkmark$ & 2 & $\checkmark$ & $\checkmark$ & $\checkmark$ \\
  Daubechies-p3    & \scs\pref{ex:dau-p3}        & \scs\prefpo{ex:dau-p3}        & $\checkmark$ & 3 & $\checkmark$ & $\checkmark$ & $\checkmark$ \\
  Symmlet-p4       & \scs\pref{ex:symmlet_p4}    & \scs\prefpo{ex:symmlet_p4}    & $\checkmark$ & 4 & $\checkmark$ & $\checkmark$ & $          $ \\
  \hline
\end{longtable}




%================================================================================
%================================================================================
%================================================================================
%================================================================================
%================================================================================

%\paragraph{Fourier Transform.}
%One of the most widely used transforms is the Fourier transform.
%The Fourier Transform is an integral operator with an exponential kernel.
%And what is so special about exponential kernels?
%Is it just that they were discovered sooner than other kernels with other transforms?
%The answer in general is ``no".
%The exponential has two properties that makes it extremely special:
%  \begin{liste}
%    \item The exponential is an eigenvalue of any LTI operator \xrefP{thm:Le=he}
%    \item The exponential generates a continuous point spectrum for the differential operator
%          \xrefP{thm:spec_D}
%  \end{liste}
%
%\thmbox{
%  \left.\begin{array}{ll}
%    1. & \text{$\opL$ is linear and} \\
%    2. & \text{$\opL$ is time-invariant}
%  \end{array}\right\}
%  \qquad\implies\qquad
%    \opL \lkerne{t}{s}
%    =
%    \mcomr{\Lh^\ast(-s)}{eigenvalue} \mcoml{\lkerne{t}{s}}{eigenvector}
%  }
%
%
%What makes wavelet system unique from other analysis systems
%(such as Fourier analysis) is its \hie{subspace architecture}.
%In this section, we present this architecture using four representations:
%
%\begin{tabular}{lp{\tw/2}ll}
%  \circOne   & lattice of subspaces             \dotfill&\pref{sec:wav_lat_subspace} & \xrefP{sec:wav_lat_subspace} \\
%  \circTwo   & lattice of projection operators  \dotfill&\pref{sec:wav_lat_op}       & \xrefP{sec:wav_lat_op} \\
%  \circThree & lattice of bases vectors         \dotfill&\pref{sec:wav_lat_bases}    & \xrefP{sec:wav_lat_bases} \\
%  \circFour  & lattice of bases coefficients    \dotfill&\pref{sec:wav_lat_coef}     & \xrefP{sec:wav_lat_coef} \\
%\end{tabular}
%
%\prefp{thm:wav_lat_iso} will show that all four of these representations are
%essentially equivalent (are isomorphisms).
%The wavelet system itself is simply a collection of the projection operators
%\xrefP{def:wav_transform} found in the wavelet operator lattice.


\fi
%  %============================================================================
% Daniel J. Greenhoe
% XeLaTeX file
%============================================================================

%======================================
\section{Frames and Bases}
%======================================







%\begin{figure}[t]
%\[\begin{array}{*{5}{>{\ds}c}}
%     && \fcolorbox{blue}{bg_blue}{\parbox[c]{3\tw/16}{\centering Hamel basis}}
%  \\ && \Uparrow 
%  \\ && \fcolorbox{blue}{bg_blue}{\parbox[c]{3\tw/16}{\centering Riesz basis}}
%  \\ && \Uparrow 
%  \\ && \fcolorbox{blue}{bg_blue}{\parbox[c]{3\tw/16}{\centering frame}}
%  \\ && \Uparrow
%  \\ && \fcolorbox{blue}{bg_blue}{\parbox[c]{3\tw/16}{\centering tight frame}}
%  \\ && \Uparrow
%  \\ && \fcolorbox{blue}{bg_blue}{\parbox[c]{3\tw/16}{\centering orthonormal basis}}
%  \\ && \Uparrow 
%  \\ && \fcolorbox{blue}{bg_blue}{\parbox[c]{3\tw/16}{\centering$0$}}
%\end{array}\]
%\end{figure}

\begin{figure}
  \psset{yunit=1.2mm,xunit=1mm}
  {\begin{pspicture}(-63,-20)(88,90)%
     \footnotesize
     \psset{%
       linecolor=blue,
       linewidth=1pt,
       cornersize=relative,
       framearc=0.25,
       gridcolor=graph,
       subgriddiv=1,
       gridlabels=4pt,
       gridwidth=0.2pt,
       }%
     \rput(0, 80){\rnode{lincombos}   {\psframebox{\parbox{60mm}{\centering
       \structe{linear combinations} \xref{def:lincombo}\\
       $\ds\vx=\sum_{\gamma\in\Lambda}\alpha_{\gamma} \vx_\gamma$
       }}}}%
     \rput(-30, 60){\rnode{hamel}   {\psframebox{\parbox{50mm}{\centering
       \structe{Hamel bases} \xref{def:hamel}\\
       $\ds\vx=\sum_{n=1}^\xN\alpha_{n} \vx_n$
       }}}}%
     \rput(30, 60){\rnode{frames}      {\psframebox{\parbox{50mm}{\centering
       \structe{frames} \xref{def:frame}\\
       $\ds{A\norm{\vx}^2 \le \sum_{n=1}^\infty \abs{\inprod{\vx}{\vx_n}}^2 \le B\norm{\vx}^2}$
       }}}}%
     \rput(50, 40){\rnode{tframes}     {\psframebox{\parbox{50mm}{\centering
       \structe{tight frames} $(A=B)$\\
       $\ds A\norm{\vx}^2\eqs\sum_{n=1}^\infty \left|\inprod{\vx}{\vx_n}\right|^2$
       \\\xref{ex:mercedes}
       }}}}%
     \rput(50, 20){\rnode{pframes}     {\psframebox{\parbox{50mm}{\centering
       \structe{Parseval frames} $(A=B=1)$\\
       $\ds\norm{\vx}^2\eqs\sum_{n=1}^\infty \left|\inprod{\vx}{\vx_n}\right|^2 $
       \\\xref{ex:mercedesA1}
       }}}}%
     %
     \rput(-30, 40){\rnode{sbases}     {\psframebox{\parbox{50mm}{\centering
       \structe{Schauder bases} \xref{def:basis_schauder}\\
       $\ds\vx\eqs\sum_{n=1}^\infty \alpha_{n} \vx_{n} $
       %$\ds\lim_{\xN\to\infty}\norm{\vx-\sum_{n=1}^\xN \alpha_{n} \vx_n}=0 $\\
       }}}}%
     %
     \rput(-20,20){\rnode{rbases}     {\psframebox{\parbox{70mm}{\centering
       \structe{Riesz bases} \xref{def:basis_riesz}\\
       %$\ds\exists\opR\st\quad\vx\eqs\sum_{n=1}^\infty \mcom{\inprod{\opR\vx_n}{\vx}}{$\alpha_n$}\opR\vx_n$%\\
       %$\ds\exists\opR\st\quad\vx\eqs\sum_{n=1}^\infty {\inprod{\opR\vx}{\vx_n}}\opR\vx_n$%\\
       $\ds \frac{1}{B}\brp{\sum_{n=1}^\infty \abs{\alpha_n}^2} \le \norm{\sum_{n=1}^\infty \alpha_n\vx_n}^2 \le \frac{1}{A}\brp{\sum_{n=1}^\infty \abs{\alpha_n}^2}$
       %\prefpp{def:basis_riesz}
       }}}}%
     %
     \rput(0,0){\rnode{obases}     {\psframebox{\parbox{60mm}{\centering
       \structe{orthonormal bases} \xref{def:basis_ortho}\\
       %$\ds\vx\eqs\sum_{n=1}^\infty \mcom{\inprod{\vx}{\vx_n}}{$\alpha_n$}\vx_n$%\\
       $\ds\vx\eqs\sum_{n=1}^\infty {\inprod{\vx}{\vx_n}}\vx_n$
       }}}}%
     %
     \rput(  0,-15){\ovalnode{zero}      {$\spZero$}}%
     %
     \ncline{lincombos}{frames}%
     \ncline{frames}{tframes}%
     \ncline{tframes}{pframes}%
     \ncline{frames}{rbases}%
     \ncline{pframes}{obases}%
     %
     \ncline{lincombos}{hamel}%
     \ncline{hamel}{sbases}%
     \ncline{sbases}{rbases}%
     \ncline{rbases}{obases}%
     \ncline{obases}{zero}%
     %
     %\psccurve[linestyle=dashed,linecolor=red,fillstyle=none]%
     %  (0,-5)(20,4)(70,60)(20,60)(15,55)(-5,38)(-20,35)(-10,20)(-25,5)%
     %\psline[linecolor=red]{->}(60,75)(60,68)%
     %\uput[135](60,75){complete spaces}%
     %\psline[linecolor=red]{->}(26,80)(15,74)%
     %\psline[linecolor=red]{->}(30,80)(30,62.5)%
     %\uput[90](28,80){analytic spaces}%
     %
     %\psgrid[unit=10\psunit](-6,-4)(6,9)%
  \end{pspicture}}
\end{figure}

%=======================================
\subsection{Linear Combinations in Linear Spaces}
%=======================================
\ifdochasnot{vector}{
A metric space is a set together with nothing else save a metric that gives the space a \hie{topology}.
A linear space (next definition) has no topology but does have some additional \hie{algebraic} structure that 
is useful in generalizing a number of mathematical concepts.
If one wishes to have both algebraic structure and a topology, then this can be accomplished by appending 
a topology (giving a \hie{topological linear space}), 
a metric (giving a \hie{metric linear space}),
an inner product (giving an \hie{inner product space} or a \hie{Hilbert space}),
or a norm (giving a \hie{normed linear space} or a \hie{Banach space}).
%---------------------------------------
\begin{definition}
\index{space!vector}
\index{space!linear}
\footnote{
  \citerppgc{kubrusly2001}{40}{41}{0817641742}{Definition 2.1 and following remarks}\\
  \citerp{haaser1991}{41} \\
  \citerpp{halmos1948}{1}{2} \\
  \citorc{peano1888}{Chapter IX}  \\
  \citorpp{peano1888e}{119}{120} \\
  \citorpp{banach1922}{134}{135}
  }
\label{def:vspace}
%---------------------------------------
%Let $\F\eqd\otriple{\setS}{+}{\cdot}$ be a field.
Let $\fieldF$ be a field.
Let $\setX$ be a set and let $+$ be an operator in $\clF{\setX^2}{\setX}$ and $\otimes$ be an operator in $\clF{\F\times\setX}{\setX}$.
\defboxp{
  The structure $\spL\eqd\linearspaceX$ is a \hid{linear space} over the field $\fieldF$ if
  %\\\indentx$\ds\begin{array}{l rcl @{\quad}C @{\quad}D@{}r@{}}
  \\\indentx$\ds\begin{array}{>{\scriptstyle}r rcl @{\quad}C @{\quad}D@{}r@{}}
    \\\cline{7-7}
    1.& \exists \vzero\in\setX \st \vx + \vzero &=& \vx
      & \forall \vx\in\setX
      & ($+$ \prope{identity})
      & \ast\vline
      \\
    2.& \exists \vy\in\setX \st \vx+\vy &=& \vzero
      & \forall \vx \in\setX
      & ($+$ \prope{inverse})
      & \vline
      \\
    3.& (\vx+\vy)+\vz &=& \vx+(\vy+\vz)
      & \forall \vx,\vy,\vz\in\setX
      & ($+$ is \prope{associative})
      & \text{ }\vline
      \\
    4.& \vx+\vy &=& \vy+\vx
      & \forall \vx,\vy\in\setX
      & ($+$ is \prope{commutative})
      & \vline
      \\\cline{7-7}
    5.& 1\cdot \vx &=& \vx
      & \forall \vx\in\setX
      & ($\cdot$ \prope{identity})
      \\
    6.& \alpha\cdot(\beta\cdot\vx) &=& (\alpha\cdot\beta)\cdot\vx
      & \forall \alpha,\beta\in\setS \text{ and } \vx\in\setX
      & ($\cdot$ \prope{associates} with $\cdot$)
      \\
    7.& \alpha\cdot(\vx+\vy) &=& (\alpha \cdot\vx)+(\alpha\cdot\vy)
      & \forall \alpha\in\setS \text{ and } \vx,\vy\in\setX
      & ($\cdot$ \prope{distributes} over $+$)
      \\
    8.& (\alpha+\beta)\cdot\vx &=& (\alpha\cdot \vx)+(\beta\cdot \vx)
      & \forall \alpha,\beta\in\setS \text{ and } \vx\in\setX
      & ($\cdot$ \prope{pseudo-distributes} over $+$)
  \end{array}$
  \\
  The set $\setX$ is called the \hid{underlying set}.
  The elements of $\setX$ are called \hid{vectors}.
  The elements of $\F$ are called \hid{scalars}.
  A linear space is also called a \hid{vector space}.
  If $\F\eqd\R$, then $\spL$ is a \hid{real linear space}.
  If $\F\eqd\C$, then $\spL$ is a \hid{complex linear space}.
  }
\end{definition}
}


A linear space \xrefP{def:vspace} in general is not equipped with a topology\ifsxref{topology}{def:topology}. 
Without a topology, it is not possible to determine whether an infinite sum of vectors 
converges\ifsxref{seq}{def:converge}.
Therefore in this section (dealing with linear spaces), 
all definitions related to sums of vectors will be valid for \emph{finite} sums \xref{def:sum} only (finite ``$\xNn$").

%--------------------------------------
\begin{definition}
\footnote{
  \citerpgc{berberian1961}{11}{0821819127}{Definition~I.4.1}\\
  \citerpg{kubrusly2001}{46}{0817641742}
  }
\label{def:lincombo}
%--------------------------------------
Let $\setxn{\vx_n\in\setX}$ be a set of vectors in a linear space $\linearspaceX$.
%Let $\spL\eqd\linearspaceX$ be a \structe{linear space} \xrefP{def:vspace}.
\defbox{\begin{array}{M} 
  A vector $\vx\in\setX$ is a \hid{linear combination} of the vectors in $\setn{\vx_n}$ if 
  \\there exists $\ds\setxn{\alpha_n\in\F}$ such that
  \\\qquad$\ds\vx=\sum_{n=1}^\xN \alpha_n \vx_n$ .
\end{array}}
\end{definition}

\ifdochasnot{subspace}{
%--------------------------------------
\begin{definition}
\label{def:span}
\label{def:linspan}
\footnote{
  \citerpgc{michel1993}{86}{048667598X}{3.3.7 Definition}\\
  \citerpg{kurdila2005}{44}{3764321989}\\
  \citerpgc{searcoid2002}{71}{185233424X}{Definition 3.2.5---more general definition}
  %\citerppgc{heil2011}{20}{21}{0817646868}{Definition 1.25}
  }
%--------------------------------------
Let $\setxn{\vx_n\in\setX}$ be a set of vectors in a linear space $\linearspaceX$.
Let $\setA$ be a subset of $\setX$.
%Let $\spL\eqd\linearspaceX$ be a \structe{linear space} \xrefP{def:vspace}.
\defbox{\begin{array}{Ml} 
  The \hid{span} of $\setn{\vx_n}$ is defined as
  & \ds
  \hxs{\linspan}\setn{\vx_n}
  \eqd 
  \set{\sum_{n=1}^\xN \alpha_{n} \vx_{n} }{\alpha_n\in\F} .
  \\
  The set $\setn{\vx_n\in\setX}$ \hid{spans} the set $\setA$ if & \setA\subseteq\linspan\setn{\vx_n} .
  %\\
  %{The \hid{closed span} $\hxs{\linspanc}\seqxZp{\vx_n}$ of $\seqxZp{\vx_n}$ in $\spO$ is the \hie{closure} of $\linspan\seqxZp{\vx_n}$ in $\spO$.}
  %\\
  %{The sequence $\seqxZp{\vx_n}$ is \hid{complete} in $\spO$ if $\linspanc\seqxZp{\vx_n}$ spans $\setX$.}
\end{array}}
\end{definition}
}

%--------------------------------------
\begin{proposition}
\footnote{
  \citerpg{kubrusly2001}{46}{0817641742}
  }
\label{prop:spanAlinspace}
%--------------------------------------
%Let $\spL\eqd\linearspaceX$ be a \structe{linear space} \xrefP{def:vspace}.
Let $\setxn{\vx_n\in\setX}$ be a set of vectors in a linear space $\linearspaceX$.
\propbox{\begin{array}{M}
  $\linspan\setn{\vx_n}$ is a \structe{linear subspace}.
\end{array}}
\end{proposition}

%--------------------------------------
\begin{definition}
\footnote{
%  \citerpg{carothers2005}{24}{0521842832}\\
  \citerppg{bachman1966}{3}{4}{0486402517}\\
  %\citerppg{bachman2000fa}{3}{4}{0486402517}\\
  \citerpg{christensen2003}{2}{0817642951}\\
  \citerpgc{heil2011}{156}{0817646868}{Definition 5.7}
  }
\label{def:linin}
%--------------------------------------
Let $\setxn{\vx_n\in\setX}$ be a set of vectors in a linear space $\spL\eqd\linearspaceX$.
%Let $\spL\eqd\linearspaceX$ be a \structe{linear space} \xrefP{def:vspace}.
\defbox{\begin{array}{M}
  The set $\setxn{\vx_n\in\setX}$ is \hid{linearly independent} in $\spL$ if
  \\$\ds\qquad\sum_{n=1}^\xN \alpha_{n} \vx_{n} = 0 \qquad\implies\qquad \alpha_1=\alpha_2=\cdots=\alpha_\xN=0$.
  \\If the vectors in $\setn{\vx_n}$ are not linearly independent, then they are \hid{linearly dependent}.
  \\An infinite set $\setxZp{\vx_n\in\setX}$ is \prope{linearly independent} if
  \\\qquad every finite subset $\setxn{\vx_{\ff(n)}}$ is \prope{linearly independent}.
  \end{array}}
\end{definition}

%--------------------------------------
\begin{definition}
\footnote{
  \citerpg{bachman1966}{4}{0486402517}\\
  %\citerpg{bachman2000fa}{4}{0486402517}\\
  \citerppgc{kubrusly2001}{48}{49}{0817641742}{Section 2.4}\\
  \citerpg{young2001}{1}{0127729550}\\
  \citerpg{carothers2005}{25}{0521842832}\\
  \citerpgc{heil2011}{125}{0817646868}{Definition 4.1}\\
  \citor{hamel1905}
  }
\label{def:basis_hamel}
\label{def:hamel}
%---------------------------------------
Let $\setxn{\vx_n\in\setX}$ be a set of vectors in a linear space $\spL\eqd\linearspaceX$.
%\defbox{\begin{array}{>{\qquad\scy}r>{\ds}lDD}
\defbox{\begin{array}{>{\qquad\scy}rMD}
  \mc{3}{M}{The set $\setn{\vx_n}$ is a \hid{Hamel basis} (also called a \hid{linear basis}) for $\spL$ if}\\ %for each $\vx\in\setX$}\\
    1. & $\setn{\vx_n}$ \prope{spans} $\spL$ & and\\
    2. & $\setn{\vx_n}$ is \prope{linearly independent}.
    %1. & \exists \setxn{\alpha_n\in\F} \st \vx = \sum_{n=1}^\xN \alpha_{n} \vx_{n} & (\prope{linear combination}) & \qquad\text{and}\\
    %2. & \sum_{n=1}^\xN \alpha_{n} \vx_{n} = 0 \implies \alpha_1=\alpha_2=\cdots=\alpha_n=0  & (\prope{linearly independent}).
  \\
  \mc{3}{M}{If in addition $\ds\vx=\sum_{n=1}^\xN \alpha_{n} \vx_n$, then$\ds\sum_{n=1}^\xN \alpha_{n} \vx_n$ is the \hid{expansion} of $\vx$ on $\setn{\vx_n}$,}\\
  \mc{3}{M}{\qquad and the elements of $\seqn{\alpha_n}$ are called the \hid{coordinates} of $\vx$ with respect to $\setn{\vx_n}$.} \\
  \mc{3}{M}{If $\alpha_\xN\neq0$, then $\xN$ is the \hid{dimension} $\dimension\spL$ of $\spL$.}
  %\mc{3}{M}{A Hamel basis is also called a \hid{linear basis}.}
  \end{array}}
\end{definition}

%---------------------------------------
\begin{theorem}
\footnote{
  \citerppgc{michel1993}{89}{90}{048667598X}{Theorem 3.3.25}
  }
\label{thm:hamel_unique}
%---------------------------------------
Let $\setxn{\vx_n}$ be a Hamel basis for a linear space $\linearspaceX$.
\thmbox{
  \brb{\vx = \sum_{n=1}^\xN \alpha_{n} \vx_{n} = \sum_{n=1}^\xN \beta_{n}  \vx_n}
  %\brbr{\begin{array}{>{\scy}r>{\ds}lD}
  %  1. & \vx = \sum_{n=1}^\xN \alpha_{n} \vx_{n} & and \\
  %  2. & \vx = \sum_{n=1}^\xN \beta_{n}  \vx_{n} &
  %\end{array}}
  \qquad\implies\qquad 
  {\mcom{\alpha_n=\beta_{n} {\scy\quad \forall n=1,2,\ldots,\xN}}{coordinates of $\vx$ are \prope{unique}}}
  \qquad \forall\vx\in\setX
  }
\end{theorem}
\begin{proof}
\begin{align*}
  \vzero
    &= \vx - \vx
  \\&= \sum_{n=1}^\xN \alpha_{n} \vx_{n} - \sum_{n=1}^\xN \beta_{n}  \vx_{n} 
  \\&= \sum_{n=1}^\xN \brp{\alpha_n-\beta_n} \vx_n
  \\&\implies \text{$\setn{\vx_n}$ is \prope{linearly dependent} if $\brp{\alpha_n-\beta_n}\neq0 \qquad\forall n=1,2,\ldots,\xN$}
  \\&\implies \text{$\brp{\alpha_n-\beta_n}=0\qquad\forall n=1,2,\ldots,\xN$ \qquad \scs(because $\setn{\vx_n}$ is a \structe{basis} and therefore must be \prope{linearly independent})}
  %  && \text{because $\setn{\vx_n}$ is a \structe{basis} and therefore must be \prope{linearly independent}}
  \\&\implies \text{$\alpha_n=\beta_n$ for $n=1,2,\ldots,\xN$}
\end{align*}
\end{proof}

%---------------------------------------
\begin{theorem}
\footnote{
  \citerppgc{michel1993}{90}{91}{048667598X}{Theorem 3.3.26}
  }
\label{thm:frm_MN}
%---------------------------------------
Let $\spL\eqd\linearspaceX$ be a \structe{linear space}.
\thmbox{\begin{array}{>{\ds}l}
  \brb{\begin{array}{>{\scy}rMD}
    1. & $\ds\set{\vx_n\in\setX}{\scy n=1,2,\ldots,\xN}$ is a \structe{Hamel basis} for $\spL$ & and \\
    2. & $\ds\set{\vy_n\in\setX}  {\scy n=1,2,\ldots,\xM}$ is a set of \prope{linearly independent} vectors in $\spL$
  \end{array}}
  \\\qquad\implies\qquad
  \brb{\begin{array}{>{\scy}rMD}
    1. & $\xMn\le\xNn$ & and \\
    2. & $\xMn=\xNn \implies \set{\vy_n}{\scy n=1,2,\ldots,\xM}$ is a \structe{basis} for $\spL$ & and\\
    3. & $\xMn\neq\xNn \implies \set{\vy_n}{\scy n=1,2,\ldots,\xM}$ is \emph{not} a {basis} for $\spL$
  \end{array}}
\end{array}}
\end{theorem}
\begin{proof}
\begin{enumerate}
  \item Proof that $\setn{\vy_1,\,\vx_1,\,\ldots,\,\vx_{\xN-1}}$ is a \structe{basis} for $\spL$: \label{item:x1phiN1}
    \begin{enumerate}
      \item Proof that $\setn{\vy_1,\,\vx_1,\,\ldots,\,\vx_{\xN-1}}$ \prope{spans} $\spL$:
        \begin{enumerate}
          \item Because $\setxn{\vx_n}$ is a \structe{basis} for $\spL$, there exists $\beta\in\F$ and $\setxn{\alpha_n\in\F}$ such that 
            \\$\ds \beta\vy_1 + \sum_{n=1}^\xN \alpha_n\vx_{n} = 0$.\label{item:frm_MN_beta}
          \item Select an $n$ such that $\alpha_n\neq0$ and renumber (if necessary) the above indices such that 
            \\$\ds\vx_{n} = -\frac{\beta}{\alpha_n}\vy_1 +  \sum_{n=1}^{\xN-1} \frac{\alpha_n}{\alpha_n}\vx_n$.
          \item Then, for any $\vy\in\setX$, we can write
            \begin{align*}
              \vy 
                &= \sum_{n=1}^\xN \gamma_{n\in\Z} \vx_n
              \\&= \brp{\sum_{n=1}^{\xN-1} \gamma_{n\in\Z} \vx_n} + \gamma_{n\in\Z} \brp{-\frac{\beta}{\alpha_n}\vy_1- \sum_{n=1}^{\xN-1} \frac{\alpha_n}{\alpha_n}\vx_n}
              \\&= -\frac{\beta\gamma_n}{\alpha_n}\vy_1 + {\sum_{n=1}^{\xN-1} \brp{\gamma_n-\frac{\alpha_n\gamma_n}{\alpha_n}} \vx_n}
              \\&= \delta\vy_1 + \sum_{n=1}^{\xN-1}\delta_{n\in\Z} \vx_n
            \end{align*}
          \item This implies that $\setn{\vy_1,\,\vx_1,\,\ldots,\,\vx_{\xN-1}}$ \prope{spans} $\spL$:
        \end{enumerate}

      \item Proof that $\setn{\vy_1,\,\vx_1,\,\ldots,\,\vx_{\xN-1}}$ is \prope{linearly independent}:
        \begin{enumerate}
          \item If $\setn{\vy_1,\,\vx_1,\,\ldots,\,\vx_{\xN-1}}$ is \prope{linearly dependent}, then there exists 
                   $\setn{\epsilon,\,\epsilon_1,\,\ldots,\,\epsilon_{\xN-1}}$ such that 
            \\$\ds\epsilon\vy_1 + \brp{\sum_{n=1}^{\xN-1}\epsilon_{n\in\Z} \vx_n} + 0\vx_{n} = 0$.  \label{item:frm_MN_epsilon}
          \item \pref{item:frm_MN_epsilon} implies that the coordinate of $\vy_1$ associated with $\vx_n$ \emph{is} $0$.
            \\$\ds\vy_1 = -\brp{\sum_{n=1}^{\xN-1}\frac{\epsilon_n}{\epsilon} \vx_n} + 0\vx_{n} = 0$.  \label{item:frm_MN_epsilon1}
          \item \pref{item:frm_MN_beta} implies that the coordinate of $\vy_1$ associated with $\vx_n$ is \emph{not} $0$.
            \\$\ds \vy_1 = -\sum_{n=1}^\xN \frac{\alpha_n}{\beta}\vx_n$.  \label{item:frm_MN_epsilon2}
          \item This implies that \pref{item:frm_MN_epsilon} (that the set is linearly dependent) is \emph{false}
                because \pref{item:frm_MN_epsilon1} and \pref{item:frm_MN_epsilon2} \emph{contradict} each other.
          \item This implies $\setn{\vy_1,\,\vx_1,\,\ldots,\,\vx_{\xN-1}}$ is \prope{linearly independent}.
        \end{enumerate}

    \end{enumerate}

  \item Proof that $\setn{\vy_1,\,\vy_2,\,\vx_1,\,\ldots,\,\vx_{\xN-2}}$ is a \structe{basis}: Repeat \pref{item:x1phiN1}.

  \item Suppose $m=n$. Proof that $\setn{\vy_1,\,\vy_2,\,\ldots,\,\vy_\xM}$ is a \structe{basis}: 
        Repeat \pref{item:x1phiN1} $\xM-1$ times.
        \label{item:frm_MN_xM}

  \item Proof that $\xM\ngtr\xN$: \label{item:frm_MN_ngtr}
    \begin{enumerate}
      \item Suppose that $\xM=\xN+1$. 
      \item Then because $\setxn{\vy_n}$ is a \structe{basis}, there exists $\set{\zeta_n}{\scy n=1,2,\ldots,\xN+1}$ such that 
        \\$\ds \sum_{n=1}^{\xN+1} \zeta_{n\in\Z} \vy_{n\in\Z} = 0$.
      \item This implies that $\set{\vy_n}{\scy n=1,2,\ldots,\xN+1}$ is \prope{linearly dependent}.
      \item This implies that $\set{\vy_n}{\scy n=1,2,\ldots,\xN+1}$ is \emph{not} a basis.
      \item This implies that $\xM\ngtr\xN$.
    \end{enumerate}

  \item Proof that $\xM\neq\xN \implies \set{\vy_n}{\scy n=1,2,\ldots,\xM}$ is \emph{not} a {basis} for $\spL$:
    \begin{enumerate}
      \item Proof that $\xM>\xN \implies \set{\vy_n}{\scy n=1,2,\ldots,\xM}$ is \emph{not} a {basis} for $\spL$: same as in \pref{item:frm_MN_ngtr}.
      \item Proof that $\xM<\xN \implies \set{\vy_n}{\scy n=1,2,\ldots,\xM}$ is \emph{not} a {basis} for $\spL$: 
        \begin{enumerate}
          \item Suppose $m=\xN-1$.
          \item Then $\set{\vy_n}{\scy n=1,2,\ldots,\xN-1}$ is a \structe{basis} and there exists $\lambda$ such that
            \\$\ds\sum_{n=1}^\xN \lambda_{n\in\Z} \vy_{n\in\Z} = 0$.
          \item This implies that $\setxn{\vy_n}$ is \prope{linearly dependent} and is \emph{not} a basis.
          \item But this contradicts \pref{item:frm_MN_xM}, therefore $\xM\neq\xN-1$.
          \item Because $\xM=\xN$ yields a basis but $\xM=\xN-1$ does not, $\xM<\xN-1$ also does not yield a basis.
        \end{enumerate}
    \end{enumerate}
\end{enumerate}
\end{proof}

%---------------------------------------
\begin{corollary}
\footnote{
  \citerpgc{kubrusly2001}{52}{0817641742}{Theorem 2.7}\\
  \citerpgc{michel1993}{91}{048667598X}{Theorem 3.3.31}
  }
\label{cor:frm_MN}
%---------------------------------------
Let $\spL\eqd\linearspaceX$ be a linear space.
\corbox{
  \mcom{
  \brb{\begin{array}{>{\scy}rMD}
    1. & $\ds\setxn{\vx_n\in\setX}$ is a \structe{Hamel basis} for $\spL$ & and \\
    2. & $\ds\set{\vy_n\in\setX}{\scy n=1,2,\ldots,\xM}$ is a \structe{Hamel basis} for $\spL$
  \end{array}}
  \qquad\implies\qquad \brb{\xNn=\xMn}
  }{(all Hamel bases for $\spL$ have the same number of vectors)}
  }
\end{corollary}
\begin{proof}
This follows from \prefpp{thm:frm_MN}.
\end{proof}

%=======================================
\subsection{Total sets in Topological Linear Spaces}
%=======================================
A linear space supports the concept of the \structe{span} of a set of vectors \xrefP{def:span}.
%The span of a set of vectors in a linear space is itself a \structe{linear space} \xrefP{prop:spanAlinspace}.
%In a \prope{finite} linear space, a set of vectors that \prope{spans} the space is also said to be \prope{total} 
%in that space.
In a topological linear space $\spO\eqd\toplinspaceX$, a set $\setA$ is said to be \prope{total} in $\spO$
if the span of $\setA$ is \prope{dense} in $\spO$.
%if $\clsp{\linspan\setA}=\spX$ \xrefP{def:total_set}.
In this case, $\setA$ is said to be a \structe{total set} or a \structe{complete set}.
However, this use of ``complete" in a ``\structe{complete set}" is not equivalent to 
the use of ``complete" in a ``\structe{complete metric space}" \ifxref{seq}{def:complete}.\footnote{%
  \citerppgc{haaser1991}{296}{297}{0486665097}{6$\cdot$Orthogonal Bases}\\
  \citerpgc{rynne2008}{78}{1848000057}{Remark 3.50}\\
  \citerpgc{heil2011}{21}{0817646868}{Remark 1.26}
  }
In this text, except for these comments and \pref{def:total_set}, 
``complete" refers to the metric space definition \ifxref{seq}{def:complete} only.


%If a \structe{total set} in $\spX$ with any one of its vectors removed is \emph{not} a total set in $\spX$,
%then that total set is a \structe{basis} for $\spX$.
If a set is both \prope{total} and \prope{linearly independent} \xrefP{def:linin} in $\spO$, then 
that set is a \structe{Hamel basis} \xrefP{def:basis_hamel} for $\spO$.

%If we also add the requirement of uniqueness, then we have a \structe{Hamel basis} (1905, \xrefp{def:basis_hamel}).
%%This is often sufficient for constructing a basis in a finite space (see \structe{Hamel basis}, \prefp{def:basis_hamel}).
%However, in an \prope{infinite} linear space, in general the concepts of completeness, the Hamel basis, and even 
%infinite summation is undefined.
%To define the \structe{total set} in an infinite linear space (\pref{def:total_set}, next definition), 
%we must first append a topology to the linear space.
%That is, we need a \structe{topological linear space} \xrefP{def:toplinspace}.
%Examples of topological linear spaces include metric linear spaces, normed linear spaces, and inner product spaces. 
%To define a basis, we also need the concept of \prope{convergence}.
%For convergence to a unique element, a topological space is not sufficient, but metric spaces 
%(which include normed linear spaces and inner product spaces) are sufficient.
%
%The \structe{Schauder basis} (1927, \xrefp{def:basis_schauder}) is defined in 
%a normed linear space. % (which also includes the inner product space).
%%\hie{Hamel bases} (1905) are often used in finite dimensional linear spaces. 
%%However, in an infinite dimensional space, Hamel bases are frought with problems.
%%One problem is that even if the linear space is \prope{separable}, a Hamel basis for that linear space may be \prope{uncountable}.
%%In a Banach space, the often preferred bases are the \hie{Schauder bases} (1927, next definition).
%Schauder bases exploit assets that are available in Banach spaces but are not in general available in linear spaces:
%a topology induced by the norm $\normn$ and the property of \prope{completeness}.
%In escence, Hamel bases are \prope{algebraic} only, while Schauder bases are both algebraic and \prope{topological}.
%
%%The number of Banach spaces over a set $\setX$ tends to be much much larger than the set of Hilbert spaces over the same set.
%%Take for example the norm $\ds\norm{x}_p\eqd\brp{\sum_{n=1}^\infty x^p}^\frac{1}{p}$ ($p\in\R$) over the set of real numbers $\R$.
%%Of these uncountably infinite number of norms, only one of them ($p=2$) generates an inner product.
%%
%%But the set of Banach spaces over a set $\setX$ is not the largest of the sets of topological spaces;
%%the set of metric spaces is still larger.
%%But one advantage that Banach spaces have over metric spaces is that the norm at the core of a Banach space is 
%%algebraic in nature
%
%Note that by definition $\sum_{n=1}^\infty \alpha \vx_{n} \eqd \lim_{\xN\to\infty} \sum_{n=1}^\xN \alpha_{n} x_n$, 
%and that this definition depends on the the operator ``$\lim$" being 
%defined \xrefP{def:strong_converge},
%which in turn requires a topology, such as is available in a \structe{Banach space} \xrefP{def:suminf}.
%Furthermore, note that in a Banach space, the equation
%$\vx\eqs\sum_{n=1}^\infty\alpha_n\vx_n$ by definition implies $\lim_{\xN\to\infty}\norm{\vx-\sum_{n=1}^\xN\alpha_n\vx_n}=0$
%(\prope{strong convergence}, \xrefp{def:strong_converge}),
%where the symbol ``$\eqs$" denotes \prope{strong convergence} \xrefP{def:eqs}.

%\ifdochasnot{series}{
%%---------------------------------------
%\begin{definition}
%\footnote{
%  \citerpg{klauder2010}{4}{0817647902}\\
%  \citerpg{kubrusly2001}{43}{0817641742}
%  }
%\label{def:suminf}
%%---------------------------------------
%Let $\topspaceX$ be a topological space and $\lim$ be the limit generated by the topology $\topT$.
%\defbox{
%  \sum_{n=1}^\infty \vx_n \eqd \lim_{\xN\to\infty} \sum_{n=1}^\xN \vx_n 
%  }
%\end{definition}
%}

%--------------------------------------
\begin{definition}
\footnote{
  \citerpgc{young2001}{19}{0127729550}{Definition 1.5.1}\\
  \citerpgc{sohrab2003}{362}{0817642110}{Definition 9.2.3}\\
  \citerpgc{gupta1998}{134}{8186358447}{Definition 2.4}\\
  \citerppgc{bachman1966}{149}{153}{0486402517}{Definition 9.3, Theorems 9.9 and 9.10}\\
  %\citerpgc{heil2011}{21}{0817646868}{Definition 1.25}
  }
\label{def:total_set}
%--------------------------------------
Let $\clsA$ be the \prope{closure} \ifxref{topology}{def:clsA} of a $\setA$ in a 
\structe{topological linear space} \ifxref{vstopo}{def:toplinspace} $\spO\eqd\toplinspaceX$.
Let $\linspan\setA$ be the \structe{span} \xref{def:span} of a set $\setA$.
\defboxt{
  A set of vectors $\setA$ is \hid{total} (or \hid{complete} or \hid{fundamental}) in $\spO$ if
  \\\indentx
  $\clsp{\linspan\setA} = \spO$ \qquad\scs(\structe{span} of $\setA$ is \prope{dense} \ifxref{topology}{def:dense} in $\spO$).
  }
\end{definition}


%=======================================
\subsection{Total sets in Banach spaces}
%=======================================
\ifdochasnot{vsnorm}{
Often in a linear space we have the option of appending additional structures
that offer useful functionality.
One of these structures is the \hie{norm} (next definition).
The norm of a vector can be described as the ``\hie{length}"
or the ``\hie{magnitude}" of the vector.
%A norm is similar to the concept of a \hie{measure}.
%\ifdochas{measure}{\footnote{{\em measure:} \prefp{def:measure}}}
%But a measure operates on a set, whereas a norm operates on a vector.

%--------------------------------------
\begin{definition}
\footnote{
  \citerpp{ab}{217}{218} \\
  \citorp{banach1932}{53}  \\
  \citorp{banach1932e}{33} \\
  \citorp{banach1922}{135}
 %\citerp{michel1993}{344} \\
 %\citerp{horn}{259}  \\
  }
\label{def:norm}
\index{space!normed vector}
\index{triangle inequality}
\index{inequality!triangle}
\index[xsym]{$\normn$}
%--------------------------------------
Let $\linearspaceX$ be a linear space and
$\abs{\cdot}\in\clF{\F}{\R}$ the absolute value function.
\ifdochas{algebra}{\footnote{
  \hie{absolute value}: \prefp{def:abs}
  }}
\defbox{\begin{array}{>{\qquad\scy}r rcl @{\qquad}C @{\qquad}D @{\qquad}D}
  \mc{7}{M}{A functional $\normn$ in $\clFxr$ is a \hid{norm} if}
    \\1. & \norm{ \vx}      &\ge& 0                     & \forall \vx \in\setX                & (\prope{strictly positive})  \nocite{michel1993}           & and  %page 115 
    \\2. & \norm{ \vx}      &=  & 0 \iff \vx=\vzero     & \forall \vx \in\setX                & (\prope{nondegenerate})                                    & and
    \\3. & \norm{\alpha\vx} &=  & \abs{\alpha}\norm{\vx}& \forall \vx \in\setX,\; \alpha\in\C & (\prope{homogeneous})                                      & and
    \\4. & \norm{\vx+\vy}   &\le& \norm{\vx}+\norm{\vy} & \forall \vx,\vy \in\setX            & (\prope{subadditive}/\prope{triangle inequality}).
  \\\mc{7}{M}{A \hib{normed linear space} is the tuple $\normspaceX$.}
\end{array}}
\end{definition}
}

%=======================================
%\subsubsection{Bases in Banach spaces}
%=======================================

\ifdochasnot{series}{
%--------------------------------------
\begin{definition}
\footnote{
  \citerpgc{bachman1966}{138, 247}{0486402517}{Definition 9.1}\\
  \citerpgc{katznelson2004}{67}{0521543592}{section 1.1}
  }
\label{def:eqs}
%--------------------------------------
Let $\spB\eqd\normspaceX$ be a \structe{Banach space}.
\defboxt{
  $\eqs$ represent \prope{strong convergence} in $\spB$. That is, 
  \\%\indentx%
    $\ds
    \vx\eqs\sum_{n=1}^\infty\alpha_n\vx_n 
    \quad\iffdef\quad 
    \lim_{\xN\to\infty}\norm{\vx-\sum_{n=1}^\xN\alpha_n\vx_n}=0
    \quad\iffdef\quad 
    \brb{\begin{array}{M}
      for every $\varepsilon>0$,\\ 
      there exists $\xM$ such that\\ 
      for all $\xN>\xM$,
      \\\indentx
      $\ds\norm{\vx-\sum_{n=1}^\xN\alpha_n\vx_n}<\varepsilon$
      \end{array}}
    $
  }
\end{definition}
  }

%--------------------------------------
\begin{definition}
\footnote{
  \citerppg{carothers2005}{24}{25}{0521842832}\\
  \citerppgc{christensen2003}{46}{49}{0817642951}{Definition 3.1.1 and page 49}\\
  \citerpgc{young2001}{19}{0127729550}{Section 6}\\
  \citerp{singer1970}{17}\\
  %\citerpgc{heil2011}{24}{0817646868}{Definition 1.32}\\
  %\citerpgc{heil2011}{131}{0817646868}{Definition 4.7}\\
  %\citerpg{christensen2008}{2}{0817646779}\\
  \citor{schauder1927}\\
  \citor{schauder1928}
  }
\label{def:basis_schauder}
\label{def:schauder}
%--------------------------------------
Let $\spB\eqd\normspaceX$ be a \structe{Banach space}.
Let $\eqs$ represent \prope{strong convergence} \xref{def:eqs} in $\spB$.
\defbox{\begin{array}{M}
  The countable set $\setxZp{\vx_n\in\setX}$ is a \hid{Schauder basis} for $\spB$ if for each $\vx\in\setX$
    \\\indentx$\begin{array}{F>{\ds}lD}
        %1. & \exists \setxn{\alpha_n\in\F} \st \lim_{\xN\to\infty}\norm{\vx - \sum_{n=1}^\xN \alpha_{n} \vx_n}=0 & (\prope{convergent in norm}) 
        1. & \exists \seqxZp{\alpha_n\in\F} \quad\st\quad \vx \eqs \sum_{n=1}^\infty \alpha_n \vx_n 
           & (\prope{strong convergence} in $\spB$)
             and%
           \\
        2. & \brb{\sum_{n=1}^\infty \alpha_{n} \vx_{n} \eqs \sum_{n=1}^\infty \beta_{n}  \vx_n} 
             \implies 
             \brb{\seqn{\alpha_n}=\seqn{\beta_n}}  
           & (\structe{coefficient functionals} are \prope{unique})%
           %&
    \end{array}$
      \\
      In this case, $\ds\sum_{n=1}^\infty \alpha_{n} \vx_n$ is the \hid{expansion} of $\vx$ on $\setxZp{\vx_n}$ and\\
      the elements of $\seqn{\alpha_n}$ are the \hid{coefficient functionals} associated with the basis $\setn{\vx_n}$.\\
      Coefficient functionals are also called \hid{coordinate functionals}.
      %A \hid{Schauder basis} may also be simply called a \hid{basis}.
    \end{array}}
\end{definition}

In a Banach space, the existence of a Schauder basis implies that the space is \prope{separable} %\ifsxref{topology}{def:separable}
\xref{thm:Bschauder==>separable}. %(next theorem).
The \label{BasisProblem} question of whether the converse is also true was posed by Banach himself in 1932,\cittrp{banach1932}{111}
and became know as ``\hie{The basis problem}".
This remained an open question for many years.
The question was finally answered some 41 years later in 1973 by \hi{Per Enflo} (University of California at Berkley), 
with the answer being ``no".
Enflo constructed a counterexample in which a separable Banach space does \emph{not} have a Schauder 
basis.\footnote{
  \citor{enflo1973}\\
  \citerppgc{lindenstrauss1977}{84}{95}{3642665594}{Section 2.d}
  }
Life is simpler in Hilbert spaces where the converse \emph{is} true:
a Hilbert space has a Schauder basis \emph{if and only if} it is separable \xref{thm:schauder<==>separable}.

%--------------------------------------
\begin{theorem}
\footnote{
  \citerpgc{bachman2000}{112}{0387988998}{3.4.8}\\
  \citerpg{giles2000}{17}{0521653754}\\
  \citerpgc{heil2011}{21}{0817646868}{Theorem 1.27}
  }
\label{thm:Bschauder==>separable}
%--------------------------------------
%Let $\seqxZp{\vx_n\in\setX}$ be a sequence of vectors in a Banach space\\
Let $\spB\eqd\normspaceX$ be a \structe{Banach space}.
Let the field of rational numbers $\Q$ be \prope{dense} in the field $\F$.
%Let $\normn$ be a function in $\clF{\setX}{\R}$ on a linear space $\spB\eqd\linearspaceX$.
%Let $\seqxZp{x_n}$ be a sequence in $\spX$.
\thmbox{\begin{array}{M}
  $\spB$ has a \structe{Schauder basis}
  $\qquad\implies\qquad$
  $\spB$ is \prope{separable}
  %If $\opair{\spB}{\normn}$ is a \prope{normed} linear space then 
  %   \\\indentx $\exists$ a sequence $\seqxZp{\vx_n}$ that is \prope{complete} in $\spB$ $\implies$ $\setX$ is \prope{separable}.
  \end{array}}
\end{theorem}
\begin{proof}
\begin{enumerate}
  \item lemma: \label{item:Bschauder==>separable_seto}
    \begin{align*}
      \seto{\set{\vx}{\text{$\exists\seqxZp{\alpha_n\Q}$ such that $\ds\lim_{\xN\to\infty}\norm{\vx-\sum_{n=1}^\xN\alpha_n\vx_n}=0$}}} 
        &= \seto{\Q\times\Zp} 
      \\&= \seto{\Z\times\Z} 
      \\&= \seto{\Z} 
      \\&= \text{\prope{countably infinite}}
    \end{align*}

  \item remainder of proof:
    \begin{align*}
      &\text{$\spB$ has a \structe{Schauder basis} $\seqxZp{\vx_n}$}
      \\&\implies \text{for every $\vx\in\spB$, there exists $\seqxZp{\alpha_n\in\F}$ such that $\ds\vx\eqs\sum_{n=1}^\infty\alpha_n\vx_n$}
        %&& \text{by definition of \structe{Schauder basis} \xref{def:schauder}}
        && \text{by \prefp{def:schauder}}
      \\&\implies \text{for every $\vx\in\spB$, there exists $\seqxZp{\alpha_n\in\F}$ such that $\ds\lim_{\xN\to\infty}\norm{\vx-\sum_{n=1}^\xN\alpha_n\vx_n}=0$}
        %&& \text{by definition of $\eqs$ \xref{def:eqs}}
        && \text{by \prefp{def:eqs}}
      \\&\implies \text{for every $\vx\in\spB$, there exists $\seqxZp{\alpha_n\in\Q}$ such that $\ds\lim_{\xN\to\infty}\norm{\vx-\sum_{n=1}^\xN\alpha_n\vx_n}=0$}
        && \text{because $\cls{\Q}=\F$}
      \\&\implies \spB = \set{\vx}{\text{$\exists \seqxZp{\alpha_n\Q}$ such that $\ds\lim_{\xN\to\infty}\norm{\vx-\sum_{n=1}^\xN\alpha_n\vx_n}=0$}}
      \\&\implies \spB = \cls{\set{\vx}{\text{$\exists \seqxZp{\alpha_n\Q}$ such that $\ds\vx=\lim_{\xN\to\infty}\sum_{n=1}^\xN\alpha_n\vx_n$}}}
      \\&\implies \text{$\spB$ is \prope{separable}}
        && \text{by \prefp{item:Bschauder==>separable_seto}}
    \end{align*}
\end{enumerate}
\end{proof}


%--------------------------------------
\begin{definition}
\footnote{
  \citerpgc{young2001}{25}{0127729550}{Definition 1.8.1, Theorem 1.8.7}
  }
\label{def:equivalent}
%--------------------------------------
Let $\setxZp{\vx_n}$ and $\setxZp{\vy_n}$ be \structe{Schauder bases} of a \structe{Banach space} $\normspaceX$.
\defboxt{
  $\setn{\vx_n}$ is \hid{equivalent} to $\setn{\vy_n}$
  \\if there exists a \prope{bounded} \prope{invertible} operator $\opR$ in $\clOxx$ such that
  %\\\indentx
  \quad$\opR\vx_n = \vy_n\qquad\scy\forall n\in\Z$
  }
\end{definition}

%--------------------------------------
\begin{theorem}
\footnote{
  \citerpgc{young2001}{25}{0127729550}{Definition 1.8.1, Theorem 1.8.7}
  }
\label{thm:equivalent_convergent}
%--------------------------------------
Let $\setxZp{\vx_n}$ and $\setxZp{\vy_n}$ be \structe{Schauder bases} of a \structe{Banach space}\\$\normspaceX$.
\thmboxt{
  $\brb{\text{$\setn{\vx_n}$ is \prope{equivalent} to $\setn{\vy_n}$}}$
  \\$\qquad\iff\qquad$
  $\brb{
    \text{$\ds\sum_{n=1}^\infty \alpha_n \vx_n$ is \prope{convergent}}
    \iff 
    \text{$\ds\sum_{n=1}^\infty \alpha_n \vy_n$ is \prope{convergent}}
    }$
  }
\end{theorem}

%======================================
\subsection{Total sets in Hilbert spaces}
%======================================
%=======================================
\subsubsection{Orthogonal sets in Inner product space}
%=======================================
\ifdochasnot{vsinprod}{
%--------------------------------------
\begin{definition}
\label{def:inprod}
\index{$\inprodn$}
\index{space!inner product}
\footnote{
  \citerpgc{istratescu1987}{111}{9027721823}{Definition 4.1.1}\\
  \citerppg{bollobas1999}{130}{131}{0521655773}\\
  \citerpg{haaser1991}{277}{0486665097} \\
  \citerp{ab}{276} \\
  \citorp{peano1888e}{72} 
  }
\indxs{\inprodn}
%--------------------------------------
Let $\spL\eqd\linearspaceX$ be a linear space.
\defbox{\begin{array}{>{\qquad}F >{\ds}rc>{\ds}l @{\qquad}C @{\qquad}D @{\qquad}D}
  \mc{7}{M}{A function $\inprodn\in\clFxxf$ is an \hid{inner product} on $\spL$ if}
  \\
   1. & \inprod{\alpha\vx}{\vy}    &=& \alpha\inprod{\vx}{{\vy}}
      & \forall \vx,\vy\in X,\;\forall\alpha\in\C
      & (\prope{homogeneous})
      & and
      \\
   2. & \inprod{\vx+\vy}{\vu} &=& \inprod{\vx}{{\vu}} + \inprod{\vy}{{\vu}}
      & \forall \vx,\vy,\vu\in X
      & (\prope{additive})
      & and
      \\
   3. & \inprod{\vx    }{\vy} &=& \inprod{\vy}{\vx}^\ast
      & \forall \vx,\vy\in X
      & (\prope{conjugate symmetric}).
      & and
      \\
   4. & \inprod{\vx    }{\vx} &\ge& 0
      & \forall \vx\in X
      & (\prope{non-negative})
      & and
      \\
   5. & \inprod{\vx    }{\vx} &=& 0 \iff \vx=\vzero
      & \forall \vx\in X
      & (\prope{non-isotropic})
  \\
  \mc{7}{M}{An inner product is also called a \hid{scalar product}.}\\
  \mc{7}{M}{The tuple $\inprodspaceX$ is an \hid{inner product space}.}
  %A function $\inprodn:X\times X\to\C$ is a \hid{hermitian form} on $\spL$ if it satisfies conditions 1--3.
  %A function $\inprodn:X\times X\to\C$ is a \hid{pre-inner product} on $\spL$ if it satisfies conditions 1--4.
\end{array}}
\end{definition}


\ifdochasnot{relation}{
%---------------------------------------
\begin{definition}
\label{def:kdelta}
%---------------------------------------
\defbox{\begin{array}{M}
The \hid{Kronecker delta function} $\hxs{\kdelta_n}$ is defined as
  \qquad
  $\ds\kdelta_n\eqd
    \brbl{\begin{array}{cMD}
      1 & for $n=0$ & and \\
      0 & for $n\ne 0$\..
    \end{array}}
  \qquad\scy
  \forall n\in\Z$
\end{array}}
\end{definition}
}

%--------------------------------------
\begin{definition}
\label{def:orthog}
%\footnote{
%  \citer{james1945}
%  }
\index{orthogonality!inner product space}
%--------------------------------------
Let $\inprodspaceX$ be an \structe{inner product space} \xref{def:inprod}.
\defboxt{
  Two vectors $\vx$ and $\vy$ in $\setX$ are \hid{orthogonal} if 
  \\\indentx$\ds
    \inprod{\vx}{\vy}=
      \brbl{\begin{array}{lM}
        0            & for $\vx\neq\vy$\\
        c\in\F\setd0 & for $\vx=\vy$
      \end{array}}$
  \\
  The notation $\vx\orthog\vy$ implies $\vx$ and $\vy$ are orthogonal.
  \\
  A set $\setY\in\psetX$ is \hid{orthogonal} if $\vx\orthog\vy\quad{\scy\forall\vx,\vy\in\setY}$.
  \\
  A set $\setY$ is \hid{orthonomal} if it is \prope{orthogonal} and $\inprod{\vy}{\vy}=1\quad{\scy\forall\vy\in\setY}$.
  \\
  A sequence $\seqxZ{\vx_n\in\setX}$ is \hid{orthogonal} if $\inprod{\vx_n}{\vx_m}=c\kdelta_{nm}$ for some $c\in\R\setd0$.
  \\
  A sequence $\seqxZ{\vx_n\in\setX}$ is \hid{orthonormal} if $\inprod{\vx_n}{\vx_m}=\kdelta_{nm}$.
  }
\end{definition}
}

%--------------------------------------
\begin{definition}
\label{def:biortho}
%--------------------------------------
Let $\spX\eqd\inprodspaceX$ be an \structe{inner product space} \xref{def:inprod}.
\defboxt{
  The sequences $\seqxZ{\vx_n\in\setX}$ and $\seqxZ{\vy_n\in\setX}$ are \hid{biorthogonal} 
  \\with respect to each other in $\spX$ if
  \indentx$\inprod{\vx_n}{\vy_m}=\kdelta_{nm}$
  }
\end{definition}


In an inner product space, \prope{orthogonality} is a special case of \prope{linear independence};
or alternatively, linear independence is a generalization of orthogonality \xrefP{thm:orthog==>linin}.

The \thme{triangle inequality}\index{inequality!triangle} 
theorem for vectors in a normed linear space
\ifdochas{vsnorm}{\xrefP{thm:norm_tri}} 
demonstrates that 
  \\\indentx$\ds\norm{\sum_{n=1}^\xN \vx_n} \le \sum_{n=1}^\xN \norm{\vx_n}.$
  \\
The \hie{Pythagorean Theorem} (\pref{thm:pythag}, next) demonstrates that this 
{\em inequality} becomes {\em equality} when the set $\setn{\vx_n}$ is orthogonal.


%--------------------------------------
\begin{theorem}[\thmd{Pythagorean Theorem}]
\footnote{
  %\citerp{pinsky2002}{305} \\
  \citerppgc{ab}{282}{283}{0120502577}{Theorem 32.7}\\
  \citerpgc{kubrusly2001}{324}{0817641742}{Proposition 5.8}\\
  \citerppgc{bollobas1999}{132}{133}{0521655773}{Theorem 3}
  }
\label{thm:pythag}
%--------------------------------------
Let $\setxn{\vx_n\in\setX}$ be a set of vectors in an \structe{inner product space} \xrefP{def:inprod} $\inprodspaceX$ and let
$\norm{\vx}\eqd\sqrt{\inprod{\vx}{\vx}}$\ifsxref{vsinprod}{def:norm=inprod}.
\thmbox{\begin{array}{>{\ds}c}
  \text{$\setn{\vx_n}$ is \prope{orthogonal}}
  \qquad\iff\qquad 
  \norm{\sum_{n=1}^\xN \vx_{n}}^2 = \sum_{n=1}^\xN \norm{\vx_n}^2
  \qquad\scriptstyle
  \forall \xN\in\Zp
  \end{array}}
\end{theorem}
\begin{proof}
%\begin{enumerate}
  1. Proof for ($\implies$) case:
    \begin{align*}
      \norm{\sum_{n=1}^\xN \vx_n}^2
        &= \inprod{\sum_{n=1}^\xN \vx_n}{\sum_{m=1}^\xN \vx_m}
        && \text{by def. of $\normn$}
      \\&= \sum_{n=1}^\xN \sum_{m=1}^\xN \inprod{\vx_n}{\vx_m}
        && \text{by def. of $\inprodn$ (\prefp{def:inprod})}
      \\&= \sum_{n=1}^\xN \sum_{m=1}^\xN \inprod{\vx_n}{\vx_m}\kdelta_{n-m}
        && \text{by left hypothesis}
      \\&= \sum_{n=1}^\xN \inprod{\vx_n}{\vx_n}
        && \text{by def. of $\kdelta$}
      \\&= \sum_{n=1}^\xN \norm{\vx_n}^2
        && \text{by def. of $\normn$}
    \end{align*}

  2. Proof for ($\impliedby$) case:
    \begin{align*}
      &4\inprod{\vx}{\vy}
      \\&= \norm{\vx+ \vy}^2 -\norm{\vx- \vy}^2 +i\norm{\vx+i\vy}^2 -i\norm{\vx-i\vy}^2
        && \text{by polarization id.}% \prefpo{thm:polar_id}}
      \\&= \left(\norm{\vx}^2 + \norm{  \vy}^2\right)
         - \left(\norm{\vx}^2 + \norm{- \vy}^2\right)
         +i\left(\norm{\vx}^2 + \norm{ i\vy}^2\right)
         -i\left(\norm{\vx}^2 + \norm{-i\vy}^2\right)
        && \text{by right hypoth.}%esis}
      \\&= \left(\norm{\vx}^2 +           \norm{\vy}^2\right)
         - \left(\norm{\vx}^2 + \abs{-1}^2\norm{\vy}^2\right)
         +i\left(\norm{\vx}^2 + \abs{ i}^2\norm{\vy}^2\right)
         -i\left(\norm{\vx}^2 + \abs{-i}^2\norm{\vy}^2\right)
        && \text{by def. of $\normn$}% \prefpo{def:norm}}
      \\&= \left(\norm{\vx}^2 + \norm{\vy}^2\right)
         - \left(\norm{\vx}^2 + \norm{\vy}^2\right)
         +i\left(\norm{\vx}^2 + \norm{\vy}^2\right)
         -i\left(\norm{\vx}^2 + \norm{\vy}^2\right)
        && \text{by def. of $\abs{\cdot}$} % \ifdochas{algebra}{\prefpo{def:abs}}
      \\&= 0
    \end{align*}
%\end{enumerate}
\end{proof}






%--------------------------------------
\begin{theorem}
\footnote{
  \citerpgc{ab}{283}{0120502577}{Corollary 32.8}\\
  \citerpgc{kubrusly2001}{352}{0817641742}{Proposition 5.34}
  }
\label{thm:orthog==>linin}
%--------------------------------------
Let $\setxn{\vx_n\in\setX}$ be a set of vectors in an \structe{inner product space} \xrefP{def:inprod} $\inprodspaceX$.
\thmbox{
  \text{$\setn{\vx_n}$ is \prope{orthogonal}}
  \qquad\implies\qquad
  \text{$\setn{\vx_n}$ is \prope{linearly independent}}
  }
\end{theorem}
\begin{proof}
\begin{enumerate}
  \item Proof using Pythagorean theorem (\prefp{thm:pythag}):
    \\Let $\seqxZp{\alpha_n}$ be a sequence with at least one nonzero element.
    \\\begin{align*}
        \norm{\sum_{n=1}^\xN \alpha_{n} \vx_n}^2
          &= \sum_{n=1}^\xN \norm{\alpha_{n} \vx_n}^2
          && \text{by left hypothesis and \prefp{thm:pythag}}
        \\&= \sum_{n=1}^\xN \abs{\alpha_n}^2\norm{\vx_n}^2
        \\&> 0
        \\&\implies \sum_{n=1}^\xN \alpha_{n} \vx_{n} \neq 0
        \\&\implies \text{$\seqxZp{\vx_n}$ is \prope{linearly independent}}
      \end{align*}

  \item Alternative proof:
    \begin{align*}
      \sum_{n=1}^\xN \alpha_n\vx_{n} = \vzero
        &\implies  \inprod{\sum_{n=1}^\xN \alpha_n\vx_n}{\vx_m} = \inprod{\vzero}{\vx_m}
      \\&\implies  \sum_{n=1}^\xN \alpha_n\inprod{\vx_n}{\vx_m} = 0
      \\&\implies  \sum_{n=1}^\xN \alpha_n\kdelta(k-m) = 0
      \\&\implies  \alpha_m = 0 \qquad \text{for $m=1,2,\ldots,\xN$}
    \end{align*}

\end{enumerate}
\end{proof}


%--------------------------------------
\begin{theorem}[\thmd{Bessel's Equality}]
\label{thm:bessel_eq}
\footnote{
  \citerpg{bachman2000}{103}{0387988998}\\
  \citerppg{pedersen2000}{38}{39}{0849371694}
  
  }
\index{equalities!Bessel's}
%--------------------------------------
Let $\setxn{\vx_n\in\setX}$ be a set of vectors in an \structe{inner product space} \xrefP{def:inprod} $\inprodspaceX$ and with
$\norm{\vx}\eqd\sqrt{\inprod{\vx}{\vx}}$\ifsxref{vsinprod}{def:norm=inprod}.
\thmbox{
  \brb{\begin{array}{M}
    $\setn{\vx_n}$ is\\ 
    \prope{orthonormal}
  \end{array}}
  \implies
  \norm{\vx-\sum_{n=1}^\xN \inprod{\vx}{\vx_n} \vx_{n} }^2
    = \norm{\vx}^2 - \sum_{n=1}^\xN |\inprod{\vx}{\vx_n}|^2
  \quad\scy \forall \vx\in\setX
  }
\end{theorem}
\begin{proof}
\begin{align*}
  &\norm{\vx-\sum_{n=1}^\xN \inprod{\vx}{\vx_n} \vx_{n} }^2
  \\&=    \norm{\vx}^2
          + \norm{\sum_{n=1}^\xN \inprod{\vx}{\vx_n} \vx_{n} }^2
          - 2\Re{\inprodr{\vx}{\sum_{n=1}^\xN \inprod{\vx}{\vx_n}\vx_n}}
    &&    \ifdochas{vsinprod}{\text{by \prefp{lem:||x+y||}}}
  \\&=    \norm{\vx}^2
          + \norm{\sum_{n=1}^\xN \inprod{\vx}{\vx_n} \vx_{n} }^2
          - 2\Re\left[\left(\sum_{n=1}^\xN \inprod{\vx}{\vx_n}\right)^\ast\;\inprod{\vx}{\vx_n}\right]
    &&    \text{by property of $\inprodn$ \xref{def:inprod}}
    %&&    \text{by \prefp{def:inprod}}
  \\&=    \norm{\vx}^2
          + \sum_{n=1}^\xN \norm{\inprod{\vx}{\vx_n} \vx_{n} }^2
          - 2\Re\left[\left(\sum_{n=1}^\xN \inprod{\vx}{\vx_n}\right)^\ast\;\inprod{\vx}{\vx_n}\right]
    &&    \text{by Pythagorean Thm. \xref{thm:pythag}}
    %&&    \text{by \pref{thm:pythag}}
  \\&=    \norm{\vx}^2
          + \sum_{n=1}^\xN \norm{\inprod{\vx}{\vx_n} \vx_{n} }^2
          - 2\Re\left(\sum_{n=1}^\xN \inprod{\vx}{\vx_n}^\ast\;\inprod{\vx}{\vx_n}\right)
    &&    \ifdochas{numsys}{\text{by \prefp{thm:conj}}}
  \\&=    \norm{\vx}^2
          + \sum_{n=1}^\xN \abs{\inprod{\vx}{\vx_n}}^2\; \mcom{\norm{\vx_{n} }^2}{$1$}
          - 2\Re\left(\sum_{n=1}^\xN \inprod{\vx}{\vx_n}^\ast\;\inprod{\vx}{\vx_n}\right)
    &&    \text{by property of $\normn$ \xref{def:norm}}
  \\&=    \norm{\vx}^2
          + \sum_{n=1}^\xN \abs{\inprod{\vx}{\vx_n}}^2\; \cdot 1
          - 2\Re\left(\sum_{n=1}^\xN \inprod{\vx}{\vx_n}^\ast\;\inprod{\vx}{\vx_n}\right)
    &&    \text{because $\setn{\vx_n}$ is \prope{orthonormal}} % \{def:orthog}
  \\&=    \norm{\vx}^2
          + \sum_{n=1}^\xN \abs{\inprod{\vx}{\vx_n}}^2
          - 2\Re\sum_{n=1}^\xN \abs{\inprod{\vx}{\vx_n}}^2 
    &&    \ifdochas{numsys}{\text{by \prefp{thm:C_abs}}}
  \\&=    \norm{\vx}^2
          + \sum_{n=1}^\xN \abs{\inprod{\vx}{\vx_n}}^2 
          - 2\sum_{n=1}^\xN \abs{\inprod{\vx}{\vx_n}}^2 
   %&&    \text{by definition of $\Re$ \ifdochas{normalg}{\prefpo{def:nalg_Re}}}
    &&    \text{because $\absn$ is real}
  \\&=    \norm{\vx}^2
          - \sum_{n=1}^\xN \abs{\inprod{\vx}{\vx_n}}^2 
\end{align*}
\end{proof}


%--------------------------------------
\begin{theorem}[\thm{Bessel's inequality}]
\label{thm:bessel_ineq}
\footnote{
  \citerppgc{giles2000}{54}{55}{0521653754}{3.13 Bessel's inequality}\\
  \citerpg{bollobas1999}{147}{0521655773}\\
  \citerpg{ab}{284}{0120502577}
  %\citerpp{pedersen2000}{38}{39}
  }
\index{inequality!Bessel's}
%--------------------------------------
Let $\setxn{\vx_n\in\setX}$ be a set of vectors in an \structe{inner product space} \xrefP{def:inprod} $\inprodspaceX$ and with
$\norm{\vx}\eqd\sqrt{\inprod{\vx}{\vx}}$\ifsxref{vsinprod}{def:norm=inprod}.
\thmbox{
  \text{$\setn{\vx_n}$ is \prope{orthonormal}}
  \quad\implies\quad
  \sum_{n=1}^\xN \abs{\inprod{\vx}{\vx_n}}^2 \le \norm{\vx}^2 
  \qquad \forall \vx\in\setX
  }
\end{theorem}
\begin{proof}
\begin{align*}
  0
    &\le \norm{\vx-\sum_{n=1}^\xN \inprod{\vx}{\vx_n} \vx_{n} }^2
    &&   \text{by definition of $\normn$ (\prefp{def:norm})}
  \\&=   \norm{\vx}^2 - \sum_{n=1}^\xN \abs{\inprod{\vx}{\vx_n}}^2
    &&   \text{by \thme{Bessel's Equality} \xref{thm:bessel_eq}}
\end{align*}
\end{proof}




The next theorem (the \hie{Best Approximation Theorem}) shows that
\begin{liste}
  \item the best sequence for representing
        a vector is the sequence of projections of the vector onto
        the sequence of basis functions
  \item the error of the projection is orthogonal to the projection.
\end{liste}

%--------------------------------------
\begin{theorem}[\thm{Best Approximation Theorem}]
\label{thm:bat}
\footnote{
  \citerpp{walter}{3}{4} \\
  \citerp{pedersen2000}{39} \\
  \citerpp{edwards1995}{94}{100} \\
  \citor{weyl1940}
  }
%--------------------------------------
Let $\setxn{\vx_n\in\setX}$ be a set of vectors in an \structe{inner product space} \xrefP{def:inprod} $\inprodspaceX$ and with
$\norm{\vx}\eqd\sqrt{\inprod{\vx}{\vx}}$\ifsxref{vsinprod}{def:norm=inprod}.
\thmbox{
  \brbr{\begin{array}{M}
  $\setn{\vx_n}$ is\\
  \prope{orthonormal}
  \end{array}}
  %\qquad\implies\qquad
  \implies
  \left\{\begin{array}{>{\scy}r>{\ds}l@{\quad}C}
    1. & \arg\min_{\seq{\alpha_n}{n=1}^\xN} \norm{ \vx - \sum_{n=1}^\xN \alpha_{n} \vx_{n} }
         = \mcoml{\seq{\inprod{\vx}{\vx_n}}{n=1}^\xN}{best $\alpha_n=\inprod{\vx}{\vx_n}$}
       & \forall\vx\in\setX
    \\[1ex]
    2. & \mcom{\ds\left(     \sum_{n=1}^\xN \inprod{\vx}{\vx_n}\vx_{n} \right)}{approximation}
         \orthog
         \mcom{\ds\left(\vx- \sum_{n=1}^\xN \inprod{\vx}{\vx_n}\vx_{n} \right)}{approximation error}
       & \forall\vx\in\setX
  \end{array}\right.
  }
\end{theorem}
\begin{proof}
\begin{enumerate}
\item Proof that $\seqn{\inprod{\vx}{\vx_n}}$ is the best sequence:
\begin{align*}
  &\norm{ \vx - \sum_{n=1}^\xN \alpha_{n} \vx_{n} }^2
  \\&= \norm{\vx}^2
      -2\Re{\inprod{\vx}{\sum_{n=1}^\xN \alpha_{n} \vx_n}}
      +\norm{\sum_{n=1}^\xN \alpha_{n} \vx_n}^2
  \\&= \norm{\vx}^2
      -2\Re\left(\sum_{n=1}^\xN \alpha_n^\ast \inprod{\vx}{\vx_n}\right)
      +\sum_{n=1}^\xN \norm{\alpha_{n} \vx_n}^2
      \quad\text{by \thme{Pythagorean Theorem} \xref{thm:pythag}}
  \\&= \norm{\vx}^2
      -2\Re\left(\sum_{n=1}^\xN \alpha_n^\ast \inprod{\vx}{\vx_n}\right)
      +\sum_{n=1}^\xN |\alpha_n|^2
      +\mcom{\Bigg[
       \sum_{n=1}^\xN \abs{\inprod{\vx}{\vx_n}}^2
      -\sum_{n=1}^\xN \abs{\inprod{\vx}{\vx_n}}^2
      \Bigg]}{0}
  \\&= \left[
       \norm{\vx}^2
      -\sum_{n=1}^\xN |\inprod{\vx}{\vx_n}|^2
       \right]
      +\sum_{n=1}^\xN \left[
        \abs{\inprod{\vx}{\vx_n}}^2
       -2\Reb{\alpha_n^\ast \inprod{\vx}{\vx_n}}
       + \abs{\alpha_n}^2
       \right]
  \\&= \left[
       \norm{\vx}^2
      -\sum_{n=1}^\xN |\inprod{\vx}{\vx_n}|^2
       \right]
      +\sum_{n=1}^\xN \left[
        |\inprod{\vx}{\vx_n}|^2
       -\alpha_n^\ast \inprod{\vx}{\vx_n}
       -\alpha_{n} \inprod{\vx}{\vx_n}^\ast
       + |\alpha_n|^2
       \right]
  \\&= \norm{\vx -\sum_{n=1}^\xN \inprod{\vx}{\vx_n}\vx_{n} }^2
      +\sum_{n=1}^\xN |\inprod{\vx}{\vx_n} - \alpha_{n} |^2
      \quad\text{by \thme{Bessel's Equality} \xref{thm:bessel_eq}}
  \\&\ge \norm{\vx -\sum_{n=1}^\xN \inprod{\vx}{\vx_n}\vx_{n} }^2
\end{align*}


\item Proof that the approximation and approximation error are orthogonal:
\begin{align*}
  &\inprod{\sum_{n=1}^\xN\inprod{\vx}{\vx_n}\vx_n}{\vx-\sum_{n=1}^\xN\inprod{\vx}{\vx_n}\vx_n}
  \\&= \inprod{\sum_{n=1}^\xN\inprod{\vx}{\vx_n}\vx_n}{\vx}
      -\inprod{\sum_{n=1}^\xN\inprod{\vx}{\vx_n}\vx_n}
              {\sum_{n=1}^\xN\inprod{\vx}{\vx_n}\vx_n}
  \\&= \sum_{n=1}^\xN\inprod{\vx}{\vx_n}^\ast \inprod{\vx}{\vx_n}
      -\sum_{n=1}^\xN \sum_{m=1}^\xN
       \inprod{\vx}{\vx_n} \inprod{\vx}{\vx_m}^\ast
       \inprod{\vx_n}{\vx_m}
  \\&= \sum_{n=1}^\xN |\inprod{\vx}{\vx_n}|^2
      -\sum_{n=1}^\xN \sum_{m=1}^\xN
       \inprod{\vx}{\vx_n} \inprod{\vx}{\vx_m}^\ast
       \kdelta_{nm}
  \\&= \sum_{n=1}^\xN |\inprod{\vx}{\vx_n}|^2
      -\sum_{n=1}^\xN |\inprod{\vx}{\vx_n}|^2
  \\&= 0
\end{align*}
\end{enumerate}
\end{proof}

%=======================================
%\subsubsection{Parseval sequences in Hilbert spaces}
%=======================================

%=======================================
\subsubsection{Orthonormal bases in Hilbert spaces}
\label{sec:hspace_bases}
%=======================================



%--------------------------------------
\begin{definition}
\label{def:basis_ortho}
\index{orthogonal}
\index{orthonormal}
\index{basis!orthonormal}
\index{basis!orthogonal}
%--------------------------------------
Let $\setxn{\vx_n\in\setX}$ be a set of vectors in an \structe{inner product space} \xrefP{def:inprod} $\spO\eqd\inprodspaceX$.
\defbox{\begin{array}{M}
  The set $\setn{\vx_n}$ is an \hid{orthogonal basis} for $\spO$ if $\setn{\vx_n}$ is \prope{orthogonal} and is \\\qquad
    a \structe{Schauder basis} for $\spO$.
  \\
  The set $\setn{\vx_n}$ is an \hid{orthonormal basis} for $\spO$ if $\setn{\vx_n}$ is \prope{orthonormal} and is \\\qquad
   a \structe{Schauder basis} for $\spO$.
\end{array}}
\end{definition}


%%--------------------------------------
%\begin{definition}
%\label{def:biorthogonal_basis}
%\label{def:biorthonormal_basis}
%\index{orthogonal}
%\index{orthonormal}
%\index{basis!biorthonormal}
%\index{basis!biorthogonal}
%%--------------------------------------
%Let $\spV\eqd\opair{\spX}{\inprodn}$ be an inner product space.
%\defbox{\begin{array}{l@{\qquad}ll@{\qquad}C@{\qquad}D}
%  \mc{5}{l}{\text{
%    A \hid{biorthogonal basis} for $\spV\eqd\opair{\spX}{\inprodn}$
%    is any pair of sets $(\setPhi,\setTh)$ that satisfies
%    }}
%  \\& 1. & \spV = \linspan \setTh
%         & 
%         & ($\setTh$ generates $\spV$)
%  \\& 2. & \inprod{\vx_n}{\vy_m} = 0 \text{ for } m\ne n
%         & \forall \vx_n\in\setPhi,\; \vy_m\in\setTh
%         & ($(\setPhi,\setTh)$ is biorthogonal)
%  \\
%  \mc{5}{l}{\text{
%    A \hid{biorthonormal basis} for $\spV\eqd\opair{\spX}{\inprodn}$
%    is any pair of sets $(\setPhi,\setTh)$ that satisfies
%    }}
%  \\& 1. & \spV = \linspan \setTh
%         & 
%         & ($\setTh$ generates $\spV$)
%  \\& 2. & \inprod{\vx_n}{\vy_m} = 
%           \left\{\begin{array}{ll}
%             0 & \text{ for } m\ne n \\
%             1 & \text{ for } m=n
%           \end{array}\right.
%         & \forall \vx_n\in\setPhi,\; \vy_m\in\setTh
%         & ($(\setPhi,\setTh)$ is biorthonormal)
%\end{array}}
%\end{definition}

%--------------------------------------
\begin{definition}
\footnote{
  \citerpgc{fabian2010}{27}{1441975144}{Theorem 1.55}\\
  \citerpg{young2001}{6}{0127729550}\\
  \citerpg{young1980}{6}{0127728503}
  }
\index{orthonormal expansion}
\index{Fourier expansion}
\label{def:hspace_fex}
\label{def:fcoef}
\label{def:fex}
%--------------------------------------
Let $\spH\eqd\inprodspaceX$ be a Hilbert space.
\defbox{\begin{array}{M}
  Suppose there exists a set $\ds\setxZp{\vx_n\in\setX}$ such that $\ds\vx\eqs\sum_{n=1}^\infty \inprod{\vx}{\vx_n} \vx_n$.\\
  Then the quantities $\inprod{\vx}{\vx_n}$ are called the \hid{Fourier coefficients} of $\vx$
  and the sum \\
  $\ds\sum_{n=1}^\infty \inprod{\vx}{\vx_n} \vx_n$ is called the \hid{Fourier expansion} of $\vx$ or the \hid{Fourier series} for $\vx$.
  \end{array}}
\end{definition}

%--------------------------------------
\begin{lemma}
\label{lem:Hfex}
%--------------------------------------
Let $\setxZp{\vx_n\in\setX}$ be a set of vectors in a \structe{Hilbert space} \ifxref{seq}{def:hilbert} $\spH\eqd\HspaceX$.
\lembox{
  \text{$\seqn{\vx_n}$ is a \structe{basis} for $\spH$}
  \qquad\implies\qquad
  \vx \eqs \ocom{\sum_{n=1}^\infty \mcom{\inprod{\vx}{\vx_n}}{Fourier coefficient}\vx_n}{Fourier expansion} \qquad\scy\forall\vx\in\setX   
  }
\end{lemma}
\begin{proof}
  \begin{align*}
    \inprod{\vx}{\vx_n}
      &= \inprod{\sum_{m\in\Z} \alpha_m \vx_m}{\vx_n}
      && \text{by left hypothesis}
    \\&= \sum_{m\in\Z} \alpha_m \inprod{\vx_m}{\vx_n}
    \\&= \sum_{m\in\Z} \alpha_m \kdelta_{n-m}
    \\&= \alpha_n
  \end{align*}
\end{proof}

%--------------------------------------
\begin{proposition}
\footnote{
  \citerppgc{han2007}{93}{94}{0821842129}{Proposition 3.11}
  }
\label{prop:parseval_reconstruction}
%--------------------------------------
Let $\setxZp{\vx_n\in\setX}$ be a set of vectors in a \structe{Hilbert space} \ifxref{seq}{def:hilbert}\\
$\spH\eqd\HspaceX$.
\propbox{\ds
  \mcom{\ds\norm{\vx}^2 \eqs\sum_{n=1}^\infty \abs{\inprod{\vx}{\vx_n}}^2}{\ide{Parseval frame}}
  \qquad\iff\qquad
  \mcom{\ds\vx\eqs\sum_{n=1}^\infty \inprod{\vx}{\vx_n}\vx_n}{\ide{Fourier expansion} \xref{def:fex}}
  \qquad\scy\forall\vx\in\setX
  }
\end{proposition}
\begin{proof}
\begin{enumerate}
  \item Proof that \ide{Parseval frame} $\impliedby$   \ide{reconstruction formula}
    \begin{align*}
      \norm{\vx}^2
        &\eqd \inprod{\vx}{\vx}
        && \text{by definition of $\normn$}
      \\&= \inprod{\sum_{n=1}^\infty \inprod{\vx}{\vx_n}\vx}{\vx_n}
        && \text{by reconstruction formula hypothesis}
      \\&\eqs\sum_{n=1}^\infty \inprod{\vx}{\vx_n}\inprod{\vx}{\vx_n}
        && \text{by property of $\inprodn$}
      \\&\eqs\sum_{n=1}^\infty \inprod{\vx}{\vx_n}\inprod{\vx}{\vx_n}^\ast
        && \text{by property of $\inprodn$}
      \\&\eqs\sum_{n=1}^\infty \abs{\inprod{\vx}{\vx_n}}^2
        && \text{by property of $\C$}
    \end{align*}

  \item Proof that \ide{Parseval frame} $\implies$ \ide{reconstruction formula}  % 2013 July 27 Saturday
    \begin{enumerate}
      \item Let $\seqxZp{\ve_n}$ be the \structe{standard othornormal basis} such that the $n$th element of $\ve_n$ is $1$ and all other elements are $0$.  \label{item:parseval_reconstruction_en}
      \item Let $\opM$ be an operator in $\spH$ such that $\ds\opM\vx\eqd\sum_{n=1}^\infty\inprod{\vx}{\vx_n}\ve_n$. \label{item:parseval_reconstruction_Mdef}
      \item lemma: $\opM$ is \prope{isometric}. Proof: \label{item:parseval_reconstruction_Miso}
        \begin{align*}
          \norm{\opM\vx}^2
            &= \norm{\sum_{n=1}^\infty \inprod{\vx}{\vx_n}\ve_n}^2
            && \text{by definition of $\opM$ \xref{item:parseval_reconstruction_Mdef}}
          \\&= \sum_{n=1}^\infty \norm{\inprod{\vx}{\vx_n}\ve_n}^2
            && \text{by \thme{Pythagorean Theorem} \xref{thm:pythag}}
          \\&= \sum_{n=1}^\infty \abs{\inprod{\vx}{\vx_n}}^2\norm{\ve_n}^2
            && \text{by \prope{homogeneous} property of norms \xref{def:norm}}
          \\&= \sum_{n=1}^\infty \abs{\inprod{\vx}{\vx_n}}^2
            && \text{by definition of \prope{orthonormal} \xref{def:orthog}}
          \\&= \norm{\vx}^2
            && \text{by Parseval frame hypothesis}
          \\&\implies \quad\text{$\opM$ is \prope{isometric}}
            && \text{by definition of \prope{isometric} \xref{def:op_isometric}}
        \end{align*}

      \item Let $\seqxZp{\vu_n}$ be an \structe{orthornormal basis} for $\spH$.  \label{item:parseval_reconstruction_un}
      \item Proof for \ide{reconstruction formula}:
        \begin{align*}
          \vx
            &= \sum_{n=1}^\infty \inprod{\vx}{\vu_n}\vu_n
            && \text{by \prefp{thm:hspace_fex}}
          \\&= \sum_{n=1}^\infty \inprod{\opM\vx}{\opM\vu_n}\vu_n
            && \text{by \prefp{item:parseval_reconstruction_Miso} and \prefp{thm:isometric_equiv}}
          \\&= \sum_{n=1}^\infty \inprod{\sum_{m=1}^\infty\inprod{\vx}{\vx_m}\ve_m}{\sum_{k=1}^\infty\inprod{\vu_n}{\vx_k}\ve_k}\vu_n
            && \text{by \prefp{item:parseval_reconstruction_Mdef}}
          \\&= \sum_{n=1}^\infty \sum_{m=1}^\infty\inprod{\vx}{\vx_m} \sum_{k=1}^\infty\inprod{\vu_n}{\vx_k}^\ast \inprod{\ve_m}{\ve_k}\vu_n
            && \text{by \prefp{def:inprod}}
          \\&= \sum_{n=1}^\infty \sum_{m=1}^\infty\inprod{\vx}{\vx_m} \inprod{\vu_n}{\vx_m}^\ast \vu_n
            && \text{by \prefp{item:parseval_reconstruction_en} and \prefp{def:orthog}}
          \\&= \sum_{n=1}^\infty \sum_{m=1}^\infty\inprod{\vx}{\vx_m} \inprod{\vx_m}{\vu_n} \vu_n
            && \text{by \prefp{def:inprod}}
          \\&= \sum_{m=1}^\infty \inprod{\vx}{\vx_m} \sum_{n=1}^\infty \inprod{\vx_m}{\vu_n} \vu_n
          \\&= \sum_{m=1}^\infty \inprod{\vx}{\vx_m} \vx_m
            && \text{by \prefp{item:parseval_reconstruction_un}}
        \end{align*}
    \end{enumerate}
\end{enumerate}
\end{proof}


%A set of orthonormal vectors may or may not span (form a basis for)
%a linear space $\spL$.
%Dedending on whether the set spans $\spL$,
%the norm of a vector is related to its projections on the members of the set
%by an inequality (set is not a basis) or by an equality (set is a basis):
%
%\begin{tabular}{llll}
%  Not a basis: &
%  $\norm{\vx}^2 \ge \sum_{n=1}^\xN |\inprod{\vx}{\vx_n}|^2$ &
%  \hie{Bessel's Inequality} \index{inequality!Bessel's} &
%  \scriptsize(\prefp{thm:bessel_ineq})
%\\
%  Is a basis: &
%  $\norm{\vx}^2 = \sum_{n=1}^\xN |\inprod{\vx}{\vx_n}|^2$ &
%  \hie{Plancherel's Equality}  &
%  \scriptsize(\prefp{thm:plancherel})
%\end{tabular}

When is a set of orthonormal vectors in a Hilbert space $\spH$ \prope{total}? 
\pref{thm:parsevalid} (next) offers some help.
%\hie{Parseval's Identity} (\prefp{thm:plancherel}) demonstrates that if a sequence $\seqxZp{\vx_n}$ is 
%\prope{orthonormal}, then $\norm{\vx}^2 \eqs\sum_{n=1}^\infty \left|\inprod{\vx}{\vx_n}\right|^2$.
%But the converse ($\seqxZp{\vx_n}$ \prope{orthonormal} $\impliedby$ $\norm{\vx}^2 \eqs\sum_{n=1}^\infty \left|\inprod{\vx}{\vx_n}\right|^2$)
%is \emph{not} true.
%That is, it is possible to have a set of vectors that satisfies Parseval's Identity, but yet are not \prope{orthonormal}.
%Such a set is illustrated in \pref{ex:mercedesA1} (next).%
%--------------------------------------
\begin{theorem}[\thmd{The Fourier Series Theorem}] %[\thmd{Parseval's Identity}]
\label{thm:fst}
\label{thm:plancherel}
\label{thm:parsevalid}
\label{thm:parseval}
\footnote{
  \citerppgc{bachman1966}{149}{155}{0486402517}{Theorem 9.12}\\
  \citerppgc{kubrusly2001}{360}{363}{0817641742}{Theorem 5.48}\\
  \citerppgc{ab}{298}{299}{0120502577}{Theorem 34.2}\\
  \citerpgc{christensen2003}{57}{0817642951}{Theorem 3.4.2}\\
  \citerppgc{berberian1961}{52}{53}{0821819127}{Theorem II\textsection8.3}\\
  \citerppgc{heil2011}{34}{35}{0817646868}{Theorem 1.50}
  %
  %\citerpg{bollobas1999}{147}{0521655773}\\
  %\citerp{mallat}{595}
  }
%--------------------------------------
Let $\setxZp{\vx_n\in\setX}$ be a set of vectors in a \structe{Hilbert space} \ifxref{seq}{def:hilbert} $\spH\eqd\HspaceX$ and let
$\norm{\vx}\eqd\sqrt{\inprod{\vx}{\vx}}$\ifsxref{vsinprod}{def:norm=inprod}.
\thmboxt{
  ${\scy(A)}$ $\setn{\vx_n}$ is \prope{orthonormal} in $\spH$
  $\qquad\implies\qquad$
  \\%\indentx
  $\brb{\begin{array}{cFrc>{\ds}lCD}
           & (1). & \cls{(\linspan\setn{\vx_n})} &=&    \spH                                                          &                        & ($\setn{\vx_n}$ is \prope{total} in $\spH$) 
    \\\iff & (2). & \inprod{\vx}{\vy}    &\eqs& \sum_{n=1}^\infty \inprod{\vx}{\vx_n}\inprod{\vy}{\vx_n}^\ast & \forall\vx,\vy\in\spX  & (\thme{Generalized Parseval's Identity})
    \\\iff & (3). & \norm{\vx}^2         &\eqs& \sum_{n=1}^\infty \abs{\inprod{\vx}{\vx_n}}^2                 & \forall\vx\in\setX     & (\thme{Parseval's Identity})
    \\\iff & (4). & \vx                  &\eqs& \sum_{n=1}^\infty \inprod{\vx}{\vx_n}\vx_n                    & \forall\vx\in\setX     & (\thme{Fourier series expansion})
  \end{array}}$
  }
\end{theorem}
\begin{proof}
\begin{enumerate}
  \item Proof that (1)$\implies$(2):
    \begin{align*}
      \inprod{\vx}{\vy}
        &\eqs \inprod{\sum_{n=1}^\infty \inprod{\vx}{\vx_n} \vx_n}
                     {\sum_{m=1}^\infty \inprod{\vy}{\vx_m} \vx_m}
        &&\text{by (A) and (1)}
      \\&\eqs \sum_{n=1}^\infty \inprod{\vx}{\vx_n} \inprod{ \vx_n}
             {\sum_{m=1}^\infty \inprod{\vy}{\vx_m} \vx_m}
        &&\text{by property of $\inprod{\cdot}{\cdot}$ \xref{def:inprod}}
      \\&\eqs \sum_{n=1}^\infty \inprod{\vx}{\vx_n} \sum_{m=1}^\infty \inprod{\vy}{\vx_m}^\ast
            \inprod{ \vx_n}{ \vx_m}
        &&\text{by property of $\inprod{\cdot}{\cdot}$ \xref{def:inprod}}
      \\&\eqs \sum_{n=1}^\infty \inprod{\vx}{\vx_n} \sum_{m=1}^\infty \inprod{\vy}{\vx_m}^\ast
            \kdelta_{mn}
        &&\text{by (A)}
      \\&\eqs \sum_{n=1}^\infty \inprod{\vx}{\vx_n} \inprod{\vy}{\vx_n}^\ast
        &&\text{by definition of $\kdelta_n$ \xref{def:kdelta}}
    \end{align*}

  \item Proof that (2)$\implies$(3):
    \begin{align*}
      \norm{\vx}^2
        &\eqd \inprod{\vx}{\vx}
        &&    \text{by definition of induced norm}
      \\&=    \sum_{n=1}^\infty \inprod{\vx}{\vx_n} \inprod{\vx}{\vx_n}^\ast
        &&    \text{by (2)}
      \\&=    \sum_{n=1}^\infty \abs{\inprod{\vx}{\vx_n}}^2
        &&    \ifdochas{numsys}{\text{by \prefp{thm:C_abs}}}
    \end{align*}
  
  \item Proof that (3)$\iff$(4) \emph{not} using (A): by \prefp{prop:parseval_reconstruction}

  \item Proof that (3)$\implies$(1) (proof by contradiction):
    \begin{enumerate}
      \item Suppose $\setn{\vx_n}$ is \prope{not total}.
      \item Then there must exist a vector $\vy$ in $\spH$ such that the set $\setB\eqd\setn{\vx_n}\setu\vy$ is \prope{orthonormal}.
      \item Then $\ds 1=\norm{y}^2 \neq \sum_{n=1}^\infty \abs{\inprod{\vy}{\vx_n}}^2 = 0$ .
      \item But this contradicts (3), and so $\setn{\vx_n}$ must be \prope{total} and (3)$\implies$(1).
    \end{enumerate}

  \item Extraneous proof that (3)$\implies$(4) (this proof is not really necessary here):
    \begin{align*}
      \norm{\vx - \sum_{n=1}^\infty \inprod{\vx}{\vx_n} \vx_{n} }^2
        &=  \norm{\vx}^2 -\sum_{n=1}^\infty |\inprod{\vx}{\vx_n}|^2
        &&  \text{by \thme{Bessel's Equality} \xref{thm:bessel_eq}}
      \\&=  0
        &&  \text{by (3)}
      \\&\implies\quad \vx\eqs\sum_{n=1}^\infty\inprod{\vx}{\vx_n}\vx_n 
        &&  \text{by definition of $\eqs$ \xref{def:eqs}}
    \end{align*} 

  \item Extraneous proof that (A)$\implies$(4) (this proof is not really necessary here)
    \begin{enumerate}
      \item The sequence
            $\sum_{n=1}^\xN \abs{\inprod{\vx}{\vx_n}}^2$
            is {\em monotonically increasing} in $n$.
      \item By Bessel's inequality (\prefpo{thm:bessel_ineq}),
            the sequence is upper bounded by $\norm{\vx}^2$:
           \[ \sum_{n=1}^\xN \abs{\inprod{\vx}{\vx_n}}^2 \le \norm{\vx}^2 \]
      \item Because this sequence is both monotonically increasing and bounded in $n$,
            it must equal its bound in the limit as $n$ approaches infinity:
           \begin{eqnarray} 
             \lim_{\xN\to\infty} \sum_{n=1}^\xN \abs{\inprod{\vx}{\vx_n}}^2
             = \norm{\vx}^2 
             \label{eq:plancherel1}
           \end{eqnarray}
      \item If we combine this result with \thme{Bessel's Equality} \xref{thm:bessel_eq} we have
        \begin{align*}
          \lim_{\xN\to\infty}
          \norm{ \vx - \sum_{n=1}^\xN \inprod{\vx}{\vx_n} \vx_{n} }^2
            &=  \norm{\vx}^2 -
                \lim_{\xN\to\infty}
                \sum_{n=1}^\xN |\inprod{\vx}{\vx_n}|^2
            &&  \text{by \thme{Bessel's Equality} \xref{thm:bessel_eq}}
          \\&=  \norm{\vx}^2 - \norm{\vx}^2
            &&  \text{by \prefp{eq:plancherel1}}
          \\&=  0
        \end{align*}
    \end{enumerate}
\end{enumerate}
\end{proof}

%--------------------------------------
\begin{proposition}[\thmd{Fourier expansion}]
\index{orthonormal expansion}
\label{thm:hspace_fex}
%--------------------------------------
Let $\setxZp{\vx_n\in\setX}$ be a set of vectors in a \structe{Hilbert space} \ifxref{seq}{def:hilbert} $\HspaceX$.
\propbox{
  \mcom{\begin{array}{M}
    $\setn{\vx_n}$ is an
    \structe{orthonormal basis} for $\spH$
  \end{array}}{(A)}
  \quad\implies\quad
  \Bigg\{
  \mcom{\vx \eqs \sum_{n=1}^\infty \alpha_{n} \vx_n}{(1)}
  \quad\iff\quad
  \mcom{\alpha_{n} = \inprod{\vx}{\vx_n}}{(2)}
  \Bigg\}
  }
\end{proposition}
\begin{proof}
\begin{enumerate}
  \item Proof that (1)$\implies$(2): by \prefp{lem:Hfex}
    %\begin{align*}
    %  \inprod{\vx}{\vx_n}
    %    &= \inprod{\sum_{m\in\Z} \alpha_m \vx_m}{\vx_n}
    %    && \text{by (A)}
    %  \\&= \sum_{m\in\Z} \alpha_m \inprod{\vx_m}{\vx_n}
    %  \\&= \sum_{m\in\Z} \alpha_m \kdelta_{n-m}
    %  \\&= \alpha_n
    %\end{align*}

  \item Proof that (1)$\impliedby$(2): %by \prefp{thm:fst}
    \begin{align*}
  \norm{ \vx - \sum_{n=1}^\infty \alpha_{n} \vx_{n\in\Z} }^2
    &=  \norm{ \vx - \sum_{n=1}^\infty \inprod{\vx}{\vx_n} \vx_{n\in\Z} }^2
    &&  \text{by right hypothesis}
  \\&=  \norm{\vx}^2 -\sum_{n=1}^\infty |\inprod{\vx}{\vx_n}|^2
    &&  \text{by Bessel's equality \prefpo{thm:bessel_eq}}
  \\&=  0
    &&  \text{by \thme{Parseval's Identity} \prefpo{thm:fst}}
  \\&\iffdef\quad \vx \eqs \sum_{n=1}^\infty \inprod{\vx}{\vx_n}\vx_n
    && \text{by definition of $\eqs$ \xref{def:eqs}}
    \end{align*}
\end{enumerate}
\end{proof}

%--------------------------------------
\begin{proposition}[\thmd{Riesz-Fischer Theorem}]
\footnote{
  \citerpg{young2001}{6}{0127729550}
  }
\label{prop:rft}
%--------------------------------------
Let $\setxZp{\vx_n\in\setX}$ be a set of vectors in a \structe{Hilbert space} \ifxref{seq}{def:hilbert} $\HspaceX$.
\propbox{
  \mcom{\begin{array}{M}
    $\setn{\vx_n}$ is an\\
    \structe{orthonormal basis}\\
    for $\spH$
  \end{array}}{(A)}
  \quad\implies\quad
  \Bigg\{
  \mcom{\sum_{n=1}^\infty \abs{\alpha_{n}}^2<\infty}{(1)}
  \quad\iff\quad
  \mcom{{\begin{array}{l}
    \scy\exists\vx\in\spH\st\\
    \alpha_{n} = \inprod{\vx}{\vx_n}
  \end{array}}}{(2)}
  \Bigg\}
  }
\end{proposition}
\begin{proof}
\begin{enumerate}
  \item Proof that (1)$\implies$(2): 
    \begin{enumerate}
      \item If (1) is true, then let $\ds\vx\eqd\sum_{n\in\Zp}\alpha_n\vx_n$.
      \item Then 
        \begin{align*}
            \inprod{\vx}{\vx_n} &= \inprod{\sum_{m\in\Zp}\alpha_m\vx_m}{\vx_n}
                                && \text{by definition of $\vx$}
                              \\&=    \sum_{m\in\Zp}\alpha_m \inprod{\vx_m}{\vx_n}
                                && \text{by \prope{homogeneous} property of $\inprodn$ \xref{def:inprod}}
                              \\&=    \sum_{m\in\Zp}\alpha_m \kdelta_{mn}
                                &&  \text{by (A)}
                              \\&=    \sum_{m\in\Zp}\alpha_n
                                &&  \text{by definition of $\kdelta$ \xref{def:kdelta}}
        \end{align*}
    \end{enumerate}

  \item Proof that (1)$\impliedby$(2): 
    \begin{align*}
      \sum_{n\in\Zp}\abs{\alpha_n}^2
        &= \sum_{n\in\Zp}\abs{\inprod{\vx}{\vx_n}}^2
        && \text{by (2)}
      \\&\le \norm{\vx}^2
        && \text{by \thme{Bessel's Inequality} \xref{thm:bessel_ineq}}
      \\&\le \infty
    \end{align*}
\end{enumerate}
\end{proof}

%--------------------------------------
\begin{theorem}
\footnote{
  \citerpg{young2001}{6}{0127729550}
  }
\label{thm:sepHiso}
%--------------------------------------
%Let $\spH$ be a \structe{Hilbert space}\ifsxref{seq}{def:hilbert}.
\thmboxt{
  All \structe{separable Hilbert spaces} are \prope{isomorphic}. That is,
  \\
  $\brb{\begin{array}{MD}
    $\spX$ is a separable\\Hilbert space & and \\
    $\spY$ is a separable\\Hilbert space & 
  \end{array}}
  \implies
  \brb{\begin{array}{FlclCD}
    \mc{6}{M}{there is a \prope{bijective} operator $\opM\in\clOxy$ such that}\\
      %\quad(1). $\opM$ is \prope{bijective} operator in $\clOxy$
      \quad(1). & \vy               &=& \opM\vx                   & \forall \vx\in\spX,\, \vy\in\spY & and\\
      \quad(2). & \norm{\opM\vx}    &=& \norm{\vx}                & \forall \vx\in\spX               & and\\
      \quad(3). & \inprod{\opM\vx}{\opM\vy} &=& \inprod{\vx}{\vy} & \forall \vx\in\spX,\, \vy\in\spY & 
  \end{array}}$
  }
\end{theorem}
\begin{proof}
\begin{enumerate}
  \item Let $\spX\eqd\HspaceX$ be a \structe{separable Hilbert space} with \structe{orthonormal basis} $\setxZp{\vx_n}$.\label{item:sepHiso_X}
      \\Let $\spY\eqd\HspaceY$ be a \structe{separable Hilbert space} with \structe{orthonormal basis} $\setxZp{\vy_n}$.\label{item:sepHiso_Y}

  \item Proof that there exists \structe{bijective} operator $\opM$ and its inverse $\opMi$ between $\setn{\vx_n}$ and $\setn{\vy_n}$:
    \begin{enumerate}
      \item Let $\opM$ be defined such that $\vy_n\eqd\opM\vx_n$.\label{item:sepHiso_M}
      \item Thus $\opM$ is a \structe{bijection} between $\setn{\vx_n}$ and $\setn{\vy_n}$.
      \item Because $\opM$ is a \prope{bijection} between $\setn{\vx_n}$ and $\setn{\vy_n}$, 
            $\opM$ has an inverse operator $\opMi$ between $\setn{\vx_n}$ and $\setn{\vy_n}$ 
            such that $\vx_n=\opMi\vy_n$.\label{item:sepHiso_Mi}
    \end{enumerate}

  \item Proof that $\opM$ and $\opMi$ are \structe{bijective} operators between $\spX$ and $\spY$:
    \begin{enumerate}
      \item Proof that $\opM$ maps  $\spX$ \prope{into} $\spY$:
        \begin{align*}
          \vx\in\spX
            &\iff \vx\eqs\sum_{n\in\Zp}\inprod{\vx}{\vx_n}\vx_n
            && \text{by \thme{Fourier expansion} \xref{thm:fst}}
          \\&\implies \exists\vy\in\spY \st \inprod{\vy}{\vy_n}=\inprod{\vx}{\vx_n}
            && \text{by \thme{Riesz-Fischer Theorem} \xref{prop:rft}}
          \\&\implies 
          \\\vy
            &= \sum_{n\in\Zp}\inprod{\vy}{\vy_n}\vy_n
            && \text{by \thme{Fourier expansion} \xref{thm:fst}}
          \\&= \sum_{n\in\Zp}\inprod{\vx}{\vx_n}\vy_n
            && \text{by \thme{Riesz-Fischer Theorem} \xref{prop:rft}}
          \\&= \sum_{n\in\Zp}\inprod{\vx}{\vx_n}\opM\vx_n
            && \text{by definition of $\opM$ \xref{item:sepHiso_M}}
          \\&= \opM\sum_{n\in\Zp}\inprod{\vx}{\vx_n}\vx_n
            && \text{by prop. of \structe{linear operators} \xref{thm:L_prop}}
          \\&= \opM\vx
            && \text{by definition of $\vx$}
        \end{align*}

      \item Proof that $\opMi$ maps $\spY$ \prope{into} $\spX$:
        \begin{align*}
          \vy\in\spY
            &\iff \vy\eqs\sum_{n\in\Zp}\inprod{\vy}{\vy_n}\vy_n
            && \text{by \thme{Fourier expansion} \xref{thm:fst}}
          \\&\implies \exists\vx\in\spX \st \inprod{\vx}{\vx_n}=\inprod{\vy}{\vy_n}
            && \text{by \thme{Riesz-Fischer Theorem} \xref{prop:rft}}
          \\&\implies 
          \\\vx
            &= \sum_{n\in\Zp}\inprod{\vx}{\vx_n}\vx_n
            && \text{by \thme{Fourier expansion} \xref{thm:fst}}
          \\&= \sum_{n\in\Zp}\inprod{\vy}{\vy_n}\vx_n
            && \text{by \thme{Riesz-Fischer Theorem} \xref{prop:rft}}
          \\&= \sum_{n\in\Zp}\inprod{\vy}{\vy_n}\opMi\vy_n
            && \text{by definition of $\opMi$ \xref{item:sepHiso_Mi}}
          \\&= \opMi\sum_{n\in\Zp}\inprod{\vy}{\vy_n}\vy_n
            && \text{by property of \structe{linear operators} \xref{thm:L_prop}}
          \\&= \opMi\vy
            && \text{by definition of $\vy$}
        \end{align*}
    \end{enumerate}

  \item Proof for (2):
    \begin{align*}
      \norm{\opM\vx}^2
        &= \norm{\opM\sum_{n\in\Zp}\inprod{\vx}{\vx_n}\vx_n}^2
        && \text{by \thme{Fourier expansion} \xref{thm:fst}}
      \\&= \norm{\sum_{n\in\Zp}\inprod{\vx}{\vx_n}\opM\vx_n}^2
        && \text{by property of \structe{linear operators} \xref{thm:L_prop}}
      \\&= \norm{\sum_{n\in\Zp}\inprod{\vx}{\vx_n}\vy_n}^2
        && \text{by definition of $\opM$ \xref{item:sepHiso_M}}
      \\&= \sum_{n\in\Zp}\abs{\inprod{\vx}{\vx_n}}^2
        && \text{by \thme{Parseval's Identity} \xref{prop:rft}}
      \\&= \norm{\sum_{n\in\Zp}\inprod{\vx}{\vx_n}\vx_n}^2
        && \text{by \thme{Parseval's Identity} \xref{prop:rft}}
      \\&= \norm{\vx}^2
        && \text{by \thme{Fourier expansion} \xref{thm:fst}}
    \end{align*}
 
  \item Proof for (3): by (2) and \prefp{thm:op_inprodiso}
\end{enumerate}
\end{proof}

%--------------------------------------
\begin{theorem}
\footnote{
  %\citerpgc{heil2011}{21}{0817646868}{Theorem 1.27}\\
  \citerpgc{bachman2000}{112}{0387988998}{3.4.8}\\
  \citerpgc{berberian1961}{53}{0821819127}{Theorem II\textsection8.3}
  }
\label{thm:schauder<==>separable}
%--------------------------------------
Let $\spH$ be a \structe{Hilbert space}\ifsxref{seq}{def:hilbert}.
\thmbox{\begin{array}{M}
  $\spH$ has a \structe{Schauder basis}
  $\qquad\iff\qquad$
  $\spH$ is \prope{separable}
  %If $\opair{\spO}{\normn}$ is an \prope{inner product} space then 
  %   \\\indentx $\exists$ an \prope{orthonormal} sequence $\seqxZp{\vx_n}$ that is \prope{complete} in $\spO$ $\iff$ $\setX$ is \prope{separable}.
  \end{array}}
\end{theorem}

%--------------------------------------
\begin{theorem}
\footnote{
  \citerpgc{kubrusly2001}{357}{0817641742}{Proposition 5.43}
  }
\label{thm:ortho<==>separable}
%--------------------------------------
Let $\spH$ be a \structe{Hilbert space}.
\thmbox{\begin{array}{M}
  $\spH$ has an \structe{orthonormal basis}
  $\qquad\iff\qquad$
  $\spH$ is \prope{separable}
  \end{array}}
\end{theorem}

%=======================================
\subsubsection{Riesz bases in Hilbert space}
%=======================================
%--------------------------------------
\begin{definition}
\footnote{
  \citerpgc{young2001}{27}{0127729550}{Definition 1.8.2}\\
  \citerpgc{christensen2003}{63}{0817642951}{Definition 3.6.1}\\
  \citerpgc{heil2011}{196}{0817646868}{Definition 7.9}
  }
\label{def:basis_riesz}
\index{basis!Riesz}
%--------------------------------------
Let $\setxZp{\vx_n\in\setX}$ be a set of vectors in a \structe{separable Hilbert space}\ifsxref{seq}{def:hilbert}
$\spH\eqd\HspaceX$.
\defboxt{
  $\setn{\vx_n}$ is a \hid{Riesz basis} for $\spH$ if
  $\setn{\vx_n}$ is \prope{equivalent} \xref{def:equivalent}\\
  to some \structe{orthonormal basis} \xref{def:basis_ortho} in $\spH$. %\footnotemark
  }
\end{definition}
%\footnotetext{\prope{equivalent}: \xref{def:equivalent}; \structe{orthonormal basis}: \xref{def:basis_ortho}}

%--------------------------------------
\begin{definition}
\footnote{
  %\citerpgc{young2001}{27}{0127729550}{Theorem 1.8.9}\\
  \citerppgc{christensen2003}{66}{68}{0817642951}{page 68 and (3.24) on page 66}\\
  %\citerppgc{heil2011}{197}{198}{0817646868}{Theorem 7.13}\\
  \citerpgc{wojtaszczyk1997}{20}{0521578949}{Definition 2.6}
  }
\label{def:rieszseq}
%--------------------------------------
Let $\seqxZp{\vx_n\in\setX}$ be a sequence of vectors in a \structe{separable Hilbert space}\ifsxref{seq}{def:hilbert}\\ 
$\spH\eqd\HspaceX$.
\defboxt{
  The sequence $\seqn{\vx_n}$ is a \hid{Riesz sequence} for $\spH$ if
    \\\qquad
    $\ds\exists A,B\in\Rp \st\qquad 
    A\brp{\sum_{n=1}^\infty \abs{\alpha_n}^2}
    \le 
    \norm{\sum_{n=1}^\infty \alpha_n\vx_n}^2
    \le
    B\brp{\sum_{n=1}^\infty \abs{\alpha_n}^2}
    \qquad\scy\forall\seqn{\alpha_n}\in\spllF.$
 %  \\
 %A \structe{Riesz sequence} $\setn{\vx_n}$ for $\spH$ is a \hid{Riesz basis} for $\spH$ if $\linspan\setn{\vx_n}=\spH$.
%  The quantities $A$ and $B$ are \hid{frame bounds}.
%  \\The quantity $A'$ is the \hid{optimal lower frame bound} if 
%  \\\qquad $A'=\sup\set{A\in\Rp}{\text{$A$ is a lower frame bound}}$.
%  \\The quantity $B'$ is the \hid{optimal upper frame bound} if 
%  \\\qquad $B'=\inf\set{B\in\Rp}{\text{$B$ is an upper frame bound}}$.
%  \\A frame is a \hid{tight frame} if $A=B$.
%  \\A frame is a \hid{normalized tight frame} (or a \hid{Parseval frame}) if $A=B=1$.
%  \\A frame $\setxZp{\vx_n}$ is an \hid{exact frame} if for some $m\in\Z$, $\setxZp{\vx_n}\setd\setn{\vx_m}$ is \emph{not} a frame.
  }
\end{definition}


%%--------------------------------------
%\begin{theorem}
%\footnote{
%  \citerpgc{young2001}{27}{0127729550}{Theorem 1.8.9}\\
%  }
%\label{thm:rbasis_rseq}
%%--------------------------------------
%Let $\spH\eqd\HspaceX$ be a \structe{separable Hilbert space}\ifsxref{seq}{def:hilbert}.
%\thmbox{
%  \brb{\begin{array}{M}$\seqxZ{\vx_n}$ is a \structe{Riesz basis}\\for $\spH$\end{array}}
%  \quad\iff\quad
%  \brb{\begin{array}{FMD}
%    (1). & $\seqn{\vx_n}$ is a \structe{Riesz sequence} in $\spH$ & and\\
%    (2). & $\linspan\seqn{\vx_n}=\spH$
%  \end{array}}
%  }
%\end{theorem}

%%--------------------------------------
%\begin{theorem}
%%\footnote{
%%  \citerpgc{wojtaszczyk1997}{20}{0521578949}{Lemma 2.7(a)}
%%  }
%\label{thm:rbasis_eq}
%%--------------------------------------
%Let $\spH\eqd\HspaceX$ be a \structe{separable Hilbert space}\ifsxref{seq}{def:hilbert}.
%\thmbox{
%  \brb{\begin{array}{FMD}
%    (A). & $\seqxZ{\vx_n}$ is a \structe{Riesz basis} for $\spH$ & and\\
%    (B). & $\seqxZ{\vy_n}$ is a \structe{Riesz basis} for $\spH$
%  \end{array}}
%  \quad\implies\quad
%  \brb{\begin{array}{M}
%    $\seqn{\vx_n}$ and $\seqn{\vy_n}$ are \prope{isomorphic}  %with respect to each other & or equivalently\\
%    %(2). & There exists an \prope{isometric} operator $\opM$ such that $\vy_n=\opM\vx_n$ and $\vx_n=\opMi\vy_n$ $\forall n\in\Z$.
%  \end{array}}
%  }
%\end{theorem}
%\begin{proof}
%\begin{enumerate}
%  \item Let $\ve_n$ be the \structe{unit vector} in $\spH$ such that the $n$th element of $\ve_n$ is $1$ and all other elements are $0$.
%  \item Let $\opM$ be an operator on $\spH$ such that $\opM\ve_n=\vx_n$.
%  \item Note that $\opM$ is \prope{isometric}, and as such $\norm{\opM\vx}=\norm{\vx}\quad{\scy\forall\vx\in\spH}$.
%  \item Let $\vy_n\eqd\brp{\opMi}^\ast$. \label{prop:rbasis_xy_Mia}
%  \item Then,
%%        \begin{align*}
%%          \inprod{\vy_n}{\vx_m}
%%            &= \inprod{\brp{\opMi}^\ast\ve_n}{\opM\ve_m}
%%          \\&= \inprod{\ve_n}{\opMi\opM\ve_m}
%%          \\&= \inprod{\ve_n}{\ve_m}
%%          \\&= \kdelta_{nm}
%%          \\&\implies \text{$\setn{\vx_n}$ and $\setn{\vy_n}$ are \prope{biorthogonal}}
%%            && \text{by \prefp{def:orthog}}
%%        \end{align*}
%%
%%      \item Proof 
%    \begin{align*}
%      \norm{\sum_{n\in\Z}\alpha_n \vy_n}
%        &= \norm{\sum_{n\in\Z}\alpha_n \brp{\opMi}^\ast\ve_n}
%        && \text{by definition of $\vy_n$ (\prefp{prop:rbasis_xy_Mia})}
%      \\&= \norm{\brp{\opMi}^\ast\sum_{n\in\Z}\alpha_n \ve_n}
%        && \text{by property of \structe{linear operator}s}
%      \\&= \norm{\sum_{n\in\Z}\alpha_n \ve_n}
%        && \text{because $\brp{\opMi}^\ast$ is \prope{isometric}}
%      \\&= \norm{\opM\sum_{n\in\Z}\alpha_n \ve_n}
%        && \text{because $\opM$ is \prope{isometric}}
%      \\&= \norm{\sum_{n\in\Z}\alpha_n \opM\ve_n}
%        && \text{by property of \structe{linear operator}s}
%      \\&= \norm{\sum_{n\in\Z}\alpha_n \vx_n}
%        && \text{by definition of $\opM$}
%      \\&\implies \text{$\setn{\vy_n}$ is a \structe{Riesz basis}}
%        &&  \text{by left hypothesis}
%    \end{align*}
%\end{enumerate}
%\end{proof}

%--------------------------------------
\begin{lemma}
\footnote{
  \citerppgc{christensen2003}{65}{66}{0817642951}{Lemma 3.6.5}
  }
\label{lem:xyextension}
%--------------------------------------
Let $\setxZp{\vx_n}$ be a sequence in a \structe{Hilbert space} $\spX\eqd\HspaceX$.
Let $\setxZp{\vy_n}$ be a sequence in a \structe{Hilbert space} $\spY\eqd\HspaceY$\ifsxref{seq}{def:hilbert}. 
Let 
\lemboxt{
  $\brb{\begin{array}{FMCD}
    (i).   & $\setn{\vx_n}$ is \prope{total} in $\spX$                                                                        &                               & and \\
    (ii).  & {\scs There exists $A>0$ such that} $\ds A\sum_{n\in\setC} \abs{a_n}^2\le\norm{\sum_{n\in\setC} a_n\vx_n}^2$    & \text{for finite $\setC$} & and \\
    (iii). & {\scs There exists $B>0$ such that} $\ds\norm{\sum_{n=1}^\infty b_n\vy_n}^2 \le B\sum_{n=1}^\infty\abs{b_n}^2$   & \forall \seqxZp{b_n}\in\spllF & 
  \end{array}}\qquad\implies$
  \\
  $\brb{\begin{array}{FMD}
    (1).  & $\opRo$ is a linear bounded operator that maps from $\linspan\setn{\vx_n}$ to $\linspan\setn{\vy_n}$    &  \\
          & \indentx where $\ds\opRo\sum_{n\in\setC}c_n\vx_n \eqd \sum_{n\in\setC}c_n\vy_n$, for some sequence $\seqn{c_n}$ and finite set $\setC$ & and \\
    (2).  & $\opR$ has a unique extension to a bounded operator $\opR$ that maps from $\spX$ to $\spY$ & and \\
    (3).  & $\normop{\opRo}\le\frac{B}{A}$ & and \\
    (4).  & $\normop{\opR}\le\frac{B}{A}$
  \end{array}}$
}
\end{lemma}
%\begin{proof}
%\begin{enumerate}
%  \item Proof that $\opRo$ is \prope{linear}: this follows directly from its definition.
%  \item Proof that $\opRo$ is \prope{bounded}: 
%    \begin{align*}
%      \normop{\opRo}
%        &\eqd \sup_{\vx\in\linspan\setn{\vx_n}}\set{\norm{\opRo\vx}}{\norm{\vx} \le 1}
%        &&    \text{by definition of $\normopn$ \xref{def:normop}}
%      \\&=    \sup_{\vx\in\linspan\setn{\vx_n}}\set{\norm{\opRo\sum_{n\in\setC}c_n\vx_n}}{\norm{\vx} \le 1}
%      \\&=    \sup_{\vx\in\linspan\setn{\vx_n}}\set{\norm{\sum_{n\in\setC}c_n\vy_n}}{\norm{\vx} \le 1}
%        &&    \text{by definition of $\opRo$ in (1)}
%      \\&\le  \sup_{\vx\in\linspan\setn{\vx_n}}\set{B\sum_{n\in\setC}\abs{c_n}^2}{\norm{\vx} \le 1}
%        &&    \text{by (iii)}
%      \\&\le  \sup_{\vx\in\linspan\setn{\vx_n}}\set{\frac{B}{A}\norm{\sum_{n\in\setC}\abs{c_n}\vx_n}^2}{\norm{\vx} \le 1}
%        &&    \text{by (ii)}
%    \end{align*}
%\end{enumerate}
%\end{proof}

%--------------------------------------
\begin{theorem}
\footnote{
  \citerpgc{young2001}{27}{0127729550}{Theorem 1.8.9}\\
  \citerpgc{christensen2003}{66}{0817642951}{Theorem 3.6.6}\\
  \citerppgc{heil2011}{197}{198}{0817646868}{Theorem 7.13}\\
  \citerppgc{christensen2008}{61}{62}{0817646779}{Theorem 3.3.7}
  }
\index{basis!Riesz}
\label{thm:rieszAB}
%--------------------------------------
Let $\setxZp{\vx_n\in\setX}$ be a set of vectors in a \structe{separable Hilbert space}\ifsxref{seq}{def:hilbert}\\ 
$\spH\eqd\HspaceX$.
\thmbox{%\begin{array}{>{\ds}l}
  \brb{\begin{array}{M}
    $\setn{\vx_n}$ is a \structe{Riesz basis}\\
    for $\spH$
  \end{array}}
  \quad\iff\quad
  \brb{\begin{array}{FMD}
    (1).  & $\setn{\vx_n}$ is \prope{total} in $\spH$
          & and
       \\
    %B. & $\setn{\vx_n}$ is a \structe{Riesz sequence} in $\spH$.
    (2).  & $\scy\exists A,B\in\Rp \st \quad\forall\seqn{\alpha_n}\in\spllF$,
        \\&  $\ds A{\sum_{n=1}^\infty \abs{\alpha_n}^2} \le \norm{\sum_{n=1}^\infty \alpha_n\vx_n}^2 \le B{\sum_{n=1}^\infty \abs{\alpha_n}^2}$ 
  \end{array}}
  %\\\qquad\iff\qquad
  %\brb{\begin{array}{FMD}
  %  (3A). & $\setn{\vx_n}$ is \prope{total} in $\spH$
  %        & and
  %        \\
  %  (3B). & $\ds \sum_{n=1}^\infty \abs{\inprod{\vx}{\vx_n}}^2 < \infty\qquad\scy\forall\vx\in\spH$
  %        & and 
  %        \\
  %  (3C). & $\exists \setn{\vy_n}$ such that $\setn{\vy_n}$ is \prope{total} in $\spH$, 
  %      \\& \prope{biorthogonal} with respect to to $\setn{\vx_n}$ and
  %      \\& $\ds \sum_{n=1}^\infty \abs{\inprod{\vx}{\vy_n}}^2 < \infty\qquad\scy\forall\vx\in\spH$.
  %\end{array}}
}
\end{theorem}
\begin{proof}
\begin{enumerate}
  \item Proof for ($\implies$) case:
    \begin{enumerate}
      \item Proof that \structe{Riesz basis} hypothesis $\implies$ (1): all bases for $\spH$ are \prope{total} in $\spH$. 
      \item Proof that \structe{Riesz basis} hypothesis $\implies$ (2):
        \begin{enumerate}
          \item Let $\seqxZp{\vu_n}$ be an \structe{orthonormal basis} for $\spH$. \label{item:rieszAB_un}
          \item Let $\opR$ be a \prope{bounded bijective} operator such that $\vx_n=\opR\vu_n$. \label{item:rieszAB_R}
          \item Proof for upper bound $B$:
            \begin{align*}
              \norm{\sum_{n=1}^\infty \alpha_n\vx_n}^2 
                &= \norm{\sum_{n=1}^\infty \alpha_n\opR\vu_n}^2 
                && \text{by definition of $\opR$ \xref{item:rieszAB_R}}
              \\&= \norm{\opR\sum_{n=1}^\infty \alpha_n\vu_n}^2 
                && \text{by \prefp{thm:L_prop}}
              \\&\le \normop{\opR}^2 \norm{\sum_{n=1}^\infty \alpha_n\vu_n}^2 
                && \text{by \prefp{thm:LxLx}}
              \\&= \normop{\opR}^2 \sum_{n=1}^\infty \norm{\alpha_n\vu_n}^2 
                && \text{by \thme{Pythagorean Theorem} \xref{thm:pythag}}
              \\&= \normop{\opR}^2 \sum_{n=1}^\infty \abs{\alpha}^2 \norm{\vu_n}^2 
                && \text{by \prope{homogeneous} property of norms \xref{def:norm}}
              \\&= \mcom{\normop{\opR}^2}{$B$} \sum_{n=1}^\infty \abs{\alpha}^2 
                && \text{by definition of \prope{orthonormality} \xref{def:orthog}}
            \end{align*}
    
          \item Proof for lower bound $A$:
            \begin{align*}
              \norm{\sum_{n=1}^\infty \alpha_n\vx_n}^2 
                &=   \frac{\normop{\opRi}^2}{\normop{\opRi}^2}\,\norm{\sum_{n=1}^\infty \alpha_n\vx_n}^2 
               %&&   \text{because $\normop{\opRi}>0$ for $\opRi\neq\opZero$ \xref{prop:op_norm}}
                &&   \text{because $\normop{\opRi}>0$ \xref{prop:op_norm}}
              \\&\ge \frac{1}{\normop{\opRi}^2}\,\norm{\opRi\sum_{n=1}^\infty \alpha_n\vx_n}^2 
                &&   \text{by \prefp{thm:LxLx}}
              \\&=   \frac{1}{\normop{\opRi}^2}\,\norm{\opRi\sum_{n=1}^\infty \alpha_n\opR\vu_n}^2 
                &&   \text{by definition of $\opR$ \xref{item:rieszAB_R}}
              \\&=   \frac{1}{\normop{\opRi}^2}\,\norm{\opRi\opR\sum_{n=1}^\infty \alpha_n\vu_n}^2 
                &&   \text{by \prefp{thm:L_prop}}
              \\&=   \frac{1}{\normop{\opRi}^2}\,\norm{\sum_{n=1}^\infty \alpha_n\vu_n}^2 
                &&   \text{by definition of inverse operator}
              \\&=   \frac{1}{\normop{\opRi}^2}\,\sum_{n=1}^\infty \norm{\alpha_n\vu_n}^2 
                &&   \text{by \thme{Pythagorean Theorem} \xref{thm:pythag}}
              \\&=   \frac{1}{\normop{\opRi}^2}\,\sum_{n=1}^\infty \abs{\alpha_n}^2 \norm{\vu_n}^2
                &&   \text{by $\normn$ \prope{homogeneous} prop. \xref{def:norm}}
              \\&=   \mcom{\frac{1}{\normop{\opRi}^2}}{$A$}\,\sum_{n=1}^\infty \abs{\alpha_n}^2 
                &&   \text{by def. of \prope{orthonormality} \xref{def:orthog}}
            \end{align*}
        \end{enumerate}
    \end{enumerate}
  \item Proof for ($\implies$) case:
    \begin{enumerate}
      \item Let $\setxZp{\vu_n}$ be an \structe{orthonormal basis} for $\spH$.
      \item Using (2) and \prefpp{lem:xyextension}, construct an bounded extension operator $\opR$ such that $\opR\vu_n=\vx_n$ for all $n\in\Zp$.
      \item Using (2) and \prefpp{lem:xyextension}, construct an bounded extension operator $\opS$ such that $\opS\vx_n=\vu_n$ for all $n\in\Zp$.
      \item Then, $\opR\opV\vx=\opV\opR\vx$ $\implies$ $\opV=\opRi$, and so $\opR$ is a bounded invertible operator
      \item and $\setn{\vx_n}$ is a \structe{Riesz sequence}.
    \end{enumerate}
\end{enumerate}
\end{proof}


%--------------------------------------
\begin{theorem}
\footnote{
  \citerpgc{wojtaszczyk1997}{20}{0521578949}{Lemma 2.7(a)}
  }
\label{thm:rbasis_xy}
%--------------------------------------
Let $\spH\eqd\HspaceX$ be a \structe{separable Hilbert space}\ifsxref{seq}{def:hilbert}.
\thmbox{
  \brb{\begin{array}{M}
    $\seqxZ{\vx_n\in\spH}$ is a\\
    \structe{Riesz basis} for $\spH$\\
    %$\ds A{\sum_{n=1}^\infty \abs{a_n}^2} \le \norm{\sum_{n=1}^\infty a_n\vx_n}^2 \le B{\sum_{n=1}^\infty \abs{a_n}^2}$\\
  \end{array}}
  \implies
  \brbl{\begin{array}{cFMD}
    \mc{4}{M}{There exists $\seqxZ{\vy_n\in\spH}$ such that}\\
    & (1). & $\seqn{\vx_n}$ and $\seqn{\vy_n}$ are \prope{biorthogonal}  & and\\
    & (2). & $\seqn{\vy_n}$ is also a \structe{Riesz basis} for $\spH$   & and \\
    & (3). & ${\scy\exists B>A>0 \st}$ \\
    &      & \mc{2}{l}{\ds A{\sum_{n=1}^\infty \abs{a_n}^2} \le \norm{\sum_{n=1}^\infty a_n\vx_n}^2 = \norm{\sum_{n=1}^\infty a_n\vy_n}^2 \le B{\sum_{n=1}^\infty \abs{a_n}^2}}\\
    &      & $\scy\forall\seqxZp{a_n}\in\spllF$
  \end{array}}
  }
\end{theorem}
\begin{proof}
\begin{enumerate}
  \item Proof for (1):
    \begin{enumerate}
      \item Let $\ve_n$ be the \structe{unit vector} in $\spH$ such that the $n$th element of $\ve_n$ is $1$ and all other elements are $0$.
      \item Let $\opM$ be an operator on $\spH$ such that $\opM\ve_n=\vx_n$.
      \item Note that $\opM$ is \prope{isometric}, and as such $\norm{\opM\vx}=\norm{\vx}\quad{\scy\forall\vx\in\spH}$.
      \item Let $\vy_n\eqd\brp{\opMi}^\ast$. \label{prop:rbasis_xy_Mia}
      \item Then,
        \begin{align*}
          \inprod{\vy_n}{\vx_m}
            &= \inprod{\brp{\opMi}^\ast\ve_n}{\opM\ve_m}
          \\&= \inprod{\ve_n}{\opMi\opM\ve_m}
          \\&= \inprod{\ve_n}{\ve_m}
          \\&= \kdelta_{nm}
          \\&\implies \text{$\setn{\vx_n}$ and $\setn{\vy_n}$ are \prope{biorthogonal}}
            && \text{by \prefp{def:orthog}}
        \end{align*}
    \end{enumerate}

  \item Proof for (3):
    \begin{align*}
      \norm{\sum_{n\in\Z}\alpha_n \vy_n}
        &= \norm{\sum_{n\in\Z}\alpha_n \brp{\opMi}^\ast\ve_n}
        && \text{by definition of $\vy_n$ (\prefp{prop:rbasis_xy_Mia})}
      \\&= \norm{\brp{\opMi}^\ast\sum_{n\in\Z}\alpha_n \ve_n}
        && \text{by property of \structe{linear operator}s}
      \\&= \norm{\sum_{n\in\Z}\alpha_n \ve_n}
        && \text{because $\brp{\opMi}^\ast$ is \prope{isometric}}
      \\&= \norm{\opM\sum_{n\in\Z}\alpha_n \ve_n}
        && \text{because $\opM$ is \prope{isometric}}
      \\&= \norm{\sum_{n\in\Z}\alpha_n \opM\ve_n}
        && \text{by property of \structe{linear operator}s}
      \\&= \norm{\sum_{n\in\Z}\alpha_n \vx_n}
        && \text{by definition of $\opM$}
      \\&\implies \text{$\setn{\vy_n}$ is a \structe{Riesz basis}}
        &&  \text{by left hypothesis}
    \end{align*}

  \item Proof for (2): by (3) and definition of \structe{Riesz basis} \xref{def:basis_riesz}
\end{enumerate}
\end{proof}

%--------------------------------------
\begin{proposition}
\footnote{
  \citerpgc{igari1996}{220}{0821821040}{Lemma 9.8}\\
  \citerppgc{wojtaszczyk1997}{20}{21}{0521578949}{Lemma 2.7(a)}
  }
\label{prop:rbasis_frame}
%--------------------------------------
Let $\setxZp{\vx_n}$ be a set of vectors in a \structe{Hilbert space}\ifsxref{seq}{def:hilbert} $\spH\eqd\HspaceX$.
\propbox{
  \brb{\begin{array}{M}
    $\setn{\vx_n}$ is a \structe{Riesz basis} for $\spH$ with\\
    $\ds A{\sum_{n=1}^\infty \abs{a_n}^2} \le \norm{\sum_{n=1}^\infty a_n\vx_n}^2 \le B{\sum_{n=1}^\infty \abs{a_n}^2}$\\
    $\scy\forall\setn{a_n}\in\spllF$
  \end{array}}
  \implies
  \brb{\begin{array}{M}
    $\setn{\vx_n}$ is a \structe{frame} for $\spH$ with\\
    $\ds\mcom{\frac{1}{B}\norm{\vx}^2 \le {\sum_{n=1}^\infty \abs{\inprod{\vx}{\vx_n}}^2} \le  \frac{1}{A}\norm{\vx}^2}{\prope{stability condition}}$\\
    $\scy\forall\vx\in\spH$
  \end{array}}
  }
\end{proposition}
\begin{proof}
\begin{enumerate}
  \item Let $\setxZp{\vy_n}$ be a \structe{Riesz basis} that is \prope{biorthogonal} to $\setxZp{\vx_n}$ \xref{thm:rbasis_xy}.\label{item:rbasis_frame_yn}

  \item Let $\ds\vx\eqd\sum_{n=1}^\infty a_n\vy_n$. \label{item:rbasis_frame_x}

  \item lemma: %$\sum_{n=1}^\infty \abs{\inprod{\vx}{\vx_n}}^2=\sum_{n=1}^\infty \abs{a_n}^2$. Proof:
        \label{item:rbasis_frame_sum}
    \begin{align*}
      \sum_{n=1}^\infty \abs{\inprod{\vx}{\vx_n}}^2
        &= \sum_{n=1}^\infty \abs{\inprod{\sum_{m=1}^\infty a_n \vy_m}{\vx_n}}^2
        && \text{by definition of $\vx$ \xref{item:rbasis_frame_x}}
      \\&= \sum_{n=1}^\infty \abs{\sum_{m=1}^\infty a_n \inprod{\vy_m}{\vx_n}}^2
        && \text{by \prope{homogeneous} property of $\inprodn$ \xref{def:inprod}}
      \\&= \sum_{n=1}^\infty \abs{\sum_{m=1}^\infty a_n \kdelta_{mn}}^2
        && \text{by definition of \prope{biorthogonal} \xref{def:biortho}}
      \\&= \sum_{n=1}^\infty \abs{a_n}^2
        && \text{by definition of $\kdelta$ \xref{def:kdelta}}
    \end{align*}

  \item Then\\
    $\begin{array}{r>{\ds}lc>{\ds}cc>{\ds}ll}
                 & A{\sum_{n=1}^\infty \abs{a_n}^2} &\le& \norm{\sum_{n=1}^\infty a_n\vx_n}^2 &\le& B{\sum_{n=1}^\infty \abs{a_n}^2}
                 & \text{by def. of $\setn{\vy_n}$ \xref{item:rbasis_frame_yn}}
      \\\implies & A{\sum_{n=1}^\infty \abs{a_n}^2} &\le& \norm{\sum_{n=1}^\infty a_n\vy_n}^2 &\le& B{\sum_{n=1}^\infty \abs{a_n}^2}
                 & \text{by def. of $\setn{\vy_n}$ \xref{item:rbasis_frame_yn}}
                %& \text{because $\setn{\vy_n}$ is a \structe{Riesz basis} \xref{thm:rbasis_xy}}
      \\\implies & A{\sum_{n=1}^\infty \abs{a_n}^2} &\le& \norm{\vx}^2 &\le& B{\sum_{n=1}^\infty \abs{a_n}^2}
                 & \text{by def. of $\vx$ \xref{item:rbasis_frame_x}}
      \\\implies & A{\sum_{n=1}^\infty \abs{\inprod{\vx}{\vx_n}}^2} &\le& \norm{\vx}^2 &\le& B{\sum_{n=1}^\infty \abs{\inprod{\vx}{\vx_n}}^2}
                 & \text{by lemma \xref{item:rbasis_frame_sum}}
      \\\implies & \frac{1}{B}\norm{\vx}^2 &\le& {\sum_{n=1}^\infty \abs{\inprod{\vx}{\vx_n}}^2} &\le&  \frac{1}{A}\norm{\vx}^2
    \end{array}$
\end{enumerate}
\end{proof}

%======================================
\subsubsection{Frames in Hilbert spaces}
%======================================


%--------------------------------------
\begin{definition}
\label{def:frame}
\footnote{
  \citerppg{young2001}{154}{155}{0127729550}\\
  \citerpgc{christensen2003}{88}{0817642951}{Definitions 5.1.1, 5.1.2}\\
  \citerppgc{heil2011}{204}{205}{0817646868}{Definition 8.2}\\
  \citerpgc{jorgensen2008}{267}{0817646825}{Definition 12.22}\\
  %\citerpgc{christensen2008}{3}{0817646779}{Definition 1.1.1}\\
  \citorp{duffin52}{343}\\
  \citerp{dgm86}{1272}
  }
\index{basis!frame}
\index{basis!tight frame}
%--------------------------------------
Let $\setxZp{\vx_n\in\setX}$ be a set of vectors in a \structe{Hilbert space}\ifsxref{seq}{def:hilbert}\\ 
$\spH\eqd\HspaceX$.
\defbox{\begin{array}{M}
  The set $\setn{\vx_n}$ is a \hid{frame} for $\spH$ if (\prope{stability condition})
    \\
    \qquad$\ds{\exists A,B\in\Rp \st
    \qquad A\norm{\vx}^2 \le \sum_{n=1}^\infty \abs{\inprod{\vx}{\vx_n}}^2 \le B\norm{\vx}^2\qquad\scy\forall\vx\in\setX}.$
    \\
  The quantities $A$ and $B$ are \hid{frame bounds}.
  \\The quantity $A'$ is the \hid{optimal lower frame bound} if 
  \\\qquad $A'=\sup\set{A\in\Rp}{\text{$A$ is a lower frame bound}}$.
  \\The quantity $B'$ is the \hid{optimal upper frame bound} if 
  \\\qquad $B'=\inf\set{B\in\Rp}{\text{$B$ is an upper frame bound}}$.
  \\A frame is a \hid{tight frame} if $A=B$.
  \\A frame is a \hid{normalized tight frame} (or a \hid{Parseval frame}) if $A=B=1$.
  \\A frame $\setxZp{\vx_n}$ is an \hid{exact frame} if for some $m\in\Z$, $\setxZp{\vx_n}\setd\setn{\vx_m}$ is \emph{not} a frame.
\end{array}}
\end{definition}




A frame is a \hie{Parseval frame} (\pref{def:frame}) if it satisfies \thme{Parseval's Identity} \xref{thm:plancherel}.
All orthonormal bases are Parseval frames \xref{thm:plancherel}; 
but not all Parseval frames are orthonormal bases \xrefP{ex:mercedesA1}.
%%--------------------------------------
%\begin{definition}
%\footnote{
%  \citerpgc{cohen2011}{28}{0817680942}{Definition 2.3}\\
%  \citerpgc{han2007}{92}{0821842129}{Definition 3.10}
%  }
%\label{def:frame_parseval}
%%--------------------------------------
%Let $\spH\eqd\HspaceX$ be a Hilbert space.
%\defbox{\ds
%  \text{A frame $\seqxZp{\vx_n}$ is a \hid{Parseval frame} in $\spH$ if}
%  \quad\mcom{\norm{\vx}^2 \eqs\sum_{n=1}^\infty \abs{\inprod{\vx}{\vx_n}}^2 \quad \forall\vx\in\setX}{\ide{Parseval's identity}}.
%  }
%\end{definition}

%--------------------------------------
\begin{example}[\exm{Mercedes frame}/\exm{Peace frame}]
\footnote{
  \citerppg{heil2011}{204}{205}{0817646868}\\
  \citerpg{byrne2005}{80}{1568812426}\\
  \citerpgc{han2007}{91}{0821842129}{Example 3.9}
  }
\label{ex:mercedesA1}
%--------------------------------------
Let $\inprodspace{\R^2}{+}{\cdot}{\R}{\dotplus}{\dottimes}{\inprodn}$ be an inner product space with
$\inprod{\opair{x_1}{y_1}}{\opair{x_2}{y_2}}\eqd x_1y_1+x_2y_2$. 
\exboxp{
Let $\vx_1\eqd\opair{0}{\sqrt{\frac{2}{3}}}$,
    $\vx_2\eqd\opair{ \sqrt{\frac{2}{3}}\frac{\sqrt{3}}{2}}{-\sqrt{\frac{2}{3}}\frac{1}{2}}$,
and $\vx_3\eqd\opair{-\sqrt{\frac{2}{3}}\frac{\sqrt{3}}{2}}{-\sqrt{\frac{2}{3}}\frac{1}{2}}$.
Then
\\\indentx$\ds\mcom{\norm{\vx}^2 = \sum_{n=1}^3 \left|\inprod{\vx}{\vx_n}\right|^2 \quad\scy\forall\vx\in\setX}{(satisfies \prope{Parseval's Identity})}$,
  but $\seqn{\vx_1,\,\vx_2,\,\vx_3}$ are \emph{not} orthonormal.

\psset{unit=0.15mm}
\begin{pspicture}(-350,-120)(250,160)
  \psset{
    linewidth=1pt,
    }
  \psline[linecolor=axis]{<->}(-110,0)(110,0)% x-axis
  \psline[linecolor=axis]{<->}(0,-110)(0,110)% y-axis
  \pscircle[linecolor=red,linestyle=dotted](0,0){100}% unit circle
  \psline[linecolor=blue,linewidth=2pt]{->}(0,0)(0,100)%
  \psline[linecolor=blue,linewidth=2pt]{->}(0,0)(86.6,-50)%
  \psline[linecolor=blue,linewidth=2pt]{->}(0,0)(-86.6,-50)%
  \uput[180](0,100){$\opair{0}{\sqrt{\frac{2}{3}}}$}%
  \uput[-30](86.6,-50){$\opair{\sqrt{\frac{2}{3}}\frac{\sqrt{3}}{2}}{-\sqrt{\frac{2}{3}}\frac{1}{2}}$}%
  \uput[-150](-86.6,-50){$\opair{-\sqrt{\frac{2}{3}}\frac{\sqrt{3}}{2}}{-\sqrt{\frac{2}{3}}\frac{1}{2}}$}%
  \uput[0](110,0){$x$}
  \uput[0](0,110){$y$}
\end{pspicture}
}
\end{example}
\begin{proof}
Let $\vv\eqd\opair{x}{y}$.
\begin{align*}
  &\sum_{n=1}^3 \left|\inprod{\vx}{\vx_n}\right|^2
  \\&= \brp{x\cdot0+y\cdot\sqrt{\frac{2}{3}}}^2 + 
       \brp{x\cdot\sqrt{\frac{2}{3}}\frac{\sqrt{3}}{2} -y\cdot\sqrt{\frac{2}{3}}\frac{1}{2}}^2 +
       \brp{-x\cdot\sqrt{\frac{2}{3}}\frac{\sqrt{3}}{2} -y\cdot\sqrt{\frac{2}{3}}\frac{1}{2}}^2
  \\&= \frac{2}{3}y^2 + 
       \brb{\brp{\frac{2}{3}}\brp{\frac{3}{4}}x^2 + \brp{\frac{2}{3}}\brp{\frac{1}{4}}y^2 -2\frac{2}{3}\frac{\sqrt{3}}{4}xy }+
       \brb{\brp{\frac{2}{3}}\brp{\frac{3}{4}}x^2 + \brp{\frac{2}{3}}\brp{\frac{1}{4}}y^2 +2\frac{2}{3}\frac{\sqrt{3}}{4}xy }
  \\&= \frac{2}{3}y^2 + \frac{1}{2}x^2 +\frac{1}{6}y^2 +\frac{1}{2}x^2 + \frac{1}{6}y^2
  \\&= x^2 +y^2
  \\&= \norm{\vv}^2
\end{align*}
\end{proof}


%--------------------------------------
\begin{example}[\exm{Mercedes frame}/\exm{Peace frame}]
\footnote{
  \citerppg{heil2011}{204}{205}{0817646868}\\
  \citerpg{byrne2005}{80}{1568812426}
  }
\label{ex:mercedes}
%--------------------------------------
Let $\inprodspace{\R^2}{+}{\cdot}{\R}{\dotplus}{\dottimes}{\inprodn}$ be an inner product space with
$\inprod{\opair{x_1}{y_1}}{\opair{x_2}{y_2}}\eqd x_1y_1+x_2y_2$. 
\exboxp{
Let $\vx_1\eqd\opair{0}{1}$,
    $\vx_2\eqd\opair{  \frac{\sqrt{3}}{2}}{\frac{1}{2}}$,
and $\vx_3\eqd\opair{- \frac{\sqrt{3}}{2}}{\frac{1}{2}}$.
Then
\\\indentx$\ds\sum_{n=1}^3 \left|\inprod{\vx}{\vx_n}\right|^2 = \frac{3}{2}\norm{\vx}^2 \quad\scy\forall\vx\in\setX$.
\\That is, $\seqn{\vx_1,\vx_2,\vx_3}$ is a \structe{tight frame} with frame bound $A=\frac{3}{2}$.

\psset{unit=0.15mm}
\begin{pspicture}(-350,-110)(250,122)
  \psset{
    linewidth=1pt,
    }
  \psline[linecolor=axis]{<->}(-110,0)(110,)% x-axis
  \psline[linecolor=axis]{<->}(0,-110)(0,110)% y-axis
  \pscircle[linecolor=red,linestyle=dotted](0,0){100}% unit circle
  \psline[linecolor=blue,linewidth=2pt]{->}(0,0)(0,100)%
  \psline[linecolor=blue,linewidth=2pt]{->}(0,0)(86.6,-50)%
  \psline[linecolor=blue,linewidth=2pt]{->}(0,0)(-86.6,-50)%
  \uput[225](0,100){$\opair{0}{1}$}%
  \uput[-30](86.6,-50){$\opair{ \frac{\sqrt{3}}{2}}{- \frac{1}{2}}$}%
  \uput[-150](-86.6,-50){$\opair{- \frac{\sqrt{3}}{2}}{- \frac{1}{2}}$}%
  \uput[0](110,0){$x$}
  \uput[0](0,110){$y$}
\end{pspicture}
}
\end{example}
\begin{proof}
Let $\vv\eqd\opair{x}{y}$.
\begin{align*}
  \sum_{n=1}^3 \left|\inprod{\vv}{\vx_n}\right|^2
    &= \brp{x\cdot0+y\cdot }^2 + 
       \brp{x\cdot \frac{\sqrt{3}}{2} -y\cdot \frac{1}{2}}^2 +
       \brp{-x\cdot \frac{\sqrt{3}}{2} -y\cdot \frac{1}{2}}^2
  \\&= y^2 + 
       \brp{\frac{3}{4}x^2 + \frac{1}{4}y^2 - \frac{\sqrt{3}}{2}xy } +
       \brp{\frac{3}{4}x^2 + \frac{1}{4}y^2 + \frac{\sqrt{3}}{2}xy }
  \\&= \frac{3}{2}\brp{x^2 +y^2}
  \\&= \frac{3}{2}\norm{\vv}^2
\end{align*}
\end{proof}



%--------------------------------------
\begin{theorem}
\footnote{
  \citerpg{christensen2003}{3}{0817642951}
  }
%--------------------------------------
Let $\setxn{\vx_n\in\setX}$ be a set of vectors in a \structe{Hilbert space} \ifxref{seq}{def:hilbert}\\
$\spH\eqd\HspaceX$.
\thmbox{\text{
  $\setn{\vx_n}$ is a \structe{frame} for $\linspan\setn{\vx_n}$.
  }}
\end{theorem}
\begin{proof}
\begin{enumerate}
  \item Upper bound: Proof that there exists $B$ such that $\sum_{n=1}^\xN \abs{\inprod{\vx}{\vx_n}}^2\le B\norm{\vx}^2 \quad\forall \vx\in\spH$:
    \begin{align*}
      \sum_{n=1}^\xN \abs{\inprod{\vx}{\vx_n}}^2
        &\le \sum_{n=1}^\xN \inprod{\vx_n}{\vx_n} \inprod{\vx}{\vx}
        &&   \text{by \thme{Cauchy-Schwarz inequality} \ifxref{vsinprod}{thm:cs}}
      \\&=   \mcom{\brb{\sum_{n=1}^\xN \norm{\vx_n}^2}}{$B$} \norm{\vx}^2
    \end{align*}

  \item Lower bound: Proof that there exists $A$ such that $A\norm{\vx}^2\le\sum_{n=1}^\xN \abs{\inprod{\vx}{\vx_n}}^2\quad\forall\vx\in\spH$:
    \begin{align*}
      \sum_{n=1}^\xN \abs{\inprod{\vx}{\vx_n}}^2
        &= {\sum_{n=1}^\xN \abs{\inprod{\vx_n}{\frac{\vx}{\norm{\vx}}}}^2} \norm{\vx}^2
      \\&\ge \mcom{\brp{\inf_\vy\set{\sum_{n=1}^\xN \abs{\inprod{\vx_n}{\vy}}^2}{\norm{\vy}=1}}}{$A$} \norm{\vx}^2
    \end{align*}

\end{enumerate}
\end{proof}

%%--------------------------------------
%\begin{definition}
%\footnote{
%  \citerpg{christensen2003}{3}{0817642951}
%  }
%\label{def:frame_S}
%%--------------------------------------
%Let $\spH\eqd\inprodspaceX$ be a Hilbert space.
%\defbox{\text{
%  The \hid{synthesis operator} \;$\opT\in\clO{\F^\xN}{\setX}$ is defined such that $\opT\setxn{\alpha_n}\eqd\sum_{n=1}^\xN \alpha_{n} \vx_n$.
%  %Let the \hid{synthesis operator}$\opT$ be defined such that $\opT\vx\eqd\setxn{\alpha_n}$ and $\vx=\sum_{n=1}^\xN \alpha_{n} \vx_n$.
%  }}
%\end{definition}







%=======================================
%\subsection{Take a second look ...}
%=======================================

%%--------------------------------------
%\begin{definition}
%\footnote{
%  \citerp{pinsky2002}{305}
%  }
%\label{def:riesz_basis}
%\index{basis!Riesz}
%%--------------------------------------
%Let $(\spV,\normn )$ a normed linear space 
%on the linear space $\spV\eqd(\setX,\F,+,\cdot)$.
%\defbox{\begin{array}{l@{\qquad}l>{\ds}l}
%  \mc{3}{l}{\text{
%    A \hid{Riesz basis} for $(\spV,\normn )$ is any set $\setTh\eqd\seq{\vx_n}{n\in\Z}$ that satisfies
%    }}
%    \\& 1.& \spV = \linspan\setTh
%            \qquad
%            \text{\scriptsize ($\setTh$ generates $\spV$).}
%    \\& 2.& \forall \seqn{\alpha_n\in\F},\; \exists A,B\in\Rp \st
%    \\&   & A\sum_n\abs{\alpha_n}^2 \le \norm{ \sum_{n\in\Z} \alpha_{n} \vx_{n} }^2 \le B\sum_n\abs{\alpha_n}^2
%  \end{array}}
%\end{definition}




%%--------------------------------------
%\begin{theorem}
%\footnote{
%  \citerp{mallat}{126}
%  }
%\index{frame!inversion}
%%--------------------------------------
%Let $\spV\eqd\opair{\spX}{\inprodn}$ be an inner product space.
%\thmbox{
%  \parbox{6\tw/16}{$\setTh$ is an orthonormal basis of $\spV$}
%  \qquad\iff\qquad
%  \parbox{6\tw/16}{$\setTh$ is a tight frame of $\spV$ with frame bound $A=1$.}
%  }
%\end{theorem}
%\begin{proof}
%\begin{align*}
%\intertext{1. Proof that $\setTh$ is an orthonormal basis $\implies$   $A=1$:}
%  A\norm{\vv}^2
%    &= \sum_{n=1}^\xN \left|\inprod{\vx_n}{\vv}\right|^2
%    && \text{by \pref{def:frame}}
%  \\&= \norm{\vv}^2
%    && \text{by left hypothesis and Parseval's Identity (\prefpo{thm:plancherel})}
%  \\ \implies & A=1
%  \\
%\intertext{2. Proof that $\setTh$ is an orthonormal basis $\impliedby$   $A=1$:}
%    A\norm{\vv}^2
%    &= \sum_{n=1}^\xN \left|\inprod{\vx_n}{\vv}\right|^2
%    && \text{by \pref{def:frame}}
%  \\&= \norm{\vv}^2
%    && \text{by right hypothesis and Parseval's Identity (\prefpo{thm:plancherel})}
%  \\\implies & \text{$\{\vx_n\}$ is orthonormal}
%    && \text{by Parseval's Identity (\prefpo{thm:plancherel})}
%\end{align*}
%\end{proof}



%%--------------------------------------
%\begin{theorem}
%\citep{dgm86}{1272}
%\index{frame!inversion}
%%--------------------------------------
%Let $\spH\eqd\inprodspaceX$ be a Hilbert space
%with tight frame $\setxn{\vx_n}$ and frame bound $A$.
%\thmbox{
%  \forall \vv\in\spV, \qquad
%  \vv = \frac{1}{A} \sum_{n\in\Z} \inprod{\vx_n}{\vv}\vx_n
%  }
%\end{theorem}
%%\begin{proof}
%%No proof at this time.
%%\end{proof}


%  %============================================================================
% Daniel J. Greenhoe
% LaTeX File
%============================================================================


%======================================
\chapter{Transversal Operators}
%======================================

\qboxnpqt
  {Ren\'e Descartes, philosopher and mathematician (1596--1650)
   \index{Descartes, Ren\'e}
   \index{quotes!Descartes, Ren\'e}
   \footnotemark}
  {../common/people/descartes_fransHals_bw_wkp_pdomain.jpg}
  {Je me plaisois surtout aux math\'ematiques,
    \`a cause de la certitude et de l'\'evidence de leurs raisons:
    mais je ne remarquois point encore leur vrai usage;
    et, pensant qu'elles ne servoient qu'aux arts m\'ecaniques,
    je m'\'etonnois de ce que leurs fondements \'etant si fermes et si solides,
    on n'avoit rien b\^ati dessus de plus relev\'e:}
  {I was especially delighted with the mathematics,
    on account of the certitude and evidence of their reasonings;
    but I had not as yet a precise knowledge of their true use;
    and thinking that they but contributed to the advancement of the mechanical arts,
    I was astonished that foundations, so strong and solid,
    should have had no loftier superstructure reared on them.}
  \citetblt{
    quote: & \citer{descartes_method} \\
    translation: & \citerc{descartes_method_eng}{part I, paragraph 10} \\
    image: & \scs\url{http://en.wikipedia.org/wiki/File:Frans_Hals_-_Portret_van_Ren\%C3\%A9_Descartes.jpg}, public domain
    }

%In a linear space, a vector (such as a function) can be represented in terms of the elements of a
%\structe{basis}.\footnote{\begin{tabular}{ll}
%  \structe{Hamel basis}:    & \prefp{def:basis_hamel}\\
%  \structe{Schauder basis}: & \prefp{def:basis_schauder}
%\end{tabular}}
%One of the most commonly used bases is the set of trigonometric functions used in Fourier analysis.
%However, these basis vectors are \prope{non-localized}; that is, their support extends over the entire set of real numbers.
%This makes representation of functions with finite support difficult in the sense that
%the partial sums of such representations tend to converge slowly,
%which in turn means that they tend to require many coefficients
%to attain an acceptable approximation in real-world applications.
%
%

%======================================
\section{Families of Functions}
%======================================
This text is largely set in the space of
\structe{Lebesgue square-integrable functions} $\spLLR$ \xref{def:spLLR}.
The space $\spLLR$ is a subspace of the space $\hxs{\spRR}$, the set of all functions with \structe{domain}
$\hxs{\R}$ (the set of real numbers) and \structe{range} $\R$.
The space $\spRR$ is a subspace of the space $\hxs{\spCC}$, the set of all functions with \structe{domain}
$\hxs{\C}$ (the set of complex numbers) and \structe{range} $\C$.
That is,
$\ds\spLLR \subseteq \spRR \subseteq \spCC$.
In general, the notation $\clFxy$ represents the set of all functions with domain $\setX$ and range $\setY$ \xref{def:spXY}.
Although this notation may seem curious, note that for finite $\setX$ and finite $\setY$, the number of functions (elements)
in $\clFxy$ is $\seto{\clFxy}=\seto{\setY}^\seto{\setX}$.
%---------------------------------------
\begin{definition}
\label{def:spXY}
%---------------------------------------
Let $\hxs{\setX}$ and $\hxs{\setY}$ be sets.
\defbox{\begin{array}{M}
  The space $\clFxy$ represents the set of all functions with \structe{domain} $\setX$ and \structe{range} $\setY$ such that
  \\\indentx$\ds \hxs{\clFxy} \eqd \set{\ff(x)}{\ff(x):\setX\to\setY}$
\end{array}}
\end{definition}

\ifdochasnot{relation}{%
%---------------------------------------
\begin{definition}
\footnote{
  \citerp{ab}{126},
  %\citerp{halmos1950}{15},
  \citerp{hausdorff1937e}{22},
  \citorp{poussin1915}{440}
  }
\label{def:setind}
\index{function!characteristic}
\index{function!indicator}
%---------------------------------------
Let $\setX$ be a set.
\defbox{\begin{array}{M}
  The \fnctd{indicator function} $\hxs{\setind}\in\clF{\psetx}{\setn{0,1}}$ is defined as
  \\\indentx$\ds
  \setind_{\setA}(x) =
    \brbl{\begin{array}{ll @{\qquad}C}
      1  & $ for $ x\in \setA       & \forall x\in\setX,\; \setA\in\psetx \\
      0  & $ for $ x\notin \setA   & \forall x\in\setX,\; \setA\in\psetx
    \end{array}}
    $
  \\
  The indicator function $\setind$ is also called the \fnctd{characteristic function}.
\end{array}}
\end{definition}
}

%=======================================
\section{Definitions and algebraic properties}
%=======================================
Much of the wavelet theory developed in this text is constructed using the \opb{translation operator} $\opTrn$
and the \opb{dilation operator} $\opDil$ (next).

%---------------------------------------
\begin{definition}
\footnote{
  \citerppgc{walnut2002}{79}{80}{0817639624}{Definition 3.39},
  \citerppg{christensen2003}{41}{42}{0817642951},
  \citerpgc{wojtaszczyk1997}{18}{0521578949}{Definitions 2.3,2.4},
  \citerpg{kammler2008}{A-21}{0521883407},
  \citerpg{bachman2000}{473}{0387988998},
  \citerpg{packer2004}{260}{0821834029}, %{section 3.1}\\
  \citerpg{zayed2004}{}{0817643044},
  \citerpgc{heil2011}{250}{0817646868}{Notation 9.4},
  \citerpg{casazza1998}{74}{0817639594},
  \citerp{goodman1993}{639},
  \citePpc{heil1989}{633}{Definition 1.3.1},
  \citerp{dai1996}{81},
  \citerpg{dai1998}{2}{0821808001}
  %\citerpg{dai1998}{21}{0821808001}
  }
\label{def:opT}
\label{def:opD}
\label{def:opTD}
%---------------------------------------
\defbox{\begin{array}{Mrc>{\ds}lCM}
    $\hxs{\opTrn_\tau}$   is a \opd{translation operator} on $\spCC$ if & \hxs{\opTrn_\tau}  \ff(x) &\eqd& \ff(x-\tau)  & \forall \ff\in\spCC & .
  \\$\hxs{\opDil_\alpha}$ is a \opd{dilation operator}    on $\spCC$ if & \hxs{\opDil_\alpha}\ff(x) &\eqd& \ff(\alpha x)& \forall \ff\in\spCC & .
  \\\mc{5}{M}{Moreover, $\hxs{\opTrn}\eqd\opTrn_1$ \quad{and}\quad $\opDil\eqd\sqrt{2}\opDil_2$} & .
\end{array}}
\end{definition}

%---------------------------------------
\begin{example}
%---------------------------------------
Let $\opT$ and $\opD$ be defined as in \prefpp{def:opTD}.
\exbox{
\begin{array}{cc}
  \includegraphics{../common/math/graphics/pdfs/opTrn.pdf}&\includegraphics{../common/math/graphics/pdfs/opDil.pdf}%
\end{array}}
\end{example}

%---------------------------------------
\begin{proposition}
\label{prop:opT_periodic}
%---------------------------------------
Let $\opTrn_\tau$ be a \fncte{translation operator} \xref{def:opT}.
\propbox{
  \sum_{n\in\Z}\opTrn_\tau^n\ff(x) = \sum_{n\in\Z}\opTrn_\tau^n\ff(x+\tau)
  \qquad \scy \forall \ff\in\spRR
  \qquad \brp{\text{\scs$\ds\sum_{n\in\Z}\opTrn_\tau^n\ff(x)$ is \prope{periodic} with period $\tau$}}
  }
\end{proposition}
\begin{proof}
\begin{align*}
  \sum_{n\in\Z}\opTrn_\tau^n\ff(x+\tau)
    &= \sum_{n\in\Z}\ff(x-n\tau+\tau)
    && \text{by definition of $\opTrn_\tau$} && \text{\xref{def:opT}}
  \\&= \sum_{m\in\Z}\ff(x-m\tau)
    && \text{where $m\eqd n-1$}         && \implies\;n=m+1
  \\&= \sum_{m\in\Z}\opTrn_\tau^m\ff(x)
    && \text{by definition of $\opTrn_\tau$} && \text{\xref{def:opT}}
\end{align*}
\end{proof}

In a linear space, every operator has an \structe{inverse}\ifsxref{relation}{def:rel_inverse}.
Although the inverse always exists as a \rele{relation}\ifsxref{relation}{def:relation},
it may not exist as a \rele{function}\ifsxrefs{relation}{def:function}or as an \rele{operator}\ifsxref{relation}{def:operator}.
But in some cases the inverse of an operator is itself an operator.
%\ifdochas{operator}{\footnote{\hie{operator inverse}: \prefp{def:op_inv}}}
The inverses of the operators $\opTrn$ and $\opDil$ both exist as operators,
as demonstrated next.

%---------------------------------------
\begin{proposition}[\thmd{transversal operator inverses}]
\label{prop:opTi}
\label{prop:opDi}
%---------------------------------------
Let $\opTrn$ and $\opDil$ be as defined in \prefp{def:opT}.
\propbox{
  \begin{array}{lrcl C@{\qquad}D}
    \mc{6}{M}{$\opTrn$ has an \ope{inverse} $\opTrni$ in $\spCC$ expressed by the relation}
      \\& \opTrni\ff(x) &=& \ff(x+1)               & \forall \ff\in\spCC  & (\hid{translation operator inverse}).
      \\
    \mc{6}{M}{$\opDil$ has an \ope{inverse} $\opDili$ in $\spCC$ expressed by the relation}
      \\& \opDili\ff(x) &=& \cwt\:\ff\brp{\half x} & \forall \ff\in\spCC  & (\hid{dilation operator inverse}).
  \end{array}
  }
\end{proposition}
\begin{proof}
\begin{align*}
  \intertext{1. Proof that $\opTrni$ is the inverse of $\opTrn$:}
  \opTrni\opTrn\ff(x)
    &= \opTrni\ff(x-1)
    && \text{by defintion of $\opTrn$}                       && \text{\xref{def:opT}}
  \\&= \ff([x+1]-1)                                          
  \\&= \ff(x)                                                
  \\&= \ff([x-1]+1)                                          
  \\&= \opTrn\ff(x+1)                                        
    && \text{by defintion of $\opTrn$}                       && \text{\xref{def:opT}}
  \\&= \opTrn\opTrni\ff(x)
  \\\implies & \opTrni\opTrn=\opI=\opTrn\opTrni
  %
  \intertext{2. Proof that $\opDili$ is the inverse of $\opDil$:}
  \opDili\opDil\ff(x)
    &= \opDili \sqrt{2}\ff\brp{2x}
    && \text{by defintion of $\opDil$}                       && \text{\xref{def:opT}}
  \\&= \brp{\cwt }\sqrt{2}\ff\brp{2\brs{\half x}}
  \\&= \ff(x)
  \\&= \sqrt{2} \brs{\cwt \ff\left(\half [2x]\right)}
  \\&= \opDil \brs{\cwt \ff\brp{\half x}}
    && \text{by defintion of $\opDil$}                       && \text{\xref{def:opT}}
  \\&= \opDil\opDili\ff(x)
  \\\implies & \opDili\opDil=\opI=\opDil\opDili
\end{align*}
\end{proof}

%---------------------------------------
\begin{proposition}
\label{prop:DjTn}
%---------------------------------------
Let $\opTrn$ and $\opDil$ be as defined in \prefp{def:opT}.\\
Let $\opDil^0=\opTrn^0\eqd\opI$ be the \ope{identity operator}.
\propbox{
  \opDil^j\opTrn^n\ff(x) = 2^{j/2}\ff\brp{2^jx-n} \qquad\scy\forall j,n\in\Z,\,\ff\in\spCC
  }
\end{proposition}
%\begin{proof}
%\begin{enumerate}
%  \item Proof for $j=0$:
%    \begin{align*}
%      \opDil^0\opTrn^n\ff(x)
%        &= \opI\opTrn^n\ff(x)
%        && \text{by definition of $\opDil^0$}
%      \\&= \opTrn^n\ff(x)
%
%        = 2^{j/2}\ff\brp{2^jx-n} \qquad\scy\forall j,n\in\Z,\,\ff\in\spCC
%
%    \end{align*}
%\end{enumerate}
%\end{proof}

%=======================================
\section{Linear space properties}
%=======================================

%20171225%\ifdochasnot{sums}{
%20171225%%--------------------------------------
%20171225%\begin{definition}
%20171225%\label{def:sum}
%20171225%\footnote{
%20171225%  \citerpgc{berberian1961}{8}{0821819127}{Definition~I.3.1},
%20171225%  \citorpc{fourier1820}{280}{``$\sum$" notation}  %{http://gallica.bnf.fr/ark:/12148/bpt6k33707/f285.image}
%20171225%  }
%20171225%%--------------------------------------
%20171225%Let $+$ be an addition operator on a tuple $\tuple{x_n}{m}{\xN}$.
%20171225%\defbox{\begin{array}{M}
%20171225%  The \hid{summation} of $\tuplen{x_n}$ from index $m$ to index $\xN$ with respect to $+$ is
%20171225%  \\\qquad$\ds
%20171225%  \sum_{n=m}^\xN x_n \eqd
%20171225%    \brbl{\begin{array}{>{\ds}l@{\qquad}M}
%20171225%      0                            & for $\xN<m$\\
%20171225%     %x_\xN                          & for $N=m$ \\
%20171225%    \brp{\sum_{n=m}^{\xN-1} x_n}+x_\xN & for $\xN\ge m$
%20171225%    \end{array}}
%20171225%  $
%20171225%\end{array}}
%20171225%\end{definition}
%20171225%}
%20171225%
%20171225%\ifdochasnot{series}{
%20171225%An infinite summation $\sum_{n=1}^\infty \fphi_n$ is meaningless outside some topological space (e.g. metric space, normed space, etc.).
%20171225%The sum $\sum_{n=1}^\infty \fphi_n$ is an abbreviation for $\lim_{\xN\to\infty}\sum_{n=1}^\xN \fphi_n$
%20171225%(the limit of partial sums). And the concept of limit is also itself meaningless outside of a topological space.
%20171225%
%20171225%%---------------------------------------
%20171225%\begin{definition}
%20171225%\footnote{
%20171225%  \citerpg{klauder2010}{4}{0817647902},
%20171225%  \citerpg{kubrusly2001}{43}{0817641742},
%20171225%  \citerppg{bachman1966}{3}{4}{0486402517}
%20171225%  %\citerppg{bachman2000fa}{3}{4}{0486402517}\\
%20171225%  }
%20171225%\label{def:suminf}
%20171225%%---------------------------------------
%20171225%Let $\topspaceX$ be a topological space and $\lim$ be the limit induced by the topology $\topT$.
%20171225%\defbox{\begin{array}{>{\ds}rc>{\ds}lc>{\ds}l}
%20171225%  \sum_{n=1}^\infty       \vx_n &\eqd& \sum_{n\in\Zp} \vx_n &\eqd& \lim_{\xN\to\infty} \sum_{n=1}^\xN \vx_n\\
%20171225%  \sum_{n=-\infty}^\infty \vx_n &\eqd& \sum_{n\in\Z}  \vx_n &\eqd& \lim_{\xN\to\infty} \brp{\sum_{n=0}^\xN \vx_n} + \brp{\lim_{\xN\to-\infty} \sum_{n=-1}^\xN \vx_n}
%20171225%\end{array}}
%20171225%\end{definition}
%20171225%}

%%---------------------------------------
%\begin{proposition}
%\label{prop:opTD_sum}
%\label{prop:opTDi_sum}
%%---------------------------------------
%%Let $\opTrn$ be the \fncte{translation operator} and $\opDil$ the \fncte{dilation operator}. % \xref{def:opD}.
%Let $\opTrn$ and $\opDil$ be as in \prefp{def:opT}.
%\propbox{\begin{array}{>{\ds}rc>{\ds}l@{\qquad}C}
%  \opDil^j\opTrn^n\sum_{m\in\Z} \ff(x) &=& \sum_{m\in\Z}\opDil^j\opTrn^n \ff(x) & \forall \ff\in\spCC,\,n,j\in\Z
%  %\opDil\sum_{n\in\Z} \ff(x) &=& \sum_{n\in\Z}\opDil \ff(x) & \forall \ff\in\spCC
%\end{array}}
%\end{proposition}
%\begin{proof}
%\begin{align*}
%  \opDil^j\opTrn^n \sum_{n\in\Z} \ff(x)
%    &= \sum_{m\in\Z} 2^{j/2}\ff\brp{2^j-n}
%  \\&= \sum_{m\in\Z}\opDil^j\opTrn^n\ff(x)
%    && \text{by \prefp{thm:L_prop}}
%\end{align*}
%\end{proof}


%--------------------------------------
\begin{proposition}
\label{prop:DjTnfg}
%--------------------------------------
Let $\opTrn$ and $\opDil$ be as in \prefp{def:opT}.
\propbox{
  \opDil^j\opTrn^n\brs{\ff\fg} = 2^{-j/2}\;\brs{\opDil^j\opTrn^n\ff}\;\brs{\opDil^j\opTrn^n\fg}
  \qquad\scy\forall j,n\in\Z,\,\ff\in\spCC
  }
\end{proposition}
\begin{proof}
  \begin{align*}
    \opDil^j\opTrn^n\brs{\ff(x)\fg(x)}
      &= 2^{j/2}
         \ff\brp{2^jx-n}
         \fg\brp{2^jx-n}
      && \text{by \prefp{prop:DjTn}}
    \\&= 2^{-j/2}
         \brs{2^{j/2}\ff\brp{2^jx-n}}
         \brs{2^{j/2}\fg\brp{2^jx-n}}
    \\&= 2^{-j/2}
         \brs{\opDil^j\opTrn^n\ff(x)} \brs{\opDil^j\opTrn^n\fg(x)}
      && \text{by \prefp{prop:DjTn}}
  \end{align*}
\end{proof}

In general the operators $\opTrn$ and $\opDil$ are \prope{noncommutative} ($\opTrn\opDil\neq\opDil\opTrn$),
as demonstrated by \pref{cnt:DTTD} (next) and \prefpp{prop:DTTD}.
%---------------------------------------
\begin{counterex}
\label{cnt:DTTD}
%---------------------------------------
\cntboxt{%
  \tbox{
    As illustrated to the right,\\ % for the function $\ff\in\clFrr$,\\ 
    it is \textbf{not} always true that\\ 
    $\opTrn\opDil=\opDil\opTrn$:
    }
  \qquad
  \begin{tabular}{ccc}
     \includegraphics{../common/math/graphics/pdfs/tentfx.pdf}%
    &\includegraphics{../common/math/graphics/pdfs/TD_tentfx.pdf}%
    &\includegraphics{../common/math/graphics/pdfs/DT_tentfx.pdf}%
  \end{tabular}%
  }
\end{counterex}

%---------------------------------------
\begin{proposition}[\thmd{commutator relation}]
\footnote{
  \citerpgc{christensen2003}{42}{0817642951}{equation (2.9)},
  \citerpg{dai1998}{21}{0821808001},
  \citerp{goodman1993}{641},
  \citerp{goodman1993feb}{110}
  }
\label{prop:DTTD}
%---------------------------------------
Let $\opTrn$ and $\opDil$ be as in \prefp{def:opT}.
\propbox{
  \begin{array}{>{\ds}rc>{\ds}lC}
    \opDil^j\opTrn^n &=& \opTrn^{2^{-j/2}n}\opDil^j & \forall j,n\in\Z\\
    \opTrn^n\opDil^j &=& \opDil^j\opTrn^{2^jn}      & \forall n,j\in\Z
  \end{array}
  }
\end{proposition}
\begin{proof}
  \begin{align*}
    \opDil^j \opTrn^{2^jn}\ff(x)
      &= 2^{j/2}\,\ff(2^jx-2^jn)
      && \text{by \prefp{prop:DjTnfg}}
    \\&= 2^{j/2}\,\ff\brp{2^j\brs{x-n}}
      && \text{by \prope{distributivity} of the field $\fieldR$} && \text{\xref{def:algebra}}
    \\&= \opTrn^n 2^{j/2}\,\ff\brp{2^jx}
      && \text{by definition of $\opTrn$}  && \text{\xref{def:opT}}
    \\&= \opTrn^n \opDil^j\ff\brp{x}
      && \text{by definition of $\opDil$}  && \text{\xref{def:opD}}
    \\
    \\
    \opDil^j \opTrn^n\ff(x)
      &= 2^{j/2}\,\ff(2^jx-n)
      && \text{by \prefp{prop:DjTnfg}}
    \\&= 2^{j/2}\,\ff\brp{2^j\brs{x-2^{-j/2}n}}
      && \text{by \prope{distributivity} of the field $\fieldR$}  && \text{\xref{def:algebra}}
    \\&= \opTrn^{2^{-j/2}n} 2^{j/2}\,\ff\brp{2^jx}
      && \text{by definition of $\opTrn$}  && \text{\xref{def:opT}}
    \\&= \opTrn^{2^{-j/2}n} \opDil^j\ff\brp{x}
      && \text{by definition of $\opDil$}  && \text{\xref{def:opD}}
  \end{align*}
\end{proof}



%=======================================
\section{Inner product space properties}
%=======================================
%\prefp{def:opT} defines the dilation operator $\opDil$ and translation operator $\opTrn$.
In an inner product space\ifsxref{vsinprod}{def:inprod},
every operator has an \ope{adjoint} \ifxref{operator}{prop:op_adjoint}
and this adjoint is always itself an operator.
In the case where the adjoint and inverse of an operator $\opU$ coincide,
then $\opU$ is said to be \prope{unitary}\ifsxref{operator}{def:op_unitary}.
And in this case, $\opU$ has several nice properties (see \pref{prop:wavstrct_TD_unitary_1} and \prefp{thm:TD_unitary}).
\pref{prop:wavstrct_Ta} (next) gives the adjoints of $\opDil$ and $\opTrn$,
and \prefpp{prop:TD_unitary} demonstrates that both
$\opDil$ and $\opTrn$ are \prop{unitary}.
Other examples of unitary operators include the \hie{Fourier Transform operator} $\opFT$
\ifsxref{harFour}{cor:ft_unitary} and the \hie{rotation matrix operator}\ifsxref{operator}{ex:operator_rotation_unitary}.
%---------------------------------------
\begin{proposition} %[Adjoints and inverses of transversal operators]
\label{prop:wavstrct_Ta}
\label{prop:wavstrct_Da}
\label{prop:opTa}
%---------------------------------------
Let $\opTrn$ be the \ope{translation operator} \xref{def:opT} with \ope{adjoint} $\hxs{\opTrna}$
and $\opDil$    the \ope{dilation operator} with \ope{adjoint} $\hxs{\opDila}$\ifsxref{operator}{def:adjoint}.
%in the inner product space $\inprodspaceX[\clFrc]$.
\propbox{
  \begin{array}{rcl C@{\qquad}D@{\qquad}D}
    \hxs{\opTrna}\ff(x) &=& \ff(x+1)
                &   \forall \ff\in\spLLR
       & (\ope{translation operator adjoint})
    \\
    \hxs{\opDila}\ff(x) &=& \cwt \:\ff\brp{\half x}
                &   \forall \ff\in\spLLR
       & (\ope{dilation operator adjoint})
  \end{array}
  }
\end{proposition}
\begin{proof}
\begin{align*}
  \intertext{1. Proof that $\opTrna\ff(x)=\ff(x+1)$:}
  \inprod{\fg(x)}{\opTrna\ff(x)}
    &= \inprod{\fg(u)}{\opTrna\ff(u)}
    && \text{by change of variable $x\rightarrow u$}
  \\&= \inprod{\opTrn\fg(u)}{\ff(u)}
    && \text{by definition of adjoint $\opTrna$}             && \text{\ifxref{operator}{def:adjoint}}
  \\&= \inprod{\fg(u-1)}{\ff(u)}
    && \text{by definition of $\opTrn$}                      && \text{\xref{def:opT}}
  \\&= \inprod{\fg(x)}{\ff(x+1)}
    && \text{where $x\eqd u-1$ $\implies$ $u=x+1$}
  \\\implies& \opTrna\ff(x) = \ff(x+1)
  \\
  \intertext{2. Proof that $\opDila\ff(x)=\cwt \:\ff\brp{\half x}$:}
  \inprod{\fg(x)}{\opDila\ff(x)}
    &= \inprod{\fg(u)}{\opDila\ff(u)}
    && \text{by change of variable $x\rightarrow u$}   
  \\&= \inprod{\opDil\fg(u)}{\ff(u)}
    && \text{by definition of $\opDila$}                     && \text{\ifxref{operator}{def:adjoint}}
  \\&= \inprod{\sqrt{2}\fg(2u)}{\ff(u)}
    && \text{by definition of $\opDil$}                      && \text{\xref{def:opD}}
  \\&= \int_{u\in\R} \sqrt{2}\fg(2u) \ff^\ast(u) \du
    && \text{by definition of $\inprodn$}                    && \text{\ifxref{vsinprod}{def:inprod}}
  \\&= \int_{x\in\R} \fg(x) \brs{\sqrt{2}\ff\brp{\frac{x}{2}} \half }^\ast \dx
    && \text{where $x=2u$}
  \\&= \inprod{\fg(x)}{\cwt  \ff\brp{\frac{x}{2}}}
    && \text{by definition of $\inprodn$}                    && \text{\ifxref{vsinprod}{def:inprod}}
  \\&\implies \opDila\ff(x)=\cwt \:\ff\brp{\frac{x}{2}}
\end{align*}
\end{proof}

%---------------------------------------
\begin{proposition}
\footnote{
  \citerpgc{christensen2003}{41}{0817642951}{Lemma 2.5.1},
  \citerpgc{wojtaszczyk1997}{18}{0521578949}{Lemma 2.5}
  }
\label{prop:TD_unitary}
%---------------------------------------
%Let $\opTrn$ be the translation operator with inverse $\hxs{\opTrni}$ and adjoint $\opTrna$
%and $\opDil$ be the dilation    operator with inverse $\hxs{\opDili}$ and adjoint $\opDila$.
Let $\opTrn$ and $\opDil$ be as in \prefpp{def:opT}.\\
Let $\opTrni$ and $\opDili$ be as in \prefpp{prop:opTi}.
\propbox{
  \begin{array}{MM}
    $\opTrn$ is \prope{unitary} in $\spLLR$ & ($\opTrni=\opTrna$ in $\spLLR$).\\
    $\opDil$ is \prope{unitary} in $\spLLR$ & ($\opDili=\opDila$ in $\spLLR$).
  \end{array}
  }
\end{proposition}
\begin{proof}
\begin{align*}
    \opTrni &= \opTrna && \text{by \prefp{prop:opTi} and \prefp{prop:wavstrct_Ta}}
  \\      &\implies \quad \text{$\opTrn$ is \prope{unitary}}
          && \text{by the definition of \prope{unitary} operators\ifsxref{operator}{def:op_unitary}}
  \\
  \\\opDili &= \opDila && \text{by \prefp{prop:opDi} and \prefp{prop:wavstrct_Da}}
  \\      &\implies \quad \text{$\opDil$ is \prope{unitary}}
          && \text{by the definition of \prope{unitary} operators\ifsxref{operator}{def:op_unitary}}
\end{align*}
\end{proof}



%=======================================
\section{Normed linear space properties}
%=======================================
%---------------------------------------
\begin{proposition}
\label{prop:opD_constant}
%---------------------------------------
Let $\opDil$ be the \fncte{dilation operator} \xref{def:opD}.
\propbox{
  \brb{\begin{array}{FMD}
    (1). & $\ds\opDil\ff(x) = \sqrt{2}\ff(x)$ & and\\
    (2). & $\ff(x)$ is \prope{continuous}
  \end{array}}
  \qquad\iff\qquad
  \brb{\text{$\ff(x)$ is a \prope{constant}}}
  \qquad\scy\forall \ff\in\spLLR
  }
\end{proposition}
\begin{proof}
\begin{enumerate}
  \item Proof that (1)$\impliedby$ \prope{constant} property:
    \begin{align*}
      \opDil\ff(x)
        &\eqd \sqrt{2}\ff\brp{2x}
        &&    \text{by definition of $\opDil$}  && \text{\xref{def:opD}}
      \\&=    \sqrt{2}\ff\brp{x}
        &&    \text{by \prope{constant} hypothesis}
    \end{align*}

  \item Proof that (2)$\impliedby$ \prope{constant} property:
    \begin{align*}
      \norm{\ff(x)-\ff(x+h)}
        &= \norm{\ff(x)-\ff(x)}
           \qquad\text{by \prope{constant} hypothesis}
      \\&= \norm{0}
      \\&= 0
           \qquad\text{by \prope{nondegenerate} property of $\normn$} && \text{\ifxref{vsnorm}{def:norm}}
      \\&\le \varepsilon
      \\&\implies \forall h>0,\,\exists\varepsilon\st\norm{\ff(x)-\ff(x+h)}<\varepsilon
      \\&\iffdef  \text{$\ff(x)$ is \prope{continuous}}
    \end{align*}

  \item Proof that (1,2)$\implies$ \prope{constant} property:
    \begin{enumerate}
      \item Suppose there exists $x,y\in\R$ such that $\ff(x)\neq\ff(y)$. \label{item:opD_constant_assumption}
      \item Let $\seqnZp{x_n}$ be a sequence with limit $x$ and $\seqnZp{y_n}$ a sequence with limit $y$ \label{item:opD_constant_xn}
      \item Then
        \begin{align*}
          0
            &< \norm{\ff(x)-\ff(y)}
            && \text{by assumption in \prefp{item:opD_constant_assumption}}
          \\&= \lim_{n\to\infty}\norm{\ff\brp{x_n}-\ff\brp{y_n}}
            && \text{by (2) and definition of $\seqn{x_n}$ and $\seqn{y_n}$ in \prefp{item:opD_constant_xn}}
          \\&= \lim_{n\to\infty}\norm{\ff\brp{2^m x_n}-\ff\brp{2^\ell y_n}}\quad\forall m,\ell\in\Z
            && \text{by (1)}
          \\&= 0
        \end{align*}
      \item But this is a \emph{contradiction}, so $\ff(x)=\ff(y)$ for all $x,y\in\R$, and $\ff(x)$ is \prope{constant}.
    \end{enumerate}
\end{enumerate}
\end{proof}

%---------------------------------------
\begin{remark}
%---------------------------------------
\remboxt{
  In \prefp{prop:opD_constant}, it is not possible to remove the \prope{continuous}\\
  constraint outright, as demonstrated by the next two counterexamples.
  }
\end{remark}

%---------------------------------------
\begin{counterex}
%---------------------------------------
Let $\ff(x)$ be a function in $\spRR$.
\cntbox{
   \begin{array}{M}
   Let $\ds\ff(x)\eqd\brbl{\begin{array}{lM}
     0 & for $x=0$\\
     1 & otherwise.
   \end{array}}$
   \\
   Then $\opDil\ff(x)\eqd\sqrt{2}\ff\brp{2x}=\sqrt{2}\ff(x)$, but $\ff(x)$ is \prope{not constant}.
   \end{array}
   \hspace{10mm}
   \begin{array}{M}
   {\footnotesize
   \psset{unit=8mm}%
   \begin{pspicture}(-2.5,-0.6)(2.5,2)%
     \psaxes[linecolor=axis,yAxis=false]{<->}(0,0)(-2.5,0)(2.5,2)%
     \psaxes[linecolor=axis,xAxis=false]{ ->}(0,0)(-2.5,0)(2.5,2)%
     \uput[0](2.5,0){$x$}%
     \psline[linecolor=blue,linestyle=dotted]   (-2.5,1)(-2,  1)%
     \psline[linecolor=blue,linestyle=solid,dotsize=5pt]{-o}(-2,  1)( 0,  1)%
     \psline[linecolor=blue,linestyle=solid,dotsize=5pt]{o-}( 0,  1)( 2,  1)%
     \psline[linecolor=blue,linestyle=dotted]   ( 2,  1)( 2.5,1)%
     \psdot[dotsize=4pt](0,0)%
     \rput[bl](0.5,0.6){$\ff(x)$}%
   \end{pspicture}}%
   \end{array}
  }
\end{counterex}

%---------------------------------------
\begin{counterex}
%---------------------------------------
Let $\ff(x)$ be a function in $\spRR$.\\ 
Let $\hxs{\Q}$ be the set of \structe{rational numbers} and
$\R\setd\Q$ the set of \structe{irrational numbers}.
\cntbox{
   \begin{array}{M}
   Let $\ds\ff(x)\eqd\brbl{\begin{array}{rM}
     1 & for $x\in\Q$\\
    -1 & for $x\in\R\setd\Q$.
   \end{array}}$
   \\
   Then $\opDil\ff(x)\eqd\sqrt{2}\ff\brp{2x}=\sqrt{2}\ff(x)$, but $\ff(x)$ is \prope{not constant}.
   \end{array}
   \hspace{10mm}
   \begin{array}{M}
   {\footnotesize
   \psset{xunit=8mm,yunit=5mm}%
   \begin{pspicture}(-2.5,-2)(2.5,2)%
     \psaxes[linecolor=axis]{<->}(0,0)(-2.5,-1.9)(2.5,1.9)%
     %\psaxes[linecolor=axis,xAxis=false]{ ->}(0,0)(-2.5,-1.9)(2.5,1.9)%
     \uput[0](2.5,0){$x$}%
     \psline[linecolor=blue,linestyle=dotted]   (-2.5,1)(2.5,  1)%
     \psline[linecolor=red,linestyle=dotted]   (-2.5,-1)(2.5, -1)%
     %\rput[bl](0.5,0.6){$\ff(x)$}%
   \end{pspicture}}%
   \end{array}
  }
\end{counterex}



%--------------------------------------
\begin{proposition}[Operator norm]
\label{prop:wavstrct_TD_unitary_1}
%--------------------------------------
%Let $\opTrn$ be the translation operator with inverse $\opTrni$ and adjoint $\opTrna$
%and $\opDil$ be the dilation    operator with inverse $\opDili$ and adjoint $\opDila$.
%Let $\inprodn$ be the inner product induced by the operator $\int$ in $\spLLR$,
%$\normn$ the norm induced by $\inprodn$, and
%$\normopn$ be the \hi{operator norm}\ifsxref{operator}{def:op_norm} induced by $\normn$.
Let $\opTrn$ and $\opDil$ be as in \prefp{def:opT}.
Let $\opTrni$ and $\opDili$ be as in \prefp{prop:opTi}.
Let $\opTrna$ and $\opDila$ be as in \prefp{prop:opTa}.
Let $\normn$ and $\inprodn$ be as in \prefp{def:spLLR}.
Let $\normopn$ be the \hi{operator norm}\ifsxref{operator}{def:op_norm} induced by $\normn$.
\propbox{
  \normop{\opTrn} = \normop{\opDil} = \normop{\opTrna} = \normop{\opDila} = \normop{\opTrni} = \normop{\opDili} = 1 
  }
%\propbox{\begin{array}{lclclD}
%    \normop{\opTrn}                   &=& \normop{\opDil}               &=& 1                 & (\prope{unit length})
%  \\\normop{\opTrna}                  &=& \normop{\opDila}              &=& 1                 & (\prope{unit length})
%  \\\normop{\opTrni}                  &=& \normop{\opDili}              &=& 1                 & (\prope{unit length})
%\end{array}}
\end{proposition}
\begin{proof}
  These results follow directly from the fact that $\opTrn$ and $\opDil$ are \prope{unitary}
  \ifsxref{operator}{prop:TD_unitary} and from
  \ifdochaselse{operator}{\prefp{thm:unitary_equiv} and \prefp{thm:unitary_prop}.}
                         {properties of unitary operators.}
\end{proof}

%--------------------------------------
\begin{theorem}
\label{thm:TD_unitary}
%--------------------------------------
%Let $\opTrn$ be the translation operator with inverse $\opTrni$ and $\opDil$ the dilation operator with inverse $\opDili$.
Let $\opTrn$ and $\opDil$ be as in \prefp{def:opT}.\\
Let $\opTrni$ and $\opDili$ be as in \prefp{prop:opTi}.
Let $\normn$ and $\inprodn$ be as in \prefp{def:spLLR}.
\thmbox{\begin{array}{Flclcl@{\qquad}C@{\qquad}D}
    1. & \norm{\opTrn\ff}                    &=& \norm{\opDil\ff}                &=& \norm{\ff}        & \forall\ff    \in\spLLR & (\prope{isometric in length})
  \\2. & \norm{\opTrn\ff-\opTrn\fg}          &=& \norm{\opDil\ff-\opDil\fg}      &=& \norm{\ff-\fg}    & \forall\ff,\fg\in\spLLR & (\prope{isometric in distance})
  \\3. & \norm{\opTrni\ff-\opTrni\fg}        &=& \norm{\opDili\ff-\opDili\fg}    &=& \norm{\ff-\fg}    & \forall\ff,\fg\in\spLLR & (\prope{isometric in distance})
  \\4. & \inprod{\opTrn\ff}{\opTrn\fg}       &=& \inprod{\opDil\ff}{\opDil\fg}   &=& \inprod{\ff}{\fg} & \forall\ff,\fg\in\spLLR & (\prope{surjective})
  \\5. & \inprod{\opTrni\ff}{\opTrni\fg}     &=& \inprod{\opDili\ff}{\opDili\fg} &=& \inprod{\ff}{\fg} & \forall\ff,\fg\in\spLLR & (\prope{surjective})
\end{array}}
\end{theorem}
\begin{proof}
  These results follow directly from the fact that $\opTrn$ and $\opDil$ are \prope{unitary}
  \xrefP{prop:TD_unitary} and from
  \ifdochaselse{operator}{\prefp{thm:unitary_equiv} and \prefp{thm:unitary_prop}.}
                         {properties of unitary operators.}
\end{proof}



%---------------------------------------
\begin{proposition}
\label{prop:vsmra_real_sa}
%---------------------------------------
%Let $\opTrn$ be the translation operator.
Let $\opTrn$ be as in \prefp{def:opT}.
%in a Hilbert space $\spH\eqd\HspaceX[\clFrc]$\ifsxref{seq}{def:hilbert}.
Let $\opAa$ be the \ope{adjoint}\ifsxrefs{operator}{def:norm_adjoint} of an operator $\opA$.
\ifdochas{operator}{Let the property ``\prope{self adjoint}" be defined as in \prefpp{def:op_selfadj}.}
\propbox{
  \begin{array}{lcl@{\qquad\qquad}C}
  \ds\brp{\sum_{n\in\Z} \opTrn^n} &=& \ds\brp{\sum_{n\in\Z} \opTrn^n}^\ast
    & \ds\brp{\text{The operator } \brs{\sum_{n\in\Z} \opTrn^n} \text{ is \prope{self-adjoint}}}
  \end{array}
  }
\end{proposition}
\begin{proof}
   %Let $\opA\eqd\brp{\sum_{n\in\Z} \opTrn^n}$.
    \begin{align*}
      %\inprod{(\opA\ff)(x)}{\fg(x)}
        \inprod{\brp{\sum_{n\in\Z} \opTrn^n}\ff(x)}{\fg(x)}
        &= \inprod{\sum_{n\in\Z} \ff(x-n)}{\fg(x)}
        && \text{by definition of $\opTrn$}  && \text{\xref{def:opT}}
      \\&= \inprod{\sum_{n\in\Z} \ff(x+n)}{\fg(x)}
        && \text{by \prope{commutative} property} && \text{\xref{def:field}}
      \\&= \sum_{n\in\Z} \inprod{\ff(x+n)}{\fg(x)}
        && \text{by \prope{additive} property of $\inprodn$} && \text{\ifxref{vsinprod}{def:inprod}}
      \\&= \sum_{n\in\Z} \inprod{\ff(u)}{\fg(u-n)}
        && \text{where $u\eqd x+n$}
      \\&= \inprodr{\ff(u)}{\sum_{n\in\Z} \fg(u-n)}
        && \text{by \prope{additive} property of $\inprodn$} && \text{\ifxref{vsinprod}{def:inprod}}
      %\\&= \inprod{\ff(u)}{\sum_{n\in\Z} \fg(u+n)}
      %  && \text{by \prop{commutative} property of addition}
      \\&= \inprodr{\ff(x)}{\sum_{n\in\Z} \fg(x-n)}
        &&  \text{by change of variable: $u\rightarrow x$}
      \\&= \inprodr{\ff(x)}{\sum_{n\in\Z} \opTrn^{n}\fg(x)}
        && \text{by definition of $\opTrn$} && \text{\xref{def:opT}}
      %\\&= \inprod{\ff(x)}{(\opA\fg)(x)}
      %  && \text{by definition of $\opA$}
      \\&\iff \brp{\sum_{n\in\Z}\opTrn^n} = \brp{\sum_{n\in\Z}\opTrn^n}^\ast
      %\\&\iff \opA=\opAa
        && \text{by definition of \ope{adjoint}} && \text{\ifsxref{operator}{prop:op_adjoint}}
      \\&\iff \text{$\brp{\sum_{n\in\Z}\opTrn^n}$ is \prope{self-adjoint}}
        && \text{by definition of \prope{self-adjoint}} && \text{\ifsxref{operator}{def:op_selfadj}}
    \end{align*}
\end{proof}




%=======================================
\section{Fourier transform properties}
%=======================================
%---------------------------------------
\begin{proposition}
\label{prop:wavstrct_BTD}
%---------------------------------------
%Let $\opTrn$ be the translation operator and $\opDil$ be the dilation operator in the space $\clFrc$.
Let $\opTrn$ and $\opDil$ be as in \prefp{def:opT}.\\
Let $\opBT$ be the \hie{two-sided Laplace transform} defined as
  $\brs{\opBT \ff}(s) \eqd \cft  \int_{\R} \ff(x) e^{-sx} \dx$.
\propbox{
  \begin{array}{F>{\ds}rc>{\ds}l c>{\ds}lCD}
      1.& \opBT\opTrn^n      &=& e^{-s n} \opBT      & &        & \forall n\in\Z &
    \\2.& \opBT\opDil^j      &=& \opDil^{-j}\opBT    & &        & \forall j\in\Z &
    \\3.& \opDil\opBT        &=& \opBT\opDili        & &        & \forall n\in\Z &
    \\4.& \opBT\opDili\opBTi &=& \opBTi\opDili\opBT  &=& \opDil & \forall n\in\Z & ($\opDili$ is \prope{similar} to $\opDil$)
    \\5.& \opDil\opBT\opDil  &=& \opDili\opBT\opDili &=& \opBT  & \forall n\in\Z &
  \end{array}
  }
\end{proposition}
\begin{proof}
\begin{align*}
  \opBT\opTrn^n \ff(x)
    &= \opBT \ff(x-n)
    && \text{by definition of $\opTrn$} && \text{\xref{def:opT}}
  \\&= \cft  \int_\R \ff(x-n) e^{-sx} \dx
    && \text{by definition of $\opBT$}
  \\&= \cft  \int_\R \ff(u) e^{-s (u+n)} \du
    && \text{where $u\eqd x-n$}
  \\&= e^{-sn}\;\brs{\cft  \int_\R \ff(u) e^{-s u} \du}
  \\&= e^{-sn}\;\opBT \ff(x)
    && \text{by definition of $\opBT$}
  \\
  \\
  \opBT\opDil^j\ff(x)
    &= \opBT \brs{2^{j/2}\,\ff\brp{2^jx}}
    && \text{by definition of $\opDil$} && \text{\xref{def:opD}}
  \\&= \cft \,\int_\R \brs{2^{j/2}\,\ff\brp{2^jx}} e^{-sx} \dx
    && \text{by definition of $\opBT$}
  \\&= \cft \,\int_\R \brs{2^{j/2}\,\ff(u)} e^{-s 2^{-j}} 2^{-j}\du
    && \text{let $u\eqd 2^jx \implies x=2^{-j}u$}
  \\&= \cwt \, \cft \,
       \int_\R \ff(u) e^{-s 2^{-j} u} \du
  \\&= \opDili\,\brs{ \cft \,
       \int_\R \ff(u) e^{-s u} \du}
    && \text{by \prefp{prop:wavstrct_Da} and} && \text{\prefp{prop:TD_unitary}}
  \\&= \opDil^{-j}\, \opBT \, \ff(x)
    && \text{by definition of $\opBT$}
  \\
  \opDil\opBT\,\ff(x)
    &= \opDil \brs{\cft  \int_\R \ff(x) e^{-sx} \dx  }
    && \text{by definition of $\opBT$}
  \\&= \frac{\sqrt{2}}{\sqrt{2\pi}} \int_\R \ff(x) e^{-2sx} \dx
    && \text{by definition of $\opDil$} && \text{\xref{def:opD}}
  \\&= \frac{\sqrt{2}}{\sqrt{2\pi}} \int_\R \ff\brp{\frac{u}{2}} e^{-s u} \half \du
    && \text{let $u\eqd 2x \implies x=\half u$}
  \\&= \cft  \int_\R \brs{\cwt  \ff\brp{\frac{u}{2}} }e^{-s u} \du
  \\&= \cft  \int_\R \brs{\opDili \ff}(u) \,e^{-s u} \du
    && \text{by \prefp{prop:wavstrct_Da} and} &&\text{\prefp{prop:TD_unitary}}
  \\&= \opBT\opDili \ff(x)
    && \text{by definition of $\opBT$}
  \\
  \opBTi\opDili\opBT
    &= \opBTi\opBT\opDil
    && \text{by previous result}
  \\&= \opDil
    && \text{by definition of operator inverse} && \text{\ifxref{operator}{def:op_inv}}
  \\
  \opBT\opDili\opBTi
    &= \opDil\opBT\opBTi
    && \text{by previous result}
  \\&= \opDil
    && \text{by definition of operator inverse} && \text{\ifxref{operator}{def:op_inv}}
  \\
  \opDil\opBT\opDil
    &= \opDil\opDili\opBT
    && \text{by previous result}
  \\&= \opBT
    && \text{by definition of operator inverse} && \text{\ifxref{operator}{def:op_inv}}
  \\
  \opDili\opBT\opDili
    &= \opDili\opDil\opBT
    && \text{by previous result}
  \\&= \opBT
    && \text{by definition of operator inverse} && \text{\ifxref{operator}{def:op_inv}}
\end{align*}
%These results follow from \prefp{prop:vsmra_real_FD}.
\end{proof}


%---------------------------------------
\begin{corollary}
%\label{prop:vsmra_real_FT}
%\label{prop:vsmra_real_FD}
\label{cor:wavstrct_FTD}
\label{cor:FTD}
%---------------------------------------
%Let $\opTrn$ be the translation operator and $\opDil$ be the dilation operator.
%in the space $\clFrc$, and
%Let $\opFT$ be the Fourier transform\ifsxref{harFour}{def:ft}. % as defined below:
Let $\opTrn$ and $\opDil$ be as in \prefp{def:opT}.
Let $\Ff(\omega)\eqd\opFT\ff(x)$ be the \ope{Fourier Transform} \xref{def:opFT} of some function $\ff\in\spLLR$ \xref{def:spLLR}.
%  \[\brs{\opFT \ff}(\omega) \eqd \cft  \int_{x\in\R} \ff(x) e^{-i\omega t} \dx\]
\corbox{
  \begin{array}{F>{\ds}rc>{\ds}l c>{\ds}l}
      1.& \opFT\opTrn^n  &=& e^{-i\omega n} \opFT
    \\2.& \opFT\opDil^j  &=& \opDil^{-j}\opFT
    \\3.& \opDil\opFT    &=& \opFT\opDili
    \\4.& \opDil         &=& \opFT\opDili\opFTi   &=& \opFTi\opDili\opFT
    \\5.& \opFT          &=& \opDil\opFT\opDil    &=& \opDili\opFT\opDili
  \end{array}
  }
\end{corollary}
\begin{proof}
These results follow directly from \prefp{prop:wavstrct_BTD} with $\opFT = \left.\opBT\right|_{s=i\omega}$.
\end{proof}

%---------------------------------------
\begin{proposition}
\label{prop:FTDf}
%---------------------------------------
%Let $\opTrn$ be the translation operator and $\opDil$ be the dilation operator.
%Let $\opTrn$ and $\opDil$ be as in \prefp{def:opT}.
%Let $\opFT$ be the Fourier transform\ifsxref{harFour}{def:ft}. % as defined below:
%Let $\Ff(\omega)\eqd\opFT\ff(x)$ for some function $\ff\in\spLLR$.
%  \[\brs{\opFT \ff}(\omega) \eqd \cft  \int_{x\in\R} \ff(x) e^{-i\omega t} \dx\]
Let $\opTrn$ and $\opDil$ be as in \prefp{def:opT}.
Let $\Ff(\omega)\eqd\opFT\ff(x)$ be the \ope{Fourier Transform} \xref{def:opFT} of some function $\ff\in\spLLR$ \xref{def:spLLR}.
\propbox{
  \opFT\opDil^j\opTrn^n \ff(x)  = \frac{1}{2^{j/2}} e^{-i\frac{\omega}{2^j} n} \Ff\brp{\frac{\omega}{2^j}}
  }
\end{proposition}
\begin{proof}
     \begin{align*}
       \opFT\opDil^j\opTrn^n\ff(x)
         &= \opDil^{-j}\opFT\opTrn^n\ff(x)
         && \text{by \prefp{cor:FTD} (3)}
       \\&= \opDil^{-j} e^{-i\omega n}\opFT\ff(x)
         && \text{by \prefp{cor:FTD} (3)}
       \\&= \opDil^{-j} e^{-i\omega n}\Ff\brp{\omega}
       \\&= 2^{-j/2} e^{-i2^{-j}\omega n}\Ff\brp{2^{-j}\omega}
         && \text{by \prefp{prop:opDi}}
     \end{align*}
\end{proof}

%---------------------------------------
\begin{proposition}
\label{prop:Fsum_af}
%---------------------------------------
Let $\opTrn$ be the translation operator \xref{def:opT}.
Let $\Ff(\omega)\eqd\opFT\ff(x)$ be the \ope{Fourier Transform} \xref{def:opFT} of a function $\ff\in\spLLR$.
Let $\Da(\omega)$ be the \ope{DTFT} \xref{def:dtft} of a sequence $\seqxZ{a_n}\in\spllR$ \xref{def:spllR}.
\propbox{
  \opFT\sum_{n\in\Z}a_n \opTrn^n\fphi(x) = \Da(\omega)\Fphi(\omega)
  \qquad\scy\forall\seqn{a_n}\in\spllR,\,\fphi(x)\in\spLLR
  }
\end{proposition}
\begin{proof}
     \begin{align*}
       \opFT\sum_{n\in\Z}a_n \opTrn^n\fphi(x)
         &= \sum_{n\in\Z}a_n \opFT\opTrn^n\fphi(x)
       \\&= \sum_{n\in\Z}a_n e^{-i\omega n}\opFT\fphi(x)
         && \text{by \prefp{cor:FTD}}
       \\&= \brs{\sum_{n\in\Z}a_n e^{-i\omega n}}\Fphi(\omega)
         && \text{by definition of $\Fphi(\omega)$}
       \\&= \Da(\omega)\Fphi(\omega)
         && \text{by definition of \ope{DTFT} \xref{def:dtft}}
     \end{align*}
\end{proof}


%======================================
%\section{Poisson Summation Formulas}
\label{sec:psf}
%======================================
%\qboxnps
%  {
%  Sim\'eon Denis Poisson (1781--1840) French mathematician and physicist
%  \index{Poisson, Sim\'eon Denis}
%  \index{quotes!Poisson, Sim\'eon Denis}
%  \footnotemark
%  }
%  {../common/people/poisson.jpg}
%  {Life is good for only two things, discovering mathematics and teaching mathematics.}
%  \footnotetext{\begin{tabular}[t]{ll}
%    %quote: & \url{http://www-history.mcs.sx-andrews.ac.uk/Biographies/Poisson.html} \\
%    quote: & \citerpg{eves1990}{486}{0030295580}\\
%    image: & \url{http://en.wikipedia.org/wiki/Poisson}
%  \end{tabular}}

%The \hie{Poisson Summation Formula} (PSF, \prefpo{thm:psf}) and
%\hie{Inverse Poisson Summation Forumla} (IPSF, \prefpo{thm:ipsf})
%are fundamental and extremely powerful theorems in Harmonic analysis.


%--------------------------------------
\begin{definition}
\label{def:opS}
%--------------------------------------
Let $\spLLRBu$ be the \structe{space of Lebesgue square-integrable functions} \xref{def:spLLRBu}.
Let $\spllR$ be the \structe{space of all absolutely square summable sequences over $\R$} \xref{def:spLLR}.
\defboxt{
  $\opS$  is the \opd{sampling operator} in $\clO{\spLLR}{\spllR}$ if
  \quad$\ds\brs{\opS\ff(x)}(n) \eqd \ff\brp{\frac{2\pi}{\tau} n}\qquad \scy\forall \ff\in\spLLRBu,\,\tau\in\Rp$
  }
\end{definition}


%--------------------------------------
\begin{theorem}[\thmd{Poisson Summation Formula}---\thmd{PSF}]
\footnote{
  \citerp{andrews}{624},
  \citerpg{knappb2005}{389}{0817632506},
  \citerp{lasser}{254},
  %\citerp{goswami}{44}  \\
  \citerpp{rudinr}{194}{195},
  \citerp{folland}{337}
  }
\label{thm:psf}
%--------------------------------------
Let $\Ff(\omega)$ be the \fncte{Fourier transform} \xref{def:opFT} of a function $\ff(x)\in\spLLR$.
Let $\opS$ be the \ope{sampling operator} \xref{def:opS}.
%Let $\opFT$ be the \hie{Fourier Transform} operator,
%    $\opFS$ the \hie{Fourier Series} operator,
%    $\opFSi$ the \hie{inverse Fourier Series} operator,
%and $\Ff(\omega)\eqd\opFT\ff(x)$ (the Fourier transform of a function $\ff(x)$).
\thmboxt{
   $\ds
   \mcom{\ds\sum_{n\in\Z} \opTrn_{\tau}^n\ff\brp{x} =
            \sum_{n\in\Z} \ff\brp{x+n\tau}}
        {summation in ``time"}
   =
   \mcom{\ds\sqrt{\frac{2\pi}{\tau}}\: \opFSi \opS\opFT\brs{\ff(x)}}
        {operator notation}
        %{\parbox{6\tw/16}{proportional to the inverse Fourier series
        % of the Fourier transform of $\ff(x)$
        % sampled at $\frac{2\pi}{\tau}$ intervals}
        %}
   =
   \mcom{\ds\frac{\sqrt{2\pi}}{\tau} \sum_{n\in\Z} \Ff\brp{\frac{2\pi}{\tau}n} e^{i\frac{2\pi}{\tau}nx}}
        {summation in ``frequency"}
  $
  %\\\qquad
  %  $\begin{array}{Ml}
  %    where & \text{$\opS\in\clO{\spLLR}{\spllR}$ is the \ope{sampling operator} defined as}\\
  %          & \brs{\opS\ff(x)}(n) \eqd \ff\brp{\frac{2\pi}{\tau} n}\qquad \forall \ff\in\spLLRBu,\,\tau\in\Rp
  %  \end{array}$
  }
\end{theorem}

\begin{proof}
%\begin{enumerate}
%\item Proof using Fourier series (\prefpo{def:opFS}):
\begin{enumerate}
  \item lemma: If $\ds \fh(x)    \eqd \sum_{n\in\Z} \ff(x+n\tau)$ then $\fh\equiv\opFSi\opFS\fh$.
        Proof: \label{ilem:psf_h}\\
        Note that $\fh(x)$ is \prope{periodic} with period $\tau$\ifsxref{sums}{lem:series_sumT}.
        %\[
        %   \fh(x+\tau) \eqd \sum_{n\in\Z} \ff(x+T+n\tau) = \sum_{n\in\Z} \ff(x+(n+1)\tau) = \sum_{n\in\Z} \ff(x+n\tau) \eqd \fh(x).
        %\]
        Because $\fh$ is periodic, it is in the domain of $\opFS$ and thus $\fh\equiv\opFSi\opFS\fh$.
  \item Proof of PSF (this theorem---\pref{thm:psf}):
    \begin{align*}
       \sum_{n\in\Z} \ff(x+n\tau)
           &= \opFSi \opFS \sum_{n\in\Z} \ff(x+n\tau)
           && \text{by \prefp{ilem:psf_h}}
         \\&= \opFSi \brn{
                \mcom{\brs{
                  \fsscale\int_0^\tau \brp{\sum_{n\in\Z} \ff(x+n\tau)} e^{-i\frac{2\pi}{\tau}kx} \dx
                  }}{$\opFS \brs{\sum_{n\in\Z} \ff(x+n\tau)}$}
                }
            && \text{by definition of $\opFS$} && \text{\xref{def:opFS}}
         \\&= \opFSi \brs{
                \fsscale \sum_{n\in\Z}
                \int_{0}^\tau \ff(x+n\tau) e^{-i\frac{2\pi}{\tau}kx} \dx
                }
         \\&= \opFSi \brs{
                \fsscale
                \sum_{n\in\Z}
                \int_{u=n\tau}^{u=(n+1)\tau} \ff(u) e^{-i\frac{2\pi}{\tau}k(u-n\tau)} \du
                }
           && \text{where $u\eqd x+n\tau$ $\implies$} && \text{$x=u-n\tau$}
         \\&= \opFSi \brs{
                \fsscale \sum_{n\in\Z}
                \cancelto{\scy1}{e^{i2\pi kn}}\;
                \int_{u=n\tau}^{u=(n+1)\tau} \ff(u) e^{-i\frac{2\pi}{\tau}ku} \du
                }
         \\&= \sqrt{\frac{2\pi}{\tau}} \opFSi \brn{
                \mcom{\brs{\fscale \int_{u\in\R} \ff(u) e^{-i\left(\frac{2\pi}{\tau}k\right)u} \du}}
                     {$\brs{\opFT\ff}\brp{\frac{2\pi}{\tau}k}$}
                }
           && \text{by evaluation of $\opFSi$} && \text{\xref{thm:opFSi}}
         \\&= \sqrt{\frac{2\pi}{\tau}}\: \opFSi \brs{\brs{\opFT\ff(x)}\brp{\frac{2\pi}{\tau}k}}
           && \text{by definition of $\opFT$} && \text{\xref{def:opFT}}
         \\&= \sqrt{\frac{2\pi}{\tau}}\: \opFSi\opS\opFT \ff
           && \text{by definition of $\opS$}
           && \text{\xref{def:opS}}
         \\&= \frac{\sqrt{2\pi}}{\tau}\:
              \sum_{n\in\Z} \Ff\brp{\frac{2\pi}{\tau}n} e^{i\frac{2\pi}{\tau}nx}
           && \text{by evaluation of $\opFSi$} && \text{\xref{thm:opFSi}}
    \end{align*}
\end{enumerate}
%\item Proof using \hie{spectral theorem}\ifdochas{operator}{ (\prefpo{thm:spectral_theorem})}:
%\begin{enumerate}
%  \item Define operator $\opA$ as
%        \[ (\opA \ff)(x) \eqd \sum_{n\in\Z} \ff(x+n) \]
%
%  \item $\opA$ can be expressed as an integral operator with
%        kernel $\fh(t,u) = \sum_{n\in\Z} \delta(u-x-n)$:
%        \begin{align*}
%          (\opA \ff)(x)
%            &= \sum_{n\in\Z} \ff(x+n)
%          \\&= \sum_{n\in\Z} \int_{u\in\R} \delta(u-x-n) \ff(u) \du
%            && \text{by \prefp{def:dirac}}
%          \\&= \int_{u\in\R} \mcom{\sum_{n\in\Z} \delta(u-x-n)}{kernel $\fh(t,u)$} \ff(u) \du
%        \end{align*}
%
%  \item $\opA$ has eigenpairs $\brp{2\pi\sum_{n\in\Z} \delta(\omega-2\pi n),\, e^{i\omega x}}$:
%        \begin{align*}
%          \opA e^{i\omega x}
%            &= \sum_{n\in\Z} e^{i\omega(x+n)}
%          \\&= \mcomr{\brp{\sum_{n\in\Z} e^{i\omega n}}}{Dirichlet kernel}\: e^{i\omega x}
%          \\&= \mcom{\brp{2\pi\sum_{n\in\Z} \delta(\omega-2\pi n)}}{eigenvalue}\:
%               \mcoml{e^{-i\omega x}}{eigenvector}
%            && \text{by \prefp{prop:dkernel}}
%        \end{align*}
%
%  \item $\opA$ is \hie{self adjoint}:%
%    \ifdochas{operator}{\footnote{\hie{self ajoint}: \prefp{def:op_selfadj}}}
%        \begin{align*}
%          \inprod{(\opA\ff)(x)}{\fg(x)}
%            &= \inprod{\sum_{n\in\Z} \ff(x+n)}{\fg(x)}
%          \\&= \sum_{n\in\Z} \inprod{\ff(x+n)}{\fg(x)}
%            && \text{by additivity property of $\inprodn$ \ifdochas{vsinprod}{\prefpo{def:inprod}}}
%          \\&= \sum_{n\in\Z} \inprod{\ff(u)}{\fg(u-n)}
%            && \text{where $u=x+n$}
%          \\&= \inprod{\ff(u)}{\sum_{n\in\Z} \fg(u-n)}
%          \\&= \inprod{\ff(u)}{\sum_{n\in\Z} \fg(u+n)}
%          \\&= \inprod{\ff(x)}{\sum_{n\in\Z} \fg(x+n)}
%          \\&= \inprod{\ff(x)}{(\opA\fg)(x)}
%          \\&\implies \opA=\opAa
%        \end{align*}
%
%  \item Because the operator $\opA$ is self adjoint, it is also \hie{normal}.%
%        \ifdochas{operator}{\footnote{\hie{normal}: \prefp{def:op_normal}}}
%
%  \item Because the operator $\opA$ is normal, it can be evaluated using the \hie{spectral theorem}
%        \footnote{\hie{spectral theorem}: \ifdochas{operator}{\prefp{thm:spectral_theorem}}}.
%        \begin{align*}
%          \sum_{n\in\Z} \ff(x+n\tau)
%            &= (\opA \ff)(x)
%            && \text{by definition of $\opA$}
%          \\&= \sum_{n\in\Z} \lambda_n \opP_n \ff
%            && \text{by spectral theorem \ifdochas{operator}{\prefpo{thm:spectral_theorem}}}
%          \\&= \sum_{n\in\Z} \brs{2\pi\sum_{m\in\Z} \delta(\omega_n-2\pi m)}\:
%                      \inprod{\ff(x)}{e^{i\omega_n t}}
%          \\&= \sum_{n\in\Z} \brs{2\pi\sum_{m\in\Z} \delta(\omega_n-2\pi m)}\:
%                      \sqrt{2\pi}(\opFT\ff)(\omega_n)
%          \\&= \brp{\sqrt{2\pi}}^\frac{3}{2}
%               \sum_{n\in\Z} \brs{\sum_{m\in\Z} \delta(\omega_n-2\pi m)}\:
%                      (\opFT\ff)(2\pi m)
%        \end{align*}
%\end{enumerate}
%\end{enumerate}
\end{proof}


%--------------------------------------
\begin{theorem}[\thmd{Inverse Poisson Summation Formula}---\thmd{IPSF}]
\footnote{
  \citerp{gauss1900werke8}{88}
  }
\label{thm:ipsf}
\index{Inverse Poisson Summation Formula}
\index{theorems!Inverse Poisson Summation Formula}
\index{IPSF}
%--------------------------------------
\\Let $\Ff(\omega)$ be the \fncte{Fourier transform} \xref{def:opFT} of a function $\ff(x)\in\spLLR$.
%Let $\opFT$ be the \hie{Fourier Transform} operator with inverse $\opFTi$,
%    $\opFS$ be the \hie{Fourier Series} operator with inverse $\opFSi$,
%and $\Ff(\omega)$ be the Fourier transform of a function $\ff(x)$
%such that $\Ff \eqd \opFT\ff$.
\thmbox{
   \mcom{\sum_{n\in\Z} \opTrn_{2\pi/\tau}^n\Ff\brp{\omega} \eqd
         \sum_{n\in\Z} \Ff\brp{\omega - \frac{2\pi}{\tau}n}}
        {summation in ``frequency"}
   \qquad=\qquad
   %\mcom{\sqrt{\tau}\: \opFSi \opS \opFTi\Ff }
   %     {\parbox{5\tw/16}{
   %     proportional to the inverse Fourier series of the
   %     inverse Fourier transform of $\Ff(\omega)$
   %     sampled at $kT$ intervals and reversed in time about $x=0$
   %     }}
   %=
   \mcom{\frac{\tau}{\sqrt{2\pi}} \sum_{n\in\Z} \ff(n\tau) e^{-i\omega n\tau}}
        {summation in ``time"}
%  \\\qquad
%    \begin{array}{>{$}l<{$}>{\ds}l}
%      where & \text{$\opS\in\clO{\spLLR}{\spllR}$ is the \ope{sampling operator} defined as}\\
%            & \brs{\opS\ff(x)}(n) \eqd \ff\brp{-n\tau}\qquad\scriptstyle \forall \ff\in\spLLRBu
%    \end{array}
  }
\end{theorem}
\begin{proof}
\begin{enumerate}
  \item lemma: \label{ilem:ipsf_h}
        If $\fh(\omega) \eqd \sum_{n\in\Z} \Ff\left( \omega + \frac{2\pi}{\tau}n  \right)$,
               then  $\fh\equiv\opFSi\opFS\fh$. Proof: \\
        Note that $\fh(\omega)$ is periodic with period $2\pi/T$:
    \[
       \fh\left(\omega+\frac{2\pi}{\tau}\right)
       \eqd \sum_{n\in\Z} \Ff\left( \omega+\frac{2\pi}{\tau}+ \frac{2\pi}{\tau}n \right)
       =      \sum_{n\in\Z} \Ff\left( \omega+ (n+1)\frac{2\pi}{\tau} \right)
       =      \sum_{n\in\Z} \Ff\left( \omega + \frac{2\pi}{\tau}n  \right)
       \eqd \fh(\omega)
    \]
    Because $\fh$ is periodic, it is in the domain of $\opFS$ and is equivalent to $\opFSi\opFS\fh$.
  \item Proof of IPSF (this theorem---\pref{thm:ipsf}):
    \begin{align*}
      &\sum_{n\in\Z} \Ff\left( \omega + \frac{2\pi}{\tau}n \right)
      \\&= \opFSi\opFS\brn{ \sum_{n\in\Z} \Ff\left( \omega + \frac{2\pi}{\tau}n \right) }
        && \text{by \prefp{ilem:ipsf_h}}
      \\&= \opFSi\brn{
             \mcom{\brs{ \sqrt{\frac{\tau}{2\pi}}\int_0^\frac{2\pi}{\tau} \sum_{n\in\Z} \Ff\brp{\omega + \frac{2\pi}{\tau}n} e^{-i\omega\frac{2\pi}{2\pi/\tau}k} \dw }}
                  {$\opFS\brs{\sum_{n\in\Z} \Ff\brp{\omega + \frac{2\pi}{\tau}n}}$}
             }
        && \text{by definition of $\opFS$} && \text{\xref{def:opFS}}
      \\&= \opFSi\brn{
             \brs{ \sqrt{\frac{\tau}{2\pi}}\sum_{n\in\Z} \int_0^\frac{2\pi}{\tau} \Ff\brp{\omega + \frac{2\pi}{\tau}n} e^{-i\omega Tk} \dw }
             }
      \\&= \opFSi\brn{
             \brs{ \sqrt{\frac{\tau}{2\pi}}\sum_{n\in\Z} \int_{u=\frac{2\pi}{\tau}n}^{u=\frac{2\pi}{\tau}(n+1)}
                   \Ff\brp{u} e^{-i\brp{u-\frac{2\pi}{\tau}n} Tk} \du }
             }
        && \text{where $u\eqd\omega + \frac{2\pi}{\tau}n$ $\implies$} && \text{$\omega=u-\frac{2\pi}{\tau}n$}
      \\&= \opFSi\brn{
             \brs{\sqrt{\frac{\tau}{2\pi}} \sum_{n\in\Z}
                  \cancelto{\scy1}{e^{i2\pi nk}}\;
                  \int_{\frac{2\pi}{\tau}n}^{\frac{2\pi}{\tau}(n+1)} \Ff(u) e^{-iu\tau k} \du
                 }
             }
      \\&= \opFSi\brn{
             \brs{\sqrt{\frac{\tau}{2\pi}}
                  \int_{\R} \Ff(u) e^{-iu \tau k} \du
                 }
             }
      \\&= \sqrt{\tau}\: \opFSi\brn{
             \mcom{\brs{
                  \fscalei
                  \int_{\R} \Ff(u) e^{iu(-\tau k)} \du
                 }}{$\brs{\opFTi\Ff}(-k\tau)$}
             }
      \\&= \sqrt{\tau}\: \opFSi \brs{ \brs{\opFTi\Ff}(-k\tau) }
        && \text{by value of $\opFTi$} && \text{\xref{thm:opFTi}}
      \\&= \sqrt{\tau}\: \opFSi \opS \opFTi \:\Ff
        && \text{by definition of $\opS$}
        && \text{\xref{def:opS}}
      \\&= \sqrt{\tau}\: \opFSi \opS \ff(x)
        && \text{by definition of $\opFT$} && \text{\xref{def:opFT}}
      \\&= \sqrt{\tau}\: \opFSi \ff(-k\tau)
        && \text{by definition of $\opS$}
        && \text{\xref{def:opS}}
      \\&= \sqrt{\tau}\: \frac{1}{\sqrt{\frac{2\pi}{\tau}}}\:
           \sum_{k\in\Z} \ff(-k\tau)e^{i2\pi\frac{1}{\frac{2\pi}{\tau}} k\omega}
        && \text{by definition of $\opFSi$} && \text{\xref{thm:opFSi}}
      \\&= \frac{\tau}{\sqrt{\frac{2\pi}{\tau}}}\:
           \sum_{k\in\Z} \ff(-k\tau)e^{i k\tau\omega}
        && \text{by definition of $\opFSi$} && \text{\xref{thm:opFSi}}
      \\&= \frac{\tau}{\sqrt{2\pi}}\sum_{m\in\Z} \ff(m\tau) e^{-i\omega m\tau}
        && \text{let $m\eqd-k$}
    \end{align*}

\end{enumerate}
\end{proof}


%---------------------------------------
\begin{remark}
%---------------------------------------
  The left hand side of the \fncte{Poisson Summation Formula} \xref{thm:psf}
  is very similar to the \fncte{Zak Transform} $\opZ$:
  \footnote{
    \citerp{janssen1988}{24},
    \citerp{zayed}{482}
    %\url{http://mathworld.wolfram.com/ZakTransform.html}
    }
  \\\indentx$\ds(\opZ\ff)(t,\omega) \eqd \sum_{n\in\Z} \ff(x+n\tau)e^{i2\pi n\omega}$
\end{remark}

%---------------------------------------
\begin{remark}
%---------------------------------------
A generalization of the \fncte{Poisson Summation Formula} \xref{thm:psf} is the
\hid{Selberg Trace Formula}.\footnote{
  \citerp{lax}{349},
  \citor{selberg1956},
  \citer{terras1999}
  }
\end{remark}

%=======================================
\section{Basis theory properties}
%=======================================
%--------------------------------------
\begin{example}[\exmd{linear functions}]
\footnote{
  \citerpgc{higgins1996}{2}{0198596995}{1.1 General introduction}
  }
\label{ex:TD_flinear}
%--------------------------------------
Let $\opTrn$ be the \structe{translation operator} \xref{def:opT}.
%Let $\ff$ be a function in $\spLLR$.
Let $\hxs{\clLcc}$ be the set of all \prope{linear} functions in $\spLLR$.
\exbox{\begin{array}{FMD}
  1. & $\setn{x,\,\opTrn x}$ is a \structe{basis} for $\clLcc$ & and \\
  2. & $\ff(x) = \ff(1)x - \ff(0)\opTrn x$                     & $\forall\ff\in\clLcc$
\end{array}}
\end{example}
\begin{proof}
By left hypothesis, $\ff$ is \prope{linear}; so let $\ff(x)\eqd ax + b$
\begin{align*}
  \ff(1)x - \ff(0)\opTrn x
    &= \ff(1)x - \ff(0)(x-1)
    && \text{by \prefp{def:opT}}
  \\&= \left.(ax+b)\right|_{x=1}x - \left.(ax+b)\right|_{x=0}(x-1)
    && \text{by left hypothesis and definition of $\ff$}
  \\&= (a+b)x - b(x-1)
  \\&= ax+bx-bx+b
  \\&= ax+b
  \\&= \ff(x)
    && \text{by left hypothesis and definition of $\ff$}
\end{align*}
\end{proof}

%--------------------------------------
\begin{example}[\exmd{Cardinal Series}]
\label{ex:TD_cardinalseries}
%--------------------------------------
Let $\opTrn$ be the \structe{translation operator} \xrefP{def:opT}.
The \prope{Paley-Wiener} class of functions $\hxs{\spPW_\sigma^2}$\ifsxref{interpo}{def:PW}
are those functions which are ``\hie{bandlimited}"
with respect to their Fourier transform\ifsxref{harFour}{def:ft}.
The cardinal series forms an orthogonal basis for such a space\ifsxref{interpo}{thm:cardinalSeries}.
The \fncte{Fourier coefficients}\ifsxref{frames}{def:fcoef} for a projection of a function $\ff$ onto the Cardinal series basis elements is particularly
simple---these coefficients are samples of $\ff(x)$ taken at regular intervals\ifsxref{interpo}{thm:t_sampling}.
In fact, one could represent the coefficients using inner product notation with the
\structe{Dirac delta distribution} $\delta$ \ifxref{relation}{def:dirac} as
follows: %\footnote{see \prefp{sec:cardinal} for more details}
  \\\indentx$\ds\inprod{\ff(x)}{\opTrn^n\delta(x)} \eqd \int_{\R} \ff(x)\delta(x-n) \dx \eqd \ff(n)$
\exbox{\begin{array}{FMD}
  1. & $\ds\setxZp{\opTrn^n\frac{\sin\brp{\pi x}}{\pi x}}$ is a \structe{basis} for $\spPW_\sigma^2$ & and\\
  2. & $\ds\ff(x) = \mcom{\sum_{n=1}^\infty \ff(n) \opTrn^n\frac{\sin\brp{\pi x}}{\pi x}}{\structe{Cardinal series}}$ & $\forall \ff\in\spPW_\sigma^2,\,\sigma\le \half $
\end{array}}
\end{example}
\ifdochas{interpo}{\begin{proof}
See \prefp{thm:cardinalSeries}.
\end{proof}}

%--------------------------------------
\begin{example}[\exmd{Fourier Series}]
\label{ex:TD_fs}
%--------------------------------------
%Define an alternative dilation operator $\opDil_n$ as
%\\\indentx$\ds\brs{\opDil_n\ff}(x)\eqd\ff(nx)$, $n\in\Z$.
\exbox{\begin{array}{Frc>{\ds}lCD}
  (1). & \mc{4}{M}{$\ds\setxZ{\opDil_n e^{ix}}$ is a \structe{basis} for $\spL\intoo{0}{2\pi}$}         & and\\
  (2). & \ff(x)   &=&    \cft \sum_{n\in\Z} \alpha_n \opDil_n e^{ix}            & \forall x\in\intoo{0}{2\pi},\,\ff\in\spL\intoo{0}{2\pi}   & where\\
       & \alpha_n &\eqd& \cft \int_0^{2\pi} \ff(x) \opDil_n e^{-ix} \dx         & \forall\ff\in\spL\intoo{0}{2\pi}
\end{array}}
\end{example}
\begin{proof}
See \prefp{thm:opFSi}.
\end{proof}

%--------------------------------------
\begin{example}[\exmd{Fourier Transform}]
\label{ex:TD_ft}
\footnote{cross reference: \prefp{def:opFT}}
%--------------------------------------
%Define an alternative dilation operator $\opDil_\omega$ as
%\\\indentx$\ds\brs{\opDil_\omega\ff}(x)\eqd\ff(\omega x)$, $\omega\in\R$.
\exbox{\begin{array}{Frc>{\ds}lCD}
  (1). & \mc{4}{M}{$\ds\set{\opDil_\omega e^{ix}}{\scy\omega\in\R}$ is a \structe{basis} for $\spLLR$}         & and\\
  (2). & \ff(x)      &=&    \cft \int_\R \Ff(\omega) \opDil_x e^{i\omega} \dw  & \forall\ff\in\spLLR   & where\\
       & \Ff(\omega) &\eqd& \cft \int_\R \ff(x) \opDil_\omega e^{-ix} \dx  & \forall\ff\in\spLLR
\end{array}}
\end{example}

%--------------------------------------
\begin{example}[\exmd{Gabor Transform}]
\footnote{
  \citeP{gabor1946},
  \citergc{qian1996}{0132543842}{Chapter 3},
  \citerpgc{forster2009}{32}{0817648909}{Definition 1.69}
  }
\label{ex:TD_gt}
%--------------------------------------
%Define an alternative dilation operator $\opDil_\omega$ and an alternative translation operator $\opTrn_\tau$ as
%\\\indentx$\brs{\opDil_\omega\ff}(x)\eqd\ff(\omega x)$, \qquad and\qquad $\brs{\opTrn_\tau\ff}(x)\eqd\ff(x-\tau)$,\qquad$\scy\omega,\tau\in\R$.
\exbox{\begin{array}{Frc>{\ds}lCD}
  (1). & \mc{4}{M}{$\ds\setbigleft{\brp{\opTrn_\tau e^{-\pi x^2}}\brp{\opDil_\omega e^{ix}}}{\scy\tau,\omega\in\R}$ is a \structe{basis} for $\spLLR$}         & and\\
  (2). & \ff(x)                  &=&    \int_\R \fG\opair{\tau}{\omega} \opDil_x e^{i\omega} \dw & \forall x\in\R,\,\ff\in\spLLR   & where\\
       & \fG\opair{\tau}{\omega} &\eqd& \int_\R \ff(x) \brp{\opTrn_\tau e^{-\pi x^2}}\brp{\opDil_\omega e^{-ix}} \dx & \forall x\in\R,\,\ff\in\spLLR
\end{array}}
\end{example}

%--------------------------------------
\begin{example}[\exmd{wavelets}]
\label{ex:TD_wavelets}
%--------------------------------------
Let $\fpsi(x)$ be a \fncte{wavelet}.
\exbox{\begin{array}{Frc>{\ds}lCD}
  (1). & \mc{4}{M}{$\ds\setbigleft{\opDil^k\opTrn^n \fpsi(x)}{\scy k,n\in\Z}$ is a \structe{basis} for $\spLLR$}         & and\\
  (2). & \ff(x)    &=&    \sum_{k\in\Z}\sum_{n\in\Z} \alpha_{k,n} \opDil^k\opTrn^n\fpsi(x) & \forall \ff\in\spLLR        & where\\
       & \alpha_n  &\eqd& \int_\R \ff(x) \opDil^k\opTrn^n\fpsi^\ast(x) \dx & \forall\ff\in\spLLR
\end{array}}
\end{example}

%%\pref{def:opDalpha} gives some generalizations of \prefp{def:opD}, as well as some additional operators used
%%in the remaining examples in this section.
%%---------------------------------------
%\begin{definition}
%\footnote{
%  \citerppgc{walnut2002}{79}{80}{0817639624}{Definition 3.39}\\
%  \citerppg{christensen2003}{41}{42}{0817642951}\\
%  \citerpg{kammler2008}{A-21}{0521883407}\\
%  \citerpg{bachman2000}{473}{0387988998}\\
%  \citerpg{packer2004}{260}{0821834029}\\ %{section 3.1}\\
%  \citerpg{zayed2004}{}{0817643044}\\
%  \citerpgc{heil2011}{250}{0817646868}{Notation 9.4}\\
%  \citerpg{casazza1998}{74}{0817639594}\\
%  \citerp{goodman1993}{639}\\
%  \citerp{dai1996}{81}\\
%  \citerpg{dai1998}{2}{0821808001}
%  %\citerpg{dai1998}{21}{0821808001}
%  }
%\label{def:opDalpha}
%\label{def:opE}
%\label{def:opM}
%%---------------------------------------
%%Let $\C$ be the set of complex numbers,
%%and $\hxs{\spLLR}$ the set of all functions with range $\C$ and domain $\C$.
%%We define the following operators in $\spLLR$.
%%Define the operators $\opTrn:\spLLR\to\spLLR$ and $\opDil:\spLLR\to\spLLR$ as follows:
%\defbox{\begin{array}{M}
%  %The operator $\opTrn$ is a \hid{translation operator} and $\opDil$ is a \hid{dilation operator} if
%  %\\\indentx
%  $\begin{array}{F >{\ds}lc>{\ds}l @{\qquad}C@{\qquad}D@{\qquad}D@{\qquad}rcl}
%    %1. & \hxs{\opTrn_\alpha}  \ff(x) &\eqd&         \ff(x-\alpha)
%    %   & \forall \ff\in\spLLR
%    %   & (\hid{generalized translation operator})
%    %   & and
%    %   & \hxs{\opTrn}&\eqd& \opTrn_1
%    %   \\
%    1. & \hxs{\opDil_\alpha}  \ff(x) &\eqd& \sqrt{\alpha}\ff(\alpha x)
%       & \forall \ff\in\spRR
%       & (\hid{generalized dilation operator})
%      %& and
%      %& \hxs{\opDil}&\eqd& \opDil_2
%       \\
%    2. & \hxs{\opE_\alpha}  \ff(x) &\eqd& e^{i2\pi\alpha x}\ff(x)
%       & \forall \ff\in\spRR
%       & (\hid{modulation operator})
%      %& and
%      %& \hxs{\opE}&\eqd& \opE_1
%       \\
%    %3. & \hxs{\opM_\ff}  \ff(x) &\eqd& \brs{\ff\circ\ff}(x)
%    %   & \forall \ff,\ff\in\spLLR
%    %   & (\hid{multiplication operator})
%    %  %& and
%    %  %& \hxs{\opM}&\eqd& \opM_x
%  \end{array}$
%  %\\
%  %Moreover, $$, $\opDil\eqd\opDil_2$, $\opE\eqd\opE_1$, and $\opM\eqd\opM_x$.
%\end{array}}
%\end{definition}
%
%
%
%%--------------------------------------
%\begin{example}[\exm{Poisson Summation Formula}]
%\label{ex:psf}
%\index{Poisson Summation Formula}
%\index{theorems!Poisson Summation Formula}
%%--------------------------------------
%Let $\opE$ be the \structe{modulation operator} and $\opTrn$ the \structe{translation operator} \xrefP{def:opT}.
%Let $\Ff(\omega)$ be the Fourier transform \ifxref{harFour}{def:opFT} of a function $\ff(x)$.
%\exbox{
%  %\mcom{\ds \sum_{n\in\Z} \Ff\brp{{2\pi}n} e^{i{2\pi}nt}}
%  \mcom{\ds \sum_{n\in\Z} \opE_n\Ff\brp{{2\pi}n}}
%       {modulated summation in ``frequency"}
%  =
%  \mcom{\ds\cft \sum_{n\in\Z} \opTrn^n\ff(x)}
%       {summation in ``time"}
%  \qquad\scy\forall \ff\in\spLLR
%  }
%\end{example}
%\begin{proof}
%See \prefp{thm:psf}.
%\end{proof}
%
%%%--------------------------------------
%%\begin{example}[\exm{Legendre Polynomials}]
%%\footnote{
%%  \citerpgc{jackson1941}{47}{0486438082}{(5)}
%%  }
%%\label{ex:legendre_polynomials}
%%%---------------------------------------
%%Let $\opM$ be the \structe{multiplication operator} \xrefP{def:opM}.
%%Let $\fP_n(x)$ be the \structe{$n$th order Legendre polynomial}.
%%\exbox{
%%  \fP_{n+1}(x)   = 2x\fP_n(x) - \frac{n}{n+1}\fP_{n-1}(x)
%%  }
%%\end{example}
%
%%--------------------------------------
%\begin{example}[\exm{B-splines}]
%\label{ex:bspline_recursion}
%%---------------------------------------
%Let $\opM$ be the \structe{multiplication operator} and $\opTrn$ the \structe{translation operator} \xrefP{def:opT}.
%Let $\fN_n(x)$ be the \structe{$n$th order cardinal B-spline}\ifsxref{spline}{def:Nn}.
%\exbox{
%  \fN_n(x)   = \frac{1}{n}x\fN_{n-1}(x) - \frac{1}{n}x\opTrn\fN_{n-1}(x) + \frac{n+1}{n}\opTrn\fN_{n-1}(x)  \qquad\scy\forall n\in\Znn\setd\setn{1},\,  \forall x\in\R
%  }
%\end{example}
%\begin{proof}
%See \prefp{thm:bspline_recursion}.
%\end{proof}
%
%%--------------------------------------
%\begin{example}[\exm{Fourier Series analysis}]
%\label{ex:fs}
%%---------------------------------------
%Let $\opDil_\alpha$ be the \structe{dilation operator} and $\opE$ the \structe{modulation operator} \xrefP{def:opE}.
%Let $\opFS$ be the Fourier Series operator\ifsxref{harPoly}{def:opFS}.
%\exbox{\begin{array}{M}
%  The \structe{inverse Fourier Series} operator $\opFSi$ is given by
%  \\\indentx
%  $\ds \ff(n) = \sum_{n\in\Z} \Ff(n) \frac{1}{\sqrt{n}}\opDil_n e^{-i2\pi t} = \sum_{n\in\Z} \opE_n \Ff(n)
%  \qquad\scy\forall\ff\in\spLLR$
%  \\where
%  \\\indentx
%  $\ds\Ff(\omega) \eqd \int_0^1 \opE_{-n} \ff(x) \dx$
%\end{array}}
%\end{example}
%\begin{proof}
%See \prefp{thm:opFSi}.
%\end{proof}



%  %============================================================================
% Daniel J. Greenhoe
% LaTeX File
%============================================================================


%======================================
\chapter{Wavelet Structures}
%======================================

\qboxnpqt
  { Jules Henri Poincar\'e (1854-1912), physicist and mathematician
    \index{Poincar\'e, Jules Henri}
    \index{quotes!Poincar\'e, Jules Henri}
    \footnotemark
  }
  {../common/people/small/poincare.jpg}
  {\ldots on fait la science avec des faits comme une maison avec des pierres ; 
   mais une accumulation de faits n'est pas plus une science qu'un tas de 
   pierres n'est une maison.}
  {Science is built up of facts, as a house is built of stones;
   but an accumulation of facts is no more a science than a heap of stones is a house.}
  \citetblt{
    quote:       & \citerc{poincare_sah}{Chapter IX, paragraph 7} \\
    translation: & \citerp{poincare_sah_eng}{141} \\
    image:       & \url{http://www-groups.dcs.st-and.ac.uk/~history/PictDisplay/Poincare.html}
    }

\qboxnps
  {
    Freeman Dyson (1923--), physicist and mathematician  %(January 1994)
    \index{Dyson, Freeman}
    \index{quotes!Dyson, Freeman}
    \footnotemark
  }
  %{../common/people/dyson_.flickr8168451.jpg}
  {../common/people/dyson_isepp-org_95-96.jpg}  %http://www.isepp.org/Media/Speaker%20Images/95-96%20Images/dyson.jpg
  %{../common/people/small/dyson.jpg}
  {The bottom line for mathematicians is that the architecture has to be right.
    In all the mathematics that I did, the essential point was to find
    the right architecture.
    It's like building a bridge.
    Once the main lines of the structure are right,
    then the details miraculously fit.
    The problem is the overall design.}
  \citetblt{
    quote: & \citerp{dyson1994}{20}  \\
    %image: & \url{http://www.flickr.com/photos/russnelson/8168451/}
    image: & \scs\url{http://www.isepp.org/Media/Speaker\%20Images/95-96\%20Images/dyson.jpg}
    }

%=======================================
\section{Introduction}
%=======================================
%=======================================
\subsection{What are wavelets?}
%=======================================
In Fourier analysis, \prope{continuous} {dilations} \xref{def:opD} of the \fncte{complex exponential} \xref{def:exp}
form a  \structe{basis} \xref{def:basis_schauder} for the \structe{space of square integrable functions} $\spLLR$ \xref{def:spLLR} 
such that
  \\\indentx$\ds\spLLR=\linspan\set{\opDil_\omega e^{ix}}{\scy\omega\in\R}$.

In Fourier series analysis \xref{thm:opFSi}, \prope{discrete} dilations of the complex exponential 
form a  basis for $\spLL{\intoo{0}{2\pi}}$ such that
  \\\indentx$\ds\spLL{\intoo{0}{2\pi}}=\linspan\setjZ{\opDil_j e^{ix}}$.

In Wavelet analysis, for some \fncte{mother wavelet} \xref{def:wavelet} $\fpsi(x)$,
  \\\indentx$\ds\spLLR=\linspan\set{\opDil_\omega\opTrn_\tau \fpsi(x)}{\omega,\tau\in\R}$.

However, the ranges of parameters $\omega$ and $\tau$ can be much reduced to the countable set $\Z$ resulting in
a \prope{dyadic} wavelet basis such that for some mother wavelet $\fpsi(x)$,
  \\\indentx$\ds\spLLR=\linspan\set{\opDil^j\opTrn^n \fpsi(x)}{j,n\in\Z}$.\\
This text deals almost exclusively with dyadic wavelets. 
Wavelets that are both \prope{dyadic} and \prope{compactly supported} have the attractive feature 
that they can be easily implemented in hardware or software by use of the 
\structe{Fast Wavelet Transform} \xref{fig:fwt}.

%\begin{minipage}{\tw-65mm}
%  In 1989, St{/'e}phane G. Mallat introduced the \structe{Multiresolution Analysis} (MRA, \prefp{def:mra})
%  method for wavelet construction. The MRA has become the dominate wavelet construction method.
%  This text uses the MRA method extensively, 
%  and combines the MRA ``scaling subspaces" \xref{def:mra} with ``wavelet subspaces" \xref{def:seqWn} 
%  to form a subspace structure as represented by the 
%  \structe{Hasse diagram} to the right.
%  The \structe{Fast Wavelet Transform} combines both sets of subspaces as well, 
%  providing the results of projections onto both wavelet and MRA subspaces.
%  %The MRA is not the only method of wavelet construction,
%  Moreover, P.G. Lemari{/'e} has proved that all wavelets with \prope{compact support} are generated by an MRA.\footnotemark
%\end{minipage}\hfill%
%\tbox{\includegraphics{graphics/latwav.pdf}}
%\footnotetext{
%  \citor{lemarie1990},
%  \citerpg{mallat}{240}{012466606X}
%  }

%=======================================
\subsection{Analyses}
%=======================================
%20171226%The MRA is an \hib{analysis} of the linear space $\spLLR$.
%20171226%An analysis of a linear space $\spX$ is any sequence $\seq{\spV_j}{j\in\Z}$ of linear subspaces of $\spX$.
%20171226%%A sequence $\seq{\spV_j}{j\in\Z}$ of linear subspaces of a linear space $\spX$
%20171226%%        is an \hib{analysis} of $\spX$.
%20171226%        %if  $\seq{\spV_j}{j\in\Z}$ is a partition of $\spX$.
%20171226%        The partial or complete reconstruction of $\spX$ from $\seq{\spV_j}{j\in\Z}$ is a \hib{synthesis}.%
%20171226%        \footnote{%
%20171226%          The word \hie{analysis} comes from the Greek word
%20171226%          {\fntagreek{>av'alusis}},
%20171226%          meaning ``dissolution" (\citerpc{perschbacher1990}{23}{entry 359}),
%20171226%          which in turn means
%20171226%          ``the resolution or separation into component parts"
%20171226%          (\citer{collins2009}, \scs\url{http://dictionary.reference.com/browse/dissolution})
%20171226%          }
%20171226%  An analysis is sometimes completely \hie{characterized} by a \hie{transform}.
%20171226%  For example, a Fourier analysis is a sequence of subspaces with sinusoidal bases.
%20171226%  Examples of subspaces in a Fourier analysis include $\spV_1=\Span\setn{e^{ix}}$, 
%20171226%  $\spV_{2.3}=\Span\setn{e^{i2.3x}}$, $\spV_{\sqrt{2}}=\Span\setn{e^{i\sqrt{2}x}}$, etc.
%20171226%  A \hib{transform} is loosely defined as a function that maps a family of functions
%20171226%  into an analysis.
%20171226%  A very useful transform (a ``\hie{Fourier transform}") for Fourier Analysis is \xref{def:opFT}
%20171226%  \\\indentx$\ds\brs{\opFT\ff}(\omega) \eqd \frac{1}{\sqrt{2\pi}} \int_\R \ff(x) e^{-i\omega x} \dx$
%20171226%
%20171226%
%20171226%%  \item A sequence $\opT$ in $\clFxy$ is a \hib{transform} \label{item:wavstrct_T}
%20171226%%        if each element in the sequence is a projection operator in $\clFxy$.
%20171226%%        An example of a transform is the \hib{cosine transform} $\opT$ in $\clFrr$ such that
%20171226%%        \begin{align*}
%20171226%%          \opT\fx(t) &\eqd \seq{\opP_j}{j\in\Z}
%20171226%%             \\&\eqd \seq{\int_{t\in\R} \fx(t)\,\mcom{\cos(nt)}{kernel} \dt}{n\in\Z}
%20171226%%             \\&\eqd \seqn{\cdots,\,
%20171226%%                           %\int_{t\in\R} \cos\brs{(-2)t}\,\fx(t) \dt,\,
%20171226%%                           \int_{t\in\R} \fx(t)\,\cos\brs{(-1)t} \dt,\,
%20171226%%                           \int_{t\in\R} \fx(t)\,                \dt,\,
%20171226%%                           \int_{t\in\R} \fx(t)\,\cos\brs{( 1)t} \dt,\,
%20171226%%                           %\int_{t\in\R} \cos\brs{(-2)t}\,\fx(t) \dt,\,
%20171226%%                           \cdots
%20171226%%                          }
%20171226%%        \end{align*}
%20171226%%        Further examples of transforms include the \hie{Fourier Transform} and various \hie{Wavelet Transforms}.
%20171226%% it is a \hib{sequence} of projection operators on $\A function $\opT$ in $\clFxy$ is a \hib{transform} if with domain $\clFxy$ and range $\clF{\setA}{\setB}$ if

    \begin{minipage}{\tw-65mm}%
      An analysis can be partially characterized by its order structure with respect
      to an order relation such as the set inclusion relation $\subseteq$.
      Most transforms have a very simple M-$n$ order structure,
      as illustrated to the right.
      The M-$n$ lattices for $n\ge3$ are \prope{modular} but not \prope{distributive}.
      Analyses typically have one subspace that is a \hie{scaling} subspace;
      and this subspace is often simply a family of constants
      (as is the case with \hi{Fourier Analysis}).
    \end{minipage}%
    \hfill\tbox{\includegraphics{graphics/latmn.pdf}}%

    \begin{minipage}{\tw-55mm}%
      A special characteristic of wavelet analysis is that there is not just one
      scaling subspace,
      %(as is with the case of Fourier and other analyses),
      but an entire sequence of scaling subspaces.
      These scaling subspaces are \prope{linearly ordered} with respect to the
      ordering relation $\subseteq$. In wavelet theory, this structure is called a \structe{multiresolution analysis},
      or \structe{MRA} \xref{def:mra}.

     The MRA was introduced by St{/'e}phane G. Mallat in 1989.
     The concept of a scaling space was perhaps first introduced by Taizo Iijima in 1959 in Japan,
    and later as the \structe{Gaussian Pyramid} by Burt and Adelson in the 1980s in the West.\footnotemark
    \end{minipage}%
    \footnotetext{%
      \citorp{mallat89}{70},
      \citor{iijima1959},
      \citor{burt1983},
      \citor{adelson1981},
      \citer{lindeberg1993},
      \citer{alvertez1993},
      \citer{guichard2012},
      \citerc{weickert1999}{historical survey}
      }
    \hfill\tbox{\includegraphics{graphics/latmra.pdf}}%

    \begin{minipage}{\tw-65mm}%
      A second special characteristic of wavelet analysis is that it's order structure
      with respect to the $\subseteq$ relation is not a simple M-$n$ lattice 
     (as is with the case of Fourier and other analyses).
      Rather, it is a lattice of the form illustrated to the right.
      This lattice is \prope{non-complemented}, \prope{non-distributive},
      \prope{non-modular}, and \prope{non-Boolean} \xref{prop:order_wavstrct}.
    \end{minipage}%
    \hfill\tbox{\includegraphics{graphics/latwav.pdf}}%

    \begin{minipage}{\tw-65mm}%
      The wavelet subspace structure is similar in form to that of the \structe{Primorial numbers},\footnotemark
      illustrated to the right by a \hie{Hasse diagram}.
      %In the world of mathematical structures,
      %there is circumstantial evidence that the order structure of wavelet analyses is quite rare,
      %if not outright unique.
      %For example, suppose we replace the wavelet subspaces with prime numbers
      %and the scaling subspaces with their products as illustrated to the right.
      %The resulting sequence $\seqn{1,\,2,\,6,\,30,\,210}$ as of 2011 July 30
      %has no matches in Neil J.A. Sloane's  \emph{Online Encyclopedia of Integer Sequences}
      %(hosted by \emph{AT\&T Research}).\footnotemark
    \end{minipage}%
    \citetblt{%
      \citeoeis{A002110}%
      }%
    \hfill\tbox{\includegraphics{graphics/latp_1235711.pdf}}%

  An analysis can be represented using three different structures:
    %\paragraph{Equivalence of lattice representations.}
    %So far we have discussed representing a wavelet analysis using three different structures:
\\\begin{tabular}{@{\qquad}ll}
  \circOne    & sequence of subspaces             \\
  \circTwo    & sequence of basis coefficients         \\
  \circThree  & sequence of basis vectors
\end{tabular}\\
These structures are isomorphic to each other, and can therefore be used interchangeably.
%(see \prefp{thm:VPb_isomorphic}).
%(see \prefp{fig:wav_VPb_isomorphic}).
%That is, a ``\hie{wavelet analysis}" can be described using any of these structures.
%However, sometimes when introducing theorems about wavelets,
%it is convenient to use elements from not just one, but from multiple lattices;
%and so it is convenient to have a ``collection" of wavelet analysis elements
%all assembled together into one formally defined tuple.
%\pref{def:wavsys} (next) does just that---it defines a \hie{wavelet analysis} in terms of a tuple with elements
%extracted from the four wavelet structures.

%---------------------------------------
% isomorphic lattices
%---------------------------------------
\mbox{}\hfill
  \tbox{\includegraphics{graphics/latp_VW.pdf}}\qquad{\Large$\thapprox$}\qquad
  \tbox{\includegraphics{graphics/latp_hg.pdf}}\qquad{\Large$\thapprox$}\qquad
  \tbox{\includegraphics{graphics/latp_pp.pdf}}
\hfill\mbox{}

\prefpp{fig:analyses} illustrate the order structures of some analyses,
        including two wavelet analyses:
\begin{figure}[th]
  \centering%
  \begin{tabular}{|c|c|}%
    \hline%
    \mc{1}{B}{Cosine analysis  (even Fourier series)}&\mc{1}{B}{Cosine polynomial analysis}%
    \\\includegraphics{graphics/baslat_cosh.pdf}&\includegraphics{graphics/baslat_cose.pdf}%
    \\\hline%
    \mc{1}{|B|}{Chebyshev polynomial analysis\cittrp{rivlin1974}{4}}&\mc{1}{|B|}{Hadamard-3 analysis}%
    \\\includegraphics{graphics/baslat_cheby.pdf}&\includegraphics{graphics/baslat_h3.pdf}%
    \\\hline
    \mc{1}{|B|}{Haar/Daubechies-$p1$ wavelet analysis} & \mc{1}{B|}{Daubechies-$p2$ wavelet analysis}%
    \\\includegraphics{graphics/baslat_d1.pdf}&\includegraphics{graphics/baslat_d2.pdf}%
    \\\hline%
  \end{tabular}%
  \caption{examples of the order structures of some analyses\label{fig:analyses}}
\end{figure}


%20171226%%=======================================
%20171226%\section{Multiresolution analysis}
%20171226%%=======================================
%20171226%%=======================================
%20171226%\subsection{Definition}
%20171226%%=======================================
%20171226%\ifdochasnot{transop}{%
%20171226%Much of the wavelet theory developed in this section is constructed using the \opb{translation operator} $\opTrn$
%20171226%and the \opb{dilation operator} $\opDil$ (next).
%20171226%%These operators are illustrated below.
%20171226%%and in \prefp{fig:haar-jn} for the \hie{Haar} MRA\cittr{haar1910}.
%20171226%%
%20171226%%---------------------------------------
%20171226%\begin{definition}
%20171226%\footnote{
%20171226%  \citerppgc{walnut2002}{79}{80}{0817639624}{Definition 3.39},
%20171226%  \citerppg{christensen2003}{41}{42}{0817642951},
%20171226%  \citerpgc{wojtaszczyk1997}{18}{0521578949}{Definitions 2.3,2.4},
%20171226%  \citerpg{kammler2008}{A-21}{0521883407},
%20171226%  \citerpg{bachman2000}{473}{0387988998},
%20171226%  \citerpg{packer2004}{260}{0821834029}, %{section 3.1}\\
%20171226%  \citerpg{zayed2004}{}{0817643044},
%20171226%  \citerpgc{heil2011}{250}{0817646868}{Notation 9.4},
%20171226%  \citerpg{casazza1998}{74}{0817639594},
%20171226%  \citerp{goodman1993}{639},
%20171226%  \citerp{dai1996}{81},
%20171226%  \citerpg{dai1998}{2}{0821808001}
%20171226%  %\citerpg{dai1998}{21}{0821808001}
%20171226%  }
%20171226%\label{def:wav_opT}
%20171226%\label{def:wav_opD}
%20171226%\label{def:wavstrct_TD}
%20171226%\label{def:opT}
%20171226%\label{def:opD}
%20171226%%---------------------------------------
%20171226%%Let $\spX\eqd\spLLR$ be the space of all \structe{square Lebesgue integrable functions} \xref{def:spLLR}.
%20171226%%Let $\opTrn$ and $\opDil$ be operators in $\clOxx$.
%20171226%%Let $\C$ be the set of complex numbers,
%20171226%%and $\hxs{\spLLR}$ the set of all functions with range $\C$ and domain $\C$.
%20171226%\defbox{\begin{array}{Frc>{\ds}lMlC}
%20171226%    1. & \mc{6}{M}{$\hxs{\opTrn}$ is the \opd{translation operator} on $\spCC$ defined as}\\
%20171226%       & \indentx\hxs{\opTrn_\tau}  \ff(x) &\eqd&         \ff(x-\tau) &and& \hxs{\opTrn}\eqd\opTrn_1 & \forall \ff\in\spCC
%20171226%       \\
%20171226%    2. & \mc{6}{M}{$\hxs{\opDil}$ is the \opd{dilation operator} on $\spCC$ defined as}\\
%20171226%       & \indentx\hxs{\opDil_\alpha} \ff(x) &\eqd& \ff(\alpha x) &and& \opDil\eqd\sqrt{2}\opDil_2 & \forall \ff\in\spCC
%20171226%\end{array}}
%20171226%\end{definition}
%20171226%
%20171226%\begin{center}\begin{tabular}{cc}
%20171226%  \includegraphics{graphics/opTrn.pdf}&\includegraphics{graphics/opDil.pdf}%
%20171226%\end{tabular}\end{center}
%20171226%}
%20171226%
%20171226%
%20171226%
%20171226%A multiresolution analysis provides ``coarse" approximations of a function in a linear space $\spLLR$ at multiple
%20171226%``scales" or ``resolutions".
%20171226%%\paragraph{Scaling function.}
%20171226%Key to this process is a sequence of \hie{scaling functions}.
%20171226%Most traditional transforms feature a single \hie{scaling function} $\fphi(x)$
%20171226%set equal to one ($\fphi(x)=1$).
%20171226%This allows for convenient representation of the most basic functions, such as constants.\citep{jawerth}{8}
%20171226%A multiresolution system, on the other hand, uses a generalized form of the scaling concept:
%20171226%\begin{dingautolist}{"AC}
%20171226%  \item Instead of the scaling function simply being set \emph{equal to unity} ($\fphi(x)=1$),
%20171226%        a multiresolution system \xref{def:mrasys} is often constructed in such a way that the scaling function 
%20171226%        $\fphi(x)$ forms a \hie{partition of unity} \xref{def:pun} such that
%20171226%        $\sum_{n\in\Z} \opTrn^n\fphi(x) = 1$.
%20171226%  \item Instead of there being \emph{just one} scaling function, there
%20171226%        is an entire sequence of scaling functions $\seqjZ{\opDil^j\fphi(x)}$, 
%20171226%        each corresponding to a different ``\hie{resolution}".
%20171226%\end{dingautolist}
%20171226%
%20171226%%--------------------------------------
%20171226%\begin{definition}% [multiresolution system]
%20171226%\footnote{
%20171226%  \citerpg{hernandez1996}{44}{0849382742},
%20171226%  \citerpgc{mallat}{221}{012466606X}{Definition 7.1},
%20171226%  \citorp{mallat89}{70},
%20171226%  \citorpgc{meyer1992}{21}{0521458692}{Definition 2.2.1},
%20171226%  \citerpgc{christensen2003}{284}{0817642951}{Definition 13.1.1},
%20171226%  \citerppgc{bachman2000}{451}{452}{0387988998}{Definition 7.7.6},
%20171226%  \citerppgc{walnut2002}{300}{301}{0817639624}{Definition 10.16},
%20171226% %\citerppgc{vidakovic}{51}{52}{0471293652}{Riesz basis: footnote on page 52}\\
%20171226%  \citerppgc{dau}{129}{140}{0898712742}{Riesz basis: page 139}
%20171226%  %\citerppgc{christensen2003}{73}{74}{0817642951}{Definition 3.8.2}\\
%20171226%  %\citerpgc{heil2011}{371}{0817646868}{Definition 12.8}\\
%20171226%  %\citerpgc{walter}{38}{1584882271}{3.1 Multiresolution Analysis}
%20171226%  }
%20171226%\label{def:seqVn}
%20171226%\label{def:mra}
%20171226%\label{def:wavstrct_phi}
%20171226%%--------------------------------------
%20171226%%Let $\spLLR$ be the space of all \structe{square Lebesgue integrable functions} \xref{def:spLLR}.
%20171226%Let $\seqjZ{\spV_j}$ be a sequence of subspaces on $\spLLR$ \xref{def:spLLR}.  %be a \prope{separable} \structe{Hilbert space}.
%20171226%Let $\clsA$ be the \structe{closure} of a set $\setA$.
%20171226%\\\defboxt{
%20171226%  The sequence $\seqjZ{\spV_j}$ is a \structd{multiresolution analysis} on $\spLLR$ if
%20171226%  \\
%20171226%  $\begin{array}{@{\qquad}F>{\ds}lCDD}
%20171226%   %1. & \spV_j \text{ is a linear subspace of $\spX$}\qquad \forall \spV_j\in\seqjZ{\spV_j}
%20171226%    %\cnto & \mc{2}{M}{$\spLLR$ is \prope{complete}}                       & ($\spLLR$ is a \structe{Hilbert Space})         & and
%20171226%    %\cntn & \mc{2}{M}{$\spLLR$ is \prope{separable}}                      &                                               & and
%20171226%      1.  & \spV_j          = \cls{\spV_j}              & \forall j\in\Z                 & (\prope{closed})                              & and 
%20171226%    \\2.  & \spV_j          \subset \spV_{j+1}          & \forall j\in\Z                 & (\prope{linearly ordered})                    & and 
%20171226%    \\3.  & \clsp{\Setu_{j\in\Z} \spV_j} = \spLLR         &                                & (\prope{dense} in $\spLLR$)    & and 
%20171226%   %\cntn & \Seti_{j\in\Z} \spV_j = \setn{\vzero}       &                                & (\structe{greatest lower bound} is $\spZero$) & and 
%20171226%    \\4.  & \ff\in\spV_j \iff    \opDil\ff\in\spV_{j+1} & \forall j\in\Z,\,\ff\in\spLLR  & (\prope{self-similar})                        & and
%20171226%   %\cntn & \ff\in\spV_j \iff    \opTrn\ff\in\spV_j     & \forall n\in\Z,\,\ff\in\spLLR  & (\prope{translation invariant})               & and 
%20171226%    \\5.  & \mc{3}{l}{\ds\exists \fphi \st \setxZ{\opTrn^n\fphi} \text{ is a \structe{Riesz basis} for $\spV_0$.}}                       & 
%20171226%  \end{array}$
%20171226%  \\
%20171226%  A \structe{multiresolution analysis} is also called an \hid{MRA}.\\
%20171226%  An element $\spV_j$ of $\seqjZ{\spV_j}$ is a \hid{scaling subspace} of the space $\spLLR$.\\
%20171226%  The pair $\hxs{\MRAspaceLLRV}$ is a \hid{multiresolution analysis space}, or \hid{MRA space}.\\
%20171226%  The function $\hxs{\fphi}$ is the \hid{scaling function} of the \structe{MRA space}.
%20171226%  }
%20171226%\end{definition}
%20171226%
%20171226%The traditional definition of the \structe{MRA} also includes the following:
%20171226%  \\\indentx$\begin{array}{F>{\ds}lCD}
%20171226%      \cntn & \ff\in\spV_j \iff    \opTrn^n\ff\in\spV_j     & \forall n,j\in\Z,\,\ff\in\spLLR  & (\prope{translation invariant})
%20171226%      \cntn & \Seti_{j\in\Z} \spV_j = \setn{\vzero}         &                                  & (\structe{greatest lower bound} is $\spZero$)
%20171226%  \end{array}$\\
%20171226%However, \pref{prop:mra_transinvar} (next) demonstrates that
%20171226%both of these follow from the \structe{MRA} as defined in \pref{def:mra}.
%20171226%
%20171226%%--------------------------------------
%20171226%\begin{proposition}
%20171226%\footnote{
%20171226%  \citerpgc{hernandez1996}{45}{0849382742}{Theorem 1.6},
%20171226%  \citerppgc{wojtaszczyk1997}{19}{28}{0521578949}{Proposition 2.14},
%20171226%  \citerppgc{pinsky2002}{313}{314}{0534376606}{Lemma 6.4.28}
%20171226%  }
%20171226%\label{prop:mra_transinvar}
%20171226%\label{prop:mra_glb}
%20171226%%--------------------------------------
%20171226%\propbox{
%20171226%  \brbr{\begin{array}{N}
%20171226%    $\seqjZ{\spV_j}$ is an \structe{MRA}\\
%20171226%    \xref{def:mra}
%20171226%  \end{array}}
%20171226%  \implies
%20171226%  \brbl{\begin{array}{F>{\ds}lCD}
%20171226%    1. & \ff\in\spV_j \iff    \opTrn^n\ff\in\spV_j   & \forall n,j\in\Z,\,\ff\in\spLLR  &(\prope{translation invariant}) and\\
%20171226%    2. & \Seti_{j\in\Z} \spV_j = \setn{\vzero}       &                                  & (\structe{greatest lower bound} is $\spZero$)
%20171226%  \end{array}}
%20171226%  }
%20171226%\end{proposition}
%20171226%\begin{proof}
%20171226%%\begin{enumerate}
%20171226%Proof for (1):
%20171226%\begin{align*}
%20171226%  &\opTrn^n\ff\in\spV_j
%20171226%  \\&\iff \opTrn^n\ff\in\linspan\set{\opDil^j\opTrn^m\fphi}{\scy m\in\Z}
%20171226%          &&
%20171226%          && \text{by definition of $\setn{\fphi}$} && \text{\xref{def:mra}}
%20171226%  \\&\iff \exists \seqxZ{\alpha_n} \st \opTrn^n\ff(x)
%20171226%          &&= \sum_{k\in\Z}\alpha_k\opDil^j\opTrn^k\fphi(x)
%20171226%          && \text{by definition of $\setn{\fphi}$} && \text{\xref{def:mra}}
%20171226%  \\&\iff \exists \seqxZ{\alpha_n} \st \ff(x)
%20171226%          &&= \opTrn^{-n}\sum_{k\in\Z}\alpha_k\opDil^j\opTrn^k\fphi(x)
%20171226%          && \text{by definition of $\opTrn$} && \text{\xref{def:opT}}
%20171226%  \\&     &&= \sum_{k\in\Z}\alpha_k\opTrn^{-n}\opDil^j\opTrn^k\fphi(x)
%20171226%          %&& \text{by \prefp{prop:opTD_sum}}
%20171226%  \\&     &&= \sum_{k\in\Z}\alpha_k\opDil^j\opTrn^{k-2n}\fphi(x)
%20171226%          && \text{by \thme{commutator relation}} && \text{\xref{prop:DTTD}}
%20171226%  \\&     &&= \sum_{\ell\in\Z}\alpha_{\ell+2n}\opDil^j\opTrn^{\ell}\fphi(x)
%20171226%          && \text{where $\ell\eqd k-2n\implies$} &&k=\ell+2n
%20171226%  \\&     &&= \sum_{\ell\in\Z}\beta_{\ell}\opDil^j\opTrn^{\ell}\fphi(x)
%20171226%          && \text{where $\beta_{\ell}\eqd\alpha_{\ell+2n}$}
%20171226%  \\&     &&\iff \quad\ff\in\spV_j
%20171226%          && \text{by def. of $\setn{\opTrn^n\fphi}$} && \text{\xref{def:mra}}
%20171226%\end{align*}
%20171226%
%20171226%Proof for (2):
%20171226%\begin{enumerate}
%20171226%  \item Let $\opP_j$ be the \ope{projection operator} that generates the scaling subspace $\spV_j$ such that \label{idef:mra_glb_Pj}
%20171226%    \\\indentx$\ds\opV_j=\set{\opP_j\ff}{\ff\in\spLLR}$
%20171226%
%20171226%  \item lemma: Functions with \prope{compact support} are \prope{dense} in $\spLLR$.
%20171226%        \label{ilem:mra_glb_dense}
%20171226%        Therefore, we only need to prove that the proposition is true for functions with support in $\intcc{-R}{R}$, for all $R>0$.
%20171226%
%20171226%  \item For some function $\ff\in\spLLR$, let $\seqxZ{\ff_n}$ be a sequence of functions in $\spLLR$ 
%20171226%        with \prope{compact support} such that
%20171226%        \\\indentx
%20171226%        $\support\ff_n\subseteq\intcc{-R}{R}$ for some $R>0$
%20171226%        \quad and\quad
%20171226%        $\ds\ff(x)=\lim_{n\to\infty}\seqn{\ff_n(x)}$.
%20171226%        \label{idef:mra_glb_ffn}
%20171226%
%20171226%  \item lemma: $\ds\Seti\spV_j=\setn{\vzero}\quad\iff\quad\lim_{j\to-\infty}\norm{\opP_j\ff}=0\quad{\scy\forall\ff\in\spLLR}$. Proof:  \label{ilem:mra_glb_VjPj}
%20171226%    \begin{align*}
%20171226%      \Seti_{j\in\Z}\spV_j 
%20171226%        &= \Seti_{j\in\Z}\set{\opP_j\ff}{\ff\in\spLLR}
%20171226%        && \text{by definition of $\spV_j$} && \text{\xref{idef:mra_glb_Pj}}
%20171226%      \\&= \lim_{j\to-\infty}\set{\opP_j\ff}{\ff\in\spLLR}
%20171226%        && \text{by definition of $\seti$} && \text{\ifxref{setstrct}{def:setop}}
%20171226%      \\&= \vzero
%20171226%      \iff \lim_{j\to-\infty}\norm{\opP_j\ff}=0
%20171226%        && \text{by \prope{nondegenerate} property of $\normn$} && \text{\xref{def:norm}}
%20171226%    \end{align*}
%20171226%  
%20171226%  \item lemma: $\ds\lim_{j\to-\infty}\norm{\opP_j\ff}=0\quad{\scy\forall\ff\in\spLLR}$. Proof:\\
%20171226%        Let $\setindAx$ be the \fncte{set indicator function} \xref{def:setind} \label{ilem:mra_glb_norm}
%20171226%    \begin{align*}
%20171226%      &\lim_{j\to-\infty}\norm{\opP_j\ff}^2
%20171226%      \\&=   \lim_{j\to-\infty}\norm{\opP_j\lim_{n\to\infty}\seqn{\ff_n}}^2
%20171226%        &&   \text{by \prefp{idef:mra_glb_ffn}}
%20171226%      \\&\le \lim_{j\to-\infty}B\sum_{n\in\Z}\abs{\inprod{\opP_j\lim_{n\to\infty}\seqn{\ff_n}}{\opDil^j\opTrn^n\fphi}}^2
%20171226%        &&   \text{by \prope{frame property}} && \text{\xref{prop:rbasis_frame}}
%20171226%      \\&=   \lim_{j\to-\infty}B\sum_{n\in\Z}\abs{\inprod{\lim_{n\to\infty}\seqn{\ff_n}}{\opDil^j\opTrn^n\fphi}}^2
%20171226%        &&   \text{by definition of $\opP_j$} && \text{\xref{idef:mra_glb_Pj}}
%20171226%      \\&=   \lim_{j\to-\infty}B\sum_{n\in\Z}\abs{\inprod{\setind_\intcc{-R}{R}(x)\lim_{n\to\infty}\seqn{\ff_n}}{\opDil^j\opTrn^n\fphi(x)}}^2
%20171226%        &&   \text{by definition of $\seqn{\ff_n}$}&& \text{\xref{idef:mra_glb_ffn}}
%20171226%      \\&=   \lim_{j\to-\infty}B\sum_{n\in\Z}\abs{\inprod{\lim_{n\to\infty}\seqn{\ff_n}}{\setind_\intcc{-R}{R}(x)\opDil^j\opTrn^n\fphi(x)}}^2
%20171226%        &&   \text{prop. of $\inprodn$ in $\spLLR$}&& \text{ \xref{def:spLLR}}
%20171226%      \\&\le \lim_{j\to-\infty}B\sum_{n\in\Z}\norm{\lim_{n\to\infty}\seqn{\ff_n}}^2\norm{\setind_\intcc{-R}{R}(x)\opDil^j\opTrn^n\fphi(x)}^2
%20171226%        &&   \text{by \thme{CS Inequality}}&& \text{\ifxref{vsinprod}{thm:cs}}
%20171226%      \\&=   \lim_{j\to-\infty}B\sum_{n\in\Z}\norm{\ff}^2\norm{\setind_\intcc{-R}{R}(x)\opDil^j\opTrn^n\fphi(x)}^2
%20171226%        &&   \text{by definition of $\seqn{\ff_n}$}&& \text{\xref{idef:mra_glb_ffn}}
%20171226%      \\&=   \lim_{j\to-\infty}B\sum_{n\in\Z}\norm{\ff}^2\norm{\brs{\mcom{\opDil^j\opDil^{-j}}{$\opI$}\setind_\intcc{-R}{R}(x)}\brs{\opDil^j\opTrn^n\fphi(x)}}^2
%20171226%        &&   \text{by property of $\opDil$} && \text{\xref{prop:opDi}}
%20171226%      \\&=   \lim_{j\to-\infty}B\sum_{n\in\Z}\norm{\ff}^2\norm{2^{j/2}\opDil^j\brb{\brs{\opDil^{-j}\setind_\intcc{-R}{R}(x)}\brs{\opTrn^n\fphi(x)}}}^2
%20171226%        &&   \mathrlap{\text{by \prefp{prop:DjTnfg}}}
%20171226%      \\&=   \lim_{j\to-\infty}B\sum_{n\in\Z}\norm{\ff}^2\norm{\opDil^j\brb{2^{j/2}2^{-j/2}\setind_\intcc{-R}{R}(2^{-j}x)\brs{\opTrn^n\fphi(x)}}}^2
%20171226%        &&   \text{by property of $\opDil$} && \text{\xref{prop:opDi}}
%20171226%      \\&=   \lim_{j\to-\infty}B\sum_{n\in\Z}\norm{\ff}^2\norm{\opDil^j\brb{\brs{\mcom{\opTrn^n\opTrn^{-n}}{$\opI$}\setind_\intcc{-R}{R}(2^{-j}x)}\brs{\opTrn^n\fphi(x)}}}^2
%20171226%        &&   \text{by property of $\opTrn$} && \text{\xref{prop:opTi}}
%20171226%      \\&=   \lim_{j\to-\infty}B\sum_{n\in\Z}\norm{\ff}^2\norm{\opDil^j\brb{\brs{\opTrn^n\setind_\intcc{-R}{R}(2^{-j}x+n)}\brs{\opTrn^n\fphi(x)}}}^2
%20171226%        &&   \text{by property of $\opTrn$} && \text{\xref{prop:opTi}}
%20171226%      \\&=   \lim_{j\to-\infty}B\sum_{n\in\Z}\norm{\ff}^2\norm{\opDil^j\opTrn^n\brb{\setind_\intcc{-R}{R}(2^{-j}x+n)\fphi(x)}}^2
%20171226%        &&   \text{by property of $\opDil$} && \text{\xref{prop:opDi}}
%20171226%      \\&=   \lim_{j\to-\infty}B\sum_{n\in\Z}\norm{\ff}^2\norm{\setind_\intcc{-R}{R}(2^{-j}x+n)\fphi(x)}^2
%20171226%        &&   \text{by \prope{unitary} prop.} && \text{\xref{thm:TD_unitary}}
%20171226%      \\&=   B\norm{\ff}^2\sum_{n\in\Z}\lim_{j\to-\infty}\norm{\setind_\intcc{-2^jR+n}{2^jR+n}(u)\fphi(2^{-j}(u-n))}^2
%20171226%        &&   \text{$u\eqd 2^jx+n\implies$} && x=2^{-j}(u-n)
%20171226%      \\&=   B\norm{\ff}^2\sum_{n\in\Z}\lim_{j\to-\infty}\int_{-2^jR+n}^{2^jR+n}\abs{\fphi(2^{-j}(u-n))}^2\du
%20171226%      \\&=   B\norm{\ff}^2\sum_{n\in\Z}\int_{n}^{n}\abs{\fphi(0)}^2\du
%20171226%      \\&=   0
%20171226%    \end{align*}
%20171226%
%20171226%  \item Final step in proof that $\ds\Seti\spV_j=\setn{\vzero}$: by \prefp{ilem:mra_glb_VjPj} and \prefp{ilem:mra_glb_norm}
%20171226%%\end{enumerate}
%20171226%\end{enumerate}
%20171226%\end{proof}
%20171226%
%20171226%
%20171226%
%20171226%%---------------------------------------
%20171226%\begin{proposition}
%20171226%\citetbl{
%20171226%  \citerppgc{wojtaszczyk1997}{28}{31}{0521578949}{Proposition 2.15}
%20171226%  }
%20171226%\label{prop:mra_UVj}
%20171226%%--------------------------------------
%20171226%%Let $\spO\eqd\mrasys$.
%20171226%Let a \structe{Riesz sequence} be defined as in \prefp{def:rieszseq}.
%20171226%\propbox{
%20171226%  \brb{\begin{array}{FMD}
%20171226%    (1). & $\seqn{\opTrn^n\fphi}$ is a \structe{Riesz sequence} &  and \\
%20171226%    (2). & $\Fphi(\omega)$ is \prope{continuous} at $0$         &  and \\
%20171226%    (3). & $\Fphi(0)\neq0$                                      &  
%20171226%  \end{array}}
%20171226%  \implies
%20171226%  \brb{\begin{array}{>{\ds}lD}
%20171226%     \cls{\brp{\Setu_{j\in\Z} \spV_j}} = \spLLR  & (\prope{dense} in $\spLLR$) 
%20171226%  \end{array}}
%20171226%  }
%20171226%\end{proposition}
%20171226%\begin{proof}
%20171226%\begin{enumerate}
%20171226%  \item Let $\opP_j$ be the \ope{projection operator} that generates the scaling subspace $\spV_j$ such that \label{item:mra_UVj_Pj}
%20171226%    \\\indentx$\ds\opV_j=\set{\opP_j\ff}{\ff\in\spH}$
%20171226%
%20171226%  \item definition: Choose $\ff\in\spLLR$ such that $\ff\orthog\Setu_{j\in\Z}\spV_j$.
%20171226%        Let $\Ff(\omega)$ be the \ope{Fourier Transform} \xref{def:opFT} of $\ff(x)$.
%20171226%        \label{idef:mra_UVj_f}
%20171226%
%20171226%  \item lemma: The function $\ff$ \xref{idef:mra_UVj_f} \emph{exists} because the set of functions that 
%20171226%        can be chosen to be $\ff$ at least contains $0$ (it is not the emptyset). Proof:
%20171226%        \label{ilem:mra_UVj_fexists}
%20171226%        \begin{align*}
%20171226%          \ff(x)=0
%20171226%            &\implies \inprodr{\ff}{\set{\fh\in\spLLR}{\fh\in\Setu_{j\in\Z}\spV_j}}
%20171226%          \\&= \inprodr{0}{\set{\fh\in\spLLR}{\fh\in\Setu_{j\in\Z}\spV_j}}
%20171226%          \\&= 0
%20171226%          \\&\implies\quad \ff\orthog\Setu_{j\in\Z}\spV_j
%20171226%          \\&\implies\quad \text{$\ff$ exists}
%20171226%        \end{align*}
%20171226%
%20171226%  \item lemma: $\norm{\opP_j\ff}=0\quad{\scy\forall j\in\Z}$. Proof:
%20171226%        \label{ilem:mra_UVj_Pf}
%20171226%    \begin{align*}
%20171226%      \norm{\opP_j\ff}
%20171226%        &= \norm{0}
%20171226%        && \text{by definition of $\ff$} &&\text{\xref{idef:mra_UVj_f}}
%20171226%      \\&= 0
%20171226%        && \text{by \prope{nondegenerate} property of $\normn$} && \text{\ifxref{vsnorm}{def:norm}}
%20171226%    \end{align*}
%20171226%
%20171226%  \item definition: Choose some function $\fg\in\spLLR$ such that $\Fg(\omega)=\Ff(\omega)\setind_\intcc{-R}{R}$ \xref{def:setind} 
%20171226%        for some $R>0$ 
%20171226%        and such that $\norm{\ff-\fg}<\varepsilon$.
%20171226%        Let $\Fg(\omega)$ be the \ope{Fourier Transform} \xref{def:opFT} of $\fg(x)$.
%20171226%        \label{idef:mra_UVj_g}
%20171226%
%20171226%  \item lemma: The function $\fg$ \xref{idef:mra_UVj_g} \emph{exists}. Proof: For some (possibly very large) $R$,
%20171226%        \label{idef:mra_UVj_gexists}
%20171226%    \begin{align*}
%20171226%      \varepsilon
%20171226%        &> \norm{\Ff(\omega)-\Fg(\omega)}
%20171226%        && \text{by definition of $\fg$} && \text{\xref{idef:mra_UVj_g}}
%20171226%      \\&= \norm{\opFT\ff(x)-\opFT\fg(x)}
%20171226%        && \text{by definition of $\Ff$ and $\Fg$} && \text{\xref{idef:mra_UVj_f}, \xref{idef:mra_UVj_g}}
%20171226%      \\&= \norm{\opFT\brs{\ff(x)-\fg(x)}}
%20171226%        && \text{by \prope{linearity} of $\opFT$} && \text{\xref{def:linop}}
%20171226%      \\&= \norm{\ff(x)-\fg(x)}
%20171226%        && \text{by \prope{unitary} property of $\opFT$} && \text{ \xref{thm:ft_unitary}}
%20171226%      \\&\implies\quad\text{$\fg$ exists}
%20171226%        && \mathrlap{\text{because it's possible to satisfy \prefp{idef:mra_UVj_g}}}
%20171226%    \end{align*}
%20171226%
%20171226%  \item lemma: $\norm{\opP_j\fg}<\varepsilon\quad{\scy\forall j\in\Z}$ for sufficiently large $R$. Proof:
%20171226%        \label{ilem:mra_UVj_ge}
%20171226%    \begin{align*}
%20171226%      \varepsilon
%20171226%        &>   \norm{\ff-\fg}
%20171226%        &&   \text{by definition of $\fg$} && \text{\xref{idef:mra_UVj_g}}
%20171226%      \\&\ge \norm{\opP_j\brs{\ff-\fg}}
%20171226%        &&   \text{by property of \ope{projection operator}s} && \text{\xref{def:opP}}
%20171226%      \\&=   \norm{\opP_j\ff-\opP_j\fg}
%20171226%        &&   \text{by \prope{additive} property of $\opP_j$} && \text{\ifxref{operator}{def:linop}}
%20171226%      \\&\ge \abs{\norm{\opP_j\ff}-\norm{\opP_j\fg}}
%20171226%        &&   \text{by \thme{Reverse Triangle Inequality}} && \text{\ifxref{vsnorm}{thm:rti}}
%20171226%      \\&=   \abs{0-\norm{\opP_j\fg}}
%20171226%        &&   \text{by \xref{ilem:mra_UVj_Pf}}
%20171226%      \\&=   \norm{\opP_j\fg}
%20171226%        &&   \text{by \prope{strictly positive} property of $\normn$} && \text{\xref{def:norm}}
%20171226%    \end{align*}
%20171226%
%20171226%  \item  lemma: $\fg=0$. Proof: \label{ilem:mra_UVj_g0}
%20171226%    \begin{align*}
%20171226%       0
%20171226%        &=   \lim_{j\to\infty}\norm{\opP_j\fg}^2
%20171226%        &&   \text{by \prefp{ilem:mra_UVj_ge}}
%20171226%      \\&\ge \lim_{j\to\infty}A\sum_{n\in\Z}\abs{\inprod{\opP_j\fg}{\opDil^j\opTrn^n\fphi}}^2
%20171226%        &&   \text{by \prope{frame property}} && \text{ \xref{prop:rbasis_frame}}
%20171226%      \\&=   \lim_{j\to\infty}A\sum_{n\in\Z}\abs{\inprod{\fg}{\opDil^j\opTrn^n\fphi}}^2
%20171226%        &&   \text{by definition of $\opP_j$} && \text{ \xref{item:mra_UVj_Pj}}
%20171226%      \\&=   \lim_{j\to\infty}A\sum_{n\in\Z}\abs{\inprod{\opFT\fg}{\opFT\opDil^j\opTrn^n\fphi}}^2
%20171226%        &&   \text{by \prope{unitary} property of $\opFT$} && \text{ \xref{thm:ft_unitary}}
%20171226%      \\&=   \lim_{j\to\infty}A\sum_{n\in\Z}\abs{\inprod{\Fg(\omega)}{2^{-j/2}e^{-i2^{-j}\omega n}\Fphi(2^{-j}\omega)}}^2
%20171226%        &&   \text{by \prefp{prop:FTDf}}
%20171226%      \\&=   \lim_{j\to\infty}A\sum_{n\in\Z}\abs{\inprod{\Fg(\omega)\Fphi^\ast(2^{-j}\omega)}{2^{-j/2}e^{-i2^{-j}\omega n}}}^2
%20171226%        &&   \text{by property of $\inprodn$ in $\spLLR$}
%20171226%      \\&=   \lim_{j\to\infty}A\norm{\Fg(\omega)\Fphi^\ast(2^{-j}\omega)}^2
%20171226%        &&   \text{by \thme{Parseval's Identity}} && \text{\xref{thm:fst}}
%20171226%      \\&=   A\norm{\Fg(\omega)\Fphi^\ast(0)}^2
%20171226%        &&   \text{by left hypothesis (2)}
%20171226%      \\&=   A\abs{\Fphi^\ast(0)}^2\,\norm{\Fg(\omega)}^2
%20171226%        &&   \text{by \prope{homogeneous} property of $\normn$} && \text{\ifxref{vsnorm}{def:norm}}
%20171226%      \\&=   A\abs{\Fphi(0)}^2\,\norm{\fg}^2
%20171226%        &&   \text{by \prope{unitary} property of $\opFT$} && \text{\xref{thm:ft_unitary}}
%20171226%      \\&\implies \norm{\fg}=0
%20171226%        &&   \text{by left hypothesis (3)}
%20171226%      \\&\iff     \fg=0
%20171226%        &&   \text{by \prope{nondegenerate} property of $\normn$} && \text{\ifxref{vsnorm}{def:norm}}
%20171226%      %\\&\implies  \cls{\brp{\Setu_{j\in\Z}\spV_j}}=\spLLR
%20171226%    \end{align*}
%20171226%
%20171226%  \item Final step in proof that $\ds\cls{\brp{\Setu_{j\in\Z} \spV_j}} = \spLLR$:
%20171226%    \begin{align*}
%20171226%      \fg
%20171226%        &=0
%20171226%        && \text{by \prefp{ilem:mra_UVj_g0}}
%20171226%      \\&\implies\ff=0
%20171226%        && \text{by definition of $\fg$} && \text{\xref{idef:mra_UVj_g}}
%20171226%      \\&\implies \cls{\brp{\Setu_{j\in\Z} \spV_j}} = \spLLR
%20171226%    \end{align*}
%20171226%\end{enumerate}
%20171226%\end{proof}
%20171226%
%20171226%%=======================================
%20171226%%\subsection{Separable Hilbert Space}
%20171226%%=======================================
%20171226%\pref{def:mra} defines an MRA on the space $\spLLR$, which is a special case of a \structe{separable Hilbert space}.
%20171226%A Hilbert space\ifsxrefs{seq}{def:hilbert}is a \structe{linear space}\ifsxrefs{vector}{def:vspace}that is 
%20171226%equipped with an \structe{inner product}\ifsxref{vsinprod}{def:inprod},
%20171226%is \prope{complete}\ifsxrefs{seq}{def:complete}with respect to the 
%20171226%\structe{metric}\ifsxrefs{metric}{def:metric}induced by the inner product,
%20171226%and contains a subset that is \prope{dense}\ifsxrefs{topology}{def:dense}in $\spLLR$.
%20171226%
%20171226%An \structe{inner product} on a linear space endows the linear space with a \structe{topology}\ifsxref{topology}{def:topology}.
%20171226%The sum such as $\sum_{n=1}^\xN \alpha_n \ff_n$ is finite and thus suitable for a finite linear space only.
%20171226%An infinite space requires an infinite sum $\sum_{n=1}^\infty \alpha_n \fphi_n$, and an infinite sum is defined
%20171226%in terms of a limit \xref{def:suminf}.
%20171226%%  \\\indentx$\ds\sum_{n=1}^\infty \alpha_n \fphi_n \eqd \lim_{\xN\to\infty}\mcom{\ds\sum_{n=1}^\xN \alpha_n \fphi_n}{partial sum}$.\\
%20171226%The limit, in turn, is defined in terms of a \structe{topology}\ifsxref{topology}{def:topology}.
%20171226%The \structe{inner product}\ifsxrefs{vsinprod}{def:inprod} induces a \structe{norm} \xref{def:norm} which induces a 
%20171226%\structe{metric}\ifsxrefs{metric}{def:metric} which induces a topology\ifsxref{metric}{thm:(X,d)->(X,t)}.
%20171226%
%20171226%%A common example of a separable Hilbert space is the space of square integrable functions, $\spLLR$.
%20171226%%And in fact, for the design examples in this book, the reader may simply set $\spLLR=\spLLR$.
%20171226%
%20171226%%%---------------------------------------
%20171226%%\begin{proposition}
%20171226%%\label{prop:Vn_separable}
%20171226%%%---------------------------------------
%20171226%%Let $\MRAspaceLLRV$ be an \structe{MRA space}.
%20171226%%\propbox{
%20171226%%  \text{$\spV_j$ is \prope{separable}}\qquad\scy\forall j\in\Z
%20171226%%  }
%20171226%%\end{proposition}
%20171226%%\begin{proof}
%20171226%%\begin{enume}
%20171226%%  \item By \pref{def:mra}, $\spLLR$ is \prope{separable}.
%20171226%%  \item So by \prefp{thm:XdYd_separable}, each $\spV_j$ is \prope{separable} as well.
%20171226%%\end{enume}
%20171226%%\end{proof}
%20171226%%
%20171226%
%20171226%
%20171226%%=======================================
%20171226%%\subsection{Closure properties}
%20171226%%=======================================
%20171226%\pref{def:mra} defines each subspace $\spV_j$ to be \prope{closed} ($\spV_j=\cls{\spV_j}$) in $\spLLR$.
%20171226%As one might imagine, the properties of \prope{completeness}\ifsxrefs{seq}{def:complete}and 
%20171226%\prope{closure}\ifsxrefs{topology}{def:clsA}%, \prefp{def:subspace_closed}
%20171226%are closely related. % (see next proposition).
%20171226%Moreover, Every \prope{complete} sequence is also \prope{bounded}\ifsxref{metric}{def:bounded},
%20171226%and so each subspace $\spV_j$ is \prope{bounded} as well. % (see \prefp{prop:Vn_bounded}).
%20171226%%Both are topological properties. Completeness is defined on sequences \xref{def:sequence}; %closure is defined on sets.
%20171226%
%20171226%
%20171226%
%20171226%%%---------------------------------------
%20171226%%\begin{proposition}
%20171226%%%---------------------------------------
%20171226%%Let $\MRAspaceLLRV$ be an \structe{MRA space}.
%20171226%%\propbox{
%20171226%%  \mcom{\spLLR=\cls\spLLR}{$\spLLR$ is \prope{closed}.}
%20171226%%  }
%20171226%%\begin{proof}
%20171226%%        The limit of an expansion (if the limit exists) may be inside the linear space or outside. \label{item:mra_Hcomplete}
%20171226%%        We would like it to be inside. That is, we would like the space $\spLLR$ to contain all its 
%20171226%%        \structe{limit points} \xref{def:limitpnt}.
%20171226%%        The space $\spLLR$ does contain all its limit points because by definition, it is \propb{complete} \xref{def:complete}.
%20171226%%        Any metric space (which includes all inner product spaces) that is \prope{complete} is also \prope{closed}
%20171226%%        \xref{thm:comcls}.
%20171226%%        And a metric space is \prope{closed} if and only if it contains all its limit points \xref{thm:cst}.
%20171226%%        An inner product space that is \prope{complete} is called a \structe{Hilbert space} \xref{def:hilbert}.
%20171226%%\end{proof}
%20171226%
%20171226%%---------------------------------------
%20171226%\begin{proposition}
%20171226%\label{prop:Vn_complete}
%20171226%%---------------------------------------
%20171226%Let $\MRAspaceLLRV$ be an \structe{MRA space}.
%20171226%\propbox{
%20171226%  \text{Each subspace $\spV_j$ is \prope{complete}.}
%20171226%  }
%20171226%\end{proposition}
%20171226%\begin{proof}
%20171226%\begin{enume}
%20171226%  \item By definition \pref{def:mra}, $\spLLR$ is \prope{complete}. 
%20171226%  \item In any metric space, (which includes all inner product spaces such as $\spLLR$),
%20171226%        a \prope{closed} subspace of a \prope{complete} metric space is itself also \prope{complete}\ifsxref{seq}{thm:comcls}.
%20171226%  \item In any \prope{complete} metric space $\spX$ (which includes all Hilbert spaces such as $\spLLR$), 
%20171226%        the two properties coincide---that is, a subspace is complete \emph{if and only if} 
%20171226%        it is closed in the space $\spX$\ifsxref{seq}{cor:comcomcls}.
%20171226%  \item So because $\spLLR$ is \prope{complete} and each $\spV_j$ is \prope{closed}, then each $\spV_j$ is also \prope{complete}.
%20171226%\end{enume}
%20171226%\end{proof}
%20171226%
%20171226%%%---------------------------------------
%20171226%%\begin{proposition}
%20171226%%\label{prop:Vn_bounded}
%20171226%%%---------------------------------------
%20171226%%Let $\MRAspaceLLRV$ be an \structe{MRA space}.
%20171226%%\propbox{\begin{array}{MMC}
%20171226%%  $\spLLR$   & is \prope{bounded}.\\
%20171226%%  $\spV_j$ & is \prope{bounded} & \forall n\in\Z .
%20171226%%\end{array}}
%20171226%%\end{proposition}
%20171226%%\begin{proof}
%20171226%%\begin{enume}
%20171226%%  \item Every \prope{complete} metric space is \prope{bounded}\ifsxref{seq}{thm:convergent==>cauchy}.
%20171226%%  \item $\spLLR$ is \prope{complete}, so it is also \prope{bounded} \xref{def:mra}.
%20171226%%  \item Each $\spV_j$ is \prope{complete}, so each $\spV_j$ is also \prope{bounded} \xref{prop:Vn_complete}.
%20171226%%\end{enume}
%20171226%%\end{proof}
%20171226%
%20171226%%=======================================
%20171226%\subsection{Order structure}
%20171226%%=======================================
%20171226%
%20171226%\begin{minipage}{\tw-58mm}%
%20171226%  A \structe{multiresolution analysis} \xref{def:mra} together with the set inclusion relation $\subseteq$
%20171226%  forms the \hie{linearly ordered set} \ifdochas{order}{\xref{def:toset}}
%20171226%  $\hxs{\opair{\seqn{\spV_j}}{\subseteq}}$, illustrated to the right by a \structe{Hasse diagram}\ifsxref{order}{def:hasse}.
%20171226%  Subspaces $\spV_j$ increase in ``size" with increasing $j$.
%20171226%  That is, they contain more and more vectors (functions) for larger and larger $j$---%
%20171226%  with the upper limit of this sequence being $\spLLR$.
%20171226%  %and the subspace $\spZero$ (smallest $n$) containing only the $\vzero$ vector.
%20171226%  Alternatively, we can say that approximation within a subspace $\spV_j$ 
%20171226%  yields greater ``\hie{resolution}" for increasing $j$.
%20171226%  %In general, the number of subspaces in such a sequence can be countably infinite (e.g. $n\in\Z$).
%20171226%\end{minipage}%
%20171226%\hfill\tbox{\includegraphics{graphics/latmra.pdf}}%
%20171226%
%20171226%The \structe{least upper bound} (\structe{l.u.b.}) of the linearly ordered set $\opair{\seqn{\spV_j}}{\subseteq}$ is $\spLLR$ \xref{def:mra}:
%20171226%  \\\indentx
%20171226%   $\ds\clsp{\Setu_{j\in\Z} \spV_j} = \spLLR$.
%20171226%   %$\ds\lim_{\xN\to\infty}\spV_j \eqd \clsp{\Setu_{j\in\Z} \spV_j} = \spLLR$.
%20171226%  \\
%20171226%
%20171226%%      \propb{upper bounded}:
%20171226%%  Furthermore, the property $\clsp{\Setu_{j\in\Z} \spV_j} = \spLLR$
%20171226%%  demonstrates that the sequence of scaling subspaces $\seqn{\spV_j}$ is \prope{upper bounded} by $\spLLR$. % \xref{def:complete_set}.
%20171226%%  Because the subspaces are nested (or linearly ordered with respect to $\subset$) such that $\spV_j\subset\spV_{j+1}$,
%20171226%%  we could define the least upper bound (or the limit) of such a sequence 
%20171226%%  as\citetbl{Many thanks to William Elliot, David C. Ullrich, and Seymour J. Shmuel Metz for help with this topic.
%20171226%%             %\url{https://groups.google.com/forum/\#!topic/sci.math/YD4N58JH5to}
%20171226%%            }
%20171226%
%20171226%The \structe{greatest lower bound} (\structe{g.l.b.}) of the linearly ordered set $\opair{\seqn{\spV_j}}{\subseteq}$ is $\spZero$ \xref{prop:mra_glb}:
%20171226%  \\\indentx
%20171226%    $\ds\Seti_{j\in\Z}\spV_j = \spZero$.
%20171226%  \\
%20171226%
%20171226%All linear subspaces contain the zero vector\ifsxref{subspace}{prop:subspace_prop}.
%20171226%So the intersection of any two subspaces must at least contain $\vzero$.
%20171226%If the intersection of any two linear subspaces $\spX$ and $\spY$ is exactly $\setn{\vzero}$, 
%20171226%then for any vector in
%20171226%the sum of those subspaces ($\vu\in\spX\adds\spY$) there are \propb{unique} vectors $\ff\in\spX$ and 
%20171226%$\fg\in\spY$ such that $\vu=\ff+\fg$.
%20171226%This is \emph{not} necessarily true if the intersection contains more than just $\setn{\vzero}$
%20171226%\ifsxref{subspace}{thm:XY0_unique}.
%20171226%
%20171226%
%20171226%%%=======================================
%20171226%%\subsection{Bases for wavelet system}
%20171226%%%=======================================
%20171226%%%%A linear space is a separable Hilbert space if and only if it has a complete basis.
%20171226%%%Note that \pref{def:mra} does not require $\lim_{\xN\to\infty}\spV_j$ to be equal to $\spLLR$, 
%20171226%%%it is only requires it to be \prope{dense} in $\spLLR$
%20171226%%%(just as the rationals are dense in the real numbers).
%20171226%%%In the set of real numbers, a countable union of closed sets is called a $\symx{\setFsigma}$ set
%20171226%%%($\setF$ stands for the French word \hie{ferm/'e} or \prope{closed}, and $\sigma$ stands for the French word \hie{somme} or sum).\citetbl{
%20171226%%%  \citerpg{carothers2000}{130}{0521497566}\\
%20171226%%%  \citerpg{givant2009}{270}{0387402934}
%20171226%%%  }                                                                             
%20171226%%
%20171226%%\prefp{def:mra} defines an MRA on the space $\spLLR$.
%20171226%%The space $\spLLR$ is an example of a Hilbert space. % $\spLLR$. 
%20171226%%A Hilbert space is a linear space equipped with an inner product 
%20171226%%and that is complete with respect to the topology induced by the inner product.
%20171226%%
%20171226%%\begin{enumerate}
%20171226%%  \item A \structb{linear space} \ifsxrefs{vector}{def:vspace} supports the expansion of a vector $\ff$
%20171226%%        (e.g. a function) in terms of a set of \structe{coordinates} $\setxn{\alpha_n}$ and a 
%20171226%%        \structe{Hamel basis} $\setxn{\fphi_n}$\ifsxrefs{frames}{def:hamel}such that \label{item:mra_hamel}
%20171226%%          \\\indentx$\ds \ff(x)=\sum_{n=1}^\xN \alpha_n \fphi_n(x)$.\\
%20171226%%        If such coordinates exist for a vector $\ff$ and basis $\setxn{\fphi_n}$, 
%20171226%%        then those coordinates are \prope{unique}\ifsxref{frames}{thm:hamel_unique}.
%20171226%%
%20171226%%  \item The Hamel basis described in \pref{item:mra_hamel} provides sufficient support for expansion in finite linear spaces, 
%20171226%%        but is problematic in infinite spaces.                           
%20171226%%        In an infinite linear space with a topology (such as a Banach space or a Hilbert space),
%20171226%%        a \structe{Schauder basis}\ifsxrefs{frames}{def:schauder}is often used.
%20171226%%        The Schauder basis is defined in terms of a special type of convergence called \prope{strong convergence}\ifsxref{seq}{def:strong_converge}.
%20171226%%        Strong convergence is defined in terms of the norm induced by the inner product: \label{item:mra_strong}
%20171226%%        \\\indentx$\ds  
%20171226%%          \ff \eqs \sum_{n=1}^\infty\alpha_n\fphi_n
%20171226%%              \eqd \lim_{\xN\to\infty}\sum_{n=1}^\xN\alpha_n\fphi_n
%20171226%%              \implies
%20171226%%              \mcom{\ds\lim_{\xN\to\infty}\norm{\ff-\sum_{n=1}^\xN\alpha_n\fphi_n}=0}{\prope{strong convergence}}
%20171226%%        $.\\
%20171226%%        %\\\indentx$\ds  
%20171226%%        %  \ff \eqs \sum_{n=1}^\infty\inprod{\ff}{\fphi_n}\fphi_n
%20171226%%        %      \eqd \lim_{\xN\to\infty}\sum_{n=1}^\xN\inprod{\ff}{\fphi_n}\fphi_n
%20171226%%        %      \implies
%20171226%%        %      \mcom{\ds\lim_{\xN\to\infty}\norm{\ff-\sum_{n=1}^\xN\inprod{\ff}{\fphi_n}\fphi_n}=0}{\prope{strong convergence}}
%20171226%%        %$.\\
%20171226%%        That is, the sum $\sum_{n=1}^\infty\alpha_n\fphi_n$ is by definition  
%20171226%%        the limit of the partial sums
%20171226%%        $\sum_{n=1}^\xN\alpha_n\fphi_n$\ifsxref{series}{def:suminf},
%20171226%%        and that these sums \prope{converge strongly} (``$\eqs$",\ifsxref{seq}{def:strong_converge}) to a limit $\ff$
%20171226%%        with respect to the topology induced by the norm $\normn$, which in turn is 
%20171226%%        induced by the inner product $\inprodn$. % \xref{def:norm=inprod}.
%20171226%%        The completeness property ensures that all of these limits $\ff$ are also in the space $\spLLR$.
%20171226%% 
%20171226%%  \item In an MRA space $\MRAspaceLLRV$, the space $\spLLR$ is separable \xref{def:mra}, and the subspaces $\spV_j$ are
%20171226%%        separable as well. % \xref{prop:Vn_separable}.
%20171226%%        The property of a space being separable is very important in analysis:
%20171226%%    \begin{enumerate}
%20171226%%      \item Every Banach space (which includes all Hilbert spaces such as $\spLLR$ and each $\spV_j$) with 
%20171226%%            a Schauder basis is \prope{separable}\ifsxref{frames}{thm:Bschauder==>separable}.\label{item:mra_Bschauder_separable}
%20171226%%      \item The converse is \emph{not} true---not every separable Banach space has a basis\ifsxrefpo{frames}{BasisProblem}.
%20171226%%    \end{enumerate}
%20171226%%
%20171226%%  \item Besides providing a topology, the \structe{inner product} also supports 
%20171226%%        the notion of a subspace geometry, 
%20171226%%        including the property of \prope{orthogonality}\ifsxref{vsinprod}{def:orthog}.
%20171226%%    \begin{enumerate}
%20171226%%      \item Orthogonality supports the \structe{Fourier expansion}\ifsxrefs{frames}{def:hspace_fex}of a vector $\ff$ over an 
%20171226%%        \structe{orthornormal basis} $\setxZp{\fphi_n}$ in the form \label{item:mra_inprod}
%20171226%%        \\\indentx$\ds  
%20171226%%          \ff \eqs \sum_{n=1}^\infty \mcom{\inprod{\ff}{\fphi_n}}{\structe{Fourier coefficient}}\fphi_n
%20171226%%          $
%20171226%%      \item In contrast to \pref{item:mra_Bschauder_separable}, life in Hilbert spaces is much simpler. 
%20171226%%            A Hilbert space has a Schauder basis \emph{if and only if} it is 
%20171226%%            separable \ifsxrefs{frames}{thm:schauder<==>separable}. 
%20171226%%            And so $\spLLR$ and each $\spV_j$ \emph{have} Schauder bases\ifsxref{frames}{thm:schauder<==>separable}.
%20171226%%      \item A special case of a Schauder basis is an orthonormal basis\ifsxref{frames}{def:basis_ortho}.
%20171226%%            A Hilbert space has an orthonormal basis if and only if it is separable\ifsxref{frames}{thm:ortho<==>separable}.
%20171226%%            And so $\spLLR$ and each $\spV_j$ \emph{have} orthonormal bases as well% 
%20171226%%            \ifdochas{frames}{ (\xref{thm:schauder<==>separable}, \xref{thm:HV_orthobasis})}.
%20171226%%      \item It is always possible to construct an orthonormal basis for a separable Hilbert space using the 
%20171226%%            \thme{Graham Schmidt orthogonalization} procedure.
%20171226%%    \end{enumerate} 
%20171226%%
%20171226%%
%20171226%%
%20171226%%\item 
%20171226%%  %Note that the intersection of any two orthogonal subspaces contains the zero vector only \xref{thm:YoZ==>YZ0}.
%20171226%%  A basis\ifsxrefs{frames}{sec:hspace_bases} $\setxn{\fphi_n}$ that is \prope{orthonormal}  possesses a number of useful properties
%20171226%%  including the following:
%20171226%%    \begin{enumerate}
%20171226%%      \item The \thmb{Pathogorean Theorem} holds such that  $\norm{\sum_{n=1}^\xN \fphi_{n}}^2 = \sum_{n=1}^\xN \norm{\fphi_n}^2$\ifsxref{frames}{thm:pythag}.
%20171226%%      \item The sequence $\setn{\fphi_n}$ is \prope{linearly independent}\ifsxref{frames}{thm:orthog==>linin}.
%20171226%%      %\item \thmb{Bessel's equality} holds such that 
%20171226%%      %      $\ds\norm{\ff-\sum_{i=n}^\xN \inprod{\ff}{\fphi_n} \fphi_n }^2 = \norm{\ff}^2 - \sum_{i=1}^\xN |\inprod{\ff}{\fphi_n}|^2$
%20171226%%      \item \thmb{Bessel's inequality} holds such that 
%20171226%%            $\ds\sum_{n=1}^\infty \abs{\inprod{\ff}{\fphi_n}}^2 \le \norm{\ff}^2$\ifsxref{frames}{thm:bessel_ineq}.
%20171226%%      \item Every vector $\ff$ in $\spLLR$ has a \structe{Fourier expansion}\ifsxrefs{frames}{def:hspace_fex}such that \\
%20171226%%            $\ds\ff\eqs\sum_{n=1}^\infty \inprod{\ff}{\fphi_n} \fphi_n$\ifsxref{frames}{thm:hspace_fex}.
%20171226%%      \item \thmb{Parseval's identity} holds \textbf{if and only if} $\setn{\fphi_n}$ is an orthonormal basis:\\
%20171226%%            $\ds\norm{\ff}^2 \eqs\sum_{n=1}^\infty \abs{\inprod{\ff}{\fphi_n}}^2  \quad\scy\forall\ff\in\setX$%
%20171226%%            \ifsxref{frames}{thm:parsevalid}.
%20171226%%      \item The \structe{Fourier expansion} of a vector $\ff$ in a Hilbert space $\spLLR$
%20171226%%            on an orthonormal basis $\setn{\fphi_n}$ 
%20171226%%            that spans a subspace $\spY\subseteq\spLLR$ is the best approximation of $\ff$ in $\spY$ with respect to
%20171226%%            the metric induced by the inner product (\thme{Best Approximation Theorem}\ifsxref{frames}{thm:bat}).
%20171226%%    \end{enumerate}
%20171226%%
%20171226%%
%20171226%%  \item \structb{Riesz basis}: \pref{def:mra} does not explicitly require an orthonormal basis.
%20171226%%         Instead, it only specifies the weaker (more general) constraint of a Riesz basis.
%20171226%%         This constraint implies simply that there is a linear mapping between the Riesz basis and an orthonormal basis.
%20171226%%         In particular, an orthonormal basis can be constructed from the Riesz basis.
%20171226%%\end{enumerate}
%20171226%
%20171226%%%--------------------------------------
%20171226%%\begin{proposition}
%20171226%%\label{thm:HV_orthobasis}
%20171226%%%--------------------------------------
%20171226%%Let $\MRAspaceLLRV$ be an MRA space.
%20171226%%\propbox{\begin{array}{MMC}
%20171226%%  $\spLLR$   & has an \structe{orthonormal basis}\\
%20171226%%  $\spV_j$ & has an \structe{orthonormal basis} & \forall j\in\Z
%20171226%%\end{array}}
%20171226%%\end{proposition}
%20171226%%\begin{proofns}
%20171226%%\begin{enume}
%20171226%%  %\item By \prefp{def:mra}, $\spLLR$ is \prope{separable}.
%20171226%%  \item $\spLLR$ is \prope{separable}.
%20171226%%  \item Therefore $\spLLR$ has an orthonormal basis\ifsxref{frames}{thm:ortho<==>separable}.
%20171226%%  \item Therefore each $\spV_j$ is \prope{separable}\ifsxref{subspace}{prop:Vn_separable}.
%20171226%%  \item So each $\spV_j$ has an orthonormal basis\ifsxref{frames}{thm:ortho<==>separable}.
%20171226%%\end{enume}
%20171226%%\end{proofns}
%20171226%
%20171226%%=======================================
%20171226%\subsection{Dilation equation}
%20171226%%=======================================
%20171226%Several functions in mathematics exhibit a kind of \prope{self-similar} or \prope{recursive} property:
%20171226%\begin{listi}
%20171226%  \item If a function $\ff(x)$ is \prope{linear}, then \xref{ex:TD_flinear}
%20171226%        \\\indentx$\ds\ff(x) = \ff(1)x - \ff(0)\opTrn x$.   %{$\setn{x,\,\opTrn x}$ is a \structe{basis} for $\clLcc$}$.
%20171226%  \item If a function $\ff(x)$ is sufficiently \prope{bandlimited}, then the \structe{Cardinal series} \xref{ex:TD_cardinalseries} demonstrates
%20171226%        \\\indentx$\ds\ff(x) = \sum_{n=1}^\infty \ff(n) \opTrn^n\frac{\sin\brs{\pi(x)}}{\pi(x)}$.
%20171226%  \item \fncte{B-splines} \xref{thm:bspline_recursion} are another example:
%20171226%        \\\indentx$\ds\fN_n(x)   = \frac{1}{n}x\fN_{n-1}(x) - \frac{1}{n}x\opTrn\fN_{n-1}(x) + \frac{n+1}{n}\opTrn\fN_{n-1}(x)  \qquad\scy\forall n\in\Znn\setd\setn{1},\,  \forall x\in\R$.
%20171226%\end{listi}
%20171226%
%20171226%The scaling function $\fphi(x)$ \xref{def:mra} also exhibits a kind of \prope{self-similar} property.
%20171226%By \prefp{def:mra}, the dilation $\opDil\ff$ of each vector $\ff$ in $\spV_0$ is in $\spV_1$.
%20171226%If $\setxZ{\opTrn^n\fphi}$ is a basis for $\spV_0$,
%20171226%then $\setxZ{\opDil\opTrn^n\fphi}$ is a basis for $\spV_1$,
%20171226%$\setxZ{\opDil^2\opTrn^n\fphi}$ is a basis for $\spV_2$, \ldots;
%20171226%and in general $\set{\opDil^j\opTrn^m\fphi}{j\in\Z}$ is a basis for $\spV_j$.
%20171226%Also, if $\fphi$ is in $\spV_0$, then it is also in $\spV_1$ (because $\spV_0\subset\spV_1$).
%20171226%And because $\fphi$ is in $\spV_1$ and because $\setxZ{\opDil\opTrn^n\fphi}$ is a basis for $\spV_1$,
%20171226%$\fphi$ is a linear combination of the elements in $\setxZ{\opDil\opTrn^n\fphi}$.
%20171226%That is, $\fphi$ can be represented as a linear combination of translated and dilated versions of itself.
%20171226%The resulting equation is called the \hie{dilation equation} (\pref{def:dilation_eq}, next).\footnote{%
%20171226%The property of \prope{translation invariance} is of particular significance in the theory of 
%20171226%\structe{normed linear spaces} (a Hilbert space is a complete normed linear space equipped with an inner product)%
%20171226%\ifdochas{vsnorm}{---see \prefp{lem:vsn_ti} and \prefp{thm:vsn_d2norm}}.
%20171226%}
%20171226%
%20171226%%--------------------------------------
%20171226%\begin{definition}
%20171226%\citetbl{
%20171226%  \citerp{jawerth}{7}
%20171226%  }
%20171226%\label{def:dilation_eq}
%20171226%%--------------------------------------
%20171226%Let $\MRAspaceLLRV$ be a \structe{multiresolution analysis space} with scaling function $\fphi$ \xref{def:mra}.
%20171226%Let $\seqxZ{h_n}$ be a \structe{sequence} \xref{def:seq} in $\spllR$ \xref{def:spllR}.
%20171226%\defboxp{
%20171226%  The equation 
%20171226%    \\\indentx$\ds\fphi(x)=\sum_{n\in\Z}  h_n \opDil \opTrn^n \fphi(x)\qquad\scy\forall x\in\R$\\
%20171226%  is called the \equd{dilation equation}.
%20171226%  It is also called the \equd{refinement equation},
%20171226%  \equd{two-scale difference equation}, and \equd{two-scale relation}.
%20171226%  }
%20171226%\end{definition}
%20171226%
%20171226%%--------------------------------------
%20171226%\begin{theorem}[\thmd{dilation equation}]
%20171226%\label{thm:h->phi}
%20171226%\label{thm:dilation_eq}
%20171226%%--------------------------------------
%20171226%%Let $\MRAspaceLLRV$ be a \structe{multiresolution analysis space} with scaling function $\fphi$ \xref{def:mra}.
%20171226%%Let $\mrasys$ be an \structe{MRA system} \xref{def:mrasys}.
%20171226%Let an \structe{MRA space} and \fncte{scaling function} be as defined in \prefp{def:mra}.
%20171226%%Let $\Fphi(\omega)$ be the \fncte{Fourier transform}\ifsxrefs{harFour}{def:ft}of $\fphi(x)$.
%20171226%%Let $\Dh(\omega)$ be the \fncte{Discrete time Fourier transform}\ifsxref{dsp}{def:dtft} of $\seqn{h_n}$.
%20171226%%\\Let $\ds\prod_{n=1}^\infty x_n \eqd \lim_{\xN\to\infty}\prod_{n=1}^\xN x_n$.
%20171226%\thmbox{
%20171226%  \brb{\begin{array}{M}
%20171226%    $\MRAspaceLLRV$ is an \structe{MRA space}\\ 
%20171226%    with \structe{scaling function} $\fphi$
%20171226%  \end{array}}
%20171226%  \quad\implies\quad
%20171226%  \mcom{\brb{\begin{array}{>{\ds}l}
%20171226%    \scy\exists \seqxZ{h_n} \st\\
%20171226%    \fphi(x)=\sum_{n\in\Z}  h_n \opDil \opTrn^n \fphi(x)\qquad
%20171226%    \scy\forall x\in\R
%20171226%  \end{array}}}{\prope{dilation equation in ``time"}}
%20171226%  }
%20171226%\end{theorem}
%20171226%\begin{proof}
%20171226%    \begin{align*}
%20171226%      \fphi &\in \spV_0
%20171226%            && \text{by \prefp{def:mra}}
%20171226%          \\&\subseteq \spV_1
%20171226%            && \text{by \prefp{def:mra}}
%20171226%          \\&= \Span\setxZ{\opDil\opTrn^n\fphi(x)}
%20171226%          \\&\implies 
%20171226%             \exists \seqxZ{h_n} \st
%20171226%                \fphi = \sum_{n\in\Z} h_n \opDil \opTrn^n \fphi
%20171226%    \end{align*}
%20171226%\end{proof}
%20171226%
%20171226%%--------------------------------------
%20171226%\begin{lemma}
%20171226%\citetbl{
%20171226%  \citerpg{mallat}{228}{012466606X}
%20171226%  }
%20171226%\label{lem:Fphi}
%20171226%%--------------------------------------
%20171226%Let $\fphi(x)$ be a function in $\spLLR$ \xref{def:spLLR}.
%20171226%Let $\Fphi(\omega)$ be the \fncte{Fourier transform}\ifsxrefs{harFour}{def:ft}of $\fphi(x)$.
%20171226%Let $\Dh(\omega)$ be the \fncte{Discrete time Fourier transform}\ifsxref{dsp}{def:dtft} of a sequence $\seqnZ{h_n}$.
%20171226%\lembox{
%20171226%  \begin{array}{>{\ds}lc rc>{\ds}lCD}
%20171226%  {\scy(A)}\quad\fphi(x)=\sum_{n\in\Z}  h_n \opDil \opTrn^n \fphi(x)\quad{\scy \forall x\in\R} %{\prope{dilation equation in ``time" (A)}}
%20171226%    &\iff&
%20171226%    \Fphi\brp{\omega} &=& \cwt \: \Dh\brp{\frac{\omega}{2}}\: \Fphi\brp{\frac{\omega}{2}}
%20171226%                        & \forall \omega\in\R
%20171226%                        & (1)
%20171226%  \\&\iff&
%20171226%    \Fphi\brp{\omega} &=& \Fphi\brp{\frac{\omega}{2^\xN}} \prod_{n=1}^\xN \cwt\:\Dh\brp{\frac{\omega}{2^n}}
%20171226%                        & \forall n\in\Zp,\,\omega\in\R
%20171226%                        & (2)
%20171226%  \end{array} 
%20171226%  }
%20171226%\end{lemma}
%20171226%\begin{proof}
%20171226%\begin{enumerate}
%20171226%  \item Proof that (A)$\implies$(1): \label{item:Fphi_A1}
%20171226%    \begin{align*}
%20171226%      \Fphi\brp{\omega}
%20171226%        &\eqd \opFT\fphi
%20171226%      \\&= \opFT\sum_{n\in\Z} h_n \opDil \opTrn^n \fphi(x)
%20171226%        && \text{by (A)}
%20171226%      \\&= \sum_{n\in\Z} h_n \opFT\opDil\opTrn^n \fphi(x)
%20171226%      \\&= \sum_{n\in\Z} h_n \mcom{\cwt e^{-i\frac{\omega}{2} n}\fphi\brp{\frac{\omega}{2}}}{$\opFT\opDil\opTrn^n \fphi(x)$}
%20171226%        && \text{by \prefp{prop:FTDf}}
%20171226%      \\&= \cwt \mcom{\brs{\sum_{n\in\Z} h_n e^{-i\frac{\omega}{2} n}}}{$\Dh(\omega/2)$} \Fphi\brp{\frac{\omega}{2}}
%20171226%      \\&= \cwt\:\Dh\brp{\frac{\omega}{2}}\: \Fphi\brp{\frac{\omega}{2}}
%20171226%        && \text{by definition of \ope{DTFT} \xref{def:dtft}}
%20171226%    \end{align*}
%20171226%
%20171226%  \item Proof that (A)$\impliedby$(1):
%20171226%    \begin{align*}
%20171226%      \fphi(x) 
%20171226%        &= \opFTi\Fphi(\omega)
%20171226%        && \text{by definition of $\Fphi(\omega)$}
%20171226%      \\&= \opFTi\cwt\:\Dh\brp{\frac{\omega}{2}}\: \Fphi\brp{\frac{\omega}{2}}
%20171226%        && \text{by (1)}
%20171226%      \\&= \opFTi\cwt\:\sum_{n\in\Z}h_n e^{-i\frac{\omega}{2} n}\: \Fphi\brp{\frac{\omega}{2}}
%20171226%        && \text{by definition of \ope{DTFT} \xref{def:dtft}}
%20171226%      \\&= \cwt\:\sum_{n\in\Z}h_n \opFTi e^{-i\frac{\omega}{2} n}\: \Fphi\brp{\frac{\omega}{2}}
%20171226%        && \text{by property of linear operators}
%20171226%      \\&= \cwt\:\sum_{n\in\Z}h_n \opFTi \opFT\opDil\opTrn^n\fphi
%20171226%        && \text{by \prefp{prop:FTDf}}
%20171226%      \\&= \sum_{n\in\Z} h_n \opDil\opTrn^n \fphi(x)
%20171226%    \end{align*}
%20171226%
%20171226%  \item Proof that (1)$\implies$(2):
%20171226%    \begin{enumerate}
%20171226%      \item Proof for $\xN=1$ case:
%20171226%        \begin{align*}
%20171226%          \brlr{\Fphi\brp{\frac{\omega}{2^\xN}}\:\prod_{n=1}^\xN \cwt \Dh\brp{\frac{\omega}{2^n}}}_{\xN=1}
%20171226%            &= \cwt\:\Dh\brp{\frac{\omega}{2}}\Fphi\brp{\frac{\omega}{2}}
%20171226%          \\&= \Fphi(\omega)
%20171226%            && \text{by (1)}
%20171226%        \end{align*}
%20171226%
%20171226%      \item Proof that [$\xN$ case]$\implies$[$\xN+1$ case]:
%20171226%        \begin{align*}
%20171226%          \Fphi\brp{\frac{\omega}{2^{\xN+1}}}\:\prod_{n=1}^{\xN+1} \cwt \Dh\brp{\frac{\omega}{2^n}}
%20171226%            &= \brs{\prod_{n=1}^{\xN} \cwt \Dh\brp{\frac{\omega}{2^n}}}
%20171226%               \mcom{\cwt \Dh\brp{\frac{\omega}{2^{N+1}}}\Fphi\brp{\frac{\omega}{2^{\xN+1}}}}{$\Fphi(\omega/2^\xN)$}
%20171226%          \\&= \Fphi(\omega/2^\xN) \prod_{n=1}^{\xN} \cwt \Dh\brp{\frac{\omega}{2^n}}
%20171226%          \\&= \Fphi(\omega)
%20171226%            && \text{by [$\xN$ case] hypothesis}
%20171226%        \end{align*}
%20171226%    \end{enumerate}
%20171226%
%20171226%  \item Proof that (1)$\impliedby$(2):
%20171226%    \begin{align*}
%20171226%      \Fphi(\omega)
%20171226%        &= \brlr{\Fphi\brp{\frac{\omega}{2^{\xN}}}\:\prod_{n=1}^{\xN} \cwt \Dh\brp{\frac{\omega}{2^n}}}_{\xN=1}
%20171226%        && \text{by (2)}
%20171226%      \\&= \Fphi\brp{\frac{\omega}{2}}\cwt \Dh\brp{\frac{\omega}{2}}
%20171226%      \\&= \cwt\Dh\brp{\frac{\omega}{2}}\Fphi\brp{\frac{\omega}{2}} 
%20171226%    \end{align*}
%20171226%
%20171226%\end{enumerate}
%20171226%\end{proof}
%20171226%
%20171226%%--------------------------------------
%20171226%\begin{lemma}
%20171226%\label{lem:Fphi_infty}
%20171226%% 2013 August 09 Friday
%20171226%% 2013 August 20 Monday: modified \implies relation
%20171226%%--------------------------------------
%20171226%Let $\fphi(x)$ be a function in $\spLLR$ \xref{def:spLLR}.
%20171226%Let $\Fphi(\omega)$ be the \fncte{Fourier transform}\ifsxrefs{harFour}{def:ft}of $\fphi(x)$.
%20171226%Let $\Dh(\omega)$ be the \fncte{Discrete time Fourier transform}\ifsxref{dsp}{def:dtft} of $\seqn{h_n}$.
%20171226%Let $\ds\prod_{n=1}^\infty x_n \eqd \lim_{\xN\to\infty}\prod_{n=1}^\xN x_n$, with respect to the standard norm in $\spLLR$.
%20171226%
%20171226%\lembox{
%20171226%  \begin{array}{>{\ds}l c rc>{\ds}lCD}
%20171226%    \brb{\begin{array}{M}
%20171226%      $\Fphi\brp{\omega} = C\prod_{n=1}^\infty \cwt\:\Dh\brp{\frac{\omega}{2^n}}$\\
%20171226%      $\scy\forall C>0,\,\omega\in\R$\qquad\qquad\scs(A)
%20171226%    \end{array}}
%20171226%      &\implies& \fphi(x)          &=& \sum_{n\in\Z}  h_n \opDil \opTrn^n \fphi(x)
%20171226%                                     & \forall x\in\R 
%20171226%                                     & (1) %{\prope{dilation equation in ``time" (A)}}
%20171226%    \\&\iff&     \Fphi\brp{\omega} &=& \cwt \: \Dh\brp{\frac{\omega}{2}}\: \Fphi\brp{\frac{\omega}{2}}
%20171226%                                     & \forall \omega\in\R
%20171226%                                     & (2)
%20171226%  \\
%20171226%      &\iff&     \Fphi\brp{\omega} &=& \Fphi\brp{\frac{\omega}{2^\xN}} \prod_{n=1}^\xN \cwt\:\Dh\brp{\frac{\omega}{2^n}}
%20171226%                                     & \forall n\in\Zp,\,\omega\in\R
%20171226%                                     & (3)
%20171226%  \end{array}
%20171226%  }
%20171226%\end{lemma}
%20171226%\begin{proof}
%20171226%  \begin{enumerate}
%20171226%    \item Proof that (1)$\iff$(2)$\iff$(3): by \prefp{lem:Fphi}
%20171226%    \item Proof that (A)$\implies$(2):
%20171226%      \begin{align*}
%20171226%        \Fphi(\omega)
%20171226%          &= C\:\prod_{n=1}^{\infty} \cwt \Dh\brp{\frac{\omega}{2^n}}
%20171226%          && \text{by left hypothesis}
%20171226%        \\&= C\:\cwt \Dh\brp{\frac{\omega}{2}} \prod_{n=1}^{\infty} \cwt \Dh\brp{\frac{\omega}{2^{n+1}}}
%20171226%        \\&= C\:\cwt \Dh\brp{\frac{\omega}{2}} \prod_{n=1}^{\infty} \cwt \Dh\brp{\frac{\omega/2}{2^{n}}}
%20171226%        \\&= \cwt \Dh\brp{\frac{\omega}{2}}\brs{C\prod_{n=1}^{\infty} \cwt \Dh\brp{\frac{\omega/2}{2^{n}}}}
%20171226%        \\&= \cwt \Dh\brp{\frac{\omega}{2}} \Fphi\brp{\frac{\omega}{2}}
%20171226%          && \text{by left hypothesis}
%20171226%      \end{align*}
%20171226%  \end{enumerate}
%20171226%\end{proof}
%20171226%
%20171226%
%20171226%%--------------------------------------
%20171226%\begin{proposition}
%20171226%\label{prop:Fphi}
%20171226%%--------------------------------------
%20171226%Let $\fphi(x)$ be a function in $\spLLR$ \xref{def:spLLR}.
%20171226%Let $\Fphi(\omega)$ be the \fncte{Fourier transform}\ifsxrefs{harFour}{def:ft}of $\fphi(x)$.
%20171226%Let $\Dh(\omega)$ be the \fncte{Discrete time Fourier transform}\ifsxref{dsp}{def:dtft} of $\seqn{h_n}$.
%20171226%Let $\ds\prod_{n=1}^\infty x_n \eqd \lim_{\xN\to\infty}\prod_{n=1}^\xN x_n$, with respect to the standard norm in $\spLLR$.
%20171226%\propbox{
%20171226%  \brb{\begin{array}{M}
%20171226%    $\Fphi(\omega)$ is\\
%20171226%    \prope{continuous}\\ 
%20171226%    at $\,\omega=0$
%20171226%  \end{array}}
%20171226%  \quad\implies\quad
%20171226%  \brb{\begin{array}{c rc>{\ds}lCD}
%20171226%        &
%20171226%    \fphi(x)          &=& \sum_{n\in\Z}  h_n \opDil \opTrn^n \fphi(x)
%20171226%                        & \forall x\in\R 
%20171226%                        & (1) %{\prope{dilation equation in ``time" (A)}}
%20171226%  \\\iff&
%20171226%    \Fphi\brp{\omega} &=& \cwt \: \Dh\brp{\frac{\omega}{2}}\: \Fphi\brp{\frac{\omega}{2}}
%20171226%                        & \forall \omega\in\R
%20171226%                        & (2)
%20171226%  \\\iff&
%20171226%    \Fphi\brp{\omega} &=& \Fphi\brp{\frac{\omega}{2^\xN}} \prod_{n=1}^\xN \cwt\:\Dh\brp{\frac{\omega}{2^n}}
%20171226%                        & \forall n\in\Zp,\,\omega\in\R
%20171226%                        & (3)
%20171226%  \\\iff&
%20171226%    \Fphi\brp{\omega} &=& \Fphi\brp{0} \prod_{n=1}^\infty \cwt\:\Dh\brp{\frac{\omega}{2^n}} 
%20171226%                        & \omega\in\R
%20171226%                        & (4)
%20171226%  \end{array}}
%20171226%  }
%20171226%\end{proposition}
%20171226%%\lembox{
%20171226%%  \brb{\begin{array}{FMCD}
%20171226%%    (A) & $\ds\fphi(x)=\sum_{n\in\Z}  h_n \opDil \opTrn^n \fphi(x)$ & \forall x\in\R & and\\ % & {\prope{dilation equation}}\\
%20171226%%    (B) & \mc{3}{M}{$\Fphi(\omega)$ is \prope{continuous} at $\omega=0$}
%20171226%%  \end{array}}
%20171226%%  \implies
%20171226%%  \brb{\begin{array}{>{\ds}lCD}
%20171226%%    \Fphi\brp{\omega} = \Fphi\brp{0} \prod_{n=1}^\infty \cwt\:\Dh\brp{\frac{\omega}{2^n}} & \forall \omega\in\R
%20171226%%  \end{array}} 
%20171226%%  }
%20171226%%\end{lemma}
%20171226%\begin{proof}
%20171226%  \begin{enumerate}
%20171226%    \item Proof that (1)$\iff$(2)$\iff$(3): by \prefp{lem:Fphi}
%20171226%    \item Proof that (3)$\implies$(4):
%20171226%      \begin{align*}
%20171226%        \Fphi\brp{0}\:\prod_{n=1}^{\infty} \cwt \Dh\brp{\frac{\omega}{2^n}}
%20171226%          &= \lim_{\xN\to\infty}\Fphi\brp{\frac{\omega}{2^{\xN}}}\:\prod_{n=1}^{\xN} \cwt \Dh\brp{\frac{\omega}{2^n}}
%20171226%          && \text{by \prope{continuity} and definition of $\prod_{n=1}^\infty x_n$}
%20171226%        \\&= \Fphi(\omega)
%20171226%          && \text{by (3) and \prefp{lem:Fphi}}
%20171226%      \end{align*}
%20171226%    \item Proof that (2)$\impliedby$(4): by \prefp{lem:Fphi_infty}
%20171226%      %\begin{align*}
%20171226%      %  \Fphi(\omega)
%20171226%      %    &= \Fphi\brp{0}\:\prod_{n=1}^{\infty} \cwt \Dh\brp{\frac{\omega}{2^n}}
%20171226%      %    && \text{by (4)}
%20171226%      %  \\&= \Fphi\brp{0}\:\cwt \Dh\brp{\frac{\omega}{2}} \prod_{n=1}^{\infty} \cwt \Dh\brp{\frac{\omega}{2^{n+1}}}
%20171226%      %  \\&= \Fphi\brp{0}\:\cwt \Dh\brp{\frac{\omega}{2}} \prod_{n=1}^{\infty} \cwt \Dh\brp{\frac{\omega/2}{2^{n}}}
%20171226%      %  \\&= \cwt \Dh\brp{\frac{\omega}{2}}\brs{ \Fphi\brp{0}\prod_{n=1}^{\infty} \cwt \Dh\brp{\frac{\omega/2}{2^{n}}}}
%20171226%      %  \\&= \cwt \Dh\brp{\frac{\omega}{2}} \Fphi\brp{\frac{\omega}{2}}
%20171226%      %    && \text{by (4)}
%20171226%      %\end{align*}
%20171226%  \end{enumerate}
%20171226%\end{proof}
%20171226%
%20171226%
%20171226%
%20171226%%\if 0
%20171226%
%20171226%
%20171226%\pref{def:hn} (next) formally defines the coefficients that appear in \prefpp{thm:dilation_eq}.
%20171226%%--------------------------------------
%20171226%\begin{definition}%[subspace coefficients]
%20171226%\label{def:hn}
%20171226%\label{def:mrasys}
%20171226%%--------------------------------------
%20171226%Let $\MRAspaceLLRV$ be a multiresolution analysis space with scaling function $\fphi$.
%20171226%Let $\seqxZ{h_n}$ be a sequence of coefficients such that $\fphi = \sum_{n\in\Z}  h_n \opDil\opTrn^n \fphi$.
%20171226%\defboxp{
%20171226%  A \structd{multiresolution system} is the tuple $\mrasys$.
%20171226%  The sequence $\seqxZ{h_n}$ is the \fnctd{scaling coefficient sequence}.
%20171226%  A multiresolution system is also called an \structd{MRA system}.
%20171226%  An \structe{MRA system} is an \structd{orthonormal MRA system} if $\setnZ{\opTrn^n\fphi}$ is \prope{orthonormal}.
%20171226%  }
%20171226%\end{definition}
%20171226%
%20171226%%Examples of \hi{multiresolution analyses} are provided in
%20171226%%\pref{ex:wavstrct_haar_sin} (next)
%20171226%%-- \prefp{ex:sw_gh_bspline}.
%20171226%
%20171226%%%--------------------------------------
%20171226%%\begin{definition}
%20171226%%\label{def:wavstrct_normcoef}
%20171226%%%--------------------------------------
%20171226%%Let $\mrasys$ be a {multiresolution system}, and $\opDil$ the dilation operator.
%20171226%%\defboxp{
%20171226%%  The \hid{normalization coefficient at resolution $n$} is the quantity 
%20171226%%  \\\indentx$\norm{\opDil^j\fphi}$.
%20171226%%  }
%20171226%%\end{definition}
%20171226%
%20171226%
%20171226%%--------------------------------------
%20171226%\begin{theorem}
%20171226%\label{thm:V0Vn}
%20171226%%--------------------------------------
%20171226%Let $\mrasys$ be an \structe{MRA system} \xref{def:mrasys}.\\
%20171226%Let $\linspan\setA$ be the \structe{linear span} \xref{def:span} of a set $\setA$.
%20171226%\thmbox{
%20171226%  \mcom{\linspan\setxZ{\opTrn^n\fphi}=\spV_0}
%20171226%       {$\setxZ{\opTrn^n\fphi}$ is a \structe{basis} for $\spV_0$}
%20171226%  \qquad\implies\qquad
%20171226%  \mcom{\linspan\setxZ{\opDil^j\opTrn^n\fphi}=\spV_j\quad{\scy\forall j\in\Znn}}
%20171226%       {$\setxZ{\opDil^j\opTrn^n\fphi}$ is a \structe{basis} for $\spV_j$}
%20171226%  }
%20171226%\end{theorem}
%20171226%\begin{proof} Proof is by induction:\citetbl{\citerpg{smith2011}{4}{1420063723}}
%20171226%\begin{enumerate}
%20171226%\item induction basis (proof for $j=0$ case):
%20171226%      %$\setxZ{\opDil^0\opTrn^n\fphi}$ is a basis for $\spV_0$
%20171226%      %\\\indentx$\ds \spV_0 = \set{\ff(x)}{\ff(x) = \sum_{n\in\Z} \fc_{n} \fphi(x-n)}.$
%20171226%  \begin{align*}
%20171226%    \brlr{\linspan\setxZ{\opDil^j\opTrn^n\fphi}}_{j=0}
%20171226%      &= \linspan\setxZ{\opTrn^n\fphi}
%20171226%    \\&= \spV_0
%20171226%      && \text{by left hypothesis}
%20171226%  \end{align*}
%20171226%
%20171226%\item induction step (proof that $j$ case $\implies$ $j+1$ case):
%20171226%      %$\setxZ{\opDil^j\opTrn^n\fphi}$ is a basis for $\spV_j$
%20171226%      %$\implies$ $\setxZ{\opDil^{j+1}\opTrn^n\fphi}$ is a basis for $\spV_{j+1}$:
%20171226%  \begin{align*}
%20171226%    &\linspan\setxZ{\opDil^{j+1}\opTrn^n\fphi}
%20171226%    \\&= \set{\ff\in\spLLR}{\exists \seqn{\alpha_n} \st \ff(x)=\sum_{n\in\Z}\alpha_n \opDil^{j+1}\opTrn^n\fphi}
%20171226%      && \text{by definition of $\linspan$} &&\text{\xref{def:span}}
%20171226%    \\&= \set{\ff\in\spLLR}{\exists \seqn{\alpha_n} \st \ff(x)=\opDil\sum_{n\in\Z}\alpha_n \opDil^{j}\opTrn^n\fphi}
%20171226%    \\&= \set{\ff\in\spLLR}{\exists \seqn{\alpha_n} \st \opDili\ff(x)=\sum_{n\in\Z}\alpha_n \opDil^{j}\opTrn^n\fphi}
%20171226%    \\&= \mathrlap{
%20171226%         \set{\brs{\opDil\ff}\in\spLLR}{\exists \seqn{\alpha_n} \st \opDili\brs{\opDil\ff(x)}=\sum_{n\in\Z}\alpha_n \opDil^{j}\opTrn^n\fphi}
%20171226%         }
%20171226%    \\&= \opDil\set{\ff\in\spLLR}{\exists \seqn{\alpha_n} \st \ff(x)=\sum_{n\in\Z}\alpha_n \opDil^{j}\opTrn^n\fphi}
%20171226%    \\&= \opDil\linspan\setxZ{\opDil^j\opTrn^n\fphi}
%20171226%      && \text{by definition of $\linspan$} &&\text{\xref{def:span}}
%20171226%    \\&= \opDil\spV_j
%20171226%      && \text{by induction hypothesis}
%20171226%    \\&= \spV_{j+1}
%20171226%      && \text{by \prope{self-similar} property} &&\text{\xref{def:mra}}
%20171226%  \end{align*}
%20171226%\end{enumerate}
%20171226%\end{proof}
%20171226%
%20171226%%--------------------------------------
%20171226%\begin{example}
%20171226%\label{ex:wavstrct_haar_sin}
%20171226%\exmx{Haar scaling function}
%20171226%%--------------------------------------
%20171226%\exbox{\begin{array}{rclm{54mm}} 
%20171226%  \mc{4}{M}{In the \hie{Haar} MRA, the scaling function $\fphi(x)$ is the \hie{pulse function}}
%20171226%  \\
%20171226%  \fphi(x) &=& \brbl{\begin{array}{lM}
%20171226%                       1  & for $x\in\intco{0}{1}$ \\
%20171226%                       0  & otherwise.
%20171226%                     \end{array}}
%20171226%  &
%20171226%  \includegraphics{graphics/pulse.pdf}
%20171226%  \\
%20171226%  \mc{4}{M}{In the subspace $\spV_j$ ($j\in\Z$) the scaling functions are}
%20171226%  \\
%20171226%  \opDil^j\fphi(x) &=& \brbl{\begin{array}{lM}
%20171226%                               \brp{2}^{j/2}   & for $x\in\intco{0}{\brp{2^{-j}}}$ \\
%20171226%                               0                  & otherwise.
%20171226%                              \end{array}}
%20171226%  &
%20171226%  \includegraphics{graphics/pulse2.pdf}%
%20171226%\end{array}}
%20171226%
%20171226%The scaling subspace $\spV_0$ is the span $\spV_0\eqd\Span\setxZ{\opTrn^n\fphi}$.
%20171226%The scaling subspace $\spV_j$ is the span $\spV_j\eqd\Span\set{\opDil^j\opTrn^n\fphi}{n\in\Z}$.
%20171226%  %$\opDil^j\fphi$ such that
%20171226%Note that $\norm{\opDil^j\opTrn^n\fphi}$ for each resolution $j$ and shift $n$ is unity:
%20171226%  \begin{align*}
%20171226%    \norm{\opDil^j\opTrn^n\fphi}^2
%20171226%      &= \norm{\fphi}^2  
%20171226%      && \text{by \prefp{thm:TD_unitary}}
%20171226%   %\\&= \int_{\intco{0}{\brp{2^{-j}}} \abs{\brp{\sqrt{2}}^j}^2 \dx
%20171226%    \\&= \int_0^1 \abs{1}^2 \dx
%20171226%      && \text{by definition of $\normn$ on $\spLLR$ \xref{def:spLLR}}
%20171226%    %\\&= \brp{2^{-j}\brp{2^j}
%20171226%    \\&= 1
%20171226%  \end{align*}
%20171226%
%20171226%\begin{minipage}{\tw-64mm}
%20171226%Let $\ff(x)=\sin(\pi x)$.
%20171226%Suppose we want to project $\ff(x)$ onto the subspaces $\spV_0$, $\spV_1$, $\spV_2$, \ldots.
%20171226%\end{minipage}%
%20171226%\hfill\tbox{\includegraphics{graphics/sin_pi_t.pdf}}%
%20171226%\\
%20171226%\begin{minipage}{\tw-68mm}
%20171226%\ragr
%20171226%The values of the transform coefficients for the subspace $\spV_j$ are %illustrated in \prefp{fig:wavstrct_Haar_sin}
%20171226%given by
%20171226%\end{minipage}%
%20171226%\hfill%
%20171226%\tbox{%
%20171226%%\begin{figure}[t]
%20171226%  %\mbox{}\\%
%20171226%  %\psset{unit=8mm}%
%20171226%  %%============================================================================
% Daniel J. Greenhoe
% LaTeX file
% sin(t)
%============================================================================
%  \psset{unit=1mm}
\begin{pspicture}(-40,-15)(40,15)%
  \footnotesize
  \psset{linecolor=blue}%
  %\rput(0,0){% axis
  %  \psset{linecolor=axis}
  %  \multirput(-30,0)(10,0){7}{\psline(0,-1)(0,1)}% markers on x axis
  %  \psline{<->}(-35,0)(35,0)% x axis
  %  \psline{<->}(0,-15)(0,15)%    y axis
  %  \psline(-1,10)(1,10)%
  %  \psline(-1,-10)(1,-10)%
  %  \uput[180](0,10){$\frac{1}{\pi}$}% y=1
  %  \uput[0](0,-10){$\frac{-1}{\pi}$}% y=1
  %  \multido{\ival=-3+1,\ipos=-30+10}{7}{%
  %    \uput[-90](\ipos,0){$\ival$}% x=
  %    }%
  %  \uput[0](40,0){$t$}%
  %  }%
  \psaxes[linecolor=axis,unit=10,labels=x]{<->}(0,0)(-3.5,-1.5)(3.5,1.5)%
  \multirput(-20,0)(20,0){3}{\psline{-o}(0,0)(0,10)}%
  \multirput(-30,0)(20,0){4}{\psline{-o}(0,0)(0,-10)}%
  \uput[180](0,10){$\frac{2}{\pi}$}% y=1
  \uput[0](0,-10){$\frac{-2}{\pi}$}% y=1
  \uput[0](35,0){$t$}%
  \rput[b](17.5,10){$\inprod{\ff(t-n)}{\sin(\pi t)}$}%
\end{pspicture}
%
%20171226%  \includegraphics{graphics/haar0_sin_t.pdf}%  8mm
%20171226%  }
%20171226%\\
%20171226%  \begin{align*}
%20171226%    \brs{\opR_j\ff(x)}(n) 
%20171226%      &=    \frac{1}{\norm{\opDil^j\opTrn^n\fphi}^2}\inprod{\ff(x)}{\opDil^j\opTrn^n\fphi} 
%20171226%    %\\&=    \frac{1}{\cancelto{1}{\norm{\fphi}^2}}
%20171226%    %        \inprod{\ff(x)}{\opDil^j\fphi\brp{x-n}}
%20171226%    %  &&    \text{by definition of $\opTrn$ \xref{def:wav_opT}} 
%20171226%    \\&=    \frac{1}{\cancelto{1}{\norm{\fphi}^2}}\inprod{\ff(x)}{2^{j/2}\fphi\brp{2^j x-n}} 
%20171226%      &&    \text{by \prefp{prop:DjTn}}
%20171226%    \\&=    2^{j/2} \inprod{\ff(x)}{\fphi\brp{2^j x-n}} 
%20171226%    \\&=    2^{j/2} 
%20171226%            \int_{2^{-j}n}^{2^{-j}(n+1)} \ff(x) \dx
%20171226%    \\&=    2^{j/2} 
%20171226%            \int_{2^{-j}n}^{2^{-j}(n+1)} \sin(\pi x) \dx
%20171226%    \\&=    2^{j/2}
%20171226%            \left. \brp{-\frac{1}{\pi}}\cos\brp{\pi x} \right|_{2^{-j} n}^{2^{-j}(n+1)}
%20171226%    \\&=    \frac{2^{j/2}}{\pi}
%20171226%            \brs{
%20171226%              \cos\brp{{2^{-j}n\pi}} -
%20171226%              \cos\brp{{2^{-j}(n+1)\pi}}  
%20171226%              }
%20171226%  \end{align*}
%20171226%
%20171226%
%20171226%
%20171226%And the projection $\opA_n\ff(x)$ of the function $\ff(x)$ onto the subspace $\spV_j$ is
%20171226%%(alternatively, the \hie{projection} of $\ff(x)$ \emph{onto} the space $\spV_j$ is) %\\
%20171226%%\begin{minipage}{\tw-95mm}
%20171226%  \begin{align*}
%20171226%    \opA_j\ff(x) 
%20171226%      &= \sum_{n\in\Z} \inprod{\ff(x)}{\opDil^j\opTrn^n\fphi} \opDil^j\opTrn^n\fphi 
%20171226%    \\&= \frac{2^{j/2}}{\pi}
%20171226%         \sum_{n\in\Z}
%20171226%         \brs{
%20171226%           \cos\brp{2^{-j}n\pi} -
%20171226%           \cos\brp{2^{-j}(n+1)\pi}  
%20171226%           } 2^{j/2}\fphi\brp{2^j x-n}
%20171226%    \\&= \frac{2^j}{\pi}
%20171226%         \sum_{n\in\Z}
%20171226%         \brs{
%20171226%           \cos\brp{2^{-j} n\pi} -
%20171226%           \cos\brp{2^{-j}(n+1)\pi}  
%20171226%           } \fphi\brp{2^j x-n}
%20171226%  \end{align*}
%20171226%%\end{minipage}%
%20171226%%\hfill%
%20171226%%\begin{minipage}{90mm}%
%20171226%%  \mbox{}\\%
%20171226%%  \psset{unit=8mm}%
%20171226%%  %\input{../common/math/graphics/square_123_pi.tex}
%20171226%%  \input{../common/wavelets/graphics/haar0_sin_a.tex}
%20171226%%\end{minipage}
%20171226%
%20171226%The transforms of $\sin(\pi x)$ into the subspaces $\spV_0$, $\spV_1$, and $\spV_2$,
%20171226%as well as the approximations in those subspaces are as illustrated in \prefpp{fig:wavstrct_haar_sin}.
%20171226%\end{example}
%20171226%\begin{figure}
%20171226%  \centering%
%20171226%  \begin{tabular}{|l|l|l|}
%20171226%    \hline
%20171226%    \mc{1}{|c|}{subspace}&\mc{1}{c|}{transform}&\mc{1}{c|}{approximation}
%20171226%    \\\hline\hline
%20171226%    $\spV_0$
%20171226%    & \includegraphics{graphics/haar0_sin_t.pdf}
%20171226%    & \includegraphics{graphics/haar0_sin_a.pdf}
%20171226%    \\\hline
%20171226%    $\spV_1$
%20171226%    & \includegraphics{graphics/haar1_sin_t.pdf}
%20171226%    & \includegraphics{graphics/haar1_sin_a.pdf}
%20171226%    \\\hline
%20171226%    $\spV_2$
%20171226%    & \includegraphics{graphics/haar2_sin_t.pdf}
%20171226%    & \includegraphics{graphics/haar2_sin_a.pdf}
%20171226%    \\\hline
%20171226%  \end{tabular}
%20171226%  \caption{
%20171226%    Projections of $\sin(\pi x)$ on Haar subspaces
%20171226%    \xref{ex:wavstrct_haar_sin}
%20171226%    \label{fig:wavstrct_haar_sin}
%20171226%    }
%20171226%\end{figure}
%20171226%
%20171226%
%20171226%
%20171226%
%20171226%%=======================================
%20171226%\subsection{Necessary Conditions}
%20171226%%=======================================
%20171226%%Next we look at  two necessary conditions in the ``time domain" for scaling coefficient design.
%20171226%%%They can be used in generating simultaneous equations for wavelet system design.
%20171226%%\\\indentx
%20171226%%  \begin{tabular}{@{\qquad}clll}
%20171226%%    \imark & \hie{admissibility condition}: & \pref{thm:admiss}        & \xref{thm:admiss} \\
%20171226%%    \imark & \hie{quadrature condition}:    & \pref{thm:wav_quadcon}   & \xref{thm:wav_quadcon}
%20171226%%  \end{tabular}
%20171226%
%20171226%%--------------------------------------
%20171226%\begin{theorem}[\thmd{admissibility condition}]
%20171226%\label{thm:admiss}
%20171226%%--------------------------------------
%20171226%%Let $\mrasys$ be a multiresolution system.
%20171226%Let $\Zh(z)$ be the \fncte{Z-transform} \xref{def:opZ} and 
%20171226%$\Dh(\omega)$ the \fncte{discrete-time Fourier transform} \xref{def:dtft} of a sequence $\seqxZ{h_n}$.
%20171226%\thmbox{\begin{array}{M}
%20171226%  %\brb{\begin{array}{M}$\mrasys$\\is an \structe{MRA system}\end{array}} &\impnotimpby&
%20171226%  $\brb{\text{$\mrasys$ is an \structe{MRA system} \xref{def:mrasys}}}$
%20171226%  \\$\ds\quad
%20171226%  \impnotimpby \mcom{\brb{\sum_{n\in\Z}  h_n  = \sqrt{2}}}{(1) \prope{admissibility} in ``time"}               
%20171226%  \iff         \mcom{\brb{\Zh(z)\Big|_{z=1}   = \sqrt{2}}}{(2) \prope{admissibility} in ``z domain"}                  
%20171226%  \iff         \mcom{\brb{\Dh(\omega)\Big|_{\omega=0} = \sqrt{2}}}{(3) \prope{admissibility} in ``frequency"}
%20171226%  $
%20171226%\end{array}}
%20171226%\end{theorem}
%20171226%\begin{proof}
%20171226%\begin{enumerate}
%20171226%  \item Proof that MRA system $\implies$ (1):
%20171226%    \begin{align*}
%20171226%      \sum_{n\in\Z} h_n
%20171226%        &= \frac{\int_\R \fphi(x) \dx}{\int_\R \fphi(x) \dx} \sum_{n\in\Z} h_n
%20171226%      \\&= \frac{1}{\int_\R \fphi(x) \dx} \int_\R \sum_{n\in\Z} h_n \fphi(x) \dx
%20171226%      \\&= \frac{1}{\int_\R \fphi(x) \dx} \int_\R \sum_{n\in\Z} h_n \frac{\sqrt{2}}{\sqrt{2}}\fphi(2y-n) 2\dy
%20171226%        && \text{let $y\eqd \frac{x+n}{2}\implies x=2y-n \implies \dx=2\dy$}
%20171226%      \\&= \frac{2}{\sqrt{2}}\frac{1}{\int_\R \fphi(x) \dx} \int_\R \sum_{n\in\Z} h_n \opDil\opTrn^n\fphi(y)\dy
%20171226%        && \text{by definitions of $\opTrn$ and $\opDil$ \xref{def:opT}}
%20171226%      \\&= \sqrt{2} \frac{1}{\int_\R \fphi(x) \dx} \int_\R \fphi(y) \dy
%20171226%        && \text{by \thme{dilation equation} \xref{thm:dilation_eq}}
%20171226%      \\&= \sqrt{2}
%20171226%    \end{align*}
%20171226%
%20171226%  \item Alternate proof that MRA system $\implies$ (1):\\
%20171226%    %Select a vector $\ff$ such that $\inprod{\fphi}{\ff}\ne 0$.
%20171226%    Let $\ff(x)\eqd 1\quad\forall x\in\R$. % be a constant vector (e.g. $\ff(x)=1$). Then \ldots
%20171226%    \begin{align*}
%20171226%      \inprod{\fphi}{\ff}
%20171226%        &= \inprod{\sum_{n\in\Z} h_n \opDil\opTrn^n \fphi}{\ff}
%20171226%        && \text{by \thme{dilation equation}} && \text{\xref{thm:dilation_eq}}
%20171226%      \\&= \sum_{n\in\Z} h_n \inprod{\opDil\opTrn^n \fphi}{\ff}
%20171226%        && \text{by \prop{linearity} of $\inprodn$} && \text{\ifxref{vsinprod}{def:inprod}}
%20171226%      \\&= \sum_{n\in\Z} h_n \inprod{ \fphi}{(\opDil\opTrn^n)^\ast\ff}
%20171226%        && \text{by definition of operator adjoint} && \text{\ifxref{operator}{thm:op_star}}
%20171226%      \\&= \sum_{n\in\Z} h_n \inprod{ \fphi}{(\opTrna)^n \opDila \ff}
%20171226%        && \text{by property of operator adjoint} && \text{\ifxref{operator}{thm:op_star}}
%20171226%      \\&= \sum_{n\in\Z} h_n \inprod{ \fphi}{(\opTrni)^n \opDili \ff}
%20171226%        && \text{by unitary property of $\opTrn$ and $\opDil$} && \text{\xref{prop:TD_unitary}}
%20171226%      \\&= \sum_{n\in\Z} h_n \inprod{ \fphi}{(\opTrni)^n \cwt  \ff}
%20171226%        && \text{because $\ff$ is a constant hypothesis} &&\text{and by \prefp{prop:opDi}}
%20171226%      \\&= \sum_{n\in\Z} h_n \inprod{ \fphi}{ \cwt \ff}
%20171226%        && \text{by $\ff(x)=1$ definition}
%20171226%      \\&= \sum_{n\in\Z} h_n \cwt \inprod{ \fphi}{ \ff}
%20171226%        && \text{by property of $\inprodn$} && \text{\ifxref{vsinprod}{def:inprod}}
%20171226%      \\&= \cwt \; \inprod{\fphi}{\ff}\; \sum_{n\in\Z} h_n
%20171226%      \\&\implies \sum_{n\in\Z} h_n = \sqrt{2}
%20171226%    \end{align*}
%20171226%
%20171226%  \item Proof that (1) $\iff$ (2) $\iff$ (3): by \prefp{prop:tzf}.
%20171226%
%20171226%  \item Proof for $\notimpliedby$ part: by \prefp{cnt:admiss}.
%20171226%\end{enumerate}
%20171226%\end{proof}
%20171226%
%20171226%%--------------------------------------
%20171226%\begin{counterex}
%20171226%\label{cnt:admiss}
%20171226%%--------------------------------------
%20171226%Let $\mrasys$ be an \structe{MRA system} \xref{def:mrasys}.
%20171226%\cntbox{\begin{array}{M}
%20171226%  $\brb{\begin{array}{lm{28mm}}
%20171226%     \seqn{h_n} \eqd \sqrt{2}\kdelta_{n-1} \eqd
%20171226%       \brbl{\begin{array}{lD}
%20171226%         \sqrt{2}     & for $n=1$  \\
%20171226%         0            & otherwise.
%20171226%       \end{array}}
%20171226%    &\includegraphics{graphics/hn1.pdf}%
%20171226%  \end{array}}
%20171226%  \quad\implies\quad
%20171226%  \brb{\fphi(x)=0}$
%20171226%  \\
%20171226%  which means
%20171226%  \\
%20171226%  $\ds\brb{\sum_{n\in\Z} h_n = \sqrt{2}} \quad\notimplies\quad \brb{\text{$\mrasys$ is an MRA system for $\spLLR$.}}$
%20171226%\end{array}}
%20171226%\end{counterex}
%20171226%\begin{proof}
%20171226%\begin{align*}
%20171226%  \fphi(x)
%20171226%    &= \sum_{n\in\Z} h_n \opDil\opTrn^n\fphi(x)
%20171226%    && \text{by \thme{dilation equation}} &&\text{\xref{thm:dilation_eq}}
%20171226%  \\&= \sum_{n\in\Z} h_n \fphi(2x-n)
%20171226%    && \text{by definitions of $\opDil$ and $\opTrn$} &&\text{\xref{def:opT}}
%20171226%  \\&= \sum_{n\in\Z} \mcom{\sqrt{2}\kdelta_{n-1}}{$\seqn{h_n}$} \fphi(2x-n)
%20171226%    && \text{by definitions of $\seqn{h_n}$}
%20171226%  \\&= \sqrt{2}\fphi(2x-1)
%20171226%    && \text{by definition of $\fphi(x)$}
%20171226%  \\\implies
%20171226%  \fphi(x) &= 0
%20171226%\end{align*}
%20171226%This implies $\fphi(x)=0$, which implies that $\mrasys$ is \emph{not} an \structe{MRA system} for $\spLLR$ because
%20171226%  \\\indentx$\ds \clsp{\Setu_{j\in\Z} \spV_j} = \clsp{\Setu_{j\in\Z} \linspan\set{\opDil^j\opTrn^n\fphi}{\scy n\in\Z}} \neq \spLLR$\\
%20171226%(the \structe{least upper bound} is \emph{not} $\spLLR$).
%20171226%\end{proof}
%20171226%
%20171226%
%20171226%
%20171226%
%20171226%%--------------------------------------
%20171226%\begin{theorem}[\thmd{Quadrature condition} in ``time"]
%20171226%\label{thm:wav_quadcon}
%20171226%\label{thm:wav_hh}
%20171226%%--------------------------------------
%20171226%Let $\mrasys$ be an \structe{MRA system} \xref{def:mrasys}.
%20171226%\thmbox{
%20171226%  \sum_{m\in\Z} h_m \sum_{k\in\Z} h_k^\ast \inprod{\fphi}{\opTrn^{2n-m+k} \fphi}
%20171226%  =\inprod{\fphi}{\opTrn^n \fphi}
%20171226%  \qquad\scy\forall n\in\Z
%20171226%  }
%20171226%\end{theorem}
%20171226%\begin{proof}
%20171226%\begin{align*}
%20171226%  \inprod{\fphi}{\opTrn^n \fphi}
%20171226%    &= \inprod{\sum_{m\in\Z} h_m \opDil \opTrn^m \fphi }{\opTrn^n \sum_{k\in\Z} h_k \opDil \opTrn^k \fphi}
%20171226%    && \text{by \thme{dilation equation}} && \text{\xref{thm:dilation_eq}}
%20171226%  \\&= \sum_{m\in\Z} h_m \sum_{k\in\Z} h_k^\ast \inprod{\opDil \opTrn^m \fphi }{\opTrn^n \opDil \opTrn^k \fphi}
%20171226%    && \text{by properties of $\inprodn$} && \text{\ifxref{vsinprod}{def:inprod}}
%20171226%  \\&= \sum_{m\in\Z} h_m \sum_{k\in\Z} h_k^\ast \inprod{\fphi }{\left(\opDil \opTrn^m \right)^\ast \opTrn^n \opDil \opTrn^k \fphi}
%20171226%    && \text{by definition of operator adjoint} && \text{\ifxref{operator}{prop:op_adjoint}}
%20171226%  \\&= \sum_{m\in\Z} h_m \sum_{k\in\Z} h_k^\ast \inprod{\fphi }{\left(\opDil \opTrn^m \right)^\ast \opDil \opTrn^{2n} \opTrn^k \fphi}
%20171226%    && \text{by \prefp{prop:DTTD}}
%20171226%  \\&= \sum_{m\in\Z} h_m \sum_{k\in\Z} h_k^\ast \inprod{\fphi }{\opTrna^m \opDila \opDil \opTrn^{2n} \opTrn^k \fphi}
%20171226%    && \text{by operator star-algebra properties} && \text{\ifxref{operator}{thm:op_star}}
%20171226%  \\&= \sum_{m\in\Z} h_m \sum_{k\in\Z} h_k^\ast \inprod{\fphi }{\opTrn^{-m} \opDil^{-1} \opDil \opTrn^{2n} \opTrn^k \fphi}
%20171226%    && \text{by \prefp{prop:TD_unitary}}
%20171226%  \\&= \sum_{m\in\Z} h_m \sum_{k\in\Z} h_k^\ast \inprod{\fphi }{\opTrn^{2n-m+k} \fphi}
%20171226%\end{align*}
%20171226%\end{proof}
%20171226%
%20171226%%%---------------------------------------
%20171226%%\begin{theorem}[Neumann Expansion Theorem]
%20171226%%\index{Neumann Expansion Theorem}
%20171226%%\thmx{Neumann Expansion Theorem}
%20171226%%\label{thm:op_net2}
%20171226%%\citep{michel1993}{415}
%20171226%%%---------------------------------------
%20171226%%Let $\opA\in\clFxx$ be an operator on a linear space $\spX$.
%20171226%%Let $\opA^0\eqd \opI$.
%20171226%%\thmbox{\begin{array}{ll}
%20171226%%  \left.\begin{array}{lrclD}
%20171226%%    1. & \opA          &\in& \oppB(\spX,\spX) & ($\opA$ is bounded) \\
%20171226%%    2. & \normop{\opA} &<&   1
%20171226%%  \end{array}\right\}
%20171226%%  \implies
%20171226%%  \left\{\begin{array}{lrc>{\ds}l}
%20171226%%    1. & (\opI-\opA)^{-1} &&\text{ exists} \\
%20171226%%    2. & \normop{(\opI-\opA)^{-1}} &\le& \frac{1}{1-\normop{\opA}} \\
%20171226%%    3. & (\opI-\opA)^{-1} &=& \sum_{n=0}^\infty \opA^\xN  \\
%20171226%%       & \mc{3}{c}{\text{ with uniform convergence}}
%20171226%%  \end{array}\right.
%20171226%%\end{array}}
%20171226%%\end{theorem}
%20171226%%
%20171226%%
%20171226%%
%20171226%%
%20171226%%
%20171226%%
%20171226%%
%20171226%%%--------------------------------------
%20171226%%\begin{theorem}
%20171226%%\label{thm:wav_net}
%20171226%%\thmx{$\sum_{n\in\Z} \abs{h_n} \ge 1$}
%20171226%%%--------------------------------------
%20171226%%Let $\wavsys$ be a \hi{wavelet system}.
%20171226%%\thmbox{
%20171226%%  \sum_{n\in\Z} \abs{h_n} \ge 1
%20171226%%  }
%20171226%%\end{theorem}
%20171226%%\begin{proof}
%20171226%%\begin{align*}
%20171226%%  &&
%20171226%%  \fphi &= \sum_{n\in\Z} h_n \opTrn^n \opDil \fphi
%20171226%%  \\\implies&&
%20171226%%  \left(\opI - \sum_{n\in\Z} h_n \opTrn^n \opDil \right)\fphi &= \vzero
%20171226%%  \\\implies&&
%20171226%%  \left(\opI - \sum_{n\in\Z} h_n \opTrn^n \opDil \right)^{-1} & \text{must not exist}
%20171226%%  \\\implies&&
%20171226%%  \normop{\sum_{n\in\Z} h_n \opTrn^n \opDil} & \ge 1
%20171226%%    && \text{by Neumann Expansion Theorem \xref{thm:op_net2}}
%20171226%%  \\\implies&&
%20171226%%  1
%20171226%%      &\le \normop{\sum_{n\in\Z} h_n \opTrn^n \opDil}
%20171226%%     &&    %\text{by Neumann Expansion Theorem \xref{thm:op_net2}}
%20171226%%  \\&&&\le \sum_{n\in\Z}  \normop{h_n \opTrn^n \opDil}
%20171226%%     &&    \text{by generalized triangle inequality \ifdochas{vsnorm}{\xref{thm:norm_tri}}}
%20171226%%  \\&&&=   \sum_{n\in\Z}  \abs{h_n}\; \cancelto{1}{\normop{ \opTrn^n \opDil}}
%20171226%%     &&    \text{by homogeneous property of norm \ifdochas{vsnorm}{\xref{def:norm}}}
%20171226%%  \\&&&=   \sum_{n\in\Z}  \abs{h_n}
%20171226%%     &&    \ifdochas{operator}{\text{by \prefp{prop:op_unitary_UV} and \prefp{thm:unitary_prop}}}
%20171226%%\end{align*}
%20171226%%\end{proof}
%20171226%
%20171226%
%20171226%
%20171226%
%20171226%\pref{thm:gen_quadcon} (next) presents the \structe{quadrature necessary conditions} of a \structe{wavelet system}.
%20171226%These relations simplify dramatically in the special case of an
%20171226%\structe{orthonormal wavelet system} \xref{thm:oquadcon}.
%20171226%%--------------------------------------
%20171226%\begin{theorem}[\thmd{Quadrature condition} in ``frequency"]
%20171226%\citetbl{
%20171226%  \citerp{chui}{135},
%20171226%  \citerp{goswami}{110}
%20171226%  }
%20171226%\label{thm:Sphi}
%20171226%%--------------------------------------
%20171226%Let $\mrasys$ be an \structe{MRA system} \xref{def:mrasys}.
%20171226%Let $\Fx(\omega)$ be the \fncte{discrete time Fourier transform}\ifsxrefs{dsp}{def:dtft}for a sequence $\seqxZ{x_n}$ in $\spllR$.
%20171226%Let $\Swphi(\omega)$ be the \fncte{auto-power spectrum} \xref{def:Swfg} of $\fphi$.
%20171226%\thmbox{\begin{array}{>{\ds}lc>{\ds}l}
%20171226%   \abs{\Dh\left(\omega     \right)}^2 \rnode[b]{noteSphi1}{\Sphi}(\omega) + \abs{\Dh\brp{\omega+\pi }}^2 \rnode[b]{noteSphi2}{\Sphi}(\omega+\pi) &=& 2\rnode[b]{noteSphi3}{\Sphi}(2\omega)
%20171226%\end{array}}
%20171226%\hfill
%20171226%\rnode[bl]{noteSphi}{\footnotesize$\brp{\begin{array}{N}
%20171226%  Note: $\Sphi(\omega)=1$\\% $\iff$ $\setxZ{\opTrn^n\fphi}$\\
%20171226%  for \prope{orthonormal} MRA\\
%20171226%  \xref{lem:oms_quadcon}
%20171226%\end{array}}$}
%20171226%\ncarc[arcangle=30,linewidth=0.5pt,linecolor=red,linestyle=dashed]{->}{noteSphi}{noteSphi1}
%20171226%\ncarc[arcangle=30,linewidth=0.5pt,linecolor=red,linestyle=dashed]{->}{noteSphi}{noteSphi2}
%20171226%\ncarc[arcangle=30,linewidth=0.5pt,linecolor=red,linestyle=dashed]{->}{noteSphi}{noteSphi3}
%20171226%\end{theorem}
%20171226%\begin{proof}
%20171226%%\begin{enumerate}
%20171226%%  \item First note that $\Dh(\omega)$ and $\Dg(\omega)$ are periodic with period $2\pi$ such that\ifsxrefs{dsp}{prop:dtft_2pi} 
%20171226%%  \label{item:qc1}
%20171226%%    \\\indentx$\begin{array}{rclC}
%20171226%%         \Dh(\omega+2\pi n) &=& \Dh(\omega) & \forall n\in\Z   \\
%20171226%%         \Dg(\omega+2\pi n) &=& \Dg(\omega) & \forall n\in\Z   
%20171226%%       \end{array}$
%20171226%%
%20171226%%\item Proof for theorem statement:
%20171226%\begin{align*}
%20171226%   &2\Sphi(2\omega)
%20171226%   \\&= 2\brp{2\pi}\sum_{n\in\Z} \left|\Fphi(2\omega+2\pi n)\right|^2
%20171226%     && \text{by \prefp{thm:Swfg}}
%20171226%   \\&= 2\brp{2\pi}\sum_{n\in\Z} \left|\cwt \Dh\left(\frac{2\omega+2\pi n}{2}\right)\Fphi\left(\frac{2\omega+2\pi n}{2}\right)\right|^2
%20171226%     && \text{by \prefp{lem:Fphi}}
%20171226%   \\&= \mathrlap{
%20171226%          {2\pi}\sum_{n\in\Ze} \left|\Dh\left(\frac{2\omega+2\pi n}{2}\right)\right|^2\left|\Fphi\left(\frac{2\omega+2\pi n}{2}\right)\right|^2 +
%20171226%          {2\pi}\sum_{n\in\Zo} \left|\Dh\left(\frac{2\omega+2\pi n}{2}\right)\right|^2\left|\Fphi\left(\frac{2\omega+2\pi n}{2}\right)\right|^2
%20171226%          }
%20171226%   \\&= \mathrlap{
%20171226%          2\pi\sum_{n\in\Z} \left|\Dh\left(\omega+2\pi n\right)\right|^2\left|\Fphi\left(\omega+2\pi n\right)\right|^2 +
%20171226%          2\pi\sum_{n\in\Z} \left|\Dh\left(\omega+2\pi n+\pi \right)\right|^2\left|\Fphi\left(\omega+2\pi n+ \pi \right)\right|^2
%20171226%          }
%20171226%   \\&= 2\pi\sum_{n\in\Z} \left|\Dh\left(\omega\right)\right|^2\left|\Fphi\left(\omega+2\pi n\right)\right|^2 +
%20171226%        2\pi\sum_{n\in\Z} \left|\Dh\left(\omega+\pi \right)\right|^2\left|\Fphi\left(\omega+2\pi n+ \pi \right)\right|^2
%20171226%     && \text{by \prefp{prop:dtft_2pi}}
%20171226%        %\indentx\text{by (\prefp{item:qc1})}
%20171226%   \\&= \left|\Dh\left(\omega     \right)\right|^2 \brp{2\pi\sum_{n\in\Z} \left|\Fphi\left(\omega    +2\pi n\right)\right|^2} +
%20171226%        \left|\Dh\left(\omega+\pi \right)\right|^2 \brp{2\pi\sum_{n\in\Z} \left|\Fphi\left(\omega+\pi+2\pi n\right)\right|^2}
%20171226%   \\&= \left|\Dh\left(\omega     \right)\right|^2 \Sphi(\omega) +
%20171226%        \left|\Dh\left(\omega+\pi \right)\right|^2 \Sphi(\omega+\pi)
%20171226%     && \text{by \prefp{thm:Swfg}}
%20171226%\end{align*}
%20171226%
%20171226%%\end{enumerate}
%20171226%\end{proof}
%20171226%
%20171226%
%20171226%%=======================================
%20171226%\subsection{Sufficient conditions}
%20171226%%=======================================
%20171226%\pref{thm:mra_rdc} (next) gives a set of \emph{sufficient} conditions on the \fncte{scaling function} \xref{def:mra} 
%20171226%$\fphi$ to generate an \structe{MRA}.
%20171226%\ifdochas{ortho}{\prefpp{thm:h_ns} provides a set of sufficient conditions on the \fncte{scaling coefficients} \xref{def:hn} $\seqnZ{h_n}$ 
%20171226%to generate an \structe{MRA}; howbeit, this set results in the more restrictive \prope{orthonormal} MRA.}
%20171226%%---------------------------------------
%20171226%\begin{theorem}
%20171226%\citetbl{
%20171226%  \citerpgc{wojtaszczyk1997}{28}{0521578949}{Theorem 2.13},
%20171226%  \citerpgc{pinsky2002}{313}{0534376606}{Theorem 6.4.27}
%20171226%  }
%20171226%\label{thm:mra_rdc}
%20171226%\label{thm:mra_sufficient_phi}
%20171226%%--------------------------------------
%20171226%%Let an \structe{MRA} be defined as in \prefp{def:mra}.
%20171226%%Let a \structe{Riesz sequence} be defined as in \prefp{def:rieszseq}.
%20171226%Let $\spV_j\eqd\linspan\setxZ{\opTrn\fphi(x)}$ \xref{def:span}.
%20171226%\thmbox{
%20171226%  \brb{\begin{array}{FMD}
%20171226%    (1). & $\seqn{\opTrn^n\fphi}$ is a \structe{Riesz sequence} \xref{def:rieszseq}    & and \\
%20171226%    (2). & $\ds\exists \seqn{h_n} \st \fphi(x)=\sum_{n\in\Z}h_n\opDil\opTrn^n\fphi(x)$ & and \\
%20171226%    (3). & $\Fphi(\omega)$ is \prope{continuous} at $0$                                & and \\
%20171226%    (4). & $\Fphi(0)\neq0$
%20171226%  \end{array}}
%20171226%  \implies
%20171226%  \brb{\begin{array}{N}
%20171226%    $\seqjZ{\spV_j}$ is an \structe{MRA}\\
%20171226%    \xref{def:mra}
%20171226%  \end{array}}
%20171226%  }
%20171226%\end{theorem}
%20171226%\begin{proof}
%20171226%For this to be true, each of the conditions in the definition of an \structe{MRA} \xref{def:mra} must be satisfied:
%20171226%\begin{enumerate}
%20171226%  \item Proof that each $\spV_j$ is \prope{closed}: by definition of $\linspan$
%20171226%
%20171226%  \item Proof that $\seqn{\spV_j}$ is \prope{linearly ordered}: 
%20171226%    \begin{align*}
%20171226%      \spV_j\subseteq\spV_{j+1}
%20171226%        &\iff \linspan\setn{\opDil^j\opTrn^n\fphi}\subseteq\linspan\setn{\opDil^{j+1}\opTrn^n\fphi}
%20171226%        &\iff (2)
%20171226%    \end{align*}
%20171226%
%20171226%  \item Proof that $\ds\Setu_{j\in\Z}\spV_j$ is \prope{dense} in $\spLLR$: by \prefp{prop:mra_UVj}
%20171226%  
%20171226%  \item Proof of \prope{self-similar} property:
%20171226%    \begin{align*}
%20171226%      \brb{\ff\in\spV_j\iff\opDil\ff\in\spV_{j+1}}
%20171226%        &\iff \ff\in\linspan\setn{\opTrn^n\fphi}\iff\opDil\ff\in\linspan\setn{\opDil\opTrn^n\fphi}
%20171226%        &\iff (2)
%20171226%    \end{align*}
%20171226%
%20171226%  \item Proof for \structe{Riesz basis}: by (1) and \prefp{prop:mra_UVj}.
%20171226%\end{enumerate}
%20171226%\end{proof}



%=======================================
%\section{Wavelet analysis}
%=======================================
%=======================================
\section{Definition}
%=======================================
The term ``wavelet" comes from the French word ``\hie{ondelette}", meaning ``small wave". 
And in essence, wavelets are ``small waves" (as opposed to the ``long waves" of Fourier analysis) 
that form a basis for the Hilbert space $\spLLR$.\citetbl{
  \citerpg{strang1996}{ix}{0961408871},
  \citerpg{atkinson2009}{191}{1441904581}
  }
%---------------------------------------
\begin{definition}
\citetbl{
  \citerpgc{wojtaszczyk1997}{17}{0521578949}{Definition 2.1}
  }
\label{def:wavelet}
\label{def:seqWn}
\label{def:wavstrct_psi}
%---------------------------------------
%Let $\MRAspaceLLRV$ be an \structe{multiresolution space} \xref{def:mra}.
Let $\opTrn$ and $\opDil$ be as defined in \prefp{def:opT}.
\defboxp{
  A function $\fpsi(x)$ in $\spLLR$ is a \fnctd{wavelet function} for $\spLLR$ if
  \\\indentx$\set{\opDil^j\opTrn^n\fpsi}{\scy j,n\in\Z}$ is a \structe{Riesz basis} for $\spLLR$.\\
  In this case, $\fpsi$ is also called the \fnctd{mother wavelet} of the basis $\set{\opDil^j\opTrn^n\fpsi}{\scy j,n\in\Z}$.
  The sequence of subspaces $\seqjZ{\spW_j}$ is the \structd{wavelet analysis} induced by $\fpsi$, 
  where each subspace $\spW_j$ is defined as
  \\\indentx$\spW_j\eqd\linspan\setnZ{\opDil^j\opTrn^n\fpsi}$ .
  }
\end{definition}

%---------------------------------------
%\begin{definition}
%---------------------------------------
%Let $\MRAspaceLLRV$ be an \structe{multiresolution space} \xref{def:mra}.
%Let the operation $\adds$ represent \hie{Minkowski addition} on $\spLLR$\ifsxref{subspace}{def:sub_add}.%
%\defboxt{
%  %The \hid{wavelet subspace} $\spW_j$ is the orthogonal complement of $\spV_j$ in $\spV_{j+1}$ such that
%  %  \\\indentx$\spV_j \adds \spW_j = \spV_{j+1}$
%  %  \\
%  The sequence $\seqjZ{\spW_j}$ is a \hid{wavelet analysis} on $\MRAspaceLLRV$ if
%  \\\indentx$\begin{array}{FMCD}
%    1. & $\spV_{j+1} = \mcom{\spV_j \adds \spW_j}{\hi{Minkowski addition}}$
%       & \forall \spW_j \in \seqxZ{\spW_j}\text{ and }\spV_j\in\seqjZ{\spV_j}
%       & and
%     %\qquad\text{\scriptsize ($\spW_j$ is the complement of $\spV_j$ in $\spV_{j+1}$)}
%    \\
%    2. & \mc{2}{M}{There exists $\fpsi\in\spLLR$ such that $\seqxZ{\opTrn^n\fpsi}$ is a \structe{Riesz basis} for $\spW_0$.}
%  \end{array}$
%  }
%\end{definition}

%%---------------------------------------
%\begin{proposition}[complemented subspaces]
%%---------------------------------------
%Let $\MRAspaceLLRV$ be an \structe{MRA space}.
%Let $\spV_j^\orthog$ be the \structe{orthogonal complement} \xref{def:sub_ocomp} of $\spV_j$.
%\propbox{
%  \spV_j^{\orthog\orthog}=\spV_j \qquad\scy\forall n\in\Z \qquad\scs\text{\prope{involutory}}
%  }
%\end{proposition}
%\begin{proof}
%\begin{enume}
%  \item By \pref{def:mra}, $\spV_j$ is \prope{closed} in $\spLLR$ ($\spV_j=\cls\spV_j$).
%  \item By \prefpp{thm:inprod_orthog}, $\spV_j^{\orthog\orthog}=\spV_j$.
%\end{enume}
%\end{proof}

A \structe{wavelet analysis} $\seqn{\spW_j}$ is often constructed from a \structe{multiresolution anaysis} \xref{def:mra}
$\seqn{\spV_j}$ under the relationship
\\\indentx$\ds\spV_{j+1} = \spV_j \adds \spW_j$,\qquad where $\adds$ is subspace addition (\ope{Minkowski addition}).\\
By this relationship alone, $\seqn{\spW_j}$ is in no way uniquely defined 
in terms of a multiresolution analysis $\seqn{\spV_j}$.
In general there are many possible complements of a subspace $\spV_j$.
To uniquely define such a wavelet subspace, one or more additional constraints are required.
One of the most common additional constraints is \hie{orthogonality}, such that
$\spV_j$ and $\spW_j$ are orthogonal to each other\ifdochas{ortho}{ (see \prefp{chp:ortho})}.




%=======================================
\section{Dilation equation}
%=======================================
Suppose $\seqxZ{\opTrn^n\fpsi}$ is a basis for $\spW_0$.
By \prefp{def:seqWn}, the wavelet subspace $\spW_0$ is contained in the 
scaling subspace $\spV_1$.
By \prefp{def:mra}, the sequence $\seqxZ{\opDil\opTrn^n\fphi}$ is a basis for $\spV_1$.
Because $\spW_0$ is contained in $\spV_1$,
the sequence $\seqxZ{\opDil\opTrn^n\fphi}$ is also a basis for $\spW_0$.

%--------------------------------------
\begin{theorem}[\thmd{wavelet dilation equation}]
\label{thm:g->psi}
%--------------------------------------
Let $\mrasys$ be a \structe{multiresolution system} \xref{def:mrasys}
and $\seqjZ{\spW_j}$ be a \structe{wavelet analysis} \xref{def:seqWn} 
with respect to\\
$\mrasys$ and with \fncte{wavelet function} $\fpsi$ \xref{def:wavelet}.
\thmbox{
  \begin{array}{l rc>{\ds}l @{\qquad}D}
    \exists \seqxZ{g_n} \st
      & \fpsi &=& \sum_{n\in\Z}  g_n \opDil \opTrn^n \fphi
      & 
  \end{array}}
\end{theorem}
\begin{proof}
\begin{align*}
  \fpsi &\in \spW_0
        && \text{by \prefp{def:seqWn}}
      \\&\subseteq \spV_1
        && \text{by \prefp{def:seqWn}}
      \\&= \Span\seqxZ{\opDil\opTrn^n\fphi(x)}
        && \text{by \prefp{def:mra} (MRA)}
      \\&\implies 
         \exists \seqxZ{g_n} \st
            \fpsi = \sum_{n\in\Z}  g_n \opDil \opTrn^n \fphi
\end{align*}

%\item Proof that $\fpsi(x) = \sqrt{2} \sum_{n\in\Z}  g_n  \fphi(2x-n)$:
%\begin{align*}
%              &  \set{\fphi(2x-n)}{n\in\Z} \text{ spans } \spV_1
%              && \text{by (1)}
%  \\
%              &  \set{\fpsi(x-n)}{n\in\Z} \text{ spans } \spW_0
%  \\ \implies & \fpsi(x)\in \spW_0 \subset \spV_1
%  \\ \implies & \text{$\fpsi(x)$ can be represented as a linear combination of $\fphi(2x-n)$}.
%\end{align*}
%\end{enumerate}
\end{proof}

A \structe{wavelet system} (next definition) consists of two subspace sequences: 
\begin{liste}
  \item A \structb{multiresolution analysis} $\seqn{\spV_j}$ \xref{def:mra}
     provides ``coarse" approximations of a function in $\spLLR$ at different ``scales" or resolutions.
  \item A \structb{wavelet analysis} $\seqn{\spW_j}$
     provides the ``detail" of the function missing from the approximation provided by a given scaling subspace
     \xref{def:seqWn}.
\end{liste}

%--------------------------------------
\begin{definition}
\label{def:wavsys}
\label{def:gn}
%--------------------------------------
Let $\mrasys$ be a \structe{multiresolution system} \xref{def:mra}
and $\seqjZ{\spW_j}$ a wavelet analysis \xref{def:seqWn}
with respect to $\seqjZ{\spV_j}$.
Let $\seqxZ{g_n}$ be a sequence of coefficients.
\defbox{\begin{array}{M}
  A \structd{wavelet system} is the tuple \quad$\ds\wavsys$\\
  and the sequence $\seqxZ{g_n}$ that satisfies the equation
  $\ds\fpsi = \sum_{n\in\Z}  g_n \opDil \opTrn^n \fphi$\\
  is the \structd{wavelet coefficient sequence}.
\end{array}}
\end{definition}

%--------------------------------------
\begin{remark}
%--------------------------------------
The pair of coefficient sequences $\opair{\seqn{h_n}}{\seqn{g_n}}$ generates 
the scaling function $\fphi(x)$ \xref{def:wavstrct_phi} 
and the wavelet function $\fpsi(x)$ \xref{def:wavstrct_psi}.
These functions in turn generate 
the multiresolution analysis $\seqn{\spV_j}$ \xref{def:seqVn}
and the wavelet analysis $\seqn{\spW_j}$ \xref{def:seqWn}.
Therefore, the coefficient sequence pair $\opair{\seqn{h_n}}{\seqn{g_n}}$ 
totally defines a wavelet system\\$\wavsys$ \xref{def:wavsys}.

Furthermore, especially in the case of orthonormal wavelets, the wavelet coefficient
sequence $\seqxZ{g_n}$ is often defined in terms of the 
scaling coefficient sequence $\seqxZ{h_n}$
in a very simple and straightforward manner.
Therefore, in the case of an orthonormal wavelet system, the coefficient
scaling sequence $\seqxZ{h_n}$ often totally defines the entire wavelet system.
And in this case, designing a wavelet system is only a matter of finding a handful of
scaling coefficients $\seqn{h_1,\,h_2,\,\ldots,\,h_n}$\ldots because once you have these,
you can generate everything else.
\end{remark}


%%---------------------------------------
%\begin{definition}
%\label{def:wav_lat_coef}
%%---------------------------------------
%Let $\seqxZ{h_n}$ be a sequence of scaling coefficients and
%    $\seqxZ{g_n}$ be the associated sequence of wavelet coefficients.
%%    $\subseteq$ the set inclusion relation,
%%    $\setu$ the set union operation, and
%%    $\seti$ the set intersection operation.
%\defbox{\begin{array}{l}
%  \text{The tupple }
%  \qquad \wavlatcoef \\
%  \text{is called the \hid{lattice of wavelet bases coefficients}.}
%\end{array}}
%\end{definition}



%\begin{figure}[t]
%\setlength{\unitlength}{8mm}
%\begin{center}
%\begin{tabular}{cc}
%   \includegraphics*[width=6\tw/16, height=6\tw/16, clip=true]{../common/wavelets/haar_sj.eps} &
%   \includegraphics*[width=6\tw/16, height=6\tw/16, clip=true]{../common/wavelets/haar_sk.eps} \\
%   \footnotesize varying dilation, constant translation   &
%   \footnotesize varying translation, constant dilation
%\end{tabular}
%\caption{
%   \label{fig:haar-jn}
%   Haar scaling function at varying dilations and translations.
%   }
%\end{center}
%\end{figure}



%=======================================
\section{Order structure}
%=======================================
%The axioms of \prefp{def:mra} generate a subspace architecture.
%These transform representation sequences may be \hie{ordered} with \hie{ordering relations}
%as follows:%
%\footnote{\hie{projection operator ordering}: \prefp{def:operator_lattice}}

\begin{minipage}{\tw-70mm}\raggedright
The \structe{wavelet system} $\wavsys$ \xref{def:wavsys} together with the 
set inclusion relation $\subseteq$ 
forms an \structe{ordered set}\ifsxref{order}{def:poset}, 
illustrated to the right by a \hie{Hasse diagram}\ifsxref{order}{def:hasse}.%\ifdochas{order}{\footnotemark}
%Under these three order relations, wavelet system generate three isomorphic lattices such 
%as are illustrated in \prefp{fig:wav_VPb_isomorphic} 
%and in the figure to the right.
\end{minipage}%
\hfill\tbox{\includegraphics{graphics/latwav.pdf}}%

%---------------------------------------
\begin{proposition}
\label{prop:order_wavstrct}
%---------------------------------------
Let $\wavsys$ be a wavelet system with order relation $\subseteq$.
The lattice $\latL\eqd\lattice{\seqn{\spV_j},\seqn{\spW_j}}{\subseteq}{\join}{\meet}$ has 
the following properties:
\propbox{\begin{array}{FM}
      1.  & $\latL$ is \prope{nondistributive}.
    \\2.  & $\latL$ is \prope{nonmodular}.
    %\cntn & $\latL$ is \prope{complemented}.
    %\cntn & $\latL$ is \prope{not uniquely complemented}.
    %\cntn & $\latL$ is \prope{nonorthocomplemented}.
    \\3.  & $\latL$ is \prope{noncomplemented}.
    \\4.  & $\latL$ is \prope{nonBoolean}.
\end{array}}
\end{proposition}
\begin{proof}
%\mbox{}\hspace{20mm}
%  \latmatlw{4}{0.5}
%    {
%           &       & \null                 \\  
%           & \null                         \\  
%     \null &       & \null &       & \null \\  
%           &       & \null                   
%    }
%    {\ncline{1,3}{2,2}\ncline{2,2}{3,1}
%     \ncline{1,3}{3,5}
%     \ncline{2,2}{3,3}
%     \ncline{4,3}{3,1}\ncline{4,3}{3,3}\ncline{4,3}{3,5}
%    }
%    {\nput{ 90}{1,3}{$1$}
%     \nput{135}{2,2}{$v$}
%     \nput{0}{3,1}{$x$}
%     \nput{ 67}{3,3}{$y$}
%     \nput{  0}{3,5}{$z$}
%     \nput{-90}{4,3}{$0$}
%    }
\begin{enumerate}
  \item Proof that $\latL$ is \prope{nondistributive}: \label{item:wavprop_nondistrib}
    \begin{enumerate}
      \item $\latL$ contains the $N5$ lattice\ifsxref{latm}{def:lat_N5}.
      \item Because $\latL$ contains the $N5$ lattice, $\latL$ is \prope{nondistributive}\ifdochas{latm}{ by \prefp{thm:latd_char_n5m3}}.
    \end{enumerate}

  \item Proof that $\latL$ is \prope{nonmodular} and \prope{nondistributive}: 
    \begin{enumerate}
      \item $\latL$ contains the $N5$ lattice\ifsxref{latm}{def:lat_N5}.
      \item Because $\latL$ contains the $N5$ lattice, $\latL$ is \prope{nonmodular}\ifdochas{latm}{ by \prefp{thm:lat_mod_iff_N5}}.
    \end{enumerate}

  \item Proof that $\latL$ is \prope{noncomplemented}:
    \begin{minipage}{50mm}
      \begin{align*}
          x' &= y' = v' = z
        \\z' &= \setn{x,y,v}
        \\x''&= \brp{x'}'
           \\&= z'
           \\&= \setn{x,y,v}
           \\&\ne  x
      \end{align*}
    \end{minipage}\qquad\tbox{\includegraphics{graphics/lat6_plat_xyzv10.pdf}}%

  %\item Proof that $\latL$ is \prope{not uniquely complemented}:\\
  %   For example, subspace $\spW_2$ in \prefp{fig:wav_VPb_isomorphic} is complemented
  %   by $\spV_1$, $\spV_2$, and $\spW_1$.
  %\item Proof that $\latL$ is \prope{orthomodular}:
  %  \begin{enumerate}
  %    \item $\latL$ does \emph{not} contain the $O_6$ lattice\ifdochas{ortholat}{ \xref{def:latoc_omod}}.
  %    \item Because $\latL$ does not contain the $O_6$ lattice, $\latL$ is \prope{orthomodular}\ifdochas{ortholat}{ by \prefp{thm:latoc_omod}}.
  %  \end{enumerate}

  \item Proof that $\latL$ is \prope{nonBoolean}:
    \begin{enumerate}
      \item $\latL$ is \prope{nondistributive} (\pref{item:wavprop_nondistrib}).
      \item Because $\latL$ is \prope{nondistributive}, it is \prope{nonBoolean}\ifdochas{boolean}{ by \prefp{def:booalg}}.
    \end{enumerate}
\end{enumerate}
\end{proof}




%=======================================
\section{Subspace algebraic structure}
%=======================================
%--------------------------------------
\begin{theorem}
\label{thm:mra_subalg}
%--------------------------------------
Let $\wavsys$ be a \structe{wavelet system} \xref{def:wavsys}.
Let $\spV_1 \adds \spV_2$ represent \fncte{Minkowski addition} of two subspaces $\spV_1$ and $\spV_2$ of a Hilbert space $\spH$.
\thmbox{\begin{array}{rc>{\ds}l D}
    \spLLR &=& \lim_{j\to\infty}\spV_j                 
             & ($\spLLR$ is equivalent to one very large scaling subspace)\\
           &=& \spV_j \adds \spW_j \adds \spW_{j+1} \adds \spW_{j+2} \adds\, \cdots 
             & $\brp{\begin{array}{D}$\spLLR$ is equivalent to one scaling space\\
                                       and a sequence of wavelet subspaces\end{array}} $\\
           &=& \cdots\,\adds \spW_{-2} \adds \spW_{-1} \adds \spW_0 \adds \spW_1 \adds \spW_2 \adds\,\cdots        
             & ($\spLLR$ is equivalent to a sequence of wavelet subspaces)
  \end{array}}
\end{theorem}
\begin{proof}
\begin{enumerate}
  \item Proof for (1):
    \begin{align*}
      \spLLR 
        &= \lim_{j\to\infty}\spV_j                 
        && \text{by \prefp{def:mra}}
    \end{align*}

  \item Proof for (2):
    \begin{align*}
      \mcom{\spV_j \adds \spW_j}{$\spV_{j+1}$} \adds \spW_{j+1} \adds \spW_{j+2} \adds \cdots
        &= \mcom{\spV_{j+1} \adds \spW_{j+1}}{$\spV_{j+2}$} \adds \spW_{j+2} \adds \spW_{j+3} \adds \cdots
      \\&= \mcom{\spV_{j+2} \adds \spW_{j+2}}{$\spV_{j+3}$} \adds \spW_{j+3} \adds \spW_{j+4} \adds \cdots
      \\&= \mcom{\spV_{j+3} \adds \spW_{j+3}}{$\spV_{j+4}$} \adds \spW_{j+4} \adds \spW_{j+5} \adds \cdots
      \\&= \mcom{\spV_{j+5} \adds \spW_{j+5}}{$\spV_{j+5}$} \adds \spW_{j+6} \adds \spW_{j+6} \adds \cdots
      \\&= \lim_{j\to\infty}\spV_{j+5} \adds \spW_{j+5} \adds \spW_{j+6} \adds \spW_{j+6} \adds \cdots
      \\&= \spLLR
    \end{align*}

  \item Proof for (3):
    \begin{align*}
      \spLLR &= \mcom{\spV_0}{$\spV_{-1}\adds\spW_{-1}$} \adds \spW_0 \adds \spW_1 \adds \spW_2 \adds \spW_3 \adds \cdots
             && \text{by (2)}
           \\&= \mcom{\spV_{-1}}{$\spV_{-2}\adds\spW_{-2}$} \spW_{-1} \adds \spW_0 \adds \spW_1 \adds \spW_2 \adds \spW_3 \adds \cdots
           \\&= \mcom{\spV_{-2}}{$\spV_{-3}\adds\spW_{-3}$} \spW_{-2} \adds \spW_{-1} \adds \spW_0 \adds \spW_1 \adds \spW_2 \adds \spW_3 \adds \cdots
           \\&= \mcom{\spV_{-3}}{$\spV_{-4}\adds\spW_{-4}$} \spW_{-3} \adds \spW_{-2} \adds \spW_{-1} \adds \spW_0 \adds \spW_1 \adds \spW_2 \adds \spW_3 \adds \cdots
           \\&\vdots
           \\&= \cdots \adds \spW_{-3} \adds \spW_{-2} \adds \spW_{-1} \adds \spW_0 \adds \spW_1 \adds \spW_2 \adds \spW_3 \adds \cdots
    \end{align*}
\end{enumerate}
\end{proof}

%--------------------------------------
\begin{remark}
%--------------------------------------
In the special case that two subspaces $\spW_1$ and $\spW_2$ are \prope{orthogonal} to each other, then 
the \fncte{subspace addition} operation $\spW_1\adds\spW_2$ is frequently expressed as
$\spW_1\oplus\spW_2$.
In the case of an \structe{orthonormal wavelet system}\ifsxref{ortho}{def:ows}, 
the expressions in \prefpp{thm:mra_subalg} could be expressed as
\\\indentx$\begin{array}{rc>{\ds}l}
    \spLLR &=& \lim_{j\to\infty}\spV_j                 \\
           &=& \spV_j \oplus \spW_j \oplus \spW_{j+1} \oplus \spW_{j+2} \oplus\, \cdots \\
           &=& \cdots\,\oplus \spW_{-2} \oplus \spW_{-1} \oplus \spW_0 \oplus \spW_1 \oplus \spW_2 \oplus\,\cdots .       
  \end{array}$
\end{remark}.


%=======================================
\section{Necessary conditions}
%=======================================

%--------------------------------------
\begin{theorem}[\thmd{quadrature condition}s in ``time"]
\label{thm:wavsys_quadcon}
%--------------------------------------
Let $\wavsys$ be a wavelet system \xref{def:wavsys}.
\thmbox{\begin{array}{F>{\ds}rc>{\ds}lC}
  1. & \sum_{m\in\Z} h_m \sum_{k\in\Z} h_k^\ast \inprod{\fphi}{\opTrn^{2n-m+k} \fphi} &=& \inprod{\fphi}{\opTrn^n \fphi} & \forall n\in\Z\\
  2. & \sum_{m\in\Z} g_m \sum_{k\in\Z} g_k^\ast \inprod{\fphi}{\opTrn^{2n-m+k} \fphi} &=& \inprod{\fpsi}{\opTrn^n \fpsi} & \forall n\in\Z\\
  3. & \sum_{m\in\Z} h_m \sum_{k\in\Z} g_k^\ast \inprod{\fphi}{\opTrn^{2n-m+k} \fphi} &=& \inprod{\fphi}{\opTrn^n \fpsi} & \forall n\in\Z
\end{array}}
\end{theorem}
\begin{proof}
\begin{enumerate}
  \item Proof for (1): by \prefp{thm:wav_quadcon}.
  \item Proof for (2): 
    \begin{align*}
      \inprod{\fpsi}{\opTrn^n \fpsi}
        &= \inprod{\sum_{m\in\Z} g_m \opDil \opTrn^m \fphi }{\opTrn^n \sum_{k\in\Z} g_k \opDil \opTrn^k \fphi}
        && \text{by \thme{wavelet dilation equation}}
        && \text{\xref{thm:g->psi}}
      \\&= \sum_{m\in\Z} g_m \sum_{k\in\Z} g_k^\ast \inprod{\opDil \opTrn^m \fphi }{\opTrn^n \opDil \opTrn^k \fphi}
        && \text{by properties of $\inprodn$}
        && \text{\xref{def:inprod}}
      \\&= \sum_{m\in\Z} g_m \sum_{k\in\Z} g_k^\ast \inprod{\fphi }{\left(\opDil \opTrn^m \right)^\ast \opTrn^n \opDil \opTrn^k \fphi}
        && \text{by def. of operator adjoint}
        && \text{\ifxref{operator}{prop:op_adjoint}}
      \\&= \sum_{m\in\Z} g_m \sum_{k\in\Z} g_k^\ast \inprod{\fphi }{\left(\opDil \opTrn^m \right)^\ast \opDil \opTrn^{2n} \opTrn^k \fphi}
        && \text{by \prefp{prop:DTTD}}
      \\&= \sum_{m\in\Z} g_m \sum_{k\in\Z} g_k^\ast \inprod{\fphi }{\opTrna^m \opDila \opDil \opTrn^{2n} \opTrn^k \fphi}
        && \text{by operator star-algebra prop.}
        && \text{\ifxref{operator}{thm:op_star}}
      \\&= \sum_{m\in\Z} g_m \sum_{k\in\Z} g_k^\ast \inprod{\fphi }{\opTrn^{-m} \opDil^{-1} \opDil \opTrn^{2n} \opTrn^k \fphi}
        && \text{by \prefp{prop:TD_unitary}}
      \\&= \sum_{m\in\Z} g_m \sum_{k\in\Z} g_k^\ast \inprod{\fphi }{\opTrn^{2n-m+k} \fphi}
    \end{align*}

  \item Proof for (3): 
    \begin{align*}
      &\inprod{\fphi}{\opTrn^n \fpsi}
      \\&= \inprod{\sum_{m\in\Z} h_m \opDil \opTrn^m \fphi }{\opTrn^n \sum_{k\in\Z} g_k \opDil \opTrn^k \fphi}
        && \text{by \prefp{thm:dilation_eq}}
        && \text{and \prefp{thm:g->psi}}
      \\&= \sum_{m\in\Z} h_m \sum_{k\in\Z} g_k^\ast \inprod{\opDil \opTrn^m \fphi }{\opTrn^n \opDil \opTrn^k \fphi}
        && \text{by properties of $\inprodn$}
        && \text{\xref{def:inprod}}
      \\&= \sum_{m\in\Z} h_m \sum_{k\in\Z} g_k^\ast \inprod{\fphi }{\left(\opDil \opTrn^m \right)^\ast \opTrn^n \opDil \opTrn^k \fphi}
        && \text{by definition of operator adjoint}
        && \text{\ifxref{operator}{prop:op_adjoint}}
      \\&= \sum_{m\in\Z} h_m \sum_{k\in\Z} g_k^\ast \inprod{\fphi }{\left(\opDil \opTrn^m \right)^\ast \opDil \opTrn^{2n} \opTrn^k \fphi}
        && \text{by \prefp{prop:DTTD}}
      \\&= \sum_{m\in\Z} h_m \sum_{k\in\Z} g_k^\ast \inprod{\fphi }{\opTrna^m \opDila \opDil \opTrn^{2n} \opTrn^k \fphi}
        && \text{by operator star-algebra properties}
        && \text{\ifxref{operator}{thm:op_star}}
      \\&= \sum_{m\in\Z} h_m \sum_{k\in\Z} g_k^\ast \inprod{\fphi }{\opTrn^{-m} \opDil^{-1} \opDil \opTrn^{2n} \opTrn^k \fphi}
        && \text{by \prefp{prop:TD_unitary}}
      \\&= \sum_{m\in\Z} h_m \sum_{k\in\Z} g_k^\ast \inprod{\fphi }{\opTrn^{2n-m+k} \fphi}
    \end{align*}
\end{enumerate}
\end{proof}



%=======================================
%\section{Fourier properties}
%=======================================

%--------------------------------------
\begin{proposition}
\label{prop:vsmra_real_Fpsi}
\label{prop:psi_g_phi}
%--------------------------------------
Let $\wavsys$ be a wavelet system.
Let $\Fphi(\omega)$ and $\Fpsi(\omega)$ be the \fncte{Fourier transform}s\ifsxrefs{harFour}{def:ft}of $\fphi(x)$ and $\fpsi(x)$, respectively.
Let $\Dg(\omega)$ be the \fncte{Discrete time Fourier transform}\ifsxrefs{dsp}{def:dtft}of $\seqn{g_n}$.
%  $\begin{array}[t]{rc>{\ds}l c>{\ds}l D}
%    \Fpsi\brp{\omega}
%      &\eqd& \opFT\fpsi
%      &\eqd& \frac{1}{\sqrt{2\pi}}\int_t \fpsi(x) e^{-i\omega t} \dx
%      &      (\structe{Fourier transform}, \prefp{def:ft})
%      \\
%    \Dg(\omega)
%      &\eqd& \opDTFT\seqn{g_n}
%      &\eqd& \sum_{n\in\Z} g_n e^{-i\omega n}
%      &      (\structe{Discrete-time Fourier Transform}).
%  \end{array}$
\propbox{
  \Fpsi\brp{\omega}
    %\eqd
    %\mcom{\opFT\fpsi = \brp{\opDili \opDTFT\seqn{g_n}} \; \brp{\opDili \opFT\fphi}}
    %     {operator notation}
    =
    {\cwt \: \Dg\brp{\frac{\omega}{2}}\: \Fphi\brp{\frac{\omega}{2}}}
    %     {traditional notation}
  }
\end{proposition}
\begin{proof}
\begin{align*}
  \Fpsi\brp{\omega}
    &\eqd \opFT\fpsi
  \\&= \opFT\sum_{n\in\Z} g_n \opDil \opTrn^n \fphi
    && \text{by \thme{wavelet dilation equation}}
    && \text{\xref{thm:g->psi}}
  \\&= \sum_{n\in\Z} g_n \opFT\opDil \opTrn^n \fphi
  \\&= \sum_{n\in\Z} g_n \opDili \opFT\opTrn^n \fphi
    && \text{by \prefp{cor:wavstrct_FTD}}
  \\&= \sum_{n\in\Z} g_n \opDili e^{-i\omega n} \opFT\fphi
    && \text{by \prefp{cor:wavstrct_FTD}}
  \\&= \sum_{n\in\Z} g_n \sqrt{2}\brp{\opDili e^{-i\omega n}} \brp{\opDili\opFT\fphi}
    && \text{by \prefp{prop:DjTnfg}}
  \\&= \sqrt{2}\brp{\opDili \sum_{n\in\Z} g_n e^{-i\omega n}} \; \brp{\opDili \opFT\fphi}
  \\&= \sqrt{2}\brp{\opDili \opDTFT\seqn{g_n}} \; \brp{\opDili \opFT\fphi}
    && \text{by definition of $\opDTFT$}
    && \text{\ifxref{dsp}{def:dtft}}
  \\&= \sqrt{2}\cwt \: \Dg\brp{\frac{\omega}{2}}\: \cwt \Fphi\brp{\frac{\omega}{2}}
    && \text{by property of $\opDil$} && \text{\xref{prop:opDi}}
  \\&= \cwt \: \Dg\brp{\frac{\omega}{2}}\: \Fphi\brp{\frac{\omega}{2}}
\end{align*}
%
%\begin{align*}
%  \Fpsi\brp{\omega}
%    &= \opF\fpsi
%  \\&= \opFT\sum_{n\in\Z} g_n \opDil \opTrn^n \fphi
%    && %\text{by \thme{dilation equation} \xref{thm:dilation_eq}}
%  \\&= \sum_{n\in\Z} g_n \opFT\opDil \opTrn^n \fphi
%  \\&= \sum_{n\in\Z} g_n \opDili \opFT\opTrn^n \fphi
%    && \text{by \prefp{prop:vsmra_real_FD}}
%  \\&= \sum_{n\in\Z} g_n \opDili e^{-i\omega n} \opFT\fphi
%    && \text{by \prefp{prop:vsmra_real_FT}}
%  \\&= \brp{\opDili \sum_{n\in\Z} g_n e^{-i\omega n}} \; \brp{\opDili \opFT\fphi}
%  \\&= \brp{\opDili \opDTFT\seqn{g_n}} \; \brp{\opDili \opFT\fphi}
%  \\&= \fscale \: \Dg\brp{\frac{\omega}{2}}\: \fscale \Fphi\brp{\frac{\omega}{2}}
%    && \text{by property of $\opDil$} && \text{\xref{prop:opDi}}
%  \\&= \frac{1}{2}\: \Dh\brp{\frac{\omega}{2}}\: \Fphi\brp{\frac{\omega}{2}}
%\end{align*}
\end{proof}

%=======================================
%\section{Immediate results}
%=======================================



%=======================================
%\subsection{Power Spectrum}
%=======================================
%%--------------------------------------
%\begin{definition}
%\citep{chui}{134}
%\label{def:wav_S}
%\index{scaling power spectrum function         }
%\index{wavelet power spectrum function         }
%\index{scaling wavelet power spectrum function }
%\index{Laurent polynomial}
%%--------------------------------------
%Let $\wavsys$ be a \hi{wavelet system}.
%Let $\Szfg(z)$ be the \fncte{complex cross-power spectrum} of $\ff$ and $\fg$ \xref{def:Szfg} in $\spLLR$
%and $\Swfg(\omega)$ be the \fncte{cross-power spectrum} of $\ff$ and $\fg$ \xref{def:Swfg} in $\spLLR$.
%\defbox{\begin{array}{>{\ds}rc>{\ds}lM}
%    \Szphi(z)  &\eqd&  \left.\Szfg(z)\right|_{\ff=\fg=\fphi} &is the \hid{scaling power spectrum function}.
%    \Szpsi(z)  &\eqd&  \left.\Szfg(z)\right|_{\ff=\fg=\fpsi} &is the \hid{wavelet power spectrum function}.
%    \Szpsi(z)  &\eqd&  \Szfg(z) &is the \hid{scaling power spectrum function}.
%    \\
%    \Spsi(\omega) &\eqd&  \sum_{n\in\Z} \Rpsi(n) \fkernea{n}{\omega}
%    &is the \hid{wavelet power spectrum function}.
%    \\
%    \Shs (\omega) &\eqd&  \sum_{n\in\Z} \Rhs (n) \fkernea{n}{\omega}
%    &is the \hid{scaling wavelet power spectrum function}.
%\end{array}}
%%\\The Laurent polynomial $\Sphi(\omega)$ is also called the \hid{Euler-Frobenius polynomial}.
%\end{definition}


%In this chapter, we don't assume the special case of orthonormality.
%But good things happen if we do happen to have orthonormality.
%One of them is that the power spectrum equations in \prefp{lem:SSS}
%simplify to constants \xref{lem:SSSo}.

\pref{thm:gen_quadcon} (next) presents the \structe{quadrature} necessary conditions of a \hi{wavelet system}.
These relations simplify dramatically in the special case of an
\structe{orthonormal wavelet system} \xref{thm:oquadcon}.
%--------------------------------------
\begin{theorem}[\thmd{Quadrature conditions} in ``frequency"]
\citetbl{
  \citerp{chui}{135},
  \citerp{goswami}{110}
  }
\label{thm:gen_quadcon}
%--------------------------------------
Let $\wavsys$ be a \hi{wavelet system}.
Let $\Fx(\omega)$ be the \fncte{discrete time Fourier transform}\ifsxrefs{dsp}{def:dtft}for a sequence $\seqxZ{x_n}$ in $\spllR$.
Let $\Swphi(\omega)$ be the \fncte{auto-power spectrum} \xref{def:Swfg} of $\fphi$,
    $\Swpsi(\omega)$ be the \fncte{auto-power spectrum} of $\fpsi$,
and $\Shs(\omega)$ be the \fncte{cross-power spectrum} of $\fphi$ and $\fpsi$.
\thmbox{\begin{array}{F>{\ds}lc>{\ds}l}
   1. & \abs{\Dh\left(\omega     \right)}^2 \Sphi(\omega) + \abs{\Dh\brp{\omega+\pi }}^2 \Sphi(\omega+\pi) &=& 2\Sphi(2\omega)
\\ 2. & \abs{\Dg\left(\omega     \right)}^2 \Sphi(\omega) + \abs{\Dg\brp{\omega+\pi }}^2 \Sphi(\omega+\pi) &=& 2\Spsi(2\omega)
\\ 3. & \Dh(\omega)\Dg^\ast(\omega)         \Sphi(\omega) + \Dh(\omega +\pi)\Dg^\ast(\omega +\pi)\Sphi(\omega+\pi) &=& 2\Shs(2\omega)
\end{array}}
\end{theorem}
\begin{proof}
\begin{enumerate}
  %\item First note that $\Dh(\omega)$ and $\Dg(\omega)$ are periodic with period $2\pi$ such that\ifsxrefs{dsp}{prop:dtft_2pi}
  %  \\$\begin{array}{rclC}
  %       \Dh(\omega+2\pi n) &=& \Dh(\omega) & \forall n\in\Z   \\
  %       \Dg(\omega+2\pi n) &=& \Dg(\omega) & \forall n\in\Z   
  %     \end{array}$

\item Proof for (1): by \prefp{thm:Sphi}.

\item Proof for (2):
\begin{align*}
   2\Spsi(2\omega)
     &\eqd 2\brp{2\pi}\sum_{n\in\Z} \left|\Fpsi(2\omega+2\pi n)\right|^2
   \\&= 2\brp{2\pi}\sum_{n\in\Z} \left|\cwt \Dg\left(\frac{2\omega+2\pi n}{2}\right)\Fphi\left(\frac{2\omega+2\pi n}{2}\right)\right|^2
        \indentx\text{by \prefp{lem:Fphi}}
   \\&= 2\pi
        \sum_{n\in\Ze} \left|\Dg\left(\frac{2\omega+2\pi n}{2}\right)\right|^2\left|\Fphi\left(\frac{2\omega+2\pi n}{2}\right)\right|^2 +
     \\&\qquad 2\pi
        \sum_{n\in\Zo} \left|\Dg\left(\frac{2\omega+2\pi n}{2}\right)\right|^2\left|\Fphi\left(\frac{2\omega+2\pi n}{2}\right)\right|^2
   \\&= 2\pi\sum_{n\in\Z} \abs{\Dg\brp{\omega+2\pi n     }}^2 \abs{\Fphi\brp{\omega+2\pi n       }}^2 +
        2\pi\sum_{n\in\Z} \abs{\Dg\brp{\omega+2\pi n+\pi }}^2 \abs{\Fphi\brp{\omega+2\pi n + \pi }}^2  
   \\&= 2\pi\sum_{n\in\Z} \abs{\Dg\brp{\omega            }}^2 \abs{\Fphi\brp{\omega+2\pi n       }}^2 +
        2\pi\sum_{n\in\Z} \abs{\Dg\brp{\omega+\pi        }}^2 \abs{\Fphi\brp{\omega+2\pi n + \pi }}^2  
   \\&= \abs{\Dg\brp{\omega     }}^2 \brp{2\pi\sum_{n\in\Z} \abs{\Fphi\brp{\omega+2\pi n       }}^2 +}
        \abs{\Dg\brp{\omega+\pi }}^2 \brp{2\pi\sum_{n\in\Z} \abs{\Fphi\brp{\omega+\pi+2\pi n   }}^2  }
   \\&= \abs{\Dg\brp{\omega     }}^2\Sphi(\omega) +
        \abs{\Dg\brp{\omega+\pi }}^2 \Sphi(\omega+\pi)
        \indentx\text{by \prefp{thm:Swfg}}
\end{align*}


\item Proof for (3):
\begin{align*}
  2\Shs(2\omega)
    &=  2\brp{2\pi}\sum_{n\in\Z} \Fphi(2\omega+2\pi n) \Fpsi^\ast(2\omega+2\pi n)
  \\&=  2\brp{2\pi}\sum_{n\in\Z}
        \cwt 
        \Dh  \left(\omega +\pi n \right)
        \Fphi\left(\omega +\pi n \right)
        \cwt 
        \Dg^\ast  \left(\omega +\pi n \right)
        \Fphi^\ast\left(\omega +\pi n \right)
        \quad\text{by \prefp{lem:Fphi}}
  \\&=  2\pi
        \sum_{n\in\Z}
        \Dh  \left(\omega +\pi n \right)
        \Dg^\ast  \left(\omega +\pi n \right)
        \left| \Fphi\left(\omega +\pi n \right) \right|^2
  \\&=  2\pi
        \sum_{n\in\Zo}
        \Dh  \left(\omega +\pi n \right)
        \Dg^\ast  \left(\omega +\pi n \right)
        \left| \Fphi\left(\omega +\pi n \right) \right|^2
      \\&\qquad+ 2\pi\sum_{n\in\Ze}
        \Dh  \left(\omega +\pi n \right)
        \Dg^\ast  \left(\omega +\pi n \right)
        \left| \Fphi\left(\omega +\pi n \right) \right|^2
  \\&=  2\pi\sum_{n\in\Z}
        \Dh  \left(\omega +2\pi n+\pi \right)
        \Dg^\ast  \left(\omega +2\pi n+\pi \right)
        \left| \Fphi\left(\omega +2\pi n+\pi \right) \right|^2
      \\&\qquad+ 2\pi\sum_{n\in\Z}
        \Dh  \left(\omega +2\pi n\right)
        \Dg^\ast  \left(\omega +2\pi n\right)
        \left| \Fphi\left(\omega +2\pi n\right) \right|^2
  \\&=  2\pi
        \sum_{n\in\Z}
        \Dh  \left(\omega +\pi \right)
        \Dg^\ast  \left(\omega +\pi \right)
        \left| \Fphi\left(\omega +2\pi n+\pi \right) \right|^2
      + 2\pi\sum_{n\in\Z}
        \Dh  \left(\omega \right)
        \Dg^\ast  \left(\omega \right)
        \left| \Fphi\left(\omega +2\pi n\right) \right|^2
  \\&=  \Dh  \left(\omega \right)
        \Dg^\ast  \left(\omega \right)
        \brp{2\pi\sum_{n\in\Z} \left| \Fphi\left(\omega +2\pi n\right) \right|^2}
      \\&\qquad+ \Dh  \left(\omega +\pi \right)
        \Dg^\ast  \left(\omega +\pi \right)
        \brp{2\pi\sum_{n\in\Z}\left| \Fphi\left(\omega +\pi+2\pi n\right) \right|^2}
  \\&=  \Dh(\omega)
        \Dg^\ast(\omega)
        \brp{2\pi\sum_{n\in\Z} \left| \Fphi(\omega +2\pi n) \right|^2}
      + \Dh  (\omega +\pi)
        \Dg^\ast (\omega +\pi)
        \brp{2\pi\sum_{n\in\Z}\left| \Fphi(\omega +\pi+2\pi n) \right|^2}
  \\&=  \Dh(\omega     ) \Dg^\ast(\omega     )\Sphi(\omega)
      + \Dh(\omega +\pi) \Dg^\ast(\omega +\pi)\Sphi(\omega+\pi)
        \indentx\text{by \prefp{thm:Swfg}}
\end{align*}

\end{enumerate}
\end{proof}

%%=======================================
%\begin{definition}
%%=======================================
%Let $\wavsys$ be a wavelet system.
%Let $\oppS\seqn{\fphi}_m$ be the \hid{span} of the basis vectors $\seqn{\fphi}_m$.
%We define the following order relations. 
%\defbox{\begin{array}{rcll}
%%  \opP_m         &\orel&  \opP_n         & \quad\text{if}\quad \opP_m\opP_n=\opP_n\opP_m=\opP_m \\
%  \spV_m         &\orela& \spV_j         & \quad\text{if}\quad \spV_m \subseteq \spV_j \\
%  \seqn{\fphi}_m &\orelb& \seqn{\fphi}_n & \quad\text{if}\quad \oppS\seqn{\fphi}_m \subseteq \oppS\seqn{\fphi}_n %\qquad\text{where $\oppS$ is the span}
%\end{array}}
%\end{definition}
%



%The sequences of subspaces discussed in this section together with
%set relations $\subseteq$, $\setu$, and $\seti$,
%for a lattice.
%This ``\hie{lattice of wavelet subspaces}" is defined next.
%%---------------------------------------
%\begin{definition}
%\label{def:wav_lat_subspace}
%%---------------------------------------
%Let $\seqjZ{\spV_j}$ be a sequence of scaling subspaces and
%    $\seqjZ{\spW_j}$ be a sequence of wavelet subspaces.
%%    $\subseteq$ the set inclusion relation,
%%    $\odot$ the set union operation, and
%%    $\seti$ the set intersection operation.
%\defbox{\begin{array}{l}
%  \text{The tupple }
%  \qquad \wavlatsubs \\
%  \text{is called the \hid{lattice of wavelet subspaces}.}
%\end{array}}
%\end{definition}


%=======================================
\section{Sufficient condition}
%=======================================
In this text, an often used sufficient condition for designing the \structe{wavelet coefficient sequence} 
$\seqn{g_n}$ \xref{def:gn} is the \prope{conjugate quadrature filter condition} \xref{def:cqf}. 
It expresses the sequence $\seqn{g_n}$ in terms of the \structe{scaling coefficient sequence} \xref{def:hn}
and a ``shift" integer $\xN$ as $g_n = \pm(-1)^n h^\ast_{\xN-n}$.
The \structe{CQF condition} has the following ``nice" properties:
\\\indentx\begin{tabular}{>{\scs}rp{\tw-30mm}}
    1. & Given a \structe{scaling coefficient sequence} $\seqn{h_n}$ \xref{def:hn}, 
         it is extremely simple to compute the \structe{wavelet coefficient sequence} $\seqn{g_n}$ \xref{def:gn}.
  \\2. & If $\setn{\opTrn\fphi}$ of a \structe{wavelet system} $\wavsys$ \xref{def:wavsys} is \prope{orthonormal} and 
         $\otriple{\seqn{g_n}}{\seqn{h_n}}{\xN}$ satisfies the \prope{CQF condition}, 
         then $\setn{\opTrn^n\fpsi}$ is also \prope{orthnormal}\ifsxref{ortho}{thm:ortho_qmr}.
  \\3. & If $\setn{\opTrn\fphi}$ of a \structe{wavelet system} $\wavsys$ \xref{def:wavsys} is \prope{orthonormal} and 
         $\otriple{\seqn{g_n}}{\seqn{h_n}}{\xN}$ satisfies the \prope{CQF condition}, 
         then the \structe{wavelet subspace} $\spW_0$ is 
         \prope{orthnormal} to the \structe{scaling subspace} $\spV_0$ ($\spW_0\orthog\spV_0$)\ifsxref{ortho}{thm:ortho_qmr}.
\end{tabular}

%---------------------------------------
\begin{theorem}
\label{thm:wavstrct_cqf}
%---------------------------------------
Let $\wavsys$ be a \structe{wavelet system} \xref{def:wavsys}.
Let $\Dg(\omega)$ be the \ope{DTFT} \xref{def:dtft} and $\Zg(z)$ the \ope{Z-transform} \xref{def:opZ} of $\seqn{g_n}$.
\thmbox{
  \begin{array}{>{\ds}rc >{\ds}rcl @{\qquad}D}
  \mcom{g_n = \pm(-1)^n h^\ast_{\xN-n},\,{\scy\xN\in\Z}}{\structe{conjugate quadrature filter}}
      &\iff&     \Dg(\omega)                   &=& \pm (-1)^\xN e^{-i\omega\xN} \Dh^\ast(\omega+\pi)\Big|_{\omega=\pi}   & (1)
    \\&\implies& \sum_{n\in\Z} (-1)^n g_n      &=& \sqrt{2}                                                              & (2)
    \\&\iff&     \Zg(z)\Big|_{z=-1}            &=& \sqrt{2}                                                              & (3)
    \\&\iff&     \Dg(\omega)\Big|_{\omega=\pi} &=& \sqrt{2}                                                              & (4)
  \end{array}}
\end{theorem}
\begin{proof}
  \begin{enumerate}
    \item Proof that CQF$\iff$(1): by \prefp{thm:cqf}

    \item Proof that CQF$\implies$(4):
      \begin{align*}
        \Dg(\pi)
          &= \Dg(\omega)\Big|_{\omega=\pi}
        \\&= \pm (-1)^\xN e^{-i\omega\xN} \Dh^\ast(\omega+\pi)\Big|_{\omega=\pi}
          && \text{by \thme{CQF theorem}}
          && \text{\xref{thm:cqf}}
        \\&= \pm (-1)^\xN e^{-i\pi\xN} \Dh^\ast(2\pi)
        \\&= \pm (-1)^\xN (-1)^\xN \Dh^\ast(0)
          && \text{by \thme{DTFT periodicity}}
          && \text{\xref{prop:dtft_2pi}}
        \\&= \sqrt{2}
          && \text{by \thme{admissibility condition}}
          && \text{\xref{thm:admiss}}
      \end{align*}

    \item Proof that (2)$\iff$(3)$\iff$(4): by \prefp{prop:dsp_zminone}
  \end{enumerate}
\end{proof}

%=======================================
\section{Support size}
%=======================================
%--------------------------------------
\begin{theorem}[\thmd{support size}]
\citetbl{
  \citerppg{mallat}{243}{244}{012466606X}
  }
\label{thm:support}
%--------------------------------------
Let $\wavsys$ be a \structe{wavelet system} \xref{def:wavsys}
induced by the \thme{CQF conditions} \xref{thm:wavstrct_cqf}.
Let $\support\ff$ be the support of a function $\ff$ \xref{def:support}.
\thmbox{
  \begin{array}{>{\ds}rc>{\ds}l}
  \support\fphi &=& \support\fh
  \\
  %\xN\in\Zo \text{ and }  g_n  = \pm(-1)^\xN\fh(\xN-n) \quad\implies\quad
  \support\fpsi &=& \intcc{\frac{\xN-(n_2-n_1)}{2}}{\frac{\xN+(n_2-n_1)}{2}}
  \end{array}
  }
\end{theorem}
\begin{proof}
\begin{enumerate}
  \item Proof that $\support\fphi = \support\fh$: by \prefpp{thm:mra_support}

  %\item Definitions: \label{item:wavstrct_support_def}
  %  \\$\begin{array}{rcl}
  %    \support\fphi &\eqd& [a,b] \\
  %    \support\fh   &\eqd& [k,m].
  %   \end{array}$

  %\item lemma: \label{ilem:mra_support_lemma}
  %  \\$\support\fphi(x)=\brs{a,b} \quad\iff\quad \support\fphi(2x)=\brs{\frac{a}{2},\frac{b}{2}}$
  %
  %\item lemma: \label{ilem:mra_support_mpy}
  %  \\$\support\brs{\lambda\fphi(x)}=\support\brs{\fphi(x)}\quad\forall\lambda\in\R\setd0$.

  \item Proof that $\support \psi=\intcc{\frac{\xN-(n_2-n_1)}{2}}{\frac{\xN+(n_2-n_1)}{2}}$:
      \begin{align*}
        \support\fpsi(x)
          &= \support \brs{ \sum_{n\in\Z} g_n \opDil\opTrn^n\fphi(x) }
          && \text{by \thme{wavelet dilation equation}} 
          && \text{\xref{thm:g->psi}}
        \\&= \support \brs{\sqrt{2}\sum_{n\in\Z}  g_n \fphi(2x-n)}
          && \text{by definition of $\opTrn$ and $\opDil$} 
          && \text{\xref{def:opTD}}
        \\&= \support \brs{\sqrt{2}\sum_{n\in\Z} \pm (-1)^\xN \fh(\xN-n)\fphi(2x-n)}
          && \text{by \thme{CQF conditions}} && \text{\xref{thm:wavstrct_cqf}}
        \\&= \support \brs{\sum_{n\in\Z} \fh(\xN-n)\fphi(2x-n)}
          && \text{by \prefpp{ilem:mra_support_mpy}}
        \\&= \cls{\set{x\in\R}{\sum_{n\in\Z} \fh(\xN-n)\fphi(2x-n)\ne 0}}
          && \text{by definition of $\support$} && \text{\xref{def:support}}
        \\&= \intcc{\frac{n_1}{2}+\frac{\xN-n_2}{2}} {\frac{n_2}{2}+\frac{\xN-n_1}{2}}
        \\&= \intcc{\frac{\xN-(n_2-n_1)}{2}}         {\frac{\xN+(n_2-n_1)}{2}}
      \end{align*}
\end{enumerate}
\end{proof}

%--------------------------------------
\begin{example}
%--------------------------------------
Here are some examples using \fncte{Daubechies wavelet function}s.
\exbox{\begin{array}{NNN}
%    \includegraphics{graphics/b3_g.pdf}
    \includegraphics{graphics/d1_psi_g.pdf}
   &\includegraphics{graphics/d2_psi_g.pdf}
   &\includegraphics{graphics/d3_psi_g.pdf}
   %&\includegraphics{graphics/d4_psi_g.pdf}
  %\\\fncte{B-spline of order 2}
  \\\fncte{Daubechies-1}
   &\fncte{Daubechies-2}
   &\fncte{Daubechies-3}
   %&\fncte{Daubechies-4 wavelet function}
  \\\xref{ex:dau-p1}&\xref{ex:dau-p2}&\xref{ex:dau-p3}
\end{array}}
\end{example}

%=======================================
\section{Examples}
%=======================================
%Under the very general constraints of this chapter, I know of no wavelet examples.
No further examples of wavelets are presented in this section. 
Examples begin in the next chapter which is about a property called the \prope{partition of unity}.
%The very minimal of requirements, it seems is a \prope{partition of unity} \xref{chp:pounity}.
Other design constraints leading to wavelets with more ``powerful" properties include 
\prope{vanishing moments} \xref{chp:vanish}, \prope{orthonormality}\ifsxref{ortho}{chp:ortho},
\prope{compact support}\ifsxref{compactp}{chp:compactp}, and \prope{minimum phase} \xref{def:ztr_minphase}.

%Here are some examples of \structe{wavelet systems} \xref{def:wavsys}:
%\begin{longtable}{|l||ll||c|c|c|c|c|}
%  \hline
%  Name   & \mc{2}{||c|}{Reference} & \rotatebox{75}{\prope{partition of unity}} 
%                                   & \rotatebox{75}{\prope{vanishing moments}} 
%                                   & \rotatebox{75}{\prope{orthonormality}} 
%                                   & \rotatebox{75}{\prope{compact support}}
%                                   & \rotatebox{75}{\prope{minimum phase}}
%%         &          &       & \xref{chp:pounity}         & \xref{chp:vanish}         & \xref{chp:ortho}       & \xref{chp:compactp}
%  \\\hline
%  %Haar             & \scs\pref{ex:pun_n=2}       & \scs\prefpo{ex:pun_n=2}       & $\checkmark$ & 1 & $\checkmark$ & $\checkmark$ & $\checkmark$ \\
%  %order 1 B-spline & \scs\pref{ex:sw_gh_tent}    & \scs\prefpo{ex:sw_gh_tent}    & $\checkmark$ & 2 & $          $ & $          $ & $          $ \\
%  %order 3 B-spline & \scs\pref{ex:sw_gh_bspline} & \scs\prefpo{ex:sw_gh_bspline} & $\checkmark$ & 4 & $          $ & $          $ & $          $ \\
%  order 0 B-spline & \pref{ex:N0_hg}                          & \prefpo{ex:N0_hg} & $\checkmark$ & 1 & $\checkmark$ & $\checkmark$ & $\checkmark$ \\
%  order 1 B-spline & \pref{ex:N1_hg}                          & \prefpo{ex:N1_hg} & $\checkmark$ & 2 & $          $ & $          $ & $          $ \\
%  order 2 B-spline & \pref{ex:N2_hg}                          & \prefpo{ex:N2_hg} & $\checkmark$ & 4 & $          $ & $          $ & $          $ \\
%  Daubechies-p2    & \ifdochas{compactp}{\pref{ex:dau-p2}}    & \ifdochas{compactp}{\prefpo{ex:dau-p2}}        & $\checkmark$ & 2 & $\checkmark$ & $\checkmark$ & $\checkmark$ \\
%  Daubechies-p3    & \ifdochas{compactp}{\pref{ex:dau-p3}}    & \ifdochas{compactp}{\prefpo{ex:dau-p3}}        & $\checkmark$ & 3 & $\checkmark$ & $\checkmark$ & $\checkmark$ \\
%  Symlet-p4        & \ifdochas{compactp}{\pref{ex:symlet_p4}} & \ifdochas{compactp}{\prefpo{ex:symlet_p4}}     & $\checkmark$ & 4 & $\checkmark$ & $\checkmark$ & $          $ \\
%  \hline
%\end{longtable}




%================================================================================
%================================================================================
%================================================================================
%================================================================================
%================================================================================

%\paragraph{Fourier Transform.}
%One of the most widely used transforms is the Fourier transform.
%The Fourier Transform is an integral operator with an exponential kernel.
%And what is so special about exponential kernels?
%Is it just that they were discovered sooner than other kernels with other transforms?
%The answer in general is ``no".
%The exponential has two properties that makes it extremely special:
%  \begin{liste}
%    \item The exponential is an eigenvalue of any LTI operator \xref{thm:Le=he}
%    \item The exponential generates a continuous point spectrum for the differential operator
%          \xref{thm:spec_D}
%  \end{liste}
%
%\thmbox{
%  \left.\begin{array}{ll}
%    1. & \text{$\opL$ is linear and} \\
%    2. & \text{$\opL$ is time-invariant}
%  \end{array}\right\}
%  \qquad\implies\qquad
%    \opL \lkerne{t}{s}
%    =
%    \mcomr{\Lh^\ast(-s)}{eigenvalue} \mcoml{\lkerne{t}{s}}{eigenvector}
%  }
%
%
%What makes wavelet system unique from other analysis systems
%(such as Fourier analysis) is its \hie{subspace architecture}.
%In this section, we present this architecture using four representations:
%
%\begin{tabular}{lp{\tw/2}ll}
%  \circOne   & lattice of subspaces             \dotfill&\pref{sec:wav_lat_subspace} & \xref{sec:wav_lat_subspace} \\
%  \circTwo   & lattice of projection operators  \dotfill&\pref{sec:wav_lat_op}       & \xref{sec:wav_lat_op} \\
%  \circThree & lattice of bases vectors         \dotfill&\pref{sec:wav_lat_bases}    & \xref{sec:wav_lat_bases} \\
%  \circFour  & lattice of bases coefficients    \dotfill&\pref{sec:wav_lat_coef}     & \xref{sec:wav_lat_coef} \\
%\end{tabular}
%
%\prefp{thm:wav_lat_iso} will show that all four of these representations are
%essentially equivalent (are isomorphisms).
%The wavelet system itself is simply a collection of the projection operators
%\xref{def:wav_transform} found in the wavelet operator lattice.



%  %============================================================================
% LaTeX File
% Daniel J. Greenhoe
%============================================================================

%======================================
\section{Operations on Sequences}
%\label{app:dsp}
%\label{app:ztrans}
%======================================
%======================================
%\subsection{Operations}
%======================================
%======================================
\subsection{Convolution operation}
%======================================
%--------------------------------------
\begin{definition}
\label{def:sequence}
\footnote{
  \citerp{bromwich1908}{1},
  \citerpgc{thomson2008}{23}{143484367X}{Definition 2.1},
  \citerpg{joshi1997}{31}{8122408265}
  }
\label{def:tuple}
\label{def:seq}
%--------------------------------------
Let $\clFyx$ be the set of all functions from a set $\setY$ to a set $\setX$.
Let $\Z$ be the set of integers.
\defbox{\begin{array}{l}
  \text{A function $\ff$ in $\clFyx$ is a \structd{sequence} over $\setX$ if $\setY=\Z$.}\\
  \text{A sequence may be denoted in the form $\ds\seqxZ{x_n}$ or simply as $\ds\seqn{x_n}$.}
  %\text{A function $\ff$ in $\clFyx$ is an \hid{n-tuple} over $\setX$ if $\setY=\setn{1,2,\ldots,\xN}$.}\\
  %\text{An n-tuple may be denoted in the form $\ds\tuplexn{x_n}$ or simply as $\ds\tuplen{x_n}$.}
\end{array}}
\end{definition}

%--------------------------------------
\begin{definition}
\footnote{
  \citerpgc{kubrusly2011}{347}{0817649972}{Example 5.K}
  }
\label{def:spllR}
\label{def:spllC}
\label{def:spllF}
%--------------------------------------
%Let $\fieldF$ be a \structe{field}. % \xref{def:field}.
Let $\fieldC$ be the \structe{field of complex numbers}.
\defboxt{
  The \structd{space of all absolutely square summable sequences} $\spllC$ over $\C$ is defined as
  \\\indentx$\ds\spllC\eqd\set{\seqxZ{x_n}}{\sum_{n\in\Z}\abs{x_n}^2 < \infty}$
  }
\end{definition}

%The space $\spllC$ is an example of a \structe{separable Hilbert space}.
%In fact, $\spllC$ is the \emph{only} separable Hilbert space in the sense that all separable Hilbert spaces
%are isomorphically equivalent.
%For example, $\spllC$ is isomorphic to $\spLLR$, the \structe{space of all absolutely square Lebesgue integrable functions}.
%%That is, their topological structure is the same.
%%Differences occur in the nature of operators on the spaces.

%--------------------------------------
\begin{definition}
\label{def:dsp_conv}
\label{def:convd}
\index{convolution}
%--------------------------------------
%Let $\seq{x_n}{n\in\Z}$ and $\seq{y_n}{n\in\Z}$ be sequences \xref{def:seq} in the space $\spllC$ \xref{def:spllR}.
\defbox{\begin{array}{M}
  The \opd{convolution} operation $\hxs{\convd}$ is defined as
  \\\indentx
  $\ds{\seqn{x_n}\convd\seqn{y_n}} \eqd \seq{\sum_{m\in\Z} x_{m} y_{n-m}}{n\in\Z}\qquad\scy\forall\seqxZ{x_n},\seqxZ{y_n}\in\spllC$
\end{array}}
\end{definition}

%======================================
\subsection{Z-transform}
%======================================
%--------------------------------------
\begin{definition}
\footnote{
  \structe{Laurent series}: \citerpg{aa}{49}{0821821466}
  }
\label{def:opZ}
%--------------------------------------
Let $\seqnZ{\fx(n)}$ be a sequence.
\defboxt{
%Let $\seq{x_n}{n\in\Z}$ be a sequence in the space $\spllC$. %over a ring $\ring$.
  The \opd{z-transform} $\opZ$ of $\seqn{\fx(n)}$ is defined as
  \\\indentx
  $\ds\brs{\hxs{\opZ}\seqn{\fx(n)}}(z) \eqd {\sum_{n\in\Z} \fx(n) z^{-n}}\qquad\scy\forall\seqn{\fx(n)}\in\spllC$
  }
\end{definition}

%--------------------------------------
\begin{proposition}
\label{prop:opZ}
%--------------------------------------
Let $X(z)\eqd\opZ\fx(n)$ be the \ope{z-transform} of $\fx(n)$.
\thmbox{
  \mcom{\brb{\begin{array}{rcl}
    \Zx(z) &\eqd& \opZ\seqn{\fx(n)}
  \end{array}}}{\xref{def:opZ}}
  \quad\implies\quad
  \brb{\begin{array}{F>{\ds}lc>{\ds}lCD}
      (1).&\opZ\seqn{\alpha\fx(n)} &=& \alpha\Zx(z)                    & \forall\seqn{x_n}\in\spllC & and
    \\(2).&\opZ\seqn{\fx[n-k]}     &=& z^{-k}\Zx(z)                    & \forall\seqn{x_n}\in\spllC & and
    \\(3).&\opZ\seqn{\fx(-n)}      &=& \Zx\brp{\frac{1}{z}}            & \forall\seqn{x_n}\in\spllC & and
    \\(4).&\opZ\seqn{\fx^\ast(n)}  &=& \Zx^\ast\brp{z^\ast}            & \forall\seqn{x_n}\in\spllC & and
    \\(5).&\opZ\seqn{\fx^\ast(-n)} &=& \Zx^\ast\brp{\frac{1}{z^\ast}}  & \forall\seqn{x_n}\in\spllC &
  \end{array}}
  }
\end{proposition}
\begin{proof}
\begin{align*}
  \alpha\Z\Zx(z)
    &\eqd \alpha \opZ \seqn{\fx(n)}                && \text{by definition of $\Zx(z)$}
  \\&\eqd \alpha \sum_{n\in\Z} \fx(n) z^{-n}       && \text{by definition of $\opZ$ operator}
  \\&\eqd \sum_{n\in\Z} \brp{\alpha\fx(n)} z^{-n}  && \text{by \prope{distributive} property}
  \\&\eqd \opZ\seqn{\alpha\fx(n)}                  && \text{by definition of $\opZ$ operator}
  \\
  z^{-k}\Zx(z)
    &= z^{-k} \opZ\seqn{\fx(n)}
    && \text{by definition of $\Zx(z)$}
    && \text{(left hypothesis)}
  \\&\eqd z^{-k}\sum_{n=-\infty}^{n=+\infty} \fx(n) z^{-n}
    && \text{by definition of $\opZ$}
    && \text{\xref{def:opZ}}
  \\&=          \sum_{n=-\infty}^{n=+\infty} \fx(n) z^{-n-k}
  \\&=          \sum_{m-k=-\infty}^{m-k=+\infty} \fx[m-k] z^{-m}
    && \text{where $m\eqd n+k$}
    && \text{$\implies$ $n=m-k$}
  \\&=          \sum_{m=-\infty}^{m=+\infty} \fx[m-k] z^{-m}
  \\&=          \sum_{n=-\infty}^{n=+\infty} \fx[n-k] z^{-n}
    && \text{where $n\eqd m$}
  \\&\eqd \opZ\seqn{\fx[n-k]}
    && \text{by definition of $\opZ$}
    && \text{\xref{def:opZ}}
  \\
  \opZ\seqn{\fx^\ast(n)}
    &\eqd \sum_{n\in\Z}\fx^\ast(n) z^{-n}
    && \text{by definition of $\opZ$}
    && \text{\xref{def:opZ}}
  \\&\eqd \brp{\sum_{n\in\Z}\fx(n) (z^\ast)^{-n}}^\ast
    && \text{by definition of $\opZ$}
    && \text{\xref{def:opZ}}
  \\&\eqd \Zx^\ast(z^\ast)
    && \text{by definition of $\opZ$}
    && \text{\xref{def:opZ}}
  \\
  \opZ\seqn{\fx(-n)}
    &\eqd \sum_{n\in\Z}\fx(-n) z^{-n}
    && \text{by definition of $\opZ$}
    && \text{\xref{def:opZ}}
  \\&= \sum_{-m\in\Z}\fx[m] z^{m}
    && \text{where $m\eqd -n$}
    && \text{$\implies$ $n=-m$}
  \\&= \sum_{m\in\Z}\fx[m] z^{m}
    && \text{because $\seqn{\fx(n)},\seqn{z^n}\in\spllC$}     && \text{\xref{def:spllC}}
  \\&= \sum_{m\in\Z}\fx[m] \brp{\frac{1}{z}}^{-m}
  \\&\eqd \Zx\brp{\frac{1}{z}}
    && \text{by definition of $\opZ$}
    && \text{\xref{def:opZ}}
  \\
  \opZ\seqn{\fx^\ast(-n)}
    &\eqd \sum_{n\in\Z}\fx^\ast(-n) z^{-n}
    && \text{by definition of $\opZ$}
    && \text{\xref{def:opZ}}
  \\&= \sum_{-m\in\Z}\fx^\ast[m] z^{m}
    && \text{where $m\eqd -n$}
    && \text{$\implies$ $n=-m$}
  \\&= \sum_{m\in\Z}\fx^\ast[m] z^{m}
    && \text{because $\seqn{\fx(n)},\seqn{z^n}\in\spllC$}     && \text{\xref{def:spllC}}
  \\&= \sum_{m\in\Z}\fx^\ast[m] \brp{\frac{1}{z}}^{-m}
  \\&= \brp{\sum_{m\in\Z}\fx[m] \brp{\frac{1}{z^\ast}}^{-m}}^\ast
  \\&\eqd \Zx^\ast\brp{\frac{1}{z^\ast}}
    && \text{by definition of $\opZ$}
    && \text{\xref{def:opZ}}
\end{align*}
\end{proof}

%--------------------------------------
\begin{proposition}[\thmd{Convolution Theorem}]
\label{prop:conv}
%--------------------------------------
Let $\convd$ be the convolution operator \xref{def:dsp_conv}.
%$\seq{x_n}{n\in\Z}$ and $\seq{y_n}{n\in\Z}$ be sequences in the space $\spllC$. %be sequences over a ring $\ring$.
\thmbox{
  \opZ\mcom{\brp{\seqn{x_n}\convd\seqn{y_n}}}{sequence convolution} = \mcom{\brp{\opZ\seqn{x_n}}\;\brp{\opZ\seqn{y_n}}}{series multiplication}
  \qquad{\scy\forall\seqxZ{x_n},\seqxZ{y_n}\in\spllC}
  }
\end{proposition}
\begin{proof}
\begin{align*}
  [\opZ(x\convd y)](z)
    &\eqd \opZ {\left(\sum_{m\in\Z} x_m y_{n-m}\right)}
    &&    \text{by \prefp{def:dsp_conv}}
  \\&\eqd \sum_{n\in\Z} \sum_{m\in\Z} x_m y_{n-m} z^{-n}
    &&    \text{by \prefp{def:opZ}}
  \\&=    \sum_{n\in\Z} \sum_{m\in\Z} x_m y_{n-m} z^{-n}
  \\&=    \sum_{m\in\Z} \sum_{n\in\Z} x_m y_{n-m} z^{-n}
  \\&=    \sum_{m\in\Z} \sum_{k\in\Z} x_m y_k z^{-(m+k)}
    &&    \text{where $k=n-m \iff n=m+k$}
  \\&=    {\left[\sum_{m\in\Z} x_m z^{-m}\right]}
          {\left[\sum_{k\in\Z} y_k z^{-k}\right]}
  \\&\eqd \brp{\opZ\seqn{x_n}}\;\brp{\opZ\seqn{y_n}}
    &&    \text{by \prefp{def:opZ}}
\end{align*}
\end{proof}

%---------------------------------------
\begin{lemma}
\label{lem:real_xyh}
%---------------------------------------
Let $\opH$ be a \prope{linear time-invariant} operator with \fncte{impulse response} $\seqn{\fh(n)}$.
Let $\seqn{\fy(n)}\eqd\seqn{\opH\fx(n)}$.
\lembox{
  \brb{\begin{array}{FMD}
      (A). & $\seqn{\fx(n)}$ and $\seqn{\fy(n)}$ are \prope{real-valued}  & and
    \\(B). & $\seqn{\fx(n)}$ and $\seqn{\fh(n)}$  are in $\spllC$         & and
    \\(C). & $\seqn{\fx(n)}\neq\seqn{\cdots,0,0,0,\cdots}$                & and
    \\(D). & $\seqn{\fh(n)}$ is \prope{linear time-invariant}
  \end{array}}
  \implies
  \brb{\begin{array}{FMD}
      (1). & $\seqn{\fh(n)}$ is \prope{real-valued} & and
    \\(2). & $\ZH(z) = \ZH^*\brp{z^\ast}$
  \end{array}}
  }
\end{lemma}
\begin{proof}
\begin{enumerate}
  \item Let $\fh_R(n)$ and $\fh_I(n)$ be the \structe{real-part} and \structe{imaginary-part}, respectively,
        of $\fh(n)$. \label{item:real_xyh_def}
  \item lemma: $\sum_{m\in\Z} \fh_I(m)\fx(n-m) = 0$ \label{ilem:real_xyh_lem}
    \begin{align*}
      &\sum_{m\in\Z} \fh_R(m)\fx(n-m) + i\sum_{m\in\Z} \fh_I(m)\fx(n-m)
      \\&= \sum_{m\in\Z} \fh(m)\fx(n-m)
        && \text{by definitions of $\fh_R$ and $\fh_I$}                && \text{\pref{item:real_xyh_def}}
      \\&= \fy(n)
        && \text{because $\opH$ is \prope{LTI}}                        && \text{hypothesis (D)}
      \\&= \fy^\ast(n)
        && \text{because $\fy$ is \prope{real-valued}}                 && \text{hypothesis (A)}
      \\&= \brp{\sum_{m\in\Z} \fh(m)\fx(n-m)}^\ast
        && \text{because $\opH$ is \prope{LTI}}                        && \text{hypothesis (D)}
      \\&= \sum_{m\in\Z} \fh^\ast(m)\fx^\ast(n-m)
        && \text{by \prope{antiautomorphic} property}                  && \text{\xref{def:staralg}}
       %&& \text{by \prope{antiautomorphic} property of *-algebras}    && \text{\xref{def:staralg}}
      \\&= \sum_{m\in\Z} \fh^\ast(m)\fx(n-m)
        && \text{because $\fy$ is \prope{real-valued}}                 && \text{hypothesis (A)}
      \\&= \sum_{m\in\Z} \fh_R(m)\fx(n-m) - i\sum_{m\in\Z} \fh_I(m)\fx(n-m)
        && \text{by definitions of $\fh_R$ and $\fh_I$}                && \text{\pref{item:real_xyh_def}}
      \\
      \implies&
      \boxed{\sum_{m\in\Z} \fh_I(m)\fx(n-m) = 0}
    \end{align*}

  \item Notes:
    \begin{enumerate}
      \item Without hypothesis (C), it is trivial to satisfy \pref{ilem:real_xyh_lem}.

      \item Without hypothesis (B), it is simple to satisfy \pref{ilem:real_xyh_lem} with
        \\$\fh(n)=\seqn{\cdots,0,0,0,i,-i,0,0,0,\cdots}$ and $\fx(n)=\seqn{\cdots,1,1,1,\cdots}$

      \item Without hypothesis (D), it is trivial to satisfy \pref{ilem:real_xyh_lem} with
        $\ds\opH\fx(n)\eqd\Real\brp{\sum_{m\in\Z}\fh(m)\fx(n-m)}$
    \end{enumerate}

  \item Proof that $\fh(n)$ is \prope{real-valued}:\label{item:real_xyh_realh}
    \begin{align*}
      \text{\pref{ilem:real_xyh_lem}}
        &\implies \ZH_I(z)\ZX(z) = 0
        && \text{by \thme{Convolution Theorem}}                  && \text{\xref{prop:conv}}
      \\&\implies \ZH_I(z) = 0
        && \text{because $\fx(n)\neq\seqn{\cdots,0,0,0,\cdots}$} && \text{hypothesis (C)}
      \\&\implies \fh_I(n) = \seqn{\cdots,0,0,0,\cdots}
      \\&\implies \text{$\fh(n)\eqd\fh_R(n)+i\fh_I(n)$ is \prope{real-valued}}
    \end{align*}

  \item Proof that $\ZH(z) = \ZH^*\brp{z^\ast}$:
    \begin{align*}
      \ZH(z)
        &\eqd \opZ\seqn{\fx(n)}
        && \text{by definition of $\ZH(z)$}
      \\&= \opZ\seqn{\fx^\ast(n)}
        && \text{because $\fx(n)$ is \prope{real-valued}} && \text{\xref{item:real_xyh_realh}}
      \\&= \ZH^*\brp{z^\ast}
        && \text{by \pref{prop:opZ}}
    \end{align*}
\end{enumerate}
\end{proof}

%---------------------------------------
\begin{lemma}
\label{lem:real_FH}
%---------------------------------------
Let $\FH(\omega)$ be the DTFT \xref{def:dtft} of a sequence $\fh(n)$.
\lembox{
  \brb{\begin{array}{M}
      $\fh(n)$ is \prope{real-valued}
  \end{array}}
  \implies
  \brb{\begin{array}{rclD}
      \FH(-\omega) &=& \FH^\ast(\omega) & (\prope{conjugate symmetric})
  \end{array}}
  }
\end{lemma}
\begin{proof}
\begin{align*}
  \FH(-\omega)
    &\eqd \sum_{n\in\Z} \fh(n) e^{-i(-\omega)n}
    && \text{by definition of $\FH(\omega)$}  &&\text{\xref{def:dtft}}
  \\&= \sum_{n\in\Z} \fh(n) e^{i\omega n}
  \\&= \brs{\sum_{n\in\Z} \fh^\ast(n) e^{i\omega n}}^\ast
    && \text{by \prope{antiautomorphic} property of *-algebras}    && \text{\xref{def:staralg}}
  \\&= \brs{\sum_{n\in\Z} \fh(n) e^{-i\omega n}}^\ast
    && \text{by \prope{real-valued} hypothesis}
  \\&\eqd \FH^\ast(\omega)
    && \text{by definition of $\FH(\omega)$}  &&\text{\xref{def:dtft}}
\end{align*}
\end{proof}

%  %============================================================================
% Daniel J. Greenhoe
% XeLaTeX file
%============================================================================

%=======================================
\chapter{Fast Wavelet Transform (FWT) }
\index{fast wavelet transform}
\index{FWT}
%=======================================
The Fast Wavelet Transform can be computed using simple 
discrete filter operations (as a conjugate mirror filter).

%--------------------------------------
\begin{definition}[Wavelet Transform]
\label{def:wt}
\index{wavelet transform}
%--------------------------------------
Let the wavelet transform 
$\opW:\{f:\R\to\C\}\to \{w:\Z^2\to\C\}$ be defined as \footnotemark
\defbox{\begin{array}{rcl}
   [\opW \ff](j,n) &\eqd& \inprod{\ff(x)}{\fpsi_{k,n}(x)} 
\end{array}}
\end{definition}
\footnotetext{
Notice that this definition is similar to the definition of transforms 
of other analysis systems:
\[
\begin{array}{cllllll}
   \imark                             &
   \mbox{Laplace Transform}            & 
   \mathcal{L}f(s)                     & 
   \eqd& \inprod{\ff(x)}{e^{sx}}       & 
   \eqd& \int_x \ff(x) e^{-sx} \dx 
   \\
   \imark                             &
   \mbox{Continuous Fourier Transform} & 
   \mathcal{F}f(\omega)                & 
   \eqd& \inprod{\ff(x)}{e^{i\omega x}} & 
   \eqd& \int_x \ff(x) e^{-i\omega x} \dx 
   \\
   \imark                             &
   \mbox{Fourier Series Transform}     & 
   \mathcal{F}_sf(k)                     & 
   \eqd& \inprod{\ff(x)}{e^{i\frac{2\pi}{T}kx}} & 
   \eqd& \int_x \ff(x) e^{-i\frac{2\pi}{T}kx} \dx 
   \\
   \imark                             &
   \mbox{Z-Transform}                  & 
   \mathcal{Z}f(z)                     & 
   \eqd& \inprod{\ff(x)}{z^n} & 
   \eqd& \sum_n \ff(x) z^{-n} 
   \\
   \imark                             &
   \mbox{Discrete Fourier Transform}                  & 
   \mathcal{F}_df(k)                     & 
   \eqd& \inprod{f(n)}{e^{i\frac{2\pi}{N}kn}} & 
   \eqd& \sum_n \ff(x) e^{-i\frac{2\pi}{N}kn} 
   \\
\end{array}
\]
}

%--------------------------------------
\begin{definition}
\index{scaling coefficients}
\index{wavelet coefficients}
\index{scaling filter coefficients}
\index{wavelet filter coefficients}
%--------------------------------------
The following relations are defined as described below:
\defbox{\begin{array}{lllrcl}
   \mbox{scaling coefficients} & v_j:\Z\to\C & \mbox{such that} & 
   v_j(n) &\eqd& \ds \inprod{f(x)}{\fphi_{j,n}(x)}
\\
   \mbox{wavelet coefficients} & w_j:\Z\to\C & \mbox{such that} & 
   w_j(n) &\eqd& \ds \inprod{f(x)}{\fpsi_{j,n}(x)}
\\
   \mbox{scaling filter coefficients} & \fhb:\Z\to\C & \mbox{such that} & 
   \fhb(n) &\eqd& \ds \fh(-n)
\\
   \mbox{wavelet filter coefficients} & \fgb:\Z\to\C & \mbox{such that} & 
   \fgb(n) &\eqd& \ds \fg(-n)
\end{array}}
\end{definition}

The scaling and wavelet filter coefficients at scale~$j$ are 
equal to the filtered and downsampled \xref{thm:downsample}
scaling filter coefficients at scale~$j+1$:%
\footnote{
  \citerp{mallat}{257},
  \citerp{burrus}{35}
  }

\begin{liste}
   \item The convolution \xref{def:convd} of $v_{j+1}(n)$ with $\fhb(n)$ 
         and then downsampling by 2 produces $v_j(n)$.
   \item The convolution of $v_{j+1}(n)$ with $\fgb(n)$ 
         and then downsampling by 2 produces $w_j(n)$.
\end{liste}
This is formally stated and proved in the next theorem.
%--------------------------------------
\begin{theorem}
\index{scaling filters}
\index{wavelet filters}
%--------------------------------------
\thmbox{
   \begin{array}{rcl}
      v_j(n) &=& [\fhb \conv v_{j+1}](2n) \\
      w_j(n) &=& [\fgb \conv v_{j+1}](2n)
   \end{array}
}
\end{theorem}
\begin{proof}
\begin{align*}
    v_j(n) 
      &= \inprod{\ff(x)}{\fphi_{j,n}(x)} 
    \\&= \inprod{\ff(x)}{\sqrt{2^j}\fphi\left(2^jx-n \right)}
    \\&= \inprod{\ff(x)}{\sqrt{2^j}\sqrt{2}\sum_m h(m)\fphi\left(2(2^jx-n)-m \right)}
    \\&= \inprod{\ff(x)}{\sum_m h(j)\sqrt{2^{j+1}}\fphi\left(2^{j+1}x-2n -m \right)}
    \\&= \sum_m h(m)\inprod{\ff(x)}{\sqrt{2^{j+1}}\fphi\left(2^{j+1}x-2n -m \right)}
    \\&= \sum_m h(m)\inprod{\ff(x)}{\fphi_{j+1,2n+m}(x)}
    \\&= \sum_m h(m)v_{j+1}(2n+m)
    \\&= \sum_p h(p-2n)v_{j+1}(p)
      && \text{let }p=2n+m \iff m=p-2n
    \\&= \sum_p \fhb(2n-p) v_{j+1}(p)
    \\&= [\fhb \conv v_{j+1}](2n)
\\ 
\\
    w_j(n)
      &= \inprod{\ff(x)}{\fpsi_{j,n}(x)} 
    \\&= \inprod{\ff(x)}{\sqrt{2^j}\fpsi\left(2^jx-n \right)}
    \\&= \inprod{\ff(x)}{\sqrt{2^j}\sqrt{2}\sum_m \fg(j)\fphi\left(2(2^jx-n)-m \right)}
    \\&= \inprod{\ff(x)}{\sum_m \fg(m)\sqrt{2^{j+1}}\fphi\left(2^{j+1}x-2n -m \right)}
    \\&= \sum_m \fg(m)\inprod{\ff(x)}{\sqrt{2^{j+1}}\fphi\left(2^{j+1}x-2n -m \right)}
    \\&= \sum_m \fg(m)\inprod{\ff(x)}{\fphi_{j+1,2n+m}(x)}
    \\&= \sum_m \fg(m)v_{j+1}(2n+m)
    \\&= \sum_p \fg(p-2n)v_{j+1}(p)
      && \text{let }p=2n+m \iff m=p-2n
    \\&= \sum_p \fgb(2n-p)v_{j+1}(p) 
    \\&= [\fgb \conv v_{j+1}](2n)
   \end{align*}
\end{proof}

\begin{figure}[h] %\color{figcolor}
\centering\gsize%============================================================================
% Daniel J. Greenhoe
% XeLaTeX file
%============================================================================
{\setlength{\unitlength}{9mm}
\color{blue}%
\footnotesize%
\begin{tabular}{c}
\begin{picture}(16,1.65)
\thicklines
\put( 5,   1.1){\makebox(2,1)[b]{$v_j(n)=\inprod{f(x)}{\fphi_{j,n}(x)}$ }}
\put( 6  , 1  ){\vector(0,-1){1  } }
\put( 6  , 0.5){\line  (1, 0){3  } }
\put( 9  , 0.5){\vector(0,-1){0.5} }
\end{picture}
\\
\begin{picture}(16,4)
\thinlines
\put( 5.5, 3  ){\framebox(1,1){$\fhb(n)$} }
\put( 8.5, 3  ){\framebox(1,1){$\fgb(n)$} }
\put( 6  , 3  ){\vector(0,-1){0.5} }
\put( 9  , 3  ){\vector(0,-1){0.5} }
\put( 5.5, 1.5){\framebox(1,1){$\downarrow 2$} }
\put( 8.5, 1.5){\framebox(1,1){$\downarrow 2$} }
\put( 6  , 1.5){\vector(0,-1){1.5} }
\put( 9  , 1.5){\line  (0,-1){0.5} }
\put( 9  , 1  ){\vector(1, 0){1  } }
\put( 0,   0  ){\makebox(5,2)[r]{$v_{k-1}(n) = \inprod{\ff(x)}{\fphi_{k-1,n}(x)}$ }}
\put(10.5, 0  ){\makebox(5,2)[l]{$w_{k-1}(n) = \inprod{\ff(x)}{\fpsi_{k-1,n}(x)}$ }}
\put( 6  , 0.5){\line  (1, 0){3  } }
\put( 9  , 0.5){\vector(0,-1){0.5} }
\end{picture}
\\
\begin{picture}(16,4)
\thinlines
\put( 5.5, 3  ){\framebox(1,1){$\fhb(n)$} }
\put( 8.5, 3  ){\framebox(1,1){$\fgb(n)$} }
\put( 6  , 3  ){\vector(0,-1){0.5} }
\put( 9  , 3  ){\vector(0,-1){0.5} }
\put( 5.5, 1.5){\framebox(1,1){$\downarrow 2$} }
\put( 8.5, 1.5){\framebox(1,1){$\downarrow 2$} }
\put( 6  , 1.5){\vector(0,-1){1.5} }
\put( 9  , 1.5){\line  (0,-1){0.5} }
\put( 9  , 1  ){\vector(1, 0){1  } }
\put( 0,   0  ){\makebox(5,2)[r]{$v_{k-2}(n) = \inprod{\ff(x)}{\fphi_{k-2,n}(x)}$ }}
\put(10.5, 0  ){\makebox(5,2)[l]{$w_{k-2}(n) = \inprod{\ff(x)}{\fpsi_{k-2,n}(x)}$ }}
\put( 6  , 0.5){\line  (1, 0){3  } }
\put( 9  , 0.5){\vector(0,-1){0.5} }
\end{picture}
\\
$\vdots$\hspace{3cm}$\vdots$\hspace{1cm}
\\
\begin{picture}(16,1.5)
\thinlines
\put( 6  , 1.5){\vector(0,-1){1.5} }
\put( 0,   0  ){\makebox(5,2)[r]{$v_{1}(n) = \inprod{\ff(x)}{\fphi_{1,n}(x)}$ }}
\put(10.5, 0  ){\makebox(5,2)[l]{$w_{1}(n) = \inprod{\ff(x)}{\fpsi_{1,n}(x)}$ }}
\put( 6  , 0.5){\line  (1, 0){3  } }
\put( 9  , 0.5){\vector(0,-1){0.5} }
\end{picture}
\\
\begin{picture}(16,4)
\thinlines
\put( 5.5  , 3  ){\framebox(1,1){$\fhb(n)$} }
\put( 8.5  , 3  ){\framebox(1,1){$\fgb(n)$} }
\put( 6  , 3  ){\vector(0,-1){0.5} }
\put( 9  , 3  ){\vector(0,-1){0.5} }
\put( 5.5, 1.5){\framebox(1,1){$\downarrow 2$} }
\put( 8.5, 1.5){\framebox(1,1){$\downarrow 2$} }
\put( 6  , 1.5){\vector(0,-1){0.5} }
\put( 9  , 1.5){\line  (0,-1){0.5} }
\put( 9  , 1  ){\vector(1, 0){1  } }
\put( 0,   0  ){\makebox(5,2)[r]{$v_0(n) = \inprod{\ff(x)}{\fphi(x-n)}$ }}
\put(10.5, 0  ){\makebox(5,2)[l]{$w_0(n) = \inprod{\ff(x)}{\fpsi(x-n)}$ }}
%\put( 6  , 0.5){\line  (1, 0){3  } }
%\put( 9  , 0.5){\vector(0,-1){0.5} }
\end{picture}
\end{tabular}}



\caption{
   $k$-Stage Fast Wavelet Transform
   \label{fig:fwt}
   }
\end{figure}
These filtering and downsampling operations are equivalent to the 
operations performed by a filter bank.  
Therefore, a filter bank can be 
used to implement a \ope{Fast Wavelet Transform} (\ope{FWT}), 
as illustrated in \prefpp{fig:fwt}.

%\begin{center}
%\begin{figure}[h] \color{figcolor}
%\begin{small}
%\begin{tabular}{c}
%   \ingr{\tw-5mm}{20mm}{../common/wavelets/pare.eps} \\
%      $\ff(x)$: truncated parabola                           \\
%   \ingr{\tw/2-10mm}{60mm}{../common/wavelets/d2_par_fwte.eps} \\
%      $[\opW \ff](k,n)$ using Haar                     \\
%   \ingr{\tw/2-10mm}{60mm}{../common/wavelets/d4_par_fwte.eps} \\
%      $[\opW \ff](k,n)$ using Daubechies-4            
%\end{tabular}
%\end{small}
%\caption{Fast Wavelet Transforms using Haar and Daubechies-4 basis}
%\end{figure}
%\end{center}





