%============================================================================
% Daniel J. Greenhoe
% XeLaTeX file
% graph theory
%============================================================================
%=======================================
%\subsection{Probability spaces}
%=======================================
  \qboxnpq
    {Joseph Leonard Doob (1910--2004), pioneer of and key contributor to mathematical probability\footnotemark}
    {../common/people/doobjl_dartmouthedu.jpg}
    {While writing my book I had an argument with Feller. 
     He asserted that everyone said ``random variable" and I asserted that everyone said ``chance variable." 
     We obviously had to use the same name in our books, so we decided the issue by a stochastic procedure. 
     That is, we tossed for it and he won.}
  \footnotetext{
    quote: \citePp{snell1997}{307}, 
           \citePp{snell2005}{251}.
    %reference:   \citerpg{suhov2008}{238}{0521847672}.
    image: \url{http://www.dartmouth.edu/~chance/Doob/conversation.html}
    }

%A \structe{metric space} (next definition) is a special case of a \structe{distance space} \xref{def:distance}.
%To be more specific, a \fncte{metric} is a \fncte{distance} function for which the \prope{triangle inequality} property holds.
%%---------------------------------------
%\begin{definition}
%\citetbl{
%  \citerp{dieudonne1969}{28},
%  \citerp{copson1968}{21},
%  \citerp{hausdorff1937e}{109},
%  \citer{frechet1928},
%  \citerp{frechet1906}{30} 
%  %\citor{hausdorff1914}\\
%  %\cithrpg{ab}{34}{0120502577} 
%  }
%\label{def:metric}
%\label{def:mspace}
%%\label{def:(X,d)}
%%---------------------------------------
%\defboxp{
%  A function $\hxs{\metricn}\in\clF{\setX\times\setX}{\Rnn}$ is a \fnctd{metric} on a \structe{set} $\setX$ if
%  \\\indentx$\begin{array}{D rcl @{\qquad}C @{\qquad}DD}
%      1. & \metric{x}{y} &\ge& 0                                 & \forall x,y   \in\setX & (\prope{non-negative})   & and 
%    \\2. & \metric{x}{y} &=  & 0  \iff x=y                       & \forall x,y   \in\setX & (\prope{nondegenerate})  & and 
%    \\3. & \metric{x}{y} &=  & \metricn(y,x)                     & \forall x,y   \in\setX & (\prope{symmetric})      & and 
%    \\4. & \metric{x}{y} &\le& \metric{x}{z}+\metric{z}{y}       & \forall x,y,z \in\setX & (\prope{subadditive}/\prope{triangle inequality}).\footnotemark
%  \end{array}$\\
%  The \structe{ordered pair} $\metspaceX$ is a \structd{metric space} if $\setX$ is a \structe{set} and $\metricn$
%  is a \fncte{metric} on $\setX$.
%  }
%\end{definition}
%\citetblt{\prope{triangle inequality}: \citorc{euclid}{Book I Proposition 20}}
%
%%---------------------------------------
%\begin{definition}
%\citetbl{
%  \citerpgc{busemann1955}{4}{0486154629}{{\scshape Comments on the axioms}},
%  %\citerpg{davis2005}{17}{0071243399} \\
%  \citerp{giles1987}{13},
%  \citerp{copson1968}{24},
%  \citerpgc{khamsi2001}{19}{0471418250}{Example 2.1}
%  }
%\index{metrics!discrete}
%\index{discrete metric}
%%\label{ex:d_discrete}
%\label{def:discretemetric}
%%---------------------------------------
%Let $\setX$ be a set and $\metricn\in\clFxxr$.
%The function $\metricn$ is the \fnctd{discrete metric} on $\clFxxr$ if
%\\\indentx$\metric{x}{y}\eqd\brb{\begin{array}{cM}
%  0 & if $x=y$\\
%  1 & otherwise
%\end{array}} \qquad\scy\forall x,y\in\setX$
%\end{definition}
%
%
%


%---------------------------------------
\begin{definition}
  \citetbl{
    \citerppg{papoulis}{21}{22}{0070484775},
    \citerpc{kolmogorov1933e2}{2}{\textsection 1. Axioms I--V}
    }
\label{def:psp}
%---------------------------------------
Let $\setX$ be a \structe{set}.
\defboxp{
  A function $\psp\in\clF{\setX}{\Rnn}$ is a \fnctd{probability function} if
    \\\indentx$\begin{array}{Flc lcl CDD}
      (1). &                   &        & \psp(\lid)     &=&      1               &                     & (\prope{normalized})     & and \\
      (2). &                   &        & \psp(x)        &\oreld& 0               & \forall x\in\setX   & (\prope{nonnegative})& and \\
      (3). & x\meet y = \lzero &\implies& \psp(x\join y) &=&      \psp(x)+\psp(y) & \forall x,y\in\setX & (\prope{additive})   & .      
    \end{array}$
  }
\end{definition}



