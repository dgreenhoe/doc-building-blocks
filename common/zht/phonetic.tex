%============================================================================
% XeLaTeX File
% Daniel J. Greenhoe
%============================================================================

%=======================================
\chapter{Phones of Mandarin Chinese}
\label{chp:phones}
%=======================================
%=======================================
\section{How many phones do you have?}
%=======================================
Chinese has about {401} phones (syllable sounds).
English has about {15,831} phones.
This result for English is from Chris Barker at 
  New York University's Department of Linguistics.
      He used the \hie{BEEP Pronouncing Dictionary}
             which apparently at the time contained ``just over" 16,000 words, but now 
             contains over 250,000 words. It can be freely downloaded from
             \url{http://svr-www.eng.cam.ac.uk/comp.speech/Section1/Lexical/beep.html}
      Chris Barker wrote a computer program to find all the syllables in the dictionary. 

%  New York University's \zcomi{Department}{系} of \zcomi{Linguistics}{語言學}.
%  %  \\\url{http://semarch.linguistics.fas.nyu.edu/barker/Syllables/index.txt}
%      He used the \zcomi{BEEP Pronouncing Dictionary}{BEEP發音字典} 
%             which apparently at the time contained ``just over" 16,000 words, but now 
%             contains over 250,000 words. It can be freely downloaded from
%             \url{http://svr-www.eng.cam.ac.uk/comp.speech/Section1/Lexical/beep.html}
%      Chris Barker wrote a \zcomi{computer program}{電腦程式} to find all the syllables in the dictionary. 
  %His \zcomi{source code}{原始碼} was available at
  %  \\\url{http://idiom.ucsd.edu/~barker/Syllables/syllabify.pl},
  %\\but does not seem to be available now.
  %    \\{\scs\fullcite{barkerEngSyll}}


%=======================================
\section{Spoken Chinese}
%=======================================
Spoken Mandarin Chinese has about 401 syllable sounds (phones).
In addition to these sounds, there are four tones plus a neutral tone;
in general, meanings of sounds change (greatly) based on the tone
that the sounds are voiced with.
The four tones are referred to as follows:\\
%\\\begin{tabular}{lll}
%  \circOne   & \hie{first  tone} & (or \hie{high    tone}) \\
%  \circTwo   & \hie{second tone} & (or \hie{rising  tone}) \\
%  \circThree & \hie{third  tone} & (or \hie{low     tone}) \\
%  \circFour  & \hie{fourth tone} & (or \hie{falling tone})
%\end{tabular}\\
%These four tones are illustrated in \prefpp{fig:tones}.
%
%\begin{figure}
{
%\centering
\setlength{\unitlength}{\tw/800}%
\thicklines
\begin{tabular}{cccc}
  \begin{picture}(170,140)(0,0)
    \thicklines
    \color{axis}%
      \put(0,0){\line(1,0){120} }%
      \put(0,0){\line(0,1){120} }%
      \put(125,  0){\makebox(0,0)[bl]{time}}%
      \put(  0,125){\makebox(0,0)[bl]{pitch}}%
    \color{blue}%
      \put(0,100){\line(1,0){100} }%
  \end{picture}
  &
  \begin{picture}(170,140)(0,0)
    \thicklines
    \color{axis}%
      \put(0,0){\line(1,0){120} }%
      \put(0,0){\line(0,1){120} }%
      \put(125,  0){\makebox(0,0)[bl]{time}}%
      \put(  0,125){\makebox(0,0)[bl]{pitch}}%
    \color{blue}%
      \put(0,50){\line(2,1){100} }%
  \end{picture}
  &
  \begin{picture}(170,140)(0,0)
    \thicklines
    \color{axis}%
      \put(0,0){\line(1,0){120} }%
      \put(0,0){\line(0,1){120} }%
      \put(125,  0){\makebox(0,0)[bl]{time}}%
      \put(  0,125){\makebox(0,0)[bl]{pitch}}%
    \color{blue}%
      \qbezier(0,75)(50,0)(100,50)
  \end{picture}
  &
  \begin{picture}(170,140)(0,0)
    \thicklines
    \color{axis}%
      \put(0,0){\line(1,0){120} }%
      \put(0,0){\line(0,1){120} }%
      \put(125,  0){\makebox(0,0)[bl]{time}}%
      \put(  0,125){\makebox(0,0)[bl]{pitch}}%
    \color{blue}%
      \put(0,100){\line(1,-2){50} }%
  \end{picture}
  \\
  \hi{first tone} & \hi{second tone} & \hi{third tone} & \hi{fourth tone}
  \\
  (\hi{high tone}) & (\hi{rising tone}) & (\hi{low tone}) & (\hi{falling tone})
\end{tabular}
%\caption{Tones of Mandarin Chinese\label{fig:tones}}
%\end{figure}
}

As previously mentioned, using the correct tone is important,
as the meaning of a character can change drastically with a change of tone,
as illustrated by \pref{ex:tones} (next) and \prefpp{ex:tonediff}.

%---------------------------------------
\begin{example}
\mbox{}\\
\label{ex:tones}
%---------------------------------------
%The sounds represented by the Pinyin letters are described below.
%in \prefpp{voc:pinyin_consonants} and \prefpp{voc:pinyin_vowels}.
%In addition to these sounds, tones are usually represented by marks over the top
%of the words, as illustrated next:
%(with Chinese character and meaning in English underneath):
\\\begin{longtable}{|l|l|>{\Large}c|*{3}{c@{}}|}
  \hline
             & \mc{1}{c|}{tone}                          & \mc{1}{c|}{marking}        & \mc{3}{c|}{examples}        \\
  \hline
  \circOne   & \parbox[t]{7\tw/32}{\raggedright\hi{first  tone}\\(\hi{high    tone})} & -                  & \zhtss{媽}{m/-a}{mother}        & \zhtss{八}{b/-a}{eight}           & \zhtss{歌}{g/-e}{song}   \\
  \circTwo   & \parbox[t]{7\tw/32}{\raggedright\hi{second tone}\\(\hi{rising  tone})} & \'{\hspace{1ex}}   & \zhtss{麻}{m/'a}{numbness}      & \zhtss{拔}{b/'a}{to uproot}       & \zhtss{格}{g\'e}{checks} \\
  \circThree & \parbox[t]{7\tw/32}{\raggedright\hi{third  tone}\\(\hi{low     tone})} & \v{\hspace{1ex}}   & \zhtss{馬}{m{/>a}}{horse}       & \zhtss{把}{b{/>a}}{handle}        & \zhtss{舸}{g\v{e}}{barge}  \\
  \circFour  & \parbox[t]{7\tw/32}{\raggedright\hi{fourth tone}\\(\hi{falling tone})} & \`{\hspace{1ex}}   & \zhtss{罵}{m/`a}{to scold}      & \zhtss{爸}{b/`a}{dad}             & \zhtss{各}{g\`e}{each}   \\
  \circFive  & \parbox[t]{7\tw/32}{\raggedright\hi{no     tone}\\(\hi{neutral tone})} &   {\hspace{1ex}}   & \zhtss{麼}{ma}{(for questions)} & \zhtss{吧}{ba}{(for suggestions)} & \zhtss{個}{ge}{(quantifier)}\\
  \hline
\end{longtable}
\end{example}

%---------------------------------------
\begin{example}
\mbox{}\\
\label{ex:tonediff}
%---------------------------------------
%Here are some examples:
%\\\begin{tabular}{|l|l|}
%  \\\hline
%     \zhtsd{語}{言}{y{/>u}}{y/'an}{language}{speech}{language}
%  &  \zhtsd{預}{言}{y/`u}{y/'an}{in advance}{speech}{prediction}
%  \\ \zhtsd{母}{親}{m{/>u}}{q/-in}{mother}{parent}{mother}
%  &  \zhtsd{木}{琴}{m/`u}{\'qin}{wood}{instrument}{xylophone}
%  \\\hline
%\end{tabular}
%
%Here are some examples:
%
\begin{longtable}{ll}
%\tcom{%
%  \zhtss{她}{t/-a}{she}
%  \zhtss{是}{sh/`i}{is}
%  \zhtss{我}{w/>o}{I}
%  \zhtss{媽}{m/-a}{mother}
%  \zhtsP}{She is my mother (m/-a).}
%  &
%\tcom{%
%  \zhtss{她}{t/-a}{she}
%  \zhtss{是}{sh/`i}{is}
%  \zhtss{我}{w/>o}{I}
%  \zhtss{馬}{m/>a}{horse}
%  \zhtsP}{She is my horse (m/>a).}
%  \\
\tcom{\zhtss{我}{w/>o}{I}%
      \zhtss{愛}{/`ai}{love}%
      \zhtsd{我}{的}{w/>o}{de}{I}{'s}{my}%
      \zhtsd{母}{親}{m{/>u}}{q/-in}{mother}{parent}{mother}%
      \zhtsP}
     {I love my mother (m{/>u} q/-in).}
&
\tcom{\zhtss{我}{w/>o}{I}%
      \zhtss{愛}{/`ai}{love}%
      \zhtsd{我}{的}{w/>o}{de}{I}{'s}{my}%
      \zhtsd{木}{琴}{m/`u}{q/'in}{wood}{instrument}{xylophone}%
      \zhtsP}
     {I love my xylophone (m/`u q/'in).}
\\
\tcom{%
  \zhtsd{今}{天}{j/-in}{ti/-an}{today}{sky}{today}
  \zhtss{她}{t/-a}{she}
  \zhtss{有}{y/>ou}{has}
  \zhtsd{比}{賽}{b/>i}{s/`ai}{compared}{race}{competition}
  \zhtsP}{Today she has a competition (b/>i s/`ai).}
  &
\tcom{%
  \zhtsd{今}{天}{j/-in}{ti/-an}{today}{sky}{today}
  \zhtss{她}{t/-a}{she}
  \zhtss{有}{y/>ou}{has}
  \zhtsd{鼻}{塞}{b/'i}{s/-ai}{nose}{plug}{stuffy nose}
  \zhtsP}{Today she has a stuffy nose (b/'i s/-ai).}
\\
\tcom{%
  \zhtss{我}{w/>o}{I}%
  \zhtsd{看}{到}{k/`an}{d/`ao}{see}{arrive}{see}%
  \zhtsd{一}{個}{y/'i}{ge}{one}{one}{a}%
  \zhtsd{小}{偷}{x/>iao}{t/-ou}{small}{steal}{thief}%
  \zhtsP}{I see a thief (x/>iao t/-ou).}%
  &
\tcom{%
  \zhtss{我}{w/>o}{I}%
  \zhtsd{看}{到}{k/`an}{d/`ao}{see}{arrive}{see}%
  \zhtsd{一}{個}{y/'i}{ge}{one}{one}{a}%
  \zhtsd{小}{頭}{x/>iao}{t/-ou}{small}{head}{small head}%
  \zhtsP}{I see a small head (x/>iao t/'ou).}%
\end{longtable}
%她是老師。
%她是老實。
%\tcom{\zhtss{我}{w/>o}{I}\zhtsd{喜}{歡}{\v{x}i}{h\=uan}{happy}{cheerful}{like}\zhtsd{我}{的}{w/>o}{de}{I}{('s)}{my}\zhtsd{老}{師}{l{/>a}o}{\=shi}{old}{teacher}{teacher}\zhtsP}
%     {I like my teacher (l{/>a}o \=shi).}
%  &
%\tcom{\zhtss{我}{w/>o}{I}\zhtsd{喜}{歡}{\v{x}i}{h\=uan}{happy}{cheerful}{like}\zhtsd{我}{的}{w/>o}{de}{I}{('s)}{my}\zhtsd{老}{是}{l{/>a}o}{\`shi}{old}{is}{always}\zhtsP}
%     {I like my always (l{/>a}o \`shi).}

\tcom{%
  \zhtss{我}{w/>o}{I}
  \zhtss{聽}{t/-ing}{hear}
  \zhtss{不}{b/`u}{not}
  \zhtss{懂}{d/>ong}{understand}
  \zhtsd{你}{的}{n/>i}{de}{you}{'s}{your}
  \zhtsd{語}{言}{y/>u}{y/'an}{speech}{statement}{language}
  \zhtsP}{I don't understand your language (y/>u y/'an).}

\tcom{%
  \zhtss{我}{w/>o}{I}
  \zhtss{聽}{t/-ing}{hear}
  \zhtss{不}{b/`u}{not}
  \zhtss{懂}{d/>ong}{understand}
  \zhtsd{你}{的}{n/>i}{de}{you}{'s}{your}
  \zhtsd{預}{言}{y/`u}{y/'an}{advance}{statement}{prediction}
  \zhtsP}{I don't understand your prediction (y/`u y/'an).}


%I wnat to read a book. / I want to cut down a tree.
%Please press the button. / Please press the cow.
\end{example}


%============================================================================
% XeLaTeX File
% Daniel J. Greenhoe
%============================================================================

%=======================================
%\section{Syllable sounds of Mandarin Chinese}
%\label{sec:phones}
%=======================================
Following is a fairly complete list of the phones (syllable sounds) of Mandarin Chinese.%
\citetbl{%
  \citer{yeh2000}\\
  \citer{shibles1994}
  }
The columns ``1, 2, 3, 4, n" represent
first tone, second tone, third tone, forth tone, and neutral, respectively.
A check ($\checkmark$) in a column signifies that a character with
the given sound (``phone") and tone exists in Mandarin Chinese.
``IPA-n" represents \hie{International Phoenetic Alphabet narrow transcription}.
``IPA-b" represents \hie{International Phoenetic Alphabet broad transcription}.
For an audio pronounciation guides of IPA glyphs, see
\\\indentx\url{http://www.internationalphoneticalphabet.org/ipa-sounds/ipa-chart-with-sounds/}
\\\indentx\url{http://www.ipachart.com/}

{\renewcommand{\tabcolsep}{4pt}
\begin{longtable}{|>{\scriptsize}r|l|>{\fntzh}l|>{\fntipa}l| *{5}{>{$}c<{$}@{\hspace{1pt}}}|
                  |>{\scriptsize}r|l|>{\fntzh}l|>{\fntipa}l| *{5}{>{$}c<{$}@{\hspace{1pt}}}|
                 }
  \hline
  \tblx&\mc{1}{B}{PinYin}&\mc{1}{B|}{ZhuYin}&\mc{1}{B|}{IPA}&\mc{1}{B}{1}&\mc{1}{B}{2}&\mc{1}{B}{3}&\mc{1}{B}{4}&\mc{1}{B}{n}%
  \tblc&\mc{1}{B}{PinYin}&\mc{1}{B|}{ZhuYin}&\mc{1}{B|}{IPA}&\mc{1}{B}{1}&\mc{1}{B}{2}&\mc{1}{B}{3}&\mc{1}{B}{4}&\mc{1}{B}{n}%
  \\\hline                                   
  \cnto &  a            & {ㄚ}       & a             & \checkmark &            &            & \checkmark & \checkmark             %& a                   
  \cntc &  ai           & {ㄞ}       & aI            & \checkmark & \checkmark & \checkmark & \checkmark &                        %& aI                  
  \cntn &  an           & {ㄢ}       & an            & \checkmark & \checkmark &            & \checkmark &                        %& an                  
  \cntc &  ang          & {ㄤ}       & aN            & \checkmark & \checkmark &            & \checkmark &                        %& aN                  
  \cntn &  ao           & {ㄠ}       & ao            & \checkmark & \checkmark & \checkmark & \checkmark &                        %& ao                  
  \cntc &  ba           & {ㄅㄚ}     & ba            & \checkmark & \checkmark & \checkmark & \checkmark & \checkmark             %& \textsubring{b}a    
  \cntn &  bai          & {ㄅㄞ}     & baI           & \checkmark & \checkmark & \checkmark & \checkmark &                        %& \textsubring{b}aI   
  \cntc &  ban          & {ㄅㄢ}     & ban           & \checkmark &            & \checkmark & \checkmark &                        %& \textsubring{b}an   
  \cntn &  bang         & {ㄅㄤ}     & baN           & \checkmark &            & \checkmark & \checkmark &                        %& \textsubring{b}aN   
  \cntc &  bao          & {ㄅㄠ}     & bao           & \checkmark & \checkmark & \checkmark & \checkmark &                        %& \textsubring{b}ao   
  \cntn &  bei          & {ㄅㄟ}     & be            & \checkmark &            & \checkmark & \checkmark &                        %& \textsubring{b}e    
  \cntc &  ben          & {ㄅㄣ}     & b@n           & \checkmark &            & \checkmark & \checkmark &                        %& \textsubring{b}@n   
  \cntn &  beng         & {ㄅㄥ}     & b2N           & \checkmark & \checkmark & \checkmark & \checkmark &                        %& \textsubring{b}2N   
  \cntc &  bi           & {ㄅㄧ}     & bi            & \checkmark & \checkmark & \checkmark & \checkmark &                        %& \textsubring{b}i    
  \cntn &  bian         & {ㄅㄧㄢ}   & biEn          & \checkmark &            & \checkmark & \checkmark &                        %& \textsubring{b}iEn  
  \cntc &  biao         & {ㄅㄧㄠ}   & biao          & \checkmark &            & \checkmark &            &                        %& \textsubring{b}iao  
  \cntn &  bie          & {ㄅㄧㄝ}   & biE           & \checkmark & \checkmark & \checkmark & \checkmark &                        %& \textsubring{b}iE   
  \cntc &  bin          & {ㄅㄧㄣ}   & bin           & \checkmark & \checkmark &            &            &                        %& \textsubring{b}in   
  \cntn &  bing         & {ㄅㄧㄥ}   & biN           & \checkmark &            & \checkmark & \checkmark &                        %& \textsubring{b}iN   
  \cntc &  bo           & {ㄅㄛ}     & bo            & \checkmark & \checkmark & \checkmark & \checkmark & \checkmark             %& \textsubring{b}O    
  \cntn &  bu           & {ㄅㄨ}     & bu            &            &            & \checkmark & \checkmark &                        %& \textsubring{b}u    
  \cntc &  ca           & {ㄘㄚ}     & /tsa          & \checkmark &            &            &            &                        %& /ts/>ha             
  \cntn &  cai          & {ㄘㄞ}     & /tsaI         & \checkmark & \checkmark & \checkmark & \checkmark &                        %& /ts/>haI            
  \cntc &  can          & {ㄘㄢ}     & /tsan         & \checkmark & \checkmark & \checkmark & \checkmark &                        %& /ts/>han            
  \cntn &  cang         & {ㄘㄤ}     & /tsaN         & \checkmark & \checkmark &            &            &                        %& /ts/>haN            
  \cntc &  cao          & {ㄘㄠ}     & /tsao         & \checkmark & \checkmark & \checkmark &            &                        %& /ts/>hao            
  \cntn &  ce           & {ㄘㄜ}     & /tsE          &            &            &            & \checkmark &                        %& /ts/>h7             
  \cntc &  cen          & {ㄘㄣ}     & /ts@n         & \checkmark & \checkmark &            &            &                        %& /ts/>h@n            
  \cntn &  ceng         & {ㄘㄥ}     & /ts2N         &            & \checkmark &            &            &                        %& /ts/>hEN            
  \cntc &  cha          & {ㄔㄚ}     & {/:tS}a       & \checkmark & \checkmark & \checkmark & \checkmark &                        %& /cha                
  \cntn &  chai         & {ㄔㄞ}     & {/:tS}aI      & \checkmark & \checkmark &            &            &                        %& /chaI               
  \cntc &  chan         & {ㄔㄢ}     & {/:tS}an      & \checkmark & \checkmark & \checkmark & \checkmark &                        %& /chan               
  \cntn &  chang        & {ㄔㄤ}     & {/:tS}aN      & \checkmark & \checkmark & \checkmark & \checkmark &                        %& /chaN               
  \cntc &  chao         & {ㄔㄠ}     & {/:tS}ao      & \checkmark & \checkmark & \checkmark & \checkmark &                        %& /chAo               
  \cntn &  che          & {ㄔㄜ}     & {/:tS}7       & \checkmark &            & \checkmark & \checkmark &                        %& /ch/>h7             
  \cntc &  chen         & {ㄔㄣ}     & {/:tS}@n      & \checkmark & \checkmark &            & \checkmark &                        %& /ch/>h@n            
  \cntn &  cheng        & {ㄔㄥ}     & {/:tS}2N      & \checkmark & \checkmark & \checkmark & \checkmark &                        %& /ch/>h2N            
  \cntc &  chi          & {ㄔ}       & {/:tS}/*r     & \checkmark & \checkmark & \checkmark & \checkmark &                        %& /ch/>h/*r           
  \cntn &  chong        & {ㄔㄨㄥ}   & {/:tS}uN      & \checkmark & \checkmark & \checkmark & \checkmark &                        %& /chuN               
  \cntc &  chou         & {ㄔㄡ}     & {/:tS}Ou      & \checkmark & \checkmark & \checkmark & \checkmark &                        %& /ch/>hOu            
  \cntn &  chu          & {ㄔㄨ}     & {/:tS}u       & \checkmark & \checkmark & \checkmark & \checkmark &                        %& /ch/>hu             
  \cntc &  chua         & {ㄔㄨㄚ}   & {/:tS}ua      & \checkmark &            &            &            &                        %& /chua               
  \cntn &  chuai        & {ㄔㄨㄞ}   & {/:tS}uaI     &            &            & \checkmark & \checkmark &                        %& /chuaI              
  \cntc &  chuan        & {ㄔㄨㄢ}   & {/:tS}uan     & \checkmark & \checkmark & \checkmark & \checkmark &                        %& /chuAn              
  \cntn &  chuang       & {ㄔㄨㄤ}   & {/:tS}uaN     & \checkmark & \checkmark & \checkmark & \checkmark &                        %& /chuAN              
  \cntc &  chui         & {ㄔㄨㄟ}   & {/:tS}ui      & \checkmark & \checkmark &            &            &                        %& /ch/>hui            
  \cntn &  chun         & {ㄔㄨㄣ}   & {/:tS}un      & \checkmark & \checkmark & \checkmark &            &                        %& /chun               
  \cntc &  chuo         & {ㄔㄨㄛ}   & {/:tS}uO      & \checkmark &            &            & \checkmark &                        %& /chuO               
  \cntn &  ci           & {ㄘ}       & /ts           & \checkmark & \checkmark & \checkmark & \checkmark &                        %& /ts/>h              
  \cntc &  cong         & {ㄘㄨㄥ}   & /tsuN         & \checkmark & \checkmark &            &            &                        %& /ts/>huN            
  \cntn &  cou          & {ㄘㄡ}     & /tso          &            &            &            & \checkmark &                        %& /ts/>hOu            
  \cntc &  cu           & {ㄘㄨ}     & /tsu          & \checkmark & \checkmark &            & \checkmark &                        %& /ts/>hu             
  \cntn &  cuan         & {ㄘㄨㄢ}   & /tsuan        &            &            &            & \checkmark &                        %& /ts/>huan           
  \cntc &  cui          & {ㄘㄨㄟ}   & /tsue         & \checkmark &            & \checkmark & \checkmark &                        %& /ts/>hui            
  \cntn &  cun          & {ㄘㄨㄣ}   & /tsun         & \checkmark & \checkmark & \checkmark & \checkmark &                        %& /ts/>hun            
  \cntc &  cuo          & {ㄘㄨㄛ}   & /tsuo         & \checkmark &            &            & \checkmark &                        %& /ts/>huO            
  \cntn &  da           & {ㄉㄚ}     & {d}a          & \checkmark & \checkmark & \checkmark & \checkmark &                        %& \textsubring{d}a    
  \cntc &  dai          & {ㄉㄞ}     & {d}aI         & \checkmark &            & \checkmark & \checkmark &                        %& \textsubring{d}aI   
  \cntn &  dan          & {ㄉㄢ}     & {d}an         & \checkmark &            & \checkmark & \checkmark &                        %& \textsubring{d}an   
  \cntc &  dang         & {ㄉㄤ}     & {d}aN         & \checkmark &            & \checkmark & \checkmark &                        %& \textsubring{d}aN   
  \cntn &  dao          & {ㄉㄠ}     & {d}ao         & \checkmark &            & \checkmark & \checkmark &                        %& \textsubring{d}ao   
  \cntc &  de           & {ㄉㄜ}     & {d}E          &            & \checkmark &            &            & \checkmark             %& \textsubring{d}7    
  \cntn &  dei          & {ㄉㄟ}     & {d}e          &            &            & \checkmark &            &                        %& \textsubring{d}e    
  \cntc &  deng         & {ㄉㄥ}     & {d}2N         & \checkmark &            & \checkmark & \checkmark &                        %& \textsubring{d}2N   
  \cntn &  di           & {ㄉㄧ}     & {d}i          & \checkmark & \checkmark & \checkmark & \checkmark &                        %& \textsubring{d}i    
  \cntc &  dian         & {ㄉㄧㄢ}   & {d}iEn        & \checkmark &            & \checkmark & \checkmark &                        %& \textsubring{d}iEn  
  \cntn &  diao         & {ㄉㄧㄠ}   & {d}iao        & \checkmark &            &            & \checkmark &                        %& \textsubring{d}iao  
  \cntc &  die          & {ㄉㄧㄝ}   & {d}iE         & \checkmark & \checkmark &            &            &                        %& \textsubring{d}iE   
  \cntn &  ding         & {ㄉㄧㄥ}   & {d}iN         & \checkmark &            & \checkmark & \checkmark &                        %& \textsubring{d}iN   
  \cntc &  diu          & {ㄉㄧㄡ}   & {d}iu         & \checkmark &            &            &            &                        %& \textsubring{d}iu   
  \cntn &  dong         & {ㄉㄨㄥ}   & {d}uN         & \checkmark &            & \checkmark & \checkmark &                        %& \textsubring{d}uN   
  \cntc &  dou          & {ㄉㄡ}     & {d}o          & \checkmark &            & \checkmark & \checkmark &                        %& \textsubring{d}Ou   
  \cntn &  du           & {ㄉㄨ}     & {d}u          & \checkmark & \checkmark & \checkmark & \checkmark &                        %& \textsubring{d}u    
  \cntc &  duan         & {ㄉㄨㄢ}   & {d}uan        & \checkmark &            & \checkmark & \checkmark &                        %& \textsubring{d}uan  
  \cntn &  dui          & {ㄉㄨㄟ}   & {d}ue         & \checkmark &            &            & \checkmark &                        %& \textsubring{d}ui   
  \cntc &  dun          & {ㄉㄨㄣ}   & {d}un         & \checkmark &            & \checkmark & \checkmark &                        %& \textsubring{d}un   
  \cntn &  duo          & {ㄉㄨㄛ}   & {d}uo         & \checkmark & \checkmark & \checkmark & \checkmark &                        %& \textsubring{d}uO   
  \cntc &  e            & {ㄜ}       & 7             & \checkmark & \checkmark & \checkmark & \checkmark &                        %& 7                   
  \cntn &  en           & {ㄣ}       & @n            & \checkmark &            &            &            &                        %& @n                  
  \cntc &  er           & {ㄦ}       & 2/*r          & \checkmark &            & \checkmark & \checkmark &                        %& 2/*r                
  \cntn &  fa           & {ㄈㄚ}     & fa            & \checkmark & \checkmark & \checkmark & \checkmark &                        %& fa                  
  \cntc &  fan          & {ㄈㄢ}     & fan           & \checkmark & \checkmark & \checkmark & \checkmark &                        %& fan                 
  \cntn &  fang         & {ㄈㄤ}     & faN           & \checkmark & \checkmark & \checkmark & \checkmark &                        %& faN                 
  \cntc &  fei          & {ㄈㄟ}     & fe            & \checkmark & \checkmark & \checkmark & \checkmark &                        %& fe                  
  \cntn &  fen          & {ㄈㄣ}     & f@n           & \checkmark & \checkmark & \checkmark & \checkmark &                        %& f@n                 
  \cntc &  feng         & {ㄈㄥ}     & fuN           & \checkmark & \checkmark & \checkmark & \checkmark &                        %& f2N                 
  \cntn &  fo           & {ㄈㄛ}     & fO            &            & \checkmark &            &            &                        %& fO                  
  \cntc &  fou          & {ㄈㄡ}     & fOu           &            & \checkmark & \checkmark &            &                        %& fOu                 
  \cntn &  fu           & {ㄈㄨ}     & fu            & \checkmark & \checkmark & \checkmark & \checkmark &                        %& fu                  
  \cntc &  ga           & {ㄍㄚ}     & ga            & \checkmark & \checkmark &            &            &                        %& \r{g}a              
  \cntn &  gai          & {ㄍㄞ}     & gaI           & \checkmark &            & \checkmark & \checkmark &                        %& \r{g}aI             
  \cntc &  gan          & {ㄍㄢ}     & gan           & \checkmark &            & \checkmark & \checkmark &                        %& \r{g}an             
  \cntn &  gang         & {ㄍㄤ}     & gaN           & \checkmark &            & \checkmark & \checkmark &                        %& \r{g}aN             
  \cntc &  gao          & {ㄍㄠ}     & gao           & \checkmark &            & \checkmark & \checkmark &                        %& \r{g}Ao             
  \cntn &  ge           & {ㄍㄜ}     & g2            & \checkmark & \checkmark & \checkmark & \checkmark &                        %& \r{g}2              
  \cntc &  gei          & {ㄍㄟ}     & ge            &            &            & \checkmark &            &                        %& \r{g}e              
  \cntn &  gen          & {ㄍㄣ}     & g@n           & \checkmark &            & \checkmark & \checkmark &                        %& \r{g}@n             
  \cntc &  geng         & {ㄍㄥ}     & g2N           & \checkmark &            & \checkmark & \checkmark &                        %& \r{g}2N             
  \cntn &  gong         & {ㄍㄨㄥ}   & guN           & \checkmark &            & \checkmark & \checkmark &                        %& \r{g}uN             
  \cntc &  gou          & {ㄍㄡ}     & gOu           & \checkmark &            & \checkmark & \checkmark &                        %& \r{g}Ou             
  \cntn &  gu           & {ㄍㄨ}     & gu            & \checkmark & \checkmark & \checkmark & \checkmark &                        %& \r{g}u              
  \cntc &  gua          & {ㄍㄨㄚ}   & gua           & \checkmark &            & \checkmark & \checkmark &                        %& \r{g}uA             
  \cntn &  guai         & {ㄍㄨㄞ}   & guaI          & \checkmark &            & \checkmark & \checkmark &                        %& \r{g}uaI            
  \cntc &  guan         & {ㄍㄨㄢ}   & guan          & \checkmark &            & \checkmark & \checkmark &                        %& \r{g}uan            
  \cntn &  guang        & {ㄍㄨㄤ}   & guaN          & \checkmark &            & \checkmark & \checkmark &                        %& \r{g}uAN            
  \cntc &  gui          & {ㄍㄨㄟ}   & gue           & \checkmark &            & \checkmark & \checkmark &                        %& \r{g}ue             
  \cntn &  gun          & {ㄍㄨㄣ}   & gun           &            &            & \checkmark & \checkmark &                        %& \r{g}un             
  \cntc &  guo          & {ㄍㄨㄛ}   & guo           & \checkmark & \checkmark & \checkmark & \checkmark &                        %& \r{g}uO             
  \cntn &  ha           & {ㄏㄚ}     & xa            & \checkmark & \checkmark & \checkmark &            &                        %& xA                  
  \cntc &  hai          & {ㄏㄞ}     & xaI           & \checkmark & \checkmark & \checkmark & \checkmark &                        %& xaI                 
  \cntn &  han          & {ㄏㄢ}     & xan           & \checkmark & \checkmark & \checkmark & \checkmark &                        %& xan                 
  \cntc &  hang         & {ㄏㄤ}     & xaN           &            & \checkmark &            & \checkmark &                        %& xAN                 
  \cntn &  hao          & {ㄏㄠ}     & xao           &            & \checkmark & \checkmark & \checkmark &                        %& xAo                 
  \cntc &  he           & {ㄏㄜ}     & x2            & \checkmark & \checkmark &            & \checkmark &                        %& x2                  
  \cntn &  hei          & {ㄏㄟ}     & xe            & \checkmark &            &            &            &                        %& xe                  
  \cntc &  hen          & {ㄏㄣ}     & x@n           &            & \checkmark & \checkmark & \checkmark &                        %& x@n                 
  \cntn &  heng         & {ㄏㄥ}     & x2N           & \checkmark & \checkmark &            & \checkmark &                        %& x2N                 
  \cntc &  hong         & {ㄏㄨㄥ}   & xuN           & \checkmark & \checkmark & \checkmark & \checkmark &                        %& xuN                 
  \cntn &  hou          & {ㄏㄡ}     & xo            &            & \checkmark & \checkmark & \checkmark &                        %& xOu                 
  \cntc &  hu           & {ㄏㄨ}     & xu            & \checkmark & \checkmark & \checkmark & \checkmark &                        %& xu                  
  \cntn &  hua          & {ㄏㄨㄚ}   & xua           & \checkmark & \checkmark &            & \checkmark &                        %& xuA                 
  \cntc &  huai         & {ㄏㄨㄞ}   & xuaI          &            & \checkmark &            & \checkmark &                        %& xuaI                
  \cntn &  huan         & {ㄏㄨㄢ}   & xuan          & \checkmark & \checkmark & \checkmark & \checkmark &                        %& xuAn                
  \cntc &  huang        & {ㄏㄨㄤ}   & xuaN          & \checkmark & \checkmark & \checkmark & \checkmark &                        %& xuAN                
  \cntn &  hui          & {ㄏㄨㄟ}   & xue           & \checkmark & \checkmark & \checkmark & \checkmark &                        %& xue                 
  \cntc &  hun          & {ㄏㄨㄣ}   & xun           & \checkmark & \checkmark & \checkmark & \checkmark &                        %& xun                 
  \cntn &  huo          & {ㄏㄨㄛ}   & xuO           & \checkmark & \checkmark & \checkmark & \checkmark &                        %& xuO                 
  \cntc &  ji           & {ㄐㄧ}     & {/dj}i        & \checkmark & \checkmark & \checkmark & \checkmark &                        %& \textsubring{/;dz}i 
  \cntn &  jia          & {ㄐㄧㄚ}   & {/dj}ia       & \checkmark & \checkmark & \checkmark & \checkmark &                        %& \textsubring{/;dz}ia
  \cntc &  jian         & {ㄐㄧㄢ}   & {/dj}iEn   n  & \checkmark &            & \checkmark & \checkmark &                        %& \textsubring{/;dz}iE
  \cntn &  jiang        & {ㄐㄧㄤ}   & {/dj}iaN   N  & \checkmark &            & \checkmark & \checkmark &                        %& \textsubring{/;dz}ia
  \cntc &  jiao         & {ㄐㄧㄠ}   & {/dj}iao   o  & \checkmark & \checkmark & \checkmark & \checkmark &                        %& \textsubring{/;dz}iA
  \cntn &  jie          & {ㄐㄧㄝ}   & {/dj}iE       & \checkmark & \checkmark & \checkmark & \checkmark &                        %& \textsubring{/;dz}iE
  \cntc &  jin          & {ㄐㄧㄣ}   & {/dj}In       & \checkmark &            & \checkmark & \checkmark &                        %& \textsubring{/;dz}In
  \cntn &  jing         & {ㄐㄧㄥ}   & {/dj}IN       & \checkmark &            & \checkmark & \checkmark &                        %& \textsubring{/;dz}IN
  \cntc &  jiong        & {ㄐㄩㄥ}   & {/dj}iuN   N  &            &            & \checkmark &            &                        %& \textsubring{/;dz}iu
  \cntn &  jiu          & {ㄐㄧㄡ}   & {/dj}iu       & \checkmark &            & \checkmark & \checkmark &                        %& \textsubring{/;dz}iu
  \cntc &  ju           & {ㄐㄩ}     & {/dj}y        & \checkmark & \checkmark & \checkmark & \checkmark &                        %& \textsubring{/;dz}y 
  \cntn &  juan         & {ㄐㄩㄢ}   & {/dj}yan   n  & \checkmark &            & \checkmark & \checkmark &                        %& \textsubring{/;dz}ya
  \cntc &  jue          & {ㄐㄩㄝ}   & {/dj}yE       & \checkmark & \checkmark &            & \checkmark &                        %& \textsubring{/;dz}yE
  \cntn &  jun          & {ㄐㄩㄣ}   & {/dj}yn       & \checkmark &            &            & \checkmark &                        %& \textsubring{/;dz}yn
  \cntc &  ka           & {ㄎㄚ}     & ka            & \checkmark &            & \checkmark & \checkmark &                        %& k/>ha               
  \cntn &  kai          & {ㄎㄞ}     & kaI           & \checkmark &            & \checkmark & \checkmark &                        %& k/>haI              
  \cntc &  kan          & {ㄎㄢ}     & kan           & \checkmark &            & \checkmark & \checkmark &                        %& k/>han              
  \cntn &  kang         & {ㄎㄤ}     & kaN           & \checkmark & \checkmark & \checkmark & \checkmark &                        %& k/>haN              
  \cntc &  kao          & {ㄎㄠ}     & kao           &            &            & \checkmark & \checkmark &                        %& k/>hao              
  \cntn &  ke           & {ㄎㄜ}     & k2            & \checkmark & \checkmark & \checkmark & \checkmark &                        %& k/>h2               
  \cntc &  ken          & {ㄎㄣ}     & k@n           &            &            & \checkmark & \checkmark &                        %& k/>h@n              
  \cntn &  keng         & {ㄎㄥ}     & k2N           & \checkmark &            & \checkmark &            &                        %& k/>h2N              
  \cntc &  kong         & {ㄎㄨㄥ}   & kuN           & \checkmark &            & \checkmark & \checkmark &                        %& k/>huN              
  \cntn &  kou          & {ㄎㄡ}     & ko            &            &            & \checkmark & \checkmark &                        %& k/>hou              
  \cntc &  ku           & {ㄎㄨ}     & ku            & \checkmark &            & \checkmark & \checkmark &                        %& k/>hu               
  \cntn &  kua          & {ㄎㄨㄚ}   & kua           & \checkmark &            & \checkmark & \checkmark &                        %& k/>huA              
  \cntc &  kuai         & {ㄎㄨㄞ}   & kuaI          &            &            &            & \checkmark &                        %& k/>huaI             
  \cntn &  kuan         & {ㄎㄨㄢ}   & kuan          & \checkmark &            & \checkmark &            &                        %& k/>huan             
  \cntc &  kuang        & {ㄎㄨㄤ}   & kuaN          & \checkmark & \checkmark &            & \checkmark &                        %& k/>huaN             
  \cntn &  kui          & {ㄎㄨㄟ}   & kue           & \checkmark & \checkmark & \checkmark & \checkmark &                        %& k/>hue              
  \cntc &  kun          & {ㄎㄨㄣ}   & kun           & \checkmark &            & \checkmark & \checkmark &                        %& k/>hun              
  \cntn &  kuo          & {ㄎㄨㄛ}   & kuO           &            &            &            & \checkmark &                        %& k/>huO              
  \cntc &  la           & {ㄌㄚ}     & la            & \checkmark & \checkmark & \checkmark & \checkmark & \checkmark             %& lA                  
  \cntn &  lai          & {ㄌㄞ}     & laI           &            & \checkmark &            & \checkmark &                        %& laI                 
  \cntc &  lan          & {ㄌㄢ}     & lan           &            & \checkmark & \checkmark & \checkmark &                        %& lan                 
  \cntn &  lang         & {ㄌㄤ}     & laN           &            & \checkmark & \checkmark & \checkmark &                        %& laN                 
  \cntc &  lao          & {ㄌㄠ}     & lao           & \checkmark & \checkmark & \checkmark & \checkmark &                        %& lao                 
  \cntn &  le           & {ㄌㄜ}     & l7            &            &            &            & \checkmark & \checkmark             %& l7                  
  \cntc &  lei          & {ㄌㄟ}     & le            & \checkmark & \checkmark & \checkmark & \checkmark &                        %& le                  
  \cntn &  leng         & {ㄌㄥ}     & l2N           &            & \checkmark & \checkmark & \checkmark &                        %& l2N                 
  \cntc &  li           & {ㄌㄧ}     & li            & \checkmark & \checkmark & \checkmark & \checkmark & \checkmark             %& li                  
  \cntn &  lia          & {ㄌㄧㄚ}   & lia           &            &            & \checkmark &            &                        %& lia                 
  \cntc &  lian         & {ㄌㄧㄢ}   & liEn          &            & \checkmark & \checkmark & \checkmark &                        %& liEn                
  \cntn &  liang        & {ㄌㄧㄤ}   & liAN          &            & \checkmark & \checkmark & \checkmark &                        %& liaN                
  \cntc &  liao         & {ㄌㄧㄠ}   & liao          & \checkmark & \checkmark & \checkmark & \checkmark &                        %& liao                
  \cntn &  lie          & {ㄌㄧㄝ}   & liE           & \checkmark & \checkmark & \checkmark & \checkmark &                        %& liE                 
  \cntc &  lin          & {ㄌㄧㄣ}   & lIn           &            & \checkmark & \checkmark & \checkmark &                        %& lIn                 
  \cntn &  ling         & {ㄌㄧㄥ}   & lIN           & \checkmark & \checkmark & \checkmark & \checkmark &                        %& lIN                 
  \cntc &  liu          & {ㄌㄧㄡ}   & liu           & \checkmark & \checkmark & \checkmark & \checkmark &                        %& liu                 
  \cntn &  long         & {ㄌㄨㄥ}   & luN           &            & \checkmark & \checkmark & \checkmark &                        %& luN                 
  \cntc &  lou          & {ㄌㄡ}     & lo            & \checkmark & \checkmark & \checkmark & \checkmark & \checkmark             %& lou                 
  \cntn &  lu           & {ㄌㄨ}     & lu            &            & \checkmark & \checkmark & \checkmark &                        %& lu                  
  \cntc &  l/:u         & {ㄌㄩ}     & ly            &            & \checkmark & \checkmark & \checkmark &                        %& ly                  
  \cntn &  l/:ue        & {ㄌㄩㄝ}   & lyE           &            &            &            & \checkmark &                        %& lyE                 
  \cntc &  luan         & {ㄌㄨㄢ}   & luan          &            & \checkmark & \checkmark & \checkmark &                        %& luan                
  \cntn &  lun          & {ㄌㄨㄣ}   & lun           & \checkmark & \checkmark &            & \checkmark &                        %& lun                 
  \cntc &  luo          & {ㄌㄨㄛ}   & luO           & \checkmark & \checkmark & \checkmark & \checkmark &                        %& luo                 
  \cntn &  ma           & {ㄇㄚ}     & ma            & \checkmark & \checkmark & \checkmark & \checkmark & \checkmark             %& ma                  
  \cntc &  mai          & {ㄇㄞ}     & maI           &            & \checkmark & \checkmark & \checkmark &                        %& maI                 
  \cntn &  man          & {ㄇㄢ}     & man           &            & \checkmark & \checkmark & \checkmark &                        %& man                 
  \cntc &  mang         & {ㄇㄤ}     & maN           &            & \checkmark & \checkmark &            &                        %& maN                 
  \cntn &  mao          & {ㄇㄠ}     & mao           & \checkmark & \checkmark & \checkmark & \checkmark &                        %& mao                 
  \cntc &  me           & {ㄇㄜ}     & m@            &            &            &            &            & \checkmark             %& m@                  
  \cntn &  mei          & {ㄇㄟ}     & me            &            & \checkmark & \checkmark & \checkmark &                        %& me                  
  \cntc &  men          & {ㄇㄣ}     & men           & \checkmark & \checkmark &            & \checkmark &                        %& men                 
  \cntn &  meng         & {ㄇㄥ}     & m2N           & \checkmark & \checkmark & \checkmark & \checkmark &                        %& m2N                 
  \cntc &  mi           & {ㄇㄧ}     & mi            & \checkmark & \checkmark & \checkmark & \checkmark &                        %& mi                  
  \cntn &  mian         & {ㄇㄧㄢ}   & miEn          &            & \checkmark & \checkmark & \checkmark &                        %& miEn                
  \cntc &  miao         & {ㄇㄧㄠ}   & miao          &            & \checkmark & \checkmark & \checkmark &                        %& miao                
  \cntn &  mie          & {ㄇㄧㄝ}   & miE           & \checkmark & \checkmark &            &            &                        %& miE                 
  \cntc &  min          & {ㄇㄧㄣ}   & min           &            & \checkmark & \checkmark &            &                        %& min                 
  \cntn &  ming         & {ㄇㄧㄥ}   & miN           &            & \checkmark & \checkmark & \checkmark &                        %& miN                 
  \cntc &  miu          & {ㄇㄧㄡ}   & miu           &            &            &            & \checkmark &                        %& miu                 
  \cntn &  mo           & {ㄇㄛ}     & mo            & \checkmark & \checkmark & \checkmark & \checkmark &                        %& mO                  
  \cntc &  mou          & {ㄇㄡ}     & mOu           &            & \checkmark & \checkmark & \checkmark &                        %& mou                 
  \cntn &  mu           & {ㄇㄨ}     & mu            &            & \checkmark & \checkmark & \checkmark &                        %& mu                  
  \cntc &  na           & {ㄋㄚ}     & na            &            & \checkmark & \checkmark & \checkmark & \checkmark             %& na                  
  \cntn &  nai          & {ㄋㄞ}     & naI           &            &            & \checkmark & \checkmark &                        %& naI                 
  \cntc &  nan          & {ㄋㄢ}     & nan           &            & \checkmark & \checkmark & \checkmark &                        %& nan                 
  \cntn &  nang         & {ㄋㄤ}     & naN           &            & \checkmark & \checkmark &            &                        %& naN                 
  \cntc &  nao          & {ㄋㄠ}     & nao           &            & \checkmark & \checkmark & \checkmark &                        %& nao                 
  \cntn &  ne           & {ㄋㄜ}     & n@            &            &            &            & \checkmark & \checkmark             %& n@                  
  \cntc &  nei          & {ㄋㄟ}     & ne            &            &            & \checkmark & \checkmark &                        %& ne                  
  \cntn &  nen          & {ㄋㄣ}     & n@n           &            &            &            & \checkmark &                        %& n@n                 
  \cntc &  neng         & {ㄋㄥ}     & n2N           &            & \checkmark &            & \checkmark &                        %& n2N                 
  \cntn &  ni           & {ㄋㄧ}     & ni            &            & \checkmark & \checkmark & \checkmark &                        %& ni                  
  \cntc &  nian         & {ㄋㄧㄢ}   & niEn          & \checkmark & \checkmark & \checkmark & \checkmark &                        %& niEn                
  \cntn &  niang        & {ㄋㄧㄤ}   & niAN          & \checkmark &            &            & \checkmark &                        %& niaN                
  \cntc &  niao         & {ㄋㄧㄠ}   & niao          &            &            & \checkmark & \checkmark &                        %& niao                
  \cntn &  nie          & {ㄋㄧㄝ}   & niE           & \checkmark &            &            & \checkmark &                        %& niE                 
  \cntc &  nin          & {ㄋㄧㄣ}   & nin           &            & \checkmark & \checkmark &            &                        %& nin                 
  \cntn &  ning         & {ㄋㄧㄥ}   & niN           &            & \checkmark & \checkmark & \checkmark &                        %& niN                 
  \cntc &  niu          & {ㄋㄧㄡ}   & niu           & \checkmark & \checkmark & \checkmark & \checkmark &                        %& niu                 
  \cntn &  nong         & {ㄋㄨㄥ}   & nuN           &            & \checkmark & \checkmark & \checkmark &                        %& nuN                 
  \cntc &  nou          & {ㄋㄡ}     & nOu           &            & \checkmark &            & \checkmark & \checkmark             %& nOu                 
  \cntn &  nu           & {ㄋㄨ}     & nu            &            & \checkmark & \checkmark & \checkmark &                        %& nu                  
  \cntc &  n/:u         & {ㄋㄩ}     & ny            &            &            & \checkmark & \checkmark &                        %& ny                  
  \cntn &  nuan         & {ㄋㄨㄢ}   & nuan          &            &            & \checkmark &            &                        %& nuAn                
  \cntc &  n/:ue        & {ㄋㄩㄝ}   & nyE           &            &            &            & \checkmark &                        %& nyE                 
  \cntn &  nuo          & {ㄋㄨㄛ}   & nuO           &            & \checkmark & \checkmark & \checkmark &                        %& nuO                 
  \cntc &  o            & {ㄛ}       & O             & \checkmark & \checkmark &            &            &                        %& o                   
  \cntn &  ou           & {ㄡ}       & o             & \checkmark &            & \checkmark & \checkmark &                        %& Ou                  
  \cntc &  pa           & {ㄆㄚ}     & pa            & \checkmark & \checkmark &            & \checkmark &                        %& p/>ha               
  \cntn &  pai          & {ㄆㄞ}     & paI           & \checkmark & \checkmark &            & \checkmark &                        %& p/>haI              
  \cntc &  pan          & {ㄆㄢ}     & pan           & \checkmark & \checkmark &            & \checkmark &                        %& p/>han              
  \cntn &  pang         & {ㄆㄤ}     & paN           & \checkmark & \checkmark &            & \checkmark &                        %& p/>haN              
  \cntc &  pao          & {ㄆㄠ}     & pao           & \checkmark & \checkmark &            & \checkmark &                        %& p/>hao              
  \cntn &  pei          & {ㄆㄟ}     & pe            & \checkmark & \checkmark &            & \checkmark &                        %& p/>he               
  \cntc &  pen          & {ㄆㄣ}     & p@n           & \checkmark & \checkmark &            &            &                        %& p/>h@n              
  \cntn &  peng         & {ㄆㄥ}     & p2N           & \checkmark & \checkmark & \checkmark & \checkmark &                        %& p/>h2N              
  \cntc &  pi           & {ㄆㄧ}     & pi            & \checkmark & \checkmark & \checkmark & \checkmark &                        %& p/>hi               
  \cntn &  pian         & {ㄆㄧㄢ}   & piEn          & \checkmark & \checkmark &            & \checkmark &                        %& p/>hiEn             
  \cntc &  piao         & {ㄆㄧㄠ}   & piao          & \checkmark & \checkmark & \checkmark & \checkmark &                        %& p/>hiao             
  \cntn &  pin          & {ㄆㄧㄣ}   & pin           & \checkmark & \checkmark & \checkmark & \checkmark &                        %& p/>hin              
  \cntc &  ping         & {ㄆㄧㄥ}   & piN           & \checkmark & \checkmark &            &            &                        %& p/>hiN              
  \cntn &  po           & {ㄆㄛ}     & pO            & \checkmark & \checkmark & \checkmark & \checkmark &                        %& p/>hO               
  \cntc &  pou          & {ㄆㄡ}     & po            & \checkmark &            & \checkmark &            &                        %& p/>hOu              
  \cntn &  pu           & {ㄆㄨ}     & pu            & \checkmark & \checkmark & \checkmark & \checkmark &                        %& p/>hu               
  \cntc &  qi           & {ㄑㄧ}     & {/ch}i        & \checkmark & \checkmark & \checkmark & \checkmark &                        %& tCi                 
  \cntn &  qia          & {ㄑㄧㄚ}   & {/ch}ia       & \checkmark & \checkmark & \checkmark & \checkmark &                        %& tCia                
  \cntc &  qian         & {ㄑㄧㄢ}   & {/ch}iEn      & \checkmark & \checkmark & \checkmark & \checkmark &                        %& tCiEn               
  \cntn &  qiang        & {ㄑㄧㄤ}   & {/ch}iaN      & \checkmark & \checkmark & \checkmark & \checkmark &                        %& tCiaN               
  \cntc &  qiao         & {ㄑㄧㄠ}   & {/ch}iao      & \checkmark & \checkmark & \checkmark & \checkmark &                        %& tCiao               
  \cntn &  qie          & {ㄑㄧㄝ}   & {/ch}iE       & \checkmark & \checkmark & \checkmark & \checkmark &                        %& tCiE                
  \cntc &  qin          & {ㄑㄧㄣ}   & {/ch}in       & \checkmark & \checkmark & \checkmark & \checkmark &                        %& tCin                
  \cntn &  qing         & {ㄑㄧㄥ}   & {/ch}iN       & \checkmark & \checkmark & \checkmark & \checkmark &                        %& tCiN                
  \cntc &  qiong        & {ㄑㄩㄥ}   & {/ch}iuN      & \checkmark & \checkmark &            &            &                        %& tCiuN               
  \cntn &  qiu          & {ㄑㄧㄡ}   & {/ch}iO       & \checkmark & \checkmark & \checkmark &            &                        %& tCiO                
  \cntc &  q/:u         & {ㄑㄩ}     & {/ch}y        & \checkmark & \checkmark & \checkmark & \checkmark &                        %& tCy                 
  \cntn &  quan         & {ㄑㄩㄢ}   & {/ch}yan      & \checkmark & \checkmark & \checkmark & \checkmark &                        %& tCyan               
  \cntc &  que          & {ㄑㄩㄝ}   & {/ch}yE       & \checkmark & \checkmark &            & \checkmark &                        %& tCyE                
  \cntn &  qun          & {ㄑㄩㄣ}   & {/ch}yn       &            & \checkmark &            &            &                        %& tCyn                
  \cntc &  ran          & {ㄖㄢ}     & /:Ran         &            & \checkmark & \checkmark &            &                        %& /:Ran               
  \cntn &  rang         & {ㄖㄤ}     & /:RaN         & \checkmark & \checkmark & \checkmark & \checkmark &                        %& /:RaN               
  \cntc &  rao          & {ㄖㄠ}     & /:Rao         &            & \checkmark & \checkmark & \checkmark &                        %& /:Rao               
  \cntn &  re           & {ㄖㄜ}     & /:R7          &            &            & \checkmark & \checkmark &                        %& /:R7                
  \cntc &  ren          & {ㄖㄣ}     & /:R@n         &            & \checkmark & \checkmark & \checkmark &                        %& /:R@n               
  \cntn &  reng         & {ㄖㄥ}     & /:R2N         & \checkmark & \checkmark &            &            &                        %& /:R2N               
  \cntc &  ri           & {ㄖ}       & /:R           &            &            &            & \checkmark &                        %& /:R                 
  \cntn &  rong         & {ㄖㄨㄥ}   & /:RuN         &            & \checkmark & \checkmark &            &                        %& /:RuN               
  \cntc &  rou          & {ㄖㄡ}     & /:Ro          &            & \checkmark & \checkmark & \checkmark &                        %& /:ROu               
  \cntn &  ru           & {ㄖㄨ}     & /:Ru          &            & \checkmark & \checkmark & \checkmark &                        %& /:Ru                
  \cntc &  ruan         & {ㄖㄨㄢ}   & /:Ruan        &            &            & \checkmark &            &                        %& /:Ruan              
  \cntn &  rui          & {ㄖㄨㄟ}   & /:Rue         &            &            & \checkmark & \checkmark &                        %& /:Rui               
  \cntc &  run          & {ㄖㄨㄣ}   & /:R2n         &            &            &            & \checkmark &                        %& /:R2n               
  \cntn &  ruo          & {ㄖㄨㄛ}   & /:RuO         &            &            &            & \checkmark &                        %& /:RuO               
  \cntc &  sa           & {ㄙㄚ}     & sa            & \checkmark &            & \checkmark & \checkmark &                        %& sa                  
  \cntn &  sai          & {ㄙㄞ}     & saI           & \checkmark &            &            & \checkmark &                        %& saI                 
  \cntc &  san          & {ㄙㄢ}     & san           & \checkmark &            & \checkmark & \checkmark &                        %& san                 
  \cntn &  sang         & {ㄙㄤ}     & saN           & \checkmark &            & \checkmark & \checkmark &                        %& saN                 
  \cntc &  sao          & {ㄙㄠ}     & sao           & \checkmark &            & \checkmark & \checkmark &                        %& sao                 
  \cntn &  se           & {ㄙㄜ}     & s7            &            &            &            & \checkmark &                        %& s7                  
  \cntc &  sen          & {ㄙㄣ}     & s@n           & \checkmark &            &            &            &                        %& s@n                 
  \cntn &  seng         & {ㄙㄥ}     & s2N           & \checkmark &            &            &            &                        %& s2N                 
  \cntc &  sha          & {ㄕㄚ}     & /:sa          & \checkmark & \checkmark & \checkmark & \checkmark &                        %& /:sa                
  \cntn &  shai         & {ㄕㄞ}     & /:saI         & \checkmark &            & \checkmark & \checkmark &                        %& /:saI               
  \cntc &  shan         & {ㄕㄢ}     & /:san         & \checkmark &            & \checkmark & \checkmark &                        %& /:san               
  \cntn &  shang        & {ㄕㄤ}     & /:saN         & \checkmark &            & \checkmark & \checkmark &                        %& /:saN               
  \cntc &  shao         & {ㄕㄠ}     & /:sao         & \checkmark & \checkmark & \checkmark & \checkmark &                        %& /:sao               
  \cntn &  she          & {ㄕㄜ}     & /:s2          & \checkmark & \checkmark & \checkmark & \checkmark &                        %& /:s2                
  \cntc &  shei         & {ㄕㄟ}     & /:se          &            & \checkmark &            &            &                        %& /:se                
  \cntn &  shen         & {ㄕㄣ}     & /:s@n         & \checkmark & \checkmark & \checkmark & \checkmark &                        %& /:s@n               
  \cntc &  sheng        & {ㄕㄥ}     & /:s2N         & \checkmark & \checkmark & \checkmark & \checkmark &                        %& /:s2N               
  \cntn &  shi          & {ㄕ}       & /:s/*r        & \checkmark & \checkmark & \checkmark & \checkmark & \checkmark             %& /:s/*r              
  \cntc &  shou         & {ㄕㄡ}     & /:sO          & \checkmark & \checkmark & \checkmark & \checkmark &                        %& /:sOu               
  \cntn &  shu          & {ㄕㄨ}     & /:su          & \checkmark & \checkmark & \checkmark & \checkmark &                        %& /:su                
  \cntc &  shua         & {ㄕㄨㄚ}   & /:sua         & \checkmark &            & \checkmark & \checkmark &                        %& /:sua               
  \cntn &  shuai        & {ㄕㄨㄞ}   & /:suaI        & \checkmark &            & \checkmark & \checkmark &                        %& /:suaI              
  \cntc &  shuan        & {ㄕㄨㄢ}   & /:suan        & \checkmark &            &            & \checkmark &                        %& /:suan              
  \cntn &  shuang       & {ㄕㄨㄤ}   & /:suaN        & \checkmark &            & \checkmark &            &                        %& /:suang             
  \cntc &  shui         & {ㄕㄨㄟ}   & /:sue         &            & \checkmark & \checkmark & \checkmark &                        %& /:sui               
  \cntn &  shun         & {ㄕㄨㄣ}   & /:sun         &            &            & \checkmark & \checkmark &                        %& /:sun               
  \cntc &  shuo         & {ㄕㄨㄛ}   & /:suo         & \checkmark &            &            & \checkmark &                        %& /:suO               
  \cntn &  si           & {ㄙ}       & sW            & \checkmark &            & \checkmark & \checkmark &                        %& sW                  
  \cntc &  song         & {ㄙㄨㄥ}   & suN           & \checkmark &            & \checkmark & \checkmark &                        %& suN                 
  \cntn &  sou          & {ㄙㄡ}     & so            & \checkmark &            & \checkmark & \checkmark &                        %& sOu                 
  \cntc &  su           & {ㄙㄨ}     & su            & \checkmark & \checkmark &            & \checkmark &                        %& su                  
  \cntn &  suan         & {ㄙㄨㄢ}   & suan          & \checkmark &            &            & \checkmark &                        %& suan                
  \cntc &  sui          & {ㄙㄨㄟ}   & sue           & \checkmark & \checkmark & \checkmark & \checkmark &                        %& sui                 
  \cntn &  sun          & {ㄙㄨㄣ}   & sun           & \checkmark &            & \checkmark & \checkmark &                        %& sun                 
  \cntc &  suo          & {ㄙㄨㄛ}   & suO           & \checkmark & \checkmark & \checkmark &            &                        %& suO                 
  \cntn &  ta           & {ㄊㄚ}     & ta            & \checkmark &            & \checkmark & \checkmark &                        %& t/>ha               
  \cntc &  tai          & {ㄊㄞ}     & taI           & \checkmark & \checkmark &            & \checkmark &                        %& t/>haI              
  \cntn &  tan          & {ㄊㄢ}     & tan           & \checkmark & \checkmark & \checkmark & \checkmark &                        %& t/>han              
  \cntc &  tang         & {ㄊㄤ}     & taN           & \checkmark & \checkmark & \checkmark & \checkmark &                        %& t/>haN              
  \cntn &  tao          & {ㄊㄠ}     & tao           & \checkmark & \checkmark & \checkmark & \checkmark &                        %& t/>hao              
  \cntc &  te           & {ㄊㄜ}     & tE            & \checkmark &            &            & \checkmark &                        %& t/>h7               
  \cntn &  teng         & {ㄊㄥ}     & t2N           &            & \checkmark &            &            &                        %& t/>h2N              
  \cntc &  ti           & {ㄊㄧ}     & ti            & \checkmark & \checkmark & \checkmark & \checkmark &                        %& t/>hi               
  \cntn &  tian         & {ㄊㄧㄢ}   & tiEn          & \checkmark & \checkmark & \checkmark &            &                        %& t/>hiEn             
  \cntc &  tiao         & {ㄊㄧㄠ}   & tiao          & \checkmark & \checkmark & \checkmark & \checkmark &                        %& t/>hiao             
  \cntn &  tie          & {ㄊㄧㄝ}   & tiE           & \checkmark &            & \checkmark &            &                        %& t/>hyE              
  \cntc &  ting         & {ㄊㄧㄥ}   & tiN           & \checkmark & \checkmark & \checkmark & \checkmark &                        %& t/>hing             
  \cntn &  tong         & {ㄊㄨㄥ}   & tuN           & \checkmark & \checkmark & \checkmark & \checkmark &                        %& t/>huN              
  \cntc &  tou          & {ㄊㄡ}     & to            & \checkmark & \checkmark &            & \checkmark &                        %& t/>hOu              
  \cntn &  tu           & {ㄊㄨ}     & tu            & \checkmark & \checkmark & \checkmark & \checkmark &                        %& t/>hu               
  \cntc &  tuan         & {ㄊㄨㄢ}   & tuan          & \checkmark & \checkmark &            &            &                        %& t/>huan             
  \cntn &  tui          & {ㄊㄨㄟ}   & tueI          & \checkmark & \checkmark & \checkmark & \checkmark &                        %& t/>hui              
  \cntc &  tun          & {ㄊㄨㄣ}   & t2n           & \checkmark & \checkmark &            & \checkmark &                        %& t/>hun              
  \cntn &  tuo          & {ㄊㄨㄛ}   & tuo           & \checkmark & \checkmark & \checkmark & \checkmark &                        %& t/>huO              
  \cntc &  wa           & {ㄨㄚ}     & wa            & \checkmark & \checkmark & \checkmark & \checkmark & \checkmark             %& wa                  
  \cntn &  wai          & {ㄨㄞ}     & waI           & \checkmark &            &            & \checkmark &                        %& waI                 
  \cntc &  wan          & {ㄨㄢ}     & wan           & \checkmark & \checkmark & \checkmark & \checkmark &                        %& wan                 
  \cntn &  wang         & {ㄨㄤ}     & waN           & \checkmark & \checkmark & \checkmark & \checkmark &                        %& waN                 
  \cntc &  wei          & {ㄨㄟ}     & we            & \checkmark & \checkmark & \checkmark & \checkmark &                        %& we                  
  \cntn &  wen          & {ㄨㄣ}     & wEn           & \checkmark & \checkmark & \checkmark & \checkmark &                        %& w@n                 
  \cntc &  weng         & {ㄨㄥ}     & w2N           & \checkmark &            & \checkmark & \checkmark &                        %& w2N                 
  \cntn &  wo           & {ㄨㄛ}     & wo            & \checkmark &            & \checkmark & \checkmark &                        %& wO                  
  \cntc &  wu           & {ㄨ}       & wu            & \checkmark & \checkmark & \checkmark & \checkmark &                        %& wu                  
  \cntn &  xi           & {ㄒㄧ}     & Ci            & \checkmark & \checkmark & \checkmark & \checkmark &                        %& Ci                  
  \cntc &  xia          & {ㄒㄧㄚ}   & Cia           & \checkmark & \checkmark &            & \checkmark &                        %& Cia                 
  \cntn &  xian         & {ㄒㄧㄢ}   & CiEn          & \checkmark & \checkmark & \checkmark & \checkmark &                        %& Cian                
  \cntc &  xiang        & {ㄒㄧㄤ}   & CiEN          & \checkmark & \checkmark & \checkmark & \checkmark &                        %& CiEN                
  \cntn &  xiao         & {ㄒㄧㄠ}   & Ciao          & \checkmark & \checkmark & \checkmark & \checkmark &                        %& Ciao                
  \cntc &  xie          & {ㄒㄧㄝ}   & CiE           & \checkmark & \checkmark & \checkmark & \checkmark &                        %& CiE                 
  \cntn &  xin          & {ㄒㄧㄣ}   & Cin           & \checkmark & \checkmark &            & \checkmark &                        %& Cin                 
  \cntc &  xing         & {ㄒㄧㄥ}   & CiN           & \checkmark & \checkmark & \checkmark & \checkmark &                        %& CiN                 
  \cntn &  xiong        & {ㄒㄩㄥ}   & CiON          & \checkmark & \checkmark &            &            &                        %& CyN                 
  \cntc &  xiu          & {ㄒㄧㄡ}   & Cio           & \checkmark &            & \checkmark & \checkmark &                        %& Ciu                 
  \cntn &  xu           & {ㄒㄩ}     & Cy            & \checkmark & \checkmark & \checkmark & \checkmark &                        %& Cy                  
  \cntc &  xuan         & {ㄒㄩㄢ}   & Cuan          & \checkmark & \checkmark & \checkmark & \checkmark &                        %& Cuan                
  \cntn &  xue          & {ㄒㄩㄝ}   & CyE           & \checkmark & \checkmark & \checkmark & \checkmark &                        %& CuE                 
  \cntc &  xun          & {ㄒㄩㄣ}   & Cyn           & \checkmark & \checkmark &            & \checkmark &                        %& Cyn                 
  \cntn &  ya           & {ㄧㄚ}     & ja            & \checkmark & \checkmark & \checkmark & \checkmark & \checkmark             %& ja                  
  \cntc &  yai          & {ㄧㄞ}     & jaI           &            & \checkmark &            &            &                        %& jaI                 
  \cntn &  yan          & {ㄧㄢ}     & jEn           & \checkmark & \checkmark & \checkmark & \checkmark &                        %& jEn                 
  \cntc &  yang         & {ㄧㄤ}     & jAN           & \checkmark & \checkmark & \checkmark & \checkmark &                        %& jAN                 
  \cntn &  yao          & {ㄧㄠ}     & jao           & \checkmark & \checkmark & \checkmark & \checkmark &                        %& jao                 
  \cntc &  ye           & {ㄧㄝ}     & jE            & \checkmark & \checkmark & \checkmark & \checkmark &                        %& jE                  
  \cntn &  yi           & {ㄧ}       & ji            & \checkmark & \checkmark & \checkmark & \checkmark &                        %& ji                  
  \cntc &  yin          & {ㄧㄣ}     & jin           & \checkmark & \checkmark & \checkmark & \checkmark &                        %& jin                 
  \cntn &  ying         & {ㄧㄥ}     & jiN           & \checkmark & \checkmark & \checkmark & \checkmark &                        %& jing                
  \cntc &  yong         & {ㄩㄥ}     & jON           & \checkmark & \checkmark & \checkmark & \checkmark &                        %& jong                
  \cntn &  you          & {ㄧㄡ}     & ju            & \checkmark & \checkmark & \checkmark & \checkmark &                        %& jOu                 
  \cntc &  yu           & {ㄩ}       & jy            & \checkmark & \checkmark & \checkmark & \checkmark &                        %& jy                  
  \cntn &  yuan         & {ㄩㄢ}     & jyan          & \checkmark & \checkmark & \checkmark & \checkmark &                        %& jyan                
  \cntc &  yue          & {ㄩㄝ}     & jyE           & \checkmark &            &            & \checkmark &                        %& jyE                 
  \cntn &  yun          & {ㄩㄣ}     & jyn           & \checkmark & \checkmark & \checkmark & \checkmark &                        %& jyn                 
  \cntc &  za           & {ㄗㄚ}     & /dza          & \checkmark &            &            &            &                        %& /;dza               
  \cntn &  zai          & {ㄗㄞ}     & /dzaI         & \checkmark &            & \checkmark & \checkmark &                        %& /;dzaI              
  \cntc &  zan          & {ㄗㄢ}     & /dzan         & \checkmark & \checkmark &            & \checkmark &                        %& /;dzan              
  \cntn &  zang         & {ㄗㄤ}     & /dzaN         & \checkmark &            &            & \checkmark &                        %& /;dzaN              
  \cntc &  zao          & {ㄗㄠ}     & /dzao         & \checkmark & \checkmark & \checkmark & \checkmark &                        %& /;dzao              
  \cntn &  ze           & {ㄗㄜ}     & /dz7          &            & \checkmark & \checkmark & \checkmark &                        %& /;dz7               
  \cntc &  zei          & {ㄗㄟ}     & /dze          &            & \checkmark &            &            &                        %& /;dze               
  \cntn &  zen          & {ㄗㄣ}     & {/dz}@n       & \checkmark &            & \checkmark &            &                        %& /;dz@n              
  \cntc &  zeng         & {ㄗㄥ}     & {/dz}2N       & \checkmark &            &            & \checkmark &                        %& /;dz2N              
  \cntn &  zha          & {ㄓㄚ}     & /:za          & \checkmark & \checkmark & \checkmark & \checkmark &                        %& /:za                
  \cntc &  zhai         & {ㄓㄞ}     & /:zaI         & \checkmark & \checkmark & \checkmark & \checkmark &                        %& /:zaI               
  \cntn &  zhan         & {ㄓㄢ}     & /:zan         & \checkmark &            & \checkmark & \checkmark &                        %& /:zan               
  \cntc &  zhang        & {ㄓㄤ}     & /:zaN         & \checkmark &            & \checkmark & \checkmark &                        %& /:zaN               
  \cntn &  zhao         & {ㄓㄠ}     & /:zao         & \checkmark & \checkmark & \checkmark & \checkmark &                        %& /:zao               
  \cntc &  zhe          & {ㄓㄜ}     & /:z7          & \checkmark & \checkmark & \checkmark & \checkmark & \checkmark             %& /:z7                
  \cntn &  zhen         & {ㄓㄣ}     & /:z@n         & \checkmark &            & \checkmark & \checkmark &                        %& /:z@n               
  \cntc &  zheng        & {ㄓㄥ}     & /:z2N         & \checkmark &            & \checkmark & \checkmark &                        %& /:z2N               
  \cntn &  zhi          & {ㄓ}       & /:z/*r        & \checkmark & \checkmark & \checkmark & \checkmark &                        %& /:z/*r              
  \cntc &  zhong        & {ㄓㄨㄥ}   & /:zuN         & \checkmark &            & \checkmark & \checkmark &                        %& /:zuN               
  \cntn &  zhou         & {ㄓㄡ}     & /:zOu         & \checkmark & \checkmark & \checkmark & \checkmark &                        %& /:zOu               
  \cntc &  zhu          & {ㄓㄨ}     & /:zzu         & \checkmark & \checkmark & \checkmark & \checkmark &                        %& /:zu                
  \cntn &  zhua         & {ㄓㄨㄚ}   & /:zua         & \checkmark &            & \checkmark &            &                        %& /:zua               
  \cntc &  zhuai        & {ㄓㄨㄞ}   & /:zuaI        & \checkmark &            & \checkmark & \checkmark &                        %& /:zuaI              
  \cntn &  zhuan        & {ㄓㄨㄢ}   & /:zuan        & \checkmark &            & \checkmark & \checkmark &                        %& /:zuan              
  \cntc &  zhuang       & {ㄓㄨㄤ}   & /:zuaN        & \checkmark &            &            & \checkmark &                        %& /:zuaN              
  \cntn &  zhui         & {ㄓㄨㄟ}   & /:zui         & \checkmark &            &            & \checkmark &                        %& /:zui               
  \cntc &  zhun         & {ㄓㄨㄣ}   & /:zun         & \checkmark &            & \checkmark &            &                        %& /:zun               
  \cntn &  zhuo         & {ㄓㄨㄛ}   & /:zuo         & \checkmark & \checkmark &            &            &                        %& /:zuO               
  \cntc &  zi           & {ㄗ}       & {/dz}W        & \checkmark &            & \checkmark & \checkmark &                        %& /;dzW               
  \cntn &  zong         & {ㄗㄨㄥ}   & {/dz}uN       & \checkmark &            & \checkmark & \checkmark &                        %& /;dzuN              
  \cntc &  zou          & {ㄗㄡ}     & {/dz}o        & \checkmark &            & \checkmark & \checkmark &                        %& /;dzOu              
  \cntn &  zu           & {ㄗㄨ}     & {/dz}u        & \checkmark & \checkmark & \checkmark &            &                        %& /;dzu               
  \cntc &  zuan         & {ㄗㄨㄢ}   & {/dz}uEn      & \checkmark &            & \checkmark & \checkmark &                        %& /;dzuEn             
  \cntn &  zui          & {ㄗㄨㄟ}   & {/dz}ui       &            &            & \checkmark & \checkmark &                        %& /;dzui              
  \cntc &  zun          & {ㄗㄨㄣ}   & {/dz}un       & \checkmark &            &            &            &                        %& /;dzun              
  \cntn &  zuo          & {ㄗㄨㄛ}   & {/dz}uo       & \checkmark & \checkmark & \checkmark & \checkmark &                        %& /;dzuO              
  &     &               &            &               &            &            &            &            &
  \\\hline
\end{longtable}
}
%  ㄅㄆㄇㄈ ㄉㄊㄋㄌㄍㄎㄏ ㄐㄑㄒ ㄓㄔㄕㄖ ㄗㄘㄙ ㄚㄛㄜㄝㄞㄟㄠㄡㄢㄣㄤㄥㄦ ㄧㄨㄩ
%                    ㄚㄟㄞㄢㄤㄠ ㄜㄟㄣㄥㄝ  ㄧ ㄛㄡㄨㄩ
%                    ㄅㄆㄇㄈ ㄉㄊㄋㄌㄍㄣ
%ㄩㄝㄡㄛㄥ

%  \cntc &  zhei  ??       & {ㄓㄟ}??     & /:zei   ??               &            &            &            & \checkmark &
%  \cntn &  lo  ??        & {ㄌㄛ}??     & lo ??                    & \checkmark &            &            &            &



%  ㄅㄆㄇㄈ ㄉㄊㄋㄌㄍㄎㄏ ㄐㄑㄒ ㄓㄔㄕㄖ ㄗㄘㄙ ㄚㄛㄜㄝㄞㄟㄠㄡㄢㄣㄤㄥㄦ ㄧㄨㄩ
%                    ㄚㄟㄞㄢㄤㄠ ㄜㄟㄣㄥㄝ  ㄧ ㄛㄡㄨㄩ
%                    ㄅㄆㄇㄈ ㄉㄊㄋㄌㄍㄣ


%%=======================================
%\section{Practice Writing}
%%=======================================
%\newpage
%\begin{tabular}{*{8}{>{\fntzht\Huge}p{25mm}}}
%  ㄔ&ㄔ&ㄔ&ㄔ&ㄔ&ㄔ&ㄔ&ㄔ\\
%  ㄓ&ㄓ&ㄓ&ㄓ&ㄓ&ㄓ&ㄓ&ㄓ\\
%  ㄏ&ㄏ&ㄏ&ㄏ&ㄏ&ㄏ&ㄏ&ㄏ\\
%  ㄈ&ㄈ&ㄈ&ㄈ&ㄈ&ㄈ&ㄈ&ㄈ
%\end{tabular}

%  {ㄅ} & b  & {ㄋ} & n &  {ㄑ} & q   &  {ㄗ} & z   & {ㄝ} & ye  & {ㄣ} & en   &{ㄩ} & \"u  \\
%  {ㄆ} & p  & {ㄌ} & l &  {ㄒ} & xi  &  {ㄘ} & c   & {ㄞ} & ai  & {ㄤ} & ang  &   &           \\
%  {ㄇ} & m  & {ㄍ} & g &  {ㄓ} & zhi &  {ㄙ} & s   & {ㄟ} & ei  & {ㄥ} & eng  &   &           \\
%  {ㄈ} & f  & {ㄎ} & k &  {ㄔ} & chi &  {ㄚ} & a   & {ㄠ} & ao  & {ㄦ} & er   &   &           \\
%  {ㄉ} & d  & {ㄏ} & h &  {ㄕ} & shi &  {ㄛ} & o   & {ㄡ} & ou  & {ㄧ} & yi   &   &           \\
%  {ㄊ} & t  & {ㄐ} & j &  {ㄖ} & re  &  {ㄜ} & e   & {ㄢ} & an  & {ㄨ} & u    &   &           \\
