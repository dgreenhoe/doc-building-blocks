%============================================================================
% Daniel J. Greenhoe
% LaTeX File
%============================================================================
%=======================================
\chapter{Analyses and Transforms}
%=======================================
\qboxnpq{
  Joseph Fourier (1768--1830)
  \index{Fourier, Joseph}
  \index{quotes!Fourier, Joseph}
  \footnotemark
  }
  {../common/people/fourier_bunzil_wkp_pdomain_gray.jpg}
  %{Mathematics compares the most diverse phenomena and discovers the secret analogies that unite them.}
  {%
  The analytical equations, unknown to the ancient geometers, which Descartes was the first to introduce into the study of curves and surfaces, 
  \ldots %are not restricted to the properties of figures, and to those properties which are the object of rational mechanics ; 
  they extend to all general phenomena. 
  There cannot be a language more universal and more simple, 
  more free from errors and from obscurities, 
  \ldots
  %that is to say more worthy to express the invariable relations of natural things. 
  %Considered from this point of view, 
  mathematical analysis is as extensive as nature itself; 
  it defines all perceptible relations, measures times, spaces, forces, temperatures ; 
  this difficult science is formed slowly, but 
  it preserves every principle which it has once acquired; 
  it grows and strengthens itself incessantly in the midst of the many variations and errors of the human mind. 
  Its chief attribute is clearness; it has no marks to express confused notions. 
  It brings together phenomena the most diverse, and discovers the hidden analogies which unite them.
  }
  \footnotetext{\begin{tabular}[t]{ll}
    quote: & \citerppc{fourier1822e}{7}{8}{Preliminary Discourse}\\
    image: & \scs\url{http://en.wikipedia.org/wiki/File:Fourier2.jpg}, public domain
  \end{tabular}}
%=======================================
\section{Abstract spaces}
%=======================================
The \structd{abstract space} was introduced by Maurice Fr/'echet in his 1906 Ph.D. thesis.\footnote{
  \citeP{frechet1906},
  \citer{frechet1928}.
  %\citerp{frechet1950}{147}
   {\fntquote``A collection of these abstract elements will be called an \emph{abstract set}.
   If to this set there is added some rule of association of these elements,
   or some relation between them,
   the set will be called an \emph{abstract space}."}---Maurice Fr/'echet
  }
An \structe{abstract space} in mathematics does not really have a rigorous definition;
but in general it is a set together with some other unifying structure.
Examples of spaces include \structe{topological space}s, \structe{metric space}s, and \structe{linear space}s 
(\structe{vector space}s).

%=======================================
\section{Lattice of subspaces}
%=======================================
An abstract space can be decomposed into one or more \structe{subspace}s. 
Roughly speaking, a subspace of an abstract space is simply a subset the abstract space 
that has the same properties of that abstract space.
The subspaces can be ordered under the ordering relation $\subseteq$ (subset or equal to relation)
to form a \structe{lattice}.

\begin{figure}[th]
\begin{center}
\includegraphics{../common/math/graphics/pdfs/latlattopxyz.pdf}%
\end{center}
\caption{
  lattice of topologies on $\setX\eqd\setn{x,y,z}$ \xref{ex:set_lat_top_xyz}
  \label{fig:set_latlat_top_xyz}
  }
\end{figure}


\begin{minipage}{\tw-45mm}
%---------------------------------------
\begin{example}
\label{ex:set_lat_top_xyz}
\footnotemark
%---------------------------------------
The power set $\psetX$ is a \structe{topology} on the set $\setX$.
%If $\setX\eqd\setn{x,y,z}$, then $\pset{\setn{x,y,z}}$ is a topology on $\setX$. 
But there are also 28 other topologies on $\setn{x,y,z}$, and these are all \structe{subspaces} of $\pset{\setn{x,y,z}}$.
Let a given topology in $\sssT{\setn{x,y,z}}$ be represented by a Hasse diagram 
as illustrated to the right, where a circle present means the indicated set is in the topology,
and a circle absent means the indicated set is not in the topology.
%\prefpp{ex:top_xyz} lists the 29 topologies $\sssT{\setn{x,y,z}}$.
The lattice of the 29 topologies $\lattice{\sssT{\setn{x,y,z}}}{\sorel}{\setu}{\seti}$
is illustrated in \prefpp{fig:set_latlat_top_xyz}. % and \prefpp{fig:set_lat_top_xyz}.
The lattice of these 29 topologies is \prope{non-distributive} (it contains the \structe{N5 lattice}).
The five topologies illustrated by red shaded nodes
%$\topT_{1}$, $\topT_{41}$, $\topT_{22}$, $\topT_{14}$, and $\topT_{77}$ (the red shaded nodes)
are also \hie{algebras of sets}\ifsxref{setstrct}{def:algsets}.
%these five sets are shaded in \pref{fig:set_latlat_top_xyz} and represented as solid dots in
%\pref{fig:set_lat_top_xyz}.
%Furthermore, \prefpp{fig:set_lat_top_xyz_d} redraws each of the 29 topological lattices 
%in a simpler unlabeled form demonstrating the distributive property of the topologies.
\end{example}%
\end{minipage}\hfill%
\footnote{%
  \citerp{isham1999}{44},
  \citerp{isham1989}{1516},
  \citerp{steiner1966}{386}
  %\citer{freese_latdraw}
  }%
\tbox{\includegraphics{../common/math/graphics/pdfs/lat2xyzdotted.pdf}}

%\includegraphics{subspace.tex}

\begin{figure}
  \centering
  \includegraphics{../common/math/graphics/pdfs/latlatalgwxyz.pdf}
  \caption{%
    lattice of \structe{algebras of sets} on $\setn{w,x,y,z}$ \xref{ex:algsets_wxyz}
    \label{fig:algsets_wxyz}
    }
\end{figure}
%---------------------------------------
\begin{example}
\label{ex:algsets_wxyz}
%---------------------------------------
The power set $\psetX$ is an \structe{algebra of sets} on the set $\setX$.
But there are also 14 other algebras of sets on $\setn{w,x,y,z}$, and these are all \structe{subspaces} of $\pset{\setn{w,x,y,z}}$.
The \structe{lattice of algebras of sets} on $\setn{w,x,y,z}$ is illustrated in \prefpp{fig:algsets_wxyz}. 
\end{example}%

%=======================================
%\section{Lattice of linear subspaces}
%=======================================
A \structe{linear subspace} is a subspace of a \structe{linear space} (\structe{vector space}).
Linear subspaces have some special properties: 
Every linear subspace contains the additive identity zero vector, and every linear subspace is \prope{convex}.

\begin{figure}[th]
  \centering
  \includegraphics{../common/math/graphics/pdfs/latr3subspaces.pdf}
  \caption{lattice of subspaces of $\R^3$ \xref{ex:r3subspaces}\label{fig:latr3subspaces}}
\end{figure}
%--------------------------------------
%--------------------------------------
\begin{minipage}{\tw-63mm}
\begin{example}\label{ex:r3subspaces}
  The 3-dimensional Euclidean space $\R^3$ contains 
  the 2-dimensional $xy$-plane and $xz$-plane subspaces,
  which in turn both contain the 1-dimensional $x$-axis subspace.
  These subspaces are illustrated in the figure to the right and in \prefpp{fig:latr3subspaces}.
\end{example}
\end{minipage}\hfill%
\tbox{\includegraphics{../common/math/graphics/pdfs/r3subspaces_xy.pdf}}

%=======================================
\section{Analyses}
%=======================================
An \opd{analysis} of a space $\spX$ is any lattice of subspaces of $\spX$.
%A sequence $\seq{\spV_j}{j\in\Z}$ of linear subspaces of a linear space $\spX$ is an \hib{analysis} of $\spX$.
        %if  $\seq{\spV_j}{j\in\Z}$ is a partition of $\spX$.
        The partial or complete reconstruction of $\spX$ from this set is a \hid{synthesis}.%
\footnote{%
  The word \hie{analysis} comes from the Greek word
  {\fntagreek{>av'alusis}},
  meaning ``dissolution" (\citerpc{perschbacher1990}{23}{entry 359}),
  which in turn means
  ``the resolution or separation into component parts"
  (\citer{collins2009}, \scs\url{http://dictionary.reference.com/browse/dissolution})
  }

\begin{figure}
  \centering
  \begin{tabular}{*{3}{>{\scs}c}}
    % \includegraphics{../common/math/graphics/latr3linear.tex}%
    %&\includegraphics{../common/math/graphics/latr3m3.tex}%
    %&\includegraphics{../common/math/graphics/latr3mra.tex}%
    \\linearly ordered analysis of $\R^3$
    &M-3 analysis of $\R^3$
    &wavelet-like analysis of $\R^3$
  \end{tabular}
  \caption{some anayses of $\R^3$ \xref{ex:r3analyses} \label{fig:r3analyses}}
\end{figure}%
%---------------------------------------
\begin{example}
\label{ex:r3analyses}
%---------------------------------------
  The lattices of subspaces illustrated in \prefpp{fig:r3analyses} are all \structe{analyses} of 
  $\R^3$.
\end{example}



%=======================================
\section{Transform}
%=======================================
%---------------------------------------
\begin{definition}
%---------------------------------------
A \hid{transform} on a space $\spX$ is a sequence of projection operators that induces 
an \structe{analysis} on $\spX$.
\end{definition}

%---------------------------------------
\begin{example}
%---------------------------------------
A \structe{Fourier analysis} is a sequence of subspaces with sinusoidal bases.
  Examples of subspaces in a Fourier analysis include $\spV_1=\Span\setn{e^{ix}}$, 
  $\spV_{2.3}=\Span\setn{e^{i2.3x}}$, $\spV_{\sqrt{2}}=\Span\setn{e^{i\sqrt{2}x}}$, etc.
  A \hib{transform} is a set of \structe{projection operators} that maps a family of functions (e.g. $\spLLR$)
  into an analysis.
  The \ope{Fourier transform}" for Fourier Analysis is \ifxref{harFour}{def:opFT}
  \\\indentx$\ds\brs{\opFT\ff}(\omega) \eqd \frac{1}{\sqrt{2\pi}} \int_\R \ff(x) e^{-i\omega x} \dx$
\end{example}

%  \item A sequence $\opT$ in $\clFxy$ is a \hib{transform} \label{item:wavstrct_T}
%        if each element in the sequence is a projection operator in $\clFxy$.
%        An example of a transform is the \hib{cosine transform} $\opT$ in $\clFrr$ such that
%        \begin{align*}
%          \opT\fx(t) &\eqd \seq{\opP_j}{j\in\Z}
%             \\&\eqd \seq{\int_{t\in\R} \fx(t)\,\mcom{\cos(nt)}{kernel} \dt}{n\in\Z}
%             \\&\eqd \seqn{\cdots,\,
%                           %\int_{t\in\R} \cos\brs{(-2)t}\,\fx(t) \dt,\,
%                           \int_{t\in\R} \fx(t)\,\cos\brs{(-1)t} \dt,\,
%                           \int_{t\in\R} \fx(t)\,                \dt,\,
%                           \int_{t\in\R} \fx(t)\,\cos\brs{( 1)t} \dt,\,
%                           %\int_{t\in\R} \cos\brs{(-2)t}\,\fx(t) \dt,\,
%                           \cdots
%                          }
%        \end{align*}
%        Further examples of transforms include the \hie{Fourier Transform} and various \hie{Wavelet Transforms}.
% it is a \hib{sequence} of projection operators on $\A function $\opT$ in $\clFxy$ is a \hib{transform} if with domain $\clFxy$ and range $\clF{\setA}{\setB}$ if


%---------------------------------------
\section{Properties of subspace order structures}
%---------------------------------------
The ordered set of all linear subspaces of a \structe{Hilbert space} is an \structe{orthomodular lattice}.
Orthomodular lattices (and hence Hilbert subspaces) have some special properties (next theorem).
One is that they satisfy \prope{de Morgan's law}.

%---------------------------------------
\begin{theorem}
\footnote{
  %\citerppg{beran1985}{30}{31}{902771715X}
  \citerppg{beran1985}{30}{33}{902771715X},
  \citorpc{birkhoffjvn1936}{830}{L74},
  %\citerppg{beran1985}{31}{33}{902771715X},
  \citorppu{beran1976}{251}{252}{http://projecteuclid.org/euclid.ndjfl/1093887530}
  }
\label{thm:latoc_char3}
%---------------------------------------
Let $\latL\eqd\latocd$ be an algebraic structure.
\thmbox{
  \brbr{\parbox{\tw/4}{$\latL$ is an \structe{orthomodular lattice}}}
  \implies
  \brbl{\begin{array}{FlclCDD}
    1. & (x\join y)^\ocop                      &=& x^\ocop \meet y^\ocop & \forall x,y\in\setX   & (\prope{de Morgan}) & and \\
    2. & (x\meet y)^\ocop                      &=& x^\ocop \join y^\ocop & \forall x,y\in\setX   & (\prope{de Morgan}) & and \\
    3. & (z^\ocop \meet y^\ocop)^\ocop \join x &=& (x \join y) \join z   & \forall x,y,z\in\setX &                     & and \\
    4. & x \meet (x \join y)                   &=& x                     & \forall x,y\in\setX   &                     & and \\
    5. & x \join (y\meet y^\ocop)              &=& x                     & \forall x,y\in\setX.
  \end{array}}
  }
\end{theorem}

\begin{figure}
{\begin{center}%
  \begin{fsL}%
\psset{unit=8mm}%
\begin{tabular}{|c|c|}%
\hline%
\mc{1}{B}{Cosine analysis  (even Fourier series)} & \mc{1}{B}{Cosine polynomial analysis}%
\\%
  \includegraphics{../common/math/graphics/pdfs/baslat_cosh.pdf}%
&%
  \includegraphics{../common/math/graphics/pdfs/baslat_cose.pdf}%
\\\hline%
\mc{1}{|B|}{Chebyshev polynomial analysis\cittrpg{rivlin1974}{4}{047172470X}}&\mc{1}{|B|}{Hadamard-3 analysis}%
\\%
  \includegraphics{../common/math/graphics/pdfs/baslat_cheby.pdf}%
&%
  \includegraphics{../common/math/graphics/pdfs/baslat_h3.pdf}%
\\\hline%
\end{tabular}%
  \end{fsL}%
\end{center}}%
\caption{some common transforms\label{fig:commontrans}}
\end{figure}

\mbox{}\\
\begin{minipage}{\tw-65mm}%
  %An analysis can also be partially characterized by its order structure with respect
  %to an order relation such as the set inclusion relation $\subseteq$.
  Most transforms have a very simple M-$n$ order structure,
  as illustrated to the right and in \prefp{fig:commontrans}.
  The M-$n$ lattices for $n\ge3$ are \prope{modular} but not \prope{distributive}.
  Analyses typically have one subspace that is a \hie{scaling} subspace;
  and this subspace is often simply a family of constants
  (as is the case with \structe{Fourier Analysis}).
  There is one noteable exception to this---MRA induced \structe{wavelet analysis}\ifsxref{wavstrct}{def:wavelet}.
\end{minipage}\hfill%
\tboxc{\includegraphics{../common/math/graphics/pdfs/latmn.pdf}}

%=======================================
\section{Operator inducing analyses}
%=======================================
\tboxc{\includegraphics{../common/math/graphics/pdfs/opTrn.pdf}}\hfill
\begin{minipage}{\tw-120mm}
  An \structe{analysis} is often defined in terms of \\a small number (e.g. 2) operators.
  Two such operators are the \ope{translation operator} and the \ope{dilation operator}\ifsxref{transop}{def:opTD}.
\end{minipage}
\hfill\tboxc{\includegraphics{../common/math/graphics/pdfs/opDil.pdf}}

%---------------------------------------
\begin{example}
%---------------------------------------
In \structe{Fourier analysis}, \prop{continuous} {dilations} \xref{def:opD} of the \fncte{complex exponential}\ifsxref{numsys}{def:exp}
form a  \structe{basis}\ifsxref{frames}{def:basis_schauder} for the \structe{space of square integrable functions} $\spLLR$\ifsxref{frames}{def:spLLR} 
such that
\quad$\ds\spLLR=\linspan\set{\opDil_\omega e^{ix}}{\scy\omega\in\R}$.
\end{example}

%---------------------------------------
\begin{example}
%---------------------------------------
In \structe{Fourier series analysis}\ifsxref{harPoly}{thm:opFSi}, \prop{discrete} dilations of the complex exponential 
form a  basis for $\spLL{\intoo{0}{2\pi}}$ such that
\quad$\ds\spLL{\intoo{0}{2\pi}}=\linspan\setjZ{\opDil_j e^{ix}}$.
\end{example}

%=======================================
\section{Wavelet analyses}
%=======================================
The term ``wavelet" comes from the French word ``\hie{ondelette}", meaning ``small wave". 
And in essence, wavelets are ``small waves" (as opposed to the ``long waves" of Fourier analysis) 
that form a basis for the Hilbert space $\spLLR$.\footnote{
  \citerpg{strang1996}{ix}{0961408871}
  \citerpg{atkinson2009}{191}{1441904581}
  }

\begin{figure}
  \centering
  \begin{tabular}{|c|c|}%
  \hline%
  \mc{1}{|B|}{Haar/Daubechies-$p1$ wavelet analysis} & \mc{1}{B|}{Daubechies-$p2$ wavelet analysis}%
  \\%
    \includegraphics{../common/math/graphics/pdfs/baslat_d1.pdf}%
  &%
    \includegraphics{../common/math/graphics/pdfs/baslat_d2.pdf}%
  \\\hline%
  \end{tabular}%
  \caption{some wavelet transforms\label{fig:wavtrans}}
\end{figure}

\begin{minipage}{\tw-55mm}%
  \textbf{A special characteristic} of wavelet analysis is that there is not just one scaling subspace,
  (as is with the case of Fourier and several other analyses),
  but an entire sequence of scaling subspaces \xref{fig:wavtrans}.
  These scaling subspaces are \prope{linearly ordered} with respect to the
  ordering relation $\subseteq$. In wavelet theory, this structure is called a \structe{multiresolution analysis},
  or \structe{MRA}\ifsxref{mra}{def:mra}.
  The MRA was introduced by St{/'e}phane G. Mallat in 1989.
  The concept of a scaling space was perhaps first introduced by Taizo Iijima in 1959 in Japan,
  and later as the \structe{Gaussian Pyramid} by Burt and Adelson in the 1980s in the West.\footnotemark
\end{minipage}%
\footnote{%
  \citorp{mallat89}{70}
  \citor{iijima1959}
  \citor{burt1983}
  \citor{adelson1981}
  \citer{lindeberg1993}
  \citer{alvertez1993}
  \citer{guichard2012}
  \citerc{weickert1999}{historical survey}
  }
\hfill%
{\begin{minipage}{50mm}%
  %\mbox{}\\% force (just above?) top of graphic to be the top of the minipage
  \fns%
  \psset{yunit=0.5\psunit}%
  \includegraphics{../common/math/graphics/pdfs/latmra.pdf}%
\end{minipage}}

The MRA has become the dominate wavelet construction method.
Moreover, P.G. Lemari/'e has proved that all wavelets with \prope{compact support} are generated by an MRA.\footnote{
  \citer{lemarie1990},
  \citerpg{mallat}{240}{012466606X}
  }

\begin{minipage}{\tw-65mm}%
  \textbf{A second special characteristic} of wavelet analysis is that it's order structure
  with respect to the $\subseteq$ relation is not a simple M-$n$ lattice 
 (as is with the case of Fourier and several other analyses).
  Rather, it is a lattice of the form illustrated to the right and in \prefpp{fig:wavtrans}.
  This lattice is \prope{non-complemented}, \prope{non-distributive},
  \prope{non-modular}, and \prope{non-Boolean} \xref{prop:order_wavstrct}.\footnotemark
\end{minipage}%
\hfill%
\tbox{\includegraphics{../common/math/graphics/pdfs/latwav.pdf}}
\footnotetext{
  \citerpgc{greenhoe2013wsd}{72}{0983801134}{Section 2.4.3 Order structure}
  }

%=======================================
%\section{Wavelets stand out}
%=======================================

\mbox{}\\
\begin{minipage}{\tw-65mm}%
  %Wavelets stand out in the world of Transforms.
  In the world of mathematical structures,
  the order structure of wavelet analyses is quite rare, but not completely unique.
  One example of a system with similar structure is the set of \structe{Primorial numbers}
  together with the $|$ (``divides") ordering relation\footnotemark
  as illustrated to the right. % by a \hie{Hasse diagram}.
  %if not outright unique.
  %For example, suppose we replace the wavelet subspaces with prime numbers
  %and the scaling subspaces with their products as illustrated to the right.
  %The resulting sequence $\seqn{1,\,2,\,6,\,30,\,210}$ as of 2011 July 30
  %has no matches in Neil J.A. Sloane's  \emph{Online Encyclopedia of Integer Sequences}
  %(hosted by \emph{AT\&T Research}).\footnotemark
\end{minipage}%
\footnote{%
  \citeoeis{A002110},
  \citerpg{greenhoe2013wsd}{30}{0983801134}
  }%
\hfill%
\tbox{\includegraphics{../common/math/graphics/pdfs/latp_1235711.pdf}}%
  %\includegraphics{../common/wavelets/graphics/latwav123.tex}%

The basis sequence of most transform are fixed with no design freedom 
For example, the Fourier Transform uses the complex exponential, Taylor Expansion uses 
monomials of the form $(x-a)^n$.
However, there are an infinite number of wavelet basis sequences---lots and lots of design freedom.
For information regarding designing wavelet basis sequences, see \citer{greenhoe2013wsd}.

However, one arguable disadvantage is that wavelets do not support a \thmb{convolution theorem}---a theorem 
enjoyed by the Fourier transforms, Laplace Transform, and Z Transform.
These other transforms induce a convolution theorem because they are defined in terms of an exponential
(e.g. $e^{-i\omega t}$, $e^{-i\omega n}$, $e^{-st}$, $z^{-n}$),
and exponentials sport the property $a^{x+y}=a^xa^y$.
