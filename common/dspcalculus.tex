%============================================================================
% XeLaTeX File
% Daniel J. Greenhoe
%============================================================================
%=======================================
\chapter{DSP Calculus}
%=======================================

%\fbox{displacement}\qquad\fbox{velocity}\qquad\fbox{acceleration}
%
%How are they related?
%
%
%%---------------------------------------
%% Definitions
%%---------------------------------------


%\defbox{\begin{array}{rc>{\ds}l}
%  \fv(t) &\eqd& \ddt \fx(t)
%  \\
%  \fa(t) &\eqd& \ddt \fv(t)
%\end{array}}
 
%=======================================
\section{Fourier Transform calculus}
%=======================================

\thmbox{\begin{array}{rc>{\ds}l}
  \fx(t) &=& \int_{u=0}^t \fv(u) \du + \mcom{\fx(0)}{initial condition}
  \\
  \fv(t) &=& \int_{u=0}^t \fa(u) \du + \mcom{\fv(0)}{initial condition}
\end{array}}
\\
\begin{proof}
\thme{Fundamental Theorem of Calculus}
\end{proof}

%---------------------------------------
\begin{proposition}
%---------------------------------------
\mbox{}\\
\propboxt{
  The \ope{Fourier Transform} of the \ope{differential operator} is
  \\\indentx$\ds
    \opFT \brs{\ddt \fx(t)} = i\omega \FX(\omega)
  $}
\end{proposition}
\begin{proof}
\begin{align*}
  \boxed{\opFT \brs{\ddt \fx(t)}} 
    &\eqd \int_{t=-\infty}^{t=+\infty} \mcom{\brs{\ddt \fx(t)}}{$\dv$} \mcom{e^{-i\omega t}}{$u$} \dt
  \\&= \brlr{\mcom{e^{-i\omega t}}{$u$} \mcom{\fx(t)}{$v$}}_{t=-\infty}^{t=+\infty}
      -\int_{t=-\infty}^{t=+\infty} \mcom{\fx(t)}{$v$} \mcom{(-i\omega)e^{-i\omega t}}{$\du$} \dt
    && \text{by \thme{Integration by Parts}}
  \\&= \cancelto{0}{e^{-i\omega \infty}}\fx(\infty) - e^{-i\omega \infty}\cancelto{0}{\fx(-\infty)} 
      -(-i\omega)\mcom{\int_{t=-\infty}^{t=+\infty} \fx(t) e^{-i\omega t} \dt}{\ope{Fourier Transform} of $\fx(t)$}
    && \text{assuming $\fx(t)$ started at $0$}
  \\&= \boxed{i\omega X(\omega)}
\end{align*}
\end{proof}

%---------------------------------------
%\newpage\mbox{} 
%---------------------------------------
%---------------------------------------
\begin{proposition}
%---------------------------------------
\mbox{}\\
\propboxt{
  The \ope{Fourier Transform} of the \ope{integration operator} is
  \\\indentx$\ds
    \opFT \int_{u=-\infty}^{u=t} \fx(u) \du = \frac{1}{i\omega} \FX(\omega)
  $}
\end{proposition}
\begin{proof}
\begin{align*}
  \boxed{\opFT \int_{u=-\infty}^{u=t} \fx(u) \du}
    &\eqd \int_{t=-\infty}^{t=+\infty} \brs{\int_{u=-\infty}^{u=t} \fx(u) \du} e^{-i\omega t} \dt
  \\&= \int_{t=-\infty}^{t=+\infty} \brs{\int_{u=-\infty}^{u=+\infty} \fx(u) \fh(t-u) \du} e^{-i\omega t} \dt
    && \brp{\begin{array}{M}where $\fh(t)$ is the\\\fncte{Heaviside function}\end{array}}
  \\&= \int_{v=-\infty}^{v=+\infty} \int_{u=-\infty}^{u=+\infty} \fx(u) \fh(v)  e^{-i\omega (u+v)} \du \dv
    && \brp{\begin{array}{Mrcl}
         where      & v&=&t-u\\ 
         $\implies$ & t&=&u+v
       \end{array}}
  \\&= \brs{\int_{v=-\infty}^{v=+\infty} \fh(v) e^{-i\omega v} \dv} \mcom{\brs{\int_{u=-\infty}^{u=+\infty} \fx(u)   e^{-i\omega u} \du }}{\ope{Fourier Transform} $X(\omega)$ of $\fx(t)$}
  \\&= \brs{\int_{v=0}^{v=+\infty} e^{-i\omega v} \dv} X(\omega)
  \\&= \brlr{\frac{1}{-i\omega}e^{-i\omega v}}_{v=0}^{v=+\infty} X(\omega)
     \quad=\quad \boxed{\frac{1}{i\omega} X(\omega)}
\end{align*}

%=======================================
\section{Digital differentiation methods}
%=======================================
%=======================================
%\subsection{Method 1: Difference}
%=======================================


Digital Differentiation Method \#1: \ope{Difference}\footnote{\citerpgc{williams1986}{69}{9780132018562}{Difference}}
 
\begin{align*}
  \fy[n]
    &\eqd \fx[n] - \fx[n-1]
  \\\\
  \opZ\brb{\fy[n]} &= \opZ\brb{\fx[n] - \fx[n-1]}
  \\
  \ZY(z) &= \ZX(z) + z^{-1}\ZX(z)
  \\\\
  \frac{\ZY(z)}{\ZX(z)} &= 1-z^{-1} \quad=\quad \boxed{\frac{z-1}{z}}
  \qquad\brb{\begin{array}{MM}
    How many zeros? & Where?\\
    How many poles? & Where?
  \end{array}}
\end{align*}

%---------------------------------------
%\newpage\mbox{} 
%---------------------------------------
 
\begin{figure}[h]
  \centering
  \includegraphics{graphics/diff.pdf}
  \caption{Digital differentiation methods\label{fig:differentiation}}
\end{figure}
Is digital differentiation equivalent to continuous differentiation?\footnote{\citerpgc{williams1986}{70}{9780132018562}{Figure 2.14(a)}}
 
\begin{align*}
  \abs{\frac{z-1}{z}}_{z=e^{i\omega}}
    &= \abs{\frac{e^{i\omega}-1}
                 {e^{i\omega}}
           }
  \\&= \abs{\frac{e^{i\omega/2}\brp{e^{i\omega/2}-e^{-i\omega/2}}}
                 {e^{i\omega}}
           }
   &&= \abs{\mcom{e^{-i\omega/2}}{phase}\,\mcom{{2\sin\brp{\frac{\omega}{2}}}}{magnitude}}
  \\&= \boxed{{2\sin\brp{\frac{\omega}{2}}}} 
    && \text{for $0\le\omega\le\pi$}
\end{align*}

%=======================================
\subsection{Digital Differentiation Method \#2: \ope{Central Difference}}
\footnote{\citerpgc{williams1986}{69}{9780132018562}{Difference}}
%=======================================
 
\begin{align*}
  \fy[n]
    &\eqd \frac{\fx[n] - \fx[n-2]}{2}
  \\\\
  Y(z) &= \frac{X(z) + z^{-1}X(z)}{2}
  \\\\
  \frac{Y(z)}{X(z)} &= \frac{1-z^{-1}}{2} \quad=\quad {\frac{z^2-1}{2z^2}}
  \\\\
                    &= \boxed{\frac{(z+1)(z-1)}{2z^2}} 
  \qquad\brb{\begin{array}{MM}
    How many zeros? & Where?\\
    How many poles? & Where?
  \end{array}}
\end{align*}


\begin{figure}
  \centering 
  \includegraphics{graphics/cendiff.pdf}
  \caption{Central difference}
\end{figure}
Central Difference = Continuous Differentiation?\footnote{\citerpgc{williams1986}{70}{9780132018562}{Figure 2.14(b)}}
 
\begin{align*}
  \abs{\frac{z^2-1}{2z^2}}_{z=e^{i\omega}}
    &= \abs{\frac{e^{2i\omega}-1}
                 {2e^{2i\omega}}
           }
     = \abs{\brp{\frac{e^{i\omega}}{e^{2i\omega}}}
            \frac{\brp{e^{i\omega}-e^{-i\omega}}}{2}
           }
  \\&= \abs{\brp{e^{-i\omega}}
            \frac{\brs{cos(\omega)+i\sin(\omega)}-\brs{\cos(\omega)+i\sin(-\omega)}}{2}
           }
  \\&= \abs{\brp{e^{-i\omega}}
            \frac{\brs{cos(\omega)+i\sin(\omega)}-\brs{\cos(\omega)-i\sin(\omega)}}{2}
           }
  \\&= \abs{\brp{e^{-i\omega+\pi/2}}
            \frac{2\sin(\omega)}{2}
           }
     = \boxed{\abs{\sin(\omega)}}
\end{align*}


%=======================================
\section{Digital integration}
%=======================================
%=======================================
\subsection{Digital Integration Method \#1: \opd{Summation}}
%=======================================

{\begin{align*}
  \fy[n]
    &\eqd \fx[n] + \mcom{\fx[n-1] + \fx[n-2] + \fx[n-3] + \fx[n-4] + \fx[n-5] + \cdots}{{$\fy[n-1]$}}
  \\
  \fy[n] &=    \fx[n] + \fy[n-1]
  \\\\
  \opZ\brb{\fy[n]} &= \opZ\brb{\fx[n] + \fy[n-1]}
  \\
  Y(z) &= X(z) + z^{-1}Y(z)
  \\
  Y(z)\brs{1-z^{-1}} &= X(z)
  \\\\
  \frac{Y(z)}{X(z)} &= \frac{1}{1-z^{-1}} \quad=\quad \boxed{\frac{z}{z-1}}
  \qquad\brb{\begin{array}{MM}
    How many zeros? & Where?\\
    How many poles? & Where?
  \end{array}}
\end{align*}}


%---------------------------------------
%\newpage\mbox{} 
%---------------------------------------
%=======================================
\subsection{Digital Integration Method \#2: \opd{Trapezoid}}
%=======================================
 
{\begin{align*}
  \fy[n]
    &\eqd \frac{\fx[n]+\fx[n-1]}{2} + \frac{\fx[n-1]+\fx[n-2]}{2} + \frac{\fx[n-2]+\fx[n-3]}{2} + \cdots
  \\&=    \sfrac{1}{2}\fx[n] + \mcom{\fx[n-1] + \fx[n-2] + \fx[n-3] + \fx[n-4] + \fx[n-5] + \cdots}{{$\fy[n-1]+\sfrac{1}{2}\fx[n-1]$}}
  \\&=    \sfrac{1}{2}\fx[n] + \fy[n-1]+\sfrac{1}{2}\fx[n-1]
  \\\\
  \fy[n]-\fy[n-1]&=    \sfrac{1}{2}\brs{\fx[n] + +\fx[n-1]} 
  \\\\
  Y(z)\brs{1-z^{-1}} &= \sfrac{1}{2} X(z)\brs{1+z^{-1}}
  \\\\
  \frac{Y(z)}{X(z)} 
    &= \brp{\frac{1}{2}} \frac{1+z^{-1}}{1-z^{-1}}
     = \boxed{\brp{\frac{1}{2}} \frac{z+1}{z-1}} 
  \qquad\brb{\begin{array}{MM}
    How many zeros? & Where?\\
    How many poles? & Where?
  \end{array}}
\end{align*}}

%=======================================
\subsection{Digital Integration Method \#3:  \opd{Simpson's Rule}}
%=======================================

% 
%{\begin{align*}
%  \fy[n]
%    &\eqd \frac{\fx[n  ]+4\fx[n -1]+\fx[n -2]}{3}   
%     +    \frac{\fx[n-1]+4\fx[n -2]+\fx[n -3]}{3}   
%     +    \frac{\fx[n-2]+4\fx[n -3]+\fx[n -4]}{3} 
%     +    \frac{\fx[n-3]+4\fx[n -4]+\fx[n -5]}{3} 
%     +    \frac{\fx[n-4]+4\fx[n -5]+\fx[n -6]}{3} 
%     +    \frac{\fx[n-5]+4\fx[n -6]+\fx[n -7]}{3} 
%     +    \frac{\fx[n-6]+4\fx[n -7]+\fx[n -8]}{3} 
%     +    \frac{\fx[n-7]+4\fx[n -8]+\fx[n -9]}{3} 
%     +    \frac{\fx[n-8]+4\fx[n -9]+\fx[n-10]}{3} 
%     +    \frac{\fx[n-9]+4\fx[n-10]+\fx[n-11]}{3} 
%     +    \cdots  
%  \\&=    \sfrac{1}{3}\fx[n]+\sfrac{5}{3}\fx[n-1] 
%        + \mcom{2\fx[n-2] + 2\fx[n-3] + 2\fx[n-4] + \cdots}
%               {$\fy[n-2]+\sfrac{5}{3}\fx[n-2]+\sfrac{1}{3}\fx[n-3]$}
%
%
%  \\&=    \sfrac{1}{3}\fx[n]+\sfrac{5}{3}\fx[n-1] 
%        + \mcom{2\fx[n-2] + 2\fx[n-3] + 2\fx[n-4] + \cdots}
%               {$\fy[n-2]+\sfrac{5}{3}\fx[n-2]+\sfrac{1}{3}\fx[n-3]$}
%
%
%  \\&=    \sfrac{1}{3}\fx[n]+\mcom{\sfrac{5}{3}\fx[n-1] + 2\fx[n-2] + 2\fx[n-3] + 2\fx[n-4] + \cdots}
%                                  {$\fy[n-1]+\sfrac{2}{3}\fx[n-1]+\sfrac{1}{3}\fx[n-2]$}
%  \\&=    \fy[n-1] - \sfrac{1}{3}\brp{\fx[n]+ 2\fx[n-1]+\fx[n-2]}
%
%
%
%
%     +    \frac{\fx[n-1]+4\fx[n-2]+\fx[n-3]}{3}   
%     +    \frac{\fx[n-2]+4\fx[n-3]+\fx[n-4]}{3} 
%     +    \cdots  
%  \\&=    \sfrac{1}{2}\fx[n] + \mcom{\fx[n-1] + \fx[n-2] + \fx[n-3] + \fx[n-4] + \fx[n-5] + \cdots}{{$\fy[n-1]+\sfrac{1}{2}\fx[n-1]$}}
%  \\&=    \sfrac{1}{2}\fx[n] + \fy[n-1]+\sfrac{1}{2}\fx[n-1]
%  \\\\
%  \fy[n]-\fy[n-1]&=    \sfrac{1}{2}\brs{\fx[n] + +\fx[n-1]} 
%  \\\\
%  Y(z)\brs{1-z^{-1}} &= \sfrac{1}{2} X(z)\brs{1+z^{-1}}
%  \\\\
%  \frac{Y(z)}{X(z)} 
%    &= \brp{\frac{1}{2}} \frac{1+z^{-1}}{1-z^{-1}}
%     = \boxed{\brp{\frac{1}{2}} \frac{z+1}{z-1}} 
%  \qquad\brb{\begin{array}{MM}
%    How many zeros? & Where?\\
%    How many poles? & Where?
%  \end{array}}
%\end{align*}}
%

%---------------------------------------
%\newpage\mbox{} 
%---------------------------------------
 
\begin{figure}[h]
  \centering
  \begin{tabular}{|c|c|}
    \hline
    \includegraphics{graphics/IntSum.pdf}&\includegraphics{graphics/IntTrap.pdf}
  \\summation integration & trapezoid integration
  \\\hline
  \end{tabular}
  \caption{Comparison of digital integration methods to analytic integration\label{fig:dspint}}
\end{figure}
Is digital summation integration equivalent to continuous integration?
Not really \xref{fig:dspint}.

\begin{align*}
  \abs{\frac{z}{z-1}}_{z=e^{i\omega}}
    &= \abs{\frac{e^{i\omega}}{e^{i\omega}-1}}
  \\&= \abs{\frac{e^{i\omega}}{e^{i\omega/2}\brp{e^{i\omega/2}-e^{-i\omega/2}}}}
   &&= \abs{\mcom{e^{i\omega/2}}{phase}\,\mcom{\frac{1}{2\sin\brp{\frac{\omega}{2}}}}{magnitude}}
  \\&= \boxed{\frac{1}{2\sin\brp{\frac{\omega}{2}}}} 
    && \text{for $0\le\omega\le\pi$}
\end{align*}


Is digital trapezoid integration equivalent to continuous integration? 
Not really \xref{fig:dspint}.
 
\begin{align*}
  \abs{\frac{1}{2}\brp{\frac{z+1}{z-1}}}_{z=e^{i\omega}}
    &= \frac{1}{2}
       \abs{\frac{e^{i\omega}+1}{e^{i\omega}-1}}
  \\&= \frac{1}{2}
       \abs{\frac{e^{i\omega/2}\brp{e^{i\omega/2}+e^{-i\omega/2}}}
                 {e^{i\omega/2}\brp{e^{i\omega/2}-e^{-i\omega/2}}}
           }
   &&= \frac{1}{2}
       \abs{\frac{2\cos\brp{\frac{\omega}{2}}}
                 {2\sin\brp{\frac{\omega}{2}}}
           }
  \\&= \boxed{\frac{1}{2} \abs{\cot\brp{\frac{\omega}{2}}}}
    && \text{for $0\le\omega\le\pi$}
\end{align*}





