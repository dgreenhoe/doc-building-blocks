%============================================================================
% NCTU - Hsinchu, Taiwan
% LaTeX File
% Daniel Greenhoe
%============================================================================

%======================================
\chapter{Electromagnetics}
\label{app:em}
\index{electromagnetics}
%======================================
Physics involves the study of principles which govern the natural world.
Some of these governing principles can be described using a concept called a ``field".
Three naturally occuring fields have been identified:
\\\indentx$\begin{array}{cM}
     \imarks & gravitational field
   \\\imarks & electric field
   \\\imarks & magnetic field
\end{array}$

Thus far no set of equations has been found
that show the relationship between all three of these fields.
However, James Maxwell has successfully constructed a set of four equations which demonstrate the
relationship between the electric and magnetic fields.
These equations show that electric and magnetic fields are intimately related and thus
the joint study of these fields is called \structe{electromagnetic field} theory.

%--------------------------------------
\section{Identities}
%--------------------------------------
The following identities
are useful in working with differential operators.
Identities will be distinguished from equations\footnote{
   An {\em identity} is a special case of an {\em equation};
   And in this sense an identity is different from an equation.
   An identity is true over the entire domain of the free variable.
   However, an equation may only be true over a portion of the domain or may even be always false.
   For example, suppose $\theta\in\R$.
   Then $\sin^2\theta + \cos^2\theta \equiv 1$ is an {\bf identity} because it is true for all $\theta\in\R$.
   The expression $\cos^2\theta=1$ is only an {\bf equation} (not an identity) because it is only true
   at integer multiples of $2\pi$.
   The expression $\cos^2\theta=2$ is an {\bf equation} which is not true for any value in the domain ($\theta\in\R$).
   Reference: \cite{smith}
   }
by using the assignment ``$\equiv$" rather than ``$=$".

%--------------------------------------
\begin{theorem}[\thmd{Stokes' Theorem}]
\label{id:stokes}
%--------------------------------------
\thmbox{
   \int_s (\curl{A}) \cdot d\mathbf{s} \equiv \oint_l {\mathbf A} \cdot d{\mathbf L}
  }
\end{theorem}

%--------------------------------------
\begin{theorem}[\thmd{Divergence Theorem}]
\label{thm:divergence}
%--------------------------------------
\thmbox{
  \int_v (\diver{A}) \; dv \equiv \oint_s {\mathbf A} \cdot d{\mathbf s}
  }
\end{theorem}

%--------------------------------------
\begin{theorem}[\thmd{Laplacian Identity}]
\label{thm:laplace}
%--------------------------------------
\thmbox{
  \curl\curl\vA \equiv \grad(\diver\vA) - \laplacian\vA
  }
\end{theorem}

%--------------------------------------
\section{Electromagnetic Field Definitions}
\index{electromagnetic fields}
%--------------------------------------
%--------------------------------------
\subsection{Vector quantities}
%--------------------------------------
Maxwell's equations describe electromagnetic properties in terms of four
vector quantities: $\E, \H, \D,$ and $\emB$.

%--------------------------------------
\begin{definition}
\label{def_e}
\index{electromagnetic fields!electric}
%--------------------------------------
\defboxt{
  The \fnctd{electric field} $\E$ describes the force per unit charge exerted by the field.
  \\\indentx$\ds \E \eqd \frac{\mathbf F}{Q} $
  \qquad where ${\mathbf F}$ is force exerted on a charge $Q$.
  }
\end{definition}

%--------------------------------------
\begin{definition}
\label{def:d}
\index{electromagnetic fields!electric flux density}
%--------------------------------------
\defboxt{
  The \fnctd{electric flux density} $\D$ specifies\\the equivalent charge per unit area.
  }
\end{definition}

%--------------------------------------
\begin{definition}
\label{def:h}
\index{electromagnetic fields!magnetic}
%--------------------------------------
\defboxt{
  The \fnctd{magnetic field} $\H$ specifies\\
  the force generated by the movement of a charged particle.
  }
\end{definition}

%--------------------------------------
\begin{definition}
\label{def:b}
\index{electromagnetic fields!magnetic flux density}
%--------------------------------------
\defboxt{
  The \fnctd{magnetic flux density} $\emB$ specifies\\ 
  the equivalent force of movement of charge per unit area exerted by a magnetic field $\H$.
  }
\end{definition}

%--------------------------------------
\subsection{Operators}
%--------------------------------------
The relationship between the electric flux density $\D$
and electric field $\E$ is described by the \ope{permittivity operator} $\DE$
as defined \pref{def:de} (next definition).

%--------------------------------------
\begin{remark}
\footnote{
  \citerp{kong}{5}
  }
\label{rem:de}
%--------------------------------------
For a very wide class of media, the relation between $\D$ and $\E$
can be described very accurately as $\D = \DE \E$.
However in general, $\D$ is a function of both $\E$ and $\H$ such that
$\D = f(\E,\H)$.
One such class of media is \prope{bianisotropic media}.
\end{remark}

%--------------------------------------
\begin{definition}
\label{def:de}
\index{permittivity}
%--------------------------------------
\defbox{\begin{array}{MrclM}
  The \opd{permittivity operator} $\DE$ is defined as
   &\ds\D &=& \DE\E
   &
  \\
  If the operation $\DE$ is \prope{invertible} then
   & \E &=& \DE^{-1} \D
   & where $\DE^{-1}$ is the inverse operation of $\DE$
\end{array}}
\end{definition}

The relationship between the magnetic flux density $\emB$
and magnetic field $\H$ is described by the \ope{permeability operator} $\emBH$
as defined in \pref{def:bh} (next definition).

%--------------------------------------
\begin{remark}
%--------------------------------------
   Similar to \pref{rem:de}, for an very wide class of media, the relation between $\emB$ and $\H$
   can be described very accurately as $\emB = \emBH \H$.
   However in general, $\emB$ is a function of both $\H$ and $\E$ such that
   $\emB = \fg(\H,\E)$ for some function $\fg$.
\end{remark}

%--------------------------------------
\begin{definition}
\label{def:bh}
\index{permeability}
%--------------------------------------
\defbox{\begin{array}{MrclM}
  The \opd{permeability operator} $\emBH$ is defined as
    &\emB &=& \emBH\H
    \\
  If the operation $\emBH$ is \prope{invertible} then
    &\H &=& \emBH^{-1} \emB
    & where $\emBH^{-1}$ is the inverse operation of $\emBH$.
\end{array}}
\end{definition}

%--------------------------------------
\subsection{Types of Media}
\index{media}
%--------------------------------------
Electromagnetic waves propagate through a \structe{media}.
A media may be classified according to whether it is \propb{linear}, \propb{homogeneous}, 
\propb{isotropic}, \propb{time-invariant},
or \propb{simple}.

%--------------------------------------
\begin{definition}
\label{def_simple}
\index{media!simple}
%--------------------------------------
\defboxt{
  A media is \propd{simple} if the operators $\DE$ and $\emBH$ are multiplicative constants
  $\epsilon$ and $\mu$ such that
  \\\indentx$\begin{array}{rclD}
    \D   &=& \epsilon \E  & and\\
    \emB &=& \mu      \H  
  \end{array}$
  }
\end{definition}

%--------------------------------------
\section{Electromagnetic Field Axioms}
\index{electromagnetic waves!laws}
%--------------------------------------
The fundamentals of electromagnetic theory are at their core based
largely on empirical results rather than on mathematical analysis.
Since they are based on experiment rather than analysis,
we present them here as ``axioms", which of course require no proof.

%--------------------------------------
\begin{axiom}[\thmd{Maxwell-Faraday Axiom}]
\label{ax:mf}
\index{electromagnetic waves!laws!Faraday}
%--------------------------------------
\axbox{
  \curl{\E} = -\pderiv{}{t}\emB
  }
\end{axiom}

%--------------------------------------
\begin{axiom}[\thmd{Maxwell-Ampere Axiom}]
\label{ax:ma}
\index{electromagnetic waves!laws!Ampere}
%--------------------------------------
\axbox{
   \curl{\H} = \pderiv{}{t}\D + \J
   \qquad\text{where $\J$ is electric current density}
   }
\end{axiom}

%--------------------------------------
\begin{axiom}[\thmd{Maxwell-Gauss-D Axiom}]
\label{ax:mgd}
\index{electromagnetic waves!laws!Gauss-D}
%--------------------------------------
\axbox{
  \diver{\D} = \rho
  \qquad\text{where $\rho$ is electric charge density}
  }
\end{axiom}

%--------------------------------------
\begin{axiom}[\thmd{Maxwell-Gauss-B Axiom}]
\label{ax:mgb}
\index{electromagnetic waves!laws!Gauss-B}
%--------------------------------------
\axbox{
  \diver{\emB} = 0
  }
\end{axiom}

%--------------------------------------
\section{Wave Equations}
%--------------------------------------
In a simple media, electric and magnetic fields propagate in the form of waves.
This can be shown using two theorems.
\begin{liste}
   \item In a \prope{linear} media,
         the time/space relationships between $\E$ and $\H$
         can be described using second order differential equations
        \xref{thm:diffwave}.
   \item In a \prope{simple} media,
         the solution to these equations are waves propagating in both time and location
        \xref{thm:swave}.
\end{liste}

%--------------------------------------
\begin{theorem}[\thmd{Electric field wave equation}]
\label{thm:diffwave}
%--------------------------------------
\thmbox{
\brb{\begin{array}{FMMD}
     (1). & $\DE$ and $\emBH$ are \propb{linear}.                                                                  & and
   \\(2). & $\DE$ and $\emBH$ are \propb{time-invariant}                                                           & and
   \\(3). & $\DE$ and $\emBH$ are \propb{invertible}       & ($\DE^{-1}$ and $\emBH^{-1}$ exist)                   & and
   \\(4). & If $\E=\vect{0}$, then $\D=\vect{0}$           & ($\D = \DE \vect{0}=\vect{0}$)                        & and
   \\(5). & If $\H=\vect{0}$, then $\emB=\vect{0}$         & ($\emB = \emBH \vect{0}=\vect{0}$)                    & and
   \\(6). & The charge density is constant in location     & ($\grad{\rho}=\vect{0}$)                              & and
   \\(7). & Current flow is constant in location and time  & ($\pderiv{}{t}\J = \vect{0}$ and $\grad\J=\vect{0}$)  &
\end{array}}
\implies
\brb{\begin{array}{rclD}
   \laplacian{\E} &=& \DE \emBH \pderiv{^2}{t^2}\E & and\\
   \laplacian{H}  &=& \DE\emBH  \pderiv{^2}{t^2}\H &
\end{array}}
}
\end{theorem}
\begin{proof}
%By Lemma ---, % \ref{lem_linear},
The condition that $\DE$ is linear and invertible implies $\DE^{-1}$ is also linear.
We now analyze the curl of the left hand side of the Maxwell-Faraday Axiom.
\begin{align*}
   \curl{\curl{E}}
     &= \grad{(\diver{E})} - \laplacian{\E}
     && \text{by \prefp{thm:laplace}}
   \\&= \grad{(\diver{\DE^{-1}\D})} - \laplacian{\E}
     && \text{because $\DE$ is invertible}
   \\&= \grad{\DE^{-1}(\diver{\D})} - \laplacian{\E}
     && \text{because $\DE^{-1}$ is linear}
   \\&= \DE^{-1}\grad{(\diver{\D})} - \laplacian{\E}
     && \text{because $\DE^{-1}$ is linear}
   \\&= \DE^{-1}\grad{\rho} - \laplacian{\E}
     && \text{by \prefp{ax:mgd}}
   \\&= \DE^{-1}0 - \laplacian{\E}
     && \text{by condition 6}
   \\&= \DE^{-1}\DE 0 - \laplacian{\E}
     && \text{by condition 4}
   \\&= 0 - \laplacian{\E}
     && \text{because $\DE^{-1}\DE=I$ is the identity operator}
   \\ &= - \laplacian{\E}
\end{align*}


We now analyze the curl of the right side of the Maxwell-Faraday Axiom.


\begin{align*}
   \curl{\left( -\pderiv{}{t}\emB \right)}
     &= -\pderiv{}{t}\curl{\emB}
     && \text{by linearity of operators}
   \\&= -\pderiv{}{t}\curl{\emBH H}
     &&\text{by \prefp{def:bh}}
   \\&= -\pderiv{}{t}\emBH \curl{H}
     && \text{by linearity of $\emBH$}
   \\&= -\emBH \pderiv{}{t} \curl{H}
     && \text{by time-invariance of $\emBH$}
   \\&= -\emBH \pderiv{}{t} \left( \pderiv{}{t}\D + \J   \right)
     && \text{by the Maxwell-Ampere Axiom}
   \\&= -\emBH \left( \pderiv{^2}{t^2}\D + \pderiv{}{t}\J   \right)
   \\&= -\emBH \left( \pderiv{^2}{t^2}\D + 0  \right)
     && \text{by condition 7}
   \\&= -\emBH \left( \pderiv{^2}{t^2}\DE\E  \right)
     && \text{by \prefp{def:de}}
   \\&= -\DE \emBH \pderiv{^2}{t^2}\E
     && \text{by time-invariance of $\DE$ }
\end{align*}



Starting with the Maxwell-Ampere Axiom and using
the results of the previous two sets of equations, we can now prove
the first equation of the theorem.

\begin{align*}
   \curl{\E}        &= -\pderiv{}{t}\emB                     && \Rightarrow \\
   \curl{\curl{\E}} &= \curl{(-\pderiv{}{t}\emB)}            && \Leftrightarrow \\
   - \laplacian{\E} &= -\DE \emBH \pderiv{^2}{t^2}\E         && \Leftrightarrow \\
   \laplacian{\E}   &= \DE \emBH \pderiv{^2}{t^2}\E
\end{align*}



%By Lemma ---, %\ref{lem_linear},
The condition that $\emBH$ is linear and invertible implies $\emBH^{-1}$ is also linear.

We now analyze the curl of the left hand side of the Maxwell-Ampere Axiom.
\begin{align*}
   \curl{\curl{H}}
   &\equiv& \grad{(\diver{H})} - \laplacian{H}
   && \text{by \prefp{thm:laplace}}
   \\ &=&
   \grad{(\diver{\emBH^{-1}\emB})} - \laplacian{H}
   && \text{because $\emBH$ is invertible}
   \\ &=&
   \grad{\emBH^{-1}(\diver{\emB})} - \laplacian{H}
   && \text{because $\emBH^{-1}$ is linear}
   \\ &=&
   \emBH^{-1}\grad{(\diver{\emB})} - \laplacian{H}
   && \text{because $\emBH^{-1}$ is \prope{linear}}
   \\ &=&
   \emBH^{-1}0 - \laplacian{H}
   && \text{by \prefp{ax:mgb}}
   \\ &=&
   \emBH^{-1}\emBH 0 - \laplacian{H}
   && \text{by condition 5}
   \\ &=&
   0 - \laplacian{H}
   && \text{because $\emBH^{-1}\emBH=I$ is the identity operator}
   \\ &=&
   - \laplacian{H}
\end{align*}

We now analyze the curl of the right side of the \thme{Maxwell-Faraday Axiom} \xref{ax:mf}.

\begin{align*}
   \curl{\left( \pderiv{}{t}\D + \J \right)}
     &= \pderiv{}{t}\curl{\D} + \curl{\J}               &&\text{by linearity of operators}
   \\&= \pderiv{}{t}\curl{\D}                           &&\text{by condition 7}
   \\&= \pderiv{}{t}\curl{\DE \E}                       &&\text{by \prefp{def:de}}
   \\&= \pderiv{}{t}\DE \curl{\E}                       &&\text{by linearity of $\DE$}
   \\&= \DE \pderiv{}{t} \curl{E}                       &&\text{by time-invariance of $\DE$}
   \\&= \DE \pderiv{}{t} \left( -\pderiv{}{t}\emB \right) &&\text{by the Maxwell-Faraday Axiom}
   \\&= -\DE \pderiv{^2}{t^2} \emB
   \\&= -\DE \pderiv{^2}{t^2} \emBH\H                     &&\text{by \prefp{def:bh}}
   \\&= -\DE\emBH \pderiv{^2}{t^2} \H                     &&\text{by \prope{time-invariance} of $\emBH$}
\end{align*}

Starting with the Maxwell-Faraday Axiom and using
the results of the previous two sets of equations, we can now prove
the second part of the theorem.

\begin{align*}
   \curl{\H}        &= \pderiv{}{t}\D + \J              && \Rightarrow \\
   \curl{\curl{\H}} &= \curl{(\pderiv{}{t}\D + \J)}     && \Leftrightarrow \\
   - \laplacian{\H} &= -\DE \emBH \pderiv{^2}{t^2}\H      && \Leftrightarrow \\
   \laplacian{\H}   &= \DE \emBH \pderiv{^2}{t^2}\H
\end{align*}
\end{proof}

\prefpp{thm:diffwave} shows that under Axioms \pref{ax:mf} -- \prefpp{ax:mgb} and
certain other general conditions, both the electric field and magnetic field can be
represented as second order differential equations in location and time.
The general solution to these equations is given in the next theorem.

%--------------------------------------
\begin{theorem}
\label{thm:swave}
\footnote{
  \citerp{inan}{21}
  }
%--------------------------------------
In a simple media, the wave equation for the electric field $\E$
has the following general solution:

\thmbox{
  \E(x,y,z,t) = \vect{p_1}(\hat{k}\cdot \vect{r} - vt) + \vect{p_2}(\hat{k}\cdot \vect{r} + vt)
  }

where $\vect{p_1}$ and $\vect{p_2}$ are any vector functions,
$\unit{k}=\unit{x}k_x+\unit{y}k_y+\unit{z}k_z$ is a unit vector in the direction of wave propagation,
$\vect{r}$ is a position vector,
and $v=1/\sqrt{\epsilon\mu}$.
\end{theorem}

\begin{proof}
According to \prefpp{thm:diffwave},
\begin{eqnarray}
   \laplacian{\E} &=& \DE \emBH \pderiv{^2}{t^2}\E .
\end{eqnarray}

Since the media is simple, the operation $\DE \emBH$ equivalent to multiplicatioin by $\epsilon\mu$ and so
\begin{align*}
   \laplacian{\E} &=& \epsilon\mu \pderiv{^2}{t^2}\E.
\end{align*}

This equation is actually three equations.
\begin{align*}
   \laplacian{E_x} &= \epsilon\mu \pderiv{^2}{t^2}E_x  && \text{x component} \\
   \laplacian{E_y} &= \epsilon\mu \pderiv{^2}{t^2}E_y  && \text{y component} \\
   \laplacian{E_z} &= \epsilon\mu \pderiv{^2}{t^2}E_z  && \text{z component}
\end{align*}

Proving any one of them proves them all.  We pick the first one.
The term $\epsilon\mu\pderiv{^2}{t^2}E_x$ can be evaluated as follows:
\begin{align*}
   \epsilon\mu\pderiv{^2}{t^2}E_x
   &= 
   \epsilon\mu\pderiv{^2}{t^2}p_{1x}(\unit{k}\cdot \vect{r} - vt) +
   \epsilon\mu\pderiv{^2}{t^2}p_{2x}(\unit{k}\cdot \vect{r} + vt)
   \\&= 
   \epsilon\mu v^2 p^"_{1x}(\unit{k}\cdot \vect{r} - vt) +
   \epsilon\mu v^2 p^"_{2x}(\unit{k}\cdot \vect{r} + vt)
   \\&= 
   p^"_{1x}(\unit{k}\cdot \vect{r} - vt) +
   p^"_{2x}(\unit{k}\cdot \vect{r} + vt)
\end{align*}

The term $\laplacian{E_x}$ can be evaluated as follows:
\begin{align*}
   \laplacian{E_x}
   &=&
   \laplacian{p_{1x}}(\unit{k}\cdot \vect{r} - vt) +
   \laplacian{p_{2x}}(\unit{k}\cdot \vect{r} - vt)
\end{align*}

The two terms on the right can be simplified.

\begin{align*}
   \laplacian{p_{1x}}(\unit{k}\cdot \vect{r} - vt)
   &= 
   \left( \pderiv{^2}{x^2}+\pderiv{^2}{y^2}+\pderiv{^2}{z^2}\right)
   p_{1x}(\unit{k}\cdot \vect{r} - vt)
   \\&= 
   \left( \pderiv{^2}{x^2}+\pderiv{^2}{y^2}+\pderiv{^2}{z^2}\right)
   p_{1x}(xk_x + yk_y + zk_z - vt)
   \\&= 
   \pderiv{^2}{x^2}p_{1x}(xk_x + yk_y + zk_z - vt) +
   \pderiv{^2}{y^2}p_{1x}(xk_x + yk_y + zk_z - vt) +
   \\& 
   \pderiv{^2}{z^2}p_{1x}(xk_x + yk_y + zk_z - vt)
   \\&= 
   k_x^2 p^"_{1x}(xk_x + yk_y + zk_z - vt) +
   k_y^2 p^"_{1x}(xk_x + yk_y + zk_z - vt) +
   \\& 
   k_z^2 p^"_{1x}(xk_x + yk_y + zk_z - vt)
   \\&= 
   (k_x^2 + k_y^2 + k_z^2)p^"_{1x}(xk_x + yk_y + zk_z - vt)
   \\&= 
   \unit{k}\cdot\unit{k}p^"_{1x}(\unit{k}\cdot \vect{r} - vt)
   \\&= 
   p^"_{1x}(\unit{k}\cdot \vect{r} - vt)
   \\ \\
   \laplacian{p_{2x}}(\unit{k}\cdot \vect{r} + vt)
   &= 
   p^"_{2x}(\unit{k}\cdot \vect{r} + vt)
\end{align*}

The term $\laplacian{E_x}$ can now be expressed as
\begin{align*}
   \laplacian{E_x}
   &= 
   \laplacian{p_{1x}}(\unit{k}\cdot \vect{r} - vt) +
   \laplacian{p_{2x}}(\unit{k}\cdot \vect{r} - vt)
   \\&= 
   p^"_{1x}(\unit{k}\cdot \vect{r} - vt) +
   p^"_{2x}(\unit{k}\cdot \vect{r} + vt)
   \\&= 
   \epsilon\mu \pderiv{^2}{t^2}E_x.
\end{align*}
\end{proof}

%--------------------------------------
\section{Effect of objects on electromagnetic waves}
\label{sec_effects}
%--------------------------------------
The following are attributes of an electromagnetic wave.
Some of these attributes can be affected by an object in the path of the wave.
Because the attributes of the wave can be affected by the object,
measurements of the attributes can be exploited
to infer some information about the object.
\begin{liste}
   \item propagation
   \item polarization
   \item permittivity
   \item permeability
\end{liste}

%--------------------------------------
\paragraph{Propagation}
%--------------------------------------
An object can affect electromagnetic wave propagation in the following ways.
\begin{liste}
   \item Reflection
   \item Refraction
   \item Diffraction
\end{liste}

%--------------------------------------
\paragraph{Reflection}
\index{electromagnetic waves!reflection}
%--------------------------------------
A single reflection is very useful for gaining information about a single surface of an object.
This is used extensively by radar and sonar systems.
Of course multiple refections could be used to gain more information about the object.
This could involve several reflections over time or an array of transmitting and receiving antennas.

%--------------------------------------
\paragraph{Refraction, permittivity, permeability}
\index{electromagnetic waves!refraction}
\index{electromagnetic waves!permittivity}
\index{electromagnetic waves!permeability}
%--------------------------------------
Refraction is very useful for determining the internal composition of an object.
The electric field wave equation tells us that
\\\indentx$\laplacian{\E} = \DE \emBH \pderiv{^2}{t^2}\E$\\
where $\DE$ is the \ope{permittivity operator} and $\emBH$ the \ope{permeability operator}.
Using numerical techniques, it may be possible to ``solve" (find the mapping for)
the operation $\DE \emBH$.
In general the operation is \prope{non-linear}.
However in many cases it may be \prope{linear} or approximately linear in which case
$\DE \emBH$ may be modeled as a matrix.
One technique for analyzing the matrix is to perform a \ope{singular value decomposition} (SVD)
and then analyze the pseudo eigenvalues and eigenvectors of the decomposition to
gain a clearer understanding of the properties of the object.  The SVD of $\DE\emBH$ can
be expresed as
\\\indentx$\ds\DE\emBH = \vect{U} \Lambda \vect{V}$\\
where $\Lambda$ is a diagonal matrix containing the pseudo-eigenvalues of $\DE\emBH$
and $\vect{U}$ and $\vect{V}$ are matrices containing
the pseudo-eigenvectors.

%--------------------------------------
\paragraph{Diffraction}
\index{electromagnetic waves!diffraction}
%--------------------------------------
An object may completely block a portion of an oncoming electromagnetic wave.
However, due to diffraction, the wave may essentially reconstruct the
hole the object made in the wave as the wave propagates farther and farther past
the object.  This effect is at least partly explained by {\em Huygen's principle}.
Information gathered from a diffracted wave could perhaps give move
information about the overall shape of an object than a single reflection could.
This is because a reflected wave only carries information about a single surface,
whereas a diffracted wave flows around an object and therefore may carry information
about the entire outer surface of the object.

%--------------------------------------
\paragraph{Polarization}
\index{electromagnetic waves!polarization}
%--------------------------------------
Qualitatively, polarization is the general ``shape" of the electric field $\E(x,y,z,t)$.
For example, FM radio uses linear polarization.
Some radar systems use circular polarization.
If $\E(x,y,z,t)$ is extremely random in magnitude and direction over time,
then the wave is said to be {\em unpolarized}.
Light from the sun is an example of a wave that is nearly unpolarized\footnote{\citerp{inan}{94}}.
A more formal (quantitative) definition of polarization is presented next.

%--------------------------------------
\begin{definition}
\label{def_polarizatioin}
%--------------------------------------
\defboxt{
  Let a \fnctd{polarization function} $\vp(x,y,z)$ be defined as
  \\\indentx$\ds   p(x,y,z) \eqd \ds\lim\limits_{T\rightarrow\infty}
                     \frac{1}{2T}
                     \int\limits_{-T}^{T} \E(x,y,z,t) \; dt$
  \\The shape of $\vect{p}(x,y,z,t)$ is the \propd{polarization} of $\E(x,y,z,t)$.
  }
\end{definition}

%--------------------------------------
\begin{remark}
\footnote{\citerp{inan}{96}}
%--------------------------------------
An object can affect the polarization of a wave.
This has been exploited in radar systems to distinguish a metal object from
clouds and ``clutter".
\end{remark}

