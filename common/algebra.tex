%============================================================================
% Daniel J. Greenhoe
% LaTeX file
%============================================================================
%======================================
\chapter{Algebraic structures}
\label{app:algebra}
\index{algebra}
%======================================
\qboxnpq
  {\href{http://en.wikipedia.org/wiki/Cardano}{Gerolamo Cardano}
   \href{http://www-history.mcs.st-andrews.ac.uk/Timelines/TimelineC.html}{(1501--1576)},
   \href{http://www-history.mcs.st-andrews.ac.uk/BirthplaceMaps/Places/Italy.html}{Italian} mathematician, physician, and astrologer
   \footnotemark
   \index{Cardano, Gerolamo}
   \index{quotes!Cardano, Gerolamo}
  }
  {../common/people/cardano.jpg}
  {In this book, learned reader, you have the rules of algebra\ldots
    It unties the knot not only where one term is equal to another or two to one
    but also where two are equal to two or three to one.\ldots
    this most abstruse and unsurpassed treasury of the entire arithmetic being
    brought to light and, as in a theater, exposed to the sight of all\ldots}
  \citetblt{
    quote: & \citerp{cardano1545e}{1} \\
    image: & \url{http://en.wikipedia.org/wiki/Image:Cardano.jpg}
    }

A set together with one or more operations forms several standard
mathematical structures:

\indentx \structe{group} $\supseteq$ \structe{ring} $\supseteq$ \structe{commutative ring} $\supseteq$ \structe{integral domain} $\supseteq$ \structe{field}

%---------------------------------------
\begin{definition}
\citetbl{
  \citerp{durbin2000}{29}
  }
\index{group}
\label{def:alg_group}
\label{def:group}
%---------------------------------------
Let $\setX$ be a set and $\Diamond:\setX\times \setX\to \setX$ be an operation on $\setX$. \\
\defboxt{
  The pair $\opair{\setX}{\Diamond}$ is a \structd{group} if
  \\\indentx$\begin{array}{Fll @{\qquad}C @{\qquad}DD}
    1. & \exists e \in \setX \st
       & e \Diamond x = x\Diamond e    = x
       & \forall x\in\setX
       & (\prope{identity} element)
       & and
       \\
    2. & \exists (-x) \in \setX \st
       & (-x) \Diamond x = x\Diamond (-x) = e
       & \forall x\in\setX
       & (\prope{inverse} element)
       & and
       \\
    3. &
       & x\Diamond(y\Diamond z) = (x\Diamond y)\Diamond z
       & \forall x,y,z\in\setX
       & (\prope{associative})
  \end{array}$
  }
\end{definition}

%---------------------------------------
\begin{definition}
\citetbl{
  \citerpp{durbin2000}{114}{115}
  }
\label{def:ring}
%---------------------------------------
Let $+:\setX\times \setX\to \setX$ and $*:\setX\times \setX\to \setX$
be operations on a set $\setX$.
Furthermore, let the operation $*$ also be represented by
juxtapostion as in $a*b\equiv ab$.\\
\defboxt{
  The triple $\otriple{\setX}{+}{*}$ is a \structd{ring} if
  \\\indentx
  $\begin{array}{F lcl @{\qquad}C @{\qquad}D @{\qquad}D}
     1. & \mc{3}{l}{\text{$\opair{\setX}{+}$ is a group.}}
        &
        & (additive group)
        & and
        \\
     2. & x(yz)  &=& (xy)z
        & \forall x,y,z\in\setX
        & (\prope{associative} with respect to $*$)
        & and
        \\
     3. & x(y+z) &=& (xy)+(xz)
        & \forall x,y,z\in\setX
        & ($*$ is \prope{left distributive} over $+$)
        & and
        \\
     4. & (x+y)z &=& (xz)+(yz)
        & \forall x,y,z\in\setX
        & ($*$ is \prope{right distributive} over $+$).
  \end{array}$}
\end{definition}

%---------------------------------------
\begin{definition}
\citetbl{
  \citerp{durbin2000}{118}
  }
\label{def:com_ring}
\index{ring!commutative}
%---------------------------------------
\defboxt{
  A triple $\otriple{\setX}{+}{*}$ is a \structd{commutative ring} if
  \\\indentx
  $\begin{array}{FMCD}
      1. & $\otriple{\setX}{+}{*}$ is a \structe{ring}  &                     &  and
    \\2. & $xy=yx$                                      & \forall x,y\in\setX & (\prope{commutative}).
  \end{array}$}
\end{definition}

%--------------------------------------
\begin{definition}
\citetbl{
  \citerpg{cohn}{312}{1852335874}
  }
\label{def:abs}
\index{ring!absolute value}
\index{ring!modulus}
%--------------------------------------
Let $R$ be a \structe{commutative ring} \xref{def:com_ring}.
\defboxt{
  A function $\absn$ in $\clFrr$ is an \fnctd{absolute value} (or \fnctd{modulus}) if
  \\\indentx
  $\begin{array}{F rcl @{\qquad}C@{\qquad}D@{\qquad}D}
    1. & \abs{ x}  &\ge& 0                     & x  \in\R & (\prope{non-negative})                              & and
  \\2. & \abs{ x}  &=  & 0 \iff x=0            & x  \in\R & (\prope{nondegenerate})                             & and
  \\3. & \abs{xy}  &=  & \abs{x}\cdot\abs{y}   & x,y\in\R & (\prope{homogeneous} / \prope{submultiplicative})   & and
  \\4. & \abs{x+y} &\le& \abs{x}+\abs{y}       & x,y\in\R & (\prope{subadditive} / \prope{triangle inequality}) &
  \end{array}$}
\end{definition}

%---------------------------------------
\begin{definition}
\footnote{
  \citerp{durbin2000}{123},
  \citor{weber1893}
  }
\label{def:field}
%---------------------------------------
\defboxt{
  The structure $F\eqd\fieldX$ is a \structd{field} if
  \\\indentx$\begin{array}{FlCDD}
   1. & \otriple{\setX}{+}{*} \text{ is a ring}
      &
      & (ring)
      & and
      \\
   2. & xy=yx
      & \forall x,y\in\setX
      & (commutative with respect to $*$)
      & and
      \\
   3. & (\setX\setd\setn{0},*) \text{ is a group}
      &
      & (group with respect to $*$).
\end{array}$}
\end{definition}

%---------------------------------------
\begin{definition}
\citetbl{
  \citerp{aa}{3},
  \citerp{michel1993}{56}
  }
\label{def:algebra}
%---------------------------------------
Let $\spV=(F,+,\cdot)$ be a vector space and
    $\otimes:\spV\times\spV\to\spV$ be a vector-vector multiplication operator.\\
An \structd{algebra} is any pair $\opair{\spV}{\otimes}$ that satisfies
($\otimes$ is represented by juxtaposition)
\defbox{\begin{array}{F rcl C DD}
   1.& (\vu\vx)\vy     &=& \vu(\vx\vy)
     & \forall \vu,\vx,\vy\in \spV
     & (\prope{associative})
     & and
     \\
   2.& \vu( \vx + \vy) &=& (\vu\vx) + (\vu\vy)
     & \forall \vu,\vx,\vy\in \spV
     & (\prope{left distributive})
     & and
     \\
   3.& ( \vu + \vx)\vy &=& (\vu\vy) + (\vx\vy)
     & \forall \vu,\vx,\vy\in \spV
     & (\prope{right distributive})
     & and
     \\
   4.& \alpha(\vx\vy)  &=& (\alpha \vx)\vy = \vx (\alpha \vy)
     & \forall \vx,\vy\in \spV \text{ and } \alpha\in F
     & (\prope{scalar commutative})
     & .
\end{array}}
\end{definition}
