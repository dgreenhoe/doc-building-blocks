%============================================================================
% LaTeX File
% Daniel J. Greenhoe
%============================================================================

%======================================
\chapter{Number Systems}
\label{app:complex}
\index{complex numbers}
%======================================

\qboxnq
  {\href{http://en.wikipedia.org/wiki/Proclus}{Proclus Lycaeus}
   (\href{http://www-history.mcs.st-andrews.ac.uk/Timelines/TimelineA.html}{412 -- 485 AD}),
   Greek philosopher
   \footnotemark
   \index{Proclus Lycaeus}
   \index{quotes!Proclus Lycaeus}
  }
  {Wherever there is number, there is beauty.}
  \citetblt{
    quote:       & \citerpg{kline1990}{131}{0195061357}
    }



The most common number systems are the following:
\\\begin{tabular}{>{$\imark$ }l *{3}{>{$}l<{$}} *{2}{>{\color{blue}\scriptsize}l}}
    The set of \structe{natural numbers} & \Zp  &\eqd& \setn{1,2,3,\ldots}                   & \pref{def:N} & \prefpo{def:N}
  \\The set of \structe{whole numbers}   & \Znn &\eqd& \setn{0,1,2,3,\ldots}                 & \pref{def:W} & \prefpo{def:W}
  \\The set of \structe{integers}        & \Z   &\eqd& \setn{\ldots,-3,-2,-1,0,1,2,3,\ldots} & \pref{def:Z} & \prefpo{def:Z}
  \\The set of \structe{rational numbers}& \Q   &\eqd& \set{\frac{n}{m}}{n\in\Z,\; m\in\Zp}  & \pref{def:Q} & \prefpo{def:Q}
  \\The set of \structe{real numbers}    & \R   &\eqd& \text{completion of } \Q              & \pref{def:R} & \prefpo{def:R}
  \\The set of \structe{complex numbers} & \C   &\eqd& \set{\opair{x}{y}}{x,y\in\R}          & \pref{def:C} & \prefpo{def:C}
  \end{tabular}

Generally there are two ways to construct these numbers systems:%
\citetbl{\citerppg{bjb}{3}{4}{0486255638}}
\begin{dingautolist}{"AC}
  \item \hib{ordinal approach}:
      Start by defining the natural numbers axiomatically using the \hie{Peano axioms}.
      From there construct the other number systems.
      The ordinal approach presents the natural numbers as having an inherent \hie{order}
      such that each natural number has a unique \hie{successor}
      that is also a natural number.

  \item \hib{cardinal approach}:
      Start by defining the real numbers axiomatically using their \hie{field properties}
      as axioms.
      From there define the natural numbers, whole numbers, integers, and rational numbers
      as subsets of the reals,
      and define the complex numbers as ordered pairs of the reals.
      The cardinal approach presents the real numbers as having inherent \hie{size}
      such that any number except zero can be sliced into smaller and smaller pieces.
\end{dingautolist}

Either approach is perfectly acceptable.
%Hilbert used the cardinal
%approach in this famous work \hie{Grundlagen der Geometrie}.\citetbl{
%  \citor{hilbert_fog} \\
%  \citor{hilbert_foge}
%  }
%Dedekind and Peano used the ordinal approach in their respective works
%\emph{Was sind und was sollen die Zahlen?} and
%\emph{Arithmetices principia, nova methodo exposita}.\citetbl{
%  \citor{dedekind1888} \\
%  \citor{dedekind1888e} \\
%  \citor{peano1889} \\
%  \citor{peano1889e}
%  }
%\citepg{amann2005}{29}{3764371536}
%
However, under the assumption that, say, four thousand years ago
most every culture on earth had some kind of counting system,
but relatively few, if any, were familiar with the real number system,
one might argue that the ordinal approach
is a more ``natural" choice for number system construction.
And indeed it is the ordinal approach that is presented in the following material.

%=======================================
\section{Natural numbers}
%=======================================
\qboxnpq
  {\href{http://www-history.mcs.st-andrews.ac.uk/Mathematicians/Dedekind.html}{Richard Dedekind}
   (\href{http://www-history.mcs.st-andrews.ac.uk/Timelines/TimelineF.html}{1831--1915}),
   \href{http://www-history.mcs.st-andrews.ac.uk/BirthplaceMaps/Places/Germany.html}{German} mathematician
   \footnotemark
   \index{Dedekind, Richard}
   \index{quotes!Dedekind, Richard}
  }
  {../common/people/dedekind.jpg}
  {I regard the whole of arithmetic as a necessary, or at least natural,
   consequence of the simplest arithmetic act, that of counting,
   and counting itself as nothing else than the successive creation
   of the infinite series of positive integers in which each individual
   is defined by the one immediately preceding;
   the simplest act is the passing from an already-formed individual to the consecutive
   new one to be formed.}
  \citetblt{
    quote:       & \citeP{dedekind1872} \\
    translation: & \citerp{dedekind1872e}{2},
                   \citerp{dedekind1872ed}{2?},
                   \citerpg{dedekind1872eo}{768}{019850537X} \\
                %& Stanley Burris, \url{http://www.math.uwaterloo.ca/~snburris/htdocs/scav/dedek/dedek.html} \\
    image:       & \url{http://turnbull.mcs.st-and.ac.uk/history/PictDisplay/Dedekind.html}
    }

%=======================================
\subsection{Definitions}
%=======================================

\qboxnpq
  { \href{http://en.wikipedia.org/wiki/Bertrand_Russell}{Bertrand Russell}
    \href{http://www-history.mcs.st-andrews.ac.uk/Timelines/TimelineF.html}{(1872--1970)},
    \href{http://www-history.mcs.st-andrews.ac.uk/BirthplaceMaps/Places/UK.html}{British mathematician},
    \index{Russull, Bertrand}
    \index{quotes!Russull, Bertrand}
    \footnotemark
  }
  {../common/people/russell1907_wkp_pdomain.jpg} % http://en.wikipedia.org/wiki/File:Russell1907-2.jpg
  %{../common/people/russell.jpg}
  {The Congress was a turning point in my intellectual life, 
  because I there met Peano.%
  \ldots
  %I already knew him by name and had seen some of his work, 
  %but had not taken the trouble to master  his notation. 
  In discussions at the Congress I observed that he was always 
  more precise than anyone else, and that he invariably got the better 
  of any argument upon which he embarked,
  As the days went by, I decided that this must be owing to his 
  mathematical logic.%
  \ldots
  %I therefore got him to give me all his works, 
  %and as soon as the Congress was over I retired to Fernhurst 
  %to study quietly every word written by him and his disciples. 
  It became clear to me that his notation afforded an instrument 
  of logical analysis such as I had been seeking for years, 
  and that by studying him I was acquiring a new and powerful 
  technique for the work that I had long wanted to do.}
\citetblt{
  quote: & \citerpp{russell1951}{216}{217}\\
  image: & \url{http://en.wikipedia.org/wiki/File:Russell1907-2.jpg}, public domain
  }

%---------------------------------------
\begin{axiom}[Peano's axioms]
\label{ax:peano}
\label{def:N}
\footnote{
  \citerpg{landau1966}{2}{082182693X},
  \citerpg{halmos1960}{46}{0387900926},
  \citerpg{thurston1956}{51}{0486458067},
  \citor{peano1889},
  \citorpg{peano1889e}{94}{158348597X},
  \citor{dedekind1888},
  \citorp{dedekind1888e}{67},
  \citerppgc{cori2001}{8}{15}{0198500513}{recursion theory}
  }
\index{Peano's axioms}
\index{axiom!Peano}
\index{number!natural}
\index[xsym]{$\Zp$}
%---------------------------------------
\axboxt{
  The set $\hxsd{\Zp}$ of \structd{natural numbers} satisfies the following axioms:
  \\\indentx$\ds\begin{array}{FlCD}
    1. & 1 \in \Zp
       &
       & ($\Zp$ is not empty)
  \\2. & \fsp{n}\in\Zp \qquad
       & \forall n\in\Zp
       & (every element $n$ in $\Zp$ has a \hie{successor} $\fsp{n}$ in $\Zp$)
  \\3. & \fsp{n}=\fsp{m} \iff n=m
       & \forall n,m\in\Zp
       & ($\fs$ is a \hie{bijection})
  \\4. & \fsp{n}\ne  1
       & \forall n\in\Zp
       & ($1$ is not a successor of any element in $\Zp$)
  \\5. & \forall \setM \subseteq \Zp
       &
       & (\hie{axiom of induction})  \index{axiom!induction}
  \\   & \mc{3}{l}{
           \left.\begin{array}{FlD}
                 (5a) & 1\in\setM                      & \emph{and}\\
                 (5b) & m\in\setM \implies \fsp{m}\in\setM
           \end{array}\right\}
           \qquad\implies\qquad
           \setM = \Zp
           }
  \end{array}$
  \\
    The set $\Zp$ is called the set of \structd{natural numbers}.
    The value $\fsp{n}$ is called the \vald{successor} of $n$.\\
    The element $1$ is called ``\vald{one}".
  }
\end{axiom}

%=======================================
\subsection{Addition}
%=======================================
\qboxnpq
  {\href{http://en.wikipedia.org/wiki/Frege}{Gottlob Frege}
   \href{http://www-history.mcs.st-andrews.ac.uk/Timelines/TimelineF.html}{(1848--1925)},
   \href{http://www-history.mcs.st-andrews.ac.uk/BirthplaceMaps/Places/Germany.html}{German} mathematician, logician, and philosopher
   \footnotemark
   \index{Frege, Gottlob}
   \index{quotes!Frege, Gottlob}
  }
  {../common/people/frege.jpg}
  {And even now I do not see how arithmetic can be scientifically established;
   how numbers can be apprehended as logical objects, and brought under review;
   unless we are permitted -- at least conditionally --
   to pass from a concept to its extension.}
  \citetblt{
    quote:       & \citor{frege1903}  \\
    translation: & \citorp{fregeGeach1952}{234},
                   \citorpg{fregeBeaney1997}{280}{0631194452}\\
    image:       & \url{http://en.wikipedia.org/wiki/Image:Frege.jpg}
    }

%---------------------------------------
\begin{definition}[Addition over $\Zp$]
\label{def:N+}
\citetbl{
  \citor{dedekind1888},
  \citorp{dedekind1888e}{97},
  %\citerpg{thurston1956}{10}{0486458067} \\
  \citerpg{landau1966}{4}{082182693X}
  }
%---------------------------------------
Let $\fsp{n}$ be the \fncte{successor} of $n$ \xref{def:N}.
\defboxt{
  Let $+:\Zp\times\Zp\to\Zp$ be an operator that satisfies the following conditions:
  \\\indentx$\begin{array}{FlclCD}
    1. & \fsp{n}   &=& n + 1       & \forall n\in\Zp   & and \\
    2. & \fs(n+m)  &=& n + \fsp{m} & \forall n,m\in\Zp.&
  \end{array}$
  \\
  The quantity $n\hxsd{+}m$ is called the \opd{sum} of $n$ and $m$.
  }
\end{definition}


%---------------------------------------
\begin{lemma}
\label{lem:N+}
\citetbl{
  \citorpp{dedekind1888e}{97}{99}
  }
%---------------------------------------
Let $n+m$ represent the \hie{sum} of $n$ and $m$.
\lembox{\begin{array}{lcl@{\qquad}C}
  \fsp{m} + n &=& m + \fsp{n}  &\forall n,m,k\in\Zp  \\
  \fs(m+n)   &=& \fsp{m}+n    &\forall n,m\in\Zp    \\
  \fsp{n}     &=& 1+n         &\forall n\in\Zp
\end{array}}
\end{lemma}
\begin{proof}
\begin{enumerate}
  \item Proof that $\fsp{m} + n = m + \fsp{n}$:
    \begin{enumerate}
      \item Let $\setM$ be the set of values for which $\fsp{m} + n = m + \fsp{n}$.
      \item Proof that $1\in\setM$:
        \begin{align*}
          \fsp{m} + 1
            &= \fs\brs{\fsp{m}}
            && \text{by definition of addition over $\Zp$ \prefpo{def:N+}}
          \\&= \fs\brs{m+1}
            && \text{by definition of addition over $\Zp$ \prefpo{def:N+}}
          \\&= m + \fs(1)
            && \text{by definition of addition over $\Zp$ \prefpo{def:N+}}
        \end{align*}

      \item Proof that $m\in\setM\implies\fsp{m}\in\setM$:
        \begin{align*}
          \fsp{m} + \fsp{n}
            &= \fsp{m} + \brs{n+ 1}
            && \text{by definition of addition over $\Zp$ \prefpo{def:N+}}
          \\&= \brs{\fsp{m} + n} + 1
            && \text{by associative property}
          \\&= \brs{m + \fsp{n}} + 1
            && \text{by left hypothesis}
          \\&= m + \brs{\fsp{n} + 1}
            && \text{by associative property}
          \\&= m + \fs\brp{\fsp{n}}
            && \text{by associative property}
        \end{align*}

      \item Therefore, by the \hie{axiom of induction} (\prefpo{def:N}),
            $\setM=\Zp$ (all of $\Zp$ has the property of $\setM$).
    \end{enumerate}

  \item Proof that $\fsp{m}+n = \fs(m+n)$:
    \begin{align*}
      \fsp{m} + n
        &= m + \fsp{n}
        && \text{by previous lemma (2)}
      \\&= m + [n + 1]
        && \text{by definition of addition over $\Zp$ \prefpo{def:N+}}
      \\&= [m + n] + 1
        && \text{by associative property}
      \\&= \fs(m + n)
        && \text{by definition of addition over $\Zp$ \prefpo{def:N+}}
    \end{align*}

  \item Proof that $1+n=\fsp{n}$:
    \begin{enumerate}
      \item Let $\setM$ be the set of values for which $1+n=\fsp{n}$.
      \item Proof that $1\in\setM$:
        \begin{align*}
          1 + 1
            &= \fs(1)
            && \text{by definition of addition over $\Zp$ \prefpo{def:N+}}
        \end{align*}

      \item Proof that $m\in\setM\implies\fsp{m}\in\setM$:
        \begin{align*}
          1 + \fsp{n}
            &= 1 + [n+1]
            && \text{by definition of addition over $\Zp$ \prefpo{def:N+}}
          \\&= [1 + n] + 1
            && \text{by associative property}
          \\&= \fsp{n} + 1
            && \text{by left hypothesis}
          \\&= \fs\brp{\fsp{n}}
            && \text{by definition of addition over $\Zp$ \prefpo{def:N+}}
        \end{align*}

      \item Therefore, by the \hie{axiom of induction} (\prefpo{def:N}),
            $\setM=\Zp$ (all of $\Zp$ has the property of $\setM$).
    \end{enumerate}

\end{enumerate}
\end{proof}



%---------------------------------------
\begin{theorem}[\thmd{Associative property of addition}]
\label{thm:N+_assoc}
\citetbl{
  \citorpp{dedekind1888e}{97}{99},
  \citerppg{landau1966}{4}{5}{082182693X},
  \citerpg{thurston1956}{53}{0486458067}
  }
%---------------------------------------
Let $n+m$ represent the \hie{sum} of $n$ and $m$.
\thmbox{\begin{array}{lcl@{\qquad}C@{\qquad}D}
  (n + m) + k   &=& n + (m + k)  & \forall n,m,k\in\Zp  & (\hie{associative}) \\
\end{array}}
\end{theorem}
\begin{proof}
    \begin{enumerate}
      \item Let $\setM$ be the set of values for which $\opair{\Zp}{+}$ is associative.
      \item Proof that $1\in\setM$:
        \begin{align*}
          (n+m)+1
            &= \fs(n+m)
            && \text{by \prefp{def:N+}}
          \\&= n + \fsp{m}
            && \text{by \prefp{def:N+}}
          \\&= n + (m+1)
            && \text{by \prefp{def:N+}}
        \end{align*}
      \item Proof that $k\in\setM\implies\fs(k)\in\setM$:
        \begin{align*}
          \brs{n+m} + \fs(k)
            &= \fs\brp{\brs{n+m} + k}
            && \text{by \prefp{def:N+}}
          \\&= \fs\brp{n+\brs{m + k}}
            && \text{by left hypothesis}
          \\&= n+\brs{\fs(m + k)}
            && \text{by \prefp{def:N+}}
          \\&= n + \brs{m + \fs(k)}
            && \text{by \prefp{def:N+}}
        \end{align*}

      \item Therefore, by the \hie{axiom of induction} (\prefpo{def:N}),
            $\setM=\Zp$ (all of $\Zp$ is associative under $+$).
    \end{enumerate}

\end{proof}


%---------------------------------------
\begin{theorem}[\thmd{Commutative property of addition}]
\label{thm:N+_com}
\citetbl{
  \citerpp{dedekind1888e}{97}{99},
  \citerppg{landau1966}{4}{5}{082182693X},
  \citerpg{thurston1956}{53}{0486458067} 
  }
%---------------------------------------
Let $n+m$ represent the \hie{sum} of $n$ and $m$.
\thmbox{\begin{array}{lcl@{\qquad}C@{\qquad}D}
   n + m        &=& m + n        & \forall n,m\in\Zp    & (\prope{commutative})
\end{array}}
\end{theorem}
\begin{proof}
    \begin{enumerate}
      \item Let $\setM$ be the set of values for which $\opair{\Zp}{+}$ is commutative.
      \item Proof that $1\in\setM$:
        \begin{align*}
          1 + n
            &= \fsp{n}
            && \text{by \prefp{lem:N+}}
          \\&= n+1
            && \text{by definition of addition over $\Zp$ \prefpo{def:N+}}
        \end{align*}

      \item Proof that $m\in\setM\implies\fsp{m}\in\setM$:
        \begin{align*}
          n + \fsp{m}
            &= n + \brs{m+1}
            && \text{by definition of addition over $\Zp$ \prefpo{def:N+}}
          \\&= \brs{n + m}+1
            && \text{by associative property}
          \\&= \brs{m + n}+1
            && \text{by left hypothesis}
          \\&= m + \brs{n+1}
            && \text{by associative property}
          \\&= m + \brs{1+n}
            && \text{by \prefp{lem:N+}}
          \\&= \brs{m + 1} + n
            && \text{by associative property}
          \\&= \fsp{m} + n
            && \text{by definition of addition over $\Zp$ \prefpo{def:N+}}
        \end{align*}

      \item Therefore, by the \hie{axiom of induction} (\prefpo{def:N}),
            $\setM=\Zp$ (all of $\Zp$ is commutative under $+$).
    \end{enumerate}
\end{proof}


%=======================================
\subsection{Multiplication}
%=======================================
%---------------------------------------
\begin{definition}[Multiplication over $\Zp$]
\label{def:Nx}
\citetbl{
  \citorp{dedekind1888e}{101},
  %\citerpg{thurston1956}{11}{0486458067} \\
  \citerpg{landau1966}{14}{082182693X}
  }
%---------------------------------------
Let $\fsp{n}$ be the \hie{successor} of $n$.
\defboxt{
  Let $\hxsd{\cdot}:\Zp\times\Zp\to\Zp$ be an operator that satisfies
  \\\indentx$\begin{array}{FlclCD}
    1. & n \cdot 1       &=& n            & \forall n\in\Zp & and
  \\2. & n \cdot \fsp{m} &=& n\cdot m + n & \forall n,m\in\Zp
  \end{array}$
  \\
  The quantity $n\cdot m$ is called the \opd{product} of $n$ and $m$.\\
  The operation $+$ is called \opd{addition}.
  }
\end{definition}



%---------------------------------------
\begin{lemma}
\label{lem:Nx}
\citetbl{
  \citorpp{dedekind1888e}{101}{102}
  }
%---------------------------------------
\lembox{\begin{array}{rcl@{\qquad}C}
  \fsp{n}m &=& nm+m    & \forall n,m\in\Zp  \\
  1 \cdot n &=& n     & \forall n\in\Zp
\end{array}}
\end{lemma}
\begin{proof}
\begin{enumerate}
  \item Proof that $\fsp{n}m = nm+m$:
    \begin{enumerate}
      \item Let $\setM$ be the set of values for which $\fsp{n}m = nm+m$.
      \item Proof that $1\in\setM$:
        \begin{align*}
          \fsp{n}\cdot1
            &= \fsp{n}
            && \text{by definition of multiplication over $\Zp$ \prefpo{def:Nx}}
          \\&= n+1
            && \text{by definition of addition over $\Zp$ \prefpo{def:N+}}
          \\&= n\cdot1+1
            && \text{by definition of multiplication over $\Zp$ \prefpo{def:Nx}}
        \end{align*}

      \item Proof that $m\in\setM\implies \fsp{n}\fsp{m} = n\fsp{m}+\fsp{m}$:
        \begin{align*}
          \fsp{n}\fsp{m}
            &= \fsp{n}m + \fsp{n}
            && \text{by definition of multiplication over $\Zp$ \prefpo{def:Nx}}
          \\&= \brs{nm+m} + \fsp{n}
            && \text{by left hypothesis}
          \\&= nm+\brs{m + \fsp{n}}
            && \text{by associative property of $+$ over $\Zp$ \prefpo{thm:N+_assoc}}
          \\&= nm+\brs{\fsp{n}+m}
            && \text{by commutative property of $+$ over $\Zp$ \prefpo{thm:N+_com}}
          \\&= nm+\brs{n+\fsp{m}}
            && \text{by \prefp{lem:N+}}
          \\&= \brs{nm+n}+\fsp{m}
            && \text{by commutative property of $+$ over $\Zp$ \prefpo{thm:N+_com}}
          \\&= n\fsp{m}+\fsp{m}
            && \text{by definition of multiplication over $\Zp$ \prefpo{def:Nx}}
        \end{align*}

      \item Therefore, by the \hie{axiom of induction} (\prefpo{def:N}),
            $\fsp{n}m = nm+m$.
    \end{enumerate}

  \item Proof that $1\cdot n = n$:
    \begin{enumerate}
      \item Let $\setM$ be the set of values for which $1\cdot n = n$.
      \item Proof that $1\in\setM$:
        \begin{align*}
          1\cdot1
            &= 1
            && \text{by definition of multiplication over $\Zp$ \prefpo{def:Nx}}
        \end{align*}

      \item Proof that $n\in\setM\implies 1\cdot\fsp{n} = \fsp{n}$:
        \begin{align*}
          1 \cdot \fsp{n}
            &= 1\cdot n + 1
            && \text{by definition of addition over $\Zp$ \prefpo{def:N+}}
          \\&= n + 1
            && \text{by left hypothesis}
          \\&= \fsp{n}
            && \text{by definition of addition over $\Zp$ \prefpo{def:N+}}
        \end{align*}

      \item Therefore, by the \hie{axiom of induction} (\prefpo{def:N}),
            $1\cdot n = n$.
    \end{enumerate}

\end{enumerate}
\end{proof}




%---------------------------------------
\begin{theorem}[Commutative property of multiplication]
\label{thm:Nx_com}
\citetbl{
  \citorp{dedekind1888e}{102},
  \citerpg{landau1966}{15}{082182693X}
  }
%---------------------------------------
Let $n\cdot m$ represent the \hie{product} of $n$ and $m$.
\thmbox{\begin{array}{lcl@{\qquad}C@{\qquad}D}
  nm    &=& mn     & \forall n,m\in\Zp    & (\hie{commutative})
\end{array}}
\end{theorem}
\begin{proof}
  \begin{enumerate}
    \item Let $\setM$ be the set of values for which $nm=mn$.
    \item Proof that $1\in\setM$:
      \begin{align*}
        n\cdot1
          &= n
          && \text{by definition of multiplication over $\Zp$ \prefpo{def:Nx}}
        \\&= 1\cdot n
          && \text{by \prefp{lem:Nx}}
      \end{align*}

    \item Proof that $m\in\setM\implies\fsp{m}\in\setM$:
      \begin{align*}
        n\cdot\fsp{m}
          &= nm + n
          && \text{by definition of multiplication over $\Zp$ \prefpo{def:Nx}}
        \\&= mn + n
          && \text{by left hypothesis}
        \\&= \fsp{m}n
          && \text{by \prefp{lem:Nx}}
      \end{align*}

    \item Therefore, by the \hie{axiom of induction} (\prefpo{def:N}),
          $\setM=\Zp$ ($nm=mn \quad\forall m\in\Zp$).
  \end{enumerate}



  %  \begin{enumerate}
  %    \item Let $\setM$ be the set of values for which $\opair{\Zp}{\cdot}$ is commutative.
  %    \item Proof that $1\in\setM$:
  %    \item Proof that $k\in\setM\implies\fs(k)\in\setM$:
  %    \item Therefore, by the \hie{axiom of induction} (\prefpo{def:N}),
  %          $\setM=\Zp$ (all of $\Zp$ is commutative under $+$).
  %  \end{enumerate}
\end{proof}


%---------------------------------------
\begin{theorem}[Distributive propterties]
\label{thm:N_distrib}
\citetbl{
  \citorpp{dedekind1888e}{102}{103},
  \citerpg{landau1966}{16}{082182693X}
  }
\index{distributive!left}
\index{distributive!right}
%---------------------------------------
Let $n+m$ represent the \hie{sum} of $n$ and $m$ and
    $n\cdot m$ represent the \hie{product} of $n$ and $m$.
\thmbox{\begin{array}{lcl@{\qquad}C@{\qquad}D}
  n(m+k) &=& nm + nk  & \forall n,m,k\in\Zp  & (\hie{left distributive})  \\
  (n+m)k &=& nk + mk  & \forall n,m,k\in\Zp  & (\hie{right distributive})
\end{array}}
\end{theorem}
\begin{proof}
\begin{enumerate}
  \item Proof that $\Zp$ is left distributive:
    \begin{enumerate}
      \item Let $\setM$ be the set of values for which $(\Zp,+,\cdot)$ is distributive.
      \item Proof that $1\in\setM$:
        \begin{align*}
          n(m+1)
            &= n\fsp{m}
            && \text{by definition of $+$ (\prefp{def:N+})}
          \\&= nm + n
            && \text{by definition of $\cdot$ (\prefp{def:Nx})}
          \\&= nm + n\cdot 1
            && \text{by definition of $\cdot$ (\prefp{def:Nx})}
        \end{align*}

      \item Proof that $k\in\setM\implies\fs(k)\in\setM$:
        \begin{align*}
          n(m+\fs(k))
            &= n\fs(m+k)
            && \text{by definition of $+$ (\prefp{def:N+})}
          \\&= n(m+k) + n
            && \text{by definition of $\cdot$ (\prefp{def:Nx})}
          \\&= (nm + nk) + n
            && \text{by left hypothesis}
          \\&= nm + (nk + n)
            && \text{by associative property of $\opair{\Zp}{+}$ (\prefp{thm:N+_assoc})}
          \\&= nm + n\fs(k)
            && \text{by definition of $\cdot$ (\prefp{def:Nx})}
        \end{align*}

      \item Therefore, by the \hie{axiom of induction} (\prefpo{def:N}),
            $\setM=\Zp$ (all of $(\Zp,+,\cdot)$ is distributive).
    \end{enumerate}

  \item Proof that $\Zp$ is right distributive:
    \begin{align*}
      (n+m)k
        &= k(n+m)
        && \text{by commutative property of multiplication over $\Zp$ (\prefpo{thm:Nx_com})}
      \\&= kn+km
        && \text{by left distributive property of $\Zp$}
      \\&= nk+mk
        && \text{by commutative property of multiplication over $\Zp$ (\prefpo{thm:Nx_com})}
    \end{align*}
\end{enumerate}
\end{proof}



%---------------------------------------
\begin{theorem}[Associative propterty of multiplication]
\label{thm:Nx_assoc}
\citetbl{
  \citorp{dedekind1888e}{103},
  \citerpg{landau1966}{16}{082182693X}
  }
%---------------------------------------
Let $n\cdot m$ represent the \hie{product} of $n$ and $m$.
\thmbox{\begin{array}{lcl@{\qquad}C@{\qquad}D}
  (nm)k &=& n(mk)  & \forall n,m,k\in\Zp  & (\hie{associative})
\end{array}}
\end{theorem}
\begin{proof}
  \begin{enumerate}
    \item Let $\setM$ be the set of values for which $n(mk)=(nm)k)$.
    \item Proof that $1\in\setM$:
      \begin{align*}
        n(m\cdot1)
          &= nm
          && \text{by definition of multiplication over $\Zp$ \prefpo{def:Nx}}
        \\&= (nm)\cdot 1
          && \text{by definition of multiplication over $\Zp$ \prefpo{def:Nx}}
      \end{align*}

    \item Proof that $k\in\setM\implies\fs(k)\in\setM$:
      \begin{align*}
        n \brs{m\fs(k)}
          &= n \brs{m(k+1)}
          && \text{by definition of addition over $\Zp$ \prefpo{def:N+}}
        \\&= n \brs{mk+m)}
          && \text{by left distributive property \prefpo{thm:N_distrib}}
        \\&= n(mk)+nm
          && \text{by left distributive property \prefpo{thm:N_distrib}}
        \\&= (nm)k+nm
          && \text{by left hypothesis}
        \\&= (nm)k+(nm)\cdot1
          && \text{by definition of multiplication over $\Zp$ \prefpo{def:Nx}}
        \\&= (nm)(k+1)
          && \text{by left distributive property \prefpo{thm:N_distrib}}
        \\&= (nm)\fs(k)
          && \text{by definition of addition over $\Zp$ \prefpo{def:N+}}
      \end{align*}

    \item Therefore, by the \hie{axiom of induction} (\prefpo{def:N}),
          $\setM=\Zp$ ($n(mk)=(nm)k\quad\forall k\in\Zp$).
  \end{enumerate}
\end{proof}



%---------------------------------------
\begin{definition}[exponents on $\Zp$]
\label{def:Nexp}
\citetbl{
  \citorp{dedekind1888e}{104}
  %\citerpg{landau1966}{14}{082182693X}
  }
%---------------------------------------
Let $\fsp{n}$ be the \hie{successor} of $n$.
\defboxt{
  Let $^:\Zp\to\Zp$ be an operator that satisfies
  \\\indentx
  $\begin{array}{FlclCD}
    1. & a^1         &=& a      & \forall a,n\in\Zp & and
  \\2. & a^{\fsp{n}} &=& a^n a  & \forall a,n\in\Zp
  \end{array}$
  \\
  The quantity $n$ is called the \vald{exponent} of $a^n$.
  The quantity $a$ is called the \vald{base} of $a^n$.
  }
\end{definition}

%---------------------------------------
\begin{theorem}
\label{thm:Nexp}
\citetbl{
  \citorpp{dedekind1888e}{104}{105}
  %\citerpg{landau1966}{14}{082182693X}
  }
%---------------------------------------
\thmbox{\begin{array}{rcl@{\qquad}C}
  a^{(n+m)}    &=& \brp{a^n} \, \brp{a^m}  & \forall n,m,a\in\Zp \\
  \brp{a^n}^m  &=& a^{(nm)}                & \forall n,m,a\in\Zp \\
  \brp{ab}^n   &=& \brp{a^n} \, \brp{b^n}  & \forall n,a,b\in\Zp
\end{array}}
\end{theorem}
\begin{proof}
\begin{enumerate}
  \item Proof that $a^{n+m} = a^n  a^m$:
    \begin{enumerate}
      \item Let $\setM$ be the set of values for which $a^{n+m} = a^n a^m$.
      \item Proof that $1\in\setM$:
        \begin{align*}
          a^{n+1}
            &= a^{\fsp{n}}
            && \text{by definition of addition over $\Zp$ \prefpo{def:N+}}
          \\&= a^n\, a
            && \text{by definition of exponents over $\Zp$ \prefpo{def:Nexp}}
          \\&= a^n\, a^1
            && \text{by definition of exponents over $\Zp$ \prefpo{def:Nexp}}
        \end{align*}

      \item Proof that $m\in\setM\implies\fsp{m}\in\setM$:
        \begin{align*}
          a^{n+\fsp{m}}
            &= a^{n + \brs{m+1}}
            && \text{by definition of addition over $\Zp$ \prefpo{def:N+}}
          \\&= a^{\brs{n + m}+1}
            && \text{by \prefp{thm:N+_com}}
          \\&= a^{\fs(n + m)}
            && \text{by definition of addition over $\Zp$ \prefpo{def:N+}}
          \\&= a^{n + m} \, a
            && \text{by definition of exponents over $\Zp$ \prefpo{def:Nexp}}
          \\&= \brs{a^n a^m}  a
            && \text{by left hypothesis}
          \\&= a^n \brs{a^m  a}
            && \text{by \prefp{thm:Nx_assoc}}
          \\&= a^n a^{\fsp{m}}
            && \text{by definition of exponents over $\Zp$ \prefpo{def:Nexp}}
        \end{align*}

      \item Therefore, by the \hie{axiom of induction} (\prefpo{def:N}),
            $\setM=\Zp$.
    \end{enumerate}


  \item Proof that $\brp{a^n}^m = a^{nm}$:
    \begin{enumerate}
      \item Let $\setM$ be the set of values for which $\brp{a^n}^m = a^{nm}$.
      \item Proof that $1\in\setM$:
        \begin{align*}
          \brp{a^n}^1
            &= \brp{a^n}^1
          \\&= a^{\fsp{n}}
            && \text{by definition of addition over $\Zp$ \prefpo{def:N+}}
          \\&= a^n\, a
            && \text{by definition of exponents over $\Zp$ \prefpo{def:Nexp}}
          \\&= a^n\, a^1
            && \text{by definition of exponents over $\Zp$ \prefpo{def:Nexp}}
        \end{align*}

      \item Proof that $m\in\setM\implies\fsp{m}\in\setM$:
        \begin{align*}
          \brp{a^n}^\fsp{m}
            &= \brp{a^n}^m \, \brp{a^n}
          \\&= a^{nm} \, a^n
            && \text{by left hypothesis}
          \\&= a^{nm+n}
            && \text{by previous property}
          \\&= a^{n\fsp{m}}
            && \text{by definition of multiplication over $\Zp$ \prefpo{def:Nx}}
        \end{align*}

      \item Therefore, by the \hie{axiom of induction} (\prefpo{def:N}),
            $\setM=\Zp$.
    \end{enumerate}


  \item Proof that $\brp{ab}^n = a^n b^n$:
    \begin{enumerate}
      \item Let $\setM$ be the set of values for which $\brp{a^n}^m = a^{nm}$.
      \item Proof that $1\in\setM$:
        \begin{align*}
          \brp{ab}^1
            &= ab
            && \text{by definition of exponents over $\Zp$ \prefpo{def:Nexp}}
          \\&= \brp{a^1}\, \brp{b^1}
            && \text{by definition of exponents over $\Zp$ \prefpo{def:Nexp}}
        \end{align*}

      \item Proof that $m\in\setM\implies\fsp{m}\in\setM$:
        \begin{align*}
          \brp{ab}^\fsp{n}
            &= \brp{ab}^n \, \brp{ab}
            && \text{by definition of exponents over $\Zp$ \prefpo{def:Nexp}}
          \\&= \brp{a^n b^n} \, \brp{ab}
            && \text{by left hypothesis}
          \\&= a^n \brp{b^n a} b
            && \text{by associative property of $\cdot$ over $\Zp$ (\prefpo{thm:Nx_assoc})}
          \\&= a^n \brp{a b^n } b
            && \text{by commutative property of $\cdot$ over $\Zp$ (\prefpo{thm:Nx_com})}
          \\&= \brp{a^n a} \, \brp{ b^n  b}
            && \text{by associative property of $\cdot$ over $\Zp$ (\prefpo{thm:Nx_assoc})}
          \\&= a^\fsp{n}  \, b^\fsp{n}
            && \text{by definition of exponents over $\Zp$ \prefpo{def:Nexp}}
        \end{align*}

      \item Therefore, by the \hie{axiom of induction} (\prefpo{def:N}),
            $\setM=\Zp$.
    \end{enumerate}
\end{enumerate}
\end{proof}

%=======================================
\subsection{Order}
%=======================================
%---------------------------------------
\begin{definition}
\label{def:N_order}
\citetbl{
  \citerpg{landau1966}{9}{082182693X}
  }
%---------------------------------------
Let $\Zp$ be the set of natural numbers.
\defboxt{
  $m \hxsd{\le} n$ \quad if \quad\\ 
  there exists $k\in\Zp$ such that $n = m+k$ or $m=n$
  }
\end{definition}

%---------------------------------------
\begin{theorem}
\label{thm:N_order}
%---------------------------------------
\thmboxt{
  $\opair{\Zp}{\le}$ is a \structe{totally ordered set}.
  }
\end{theorem}


%=======================================
\section{Whole Numbers}
%=======================================
%---------------------------------------
\begin{definition}
\label{def:W}
\indxs{\Znn}
%---------------------------------------
Let $\Zp$ be the set of natural numbers.
\defboxt{
  The set of \structd{whole numbers} $\hxsd{\Znn}$ is defined as $\Znn\eqd\Zp\setu0$ where\\
  $\hxsd{0}$ is the element that satisfies the condition $\fs(0)=1$.
  }
\end{definition}

%---------------------------------------
\begin{theorem}[Division]
\citetbl{
  \citerpg{amann2005}{35}{3764371536}
  }
\label{thm:N_div}
%---------------------------------------
\thmbox{
  \forall n,m\in\Znn \qquad \exists r,q\in\Znn \qquad \st \qquad n=qm+r \text{ and } r <q m
  }
\end{theorem}
\begin{proof}
\begin{enumerate}
  \item Let $\setM$ be defined as
    \[ \setM \eqd \set{n\in\Z}{\forall m\in\Z,\; \exists q,r\in\Znn \;\st\; n=qm+r}. \]

  \item Proof that $0\in\setM$:  $0 = 0\cdot m + r$

  \item Proof that $n\in\setM\implies\fsp{n}\in\setM$:
    \begin{align*}
      \fsp{n}
        &= \fs\brp{qm+r}
        && \text{left hypothesis}
      \\&= qm + \fs(r)
        && \text{by \prefp{def:N+}}
      \\&= \left\{\begin{array}{ll}
             qm + \fs(r)     & \text{for $\fs(r)<q m$}\\
             qm + m          & \text{for $\fs(r)=m$}
           \end{array}\right.
      \\&= \left\{\begin{array}{ll}
             qm + \fs(r)     & \text{for $\fs(r)<q m$}\\
             \fs(q) m        & \text{for $\fs(r)=m$}
           \end{array}\right.
        && \text{by \prefp{lem:Nx}}
    \end{align*}
\end{enumerate}
\end{proof}


%=======================================
\section{Integers}
%=======================================
%---------------------------------------
\begin{definition}
%---------------------------------------
\defbox{
  \text{The \opd{negative} number $\hxsd{-}n$ is the number that satisfies $n + (-n)=0$ forall $n\in\Zp$.}
  }
\end{definition}



%---------------------------------------
\begin{definition}
\label{def:Z}
\index[xsym]{$\Z$}
%---------------------------------------
\defboxt{
  The set of integers $\hxsd{\Z}$ is defined as \footnotemark
  \\\indentx
  $\ds\Z \eqd \Znn \setu \set{-n}{n\in\Zp}$
  }
\end{definition}
\footnotetext{
  note about $\Z$: ``Z" stands for the German word \hie{zahlen} which means \emph{numbers}.
  \\Reference: \citerpg{stillwell2002}{404}{0387953361}
  %\\Reference: \citerpg{pugh2002}{1}{0387952977}
  }



%=======================================
\section{Rational Numbers}
%=======================================
%---------------------------------------
\begin{definition}
\label{def:frac_eq}
\index{fraction!equivalence}
%---------------------------------------
\defboxt{
  The fractions $\frac{m}{n}$ and $\frac{p}{q}$ are \propd{equivalent} if $mq=np$. \\
  The equivalence class reprented by the member $\frac{m}{n}$ is denoted
  $\brs{\frac{m}{n}} \eqd \set{\frac{p}{q}}{\frac{p}{q}=\frac{m}{n}}$.
  }
\end{definition}

%---------------------------------------
\begin{definition}
\label{def:frac_order}
\index{fraction!order}
%---------------------------------------
\defbox{
  \frac{m}{n} \orel \frac{p}{q}
  \qquad\text{if}\qquad
  mq \orela np
  }
\end{definition}


A \hie{rational number} is any number $q$ that is a ratio of two integers,
the denominator not being $0$.
This is formally defined next.
%---------------------------------------
\begin{definition}
\label{def:Q}
\index{set!rational numbers}
\indxs{\Q}
%---------------------------------------
\defboxt{
  The set of rational numbers $\Q$ is defined as\footnotemark
  \\\indentx$\ds\Q\eqd\set{\brs{\frac{n}{m}}}{n\in\Z \text{ and } m\in\Zp}.$
  \\
  A \vald{rational number} is any element of $\Q$.
  }
\end{definition}
\footnotetext{
  note about $\Q$: ``Q" stands for the English word ``quotient".
  \\Reference: \citerpg{pugh2002}{2}{0387952977}
  }

%---------------------------------------
\begin{definition}
\label{def:Q+x}
\citetbl{
  \citerpg{strichartz1995}{19}{0867204710},
  \citerpp{hobson1921}{15}{17}
  }
%---------------------------------------
Let $\Q$ be the set of rational numbers (\prefp{def:Q}).
Let $\oplus$  be the operation of addition on the set of intergers $\Z$.
Let $\otimes$ be the operation of multiplication on $\Z$.
\defboxt{
  Let the \opd{addition operator} $+$ and \opd{multiplication operator} $\times$ be defined as follows:
  \\\indentx
  $\ds\brbr{\begin{array}{rc>{\ds}lCD}
    r &\eqd& \frac{n_r}{d_r} & n_r,d_r\in\Z & and \\
    s &\eqd& \frac{n_s}{d_s} & n_s,d_s\in\Z &
  \end{array}}
  \qquad\implies\qquad
  \brbl{\begin{array}{rc>{\ds}l}
    r + s      &\eqd& \frac{\brp{n_r \otimes d_s} \oplus \brp{n_s \otimes d_r}}{d_r \otimes d_s}\\
    r \times s &\eqd& \frac{n_r \otimes n_s}{d_r \otimes d_s}
  \end{array}}$
  }
\end{definition}

%=======================================
\section{Real numbers}
%=======================================
%\qboxnps
%  {\href{http://en.wikipedia.org/wiki/Hermite}{Charles Hermite}
%   \href{http://www-history.mcs.st-andrews.ac.uk/Timelines/TimelineF.html}{(1822 -- 1901)},
%   \href{http://www-history.mcs.st-andrews.ac.uk/BirthplaceMaps/Places/France.html}{French} mathematician
%    \index{Hermite, Charles}
%    \index{quotes!Hermite, Charles}
%    \footnotemark
%  }
%  {../common/people/hermite.jpg}
%  {I shall risk nothing on an attempt to prove the transcendence of the number $\pi$.
%   If others undertake this enterprise, no one will be happier than I at their success,
%   but believe me, my dear friend, it will not fail to cost them some efforts.}
%  \citetblt{
%   %quote: & \url{http://lagrange.math.trinity.edu/aholder/misc/quotes.shtml} \\
%    quote: & \citerp{simmons2007}{196} \\
%    image: & \url{http://www-groups.dcs.st-and.ac.uk/~history/PictDisplay/Hermite.html}
%    }

\qboxnpqt
  {Lindemann's recollection, published more than ten years after Leopold Kronecker's (pictured) death, of an encounter with 
    \href{http://en.wikipedia.org/wiki/Kronecker}{Kronecker}
    \href{http://www-history.mcs.st-andrews.ac.uk/Timelines/TimelineF.html}{(1823--1891)},
    \href{http://www-history.mcs.st-andrews.ac.uk/BirthplaceMaps/Places/Germany.html}{German} mathematician,
    after Lindemann had proved that $\pi$ is a transcendental number.~\footnotemark
    \index{Kronecker, Leopold}% 1882
    \index{quotes!Kronecker, Leopold}% 1882
  }
  {../common/people/kronecker.jpg}
  {Sp{\:}ter ging Kronecker noch weiter, indem er die Existenz irrationaler Zahlen leugnete;
   so sagte er mir in seiner lebhaften und zu Paradoxen geneigten Art einmal:
   ``Was n{\:u}tzt uns Ihre sch{/:o}ne Untersuchung {/:u}ber die Zahl \txpi?
   Wozu das Nachdenken {/:u}ber solche Probleme, 
   wenne es doch gar keine irrationalen Zahlen gibt?"}
  {Later, Kronecker went even further and denied the existence of irrational numbers; 
   thus, he once said to me in his lively and paradoxical way,
   ``Of what use to us are your beautiful researches about the number $\pi$?
     Why consider such problems when in fact there are no irrational numbers?" }
  %{Of what use is your beautiful investigation regarding $\pi$?
   %Why study such problems since irrational numbers are non-existent?}
  \citetblt{
   quote:      & \citorpc{poincare_sah_de}{246}{note 3}\\
   translation: & \citerpg{edwards2005}{203}{0387219781}\\
   %quote: & \citerpu{kline1990}{198}{} \\
   %quote: & \citerpg{kline1982}{232}{0195030850}\\
   %quote: & \url{http://www-groups.dcs.st-and.ac.uk/~history/HistTopics/Beginnings_of_set_theory.html} \\
    image: & \url{http://www-history.mcs.st-andrews.ac.uk/PictDisplay/Kronecker.html}
    }

\qboxnpq
  {\href{http://www-history.mcs.st-andrews.ac.uk/Mathematicians/Dedekind.html}{Richard Dedekind}
   (\href{http://www-history.mcs.st-andrews.ac.uk/Timelines/TimelineF.html}{1831--1915}),
   \href{http://www-history.mcs.st-andrews.ac.uk/BirthplaceMaps/Places/Germany.html}{German} mathematician
   \footnotemark
   \index{Dedekind, Richard}
   \index{quotes!Dedekind, Richard}
  }
  {../common/people/dedekind.jpg}
  {I found myself for the first time obliged to lecture upon the
   elements of the differential calculus and felt more keenly than ever before the
   lack of a really scientific foundation for arithmetic.\ldots
   For myself this feeling
   of dissatisfaction was so overpowering that I made the fixed resolve to keep
   meditating on the question till I should find a purely arithmetic and perfectly
   rigorous foundation for the principles of infinitesimal analysis.\ldots
   It then only remained to
   discover its true origin in the elements of arithmetic and thus at the same time
   to secure a real definition of the essence of continuity.
   I succeeded Nov. 24, 1858,\ldots}
  \citetblt{
    quote:       & \citorp{dedekind1872e}{1} \\
    image:       & \url{http://turnbull.mcs.st-and.ac.uk/history/PictDisplay/Dedekind.html}
    }


%=======================================
\subsection{Construction}
%=======================================
%---------------------------------------
\begin{definition}[Dedekind cuts]
\label{def:cut}
\citetbl{
  \citerpg{landau1966}{43}{082182693X},
  \citerpg{rudinp}{17}{007054235X}
  }
\index{Dedekind cuts}
%---------------------------------------
\defboxt{
  A set $\setL$ is a \structd{cut} if
  \\\indentx$\ds\begin{array}{FlDD}
    1. & \setL \subsetneq \Q       & ($\setL$ is a proper subset of $\Q$) & and \\
    2. & \setL \ne  \emptyset      & ($\setL$ is non-empty)               & and \\
    3. & \text{$r\in\setL$ and $q\in\Q$ and $q<r$} \quad\implies\quad q\in\setL &  & and \\
    4. & \forall r\in\setL,\; \exists q\in\setL \st r<q & ($\setL$ is open)
  \end{array}$
  }
\end{definition}

%---------------------------------------
\begin{lemma}
\citetbl{
  \citerpg{rudinp}{17}{007054235X}
  }
%---------------------------------------
Let $\opair{\Q}{\orel}$ be the ordered set of rationals.
Let $\setL$ be a cut on $\Q$.
\lembox{\begin{array}{lElcl}
  p\in\setL     & and & q\notin\setL &\implies& p<q \\
  q\notin\setL & and & q<r           &\implies& r\notin\setL
\end{array}}
\end{lemma}
\begin{proof}
\begin{align*}
  p\ge q
    &\implies q\in\setL
    && \text{by (3) in \prefp{def:cut}}
  \\&\implies \text{contradiction of left hypothesis}
  \\
  \\
  r\in\setL
    &\implies q\in\setL
    && \text{by (3) in \prefp{def:cut}}
  \\&\implies \text{contradiction of left hypothesis}
\end{align*}
\end{proof}


%---------------------------------------
\begin{definition}
\label{def:R}
\citetbl{
  \citerpg{pugh2002}{12}{0387952977}
  }
\index{sets!real numbers}
\index{real numbers}
%---------------------------------------
\defboxt{\indxs{\R}
  The set of real numbers $\R$ is the set of all cuts on $\Q$. That is,
  \\\indentx$\ds\R\eqd\set{\setL}{\text{$\setL$ is a cut}}$
  \\
  Any element of $\R$ (any cut) is a \vald{real number}.
  }
\end{definition}



%=======================================
\subsection{Order structure}
%=======================================
%---------------------------------------
\begin{definition}
\label{def:R_orderrel}
\citetbl{
  \citerpg{pugh2002}{13}{0387952977}
  }
%---------------------------------------
Let $\opair{\Q}{\orela}$ be the ordered set of rational numbers.
\defbox{
  x \orel y
  \quad\text{if} \quad
  x \subseteq y
  \qquad\text{for $x,y\in\R$ ($x$ and $y$ are cuts on $\Q$)}
  }
\end{definition}

%---------------------------------------
\begin{theorem}
%---------------------------------------
Let $\orel$ be the relation defined in \prefpp{def:R_orderrel}.
\thmboxt{
  $\opair{\R}{\orel}$ is a \hie{totally ordered set}. In particular,
  \\\indentx$\begin{array}{Fl@{\qquad}C@{\qquad}DD@{}r@{}D}
    \cline{6-6}
    1. & x \orel x
       & \forall x\in\R
       & (\prop{reflexive})
       & and \hspace{2ex}
       & \text{ }\vline
       &
       \\
    2. & x \orel y \text{ and } y \orel z \implies x \orel z
       & \forall x,y,z\in\R
       & (\prop{transitive})
       & and
       & \vline
       & \text{\hspace{2ex}partial order}
       \\
    3. & x \orel y \text{ and } y \orel x \implies x=y
       & \forall x,y\in\R
       & (\prop{antisymmetric})
       & and
       & \vline
       &
    \\\cline{6-6}
    4. & x,y\in\R \quad\implies\quad x\orel y \text{ or } y\orel x
       & \forall x,y\in\R
       & (\prop{comparable}).
       &
       &
  \end{array}$
  }
\end{theorem}



%---------------------------------------
\begin{theorem}
\label{thm:R_lub_prop}
\citetbl{
  \citerpg{pugh2002}{13}{0387952977},
  \citerpg{rudinp}{18}{007054235X}
  }
%---------------------------------------
\thmboxt{
  The set $\R$ has the least upper bound property. \footnotemark
  }
\end{theorem}
\footnotetext{
  \hie{least upper bound property}: \prefpp{def:lsb_prop}
  }

%=======================================
\subsection{Arithmetic}
%=======================================
%---------------------------------------
\begin{definition}[real addition]
%---------------------------------------
\defbox{
  x + y = \set{a+b}{a\in x,\,b\in y}
  \qquad\forall x,y\in\R
  }
\end{definition}




%=======================================
\subsection{Least upper bound properties}
%=======================================
%---------------------------------------
\begin{theorem}
\label{thm:R_abe}
\citetbl{
  \citerpg{apostol1975}{3}{9861541039}
  }
%---------------------------------------
Let $a,b,\epsilon\in\R$.
\thmbox{\begin{array}{llcl}
  a \le b+\epsilon & \forall \epsilon>0 & \implies & a \le b \\
  a <   b+\epsilon & \forall \epsilon>0 & \implies & a \le b
\end{array}}
\end{theorem}
\begin{proof}
\begin{enumerate}
  \item Choose $\epsilon=\frac{a-b}{2}$.
  \item
    \begin{align*}
      b<a
        &\implies b+\epsilon
         && = b + \frac{a-b}{2}
        &&    \text{by choice of $\epsilon=\frac{a-b}{2}$}
      \\&&& = \frac{a+b}{2}
      \\&&& < \frac{a+a}{2}
         && \text{by $b<a$ assumption}
      \\&&& = a
      \\&\implies a> b+\epsilon
      \\&\implies \text{contradiction of left hypotheses}
      \\&\implies a \le b
    \end{align*}
\end{enumerate}
\end{proof}

%---------------------------------------
\begin{theorem}
\label{thm:R_sup_axs}
\citetbl{
  \citerpg{apostol1975}{9}{9861541039}
  }
%---------------------------------------
Let $\setS\in\psetr$ be a subset of the set of real numbers $\R$.
\thmbox{
  a<\supS
  \qquad\implies\qquad
  \exists x\in\setS \st a<x\le\supS
  }
\end{theorem}
\begin{proof}
\begin{align*}
  \lnot\brp{\exists x\in\setS \st a<x\le\supS}
    &\implies \forall x\in\setS,\;  a<x\le\supS
  \\&\implies \forall x\in\setS,\;  x\le a<\supS
  \\&\implies \text{$a$ is a least upper bound for $\setS$}
  \\&\implies a=\supS
  \\&\implies \text{contradiction of left hypothesis}
  \\&\implies \exists x\in\setS \st a<x\le\supS
\end{align*}
\end{proof}

%---------------------------------------
\begin{theorem}
\label{thm:R_sup_ABC}
\citetbl{
  \citerpg{apostol1975}{10}{9861541039}
  }
%---------------------------------------
Let $\setA\in\psetr$ and $\setB\in\psetr$ be subsets of the set of real numbers $\R$.
Define $\setA+\setB\eqd\set{x+y}{x\in\setA,\,y\in\setB}$.
\thmbox{
  \brbr{\begin{array}{lD}
    \setC=\setA+\setB         & and \\
    \text{$\supA$ exists} & and \\
    \text{$\supB$ exists}
  \end{array}}
  \qquad\implies\qquad
  \brbl{\begin{array}{lD}
    \text{$\supC$ exists} & and \\
    \supC = \supA+\supB
  \end{array}}
  }
\end{theorem}
\begin{proof}
\begin{enumerate}
  \item Proof that $\supC$ exists and $\supC\le\supA+\supB$:
    \begin{align*}
      z = x+y
        &\le \supA+\supB
        &\implies \text{$\supC$ exists and $\supC\le\supA+\supB$}
    \end{align*}

  \item Proof that $\supA+\supB\le \supC$:
    \begin{align*}
      &\exists x,y \st \brp{\supA-\epsilon} + \brp{\supB-\epsilon} < x+y \le \supA+\supB
      \\&\implies \supA + \supB - 2\epsilon < x+y \le \supC
      \\&\implies \supA + \supB \le \supC
        && \text{by \prefp{thm:R_abe}}
    \end{align*}
\end{enumerate}
\end{proof}

%---------------------------------------
\begin{theorem}
%---------------------------------------
Let $\setS\in\psetr$ and $\setT\in\psetr$ be subsets of the set of real numbers $\R$.
\thmbox{
  \brbr{\begin{array}{lD}
    s \le t \quad \forall  s\in\setS,\, t\in\setT  & and \\
    \text{$\supT$ exists}
  \end{array}}
  \qquad\implies\qquad
  \brbl{\begin{array}{lD}
    \text{$\supS$ exists} & and  \\
    \supS \le \supT
  \end{array}}
  }
\end{theorem}
\begin{proof}
\begin{align*}
  s \le t \le \supT
    & \implies \text{$\supS$ exists and $\supS\le\supT$}
\end{align*}
\end{proof}


%=======================================
\subsection{Completeness}
%=======================================
%---------------------------------------
\begin{theorem}
\citepg{rudinp}{2}{007054235X}
%---------------------------------------
The set of rational numbers $\Q$ is not complete in the set of real numbers $\R$.
\end{theorem}
\begin{proof}
The proof is by an example of the number $\sqrt{2}$,
which we will show is not a rational number.
\begin{align*}
\intertext{1. Proof that [$n^2$ is even] $\implies$ [$n$ is even]:}
  \text{$n^2$ is even}
    &\implies \exists m\in\Z \st n^2 = 2m^2
  \\&\implies n   = \sqrt{2m^2}
  \\&\implies \exists p\in\Z \st n = \sqrt{2\cdot\mcom{2\cdot p^2}{$m^2$}}
  \\&\implies n = 2p
  \\&\implies \text{$n$ is even}
%
\intertext{2. Proof that $\sqrt{2}\notin\Q$:}
  \sqrt{2} \in\Q
    &\implies \exists n\in\Z, m\in\Z\setd\setn{0}, \text{ not both even } \st \sqrt{2} = \frac{n}{m}
  \\&\implies n^2 = 2m^2
  \\&\implies n^2  \text{ is even}
  \\&\implies n    \text{ is even}
  \\&\implies n^2  \text{ is divisible by $4$}
  \\&\implies 2m^2 \text{ is divisible by $4$}
  \\&\implies m^2  \text{ is even}
  \\&\implies m    \text{ is even}
  \\&\implies      \text{both $n$ and $m$ are even}
  \\&\implies \sqrt{2} \notin\Q
\end{align*}
\end{proof}


%---------------------------------------
\begin{theorem}[The Archimedean Property / Eudoxus axiom]
\citetbl{
  \citerp{ab}{17} \\
  \citerpg{carothers2000}{5}{0521497566}
  }
\label{thm:archprop}
\index{Archimedean Property} \index{properties!Archimedean} \index{axiom!Achimedean property}
\index{theorems!Archimedean Property}
\index{Eudoxus axiom}        \index{axiom!Eudoxus}
\index{theorems!Eudoxus axiom}
%---------------------------------------
\thmbox{
  \exists n\in\Zp \qquad\st\qquad nx > y
  \qquad\forall x,y\in\Rp
  }
\end{theorem}
\begin{proof}
\begin{enumerate}
  \item Proof that if $\supA$ of a set $\setA$ exists, then
        $\forall\epsilon>0,\,\exists x \st \supA-\epsilon<x\le\supA$:
        \label{item:archprop_supA}
    \begin{align*}
      \lnot\brp{\exists x \st \supA-\epsilon<x\le\supA}
        &\implies \nexists x \st \supA-\epsilon < x
      \\&\implies x \le \supA-\epsilon  \qquad\forall x\in\setA
      \\&\implies \text{($\supA-\epsilon$) is an upper bound of $\setA$}
      \\&\implies \supA\le\supA-\epsilon
      \\&\implies \text{impossibility}
      \\&\implies \exists x \st (\supA-\epsilon)<x\le\supA
    \end{align*}

  \item Proof that $\Zp$ is unbounded above:\label{item:archprop1}
    \begin{align*}
      \lnot\brp{\text{$\Zp$ is unbounded above}}
        &\implies \text{$\Zp$ is bounded above}
      \\&\implies \exists x\in\R \st n\leq x \quad\forall n\in\Zp
      \\&\implies \exists s\in\R \st s=\sup\Zp
        && \text{by \prefp{thm:R_lub_prop}}
      \\&\implies \exists n \st s-1 < n
        && \text{by \pref{item:archprop_supA}}
      \\&\implies \exists n \st s < n+1
      \\&\implies \exists n \st s < n+1 \le s
      \\&\implies s < s
      \\&\implies \text{impossibility}
      \\&\implies \text{$\Zp$ is unbounded above}
    \end{align*}

  \item Proof that $\exists n\in\Zp \st nx > y$:
    \begin{align*}
      \lnot\brs{\exists n\in\Zp \st nx > y}
        &\implies \nexists n\in\Zp \st nx > y
      \\&\implies nx \le y \quad\forall n\in\Zp
      \\&\implies n \le \frac{y}{x} \quad\forall n\in\Zp
      \\&\implies \text{$\Zp$ is bounded above by $\frac{y}{x}$}
      \\&\implies \text{impossibility}
        && \text{by \pref{item:archprop1}}
      \\&\implies \exists n\in\Zp \st nx > y
    \end{align*}

\end{enumerate}
\end{proof}

%---------------------------------------
\begin{theorem}
\citetbl{
  \citerpg{carothers2000}{5}{0521497566},
  \citerp{ab}{17}
  }
\label{thm:ran_xry}
%---------------------------------------
Let $\R$ be the set of real numbers and $\Q$ the set of rational numbers.
\thmbox{
  \left.\begin{array}{lD}
    x,y\in\R & and \\
    x<y
  \end{array}\right\}
  \qquad\implies\qquad
  \exists r\in\Q \st x< r< y
  }
\end{theorem}
\begin{proof}
\begin{align*}
  x,y\in\R \text{ and } x< y
    &\implies \forall x,y\in\R \quad \exists q\in\Zp \st q(y-x) > 1
    & \text{by the Archimedean property \prefpo{thm:archprop}}
  \\&\implies qy - qx > 1
  \\&\implies \exists p\in\Zp \st qy < p < qx
  \\&\implies y < \frac{p}{q} < x
  \\&\implies \exists r\in\Q \st x< r< y
\end{align*}
\end{proof}


%---------------------------------------
\begin{theorem}
\citetbl{
  \citerpg{carothers2000}{}{0521497566}
  }
%---------------------------------------
\thmbox{
  \brb{\begin{array}{FMD}
    1. & $\seq{x_n}{n\in\Z}$ is \propb{monotone} & and \\
    2. & $\seq{x_n}{n\in\Z}$ is \propb{bounded}
  \end{array}}
  \qquad\implies\qquad
  \brb{\begin{array}{M}
    $\seq{x_n}{n\in\Z}$\\
    \propb{converges}
  \end{array}}
  }
\end{theorem}
\begin{proof}
\attention
\begin{enumerate}
  \item Proof for monotonic \emph{increasing} sequences:
    \begin{enumerate}
      \item By boundness hypothesis, $\sup\seqn{x_n}$ exists in $\R$ such that
        \[x_m-\epsilon < \sup\seqn{x_n} \qquad \forall \epsilon>0,\, m\in\Z  \]

      \item By \prefpp{thm:ran_xry}, there exists $x\in\R$ between $x_m$ and $\sup\seqn{x_n}$ such that
        \[x_m-\epsilon < x \le \sup\seqn{x_n}  \qquad \forall \epsilon>0,\, m\in\Z  \]

      \item $x_m-x < \epsilon $
    \end{enumerate}

  \item Proof for monotonic \emph{decreasing} sequences:

\end{enumerate}

\end{proof}


%%---------------------------------------
%\begin{definition}
%\label{def:R}
%%---------------------------------------
%\defbox{
%  \text{The \structd{real numbers} are defined as
%  the completion of $\Q$ under the measure $\absn{}$.}
%  }
%\end{definition}
%


%=======================================
\subsection{Normed algebra structure}
%=======================================
%--------------------------------------
\begin{definition}
\label{def:R_abs}
\citetbl{
  \citerpg{apostol1975}{13}{9861541039}
  }
\indxs{\absn\in\clF{\R}{\R}}
%--------------------------------------
\defboxt{
  The \fnctd{absolute value} $\hxsd{\absn}\in\clF{\R}{\R}$ is defined as
  \\\indentx$\ds
  \abs{x} \eqd
  \brbl{\begin{array}{rEl}
     x & for & x\ge 0 \\
    -x & for & x<0
  \end{array}}
  $
  }
\end{definition}

%--------------------------------------
\begin{lemma}
\label{lem:R_abs}
\citetbl{
  \citerpg{apostol1975}{13}{9861541039}
  }
%--------------------------------------
\lembox{
  \abs{x}\le a \qquad\iff\qquad -a \le x\le a
  \qquad\scriptstyle\forall x\in\R
  }
\end{lemma}
\begin{proof}
\begin{enumerate}
  \item Proof that $\abs{x}\le a \;\implies\; -a \le x\le a$
    \begin{align*}
      \abs{x}\le a
        &\implies \brb{\begin{array}{rEl}
                     x & for & x\ge 0 \\
                    -x & for & x<0
                  \end{array}} \le a
        && \text{by \prefp{def:R_abs}}
      \\&\implies \brb{\begin{array}{rEl}
                     x\le a & for & x\ge 0 \\
                    (-x)\le a & for & x<0
                  \end{array}}
      \\&\implies \brb{\begin{array}{rEl}
                     x\le a & for & x\ge 0 \\
                    (x\ge -a) & for & x<0
                  \end{array}}
      \\&\implies -a\le x \le a
    \end{align*}

  \item Proof that $\abs{x}\le a \;\impliedby\; -a \le x\le a$
    \begin{align*}
      \abs{x}
        &= \brb{\begin{array}{rEl}
                  x & for & x\ge 0 \\
                 -x & for & x<0
                \end{array}}
        && \text{by \prefp{def:R_abs}}
      \\&\le \brb{\begin{array}{rEl}
                  a & for & x\ge 0 \\
                  a & for & x<0
                \end{array}}
        && \text{by left hypothesis}
      \\&= a
    \end{align*}
\end{enumerate}
\end{proof}

%--------------------------------------
\begin{theorem}[normed algebra properties]
\label{thm:R_norm}
\citetbl{
  \citerpg{apostol1975}{13}{9861541039}
  }
%--------------------------------------
Let $\absn\in\clF{\R}{\Rnn}$ be the absolute value function.
\thmboxt{
  The pair $\opair{\R}{\absn}$ is a \structe{normed algebra}. In particular
  \\\indentx
  $\ds\begin{array}{Frcl@{\qquad}C@{\qquad}D}
      1. & \abs{x}   &\ge& 0                 & \forall x \in\R    & (\prop{non-negative})
    \\2. & \abs{x}   &=&   0 \iff x=0        & \forall x \in\R    & (\prop{nondegenerate})
    \\3. & \abs{xy}  &=&   \abs{x}\:\abs{y}  & \forall x,y\in\R   & (\prop{homogeneous} / \prop{multiplicative condition})
    \\4. & \abs{x+y} &\le& \abs{x} + \abs{y} & \forall x,y\in\R   & (\prop{subadditive} / \prop{triangle inequality})
  \end{array}$
  }
\end{theorem}
\begin{proof}
  \begin{enumerate}
    \item Proof that $\abs{x}\ge0$: true by \prefp{def:R_abs}.

    \item Proof that $\abs{x}=0\iff x=0$: true by \prefp{def:R_abs}.

    \item Proof that $\abs{xy}=\abs{x}\abs{y}$:
      \begin{align*}
        \abs{xy}
          &= \brp{\begin{array}{rEl}
                xy & for & xy\ge 0 \\
               -xy & for & xy<0
             \end{array}}
           && \text{by definition of $\absn$ \prefpo{def:R_abs}}
        \\&= \brp{\begin{array}{rEl}
                x  & for & x \ge 0 \\
               -x  & for & x <0
             \end{array}}
             \brp{\begin{array}{rEl}
                 y & for &  y\ge 0 \\
               - y & for &  y<0
             \end{array}}
        \\&= \abs{x}\,\abs{y}
           && \text{by definition of $\absn$ \prefpo{def:R_abs}}
      \end{align*}

    \item Proof that $\abs{x+y}\le \abs{x}+\abs{y}$:
      \begin{enumerate}
        \item Start with these inequalities:
          \[\begin{array}{lcccl}
            -\abs{x} &\le& x &\le& \abs(x) \\
            -\abs{y} &\le& y &\le& \abs(y)
          \end{array}\]

        \item Add the above two equations to get the following:
          \[-\brp{\abs{x}+\abs{y}} \le x+y \le \brp{\abs{x}+\abs{y}} \]

        \item Then by \prefpp{lem:R_abs},
          \[ \abs{x+y} \le \abs{x} + \abs{y} \]
      \end{enumerate}

  \end{enumerate}
\end{proof}

%=======================================
\section{Complex numbers}
%=======================================
\qboxnpqt
  {
    Leonhard Euler (1707--1783), mathematician
    \index{Euler, Leonhard}
    \index{quotes!Euler, Leonhard}
    \footnotemark
  }
  %{../common/people/euler.jpg}
  {../common/people/euler1753bw_handmann_wkp_pdomain.jpg}
  {Weil nun alle m\"oglichen Zahlen, die man sich nur immer vorstellen mag,
    entweder gr\"o{\ss}er oder kleiner als 0, oder etwa 0 selbst sind, so ist klar, da{\ss}
    die Quadratwurzeln von Negativzahlen nicht einmal zu den m\"oglichen Zahlen
    gerechnet werden k\"onnen.
    Folglich m\"ussen wir sagen, da{\ss} dies unm\"ogliche Zahlen sind.
    Und dieser Umstand leitet uns auf den Begri{\ss} von solchen Zahlen,
    welche ihrer Natur nach unm\"oglich sind, und gew\"ohnlich imagin\"are
    oder eingebildete Zahlen genannt werden, weil sie blo\ss in der Einbildung
    vorhanden sind.}
  {And, since all numbers which it is possible to conceive,
    are either greater or less than 0, or are 0 itself,
    it is evident that we cannot rank the square root of a negative
    number amongst possible numbers, and we must therefore say that
    it is an \prop{impossible quantity}.
    In this manner we are led to the idea of numbers,
    which from their nature are impossible;
    and therefore they are usually called \hie{imaginary quantities},
    because they exist merely in the imagination.}
  \footnotetext{\begin{tabular}[t]{ll}
    quote: & \citorp{euler1770}{60} \\
    %www.math.uni-konstanz.de/~hoffmann/Funktionentheorie/kap1.pdf
    %www.springer.com/cda/content/document/cda_downloaddocument/9783540435549-c1.pdf?SGWID=0-0-45-130631-p2253211
    translation: & \citorp{euler1770e}{43} \\
    %image: & \url{http://en.wikipedia.org/wiki/Image:Leonhard_Euler.jpg}
    image: & \url{http://en.wikipedia.org/wiki/File:Leonhard_Euler.jpg}, public domain
  \end{tabular}}

%=======================================
\subsection{Definitions}
%=======================================
%--------------------------------------
\begin{definition}
\label{def:C}
\citetbl{
  \citerpg{landau1966}{92}{082182693X},
  \citerpg{rudinp}{12}{007054235X}
  }
\index{number!complex}
%--------------------------------------
\defboxt{%\indxs{\C}
  The set of \structd{complex numbers} $\hxsd{\C}$ is defined as
  \\\indentx$\ds
      \C \eqd \set{\opair{a}{b}}{a,b\in\R}
      \qquad
      \text{\scriptsize(the set of all ordered pairs of real numbers)}
    $
  \\
  A \vald{complex number} is any element of $\C$.
  }
\end{definition}




Addition and multiplication over $\C$ are defined next.
%--------------------------------------
\begin{definition}
\label{def:C+}
\label{def:Cx}
\citetbl{
  \citerppg{landau1966}{93}{96}{082182693X},
  \citerpg{thurston1956}{121}{0486458067},
  %\citerpg{rudinp}{12}{007054235X} \\
  \citorpg{bottazzini}{180}{0387963022},
  \citorpp{hamilton1837}{88}{93}
  }
\index{addition!complex}
\index{addition!multiplication}
%--------------------------------------
Let $\C$ be the set of complex numbers.
\defboxt{
  \hie{Addition} and \hie{multiplication} on $\C$ are defined as follows:
  \\
  $\ds\begin{array}{rcl cl C}
   \opair{x_r}{x_i} &\hxsd{+}&     \opair{y_r}{y_i} &\eqd& \opair{x_r+y_r}{x_i+y_i}                   & \forall \opair{x_r}{x_i},\opair{y_r}{y_i}\in\C \\ %& (\hie{complex addition})\\
   \opair{x_r}{x_i} &\hxsd{\cdot}& \opair{y_r}{y_i} &\eqd& \opair{x_r y_r-x_i y_i}{x_r y_i + x_i y_r} & \forall \opair{x_r}{x_i},\opair{y_r}{y_i}\in\C    %& (\hie{complex multiplication})
  \end{array}$
  }
\end{definition}


%--------------------------------------
\begin{definition}
\label{def:i}
\citetbl{
  \citerpg{landau1966}{133}{082182693X},
  %\citepg{rudinp}{13}{007054235X} \\
  \citorp{euler1777}{184},
  \citor{cardano1545},
  \citor{cardano1545e}
  %\citor{cardano1545mit} \\
  }
%--------------------------------------
Let $\C$ be the set of complex numbers.
\defboxt{\indxs{i}
  The \vald{imaginary number} $\hxsd{i}$ in $\C$ is defined as
   \\\indentx$\ds i \eqd \opair{0}{1} $
   }
\end{definition}

%--------------------------------------
\begin{proposition}
\citetbl{
  \citerpg{landau1966}{133}{082182693X}
  %\citerpg{rudinp}{14}{007054235X}
  }
%--------------------------------------
Let $\C$ be the set of complex numbers and $i$ the imaginary number.
\propbox{\begin{array}{rcl@{\qquad}C@{\qquad}D}
     i^2   &=& -1     &                     &
\end{array}}
\end{proposition}
\begin{proof}
\begin{align*}
  i^2
    &= \opair{0}{1}\opair{0}{1}
    && \text{by \prefp{def:i}}
  \\&= (0\cdot0 -1\cdot1, 0\cdot1+1\cdot0)
    && \text{by \prefp{thm:C+}}
  \\&= (-1, 0)
  \\&= -1
\end{align*}
\end{proof}


%--------------------------------------
\begin{theorem}
\citetbl{
  \citerpg{landau1966}{133}{082182693X}
  %\citerpg{rudinp}{14}{007054235X}
  }
%--------------------------------------
\thmbox{\begin{array}{rcl@{\qquad}C@{\qquad}D}
  \opair{a}{b} &=& a + ib & \forall \opair{a}{b}\in\C  & (\hie{rectangular coordinates})
\end{array}}
\end{theorem}
\begin{proof}
\begin{align*}
  \opair{a}{b}
    &= a(1,0) + b\opair{0}{1}
  \\&= a1 + bi
    && \text{by \prefp{def:i}}
  \\&= a + ib
\end{align*}
\end{proof}




%=======================================
\subsection{Field structure}
%=======================================



%--------------------------------------
\begin{theorem}[Additive abelian group]
\label{thm:C+}
\index{group!additive}
\index{complex numbers!group}
\citetbl{\citerppg{landau1966}{93}{94}{082182693X}}
%--------------------------------------
\thmboxt{%
  $\opair{\C}{+}$ is a \structe{commutative group} under addition; in particular
  \\\indentx$\begin{array}{FlCD}
     1.& (x+y)+z = x+(y+z)
       & \forall x,y,z\in\C
       & (\hie{associative})
       \\
     2.& x + \opair{0}{0} = \opair{0}{0} + x = x
       & \forall x\in\C
       & ($\opair{0}{0}$ is the \hie{additive identity element})
       \\
     3.& -\opair{a}{b} = \opair{-a}{-b}
       & \forall \opair{a}{b} \in\C
       & ($\opair{-a}{-b}$ is the \hie{additive inverse} of $\opair{a}{b}$)
       \\
     4.& x+y = y+x
       & \forall x,y\in\C
       & (\hie{commutative})
  \end{array}$
  }
\end{theorem}
\begin{proof}
\begin{align*}
  \intertext{1. Proof that $\opair{\C}{+}$ is \hie{associative}:}
    (x+y)+z
      &= \brs{\opair{x_r}{x_i}+\opair{y_r}{y_i}} + \opair{z_r}{z_i}
    \\&= \opair{x_r+y_r}{x_i+y_i} + \opair{z_r}{z_i}
      && \text{by definition of complex addition (\prefpo{def:C+})}
    \\&= \opair{[x_r+y_r]+z_r}{[x_i+y_i]+z_i}
      && \text{by definition of complex addition (\prefpo{def:C+})}
    \\&= \opair{x_r+[y_r+z_r]}{x_i+[y_i+z_i]}
    \\&= \opair{x_r}{x_i} + \opair{y_r+z_r}{y_i+z_i}
      && \text{by definition of complex addition (\prefpo{def:C+})}
    \\&= x + (y+z)
%
  \intertext{2. Proof that $\opair{0}{0}\in\C$ is the additive identity element:}
   \opair{a}{b} + \opair{0}{0}
     &= \opair{a+0}{b+0}
      && \text{by definition of complex addition (\prefpo{def:C+})}
   \\&= \opair{a}{b}
     \\
   \opair{0}{0} + \opair{a}{b}
     &= \opair{0+a}{0+b}
      && \text{by definition of complex addition (\prefpo{def:C+})}
   \\&= \opair{a}{b}
  %
  \intertext{3. Proof that $\opair{-a}{-b}\in\C$ is the inverse element of $\opair{a}{b}$:}
   \opair{a}{b}+\opair{-a}{-b}
     &= \opair{a-a}{b-b}
     && \text{by definition of complex addition (\prefpo{def:C+})}
   \\&= \opair{0}{0}
     \\
   \opair{-a}{-b}+\opair{a}{b}
     &= \opair{-a+a}{-b+b}
     && \text{by definition of complex addition (\prefpo{def:C+})}
   \\&= \opair{0}{0}
  %
  \intertext{4. Proof that $\opair{\C}{+}$ is \hie{commutative}:}
    x+y
      &= \opair{x_r}{x_i}+\opair{y_r}{y_i}
    \\&= \opair{x_r+y_r}{x_i+y_i}
      && \text{by definition of complex addition (\prefpo{def:C+})}
    \\&= \opair{y_r+x_r}{y_i+x_i}
    \\&= \opair{y_r}{y_i} + \opair{x_r}{x_i}
      && \text{by definition of complex addition (\prefpo{def:C+})}
    \\&= y+x
\end{align*}
\end{proof}


%--------------------------------------
\begin{theorem}[Multiplicative abelian group]
\label{thm:Cx}
\index{group!multiplicative}
\index{complex numbers!group}
\citetbl{
  \citerppg{landau1966}{96}{98}{082182693X}  % does not include inverse relationship (3) ?
  }
%--------------------------------------
\thmboxt{
  $\opair{\C}{\cdot}$ is a \hie{commutative group} under multiplication; in particular
  \\\indentx$\begin{array}{FlCD}
     1.& (xy)z = x(yz)
       & \forall x,y,z\in\C
       & (\hie{associative})
       \\
     2.& x \opair{1}{0} = \opair{1}{0} x = x
       & \forall x\in\C
       & ($\opair{1}{0}$ is the \hie{multiplicative identity element})
       \\
     3.& \opair{a}{b}^{-1} = \frac{1}{a^2+b^2}\opair{a}{-b}
       & \forall \opair{a}{b} \in\C
       & ($\frac{1}{a^2+b^2}\opair{a}{-b}$ is the \hie{multiplicative inverse} of $\opair{a}{b}$)
       \\
     4.& xy = yx
       & \forall x,y\in\C
       & (\hie{commutative})
  \end{array}$
  }
\end{theorem}
\begin{proof}
\begin{enumerate}
  \item Proof that $\opair{\C}{\times}$ is \hie{associative}:
    \begin{align*}
        (xy)z
          &= \brs{\opair{x_r}{x_i}\opair{y_r}{y_i}}\opair{z_r}{z_i}
        \\&= \mcom{\opair{x_ry_r-x_iy_i}{x_ry_i+x_iy_r}}{$xy$}
             \mcom{\opair{z_r}{z_i}}{$z$}
          %&& \text{by definition of complex multiplication (\prefpo{def:C+})}
          && \text{by \prefp{def:C+}}
        \\&= \opair{(x_ry_r-x_iy_i)z_r-(x_ry_i+x_iy_r)z_i}{(x_ry_r-x_iy_i)z_i+(x_ry_i+x_iy_r)z_r}
          %&& \text{by definition of complex multiplication (\prefpo{def:C+})}
          && \text{by \prefp{def:C+}}
        \\&= \opair{x_ry_rz_r-x_iy_iz_r-x_ry_iz_i-x_iy_rz_i}{x_ry_rz_i-x_iy_iz_i+x_ry_iz_r+x_iy_rz_r}
        \\&= \opair{x_r(y_rz_r-y_iz_i) - x_i(y_rz_i+y_iz_r)}
                   {x_r(y_rz_i+y_iz_r) + x_i(y_rz_r-y_iz_i)}
        \\&= \mcom{\opair{x_r}{x_i}}{$x$}
             \mcom{\opair{y_rz_r-y_iz_i}{y_rz_i+y_iz_r}}{$yz$}
        \\&= \opair{x_r}{x_i}\brs{\opair{y_r}{y_i}\opair{z_r}{z_i}}
          %&& \text{by definition of complex multiplication (\prefpo{def:C+})}
          && \text{by \prefp{def:C+}}
        \\&= x(yz)
      \end{align*}
%
  \item Proof that $\opair{1}{0}\in\C$ is the additive identity element:
    \begin{align*}
      \opair{a}{b}\opair{1}{0}
        &= \opair{a\cdot1-b\cdot0}{a\cdot0+b\cdot1}
         && \text{by definition of complex multiplication (\prefpo{def:Cx})}
      \\&= \opair{a}{b}
        \\
      \opair{1}{0}\opair{a}{b}
        &= \opair{1\cdot a-0\cdot b}{1\cdot b+0\cdot a}
         && \text{by definition of complex multiplication (\prefpo{def:Cx})}
      \\&= \opair{a}{b}
    \end{align*}
  %
  \item  Proof that $\frac{1}{a^2+b^2}\opair{a}{-b}\in\C$ is the inverse element of $\opair{a}{b}$:
    \begin{align*}
         \opair{a}{b}\brs{\frac{1}{a^2+b^2}\opair{a}{-b}}
        &= \frac{1}{a^2+b^2}\opair{aa+bb}{-ab+ba}
        && \text{by definition of complex multiplication (\prefpo{def:C+})}
      \\&= \frac{1}{a^2+b^2}\opair{a^2+b^2}{0}
      \\&= \opair{\frac{a^2+b^2}{a^2+b^2}}{\frac{0}{a^2+b^2}}
      \\&= \opair{1}{0}
        \\
      \brs{\frac{1}{a^2+b^2}\opair{a}{-b}}\opair{a}{b}
        &= \frac{1}{a^2+b^2}\opair{aa+bb}{ab-ba}
        && \text{by definition of complex multiplication (\prefpo{def:C+})}
      \\&= \frac{1}{a^2+b^2}\opair{a^2+b^2}{0}
      \\&= \opair{\frac{a^2+b^2}{a^2+b^2}}{\frac{0}{a^2+b^2}}
      \\&= \opair{1}{0}
    \end{align*}
  
  \item Proof that $\opair{\C}{\times}$ is \hie{commutative}:
    \begin{align*}
      xy
        &= \opair{x_r}{x_i}\opair{y_r}{y_i}
      \\&= \opair{x_ry_r-x_iy_i}{x_ry_i+x_iy_r}
        && \text{by definition of complex multiplication (\prefpo{def:C+})}
      \\&= \opair{y_rx_r-y_ix_i}{y_ix_r+y_rx_i}
      \\&= \opair{y_r}{y_i}\opair{x_r}{x_i}
      \\&= yx
    \end{align*}

  \item Proof that $(\C,\times)$ forms a multiplicative group:
    \begin{enumerate}
      \item Proof that $(1,0)$ is the multiplicative identity element:
        \begin{align*}
          \opair{a}{b}(1,0) &= (a1-b0,a0+b1) &&= \opair{a}{b} \\
          (1,0)\opair{a}{b} &= (1a-0b,0a+1b) &&= \opair{a}{b}
        \end{align*}
  
      \item Proof that $\frac{1}{a^2+b^2}\opair{a}{-b}$ is the multiplicative inverse element of $\opair{a}{b}$:
        \begin{align*}
          \opair{a}{b}\left[\frac{1}{a^2+b^2}\opair{a}{-b}\right] &= \frac{1}{a^2+b^2}(a^2+b^2,-ab+ba) &&= (1,0) \\
          \left[\frac{1}{a^2+b^2}\opair{a}{-b}\right]\opair{a}{b} &= \frac{1}{a^2+b^2}(a^2+b^2,ab-ba)  &&= (1,0)
        \end{align*}

      \item Proof that $(\C,\times)$ is closed:
        \begin{align*}
          \opair{a}{b}\opair{c}{d} &= (ac-bd, ad+bc)\in\C.
        \end{align*}
  \end{enumerate}
\end{enumerate}
\end{proof}



%--------------------------------------
\begin{proposition}[Distributive property]
\label{prop:C_distrib}
\index{distributive}
\index{complex!distributive}
\citetbl{
  \citerpg{landau1966}{99}{082182693X}
  }
%--------------------------------------
\propbox{\begin{array}{rclcl@{\qquad}C@{\qquad}D}
  z(x+y) &=& zx &+& zy  & \forall x,y,z\in\C  & (\hie{left distributive}) \\
  (x+y)z &=& xz &+& yz  & \forall x,y,z\in\C  & (\hie{right distributive})
\end{array}}
\end{proposition}
\begin{proof}
\begin{align*}
  z(x+y)
    &= \opair{z_r}{z_i}\brs{\opair{x_r}{x_i}+\opair{y_r}{y_i}}
  \\&= \opair{z_r}{z_i}\opair{x_r+y_r}{x_i+y_i}
    && \text{by def. of complex add. \xref{def:C+}}
  \\&= \opair{z_r[x_r+y_r] - z_i[x_i+y_i]}{z_i[x_r+y_r] + z_r[x_i+y_i]}
    && \text{by def. of complex mult. \xref{def:Cx}}
  \\&= \opair{z_rx_r - z_ix_i + z_ry_r - z_iy_i}{z_ix_r + z_rx_i + z_iy_r + z_ry_i}
  \\&= \opair{z_rx_r - z_ix_i}{z_ix_r + z_rx_i} + \opair{z_ry_r - z_iy_i}{z_iy_r + z_ry_i}
    && \text{by def. of complex add. \xref{def:C+}}
  \\&= \opair{z_r}{z_i}\opair{x_r}{x_i} + \opair{z_r}{z_i}\opair{y_r}{y_i}
    && \text{by def. of complex mult. \xref{def:Cx}}
  \\&= zx + zy
  \\
  (x+y)z
    &= z(x+y)
    && \text{by commutative prop. of \prefp{thm:Cx}}
  \\&= zx+zy
    && \text{by previous result}
  \\&= xz+yz
    && \text{by commutative prop. of \prefp{thm:Cx}}
\end{align*}
\end{proof}


%--------------------------------------
\begin{theorem}
\index{complex numbers!field}
\index{field}
%--------------------------------------
\thmbox{\begin{array}{l>{$}l<{$}}
  (\C,+,\times) & is a field.
  %(\C,+,\times) & is a vector space over the field $\R$. \\
  %(\C,+,\times) & is an algebra. \\
  \end{array}}
\end{theorem}
\begin{proof}
\begin{enumerate}
  \item By \prefpp{thm:C+}, $(\C,+,\times)$ is an additive group.
  \item By \prefpp{thm:Cx}, $(\C,+,\times)$ is a multiplicative group.
  \item By \prefpp{prop:C_distrib}, $(\C,+,\times)$ is distributive.
  \item These properties collectively imply that $(\C,+,\times)$ is a field.
\end{enumerate}
\end{proof}

%--------------------------------------
\begin{proposition}
\citetbl{
  \citerpg{apostol1975}{16}{9861541039}
  }
%--------------------------------------
\propbox{
  \opair{0}{0}\opair{a}{b} = \opair{a}{b}\opair{0}{0} = \opair{0}{0}
  \qquad\scriptstyle
  \forall \opair{a}{b}\in\C
  }
\end{proposition}
\begin{proof}
\begin{align*}
  \opair{0}{0}\opair{a}{b}
    &= \opair{0 \cdot a-0\cdot b}{0\cdot b + 0\cdot a}
    && \text{by \prefp{def:Cx}}
  \\&= \opair{0 }{0}
    && \text{by \prefp{def:Cx}}
  \\
  \opair{a}{b} \opair{0}{0}
    &= \opair{a \cdot 0-b\cdot 0}{b\cdot 0 + a\cdot 0}
    && \text{by \prefp{def:Cx}}
  \\&= \opair{0 }{0}
    && \text{by \prefp{def:Cx}}
\end{align*}
\end{proof}

%=======================================
\subsection{Normed algebra structure}
%=======================================
%--------------------------------------
\begin{definition}
\label{def:C_abs}
\citetbl{
  \citerpg{landau1966}{108}{082182693X}
  }
\indxs{\absn\in\clF{\C}{\R}}
%--------------------------------------
\defboxt{
  The \opd{absolute value} $\hxsd{\absn}\in\clF{\C}{\R}$ is defined as \\
  \indentx$\ds
  \abs{\opair{a}{b}} \eqd \sqrt{a^2+b^2} \quad\scriptstyle \forall \opair{a}{b}\in\C
  $
  }
\end{definition}


%--------------------------------------
\begin{theorem}[normed algebra properties]
\label{thm:C_norm}
\citetbl{
  \citerppg{landau1966}{108}{110}{082182693X},
  \cithrpg{rudinp}{14}{007054235X}
  }
%--------------------------------------
Let $\absn\in\clF{\C}{\Rnn}$ be the absolute value function.
\thmboxt{
  The pair $\opair{\C}{\absn}$ is a \hie{normed algebra}. In particular
  \\\indentx
  $\ds\begin{array}{Frcl@{\qquad}C@{\qquad}D}
      1. & \abs{x}   &\ge& 0                 & \forall x \in\C    & (\prop{non-negative})
    \\2. & \abs{x}   &=&   0 \iff x=0        & \forall x\in\C     & (\prop{nondegenerate})
    \\3. & \abs{xy}  &=&   \abs{x}\:\abs{y}  & \forall x,y\in\C   & (\prop{homogeneous} / \prop{multiplicative condition})
    \\4. & \abs{x+y} &\le& \abs{x} + \abs{y} & \forall x,y\in\C   & (\prop{subadditive} / \prop{triangle inequality})
  \end{array}$
  }
\end{theorem}
\begin{proof}
\begin{align*}
  \intertext{1. Proof that $\abs{x}\ge0$:}
    \abs{x}^2
      &= a^2 + b^2
      && \text{by definition of $\abs{x}$}
    \\&\ge 0
    \\
  \intertext{2. Proof that $\abs{x}=0\iff x=0$:}
    \abs{x}=0
      &\implies \abs{x}^2=0
    \\&\implies \abs{x} =0
    \\&\implies a^2+b^2 =0
    \\&\implies a=b=0
    \\&\implies \opair{a}{b}=0
    \\&\implies x=0
    \\
    x = 0
      &\implies \opair{a}{b} = 0
    \\&\implies a=b=0
    \\&\implies a^2+b^2=0
    \\&\implies \abs{x}^2=0
    \\&\implies \abs{x}=0
    \\
  \intertext{3. Proof that $\abs{xy}=\abs{x}\abs{y}$:}
    \abs{xy}^2
      &= \abs{\opair{a}{b}\opair{c}{d}}^2
      && \text{by definition of $\abs{x}$}
    \\&= \abs{(ac-bd,ad+bc)}^2
      && \text{by \prefp{def:C+}}
    \\&= (ac-bd)^2 + (ad+bc)^2
      && \text{by definition of $\abs{x}$}
    \\&= \mcom{(ac)^2 + (bd)^2 - 2(ac)(bd)}{$(ac-bd)^2$} +
         \mcom{(ad)^2 + (bc)^2 + 2(ad)(bc)}{$(ad+bc)^2$}
    \\&= (ac)^2 + (ad)^2 + (bc)^2 + (bd)^2
    \\&= (a^2+b^2)(c^2+d^2)
    \\&= \abs{x}^2 \abs{y}^2
      && \text{by definition of $\abs{x}$}
    \\
  \intertext{4. Proof that $\abs{x+y}\le \abs{x}+\abs{y}$:}
    \abs{x+y}^2
      &= \abs{\opair{a}{b}+\opair{c}{d}}^2
      && \text{by definition of $x,y$}
    \\&= \abs{(a+c,b+d)}^2
      && \text{by \prefp{def:C+}}
    \\&= (a+c)^2 + (b+d)^2
      && \text{by definition of $\abs{x}$}
    \\&= \mcom{a^2 + c^2 + 2ac}{$(a+c)^2$} + \mcom{b^2 + d^2 + 2bd}{$(b+d)^2$}
    \\&= \mcom{a^2 + b^2}{$\abs{x}^2$} + \mcom{c^2 + d^2}{$\abs{y}^2$} +
         \mcom{2ac + 2bd}{$2\Re(xy^\invo)$}
    \\&= \abs{x}^2 + \abs{y}^2 + 2\Re(xy^\invo)
    \\&\le \abs{x}^2 + \abs{y}^2 + 2\abs{xy^\invo}
      && \text{by \prefp{thm:C_abs}}
    \\&\le \abs{x}^2 + \abs{y}^2
      && \text{by $\abs{\cdot}\ge0$ property (1)}
\end{align*}
\end{proof}


%--------------------------------------
\begin{proposition}
\label{prop:C_norm_div}
\citetbl{
  \citerpg{apostol1975}{18}{9861541039}
  }
%--------------------------------------
Let $\absn\in\clF{\Rnn}{\C}$ be the absolute value function.
\propbox{
  \abs{\frac{x}{y}} = \frac{\abs{x}}{\abs{y}}
  \qquad
  \forall x,y\in\C
  }
\end{proposition}
\begin{proof}
\begin{align*}
  \frac{\abs{x}}{\abs{y}}
    &= \frac{\abs{y\,\frac{x}{y}}}{\abs{y}}
  \\&= \frac{\abs{y}\,\abs{\frac{x}{y}}}{\abs{y}}
    && \text{by homogeneous property of \prefp{thm:C_norm}}
  \\&= \abs{\frac{x}{y}}
\end{align*}
\end{proof}



%=======================================
\subsection{Star-algebra structure}
%=======================================

%%--------------------------------------
%\begin{lemma}
%\label{lem:C_basis}
%%--------------------------------------
%The elements $(1,0)$ and $\opair{0}{1}$ form a {\bf basis} for $\C$.
%\end{lemma}

%--------------------------------------
\begin{definition}
\label{def:conjugate}
\citetbl{
  \citerpg{landau1966}{106}{082182693X}
  }
\index[xsym]{$\invo$}
\index{complex numbers!conjugate}
%--------------------------------------
Let $\opair{a}{b}\in\C$.
The \opd{conjugate} operator $\hxsd{\invo}:\C\to\C$ is defined as
\defbox{
  \opair{a}{b}^\invo \eqd \opair{a}{-b} \qquad \forall \opair{a}{b}\in\C
  }
\end{definition}


%--------------------------------------
\begin{theorem}
\label{thm:conj_real}
\citetbl{
  \citerpg{berberian1961}{25}{0821819127}
  }
%--------------------------------------
\thmbox{
  x = x^\invo
  \qquad\iff\qquad
  x\in\R
  }
\end{theorem}
\begin{proof}
\begin{align*}
  \intertext{1. Proof that $x=x^\invo\implies x\in\R$}
    x = \opair{a}{b}
      &\implies x^\invo = \opair{a}{b}^\invo
      &&        \text{by definition of $x$}
    \\&\implies x^\invo = \opair{a}{-b}
      &&        \text{by \prefp{def:conjugate}}
    \\&\implies b = -b
      &&        \text{by left hypothesis}
    \\&\implies b = 0
      &&        \text{because $b\in\R$}
    \\&\implies x=(a,0)
    \\&\implies x\in\R
      &&        \text{by \prefp{def:C}}
    \\
  \intertext{2. Proof that $x=x^\invo\impliedby x\in\R$}
    x
      &= \opair{a}{b}
      && \text{by definition of $x$}
    \\&= (a,0)
      && \text{by right hypothesis}
    \\&= (a,-0)
    \\&= (a,0)^\invo
      && \text{by \prefp{def:conjugate}}
    \\&= \opair{a}{b}^\invo
      && \text{by right hypothesis}
    \\&= x^\invo
      && \text{by definition of $x$}
\end{align*}
\end{proof}


%---------------------------------------
\begin{theorem}[$\invo$-algebra properties]
\label{thm:conj}
\citetbl{
  \citerppg{landau1966}{106}{107}{082182693X}
  }
\index{involution}
\index{$\invo$-algebra}
\index{star algebra}
%---------------------------------------
Let $\invo:\C\to\C$ be the conjugate operator.
\thmboxt{
  The pair $\opair{\C}{\invo}$ is a \hie{$\invo$-algebra}\footnotemark. In particular,
  \\\indentx
  $\ds\begin{array}{rcl @{\qquad}C @{\qquad}D}
    x^{\invo\invo} &=& x                & \forall x\in\C    & (\prop{involutary})        \\
    (x+y)^\invo   &=& x^\invo + y^\invo  & \forall x,y\in\C  & (\prop{distributive})      \\
    (xy)^\invo    &=& y^\invo x^\invo    & \forall x,y\in\C  & (\prop{antiautomorphic})   \\
    %(xy)^\invo    &=& x^\invo y^\invo    & \forall x,y\in\C  & (\prop{conjugate linear}) \\
  \end{array}$
  }
\footnotetext{\hie{$\invo$-algebra}: \prefp{def:star_algebra}}
\end{theorem}
\begin{proof}
\begin{align*}
   x^{\invo\invo}
     &= (x^\invo)^\invo
   \\&= \brs{\opair{x_r}{x_i}^\invo}^\invo
    && \text{where $x\eqd\opair{x_r}{x_i}$}
   \\&= \opair{x_r}{-x_i}^\invo
    && \text{by definition of conjugate (\prefpo{def:conjugate})}
   \\&= \opair{x_r}{x_i}
    && \text{by definition of conjugate (\prefpo{def:conjugate})}
   \\&= x
    && \text{by the definition $x\eqd\opair{x_r}{x_i}$}
\\
\\
   (x+y)^\invo
     &= [\opair{x_r}{x_i} + \opair{y_r}{y_i}]^\invo
    && \text{where $x\eqd\opair{x_r}{x_i}$ and $y\eqd\opair{y_r}{y_i}$}
   \\&= (x_r+y_r,x_i+y_i)^\invo
     && \text{by definition of complex addition (\prefpo{def:C+})}
   \\&= (x_r+y_r,-x_i-y_i)
     && \text{by definition of complex conjugate (\prefpo{def:conjugate})}
   \\&= \opair{x_r}{-x_i} + \opair{y_r}{-y_i}
     && \text{by definition of complex addition (\prefpo{def:C+})}
   \\&= \opair{x_r}{x_i}^\invo + \opair{y_r}{y_i}^\invo
     && \text{by definition of complex conjugate (\prefpo{def:conjugate})}
   \\&= x^\invo + y^\invo
     && \text{by the definitions $x\eqd\opair{x_r}{x_i}$ and $y\eqd\opair{y_r}{y_i}$}
\\
\\
   (xy)^\invo
     &= [\opair{x_r}{x_i}\opair{y_r}{y_i}]^\invo
    && \text{where $x\eqd\opair{x_r}{x_i}$ and $y\eqd\opair{y_r}{y_i}$}
   \\&= (x_ry_r-x_iy_i, x_ry_i+x_iy_r)^\invo
     && \text{by definition of complex multiplication (\prefpo{def:Cx})}
   \\&= (x_ry_r-x_iy_i, -x_ry_i-x_iy_r)
     && \text{by definition of complex conjugate (\prefpo{def:conjugate})}
   \\&= \opair{x_r}{-x_i}\opair{y_r}{-y_i}
     && \text{by definition of complex multiplication (\prefpo{def:Cx})}
   \\&= \opair{y_r}{-y_i}\opair{x_r}{-x_i}
     && \text{by \prefp{thm:C+}}
   \\&= \opair{y_r}{y_i}^\invo \opair{x_r}{x_i}^\invo
     && \text{by definition of complex conjugate (\prefpo{def:conjugate})}
   \\&= y^\invo x^\invo.
     && \text{by the definitions $x\eqd\opair{x_r}{x_i}$ and $y\eqd\opair{y_r}{y_i}$}
\end{align*}
\end{proof}



%---------------------------------------
\begin{theorem}
\label{thm:C_inv_conj}
\citetbl{
  \citerpg{landau1966}{107}{082182693X}
  }
\index{conjugate}
\index{antilinear} \index{semilinear}
%---------------------------------------
Let $\invo:\C\to\C$ be the conjugate operator.
\thmbox{\begin{array}{rcl @{\qquad}C @{\qquad\qquad}D@{}r@{}>{\quad}D}
  (-x)^\invo         &=& -(x^\invo)            & \forall x\in\C    & (conjugate of additive inverse)    \\
  \brp{x^{-1}}^\invo &=& \brp{x^\invo}^{-1}    & \forall x,y\in\C  & (conjugate of multiplicative inverse)
\end{array}}
\end{theorem}
\begin{proof}
\begin{align*}
   (-x)^\invo
     &= \brs{-\opair{x_r}{x_i}}^\invo
    && \text{where $x\eqd\opair{x_r}{x_i}$}
   \\&= \opair{-x_r}{-x_i}^\invo
   \\&= \opair{-x_r}{x_i}
    && \text{by definition of conjugate (\prefpo{def:conjugate})}
   \\&= -\opair{x_r}{-x_i}
   \\&= -\brs{\opair{x_r}{x_i}^\invo}
    && \text{by definition of conjugate (\prefpo{def:conjugate})}
   \\&= -\brp{x^\invo}
    && \text{by the definition $x\eqd\opair{x_r}{x_i}$}
\\
\\
  \brp{x^{-1}}^\invo
    &= \brp{\opair{x_r}{x_i}^{-1}}^\invo
    && \text{where $x\eqd\opair{x_r}{x_i}$}
  \\&= \brp{\frac{1}{x_r^2+x_i^2}\opair{x_r}{-x_i}}^\invo
    && \text{by \prefp{thm:Cx}}
  \\&= \brp{\frac{1}{x_r^2+x_i^2}}^\invo \opair{x_r}{-x_i}^\invo
    && \text{by previous property}
  \\&= \frac{1}{x_r^2+x_i^2}\opair{x_r}{x_i}
    && \text{by definition of conjugate (\prefpo{def:conjugate})}
  \\&= \opair{x_r}{-x_i}^{-1}
    && \text{by \prefp{thm:Cx}}
  \\&= \brp{\opair{x_r}{x_i}^\invo}^{-1}
    && \text{by definition of conjugate (\prefpo{def:conjugate})}
  \\&= \brp{x^\invo}^{-1}
    && \text{by definition $x\eqd\opair{x_r}{x_i}$}
\end{align*}
\end{proof}




%---------------------------------------
\begin{corollary}
\label{cor:C_conj}
\citetbl{
  \citerpg{landau1966}{107}{082182693X}
  }
%---------------------------------------
Let $\invo:\C\to\C$ be the conjugate operator.
\corbox{\begin{array}{rcl @{\qquad}C}
  \brp{\frac{x}{y}}^\invo &=& \frac{x^\invo}{y^\invo} & \forall x,y\in\C
\end{array}}
\end{corollary}
\begin{proof}
\begin{align*}
  \brp{\frac{x}{y}}^\invo
    &= \brp{x\frac{1}{y}}^\invo
  \\&= \brs{x^\invo} \brs{\brp{\frac{1}{y}}^\invo}
    && \text{by \prefp{thm:conj}}
  \\&= \brs{x^\invo} \brs{\brp{y^{-1}}^\invo}
  \\&= \brs{x^\invo} \brs{\brp{y^\invo}^{-1}}
    && \text{by \prefp{thm:conj}}
  \\&= \frac{x^\invo}{y^\invo}
\end{align*}
\end{proof}



%=======================================
%\subsection{$C^\invo$ algebraic structure}
\subsection{C-star algebraic structure}
%=======================================
%--------------------------------------
\begin{theorem}
\label{thm:C_abs}
\citetbl{
  \citerpg{rudinp}{14}{007054235X}
  }
%--------------------------------------
Let $\abs{\cdot}:\C\to\Rnn$ be the absolute value function on $\C$.
\thmbox{\begin{array}{Frcl@{\qquad}C}
    1. & \abs{x^\invo}&=&   \abs{x}    & \forall x\in\C
  \\2. & \abs{x}^2   &=&   xx^\invo    & \forall x\in\C
\end{array}}
\end{theorem}
\begin{proof}
\begin{align*}
  \abs{x^\invo}
    &= \abs{\opair{a}{b}^\invo}
    && \text{by definition of $x$}
  \\&= \abs{\opair{a}{-b}}
    && \text{by \prefp{def:conjugate}}
  \\&= \sqrt{a^2+(-b)^2}
    && \text{by definition of $\abs{x}$}
  \\&= \sqrt{a^2+b^2}
  \\&= \abs{\opair{a}{b}}
    && \text{by definition of $\abs{x}$}
  \\&= \abs{x}
    && \text{by definition of $x$}
  \\
  \\
  \abs{x}^2
    &= \abs{\opair{a}{b}}^2
    && \text{by definition of $x$}
  \\&= a^2 + b^2
    && \text{by \prefp{def:C_abs}}
  \\&= (a^2 + b^2,0)
  \\&= (a^2+b^2,ba-ab)
  \\&= \opair{a}{b}\opair{a}{-b}
  \\&= \opair{a}{b}\opair{a}{b}^\invo
    && \text{by \prefp{def:C_abs}}
  \\&= xx^\invo
    && \text{by definition of $x$}
\end{align*}
\end{proof}




%=======================================
\subsection{Hermitian structure}
%=======================================
The pair $\opair{\C}{\ast}$ is a special case of a \hie{star-algebra} (\hie{$\ast$-algebra}).
In a star-algebra, the real and imaginary components of an element $x$
are defined as follows:%
\ifdochas{normalg}{\footnote{$\Re$ and $\Im$: \prefpp{def:nalg_Re}}}
%--------------------------------------
%\begin{definition}[Hermitian components]
%\label{def:C_Re}
%\label{def:C_Im}
%\index{Hermitian components}
%--------------------------------------
  \[
    \Re {x} \eqd  \frac{1}{2 } \Big(x + x^\invo \Big)
    \qquad\qquad
    \Im {x} \eqd  \frac{1}{2i} \Big(x - x^\invo \Big).
  \]

%--------------------------------------
\begin{theorem}
\label{thm:Rex}
\citetbl{
  \citerp{munkres2000}{87}
  }
%--------------------------------------
Let $x\eqd\opair{a}{b}\in\C$.
Then
\thmbox{\begin{array}{rcl@{\qquad}C}
  \Re \opair{a}{b} &=& a & \forall \opair{a}{b}\in\C  \\
  \Im \opair{a}{b} &=& b & \forall \opair{a}{b}\in\C
\end{array}}
\end{theorem}
\begin{proof}
\begin{align*}
  \Re\opair{a}{b}
    &= \frac{1}{2 } \brp{\opair{a}{b} + \opair{a}{b}^\invo}
    && \text{by \prefp{def:nalg_Re}}
  \\&= \frac{1}{2 } \brp{\opair{a}{b} + \opair{a}{-b}}
    && \text{by \prefp{def:conjugate}}
  \\&= \frac{1}{2 } \opair{a+a}{b-b}
    && \text{by \prefp{def:C+}}
  \\&= \frac{1}{2 } \opair{2a}{0}
  \\&= a
  \\
  \\
  \Im\opair{a}{b}
    &= \frac{1}{2 } \brp{\opair{a}{b} - \opair{a}{b}^\invo}
    && \ifdochas{normalg}{\text{by \prefp{def:nalg_Im}}}
  \\&= \frac{1}{2 } \brp{\opair{a}{b} - \opair{a}{-b}}
    && \text{by \prefp{def:conjugate}}
  \\&= \frac{1}{2 } \opair{a-a}{b+b}
    && \text{by \prefp{def:C+}}
  \\&= \frac{1}{2 } \opair{0}{2b}
  \\&= b
\end{align*}
\end{proof}


%--------------------------------------
\begin{theorem}
%\label{thm:C_abs}
\citepg{rudinp}{14}{007054235X}
%--------------------------------------
Let $\abs{\cdot}:\C\to\Rnn$ be the absolute value function on $\C$.
\thmbox{\begin{array}{Frcl@{\qquad}C}
    1. & \abs{\Re x} &\le& \abs{x}    & \forall x\in\C
  \\2. & \abs{\Im x} &\le& \abs{x}    & \forall x\in\C
\end{array}}
\end{theorem}
\begin{proof}
\begin{align*}
  \abs{\Re(x)}
    &= \abs{\Re\opair{a}{b}}
    && \text{by definition of $x$}
  \\&= \abs{a}
    && \text{by \prefp{thm:Rex}}
  \\&= \sqrt{a^2}
  \\&\le \sqrt{a^2+b^2}
  \\&= \abs{\opair{a}{b}}
    && \text{by definition of $\abs{x}$}
  \\&= \abs{x}
    && \text{by definition of $x$}
  \\
  \\
  \abs{\Im(x)}
    &= \abs{\Im\opair{a}{b}}
    && \text{by definition of $x$}
  \\&= \abs{b}
    && \text{by \prefp{thm:Rex}}
  \\&= \sqrt{b^2}
  \\&\le \sqrt{a^2+b^2}
  \\&= \abs{\opair{a}{b}}
    && \text{by definition of $\abs{x}$}
  \\&= \abs{x}
    && \text{by definition of $x$}
\end{align*}
\end{proof}


%=======================================
\section{Literature}
%=======================================
\begin{survey}
\begin{enumerate}
  \item Classic/standard texts: \label{lit:numsys_classic}
    \\\citor{dedekind1872}
    \\\citor{dedekind1888}
    \\\citer{landau1966}
    \\\citer{thurston1956}
    \\\citer{birkhoff1996}

  \item English translations and reprints of \pref{lit:numsys_classic}:
    \\\citor{dedekind1872e}
    \\\citor{dedekind1872ed}
    \\\citor{dedekind1872eo}
    \\\citor{dedekind1888e}
    \\\citor{dedekind1888ed}
    \\\citer{thurston2007}
\end{enumerate}
\end{survey}