%============================================================================
% Daniel J. Greenhoe
% LaTeX file
%============================================================================


%=======================================
\chapter{Integration}
%=======================================

%=======================================
%\section{Definitions}
%=======================================

\footnote{The name \hie{integral calculus} and its operational symbol $\int$ were the 
  product of a collaboration between 
  \href{http://en.wikipedia.org/wiki/Gottfried_Leibniz}{Gottfried Leibnitz} (1646--1716) 
  and \href{http://en.wikipedia.org/wiki/Johann_Bernoulli}{Johann Bernoulli} (1667--1748).
  Leibnitz preferred the terminology \hie{calculus summatorius} (summation calculus)
  and the operational symbol $\int$ (an enlongated ``S").
  Bernoulli preferred the terminology \hie{calculus integralis} (integral calculs)
  and the operational symbol $I$.
  In the end, a compromise was reached which is the currently used terminology 
  ``integral calculs" with symbol $\int$.
  Reference: \citerpp{cajori2}{181}{182}
  }
%---------------------------------------
\begin{definition}
\label{def:intR}
\citetbl{
  \citerp{jahnke2003}{262},
  \citor{cauchy1823}
  }
\index{integral operators!Cauchy}
%---------------------------------------
Let $\ms$ be a measure space.
Let $\seq{\ssetP_n}{n\in\Z}$ be sequence of increasingly fine partitions 
of a set $\setE\subseteq\setX$.
\defboxp{\indxs{\intC}
  The \hid{Cauchy integral operator} $\intC$ of a function $\ff$ over a set $\setE\subseteq\setX$
  on the measure space $\ms$ is
  \[ \intC_\setE \ff(x) \dmu \eqd \sum_n \ff(x_n) \mu(\setE_n)  \]
  where 
  $x_n\in\setE_n$ and 
  $\ds\setE_n \in \lim_{m\to\infty}\ssetP_m \eqd \set{\setE_n}{n\in\Z}$
  }
\end{definition}

%---------------------------------------
\begin{definition}
\label{def:intR}
\citetbl{
  \citerp{jahnke2003}{264},
  \citor{riemann1854}
  }
\index{integral operators!Riemann}
%---------------------------------------
Let $\ms$ be a measure space.
Let $\seq{\ssetP_n}{n\in\Z}$ be sequence of increasingly fine partitions 
of a set $\setE\subseteq\setX$.
\defboxp{\indxs{\intR}
  \[\begin{array}{>{\ds}rc>{\ds}l@{\qquad}D}
    \intR_E      \ff \dmu  &=& \inf_{x_i \in\setP_i} \setn{\sum_n \ff(x_n) \mu(\setE_n) } 
      & (lower integral)
      \\
    \intR_E^\ast \ff \dmu  &=& \sup_{x_i \in\setP_i} \setn{\sum_n \ff(x_n) \mu(\setE_n) }
      & (upper integral)
  \end{array}\]
  where 
  $x_n\in\setE_n$ and 
  $\ds\setE_n \in \lim_{m\to\infty}\ssetP_m \eqd \set{\setE_n}{n\in\Z}$.
  The sum $\intR_\setE\ff \dmu$ is \hid{Riemann integrable} if
  $\intR_\setE\ff\dmu = \intR_\setE^\ast \ff\dmu$ and in this case 
  the \hid{Riemann integral operator} $\intR$ of $\ff$ over $\setE\subseteq\setX$
  on $\ms$ is $\intR_E\ff\dmu$.
  }
\end{definition}

%---------------------------------------
\begin{definition}
\label{def:intL}
\citetbl{
  \citer{lebesgue1902},
  \citer{lebesgue1902b} 
  }
%---------------------------------------
Let $\ms$ be a measure space.
Let $\seq{\ssetP_n}{n\in\Z}$ be sequence of increasingly fine partitions 
of a set $\setE\subseteq\setX$.
\defboxp{\indxs{\intL}
  The \hid{Lebesgue integral operator} $\intL$ of $\ff$ over $\setE\subseteq\setX$
  on $\ms$ is 
  \[
    \intL_\setE  \ff \dmu  \eqd \sum_{y\in\setY} y \mu\brp{\ff^{-1}(y)}
  \]
  }
\end{definition}


\begin{figure}[ht]%\color{figcolor}
\begin{center}
\begin{fsL}
\setlength{\unitlength}{\tw/1100}
\begin{picture}(1000,430)(0,-30)
  %{\color{graphpaper}\graphpaper[20](0,0)(1000,400)}
  \thicklines
  \put(   0 ,   0 ){\line(1,0){1000} }%
  \put(   0 ,   0 ){\line(0,1){400} }%
  \thicklines
  \put(   0 , -10 ){\makebox(0,0)[t]{$0$}}%
  \put( 100 , -10 ){\makebox(0,0)[t]{$1$}}%
  \put( 200 , -10 ){\makebox(0,0)[t]{$2$}}%
  \put( 300 , -10 ){\makebox(0,0)[t]{$3$}}%
  \put( 400 , -10 ){\makebox(0,0)[t]{$4$}}%
  \put( 500 , -10 ){\makebox(0,0)[t]{$5$}}%
  \put( 600 , -10 ){\makebox(0,0)[t]{$6$}}%
  \put( 700 , -10 ){\makebox(0,0)[t]{$7$}}%
  \put( 800 , -10 ){\makebox(0,0)[t]{$8$}}%
  \put( 900 , -10 ){\makebox(0,0)[t]{$9$}}%
  \put(1000 , -10 ){\makebox(0,0)[t]{$10$}}%
  {\color{blue}%
    \qbezier(0,50)(300,600)(600,200)%
    \qbezier(600,200)(800,-67)(1000,400)%
    }%
  {\color{red}%
    \put( 150, 30){\makebox(0,0)[c]{$\leftarrow\Delta x\rightarrow$}}%
    \put(   0,  0){\line(0,1){ 50}} \put(   0, 50){\line(1,0){100}} \put(   0, 50){\circle*{20}}%
    \put( 100,  0){\line(0,1){200}} \put( 100,200){\line(1,0){100}} \put( 100,200){\circle*{20}}%
    \put( 200,  0){\line(0,1){310}} \put( 200,310){\line(1,0){100}} \put( 200,310){\circle*{20}}%
    \put( 300,  0){\line(0,1){360}} \put( 300,360){\line(1,0){100}} \put( 300,360){\circle*{20}}%
    \put( 400,  0){\line(0,1){360}} \put( 400,360){\line(1,0){100}} \put( 400,360){\circle*{20}}%
    \put( 500,  0){\line(0,1){360}} \put( 500,310){\line(1,0){100}} \put( 500,310){\circle*{20}}%
    \put( 600,  0){\line(0,1){310}} \put( 600,200){\line(1,0){100}} \put( 600,200){\circle*{20}}%
    \put( 700,  0){\line(0,1){200}} \put( 700,110){\line(1,0){100}} \put( 700,110){\circle*{20}}%
    \put( 800,  0){\line(0,1){110}} \put( 800,110){\line(1,0){100}} \put( 800,110){\circle*{20}}%
    \put( 900,  0){\line(0,1){210}} \put( 900,210){\line(1,0){100}} \put( 900,210){\circle*{20}}%
    \put(1000,  0){\line(0,1){400}} \put(1000,400){\circle*{20}}%
    %
    \put(  50 , 60 ){\makebox(0,0)[b]{$\frac{1}{2}$}}
    \put( 150 ,210 ){\makebox(0,0)[b]{$2$}}
    \put( 250 ,320 ){\makebox(0,0)[b]{$3$}}
    \put( 350 ,380 ){\makebox(0,0)[b]{$3\frac{1}{2}$}}
    \put( 450 ,380 ){\makebox(0,0)[b]{$3\frac{1}{2}$}}
    \put( 550 ,320 ){\makebox(0,0)[b]{$3$}}
    \put( 650 ,210 ){\makebox(0,0)[b]{$2$}}
    \put( 750 ,120 ){\makebox(0,0)[b]{$1$}}
    \put( 850 ,120 ){\makebox(0,0)[b]{$1$}}
    \put( 950 ,220 ){\makebox(0,0)[b]{$2$}}
    }%
\end{picture}
\end{fsL}
\end{center}
\caption{
   Curve for example \prefpp{ex:int_curve}
   \label{fig:int_curve}
   }
\end{figure}

%---------------------------------------
\begin{example}
\label{ex:int_curve}
%---------------------------------------
Suppose we want to compute the area under the curve in \prefpp{fig:int_curve}
from $x=0$ to $x=10$.
This can be accomplished using either Riemann or Lebesgue integration. 
Riemann integration adds up the areas one by one, starting from $x=0$ and ending 
with $x=10$ such that
\begin{align*}
  \intr_0^{10} 
    &= \sum_{n=0}^{n=9} x_n \mcom{\mu\set{x\in\setE}{x_n\le x<x_{n+1}}}{$\Delta x=1$}
  \\&= \sum_{n=0}^{n=9} x_n \cdot 1
  \\&= \frac{1}{2} + 2 + 3 + 3\frac{1}{2} + 3\frac{1}{2} + 3 + 2 + 1 + 1 + 2
  \\&= 21\frac{1}{2}
\end{align*}
On the other hand, Lebesgue integration first groups together all equal values
into their own set and then sums the value of each set times the size of the set such that
\begin{align*}
  \int_0^{10} 
    &= \sum_{k=1}^{n=5} y_k \mu\set{x\in\setE}{\ff(x)=y_k}
  \\&= \mcom{\frac{1}{2}}{$y_1$}  \times 1 + 
       \mcom{1}{$y_2$}            \times 2 +   
       \mcom{2}{$y_3$}            \times 3 + 
       \mcom{3}{$y_4$}            \times 2 + 
       \mcom{3\frac{1}{2}}{$y_5$} \times 2
  \\&= 21\frac{1}{2}
\end{align*}
Of course in this case and in the case of all other ``well behaved" functions, 
the two approaches yield the same result.
\end{example}


\begin{figure}
\centering
\begin{tabular}{*{11}{c}}
  $\scriptstyle \seto{ 2}=1$ &
  $\scriptstyle \seto{ 3}=2$ &
  $\scriptstyle \seto{ 4}=3$ &
  $\scriptstyle \seto{ 5}=4$ &
  $\scriptstyle \seto{ 6}=5$ &
  $\scriptstyle \seto{ 7}=6$ &
  $\scriptstyle \seto{ 8}=5$ &
  $\scriptstyle \seto{ 8}=4$ &
  $\scriptstyle \seto{10}=3$ &
  $\scriptstyle \seto{11}=2$ &
  $\scriptstyle \seto{12}=1$  
  \\                  &                  &                  &                  &                  & \diceF\diceA &                  &                  &                  &                  &                  
  \\                  &                  &                  &                  & \diceE\diceA & \diceE\diceB & \diceF\diceB &                  &                  &                  &                  
  \\                  &                  &                  & \diceD\diceA & \diceD\diceB & \diceD\diceC & \diceE\diceC & \diceF\diceC &                  &                  &                  
  \\                  &                  & \diceC\diceA & \diceC\diceB & \diceC\diceC & \diceC\diceD & \diceD\diceD & \diceE\diceD & \diceF\diceD &                  &                  
  \\                  & \diceB\diceA & \diceB\diceB & \diceB\diceC & \diceB\diceD & \diceB\diceE & \diceC\diceE & \diceD\diceE & \diceE\diceE & \diceF\diceE &                  
  \\ \diceA\diceA & \diceA\diceB & \diceA\diceC & \diceA\diceD & \diceA\diceE & \diceA\diceF & \diceB\diceF & \diceC\diceF & \diceD\diceF & \diceE\diceF & \diceF\diceF
\end{tabular}
  \caption{
    Pair of dice distribution for \prefpp{ex:int_two_dice}
    \label{fig:int_two_dice}
    }
\end{figure}


%---------------------------------------
\begin{example}
\label{ex:int_two_dice}
\hspace{1pt}\\
%---------------------------------------
\begin{minipage}{6\tw/16}%
  \begin{tabular}{c|*{6}{c}|c}
             & \diceA & \diceB & \diceC & \diceD & \diceE & \diceF \\
    \hline     
    \diceA &    2     &     3    &    4     &     5    &     6    &     7    \\
    \diceB &    3     &     4    &    5     &     6    &     7    &     8    \\
    \diceC &    4     &     5    &    6     &     7    &     8    &     9    \\
    \diceD &    5     &     6    &    7     &     8    &     9    &    10    \\
    \diceE &    6     &     7    &    8     &     9    &    10    &    11    \\
    \diceF &    7     &     8    &    9     &    10    &    11    &    12    \\
  \end{tabular}%
\end{minipage}%
\hfill
\begin{minipage}{9\tw/16}%
Suppose we want to find the sum of all possible outcomes of the sum of a pair of dice.
All the possible outcomes are summarized in the table at the left.
Riemann integration would start in the upper left hand corner (\diceA\diceA)
and sum across each row such that:
\begin{align*}
  \intr_{\setE} \ff(x) \dx
    &= \sum_{n=1}^{36} \ff(x_n) \cdot 1
  \\&= \text{\diceA\diceA} + \text{\diceA\diceB} + 
       \text{\diceA\diceC} + \cdots +
       \text{\diceF\diceE} + \text{\diceF\diceF}
  \\&= \mcom{2 + 3 + 4 + \cdots + 11 + 12}{36 terms}
  \\&= 250
\end{align*}
\end{minipage}\\
Lebesgue integration, on the other hand, groups like values into sets and 
thus actually adds diagonally---because like values occur along diagonal lines.
This organization of like values is illustrated in \prefpp{fig:int_two_dice}
and calculated below:
\begin{align*}
  \int_{\setE} \ff \dd{\mu}
    &= \sum_{k=2}^{k=12} k \mu\set{\text{sum of dice pair}}{\text{sum }=k}
  \\&= \mcom{2\times1 + 3\times2 +  4\times3 +  5\times4 + 6\times5 + 7\times6 + 
       8\times5 + 9\times4 + 10\times3 + 11\times2 + 12\times1}{11 terms}
  \\&= 250 
\end{align*}
\end{example}

%---------------------------------------
\begin{example}[Salt and pepper function/Dirichlet monster]
\citetbl{
  \citerp{jahnke2003}{263},
  \citor{dirichlet1829},
  \citor{dirichlet1829b} 
  }
\index{salt and pepper function}
\index{Dirichlet monster}
\index{functions!salt and pepper}
\index{functions!Dirichlet monster}
%---------------------------------------
\exbox{
  \ff(x) \eqd \left\{\begin{array}{ll}
    0 & \text{for $x$ rational} \\
    1 & \text{for $x$ irrational}
  \end{array}\right.
  \qquad\implies\qquad
  \text{$\ff$ is not Riemann integrable}
  }
\end{example}

%---------------------------------------
\begin{definition}
\label{def:spLLR}
%---------------------------------------
Let $\int_\setA \ff \dmu$ be the \structe{Lebesgue integral} on a \structe{measure space} $\ms$.
\defbox{\begin{array}{M}
  The space $\spLLR$ is defined as
  \\\indentx$\ds\set{\ff\in\clFrr}{\int_\R \abs{\ff(x)}^2 \dmu < \infty}$
\end{array}}
\end{definition}

%=======================================
%\section{Fredholm integral operators}
%=======================================
%---------------------------------------
\begin{definition}
\label{def:op_int}
\citetbl{
  \citerp{michel1993}{425}
  }
\index{Fredholm integral equation of the first kind|textbf}
%---------------------------------------
\defbox{\begin{array}{l@{\qquad}l}
  \mc{2}{l}{\text{A \hid{Fredholm operator} $\opK$ is defined as}}
  \\&
  \brs{\opK\ff}(t) 
  \eqd 
  \mcom{\int_a^b \mcom{\kappa(t,s)}{kernel} \ff(s) \dd{s} }
       {Fredholm integral equation of the first kind \footnotemark}
  \qquad \forall \ff\in \spL_2([a,b])
  \end{array}}
\footnotetext{
   The equation $\int_u \kappa(t,s) \ff(s) \dd{s}$ is a 
   \hid{Fredholm integral equation of the first kind} and
   $\kappa(t,u)$ is the {\bf kernel} of the equation.
   References: \\
   \begin{tabular}{l}
     \citer{fredholm1900} \\
     \citerp{fredholm1903}{365} \\
     \citerp{michel1993}{97} \\
     \citerp{keener}{101}
   \end{tabular}
   }
\end{definition}

%---------------------------------------
\begin{example}
%---------------------------------------
Examples of \prop{Fredholm operators} include
\[
\begin{array}{llrclrcl}
  1. & \text{Fourier Transform}
     & [\opFT \fx](f) &=& \int_t \fx(t) e^{-i2\pi ft}\dt 
     & \kappa(t,f)    &=& e^{-i2\pi ft}
\\
  2. & \text{Inverse Fourier Transform}
     & [\opFTi \Fx](t) &=& \int_f \Fx(f) e^{i2\pi ft}\df 
     & \kappa(f,t)    &=& e^{i2\pi ft}
\\
  3. & \text{Laplace operator}
     & [\opL \fx](s) &=& \int_t \fx(t) e^{-st}\dt 
     & \kappa(t,s)   &=& e^{-st}
\\
  4. & \text{autocorrelation operator}
     & [\opR \fx](t) &=& \int_s R(t,s) \fx(s) \dds 
     & \kappa(t,s)   &=& R(t,s) %\text{ (autocorrelation function)}
\end{array}
\]
\end{example}


%---------------------------------------
\begin{theorem}
\label{thm:intop_KkKk}
%---------------------------------------
Let $\opK$ be a \prop{Fredholm operator} with kernel $\kappa(t,s)$ and adjoint $\opKa$.
\formbox{
  \brs{\opK\ff}(t) = \int_\setA \kappa(t,s) \ff(s) \dds
  \qquad\iff\qquad
  \brs{\opKa\ff}(t) = \int_\setA \kappa^\ast(s,t) \ff(s) \dds
  }
\end{theorem}
\begin{proof}
\begin{align*}
   \brs{\opK\ff}(t) 
     &= \int_\setA \kappa(t,s) \ff(s) \dds
   \\\iff
   \inprod{\brs{\opK \ff}(t)}{\fg(t)}
     &= \inprod{\int_s \kappa(t,s) \ff(s) \dds}{\fg(t)}
     && \text{by left hypothesis}
   \\&= \int_s \ff(s) \inprod{\kappa(t,s)  }{\fg(t)} \dds
     && \text{by additivity property of $\inprodn$\ifsxref{vsinprod}{def:inprod}}
   \\&= \int_s \ff(s) \inprod{\fg(t)}{\kappa(t,s)}^\ast \dds
     && \text{by conjugate symmetry property of $\inprodn$\ifsxref{vsinprod}{def:inprod}}
   \\&= \inprod{ \ff(s) }{\inprod{\fg(t)}{\kappa(t,s)} }
     && \text{by local definition of $\inprodn$}
   \\&= \inprod{ \ff(s) }{\mcom{\int_t \kappa^\ast(t,s) \fg(t) \dt}{$\brs{\opKa\fg}(s)$} }
     && \text{by local definition of $\inprodn$}
   \\\iff
   \brs{\opKa\fg}(s) 
     &= \int_\setA \kappa^\ast(t,s) \fg(t) \dt
     && \text{by right hypothesis}
   \\\iff
   \brs{\opKa\fg}(\sigma) 
     &= \int_\setA \kappa^\ast(\tau,\sigma) \fg(\tau) \dtau
     && \text{by change of variable: $\tau=t,\, \sigma=s$}
   \\\iff
   \brs{\opKa\ff}(t) 
     &= \int_\setA \kappa^\ast(s,t) \ff(s) \dds
     && \text{by change of variable: $t=\sigma,\, s=\tau,\, \ff=\fg$}
\end{align*}
\end{proof}

%---------------------------------------
\begin{theorem}
\label{thm:intop_KKkk}
\citetbl{
  \citerp{michel1993}{430}
  }
%---------------------------------------
Let $\opK$ be an \prop{Fredholm operator} with kernel $\kappa(t,s)$ and adjoint $\opKa$.
\formbox{
  \mcom{\opK=\opKa}{$\opK$ is self-adjoint}
  \qquad\iff\qquad
  \mcom{\kappa(t,s) = \kappa^\ast(s,t)}{kernel is \prop{conjugate symmetric}}
  }
\end{theorem}
\begin{proof}
\begin{align*}
  \opK = \opKa
    &\iff \int_\setA \kappa(t,s) \ff(s) \dds = \int_\setA \kappa^\ast(s,t) \ff(s) \dds
    &&    \text{by \prefp{thm:intop_KkKk}}
  \\&\iff \kappa(t,s) =  \kappa^\ast(s,t) 
\end{align*}
\end{proof}



%---------------------------------------
\begin{theorem}[Mercer's Theorem]
\label{thm:mercer}
\citetbl{
  \citerp{gohberg}{198},
  \citerpp{courant}{138}{140},
  \citerp{mercer1909}{439}
  }
\index{Mercer's Theorem}
\index{theorems!Mercer's Theorem}
%---------------------------------------
Let $\opK$ be an \prop{Fredholm operator} with kernel $\kappa(t,s)$ and
eigensystem $\seq{(\lambda_n,\phi_n(t))}{n\in\Z}$.
\formbox{
  \left.\begin{array}{ll}
    1.& \mcom{\int_a^b \int_a^b \kappa(t,s) \ff(t) \ff^\ast(s) \ds\dt \ge 0}{positive} \quad \text{and }\\
    2.& \kappa(t,s) \text{ is continuous on } [a,b]\times[a,b]
  \end{array}\right\}
  \implies
  \left\{\begin{array}{ll}
    1.& \kappa(t,s) = \sum_n \lambda_n \fphi_n(t) \fphi_n^\ast(s) \quad\text{and}\\
    2.& \kappa(t,s) \text{ converges absolutely and} \\
      & \text{uniformly on $[a,b]\times[a,b]$}
  \end{array}\right.
  }
\end{theorem}
\begin{proof}

\end{proof}