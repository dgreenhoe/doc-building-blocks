%============================================================================
% LaTeX File
% Daniel Greenhoe
%============================================================================
%======================================
\chapter{Narrowband Signals}
%======================================
\begin{figure}[ht] \color{figcolor}
\begin{center}
\begin{fsL}
\setlength{\unitlength}{0.1mm}
\begin{picture}(700,200)
  %\graphpaper[10](0,0)(600,200)
  \thicklines
  \put(   0 , 100 ){\line(1,0){600} }
  \put( 300 ,   0 ){\line(0,1){200} }
  \put(   0 ,   0 ){\makebox( 200, 90)[t]{$-f_c$} }
  \put( 400 ,   0 ){\makebox( 200, 90)[t]{$+f_c$} }
  \put( 610 ,   0 ){\makebox( 100,200)[l]{$f$} }
  \put( 310 , 100 ){\makebox( 100,100)[tl]{$|\Fx(f)|$} }
  \thinlines
  \put( 100 ,  90 ){\line( 0,1){20} }
  \put( 500 ,  90 ){\line( 0,1){20} }
  \put( 470 , 100 ){\line( 1,2){30} }
  \put( 530 , 100 ){\line(-1,2){30} }
  \put(  70 , 100 ){\line( 1,2){30} }
  \put( 130 , 100 ){\line(-1,2){30} }
\end{picture}
\end{fsL}
\end{center}
\caption{
   Narrowband signal
   \label{fig:narrowband}
   }
\end{figure}

Communication systems are often assumed to be \prope{narrowband} (next definition) meaning
the bandwidth of the information carrying signal is ``small" compared
to the carrier frequency \xref{fig:narrowband}.

%---------------------------------------
\begin{definition}
\label{def:narrowband}
\index{narrowband}
%---------------------------------------
Let $\fx:\R\to\R$ be an information carrying waveform,
$\Fx(f)=[\opFT\fx](f)$ and $\f_c\in\R$.
\defboxt{
  The signal $\fx(t)$ is \propd{narrowband} if
  \\\indentx$\begin{array}{FMD}
     (1).& The energy of $\Fx(f)$ is located in the vicinity of frequency $\pm f_c$ & and\\
     (2).& the bandwidth of $\Fx(f)$ is ``small" compared to $f_c$.
  \end{array}$\\
  If $\fx(t)$ is the transmitted signal in a communication system
  $\opSys=\opR\opC\opT$ such that $\fx(t) = \opT\vu$,\\ and $\fx(t)$ is \prope{narrowband}, then
  $\opSys$ is a \structd{narrowband system}.
  }
\end{definition}

%=======================================
\section{Time representation}
\index{narrowband!time representation}
%=======================================
\begin{minipage}{\tw/3}%
  \color{figcolor}
  \begin{center}
  \begin{fsL}
  \setlength{\unitlength}{0.20mm}
  \begin{picture}(250,240)(-100,-100)
    %\graphpaper[10](-100,-100)(200,200)
    \thinlines
    \put(-100,   0){\line(1,0){200} }
    \put(   0,-100){\line(0,1){200} }
    \put( 105,   0){\makebox(0,0)[l]{$\Re$}}
    \put(   0, 105){\makebox(0,0)[b]{$\Im$}}
    \put(   0,   0){\vector(1,1){80} }
    \qbezier[16](80, 0)(80,40)(80,80)
    \qbezier[16]( 0,80)(40,80)(80,80)

    \put( 80,-5){\line(0,1){10} }
    \put(-5, 80){\line(1,0){10} }
    \put( 80,-10){\makebox(0,0)[t ]{$\fp(t)$}}
    \put(-10, 80){\makebox(0,0)[r ]{$\fq(t)$}}
    \put( 40, 40){\makebox(0,0)[br]{$\fa(t)$}}
    \put( 30,  5){\makebox(0,0)[bl]{$\phi(t)$} }
    \put( 85,85){\makebox(0,0)[bl]{$\fx_l(t) = \fp(t) + i\fq(t)$} }
  \end{picture}
  \end{fsL}
  \end{center}
  %\caption{
  %   Time representation components for narrowband signal
  %   \label{fig:PQ}
  %   }
  %\end{figure}
\end{minipage}%
\begin{minipage}{2\tw/3}%
  Narrowband signals have three common time representations
  (next definition).
  These three forms are equivalent under some simple relations
  (next proposition).
\end{minipage}%

%---------------------------------------
\begin{definition}
\label{def:narrowband_quad}
%---------------------------------------
Let $\fx(t)$ be a \structe{narrowband signal} \xref{def:narrowband}.
\defboxt{
  Let $\fx(t) \eqd \fa(t)\cos\brs{2\pi f_c t + \phi(t) }$. Then
  \\\indentx$\begin{array}{lMMrcl}
      \fa(t)  & is the \fnctd{amplitude}            & of $\fx(t)$
    \\\phi(t) & is the \fnctd{phase}                & of $\fx(t)$
    \\\fp(t)  & is the \fnctd{quadrature component} & of $\fx(t)$ where & \fp(t)   &\eqd& \fa(t)\cos\brs{\phi(t)}
    \\\fq(t)  & is the \fnctd{inphase component}    & of $\fx(t)$ where & \fq(t)   &\eqd& \fa(t)\sin\brs{\phi(t)}
    \\\fx_l(t)& is the \fnctd{complex envelope}     & of $\fx(t)$ where & \fx_l(t) &\eqd& \fp(t) + i\fq(t)
  \end{array}$
}
\end{definition}

A narrowband signal $\fx:\R\to\R$ can be represented
by any of the following three \textbf{equivalent canonical forms}:
%---------------------------------------
\begin{proposition}
\label{prop:canforms}
%---------------------------------------
Let $\fx(t)$ be defined as in \pref{def:cenv}.
\propbox{\begin{array}{rc>{\ds}lM}
   \fx(t) &\eqd& \fa(t)\cos\brs{2\pi f_c t + \phi(t)}            & (\structe{amplitude and phase form})
        \\&=&    \fp(t)\cos(2\pi f_c t) - \fq(t)\sin(2\pi f_c t) & (\structe{quadrature form})
        \\&=&    \Reb{\fx_l(t)e^{i2\pi f_c t}}                   & (\structe{complex envelope form})
\end{array}}
\end{proposition}
\begin{proof}
\begin{align*}
  \fx(t)
    &\eqd \mcom{\boxed{\fa(t)\cos\brs{2\pi f_c t + \phi(t)}}}{amplitude-phase form}
    && \text{amplitude and phase form}
    && \text{\xref{def:cenv}}
  \\&= \mcom{\fa(t)\cos[\phi(t)]}{$\fp(t)$} \cos[2\pi f_c t]
      -\mcom{\fa(t)\sin[\phi(t)]}{$\fq(t)$} \sin[2\pi f_c t]
    && \text{by \thme{double angle formulas}}
    && \text{\ifxref{harTrig}{thm:trig_double}}
  \\&= \mcom{\boxed{\fp(t)\cos[2\pi f_c t] -\fq(t)\sin[2\pi f_c t]}}{quadrature form}
    && \text{quadrature form}
    && \text{\xref{def:cenv}}
  \\&= \Real\brp{[\fp(t)+i\fq(t)][\cos(2\pi f_c t) + i\sin(2\pi f_c t)]}
    && \text{by definitions of $\Real$}
    && \text{\xref{def:Re}}
  \\&= \mcom{\boxed{\Reb{\fx_l(t)e^{i2\pi f_c t}}}}{complex envelope form}
    && \text{by \thme{Euler's identity}}
    && \text{\ifxref{harTrig}{thm:eid}}
    %&& \text{complex envelope form}
    %&& \text{\xref{def:cenv}}
\end{align*}
\end{proof}

The three canonical forms in \prefpp{prop:canforms} are now designated formally:
%---------------------------------------
\begin{definition}
\label{def:canforms}
\label{def:cenv}
%---------------------------------------
Let $\fx(t)$ be a \structe{narrowband signal}.
\defbox{\begin{array}{>{\ds}lMMMM}
     \fx(t)\eqd\fa(t)\cos\brs{2\pi f_c t + \phi(t)}            & is the & \structd{amplitude and phase form}     & of & $\fx(t)$
   \\\fx(t)\eqd\fp(t)\cos(2\pi f_c t) - \fq(t)\sin(2\pi f_c t) & is the & \structd{quadrature form}\footnotemark & of & $\fx(t)$
   \\\fx(t)\eqd\Reb{\fx_l(t)e^{i2\pi f_c t}}                   & is the & \structd{complex envelope form}        & of & $\fx(t)$
\end{array}}
\footnotetext{
$\fx(t) = \fp(t)\cos(2\pi f_c t) - \fq(t)\sin(2\pi f_c t) $
is also known as \structe{Rice's representation}.\\
Reference: \citerp{srv}{23}
}
\end{definition}

\prefpp{prop:canforms} gave the three canonical forms \xref{def:canforms} in terms of a
modulated narrowband signal $\fx(t)$ and some quadrature components defined in \prefpp{def:narrowband_quad}.
\pref{prop:canforms_quad} (next) gives some relationships between these components.
%---------------------------------------
\begin{proposition}
\label{prop:canforms_quad}
%---------------------------------------
\propbox{\begin{array}{rc>{\ds}l | rc>{\ds}l}
    \fx_l(t) &=& \mc{3}{l}{\fa(t)e^{i\phi(t)}}
  \\\fa(t)   &=& \sqrt{\fp^2(t) + \fq^2(t)} & \phi(t) &=& \arctan\frac{\fq(t)}{\fp(t)}
  \\\fp(t)   &=& \fa(t)\cos\phi             & \fq(t)  &=& \fa(t)\sin\phi
  \\\fp(t)   &=& \Reb{\fx_l(t)}             & \fq(t)  &=& \Imb{\fx_l(t)}
\end{array}}
\end{proposition}
\begin{proof}
\begin{align*}
   \fp
     &=  \Reb{\fp + i\fq}
    &&=  \Reb{\fx_l}
\\
   \fq
     &=  \Imb{\fp + i\fq}
    &&=  \Imb{\fx_l}
\\
   \fp
     &= \Reb{\fp + i\fq}
    &&= \Reb{\fa e^{i\phi}}
    &&= \Reb{\fa\cos\phi + i\fa\sin\phi}
    &&= \fa\cos\phi
\\
   \fq
     &= \Imb{\fp + i\fq}
    &&= \Imb{\fa e^{i\phi}}
    &&= \Imb{\fa\cos\phi + i\fa\sin\phi}
    &&= \fa\sin\phi
\\
   \fa^2
     &= \fa^2 (\cos^2\phi + \sin^2\phi)
    &&= (\fa \cos\phi)^2 + (\fa\sin\phi)^2
    &&= \fp^2 + \fq^2
\\
   \tan\phi
     &= \frac{\sin\phi}{\cos\phi}
    &&= \frac{\fa\sin\phi}{\fa\cos\phi}
    &&= \frac{\fq}{\fp}
\end{align*}
\end{proof}

%---------------------------------------
\begin{minipage}{\tw-60mm}
\begin{remark}
%---------------------------------------
In practice (with real hardware), you will likely first have access to
the quadrature components $\fp(t)$ and $\fq(t)$.
Take for example the \hie{Analog Devices ADL5387 Quadrature Demodulator} and
evaluation board, as illustrated to the right.\footnotemark
Note that \fncte{quadrature component} $\fp(t)$ is available at connector ``Q OUTPUT"
     and  \fncte{inphase component}    $\fq(t)$ is available at connector ``I OUTPUT".
\end{remark}
\end{minipage}\hfill\tbox{\includegraphics[width=55mm]{../common/ic/ADL5387.jpg}}
\footnotetext{diagram copied from \citeP{ADL5387}}

%======================================
\section{Frequency Representation}
\index{narrowband!frequency representation}
%======================================
Any \prope{real-valued} time signal $\fx:\R\to\R$
is always \prope{hermitian symmetric} in frequency
such that $\Fx(f) =  \Fx^\ast(-f)$ \xref{fig:freq_rep}.

\begin{figure}[ht] \color{figcolor}
\begin{center}
\begin{fsL}
\setlength{\unitlength}{0.1mm}
\begin{tabular}{cc}
\begin{picture}(700,200)
  %\graphpaper[10](0,0)(600,200)
  \thinlines
  \put(   0 , 100 ){\line(1,0){600} }
  \put( 300 ,  20 ){\line(0,1){180} }
  \put(   0 ,  00 ){\makebox( 200, 90)[t]{$-f_c$} }
  \put( 400 ,  00 ){\makebox( 200, 90)[t]{$+f_c$} }
  \put( 610 ,   0 ){\makebox( 100,200)[l]{$f$} }
  \put( 310 , 100 ){\makebox( 100,100)[tl]{$\Reb{\Fx(f)}$} }
  \thinlines
  \put( 100 ,  90 ){\line( 0,1){20} }
  \put( 500 ,  90 ){\line( 0,1){20} }
  \put( 470 , 100 ){\line( 1,2){30} }
  \put( 530 , 100 ){\line(-1,2){30} }
  \put(  70 , 100 ){\line( 1,2){30} }
  \put( 130 , 100 ){\line(-1,2){30} }
\end{picture}
&
\begin{picture}(700,200)
  \thinlines
  \put(   0 , 100 ){\line(1,0){600} }
  \put( 300 ,  20 ){\line(0,1){180} }
  \put(   0 ,  00 ){\makebox( 200, 90)[t]{$-f_c$} }
  \put( 400 ,  00 ){\makebox( 200, 90)[t]{$+f_c$} }
  \put( 610 ,  00 ){\makebox( 100,200)[l]{$f$} }
  \put( 310 , 100 ){\makebox( 100,100)[tl]{$\Imb{\Fx(f)}$} }
  \thinlines
  \put( 100 ,  90 ){\line( 0, 1){20} }
  \put( 500 ,  90 ){\line( 0, 1){20} }
  \put( 470 , 100 ){\line( 1, 2){30} }
  \put( 530 , 100 ){\line(-1, 2){30} }
  \put(  70 , 100 ){\line( 1,-2){30} }
  \put( 130 , 100 ){\line(-1,-2){30} }
\end{picture}
\\
\begin{picture}(700,200)
  \thinlines
  \put(   0 , 100 ){\line(1,0){600} }
  \put( 300 ,  20 ){\line(0,1){180} }
  \put(   0 ,  00 ){\makebox( 200, 90)[t]{$-f_c$} }
  \put( 400 ,  00 ){\makebox( 200, 90)[t]{$+f_c$} }
  \put( 610 ,  00 ){\makebox( 100,200)[l]{$f$} }
  \put( 310 , 100 ){\makebox( 100,100)[tl]{$|\Fx(f)|$} }
  \thinlines
  \put( 100 ,  90 ){\line( 0,1){20} }
  \put( 500 ,  90 ){\line( 0,1){20} }
  \put( 470 , 100 ){\line( 1,2){30} }
  \put( 530 , 100 ){\line(-1,2){30} }
  \put(  70 , 100 ){\line( 1,2){30} }
  \put( 130 , 100 ){\line(-1,2){30} }
\end{picture}
&
\begin{picture}(700,200)
  \thinlines
  \put(   0 , 100 ){\line(1,0){600} }
  \put( 300 ,  20 ){\line(0,1){180} }
  \put(   0 ,   0 ){\makebox( 200, 90)[t]{$-f_c$} }
  \put( 400 ,   0 ){\makebox( 200, 90)[t]{$+f_c$} }
  \put( 610 ,   0 ){\makebox( 100,200)[l]{$f$} }
  \put( 310 , 100 ){\makebox( 100,100)[tl]{$\angle\Fx(f)$} }
  \thinlines
  \put( 100 ,  90 ){\line( 0, 1){20} }
  \put( 500 ,  90 ){\line( 0, 1){20} }
  \put( 470 , 100 ){\line( 0,-1){60} }
  \put( 470 ,  40 ){\line( 1, 2){60} }
  \put( 530 , 100 ){\line( 0, 1){60} }
  \put(  70 , 100 ){\line( 0,-1){60} }
  \put(  70 ,  40 ){\line( 1, 2){60} }
  \put( 130 , 100 ){\line( 0, 1){60} }
\end{picture}
\end{tabular}
\end{fsL}
\end{center}
\caption{
   Frequency characteristics of any real-valued signal $\fx(t)$
   \label{fig:freq_rep}
   }
\end{figure}

%=======================================
\section{Lowpass representation}
\index{narrowband!lowpass representation}
%=======================================
The \fncte{complex envelope} $\fx_l:\R\to\C$ \xref{def:cenv} of a narrowband signal
$\fx:\R\to\R$ is sometimes called the \fncte{lowpass representation}
of $\fx(t)$.
Because all the information carried by $\fx(t)$
is contained within a narrow band of $\Fx(f)$,
the lowpass representation $\fx_l(t)$ along with the parameter $f_c$
is a sufficient representation of $\fx(t)$
and thus the high frequency factor $e^{i2\pi f_c t}$ may for engineering purposes often be ignored.

The sufficiency of the low-pass representation $\fx_l(t)$
is demonstrated in that
\begin{enume}
   \item $\fx_l(t)$ together with $f_c$ is sufficient to represent
         $\fx(t)$ in time (by \prefpp{def:canforms})
   \item $\Fx_l(f)$ together with $f_c$ is sufficient to represent
         $\Fx(f)$ in frequency (\prefpp{thm:xxl})
   \item $\fx_l(t)$ is sufficient to calculate the energy in $\fx(t)$
         (\prefpp{thm:xxl})
   \item $\fx_l(t)$ and the impulse response $\fh(t)$ of an LTI operation
         is sufficient to calculate the output of the LTI operation
         on $\fx(t)$ (\prefpp{thm:cmplxEnvLTI}).
\end{enume}
%---------------------------------------
\begin{theorem}
\label{thm:xxl}
%---------------------------------------
Let $\fx:\R\to\R$ be a signal with center frequency $f_c\in\R$
and $\fx_l:\R\to\C$ the complex envelope of $\fx(t)$ \xref{def:cenv}.
\thmbox{
  \brb{\begin{array}{M}
    $\fx(t)$ is\\\prope{narrowband}
  \end{array}}
  \implies
  \brb{\begin{array}{rcl}
     \Fx(f)        &=&    \frac{1}{2}\Fx_l(f-f_c) + \frac{1}{2}\Fx_l^\ast(-f-f_c)
  \\\opE{\fx(t)}   &\eqa& \frac{1}{2}E{\fx_l(t)}
  \\\abs{\Fx(f)}^2 &=&    \frac{1}{4} \abs{\Fx_l( f-f_c)}^2 + \frac{1}{4} \abs{\Fx_l(-f-f_c)}^2
  \\\angle\Fx(f)   &=&    \brbr{\begin{array}{rM}
                             \angle\Fx_l(f-f_c) & for $f\eqa +f_c$ \\
                            -\angle\Fx_l(f+f_c) & for $f\eqa -f_c$
                          \end{array}}
  \end{array}}
  }
\end{theorem}
\begin{proof}
\begin{enumerate}
  \item Proof that $\opE\fx(t)\eqa\frac{1}{2}\opE\fx_l(t)$:
    \begin{align*}
       \opE{\fx(t)}
            &\eqd \norm{\fx(t)}^2
          \\&=    \norm{\Reb{ \fx_l(t)e^{j2\pi f_ct}}}^2
          \\&=    \norm{\frac{1}{2}\fx_l(t)e^{j2\pi f_ct} + \frac{1}{2}\fx_l^\ast(t)e^{-j2\pi f_ct} }^2
            && \text{by \thme{Euler formulas}}
            && \text{\xref{cor:eform}}
          \\&= \mathrlap{   \norm{ \frac{1}{2}\fx_l(t)e^{j2\pi f_ct} }^2
             +    \norm{ \frac{1}{2}\fx_l^\ast(t)e^{-j2\pi f_ct} }^2
             +    2\Reb{\inprod{\frac{1}{2}\fx_l(t)e^{j2\pi f_ct}}{\frac{1}{2}\fx_l^\ast(t)e^{-j2\pi f_ct}}}
            \quad\text{by \thme{Polar Identity}}
                     \ifsxref{vsinprod}{lem:polarid}}
          \\&=    \frac{1}{4}\norm{ \fx_l(t) }^2
             +    \frac{1}{4}\norm{ \fx_l(t) }^2
             +    \frac{1}{2}\mcom{\Reb{\inprod{\fx_l(t)e^{j2\pi f_ct}}{\fx_l^\ast(t)e^{-j2\pi f_ct}}}}{$\eqa0$ after \ope{low-pass filtering}}
          \\&\eqa \frac{1}{2}\norm{ \fx_l(t) }^2
          \\&\eqd \frac{1}{2}\opE{\fx_l(t)}
    \end{align*}

  \item lemma: $\Fx(f)=\frac{1}{2}\Fx_l(f-f_c) + \frac{1}{2}\Fx_l^\ast(-f-f_c)$. Proof: \label{ilem:xxl}
    \begin{align*}
      \boxed{\Fx(f)}
           &\eqd [\opFT{\fx(t)}](f)
           && \text{by definition of $\Fx(f)$}
         \\&\eqd \inprod{\fx(t)}{e^{j2\pi ft}}
           && \text{by definition of \ope{Fourier Transform}}
         \\&\eqd \inprod{\Reb{ \fx_l(t)e^{j2\pi f_ct}}}
                              {e^{j2\pi ft}}
           && \text{by definition of \fncte{complex envelope} $\fx_l(t)$}
         \\&=    \inprod{\frac{1}{2}\left[ \fx_l(t)e^{j2\pi f_ct} + \fx_l^\ast(t)e^{-j2\pi f_ct}   \right]}
                              {e^{j2\pi ft}}
            && \text{by \thme{Euler formulas}
                     \xref{cor:eform}}
         \\&=    \frac{1}{2} \inprod{\fx_l(t)     e^{ j2\pi f_ct}}{ e^{j2\pi ft}}
               + \frac{1}{2} \inprod{\fx_l^\ast(t)e^{-j2\pi f_ct}}{ e^{j2\pi ft}}
           && \text{by \prope{additive} property of $\inprodn$}
           && \text{\ifxref{vsinprod}{def:inprod}}
         \\&\eqd \frac{1}{2} \int_{t\in\R}  \fx_l(t)e^{-j2\pi(f-f_c)t} \dt
            +    \frac{1}{2} \left[ \int_{t\in\R}  \fx_l(t) e^{-j2\pi(- f-f_c)t}  \dt \right]^\ast
           && \text{by definition of $\inprodn$}
         \\&\eqd \boxed{\frac{1}{2}\Fx_l(f-f_c) + \frac{1}{2}\Fx_l^\ast(-f-f_c)}
           && \text{by definition of \ope{Fourier Transform}}
    \end{align*}

  \item
    \begin{align*}
       \Reb{\Fx(f)}
         &= \Reb{\frac{1}{2}\Fx_l(f-f_c) + \frac{1}{2}\Fx_l^\ast(-f-f_c)}
         && \text{by \pref{ilem:xxl}}
       \\&= \frac{1}{2}\Reb{\Fx_l(f-f_c)} + \frac{1}{2}\Reb{\Fx_l^\ast(-f-f_c)}
       \\&= \frac{1}{2}\Reb{\Fx_l(f-f_c)} + \frac{1}{2}\Reb{\Fx_l(-f-f_c)}
       \\
       \Imb{\Fx(f)}
         &= \Imb{\frac{1}{2}\Fx_l(f-f_c) + \frac{1}{2}\Fx_l^\ast(-f-f_c)}
         && \text{by \pref{ilem:xxl}}
       \\&= \frac{1}{2}\Imb{\Fx_l(f-f_c)} + \frac{1}{2}\Imb{\Fx_l^\ast(-f-f_c)}
       \\&= \frac{1}{2}\Imb{\Fx_l(f-f_c)} - \frac{1}{2}\Imb{\Fx_l(-f-f_c)}
       \\
       \abs{\Fx(f)}^2
         &= \left| \frac{1}{2}\Fx_l(f-f_c) + \frac{1}{2}\Fx_l^\ast(-f-f_c) \right|^2
         && \text{by \pref{ilem:xxl}}
       \\&= \frac{1}{4}
            \left[\Fx_l(f-f_c) + \frac{1}{2}\Fx_l^\ast(-f-f_c) \right]
            \left[\Fx_l(f-f_c) + \frac{1}{2}\Fx_l^\ast(-f-f_c) \right]^\ast
       \\&=\frac{1}{4} \Big[
            \Fx_l     ( f-f_c) \Fx_l^\ast( f-f_c) +
            \Fx_l     ( f-f_c) \Fx_l     (-f-f_c) +
       \\&\qquad \Fx_l^\ast(-f-f_c) \Fx_l^\ast( f-f_c) +
            \Fx_l^\ast(-f-f_c) \Fx_l     (-f-f_c)
            \Big]
       \\&= \frac{1}{4} \brs{
            \abs{\Fx_l( f-f_c)}^2 +
            2\Reb{\Fx_l     ( f-f_c) \Fx_l     (-f-f_c)} +
            \abs{\Fx_l^\ast(-f-f_c)}^2
            }
       \\&= \frac{1}{4} \brs{
            \abs{\Fx_l( f-f_c)}^2 + \abs{\Fx_l(-f-f_c)}^2 + 0
            }
       \\
       \angle\Fx(f)
         &= \angle\left[ \frac{1}{2}\Fx_l(f-f_c) + \frac{1}{2}\Fx_l^\ast(-f-f_c) \right]
       \\&= \angle\left[ \Fx_l(f-f_c) + \Fx_l^\ast(-f-f_c) \right]
       \\&= \atan\frac{\Imb{\Fx_l(f-f_c) + \Fx_l^\ast(-f-f_c) }}
                      {\Reb{\Fx_l(f-f_c) + \Fx_l^\ast(-f-f_c) }}
         && \text{by definition of $\angle$}
        %&& \text{by defs. of $\Real$ and $\Imag$ \ifsxref{normalg}{def:Re}}
       \\&= \atan\frac{\Imb{\Fx_l(f-f_c)} + \Imb{\Fx_l^\ast(-f-f_c) }}
                      {\Reb{\Fx_l(f-f_c)} + \Reb{\Fx_l^\ast(-f-f_c) }}
       \\&= \atan\frac{\Imb{\Fx_l(f-f_c)} - \Imb{\Fx_l(-f-f_c) }}
                      {\Reb{\Fx_l(f-f_c)} + \Reb{\Fx_l(-f-f_c) }}
       \\&= \left\{\begin{array}{rl}
             \angle\Fx_l(f-f_c) &:f\eqa +f_c \\
            -\angle\Fx_l(f+f_c) &:f\eqa -f_c
            \end{array}\right.
    \end{align*}
\end{enumerate}
\end{proof}

%---------------------------------------
\begin{theorem}{\bf Lowpass LTI theorem.}
\label{thm:cmplxEnvLTI}
\index{lowpass LTI theorem}
%---------------------------------------
\begin{enume}
  \item Let $\fx:\R\to\R$ be a narrowband signal
        at center frequency $f_c\in\R$,
        with complex envelope $\fx_l:\R\to\C$,
        and Fourier transform $\Fx:\R\to\C$.
  \item Let $\fh:\R\to\R$ be the narrowband impulse response
        of an LTI operation such that $\fh(t)$ is located
        at center frequency $f_c\in\R$,
        has complex envelope $\fh_l:\R\to\C$,
        and Fourier transform $\Fh:\R\to\C$.
  \item Let $\fy:\R\to\R$ be the response of
        the LTI operation on $\fx(t)$.
        Let the complex envelope of $\fy(t)$ be $\fy_l:\R\to\C$
        and the Fourier transform $\Fy:\R\to\C$.
\end{enume}
Then
\thmbox{
\begin{array}{rcl}
  \fy_l(t) &=& \frac{1}{2} \fh_l(t) \conv \fx_l(t) \\
  \Fy_l(f) &=& \frac{1}{2} \Fh_l(f)       \Fx_l(f)
\end{array}
}
\end{theorem}
\begin{proof}
\begin{eqnarray*}
   \Reb{ \fy_l(t)e^{i2\pi f_ct}}
     &=& \fy(t)
   \\&=& \fh(t) \conv \fx(t)
   \\&=& \int_u \fh(u)\fx(t-u)\du
   \\&=& \int_u \Reb{ \fh_l(u)e^{i2\pi \ff_cu}}
                \Reb{ \fx_l(t-u)e^{i2\pi \ff_c(t-u)}} \du
   \\&=& \frac{1}{4}
         \int_u \left[\fh_l(t)  e^{i2\pi f_ct    } + \fh_l^\ast(t  )e^{-i2\pi f_c t }\right]
                \left[\fx_l(t-u)e^{i2\pi f_c(t-u)} + \fx_l^\ast(t-u)e^{-i2\pi f_c(t-u)}\right] \du
   \\&=& \frac{1}{4}\int_u \fh_l(u)     e^{ i2\pi \ff_cu} \fx_l(t-u)     e^{ i2\pi f_c(t-u)} \du
       + \frac{1}{4}\int_u \fh_l(u)     e^{ i2\pi \ff_cu} \fx_l^\ast(t-u)e^{-i2\pi f_c(t-u)} \du +
   \\&&  \frac{1}{4}\int_u \fh_l^\ast(u)e^{-i2\pi \ff_cu} \fx_l(t-u)     e^{ i2\pi f_c(t-u)} \du
       + \frac{1}{4}\int_u \fh_l^\ast(u)e^{-i2\pi \ff_cu} \fx_l^\ast(t-u)e^{-i2\pi f_c(t-u)} \du
%
   \\&=& \frac{1}{4}e^{ i2\pi f_ct} \int_u \fh_l(u)\fx_l(t-u)                         \du
       + \frac{1}{4}e^{-i2\pi f_ct} \int_u \fh_l(u) e^{ i4\pi \ff_cu} \fx_l^\ast(t-u) \du +
   \\&&  \frac{1}{4}e^{ i2\pi f_ct} \int_u \fh_l^\ast(u)e^{-i4\pi \ff_cu} \fx_l(t-u)  \du
       + \frac{1}{4}e^{-i2\pi f_ct} \int_u \fh_l^\ast(u) \fx_l^\ast(t-u)              \du
%
   \\&=& \frac{1}{4}      e^{ i2\pi f_ct} \int_u \fh_l(u)\fx_l(t-u)                         \du
        +\frac{1}{4}\left(e^{ i2\pi f_ct} \int_u \fh_l(u)\fx_l(t-u)                         \du \right)^\ast +
   \\&&  \frac{1}{4}      e^{ i2\pi f_ct} \int_u \fh_l^\ast(u) e^{-i4\pi \ff_cu} \fx_l(t-u) \du
       + \frac{1}{4}\left(e^{ i2\pi f_ct} \int_u \fh_l^\ast(u) e^{-i4\pi \ff_cu} \fx_l(t-u) \du \right)^\ast
%
   \\&=& \frac{1}{2}\Reb{ e^{ i2\pi f_ct} \int_u \fh_l(u)\fx_l(t-u)                         \du }
        +\frac{1}{2}\Reb{ e^{ i2\pi f_ct} \int_u \fh_l^\ast(u) e^{-i4\pi \ff_cu} \fx_l(t-u) \du }
   \\&=& \frac{1}{2}\Reb{ e^{ i2\pi f_ct} [\fh_l\conv\fx_l](t) }
        +\frac{1}{2}\Reb{ e^{ i2\pi f_ct} \int_u \fh_l^\ast(u) \fx_l(t-u) e^{-i4\pi \ff_cu} \du }
   \\&\eqa& \frac{1}{2}\Reb{ e^{ i2\pi f_ct} [\fh_l\conv\fx_l](t) }
        + 0 ?
\end{eqnarray*}
\attention
Note that convolving $\fx_l(t)$ with $\fh(t)$ directly does
not work (we still need the factor $e^{i2\pi \ff_c(t)}$).
\begin{eqnarray*}
   \Reb{ \fy_l(t)e^{i2\pi f_ct}}
     &=& \fy(t)
   \\&=& \fh(t) \conv \fx(t)
   \\&=& \int_u \fh(u)\fx(t-u)\du
   \\&=& \int_u \fh(u)
                \Reb{ \fx_l(t-u)e^{i2\pi \ff_c(t-u)}} \du
   \\&=& \Reb{ \int_u \fh(u)
                \fx_l(t-u)e^{i2\pi \ff_c(t-u)} \du }
   \\&=& \Reb{ \fh(t)\conv\left[\fx_l(t)e^{i2\pi \ff_c(t)}\right]}
\end{eqnarray*}
\end{proof}



%======================================
\section{Narrowband noise processes}
%======================================
A narrowband noise process $\fn(t)$ can be represented in any of the
three canonical forms presented in \prefpp{def:canforms}
(\prefpo{def:canforms}):
\begin{align*}
   \fn(t)
     &= \fa(t)\cos[2\pi f_ct + \phi(t)]
     && \text{(amplitude and phase)}
   \\&= \fp(t)\cos(2\pi f_ct) - \fq(t) \sin(2\pi f_c t)
     && \text{(quadrature)}
   \\&= \Real\left(\fn_l(t)e^{j2\pi f_ct}\right)
     && \text{(complex envelope).}
\end{align*}



\begin{figure}[ht] \color{figcolor}
\begin{center}
\begin{fsL}
\setlength{\unitlength}{0.1mm}
\begin{tabular}{cccc}
\begin{picture}(230,200)(-100,-100)
  %\graphpaper[10](0,0)(200,200)
  \thicklines
  \put(-100 ,   0){\line    (   1,   0){200} }
  \put(   0 ,-100){\line    (   0,   1){200} }
  \put( 110 ,   0){\makebox (   0,   0)[l]{$\tau$} }
  \put(  40 ,  40){\makebox (   0,   0)[bl]{$\Rpp(\tau)$} }
  \thinlines
  \qbezier(  0, 80)( 40, 20)( 80,  0)
  \qbezier(  0, 80)(-40, 20)(-80,  0)
\end{picture}
&
\begin{picture}(230,200)(-100,-100)
  %\graphpaper[10](0,0)(200,200)
  \thicklines
  \put(-100 ,   0){\line    (   1,   0){200} }
  \put(   0 ,-100){\line    (   0,   1){200} }
  \put( 110 ,   0){\makebox (   0,   0)[l]{$\tau$} }
  \put(  10 , -20){\makebox (   0,   0)[tl]{$\Rpq(\tau)$} }
  \thinlines
  \put(- 40 ,- 80){\line    (   1,   2){ 80} }
  \qbezier( 40, 80)( 50, 30)( 80,  0)
  \qbezier(-80,  0)(-50,-30)(-40,-80)
\end{picture}
&
\begin{picture}(230,200)(-100,-100)
  %\graphpaper[10](0,0)(200,200)
  \thicklines
  \put(-100 ,   0){\line    (   1,   0){200} }
  \put(   0 ,-100){\line    (   0,   1){200} }
  \put( 110 ,   0){\makebox (   0,   0)[l]{$\tau$} }
  \put(  10 ,  50){\makebox (   0,   0)[l]{$\Rqp(\tau)$} }
  \thinlines
  \put(- 40 ,  80){\line    (   1,  -2){ 80} }
  \qbezier( 40,-80)( 50,-30)( 80,  0)
  \qbezier(-80,  0)(-50, 30)(-40, 80)
\end{picture}
&
\begin{picture}(230,200)(-100,-100)
  %\graphpaper[10](0,0)(200,200)
  \thicklines
  \put(-100 ,   0){\line    (   1,   0){200} }
  \put(   0 ,-100){\line    (   0,   1){200} }
  \put( 110 ,   0){\makebox (   0,   0)[l]{$\tau$} }
  \put(  40 ,  40){\makebox (   0,   0)[bl]{$\Rqq(\tau)$} }
  \thinlines
  \qbezier(  0, 80)( 40, 20)( 80,  0)
  \qbezier(  0, 80)(-40, 20)(-80,  0)
\end{picture}
\\
%\mc{4}{c}{
%  \setlength{\unitlength}{0.2mm}
%  \begin{picture}(250,240)(-100,-100)
%    %\graphpaper[10](-100,-100)(200,200)
%    \thinlines
%    \put(-100,   0){\line(1,0){200} }
%    \put(   0,-100){\line(0,1){200} }
%    \put( 105,   0){\makebox(0,0)[l]{$\cos2\pi f_ct$}}
%    \put(   0, 105){\makebox(0,0)[b]{$\sin2\pi f_ct$}}
%    \put(   0,   0){\vector(1,1){80} }
%    \qbezier[16](80, 0)(80,40)(80,80)
%    \qbezier[16]( 0,80)(40,80)(80,80)
%    \put( 80,-5){\line(0,1){10} }
%    \put(-5, 80){\line(1,0){10} }
%    \put( 80,-10){\makebox(0,0)[t ]{$\Rpp(\tau)$}}
%    \put(-10, 80){\makebox(0,0)[r ]{$\Rpq(\tau)$}}
%    \put( 40, 40){\makebox(0,0)[br]{$\Rxx[nn](\tau)$}}
%    %\put( 30,  5){\makebox(0,0)[bl]{$\phi(t)$} }
%    \put( 85,85){\makebox(0,0)[lb]{$\Rzz(\tau)=2\Rpp(\tau)-2i\Rpq(\tau)$}}
%  \end{picture}
%  }
\end{tabular}
\end{fsL}
\end{center}
\caption{
   Correlations of inphase component $\fp(t)$ and quadrature component $\fq(t)$
   \label{fig:Rpq}
   }
\end{figure}

%--------------------------------------
\begin{theorem}
%--------------------------------------
Let $\fn:\R\to\R$ be a narrowband noise process with
quadrature components $\fp:\R\to\R$ and $\fq:\R\to\R$ and
complex envelope $\fz:\R\to\C$
such that
\begin{eqnarray*}
  n(t) &=& \fp(t)\cos(2\pi f_ct) - \fq(t) \sin(2\pi f_c t) \\
       &=& \Reb{\fz(t)e^{i2\pi f_ct}} \\
  \Rxy(\tau) &\eqd& \pE\brs{\fx(t+\tau)\fy^\ast(t)}.
\end{eqnarray*}
Then (see \prefpp{fig:Rpq})
\thmbox{
\begin{array}{llD}
   1. & \pE\brs{p(t)} = \pE\brs{q(t)} = 0                                        & (component means are zero                 ) \\
   2. & \Rpp(\tau)=\Rqq(\tau)                                            & (autocorrelations are equal               ) \\
   3. & \Rpq(\tau)=-\Rqp(\tau)                                           & (crosscorrelations are additive inverses  ) \\
   4. & \Rpp(\tau)=\Rpp(-\tau)                                           & (autocorrelations are symmetric           ) \\
   5. & \Rpq(\tau)=-\Rpq(-\tau), \Rqp(\tau)=-\Rqp(-\tau)                 & (crosscorrelations are anti-symmetric     ) \\
   6. & \Rpq(0)=0                                                        & (components are uncorrelated for $\tau=0$ ) \\
   7. & \Rxx[nn](\tau) = \Rpp(\tau)\cos(2\pi f_c\tau) + \Rpq(\tau)\sin(2\pi f_c\tau) & (noise autocorrelation                    ) \\
   8. & \Rzz(\tau) = 2\Rpp(\tau) - 2i\Rpq(\tau)                          & (complex envelope autocorrelation         ).
\end{array}
}
\end{theorem}
\begin{proof}
\begin{eqnarray*}
   0 &=& \pE\brs{n(t)}
   \\&=& \pE\brs{p(t)\cos(2\pi f_ct) - q(t)\sin(2\pi f_c t)}
   \\&=& \pE\brs{p(t)\cos(2\pi f_ct)} - \pE\brs{q(t)\sin(2\pi f_c t)}
   \\&=& \pE\brs{p(t)}\cos(2\pi f_ct) - \pE\brs{q(t)}\sin(2\pi f_c t)
\end{eqnarray*}

\begin{eqnarray*}
   \Rxx[nn](\tau)
     &=& \pE\brs{n(t+\tau)n(t)}
   \\&=& \pE\brs{
         \left( p(t+\tau)\cos(2\pi f_ct + 2\pi f_c\tau) - q(t) \sin(2\pi f_ct + 2\pi f_c\tau)  \right)
         \left( p(t     )\cos(2\pi f_ct               ) - q(t) \sin(2\pi f_ct               )  \right)
         }
   \\&=& \pE\brs{ p(t+\tau)p(t) \cos(2\pi f_ct + 2\pi f_c\tau) \cos(2\pi f_ct) }
        -\pE\brs{ p(t+\tau)q(t) \cos(2\pi f_ct + 2\pi f_c\tau) \sin(2\pi f_ct) }
   \\&& -\pE\brs{ q(t+\tau)p(t) \sin(2\pi f_ct + 2\pi f_c\tau) \cos(2\pi f_ct) }
        +\pE\brs{ q(t+\tau)q(t) \sin(2\pi f_ct + 2\pi f_c\tau) \sin(2\pi f_ct) }
%
   \\&=& \Rpp(\tau) \pE\brs{ \cos(2\pi f_ct + 2\pi f_c\tau) \cos(2\pi f_ct) }
        -\Rpq(\tau) \pE\brs{ \cos(2\pi f_ct + 2\pi f_c\tau) \sin(2\pi f_ct) }
   \\&& -\Rqp(\tau) \pE\brs{ \sin(2\pi f_ct + 2\pi f_c\tau) \cos(2\pi f_ct) }
        +\Rqq(\tau) \pE\brs{ \sin(2\pi f_ct + 2\pi f_c\tau) \sin(2\pi f_ct) }
%
   \\&=& \frac{1}{2} \Rpp(\tau) \left[ \cos(2\pi f_c\tau) + \cos(4\pi f_ct + 2\pi f_c\tau) \right]
        -\frac{1}{2} \Rpq(\tau) \left[-\sin(2\pi f_c\tau) + \sin(4\pi f_ct + 2\pi f_c\tau) \right]
   \\&& -\frac{1}{2} \Rqp(\tau) \left[ \sin(2\pi f_c\tau) + \sin(4\pi f_ct + 2\pi f_c\tau) \right]
        +\frac{1}{2} \Rqq(\tau) \left[ \cos(2\pi f_c\tau) - \cos(4\pi f_ct + 2\pi f_c\tau) \right]
%
   \\&=& \frac{1}{2}\left[ \Rpp(\tau) + \Rqq(\tau)\right] \cos(2\pi f_c\tau)
        +\frac{1}{2}\left[ \Rpq(\tau) - \Rqp(\tau)\right] \sin(2\pi f_c\tau)
   \\&& +\frac{1}{2}\left[ \Rpp(\tau) - \Rqq(\tau)\right] \cos(4\pi f_ct + 2\pi f_c\tau)
        -\frac{1}{2}\left[ \Rpq(\tau) + \Rqp(\tau)\right] \sin(4\pi f_ct + 2\pi f_c\tau)
\end{eqnarray*}

Because $\Rxx[nn](\tau)$ is not a function of $t$,
the last two terms must be zero for all $t$,
which implies
\begin{eqnarray*}
   \Rpp(\tau) &=& \Rqq(\tau)   \\
   \Rpq(\tau) &=& -\Rqp(\tau).
\end{eqnarray*}

From these we have
\begin{eqnarray*}
\Rxx[nn](\tau)
     &=& \frac{1}{2}\left[ \Rpp(\tau) + \Rqq(\tau)\right] \cos(2\pi f_c\tau)
        +\frac{1}{2}\left[ \Rpq(\tau) - \Rqp(\tau)\right] \sin(2\pi f_c\tau)
   \\&& +\frac{1}{2}\left[ \Rpp(\tau) - \Rqq(\tau)\right] \cos(4\pi f_ct + 2\pi f_c\tau)
        -\frac{1}{2}\left[ \Rpq(\tau) + \Rqp(\tau)\right] \sin(4\pi f_ct + 2\pi f_c\tau)
%
   \\&=& \Rpp(\tau)\cos(2\pi f_c\tau) +\Rpq(\tau)\sin(2\pi f_c\tau)
\end{eqnarray*}

\begin{align*}
   \Rpq(\tau)
     &=      -\Rqp(\tau)
   \\&\eqd -\pE\brs{q(t+\tau) p(t)}
   \\&=       \pE\brs{p(t) q(t+\tau)}
   \\&\eqd -\Rpq(-\tau)
\end{align*}

This implies
$\Rpq(\tau)$ is odd-symmetric. \\
\begin{eqnarray*}
     &&          \Rpq(\tau)= -\Rpq(-\tau)
   \\&\implies&  \Rpq(0)= -\Rpq(0)
   \\&\implies&  \Rpq(0)=0.
\end{eqnarray*}


\begin{eqnarray*}
   \Rzz(\tau)
     &\eqd& \pE\brs{z(t+\tau)z^\ast(t)}
   \\&=&      \pE\brs{\left( x(t+\tau) + iy(t+\tau)\right)\left( x(t) + iy(t)\right)^\ast}
   \\&=&      \pE\brs{\left( x(t+\tau) + iy(t+\tau)\right)\left( x^\ast(t) - iy^\ast(t)\right)}
%
   \\&=&      \pE\brs{ x(t+\tau)x^\ast(t) }
       -     i\pE\brs{ x(t+\tau)y^\ast(t) }
       +     i\pE\brs{ y(t+\tau)x^\ast(t) }
       +      \pE\brs{ y(t+\tau)y^\ast(t) }
%
   \\&\eqd& \Rpp(\tau) - i\Rpq(\tau) + i\Rqp(\tau) + \Rqq(\tau)
   \\&=&      \Rpp(\tau) - i\Rpq(\tau) - i\Rpq(\tau) + \Rqq(\tau)
   \\&=&      2\Rpp(\tau) - 2i\Rpq(\tau)
\end{eqnarray*}
\end{proof}




