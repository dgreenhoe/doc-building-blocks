%============================================================================
% LaTeX File
% Daniel J. Greenhoe
%============================================================================

%======================================
\chapter{Bandlimited Channel (ISI)}
\index{Intersymbol interference}
\index{ISI}
\index{channel!bandlimited}
\label{chp:isi}
%======================================
\begin{figure}[ht]
\begin{center}
\begin{fsK}
\setlength{\unitlength}{0.12mm}
\begin{picture}(1500,250)(-50,-100)%
  \thicklines%
  %\graphpaper[10](0,0)(500,100)%
  \color{blue}%
    \multiput(-100,0)(200,0){7}{%
      \put(   0,   0){\vector(1,0){100} }%
      \put( 100, -50){\framebox(100,100){}}%
      }%
    \put(  1300,   0){\vector(1,0){100} }%
  \color{blue}%
    \put(-100,10){\makebox(100,0)[b]{$\{u_n\}$}}%
    \put( 100,10){\makebox(100,0)[b]{$\{y_n\}$}}%
    \put( 300,10){\makebox(100,0)[b]{$\{a_n\}$}}%
    \put( 500,10){\makebox(100,0)[b]{$\fs(t;u)$}}%
    \put(1100,10){\makebox(100,0)[b]{$\fr(t;u)$}}%
    \put(1300,10){\makebox(100,0)[b]{$\{\hat{u}_n\}$}}%
  \color{blue}%
    \put(  00, 0){\makebox(100,0)[c]{precoder}}%
    \put( 200, 0){\makebox(100,0)[c]{mapper}}%
    \put( 400, 0){\makebox(100,0)[c]{impulse}}%
    \put( 600, 0){\makebox(100,0)[c]{$\hht(t)$}}%
    \put( 800, 0){\makebox(100,0)[c]{$\hhc(t)$}}%
    \put(1000, 0){\makebox(100,0)[c]{$\hhr(t)$}}%
    \put(1200, 0){\makebox(100,0)[c]{decoder}}%
  \color{red}%
    \put(-20,-100){\dashbox{4}(740,200)[b]{transmitter}}%
    \put(780,-100){\dashbox{4}(140,200)[b]{channel}}%
    \put(980,-100){\dashbox{4}(340,200)[b]{receiver}}%
    \put(580,-150){\dashbox{4}(540,300)[b]{$\hhcmp\eqd\hht\conv\hhc\conv\hhr$}}%
\end{picture}%
\end{fsK}
\end{center}
\caption{
   ISI system model
   \label{fig:isi_model}
   }
\end{figure}




\if 0


\begin{figure}[ht]\color{figcolor}
\begin{center}
\begin{fsK}
\setlength{\unitlength}{0.2mm}
\begin{picture}(700,250)(0,-100)
  \thicklines
  %\graphpaper[10](0,0)(500,100)

  \put(   0 ,   0 ){\vector(1,0){100} }
  \put(   0 ,   0 ){\makebox(100,  50)[b]{$a_n$} }
  \put( 100 , -50 ){\framebox(100,100){transmitter}}
  \put( 100 , -40 ){\makebox( 100, 80)[b]{\opT} }
  \put( 200 ,   0 ){\vector(1,0){100} }
  \put( 200 ,   0 ){\makebox(100,  50)[b]{$s(t)$} }
  \put( 300 , -50 ){\framebox(100,100){}}
  \put( 300 , -40 ){\makebox( 100, 80)[b]{\opC} }
  \put( 400 ,   0 ){\vector(1,0){100} }
  \put( 400 ,   0 ){\makebox(100,  50)[b]{$r(t)$} }
  \put( 500 , -50 ){\framebox(100,100){receiver}}
  \put( 500 , -40 ){\makebox( 100, 80)[b]{\opR} }
  \put( 600 ,   0 ){\vector(1,0){100} }
  \put( 600 ,   0 ){\makebox(100,  50)[b]{$r(nT_2)$} }

  \put( 150 ,  80 ){\vector(0,-1){30} }
  \put( 100 ,  80 ){\makebox(100, 50)[b]{signal rate $\ds \frac{1}{T_1}\le 2W$} }
  \put( 550 ,  80 ){\vector(0,-1){30} }
  \put( 500 ,  80 ){\makebox(100, 50)[b]{sampling rate $\ds \frac{1}{T_2}\ge 2W$} }

  \put( 300 ,-25 ){
  \setlength{\unitlength}{0.08mm}
  \begin{picture}(200,170)(-100,-50)
  \put(-80 ,  0 ){\line(1,0){160} }
  \put(  0 ,-10 ){\line(0,1){130} }
  \put(-50 ,  0 ){\line(0,1){100} }
  \put(-50 ,100 ){\line(1,0){100} }
  \put( 50 ,  0 ){\line(0,1){100} }
  \put(-50 ,-10 ){\makebox(0,0)[t]{$-W$} }
  \put( 50 ,-10 ){\makebox(0,0)[t]{$+W$} }
  \end{picture}
  }

  \put( 200 ,-60 ){\makebox(0, 0)[lt]{$\ds s(t)=\sum_n a_n \hhcmp(t-nT)$} }
  \put( 400 ,-60 ){\makebox(0, 0)[lt]{$\ds r(t)=\sum_n a_n \hhcmp'(t-nT)$} }
\end{picture}
\end{fsK}
\end{center}
\caption{
   ISI system model
%   \label{fig:isi_model}
   }
\end{figure}

\fi

%--------------------------------------
%\section{Introduction}
%\label{sec:intro}
%--------------------------------------
\paragraph{System disturbances.}
There are two fundamental disturbances in any communication system
which increase the probability of communication error:
\begin{dingautolist}{"AC}
   \item noise
   \item intersymbol interference (ISI)
\end{dingautolist}
Noise is produced by a number of sources;
one of them being {\em thermal noise} and therefore can
never be eliminated in any system which operates
above $-273^\circ$ C (absolute zero).
ISI is produced as a result of band-limited communication channels.
Unlike noise, it is possible to completely eliminate ISI
by the proper selection of the symbol waveform used to
carry information through the channel.

This chapter describes the cause of ISI in a communication system
and discusses techniques of designing signaling waveforms
with no ISI.
Three solutions are presented and are summarized in the following table:

\setlength{\unitlength}{0.1mm}
\begin{fsL}\color{figcolor}
\begin{tabular}{|c|c|c|}
\hline
  zero ISI solution & duobinary solution & modified duobinary solution \\
\hline
   $\hhcmp(nT) =
   \left\{
      \begin{array}{rcl}
          1 & : & n = 0 \\
          0 & : & \mbox{otherwise}
      \end{array}
   \right.$
&
   $\hhcmp(nT) =
   \left\{
      \begin{array}{rcl}
          1 & : & n = 0,1 \\
          0 & : & \mbox{otherwise}
      \end{array}
   \right.$
&
   $\hhcmp(nT) =
   \left\{
      \begin{array}{rcl}
           1 & : & n = -1 \\
          -1 & : & n = +1 \\
           0 & : & \mbox{otherwise}
      \end{array}
   \right.$
\\
\begin{picture}(400,200)(-200,-100) \color{figcolor}
  %\graphpaper[10](-100,0)(500,150)
  \thicklines
  \put(-150 ,   0 ){\line(1,0){300} }
  \put( 160 ,   0 ){\makebox(0,0)[l]{$n$}}
  \put(-100 , -10 ){\makebox(0,0)[t]{$-1$}}
  \put(   0 , -10 ){\makebox(0,0)[t]{$0$}}
  \put( 100 , -10 ){\makebox(0,0)[t]{$+1$}}
  \color{red}
    \put(   0 ,   0 ){\line(0,1){ 50} }
    \put(   0 ,  50 ){\circle*{10}}
\end{picture}
&
\begin{picture}(400,200)(-200,-100) \color{figcolor}
  %\graphpaper[10](-100,0)(500,150)
  \thicklines
  \put(-150 ,   0 ){\line(1,0){300} }
  \put( 160 ,   0 ){\makebox(0,0)[l]{$n$}}
  \put(-100 , -10 ){\makebox(0,0)[t]{$-1$}}
  \put(   0 , -10 ){\makebox(0,0)[t]{$0$}}
  \put( 100 , -10 ){\makebox(0,0)[t]{$+1$}}
  \color{red}
    \put(   0 ,   0 ){\line(0,1){ 50} } \put(   0 ,  50 ){\circle*{10}}
    \put( 100 ,   0 ){\line(0,1){ 50} } \put( 100 ,  50 ){\circle*{10}}
\end{picture}
&
\begin{picture}(400,200)(-200,-100) \color{figcolor}
  %\graphpaper[10](-100,0)(500,150)
  \thicklines
  \put(-150 ,   0 ){\line(1,0){300} }
  \put( 160 ,   0 ){\makebox(0,0)[l]{$n$}}
  \put(-100 , -10 ){\makebox(0,0)[t]{$-1$}}
  \put(   0 , -10 ){\makebox(0,0)[t]{$0$}}
  \put( 100 , -10 ){\makebox(0,0)[t]{$+1$}}
  \color{red}
    \put(-100 ,   0 ){\line(0, 1){ 50} } \put(-100 ,  50 ){\circle*{10}}
    \put( 100 ,   0 ){\line(0,-1){ 50} } \put( 100 , -50 ){\circle*{10}}
\end{picture}
\\
   $\bigg\Updownarrow$ & $\bigg\Updownarrow$ & $\bigg\Updownarrow$
\\
   $\frac{1}{T}\sum_n \ft{\hhcmp}\left(f+ \frac{n}{T} \right) = 1$
&
   $\frac{1}{T}\sum_n \ft{\hhcmp}\left(f+ \frac{n}{T} \right)
   = 2 e^{-i\pi fT} \cos(\pi fT)$
&
   $\frac{1}{T}\sum_n \ft{\hhcmp}\left(f+ \frac{n}{T} \right)
   = i2 \sin( 2\pi fT )$
\\
\begin{picture}(400,150)(-200,-50) \color{figcolor}
  %\graphpaper[10](-100,0)(500,150)
  \thicklines
  \put(-150 ,   0 ){\line(1,0){300} }
  \put(   0 , -20 ){\line(0,1){100} }
  \put( 160 ,   0 ){$f$}
  \put( -20 ,  60 ){$1$}
  \put(-100 , -10 ){\makebox(0,0)[t]{$-W=-\frac{1}{2T}$}}
  \put( 100 , -10 ){\makebox(0,0)[t]{$W=\frac{1}{2T}$}}
  \color{red}
    \put(-100 ,   0 ){\line(0,1){ 50} }
    \put(-100 ,  50 ){\line(1,0){200} }
    \put( 100 ,   0 ){\line(0,1){ 50} }
\end{picture}
&
\begin{picture}(400,150)(-200,-50) \color{figcolor}
  %\graphpaper[10](-100,0)(500,150)
  \thicklines
  \put(-150 ,   0 ){\line(1,0){300} }
  \put(   0 , -20 ){\line(0,1){100} }
  \put( 160 ,   0 ){$f$}
  %\put(  20 ,  60 ){$\ds|\ft{\hhcmp}(f)|=\frac{1}{W}\cos\left(\frac{\pi}{2W}f\right)\chi_B(f)$}
  \put( -20 ,  60 ){$2$}
  \put(-200 , -50 ){\makebox(200,40)[t]{$-W=-\frac{1}{2T}$}}
  \put(   0 , -50 ){\makebox(200,40)[t]{$W=\frac{1}{2T}$}}
  \color{red}
    \qbezier(-100,   0)(   0, 100)( 100,   0)
\end{picture}
&
\begin{picture}(400,150)(-200,-50) \color{figcolor}
  %\graphpaper[10](-100,0)(500,150)
  \thicklines
  \put(-150 ,   0 ){\line(1,0){300} }
  \put(   0 , -20 ){\line(0,1){100} }
  \put( 160 ,   0 ){$f$}
  %\put(  20 ,  60 ){$\ds -i\ft{\hhcmp}(f)=\frac{1}{W}\sin\left(\frac{\pi}{W}f\right)\chi_B(f)$}
  \put( -20 ,  60 ){$2$}
  \put(-200 , -50 ){\makebox(200,40)[t]{$-W=-\frac{1}{2T}$}}
  \put(   0 , -50 ){\makebox(200,40)[t]{$W=\frac{1}{2T}$}}
  \color{red}
    \qbezier(-100,   0)( -50, 100)(   0,   0)
    \qbezier(   0,   0)(  50, 100)( 100,   0)
\end{picture}\\
\hline
Section~\ref{sec:isi_zero} page~\pageref{sec:isi_zero} &
Section~\ref{sec:isi_duo}  page~\pageref{sec:isi_duo}  &
Section~\ref{sec:isi_mduo} page~\pageref{sec:isi_mduo}
\\
\hline
\end{tabular}
\end{fsL}

%=======================================
\section{Description of ISI}
\label{sec:chan-model}
%=======================================
The channel model is illustrated in \prefpp{fig:isi_model}.
The signal received at the decoder is
\[ \fr(t;u) = \sum_n a_n \hhcmp(t-nT). \]
We arbitrarily scale $\hhcmp(t)$ such that
\[ \hhcmp(0)=1. \]
If this signal is sampled at intervals $T$, we have
\begin{align*}
  \fr(nT)
    &= \left.\fr(t)\right|_{t=nT}
  \\&= \left. \sum_m a_m \hhcmp(t-mT) \right|_{t=nT}
  \\&= \sum_m a_m \hhcmp(nT-mT)
  \\&= a_n \hhcmp(0) + \sum_{m\not=n} a_m \hhcmp(nT-mT)
  \\&= \mcom{a_n}{desired} + \mcom{\sum_{m\not=n} a_m \hhcmp(nT-mT)}{ISI (not wanted)}
\end{align*}
At the sampling intervals, we only want $a_n$, not the other terms.
These other terms are referred to as {\em Intersymbol Interference} (ISI).

%--------------------------------------
\begin{definition}
\index{Intersymbol Interference}
\index{ISI}
%--------------------------------------
{\bf Intersymbol interference} (ISI) is a communication system characteristic
such that
a received signal sample $\fr(nT)$ is a function
of one or more information values $a_m, m\ne n$.
If $\fr(nT)$ is a function of $a_n$ alone,
then we say the system has {\bf zero ISI}.
\end{definition}

If $\hhcmp(t)$ is properly designed,
the communication system will have zero ISI.


%=======================================
\section{Zero-ISI solution}
\label{sec:isi_zero}
%=======================================
%=======================================
\subsection{Constraints}
%=======================================
Previously we stated that for zero ISI,
  \[\mcom{a_n}{desired} + \mcom{\sum_{m\not=n} a_m \hhcmp(nT-mT)}{ISI (not wanted)}\]
This equation is satisfied if and only if
\[
   \hhcmp(nT) =
   \left\{\begin{array}{l@{\hspace{4ex}\mbox{for}\hspace{4ex}}l}
     1 &   n=    0 \\
     0 &   n\neq 0
   \end{array}\right.
\]
Also, the channel imposes a band-width constraint $W$.
These considerations can be combined into two fundamental constraints
on the signaling pulse $\hhcmp(t)$:
\[
\begin{array}{llrcl}
   \circOne & \text{\hib{sampling constraint}:}
      & \hhcmp(nT) &=&
          \left\{\begin{array}{l@{\hspace{4ex}\mbox{for}\hspace{4ex}}l}
            1 &   n=    0 \\
            0 &   n\neq 0
          \end{array}\right.
\\
   \circTwo & \text{\hib{bandwidth constraint}:}
      & [\opFT{\hhcmp}](f) &=& 0 \mbox{ for } |f|\ge W.
\end{array}
\]
These two constraints are in conflict with each other.
The sampling constraint is quite easy to satisfy by
designing $\hhcmp$ with support (region on $t$ where $\hhcmp(t)\neq0$)
only within $[0, T)$.
However, giving $\hhcmp$ small support makes $\ft{\hhcmp}$ have large
bandwidth, violating the bandwidth constraint.
However, \pref{thm:unity} (next) gives a criterion which allows
both constraints to be satisfied simultaneously.

%--------------------------------------
\begin{theorem}[Partition of unity criterion]
\footnote{
  \citerp{proakis}{557}
  }
\label{thm:unity}
\index{partition of unity criterion}
%--------------------------------------
Let $\ft{\hhcmp}(f)$ be the Fourier Transform of a function
$\hhcmp(t)$ and $T\in\R$ a constant.
Then
\thmbox{
  \left[
  \hhcmp(nT) =
  \left\{
     \begin{tabular}{lcl}
        1 & : & $n=0$ \\
        0 & : & $n\neq 0$
     \end{tabular}
  \right.
  \right]
  \qquad\iff\qquad
  \left[
    \frac{1}{T}\sum_n \ft{\hhcmp}\left(f+ \frac{n}{T} \right) = 1.
  \right]
  }
\end{theorem}
\begin{proof}
This theorem is easily proven using the
\hie{Inverse Poisson's Summation Formula} (\hie{IPSF})
(\prefp{thm:ipsf})
which states
\begin{eqnarray*}
   \sum_n \ft{\hhcmp}\left( f+ \frac{n}{T} \right)
       &=& T
           \sum_n \hhcmp(nT)
           e^{-i2\pi fnT}
\end{eqnarray*}

\begin{enumerate}
\item Prove ``only if" case ($\implies$):

\begin{align*}
  \frac{1}{T}\sum_n \ft{\hhcmp}\left( f + \frac{n}{T} \right)
     &= \sum_n \hhcmp(nT) e^{-i2\pi fnT}
     && \text{by IPSF}
   \\&= \hhcmp(0) + \sum_{n\ne0} \hhcmp(nT) e^{-i2\pi fnT}
   \\&= 1
     && \text{by left hypothesis}
\end{align*}


\item Prove ``if" case ($\impliedby$):

\begin{align*}
   1
     &= \frac{1}{T}\sum_n \ft{\hhcmp}\left( f+\frac{n}{T} \right)
     && \text{by right hypothesis}
   \\&= \sum_n \hhcmp(nT) e^{-i2\pi fnT}
     && \text{by IPSF}
   \\&= \hhcmp(0) + \sum_{n\ne 0} \hhcmp(nT) e^{-i2\pi fnT}
   \\&= \hhcmp(0) + \sum_{n\ne 0} \hhcmp(nT) \cos(2\pi fnT)
                     - i\sum_{n\ne 0} \hhcmp(nT) \sin(2\pi fnT)
   \\ \implies
   \hhcmp(nT)
       &= \left\{
        \begin{tabular}{lcl}
           1 & : & $n=0$    \\
           0 & : & $n\neq0$
        \end{tabular}
        \right.
       && \text{because ``1" is real for all $f$}
%     \\\implies \mbox{Equation (\ref{eq:I}) }
\end{align*}
\end{enumerate}
\end{proof}

%=======================================
\subsection{Signaling rate limits}
%=======================================
%--------------------------------------
\begin{definition}
\label{def:chi}
\footnote{
  \citerp{ab}{126}
  }
\index{$\chi$ function}
%--------------------------------------
The {\bf characteristic function} $\chi_A:X\to\{0,1\}$ of set $A$ is defined as

\defbox{
  \chi_A(x) \eqd
  \left\{
  \begin{array}{l@{\hspace{4ex}\mbox{for}\hspace{4ex}}l}
     1 & x\in A\subseteq X    \\
     0 & x\notin A\subseteq X
  \end{array}
  \right.
}
\end{definition}

Next are two complimentary theorems;
both of which are closely related to the partition of unity criterion:

\begin{tabular}{lllp{6cm}}
   1. & Nyquist signaling theorem & (\prefpp{thm:nst}) &
   A signal may be transmitted with zero-ISI if the
   signaling rate is less than or equal to $2W$.
\\
   2. & Shannon sampling theorem & (\prefpp{thm:sst}) &
   Perfect reconstruction of a sampled signal is
   possible if the sampling rate is greater than or equal to $2W$.
\end{tabular}

%--------------------------------------
\begin{theorem}[Nyquist signaling theorem]
\footnote{
  \citerp{proakis}{13}
  }
\index{Nyquist signaling theorem}
\label{thm:nst}
%--------------------------------------
\attention
Let $s(t)$ be a signal of the form
   \[ s(t) = \sum_n a_n \hhcmp(t-nT_1) \]
and with bandwidth
  \[ [\opFT s](f) = 0 \mbox{ for } |f|\ge W.  \]

Then there exists $\hhcmp(t)$ such that if
   \[  \frac{1}{T_1} \le 2W \]
then
   \[ s(t) = \sum_n s(nT_1) \hhcmp(t-nT_1). \]

Furthermore, if
   \[  \frac{1}{T_1} = 2W \]
then
   \[
      s(t) = \sum_n s(nT_1) \frac{\sin\left[ \frac{\pi}{T_1}(t-nT_1) \right]}
                                 {           \frac{\pi}{T_1}(t-nT_1)}.
   \]
\end{theorem}


\begin{proof}
The upper signaling rate bound (equality) is proven by the
partition of unity criterion.
Given a signaling rate $1/T$, the pulse shape with the smallest
bandwidth that forms a partition of unity in the frequency domain
is the sync function in the time domain,
which is a rectangular pulse in frequency domain given by
\[ \frac{1}{2W} \chi_{[-W,+W]}(f). \]
\end{proof}

%--------------------------------------
\begin{theorem}[Shannon sampling theorem]
\footnote{
  \citerp{proakis}{13}
  }
\index{Shannon sampling theorem}
\label{thm:sst}
%--------------------------------------
Let $r(t)$ be a signal with bandwidth
         \[ [\opFT r](f) = 0 \mbox{ for } |f|\ge W  \]
and sampled at time intervals $T_2$.

Then there exists $\hhcmp(t)$ such that if
   \[  \frac{1}{T_2} \ge 2W \]
then
   \[ s(t) = \sum_n s(nT_2) \hhcmp(t-nT_2). \]

Furthermore, if
   \[  \frac{1}{T_2} = 2W \]
then
   \[
      s(t) = \sum_n s(nT_2) \frac{\sin\left[ \frac{\pi}{T_2}(t-nT_2) \right]}
                               {           \frac{\pi}{T_2}(t-nT_2)}.
   \]
\end{theorem}


%=======================================
\subsection{Zero-ISI system impulse responses}
%=======================================
Using Partition of Unity Theorem \ref{thm:unity},
we can design ISI waveforms in the frequency domain and thus
easily satisfy both the constraints given in Section \ref{sec:isi_zero}.

%--------------------------------------
\subsubsection{Nyquist Rate zero-ISI waveform}
\index{Nyquist rate}
%--------------------------------------
The maximum signaling rate is $1/T=2W$ (Nyquist Signaling Theorem).
If we signal at this maximum rate, there is only one waveform $\hhcmpf$
which satisfies the partition of unity condition:
$\hhcmpf(f) = \chi_{[-1/2T,1/2T)}(f)$.
In the time domain this is the sinc function
\[
   \hhcmp(t)=\frac{1}{T} \frac{\sin\left(\frac{\pi}{T}t\right)}{\frac{\pi}{T}t}
\]


\begin{figure}[ht]\color{figcolor}
\begin{center}
\begin{tabular}{ccc}
   \epsfig{file=sinc.eps,    height=4cm, width=4.5cm}  &
   \epsfig{file=pulse.eps,   height=4cm, width=4.5cm}  &
   \epsfig{file=pulse_ss.eps,height=4cm, width=4.5cm}
\\
   $\hhcmp(t)=\frac{1}{T} \frac{\sin\left(\frac{\pi}{T}t\right)}{\frac{\pi}{T}t}$ &
   $\hhcmpf(f)=\chi_{[-1/2T,1/2T)}(f)$ &
   $\sum_n \hhcmpf\left( f-\frac{n}{T} \right)$
\end{tabular}
\end{center}
\caption{
  Nyquist rate zero-ISI signaling waveform
  \label{fig:pulse}
}
\end{figure}




%--------------------------------------
\subsubsection{Raised cosine zero-ISI waveforms}
\index{raised cosine}
%--------------------------------------
\begin{figure}[ht]\color{figcolor}
\begin{center}
\begin{tabular}{cc}
   \epsfig{file=rcos3d.eps,   height=6cm, width=7cm} &
   \epsfig{file=ift_rcos.eps, height=6cm, width=7cm} \\
   $\ft{\hhcmp}(f)$ & $\hhcmp(t)$
\end{tabular}
\caption{
  Raised cosine for various roll-off factors $\beta$
  \label{fig:rcos3d}
}
\end{center}
\end{figure}

\begin{figure}[ht]\color{figcolor}
\setlength{\unitlength}{\tw/600}
\begin{footnotesize}
\begin{center}
\begin{picture}(500,140)(-250,-20)
  %{\color{clr:graphpaper}\graphpaper[10](-250,0)(500,150)}
  \thicklines
  \put(-200,   0){\line(1,0){400} }
  \put(   0, -20){\line(0,1){150} }
  \put( 210,   0){\makebox(0,0)[l]{$f$}}

  {\color{red}
    \put(-50, 100){\line(1,0){100} }
    \qbezier(50,100)(80,100)(100,50)
    \qbezier(100,50)(120,0)(150,0)
    \qbezier(-50,100)(-80,100)(-100,50)
    \qbezier(-100,50)(-120,0)(-150,0)
  }

  \put(  50, -10){\makebox(0,0)[t]{$\frac{1-\beta}{2T}$}}
  \put( 100, -10){\makebox(0,0)[t]{$\frac{1}{2T}$}}
  \put( 150, -10){\makebox(0,0)[t]{$\frac{1+\beta}{2T}$}}
  \put(- 50, -10){\makebox(0,0)[t]{$\frac{1-\beta}{-2T}$}}
  \put(-100, -10){\makebox(0,0)[t]{$\frac{1}{-2T}$}}
  \put(-150, -10){\makebox(0,0)[t]{$\frac{1+\beta}{-2T}$}}

  \put( 5,  50){\makebox(0,0)[l]{$\frac{T}{2}$}}
  \put( 5, 105){\makebox(0,0)[lb]{$T$}}

  %\put( 50, 110){\makebox(0,0)[b]{$\longleftarrow T \longrightarrow$}}
  \put(100, 115){\makebox(0,0)[b]{half cycle}}
  \put(100, 110){\makebox(0,0)[t]{raised cosine}}
  %\put(230, 110){\makebox(0,0)[b]{$\longleftarrow 0 \longrightarrow$}}

  %\put(- 50, 110){\makebox(0,0)[b]{$\longleftarrow T \longrightarrow$}}
  \put(-100, 115){\makebox(0,0)[b]{half cycle}}
  \put(-100, 110){\makebox(0,0)[t]{raised cosine}}
  %\put(-230, 110){\makebox(0,0)[b]{$\longleftarrow 0 \longrightarrow$}}

  \put(-150, -5){\line(0,1){10} }
  \put(-100, -5){\line(0,1){10} }
  \put(- 50, -5){\line(0,1){10} }
  \put( 150, -5){\line(0,1){10} }
  \put( 100, -5){\line(0,1){10} }
  \put(  50, -5){\line(0,1){10} }

  \qbezier[32](- 50,0)(- 50,65)(- 50,130)
  \qbezier[16](-100,0)(-100,25)(-100,50)
  \qbezier[32](-150,0)(-150,65)(-150,130)
  \qbezier[32]( 50,0)( 50,65)( 50,130)
  \qbezier[16](100,0)(100,25)(100,50)
  \qbezier[32](150,0)(150,65)(150,130)
  \qbezier[60](-100,50)(0,50)(100,50)
\end{picture}
\caption{
  Raised cosine
  \label{fig:rcos}
}
\end{center}
\end{footnotesize}
\end{figure}


\begin{figure}[ht]\color{figcolor}
\setlength{\unitlength}{\tw/600}
\begin{footnotesize}
\begin{center}
\begin{picture}(500,140)(-250,-20)
  %\graphpaper[10](-100,0)(500,150)
  \thicklines
  \put(-250,   0){\line(1,0){500} }
  \put(   0, -20){\line(0,1){150} }
  \put( 260,   0){\makebox(0,0)[l]{$f$}}

  \put(  50, -10){\makebox(0,0)[t]{$\frac{1}{4T}$}}
  \put( 100, -10){\makebox(0,0)[t]{$\frac{1}{2T}$}}
  \put( 150, -10){\makebox(0,0)[t]{$\frac{3}{4T}$}}
  \put( 200, -10){\makebox(0,0)[t]{$\frac{1}{T}$}}

  \put(- 50, -10){\makebox(0,0)[t]{$\frac{-1}{4T}$}}
  \put(-100, -10){\makebox(0,0)[t]{$\frac{-1}{2T}$}}
  \put(-150, -10){\makebox(0,0)[t]{$\frac{-3}{4T}$}}
  \put(-200, -10){\makebox(0,0)[t]{$\frac{-1}{T}$}}

  \put( 5,  50){\makebox(0,0)[l]{$\frac{T}{2}$}}
  \put( 5, 105){\makebox(0,0)[lb]{$T$}}


  \put(-200, -5){\line(0,1){10} }
  \put(-150, -5){\line(0,1){10} }
  \put(-100, -5){\line(0,1){10} }
  \put(- 50, -5){\line(0,1){10} }
  \put( 200, -5){\line(0,1){10} }
  \put( 150, -5){\line(0,1){10} }
  \put( 100, -5){\line(0,1){10} }
  \put(  50, -5){\line(0,1){10} }

  \qbezier[60](-100,50)(0,50)(100,50)

  \color[rgb]{0,0,1}
    \put(-100, 100){\line(1,0){200} }
    \put(-100,   0){\line(0,1){100} }
    \put(100,    0){\line(0,1){100} }
    \put(140,   90){\vector(-1,0){40} }
    \put(150,   90){\makebox(0,0)[l]{$\beta=0$}}
  \color{red}
    \put(-50, 100){\line(1,0){100} }
    \qbezier(50,100)(80,100)(100,50)
    \qbezier(100,50)(120,0)(150,0)
    \qbezier(-50,100)(-80,100)(-100,50)
    \qbezier(-100,50)(-120,0)(-150,0)
    \put(140,   70){\vector(-1,0){49} }
    \put(150,   70){\makebox(0,0)[l]{$\beta=\frac{1}{2}$}}
  \color[rgb]{0,0.5,0}
    \qbezier(0,100)(60,100)(100,50)
    \qbezier(100,50)(140,0)(200,0)
    \qbezier(0,100)(-60,100)(-100,50)
    \qbezier(-100,50)(-140,0)(-200,0)
    \put(140,   30){\vector(-1,0){20} }
    \put(150,   30){\makebox(0,0)[l]{$\beta=1$}}
\end{picture}
\caption{
  Raised cosine for various $\beta$ values
  %\label{fig:rcos}
}
\end{center}
\end{footnotesize}
\end{figure}



\begin{figure}[ht]\color{figcolor}
\setlength{\unitlength}{0.1mm}
\begin{footnotesize}
\begin{center}
\begin{picture}(550,200)(-250,-50)
  %\graphpaper[10](-100,0)(500,150)
  \thicklines
  \put(-620,   0){\line(1,0){1240} }
  \put(   0, -20){\line(0,1){150} }
  \put( 640,   0){\makebox(0,0)[l]{$f$}}

  \put( 100, -10){\makebox(0,0)[t]{$\frac{1}{2T}$}}
  \put( 200, -10){\makebox(0,0)[t]{$\frac{1}{T}$}}
  \put( 300, -10){\makebox(0,0)[t]{$\frac{3}{2T}$}}
  \put( 400, -10){\makebox(0,0)[t]{$\frac{2}{ T}$}}
  \put( 500, -10){\makebox(0,0)[t]{$\frac{5}{2T}$}}
  \put( 600, -10){\makebox(0,0)[t]{$\frac{3}{ T}$}}

  \put(-100, -10){\makebox(0,0)[t]{$\frac{-1}{2T}$}}
  \put(-200, -10){\makebox(0,0)[t]{$\frac{-1}{T}$}}
  \put(-300, -10){\makebox(0,0)[t]{$\frac{-3}{2T}$}}
  \put(-400, -10){\makebox(0,0)[t]{$\frac{-2}{ T}$}}
  \put(-500, -10){\makebox(0,0)[t]{$\frac{-5}{2T}$}}
  \put(-600, -10){\makebox(0,0)[t]{$\frac{-3}{ T}$}}

  \put( 5, 105){\makebox(0,0)[lb]{$T$}}

  \multiput(-600,0)(100,0){13}{\put(0, -5){\line(0,1){10} }}

  \qbezier[140](-600,100)(0,100)(600,100)
  \put(550, 50){\makebox(0,0)[t]{sum of shifted raised cosines}}
  \put(550, 60){\vector(0,1){40}}

  \color{red}

  \multiput(-400,0)(200,0){5}
  {
    \put(-50, 100){\line(1,0){100} }
    \qbezier(50,100)(80,100)(100,50)
    \qbezier(100,50)(120,0)(150,0)
    \qbezier(-50,100)(-80,100)(-100,50)
    \qbezier(-100,50)(-120,0)(-150,0)
  }

\end{picture}
\caption{
  Sum of shifted raised cosines
  %\label{fig:rcos}
}
\end{center}
\end{footnotesize}
\end{figure}





The {\bf Raised Cosine} is the Fourier Transform of
one of the most widely used signaling waveforms.\footnote{
  Note: The raised cosine is similar to the {\em Meyer wavelet}.
  ref: \cite[page 65]{vidakovic}
  }
In the frequency domain it has the form\footnote{\citerp{proakis}{560}}
\[
   \hhcmpf(f) =
   \left\{
      \begin{array}{lclcccl}
         T
         &:&
          0 &\le& |f| &\le& \frac{1-\beta}{2T}
         \\
         \ds \frac{T}{2}
           \left[ 1 + \cos\left(
               \frac{\pi T}{\beta}\left[ |f|-\frac{1-\beta}{2T}\right]
             \right)
           \right]
         &:&
         \frac{1-\beta}{2T} &\le& |f| &\le& \frac{1+\beta}{2T}
         \\
         0
         &:&
          && |f| &>& \frac{1+\beta}{2T}
      \end{array}
   \right.
\]

The value $\beta\in[0,1]$ is the {\em roll-off factor}.
The raised cosine for various roll-off factors $\beta$ is
illustrated in Figure \ref{fig:rcos3d}.

Shifted versions of $\hhcmpf(f)$ sum to unity because the
cosine regions sum to unity:
\begin{eqnarray*}
  \frac{1}{2}[1 + \cos(\theta)] + \frac{1}{2}[1 + \cos(\theta+\pi)]
    &=& \frac{1}{2}[1 + \cos(\theta)] + \frac{1}{2}[1 - \cos(\theta)]
     = 1
\end{eqnarray*}

The inverse Fourier transform of the raised cosine filter is
illustrated in Figure \ref{fig:rcos3d}.
These waveforms are the signaling waveforms $\hhcmp$.
Notice how they becoming smoother in frequency
but wider in time with increasing $\beta$;





%--------------------------------------
\subsubsection{B-Spline zero-ISI waveforms}
\index{B-splines}
%--------------------------------------
B-Splines are formed by repeatedly convolving the $\chi$ function with itself.
%--------------------------------------
\begin{definition}
\label{def:Bspline}
%--------------------------------------
A {\bf B-spline} $\beta_m(f)$ of order $m$ is the
characteristic function $\theta=\chi(f)_{[-1/2T,1/2T)}$
convolved with itself $m$ times.
That is, if $\ast$ is the convolution operation, then
\begin{align*}
   \beta_0 &\eqd \theta  \\
   \beta_1 &\eqd \theta\ast\theta           &=& \beta_0\ast\theta\\
   \beta_2 &\eqd \theta\ast\theta\ast\theta &=& \beta_1\ast\theta \\
   \beta_3 &\eqd \theta\ast\theta\ast\theta\ast\theta &=& \beta_2\ast\theta \\
           & \vdots
\end{align*}
\end{definition}

\begin{figure}[ht]\color{figcolor}
\begin{center}
\begin{tabular}{ccccccccc}
   \epsfig{file=sp_0_s.eps,    height=3cm, width=3cm}  &
   \epsfig{file=sp_1_s.eps,    height=3cm, width=3cm}  &
   \epsfig{file=sp_2_s.eps,    height=3cm, width=3cm}  &
   \epsfig{file=sp_3_s.eps,    height=3cm, width=3cm}
\\
   $\beta_0$ &
   $\beta_1$ &
   $\beta_2$ &
   $\beta_3$
\end{tabular}
\end{center}
\caption{
  B-Splines of order 0,1,2,3
  \label{fig:Bspline}
}
\end{figure}

All B-Splines form a partition of unity.\footnote{\citerp{goswami}{46}}
and their inverse Fourier Transforms may therefore be used
as signaling waveforms $\hhcmp(t)$.
%--------------------------------------
\begin{theorem}
%--------------------------------------
All B-Splines $\beta_m$ of order $m\in\{0,1,2,\ldots\}$
form a partition of unity.
\end{theorem}
\begin{proof}
\begin{enume}
   \item A B-Spline $\ft{\beta}_m$ of order $m$
         is the $\chi$ function convolved with itself $m$ times.
   \item This implies that the inverse Fourier Transform $\beta_m$ is
         \[ \beta_m(t) =
             \left[
             \frac{2}{T}
             \frac{\sin\left(\frac{2\pi}{T}t\right)}{\frac{2\pi}{T}t}
             \right]^{m+1}
         \]
   \item This equation satisfies the Partition of Unity criterion (Theorem \ref{thm:unity}).
         \[ \beta_m(nT) =
             \left[
             \frac{2}{T}
             \frac{\sin\left(2\pi n\right)}{2\pi n}
             \right]^{m+1}
             = \left\{
                  \begin{tabular}{lcl}
                     $(2/T)^m$ & : & $n=0$ \\
                     $0$       & : & $n\neq0$
                  \end{tabular}
               \right.
         \]
   \item Therefore, $\beta_m$ forms a partition of unity for all $m=0,1,2,\ldots$.
\end{enume}
\end{proof}

Because $\beta_m$ form a partition of unity,
we can use their inverse Fourier transforms as signaling waveforms $\hhcmp_m$.
That is, if $\hhcmpf_m=\beta_m$ then
\[ \hhcmp_m \eqd \opFTi{\hhcmpf_m} \eqd \opFTi{\beta_m} =
             \left[
             \frac{2}{T}
             \frac{\sin\left(\frac{2\pi}{T}t\right)}{\frac{2\pi}{T}t}
             \right]^{m+1}
\]
are valid signaling waveforms.

\begin{figure}[ht]\color{figcolor}
\begin{center}
\begin{tabular}{ccccccccc}
   \epsfig{file=sp_0_s.eps,    height=3cm, width=3cm}  &
   \epsfig{file=sp_1_s.eps,    height=3cm, width=3cm}  &
   \epsfig{file=sp_2_s.eps,    height=3cm, width=3cm}  &
   \epsfig{file=sp_3_s.eps,    height=3cm, width=3cm}
\\
   $\hhcmpf_0=\beta_0$ &
   $\hhcmpf_1=\beta_1$ &
   $\hhcmpf_2=\beta_2$ &
   $\hhcmpf_3=\beta_3$
\\
   \epsfig{file=sinc_p1.eps,    height=3cm, width=3cm}  &
   \epsfig{file=sinc_p2.eps,    height=3cm, width=3cm}  &
   \epsfig{file=sinc_p3.eps,    height=3cm, width=3cm}  &
   \epsfig{file=sinc_p4.eps,    height=3cm, width=3cm}  &
\\
   $\hhcmp_0$ &
   $\hhcmp_1$ &
   $\hhcmp_2$ &
   $\hhcmp_3$
\end{tabular}
\end{center}
\caption{
  B-Splines signaling waveforms in frequency and time domains
  \label{fig:Bspline_ft}
}
\end{figure}

%--------------------------------------
\subsubsection{Wavelet scaling function zero-ISI waveforms}
\index{wavelets}
\index{wavelets!scaling functions}
%--------------------------------------
Wavelets consists of two families of functions:
the \hie{scaling functions} $\phi_{m,n}(t)$ and
the \hie{wavelet functions} $\psi_{m,n}(t)$.
Each member of the family may be scaled by $2^m$
and translated by $n$.
There are many scaling and wavelet functions available.
Most scaling functions $\phi$ satisfy
the partition of unity criterion\footnote{\citerpc{jawerth}{8}{???}}.  \attention
The inverse Fourier Transform of scaling functions may therefore
be used as signaling waveforms.
%Some Daubechies scaling functions are shown in \prefpp{fig:dau}.
%
%\begin{figure}[ht]\color{figcolor}
%\begin{center}
%\begin{tabular}{ccccccccc}
%   %20171226%\epsfig{file=../common/20090604/common/wavelets/d1sc_x3125.eps,    height=3cm, width=3cm}  &
%   %20171226%\epsfig{file=../common/20090604/common/wavelets/d2sc_x3125.eps,    height=3cm, width=3cm}  &
%   %20171226%\epsfig{file=../common/20090604/common/wavelets/d3sc_x3125.eps,    height=3cm, width=3cm}  &
%   %20171226%\epsfig{file=../common/20090604/common/wavelets/d4sc_x3125.eps,    height=3cm, width=3cm}
%\\
%   Daubechies-2  &
%   Daubechies-4  &
%   Daubechies-8  &
%   Daubechies-16
%\end{tabular}
%\end{center}
%\caption{
%  Daubechies wavelets
%  \label{fig:dau}
%}
%\end{figure}

One advantage of using wavelet zero-ISI waveforms is that a {\em fast wavelet transform}
(FWT) is available requiring only order $\log n$ operations,
even faster than the fast fourier transform's $n\log n$ operations.
The availability of the FWT in addition to the wavelet's natural
signal analysis capability, may allow the system to make further use of the
incoming waveforms for channel estimation, channel equalization, and
symbol detection.

%=======================================
\section{Duobinary ISI solution}
\index{duobinary}
\label{sec:isi_duo}
\label{sec:duobinary}
%=======================================
%--------------------------------------
\subsection{Constraints}
%--------------------------------------
The received waveform $r(t)$ is of the form
\begin{eqnarray*}
   r(t) &=& \sum_m a_m \hhcmp(t-mT).
\end{eqnarray*}

At sampling instants $t=nT$, $r(t)$ has the form

\begin{eqnarray*}
   r(nT)
     &=& \left. r(t)                               \right|_{t=nT}
   \\&=& \sum_m a_m \hhcmp(nT-mT)
   \\&=& \left. a_m\hhcmp(nT-mT)\right|_{m=n}     +
         \left. a_m\hhcmp(nT-mT)\right|_{m=n-1}   +
         \sum_{m\ne n,n-1} a_m \hhcmp(nT-mT)
   \\&=& a_n\hhcmp(nT-nT) + a_{n-1}\hhcmp(nT-(n-1)T) + \sum_{m\ne n,n-1} a_m \hhcmp(nT-mT)
   \\&=& a_n\hhcmp(0) +
         a_{n-1}\hhcmp(T) +
         \sum_{m\ne n,n-1} a_m \hhcmp(nT-mT)
\end{eqnarray*}

We place the following constraints on the signaling waveform $\hhcmp(t)$:
\[
\begin{array}{llrcl}
   \circOne & \mbox{\bf sampling constraint:}
      & \hhcmp(nT) &=&
          \left\{\begin{array}{l@{\hspace{4ex}}l}
            1 &   \mbox{for } n=0,1 \\
            0 &   \mbox{otherwise}
          \end{array}\right.
\\
   \circTwo & \mbox{\bf bandwidth constraint:}
      & [\opFT{\hhcmp}](f) &=& 0 \mbox{ for } |f|\ge W.
\end{array}
\]

These two constraints are in conflict with each other.
However, they are both satisfied if the criterion
in \prefpp{thm:db_zeroISI} is met.

%--------------------------------------
\subsection{Criterion}
%--------------------------------------
%--------------------------------------
\begin{theorem}
\label{thm:db_zeroISI}
%--------------------------------------
Let $\ft{\hhcmp}(f)$ be the Fourier Transform of a function
$\hhcmp(t)$ and $T\in\R$ a constant.
Then
\thmbox{
   \brs{
   \hhcmp(nT) =
   \left\{
      \begin{tabular}{lcl}
         1 & : & $n=0,1$ \\
         0 & : & otherwise
      \end{tabular}
   \right.
   }
\iff
   \brs{
   \frac{1}{T}\sum_n \ft{\hhcmp}\left(f+ \frac{n}{T} \right)
     = 2e^{-i\pi fT}\cos\left(\pi fT\right).
   }
}

\end{theorem}
\begin{proof}
This theorem is easily proven using the
{\em Inverse Poisson's Summation Formula}(IPSF)
(\prefp{thm:ipsf})
which states
\begin{eqnarray*}
   \sum_n \ft{\hhcmp}\left( f+ \frac{n}{T} \right)
      &=& T \sum_n \hhcmp(nT) e^{-i2\pi fnT}
\end{eqnarray*}

\begin{enumerate}
\item Prove ``only if" case ($\implies$):

\begin{align*}
  \sum_n \ft{\hhcmp}\left( f + \frac{n}{T} \right)
    &= T\sum_n \hhcmp(nT) e^{-i2\pi fnT}
    && \text{by IPSF}
  \\&= T\left[1 + e^{-i2\pi fT}\right]
    && \text{by left hypothesis}
  \\&= 2Te^{-i\pi fT}\left( \frac{1}{2}e^{i\pi fT} + \frac{1}{2}e^{-i\pi fT}  \right)
  \\&= 2Te^{-i\pi fT}\cos\left(\pi fT\right)
    && \text{by \thme{Euler formulas} \prefp{cor:eform}}
\end{align*}


\item Prove ``if" case ($\impliedby$):

\begin{align*}
     2e^{-i\pi fT}\cos\left(\pi fT\right)
       &=  \frac{1}{T}\sum_n \ft{\hhcmp}\left( f+\frac{n}{T} \right)
       && \text{by right hypothesis}
     \\&= \frac{1}{T}T \sum_n \hhcmp(nT) e^{-i2\pi fnT}
       && \text{by IPSF}
     \\&= 2e^{-i\pi fT} \sum_n \hhcmp(nT) \frac{1}{2}e^{i\pi fT}e^{-i2\pi fnT}
     \\&= 2e^{-i\pi fT} \sum_n \hhcmp(nT) \frac{1}{2}e^{-i\pi fT(2n-1)}
     \\&= 2e^{-i\pi fT}
             \left[
               \hhcmp(0) \frac{1}{2}e^{i\pi fT} +
               \hhcmp(T) \frac{1}{2}e^{-i\pi fT} +
               \sum_{n\ne 0,1} \hhcmp(nT) \frac{1}{2}e^{-i\pi fT(2n-1)}
             \right]
     \\&\implies
\\
   \hhcmp(nT)
     &= \left\{\begin{array}{lcl}
           1 & : & n=0,1    \\
           0 & : & \text{otherwise}
        \end{array}\right.
       && \text{because $\cos(\pi fT)$ has no imaginary part}
\end{align*}
\end{enumerate}
\end{proof}

%--------------------------------------
\subsection{Signaling waveform}
%--------------------------------------
\begin{figure}[ht]\color{figcolor}
\begin{center}
\begin{fsL}
\setlength{\unitlength}{0.2mm}
\begin{picture}(400,120)(-200,-50)
  %\graphpaper[10](-100,0)(500,150)
  \thicklines
  \put(-150 ,   0 ){\line(1,0){300} }
  \put(   0 , -20 ){\line(0,1){100} }
  \put( 160 ,   0 ){$f$}
  \put(  20 ,  60 ){$\ds|\ft{\hhcmp}(f)|=\frac{1}{W}\cos\left(\frac{\pi}{2W}f\right)\chi_B(f)$}
  \put( -20 ,  60 ){$\frac{1}{W}$}
  \put(-200 , -50 ){\makebox(200,40)[t]{$-W=-\frac{1}{2T}$}}
  \put(   0 , -50 ){\makebox(200,40)[t]{$W=\frac{1}{2T}$}}
  \color{red}
    \qbezier(-100,   0)(   0, 100)( 100,   0)
\end{picture}
\end{fsL}
\end{center}
\caption{
  Duobinary waveform $\ft{\hhcmp}(f)$ at Nyquist rate
   \label{fig:db_zeroISI}
   }
\end{figure}

The next theorem specifies a signaling waveform which satisfies the
criterion at the Nyquist rate
\[ W  = \frac{1}{2T}. \]
Unlike the zero-ISI Nyquist rate signaling waveform
(\prefpp{fig:pulse}),
the duobinary Nyquist rate signaling waveform
(\prefpp{fig:db_zeroISI})
can be easily approximated in real systems.
%--------------------------------------
\begin{theorem}
%--------------------------------------
The waveform $\hhcmp(t)$ with Fourier transform $\ft{\hhcmp}(f)$
(see \prefpp{fig:db_zeroISI})
satisfies the criterion stated in \prefpp{thm:db_zeroISI},
where

\begin{eqnarray*}
   \ft{\hhcmp}(f) &=& \left\{
   \begin{tabular}{ll}
      $\ds 2Te^{-i\pi Tf}\cos(\pi Tf)$ & : $\frac{-1}{2T} \le f < \frac{1}{2T}$ \\
      $\ds 0$ & : otherwise
   \end{tabular}
   \right.
\end{eqnarray*}

\begin{eqnarray*}
   \hhcmp(t)
     &=&    \frac{\sin\left[\frac{\pi}{T}t\right]}{\frac{\pi}{T}t} +
            \frac{\sin\left[\frac{\pi}{T}(t-T)\right]}{\frac{\pi}{T}(t-T)}
   \\&\eqd& \sinc{\frac{\pi}{T}t} + \sinc{\frac{\pi}{T}(t-T)}
\end{eqnarray*}
\end{theorem}

\begin{proof}
Let $B=[-1/2T, +1/2T)$ such that
\[
\chi_B(f) \eqd
\left\{
\begin{tabular}{llll}
   1 & : $f\in [-1/2T, +1/2T)$ \\
   0 & : otherwise.
\end{tabular}
\right.
\]

Then
First, observe that $\ft{\hhcmp}(f)$ satisfies
the criterion of \prefpp{thm:db_zeroISI}:

\begin{align*}
   \sum_n \ft{\hhcmp}\left( f + \frac{n}{T} \right)
     &= \sum_n 2T e^{-i\pi T\left(f+\frac{n}{T}\right)}
        \cos\left[\pi T(f+\frac{n}{T})\right]
        \chi_B\left(f+\frac{n}{T}\right)
   \\&= 2T \sum_n
        e^{-i\pi Tf}
        e^{-i\pi n }
        \left[\cos(\pi Tf)\cos(\pi n)-\sin(\pi Tf)\sin(\pi n) \right]
        \chi_B\left(f+\frac{n}{T}\right)
   \\&= 2T e^{-i\pi Tf} \sum_n
        (-1)^n
        \left[\cos(\pi Tf)(-1)^n -\sin(\pi Tf)\cdot0 \right]
        \chi_B\left(f+\frac{n}{T}\right)
   \\&= 2T e^{-i\pi Tf} \sum_n
        \cos(\pi Tf)
        \chi_B\left(f+\frac{n}{T}\right)
   \\&= 2T e^{-i\pi Tf} \cos(\pi Tf) \sum_n
        \chi_B\left(f+\frac{n}{T}\right)
   \\&= 2T e^{-i\pi Tf} \cos(\pi Tf)
   %\\&\iff& Equation~\ref{eq:db_II}.
\end{align*}

The signaling waveform $\hhcmp(t)$ can be found by taking
the inverse Fourier Transform of $\ft{\hhcmp}(f)$:

\begin{align*}
   \hhcmp(t)
     &= [\opFTi\ft{\hhcmp}](t)
   \\&= \int_f \hhcmp(f) e^{i2\pi ft} \df
   \\&= \int_\frac{-1}{2T}^\frac{1}{2T}
        2T e^{-i\pi Tf} \cos(\pi Tf) e^{i2\pi ft} \df
   \\&= 2T\int_\frac{-1}{2T}^\frac{1}{2T}
        e^{-i\pi Tf}
        \frac{1}{2}\left[ e^{i\pi Tf} + e^{-i\pi Tf}  \right]
        e^{i2\pi ft} \df
   \\&= T\int_\frac{-1}{2T}^\frac{1}{2T}
        \left[ 1 + e^{-i2\pi Tf}  \right]
        e^{i2\pi ft} \df
   \\&= T\int_\frac{-1}{2T}^\frac{1}{2T}
        e^{i2\pi ft} + e^{i2\pi(t-T)f} \df
   \\&= \left. T\frac{e^{i2\pi ft    }}{i2\pi t    } \right|_\frac{-1}{2T}^\frac{1}{2T}  +
        \left. T\frac{e^{i2\pi f(t-T)}}{i2\pi (t-T)} \right|_\frac{-1}{2T}^\frac{1}{2T}
   \\&= \frac{e^{i\frac{\pi}{T}t    }-e^{-i\frac{\pi}{T}t    }}{i2\frac{\pi}{T} t    }   +
        \frac{e^{i\frac{\pi}{T}(t-T)}-e^{-i\frac{\pi}{T}(t-T)}}{i2\frac{\pi}{T} (t-T)}
   \\&= \frac{\sin[\frac{\pi}{T}t]    }{\frac{\pi}{T} t    }   +
        \frac{\sin[\frac{\pi}{T}(t-T)]}{\frac{\pi}{T} (t-T)}
\end{align*}
\end{proof}

%--------------------------------------
\subsection{Detection}
%--------------------------------------
\begin{figure}[ht]\color{figcolor}
\begin{center}
\begin{fsK}
\setlength{\unitlength}{0.15mm}
\begin{picture}(750,200)(-50,-150)
  \thicklines
  %\graphpaper[10](0,0)(700,100)

  \put(-50,   0 ){\vector  (   1,  0)   {90}          }
  \put(   0,-150 ){\dashbox{4}( 200,200)   {  }           }
  \put(   0,-140 ){\makebox ( 200,200)[b]{precoder}     }
  \put(   0, -50 ){\makebox ( 100,100)   {$+$}          }
  \put(   0,   5 ){\makebox (  50,  0)[b]{$+$}          }
  \put(   0, -50 ){\makebox (  45, 40)[r]{$-$}          }
  \put(  50,   0 ){\circle{20}                          }
  \put(  50,-100 ){\vector  (   0,  1)   { 90}          }
  \put(  60,   0 ){\line    (   1,  0)   { 10}          }
  \put(  90,   0 ){\line    (   1,  0)   {110}          }
  \put(  70, -10 ){\framebox(  20, 20)   {   }          }
  \put(   0,   0 ){\makebox ( 160, 50)   {mod M}        }
  \put(  80, -70 ){\framebox(  40, 40)   {$z^{-1}$}     }
  \put( 100,-100 ){\line    (   0,  1)   { 30}          }
  \put( 100, -30 ){\line    (   0,  1)   { 30}          }
  \put(  50,-100 ){\line    (   1,  0)   { 50}          }

  \put( 200,    0 ){\vector  (   1,  0)   {50}           }
  \put( 250,  -50 ){\framebox( 100,100)   {mapping}      }
  \put( 350,    0 ){\vector  (   1,  0)   {50}           }
  \put( 400,  -50 ){\framebox( 100,100)   {modulator}    }

  \put( 550,    0){\usebox{\picAWGN}}

  \put(- 50,   10 ){\makebox (  50, 50)[b]{$u_n$}        }
  \put( 200,   10 ){\makebox (  50, 50)[b]{$y_n$}        }
  \put( 350,   10 ){\makebox (  50, 50)[b]{$a_n$}        }
  \put( 500,   10 ){\makebox (  50, 50)[b]{$s(t)$}       }
  \put( 650,   10 ){\makebox (  50, 50)[b]{$r(t)$}       }
\end{picture}
\end{fsK}
\end{center}
\caption{
   Duobinary Detection Model
   \label{fig:db_detect}
   }
\end{figure}

Detection of a received signal using duobinary modulation
presents a special problem because each received
symbol at time period $n$
is a function of both the time $n$ and $n-1$ transmitted symbols
(has single symbol ISI).
In this case and if channel noise is zero,
detection can still be performed without error using
the algorithm described below and illustrated in
\prefpp{fig:db_detect}.

%---------------------------------------
\begin{lemma}
%---------------------------------------
\lembox{ (a+b)\mod = (a\mod + b\mod)\mod}
\end{lemma}
\begin{proof}
\[
\begin{array}{rclcrcl}
  a       &=& Mq_1 + r_1   &\iff& a        \mod &=& r_1   \\
  b       &=& Mq_2 + r_2   &\iff& b        \mod &=& r_2   \\
  a+b     &=& Mq_3 + r_3   &\iff& (a+b)    \mod &=& r_3   \\
  r_1+r_2 &=& Mq_4 + r_4   &\iff& (r_1+r_2)\mod &=& r_4
\end{array}
\]
%\begin{eqnarray*}
%%  (a+b) \mod
%%    &=& (Mq_1+r_1 + Mq_2 + r_2)\mod
%%  \\&=& [M(q_1+q_2) +r_1 + r_2]\mod
%%  \\&=& [M(q_1+q_2) +Mq_4+r_4]\mod
%%  \\&=& [M(q_1+q_2+q_4) +r_4]\mod
%%  \\&=& r_4
%%  \\&=& (r_1 + r_2)\mod
%%  \\&=& (a\mod + b\mod)\mod
%\end{eqnarray*}
\end{proof}

%---------------------------------------
\begin{theorem}
%---------------------------------------
Let $u_n\in\{0,1,\ldots,M-1\}$ be the data transmitted
using {\em duobinary} symbol signaling.
Let
\begin{eqnarray*}
   \fr(t)&\eqd& \fs(t;u) + \fn(t)               \\
   r_n &\eqd& \left. r(t)\right|_{t=nT} = r(nT) \\
   y_n &\eqd& (u_n - y_{n-1})\mod               \\
   a_n &\eqd& 2y_n -M + 1                       \\
   n_n &\eqd& \left. n(t)\right|_{t=nT} = n(nT) \\
   S_n   &\eqd& \sum_{k=-\infty}^n (-1)^{n-k} u_k.
\end{eqnarray*}
Then
\thmbox{
  r_n|u_n,S_{n-1} =
  2\Big[ [u_n\mod + (- S_{n-1})\mod]\mod + S_{n-1}\mod - (M - 1) \Big] + n_n
  }

If $\fn(t)$ is a white Gaussian random process, then
\thmbox{
  r_n \sim \pN
    {2\Big[ [u_n\mod + (- S_{n-1})\mod]\mod + S_{n-1}\mod - (M - 1) \Big]}
    {\sigma^2}
  }
\end{theorem}
\begin{proof}

The sequence $\{y_n\}$ is the precoded sequence:
\begin{align*}
   y_n
     &= \left( u_n - y_{n-1} \right)\mod
   \\&= \left[ u_n - (u_{n-1}-y_{n-2}) \right]\mod
   \\&= \left( u_n - u_{n-1} + u_{n-2}-y_{n-3} \right)\mod
   \\&= \left( u_n - u_{n-1} + u_{n-2}-u_{n-3}+y_{n-4} \right)\mod
   \\&= \left( \sum_{k=-\infty}^n (-1)^{n-k} u_k \right)\mod
   \\&= S_n \mod
\end{align*}

A mapping is performed on each $y_n$ to produce $a_n$:
\[ a_n = 2y_n - M + 1.\]

The modulator uses the duobinary signaling waveform $\hhcmp(t)$
and $a_n$ to produce the transmitted signal $s(t)$ at signaling
rate $1/T$:
\[ s(t) = \sum_n a_n \hhcmp(t-nT). \]


Before going further, here is a useful relation:
\begin{align*}
  S_n
    &\eqd \sum_{k=-\infty}^{n} (-1)^{n-k} u_k
  \\&=    u_n + \sum_{k=-\infty}^{n-1} (-1)^{n-k} u_k
  \\&=    u_n - \sum_{k=-\infty}^{n-1} (-1)(-1)^{n-k} u_k
  \\&=    u_n - \sum_{k=-\infty}^{n-1} (-1)^{-1}(-1)^{n-k} u_k
  \\&=    u_n - \sum_{k=-\infty}^{n-1} (-1)^{n-1-k} u_k
  \\&\eqd u_n - S_{n-1}
\end{align*}


The received signal samples $\fr_n$ are as follows:
\begin{align*}
   \fr_n
     &= \left. \fr(t) \right|_{t=nT}
   \\&= \left[ \fs(t) + n(t) \right]_{t=nT}
   \\&= \left[ \sum_m a_n \hhcmp(t-mT) + n(t) \right]_{t=nT}
   \\&= \sum_m a_m \hhcmp(nT-mT) + n(nT)
   \\&= a_n\hhcmp(0) + a_{n-1}\hhcmp(T) + n_n
   \\&= a_n + a_{n-1} + n_n
   \\&= \left( 2y_n - M + 1 \right) + \left(2y_{n-1} - M + 1 \right) + n_n
   \\&= 2\left( y_n + y_{n-1} - M + 1 \right) + n_n
   \\&= 2\left[
          \left( \sum_{k=-\infty}^n (-1)^{n-k} u_k \right)\mod +
          \left( \sum_{k=-\infty}^{n-1} (-1)^{n-1-k} u_k \right)\mod
          - M + 1 \right] + n_n
   \\&= 2\Big[ S_n\mod + S_{n-1}\mod - M + 1 \Big] + n_n
   \\&= 2\Big[ (u_n - S_{n-1})\mod + S_{n-1}\mod - (M - 1) \Big] + n_n
   \\&= 2\Big[ [u_n\mod + (- S_{n-1})\mod]\mod + S_{n-1}\mod - (M - 1) \Big] + n_n
\end{align*}

Thus, $(r_n|u_n,S_{n-1})$ have Gaussian distribution with means
\[ \Eb{r_n|u_n,S_n} =  2\left[
           \left( u_n + S_{n-1} \right)\mod + \left( M   - S_{n-1}\right)\mod
           - (M - 1) \right]. \]
\end{proof}

That is the good news.
The bad news is that in general we don't know $S_n$.
However, the additional good news is that it doesn't matter what
$S_{n-1}$ is because the values $\Eb{r_n|u_n}$ are
always distinct from the values $\Eb{r_m|v_m}$
if $u_n\ne v_n$.
That is
\begin{align*}
   & (u_n\ne v_n)\implies
 \\&
   \Eb{r_n|u_n,S_{n-1}} \ne \Eb{r_n|v_n,S_{n-1}}
   && \forall S_{n-1}
\end{align*}

For ML optimization, we are interested in the distributions
$\pp(r_n|u_n)$.
However, what we conveniently have is $\pp(r_n|u_n,S_{n-1})$.
If we assume that all values of $S_{n-1}\in\{0,1,\ldots,M-1\}$
are equally likely, we can convert from the latter to the
former by the relation:

\begin{align*}
   \pp(r_n|u_n)
      &= \frac{\pp(r_n,u_n)}{\pp(u_n)}
    \\&= \frac{\pp(u_n|r_n)\pp(r_n)}
              {\pp(u_n)}
    \\&= \frac{\pp(u_n|r_n)\pp(r_n)}
              {\pp(u_n)}
    \\&= \frac{\sum_{s=0}^{M-1}\pp(u_n,S_{n-1}=s|r_n)\pp(r_n)}
              {\pp(u_n)}
    \\&= \frac{\sum_{s=0}^{M-1}\pp(r_n|u_n,S_{n-1}=s)\pp(r_n)\pp(u_n,S_{n-1})}
              {\pp(u_n)\pp(r_n)}
    \\&= \frac{\sum_{s=0}^{M-1}\pp(r_n|u_n,S_{n-1}=s)\pp(u_n)\pp(S_{n-1})}
              {\pp(u_n)}
    \\&= \sum_{m=0}^{M-1}\pp(r_n|u_n,S_{n-1}=m)\pp(S_{n-1})
    \\&= \frac{1}{M}\sum_{m=0}^{M-1}\pp(r_n|u_n,S_{n-1}=m)
\end{align*}

%--------------------------------------
\subsubsection{Detection in the case $M=2$}
%--------------------------------------
For the case $M=2$, we have the following mean values:
\[
\begin{array}{cc|rrr}
   u_n & S_{n-1}\mod[2] & \mc{3}{c}{\Eb{r_n|u_n,S_{n-1}}} \\
   \hline
   \hline
   0 & 0 && -2 \\
   0 & 1 &&  2 \\
   1 & 0 &&  0 \\
   1 & 1 &&  0
\end{array}
\]

This gives distributions  (see \prefpp{fig:db_pdf_M=2})
\begin{eqnarray*}
   (r_n|u_n=0) &\sim& \frac{1}{2}\pN{-2}{\sigma^2}+
                      \frac{1}{2}\pN{ 2}{\sigma^2} \\
   (r_n|u_n=1) &\sim&            \pN{ 0}{\sigma^2}.
\end{eqnarray*}

\begin{figure}[ht]\color{figcolor}
\begin{center}
\begin{fsL}
\setlength{\unitlength}{0.2mm}
\begin{picture}(440,160)(-200,-30)
  %\graphpaper[10](-100,0)(500,150)
  \thicklines
  \put( 220 ,   0 ){$r_n$}
  \put(-200 ,   0 ){\line(1,0){400} }
  \put(   0 , -10 ){\line(0,1){150} }

  \put( -50 , -10 ){\line(0,-1){40} }
  \put(  50 , -10 ){\line(0,-1){40} }

  \put(-100 , -30 ){$D_0$ }
  \put( 100 , -30 ){$D_0$ }
  \put(   0 , -30 ){$D_1$ }

  %\qbezier[12]( 60,  0)( 60, 30)( 60, 60)
  %\qbezier[12](  0, 60)( 30, 60)( 60, 60)

  \color{red}

  \qbezier(-200,   0)(-160,   0)(-140,  30)  % left   side
  \qbezier(-140,  30)(-100,  90)( -60,  30)  % center at -100
  \qbezier( -60,  30)( -40,   0)(   0,   0)  % right  side

  \qbezier(-100,   0)( -60,   0)( -40,  60)  % left   side
  \qbezier( -40,  60)(   0, 180)(  40,  60)  % center at    0
  \qbezier(  40,  60)(  60,   0)( 100,   0)  % right  side

  \qbezier(  00,   0)(  40,   0)(  60,  30)  % left   side
  \qbezier(  60,  30)( 100,  90)( 140,  30)  % center at  100
  \qbezier( 140,  30)( 160,   0)( 200,   0)  % right  side
\end{picture}
\end{fsL}
\end{center}
\caption{
  Duobinary receiver distributions for $M=2$
   \label{fig:db_pdf_M=2}
   }
\end{figure}

%--------------------------------------
\subsubsection{Detection in the case $M=4$}
%--------------------------------------
For the case $M=4$, we have the following mean values:
\[
\begin{array}{cc|rrr}
   u_n & S_{n-1}\mod[4] & \mc{3}{c}{\Eb{r_n|u_n,S_{n-1}}} \\
   \hline
   \hline
   0 & 0 && -6 \\
   0 & 1 &&  2 \\
   0 & 2 &&  2 \\
   0 & 3 &&  2 \\
\hline
   1 & 0 && -4 \\
   1 & 1 && -4 \\
   1 & 2 &&  4 \\
   1 & 3 &&  4 \\
\hline
   2 & 0 && -2 \\
   2 & 1 && -2 \\
   2 & 2 && -2 \\
   2 & 3 &&  6 \\
\hline
   3 & 0 &&  0 \\
   3 & 1 &&  0 \\
   3 & 2 &&  0 \\
   3 & 3 &&  0
\end{array}
\]

This gives distributions (see \prefpp{fig:db_pdf_M=4})
\begin{eqnarray*}
   (r_n|u_n=0) &\sim& \frac{1}{4}\pN{-6}{\sigma^2}+\frac{3}{4}\pN{2}{\sigma^2} \\
   (r_n|u_n=1) &\sim& \frac{1}{2}\pN{-4}{\sigma^2}+\frac{1}{2}\pN{4}{\sigma^2} \\
   (r_n|u_n=2) &\sim& \frac{1}{4}\pN{ 6}{\sigma^2}+\frac{3}{4}\pN{-2}{\sigma^2} \\
   (r_n|u_n=3) &\sim& \pN{ 0}{\sigma^2}.
\end{eqnarray*}


\begin{figure}[ht]\color{figcolor}
\begin{center}
\begin{fsL}
\setlength{\unitlength}{0.15mm}
\begin{picture}(840,160)(-400,-30)
  %\graphpaper[10](-100,0)(500,150)
  \thicklines
  \put( 420 ,   0 ){$r_n$}
  \put(-400 ,   0 ){\line(1,0){800} }
  \put(   0 , -10 ){\line(0,1){150} }

  \put(-250 , -10 ){\line(0,-1){40} }
  \put(-150 , -10 ){\line(0,-1){40} }
  \put( -50 , -10 ){\line(0,-1){40} }
  \put(  50 , -10 ){\line(0,-1){40} }
  \put( 150 , -10 ){\line(0,-1){40} }
  \put( 250 , -10 ){\line(0,-1){40} }

  \put(-300 , -30 ){$D_0$ }
  \put(-200 , -30 ){$D_1$ }
  \put(-100 , -30 ){$D_2$ }
  \put(   0 , -30 ){$D_3$ }
  \put( 100 , -30 ){$D_0$ }
  \put( 200 , -30 ){$D_1$ }
  \put( 300 , -30 ){$D_2$ }

  \color{red}

  %\qbezier[12]( 60,  0)( 60, 30)( 60, 60)
  %\qbezier[12](  0, 60)( 30, 60)( 60, 60)

  \qbezier(-400,   0)(-360,   0)(-340,  15)  % left   side
  \qbezier(-340,  15)(-300,  45)(-260,  15)  % center at -300
  \qbezier(-260,  15)(-240,   0)(-200,   0)  % right  side

  \qbezier(-300,   0)(-260,   0)(-240,  15)  % left   side
  \qbezier(-240,  15)(-200,  45)(-160,  15)  % center at -200
  \qbezier(-160,  15)(-140,   0)(-100,   0)  % right  side

  \qbezier(-200,   0)(-160,   0)(-140,  45)  % left   side
  \qbezier(-140,  45)(-100, 135)( -60,  45)  % center at -100
  \qbezier( -60,  45)( -40,   0)(   0,   0)  % right  side

  \qbezier(-100,   0)( -60,   0)( -40,  60)  % left   side
  \qbezier( -40,  60)(   0, 180)(  40,  60)  % center at    0
  \qbezier(  40,  60)(  60,   0)( 100,   0)  % right  side

  \qbezier(   0,   0)(  40,   0)(  60,  45)  % left   side
  \qbezier(  60,  45)( 100, 135)( 140,  45)  % center at  100
  \qbezier( 140,  45)( 160,   0)( 200,   0)  % right  side

  \qbezier( 100,   0)( 140,   0)( 160,  45)  % left   side
  \qbezier( 160,  45)( 200, 135)( 240,  45)  % center at  200
  \qbezier( 240,  45)( 260,   0)( 300,   0)  % right  side

  \qbezier( 200,   0)( 240,   0)( 260,  15)  % left   side
  \qbezier( 260,  15)( 300,  45)( 340,  15)  % center at  300
  \qbezier( 340,  15)( 360,   0)( 400,   0)  % right  side

\end{picture}
\end{fsL}
\end{center}
\caption{
  Duobinary receiver distributions for $M=4$
   \label{fig:db_pdf_M=4}
   }
\end{figure}

%=======================================
\section{Modified Duobinary ISI solution}
\index{modified duobinary}
\label{sec:isi_mduo}
%=======================================
%--------------------------------------
\subsection{Constraints}
%--------------------------------------
The received waveform $r(t)$ is of the form
\begin{eqnarray*}
   r(t) &=& \sum_m a_m \hhcmp(t-mT).
\end{eqnarray*}

At sampling instants $t=nT$, $r(t)$ has the form

\begin{eqnarray*}
   r(nT)
     &=& \left. r(t)                               \right|_{t=nT}
   \\&=& \sum_m a_m \hhcmp(nT-mT)
   \\&=& \left. a_m\hhcmp(nT-mT)\right|_{m=n}     +
         \left. a_m\hhcmp(nT-mT)\right|_{m=n-1}   +
         \sum_{m\ne n-1,n+1} a_m \hhcmp(nT-mT)
   \\&=& a_{n-1}\hhcmp(nT-(n-1)T) +
         a_{n+1}\hhcmp(nT-(n+1)T) +
         \sum_{m\ne n-1,n,n+1} a_m \hhcmp(nT-mT)
   \\&=& a_{n+1}\hhcmp(-T) +
         a_{n-1}\hhcmp(T) +
         \sum_{m\ne n-1,n+1} a_m \hhcmp(nT-mT)
\end{eqnarray*}

We place the following constraints on the signaling waveform $\hhcmp(t)$:

We place the following constraints on the signaling waveform $\hhcmp(t)$:
\[
\begin{array}{llrcl}
   1. & \mbox{\bf sampling constraint:}
      & \hhcmp(nT) &=&
          \left\{\begin{array}{l@{\hspace{4ex}}l}
            +1 &   \mbox{for } n=-1 \\
            -1 &   \mbox{for } n=+1 \\
             0 &   \mbox{otherwise}
          \end{array}\right.
\\
   2. & \mbox{\bf bandwidth constraint:}
      & [\opFT{\hhcmp}](f) &=& 0 \mbox{ for } |f|\ge W.
\end{array}
\]
These two constraints are in conflict with each other.
However, they are both satisfied if the criterion
in \prefpp{thm:mdb_zeroISI} is met.

%--------------------------------------
\subsection{Criterion}
%--------------------------------------
%--------------------------------------
\begin{theorem}
\label{thm:mdb_zeroISI}
%--------------------------------------
Let $\ft{\hhcmp}(f)$ be the Fourier Transform of a function
$\hhcmp(t)$ and $T\in\R$ a constant.
\thmbox{
   \brs{
   \hhcmp(nT) =
   \left\{
      \begin{array}{lcl}
         +1 & : & n=-1 \\
         -1 & : & n=+1 \\
          0 & : & \mbox{otherwise}
      \end{array}
   \right.
   }
\iff
   \brs{
   \frac{1}{T}\sum_n \ft{\hhcmp}\left(f+ \frac{n}{T} \right)
     = i2\sin(2\pi fT).
   }
}
\end{theorem}
\begin{proof}
This theorem is easily proven using the
\thme{Inverse Poisson's Summation Formula} (IPSF) \ifsxref{transop}{thm:ipsf}
which states
\begin{eqnarray*}
   \sum_{n\in\Z} \ft{\hhcmp}\left( f+ \frac{n}{T} \right)
      &=& T \sum_n \hhcmp(nT) e^{-i2\pi fnT}
\end{eqnarray*}

\begin{enumerate}
  \item ``Only if" case ($\implies$):
\begin{align*}
   \sum_n \ft{\hhcmp}\left( f + \frac{n}{T} \right)
       &= T\sum_n \hhcmp(nT) e^{-i2\pi fnT}
       && \text{by IPSF}
     \\&= T\left[
                  \hhcmp(-1T) e^{-i2\pi f(-1)T} +
                  \hhcmp(1T) e^{-i2\pi f1T} +
                  \sum_{n\ne n-1,n+1} \hhcmp(nT) e^{-i2\pi fnT}
               \right]
     \\&= T\left[
                  (1)  e^{-i2\pi f(-1)T}
                  (-1) e^{-i2\pi f1T}
               \right]
       && \text{by left hypothesis}
     \\&= T\left[e^{i2\pi fT} -e^{-i2\pi fT} \right]
     \\&= i2T \sin( 2\pi fT )
       && \text{by \thme{Euler formulas} \prefp{cor:eform}}
\end{align*}

\item ``If" case ($\impliedby$):
\begin{align*}
   i2T \sin( 2\pi fT )
       &= \sum_n \ft{\hhcmp}\left( f+\frac{n}{T} \right)
       && \text{by right hypothesis}
     \\&= T \sum_n \hhcmp(nT) e^{-i2\pi fnT}
       && \text{by IPSF}
     \\&= i2T \sum_n \hhcmp(nT) \frac{1}{2i}e^{-i2\pi fnT}
     \\&= i2T \left[
                 \frac{\hhcmp(-T)e^{i2\pi fT} + \hhcmp( T)e^{-i2\pi fT}}
                      {2i}
                 + \sum_{n\ne-1,1} \hhcmp(nT) \frac{1}{2i}e^{-i2\pi fnT}
              \right]
     \\ \implies
\\
   \hhcmp(nT)
   &= \left\{
      \begin{array}{rcccr}
          1 & : & n &=&   -1 \\
         -1 & : & n &=&    1 \\
          0 & : & \mc{3}{l}{\mbox{otherwise}}
      \end{array} \right.
       && \text{because $\sin(2\pi fT)$ has no imaginary part}
\end{align*}
\end{enumerate}
\end{proof}

%--------------------------------------
\subsection{Signaling waveform}
%--------------------------------------
\begin{figure}[ht]\color{figcolor}
\begin{center}
\begin{fsL}
\setlength{\unitlength}{0.2mm}
\begin{picture}(400,120)(-200,-50)
  %\graphpaper[10](-100,0)(500,150)
  \thicklines
  \put(-150 ,   0 ){\line(1,0){300} }
  \put(   0 , -80 ){\line(0,1){160} }
  \put( 160 ,   0 ){$f$}
  \put(  20 ,  60 ){$\ds -i\ft{\hhcmp}(f)=\frac{1}{W}\sin\left(\frac{\pi}{W}f\right)\chi_B(f)$}
  \put( -20 ,  60 ){$\frac{1}{W}$}
  \put(-200 , -50 ){\makebox(200,40)[t]{$-W=-\frac{1}{2T}$}}
  \put(   0 , -50 ){\makebox(200,40)[t]{$W=\frac{1}{2T}$}}

  \color{red}
  \qbezier(-100,   0)( -50,-100)(   0,   0)
  \qbezier(   0,   0)(  50, 100)( 100,   0)
\end{picture}
\end{fsL}
\end{center}
\caption{
  Modified duobinary waveform $\ft{\hhcmp}(f)$ at Nyquist rate
   \label{fig:mdb_zeroISI}
   }
\end{figure}



The next theorem specifies a signaling waveform which satisfies the
criterion at the Nyquist rate
\[ W  = \frac{1}{2T}. \]
Like the duobinary Nyquist rate signaling waveform
(\prefpp{fig:db_zeroISI}),
the modified duobinary Nyquist rate signaling waveform
(\prefpp{fig:mdb_zeroISI})
can be easily approximated in real systems.
Unlike the duobinary Nyquist rate signaling waveform,
the modified duobinary Nyquist rate signaling waveform
has no DC component making it a better candidate
for channels that attenuate DC
(for example, capacitively coupled channels).

%--------------------------------------
\begin{theorem}
%--------------------------------------
The waveform $\hhcmp(t)$ with Fourier transform $\ft{\hhcmp}(f)$
(see \prefpp{fig:mdb_zeroISI})
satisfies the criterion stated in \prefpp{thm:mdb_zeroISI},
where

\thmbox{\begin{array}{rcl}
   \ft{\hhcmp}(f) &=& \left\{
   \begin{array}{ll}
      \ds i2T\sin(2\pi fT) & : \frac{-1}{2T} \le f < \frac{1}{2T} \\
      \ds 0                & : \mbox{ otherwise}.
   \end{array}
   \right.
\\ \\ \ds
   \hhcmp(t)
     &=& \frac{\sin[\frac{\pi}{T}(t+T)]} {\frac{\pi}{T} (t+T)}   -
         \frac{\sin[\frac{\pi}{T}(t-T)]} {\frac{\pi}{T} (t-T)}
\\
   \\&=& \sinc{\frac{\pi}{T}(t+T)}   -
         \sinc{\frac{\pi}{T}(t-T)}
\end{array}
}
\end{theorem}
\begin{proof}
Let $B=[-1/2T, +1/2T)$ such that
\[
\chi_B(f) \eqd
\left\{
\begin{tabular}{llll}
   1 & : $f\in [-1/2T, +1/2T)$ \\
   0 & : otherwise.
\end{tabular}
\right.
\]

Then
First, observe that $\ft{\hhcmp}(f)$ satisfies
the criterion of \prefpp{thm:mdb_zeroISI}:

\begin{eqnarray*}
   \sum_n \ft{\hhcmp}\left( f + \frac{n}{T} \right)
     &=& \sum_n i2T
         \sin[2\pi (f+\frac{n}{T})T]
         \chi_B\left(f+\frac{n}{T}\right)
   \\&=& i2T  \sum_n
         \sin(2\pi fT + 2\pi n)
         \chi_B\left(f+\frac{n}{T}\right)
   \\&=& i2T  \sum_n \left[
         \sin(2\pi fT)\cos(2\pi n) + \cos(2\pi fT)\sin(2\pi n)
         \right]
         \chi_B\left(f+\frac{n}{T}\right)
   \\&=& i2T  \sum_n \left[
         \sin(2\pi fT)\cdot1 + \cos(2\pi fT)\cdot0
         \right]
         \chi_B\left(f+\frac{n}{T}\right)
   \\&=& i2T  \sum_n \sin(2\pi fT)
         \chi_B\left(f+\frac{n}{T}\right)
   \\&=& i2T \sin(2\pi fT) \sum_n
         \chi_B\left(f+\frac{n}{T}\right)
   \\&=& i2T \sin(2\pi fT)
%   \\&\iff& Equation~\ref{eq:mdb_II}.
\end{eqnarray*}

The signaling waveform $\hhcmp(t)$ can be found by taking
the inverse Fourier Transform of $\ft{\hhcmp}(f)$:

\begin{align*}
   \hhcmp(t)
     &= [\opFTi\ft{\hhcmp}](t)
   \\&= \int_f \hhcmp(f) e^{i2\pi ft} \df
   \\&= \int_\frac{-1}{2T}^\frac{1}{2T}
        i2T \sin(2\pi Tf) e^{i2\pi ft} \df
   \\&= i2T\int_\frac{-1}{2T}^\frac{1}{2T}
        \frac{1}{2i}\left[ e^{i2\pi Tf} - e^{-i2\pi Tf}  \right]
        e^{i2\pi ft} \df
   \\&= T\int_\frac{-1}{2T}^\frac{1}{2T}
        \left[ e^{i2\pi Tf} - e^{-i2\pi Tf}  \right]
        e^{i2\pi ft} \df
   \\&= T\int_\frac{-1}{2T}^\frac{1}{2T}
        \left[ e^{i2\pi f(t+T)} - e^{i2\pi f(t-T)}  \right]
        \df
   \\&= \left. T\frac{e^{i2\pi f(t+T)}}{i2\pi (t+T)} \right|_\frac{-1}{2T}^\frac{1}{2T}  -
        \left. T\frac{e^{i2\pi f(t-T)}}{i2\pi (t-T)} \right|_\frac{-1}{2T}^\frac{1}{2T}
   \\&= \frac{e^{i\frac{\pi}{T} (t+T)} - e^{-i\frac{\pi}{T} (t+T)}}{2i\frac{\pi}{T} (t+T)}   -
        \frac{e^{i\frac{\pi}{T} (t-T)} - e^{-i\frac{\pi}{T} (t-T)}}{2i\frac{\pi}{T} (t-T)}
   \\&= \frac{2i\sin[\frac{\pi}{T}(t+T)]} {2i\frac{\pi}{T} (t+T)}   -
        \frac{2i\sin[\frac{\pi}{T}(t-T)]} {2i\frac{\pi}{T} (t-T)}
   \\&= \frac{\sin[\frac{\pi}{T}(t+T)]} {\frac{\pi}{T} (t+T)}   -
        \frac{\sin[\frac{\pi}{T}(t-T)]} {\frac{\pi}{T} (t-T)}
\end{align*}
\end{proof}
