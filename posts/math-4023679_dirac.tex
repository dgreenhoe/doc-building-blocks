Daniel J. Greenhoe
https://math.stackexchange.com/questions/4023679/

$\newcommand{\eqd}{\triangleq}$
$\newcommand{\eqa}{\approx}$
$\newcommand{\abs}[1]{{\left\lvert #1 \right\rvert}}$
$\newcommand{\brp}[1]{{\left(#1\right)}}$
$\newcommand{\brs}[1]{{\left[#1\right]}}$
$\newcommand{\brlr}[1]{\left.#1\right|}$
$\newcommand{\brbl}[1]{\left\{#1\right.}$
$\newcommand{\deriv} [2]   {{\frac{\mathrm{d}#1}{\mathrm{d}#2} }}$
$\newcommand{\R}{\Bbb{R}}$
$\newcommand{\Z}{\Bbb{Z}}$
$\newcommand{\Rx} {{\R^\ast}}$
$\newcommand{\intcc} [2]  {{\left[#1:#2\right]}}$
$\newcommand{\intoo} [2]  {{\left(#1:#2\right)}}$
$\newcommand{\intoc} [2]  {{\left(#1:#2\right]}}$
$\newcommand{\intco} [2]  {{\left[#1:#2\right)}}$
$\newcommand{\ff}{\mathrm{f}}$
$\newcommand{\fy}{\mathrm{y}}$
$\newcommand{\fF}{\mathrm{F}}$
$\newcommand{\fG}{\mathrm{G}}$
$\newcommand{\fg}{\mathrm{g}}$
$\newcommand{\fphi}{\mathrm{\phi}}$
$\newcommand{\dx}{\mathrm{dx}}$
$\newcommand{\du}{\mathrm{du}}$
$\newcommand{\dv}{\mathrm{dv}}$
$\newcommand{\dt}{\mathrm{dt}}$
$\newcommand{\dtau}{\mathrm{d\tau}}$
$\newcommand{\kdelta}{\bar{\delta}}$
$\newcommand{\setu}{\cup}$
$\newcommand{\setn}[1]{{\left\{{#1}\right\}}}$
$\newcommand{\seqn}[1]{{\left[{#1}\right]}}$
$\newcommand{\inprod}[2]{\left\langle{#1}\,|\,{#2}\right\rangle}$

Building on what others have already said, the two functions are just thattwo distinct functions. 
One is the real-valued _Kronecker Delta_ function $\kdelta:\Z\to\R$ defined as 
$$\kdelta(n)\eqd\brbl{\begin{array}{cl}
  1 & \text{if $n=0$}\\
  0 & \text{otherwise}
\end{array}}$$
and the other is the _Dirac Delta_ $\delta:\R\to\Rx$, which obviously is _not_ a real-valued function at all.
This is not so much a problem in and of itself 
(functions can be mappings to real numbers, complex numbers, colors, faces of a die, ...)
but in the case of $\delta$, it maps not into a well behaved set like the real numbers $\R$, 
but rather into the set of 
[extended real numbers](https://books.google.com/books?vid=ISBN0792336585&pg=PA8)
$\Rx\eqd\R\setu\setn{-\infty,+\infty}$. 
The set $\Rx$ is an Alice-in-Wonderland type world and when working therein requires great care
and proper safety equipment. 
In $\Rx$ crazy things can happen like $\infty+\infty=\infty$.

Because $\delta(x)$ maps to $\Rx$, many would not call it a _function_ at all, but rather an
_extended function_, or a distribution 
(see for example [Mallat](https://books.google.com/books?id=hbVOfWQNtB8C) Appendix A.7).
A distribution may be something of a loose cannon when left to its own devices,
but when constrained to the inside of an integral
(or more generally inside an inner product)
it becomes quite well-behaved, predictable, and useful.

Building on what @Anixx and @Thomas-Andrews have already pointed out, 
the Dirac Delta distribution can be defined, together with $\int\dx$, as an 
_operator_ (a [mapping](https://books.google.com/books?vid=ISBN0817646868&pg=PA42) from one 
[linear space](https://books.google.com/books?vid=ISBN0817641742) to another/same linear space)
with the following [definition](https://books.google.com/books?vid=ISBN0521649714):
$$\inprod{\ff(x)}{\delta(x)} \eqd \int_{\R}\ff(x)\delta(x)\dx \eqd \ff(0)$$

So the Dirac Delta distribution $\delta(x)$ and the Kronecker delta function are distinctly different.
But that being said (that they are very different in form), they are very similar in function.
In particular, note the following:

(1) Both "sum" to 1:
\begin{align*}
  \boxed{\int_{\R} \delta(x)\dx}
    &= \int_{\R} 1\cdot\delta(x)\dx
  \\&\eqd \boxed{1}
    && \text{by definition of $\delta$}
  \\
  \\
  \boxed{\sum_{n\in\Z}\kdelta(n)}
    &= \kdelta(0)
    && \text{by definition of $\kdelta$}
  \\&= \boxed{1}
\end{align*}

(2) Both have a similar time-shift property:
\begin{align*}
   \boxed{\int_{\R} \delta\brp{x-a}\ff(x)\dx}
     &= \int_\R \delta\brp{u}\ff(u+a)\du
     && \text{where $u=x-a$ $\implies$ $\dx=\du$}
   \\&= \boxed{\ff(a)}
  \\
  \\
  \boxed{\sum_{n\in\Z}\kdelta(n-k)y(n)}
    &= \sum_{n\in\Z}\kdelta(m)y(m+k)
    && \text{where $m=n-k$ $\implies$ $n=m+k$}
  \\&= \kdelta(0)y(0+k)
    && \text{by definition of $\kdelta$}
  \\&= \boxed{y(k)}
\end{align*}

(3) Both induce a projection operator.
A projection operator $P$ is a linear operator [such that](https://books.google.com/books?id=yzr0BwAAQBAJ&pg=PA73)
$P^2=P$:
\begin{align*}
  \boxed{P^2}\ff(x)
    &= PP\ff(x)
  \\&\eqd P\int_\R \delta(x)\ff(x) \dx
    && \text{where here $P$ is the Dirac Delta operator}
  \\&\eqd P\ff(0)
    && \text{by definition of $\delta$}
  \\&\eqd \int_\R \delta(x)\ff(0)\dx
    && \text{where here $P$ is the Dirac Delta operator}
  \\&= \ff(0) \int_\R \delta(x)\dx
    && \text{by linearity of the integral operator}
  \\&= \ff(0)
    && \text{because $\int_{\R} \delta(x)\dx=1$}
  \\&= \int_\R\delta(x)\ff(x)\dx
    && \text{by definition of $\delta$}
  \\&= \boxed{P}\ff(x)
    && \text{where here $P$ is the Dirac Delta operator}
  \\
  \\
  \boxed{P^2}\fy(n)
    &= PP\fy(n)
  \\&\eqd P\sum_{n\in\Z} \kdelta(n)\fy(n) 
    && \text{where here $P$ is the Kronecker Delta operator}
  \\&\eqd P\fy(0)
    && \text{by definition of $\kdelta$}
  \\&\eqd \sum_{n\in\Z} \kdelta(n)\fy(0)
    && \text{where here $P$ is the Kronecker Delta operator}
  \\&= \fy(0) \sum_{n\in\Z} \kdelta(n)
    && \text{by linearity of the summation operator}
  \\&= \fy(0)
    && \text{because $\sum_{n\in\Z} \kdelta(n)=1$}
  \\&= \sum_{n\in\Z}\kdelta(n)\fy(n)
    && \text{by definition of $\kdelta$}
  \\&= \boxed{P}\fy(n)
    && \text{where here $P$ is the Kronecker Delta operator}
\end{align*} 

(4) Both can be used for sampling.
The Dirac can be used�inside an integral�as an projection operator to map
a function $\ff(t)$ to a single point $\ff(a)$ for some value $t=a$.
That is, it can be used to _sample_ $\ff(t)$ at a given time $t=a$.
In the field of Digital Signal Processing
(DSP) a continuous time function $\ff(t)$ can be transformed (mapped) to a  
sequence 
([a function with domain $\Z$](https://books.google.com/books?vid=ISBN8122408265&pg=PA31))
$\seqn{\ldots, x_{n-1}, x_n, x_{n+1}, \ldots}$,
where each element $x_k$ of this sequence is
$$x_k \eqd \int_\R \ff(t)\delta(t-kT)\dt = \ff(kT)$$
Likewise, the Kronecker delta can be used to sample (OK, maybe a bit of a stretch here?) 
a sequence $\seqn{\ldots, x_{n-1}, x_n, x_{n+1}, \ldots}$
in the sense
$$y(n)= \cdots + x_{-2}\kdelta(n+2) + x_{-1}\kdelta(n+1) + x_0\kdelta(n) + x_1\kdelta(n-1) + x_2\kdelta(n-2) + \cdots$$

