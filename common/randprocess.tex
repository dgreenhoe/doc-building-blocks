%============================================================================
% LaTeX File
% Daniel J. Greenhoe
%============================================================================

%======================================
\chapter{Continuous Random Processes}
\label{app:random_processes}
%======================================
%=======================================
\section{Definitions}
%=======================================
%---------------------------------------
\begin{definition}
\index{random variable}
\index{random process}
\footnote{
  \citerpg{papoulis}{63}{0070484775},
  \citerpg{papoulis}{285}{0070484775}
  }
%---------------------------------------
Let $\ps$ be a \structe{probability space}.\\
\defboxt{
  The function $\rvx:\pso\to\R$ is a \fnctd{random variable}.\\
  The function $\rvy:\R\times\pso\to\R$ is a \fnctd{random process}.
  }
\end{definition}

The random process $\rvx(t,\omega)$, where $t$ commonly represents time
and $\omega\in\pso$ is an outcome of an experiment,
can take on more specialized forms depending on whether
$t$ and $\omega$ are fixed or allowed to vary.
These forms are illustrated in \prefp{fig:X(t,w)}\footnote{\citerppg{papoulis}{285}{286}{0070484775}}
and \prefp{fig:X(t,w)graph}.

\begin{figure}[ht]\color{figcolor}
\begin{center}
   \begin{tabular}{|c||c|c|}
      \hline
         $\rvx(t,\omega)$ &  fixed $t$      & variable $t$   \\
      \hline
      \hline
         fixed    $\omega$ & number          & time function  \\
      \hline
         variable $\omega$ & random variable & random process \\
      \hline
   \end{tabular}
\caption{
   Specialized forms of a random process $\rvx(t,\omega)$
   \label{fig:X(t,w)}
   }
\end{center}
\end{figure}

\begin{figure}[ht]\color{figcolor}
\begin{center}
\includegraphics[height=8cm,width=12cm]{../common/x_tw.eps}
\end{center}
\caption{
  Example of a random process $\rvx(t,\omega)$
  \label{fig:X(t,w)graph}
}
\end{figure}

%---------------------------------------
\begin{definition}
\footnote{
  \citerpgc{papoulis1984}{216}{0070484686}{$R_{xy}(t_1,t_2)=E\brb{\rvx(t_1)\rvy^\ast(t_2)}$ (9-35)}
  }
\label{def:Rxx}
\label{def:Rxy}
%---------------------------------------
Let $\rvx(t)$ and $\rvy(t)$ be \fncte{random process}es.
Let $\pE$ be the \fncte{expectation operator} \xref{def:pE}.
\defbox{
  \begin{array}{MlMlc>{\ds}l}
       The \fnctd{mean}                      & \pmeanx(t) & of $\rvx(t)$              is & \pmeanx(t)&\eqd& \pE\brs{\rvx(t)}                    %& \text{(\prope{mean functional})} \\
     \\The \fnctd{cross-correlation}         & \Rxy(t,u)  & of $\rvx(t)$ and $\rvy(t)$is & \Rxy(t,u) &\eqd& \pE\brs{\rvx(t)\rvy^\ast(u)}        %& \text{(\prope{cross-correlation bilinear functional})} \\
     \\The \fnctd{auto-correlation function} & \Rxx(t,u)  & of $\rvx(t)$              is & \Rxx(t,u) &\eqd& \pE\brs{\rvx(t)\rvx^\ast(u)}        %& \text{(\prope{auto-correlation bilinear functional})} \\
  \end{array}
  }
\end{definition}

%---------------------------------------
\begin{definition}
%\label{def:wss}
%---------------------------------------
\defboxt{
  A random process $\rvx(t)$ is \propd{wide sense stationary} (\propd{WSS}) if
  \\\indentx$\begin{array}{FlMMD}
   (1).& \pmeanx(t)     & is \prope{constant} with respect to $t$ & (\prope{stationary in the mean})    & and \\
   (2).& \Rxx(t+\tau,t) & is \prope{constant} with respect to $t$ & (\prope{stationary in correlation})
  \end{array}$
  }
\end{definition}

If a process $\rvx(t)$ is \prope{wide sense stationary}, mean and correlation are often written
$\pmeanx$ and $\Rxx(\tau)$, respectively.
If a pair of processes $\rvx(t)$ and $\rvy(t)$ are \prope{WSS},
then their cross-correlation is commonly written $\Rxy(\tau)$.

%=======================================
\section{Properties}
%=======================================
%---------------------------------------
\begin{theorem}
\label{thm:Rxx_prop}
\index{cross-correlation}
\index{symmetric!conjugate}
\index{conjuage symmetric}
%---------------------------------------
Let $\rvx(t)$ and $\fy(t)$ be random processes with
cross-correlation $\Rxy(t,u)$ and
let $\Rxx(t,u)$ be the auto-correlation of $\rvx(t)$.
\thmbox{
\begin{array}{rcll}
   \Rxx(t,u) &=& \Rxx^\ast(u,t) & \text{(\prope{conjugate symmetric})}\\
   \Rxy(t,u) &=& \Ryx^\ast(u,t) &
\end{array}
}
\end{theorem}
\begin{proof}
\begin{align*}
   \Rxx(t,u)
      &\eqd \pE\brs{\rvx(t) \rvx^\ast(u)}
      &=    \pE\brs{\rvx^\ast(u) \rvx(t)}
      &=    \brp{ \pE\brs{\rvx(u) \rvx^\ast(t)}}^\ast
      &\eqd \Rxx^\ast(u,t)
\\
   \Rxy(t,u)
      &\eqd \pE\brs{\rvx(t) \rvy^\ast(u)}
      &=    \pE\brs{\rvy^\ast(u) \rvx(t) }
      &=    \brp{\pE\brs{\rvy(u) \rvx^\ast(t)}}^\ast
      &\eqd \Ryx^\ast(u,t)
\end{align*}
\end{proof}

