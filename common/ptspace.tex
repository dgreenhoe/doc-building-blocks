%============================================================================
% LaTeX File
% Daniel J. Greenhoe
%============================================================================

%======================================
\chapter{Distance Spaces with Power Triangle Inequalities}
\label{chp:trirel}
\label{chp:pdspace}
%======================================
%======================================
\section{Definitions}
\label{sec:pdspace_def}
%======================================
%%---------------------------------------
%\begin{definition}
%\footnote{
%
%  }
%\label{def:ftri}
%%---------------------------------------
%%Let $\dspaceX$ be a \structe{distance space} \xref{def:dspace}.
%Let $\Rp$ be the set of all \structe{positive real numbers} 
%and $\Rx$ be the set of \structe{extended real numbers} \xref{def:Rx}.
%\defboxp{
%  A function $\ptf$ is a \fnctd{triangle function} if
%  \\\indentx$\begin{array}{FrclCD}
%    1. & \ftri(\ff,\fg) &=&     \ftri(\fg,\ff)  & \forall \ff,\fg\in  & (\prope{symmetric})
%  \\2. & \ftri(\ff,\fg) &\orel& \ftri(\fh,\fk)\quad\text{whenever $\ff\orel\fh$ and $\fg\orel\fk$} & \forall \ff,\fg,\fh,\fk\in & (\prope{order preserving})
%  \\3. & 
%  \end{array}$
%  }
%\end{definition}

This chapter introduces a relation called the \rele{power triangle inequality} \xref{def:trirel}.
It is a generalization of other common relations, including the \rele{triangle inequality} \xref{def:trirels}.
The \rele{power triangle inequality} is defined in terms of a function herein called the \fncte{power triangle function} (next definition).
This function is a special case of the \fncte{power mean} with $\xN=2$ and $\lambda_1=\lambda_2=\frac{1}{2}$ \xref{def:pmean}.
\fncte{Power mean}s have the attractive properties of being \prope{continuous} and \prope{strictly monontone} with respect
to a free parameter $p\in\Rx$ \xref{thm:pmean_continuous}.
This fact is inherited and exploited by the \rele{power triangle inequality} \xref{cor:tri_mono}.
%---------------------------------------
\begin{definition}
\footnote{
  \citeP{greenhoe2015pds}
  }
\label{def:ptf}
%---------------------------------------
Let $\dspaceX$ be a \structe{distance space} \xref{def:dspace}.
Let $\Rp$ be the set of all \structe{positive real numbers} 
and $\Rx$ be the set of \structe{extended real numbers} $\brp{\Rx\eqd\R\setu\setn{-\infty,\,\infty}}$. %\xref{def:Rx}.
\defboxp{
  The \fnctd{power triangle function} $\ptf$ on $\dspaceX$ is %, for some $\otriple{x}{y}{z}\in\setX^3$, 
  defined as
  \\\indentx$\begin{array}{rc>{\ds}lCC}
    \ptfD
      &\eqd& 2\sigma\brs{\frac{1}{2}\distancep{p}{x}{z} + \frac{1}{2}\distancep{p}{z}{y}}^\frac{1}{p} 
      &      \forall \opair{p}{\sigma}\in\Rx\times\Rnn, 
      &      x,y,z\in\setX
  \end{array}$
  }
\end{definition}

%---------------------------------------
\begin{remark}
\footnote{
  \citerp{sherstnev1962}{4},
  %\citer{sherstnev1964},
  \citerpgc{schweizer1983}{9}{0486143759}{(1.6.1)--(1.6.4)},
  \citePp{bessenyei2014}{2}
  }
%---------------------------------------
In the field of \hie{probabilistic metric spaces}, 
a function called he \fncte{triangle function} was introduced by Sherstnev in 1962.
However, the \fncte{power triangle function} as defined in this present paper is \emph{not} a special case of
(is not compatible with) the \fncte{triangle function} of Sherstnev.
Another definition of \fncte{triangle function} has been offered by Bessenyei in 2014
with special cases of $\Phi(u,v)\eqd c(u+v)$ and $\Phi(u,v)\eqd(u^p+v^p)^\frac{1}{p}$,
which \emph{are} similar to the definition of \fncte{power triangle function} offered in this present paper.
\end{remark}

%---------------------------------------
\begin{definition}
\label{def:trirel}
\label{def:pdspace}
\label{def:ptineq}
%---------------------------------------
Let $\dspaceX$ be a \structe{distance space}.
Let $\clRxxx$ be the set of all trinomial \structe{relation}s\ifsxref{relation}{def:clRxy} on $\setX$.
\defboxp{
  A relation $\trirelD$ in $\clRxxx$
  is a \reld{power triangle inequality} on $\dspaceX$ if
  \\\indentx$\ds\trirelD\eqd\set{\otriple{x}{y}{z}\in\setX^3}{\distance{x}{y}\le\ptfD}$
    \qquad for some $\opair{p}{\sigma}\in\Rx\times\Rp$.
  \\
  The tupple $\pdspaceX$ is a \structd{power distance space} and $\distancen$ a \fnctd{power distance} or \fnctd{power distance function} if 
  $\dspaceX$ is a \structe{distance space} in which the 
  \structe{triangle relation} $\trirelD$ holds.
 %\\\indentx$\brb{x,y,z\in\setX} \quad\implies\quad \brb{\distance{x}{y}\le\ptfD}$
  %The notation $\trirel(p,\sigma;x,y,z)$ indicates that the triple $\otriple{x}{y}{z}$ is in 
  %the \rele{power triangle inequality} $\trirelD$ ($\otriple{x}{y}{z}\in\trirelD$).
  %
  }
\end{definition}

%This paper introduces a new relation called the \rele{power triangle inequality} (next).
%%---------------------------------------
%\begin{definition}
%\label{def:trirel}
%%---------------------------------------
%Let $\clRxxx$ be the set of all trinomial \structe{relation}s \xref{def:clRxy} on a 
%\structe{set} $\setX$.
%Let $\dspaceX$ be a \structe{distance space} \xref{def:dspace} on $\setX$.
%\defboxp{
%  A relation $\trirel(x,y,z)$ in $\clRxxx$ is a \reld{power triangle inequality} on $\dspaceX$ if
%  $\otriple{x}{x}{z}\in\trirel\quad\forall x,z\in\setX$.
%  }
%\end{definition}
%
%%---------------------------------------
%\begin{definition}
%\footnote{
%  \citePp{bessenyei2014}{2}
%  \citePpc{czerwik1993}{5}{\structe{b-metric}; (1),(2),(5)},
%  \citeP{fagin2003p},
%  \citePc{fagin2003}{Definition 4.2 (Relaxed metrics)},
%  \citePpc{xia2009}{453}{Definition 2.1},
%  \citerpgc{heinonen2001}{109}{0387951040}{14.1 Quasimetric spaces.},
%  \citerpgc{kirk2014}{113}{3319109278}{Definition 12.1},
%  \citerpg{deza2014}{7}{3662443422}
%  }
%%---------------------------------------
%Special cases of \structe{power triangle inequality}s \xref{def:trirel} on a \structe{distance space} $\dspaceX$ 
%are defined as follows:
%\\\defboxp{$\begin{array}{Fl@{\hspace{2pt}}lM}
% %1. &\trirel\eqd\setX^3                                                                                                              &\big\  & is the \reld{trivial power triangle inequality}.\\
%  1. &\trirel\eqd\big\{\otriple{x}{y}{z}\in\setX^3\,|\,\distance{x}{y} \orel \distance{x}{z}+\distance{z}{y}                                   &\big\} & is the \reld{triangle inequality}.\\
%  2. &\trirel\eqd\big\{\otriple{x}{y}{z}\in\setX^3\,|\,\distance{x}{y} \orel \sigma\brs{\distance{x}{z}+\distance{z}{y}}            ,\,\sigma>0&\big\} & is the \reld{relaxed triangle inequality}.\\
%  3. &\trirel\eqd\big\{\otriple{x}{y}{z}\in\setX^3\,|\,\distance{x}{y} \orel \max\setn{\distance{x}{z},\,\distance{z}{y}}                      &\big\} & is the \reld{inframetric inequality}.\\
%  4. &\trirel\eqd\big\{\otriple{x}{y}{z}\in\setX^3\,|\,\distance{x}{y} \orel \sigma\max\setn{\distance{x}{z},\,\distance{z}{y}}     ,\,\sigma>0&\big\} & is the \reld{\txsigma-inframetric inequality}.\\
%  5. &\trirel\eqd\big\{\otriple{x}{y}{z}\in\setX^3\,|\,\distance{x}{y} \orel \brs{\distancen^p(x,z) + \distancen^p(z,y)}^\frac{1}{p},\,p>0     &\big\} & is the \reld{$p^{th}$ order triangle inequality}.
%\end{array}$}
%\end{definition}

The \fncte{power triangle function} can be used to define some standard inequalities (next definition).
See \prefpp{cor:ftri_means} for some justification of the definitions.
%---------------------------------------
\begin{definition}
\footnote{
  \citePp{bessenyei2014}{2},
  \citePpc{czerwik1993}{5}{\structe{b-metric}; (1),(2),(5)},
  \citeP{fagin2003p},
  \citePc{fagin2003}{Definition 4.2 (Relaxed metrics)},
  \citePpc{xia2009}{453}{Definition 2.1},
  \citerpgc{heinonen2001}{109}{0387951040}{14.1 Quasimetric spaces.},
  \citerpgc{kirk2014}{113}{3319109278}{Definition 12.1},
  \citerpg{deza2014}{7}{3662443422},
  \citePp{hoehn1985}{151},
  \citerpgc{gibbons1977}{51}{1611971101}{\ope{square-mean-root} (\ope{SMR}) (2.4.1)},
  \citerc{euclid}{triangle inequality---Book I Proposition 20}
  }
\label{def:trirels}
%---------------------------------------
Let $\trirelD$ be a \rele{power triangle inequality} on 
a \structe{distance space} $\dspaceX$.
\\\defboxp{$\begin{array}{*{2}{Fl@{\,}r@{\,}@{\,}c@{\,}@{\,}l@{\,}M}}
  1. & \trirel(& \infty      ,&\sfrac{\sigma}{2}  ;\distancen&)&{\scs is the} \reld{\txsigma-inframetric inequality} &  6. & \trirel(& 1           ,&1          ;\distancen &)&{\scs is the} \reld{triangle inequality}
\\2. & \trirel(& \infty      ,&\frac{1}{2}        ;\distancen&)&{\scs is the} \reld{inframetric inequality}          &  7. & \trirel(& 0           ,&\frac{1}{2};\distancen &)&{\scs is the} \reld{geometric inequality}           
\\3. & \trirel(& 2           ,&\sfrac{\sqrt{2}}{2};\distancen&)&{\scs is the} \reld{quadratic inequality}            &  8. & \trirel(&-1           ,&\frac{1}{4};\distancen &)&{\scs is the} \reld{harmonic inequality}            
\\4. & \trirel(& 1           ,&\sigma             ;\distancen&)&{\scs is the} \reld{relaxed triangle inequality}     &  9. & \trirel(&-\infty      ,&\frac{1}{2};\distancen &)&{\scs is the} \reld{minimal  inequality}            
\\5. & \trirel(& \sfrac{1}{2},&2                  ;\distancen&)&{\scs is the} \reld{square mean root inequality}     &     &         &              &                       & &                                      
%\\5. & \trirel(& \sfrac{1}{2},&2                  ;\distancen&)&\mc{6}{@{\,}M}{{\scs is the} \reld{square mean root inequality}}    
\end{array}$}
\end{definition}

%%---------------------------------------
%\begin{definition}
%\footnote{
%  \citePp{bessenyei2014}{2},
%  \citePpc{czerwik1993}{5}{\structe{b-metric}; (1),(2),(5)},
%  \citeP{fagin2003p},
%  \citePc{fagin2003}{Definition 4.2 (Relaxed metrics)},
%  \citePpc{xia2009}{453}{Definition 2.1},
%  \citerpgc{heinonen2001}{109}{0387951040}{14.1 Quasimetric spaces.},
%  \citerpgc{kirk2014}{113}{3319109278}{Definition 12.1},
%  \citerpg{deza2014}{7}{3662443422}
%  }
%\label{def:trirels}
%%---------------------------------------
%Let $\dspaceX$ be a \structe{distance space} \xref{def:dspace}.
%\\\defboxp{$\begin{array}{FrcllM}
%  1. &\distance{x}{y} &\orel& \distance{x}{z}+\distance{z}{y}                         & \forall x,y,z\in\setX,\,          & is the \reld{triangle inequality}.\\
%  2. &\distance{x}{y} &\orel& \sigma\brs{\distance{x}{z}+\distance{z}{y}}             & \forall x,y,z\in\setX,\, \sigma>0 & is the \reld{relaxed triangle inequality}.\\
%  3. &\distance{x}{y} &\orel& \max\setn{\distance{x}{z},\,\distance{z}{y}}            & \forall x,y,z\in\setX,\,          & is the \reld{inframetric inequality}.\\
%  4. &\distance{x}{y} &\orel& \sigma\max\setn{\distance{x}{z},\,\distance{z}{y}}      & \forall x,y,z\in\setX,\, \sigma>0 & is the \reld{\txsigma-inframetric inequality}.\\
%  5. &\distance{x}{y} &\orel& \brs{\distancen^p(x,z) + \distancen^p(z,y)}^\frac{1}{p} & \forall x,y,z\in\setX,\, p>0      & is the \reld{$p^{th}$ order triangle inequality}.
%\end{array}$}
%\end{definition}

%---------------------------------------
\begin{definition}
\footnote{
  \structb{metric space}:
  \citerp{dieudonne1969}{28},
  \citerp{copson1968}{21},
  \citorp{hausdorff1937e}{109},
  \citer{frechet1928},
  \citePp{frechet1906}{30} 
  %\citor{hausdorff1914}\\
  %\cithrpg{ab}{34}{0120502577} 
  %\citerc{euclid}{triangle inequality---Book I Proposition 20}
  \structb{near metric space}:
  \citePpc{czerwik1993}{5}{\structe{b-metric}; (1),(2),(5)},
  \citeP{fagin2003p},
  \citePc{fagin2003}{Definition 4.2 (Relaxed metrics)},
  \citePpc{xia2009}{453}{Definition 2.1},
  \citerpgc{heinonen2001}{109}{0387951040}{14.1 Quasimetric spaces.},
  \citerpgc{kirk2014}{113}{3319109278}{Definition 12.1},
  \citerpg{deza2014}{7}{3662443422}
  }
%\label{def:metric}
\label{def:mspace}
\label{def:nmetric}
\label{def:pdspace_spaces}
%---------------------------------------
Let $\dspaceX$ be a \structe{distance space} \xref{def:dspace}.
\defbox{\begin{array}{FMMMM}
    1. & $\dspaceX$ is a &\structd{metric space}               & if the \rele{triangle inequality}             & holds in $\setX$.
  \\2. & $\dspaceX$ is a &\structd{near metric space}          & if the \rele{relaxed triangle inequality}     & holds in $\setX$.
  \\3. & $\dspaceX$ is an&\structd{inframetric space}          & if the \rele{inframetric inequality}          & holds in $\setX$.
  \\4. & $\dspaceX$ is a &\structd{\txsigma-inframetric space} & if the \rele{\txsigma-inframetric inequality} & holds in $\setX$.
\end{array}}
\end{definition}

%%---------------------------------------
%\begin{definition}
%\footnote{
%  \citerp{dieudonne1969}{28},
%  \citerp{copson1968}{21},
%  \citorp{hausdorff1937e}{109},
%  \citer{frechet1928},
%  \citePp{frechet1906}{30} 
%  %\citor{hausdorff1914}\\
%  %\cithrpg{ab}{34}{0120502577} 
%  \citerc{euclid}{triangle inequality---Book I Proposition 20}
%  }
%\label{def:metric}
%%---------------------------------------
%%Let $\setX$ be a set.
%\defboxp{A \fncte{distance space} \xref{def:dspace} $\dspaceX$ is a \structd{metric space} if
%the \prope{triangle inequality} \xref{def:trirels} holds for all of $\setX$.
%In this case, $\distancen$ is a \fncte{metric}.}
%%  \\\indentx$\begin{array}{rcl CDDr@{}l@{\,}D}
%%        \distance{x}{y} &\le& \distance{x}{z}+\distance{z}{y}       & \forall x,y,z \in\setX & (\prope{subadditive}/\prope{triangle inequality}).\footnotemark
%%    \end{array}$\\
%%  The ordered pair $\dspaceX$ is a \structd{metric space} if $\distancen$ is a \fncte{metric}.
%\end{definition}


%%---------------------------------------
%\begin{definition}
%\label{def:metric}
%\label{def:(X,d)}
%\footnote{
%  \citerp{dieudonne1969}{28},
%  \citerp{copson1968}{21},
%  \citorp{hausdorff1937e}{109},
%  \citer{frechet1928},
%  \citePp{frechet1906}{30} 
%  %\citor{hausdorff1914}\\
%  %\cithrpg{ab}{34}{0120502577} 
%  \citerc{euclid}{triangle inequality---Book I Proposition 20}
%  }
%\index{space!metric}
%%---------------------------------------
%Let $\setX$ be a set.
%\defboxp{
%  A function $\distancen$ in the set $\clFxr$ is a \fnctd{metric} if 
%  \\\indentx$\begin{array}{F rcl CDDr@{}l@{\,}D}
%        \cline{8-8}
%        1. & \distance{x}{y} &\ge& 0                                 & \forall x,y   \in\setX & (\prope{non-negative})   & and &&\vline
%      \\2. & \distance{x}{y} &=  & 0  \iff x=y                       & \forall x,y   \in\setX & (\prope{nondegenerate})  & and &&\vline & \fncte{distance}
%      \\3. & \distance{x}{y} &=  & \distancen(y,x)                     & \forall x,y   \in\setX & (\prope{symmetric})      & and &&\vline
%      \\\cline{8-8}
%        4. & \distance{x}{y} &\le& \distance{x}{z}+\distance{z}{y}       & \forall x,y,z \in\setX & (\prope{subadditive}/\prope{triangle inequality}).\footnotemark
%    \end{array}$\\
%  The ordered pair $\dspaceX$ is a \structd{metric space} if $\distancen$ is a \fncte{metric}.
%  }
%\end{definition}
%%\footnotetext{\citorc{euclid}{Book I Proposition 20}}



%%---------------------------------------
%\begin{definition}
%\label{def:qmetric}
%\footnote{
%  \citerppg{deza2014}{6}{7}{3662443422},
%  \citerpg{deza2006}{4}{0444520872},
%  \citePpc{wilson1931}{675}{\textsection 1.},
%  \citeP{ribeiro1943},
%  \citePpc{kelly1963}{71}{Introduction},
%  \citeP{patty1967},
%  \citePp{stoltenberg1969}{65},
%  \citerpgc{grabiec2006}{3}{1594549176}{Introduction},
%  }
%\index{space!metric}
%%---------------------------------------
%%Let $\setX$ be a set and $\Rnn$ the set of non-negative real numbers.
%\defboxp{
%  A function $\hxs{\distancen}\in\clF{\setX\times\setX}{\R}$ \xref{def:clFxy} is an \fnctd{asymmetric metric} on a \structe{set} $\setX$ if
%  \\\indentx$\begin{array}{F rcl CDD}
%        1. & \distance{x}{y} &\ge& 0                                 & \forall x,y   \in\setX & (\prope{non-negative})   & and 
%      \\2. & \distance{x}{y} &=  & 0  \iff x=y                       & \forall x,y   \in\setX & (\prope{nondegenerate})  & and 
%      \\3. & \distance{x}{y} &\le& \distance{x}{z}+\distance{z}{y}       & \forall x,y,z \in\setX & (\prope{subadditive} / \prope{triangle inequality}).
%    \end{array}$
%  \\The ordered pair $\dspaceX$ is an \structd{asymmetric metric space} if $\distancen$ is an \fncte{asymmetric metric} on $\setX$.\\
%  An \fncte{asymmetric metric} is also called a \fnctd{directed metric}.\footnotemark
%  }
%\footnotetext{
%  In the literature, the term \fnctd{quasi-metric} can refer to either what is here called a 
%  \fncte{asymmetric metric} \xref{def:qmetric} or a \fncte{near metric} \xref{def:nmetric}.
%  This paper avoids the terminology conflict by not further using the term \fncte{quasi-metric}.
%  }
%\end{definition}
%
%%---------------------------------------
%\begin{definition}
%\footnote{
%  \citePpc{czerwik1993}{5}{\structe{b-metric}; (1),(2),(5)},
%  \citeP{fagin2003p},
%  \citePc{fagin2003}{Definition 4.2 (Relaxed metrics)},
%  \citePpc{xia2009}{453}{Definition 2.1},
%  \citerpgc{heinonen2001}{109}{0387951040}{14.1 Quasimetric spaces.},
%  \citerpgc{kirk2014}{113}{3319109278}{Definition 12.1},
%  \citerpg{deza2014}{7}{3662443422}
%  }
%\label{def:nmetric}
%%---------------------------------------
%%Let $\setX$ be a set.
%\defboxp{A \fncte{distance space} \xref{def:dspace} $\dspaceX$ is a \structd{near metric space} if
%the \prope{relaxed triangle inequality} \xref{def:trirels} holds for all of $\setX$.
%In this case, $\distancen$ is a \fncte{near metric}.}
%%\defboxp{
%%  A function $\distancen$ in the set $\clFxr$ is a \fnctd{near metric} if for some $\sigma\ge1$
%%  \\\indentx$\begin{array}{F rcl CDDr@{}l@{\,}D}
%%        \cline{8-8}
%%        1. & \distance{x}{y} &\ge& 0                                 & \forall x,y   \in\setX & (\prope{non-negative})   & and &&\vline
%%      \\2. & \distance{x}{y} &=  & 0  \iff x=y                       & \forall x,y   \in\setX & (\prope{nondegenerate})  & and &&\vline & \fncte{distance}
%%      \\3. & \distance{x}{y} &=  & \distancen(y,x)                     & \forall x,y   \in\setX & (\prope{symmetric})      & and &&\vline
%%      \\\cline{8-8}
%%        4. & \distance{x}{y} &\le& \sigma\brs{\distance{x}{z}+\distance{z}{y}} & \forall x,y,z \in\setX & (\prope{relaxed subadditive} / \prope{relaxed triangle inequality}).
%%    \end{array}$\\
%%  The pair $\dspaceX$ is a \structd{near metric space} if $\distancen$ is a \fncte{near metric}.
%%  }
%\end{definition}

%=======================================
\section{Properties}
\label{sec:pdspace_prop}
%=======================================
%=======================================
\subsection{Relationships of the power triangle function}
\label{sec:pdspace_propr}
%=======================================
%%% We start with a proposition (next), that while it may be somewhat interesting, 
%%% it is debatable whether it is significant in the sense of Hardy's seriousness criterion.\footnote{
%%%   ``The ``seriousness" of a mathematical theorem lies,
%%%     not in its practical consequences, which are usually negligible,
%%%     but in the \emph{significance} of the mathematical ideas which it connects.
%%%     We may say, roughly, that a mathematical idea is ``significant" if it can be
%%%     connected, in a natural illuminating way,
%%%     with a large complex of other mathematical ideas.
%%%   --- G. H. Hardy (1940). Reference: \citerc{hardy1940}{section 11}
%%% }
%%% %---------------------------------------
%%% \begin{proposition}
%%% \label{prop:tri_mpf}
%%% %---------------------------------------
%%% Let $\pdspaceX$ be a \fncte{power distance space} \xref{def:pdspace}
%%% with \fncte{power triangle function} $\ptf$ \xref{def:ptf}
%%% \propbox{
%%%   \brb{\text{$\pwdfn$ is a \structe{metric} on $\setX$}} 
%%%   \quad\implies\quad
%%%   \brb{\begin{array}{M}
%%%     $\pwdfa{\opair{x}{z}}{\opair{z}{y}}\eqd\ptfD$\\
%%%     is a \fncte{power distance function} on $\setX^2$
%%%   \end{array}}
%%%   }
%%% \end{proposition}
%%% \begin{proof}
%%% %{thm:met_power}
%%% \begin{enumerate}
%%%   %\item Proof that $\opair{x}{z}=\opair{z}{y}\implies\pwdf{\opair{x}{z}}{\opair{z}{y}}=0$ for $r\in\intco{1}{\infty}$:
%%%   %  \begin{align*}
%%%   %    \pwdfn(\opair{x}{z},\opair{z}{y})
%%%   %      &\eqd \brp{\sum_{n=1}^\xN \lambda_n \pwdfan^r(x_n,y_n)}^\frac{1}{r}
%%%   %      &&    \text{by definition of $\pwdfn$}
%%%   %    \\&=    \brp{\sum_{n=1}^\xN \lambda_n \pwdfan^r(x_n,x_n)}^\frac{1}{r}
%%%   %      &&    \text{by $\opair{x}{z}=\opair{z}{y}$ hypothesis}
%%%   %    \\&=    \brp{\sum_{n=1}^\xN 0}^\frac{1}{r}
%%%   %      &&    \text{because $\pwdfan$ is \prope{nondegenerate}}
%%%   %    \\&=    0
%%%   %  \end{align*}
%%%   %
%%%   \item Proof that $\pwdfa{\opair{x}{z}}{\opair{z}{y}}=0\implies\opair{x}{z}=\opair{z}{y}$ for $r\in\intco{1}{\infty}$:
%%%      \begin{align*}
%%%       0 &=    \pwdfa{\opair{x}{z}}{\opair{z}{y}}
%%%         &&    \forall x,y,z\in\setX
%%%         &&    \text{by $\pwdfa{\opair{x}{z}}{\opair{z}{y}}=0$ hypothesis}
%%%       \\&\eqd \ptfD
%%%         &&    \forall x,y,z\in\setX
%%%         &&    \text{by definition of $\pwdfan$}
%%%       \\&\eqd 2\sigma\brs{\sfrac{1}{2}\pwdfp{p}{x}{z} + \sfrac{1}{2}\pwdfp{p}{z}{y}}^\frac{1}{p}
%%%         &&    \forall x,y,z\in\setX
%%%         &&    \text{by definition of $\ptf$ \xref{def:ptf}}
%%%       \\&\implies \pwdf{x}{y}= 0 
%%%         &&    \forall x,y\in\setX
%%%         &&    \text{because $\pwdfn$ is \prope{non-negative}}
%%%       \\&\implies \opair{x}{z}=\opair{z}{y}
%%%         &&    \forall x,y,z\in\setX
%%%         &&    \text{because $\pwdfn$ is \prope{nondegenerate}}
%%%     \end{align*}
%%%   
%%%   \item Proof that $\pwdfa{\opair{x}{z}}{\opair{z}{y}}=0\impliedby\opair{x}{z}=\opair{z}{y}$ for $r\in\intco{1}{\infty}$:
%%%      \begin{align*}
%%%       \pwdfa{\opair{x}{z}}{\opair{z}{y}}
%%%         &= \pwdfa{\opair{x}{z}}{\opair{x}{z}}
%%%         && \text{by right hypothesis}
%%%       \\&\eqd \ptf(p,\sigma,x,x,z;\pwdfn)
%%%         && \text{by definition of $\pwdfan$}
%%%       \\&\eqd 2\sigma\brs{\sfrac{1}{2}\pwdfp{p}{x}{z} + \sfrac{1}{2}\pwdfp{p}{z}{x}}^\frac{1}{p}
%%%         &&    \text{by definition of $\ptf$ \xref{def:ptf}}
%%%       \\&= 2\sigma\brs{\sfrac{1}{2}\pwdfp{p}{x}{z} + \sfrac{1}{2}\pwdfp{p}{x}{z}}^\frac{1}{p}
%%%         &&    \text{by \prope{symmetric} property of $\pwdfn$ \xref{def:pwdf}}
%%%       \\&= 2\sigma\pwdf{x}{z}
%%%     \end{align*}
%%%   
%%%   \item Proof that $\pwdfn$ satisfies the triangle inequality property  for $r=1$:
%%%     \begin{align*}
%%%       \pwdf{\opair{x}{z}}{\opair{z}{y}}
%%%         &\eqd \brp{\sum_{n=1}^\xN \lambda_n \pwdfan^r(x_n,y_n)}^\frac{1}{r}
%%%         &&    \text{by definition of $\pwdfn$}
%%%       \\&=    \sum_{n=1}^\xN \lambda_n \pwdfa{x_n}{y_n}
%%%         &&    \text{by $r=1$ hypothesis}
%%%       \\&\leq \sum_{n=1}^\xN \lambda_n \brs{\pwdfa{z_n}{x_n}+\pwdfa{z_n}{y_n}}
%%%         &&    \text{by \prope{triangle inequality}}
%%%       \\&=    \sum_{n=1}^\xN \lambda_n \pwdfa{z_n}{x_n} + \sum_{n=1}^\xN \lambda_n \pwdfa{z_n}{y_n}
%%%       \\&=    \brp{\sum_{n=1}^\xN \lambda_n \pwdfan^r(z_n,x_n)}^\frac{1}{r} + \brp{\sum_{n=1}^\xN \lambda_n \pwdfan^r(z_n,y_n)}^\frac{1}{r}
%%%         &&    \text{by $r=1$ hypothesis}
%%%       \\&\eqd \pwdf{\tuplen{z_n}}{\opair{x}{z}} + \pwdf{\tuplen{z_n}}{\opair{z}{y}}
%%%         &&    \text{by definition of $\pwdfn$}
%%%     \end{align*}
%%% 
%%%   \item Proof that $\pwdfn$ satisfies the triangle inequality property  for $r\in\intoo{1}{\infty}$:
%%%     \begin{align*}
%%%       &\pwdf{\opair{x}{z}}{\opair{z}{y}}
%%%       \\&\eqd \brp{\sum_{n=1}^\xN \lambda_n \pwdfan^r(x_n,y_n)}^\frac{1}{r}
%%%         &&    \text{by definition of $\pwdfn$}
%%%       \\&\leq \brp{\sum_{n=1}^\xN \lambda_n \brs{\pwdfan(z_n,x_n)+\pwdfan(z_n,y_n)}^r}^\frac{1}{r}
%%%         &&    \text{by \prope{subaddtive} property \xref{def:metric}}
%%%       \\&=    \brp{\sum_{n=1}^\xN \brs{\lambda_n^\frac{1}{r} \pwdfan(z_n,x_n)+\lambda_n^\frac{1}{r}\pwdfan(z_n,y_n)}^r}^\frac{1}{r}
%%%         &&    \text{by \prope{subaddtive} property \xref{def:metric}}
%%%       \\&\leq \brp{\sum_{n=1}^\xN \brs{\lambda_n^\frac{1}{r}\pwdfan(z_n,x_n)}^r}^\frac{1}{r} 
%%%             + \brp{\sum_{n=1}^\xN \brs{\lambda_n^\frac{1}{r}\pwdfan(z_n,y_n)}^r}^\frac{1}{r}
%%%         &&    \text{by \thme{Minkowski's inequality}} %{ \xref{thm:lp_minkowski}}
%%%       \\&\leq \brp{\sum_{n=1}^\xN \lambda_n \pwdfan^r(z_n,x_n)}^\frac{1}{r} 
%%%             + \brp{\sum_{n=1}^\xN \lambda_n \pwdfan^r(z_n,y_n)}^\frac{1}{r}
%%%       \\&\eqd \pwdf{\tuplen{z_n}}{\opair{x}{z}} + \pwdf{\tuplen{z_n}}{\opair{z}{y}}
%%%         &&    \text{by definition of $\pwdfn$}
%%%     \end{align*}
%%% 
%%%   \item Proof for the $r=\infty$ case:
%%%     \begin{enumerate}
%%%       \item Proof that $\pwdf{\opair{x}{z}}{\opair{z}{y}} = \max\opair{x}{z}$: by \prefp{thm:seq_Mr}
%%%       
%%%   \item Proof that $\opair{x}{z}=\opair{z}{y}\implies\pwdf{\opair{x}{z}}{\opair{z}{y}}=0$:
%%%     \begin{align*}
%%%       \pwdf{\opair{x}{z}}{\opair{z}{y}}
%%%         &\eqd \max\set{\pwdfa{x_n}{y_n}}{n=1,2,\ldots,\xN}
%%%         &&    \text{by definition of $\pwdfn$}
%%%       \\&=    \max\set{\pwdfa{x_n}{x_n}}{n=1,2,\ldots,\xN}
%%%         &&    \text{by $\opair{x}{z}=\opair{z}{y}$ hypothesis}
%%%       \\&=    0
%%%         &&    \text{because $\pwdfan$ is \prope{nondegenerate}}
%%%       \\
%%%     \end{align*}
%%%   \item Proof that $\opair{x}{z}=\opair{z}{y}\impliedby\pwdf{\opair{x}{z}}{\opair{z}{y}}=0$:
%%%     \begin{align*}
%%%       0
%%%         &=    \pwdf{\opair{x}{z}}{\opair{z}{y}}
%%%         &&    \text{by $\pwdf{\opair{x}{z}}{\opair{z}{y}}=0$ hypothesis}
%%%       \\&\eqd \max\set{\pwdfa{x_n}{y_n}}{n=1,2,\ldots,\xN}
%%%         &&    \text{by definition of $\pwdfn$}
%%%       \\\implies \pwdfa{x_n}{y_n}&=0 \text{ for } n=1,2,\ldots,\xN
%%%       \\\implies \opair{x}{z}&=\opair{z}{y}
%%%         &&    \text{because $\pwdfan$ is \prope{nondegenerate}}
%%%       \\
%%%     \end{align*}
%%%   \item Proof that $\pwdfn$ satisfies the triangle inequality property:
%%%     \begin{align*}
%%%       &\pwdf{\opair{x}{z}}{\opair{z}{y}}
%%%       \\&\eqd \max\set{\pwdfa{x_n}{y_n}}{n=1,2,\ldots,\xN}
%%%         &&    \text{by definition of $\pwdfn$}
%%%       \\&\le  \max\set{\pwdfa{x_n}{z_n}+\pwdfa{z_n}{y_n}}{n=1,2,\ldots,\xN}
%%%         %&&    \text{because $\pwdfan$ satisfies the triangle inequality property}
%%%         &&    \text{by \prope{subadditive} property}
%%%       \\&\le  \max\set{\pwdfa{x_n}{z_n}}{n=1,2,\ldots,\xN}
%%%           +   \max\set{\pwdfa{z_n}{y_n}}{n=1,2,\ldots,\xN}
%%%         &&    \text{by \prope{non-negative} property}
%%%       \\&=    \max\set{\pwdfa{z_n}{x_n}}{n=1,2,\ldots,\xN}
%%%           +   \max\set{\pwdfa{z_n}{y_n}}{n=1,2,\ldots,\xN}
%%%         &&    \text{by \prope{symmetry} property}
%%%       \\&\eqd \pwdfn(\tuplen{z_n},\opair{x}{z}) + \pwdfn(\tuplen{z_n},\opair{z}{y})
%%%         &&    \text{by definition of $\pwdfn$}
%%%     \end{align*}
%%%   \end{enumerate}
%%% 
%%%   \item And so by \prefpp{thm:metric_equiv}, $\pwdfn$ is a \fncte{metric} for $r\in\intcc{1}{\infty}$.
%%% \end{enumerate}
%%% 
%%% 
%%% 
%%% \end{proof}
%---------------------------------------
\begin{corollary}
\label{cor:tri_mono}
%---------------------------------------
Let $\ptfD$ be the \fncte{power triangle function} \xref{def:ptf} in the \structe{distance space} \xref{def:dspace} $\dspaceX$.
Let $\omsR$ be the \structe{ordered metric space} with the usual ordering relation $\le$ and usual metric $\absn$ on $\R$.
\corboxp{
  The function $\ptfD$ 
  is \propb{continuous} and \propb{strictly monotone} in $\omsR$ with respect to both the variables $p$ and $\sigma$.
  }
\end{corollary}
\begin{proof}
\begin{enumerate}
  \item Proof that $\ptfD$ is \prope{continuous} and \prope{strictly monotone} with respect to $p$: This follows directly from \prefpp{thm:seq_Mr}.
  \item Proof that $\ptfD$ is \prope{continuous} and \prope{strictly monotone} with respect to $\sigma$:
    \begin{align*}
      \ptfD
        &\eqd 2\sigma\mcom{\brs{\frac{1}{2}\distancep{p}{x}{z} + \frac{1}{2}\distancep{p}{z}{y}}^\frac{1}{p}}{$\ff(p,x,y,z)$}
        && \text{by definition of $\ptf$ \xref{def:ptf}}
      \\&=  2\sigma\ff(p,x,y,z)
        && \text{where $\ff$ is defined as above}
      \\&\implies\text{$\ptf$ is \prope{affine} with respect to $\sigma$}
      \\&\mathrlap{\implies\text{$\ptf$ is \prope{continuous} and \prope{strictly monotone} with respect to $\sigma$:}}
    \end{align*} 
\end{enumerate}
\end{proof}

%---------------------------------------
\begin{corollary}
\label{cor:ftri_means}
%---------------------------------------
Let $\ptfD$ be the \fncte{power triangle function} \xrefh{def:ptf} in the \structe{distance space} \xrefh{def:dspace} $\dspaceX$.
\\\corboxp{$
   \ptfD =
    \brbl{\begin{array}{>{\ds}lDlD}
      2\sigma\max\setn{\distancen(x,z),\,\distancen(z,y)}                                 &for& p=\infty,  & (\fncte{maximum}, {\scs corresponds to} \structe{inframetric space})\\
      2\sigma\brs{\frac{1}{2}\distancep{2}{x}{z} + \frac{1}{2}\distancep{2}{z}{y}}^\frac{1}{2} &for& p=2,       & (\fncte{quadratic mean})\\
       \sigma\brs{\distancen(x,z) + \distancen(z,y)}                                      &for& p=1,       & (\fncte{arithmetic mean}, corresponds to \structe{near metric space})\\
      2\sigma     \sqrt{\distancen(x,z)}\,\sqrt{\distancen(z,y)}                          &for& p=0        & (\fncte{geometric mean})  \\
      4\sigma     \brs{\frac{1}{\distancen(x,z)} + \frac{1}{\distancen(z,y)}}^{-1}        &for& p=-1       & (\fncte{harmonic mean})   \\
      2\sigma\min\setn{\distancen(x,z),\, \distancen(z,y)}                                &for& p=-\infty, & (\fncte{minimum})         \\
    \end{array}}
  $}
\end{corollary}
\begin{proof}
These follow directly from \prefpp{thm:seq_Mr}.
\end{proof}

%---------------------------------------
\begin{corollary}
%---------------------------------------
Let $\dspaceX$ be a \structe{distance space}.
% with \rele{power triangle inequality} \xrefh{def:trirel} $\trirelD$
%defined in terms a \fncte{power triangle function} \xrefh{def:ptf} $\ptfD$ on $\dspaceX$.
\corbox{\begin{array}{rclclcl}
        \ds 2\sigma\min\setn{\distancen(x,z),\, \distancen(z,y)}                         
   &\le&\ds 4\sigma     \brs{\frac{1}{\distancen(x,z)} + \frac{1}{\distancen(z,y)}}^{-1} 
   &\le&\ds 2\sigma     \sqrt{\distancen(x,z)}\,\sqrt{\distancen(z,y)}                   
 \\&\le&\ds  \sigma\brs{\distancen(x,z) + \distancen(z,y)}                               
   &\le&\ds 2\sigma\max\setn{\distancen(x,z),\,\distancen(z,y)}                          
\end{array}}
\end{corollary}
\begin{proof}
These follow directly from \prefpp{cor:means}.
\end{proof}

%%---------------------------------------
%\begin{proposition}
%%---------------------------------------
%Let $\dspaceX$ be a \structe{distance space} \xref{def:dspace}.
%\\\propboxp{
%  $\sigma\brs{\lambda_1\distancen^p(x,z) + \lambda_2\distancen^p(z,y)}^\frac{1}{p} =$
%  \\\indentx$\brb{\begin{array}{lDllD}
%     %\brs{\distancen^p(x,z) + \distancen^p(z,y)}^\frac{1}{p}          &for&            & \lambda=1            \sigma=1, & (right side of {\rele{$p^{th}$ order triangle inequality}})\\
%     %\max\setn{\distancen(x,z) + \distancen(z,y)}                     &for& p=\infty,  & \lambda=\frac{1}{2}  \sigma=1, & (right side of \rele{inframetric inequality})\\
%      \sigma\max\setn{\distancen(x,z),\,\distancen(z,y)}               &for& p=\infty,  &                                & (right side of \rele{\txsigma-inframetric inequality})\\
%     %          \distancen(x,z) + \distancen(z,y)                      &for& p=1,       & \lambda=1            \sigma=1, & (right side of \rele{triangle    inequality})\\
%      \frac{1}{2}\sigma\brs{\distancen(x,z) + \distancen(z,y)}                       &for& p=1,       & \lambda_1=\lambda_2=\frac{1}{2}& (right side of \rele{relaxed triangle inequality})\\
%      \sigma     \brs{\distancen(x,z)}^{\lambda_1} \brs{\distancen(z,y)}^{\lambda_2} &for& p=0        &                                & \\
%      \sigma     \brs{\frac{\lambda_1}{\distancen(x,z)} + \frac{\lambda_2}{\distancen(z,y)}}^{-1}&for& p=-1 &                 & \\
%      \sigma\min\setn{\distancen(x,z),\, \distancen(z,y)}              &for& p=-\infty, &                                & 
%    \end{array}}$
%  }
%\end{proposition}
%\begin{proof}
%These follow from direct parameter ($\sigma$, $p$) substitution and/or \prefpp{thm:seq_Mr}.
%\begin{align*}
%  \brb{\sigma\brs{\lambda\distancen^p(x,z) + \lambda\distancen^p(z,y)}^\frac{1}{p}}_{p=\infty}
%    &= \sigma\brb{\brs{2\lambda}^\frac{1}{p}\brs{\frac{1}{2}\distancen^p(x,z) + \frac{1}{2}\distancen^p(z,y)}^\frac{1}{p}}_{p=\infty}
%   &&\ocom{= \sigma\max\setn{\brs{\distancen(x,z),\,\distancen(z,y)}}}{by \prefp{thm:seq_Mr}}
%  %&& \text{by \prefpp{thm:seq_Mr}}
%  \\
%  \brb{\sigma\brs{\lambda\distancen^p(x,z) + \lambda\distancen^p(z,y)}^\frac{1}{p}}_{p=0,\lambda=\frac{1}{2}}
%    &= \sigma\brb{\brs{\frac{1}{2}\distancen^p(x,z) + \frac{1}{2}\distancen^p(z,y)}^\frac{1}{p}}_{p=0}
%   &&= \sigma\distancen^\frac{1}{2}(x,z)\distancen^\frac{1}{2}(z,y)
%   %&& \text{by \prefpp{thm:seq_Mr}}
%  \\
%  \brb{\sigma\brs{\lambda\distancen^p(x,z) + \lambda\distancen^p(z,y)}^\frac{1}{p}}_{p=-1}
%    &= \sigma\brb{\brs{\lambda}^\frac{1}{p}\brs{\distancen^p(x,z) + \distancen^p(z,y)}^\frac{1}{p}}_{p=-1}
%   &&= \frac{\sigma}{\lambda}\brs{\frac{1}{\distancen(x,z)} + \frac{1}{\distancen(z,y)}}^{-1}
%   %&& \text{by \prefpp{thm:seq_Mr}}
%  \\
%  \brb{\sigma\brs{\lambda\distancen^p(x,z) + \lambda\distancen^p(z,y)}^\frac{1}{p}}_{p=-\infty}
%    &= \sigma\brb{\brs{2\lambda}^\frac{1}{p}\brs{\frac{1}{2}\distancen^p(x,z) + \frac{1}{2}\distancen^p(z,y)}^\frac{1}{p}}_{p=-\infty}
%   &&= \sigma\min\setn{\brs{\distancen(x,z),\,\distancen(z,y)}}
%   %&& \text{by \prefpp{thm:seq_Mr}}
%\end{align*}
%\end{proof}

%=======================================
\subsection{Properties of power distance spaces}
\label{sec:pdspace_propd}
%=======================================
The \prope{power triangle inequality} property of a \structe{power distance space}
axiomatically endows a metric with an upper bound. 
\pref{lem:pdspace_ineq} (next) demonstrates that there is a complementary lower bound 
somewhat similar in form to the \prope{power triangle inequality} upper bound.
In the special case where $2\sigma=2^\frac{1}{p}$, 
%For the case $\opair{p}{\sigma}=\opair{1}{1}$, 
the lower bound helps provide a simple proof of the \prope{continuity} of 
a large class of \fncte{power distance function}s \xref{thm:pdspace_continuous}.
The inequality $2\sigma\le2^\frac{1}{p}$ is a special relation in this paper and appears repeatedly in this paper;
it appears as an inequality in \prefpp{lem:tri_open}, \prefpp{cor:tspace_base} and \prefpp{cor:oball_open},
and as an equality in \pref{lem:pdspace_ineq} (next) and \prefpp{thm:pdspace_continuous}.
It is plotted in \prefpp{fig:sigmap}.
\begin{figure}[t]
  %\gsize%
  \footnotesize%
  \centering%
  \includegraphics{../common/math/graphics/pdfs/trirel_sigmap.pdf}
  \caption{$\sigma = \frac{1}{2}(2^{\frac{1}{p}}) = 2^{\frac{1}{p}-1}$ or $p=\frac{\ln2}{\ln(2\sigma)}$ 
  %(see  \prefpp{thm:openball_openset})
  \label{fig:sigmap}
  \scs(see \prefp{lem:pdspace_ineq}, \prefp{lem:tri_open}, \prefp{cor:oball_open}, \prefp{cor:tspace_base}, and \prefp{thm:pdspace_continuous}).
  }
\end{figure}
%---------------------------------------
\begin{lemma}
\footnote{
  in \structe{metric space} ($\opair{p}{\sigma}=\opair{1}{1}$):
  \citerpg{dieudonne1969}{28}{1406727911},
  \citerpg{michel1993}{266}{048667598X},
  \citerpgc{berberian1961}{37}{0821819127}{Theorem~II.4.1}
  }
\label{lem:pdspace_ineq}
%---------------------------------------
Let $\pdspaceX$ be a \structe{power triangle triangle space} \xref{def:pdspace}. 
Let $\absn$ be the \fncte{absolute value} function \xref{def:abs}.
Let $\max\setn{x,y}$ be the maximum and $\min\setn{x,y}$ the minimum of any $x,y\in\Rx$.
Then, for all $\opair{p}{\sigma}\in\Rx\times\Rp$,
%Then, for all $x,y,z\in\setX$
\lembox{\begin{array}{Frc>{\ds}lCD}
  %\opair{p}{\sigma}\in\R\times\Rp \quad\implies\quad
  1.&
  \distancep{p}{x}{y} &\ge& 
    \max\brb{0,\,
      \frac{2}{(2\sigma)^p}\distancep{p}{x}{z}-\distancep{p}{z}{y},\,
      \frac{2}{(2\sigma)^p}\distancep{p}{y}{z}-\distancep{p}{z}{x}
      }
  &\forall x,y,z\in\setX
  & and
  \\
  2.&
  %\opair{p}{\sigma}\in\set{\R\times\Rp}{p\neq0,\,2\sigma=2^\frac{1}{p}}\setu\setn{\opair{\infty}{\sfrac{1}{2}},\,\opair{-\infty}{\sfrac{1}{2}}} 
  %\quad\implies\quad
  \distance{x}{y}&\ge&\abs{\distance{x}{z} - \distance{z}{y}} 
  \quad\text{if \;$p\neq0$\; and \;$2\sigma=2^\frac{1}{p}$}
  &\forall x,y,z\in\setX .
\end{array}}
\end{lemma}
\begin{proof}
\begin{enumerate}
  \item lemma: $\frac{2}{(2\sigma)^p}\distancep{p}{x}{z}-\distancep{p}{z}{y} \le \distancep{p}{x}{y}\quad\forall\opair{p}{\sigma}\in\Rx\times\Rp$: 
        Proof:\label{ilem:pdspace_ineq}
    %\begin{enumerate}
      %\item Proof for $\opair{p}{\sigma}\in(\Rx\setd\setn{0})\times\Rp$: 
        \begin{align*}
          \frac{2}{(2\sigma)^p}\distancep{p}{x}{z}-\distancep{p}{z}{y}
            &\le \frac{2}{(2\sigma)^p}\brs{2\sigma\brs{\sfrac{1}{2}\distancep{p}{x}{y}+\sfrac{1}{2}\distancep{p}{y}{z}}^\frac{1}{p}}^p - \distancep{p}{z}{y}
            && \text{by \prope{power triangle inequality}}
          \\&=   \frac{2(2\sigma)^p}{(2\sigma)^p}\brs{\sfrac{1}{2}\distancep{p}{x}{y}+\sfrac{1}{2}\distancep{p}{y}{z}} - \distancep{p}{z}{y}
          \\&=   \brs{\distancep{p}{x}{y}+\distancep{p}{y}{z}} - \distancep{p}{y}{z}
            && \text{by \prope{symmetric} property of $\distancen$}
          \\&=   \distancep{p}{x}{y}
        \end{align*}

      %\item Proof for $\opair{p}{\sigma}\in\setn{0}\times\Rp$: For this case the lemma trivially holds because 
      %  \\\indentx$
      %      \brs{\frac{2}{(2\sigma)^p}\distancep{p}{x}{z}-\distancep{p}{z}{y}}_{p=0} 
      %         = \frac{2}{1}-1 = 1 \le 1 =   \brs{\distancep{p}{x}{y}}_{p=0}
      %      $

      %\item Proof for $\opair{p}{\sigma}=\opair{-\infty}{\frac{1}{2}}$:
      %  \begin{align*}
      %    \brs{\frac{2}{(2\sigma)^p}\distancep{p}{x}{z}-\distancep{p}{z}{y}}_{\opair{p}{\sigma}=\opair{-\infty}{\frac{1}{2}}}
      %      &= \brs{\frac{2}{1}\distancep{p}{x}{z}-\distancep{p}{z}{y}}_{\opair{p}{\sigma}=\opair{-\infty}{\frac{1}{2}}}
      %    \\&\le \brs{2\brs{2\sigma\brs{\sfrac{1}{2}\distancep{p}{x}{y}+\sfrac{1}{2}\distancep{p}{y}{z}}^\frac{1}{p}}^p - \distancep{p}{z}{y}}_{\opair{p}{\sigma}=\opair{-\infty}{\frac{1}{2}}}
      %      && \text{by \prope{power triangle inequality}}
      %    \\&= \brs{\brs{\distancep{p}{x}{y}+\distancep{p}{y}{z}} - \distancep{p}{z}{y}}_{p=-\infty}
      %    \\&= \brs{\brs{\distancep{p}{x}{y}+\distancep{p}{y}{z}} - \distancep{p}{y}{z}}_{p=-\infty}
      %      && \text{by \prope{symmetric} property of $\distancen$}
      %    \\&=   \distancep{p}{x}{y}
      %  \end{align*}
    %\end{enumerate}

  \item Proof for $\opair{p}{\sigma}\in\Rx\times\Rp$ case:  \label{item:pdspace_max}
    \\\indentx$\begin{array}{rclclM}
      \distancep{p}{x}{y} & &                    &\ge& \frac{2}{(2\sigma)^p}\distancep{p}{x}{z}-\distancep{p}{z}{y} & \text{by \pref{ilem:pdspace_ineq}}
    \\\distancep{p}{x}{y} &=&\distancep{p}{y}{x} &\ge& \frac{2}{(2\sigma)^p}\distancep{p}{y}{z}-\distancep{p}{z}{x} & \text{by \prope{commutative} property of $\distancen$ and \pref{ilem:pdspace_ineq}}
    \\\distancep{p}{x}{y} & &                    &\ge& 0                                                            & \text{by \prope{non-negative} property of $\distancen$ \xref{def:dspace}}
    \end{array}$\\
  The rest follows because $\fg(x)\eqd x^\frac{1}{p}$ is \prope{strictly monotone} in $\clFrr$.

  \item Proof for $2\sigma=2^\frac{1}{p}$ case:
        \begin{align*}
          \distance{x}{y} 
            &\ge  \max\brb{0,\,
                    \frac{2}{(2\sigma)^p}\distancep{p}{x}{z}-\distancep{p}{z}{y},\,
                    \frac{2}{(2\sigma)^p}\distancep{p}{y}{z}-\distancep{p}{z}{x}
                    }^{\frac{1}{p}}
            && \text{by \prefpp{item:pdspace_max}}
          \\&=    \max\brb{0,\,
                    \distance{x}{z}-\distance{z}{y},\,
                    \distance{y}{z}-\distance{z}{x}
                    }
            && \text{by $2\sigma=2^\frac{1}{p}$ hyp. $\iff\frac{2}{(2\sigma)^p}=1$}
          \\&=    \max\brb{0,\,
                    \brp{\distance{x}{z}-\distance{z}{y}},\,
                   -\brp{\distance{x}{z}-\distance{z}{y}}
                    }
            && \text{by \prope{symmetric} property of $\distancen$}
          \\&= \abs{\brp{\distance{x}{z}-\distance{z}{y}}}
        \end{align*}
\end{enumerate}
\end{proof}


%\begin{figure}[h]
%  \gsize%
%  \centering%
%  %%============================================================================
% Daniel J. Greenhoe
% LaTeX file
%============================================================================
  \begin{pspicture}(-1,-3.5)(5.5,4)%
    \psaxes[linecolor=axis,labels=all,ticks=all,showorigin=true]{<->}(0,0)(-1.5,-3.5)(5,3.5)%
    \psplot[plotpoints=128,linecolor=blue]{0.63}{5}{2 log x log 2 log add div}% 
    %\psplot[plotpoints=1024,linecolor=blue,linewidth=0.75pt]{0.00000001}{4}{2 ln x 0.01 mul ln 2 ln add div}% 
    \psplot[plotpoints=256,linecolor=blue]{0.000001}{0.4}{2 ln x 2 mul ln div}% 
    %\psplot[plotpoints=128,linecolor=blue]{0.01}{0.4}{2 log x log 2 log add div}% 
    %\psline(0,0)
    \pnode(0.06766764,-0.34657359){ipnt}% inflection point at (ln2/(-2), (1/2)e^{-2})
    \pnode(1,-0.7){lpnt}% label for inflection point
    \psline[linestyle=dotted,dotsep=2pt,linecolor=red](0,1)(1,1)%
    \psline[linestyle=dotted,dotsep=2pt,linecolor=red](1,0)(1,1)%
    \psline[linestyle=dotted,dotsep=2pt,linecolor=red](0.5,-3.5)(0.5,3.5)%
    \psline[linecolor=red,linewidth=0.75pt]{->}(lpnt)(ipnt)%
    \rput[tl]{0}(lpnt){\begin{tabular}{@{}c}%
      second order inflection point at\\% 
      $\opair{\sigma}{p}=\opair{\frac{1}{2}e^{-2}}{-\frac{1}{2}\ln2}$%
      \end{tabular}}%
    \uput[0]{0}(0,3.5){$p$}%
    \uput[0]{0}(5,0){$\sigma$}%
    \uput[-60]{0}(0.5,0){$\frac{1}{2}$}%
  \end{pspicture}%

%  %============================================================================
% Daniel J. Greenhoe
% LaTeX file
%============================================================================
  \begin{pspicture}(-8,-0.65)(8,5.2)%
    %-----------------------------------
    % plots
    %-----------------------------------
    \psaxes[linecolor=axis,labels=all,ticks=all,showorigin=true,xAxis=false,subticks=2]{->}(0,0)(-3.5,0)(3.5,5)%
    \psaxes[linecolor=axis,labels=all,ticks=all,showorigin=true,yAxis=false,subticks=2]{<->}(0,0)(-3.5,0)(3.5,5)%
    \psplot[plotpoints=128,linecolor=blue,arrows=<->]{0.3}{3.5}{2 1 x div exp 2 div}%
    \psplot[plotpoints=256,linecolor=blue,arrows=<-o]{-3.5}{-0.001}{2 1 x div exp 2 div}%
    %-----------------------------------
    % label points
    %-----------------------------------
    \pnode(-1,4){Lgeo}\pnode(0,0.5){Pgeo}% % geometric inequality
    \pnode(-1,3){Lip}\pnode(-0.34657359,0.06766764){Pip}% inflection point at (ln2/(-2), (1/2)e^{-2})
    \pnode(-1.5,2){Lhar}\pnode(-1,0.25){Phar}% % harmonic inequality
    %
    \pnode(1,4){Lsmr}\pnode(0.5,2){Psmr}% % square mean root inequality
    \pnode(1.4,3){Lms}\pnode(1,1){Pms}% % triangle inequality
    \pnode(1.5,2){Lqm}\pnode(2,0.7071){Pqm}% % quadratic mean inequality
    %-----------------------------------
    % label arrows
    %-----------------------------------
    \psline[linestyle=dotted,dotsep=2pt,linecolor=red](0,1)(1,1)%
    \psline[linestyle=dotted,dotsep=2pt,linecolor=red](1,0)(1,1)%
    \psline[linestyle=dotted,dotsep=2pt,linecolor=red](0.5,0)(0.5,2)%
    \psline[linestyle=dotted,dotsep=2pt,linecolor=red](0,2)(0.5,2)%
    \psline[linestyle=dotted,dotsep=2pt,linecolor=red](2,0)(2,0.707)%
    \psline[linestyle=dotted,dotsep=2pt,linecolor=red](0,0.707)(2,0.707)%
    \psline[linestyle=dotted,dotsep=2pt,linecolor=red](-3.5,0.5)(3.5,0.5)%
    \psline[linecolor=red,linewidth=0.75pt]{->}(Lip)(Pip)%
    \psline[linecolor=red,linewidth=0.75pt]{->}(Lms)(Pms)%
    \psline[linecolor=red,linewidth=0.75pt]{->}(Lsmr)(Psmr)%
    \psline[linecolor=red,linewidth=0.75pt]{->}(Lgeo)(Pgeo)%
    %\psline[linecolor=red,linewidth=0.75pt]{->}(Lqm)(Pqm)%
    \psline[linecolor=red,linewidth=0.75pt]{->}(2.2,1.89)(Pqm)%
    \psline[linecolor=red,linewidth=0.75pt]{->}(Lhar)(Phar)%
    %-----------------------------------
    % labels
    %-----------------------------------
    \rput[r]{0}(Lip){$2^\mathrm{nd}$ order inflection point at $\opair{-\frac{1}{2}\ln2}{\frac{1}{2}e^{-2}}$}%
    %\rput[r]{0}(Lip){\begin{tabular}{r@{}}%
    %  second order inflection point at\\%
    %  $\opair{p}{\sigma}=\opair{-\frac{1}{2}\ln2}{\frac{1}{2}e^{-2}}\approx\opair{-0.347}{0.0677}$
    %  \end{tabular}}%
    \rput[l]{0}(Lms){\structe{triangle inequality} at $\opair{p}{\sigma}=\opair{1}{1}$}%
    \rput[l]{0}(Lsmr){\structe{square mean root (SMR) inequality} at $\opair{\frac{1}{2}}{2}$}%
    \rput[tl]{0}(3.6,0.82){\begin{tabular}{@{}r@{}}%
      to \structe{inframetric inequality} at\\%
      $\opair{p}{\sigma}=\opair{\infty}{\frac{1}{2}}$
      \end{tabular}}%
    \rput[tr]{0}(-3.6,0.65){\begin{tabular}{@{}r@{}}%
      to \structe{minimal inequality} at\\%
      $\opair{p}{\sigma}=\opair{\infty}{\frac{1}{2}}$
      \end{tabular}}%
    \rput[l]{0}(Lqm){\structe{quadratic mean inequality} at $\opair{2}{\frac{\sqrt{2}}{2}}$}%
    \rput[r]{0}(Lgeo){\structe{geometric inequality} at $\opair{p}{\sigma}=\opair{0}{\frac{1}{2}}$}%
    \rput[r]{0}(Lhar){\structe{harmonic inequality} at $\opair{p}{\sigma}=\opair{-1}{\frac{1}{4}}$}%
    \uput[0]{0}(3.5,0){$p$}%
    \uput[180]{0}(0,5){$\sigma$}%
    \uput[0]{0}(0,0.5){$\frac{1}{2}$}%
  \end{pspicture}%

%  \caption{$\sigma = 2^{\frac{1}{p}-1}$ or $p=\frac{\ln2}{\ln(2\sigma)}$ for \prefpp{thm:pdspace_continuous}\label{fig:sigmap}}
%\end{figure}
%\begin{figure}[h]
%  \centering%
%  \gsize%
%  \psset{unit=2mm}%============================================================================
% Daniel J. Greenhoe
% LaTeX file
% nominal unit = 30mm
%============================================================================
\begin{pspicture}(-1.1,-1.1)(1.1,1.1)%
  %-------------------------------------
  % options
  %-------------------------------------
  \psset{
    dotsize=5pt,
    %labelsep=5pt,
    }
  %-------------------------------------
  % axes
  %-------------------------------------
  %\psline[linecolor=axis]{<->}(-120,0)(120,0)% x-axis
  %\psline[linecolor=axis]{<->}(0,-120)(0,120)% y-axis
  %-------------------------------------
  % nodes
  %-------------------------------------
  \pnode(  0,  0){o}% origin (and center of outer ball)
  \pnode(  0, .55){p}% point p (and center of inner ball)
  \pnode(-.25, .45){q}% point q
  \pnode(  0, 1.00){otop}% top of outer ball
  \pnode(  0,-1.00){bottom}% top of outer ball
  \pnode(.71,-.71){obr}% bottom right corner of outer ball
  \pnode(-.71,-.71){obl}% bottom right corner of outer ball
  \pnode( .2828,.8328){itr}% bottom right corner of inner ball
  \pnode( .2828,.2672){ibr}% bottom right corner of inner ball
  \pnode(-.2828,.2672){ibl}% bottom right corner of inner ball
  \pnode(1.00,   0){right}% right side of outer ball
  %-------------------------------------
  % objects
  %-------------------------------------
  \pscircle[linecolor=blue,linestyle=dashed](o){1}% B(o,r) (outer ball)
  \pscircle[linecolor=red,linestyle=dashed](p){0.40}%   B(p,r_p) (ball centered at p with radius r_p)
  %\ncline{->}{o}{q}     \naput{$\metric{\theta}{q}$}%
  \ncline{->}{p}{itr}   \nbput{$r_p\le r-\metric{\theta}{p}$}%
  \ncline{->}{o}{obr}   \ncput*{$r$}% radius line
  \ncline{->}{o}{p}     \nbput{$\metric{\theta}{p}$}% 
  \psdot(o)             \uput{5pt}[225](o){$\theta$}% origin (and center of outer ball)
  \psdot(p)             \uput{5pt}[135](p){$p$}% point p
  %\psdot(q)             \uput{5pt}[ 90](q){$q$}% point q
  \uput{5pt}[45](obl){$\ball{\theta}{r}$}
  \uput[-135](ibl){$\ball{p}{r_p}$}
\end{pspicture}%
%
%  %\input{ballinball1.tex}
%  \caption{
%  %[Every point in an \sete{open ball} \xref{def:ball} is contained in an {open ball} that is contained in the original open ball]
%  %Every point in an open ball is contained in an open ball that is contained in the original open ball
%  \structe{open ball} that is \prope{open} (see \prefp{thm:openball_openset})
%  \label{fig:ms_openball_openset}
%  }
%\end{figure}
%---------------------------------------
\begin{theorem}
\label{thm:openball_openset}
%---------------------------------------
Let $\pdspaceX$ be a \structe{power distance space} \xref{def:pdspace}.
Let $\balln$ be an \structe{open ball} \xref{def:ball} on $\dspaceX$. 
Then for all $\opair{p}{\sigma}\in(\Rx\setd\setn{0})\times\Rp$,
\thmbox{
  \brb{\begin{array}{FlD}
    A. & 2\sigma \le 2^{\frac{1}{p}} & and\\
    B. & q\in\ball{\theta}{r}  
  \end{array}}
  \quad\implies\quad
  \brb{\begin{array}{Fl}
    1. & \exists r_q\in\Rp \st\\ 
       & \ball{q}{r_q}\subseteq\ball{\theta}{r}
  \end{array}}
  \quad\implies\quad
  \brb{\begin{array}{Fl}
       &\\
    B. & q\in\ball{\theta}{r}  
  \end{array}}
  }
\end{theorem}
\begin{proof}
\begin{enumerate}
  \item lemma: \label{ilem:openball_openset_lem}
    \begin{align*}
      q\in\ball{\theta}{r}
        &\iff \distance{\theta}{q} < r
       && \text{by definition of \structe{open ball} \xref{def:ball}}
      \\&\iff 0< r-\distance{\theta}{q} 
        && \text{by field property of real numbers}
      \\&\iff \exists r_q\in\Rp \st 0< r_q< r-\distance{\theta}{q}
        &&\text{by \thme{The Archimedean Property}\footnotemark}
    \end{align*}
    %\begin{align*}
    %  q\in\ball{\theta}{r}
    %    &\iff \distance{\theta}{q} < r
    %    && \text{by definition of \structe{open ball}}
    %   %&& \text{by definition of \structe{open ball} \xref{def:ball}}
    %  \\&\iff (2\sigma)^p\distance{\theta}{q} < (2\sigma)^p r
    %    && \text{because $(2\sigma)^p>0$}
    %  \\&\iff 0<(2\sigma)^p r-(2\sigma)^p\distance{\theta}{q} 
    %    && \text{by field property of real numbers}
    %  \\&\iff \exists r_q\in\Rp \st 0<(2\sigma)^p r_q<(2\sigma)^p r-(2\sigma)^p\distance{\theta}{q}
    %    &&\text{by \thme{The Archimedean Property}\footnotemark}
    %\end{align*}
\footnotetext{
  \citerpgc{ab}{17}{0120502577}{Theorem 3.3 (``\thme{The Archimedean Property}") and Theorem 3.4},
  \citerpgc{zorich2004}{53}{3540403868}{$6^\circ$ (``\thme{The principle of Archimedes}") and $7^\circ$}
  }

  \item Proof that $(A),(B)\implies(1)$:
    \begin{align*}
      \ball{q}{r_q}
        &\eqd \set{x\in\setX}{\distance{q}{x}<r_q}
        &&    \text{by definition of \structe{open ball} \xref{def:ball}}
      \\&=    \set{x\in\setX}{\distancep{p}{q}{x}< r_q^p\in\Rp}
        &&    \text{because $\ff(x)\eqd x^p$ is \prope{monotone}}
      \\&\subseteq \set{x\in\setX}{\distancep{p}{q}{x}< r^p-\distancep{p}{\theta}{q}}
        &&    \text{by hypothesis B and \prefp{ilem:openball_openset_lem}}
      \\&=    \set{x\in\setX}{\distancep{p}{\theta}{q}+\distancep{p}{q}{x}< r^p}
        &&    \text{by field property of real numbers}
      \\&=    \set{x\in\setX}{\brs{\distancep{p}{\theta}{q}+\distancep{p}{q}{x}}^{\frac{1}{p}}<r}
        &&    \text{because $\ff(x)\eqd x^{\frac{1}{p}}$ is \prope{monotone}}
      \\&\subseteq \set{x\in\setX}{2^{1-\sfrac{1}{p}}\sigma\brs{\distancep{p}{\theta}{q}+\distancep{p}{q}{x}}^{\frac{1}{p}}<r}
        &&    \text{by hypothesis A which implies $2^{1-\sfrac{1}{p}}\sigma\le1$}
      \\&= \set{x\in\setX}{2\sigma\brs{\sfrac{1}{2}\distance{\theta}{q}{x}+\sfrac{1}{2}\distancep{p}{q}{x}}^{\frac{1}{p}}<r}
        &&    \text{because $2^{1-\sfrac{1}{p}}\sigma=2\sigma(\sfrac{1}{2})^{\frac{1}{p}}$}
      \\&\eqd \set{x\in\setX}{\ptf(p,\sigma,\theta,x,q)<r}
        && \text{by definition of $\ptf$ \xref{def:ptf}}
      \\&\subseteq \set{x\in\setX}{\distance{\theta}{x}<r}
        && \text{by definition of $\pdspaceX$ \xref{def:pdspace}}
      \\&\eqd \ball{\theta}{r}
        &&    \text{by definition of \structe{open ball} \xref{def:ball}}
    \end{align*}
    %\begin{align*}
    %  \ball{q}{r_q}
    %    &\eqd \set{x\in\setX}{\distance{q}{x}<r_q\in\Rp}
    %    &&    \text{by definition of \structe{open ball} \xref{def:ball}}
    %  \\&=    \set{x\in\setX}{(2\sigma)^p\distancep{p}{q}{x}<(2\sigma)^p r_q^p\in\Rp}
    %    &&    \text{because $\ff(x)\eqd x^p$ is \prope{monotone}}
    %  \\&\subseteq \set{x\in\setX}{(2\sigma)^p\distancep{p}{q}{x}<(2\sigma)^p r^p-(2\sigma)^p\distancep{p}{\theta}{q}}
    %    &&    \text{by left hypothesis and \prefp{ilem:openball_openset_lem}}
    %  \\&=    \set{x\in\setX}{(2\sigma)^p\distance{\theta}{q}{x}+(2\sigma)^p\distancep{p}{q}{x}<(2\sigma)^p r^p}
    %    &&    \text{by property of real numbers}
    %  \\&=    \set{x\in\setX}{\distance{\theta}{q}{x}+\distancep{p}{q}{x}<r^p}
    %    &&    \text{by property of real numbers\footnotemark}
    %  \\&=    \set{x\in\setX}{\brs{\distance{\theta}{q}{x}+\distancep{p}{q}{x}}^{\frac{1}{p}}<r}
    %    &&    \text{by property of real numbers}
    %  \\&\subseteq \set{x\in\setX}{\brs{\sfrac{1}{2}\distance{\theta}{q}{x}+\sfrac{1}{2}\distancep{p}{q}{x}}^{\frac{1}{p}}<r}
    %    && \text{by property of real numbers}
    %  \\&\subseteq \set{x\in\setX}{2\sigma\brs{\sfrac{1}{2}\distance{\theta}{q}{x}+\sfrac{1}{2}\distancep{p}{q}{x}}^{\frac{1}{p}}<r}
    %    && \text{by $\sigma\in\intoc{0}{\sfrac{1}{2}}$ hypothesis}
    %  \\&\eqd \set{x\in\setX}{\ptf(p,\sigma,\theta,x,q)<r}
    %    && \text{by definition of $\ptf$ \xref{def:ptf}}
    %  \\&\subseteq \set{x\in\setX}{\distance{\theta}{x}<r}
    %    && \text{by definition of $\pdspaceX$ \xref{def:pdspace}}
    %  \\&\eqd \ball{\theta}{r}
    %    &&    \text{by definition of \structe{open ball} \xref{def:ball}}
    %\end{align*}
    %\footnotetext{
    %  Here the $(2\sigma)^2$ factor disappears, and with it some hope of relaxing the $\sigma\in\intoc{0}{\sfrac{1}{2}}$ 
    %  constraint.
    %  }

  \item Proof that $(B)\impliedby(1)$:
    \begin{align*}
      q &\in    \set{x\in\setX}{\distance{q}{x}=0}
        &&    \text{by \prope{nondegenerate} property \xref{def:dspace}}
      \\&\subseteq \set{x\in\setX}{\distance{q}{x}<r_q}
        &&    \text{because $r_q>0$}
      \\&\eqd \ball{q}{r_q}
        &&    \text{by definition of \structe{open ball} \xref{def:ball}}
      \\&\subseteq \ball{\theta}{r}
        &&    \text{by hypothesis 2}
    \end{align*}
\end{enumerate}
\end{proof}

%---------------------------------------
\begin{corollary}
\label{cor:tspace_openball}
%---------------------------------------
Let $\pdspaceX$ be a \structe{power distance space}.
Then for all $\opair{p}{\sigma}\in(\Rx\setd\setn{0})\times\Rp$,
\thmbox{
  %\brb{2^{1-\sfrac{1}{p}}\sigma \le 1}
  \brb{2\sigma \le 2^{\frac{1}{p}}}
  \quad\implies\quad
  \brb{\text{every \structe{open ball} in $\dspaceX$ is \prope{open}}}
  }
\end{corollary}
\begin{proof}
This follows from \prefpp{thm:openball_openset} and \prefpp{thm:baseoball}.
\end{proof}

%\prefpp{cor:tspace_base} shows that in a \structe{distance space} \xref{def:dspace} $\dspaceX$ with 
%the metric $\distancen$ always induces a topology $\topT$ on $\setX$.
%The set $\setX$ together with  topology $\topT$ is a \structe{topological space}.
%More specifically, the set of \structe{open balls} in a metric space form a \structe{base} for a \structe{topological space}.
%Therefore, {\em every} \structe{metric space} \xref{def:metric}\index{space!metric} {\em is} a topological space,
%and everything that is true of a topological space is also true for all \structe{metric space}\index{space!metric}s.
%---------------------------------------
\begin{corollary}
\label{cor:tspace_base}
%\label{thm:(X,d)->(X,t)}
%---------------------------------------
Let $\pdspaceX$ be a \structe{power distance space}.
Let $\baseB$ be the set of all \structe{open ball}s in $\dspaceX$.
Then for all $\opair{p}{\sigma}\in(\Rx\setd\setn{0})\times\Rp$,
%Let $\ball{x}{r}$ be an \structe{open ball} \xrefh{def:ball} on $\dspaceX$.
%\propboxp{
%  If for some $p\in\Rx$ and $\sigma\in\Rp$, $\ptfD\le\ptf(1,1;x,y,z)$ ${\scy\forall x,y,z\in\setX}$ then
%  the set of all \structe{open balls} in $\dspaceX$ is a \structe{base} for the topological space $\topspaceX$ where
%  \\\indentx
%  $\ds\topT \eqd \set{\setU\in\psetX}{\text{$\setU$ is the union of balls in $\dspaceX$}}$.
%  }
%\\Let $\baseB\eqd\set{\ball{x}{r}}{x\in\setX,\,r\in\Rp}$.
%Let $\topT \eqd \set{\setU\in\psetX}{\text{$\setU$ is the union of elements in $\baseB$}}$.
\corbox{
  \brb{2\sigma \le 2^{\frac{1}{p}}}
  \quad\implies\quad
  \brb{\begin{array}{M}
    $\baseB$ is a \structe{base} for $\topspaceX$
    %2. & $\topspaceX$ is a \structe{topological space} & and \\
  \end{array}}
  }
\end{corollary}
\begin{proof}
\begin{enumerate}
  \item The set of all \structe{open balls} in $\dspaceX$ is a \structe{base} for $\topspaceX$ by 
        \prefpp{cor:tspace_openball} and \prefpp{thm:basex}.
        
  \item $\topT$ is a topology on $\setX$ by \prefpp{def:baseB}.
\end{enumerate}
\end{proof}

\begin{figure}[h]
  \center
  \includegraphics{../common/math/graphics/pdfs/ballinopen.pdf}
  \caption{\structe{open set} (see \prefp{lem:tri_open}) \label{fig:ms_open} }
\end{figure}
\pref{lem:tri_open} (next) demonstrates that every point in an open set is contained in an open ball that is 
contained in the original open set (see also \prefp{fig:ms_open}).
%---------------------------------------
\begin{lemma}
\label{lem:tri_open}
%---------------------------------------
Let $\pdspaceX$ be a \structe{power distance space}.
Let $\balln$ be an \structe{open ball} on $\dspaceX$.
Then for all $\opair{p}{\sigma}\in(\Rx\setd\setn{0})\times\Rp$,
\\\lemboxp{$%
  \brb{\begin{array}{FlD}
    A. & 2\sigma \le 2^{\frac{1}{p}} & and\\
    B. & \mc{2}{M}{$\setU$ is \prope{open} in $\dspaceX$}
  \end{array}}
  \implies
  \brb{\begin{array}{Fl}
    1. & \forall x\in\setU,\; \exists r\in\Rp \st\\ 
       & \ball{x}{r}\subseteq \setU
  \end{array}}
  \implies
  \brb{\begin{array}{FM}
    B. & $\setU$ is\\ 
       & \prope{open} in $\dspaceX$
  \end{array}}
  $}
\end{lemma}
\begin{proof}
\begin{enumerate}
  \item Proof that for ($(A),(B)\implies(1)$):
    \begin{align*}
      \setU
        &= \Setu\set{\ball{x_\gamma}{r_\gamma}}{\ball{x_\gamma}{r_\gamma}\subseteq\setU}
        && \text{by left hypothesis and \prefp{cor:tspace_base}}
      \\&\supseteq \ball{x}{r}
        && \text{because $x$ must be in one of those balls in $\setU$}
        %&& \text{by \prefp{thm:openball_openset}}
    \end{align*}

  \item Proof that ($(B)\impliedby(1)$) case:
    \begin{align*}
      \setU 
        &= \Setu\set{x\in\setX}{x\in\setU}
        && \text{by definition of union operation $\Setu$}
      \\&= \Setu\set{\ball{x}{r}}{x\in\setU\text{ and }\ball{x}{r}\subseteq\setU}
        && \text{by hypothesis (1)}
      \\&\implies \text{$\setU$ is \prope{open}}
        && \text{by \prefp{cor:tspace_base} and \prefp{cor:dspace_open}}
    \end{align*}
\end{enumerate}
\end{proof}

%---------------------------------------
\begin{corollary}
\footnote{
  in \structe{metric space} ($\opair{p}{\sigma}=\opair{1}{1}$):
  \citerppg{rosenlicht}{40}{41}{0486650383},
  \citerpg{ab}{35}{0120502577}
  }
\label{cor:oball_open}
%\label{prop:cball_closed}
%---------------------------------------
Let $\pdspaceX$ be a \structe{power distance space}.
Let $\balln$ be an \structe{open ball} \xrefh{def:ball} on $\dspaceX$.
Then for all $\opair{p}{\sigma}\in(\Rx\setd\setn{0})\times\Rp$,
\corbox{
  \brb{2\sigma \le 2^{\frac{1}{p}}}
  \quad\implies\quad
  \brb{\begin{array}{M}
    every \structe{open ball} $\ball{x}{r}$  in $\dspaceX$ is \prope{open}
  \end{array}}
  }
\end{corollary}
\begin{proof}
    \begin{align*}
        &\text{The union of any set of open balls is open}    && \text{by \prefp{cor:tspace_base}}
      \\&\qquad\text{$\implies$ the union of a set of just one open ball is open}
      \\&\qquad\text{$\implies$ every open ball is open.}
    \end{align*}
\end{proof}


%---------------------------------------
\begin{theorem}
\footnote{
  in \structe{metric space}:
  \citerpg{rosenlicht}{45}{0486650383},
  \citerpgc{giles1987}{37}{0521359287}{3.2 Definition}
  %\citerpgc{khamsi2001}{13}{0471418250}{Definition 2.1}
  %\citerpgc{thomson2008}{30}{143484367X}{Definition 2.1}
  %\citor{cauchy1821}
%  ``$\to$" symbol: \citorpc{leathem1905}{13}{section III.11}  % referenced by bromwich1955 page 3
  }
\label{thm:ms_converge}
\index{convergence!metric space}
%---------------------------------------
Let $\pdspaceX$ be a \structe{power distance space}.
Let $\topspaceX$ be a \structe{topological space induced by $\dspaceX$}. % \xref{def:dspacetop}.
Let $\seqxZ{x_n\in\setX}$ be a  sequence in $\dspaceX$.
\thmbox{
  \mcom{\text{$\seqn{x_n}$ converges to a limit $x$}}{\xref{def:converge}}
  \quad\iff\quad
  \brb{\begin{array}{M}
    for any $\varepsilon\in\Rp$, there exists $\xN\in\Z$\\ 
    such that for all $n>\xN$,
    \quad
    $\distance{x_n}{x}<\varepsilon$
  \end{array}}
  }
\end{theorem}
\begin{proof}
%\begin{enumerate}
  %\item Proof that $\seqn{x_n}\to x$ $\implies$   $\distance{x_n}{x}<\varepsilon$:
    \begin{align*}
      \seqn{x_n}\to x
        &\iff x_n\in\setU \quad\forall\setU\in\setN_x,\,n>\xN
        && \text{by \prefp{def:converge}}
      \\&\iff \exists \ball{x}{\varepsilon} \st x_n\in\ball{x}{\varepsilon} \forall n>\xN
        %&& \text{by \prefpp{def:dspace_open}}
        && \text{by \prefp{lem:tri_open}}
      \\&\iff \distance{x_n}{x}<\varepsilon
        && \text{by \prefp{def:ball}}
    \end{align*}

%  \item Proof that $\seqn{x_n}\to x$ $\impliedby$ $\distance{x_n}{x}<\varepsilon$:
%    \begin{align*}
%      \distance{x_n}{x}<\varepsilon
%      \\&\implies \seqn{x_n}\to x
%    \end{align*}
%\end{enumerate}
\end{proof}


In \structe{distance space}s \xref{def:dspace}, not all \prope{convergent} sequences are \prope{Cauchy} \xref{ex:dspace_1n}.
However in a distance space with any \rele{power triangle inequality} \xref{def:trirel}, all \prope{convergent} sequences are
\prope{Cauchy} (next theorem).
%A sequence that is \prope{convergent} is always \prope{Cauchy} (next theorem).
%However, in a metric space, the converse is not true---a sequence that is \prope{convergent} is not in general \prope{Cauchy}.
%This is in contrast to the special case of a real sequence in the metric space $\dspace{\R}{\abs{x-y}}$. %with the usual metric ($\distance{x}{y}\eqd\abs{x-y}$).
%In this case, all Cauchy sequences are convergent and the Cauchy property is referred to as the 
%\hie{Cauchy condition}.\footnote{
%  \citerppc{whittaker1915}{13}{15}{2.22}
%  }
%---------------------------------------
\begin{theorem}
\footnote{
  in \structe{metric space}:
  \citerpgc{giles1987}{49}{0521359287}{Theorem 3.30},
  \citerpg{rosenlicht}{51}{0486650383},
  \citerppgc{apostol1975}{72}{73}{0201002884}{Theorem 4.6}
  }
\label{thm:convergent==>cauchy}
%---------------------------------------
Let $\pdspaceX$ be a \structe{power distance space}.
%Let $\dspaceX$ be a \structe{distance space} \xrefh{def:dspace} with \rele{power triangle inequality} \xrefh{def:trirel} $\trirelD$
%defined in terms a \fncte{power triangle function} \xrefh{def:ptf} $\ptfD$ on $\dspaceX$.
Let $\balln$ be an \structe{open ball} \xrefh{def:ball} on $\dspaceX$.
\thmboxp{
  For any $\opair{p}{\sigma}\in\Rx\times\Rp$,
  \\\indentx$
  \brb{\begin{array}{M}
    $\seqn{x_n}$ is \prope{convergent}\\ 
    in $\dspaceX$
  \end{array}}
  \implies
  \brb{\begin{array}{M}
    $\seqn{x_n}$ is \prope{Cauchy}\\ 
    in $\dspaceX$
  \end{array}}
  \implies
  \brb{\begin{array}{M}
    $\seqn{x_n}$ is \prope{bounded}\\ 
    in $\dspaceX$
  \end{array}}
$}
\end{theorem}
\begin{proof}
\begin{enumerate}
  \item Proof that \prope{convergent} $\implies$ \prope{Cauchy}:
    \begin{align*}
      \distance{x_n}{x_m}
        &\le \ptf(p,\sigma;x_n,x_m,x)
        &&   \text{by definition of \rele{power triangle inequality} \xref{def:trirel}}
      \\&\eqd 2\sigma\brs{\frac{1}{2}\distancep{p}{x_n}{x} + \frac{1}{2}\distancep{p}{x_m}{x}}^\frac{1}{p}
        && \text{by definition of \fncte{power triangle function} \xref{def:ptf}}
      \\&<    2\sigma\brs{\frac{1}{2}\varepsilon^p + \frac{1}{2}\varepsilon^p}^\frac{1}{p}
        && \text{by \prope{convergence} hypothesis \xref{def:converge}}
      \\&=    2\sigma\varepsilon
        && \text{by definition of \prope{convergence} \xref{def:converge}}
      \\&\implies \text{\prope{Cauchy}}
        && \text{by definition of \prope{Cauchy} \xref{def:cauchy}}
      \\
      \distance{x_n}{x_m}
        &\le \ptf(\infty,\sigma;x_n,x_m,x)
        &&   \text{by definition of \rele{power triangle inequality} at $p=\infty$}
      \\&=  2\sigma\max\setn{\distancen(x_n,x),\,\distancen(x_m,x)}
        && \text{by \prefpp{cor:ftri_means}}
      \\&=  2\sigma\max\setn{\varepsilon,\,\varepsilon}
        && \text{by \prope{convergent} hypothesis \xref{def:converge}}
      \\&=  2\sigma\varepsilon
        && \text{by definition of $\max$}
      \\
      \distance{x_n}{x_m}
        &\le \ptf(-\infty,\sigma;x_n,x_m,x)
        &&   \text{by definition of \rele{power triangle inequality} at $p=-\infty$}
      \\&=  2\sigma\min\setn{\distancen(x_n,x),\,\distancen(x_m,x)}
        && \text{by \prefpp{cor:ftri_means}}
      \\&=  2\sigma\min\setn{\varepsilon,\,\varepsilon}
        && \text{by \prope{convergent} hypothesis \xref{def:converge}}
      \\&=  2\sigma\varepsilon
        && \text{by definition of $\min$}
      \\
      \distance{x_n}{x_m}
        &\le \ptf(0,\sigma;x_n,x_m,x)
        &&   \text{by definition of \rele{power triangle inequality} at $p=0$}
      \\&=  2\sigma\sqrt{\distancen(x_n,x)}\,\sqrt{\distancen(x_m,x)}
        && \text{by \prefpp{cor:ftri_means}}
      \\&=  2\sigma\sqrt{\varepsilon}\,\sqrt{\varepsilon}
        && \text{by \prope{convergent} hypothesis \xref{def:converge}}
      \\&=  2\sigma\varepsilon
        && \text{by property of $\R$}
    \end{align*}

  \item Proof that \prope{Cauchy} $\implies$ \prope{bounded}: by \prefpp{prop:cauchy==>bounded}.

\end{enumerate}
\end{proof}

%---------------------------------------
\begin{theorem}
\footnote{
  in \structe{metric space}:
  \citerpg{rosenlicht}{52}{0486650383}
  }
%---------------------------------------
Let $\pdspaceX$ be a \structe{power distance space}.
Let $\ff\in\clFzz$ be a \prope{strictly monotone} function such that $\ff(n)<\ff(n+1)$.
\thmboxt{
For any $\opair{p}{\sigma}\in\Rx\times\Rp$
\\\indentx$
  \brb{\begin{array}{FMD}
    1. & $\seq{x_n}{n\in\Z}$ is \prope{Cauchy}     & and \\
    2. & $\seq{x_{\ff(n)}}{n\in\Z}$ is \prope{convergent}
  \end{array}}
  \qquad\implies\qquad
  \brb{\seq{x_n}{n\in\Z} \text{ is \prope{convergent}.}}
  $}
\end{theorem}
\begin{proof}
\begin{align*}
  \distance{x_n}{x}
    &= \distance{x}{x_n}
    && \text{by \prope{symmetric} property of $\distancen$}
  \\&\le \ptf(p,\sigma;x,x_n,x_{\ff(n)})
    &&   \text{by definition of \rele{power triangle inequality} \xref{def:trirel}}
  \\&\eqd 2\sigma\brs{\frac{1}{2}\distancep{p}{x}{x_{\ff(n)}} + \frac{1}{2}\distancep{p}{x_{\ff(n)}}{x_n}}^\frac{1}{p}
    && \text{by definition of \fncte{power triangle function} \xref{def:ptf}}
  \\&=  2\sigma\brs{\frac{1}{2}\varepsilon  + \frac{1}{2}\distancep{p}{x_{\ff(n)}}{x_n}}^\frac{1}{p}
    && \text{by left hypothesis 2}
  \\&=  2\sigma\brs{\frac{1}{2}\varepsilon^p  + \frac{1}{2}\varepsilon^p}^\frac{1}{p}
    && \text{by left hypothesis 1}
  \\&= 2\sigma\varepsilon
  \\&\implies \text{\prope{convergent}}
    && \text{by definition of \prope{convergent} \xref{def:converge}}
\end{align*}
\end{proof}

%---------------------------------------
\begin{theorem}
\footnote{
  in \structe{metric space} ($\opair{p}{\sigma}=\opair{1}{1}$ case):
  \citerpgc{berberian1961}{37}{0821819127}{Theorem~II.4.1}
  %\citerp{pedersen2000}{4}
  }
\label{thm:pdspace_continuous}
%---------------------------------------
Let $\pdspaceX$ be a \structe{power distance space}.
Let $\opair{\R}{\distancean}$ be a metric space of real numbers with the usual metric 
$\distancea{x}{y}\eqd\abs{x-y}$.
%Let 
%\\\indentx$\ff(p,\sigma,x,y,z)\eqd \brp{\max\brb{
%      \frac{2}{(2\sigma)^p}\distancep{p}{x}{z}-\distancep{p}{z}{y},\,
%      \frac{2}{(2\sigma)^p}\distancep{p}{y}{z}-\distancep{p}{z}{x}
%      }}^\frac{1}{p}$.\\
%\\\indentx$\ff(p,\sigma,x,y,z)\eqd \brp{\min\brb{0,\,
%    \max\brb{
%      \frac{2}{(2\sigma)^p}\distancep{p}{x}{z}-\distancep{p}{z}{y},\,
%      \frac{2}{(2\sigma)^p}\distancep{p}{y}{z}-\distancep{p}{z}{x}
%      }}}^{\frac{1}{p}}$.\\
Then
\thmbox{
%For all $p\in\Rx$, $\sigma\in\Rp$\\
%  \brb{\begin{array}{lC}
%    \ff(p,\sigma,x,y,z) \ge \abs{\distance{x}{z}-\distance{z}{y}} & \forall x,y,z\in\setX
%  \end{array}}
  \brb{2\sigma=2^\frac{1}{p}}
  \quad\implies\quad
  \brb{\begin{array}{M}
    $\distancen$ is \prope{continuous} in $\opair{\R}{\distancean}$
  \end{array}}
  }
\end{theorem}
\begin{proof}
\begin{align*}
  \abs{\distance{x}{y}-\distance{x_n}{y_n}}
    &\le \abs{\distance{x}{y}-\distance{x_n}{y}} + \abs{\distance{x_n}{y}-\distance{x_n}{y_n}}
    && \text{by \prope{triangle inequality} of $\metspace{\R}{\abs{x-y}}$}
  \\&=   \abs{\distance{x}{y}-\distance{y}{x_n}} + \abs{\distance{y}{x_n}-\distance{x_n}{y_n}}
    && \text{by \prope{commutative} property of $\distancen$ \xref{def:dspace}}
  \\&\le \distance{x}{x_n} + \distance{y}{y_n}
    && \text{by $2\sigma=2^\frac{1}{p}$ and \prefpp{lem:pdspace_ineq}}
  \\&=   0\qquad\text{as $n\to\infty$}
\end{align*}
\end{proof}

%%---------------------------------------
%\begin{theorem}
%\footnote{
%  in \structe{metric space} ($p=\sigma=1$ case):
%  \citerpgc{berberian1961}{37}{0821819127}{Theorem~II.4.1}
%  %\citerp{pedersen2000}{4}
%  }
%\label{thm:pdspace_continuous}
%%---------------------------------------
%Let $\pdspaceX$ be a \structe{power distance space}.
%Let $\opair{\R}{\distancean}$ be a metric space of real numbers with the usual metric 
%$\distancea{x}{y}\eqd\abs{x-y}$.
%\thmbox{
%%For all $p\in\Rx$, $\sigma\in\Rp$\\
%  \brb{\begin{array}{FlD}
%    1. & \opair{\seqn{x_n}}{\seqn{y_n}} \to \opair{x}{y}         & and \\
%    2. & \opair{p}{\sigma}\in(\R\setd\setn{-\infty,0})\times\Rp  & and \\
%    3. & 2\sigma = 2^{\frac{1}{p}}                             &
%  \end{array}}
%  \implies
%  \brb{\begin{array}{FMD}
%    1. & $\seqn{\distance{x_n}{y_n}} \to \distance{x}{y}$ & and \\
%    2. & $\distancen$ is \prope{continuous} in $\opair{\R}{\distancean}$ & and\\
%    3. & $p=\infty \quad\implies\quad \sigma=\frac{1}{2}$
%  \end{array}}
%  \quad%
%  {\begin{array}{C}
%    \forall x,y,x_n,y_n\in\setX 
%    %\forall p\in\Rx & and\\
%    %\forall \sigma\in\Rp
%  \end{array}}
%  }
%\end{theorem}
%\begin{proof}
%\begin{enumerate}
%  \item lemma: \label{ilem:metric_continuous}
%\begin{align*}
%  &\distancea{\distance{x}{y}}{\distance{x_n}{y_n}}
%  \\&\eqd \abs{\distance{x}{y}-\distance{x_n}{y_n}}
%    && \text{by definition of $\distancean$}
%  \\&\le \abs{                         % p(x-y) = |x-y| (usual metric)
%           \ptf(p,\sigma;x,y,x_n)   % d(x,y) <= \ptf(p,\sigma;x,y,x_n)
%          -\distance{x_n}{y_n}         %
%           }
%    &&   \text{by \prope{power triangle inequality} \xref{def:trirel}}
%  \\&=   \abs{
%           2\sigma\brs{\frac{1}{2}\distancen^p(x,x_n)+\frac{1}{2}\distancen^p(x_n,y)}^\frac{1}{p}  % \ptf(p,\sigma;x,y,x_n)
%          -\distance{x_n}{y_n}
%           }
%    &&   \text{by definition of $\ptf$ \xref{def:ptf}}
%  \\&\le \abs{
%           2\sigma\brs{\frac{1}{2}\distancen^p(x,x_n)+\frac{1}{2}
%           \ptf^p(p,\sigma;x_n,y,y_n)                           % \distancen^p(x_n,y) <=\ptf(p,\sigma;x_n,y,y_n)
%           }^\frac{1}{p}  % \ptf(p,\sigma;x,y,x_n)
%          -\distance{x_n}{y_n}
%           }
%    &&   \text{by \prope{power triangle inequality} \xref{def:trirel}}
%\end{align*}
%
%  \item Proof for the $p\in\Rx\setd\setn{-\infty,0,\infty}$ case:
%    \begin{align*}
%      &\distancea{\distance{x}{y}}{\distance{x_n}{y_n}}
%    %\\&\eqd \abs{\distance{x}{y}-\distance{x_n}{y_n}}
%    \\&\le \lim_{\varepsilon\to0}\abs{
%             2\sigma\brs{\frac{1}{2}\distancen^p(x,x_n)
%            +\frac{1}{2}\ptf^p(p,\sigma;x_n,y,y_n)                           % \distancen^p(x_n,y) <=\ptf(p,\sigma;x_n,y,y_n)
%             }^\frac{1}{p}  % \ptf(p,\sigma;x,y,x_n)
%            -\distance{x_n}{y_n}
%             }
%      &&   \text{by \prefp{ilem:metric_continuous}}
%    \\&\eqd \mathrlap{\lim_{\varepsilon\to0}\abs{
%             2\sigma\brs{\frac{1}{2}\distancen^p(x,x_n)+\frac{1}{2}
%             \brp{2\sigma\brs{\frac{1}{2}\distancen^p(x_n,y_n)+\frac{1}{2}\distancen^p(y_n,y)}^\frac{1}{p}}^p  % \ptf(p,\sigma;x_n,y,y_n)
%             }^\frac{1}{p}  % \ptf(p,\sigma;x,y,x_n)
%            -\distance{x_n}{y_n}
%             }
%      \quad\text{by def. of $\ptf$ \xref{def:ptf}}}
%    \\&=  \lim_{\varepsilon\to0}\abs{
%             2\sigma\brs{\frac{1}{2}\varepsilon+\frac{1}{2}
%             \brp{2\sigma\brs{\frac{1}{2}\distancen^p(x_n,y_n)+\frac{1}{2}\varepsilon}^\frac{1}{p}}^p  % \ptf(p,\sigma;x_n,y,y_n)
%             }^\frac{1}{p}  % \ptf(p,\sigma;x,y,x_n)
%            -\distance{x_n}{y_n}
%             }
%      &&   \text{by $\opair{\seqn{x_n}}{\seqn{y_n}}\to\opair{x}{y}$ hypothesis}
%    \\&=  \abs{
%             2\sigma\brs{\frac{1}{2}
%             \brp{2\sigma\brs{\frac{1}{2}\distancen^p(x_n,y_n)}^\frac{1}{p}}^p  % \ptf(p,\sigma;x_n,y,y_n)
%             }^\frac{1}{p}  % \ptf(p,\sigma;x,y,x_n)
%            -\distance{x_n}{y_n}
%             }
%    \\&=  \abs{
%             2\sigma\brs{\brp{\frac{1}{2}}^\frac{1}{p}
%             \brp{2\sigma\brs{\frac{1}{2}\distancen^p(x_n,y_n)}^\frac{1}{p}} 
%             }
%            -\distance{x_n}{y_n}
%             }
%    \\&=  \abs{
%             2\sigma\brs{\brp{\frac{1}{2}}^\frac{1}{p}
%             \brp{2\sigma\brs{\brp{\frac{1}{2}}^\frac{1}{p}\distancen(x_n,y_n)}} 
%             }
%            -\distance{x_n}{y_n}
%             }
%    \\&=  \abs{4\sigma^2{\brp{\frac{1}{2}}^\frac{2}{p}{{\distancen(x_n,y_n)}} }-\distance{x_n}{y_n}  }
%    \\&=  \abs{4\sigma^2\brp{\frac{1}{2}}^\frac{2}{p}-1}\distance{x_n}{y_n}
%    \\&\implies \text{convergence when $4\sigma^2 = 2^\frac{2}{p}$}
%     &&\text{(see \prefp{fig:sigmap})}
%    \\&\implies \text{convergence when $p=\frac{\ln2}{\ln(2\sigma)}$} 
%     &&\text{(see \prefp{fig:sigmap})}
%    \end{align*}
%
%  \item Proof for $\opair{p}{\sigma}=\opair{\infty}{\frac{1}{2}}$ case:
%    \begin{align*}
%      &\distancea{\distance{x}{y}}{\distance{x_n}{y_n}}
%  \\&\le \abs{
%           2\sigma\brs{\frac{1}{2}\distancen^p(x,x_n)
%          +\frac{1}{2}\ptf^p(p,\sigma;x_n,y,y_n)                           % \distancen^p(x_n,y) <=\ptf(p,\sigma;x_n,y,y_n)
%           }^\frac{1}{p}  % \ptf(p,\sigma;x,y,x_n)
%          -\distance{x_n}{y_n}
%           }
%    &&   \text{by \prefp{ilem:metric_continuous}}
%  \\&= \mathrlap{\abs{
%           2\sigma\brp{\frac{1}{2}\distancen^p(x,x_n)
%          +\frac{1}{2}\brs{2\sigma\max\setn{\distancen(x_n,y_n),\,\distancen(y_n,y)}}^p
%           }^\frac{1}{p}  % \ptf(p,\sigma;x,y,x_n)
%          -\distance{x_n}{y_n}
%           }
%    \quad\text{by \prefp{cor:ftri_means}}}
%  \\&= \abs{
%           2\sigma\brs{\frac{1}{2}\distancen^p(x,x_n)
%          +\frac{1}{2}\brs{2\sigma\distancen(x_n,y_n)}^p
%           }^\frac{1}{p}  % \ptf(p,\sigma;x,y,x_n)
%          -\distance{x_n}{y_n}
%           }
%    &&   \text{by $\opair{\seqn{x_n}}{\seqn{y_n}}\to\opair{x}{y}$ hypothesis}
%  \\&= \abs{2\sigma\max\setn{\distancen(x,x_n),\,2\sigma\distancen(x_n,y_n)}
%          -\distance{x_n}{y_n}
%           }
%    && \text{by \prefp{thm:seq_Mr}}
%  \\&= \abs{4\sigma^2\distancen(x_n,y_n)-\distance{x_n}{y_n}}
%    && \text{by $\opair{\seqn{x_n}}{\seqn{y_n}}\to\opair{x}{y}$ hypothesis}
%  \\&= \abs{4\sigma^2-1}\distancen(x_n,y_n)
%    && \text{by \prope{non-negative} property of $\distancen$ \xref{def:distance}}
%  \\&= 0\quad\text{for}\quad\sigma=\frac{1}{2}
%    \end{align*}
%
%  \item Failed proof for $p=-\infty$ case:
%    \begin{align*}
%      &\distancea{\distance{x}{y}}{\distance{x_n}{y_n}}
%  \\&\le \abs{
%           2\sigma\brs{\frac{1}{2}\distancen^p(x,x_n)
%          +\frac{1}{2}\ptf^p(p,\sigma;x_n,y,y_n)                           % \distancen^p(x_n,y) <=\ptf(p,\sigma;x_n,y,y_n)
%           }^\frac{1}{p}  % \ptf(p,\sigma;x,y,x_n)
%          -\distance{x_n}{y_n}
%           }
%    &&   \text{by \prefp{ilem:metric_continuous}}
%  \\&= \mathrlap{\abs{
%           2\sigma\brp{\frac{1}{2}\distancen^p(x,x_n)
%          +\frac{1}{2}\brs{2\sigma\min\setn{\distancen(x_n,y_n),\,\distancen(y_n,y)}}^p
%           }^\frac{1}{p}  % \ptf(p,\sigma;x,y,x_n)
%          -\distance{x_n}{y_n}
%           }
%    \quad\text{by \prefp{cor:ftri_means}}}
%  \\&= \abs{
%           2\sigma\brs{\frac{1}{2}\distancen^p(x,x_n)
%          +\frac{1}{2}\brs{2\sigma\distancen(y_n,y)}^p
%           }^\frac{1}{p}
%          -\distance{x_n}{y_n}
%           }
%    && \text{by $\opair{\seqn{x_n}}{\seqn{y_n}}\to\opair{x}{y}$ hypothesis}
%  \\&= \abs{2\sigma\min\setn{\distancen(x,x_n),\,2\sigma\distancen(y_n,y)} -\distance{x_n}{y_n} }
%    && \text{by \prefp{thm:seq_Mr}}
%  \\&= \abs{\distancen(x_n,y_n)}
%    && \text{by $\opair{\seqn{x_n}}{\seqn{y_n}}\to\opair{x}{y}$ hypothesis}
%  \\&= \distancen(x_n,y_n)
%    && \text{by \prope{non-negative} property of $\distancen$ \xref{def:distance}}
%  \\&\neq 0
%    && \text{$\neq0$ for any value of $\sigma$}
%    \end{align*}
%
%  \item Failed proof for the $p=0$ case:
%    \begin{enumerate}
%      \item demonstration of failed limit:
%    \begin{align*}
%      &\distancea{\distance{x}{y}}{\distance{x_n}{y_n}}
%  \\&\le \abs{
%           2\sigma\brs{\frac{1}{2}\distancen^p(x,x_n)
%          +\frac{1}{2}\ptf^p(p,\sigma;x_n,y,y_n)                           % \distancen^p(x_n,y) <=\ptf(p,\sigma;x_n,y,y_n)
%           }^\frac{1}{p}  % \ptf(p,\sigma;x,y,x_n)
%          -\distance{x_n}{y_n}
%           }
%    &&   \text{by \prefp{ilem:metric_continuous}}
%  \\&= \mathrlap{\abs{
%           2\sigma\brp{\frac{1}{2}\distancen^p(x,x_n)
%          +\frac{1}{2}\brs{2\sigma\sqrt{\distancen(x_n,y_n)}\sqrt{\distancen(y_n,y)}}^p
%           }^\frac{1}{p}  % \ptf(p,\sigma;x,y,x_n)
%          -\distance{x_n}{y_n}
%           }
%    \quad\text{by \prefp{cor:ftri_means}}}
%  \\&= \abs{
%           2\sigma\sqrt{\distancen(x,x_n)}
%          \sqrt{2\sigma\sqrt{\distancen(x_n,y_n)}\sqrt{\distancen(y_n,y)}}
%          -\distance{x_n}{y_n}
%           }
%    &&   \text{by \prefp{cor:ftri_means}}
%  \\&= \abs{
%           2\sigma\sqrt{\varepsilon}
%          \sqrt{2\sigma\sqrt{\distancen(x_n,y_n)}\sqrt{\varepsilon}}
%          -\distance{x_n}{y_n}
%           }
%    &&   \text{by \prefp{cor:ftri_means}}
%  \\&= \mathrlap{\distancen(x_n,y_n)
%    \qquad\text{by \prope{non-negative} property of $\distancen$ \xref{def:distance}}}
%  \\&\neq 0
%    && \text{$\neq0$ for any value of $\sigma$}
%    \end{align*}
%
%      \item For whatever (if anything) it's worth, note that the constraint $4\sigma^2=2^{\frac{2}{p}}$ is \prope{discontinuous} at $p=0$:
%        \\\indentx$\ds
%          \lim_{p\to0^+}\sigma
%            = \lim_{p\to0^+}\sqrt{\frac{1}{2}2^{\frac{2}{p}}} 
%            = \infty
%            \neq 0
%            = \lim_{p\to0^-}\sqrt{\frac{1}{2}2^{\frac{2}{p}}} 
%            = \lim_{p\to0^-}\sigma
%          $ 
%    \end{enumerate}
%
%  \item Proof that $\distancen$ is \prope{continuous}: this follows directly from \prefpp{thm:limcont}.
%\end{enumerate}
%\end{proof}

In \structe{distance space}s and \structe{topological space}s, limits of convergent sequences are in general \prope{not unique} 
\xxref{ex:dspace_01}{ex:seq_xt31}.
However \pref{thm:xn_to_xy} (next) demonstrates that, in a \structe{power distance space},
limits \emph{are} unique.
%---------------------------------------
\begin{theorem}[\thmd{Uniqueness of limit}]
\footnote{
  in \structe{metric space}:
  \citerpg{rosenlicht}{46}{0486650383},
  \citerpgc{thomson2008}{32}{143484367X}{Theorem 2.8}
  }
\label{thm:xn_to_xy}
%---------------------------------------
Let $\pdspaceX$ be a \structe{power distance space}.
Let $x,y,\in \setX$ and let $\seqn{x_n\in\setX}$ be an $\setX$-valued sequence.
%Then for all $\opair{p}{\sigma}\in(\Rx\setd\setn{0})\times\Rp$,
\thmbox{
  \brb{\begin{array}{FlD}
    1. & \brb{\opair{\seqn{x_n}}{\seqn{y_n}}\to\opair{x}{y}} & and\\
    2. & \opair{p}{\sigma}\in\Rx\times\Rp                    &
  \end{array}}
  \quad\implies\quad
  \brb{x=y}%{the \structe{limit} of a \prope{convergent} sequence is \prope{unique}}
  }
\end{theorem}
\begin{proof}
  \begin{enumerate}
    \item lemma: Proof that for all $\opair{p}{\sigma}\in\Rx\times\Rp$ and for any $\varepsilon\in\Rp$, 
          there exists $\xN$ such that $\distance{x}{y} <2\sigma\varepsilon$\label{ilem:xn_to_xy_2e}:
      \begin{align*}
        \distance{x}{y} 
          &\le  \ptf(p,\sigma;x,y,x_n)
          &&    \text{by definition of \ineqe{power triangle inequality} \xref{def:trirel}}
        \\&\eqd 2\sigma\brs{\frac{1}{2}\distancep{p}{x}{x_n} + \frac{1}{2}\distancep{p}{x_n}{y}}^\frac{1}{p}
          &&    \text{by definition of \fncte{power triangle function} \xref{def:ptf}}
        \\&<    2\sigma\brs{\frac{1}{2}\varepsilon^p + \frac{1}{2}\varepsilon^p}^\frac{1}{p}
          &&    \text{by left hypothesis and for $p\in\Rx\setd\setn{-\infty,0,\infty}$}
        \\&=    2\sigma\varepsilon
        \\
        \distance{x}{y} 
          &\le  \ptf(\infty,\sigma;x,y,x_n)
          &&    \text{by definition of \ineqe{power triangle inequality} at $p=\infty$}
        \\&=    2\sigma\max\setn{\distancen(x,x_n),\,\distancen(x_n,y)}
          &&    \text{by \prefpp{cor:ftri_means}}
        \\&<    2\sigma\varepsilon
          &&    \text{by left hypothesis}
        \\
        \distance{x}{y} 
          &\le  \ptf(-\infty,\sigma;x,y,x_n)
          &&    \text{by definition of \ineqe{power triangle inequality} at $p=-\infty$}
        \\&=    2\sigma\min\setn{\distancen(x,x_n),\,\distancen(x_n,y)}
          &&    \text{by \prefpp{cor:ftri_means}}
        \\&<    2\sigma\varepsilon
          &&    \text{by left hypothesis}
        \\
        \distance{x}{y} 
          &\le  \ptf(0,\sigma;x,y,x_n)
          &&    \text{by definition of \ineqe{power triangle inequality} at $p=0$}
        \\&=    2\sigma\sqrt{\distancen(x,x_n)}\,\sqrt{\distancen(x_n,y)}
          &&    \text{by \prefpp{cor:ftri_means}}
        \\&=    2\sigma\sqrt{\varepsilon}\,\sqrt{\varepsilon}
          &&    \text{by left hypothesis}
        \\&<    2\sigma\varepsilon
          &&    \text{by property of real numbers}
      \end{align*}
      %\begin{align*}
      %  &\seqn{x_n} \to x \text{ and }  \seqn{x_n} \to y
      %  \\&\implies \exists\xN \st \forall n>\xN,\;  \distance{x}{x_n}<\varepsilon \text{ and }  \distance{x_n}{y}<\varepsilon
      %  \\&\implies \exists\xN \st \forall n>\xN,\;  \distance{x}{y} \le \ptf(p,\sigma;x,y,x_n)
      %    && \text{by def. of p.t.ineq. \xref{def:trirel}}
      %  \\&\implies \exists\xN \st \forall n>\xN,\;  \distance{x}{y} \le 2\sigma\brs{\frac{1}{2}\distancen^p(x,x_n) + \frac{1}{2}\distancen^p(x_n,y)}^\frac{1}{p}
      %    && \text{by def. of p.t.ineq. \xref{def:trirel}}
      %  \\&\implies \exists\xN \st \forall n>\xN,\;  \distance{x}{y} \le 2\sigma\brs{\frac{1}{2}\varepsilon^p + \frac{1}{2}\varepsilon^p}^\frac{1}{p}
      %    && \text{by left hypothesis}
      %  \\&\implies \exists\xN \st \forall n>\xN,\;  \distance{x}{y} \le 2\sigma\varepsilon
      %\end{align*}
  
    \item Proof that $x=y$ (proof by contradiction): %$\distance{x}{y}=0$:
      \begin{align*}
        x\neq y
          &\implies \distance{x}{y}\neq0
          && \text{by the \prope{nondegenerate} property of $\distancen$ \xref{def:distance}}
        \\&\implies \distance{x}{y}>0
          && \text{by \prope{non-negative} property of $\distancen$ \xref{def:distance}}
        \\&\implies \exists \varepsilon \st \distance{x}{y}>2\sigma\varepsilon
        \\&\implies \text{\vale{contradiction} to \prefp{ilem:xn_to_xy_2e}}
        \\&\implies \distance{x}{y}=0
        \\&\implies x=y
      \end{align*}
      %\begin{enumerate}
      %  \item If $\distance{x}{y}>0$, then we could choose an arbitrarily small $\varepsilon$ such that
      %    \quad$\distance{x}{y} >2\varepsilon$
      %  \item But this would contradict the earlier result of $\distance{x}{y} <2\varepsilon$.
      %  \item Therefore, $\distance{x}{y}=0$ (proof by contradiction).
      %\end{enumerate}
  
    %\item Proof that $x=y$:
    %  \begin{align*}
    %    \distance{x}{y}=0
    %      &\implies x=y
    %      && \text{by the \prope{nondegenerate} property of $\distancen$ \xref{def:distance}}
    %  \end{align*}
  \end{enumerate}
\end{proof}


%%=======================================
%\subsection{Properties of selected distance spaces}
%%=======================================
%A \structe{distance space} does not necessarily have all the nice properties (next theorem)
%that a \structe{metric space} has \xref{thm:tri_metric}.
%%---------------------------------------
%\begin{theorem}
%%---------------------------------------
%Let $\seqxZ{x_n}$ be a \structe{sequence} in a \structe{distance space} \xref{def:dspace} $\dspaceX$.
%\\\thmboxp{$\begin{array}{FMcMM}
%  1. & $\distancen$   is a \fncte{distance} in $\dspaceX$ &\notimplies& $\distancen$ is \prope{continuous} in $\dspaceX$  & \xref{ex:dspace_21}.\\
%  2. & $\balln$ is an \structe{open ball} in $\dspaceX$ &\notimplies& $\balln$ is \prope{open} in $\dspaceX$          & \xref{ex:dspace_1n}.\\
%  3. & $\baseB$ is the set of all \structe{open ball}s in $\dspaceX$ &\notimplies& $\baseB$ is a \structe{base} for a topology on $\setX$ & \xref{ex:dspace_1n}.\footnotemark\\
%  4. & $\seqn{x_n}$ is \prope{convergent} in $\dspaceX$ &\notimplies& limit is \prope{unique}                           & \xref{ex:dspace_01}.\\
%  5. & $\seqn{x_n}$ is \prope{convergent} in $\dspaceX$ &\notimplies& $\seqn{x_n}$ is \prope{Cauchy} in $\dspaceX$    & \xref{ex:dspace_1n}.\\
%\end{array}$}
%%In a \structe{metric space} \xref{def:metric}, if a \fncte{sequence} has a limit, that limit is unique.
%%This is not true in general for \structe{distance space}s \xref{def:distance}.
%%See \prefpp{ex:dspace_01} for a distance space which converges to two distinct limits.
%\footnotetext{
%  \citePpc{heath1961}{810}{{\scshape Theorem}},
%  \citePpc{galvin1984}{71}{{\scshape 2.3 Lemma}}
%  }
%\end{theorem}
%
%%---------------------------------------
%\begin{theorem}
%\label{thm:tri_metric}
%%---------------------------------------
%Let $\seqxZ{x_n}$ be a \structe{sequence} in a \structe{distance space} \xref{def:dspace} $\dspaceX$.
%\\If $\dspaceX$ is a \structd{metric space} \xref{def:metric} then
%\\\thmboxp{$\begin{array}{FMcMM}
%  1. & \mc{3}{M}{$\distancen$ is \prope{continuous} in $\dspaceX$      }                                            & \xref{thm:pdspace_continuous}.\\
%  2. & $\balln$ is an \structe{open ball} in $\dspaceX$ &\implies& $\balln$ is \prope{open} in $\dspaceX$& \xref{cor:oball_open}.\\
% %3. & \mc{3}{M}{$\baseB$ is a \structe{base} for a topology on $\setX$}                                            & \xref{ex:dspace_1n}.\footnotemark\\
%  3. & $\baseB$ is the set of all \structe{open ball}s in $\dspaceX$ &\implies& $\baseB$ is a \structe{base} for a topology on $\setX$ & \xref{cor:tspace_base}.\\
%  4. & $\seqn{x_n}$ is \prope{convergent} in $\dspaceX$ &\implies& limit is \prope{unique}                        & \xref{thm:xn_to_xy}.\\
%  5. & $\seqn{x_n}$ is \prope{convergent} in $\dspaceX$ &\implies& $\seqn{x_n}$ is \prope{Cauchy} in $\dspaceX$ & \xref{prop:convergent==>cauchy}.\\
%\end{array}$}
%\end{theorem}
%
%%---------------------------------------
%\begin{theorem}
%\label{thm:tri_nmetric}
%%---------------------------------------
%Let $\seqxZ{x_n}$ be a \structe{sequence} in a \structe{distance space} \xref{def:dspace} $\dspaceX$.
%\\If $\dspaceX$ is a \structd{near metric space} \xref{def:nmetric} then
%\\\thmboxp{$\begin{array}{FMcMM}
%  1. & \mc{3}{M}{$\distancen$ is \prope{continuous} in $\dspaceX$      }                                            & \xref{prop:nmetric_metric_continuous}.\\
%  2. & $\balln$ is an \structe{open ball} in $\dspaceX$ &\implies& $\balln$ is \prope{open} in $\dspaceX$& \xref{prop:nmetric_oball_open}.\\
%  3. & $\baseB$ is the set of all \structe{open ball}s in $\dspaceX$ &\implies& $\baseB$ is a \structe{base} for a topology on $\setX$ & \xref{prop:nmetric_(X,d)->(X,t)}.\\
%  4. & $\seqn{x_n}$ is \prope{convergent} in $\dspaceX$ &\implies& limit is \prope{unique}                        & \xref{prop:nmetric_xn_to_xy}.\\
%  5. & $\seqn{x_n}$ is \prope{convergent} in $\dspaceX$ &\implies& $\seqn{x_n}$ is \prope{Cauchy} in $\dspaceX$ & \xref{prop:nmetric_convergent==>cauchy}.\\
%\end{array}$}
%\end{theorem}
%

%=======================================
\section{Examples}
%=======================================
It is not always possible to find a \rele{triangle relation} \xref{def:trirel} $\trirelD$
that holds in every \structe{distance space} \xref{def:dspace}, as demonstrated by 
\pref{ex:pdspace_01} and \pref{ex:pdspace_1n} (next two examples).
%---------------------------------------
\begin{example}
\label{ex:pdspace_01}
%---------------------------------------
Let $\distance{x}{y}\in\clF{\R\times\R}{\R}$ be defined such that
\\\indentx$\distance{x}{y} \eqd \brb{\begin{array}{lMD}
      y         & $\forall \opair{x}{y}\in\setn{4}\times\intoc{0}{2}$ & (\structe{vertical half-open interval})\\
      x         & $\forall \opair{x}{y}\in\intoc{0}{2}\times\setn{4}$ & (\structe{horizontal half-open interval})\\
      \abs{x-y} & otherwise                                           & (\prope{Euclidean})
    \end{array}}$.
\\
Note the following about the pair $\dspaceRd$:
\begin{enumerate}
  \item By \prefpp{ex:dspace_01}, $\dspaceRd$ is a \structe{distance space}, but not a \structe{metric space}---that is,
        the \rele{triangle relation} $\trirel(1,1;\distancen)$ does not hold in $\dspaceRd$.
  \item Observe further that $\dspaceRd$ is \emph{not} a \structe{power distance space}.
        In particular, the \rele{triangle relation} $\trirelD$ does not hold in $\dspaceRd$
        for any finite value of $\sigma$ (does not hold for any $\sigma\in\Rp$):
       \begin{align*}
         \distance{0}{4} 
           = 4 
            \nle 0
            = \lim_{\varepsilon\to0}2\sigma\varepsilon
           &= \lim_{\varepsilon\to0}2\sigma\brs{\sfrac{1}{2}\abs{0-\varepsilon}^p + \sfrac{1}{2}{\varepsilon}^p}^\frac{1}{p}
         \\&\eqd \lim_{\varepsilon\to0}2\sigma\brs{\sfrac{1}{2}\distancep{p}{0}{\varepsilon} + \sfrac{1}{2}\distancep{p}{\varepsilon}{4}}^\frac{1}{p}
            \eqd \lim_{\varepsilon\to0}\trirel(p,\sigma;0,4,\varepsilon;\distancen)
       \end{align*}
\end{enumerate}
\end{example}

%---------------------------------------
\begin{example}
\label{ex:pdspace_1n}
%---------------------------------------
Let $\distance{x}{y}\in\clF{\R\times\R}{\R}$ be defined such that
\\\indentx$\distance{x}{y} \eqd \brb{\begin{array}{lMD}
      \abs{x-y} & for $x=0$ or $y=0$ or $x=y$ & (\prope{Euclidean})\\
      1         & otherwise                   & (\prope{discrete})
    \end{array}}$.\\
Note the following about the pair $\dspaceRd$:
\begin{enumerate}
  \item By \prefpp{ex:dspace_1n}, $\dspaceRd$ is a \structe{distance space}, but not a \structe{metric space}---that is,
        the \rele{triangle relation} $\trirel(1,1;\distancen)$ does not hold in $\dspaceRd$.
  \item Observe further that $\dspaceRd$ is \emph{not} a \structe{power distance space}---that is,
        the \rele{triangle relation} $\trirelD$ does not hold in $\dspaceRd$
        for any value of $\opair{p}{\sigma}\in\Rx\times\Rp$.
    \begin{enumerate}
      \item Proof that $\trirelD$ does not hold for any $\opair{p}{\sigma}\in\setn{\infty}\times\Rp$:\label{item:pdspace_1n_infty}
        \begin{align*}
          \lim_{n,m\to\infty} \distance{\sfrac{1}{n}}{\sfrac{1}{m}}
            &\eqd 1 \nleq 0 = 2\sigma\max\setn{0,0}
            && \text{by definition of $\distancen$}
         %\\&\nleq0
          \\&=2\sigma\lim_{n,m\to\infty} \max\setn{\distance{\sfrac{1}{n}}{0},\,\distance{0}{\sfrac{1}{m}}}
            && \text{by \prefpp{cor:ftri_means}}
          \\&\ge\lim_{n,m\to\infty}2\sigma \brs{\sfrac{1}{2}\distancep{p}{\sfrac{1}{n}}{0}+\sfrac{1}{2}\distancep{p}{0}{\sfrac{1}{m}}}^{\frac{1}{p}}
           %&&\text{by \prope{continuous} and \prope{strictly monotone} properties of $\ptf$ \xref{cor:tri_mono}}
            &&\text{by \prefpp{cor:tri_mono}}
          \\&\eqd\lim_{n,m\to\infty}\ptf(p,\sigma,\sfrac{1}{n},\sfrac{1}{m},0)
            &&\text{by definition of $\ptf$ \xref{def:ptf}}
        \end{align*}
      \item Proof that $\trirelD$ does not hold for any $\opair{p}{\sigma}\in\Rx\times\Rp$:
            By \prefpp{cor:tri_mono}, the \fncte{triangle function} \xref{def:ptf} $\ptfD$ 
            is \prope{continuous} and \prope{strictly monotone} in $\omsR$ with respect to the variable $p$.
            Item~\ref{item:pdspace_1n_infty} demonstrates that $\trirelD$ fails to hold at the best case of $p=\infty$, 
            and so by \pref{cor:tri_mono}, it doesn't hold for any other value of $p\in\Rx$ either.
  \end{enumerate}
\end{enumerate}
\end{example}

%---------------------------------------
\begin{example}
\label{ex:pdspace_21}
%---------------------------------------
Let $\distancen$ be a function in $\clF{\R\times\R}{\R}$ such that
\\\indentx$\distance{x}{y} \eqd \brb{\begin{array}{rMD}
      2\abs{x-y} & $\forall \opair{x}{y}\in\setn{\opair{0}{1},\,\opair{1}{0}}$ & (\prope{dilated Euclidean})\\
       \abs{x-y} & otherwise                                                   & (\prope{Euclidean})
    \end{array}}$.
\\
Note the following about the pair $\dspaceRd$:
\begin{enumerate}
  \item By \prefpp{ex:dspace_21}, $\dspaceRd$ is a \structe{distance space}, but not a \structe{metric space}---that is,
        the \rele{triangle relation} $\trirel(1,1;\distancen)$ does not hold in $\dspaceRd$.
  \item But observe further that $\pdspace{\R}{\distancen}{1}{2}$ \emph{is} a \structe{power distance space}:\label{item:pdspace_21}
    \begin{enumerate}
      \item Proof that $\trirel(1,2;\distancen)$ \xref{def:trirel} holds for all $\opair{x}{y}\in\setn{\opair{0}{1},\,\opair{1}{0}}$:
        \begin{align*}
          \distance{1}{0}
            &= \distance{0}{1}
             \eqd 2\abs{0-1}
             = 2
            && \text{by definition of $\distancen$}
            %&& \text{by definition of $\absn$ \xref{def:abs}}
          \\&\le 2
             \le 2\brp{\abs{0-z} + \abs{z-1}}\quad\scy\forall z\in\R
            && \text{by definition of $\absn$ \xref{def:abs}}
          \\&=   2\sigma\brp{\sfrac{1}{2}\abs{0-z}^p + \sfrac{1}{2}\abs{z-1}^p}^\frac{1}{p}\quad\scy\forall z\in\R
            && \text{for $\opair{p}{\sigma}=\opair{1}{2}$}
          \\&\eqd 2\sigma\brp{\sfrac{1}{2}\distancep{p}{0}{z} + \distancep{p}{z}{1}}^\frac{1}{p}\quad\scy\forall z\in\R%,\,\opair{p}{\sigma}=\opair{1}{2}
            && \text{for $\opair{p}{\sigma}=\opair{1}{2}$ and by definition of $\distancen$}
          \\&\eqd \ptf(1,2;0,1,z)
            && \text{by definition of $\ptf$ \xref{def:ptf}}
        \end{align*}

      \item Proof that $\trirel(1,2;\distancen)$ holds for all other $\opair{x}{y}\in\Rx\times\Rp$:
        \begin{align*}
          \distance{x}{y}
            &\eqd 2\abs{x-y}
            && \text{by definition of $\distancen$}
          \\&\le \brp{\abs{x-z} + \abs{z-y}}
            && \text{by property of \structe{Euclidean metric space}s}
          \\&=   2\sigma\brp{\sfrac{1}{2}\abs{0-z}^p + \sfrac{1}{2}\abs{z-1}^p}^\frac{1}{p}
            && \text{for $\opair{p}{\sigma}=\opair{1}{1}$}
          \\&\eqd \ptf(1,1;x,y,z)
            && \text{by definition of $\ptf$ \xref{def:ptf}}
          \\&\le  \ptf(1,2;x,y,z)
            && \text{by \prefpp{cor:tri_mono}}
        \end{align*}
    \end{enumerate}

  \item In $\dspaceX$, the limits of \prope{convergent} sequences are \prope{unique}.
        This follows directly from the fact that $\pdspace{\R}{\distancen}{1}{2}$ is a \structe{power distance space} \xref{item:pdspace_21}
        and by \prefp{thm:xn_to_xy}.

  \item In $\dspaceX$, \prope{convergent} sequences are \prope{Cauchy}.
        This follows directly from the fact that $\pdspace{\R}{\distancen}{1}{2}$ is a \structe{power distance space} \xref{item:pdspace_21}
        and by \prefp{thm:convergent==>cauchy}.
\end{enumerate}
\end{example}

%---------------------------------------
\begin{example}
\label{ex:pdspace_xy2}
%---------------------------------------
Let $\distancen$ be a function in $\clF{\R\times\R}{\R}$ such that $\distance{x}{y}\eqd(x-y)^2$.
Note the following about the pair $\dspaceRd$:
\begin{enumerate}
  \item It was demonstrated in \prefpp{ex:dspace_xy2} that $\dspaceRd$ is a \structe{distance space},
        but that it is \emph{not} a \structe{metric space} because the \prope{triangle inequality} does not hold.

  \item However, the tuple $\pdspace{\R}{\distancen}{p}{\sigma}$ \emph{is} a 
        \structe{power distance space} \xref{def:pdspace} for any $\opair{p}{\sigma}\in\Rx\times\intco{2}{\infty}$:
        In particular, for all $x,y,z\in\R$, the \rele{power triangle inequality} \xref{def:ptineq} must hold.
        The ``worst case" for this is when a third point $z$ is exactly ``halfway between" $x$ and $y$ in $\distance{x}{y}$; 
        that is, when $z=\frac{x+y}{2}$:
        \begin{align*}
          \brp{x-y}^2
            &\eqd \distance{x}{y}
            && \text{by definition of $\distancen$}
          \\&\le  \ptfD
            && \text{by definition \prope{power triangle inequality}}
          \\&\eqd 2\sigma\brs{\sfrac{1}{2}\distancep{p}{x}{z}+\sfrac{1}{2}\distancep{p}{z}{y}}^\frac{1}{p}
            && \text{by definition $\ptf$ \xref{def:ptf}}
          \\&\eqd  2\sigma\brs{\sfrac{1}{2}\brp{x-z}^{2p}+\sfrac{1}{2}\brp{z-y}^{2p}}^\frac{1}{p}
            && \text{by definition of $\distancen$}
          \\&=     2\sigma\brs{\sfrac{1}{2}\abs{x-z}^{2p}+\sfrac{1}{2}\abs{z-y}^{2p}}^\frac{1}{p}
            && \text{because $(x)^2 = \abs{x}^2$ for all $x\in\R$}
          \\&=     2\sigma\brs{\sfrac{1}{2}\abs{x-\frac{x+y}{2}}^{2p}+\sfrac{1}{2}\abs{\frac{x+y}{2}-y}^{2p}}^\frac{1}{p}
            && \text{because $z=\frac{x+y}{2}$ is the ``worst case" scenario}
          \\&=     2\sigma\brs{\sfrac{1}{2}\abs{\frac{y-x}{2}}^{2p}+\sfrac{1}{2}\abs{\frac{x-y}{2}}^{2p}}^\frac{1}{p}
          \\&=     2\sigma\brs{\abs{\frac{x-y}{2}}^{2p}}^\frac{1}{p}
             =     \frac{2\sigma}{4}\abs{x-y}^{2}
          %\\&\implies\text{$\sigma\ge2$ and $p\in\Rx$}
          \\&\implies \opair{p}{\sigma}\in\Rx\times\intco{2}{\infty}
        \end{align*}

  \item The \fncte{power distance function} $\distancen$ is \prope{continuous} in 
        $\pdspace{\R}{\distancen}{p}{\sigma}$
        for any $\opair{p}{\sigma}$ such that $\sigma\ge2$ and $2\sigma=p^\frac{1}{p}$.
        This follows directly from \prefpp{thm:pdspace_continuous}.
\end{enumerate}
\end{example}





