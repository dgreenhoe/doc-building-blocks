%============================================================================
% Daniel J. Greenhoe
% XeLaTeX file
% stochastic systems
%============================================================================

%=======================================
\section{Random variables on outcome subspaces}
%=======================================

%=======================================
\subsection{Definitions}
%=======================================
The traditional \fncte{random variable} \xref{def:rvt} is a mapping from a \structe{probability space} \xref{def:ps}
to the \structe{real line} \xref{def:Rline}.
This paper extends this definition to include functions with additional structure in the domain and 
expanded structure in the range (next definition).
%---------------------------------------
\begin{definition}
\label{def:ocsrv}
%---------------------------------------
%Let $\clFxy$ be the set of all functions with domain $\setX$ and range $\setY$.\\
\defboxp{
  A function $\rvX\in\clOCSgh$ \xref{def:clFxy} is an \fnctd{outcome random variable} if
  $\ocsG$ is an \structe{outcome subspace} \xref{def:ocsm} and
  $\omsH$ is an \structe{ordered quasi-metric space} \xref{def:oms}.
  }
\end{definition}

The definitions of \fncte{outcome expected value} and \fncte{outcome variance} (next definition) of an 
\fncte{outcome random variable}
are, in essence, identical to 
the \fncte{outcome center} \xref{def:ocsE} and \fncte{outcome variance} \xref{def:ocsVar} of 
\structe{outcome subspace}s \xref{def:ocs}
that \fncte{outcome random variable}s map from and by induction, to.
%---------------------------------------
\begin{definition}
\label{def:ocsE}
\label{def:ocsEa}
\label{def:ocsVar}
%---------------------------------------
Let $\ocsG$ be an \structe{outcome subspace} \xref{def:ocs},
    $\omsH$ an \structe{ordered quasi-metric space} \xref{def:oms},
and $\rvX$ be an \structe{outcome random variable} \xref{def:ocsrv} in $\in\clOCSgh$.
Let $\ocsH\eqd\ocsD$ be the \structe{outcome subspace} induced by $\omsH$, $\ocsG$, and $\rvX$.
Let $\ocsE_x$ be a function from $\ocso$ to the power set $\pset{\ocso}$.
%Let $\ocscen(\ocsH)$, $\ocscena(\ocsH)$, and $\ocsVaro(\ocsH)$ be the 
%\structe{outcome center}, \structd{arithmetic center} \xref{def:ocscen}, and 
%\structe{outcome variance} \xref{def:ocsVarG}, 
%respectively, of $\ocsH$.
\defbox{\begin{array}{MlM>{\ds}l}
  The \fnctd{outcome expected value}   &\ocsE (\rvX)       & of $\rvX$ is &\ocsE (\rvX) \eqd \argmin_{x\in\ocso}\max_{y\in\ocso}\metric{x}{y}\psp(y).\\
  The \fnctd{outcome variance}         &\ocsVar(\rvX;\pE_x)& of $\rvX$ is &\ocsVar(\rvX)\eqd \sum_{x\in\ocso}\metricsq{\pE_x(\rvX)}{x}\psp(x).\\
  \mc{4}{M}{Moreover, $\ocsVar(\rvX)\eqd\ocsVar(\rvX;\ocsE)$, where $\ocsE$ is the \fncte{outcome expected value} function.}      
  %The \fnctd{outcome expected value} &\ocsE (\rvX)  & of $\rvX$ is &\ocsE (\rvX) \eqd \ocscen (\ocsH). \\
  %The \fnctd{outcome variance}       &\ocsVar(\rvX) & of $\rvX$ is &\ocsVar(\rvX)\eqd \ocsVaro(\ocsH).
%  The \fnctd{arithmetic expected value} &\ocsEa(\rvX)       & of $\rvX$ is &\ocsEa(\rvX) \eqd \ocscena(\ocsH).\\
%  The \fnctd{arithmetic expected value} &\ocsEa(\rvX)       & of $\rvX$ is &\ocsEa(\rvX) \eqd \argmin_{x\in\ocso}\sum_{y\in\ocso}\metric{x}{y}\psp(y).\\
%  \mc{4}{M}{Also, $\ocsVar(\rvX)\eqd\ocsVar(\rvX;\ocsE)$.}      
\end{array}}
\end{definition}

%%---------------------------------------
%\begin{definition}
%\label{def:ocsE}
%\label{def:ocsEa}
%\label{def:ocsVar}
%%---------------------------------------
%Let $\rvX\in\clOCSgh$ be a \structe{random variable} \xref{def:ocsrv}.
%Let $\omsH'$ be the \structe{outcome subspace} \xref{def:ocs} generated by 
%the \structe{ordered quasi-metric space} $\omsH$ \xref{def:oms}, 
%the \structe{outcome subspace} $\ocsG$,
%and the \fncte{random variable} $\rvX$.
%Let $\ocscen(\omsH')$ be the \structe{center} \xref{def:ocscen},
%    $\ocscena(\omsH')$ the \structe{arithmetic center} \xref{def:ocscena}, 
%and $\ocsVaro(\omsH')$ the \fncte{outcome variance} \xref{def:ocsVarG} 
%of $\omsH'$.
%\\\defboxt{
%  The \fnctd{expected value} $\ocsE(\rvX)$ of $\rvX$ and 
%  the \fnctd{arithmetic expected value} $\ocsEa(\rvX)$ of $\rvX$ is
%  \\\indentx$\ds\ocsE (\rvX)\eqd \ocscen(\omsH')  \qquad\qquad   \ds\ocsEa(\rvX)\eqd \ocscena(\omsH')$.\\
%  The \fnctd{variance} $(\rvX)$ of $\rvX$ is 
%  \\\indentx$\ds\ocsVar(\rvX)\eqd\ocsVaro(\omsH)$.
%  }
%\end{definition}

%=======================================
\subsection{Properties}
%=======================================
%%---------------------------------------
%\begin{theorem}
%\label{thm:E}
%%---------------------------------------
%\mbox{}\\\begin{tabular}{lMll}
%    Let & \ocsG\eqd\ocs{\ocso_\ocsG}{\metricn_\ocsG}{\orel_\ocsG}{\psp_\ocsG} & be an \structe{outcome subspace}                         & \xref{def:ocs}.
%  \\Let & \omsH\eqd\oms{\ocso_\omsH}{\metricn_\omsH}{\orel_\omsH}             & be an \structe{ordered quasi-metric space}               & \xref{def:oqms}.
%  \\Let & \omsK\eqd\oms{\ocso_\omsK}{\metricn_\omsK}{\orel_\omsK}             & be an \structe{ordered quasi-metric space}              & \xref{def:oqms}.
%  \\Let & \rvX\in\clOCSgh                                                     & be a \structe{random variable} from $\ocsG$ onto $\omsH$ & \xref{def:ocsrv}.
%  \\Let & \ff\in\clF{\ocso_\omsH}{\ocso_\omsK}                                & be a function from $\ocso_\omsH$ onto $\ocso_\omsK$.      &
%  \\Let & \fg\in\clF{\R}{\R}                                                  & be a function from $\R$ into $\R$.                        &
%  \\Let & \ocsH\eqd\ocs{\ocso_\omsH}{\metricn_\omsH}{\orel_\omsH}{\psp_\omsH} & \mc{2}{l}{be an \structe{outcome subspace} induced by $\ocsG$, $\omsH$, and $\rvX$.}
%  \\Let & \ocsK\eqd\ocs{\ocso_\omsK}{\metricn_\omsK}{\orel_\omsK}{\psp_\omsK} & \mc{2}{l}{be an \structe{outcome subspace} induced by $\omsK$, $\ocsH$ and $\ff$.}
%\end{tabular}
%\thmbox{
%  \brb{\begin{array}{FMDD}
%    1. & $\ff$ is \prope{bijective}                           &  & and\\
%    2. & $\fg$ is \prope{isotone}                            &  & and\\
%    3. & $\metricn_\omsH\brp{\ff(x),\ff(y)} = \metricn_\omsH\brp{x,y}$ &
%    %1. & $\omsH$ and $\omsK$ are \prope{isometric}            & \xref{def:isometry} & and\\
%    %2. & $\ff$ is an \prope{isometry} in $\clF{\omsH}{\omsK}$ & \xref{def:isometry} &
%  \end{array}}
%  \quad\implies\quad
%  \brb{\ocsE\brs{\ff(\rvX)} = \ff\brs{\ocsE(\rvX)}}
%  }
%\end{theorem}
%\begin{proof}
%\begin{align*}
%  \ocsE\brs{\ff(\rvX)}
%    &= \argmin_{x\in\ocso_\omsK}\max_{y\in\ocso_\omsK} \metricn_\omsK(x,y)\psp_\omsK(y)
%    && \text{by definition of $\ocsE$ \xref{def:ocsE} and $\omsK$}
%  \\&= \ff\brs{\argmin_{x\in\ocso_\omsH}\max_{y\in\ocso_\omsH} \metricn_\omsK\brp{\ff(x),\ff(y)}\psp_\omsK(\ff(y))}
%    && \text{by $\ff$ \prope{bijection} hypothesis}
%  \\&= \ff\brs{\argmin_{x\in\ocso_\omsH}\max_{y\in\ocso_\omsH} \metricn_\omsH\brp{\ff(x),\ff(y)}\psp_\omsH(y)}
%    && \text{by $\ff$ \prope{bijection} hypothesis}
%  \\&= \ff\brs{\argmin_{x\in\ocso_\omsH}\max_{y\in\ocso_\omsH} \fg\brs{\metricn_\omsH(x,y)}\psp_\omsH(y)}
%    && \text{by $\metricn_\omsH$ hypothesis}
%  \\&= \ff\brs{\argmin_{x\in\ocso_\omsH}\fg\brs{\max_{y\in\ocso_\omsH} \metricn_\omsH(x,y)}\psp_\omsH(y)}
%    && \text{by $\fg$ is \prope{isotone} hypothesis}
%  \\&= \ff\brs{\argmin_{x\in\ocso_\omsH}\max_{y\in\ocso_\omsH} \metricn_\omsH(x,y)\psp_\omsH(y)}
%    && \text{by $\fg$ is \prope{isotone} hypothesis}
%  \\&= \ff\ocsE(\rvX)
%    && \text{by definition of $\ocsE$ \xref{def:ocsE} and $\rvX$}
%\end{align*}
%\end{proof}

%---------------------------------------
\begin{theorem}
\label{thm:ocsVar}
% 2015 February 03 Tuesday ~6:00 PM Taiwan
%---------------------------------------
Let $\rvX\in\clOCSgh$ be a \fncte{random variable} \xref{def:ocsrv}
on an \structe{ordered quasi-metric space} \xref{def:oqms} $\omsH$.
Let $\ocsH\eqd\ocsD$ be the \structe{outcome subspace} \xref{def:ocs} induced by $\omsH$, $\ocsG$, and $\rvX$.
Let $\pVar(\rvX)$ be the \fncte{traditional variance} \xref{def:pVar} of $\rvX$.
Let $\ocsVar(\rvX)$ be the \fncte{outcome subspace variance} of $\rvX$ \xref{def:ocsVar}.
\thmbox{
  \brb{\begin{array}{M}
    $\omsH\eqd\omsR$ is the \structe{real line} \xref{def:Rline}
  \end{array}}
  \quad\implies\quad
  \brb{\begin{array}{c}
    \ocsVar(\rvX;\pE) = \pVar(\rvX)
  \end{array}}
  }
\end{theorem}
\begin{proof}
\begin{align*}
  \ocsVar(\rvX;\pE)
    &\eqd \sum_{x\in\omsH} \metricsq{\pE(\rvX)}{x}\psp(x)
    && \text{by definition of $\ocsVar$ \xref{def:ocsVar}}
  \\&= \sum_{x\in\R} \abs{\pE(\rvX)-x}^2\psp(x)
    && \text{by definition of real line $\omsH$ \xref{def:Rline}}
  \\&= \int_{\R} \brp{x-\pE(\rvX)}^2\psp(x) \dx
    && \text{by definition of Lebesgue integration on $\R$}
  \\&= \pVar(\rvX)
    && \text{by definition of $\pVar$ \xref{def:pVar}}
\end{align*}
\end{proof}

%---------------------------------------
\begin{remark}
\label{rem:pEocsE}
%---------------------------------------
Despite the correspondence of traditional variance and outcome variance on the \structe{real line} 
as demonstrated in \prefpp{thm:ocsVar},
the situation is different for expected values.
Even when both are calculated on the same \structe{real line},
the \vale{traditional expected value} $\pE(\rvX)$ \xref{def:pE} 
and the \vale{outcome expected value} $\ocsE(\rvX)$ \xref{def:ocsE} don't always yield the same value.
Demonstrations of this include \prefpp{ex:wdie_xy} and \prefpp{ex:lcg7x1m9_xyz}.
However, there is one common situation in which the two statistics do correspond (next theorem).
\end{remark}

%---------------------------------------
\begin{theorem}
\label{thm:pEocsE}
%---------------------------------------
Let $\rvX$, $\omsH$, $\ocsG$ be defined as in \prefpp{thm:ocsVar}.
Let $\pE(\rvX)$ be the \fncte{traditional expected value} \xref{def:pE}
and $\ocsE(\rvX)$ the \fncte{outcome expected value} of $\rvX$ \xref{def:ocsE}.
\thmbox{
  \brb{\begin{array}{FlDD}
    1. & \omsH\eqd\omsR                             & (\structe{real line} \xrefnp{def:Rline}) & and \\
    2. & \psp(a-x)=\psp(a+x)\quad\scy\forall x\in\R & (\prope{symmetric} about $a$)
  \end{array}}
  \quad\implies\quad
  \brb{\begin{array}{rclcl}
    \ocsE(\rvX) &=& a &=& \pE(\rvX)
  \end{array}}
  }
\end{theorem}
\begin{proof}
\begin{align*}
  \boxed{\ocsE(\rvX)}
    &\eqd \argmin_{x\in\R}\max_{y\in\R}\metric{x}{y}\psp(y)
    && \text{by definition of $\ocsE$ \xref{def:ocsE}}
  \\&= \argmin_{x\in\R}\max_{y\in\R}\abs{x-y}\psp(y)
    && \text{by definition of \structe{real line} \xref{def:Rline}}
  \\&= \boxed{a}
    %&& \text{\gsize\centering\psset{unit=6mm}{%============================================================================
% Daniel J. Greenhoe
% XeLaTeX file
%============================================================================
{%\psset{yunit=2\psunit}%
\begin{pspicture}(-2.5,-0.3)(7,2.5)%
  \psset{%
    labelsep=1pt,
    linewidth=1pt,
    }%
  \psaxes[linecolor=axis,yAxis=false,labels=none,ticks=none]{<->}(0,0)(-2.5,0)(6.5,2.5)% x axis
  \psaxes[linecolor=axis,xAxis=false,labels=none,ticks=none]{->}(0,0)(-2.5,0)(6.5,2.5)% y axis
  \rput(1,0){% g(y)=|x-y|
    \psline(-2,2)(0,0)(2,2)%
    \uput[-90]{0}(0,0){$x$}%
    \uput[-22]{0}(2,2){$\ff(y)\eqd\abs{x-y}$}
    }%
  \rput(1.75,0){% Gaussian like distribution
    %\psplot[plotpoints=64]{-2}{6}{2.718 x 2 sub 2 exp neg 2 div exp 2.507 div}% Gaussian distribution with sigma=1 and mean=2
    \psplot[plotpoints=64]{-4}{4}{2.718 x 2 exp neg 2 div exp }% 
    \uput[-90]{0}(0,0){$a$}%
    \psline[linestyle=dotted,linecolor=red](0,1)(0,0)%
    \uput[45]{0}(1.2,0.5){$\psp(y)$}
    }%
  %\psline[linestyle=dotted,linecolor=red](0,0.75)(3.5,0.75)%
  %
  \uput[0]{0}(6.5,0){$y$}%
  %\rput[t](3.5,2){$\ds\ff(x)\eqd\max_{y\in\R}\abs{x-y}\psp(y)$ minimized when $x=a$}%
\end{pspicture}}%
}}
    && \text{\gsize\centering\psset{unit=6mm}{\includegraphics{sto/graphics/pEocsE_max.pdf}}}
  \\&&&\text{because $\fh(x)\eqd\max_{y\in\R}\abs{x-y}\psp(y)$ is minimized when $x=a$}%
  \\&= \boxed{\pE(\rvX)}
    && \text{by \prefpp{prop:pspsym}}
%{prop:pspsym}
\end{align*}
\end{proof}

%---------------------------------------
\begin{theorem}
\label{thm:EfX}
%---------------------------------------
\mbox{}\\\begin{tabular}{lMll}
    Let & \ocsG\eqd\ocs{\ocso_\ocsG}{\metricn_\ocsG}{\orel_\ocsG}{\psp_\ocsG} & be an \structe{outcome subspace}                         & \xref{def:ocs}.
  \\Let & \omsH\eqd\oms{\ocso_\omsH}{\metricn_\omsH}{\orel_\omsH}             & be an \structe{ordered quasi-metric space}               & \xref{def:oqms}.
  \\Let & \omsK\eqd\oms{\ocso_\omsK}{\metricn_\omsK}{\orel_\omsK}             & be an \structe{ordered quasi-metric space}              & \xref{def:oqms}.
  \\Let & \rvX\in\clOCSgh                                                     & be a \structe{random variable} from $\ocsG$ onto $\omsH$ & \xref{def:ocsrv}.
  \\Let & \ff\in\clF{\ocso_\omsH}{\ocso_\omsK}                                & be a function from $\ocso_\omsH$ onto $\ocso_\omsK$ (\fncte{pullback}) & \xref{thm:met_sumpf}.
  \\Let & \fphi\in\clF{\R}{\R}                                                & be a function from $\R$ into $\R$ (\fncte{pushforward})  & \xref{def:mpf}.
  \\Let & \ocsH\eqd\ocs{\ocso_\omsH}{\metricn_\omsH}{\orel_\omsH}{\psp_\omsH} & \mc{2}{l}{be an \structe{outcome subspace} induced by $\ocsG$, $\omsH$, and $\rvX$.}
  \\Let & \ocsK\eqd\ocs{\ocso_\omsK}{\metricn_\omsK}{\orel_\omsK}{\psp_\omsK} & \mc{2}{l}{be an \structe{outcome subspace} induced by $\omsK$, $\ocsH$ and $\ff$.}
\end{tabular}
\thmbox{
  \brb{\begin{array}{FMDD}
    1. & $\ff$ is \prope{injective}                           &  & and\\
    2. & $\fphi$ is \prope{strictly isotone}                  &  & and\\
    3. & $\metricn_\omsH\brp{\ff(x),\ff(y)}\psp(y) = \fphi\brs{\metricn_\omsH\brp{x,y}\psp(y)}$ &
    %1. & $\omsH$ and $\omsK$ are \prope{isometric}            & \xref{def:isometry} & and\\
    %2. & $\ff$ is an \prope{isometry} in $\clF{\omsH}{\omsK}$ & \xref{def:isometry} &
  \end{array}}
  \quad\implies\quad
  \brb{\ocsE\brs{\ff(\rvX)} = \ff\brs{\ocsE(\rvX)}}
  }
\end{theorem}
\begin{proof}
\begin{align*}
  \ocsE\brs{\ff(\rvX)}
    &= \argmin_{x\in\ocso_\omsK}\max_{y\in\ocso_\omsK} \metricn_\omsK(x,y)\psp_\omsK(y)
    && \text{by definition of $\ocsE$ \xref{def:ocsE} and $\omsK$}
  \\&= \ff\brs{\argmin_{x\in\ocso_\omsH}\max_{y\in\ocso_\omsH} \metricn_\omsK\brp{\ff(x),\ff(y)}\psp_\omsK(\ff(y))}
    && \text{by $\ff$ \prope{bijection} hypothesis}
  \\&= \ff\brs{\argmin_{x\in\ocso_\omsH}\max_{y\in\ocso_\omsH} \metricn_\omsH\brp{\ff(x),\ff(y)}\psp_\omsH(y)}
    && \text{by $\ff$ \prope{bijection} hypothesis}
  \\&= \ff\brs{\argmin_{x\in\ocso_\omsH}\max_{y\in\ocso_\omsH} \fphi\brs{\metricn_\omsH(x,y)\psp_\omsH(y)}}
    && \text{by $\metricn_\omsH$ hypothesis}
  \\&= \ff\brs{\argmin_{x\in\ocso_\omsH}\max_{y\in\ocso_\omsH} \metricn_\omsH(x,y)\psp_\omsH(y)}
    && \text{by $\fphi$ is \prope{strictly isotone} hypothesis and \prefp{lem:argminmaxphi}}
  \\&= \ff\ocsE(\rvX)
    && \text{by definition of $\ocsE$ \xref{def:ocsE} and $\rvX$}
\end{align*}
\end{proof}


%%---------------------------------------
%\begin{corollary}
%\label{cor:ocsrv_Eax_group}
%%---------------------------------------
%Let $\omsH\eqd\omsD$ be an \structe{ordered metric space} \xref{def:oms}
%and $\rvX\in\clOCSgh$ a \structe{random variable} \xref{def:ocsrv} onto $\omsH$.
%Let $\addmult$ be a binary operator on $\ocso$ (a function in the set $\clF{\ocso\times\ocso}{\ocso}$).
%\thmbox{
%  \brb{\begin{array}{FMD}
%    1. & $\opair{\ocso}{\addmult}$ is a \structe{group} & and\\
%    2. & for some $a\in\ocso$, &\\
%       & $\ff(x)\eqd a\addmult x$ generates a \fncte{permutation} of $\opair{\ocso}{\addmult}$
%    %(either \prope{multiplicative} or \prope{additive group)
%  \end{array}}
%  \quad\implies\quad
%  \brb{\begin{array}{l}
%    \ocsE(a\addmult\rvX) = a\addmult\ocsE(\rvX)
%  \end{array}}
%  }
%\end{corollary}
%\begin{proof}
%\begin{align*}
%  \ocsE(a\addmult\rvX)
%    &= a\addmult\ocsE(\rvX)
%    && \text{because $\ff(x)=a\addmult x$ is an \prope{isometry} on $\ocsH$ and by \prefpp{thm:ocsrv_isometry}}
%\end{align*}
%\end{proof}


%---------------------------------------
\begin{corollary}
\label{cor:ocsrv_Eax_R}
%---------------------------------------
Let $\omsH$ be an \structe{ordered metric space} \xref{def:oms}
and $\rvX\in\clOCSgh$ a \structe{random variable} \xref{def:ocsrv} onto $\omsH$.
Let $\oms{\R}{\absn}{\le}$ be the \structe{real line ordered metric space} \xref{def:Rline}.
\thmbox{
    \mcom{\omsH=\oms{\R}{\absn}{\le}}{(real line)}
  \quad\implies\quad
  \brb{\begin{array}{lC}
    \ocsE(a\rvX) = a\ocsE(\rvX) & \forall a\in\Rnn
  \end{array}}
  }
\end{corollary}
\begin{proof}
\begin{enumerate}
  \item Proof for $a=0$ case: 
    \begin{align*}
      \ocsE(0\cdot\rvX)
        &= \argmin_{x\in0\cdot\omsH}\max_{y\in0\cdot\omsH} \metricn(x,y)\psp(y)
        && \text{by definition of $\ocsE$ \xref{def:ocsE}}
      \\&= \argmin_{x\in\setn{0}}\max_{y\in\setn{0}} \metricn(x,y)\psp(y)
      \\&= \argmin_{x\in\setn{0}}\max_{y\in\setn{0}} \metricn(0,0)\psp(y)
      \\&= \argmin_{x\in\setn{0}}\max_{y\in\setn{0}} 0\psp(y)
        && \text{by \prope{nondegenerate} property of $\metricn$ \xref{def:metric}}
      \\&= 0
      \\&= 0\cdot\ocsE(\rvX)
    \end{align*}

  \item Proof for $a>0$ case:
    $\metricn(\ff(x),\ff(y))\psp(y) \eqd \abs{ax-ay}\psp(y) = \abs{a}\abs{x-y}\psp(y) \eqd \abs{a}\abs{\metricn(x,y)\psp(y)}$
    \begin{align*}
      \ocsE(a\rvX)
        &= a\ocsE(\rvX)
        && \text{because $\ff(x)=ax$ is \prope{strictly isotone} on the real line and by \prefpp{thm:EfX}}
    \end{align*}
\end{enumerate}
\end{proof}

%In a ring or even a multiplicative group, $\ocsE(a\rvX)=a\ocsE(\rvX)$,
%as demonstrated in the next theorem and illusrated in \prefpp{ex:pspinner_x2x}.
%%---------------------------------------
%\begin{corollary}
%\label{cor:ocsrv_2x}
%%---------------------------------------
%Let $\omsH$ be a \structe{multiplicative group} such that $a\omsH=\setn{ah_1,ah_2,\ldots ah_\xN}$ for all $x,h_n\in\omsH$.
%Let $\rvX\in\clOCSgh$ be a \structe{random variable} \xref{def:ocsrv}.
%Let $\R$ be the set of real numbers and let $\ff$ be a function in $\clF{\omsH}{\R}$.
%\corbox{
%  \brb{\begin{array}{rclC}
%    \metric{ax}{ay} &=& \ff(a)\metric{x}{y}
%  \end{array}}
%  \quad\implies\quad
%  \ocsE(a\rvX) = \abs{\ff(a)}\ocsE(\rvX)
%  }
%\end{corollary}
%\begin{proof}
%\begin{align*}
%  \ocsE(a\rvX) 
%    &= a\argmin_{x\in a\omsH}\max_{y\in a\omsH} \metric{x}{y}\psp_{a\omsH}(y)
%    && \text{by definition of $\ocsE$ \xref{def:ocsE}}
%  \\&= a\argmin_{x\in\omsH}\max_{y\in\omsH} \metric{ax}{ay}\psp_{\omsH}(ay)
%    && \text{by definition of $\ocsE$ \xref{def:ocsE}}
%  \\&= a\argmin_{x\in\omsH}\max_{y\in\omsH} \metric{ax}{ay}\psp_{\rvX}(y)
%  \\&= a\argmin_{x\in\omsH}\max_{y\in\omsH} \abs{\metric{ax}{ay}}\psp(y)
%    && \text{by \prope{non-negative} property of $\metricn$ \xref{def:metric}}
%  \\&= a\argmin_{x\in\omsH}\max_{y\in\omsH} \abs{\ff(a)\metric{x}{y}}\psp(y)
%    && \text{by left hypothesis}
%  \\&= a\argmin_{x\in\omsH}\max_{y\in\omsH} \abs{\ff(a)}\abs{\metric{x}{y}}\psp(y)
%    && \text{by property of $\absn$}
%  \\&= a\abs{\ff(a)}\argmin_{x\in\omsH}\max_{y\in\omsH} {\metric{x}{y}}\psp(y)
%  \\&\eqd a\ocsE(\rvX)
%\end{align*}
%\end{proof}

%\if 0
%=======================================
\subsection{Problem statement}
%=======================================
The \fncte{traditional random variable} $\rvX$ \xref{def:rvt} is a function that maps from a 
\structe{stochastic process} to the \structe{real line} \xref{def:Rline}.
The traditional expectation value $\pE(\rvX)$ of $\rvX$ is then often a poor choice of a statistic
when the stochastic process that $\rvX$ maps from is a structure other than the real line or 
some substructure of the real line.
There are two fundamental problems:
\begin{enumerate}
  \item A traditional random variable $\rvX$ maps \textbf{to} 
        the \prope{linearly ordered} real line.
        However, $\rvX$ often maps \textbf{from} a random process that is \prope{non-linearly ordered}
        (or even \prope{unordered} \xxrefnp{def:order}{def:chain}).

  \item A traditional random variable $\rvX$ maps \textbf{to} 
        the real line with a \structe{metric geometry} \xref{rem:mgeo} 
        induced by the \fncte{usual metric} \xref{def:d_usual}.
        But many random processes have a fundamentally different \structe{metric geometry},
        a common one being that induced by the \fncte{discrete metric} \xref{def:dmetric}.
\end{enumerate}
Thus, the order structure of the domain and range of $\rvX$ are often fundamentally dissimilar,
leading to statistics, such as $\pE(\rvX)$, that are
of poor quality with regards to qualitative intuition and quantitative variance (expected error) measurements,
and of dubious suitability for tasks such as decision making, prediction, and hypothesis testing.

%---------------------------------------
\begin{remark}
%---------------------------------------
Unlike in traditional statistical processing,
it in general \textbf{not true} that $\ocsE(\rvX+\rvY)=\ocsE(\rvX)+\ocsE(\rvY$).
See \prefpp{ex:rline_11312a} for a counter example.
\end{remark}

%=======================================
%\subsection{A poor solution}
%\paragraph{A poor solution.}
%=======================================
%---------------------------------------
\begin{remark}
%---------------------------------------
A possible solution to the traditional random variable order and metric geometry problem is to allow the 
random variable to map into the \structe{complex plane} \xref{ex:Cplane} with the usual metric, 
rather than into the real line only. 
However, this is a poor solution, as demonstrated in \prefpp{ex:gsp_C}.
\end{remark}

%=======================================
\subsection{Examples}
%=======================================
\begin{tabstr}{0.75}
%\if 0
%=======================================
\subsubsection{Fair die examples}
%=======================================
\begin{figure}[h]
  \centering%
  \gsize%
  %{%============================================================================
% Daniel J. Greenhoe
% LaTeX file
% real dice mappings to real line and integer line
%============================================================================
\begin{pspicture}(-4.3,-1.5)(4.3,1.8)%
  %---------------------------------
  % options
  %---------------------------------
  \psset{%
    %radius=1.25ex,
    labelsep=2.5mm,
    linecolor=blue,%
    }%
  %---------------------------------
  % dice graph
  %---------------------------------
  \rput(0,0){%\psset{unit=2\psunit}%
    \Cnode[radius=1.25ex,fillstyle=solid,fillcolor=snode](-0.8660,-0.5){D4}%
    \Cnode[radius=1.25ex,fillstyle=solid,fillcolor=snode](-0.8660,0.5){D5}%
    \Cnode[radius=1.25ex,fillstyle=solid,fillcolor=snode](0,1){D6}%
    \Cnode[radius=1.25ex,fillstyle=solid,fillcolor=snode](0.8660,0.5){D3}%
    \Cnode[radius=1.25ex,fillstyle=solid,fillcolor=snode](0.8660,-0.5){D2}%
    \Cnode[radius=1.25ex,fillstyle=solid,fillcolor=snode](0,-1){D1}%
    }
  \uput{5mm}[90](D6){$\ocsG$}%
  \rput(D6){$\diceF$}%
  \rput(D5){$\diceE$}%
  \rput(D4){$\diceD$}%
  \rput(D3){$\diceC$}%
  \rput(D2){$\diceB$}%
  \rput(D1){$\diceA$}%
  %
  \ncline{D5}{D6}%
  \ncline{D4}{D5}\ncline{D4}{D6}%
  \ncline{D3}{D4}\ncline{D3}{D5}\ncline{D3}{D6}%
  \ncline{D2}{D3}\ncline{D2}{D4}\ncline{D2}{D5}\ncline{D2}{D6}%
  \ncline{D1}{D2}\ncline{D1}{D3}\ncline{D1}{D4}\ncline{D1}{D5}\ncline{D1}{D6}%
  %
  \uput[158](D6){$\frac{1}{6}$}
  \uput[150](D5){$\frac{1}{6}$}
  \uput[210](D4){$\frac{1}{6}$}
  \uput[ 22](D3){$\frac{1}{6}$}
  \uput[-45](D2){$\frac{1}{6}$}
  \uput[-158](D1){$\frac{1}{6}$}
  %---------------------------------
  % Real line
  %---------------------------------
  \rput(-2.75,0){\psset{unit=0.35\psunit}%
    \psline{<->}(0,-3.5)(0,3.5)%
    \pnode(0,2.5){R6}%
    \pnode(0,1.5){R5}%
    \pnode(0,0.5){R4}%
    \pnode(0,0){R34}%
    \pnode(0,-0.5){R3}%
    \pnode(0,-1.5){R2}%
    \pnode(0,-2.5){R1}%
    \Cnode[fillstyle=solid,linecolor=snode,fillcolor=snode,radius=0.5ex](0,0){RC}%
    \uput[135](0,3.5){$\omsR$}%
    }%
  \pscircle[fillstyle=none,linecolor=red,fillcolor=red](RC){0.6ex}%
  %\ncline{R5}{R6}%
  %\ncline{R4}{R5}%
  %\ncline{R3}{R4}%
  %\ncline{R2}{R3}%
  %\ncline{R1}{R2}%
  \rput(R6){\psline[linewidth=1pt](-0.1,0)(0.1,0)}%
  \rput(R5){\psline[linewidth=1pt](-0.1,0)(0.1,0)}%
  \rput(R4){\psline[linewidth=1pt](-0.1,0)(0.1,0)}%
  \rput(R3){\psline[linewidth=1pt](-0.1,0)(0.1,0)}%
  \rput(R2){\psline[linewidth=1pt](-0.1,0)(0.1,0)}%
  \rput(R1){\psline[linewidth=1pt](-0.1,0)(0.1,0)}%
  %
  \uput[180](R6){$6$}%
  \uput[180](R5){$5$}%
  \uput[180](R4){$4$}%
  \uput[180](R3){$3$}%
  \uput[180](R2){$2$}%
  \uput[180](R1){$1$}%
  \uput[0](R34){$3.5$}%
  %
  \ncarc[arcangle=-22,linewidth=0.75pt,linecolor=red]{->}{D6}{R6}%
  \ncarc[arcangle=-22,linewidth=0.75pt,linecolor=red]{->}{D5}{R5}%
  \ncarc[arcangle=-22,linewidth=0.75pt,linecolor=red]{->}{D4}{R4}%
  \ncarc[arcangle= 22,linewidth=0.75pt,linecolor=red]{->}{D3}{R3}%
  \ncarc[arcangle= 22,linewidth=0.75pt,linecolor=red]{->}{D2}{R2}%
  \ncarc[arcangle= 22,linewidth=0.75pt,linecolor=red]{->}{D1}{R1}%
  %
  \rput(-1.8,1){$\rvX(\cdot)$}%
  %---------------------------------
  % integer line
  %---------------------------------
  \rput(2.75,0){\psset{unit=0.35\psunit}%
    \pnode(0,3.5){I7}%
    \Cnode(0,2.5){I6}%
    \Cnode(0,1.5){I5}%
    \Cnode[fillstyle=solid,fillcolor=snode](0,0.5){I4}%
    \Cnode[fillstyle=solid,fillcolor=snode](0,-0.5){I3}%
    \Cnode(0,-1.5){I2}%
    \Cnode(0,-2.5){I1}%
    \pnode(0,-3.5){I0}%
    }%
  \uput[45](I7){$\omsZ$}%
  \ncline[linestyle=dotted]{I6}{I7}%
  \ncline{I5}{I6}%
  \ncline{I4}{I5}%
  \ncline{I3}{I4}%
  \ncline{I2}{I3}%
  \ncline{I1}{I2}%
  \ncline[linestyle=dotted]{I1}{I0}%
  %
  \uput[0](I6){$6$}%
  \uput[0](I5){$5$}%
  \uput[0](I4){$4$}%
  \uput[0](I3){$3$}%
  \uput[0](I2){$2$}%
  \uput[0](I1){$1$}%
  %
  \ncarc[arcangle= 22,linewidth=0.75pt,linecolor=green]{->}{D6}{I6}%
  \ncarc[arcangle= 22,linewidth=0.75pt,linecolor=green]{->}{D5}{I5}%
  \ncarc[arcangle= 22,linewidth=0.75pt,linecolor=green]{->}{D4}{I4}%
  \ncarc[arcangle=-22,linewidth=0.75pt,linecolor=green]{->}{D3}{I3}%
  \ncarc[arcangle=-22,linewidth=0.75pt,linecolor=green]{->}{D2}{I2}%
  \ncarc[arcangle=-22,linewidth=0.75pt,linecolor=green]{->}{D1}{I1}%
  %
  \rput(1.8,1){$\rvY(\cdot)$}%
\end{pspicture}%}%
  {\includegraphics{sto/graphics/fairdieXRYZ.pdf}}%
  \caption{random variable mappings from the fair die to the real line and integer line \label{fig:fairdieXRYZ}}
\end{figure}
%---------------------------------------
\begin{example}[\exmd{fair die mappings to real line and integer line}]
\label{ex:fairdieXRYZ} %\mbox{}\\
%---------------------------------------
Let $\ocsG$ be the \structe{fair die outcome subspace} \xref{ex:fairdie}.
Let $\rvX\in\clOCSgr$ be a \fncte{random variable} \xref{def:ocsrv} mapping from $\ocsG$ to the \structe{real line} \xref{def:Rline},
and $\rvY\in\clOCSgz$ be a \fncte{random variable} \xref{def:ocsrv} mapping from $\ocsG$ to the \structe{integer line} \xref{def:Zline},
as illustrated in \prefpp{fig:fairdieXRYZ}.
%
Let $\pE$   be the \fncte{traditional expected value} function \xref{def:pE},
    $\pVar$ the \fncte{traditional variance} function \xref{def:pVar},
    $\ocsE$ the \fncte{outcome expected value} function \xref{def:ocsE}, 
and $\ocsVar$ the \fncte{outcome variance} function \xref{def:ocsVar}.
%
This yields the following statistics:
\\\indentx$\begin{array}{>{\gsize}Mrcc rcccl}
  geometry of $\ocsG$:                                    & \ocscen (\ocsG) &=& \mc{4}{l}{\setn{\diceA,\diceB,\diceC,\diceD,\diceE,\diceF}}  \\
  traditional      statistics on \structe{real line}:     & \pE  (\rvX)     &=& 3.5           & \ocsVar(\rvX;\pE)   &=& \frac{35}{12}&\approx& 2.917 \\
  outcome subspace statistics on \structe{real line}:     & \ocsE (\rvX)    &=& \setn{3.5}    & \ocsVar(\rvX;\ocsE) &=& \frac{35}{12}&\approx& 2.917 \\
  outcome subspace statistics on \structe{integer line}:  & \ocsE (\rvY)    &=& \setn{3,\,4}  & \ocsVar(\rvY;\ocsE) &=& \frac{20}{12}&\approx& 1.667 
\end{array}$
\end{example}
\begin{proof}
    \begin{align*}
      \ocscen(\ocsG)
        %&=\ocscena(\ocsG)
        %&&\text{by \prefp{ex:fairdie}} 
        &=\setn{\diceA,\diceB,\diceC,\diceD,\diceE,\diceF}
        &&\text{by \prefp{ex:fairdie}} 
      \\
      \pE(\rvX)
        &\eqd\sum_{x\in\R} x\psp(x)
        && \text{by definition of $\pE$ \xref{def:pE}}
      \\&= \mathrlap{
           \sum_{x\in\R} x \frac{1}{6}
         = \frac{1+2+3+4+5+6}{6}
         = \frac{21}{6}
         = \frac{7}{2}
         = 3.5}
      \\
      \ocsVar(\rvX;\pE)
        &= \pVar(\rvX)
        && \text{by \prefp{thm:ocsVar}}
      \\&\eqd \sum_{x\in\R} \brs{x-\pE(\rvX)}^2 \psp(x)
        &&\text{by definition of $\pVar$ \xref{def:pVar}}
      \\&= \sum_{x\in\R} \brp{x-\frac{7}{2}}^2 \frac{1}{6}
        &&\text{by $\pE(\rvX)$ result}
      \\&= \mathrlap{\brs{\brp{1-\frac{7}{2}}^2 + \brp{2-\frac{7}{2}}^2 + \brp{3-\frac{7}{2}}^2 + \brp{4-\frac{7}{2}}^2 + \brp{5-\frac{7}{2}}^2 + \brp{6-\frac{7}{2}}^2 }\frac{1}{6}}
      \\&= \mathrlap{\brs{(2-7)^2 + (4-7)^2 + (6-7)^2 + (8-7)^2 + (10-7)^2 + (12-7)^2}\frac{1}{2^2\times6}}
      \\&= \mathrlap{\frac{25+9+1+1+9+25}{24}
         = \frac{70}{24}
         = \frac{35}{12}
        \approx 2.917}
  %  \end{align*}
  %
  %\item \structe{outcome subspace} statistics of random variable mapping $\rvX$ to \structe{real line} \xref{def:Rline}:
  %  \begin{align*}
      \\
      \ocsE(\rvX)
        &= \pE(\rvX)
        && \text{by \prefpp{thm:pEocsE}}
      \\&= \setn{\frac{7}{2}}=\setn{3.5}
        && \text{by $\pE(\rvX)$ result}
      %  && \text{by $\ocsE(\rvX)$ result}
      %  &\eqd \argmin_{x\in\R}\max_{y\in\R}\metric{x}{y}\psp(y)
      %  && \text{by definition of $\ocsE$ \xref{def:ocsE}}
      %\\&\eqd \argmin_{x\in\R}\max_{y\in\R}\abs{x-y}\frac{1}{6}
      %  && \text{by definition of \structe{real line} \xref{def:Rline} and $\ocsG$}
      %\\&= \argmin_{x\in\R}\max_{y\in\R}\abs{x-y}
      %  && \text{because $\fphi(x)=\frac{1}{6}x$ is \prope{strictly isotone} and by \prefp{lem:argminmaxphi}}
      %\\&= \argmin_{x\in\R}\brbl{\begin{array}{lM}
      %       \abs{x-1} & for $x\ge3.5$\\
      %       \abs{x-6} & for $x<3.5$
      %     \end{array}}
      %\\&= \setn{3.5}
      %  && \text{because expression is minimized at $x=3.5$}
      \\
      \ocsVar(\rvX;\ocsE)
        &\eqd \sum_{x\in\R}\metricsq{\ocsE(\rvX)}{x}\psp(x)
        && \text{by definition of $\ocsVar$ \xref{def:ocsVar}}
      \\&= \sum_{x\in\R}\metricsq{\pE(\rvX)}{x}\psp(x)
        && \text{by $\pE(\rvX)$ and $\ocsE(\rvX)$ results}
      \\&\ocsVar(\rvX;\pE)
        && \text{by definition of $\ocsVar$ \xref{def:ocsVar}}
      \\&= \frac{35}{12} \approx 2.917
        && \text{by $\pVar(\rvX)$ result}
%    \end{align*}
%
%  \item \structe{outcome subspace} statistics of random variable mapping $\rvY$ to \structe{integer line} \xref{def:Zline}:
%    \begin{align*}
      \\
      \ocsE(\rvY)
        &\eqd \argmin_{x\in\Z}\max_{y\in\omsH}\metric{x}{y}\psp(y)
        &&\text{by definition of $\ocsE$ \xref{def:ocsE}}
      \\&=\argmin_{x\in\Z}\max_{y\in\omsH}\abs{x-y}\frac{1}{6}
        && \text{by definition of \structe{integer line} \xref{def:Zline} and $\ocsG$}
      \\&=\argmin_{x\in\Z}\max_{y\in\omsH}\abs{x-y}
        && \text{because $\fphi(x)=\frac{1}{6}x$ is \prope{strictly isotone} and by \prefp{lem:argminmaxphi}}
      \\&=\mathrlap{\argmin_{x\in\Z}\max_{y\in\omsH}
             \setn{\begin{array}{cccccc}
                0 & 1 & 2 & 3 & 4 & 5 \\
                1 & 0 & 1 & 2 & 3 & 4 \\
                2 & 1 & 0 & 1 & 2 & 3 \\
                3 & 2 & 1 & 0 & 1 & 2 \\
                4 & 3 & 2 & 1 & 0 & 1 \\
                5 & 4 & 3 & 2 & 1 & 0 
             \end{array}}
      \quad= \argmin_{x\in\Z}
             \setn{\begin{array}{c}
                5\\
                4\\
                3\\
                3\\
                4\\
                5
             \end{array}}
      \quad= \argmin_{x\in\Z}
             \setn{\begin{array}{c}
                 \\
                 \\
                3\\
                4\\
                 \\
                \mbox{}
             \end{array}}}
      %\\
      %\ocsEa(\rvY)
      %  &\eqd \argmin_{x\in\omsH}\sum_{y\in\omsH}\metric{x}{y}\psp(y)
      %  &&\text{by definition of $\ocsE$ \xref{def:ocsE}}
      %\\&=\argmin_{x\in\omsH}\sum_{y\in\omsH}\abs{x-y}\frac{1}{6}
      %  && \text{by definition of \structe{integer line} \xref{def:Zline} and $\ocsG$}
      %\\&=\argmin_{x\in\omsH}\sum_{y\in\omsH}\abs{x-y}
      %  && \text{because $\fphi(x)=\frac{1}{6}x$ is \prope{strictly isotone} and by \prefp{lem:argminmaxphi}}
      %\\&=\mathrlap{\argmin_{x\in\omsH}\max_{y\in\omsH}
      %       \setn{\begin{array}{*{11}{@{\,}c}}
      %          0 &+& 1 &+& 2 &+& 3 &+& 4 &+& 5 \\
      %          1 &+& 0 &+& 1 &+& 2 &+& 3 &+& 4 \\
      %          2 &+& 1 &+& 0 &+& 1 &+& 2 &+& 3 \\
      %          3 &+& 2 &+& 1 &+& 0 &+& 1 &+& 2 \\
      %          4 &+& 3 &+& 2 &+& 1 &+& 0 &+& 1 \\
      %          5 &+& 4 &+& 3 &+& 2 &+& 1 &+& 0 
      %       \end{array}}
      %\quad= \argmin_{x\in\omsH}
      %       \setn{\begin{array}{c}
      %         15\\
      %         11\\
      %          9\\
      %          9\\
      %         11\\
      %         15
      %       \end{array}}
      %\quad= \argmin_{x\in\omsH}
      %       \setn{\begin{array}{c}
      %           \\
      %           \\
      %          3\\
      %          4\\
      %           \\
      %         \mbox{}
      %       \end{array}}}
      \\
      \ocsVar(\rvY;\ocsE)
        &\eqd \sum_{x\in\Z}\metricsq{\ocsE(\rvY)}{x}\psp(x)
        && \text{by definition of $\ocsVar$ \xref{def:ocsVar}}
      \\&= \sum_{x\in\Z}\metricsq{\setn{3,4}}{x}\frac{1}{6}
        && \text{by $\ocsE(\rvY)$ result and definition of $\ocsG$}
      \\&= \mathrlap{\frac{1}{6}\brp{\abs{3-1}^2+\abs{3-2}^2+\abs{3-3}^2+\abs{4-4}^2+\abs{4-5}^2+\abs{4-6}^2}}
      \\&= \frac{1}{6}\brp{4+1+0+0+1+4} = \frac{10}{6} = \frac{5}{3}
  \end{align*}
%\end{enumerate}
\end{proof}


The random variable mappings in \prefpp{ex:fairdieXRYZ} have two fundamental problems:
%(see figure to right where the die structure has a discrete metric topology illustrated with an undirected graph):
\begin{enumerate}
  \item The order structure of the \structe{fair die}
and the order structure of \structe{real line}
are inherently dissimilar in that 
while the \prope{bijective} \xref{def:ftypes} mapping $\rvX$ is trivially \prope{order preserving} \xref{def:orderpre}, 
its inverse is \prope{not order preserving}.
And this is a problem.
In the \prope{linearly ordered} \xref{def:chain} range of $\rvX$, it is true that $\rvX(\diceA)=1<2=\rvX(\diceB)$.
But in the unordered domain of $\rvX$ $\setn{\diceA,\diceB,\ldots,\diceF}$, it is \emph{not} true that $\diceA<\diceB$;
rather $\diceA$ and $\diceB$ are simply symbols without order.
This causes problems when we attempt to use the random variable to make statistical inferences involving moments \xref{def:ocsmom}.
The \fncte{traditional expected value} \xref{def:pE} of a \structe{fair die} \xref{ex:fairdie} 
is $\pE(X)=\frac{1}{6}(1+2+\cdots+6)=3.5$.
This implies that we expect the outcome of $\diceC$ or $\diceD$ more than we expect the outcome of say 
$\diceA$ or $\diceB$.
But these results have no relationship with reality or with intuition because the values of a fair die are merely symbols.
For a fair die, we would expect any pair of values equally.
We would not expect the outcome [$\diceC$ or $\diceD$] more than we would expect the outcome [$\diceA$ or $\diceB$],
or more than we would expect any other outcome pair.

  \item The metric geometry \xref{rem:dietop} of the \structe{fair die outcome subspace} is very dissimilar to the 
metric geometry of the \structe{real line} \xref{def:Rline} that it is mapped to by the random variable $\rvX$.
And this is a problem.
In the metric geometry of the fair die induced by the \fncte{discrete metric} \xref{def:dmetric}, 
$\diceA$ is no closer to $\diceB$ than it is to $\diceC$
($\metric{\diceA}{\diceB}=1=\metric{\diceA}{\diceC}$).
However in the metric geometry of the real line induced by the \fncte{usual metric} $\metric{x}{y}\eqd\abs{x-y}$ \xref{def:d_usual},
$\rvX(\diceA)=1$ is closer to $\rvX(\diceB)=2$ than it is to $\rvX(\diceC)=3$
($\abs{1-2}=1\neq2=\abs{1-3}$).
\end{enumerate}

%with no topology (and hence no metric) and 
%with no order, whereas the set of real numbers 
%\emph{does} have a topology 
%(induced by the metric $\metric{x}{y}\eqd\abs{x-y}$) and 
%\emph{does} have an order structure 
%(a linear order structure induced by the standard ordering relation $\orel$ on $\R$).
%\\\begin{minipage}{\tw-53mm}%
%\end{minipage}%
%\hfill%


\begin{figure}[h][h]
  \centering%
  \gsize%
  \begin{tabular}{c@{\qquad\qquad}c}
    %{%============================================================================
% Daniel J. Greenhoe
% LaTeX file
% discrete metric real dice mapping to linearly ordered O6c
%============================================================================
{%\psset{unit=0.5\psunit}%
\begin{pspicture}(-3.2,-1.5)(3.2,1.5)%
  %---------------------------------
  % options
  %---------------------------------
  \psset{%
    radius=1.25ex,
    labelsep=2.5mm,
    linecolor=blue,%
    }%
  %---------------------------------
  % dice graph
  %---------------------------------
  \rput(-1.75,0){%\psset{unit=2\psunit}%
    %\rput{ 210}(0,0){\rput(1,0){\Cnode[fillstyle=solid,fillcolor=snode](0,0){D4}}}%
    %\rput{ 150}(0,0){\rput(1,0){\Cnode[fillstyle=solid,fillcolor=snode](0,0){D5}}}%
    %\rput{  90}(0,0){\rput(1,0){\Cnode[fillstyle=solid,fillcolor=snode](0,0){D6}}}%
    %\rput{  30}(0,0){\rput(1,0){\Cnode[fillstyle=solid,fillcolor=snode](0,0){D3}}}%
    %\rput{ -30}(0,0){\rput(1,0){\Cnode[fillstyle=solid,fillcolor=snode](0,0){D2}}}%
    %\rput{ -90}(0,0){\rput(1,0){\Cnode[fillstyle=solid,fillcolor=snode](0,0){D1}}}%
    \uput{1}[210](0,0){\Cnode[fillstyle=solid,fillcolor=snode](0,0){D4}}%
    \uput{1}[150](0,0){\Cnode[fillstyle=solid,fillcolor=snode](0,0){D5}}%
    \uput{1}[ 90](0,0){\Cnode[fillstyle=solid,fillcolor=snode](0,0){D6}}%
    \uput{1}[ 30](0,0){\Cnode[fillstyle=solid,fillcolor=snode](0,0){D3}}%
    \uput{1}[-30](0,0){\Cnode[fillstyle=solid,fillcolor=snode](0,0){D2}}%
    \uput{1}[-90](0,0){\Cnode[fillstyle=solid,fillcolor=snode](0,0){D1}}%
    \uput{1.25}[120](0,0){$\ocsG$}%
    }
  \rput(D6){$\diceF$}%
  \rput(D5){$\diceE$}%
  \rput(D4){$\diceD$}%
  \rput(D3){$\diceC$}%
  \rput(D2){$\diceB$}%
  \rput(D1){$\diceA$}%
  %
  \ncline{D5}{D6}%
  \ncline{D4}{D5}\ncline{D4}{D6}%
  \ncline{D3}{D4}\ncline{D3}{D5}\ncline{D3}{D6}%
  \ncline{D2}{D3}\ncline{D2}{D4}\ncline{D2}{D5}\ncline{D2}{D6}%
  \ncline{D1}{D2}\ncline{D1}{D3}\ncline{D1}{D4}\ncline{D1}{D5}\ncline{D1}{D6}%
  %
  \uput[158](D6){$\frac{1}{6}$}
  \uput[150](D5){$\frac{1}{6}$}
  \uput[210](D4){$\frac{1}{6}$}
  \uput[ 22](D3){$\frac{1}{6}$}
  \uput[-45](D2){$\frac{1}{6}$}
  \uput[-158](D1){$\frac{1}{6}$}
  %---------------------------------
  % range graph
  %---------------------------------
  \rput(1.75,0){%\psset{unit=2\psunit}%
    %\Cnode[fillstyle=solid,fillcolor=snode](-0.8660,-0.5){E4}%
    %\Cnode[fillstyle=solid,fillcolor=snode](-0.8660,0.5){E5}%
    %\Cnode[fillstyle=solid,fillcolor=snode](0,1){E6}%
    %\Cnode[fillstyle=solid,fillcolor=snode](0.8660,0.5){E3}%
    %\Cnode[fillstyle=solid,fillcolor=snode](0.8660,-0.5){E2}%
    %\Cnode[fillstyle=solid,fillcolor=snode](0,-1){E1}%
    \uput{1}[210](0,0){\Cnode[fillstyle=solid,fillcolor=snode](0,0){E4}}%
    \uput{1}[150](0,0){\Cnode[fillstyle=solid,fillcolor=snode](0,0){E5}}%
    \uput{1}[ 90](0,0){\Cnode[fillstyle=solid,fillcolor=snode](0,0){E6}}%
    \uput{1}[ 30](0,0){\Cnode[fillstyle=solid,fillcolor=snode](0,0){E3}}%
    \uput{1}[-30](0,0){\Cnode[fillstyle=solid,fillcolor=snode](0,0){E2}}%
    \uput{1}[-90](0,0){\Cnode[fillstyle=solid,fillcolor=snode](0,0){E1}}%
    \uput{1.25}[60](0,0){$\omsH$}%
    }
  \rput(E6){$6$}%
  \rput(E5){$5$}%
  \rput(E4){$4$}%
  \rput(E3){$3$}%
  \rput(E2){$2$}%
  \rput(E1){$1$}%
  %\rput(E0){$0$}%
  %
  \ncline{E5}{E6}%
  \ncline{E4}{E5}\ncline{E4}{E6}%
  \ncline{E3}{E4}\ncline{E3}{E5}\ncline{E3}{E6}%
  \ncline{E2}{E3}\ncline{E2}{E4}\ncline{E2}{E5}\ncline{E2}{E6}%
  \ncline{E1}{E2}\ncline{E1}{E3}\ncline{E1}{E4}\ncline{E1}{E5}\ncline{E1}{E6}%
  %\ncline{E0}{E1}\ncline{E0}{E2}\ncline{E0}{E3}\ncline{E0}{E4}\ncline{E0}{E5}\ncline{E0}{E6}%
  %
  \uput[ 22](E6){$\frac{1}{6}$}
  \uput[135](E5){$\frac{1}{6}$}
  \uput[200](E4){$\frac{1}{6}$}
  \uput[ 30](E3){$\frac{1}{6}$}
  \uput[-30](E2){$\frac{1}{6}$}
  \uput[-22](E1){$\frac{1}{6}$}
  %\uput[  0](E0){$\sfrac{0}{6}$}
  %---------------------------------
  % mapping from die to O6c
  %---------------------------------
  \ncarc[arcangle= 22,linewidth=0.75pt,linecolor=red]{->}{D6}{E6}%
  \ncarc[arcangle= 22,linewidth=0.75pt,linecolor=red]{->}{D5}{E5}%
  \ncarc[arcangle= 22,linewidth=0.75pt,linecolor=red]{->}{D4}{E4}%
  \ncarc[arcangle=-22,linewidth=0.75pt,linecolor=red]{->}{D3}{E3}%
  \ncarc[arcangle=-22,linewidth=0.75pt,linecolor=red]{->}{D2}{E2}%
  \ncarc[arcangle=-22,linewidth=0.75pt,linecolor=red]{->}{D1}{E1}%
  %---------------------------------
  % labels
  %---------------------------------
  \rput(0,0){$\rvX(\cdot)$}%
  %\ncline[linestyle=dotted,nodesep=1pt]{->}{xzlabel}{xz}%
  %\ncline[linestyle=dotted,nodesep=1pt]{->}{ylabel}{y}%
\end{pspicture}
}%}&{%============================================================================
% Daniel J. Greenhoe
% LaTeX file
% discrete metric real dice mapping to linearly ordered O6c
%============================================================================
{%\psset{unit=0.5\psunit}%
\begin{pspicture}(-3.2,-1.5)(3.2,1.5)%
  %---------------------------------
  % options
  %---------------------------------
  \psset{%
    radius=1.25ex,
    labelsep=2.5mm,
    linecolor=blue,%
    }%
  %---------------------------------
  % dice graph
  %---------------------------------
  \rput(-1.75,0){%\psset{unit=2\psunit}%
    \uput{1}[210](0,0){\Cnode[fillstyle=solid,fillcolor=snode](0,0){D4}}%
    \uput{1}[150](0,0){\Cnode[fillstyle=solid,fillcolor=snode](0,0){D5}}%
    \uput{1}[ 90](0,0){\Cnode[fillstyle=solid,fillcolor=snode](0,0){D6}}%
    \uput{1}[ 30](0,0){\Cnode[fillstyle=solid,fillcolor=snode](0,0){D3}}%
    \uput{1}[-30](0,0){\Cnode[fillstyle=solid,fillcolor=snode](0,0){D2}}%
    \uput{1}[-90](0,0){\Cnode[fillstyle=solid,fillcolor=snode](0,0){D1}}%
    \uput{1.25}[120](0,0){$\ocsG$}%
    }
  \rput(D6){$\diceF$}%
  \rput(D5){$\diceE$}%
  \rput(D4){$\diceD$}%
  \rput(D3){$\diceC$}%
  \rput(D2){$\diceB$}%
  \rput(D1){$\diceA$}%
  %
  \ncline{D5}{D6}%
  \ncline{D4}{D5}\ncline{D4}{D6}%
  \ncline{D3}{D4}\ncline{D3}{D5}\ncline{D3}{D6}%
  \ncline{D2}{D3}\ncline{D2}{D4}\ncline{D2}{D5}\ncline{D2}{D6}%
  \ncline{D1}{D2}\ncline{D1}{D3}\ncline{D1}{D4}\ncline{D1}{D5}\ncline{D1}{D6}%
  %
  \uput[158](D6){$\frac{1}{6}$}
  \uput[150](D5){$\frac{1}{6}$}
  \uput[210](D4){$\frac{1}{6}$}
  \uput[ 22](D3){$\frac{1}{6}$}
  \uput[-45](D2){$\frac{1}{6}$}
  \uput[-158](D1){$\frac{1}{6}$}
  %---------------------------------
  % range graph
  %---------------------------------
  \rput(1.75,0){%\psset{unit=2\psunit}%
    \uput{1}[210](0,0){\Cnode(0,0){E4}}%
    \uput{1}[150](0,0){\Cnode(0,0){E5}}%
    \uput{1}[ 90](0,0){\Cnode(0,0){E6}}%
    \uput{1}[ 30](0,0){\Cnode(0,0){E3}}%
    \uput{1}[-30](0,0){\Cnode(0,0){E2}}%
    \uput{1}[-90](0,0){\Cnode(0,0){E1}}%
    \Cnode[fillstyle=solid,fillcolor=snode](0,0){E0}%
    \uput{1.25}[60](0,0){$\omsH$}%
    }
  \rput(E6){$6$}%
  \rput(E5){$5$}%
  \rput(E4){$4$}%
  \rput(E3){$3$}%
  \rput(E2){$2$}%
  \rput(E1){$1$}%
  \rput(E0){$0$}%
  %
  %\ncline{E5}{E6}%
  %\ncline{E4}{E5}\ncline{E4}{E6}%
  %\ncline{E3}{E4}\ncline{E3}{E5}\ncline{E3}{E6}%
  %\ncline{E2}{E3}\ncline{E2}{E4}\ncline{E2}{E5}\ncline{E2}{E6}%
  %\ncline{E1}{E2}\ncline{E1}{E3}\ncline{E1}{E4}\ncline{E1}{E5}\ncline{E1}{E6}%
  \ncline{E0}{E1}\naput{$\frac{1}{2}$}
  \ncline{E0}{E2}\naput{$\frac{1}{2}$}
  \ncline{E0}{E3}\naput{$\frac{1}{2}$}
  \ncline{E0}{E4}\naput{$\frac{1}{2}$}
  \ncline{E0}{E5}\naput{$\frac{1}{2}$}
  \ncline{E0}{E6}\naput{$\frac{1}{2}$}%
  %
  \uput[ 22](E6){$\frac{1}{6}$}
  \uput[135](E5){$\frac{1}{6}$}
  \uput[200](E4){$\frac{1}{6}$}
  \uput[ 30](E3){$\frac{1}{6}$}
  \uput[-30](E2){$\frac{1}{6}$}
  \uput[-22](E1){$\frac{1}{6}$}
  \uput[  0](E0){$\sfrac{0}{6}$}
  %---------------------------------
  % mapping from die to O6c
  %---------------------------------
  \ncarc[arcangle= 22,linewidth=0.75pt,linecolor=red]{->}{D6}{E6}%
  \ncarc[arcangle= 22,linewidth=0.75pt,linecolor=red]{->}{D5}{E5}%
  \ncarc[arcangle= 22,linewidth=0.75pt,linecolor=red]{->}{D4}{E4}%
  \ncarc[arcangle=-11,linewidth=0.75pt,linecolor=red]{->}{D3}{E3}%
  \ncarc[arcangle=-11,linewidth=0.75pt,linecolor=red]{->}{D2}{E2}%
  \ncarc[arcangle=-22,linewidth=0.75pt,linecolor=red]{->}{D1}{E1}%
  %---------------------------------
  % labels
  %---------------------------------
  \rput(0,0){$\rvX(\cdot)$}%
  %\ncline[linestyle=dotted,nodesep=1pt]{->}{xzlabel}{xz}%
  %\ncline[linestyle=dotted,nodesep=1pt]{->}{ylabel}{y}%
\end{pspicture}
}%} \\
    {\includegraphics{sto/graphics/fairdieXO6.pdf}}&{\includegraphics{sto/graphics/fairdieXO6c.pdf}} \\
    (A) mapping to isomorphic structure  & (B) mapping to extended structure
  \end{tabular}
  \caption{order preserving random variable mappings from \structe{fair die}\label{fig:fairdieXO6}}
\end{figure}
%---------------------------------------
\begin{example}[\exmd{fair die mapping to isomorphic structure}]
%\mbox{}\\
\label{ex:fdieXO6}
%---------------------------------------
%\begin{minipage}{\tw-65mm}%
%Let $\ocsG\eqd\ocsD$ 
Let $\ocsG\eqd\ocs{\setn{\diceA,\diceB,\diceC,\diceD,\diceE,\diceF}}{\metricn}{\emptyset}{\psp}$ 
be a \structe{fair die outcome subspace} \xref{ex:fairdie},
and $\omsH\eqd\oms{\setn{1,2,3,4,5,6}}{\metricn}{\emptyset}$ 
be an \structe{unordered metric space} \xref{def:oms}.
\prefpp{ex:fairdieXRYZ} presented mappings from $\ocsG$ to structures with structures dissimilar to $\ocsG$.
%structure \prope{linearly ordered lattice}s \xref{def:lattice}.
\prefp{fig:fairdieXO6} (A) illustrates a mapping to the isomorphic structure
$\ocsH\eqd\ocs{\setn{1,2,3,4,5,6}}{\metricn}{\emptyset}{\rvX(\psp)}$,
yielding the following statistics:
\\\indentx$
  \ocsE(\rvX)  = \setn{1,2,3,4,5,6} \qquad \ocsVar(\rvX)  = 0
  %\ocsEa(\rvX) &=& \setn{1,2,3,4,5,6}\\
  $\\
%\end{minipage}\hfill%
%\begin{tabular}{c}
%  \gsize%
%  %\psset{unit=5mm}%
%  {%============================================================================
% Daniel J. Greenhoe
% LaTeX file
% discrete metric real dice mapping to linearly ordered O6c
%============================================================================
{%\psset{unit=0.5\psunit}%
\begin{pspicture}(-3.2,-1.5)(3.2,1.5)%
  %---------------------------------
  % options
  %---------------------------------
  \psset{%
    radius=1.25ex,
    labelsep=2.5mm,
    linecolor=blue,%
    }%
  %---------------------------------
  % dice graph
  %---------------------------------
  \rput(-1.75,0){%\psset{unit=2\psunit}%
    %\rput{ 210}(0,0){\rput(1,0){\Cnode[fillstyle=solid,fillcolor=snode](0,0){D4}}}%
    %\rput{ 150}(0,0){\rput(1,0){\Cnode[fillstyle=solid,fillcolor=snode](0,0){D5}}}%
    %\rput{  90}(0,0){\rput(1,0){\Cnode[fillstyle=solid,fillcolor=snode](0,0){D6}}}%
    %\rput{  30}(0,0){\rput(1,0){\Cnode[fillstyle=solid,fillcolor=snode](0,0){D3}}}%
    %\rput{ -30}(0,0){\rput(1,0){\Cnode[fillstyle=solid,fillcolor=snode](0,0){D2}}}%
    %\rput{ -90}(0,0){\rput(1,0){\Cnode[fillstyle=solid,fillcolor=snode](0,0){D1}}}%
    \uput{1}[210](0,0){\Cnode[fillstyle=solid,fillcolor=snode](0,0){D4}}%
    \uput{1}[150](0,0){\Cnode[fillstyle=solid,fillcolor=snode](0,0){D5}}%
    \uput{1}[ 90](0,0){\Cnode[fillstyle=solid,fillcolor=snode](0,0){D6}}%
    \uput{1}[ 30](0,0){\Cnode[fillstyle=solid,fillcolor=snode](0,0){D3}}%
    \uput{1}[-30](0,0){\Cnode[fillstyle=solid,fillcolor=snode](0,0){D2}}%
    \uput{1}[-90](0,0){\Cnode[fillstyle=solid,fillcolor=snode](0,0){D1}}%
    \uput{1.25}[120](0,0){$\ocsG$}%
    }
  \rput(D6){$\diceF$}%
  \rput(D5){$\diceE$}%
  \rput(D4){$\diceD$}%
  \rput(D3){$\diceC$}%
  \rput(D2){$\diceB$}%
  \rput(D1){$\diceA$}%
  %
  \ncline{D5}{D6}%
  \ncline{D4}{D5}\ncline{D4}{D6}%
  \ncline{D3}{D4}\ncline{D3}{D5}\ncline{D3}{D6}%
  \ncline{D2}{D3}\ncline{D2}{D4}\ncline{D2}{D5}\ncline{D2}{D6}%
  \ncline{D1}{D2}\ncline{D1}{D3}\ncline{D1}{D4}\ncline{D1}{D5}\ncline{D1}{D6}%
  %
  \uput[158](D6){$\frac{1}{6}$}
  \uput[150](D5){$\frac{1}{6}$}
  \uput[210](D4){$\frac{1}{6}$}
  \uput[ 22](D3){$\frac{1}{6}$}
  \uput[-45](D2){$\frac{1}{6}$}
  \uput[-158](D1){$\frac{1}{6}$}
  %---------------------------------
  % range graph
  %---------------------------------
  \rput(1.75,0){%\psset{unit=2\psunit}%
    %\Cnode[fillstyle=solid,fillcolor=snode](-0.8660,-0.5){E4}%
    %\Cnode[fillstyle=solid,fillcolor=snode](-0.8660,0.5){E5}%
    %\Cnode[fillstyle=solid,fillcolor=snode](0,1){E6}%
    %\Cnode[fillstyle=solid,fillcolor=snode](0.8660,0.5){E3}%
    %\Cnode[fillstyle=solid,fillcolor=snode](0.8660,-0.5){E2}%
    %\Cnode[fillstyle=solid,fillcolor=snode](0,-1){E1}%
    \uput{1}[210](0,0){\Cnode[fillstyle=solid,fillcolor=snode](0,0){E4}}%
    \uput{1}[150](0,0){\Cnode[fillstyle=solid,fillcolor=snode](0,0){E5}}%
    \uput{1}[ 90](0,0){\Cnode[fillstyle=solid,fillcolor=snode](0,0){E6}}%
    \uput{1}[ 30](0,0){\Cnode[fillstyle=solid,fillcolor=snode](0,0){E3}}%
    \uput{1}[-30](0,0){\Cnode[fillstyle=solid,fillcolor=snode](0,0){E2}}%
    \uput{1}[-90](0,0){\Cnode[fillstyle=solid,fillcolor=snode](0,0){E1}}%
    \uput{1.25}[60](0,0){$\omsH$}%
    }
  \rput(E6){$6$}%
  \rput(E5){$5$}%
  \rput(E4){$4$}%
  \rput(E3){$3$}%
  \rput(E2){$2$}%
  \rput(E1){$1$}%
  %\rput(E0){$0$}%
  %
  \ncline{E5}{E6}%
  \ncline{E4}{E5}\ncline{E4}{E6}%
  \ncline{E3}{E4}\ncline{E3}{E5}\ncline{E3}{E6}%
  \ncline{E2}{E3}\ncline{E2}{E4}\ncline{E2}{E5}\ncline{E2}{E6}%
  \ncline{E1}{E2}\ncline{E1}{E3}\ncline{E1}{E4}\ncline{E1}{E5}\ncline{E1}{E6}%
  %\ncline{E0}{E1}\ncline{E0}{E2}\ncline{E0}{E3}\ncline{E0}{E4}\ncline{E0}{E5}\ncline{E0}{E6}%
  %
  \uput[ 22](E6){$\frac{1}{6}$}
  \uput[135](E5){$\frac{1}{6}$}
  \uput[200](E4){$\frac{1}{6}$}
  \uput[ 30](E3){$\frac{1}{6}$}
  \uput[-30](E2){$\frac{1}{6}$}
  \uput[-22](E1){$\frac{1}{6}$}
  %\uput[  0](E0){$\sfrac{0}{6}$}
  %---------------------------------
  % mapping from die to O6c
  %---------------------------------
  \ncarc[arcangle= 22,linewidth=0.75pt,linecolor=red]{->}{D6}{E6}%
  \ncarc[arcangle= 22,linewidth=0.75pt,linecolor=red]{->}{D5}{E5}%
  \ncarc[arcangle= 22,linewidth=0.75pt,linecolor=red]{->}{D4}{E4}%
  \ncarc[arcangle=-22,linewidth=0.75pt,linecolor=red]{->}{D3}{E3}%
  \ncarc[arcangle=-22,linewidth=0.75pt,linecolor=red]{->}{D2}{E2}%
  \ncarc[arcangle=-22,linewidth=0.75pt,linecolor=red]{->}{D1}{E1}%
  %---------------------------------
  % labels
  %---------------------------------
  \rput(0,0){$\rvX(\cdot)$}%
  %\ncline[linestyle=dotted,nodesep=1pt]{->}{xzlabel}{xz}%
  %\ncline[linestyle=dotted,nodesep=1pt]{->}{ylabel}{y}%
\end{pspicture}
}%}%
%\end{tabular}
%\\
Here, $\ocsE(\rvX)$ equals the entire base set of $\omsH$, 
indicating a statistic carrying no information about an expected outcome.
That is, there is no best guess concerning outcome.
This is much different than the traditional probability of $3.5$ \xref{ex:fairdieXRYZ} which 
deceptively suggests a likely outcome of $\diceC$ or $\diceD$.
And one could easily argue that no information is much better than misleading information.
\end{example}
\begin{proof}
    \begin{align*}
      \ocsE(\rvX)
        &\eqd \argmin_{x\in\omsH}\max_{y\in\omsH}\metric{x}{y}\psp(y)
        &&\text{by definition of $\ocsE$ \xref{def:ocsE}}
      \\&= \argmin_{x\in\ocsG}\max_{y\in\ocsG}\metric{x}{y}\psp(y)
        &&\text{because $\ocsG$ and $\omsH$ are \prope{isomorphic}}
      \\&= \rvX\brs{\ocscen(\ocsG)}
        &&\text{by definition of $\ocscen$ \xref{def:ocscen}}
      \\&= \rvX\brs{\setn{\diceA,\diceB,\diceC,\diceD,\diceE,\diceF}}
        &&\text{by \prefpp{ex:fairdieXRYZ}}
      \\&= \setn{1,2,3,4,5,6}
        &&\text{by definition of $\rvX$}
      \\
      \ocsVar(\rvX)
        &\eqd \sum_{x\in\omsH}\metricsq{\ocsE(\rvX)}{x}\psp(x)
        &&\text{by definition of $\ocsVar$ \xref{def:ocsVar}}
      \\&=\sum_{x\in\ocsG}\metricsq{\rvX(\ocscen(\ocsG)}{x}\psp(x)
        &&\text{because $\ocsG$ and $\omsH$ are \prope{isomorphic}}
      \\&\eqd\ocsVar(\ocsG)
        && \text{by definition of $\ocsVar$ \xref{def:ocsVarG}}
      \\&= 0
        &&\text{by \prefpp{ex:fairdieXRYZ}}
      %\\
      %\ocsEa(\rvX)
      %  &\eqd \argmin_{x\in\omsH}\sum_{y}\ocsmom(x,y)
      %  &&\text{by definition of $\ocsEa$ \xref{def:ocsEa}}
      %\\&\eqd \argmin_{x\in\omsH}\max_{y\in\omsH}\metric{x}{y}\psp(y)
      %  &&\text{by definition of $\ocsmom$ \xref{def:ocsmom}}
      %\\&=\argmin_{x\in\omsH}\max_{y\in\omsH}\metric{x}{y}\frac{1}{6}
      %  &&\text{by definition of $\ocsG$}
      %\\&= \argmin_{x\in\omsH}\max_{y\in\omsH}\metric{x}{y}
%        && \text{because $\fphi(x)=\frac{1}{6}x$ is \prope{strictly isotone} and by \prefp{lem:argminmaxphi}}
      %\\&=\mathrlap{\argmin_{x\in\omsH}
      %       \setn{\begin{array}{*{11}{c}}
      %         {0} &+& {1} &+& {1} &+& {1} &+& {1} &+& {2}\\
      %         {1} &+& {0} &+& {1} &+& {1} &+& {2} &+& {1}\\
      %         {1} &+& {1} &+& {0} &+& {2} &+& {1} &+& {1}\\
      %         {1} &+& {1} &+& {2} &+& {0} &+& {1} &+& {1}\\
      %         {1} &+& {2} &+& {1} &+& {1} &+& {0} &+& {1}\\
      %         {2} &+& {1} &+& {1} &+& {1} &+& {1} &+& {0}
      %       \end{array}}
      %  %&&\text{by \prefpp{item:fdieXO6_gcen}}
      %\quad= \argmin_{x\in\omsH}
      %       \setn{\begin{array}{c}
      %         6\\
      %         6\\
      %         6\\
      %         6\\
      %         6\\
      %         6
      %       \end{array}}}
      %\\&= \setn{1,2,3,4,5,6}
    \end{align*}
\end{proof}


Although all the coefficients of the polynomial equation $x^2-2x+2=0$ are in the set of real numbers $\R$,
the solutions of the equation ($x=1+i$ and $x=1-i$) are not. 
Rather, the two solutions are in the \structe{complex plane} $\R^2$ \xref{ex:Cplane}, of which $\R$ is a substructure.
This is an example of extending a structure (from $\R$ to $\R^2$) to achieve more useful results.
The same idea can be applied to a random variable $\rvX\in\clOCSgh$.
The definition of an \fncte{outcome random variable} \xref{def:ocsrv} does not require a bijection
between $\ocsG$ and $\omsH$;
rather, it only requires that the mapping be ``into" the base set of $\omsH$ \xref{def:ftypes}.
In \prefpp{ex:fdieXO6} in which $\ocsG$ is isomorphic to $\ocsH$, 
the expected value of $\rvX$ is a set with six values.
However, we could extend $\omsH$, while still preserving the order and metric geometry of $\ocsG$,
to produce a random variable with a simpler expected value (next example). 
\\
%---------------------------------------
\begin{example}[\exmd{fair die mapping with extended range}]
\label{ex:fdieXO6c}
%---------------------------------------
Let $\ocsG\eqd\ocs{\setn{\diceA,\diceB,\diceC,\diceD,\diceE,\diceF}}{\metricn}{\emptyset}{\psp}$ 
be a \structe{fair die outcome subspace} \xref{ex:fairdie},
and $\omsH\eqd\oms{\setn{1,2,3,4,5,6,0}}{\metrican}{\emptyset}$ 
be an \structe{unordered metric space} \xref{def:oms}.
\prefp{fig:fairdieXO6} (B) illustrates a random variable mapping $\rvX$ from $\ocsG$ to the extended structure $\omsH$,
yielding the following statistics:
\\\indentx$\ocsE(\rvX) = \setn{0} \qquad \ocsVar(\rvX)  = \frac{1}{4}$\\
As in \pref{ex:fdieXO6}, order and metric geometry are still preserved.
Here, an expected value of $\setn{0}$ simply means that no real physical value is expected 
more or less than any other real physical value.
Note also that the variance (expected error) is more than 11 times smaller than that of 
the corresponding statistical estimates on the real line 
($\sfrac{3}{12}$ versus $\sfrac{35}{12}$ \xrefnp{ex:fairdieXRYZ}).
%\end{minipage}\hfill%
%\begin{tabular}{c}
%  \gsize%
%  %\psset{unit=5mm}%
%  {%============================================================================
% Daniel J. Greenhoe
% LaTeX file
% discrete metric real dice mapping to linearly ordered O6c
%============================================================================
{%\psset{unit=0.5\psunit}%
\begin{pspicture}(-3.2,-1.5)(3.2,1.5)%
  %---------------------------------
  % options
  %---------------------------------
  \psset{%
    radius=1.25ex,
    labelsep=2.5mm,
    linecolor=blue,%
    }%
  %---------------------------------
  % dice graph
  %---------------------------------
  \rput(-1.75,0){%\psset{unit=2\psunit}%
    \uput{1}[210](0,0){\Cnode[fillstyle=solid,fillcolor=snode](0,0){D4}}%
    \uput{1}[150](0,0){\Cnode[fillstyle=solid,fillcolor=snode](0,0){D5}}%
    \uput{1}[ 90](0,0){\Cnode[fillstyle=solid,fillcolor=snode](0,0){D6}}%
    \uput{1}[ 30](0,0){\Cnode[fillstyle=solid,fillcolor=snode](0,0){D3}}%
    \uput{1}[-30](0,0){\Cnode[fillstyle=solid,fillcolor=snode](0,0){D2}}%
    \uput{1}[-90](0,0){\Cnode[fillstyle=solid,fillcolor=snode](0,0){D1}}%
    \uput{1.25}[120](0,0){$\ocsG$}%
    }
  \rput(D6){$\diceF$}%
  \rput(D5){$\diceE$}%
  \rput(D4){$\diceD$}%
  \rput(D3){$\diceC$}%
  \rput(D2){$\diceB$}%
  \rput(D1){$\diceA$}%
  %
  \ncline{D5}{D6}%
  \ncline{D4}{D5}\ncline{D4}{D6}%
  \ncline{D3}{D4}\ncline{D3}{D5}\ncline{D3}{D6}%
  \ncline{D2}{D3}\ncline{D2}{D4}\ncline{D2}{D5}\ncline{D2}{D6}%
  \ncline{D1}{D2}\ncline{D1}{D3}\ncline{D1}{D4}\ncline{D1}{D5}\ncline{D1}{D6}%
  %
  \uput[158](D6){$\frac{1}{6}$}
  \uput[150](D5){$\frac{1}{6}$}
  \uput[210](D4){$\frac{1}{6}$}
  \uput[ 22](D3){$\frac{1}{6}$}
  \uput[-45](D2){$\frac{1}{6}$}
  \uput[-158](D1){$\frac{1}{6}$}
  %---------------------------------
  % range graph
  %---------------------------------
  \rput(1.75,0){%\psset{unit=2\psunit}%
    \uput{1}[210](0,0){\Cnode(0,0){E4}}%
    \uput{1}[150](0,0){\Cnode(0,0){E5}}%
    \uput{1}[ 90](0,0){\Cnode(0,0){E6}}%
    \uput{1}[ 30](0,0){\Cnode(0,0){E3}}%
    \uput{1}[-30](0,0){\Cnode(0,0){E2}}%
    \uput{1}[-90](0,0){\Cnode(0,0){E1}}%
    \Cnode[fillstyle=solid,fillcolor=snode](0,0){E0}%
    \uput{1.25}[60](0,0){$\omsH$}%
    }
  \rput(E6){$6$}%
  \rput(E5){$5$}%
  \rput(E4){$4$}%
  \rput(E3){$3$}%
  \rput(E2){$2$}%
  \rput(E1){$1$}%
  \rput(E0){$0$}%
  %
  %\ncline{E5}{E6}%
  %\ncline{E4}{E5}\ncline{E4}{E6}%
  %\ncline{E3}{E4}\ncline{E3}{E5}\ncline{E3}{E6}%
  %\ncline{E2}{E3}\ncline{E2}{E4}\ncline{E2}{E5}\ncline{E2}{E6}%
  %\ncline{E1}{E2}\ncline{E1}{E3}\ncline{E1}{E4}\ncline{E1}{E5}\ncline{E1}{E6}%
  \ncline{E0}{E1}\naput{$\frac{1}{2}$}
  \ncline{E0}{E2}\naput{$\frac{1}{2}$}
  \ncline{E0}{E3}\naput{$\frac{1}{2}$}
  \ncline{E0}{E4}\naput{$\frac{1}{2}$}
  \ncline{E0}{E5}\naput{$\frac{1}{2}$}
  \ncline{E0}{E6}\naput{$\frac{1}{2}$}%
  %
  \uput[ 22](E6){$\frac{1}{6}$}
  \uput[135](E5){$\frac{1}{6}$}
  \uput[200](E4){$\frac{1}{6}$}
  \uput[ 30](E3){$\frac{1}{6}$}
  \uput[-30](E2){$\frac{1}{6}$}
  \uput[-22](E1){$\frac{1}{6}$}
  \uput[  0](E0){$\sfrac{0}{6}$}
  %---------------------------------
  % mapping from die to O6c
  %---------------------------------
  \ncarc[arcangle= 22,linewidth=0.75pt,linecolor=red]{->}{D6}{E6}%
  \ncarc[arcangle= 22,linewidth=0.75pt,linecolor=red]{->}{D5}{E5}%
  \ncarc[arcangle= 22,linewidth=0.75pt,linecolor=red]{->}{D4}{E4}%
  \ncarc[arcangle=-11,linewidth=0.75pt,linecolor=red]{->}{D3}{E3}%
  \ncarc[arcangle=-11,linewidth=0.75pt,linecolor=red]{->}{D2}{E2}%
  \ncarc[arcangle=-22,linewidth=0.75pt,linecolor=red]{->}{D1}{E1}%
  %---------------------------------
  % labels
  %---------------------------------
  \rput(0,0){$\rvX(\cdot)$}%
  %\ncline[linestyle=dotted,nodesep=1pt]{->}{xzlabel}{xz}%
  %\ncline[linestyle=dotted,nodesep=1pt]{->}{ylabel}{y}%
\end{pspicture}
}%}%
%\end{tabular}
\end{example}
\begin{proof}
    \begin{align*}
      \ocsE(\rvX)
        &\eqd \argmin_{x\in\omsH}\max_{y\in\omsH}\metric{x}{y}\psp(y)
        &&\text{by definition of $\ocsE$ \xref{def:ocsE}}
      \\&= \argmin_{x\in\omsH}\max_{y\in\omsH\setd\setn{0}}\metric{x}{y}\psp(y)
        &&\text{because $\psp(0)=0$}
      \\&= \argmin_{x\in\omsH}\max_{y\in\omsH\setd\setn{0}}\metric{x}{y}\frac{1}{6}
      \\&= \argmin_{x\in\omsH}\max_{y\in\omsH\setd\setn{0}}\metric{x}{y}
        && \text{because $\ff(x)=\frac{1}{6}x$ is \prope{strictly isotone} and by \prefpp{lem:argminmaxphi}}
      %\\&= \argmin_{x\in\omsH}\max_{y\in\omsH}
      %       \setn{\begin{array}{cccccc}
      %         \metricn(1,1) &\metricn(1,2) &\metricn(1,3) &\metricn(1,4) &\metricn(1,5) &\metricn(1,6) \\
      %         \metricn(2,1) &\metricn(2,2) &\metricn(2,3) &\metricn(2,4) &\metricn(2,5) &\metricn(2,6) \\
      %         \metricn(3,1) &\metricn(3,2) &\metricn(3,3) &\metricn(3,4) &\metricn(3,5) &\metricn(3,6) \\
      %         \metricn(4,1) &\metricn(4,2) &\metricn(4,3) &\metricn(4,4) &\metricn(4,5) &\metricn(4,6) \\
      %         \metricn(5,1) &\metricn(5,2) &\metricn(5,3) &\metricn(5,4) &\metricn(5,5) &\metricn(5,6) \\
      %         \metricn(6,1) &\metricn(6,2) &\metricn(6,3) &\metricn(6,4) &\metricn(6,5) &\metricn(6,6) \\
      %         \metricn(0,1) &\metricn(0,2) &\metricn(0,3) &\metricn(0,4) &\metricn(0,5) &\metricn(0,6) 
      %       \end{array}}
      \\&=\mathrlap{\argmin_{x\in\omsH}\max_{y\in\omsH}
             \setn{\begin{array}{cccc}
               \metricn(1,1) &\metricn(1,2) &\cdots &\metricn(1,6) \\
               \metricn(2,1) &\metricn(2,2) &\cdots &\metricn(2,6) \\
               \metricn(3,1) &\metricn(3,2) &\cdots &\metricn(3,6) \\
               \metricn(4,1) &\metricn(4,2) &\cdots &\metricn(4,6) \\
               \metricn(5,1) &\metricn(5,2) &\cdots &\metricn(5,6) \\
               \metricn(6,1) &\metricn(6,2) &\cdots &\metricn(6,6) \\
               \metricn(0,1) &\metricn(0,2) &\cdots &\metricn(0,6) 
             \end{array}}
      = \argmin_{x\in\omsH}\max_{y\in\omsH}
             \setn{\begin{array}{*{6}{@{\,\,}c}}
               {0}&{1}&{1}&{1}&{1}&{2}\\
               {1}&{0}&{1}&{1}&{2}&{1}\\
               {1}&{1}&{0}&{2}&{1}&{1}\\
               {1}&{1}&{2}&{0}&{1}&{1}\\
               {1}&{2}&{1}&{1}&{0}&{1}\\
               {2}&{1}&{1}&{1}&{1}&{0}\\
               {1}&{1}&{1}&{1}&{1}&{1}
             \end{array}}
      = \argmin_{x\in\omsH}
             \setn{\begin{array}{c}
               {2}\\
               {2}\\
               {2}\\
               {2}\\
               {2}\\
               {2}\\
               {1}
             \end{array}}}
      \\&= \setn{0}
      %\\
      %\ocsEa(\rvX)
      %  &\eqd \argmin_{x\in\omsH}\sum_{y}\ocsmom(x,y)
      %  &&    \text{by definition of $\ocsEa$ \xref{def:ocsEa}}
      %\\&=\mathrlap{\argmin_{x\in\omsH}
      %       \setn{\begin{array}{*{13}{@{\,}c}}
      %         {0} &+& {1} &+& {1} &+& {1} &+& {1} &+& {2} &+& 0\\
      %         {1} &+& {0} &+& {1} &+& {1} &+& {2} &+& {1} &+& 0\\
      %         {1} &+& {1} &+& {0} &+& {2} &+& {1} &+& {1} &+& 0\\
      %         {1} &+& {1} &+& {2} &+& {0} &+& {1} &+& {1} &+& 0\\
      %         {1} &+& {2} &+& {1} &+& {1} &+& {0} &+& {1} &+& 0\\
      %         {2} &+& {1} &+& {1} &+& {1} &+& {1} &+& {0} &+& 0\\
      %         {1} &+& {1} &+& {1} &+& {1} &+& {1} &+& {1} &+& 0
      %       \end{array}}
      %= \argmin_{x\in\omsH}
      %       \setn{\begin{array}{c}
      %         6\\
      %         6\\
      %         6\\
      %         6\\
      %         6\\
      %         6\\
      %         6
      %       \end{array}}}
      %\\&= \setn{0,1,2,3,4,5,6}
  \\
  \ocsVar(\rvX)
    &\eqd \sum_{x\in\omsH}\metricsq{\ocsE(\rvX)}{x}\psp(x)
    && \text{by definition of $\ocsVar$ \xref{def:ocsVar}}
  \\&= \sum_{x\in\omsH}\metricsq{\setn{0}}{x}\psp(x)
    && \text{by $\ocsE(\rvX)$ result}
  \\&= \sum_{x\in\omsH\setd\setn{0}}\metricsq{\setn{0}}{x}\frac{1}{6}
    && \text{by definition of $\ocsG$}
  \\&= 6 \brp{\frac{1}{2}}^2 \frac{1}{6} = \frac{1}{4}
    && \text{by definition of $\omsH$}
\end{align*}
%\end{enumerate}
\end{proof}

%%---------------------------------------
%\begin{minipage}{\tw-65mm}%
%\begin{example}[\exmd{fair die mapping with extended range 2}]
%\label{ex:dieXO6c2}\mbox{}\\
%%---------------------------------------
%%\prefpp{ex:realdie} presented a mapping from a real die to a linearly ordered lattice.
%In this example, the range of the random variable $\rvX$ has been extended as compared to 
%\prefpp{ex:dieXO6}, yielding simpler statistical results (see illustration to the right).
%As in \pref{ex:dieXO6}, order and topology are still preserved.
%\\\indentx$\begin{array}{rclD}
%  \ocsE(\rvX)  &=& \setn{0} & \\%(shaded in \prefp{fig:pairdice})\\
%  \ocsEa(\rvX) &=& \setn{0}\\
%  \ocsVar(\rvX)  &=& 1
%\end{array}$
%\end{example}
%\end{minipage}\hfill%
%\begin{tabular}{c}
%  \gsize%
%  %\psset{unit=5mm}%
%  {%============================================================================
% Daniel J. Greenhoe
% LaTeX file
% discrete metric real dice mapping to linearly ordered O6c
%============================================================================
{%\psset{unit=0.5\psunit}%
\begin{pspicture}(-3.2,-1.5)(3.2,1.5)%
  %---------------------------------
  % options
  %---------------------------------
  \psset{%
    radius=1.25ex,
    labelsep=2.5mm,
    linecolor=blue,%
    }%
  %---------------------------------
  % dice graph
  %---------------------------------
  \rput(-1.75,0){%\psset{unit=2\psunit}%
    \Cnode[fillstyle=solid,fillcolor=snode](-0.8660,-0.5){D4}%
    \Cnode[fillstyle=solid,fillcolor=snode](-0.8660,0.5){D5}%
    \Cnode[fillstyle=solid,fillcolor=snode](0,1){D6}%
    \Cnode[fillstyle=solid,fillcolor=snode](0.8660,0.5){D3}%
    \Cnode[fillstyle=solid,fillcolor=snode](0.8660,-0.5){D2}%
    \Cnode[fillstyle=solid,fillcolor=snode](0,-1){D1}%
    }
  \rput(D6){$\diceF$}%
  \rput(D5){$\diceE$}%
  \rput(D4){$\diceD$}%
  \rput(D3){$\diceC$}%
  \rput(D2){$\diceB$}%
  \rput(D1){$\diceA$}%
  %
  \ncline{D5}{D6}%
  \ncline{D4}{D5}\ncline{D4}{D6}%
  \ncline{D3}{D5}\ncline{D3}{D6}%
  \ncline{D2}{D3}\ncline{D2}{D4}\ncline{D2}{D6}%
  \ncline{D1}{D2}\ncline{D1}{D3}\ncline{D1}{D4}\ncline{D1}{D5}%
  %
  \uput[158](D6){$\frac{1}{6}$}
  \uput[150](D5){$\frac{1}{6}$}
  \uput[210](D4){$\frac{1}{6}$}
  \uput[ 22](D3){$\frac{1}{6}$}
  \uput[-45](D2){$\frac{1}{6}$}
  \uput[-158](D1){$\frac{1}{6}$}
  %---------------------------------
  % range graph
  %---------------------------------
  \rput(1.75,0){%\psset{unit=2\psunit}%
    \Cnode(-0.8660,-0.5){E4}%
    \Cnode(-0.8660,0.5){E5}%
    \Cnode(0,1){E6}%
    \Cnode(0.8660,0.5){E3}%
    \Cnode(0.8660,-0.5){E2}%
    \Cnode(0,-1){E1}%
    \Cnode[fillstyle=solid,fillcolor=snode](0,0){E0}%
    }
  \rput(E6){$6$}%
  \rput(E5){$5$}%
  \rput(E4){$4$}%
  \rput(E3){$3$}%
  \rput(E2){$2$}%
  \rput(E1){$1$}%
  \rput(E0){$0$}%
  %
  \ncline{E5}{E6}%
  \ncline{E4}{E5}\ncline{E4}{E6}%
  \ncline{E3}{E4}\ncline{E3}{E5}\ncline{E3}{E6}%
  \ncline{E2}{E3}\ncline{E2}{E4}\ncline{E2}{E5}\ncline{E2}{E6}%
  \ncline{E1}{E2}\ncline{E1}{E3}\ncline{E1}{E4}\ncline{E1}{E5}\ncline{E1}{E6}%
  \ncline{E0}{E1}\ncline{E0}{E2}\ncline{E0}{E3}\ncline{E0}{E4}\ncline{E0}{E5}\ncline{E0}{E6}%
  %
  \uput[ 22](E6){$\frac{1}{6}$}
  \uput[135](E5){$\frac{1}{6}$}
  \uput[200](E4){$\frac{1}{6}$}
  \uput[ 30](E3){$\frac{1}{6}$}
  \uput[-30](E2){$\frac{1}{6}$}
  \uput[-22](E1){$\frac{1}{6}$}
  \uput[  0](E0){$\sfrac{0}{6}$}
  %---------------------------------
  % mapping from die to O6c
  %---------------------------------
  \ncarc[arcangle= 22,linewidth=0.75pt,linecolor=red]{->}{D6}{E6}%
  \ncarc[arcangle= 22,linewidth=0.75pt,linecolor=red]{->}{D5}{E5}%
  \ncarc[arcangle= 22,linewidth=0.75pt,linecolor=red]{->}{D4}{E4}%
  \ncarc[arcangle=-22,linewidth=0.75pt,linecolor=red]{->}{D3}{E3}%
  \ncarc[arcangle=-22,linewidth=0.75pt,linecolor=red]{->}{D2}{E2}%
  \ncarc[arcangle=-22,linewidth=0.75pt,linecolor=red]{->}{D1}{E1}%
  %---------------------------------
  % labels
  %---------------------------------
  \rput(0,0){$\rvX(\cdot)$}%
  %\ncline[linestyle=dotted,nodesep=1pt]{->}{xzlabel}{xz}%
  %\ncline[linestyle=dotted,nodesep=1pt]{->}{ylabel}{y}%
\end{pspicture}
}%}%
%\end{tabular}


%=======================================
\subsubsection{Real die examples}
%=======================================
\begin{figure}[h]
  \centering%
  \gsize%
  %{%============================================================================
% Daniel J. Greenhoe
% LaTeX file
% discrete metric real die mapping to 4 outcome spaces
%============================================================================
{%\psset{unit=0.5\psunit}%
\begin{pspicture}(-6,-3.6)(6,3.8)%
  %---------------------------------
  % options
  %---------------------------------
  \psset{%
    radius=1.25ex,
    labelsep=2.5mm,
    linecolor=blue,%
    fillstyle=none,
    }%
  %---------------------------------
  % die graph
  %---------------------------------
  \rput(0,0){%\psset{unit=2\psunit}%
    \uput{1}[210](0,0){\Cnode[fillstyle=solid,fillcolor=snode](0,0){D4}}%
    \uput{1}[150](0,0){\Cnode[fillstyle=solid,fillcolor=snode](0,0){D5}}%
    \uput{1}[ 90](0,0){\Cnode[fillstyle=solid,fillcolor=snode](0,0){D6}}%
    \uput{1}[ 30](0,0){\Cnode[fillstyle=solid,fillcolor=snode](0,0){D3}}%
    \uput{1}[-30](0,0){\Cnode[fillstyle=solid,fillcolor=snode](0,0){D2}}%
    \uput{1}[-90](0,0){\Cnode[fillstyle=solid,fillcolor=snode](0,0){D1}}%
    \rput(0,0){$\ocsG$}%
    }
  \rput(D6){$\diceF$}%
  \rput(D5){$\diceE$}%
  \rput(D4){$\diceD$}%
  \rput(D3){$\diceC$}%
  \rput(D2){$\diceB$}%
  \rput(D1){$\diceA$}%
  %
  \ncline{D5}{D6}%
  \ncline{D4}{D5}\ncline{D4}{D6}%
  \ncline{D3}{D5}\ncline{D3}{D6}%
  \ncline{D2}{D3}\ncline{D2}{D4}\ncline{D2}{D6}%
  \ncline{D1}{D2}\ncline{D1}{D3}\ncline{D1}{D4}\ncline{D1}{D5}%
  %
  \uput[ 210](D4){$\frac{1}{6}$}
  \uput[ 150](D5){$\frac{1}{6}$}
  \uput[  90](D6){$\frac{1}{6}$}
  \uput[  30](D3){$\frac{1}{6}$}
  \uput[ -30](D2){$\frac{1}{6}$}
  \uput[ -90](D1){$\frac{1}{6}$}
  %---------------------------------
  % isomorphic mapping Y
  %---------------------------------
  \rput(-4.5,0){%\psset{unit=2\psunit}%
    \uput{1}[210](0,0){\Cnode[fillstyle=solid,fillcolor=snode](0,0){Y4}}%
    \uput{1}[150](0,0){\Cnode[fillstyle=solid,fillcolor=snode](0,0){Y5}}%
    \uput{1}[ 90](0,0){\Cnode[fillstyle=solid,fillcolor=snode](0,0){Y6}}%
    \uput{1}[ 30](0,0){\Cnode[fillstyle=solid,fillcolor=snode](0,0){Y3}}%
    \uput{1}[-30](0,0){\Cnode[fillstyle=solid,fillcolor=snode](0,0){Y2}}%
    \uput{1}[-90](0,0){\Cnode[fillstyle=solid,fillcolor=snode](0,0){Y1}}%
    \rput(0,0){$\omsH$}%
    }
  \rput(Y6){$6$}%
  \rput(Y5){$5$}%
  \rput(Y4){$4$}%
  \rput(Y3){$3$}%
  \rput(Y2){$2$}%
  \rput(Y1){$1$}%
  %
  \ncline{Y5}{Y6}%
  \ncline{Y4}{Y5}\ncline{Y4}{Y6}%
  \ncline{Y3}{Y5}\ncline{Y3}{Y6}%
  \ncline{Y2}{Y3}\ncline{Y2}{Y4}\ncline{Y2}{Y6}%
  \ncline{Y1}{Y2}\ncline{Y1}{Y3}\ncline{Y1}{Y4}\ncline{Y1}{Y5}%
  %
  \uput[210](Y4){$\frac{1}{6}$}%
  \uput[150](Y5){$\frac{1}{6}$}%
  \uput[ 90](Y6){$\frac{1}{6}$}%
  \uput[ 30](Y3){$\frac{1}{6}$}%
  \uput[-30](Y2){$\frac{1}{6}$}%
  \uput[-90](Y1){$\frac{1}{6}$}%
  %
  \ncarc[arcangle=-22,linewidth=0.75pt,linecolor=blue]{->}{D6}{Y6}%
  \ncarc[arcangle=-67,linewidth=0.75pt,linecolor=blue]{->}{D5}{Y5}%
  \ncarc[arcangle= 67,linewidth=0.75pt,linecolor=blue]{->}{D4}{Y4}%
  \ncarc[arcangle=-67,linewidth=0.75pt,linecolor=blue]{->}{D3}{Y3}%
  \ncarc[arcangle= 67,linewidth=0.75pt,linecolor=blue]{->}{D2}{Y2}%
  \ncarc[arcangle= 22,linewidth=0.75pt,linecolor=blue]{->}{D1}{Y1}%
  %---------------------------------
  % random variable mapping Z to extended outcome space mapping Z
  %---------------------------------
  \rput(4.5,0){%\psset{unit=2\psunit}%
    \uput{1}[210](0,0){\Cnode(0,0){Z4}}%
    \uput{1}[150](0,0){\Cnode(0,0){Z5}}%
    \uput{1}[ 90](0,0){\Cnode(0,0){Z6}}%
    \uput{1}[ 30](0,0){\Cnode(0,0){Z3}}%
    \uput{1}[-30](0,0){\Cnode(0,0){Z2}}%
    \uput{1}[-90](0,0){\Cnode(0,0){Z1}}%
    \Cnode[fillstyle=solid,fillcolor=snode](0,0){Z0}%
    \uput{1.25}[60](0,0){$\omsK$}%
    }%
  \rput(Z6){$6$}%
  \rput(Z5){$5$}%
  \rput(Z4){$4$}%
  \rput(Z3){$3$}%
  \rput(Z2){$2$}%
  \rput(Z1){$1$}%
  \rput(Z0){$0$}%
  %
  \ncline{Z5}{Z6}%
  \ncline{Z4}{Z5}\ncline{Z4}{Z6}%
  \ncline{Z3}{Z5}\ncline{Z3}{Z6}%
  \ncline{Z2}{Z3}\ncline{Z2}{Z4}\ncline{Z2}{Z6}%
  \ncline{Z1}{Z2}\ncline{Z1}{Z3}\ncline{Z1}{Z4}\ncline{Z1}{Z5}%
  \ncline{Z0}{Z1}\ncline{Z0}{Z2}\ncline{Z0}{Z3}\ncline{Z0}{Z4}\ncline{Z0}{Z5}\ncline{Z0}{Z6}%
  %
  \uput[210](Z4){$\frac{1}{6}$}
  \uput[150](Z5){$\frac{1}{6}$}
  \uput[ 90](Z6){$\frac{1}{6}$}
  \uput[ 30](Z3){$\frac{1}{6}$}
  \uput[-30](Z2){$\frac{1}{6}$}
  \uput[-90](Z1){$\frac{1}{6}$}
  \uput[  0](Z0){$\sfrac{0}{6}$}
  %
  \ncarc[arcangle= 22,linewidth=0.75pt,linecolor=green]{->}{D6}{Z6}%
  \ncarc[arcangle= 67,linewidth=0.75pt,linecolor=green]{->}{D5}{Z5}%
  \ncarc[arcangle=-67,linewidth=0.75pt,linecolor=green]{->}{D4}{Z4}%
  \ncarc[arcangle= 67,linewidth=0.75pt,linecolor=green]{->}{D3}{Z3}%
  \ncarc[arcangle=-67,linewidth=0.75pt,linecolor=green]{->}{D2}{Z2}%
  \ncarc[arcangle=-22,linewidth=0.75pt,linecolor=green]{->}{D1}{Z1}%
  %---------------------------------
  % random variable mapping W from G to real line 
  %---------------------------------
  \rput(0,-3){%\psset{unit=0.75\psunit}%
    \multirput(-2.5,0)(1,0){6}{\psline(0,-0.1)(0,0.1)}%
    \pnode(3.5,0){WB}%
    \pnode(2.5,0){W6}%
    \pnode(1.5,0){W5}%
    \pnode(0.5,0){W4}%
    \pnode(0,0){W34}%
    \pnode(-0.5,0){W3}%
    \pnode(-1.5,0){W2}%
    \pnode(-2.5,0){W1}%
    \pnode(-3.5,0){WA}%
    }
  \uput[-90](W6){$6$}%
  \uput[-90](W5){$5$}%
  \uput[-90](W4){$4$}%
  \uput[-90](W3){$3$}%
  \uput[-90](W2){$2$}%
  \uput[-90](W1){$1$}%
  %
  \ncline{<->}{WA}{WB}%
  \pscircle[fillstyle=solid,linecolor=snode,fillcolor=snode](W34){1ex}%
  \pscircle[fillstyle=none,linecolor=red,fillcolor=red](W34){1ex}%
  %\rput(W233){\pscircle[fillstyle=none,linecolor=red,fillcolor=red](0,0){1ex}}%
  %
  \uput[ 0](WB){$\omsR$}
  \uput[90](W6){$\frac{1}{6}$}
  \uput[90](W5){$\frac{1}{6}$}
  \uput[90](W4){$\frac{1}{6}$}
  \uput[90](W3){$\frac{1}{6}$}
  \uput[90](W2){$\frac{1}{6}$}
  \uput[90](W1){$\frac{1}{6}$}
  %
  \ncarc[arcangle= 67,linewidth=0.75pt,linecolor=purple]{->}{D6}{W6}%
  \ncarc[arcangle=-67,linewidth=0.75pt,linecolor=purple]{->}{D5}{W5}%
  \ncarc[arcangle= 22,linewidth=0.75pt,linecolor=purple]{->}{D4}{W4}%
  \ncarc[arcangle= 45,linewidth=0.75pt,linecolor=purple]{->}{D3}{W3}%
  \ncarc[arcangle= 22,linewidth=0.75pt,linecolor=purple]{->}{D2}{W2}%
  \ncarc[arcangle=-10,linewidth=0.75pt,linecolor=purple]{->}{D1}{W1}%
  %
  %---------------------------------
  % random variable mapping X from G to integer line 
  %---------------------------------
  \rput(0,3){%\psset{unit=0.75\psunit}%
    \pnode(3.5,0){XB}%
    \Cnode(2.5,0){X6}%
    \Cnode(1.5,0){X5}%
    \Cnode[fillstyle=solid,fillcolor=snode](0.5,0){X4}%
    \Cnode[fillstyle=solid,fillcolor=snode](-0.5,0){X3}%
    \Cnode(-1.5,0){X2}%
    \Cnode(-2.5,0){X1}%
    \pnode(-3.5,0){XA}%
    }
  \rput(X6){$6$}%
  \rput(X5){$5$}%
  \rput(X4){$4$}%
  \rput(X3){$3$}%
  \rput(X2){$2$}%
  \rput(X1){$1$}%
  %
  \ncline[linestyle=dotted]{X6}{XB}%
  \ncline{X5}{X6}%
  \ncline{X4}{X5}%
  \ncline{X3}{X4}%
  \ncline{X2}{X3}%
  \ncline{X1}{X2}%
  \ncline[linestyle=dotted]{X1}{XA}%
  %
  \uput[ 0](XB){$\omsZ$}
  \uput[90](X6){$\frac{1}{6}$}
  \uput[90](X5){$\frac{1}{6}$}
  \uput[90](X4){$\frac{1}{6}$}
  \uput[90](X3){$\frac{1}{6}$}
  \uput[90](X2){$\frac{1}{6}$}
  \uput[90](X1){$\frac{1}{6}$}
  %
  \ncarc[arcangle= 10,linewidth=0.75pt,linecolor=red]{->}{D6}{X6}%
  \ncarc[arcangle= 22,linewidth=0.75pt,linecolor=red]{->}{D5}{X5}%
  \ncarc[arcangle= 67,linewidth=0.75pt,linecolor=red]{->}{D4}{X4}%
  \ncarc[arcangle=-45,linewidth=0.75pt,linecolor=red]{->}{D3}{X3}%
  \ncarc[arcangle=-67,linewidth=0.75pt,linecolor=red]{->}{D2}{X2}%
  \ncarc[arcangle= 67,linewidth=0.75pt,linecolor=red]{->}{D1}{X1}%
  %
  %---------------------------------
  % labels
  %---------------------------------
  \rput(2.3,0){$\rvZ(\cdot)$}%
  \rput(-2.3,0){$\rvY(\cdot)$}%
  \rput(0,2){$\rvX(\cdot)$}%
  \rput(0,-2.2){$\rvW(\cdot)$}%
\end{pspicture}
}%}%
  {\includegraphics{sto/graphics/rdie_wxyz.pdf}}%
  \caption{random variable mappings from the \structe{real die outcome subspace} to several 
  \structe{ordered metric space}s \xref{ex:realdieXRYZ} \label{fig:realdieXRYZ}}
\end{figure}
%\begin{figure}[h]
%  \centering%
%  \gsize%
%  {%============================================================================
% Daniel J. Greenhoe
% LaTeX file
% real dice mappings to real line and integer line
%============================================================================
\begin{pspicture}(-3,-1.5)(3.2,1.5)%
  %---------------------------------
  % options
  %---------------------------------
  \psset{%
    %radius=1.25ex,
    labelsep=2.5mm,
    linecolor=blue,%
    }%
  %---------------------------------
  % dice graph
  %---------------------------------
  \rput(0,0){%\psset{unit=2\psunit}%
    \Cnode[radius=1.25ex,fillstyle=solid,fillcolor=snode](-0.8660,-0.5){D4}%
    \Cnode[radius=1.25ex,fillstyle=solid,fillcolor=snode](-0.8660,0.5){D5}%
    \Cnode[radius=1.25ex,fillstyle=solid,fillcolor=snode](0,1){D6}%
    \Cnode[radius=1.25ex,fillstyle=solid,fillcolor=snode](0.8660,0.5){D3}%
    \Cnode[radius=1.25ex,fillstyle=solid,fillcolor=snode](0.8660,-0.5){D2}%
    \Cnode[radius=1.25ex,fillstyle=solid,fillcolor=snode](0,-1){D1}%
    }
  \rput(D6){$\diceF$}%
  \rput(D5){$\diceE$}%
  \rput(D4){$\diceD$}%
  \rput(D3){$\diceC$}%
  \rput(D2){$\diceB$}%
  \rput(D1){$\diceA$}%
  %
  \ncline{D5}{D6}%
  \ncline{D4}{D5}\ncline{D4}{D6}%
  \ncline{D3}{D5}\ncline{D3}{D6}%
  \ncline{D2}{D3}\ncline{D2}{D4}\ncline{D2}{D6}%
  \ncline{D1}{D2}\ncline{D1}{D3}\ncline{D1}{D4}\ncline{D1}{D5}%
  %
  \uput[158](D6){$\frac{1}{6}$}
  \uput[150](D5){$\frac{1}{6}$}
  \uput[210](D4){$\frac{1}{6}$}
  \uput[ 22](D3){$\frac{1}{6}$}
  \uput[-45](D2){$\frac{1}{6}$}
  \uput[-158](D1){$\frac{1}{6}$}
  %---------------------------------
  % Real line
  %---------------------------------
  \rput(-2.75,0){\psset{unit=0.35\psunit}%
    \psline{<->}(0,-3.5)(0,3.5)%
    \pnode(0,2.5){R6}%
    \pnode(0,1.5){R5}%
    \pnode(0,0.5){R4}%
    \pnode(0,0){R34}%
    \pnode(0,-0.5){R3}%
    \pnode(0,-1.5){R2}%
    \pnode(0,-2.5){R1}%
    \Cnode[fillstyle=solid,linecolor=snode,fillcolor=snode,radius=0.5ex](0,0){RC}%
    }%
  %\ncline{R5}{R6}%
  %\ncline{R4}{R5}%
  %\ncline{R3}{R4}%
  %\ncline{R2}{R3}%
  %\ncline{R1}{R2}%
  \rput(R6){\psline[linewidth=1pt](-0.1,0)(0.1,0)}%
  \rput(R5){\psline[linewidth=1pt](-0.1,0)(0.1,0)}%
  \rput(R4){\psline[linewidth=1pt](-0.1,0)(0.1,0)}%
  \rput(R3){\psline[linewidth=1pt](-0.1,0)(0.1,0)}%
  \rput(R2){\psline[linewidth=1pt](-0.1,0)(0.1,0)}%
  \rput(R1){\psline[linewidth=1pt](-0.1,0)(0.1,0)}%
  %
  \uput[180](R6){$6$}%
  \uput[180](R5){$5$}%
  \uput[180](R4){$4$}%
  \uput[180](R3){$3$}%
  \uput[180](R2){$2$}%
  \uput[180](R1){$1$}%
  \uput[0](R34){$3.5$}%
  %
  \ncarc[arcangle=-22,linewidth=0.75pt,linecolor=red]{->}{D6}{R6}%
  \ncarc[arcangle=-22,linewidth=0.75pt,linecolor=red]{->}{D5}{R5}%
  \ncarc[arcangle=-22,linewidth=0.75pt,linecolor=red]{->}{D4}{R4}%
  \ncarc[arcangle= 22,linewidth=0.75pt,linecolor=red]{->}{D3}{R3}%
  \ncarc[arcangle= 22,linewidth=0.75pt,linecolor=red]{->}{D2}{R2}%
  \ncarc[arcangle= 22,linewidth=0.75pt,linecolor=red]{->}{D1}{R1}%
  %
  \rput(-1.8,1){$\rvX(\cdot)$}%
  %---------------------------------
  % integer line
  %---------------------------------
  \rput(2.75,0){\psset{unit=0.35\psunit}%
    \pnode(0,3.5){I7}%
    \Cnode(0,2.5){I6}%
    \Cnode(0,1.5){I5}%
    \Cnode[fillstyle=solid,fillcolor=snode](0,0.5){I4}%
    \Cnode[fillstyle=solid,fillcolor=snode](0,-0.5){I3}%
    \Cnode(0,-1.5){I2}%
    \Cnode(0,-2.5){I1}%
    \pnode(0,-3.5){I0}%
    }%
  \ncline[linestyle=dotted]{I6}{I7}%
  \ncline{I5}{I6}%
  \ncline{I4}{I5}%
  \ncline{I3}{I4}%
  \ncline{I2}{I3}%
  \ncline{I1}{I2}%
  \ncline[linestyle=dotted]{I1}{I0}%
  %
  \uput[0](I6){$6$}%
  \uput[0](I5){$5$}%
  \uput[0](I4){$4$}%
  \uput[0](I3){$3$}%
  \uput[0](I2){$2$}%
  \uput[0](I1){$1$}%
  %
  \ncarc[arcangle= 22,linewidth=0.75pt,linecolor=green]{->}{D6}{I6}%
  \ncarc[arcangle= 22,linewidth=0.75pt,linecolor=green]{->}{D5}{I5}%
  \ncarc[arcangle= 22,linewidth=0.75pt,linecolor=green]{->}{D4}{I4}%
  \ncarc[arcangle=-22,linewidth=0.75pt,linecolor=green]{->}{D3}{I3}%
  \ncarc[arcangle=-22,linewidth=0.75pt,linecolor=green]{->}{D2}{I2}%
  \ncarc[arcangle=-22,linewidth=0.75pt,linecolor=green]{->}{D1}{I1}%
  %
  \rput(1.8,1){$\rvY(\cdot)$}%
\end{pspicture}%}%
%  \caption{random variable mappings from the real die to the real line and integer line \label{fig:realdieXRYZ}}
%\end{figure}
%---------------------------------------
%\begin{minipage}{\tw-50mm}%
\begin{example}[\exmd{real die mappings}]
\label{ex:realdieXRYZ} %\mbox{}\\
%---------------------------------------
Let $\ocsG$ be the \structe{real die outcome subspace} \xref{ex:realdie}.
Let $\rvW$, $\rvX$, $\rvY$ and $\rvZ$ be \fncte{random variable} \xref{def:ocsrv} mappings 
as illustrated in \prefpp{fig:realdieXRYZ}.
%Let $\omsH$ be the \structe{ordered metric space} \xref{def:oms} induced by the \structe{real line} \xref{def:Rline}.
%Let $\omsK$ be the \structe{ordered metric space} induced by the \structe{integer line} \xref{def:Zline}.
%Let $\rvX$ be a random variable in $\clOCSgh$ (a mapping from $\ocsG$ to $\omsH$).
%Let $\rvY$ be a random variable in $\clOCSgk$.
%These structures are illustrated in \prefpp{fig:realdieXRYZ}.
Let $\pE$, $\pVar$, $\ocsE$, and $\ocsVar$ be defined as in \prefpp{ex:fairdieXRYZ}.
This yields the following statistics:
\\$\begin{array}{>{\gsizes}Mlcl@{\quad}lcccl}
  geometry of $\ocsG$:                                  & \ocscen (\ocsG) &=& \mc{4}{l}{\setn{\diceA,\diceB,\diceC,\diceD,\diceE,\diceF}}              \\
  traditional statistics on real line:                  & \pE   (\rvW) &=& 3.5                & \ocsVar(\rvW;\pE)   &=& \frac{35}{12} &\approx& 2.917  \\
  outcome subspace statistics on real line:             & \ocsE (\rvW) &=& \setn{3.5}         & \ocsVar(\rvW;\ocsE) &=& \frac{35}{12} &\approx& 2.917  \\
  outcome subspace statistics on integer line:          & \ocsE (\rvX) &=& \setn{3,\,4}       & \ocsVar(\rvX;\ocsE) &=& \frac{20}{12} &\approx& 1.667  \\
  outcome subspace statistics on isomorphic structure:  & \ocsE (\rvY) &=& \setn{1,2,\ldots,6} & \ocsVar(\rvY;\ocsE) &=& 0             &       &        \\
  outcome subspace statistics on extended   structure:  & \ocsE (\rvZ) &=& \setn{0}           & \ocsVar(\rvZ;\ocsE) &=& 1             &       &        
\end{array}$
\\
Similar to \prefpp{ex:fdieXO6}, the statistic $\ocsE(\rvZ)=\setn{0}$ 
indicates a statistic carrying no information about an expected outcome.
Again, one could easily argue that no information is much better than misleading information.
\end{example}
\begin{proof}
%\begin{enumerate}
%  \item geometry of \structe{real die} \structe{outcome subspace} $\ocsG$ \xref{def:ocs}:\label{item:rdieXO6_gcen}
    \begin{align*}
      \ocscen(\ocsG) 
        &=\setn{\diceA,\diceB,\diceC,\diceD,\diceE,\diceF}
        &&\text{by \prefpp{ex:realdie}}
      \\
      \pE(\rvW) 
      %  &\eqd\sum_{x\in\omsH} x\psp(x)
      %  && \text{by definition of $\pE$ \xref{def:pE}}
        &= \frac{7}{2} = 3.5
        && \text{by $\pE(\rvX)$ result of \prefp{ex:fairdieXRYZ}}
      \\
      \ocsVar(\rvW;\pE)
      %  &= \pVar(\rvW)
      %  && \text{by \prefp{thm:ocsVar}}
      %\\&\eqd \sum_{x\in\omsH} \brs{x-\pE(\rvX)}^2 \psp(x)
      %  &&\text{by definition of $\pVar$ \xref{def:pVar}}
      %\\&= \sum_{x\in\omsH} \brp{x-\frac{7}{2}}^2 \frac{1}{6}
      %  &&\text{by $\pE(\rvX)$ result}
      %\\&= \mathrlap{\brs{\brp{1-\frac{7}{2}}^2 + \brp{2-\frac{7}{2}}^2 + \brp{3-\frac{7}{2}}^2 + \brp{4-\frac{7}{2}}^2 + \brp{5-\frac{7}{2}}^2 + \brp{6-\frac{7}{2}}^2 }\frac{1}{6}}
      %\\&= \mathrlap{\brs{(2-7)^2 + (4-7)^2 + (6-7)^2 + (8-7)^2 + (10-7)^2 + (12-7)^2}\frac{1}{2^2\times6}}
      %\\&= \mathrlap{\frac{25+9+1+1+9+25}{24}
      %   = \frac{70}{24}
      %   = \frac{35}{12}
      %  \approx 2.917}
        &= \frac{35}{12}\approx 2.917
        && \text{by $\pVar(\rvX)$ result of \prefp{ex:fairdieXRYZ}}
      \\
      \ocsE(\rvW)
      %  &\eqd \argmin_{x\in\omsH}\max_{y\in\omsH}\metric{x}{y}\psp(y)
      %  &&\text{by definition of $\ocsE$ \xref{def:ocsE}}
        &= \setn{3.5}
        && \text{by $\ocsE(\rvX)$ result of \prefp{ex:fairdieXRYZ}}
      \\
      \ocsVar(\rvW;\ocsE)
      %  &\eqd \sum_{x\in\omsH}\metricsq{\ocsE(\rvX)}{x}\psp(x)
      %  && \text{by definition of $\ocsVar$ \xref{def:ocsVar}}
        &= \frac{35}{12} \approx 2.917
        && \text{by $\ocsVar(\rvX;\ocsE)$ result of \prefp{ex:fairdieXRYZ}}
      \\
      \ocsE(\rvX)
      %  &\eqd \argmin_{x\in\omsH}\max_{y\in\omsH}\metric{x}{y}\psp(y)
      %  &&\text{by definition of $\ocsE$ \xref{def:ocsE}}
        &= \setn{3,\,4}
        && \text{by $\ocsE(\rvY)$ result of \prefp{ex:fairdieXRYZ}}
      \\
      \ocsVar(\rvX;\ocsE)
      %  &\eqd \sum_{x\in\omsH}\metricsq{\ocsE(\rvX)}{x}\psp(x)
      %  && \text{by definition of $\ocsVar$ \xref{def:ocsVar}}
        &= \frac{5}{3} \approx 1.667
        && \text{by $\ocsVar(\rvY;\ocsE)$ result of \prefp{ex:fairdieXRYZ}}
      \\
      \ocsE(\rvY)
        &= \rvY\brp{\setn{\diceA,\diceB,\diceC,\diceD,\diceE,\diceF}}
        && \text{by $\ocscen(\ocsG)$ result of \prefp{ex:realdie}}
      \\&= \setn{1,2,3,4,5,6}
        && \text{by definition of $\rvY$}
      \\
      \ocsVar(\rvY;\ocsE)
        &\eqd \sum_{x\in\omsH}\metricsq{\ocsE(\rvY)}{x}\psp(x)
        && \text{by definition of $\ocsVar$ \xref{def:ocsVar}}
      \\&= \sum_{x\in\omsH}\metricsq{\omsH}{x}\psp(x)
        && \text{by $\ocsE(\rvY)$ result}
      \\&= \sum_{x\in\omsH}0^2{x}\psp(x)
        && \text{by \prope{nondegenerate} property of \fncte{quasi-metric}s \xref{def:qmetric}}
      \\&= 0
    \\
    \ocsE(\rvZ)
      &\eqd \argmin_{x\in\omsK}\max_{y\in\omsK}\metric{x}{y}\psp(y)
      &&\text{by definition of $\ocsE$ \xref{def:ocsE}}
    \\&= \argmin_{x\in\omsK}\max_{y\in\omsK\setd\setn{0}}\metric{x}{y}\psp(y)
      &&\text{because $\psp(0)=0$}
    \\&= \argmin_{x\in\omsK}\max_{y\in\omsK\setd\setn{0}}\metric{x}{y}\frac{1}{6}
      && \text{by definition of $\ocsG$ and $\rvZ$}
    \\&= \argmin_{x\in\omsK}\max_{y\in\omsK\setd\setn{0}}\metric{x}{y}
        && \text{because $\ff(x)=\frac{1}{6}x$ is \prope{strictly isotone} and by \prefpp{lem:argminmaxphi}}
    \\&=\mathrlap{\argmin_{x\in\omsK}\max_{y\in\omsK}
           \setn{\begin{array}{*{4}{@{\,}c}}
             \metricn(1,1) &\metricn(1,2) &\cdots&\metricn(1,6) \\
             \metricn(2,1) &\metricn(2,2) &\cdots&\metricn(2,6) \\
             \metricn(3,1) &\metricn(3,2) &\cdots&\metricn(3,6) \\
             \metricn(4,1) &\metricn(4,2) &\cdots&\metricn(4,6) \\
             \metricn(5,1) &\metricn(5,2) &\cdots&\metricn(5,6) \\
             \metricn(6,1) &\metricn(6,2) &\cdots&\metricn(6,6) \\
             \metricn(0,1) &\metricn(0,2) &\cdots&\metricn(0,6) 
           \end{array}}
    = \argmin_{x\in\omsK}\max_{y\in\omsK}
           \setn{\begin{array}{*{6}{@{\,\,}c}}
              0 & 1 & 1 & 1 & 1 & 2\\
              1 & 0 & 1 & 1 & 2 & 1\\
              1 & 1 & 0 & 2 & 1 & 1\\
              1 & 1 & 2 & 0 & 1 & 1\\
              1 & 2 & 1 & 1 & 0 & 1\\
              2 & 1 & 1 & 1 & 1 & 0\\
              1 & 1 & 1 & 1 & 1 & 1
           \end{array}}
    = \argmin_{x\in\omsK}
           \setn{\begin{array}{c}
             {2}\\
             {2}\\
             {2}\\
             {2}\\
             {2}\\
             {2}\\
             {1}
           \end{array}}
    = \setn{\begin{array}{c}
              \mbox{ }\\
              \mbox{ }\\
              \mbox{ }\\
              \mbox{ }\\
              \mbox{ }\\
              \mbox{ }\\
              0 
           \end{array}}}
    %\\
    %\ocsEa(\rvZ)
    %  &\eqd \argmin_{x\in\omsK}\sum_{y}\ocsmom(x,y)
    %  &&\text{by definition of $\ocsEa$ \xref{def:ocsEa}}
    %\\&= \argmin_{x\in\omsK}
    %       \setn{\begin{array}{*{11}{@{\hspace{2pt}}c}}
    %         0 &+& 1 &+& 1 &+& 1 &+& 1 &+& 2 \\
    %         1 &+& 0 &+& 1 &+& 1 &+& 2 &+& 1 \\
    %         1 &+& 1 &+& 0 &+& 2 &+& 1 &+& 1 \\
    %         1 &+& 1 &+& 2 &+& 0 &+& 1 &+& 1 \\
    %         1 &+& 2 &+& 1 &+& 1 &+& 0 &+& 1 \\
    %         2 &+& 1 &+& 1 &+& 1 &+& 1 &+& 0 \\
    %         1 &+& 1 &+& 1 &+& 1 &+& 1 &+& 1 
    %       \end{array}}
    %\quad= \argmin_{x\in\omsK}
    %       \setn{\begin{array}{c}
    %         6\\
    %         6\\
    %         6\\
    %         6\\
    %         6\\
    %         6\\
    %         6
    %       \end{array}}
    %\\&= \setn{0,1,2,3,4,5,6}
%  \end{align*}
   \\
%  \begin{align*}
    \ocsVar(\rvZ)
      &\eqd \sum_{x\in\omsK}\metricsq{\ocsE(\rvZ)}{x}\psp(x)
      && \text{by definition of $\ocsVar$ \xref{def:ocsVar}}
    \\&= \sum_{x\in\omsK\setd\setn{0}}\metricsq{\setn{0}}{x}\psp(x)
      && \text{by $\ocsE(\rvZ)$ result}
    \\&= 6\times1^2\times\frac{1}{6} = 1
      && \text{by $\ocsE(\rvZ)$ result}
  \end{align*}
%\end{enumerate}
\end{proof}



\begin{figure}[h]
  \gsize%
  \centering%
  %{%============================================================================
% Daniel J. Greenhoe
% LaTeX file
% discrete metric real dice mapping to linearly ordered L6
%============================================================================
{%\psset{unit=0.5\psunit}%
\begin{pspicture}(-6.2,-1.8)(6,2)%
  %---------------------------------
  % options
  %---------------------------------
  \psset{%
    linecolor=blue,%
    radius=1.25ex,
    labelsep=2.5mm,
    }%
  %---------------------------------
  % dice graph
  %---------------------------------
  \rput(0,0){%\psset{unit=2\psunit}%
    \uput{1}[210](0,0){\Cnode[fillstyle=solid,fillcolor=snode](0,0){D4}}%
    \uput{1}[150](0,0){\Cnode(0,0){D5}}%
    \uput{1}[ 90](0,0){\Cnode(0,0){D6}}%
    \uput{1}[ 30](0,0){\Cnode(0,0){D3}}%
    \uput{1}[-30](0,0){\Cnode(0,0){D2}}%
    \uput{1}[-90](0,0){\Cnode(0,0){D1}}%
    \rput(0,0){$\ocsG$}%
    }
  \rput(D6){$\diceF$}%
  \rput(D5){$\diceE$}%
  \rput(D4){$\diceD$}%
  \rput(D3){$\diceC$}%
  \rput(D2){$\diceB$}%
  \rput(D1){$\diceA$}%
  %
  \ncline{D5}{D6}%
  \ncline{D4}{D5}\ncline{D4}{D6}%
  \ncline{D3}{D5}\ncline{D3}{D6}%
  \ncline{D2}{D3}\ncline{D2}{D4}\ncline{D2}{D6}%
  \ncline{D1}{D2}\ncline{D1}{D3}\ncline{D1}{D4}\ncline{D1}{D5}%
  %
  \uput[  90](D6){$\frac{1}{30}$}
  \uput[ 150](D5){$\frac{1}{50}$}
  \uput[ 210](D4){$\frac{3}{5}$}
  \uput[  30](D3){$\frac{1}{30}$}
  \uput[ -60](D2){$\frac{1}{20}$}
  \uput[ -90](D1){$\frac{1}{10}$}
  %---------------------------------
  % mapping X to real line
  %---------------------------------
  \rput(-4.5,-1.75){\psset{unit=0.5\psunit}%
    \pnode(0,7){X7}%
    \pnode(0,6){X6}%
    \pnode(0,5){X5}%
    \pnode(0,4){X4}%
    \Cnode*[linecolor=snode,fillstyle=solid,fillcolor=snode](0,3){X3}%
    \pnode(0,2){X2}%
    \pnode(0,1){X1}%
    \pnode(0,0){X0}%
    \pscircle[fillstyle=none,linecolor=red,fillcolor=red](0,3.57){1ex}%
    }%
  \uput[180](X7){$\omsR$}%
  \ncline{->}{X6}{X7}%
  \ncline{X5}{X6}%
  \ncline{X4}{X5}%
  \ncline{X3}{X4}%
  \ncline{X2}{X3}%
  \ncline{X1}{X2}%
  \ncline{->}{X1}{X0}%
  %
  \rput(X6){\psline[linewidth=1pt](-0.1,0)(0.1,0)}%
  \rput(X5){\psline[linewidth=1pt](-0.1,0)(0.1,0)}%
  \rput(X4){\psline[linewidth=1pt](-0.1,0)(0.1,0)}%
  \rput(X3){\psline[linewidth=1pt](-0.1,0)(0.1,0)}%
  \rput(X2){\psline[linewidth=1pt](-0.1,0)(0.1,0)}%
  \rput(X1){\psline[linewidth=1pt](-0.1,0)(0.1,0)}%
  %
  \uput[180](X6){$6$}%
  \uput[180](X5){$5$}%
  \uput[180](X4){$4$}%
  \uput[180](X3){$3$}%
  \uput[180](X2){$2$}%
  \uput[180](X1){$1$}%
  %
  %\uput[ 158](X6){$\psp(6)=\frac{1}{30}$}
  %\uput[ 150](X5){$\psp(5)=\frac{1}{50}$}
  %\uput[ 210](X4){$\psp(4)=\frac{3}{5}$}
  %\uput[  22](X3){$\psp(3)=\frac{1}{30}$}
  %\uput[ -45](X2){$\psp(2)=\frac{1}{20}$}
  %\uput[-158](X1){$\psp(1)=\frac{1}{10}$}
  %
  \ncarc[arcangle=-22,linewidth=0.75pt,linecolor=red]{->}{D6}{X6}%
  \ncarc[arcangle=-22,linewidth=0.75pt,linecolor=red]{->}{D5}{X5}%
  \ncarc[arcangle=-22,linewidth=0.75pt,linecolor=red]{->}{D4}{X4}%
  \ncarc[arcangle= 22,linewidth=0.75pt,linecolor=red]{->}{D3}{X3}%
  \ncarc[arcangle= 45,linewidth=0.75pt,linecolor=red]{->}{D2}{X2}%
  \ncarc[arcangle= 22,linewidth=0.75pt,linecolor=red]{->}{D1}{X1}%
  %---------------------------------
  % isomorphic mapping Y
  %---------------------------------
  \rput(4.5,0){%\psset{unit=2\psunit}%
    \uput{1}[210](0,0){\Cnode[fillstyle=solid,fillcolor=snode](0,0){Y4}}%
    \uput{1}[150](0,0){\Cnode(0,0){Y5}}%
    \uput{1}[ 90](0,0){\Cnode(0,0){Y6}}%
    \uput{1}[ 30](0,0){\Cnode(0,0){Y3}}%
    \uput{1}[-30](0,0){\Cnode(0,0){Y2}}%
    \uput{1}[-90](0,0){\Cnode(0,0){Y1}}%
    \rput(0,0){$\omsK$}%
    }
  \rput(Y6){$6$}%
  \rput(Y5){$5$}%
  \rput(Y4){$4$}%
  \rput(Y3){$3$}%
  \rput(Y2){$2$}%
  \rput(Y1){$1$}%
  %
  \ncline{Y5}{Y6}%
  \ncline{Y4}{Y5}\ncline{Y4}{Y6}%
  \ncline{Y3}{Y5}\ncline{Y3}{Y6}%
  \ncline{Y2}{Y3}\ncline{Y2}{Y4}\ncline{Y2}{Y6}%
  \ncline{Y1}{Y2}\ncline{Y1}{Y3}\ncline{Y1}{Y4}\ncline{Y1}{Y5}%
  %
  \uput[  90](Y6){$\frac{1}{30}$}
  \uput[ 150](Y5){$\frac{1}{50}$}
  \uput[ 210](Y4){$\frac{3}{5}$}
  \uput[  30](Y3){$\frac{1}{30}$}
  \uput[ -60](Y2){$\frac{1}{20}$}
  \uput[ -90](Y1){$\frac{1}{10}$}
  %
  \ncarc[arcangle= 22,linewidth=0.75pt,linecolor=green]{->}{D6}{Y6}%
  \ncarc[arcangle= 67,linewidth=0.75pt,linecolor=green]{->}{D5}{Y5}%
  \ncarc[arcangle=-67,linewidth=0.75pt,linecolor=green]{->}{D4}{Y4}%
  \ncarc[arcangle= 67,linewidth=0.75pt,linecolor=green]{->}{D3}{Y3}%
  \ncarc[arcangle=-67,linewidth=0.75pt,linecolor=green]{->}{D2}{Y2}%
  \ncarc[arcangle=-22,linewidth=0.75pt,linecolor=green]{->}{D1}{Y1}%
  %---------------------------------
  % labels
  %---------------------------------
  \rput(-2.3,0){$\rvX(\cdot)$}%
  \rput(2.3,0){$\rvY(\cdot)$}%
  %\ncline[linestyle=dotted,nodesep=1pt]{->}{xzlabel}{xz}%
  %\ncline[linestyle=dotted,nodesep=1pt]{->}{ylabel}{y}%
\end{pspicture}
}%}%
  {\includegraphics{sto/graphics/wdie_xy.pdf}}%
  \caption{\structe{weighted die} mappings \xref{ex:wdie_xy}\label{fig:wdie_xy}}
\end{figure}
%---------------------------------------
\begin{example}[\exmd{weighted die mappings}]
\label{ex:wdie_xy}
%---------------------------------------
Let $\ocsG$ be \structe{weighted die outcome subspace} \xref{ex:wdie},
and $\rvX$ and $\rvY$ be \fncte{random variable}s, as illustrated in \prefpp{fig:wdie_xy}.
Let $\pE$, $\pVar$, $\ocsE$, and $\ocsVar$ be defined as in \prefpp{ex:fairdieXRYZ}.
This yields the following statistics:
\\$\begin{array}{>{\gsizes}Mlcl @{\quad} rcccl}
  geometry of $\ocsG$:                                & \ocscen (\ocsG) &=& \mc{4}{l}{\setn{\diceD}}  \\
  traditional statistics on real line:                & \pE   (\rvX)    &=& 3                                       & \ocsVar(\rvX;\pE)   &=& \frac{143}{100} &=&       1.43  \\
  outcome subspace statistics on real line:           & \ocsE (\rvX)    &=& \setn{\frac{25}{7}} \approx \setn{3.57} & \ocsVar(\rvX;\ocsE) &=& \frac{3361}{2940} &\approx& 1.143 \\
  outcome subspace stats. on isomorphic structure & \ocsE (\rvY)    &=& \setn{4}                                & \ocsVar(\rvX;\ocsE) &=& \frac{101}{300} &\approx& 0.337   
\end{array}$
\\
The statistic $\pE(\rvX)=3$ evaluated on the \structe{real line} is arguably very poor because
it suggests that we ``expect" the event $\diceC$ rather than $\diceD$, even though 
$\psp(\diceD)$ is very large, $\psp(\diceC)$ is very small, 
and the physical distance $\metric{\diceD}{\diceC}=2$ on the die from $\diceD$ to $\diceC$ is twice as much as it is to any of the other 
four die faces.
%Intuitively and according to the variance (expected error) measures, even
If we retain use of the real line but replace the \fncte{traditional expected value} $\pE(\rvX)$
with the \fncte{outcome expected value} $\ocsE(\rvX)$,
a small but significant improvement is made ($\ocsVar(\rvX;\ocsE)\approx1.143<1.43=\ocsVar(\rvX;\pE)$).
%
Arguably a better choice still is to abandon the real line altogether
in favor of the isomorphic structure $\omsK$ and the statistic $\ocsE(\rvY)=\setn{4}$ evaluated on $\omsK$,
yielding not only an intuitively better result but also a variance $\ocsVar(\rvY;\ocsE)$ 
that is more than 4 times smaller than that of $\pE(\rvX)$
($\ocsVar(\rvY;\ocsE)\approx0.337<1.43=\ocsVar(\rvX;\pE)$).
%\begin{tabular}{c}
%  \gsize%
%  \psset{unit=5mm}%
%  {%============================================================================
% Daniel J. Greenhoe
% LaTeX file
% discrete metric real dice mapping to linearly ordered L6
%============================================================================
{%\psset{unit=0.5\psunit}%
\begin{pspicture}(0,-2.6)(8.5,2.6)%
  %---------------------------------
  % options
  %---------------------------------
  \psset{%
    linecolor=blue,%
    radius=1.25ex,
    labelsep=2.5mm,
    }%
  %---------------------------------
  % dice graph
  %---------------------------------
  \rput(2.4,0){\psset{unit=2\psunit}%
    \Cnode[fillstyle=solid,fillcolor=snode](-0.8660,-0.5){D4}%
    \Cnode(-0.8660,0.5){D5}%
    \Cnode(0,1){D6}%
    \Cnode(0.8660,0.5){D3}%
    \Cnode(0.8660,-0.5){D2}%
    \Cnode(0,-1){D1}%
    }
  \rput(D6){$\diceF$}%
  \rput(D5){$\diceE$}%
  \rput(D4){$\diceD$}%
  \rput(D3){$\diceC$}%
  \rput(D2){$\diceB$}%
  \rput(D1){$\diceA$}%
  %
  \ncline{D5}{D6}%
  \ncline{D4}{D5}\ncline{D4}{D6}%
  \ncline{D3}{D5}\ncline{D3}{D6}%
  \ncline{D2}{D3}\ncline{D2}{D4}\ncline{D2}{D6}%
  \ncline{D1}{D2}\ncline{D1}{D3}\ncline{D1}{D4}\ncline{D1}{D5}%
  %
  \uput[ 158](D6){$\frac{1}{30}$}
  \uput[ 150](D5){$\frac{1}{50}$}
  \uput[ 210](D4){$\frac{3}{5}$}
  \uput[  22](D3){$\frac{1}{30}$}
  \uput[ -45](D2){$\frac{1}{20}$}
  \uput[-158](D1){$\frac{1}{10}$}
  %---------------------------------
  % L6 lattice
  %---------------------------------
  \rput(8,0){\psset{unit=0.7\psunit}%
    \pnode(0,3.5){L7}%
    \pnode(0,2.5){L6}%
    \pnode(0,1.5){L5}%
    \pnode(0,0.5){L4}%
    \Cnode*[linecolor=snode,fillstyle=solid,fillcolor=snode](0,-0.5){L3}%
    \pnode(0,-1.5){L2}%
    \pnode(0,-2.5){L1}%
    \pnode(0,-3.5){L0}%
    }%
  \ncline{->}{L6}{L7}%
  \ncline{L5}{L6}%
  \ncline{L4}{L5}%
  \ncline{L3}{L4}%
  \ncline{L2}{L3}%
  \ncline{L1}{L2}%
  \ncline{->}{L1}{L0}%
  %
  \rput(L6){\psline[linewidth=1pt](-0.1,0)(0.1,0)}%
  \rput(L5){\psline[linewidth=1pt](-0.1,0)(0.1,0)}%
  \rput(L4){\psline[linewidth=1pt](-0.1,0)(0.1,0)}%
  \rput(L3){\psline[linewidth=1pt](-0.1,0)(0.1,0)}%
  \rput(L2){\psline[linewidth=1pt](-0.1,0)(0.1,0)}%
  \rput(L1){\psline[linewidth=1pt](-0.1,0)(0.1,0)}%
  %
  \uput[0](L6){$6$}%
  \uput[0](L5){$5$}%
  \uput[0](L4){$4$}%
  \uput[0](L3){$3$}%
  \uput[0](L2){$2$}%
  \uput[0](L1){$1$}%
  %
  %\uput[ 158](L6){$\psp(6)=\frac{1}{30}$}
  %\uput[ 150](L5){$\psp(5)=\frac{1}{50}$}
  %\uput[ 210](L4){$\psp(4)=\frac{3}{5}$}
  %\uput[  22](L3){$\psp(3)=\frac{1}{30}$}
  %\uput[ -45](L2){$\psp(2)=\frac{1}{20}$}
  %\uput[-158](L1){$\psp(1)=\frac{1}{10}$}
  %---------------------------------
  % mapping from die to L6
  %---------------------------------
  \ncarc[arcangle= 22,linewidth=0.75pt,linecolor=red]{->}{D6}{L6}%
  \ncarc[arcangle= 22,linewidth=0.75pt,linecolor=red]{->}{D5}{L5}%
  \ncarc[arcangle= 22,linewidth=0.75pt,linecolor=red]{->}{D4}{L4}%
  \ncarc[arcangle=-22,linewidth=0.75pt,linecolor=red]{->}{D3}{L3}%
  \ncarc[arcangle=-22,linewidth=0.75pt,linecolor=red]{->}{D2}{L2}%
  \ncarc[arcangle=-22,linewidth=0.75pt,linecolor=red]{->}{D1}{L1}%
  %\ncline[linewidth=0.75pt,linecolor=red]{->}{b}{lb}%
  %---------------------------------
  % labels
  %---------------------------------
  \rput(6.5,1.25){$\rvX(\cdot)$}%
  %\ncline[linestyle=dotted,nodesep=1pt]{->}{xzlabel}{xz}%
  %\ncline[linestyle=dotted,nodesep=1pt]{->}{ylabel}{y}%
\end{pspicture}
}%}%
%\end{tabular}
\end{example}
\begin{proof}
    \begin{align*}
      \ocscen(\ocsG)
      %  &\eqd \argmin_{x\in\ocsG}\max_{y\in\omsH}\ocsmom(x,y)
      %  &&\text{by definition of $\ocscen$ \xref{def:ocscen}}
      %\\&\eqd \argmin_{x\in\omsH}\max_{y\in\omsH}\metric{x}{y}\psp(y)
      %  &&\text{by definition of $\ocsmom$ \xref{def:ocsmom}}
        &= \setn{\diceD}
        &&\text{by \prefpp{ex:wdie}}
      %\\
      %\ocscena(\ocsG)
      %  &\eqd \argmin_{x\in\ocsG}\max_{y\in\omsH}\ocsmom(x,y)
      %  &&\text{by definition of $\ocscena$ \xref{def:ocscena}}
      %\\&\eqd \argmin_{x\in\omsH}\max_{y\in\omsH}\metric{x}{y}\psp(y)
      %  &&\text{by definition of $\ocsmom$ \xref{def:ocsmom}}
      %\\&= \setn{\diceD}
      %  &&\text{by \prefpp{ex:wdie}}
      %\\
      %\ocsVar(\ocsG)
      %  &\eqd \sum_{x\in\ocsG}\brs{\metric{\diceD}{x}}^2\psp(x)
      %  &&\text{by definition of $\ocsVar$ \xref{def:ocsVar}}
      %\\&= \frac{101}{300}
      %   \approx 0.33
      %  &&\text{by \prefpp{ex:wdie}}
      \\
      \pE(\rvX) 
        &\eqd \int_{\R} x\psp(x)\dx
        && \text{by definition of $\pE$ \xref{def:pE}}
      \\&= \sum_{x\in\Z} x\psp(x)
        && \text{by definition of $\psp$}
      \\&= \mathrlap{1\times\frac{1}{10}+2\times\frac{1}{20}+3\times\frac{1}{30}+4\times\frac{3}{5}+5\times\frac{1}{50}+6\times\frac{1}{30}}
      \\&=\mathrlap{\frac{1}{300}\brp{1\times30 + 2\times15 + 3\times10 + 4\times180 + 5\times6 + 6\times10}
         = \frac{900}{300} = 3}
      \\
      \ocsVar(\rvX;\pE)
        &= \pVar(\rvX)
        && \text{by \prefp{thm:ocsVar}}
      \\&\eqd \int_{\R} \brs{x-\pE(\rvX)}^2\psp(x)
        && \text{by definition of $\pVar$ \xref{def:pVar}}
      \\&= \sum_{x\in\Z} \brs{x-\pE(\rvX)}^2\psp(x)
        && \text{by definition of $\psp$}
      \\&= \mathrlap{\frac{1}{10}(1-3)^2 + \frac{1}{20}(2-3)^2 + \frac{1}{30}(3-3)^2 + \frac{3}{5}(4-3)^2 + \frac{1}{50}(5-3)^2 + \frac{1}{30}(6-3)^2} 
      \\&= \mathrlap{\frac{1}{300}\brp{120+15+0+180+24+90} = \frac{429}{300}=\frac{143}{100}=1.43}
      \\
      \ocsE(\rvX)
        &\eqd \argmin_{x\in\R}\max_{y\in\R}\metric{x}{y}\psp(y)
        &&\text{by definition of $\ocsE$ \xref{def:ocsE}}
      \\&\eqd \argmin_{x\in\R}\max_{y\in\R}\abs{x-y}\psp(y)
        &&\text{by definition standard metric on real line \xref{def:Rline}}
      \\&= \argmin_{x\in\R}\brbl{\begin{array}{lM}
             \abs{x-1}\frac{1}{10} & for $\frac{25}{7}\le x\le\frac{23}{5}$\\
             \abs{x-4}\frac{3}{5}  & otherwise
           \end{array}}
        %&& \text{\psset{unit=6mm}\gsize%============================================================================
% Daniel J. Greenhoe
% XeLaTeX file
%============================================================================
{%\psset{yunit=2\psunit}%
\begin{pspicture}(-1,0)(8,2.75)%
  \psset{%
    labelsep=1pt,
    linewidth=1pt,
    }%
  \psaxes[linecolor=axis,yAxis=false,labels=none]{->}(0,0)(0,0)(7.75,2.75)% x axis
  \psaxes[linecolor=axis,xAxis=false,labels=none]{->}(0,0)(0,0)(7.75,2.75)% y axis
  %
  \psline(0,2.4)(3.571,0.2571)(4.6,0.36)(7,1.8)%
  %
  \psline[linestyle=dotted,linecolor=red](3.571,0.2571)(3.571,0)%
  \psline[linestyle=dotted,linecolor=red](3.571,0.2571)(0,0.2571)%
  \psline[linestyle=dotted,linecolor=red](4.6,0.36)(4.6,0)%
  \psline[linestyle=dotted,linecolor=red](4.6,0.36)(0,0.36)%
  \psline[linestyle=dotted,linecolor=red](7,1.8)(7,0)%
  \psline[linestyle=dotted,linecolor=red](7,1.8)(0,1.8)%
  %
  \uput[0]{0}(7.75,0){$x$}%
  \uput[-90]{0}(3.3571,0){$\frac{25}{7}$}%
  \uput[-90]{0}(4.6,0){$\frac{23}{5}$}%
  \uput[180]{0}(0,2.4){$2.4$}%
  \uput[180]{0}(0,1.8){$1.8$}%
  \uput[150]{0}(0,0.36){$\sfrac{9}{25}$}%
  \uput[195]{0}(0,0.2571){$\sfrac{9}{35}$}%
  \rput[t](3.5,2.75){$\ds\ff(x)\eqd\max_{y\in\R}\abs{x-y}\psp(y)$}%
\end{pspicture}}%
}
        && \text{\psset{unit=6mm}\gsize\includegraphics{sto/graphics/wdie_max.pdf}}
      \\&= \setn{\frac{25}{7}} \approx \setn{3.5714}
        && \text{because $\ff(x)$ is minimized at argument $x=\frac{25}{7}$}
      \\
      \ocsVar(\rvX;\ocsE)
        &\eqd \sum_{x\in\R}\metricsq{\ocsE(\rvX)}{x}\psp(x)
        && \text{by definition of $\ocsVar$ \xref{def:ocsVar}}
      \\&= \sum_{x\in\R}\metricsq{\frac{25}{7}}{x}\psp(x)
        && \text{by $\ocsE(\rvX)$ result}
      \\&= \mathrlap{
           \brp{\frac{25}{7}-1}^2\frac{1}{10} +
           \brp{\frac{25}{7}-2}^2\frac{1}{20} +
           \brp{\frac{25}{7}-3}^2\frac{1}{30} +
           \brp{\frac{25}{7}-4}^2\frac{3}{5}  +
           \brp{\frac{25}{7}-5}^2\frac{1}{50} +
           \brp{\frac{25}{7}-6}^2\frac{1}{30} 
           }
      %\\&= \mathrlap{\frac{1}{49\times300}
      %      30\brp{25- 7}^2 +
      %      15\brp{25-14}^2 +
      %      10\brp{25-21}^2 +
      %     180\brp{25-28}^2 +
      %       6\brp{25-35}^2 +
      %      10\brp{25-42}^2 
      %\\&= \mathrlap{\frac{1}{49\times300}
      %      30\brp{18}^2 +
      %      15\brp{11}^2 +
      %      10\brp{ 4}^2 +
      %     180\brp{ 3}^2 +
      %       6\brp{10}^2 +
      %      10\brp{17}^2 +
      %     }
      \\&= \mathrlap{\frac{16805}{49*300} = \frac{3361}{49*60} = \frac{3361}{2940} \approx 1.143}
      \\
      \ocsE(\rvY)
        &= \rvY\brp{\ocscen(\ocsG)}
        && \text{because $\ocsG$ and $\ocsH$ are \prope{isomorphic}}
      \\&= \rvY\brp{\setn{\diceD}}
        &&\text{by \prefpp{ex:wdie}}
      \\&= \setn{4}
        &&\text{by definition of $\rvY$}
      %\\
      %\ocsEa(\rvY)
      %  &\eqd \argmin_{x\in\omsH}\max_{y\in\omsH}\ocsmom(x,y)
      %  &&\text{by definition of $\ocsEa$ \xref{def:ocsEa}}
      %\\&= \rvY\brs{\argmin_{x\in\ocsG}\max_{y\in\ocsG}\ocsmom(x,y)}
      %  &&\text{because $\ocsG$ and $\omsH$ are \prope{isomorphic}}
      %\\&= \rvY\brs{\ocscena(\ocsG)}
      %  &&\text{by definition of $\ocscena$ \xref{def:ocscena}}
      %\\&= \rvY\brs{\setn{\diceD}}
      %  &&\text{by \prefpp{item:wdie_xy_geo}}
      %\\&= \setn{4}
      %  &&\text{by definition of $\rvY$}
      \\
      \ocsVar(\rvY;\ocsE)
        &= \ocsVaro(\ocsG)
         = \frac{101}{300} \approx 0.337
        &&\text{by \prefpp{ex:wdie}}
    \end{align*}
\end{proof}

%=======================================
\subsubsection{Spinner examples}
%=======================================
\begin{figure}[h]
  \centering%
  \gsize%
  %\psset{unit=15mm}%
%  \gsize\psset{unit=5mm}{%============================================================================
% Daniel J. Greenhoe
% LaTeX file
% spinner 6 mapping to linearly ordered L6
%============================================================================
{%\psset{unit=0.5\psunit}%
\begin{pspicture}(0,-0.4)(8.5,5.5)%
  %---------------------------------
  % options
  %---------------------------------
  \psset{%
    linecolor=blue,%
    radius=1.25ex,
    labelsep=2.5mm,
    }%
  %---------------------------------
  % spinner graph
  %---------------------------------
  \rput(2.4,2.5){\psset{unit=2\psunit}%
    \Cnode[fillstyle=solid,fillcolor=snode](-0.8660,-0.5){D6}%
    \Cnode[fillstyle=solid,fillcolor=snode](-0.8660,0.5){D5}%
    \Cnode[fillstyle=solid,fillcolor=snode](0,1){D4}%
    \Cnode[fillstyle=solid,fillcolor=snode](0.8660,0.5){D3}%
    \Cnode[fillstyle=solid,fillcolor=snode](0.8660,-0.5){D2}%
    \Cnode[fillstyle=solid,fillcolor=snode](0,-1){D1}%
    }
  \rput[-150](D6){$\circSix$}%
  \rput[ 150](D5){$\circFive$}%
  \rput[  90](D4){$\circFour$}%
  \rput[  30](D3){$\circThree$}%
  \rput[   0](D2){$\circTwo$}%
  \rput[ -90](D1){$\circOne$}%
  %
  \ncline{D6}{D1}%
  \ncline{D5}{D6}%
  \ncline{D4}{D5}%
  \ncline{D3}{D4}%
  \ncline{D2}{D3}%
  \ncline{D1}{D2}%
  %
  \uput[ 158](D6){$\frac{1}{6}$}
  \uput[ 150](D5){$\frac{1}{6}$}
  \uput[ 210](D4){$\frac{1}{6}$}
  \uput[  22](D3){$\frac{1}{6}$}
  \uput[ -45](D2){$\frac{1}{6}$}
  \uput[-158](D1){$\frac{1}{6}$}
  %---------------------------------
  % L6 lattice
  %---------------------------------
  \rput(8,0){%
    \Cnode(0,5){L6}%
    \Cnode(0,4){L5}%
    \Cnode[fillstyle=solid,fillcolor=snode](0,3){L4}%
    \Cnode[fillstyle=solid,fillcolor=snode](0,2){L3}%
    \Cnode(0,1){L2}%
    \Cnode(0,0){L1}%
    }%
  \ncline{L5}{L6}%
  \ncline{L4}{L5}%
  \ncline{L3}{L4}%
  \ncline{L2}{L3}%
  \ncline{L1}{L2}%
  %
  \rput(L6){$6$}%
  \rput(L5){$5$}%
  \rput(L4){$4$}%
  \rput(L3){$3$}%
  \rput(L2){$2$}%
  \rput(L1){$1$}%
  %---------------------------------
  % mapping from die to L6
  %---------------------------------
  \ncarc[arcangle= 22,linewidth=0.75pt,linecolor=red]{->}{D6}{L6}%
  \ncarc[arcangle= 22,linewidth=0.75pt,linecolor=red]{->}{D5}{L5}%
  \ncarc[arcangle= 22,linewidth=0.75pt,linecolor=red]{->}{D4}{L4}%
  \ncarc[arcangle=-22,linewidth=0.75pt,linecolor=red]{->}{D3}{L3}%
  \ncarc[arcangle=-22,linewidth=0.75pt,linecolor=red]{->}{D2}{L2}%
  \ncarc[arcangle=-22,linewidth=0.75pt,linecolor=red]{->}{D1}{L1}%
  %\ncline[linewidth=0.75pt,linecolor=red]{->}{b}{lb}%
  %---------------------------------
  % labels
  %---------------------------------
  \rput(6.5,1.25){$\rvX(\cdot)$}%
  %\ncline[linestyle=dotted,nodesep=1pt]{->}{xzlabel}{xz}%
  %\ncline[linestyle=dotted,nodesep=1pt]{->}{ylabel}{y}%
\end{pspicture}
}%}%
%  \gsize\psset{unit=15mm}{%============================================================================
% Daniel J. Greenhoe
% LaTeX file
% discrete metric real dice mapping to linearly ordered O6c
%============================================================================
{%\psset{unit=0.5\psunit}%
\begin{pspicture}(-3.3,-1.45)(3.3,1.45)%
  %---------------------------------
  % options
  %---------------------------------
  \psset{%
    radius=1.25ex,
    labelsep=2.5mm,
    linecolor=blue,%
    }%
  %---------------------------------
  % dice graph
  %---------------------------------
  \rput(-1.75,0){%\psset{unit=2\psunit}%
    \Cnode[fillstyle=solid,fillcolor=snode](-0.8660,-0.5){D6}%
    \Cnode[fillstyle=solid,fillcolor=snode](-0.8660,0.5){D5}%
    \Cnode[fillstyle=solid,fillcolor=snode](0,1){D4}%
    \Cnode[fillstyle=solid,fillcolor=snode](0.8660,0.5){D3}%
    \Cnode[fillstyle=solid,fillcolor=snode](0.8660,-0.5){D2}%
    \Cnode[fillstyle=solid,fillcolor=snode](0,-1){D1}%
    }
  \rput(D6){\circSix}% 
  \rput(D5){\circFive}%
  \rput(D4){\circFour}%
  \rput(D3){\circThree}%
  \rput(D2){\circTwo}% 
  \rput(D1){\circOne}% 
  %
  \ncline{D6}{D1}\nbput[labelsep=0pt,nrot=:U]{${\scy\metric{\circSix}  {\circOne}=}1$}%
  \ncline{D6}{D5}\naput[labelsep=0pt,nrot=:U]{${\scy\metric{\circFive} {\circSix}=}1$}%
  \ncline{D5}{D4}\naput[labelsep=0pt,nrot=:U]{${\scy\metric{\circFour} {\circFive}=}1$}%
  \ncline{D4}{D3}\naput[labelsep=0pt,nrot=:U]{${\scy\metric{\circThree}{\circFour}=}1$}%
  \ncline{D2}{D3}\nbput[labelsep=0pt,nrot=:U]{${\scy\metric{\circTwo}  {\circThree}=}1$}%
  \ncline{D1}{D2}\nbput[labelsep=0pt,nrot=:U]{${\scy\metric{\circOne}  {\circTwo}=}1$}%
  %
  \uput[ 180](D6){${\scy\psp(\circSix)=}\frac{1}{6}$}
  \uput[ 150](D5){${\scy\psp(\circFive)=}\frac{1}{6}$}
  \uput[  90](D4){${\scy\psp(\circFour)=}\frac{1}{6}$}
  \uput[  22](D3){${\scy\psp(\circThree)=}\frac{1}{6}$}
  \uput[ -45](D2){${\scy\psp(\circTwo)=}\frac{1}{6}$}
  \uput[ -90](D1){${\scy\psp(\circOne)=}\frac{1}{6}$}
  %---------------------------------
  % range graph
  %---------------------------------
  \rput(1.75,0){%\psset{unit=2\psunit}%
    \Cnode(-0.8660,-0.5){E6}%
    \Cnode(-0.8660,0.5){E5}%
    \Cnode(0,1){E4}%
    \Cnode(0.8660,0.5){E3}%
    \Cnode(0.8660,-0.5){E2}%
    \Cnode(0,-1){E1}%
    \Cnode[fillstyle=solid,fillcolor=snode](0,0){E0}%
    }
  \rput(E6){$6$}%
  \rput(E5){$5$}%
  \rput(E4){$4$}%
  \rput(E3){$3$}%
  \rput(E2){$2$}%
  \rput(E1){$1$}%
  \rput(E0){$0$}%
  %
  \ncline{E6}{E1}\nbput[labelsep=0pt,nrot=:U]{${\scy\metric{6}{1}=}1$}%
  \ncline{E6}{E5}\naput[labelsep=0pt,nrot=:U]{${\scy\metric{5}{6}=}1$}%
  \ncline{E5}{E4}\naput[labelsep=0pt,nrot=:U]{${\scy\metric{4}{5}=}1$}%
  \ncline{E4}{E3}\naput[labelsep=0pt,nrot=:U]{${\scy\metric{3}{4}=}1$}%
  \ncline{E2}{E3}\nbput[labelsep=0pt,nrot=:U]{${\scy\metric{2}{3}=}1$}%
  \ncline{E1}{E2}\nbput[labelsep=0pt,nrot=:U]{${\scy\metric{1}{2}=}1$}%
  %
  \ncline{E0}{E1}\naput[labelsep=0pt,nrot=:U]{${\scy\metric{0}{1}=}\frac{3}{2}$}%
  \ncline{E0}{E2}\naput[labelsep=0pt,nrot=:U]{${\scy\metric{0}{2}=}\frac{3}{2}$}%
  \ncline{E0}{E3}\naput[labelsep=0pt,nrot=:U]{${\scy\metric{0}{3}=}\frac{3}{2}$}%
  \ncline{E0}{E4}\naput[labelsep=0pt,nrot=:U]{${\scy\metric{0}{4}=}\frac{3}{2}$}%
  \ncline{E5}{E0}\nbput[labelsep=0pt,nrot=:U]{${\scy\metric{0}{5}=}\frac{3}{2}$}%
  \ncline{E6}{E0}\nbput[labelsep=0pt,nrot=:U]{${\scy\metric{0}{6}=}\frac{3}{2}$}%
  %
  \uput[180](E6){${\scy\psp(6)=}\frac{1}{6}$}
  \uput[135](E5){${\scy\psp(5)=}\frac{1}{6}$}
  \uput[ 90](E4){${\scy\psp(4)=}\frac{1}{6}$}
  \uput[ 30](E3){${\scy\psp(3)=}\frac{1}{6}$}
  \uput[-30](E2){${\scy\psp(2)=}\frac{1}{6}$}
  \uput[-90](E1){${\scy\psp(1)=}\frac{1}{6}$}
  \uput[ 15](E0){${\scy\psp(0)=}\sfrac{0}{6}$}
  %---------------------------------
  % mapping from die to O6c
  %---------------------------------
  \ncarc[arcangle= 22,linewidth=0.75pt,linecolor=red]{->}{D6}{E6}%
  \ncarc[arcangle= 22,linewidth=0.75pt,linecolor=red]{->}{D5}{E5}%
  \ncarc[arcangle= 22,linewidth=0.75pt,linecolor=red]{->}{D4}{E4}%
  \ncarc[arcangle=-22,linewidth=0.75pt,linecolor=red]{->}{D3}{E3}%
  \ncarc[arcangle=-22,linewidth=0.75pt,linecolor=red]{->}{D2}{E2}%
  \ncarc[arcangle=-22,linewidth=0.75pt,linecolor=red]{->}{D1}{E1}%
  %---------------------------------
  % labels
  %---------------------------------
  \rput(0,0){$\rvX(\cdot)$}%
  %\ncline[linestyle=dotted,nodesep=1pt]{->}{xzlabel}{xz}%
  %\ncline[linestyle=dotted,nodesep=1pt]{->}{ylabel}{y}%
\end{pspicture}
}%}%
 %%============================================================================
% Daniel J. Greenhoe
% LaTeX file
% discrete metric real dice mapping to linearly ordered O6c
%============================================================================
{\psset{unit=1.5\psunit}%
\begin{pspicture}(-5.3,-2.5)(5.3,2.25)%
  %---------------------------------
  % options
  %---------------------------------
  \psset{%
    radius=1.25ex,
    labelsep=2.5mm,
    linecolor=blue,%
    }%
  %---------------------------------
  % spinner structure H
  %---------------------------------
  \rput(0,0){%\psset{unit=2\psunit}%
    \uput{1}[210](0,0){\Cnode[fillstyle=solid,fillcolor=snode](0,0){D6}}%
    \uput{1}[150](0,0){\Cnode[fillstyle=solid,fillcolor=snode](0,0){D5}}%
    \uput{1}[ 90](0,0){\Cnode[fillstyle=solid,fillcolor=snode](0,0){D4}}%
    \uput{1}[ 30](0,0){\Cnode[fillstyle=solid,fillcolor=snode](0,0){D3}}%
    \uput{1}[-30](0,0){\Cnode[fillstyle=solid,fillcolor=snode](0,0){D2}}%
    \uput{1}[-90](0,0){\Cnode[fillstyle=solid,fillcolor=snode](0,0){D1}}%
    \rput(0,0){$\ocsG$}%
    }
  \rput(D6){\circSix}% 
  \rput(D5){\circFive}%
  \rput(D4){\circFour}%
  \rput(D3){\circThree}%
  \rput(D2){\circTwo}% 
  \rput(D1){\circOne}% 
  %
  \ncline{D6}{D1}\nbput[labelsep=0pt,nrot=:U]{${\scy\metric{\circSix}  {\circOne}=}1$}%
  \ncline{D6}{D5}\naput[labelsep=0pt,nrot=:U]{${\scy\metric{\circFive} {\circSix}=}1$}%
  \ncline{D5}{D4}\naput[labelsep=0pt,nrot=:U]{${\scy\metric{\circFour} {\circFive}=}1$}%
  \ncline{D4}{D3}\naput[labelsep=0pt,nrot=:U]{${\scy\metric{\circThree}{\circFour}=}1$}%
  \ncline{D2}{D3}\nbput[labelsep=0pt,nrot=:U]{${\scy\metric{\circTwo}  {\circThree}=}1$}%
  \ncline{D1}{D2}\nbput[labelsep=0pt,nrot=:U]{${\scy\metric{\circOne}  {\circTwo}=}1$}%
  %
  \uput[ 180](D6){${\scy\psp(\circSix)=}\frac{1}{6}$}
  \uput[ 150](D5){${\scy\psp(\circFive)=}\frac{1}{6}$}
  \uput[  90](D4){${\scy\psp(\circFour)=}\frac{1}{6}$}
  \uput[  22](D3){${\scy\psp(\circThree)=}\frac{1}{6}$}
  \uput[ -45](D2){${\scy\psp(\circTwo)=}\frac{1}{6}$}
  \uput[ -90](D1){${\scy\psp(\circOne)=}\frac{1}{6}$}
  %---------------------------------
  % W mapping to real line
  %---------------------------------
  \rput(0,-2){
    \psline{<->}(-3,0)(3,0)%
    \multirput(-2.5,0)(1,0){6}{\psline(0,-0.1)(0,0.1)}%
    \pnode( 3,0){RB}%
    \pnode( 2.5,0){R6}%
    \pnode( 1.5,0){R5}%
    \pnode( 0.5,0){R4}%
    \pnode(0,0){R34}%
    \pnode(-0.5,0){R3}%
    \pnode(-1.5,0){R2}%
    \pnode(-2.5,0){R1}%
    \pnode(-3,0){RA}%
    }
  %
  \pscircle[fillstyle=solid,linecolor=snode,fillcolor=snode](R34){1ex}%
  \pscircle[fillstyle=none,linecolor=red,fillcolor=red](R34){1ex}%
  %
  \uput[-90](R6){$6$}%
  \uput[-90](R5){$5$}%
  \uput[-90](R4){$4$}%
  \uput[-90](R3){$3$}%
  \uput[-90](R2){$2$}%
  \uput[-90](R1){$1$}%
  \uput[0](RB){$\omsR$}%
  %
  \ncarc[arcangle= 22,linewidth=0.75pt,linecolor=purple]{->}{D6}{R6}%
  \ncarc[arcangle= 22,linewidth=0.75pt,linecolor=purple]{->}{D5}{R5}%
  \ncarc[arcangle= 22,linewidth=0.75pt,linecolor=purple]{->}{D4}{R4}%
  \ncarc[arcangle= 22,linewidth=0.75pt,linecolor=purple]{->}{D3}{R3}%
  \ncarc[arcangle= 22,linewidth=0.75pt,linecolor=purple]{->}{D2}{R2}%
  \ncarc[arcangle=-22,linewidth=0.75pt,linecolor=purple]{->}{D1}{R1}%
  %---------------------------------
  % X mapping to integer line
  %---------------------------------
  \rput(0,2){%\psset{unit=2\psunit}%
    \pnode(3,0){XB}%
    \Cnode(2.5,0){X6}%
    \Cnode(1.5,0){X5}%
    \Cnode[fillstyle=solid,fillcolor=snode](0.5,0){X4}%
    \Cnode[fillstyle=solid,fillcolor=snode](-0.5,0){X3}%
    \Cnode(-1.5,0){X2}%
    \Cnode(-2.5,0){X1}%
    \pnode(-3,0){XA}%
    }%
  \ncline[linestyle=dotted]{X6}{XB}%
  \ncline{X5}{X6}%
  \ncline{X4}{X5}%
  \ncline{X3}{X4}%
  \ncline{X2}{X3}%
  \ncline{X1}{X2}%
  \ncline[linestyle=dotted]{X1}{XA}%
  %
  \rput(X6){$6$}%
  \rput(X5){$5$}%
  \rput(X4){$4$}%
  \rput(X3){$3$}%
  \rput(X2){$2$}%
  \rput(X1){$1$}%
  %
  \ncarc[arcangle=-22,linewidth=0.75pt,linecolor=red]{->}{D6}{X6}%
  \ncarc[arcangle= 22,linewidth=0.75pt,linecolor=red]{->}{D5}{X5}%
  \ncarc[arcangle=-22,linewidth=0.75pt,linecolor=red]{->}{D4}{X4}%
  \ncarc[arcangle=-22,linewidth=0.75pt,linecolor=red]{->}{D3}{X3}%
  \ncarc[arcangle= 22,linewidth=0.75pt,linecolor=red]{->}{D2}{X2}%
  \ncarc[arcangle= 22,linewidth=0.75pt,linecolor=red]{->}{D1}{X1}%
  %
  %\uput[90](X6){${\scy\psp(6)=}\frac{1}{6}$}
  %\uput[90](X5){${\scy\psp(5)=}\frac{1}{6}$}
  %\uput[90](X4){${\scy\psp(4)=}\frac{1}{6}$}
  %\uput[90](X3){${\scy\psp(3)=}\frac{1}{6}$}
  %\uput[90](X2){${\scy\psp(2)=}\frac{1}{6}$}
  %\uput[90](X1){${\scy\psp(1)=}\frac{1}{6}$}
  \uput[0](XB){$\omsZ$}%
  %---------------------------------
  % Y mapping to isomorphic structure H
  %---------------------------------
  \rput(-3.5,0){%\psset{unit=2\psunit}%
    \uput{1}[210](0,0){\Cnode[fillstyle=solid,fillcolor=snode](0,0){Y6}}%
    \uput{1}[150](0,0){\Cnode[fillstyle=solid,fillcolor=snode](0,0){Y5}}%
    \uput{1}[ 90](0,0){\Cnode[fillstyle=solid,fillcolor=snode](0,0){Y4}}%
    \uput{1}[ 30](0,0){\Cnode[fillstyle=solid,fillcolor=snode](0,0){Y3}}%
    \uput{1}[-30](0,0){\Cnode[fillstyle=solid,fillcolor=snode](0,0){Y2}}%
    \uput{1}[-90](0,0){\Cnode[fillstyle=solid,fillcolor=snode](0,0){Y1}}%
    \rput(0,0){$\ocsH$}%
    }%
  \rput(Y6){$6$}%
  \rput(Y5){$5$}%
  \rput(Y4){$4$}%
  \rput(Y3){$3$}%
  \rput(Y2){$2$}%
  \rput(Y1){$1$}%
  %
  \ncline{Y6}{Y1}\nbput[labelsep=0pt,nrot=:U]{${\scy\metric{6}{1}=}1$}%
  \ncline{Y6}{Y5}\naput[labelsep=0pt,nrot=:U]{${\scy\metric{5}{6}=}1$}%
  \ncline{Y5}{Y4}\naput[labelsep=0pt,nrot=:U]{${\scy\metric{4}{5}=}1$}%
  \ncline{Y4}{Y3}\naput[labelsep=0pt,nrot=:U]{${\scy\metric{3}{4}=}1$}%
  \ncline{Y2}{Y3}\nbput[labelsep=0pt,nrot=:U]{${\scy\metric{2}{3}=}1$}%
  \ncline{Y1}{Y2}\nbput[labelsep=0pt,nrot=:U]{${\scy\metric{1}{2}=}1$}%
  %
  \uput[180](Y6){${\scy\psp(6)=}\frac{1}{6}$}
  \uput[135](Y5){${\scy\psp(5)=}\frac{1}{6}$}
  \uput[ 90](Y4){${\scy\psp(4)=}\frac{1}{6}$}
  \uput[ 30](Y3){${\scy\psp(3)=}\frac{1}{6}$}
  \uput[-30](Y2){${\scy\psp(2)=}\frac{1}{6}$}
  \uput[-90](Y1){${\scy\psp(1)=}\frac{1}{6}$}
  %
  \ncarc[arcangle= 22,linewidth=0.75pt,linecolor=blue]{->}{D6}{Y6}%
  \ncarc[arcangle= 22,linewidth=0.75pt,linecolor=blue]{->}{D5}{Y5}%
  \ncarc[arcangle=-22,linewidth=0.75pt,linecolor=blue]{->}{D4}{Y4}%
  \ncarc[arcangle=-22,linewidth=0.75pt,linecolor=blue]{->}{D3}{Y3}%
  \ncarc[arcangle=-22,linewidth=0.75pt,linecolor=blue]{->}{D2}{Y2}%
  \ncarc[arcangle= 22,linewidth=0.75pt,linecolor=blue]{->}{D1}{Y1}%
  %---------------------------------
  % Z mapping to extended structure H
  %---------------------------------
  \rput(3.5,0){%\psset{unit=2\psunit}%
    \uput{1}[210](0,0){\Cnode(0,0){Z6}}%
    \uput{1}[150](0,0){\Cnode(0,0){Z5}}%
    \uput{1}[ 90](0,0){\Cnode(0,0){Z4}}%
    \uput{1}[ 30](0,0){\Cnode(0,0){Z3}}%
    \uput{1}[-30](0,0){\Cnode(0,0){Z2}}%
    \uput{1}[-90](0,0){\Cnode(0,0){Z1}}%
    \Cnode[fillstyle=solid,fillcolor=snode](0,0){Z0}%
    \uput{1.25}[60](0,0){$\ocsK$}%
    }
  \rput(Z6){$6$}%
  \rput(Z5){$5$}%
  \rput(Z4){$4$}%
  \rput(Z3){$3$}%
  \rput(Z2){$2$}%
  \rput(Z1){$1$}%
  \rput(Z0){$0$}%
  %
  \ncline{Z6}{Z1}\nbput[labelsep=0pt,nrot=:U]{${\scy\metric{6}{1}=}1$}%
  \ncline{Z6}{Z5}\naput[labelsep=0pt,nrot=:U]{${\scy\metric{5}{6}=}1$}%
  \ncline{Z5}{Z4}\naput[labelsep=0pt,nrot=:U]{${\scy\metric{4}{5}=}1$}%
  \ncline{Z4}{Z3}\naput[labelsep=0pt,nrot=:U]{${\scy\metric{3}{4}=}1$}%
  \ncline{Z2}{Z3}\nbput[labelsep=0pt,nrot=:U]{${\scy\metric{2}{3}=}1$}%
  \ncline{Z1}{Z2}\nbput[labelsep=0pt,nrot=:U]{${\scy\metric{1}{2}=}1$}%
  %
  \ncline{Z0}{Z1}\naput[labelsep=0pt,nrot=:U]{${\scy\metric{0}{1}=}\frac{3}{2}$}%
  \ncline{Z0}{Z2}\naput[labelsep=0pt,nrot=:U]{${\scy\metric{0}{2}=}\frac{3}{2}$}%
  \ncline{Z0}{Z3}\naput[labelsep=0pt,nrot=:U]{${\scy\metric{0}{3}=}\frac{3}{2}$}%
  \ncline{Z0}{Z4}\naput[labelsep=0pt,nrot=:U]{${\scy\metric{0}{4}=}\frac{3}{2}$}%
  \ncline{Z5}{Z0}\nbput[labelsep=0pt,nrot=:U]{${\scy\metric{0}{5}=}\frac{3}{2}$}%
  \ncline{Z6}{Z0}\nbput[labelsep=0pt,nrot=:U]{${\scy\metric{0}{6}=}\frac{3}{2}$}%
  %
  \uput[180](Z6){${\scy\psp(6)=}\frac{1}{6}$}
  \uput[135](Z5){${\scy\psp(5)=}\frac{1}{6}$}
  \uput[ 90](Z4){${\scy\psp(4)=}\frac{1}{6}$}
  \uput[ 30](Z3){${\scy\psp(3)=}\frac{1}{6}$}
  \uput[-30](Z2){${\scy\psp(2)=}\frac{1}{6}$}
  \uput[-90](Z1){${\scy\psp(1)=}\frac{1}{6}$}
  \uput[ 15](Z0){${\scy\psp(0)=}\sfrac{0}{6}$}
  %
  \ncarc[arcangle= 22,linewidth=0.75pt,linecolor=green]{->}{D6}{Z6}%
  \ncarc[arcangle= 22,linewidth=0.75pt,linecolor=green]{->}{D5}{Z5}%
  \ncarc[arcangle= 22,linewidth=0.75pt,linecolor=green]{->}{D4}{Z4}%
  \ncarc[arcangle=-22,linewidth=0.75pt,linecolor=green]{->}{D3}{Z3}%
  \ncarc[arcangle=-22,linewidth=0.75pt,linecolor=green]{->}{D2}{Z2}%
  \ncarc[arcangle=-22,linewidth=0.75pt,linecolor=green]{->}{D1}{Z1}%
  %---------------------------------
  % labels
  %---------------------------------
  \rput(0,-1.5){$\rvW(\cdot)$}%
  \rput(0, 1.5){$\rvX(\cdot)$}%
  \rput(-2.25,0){$\rvY(\cdot)$}%
  \rput(1.75,0){$\rvZ(\cdot)$}%
  %\ncline[linestyle=dotted,nodesep=1pt]{->}{xzlabel}{xz}%
  %\ncline[linestyle=dotted,nodesep=1pt]{->}{ylabel}{y}%
\end{pspicture}%
}%%
  \includegraphics{sto/graphics/ocsrv_spinner_xy.pdf}%
  \caption{Six value fair spinner with assorted random variable mappings \xref{ex:spinner_xy} \label{fig:spinner_xy}}
\end{figure}
%---------------------------------------
%\begin{minipage}{\tw-48mm}%
\begin{example}[\exmd{spinner mappings}]
\label{ex:spinner_xy} %\mbox{}\\
%\label{ex:spinnerXO6c}%\mbox{}\\
%---------------------------------------
A six value board game spinner has a cyclic structure as illustrated in \prefpp{fig:spinner_xy}.
Again, the order and metric geometry of the real line mapped to by the random variable $\rvX$
is very dissimilar to that of the \structe{outcome subspace} that it is supposed to represent.
Therefore, statistical inferences based on $\rvX$ will likely result in values that are
arguably unacceptable.
Both random variables $\rvY$ and $\rvZ$ map to structures in which order and metric geometry are preserved.
The mappings yield the following statistics:
\\$\begin{array}{>{\gsizes}Mrcl rcccl}
  geometry of $\ocsG$:                                 & \ocscen (\ocsG) &=& \mc{4}{l}{\setn{\circOne,\circTwo,\circThree,\circFour,\circFive,\circSix}} \\
  traditional statistics on real line:                 & \pE   (\rvW)    &=& 3.5                & \ocsVar(\rvW;\pE)   &=& \frac{35}{12}&\approx& 2.917 \\
  outcome subspace statistics on real line:            & \ocsE (\rvW)    &=& \setn{3.5}         & \ocsVar(\rvW;\ocsE) &=& \frac{35}{12}&\approx& 2.917 \\
  outcome subspace statistics on integer line:         & \ocsE (\rvX)    &=& \setn{3,\,4}       & \ocsVar(\rvX;\ocsE) &=& \frac{20}{12}&\approx& 1.667 \\
  outcome subspace stats. on  isomorphic structure:& \ocsE (\rvY)    &=& \setn{1,2,3,4,5,6} & \ocsVar(\rvY;\ocsE) &=& 0            &       &       \\
  outcome subspace stats. on  extended structure:  & \ocsE (\rvZ)    &=& \setn{0}           & \ocsVar(\rvZ;\ocsE) &=& \frac{9}{4}  &=&       2.25  
\end{array}$
\end{example}
\begin{proof}
\begin{align*}
  \ocscen(\ocsG)
    &= \setn{\circOne,\circTwo,\circThree,\circFour,\circFive,\circSix}
    && \text{by \prefpp{ex:spinner}}
    \\
  \pE(\rvW)
      &\eqd \sum_{x\in\R} x\psp(x)
      && \text{by definition of $\pE$ \xref{def:pE}}
    \\&= \frac{7}{2} = 3.5
      && \text{by \exme{fair die} example \prefpp{ex:fairdieXRYZ}}
    \\
  \ocsVar(\rvW;\pE)
      &= \pVar(\rvX)
      && \text{by \prefp{thm:ocsVar}}
    \\&\eqd \sum_{x\in\R} \brs{x-\pE(\rvX)}^2\psp(x)\dx
      && \text{by definition of $\pVar$ \xref{def:pVar}}
    \\&= \frac{35}{12}\approx2.917
      && \text{by \exme{fair die} example \prefpp{ex:fairdieXRYZ}}
    \\
  \ocsE(\rvW)
      &= \pE(\rvW)
      && \text{because on \structe{real line}, $\psp$ is \prope{symmetric}, and by \prefp{thm:pEocsE}}
    \\&= \setn{3.5}
      && \text{by $\pE(\rvW)$ result}
    \\
  \ocsVar(\rvW;\ocsE)
      &= \ocsVar(\rvW;\pE)
      && \text{because $\ocsE(\rvW)=\pE(\rvW)$}
    \\&= \frac{35}{12} \approx 2.917
      && \text{by $\ocsVar(\rvW;\pE)$ result}
    \\
  \ocsE(\rvX)
      &\eqd \argmin_{x\in\Z}\max_{y\in\Z}\metric{x}{y}\psp(y)
      && \text{by definition of $\ocsE$ \xref{def:ocsE}}
    \\&\eqd \argmin_{x\in\Z}\max_{y\in\Z}\abs{x-y}\frac{1}{6}
      && \text{by definition of \structe{integer line} \xref{def:Zline} and $\ocsG$}
    \\&= \setn{3,\,4}
      && \text{by \exme{fair die} example \prefpp{ex:fairdieXRYZ}}
    \\
  \ocsVar(\rvX;\ocsE)
      &= \frac{5}{3} \approx 1.667
      && \text{by \exme{fair die} example \prefpp{ex:fairdieXRYZ}}
    \\
  \ocsE(\rvY)
      &= \rvY\brs{\ocscen(\ocsG)}
      && \text{because $\ocsG$ and $\omsH$ are \prope{isomorphic} under mapping $\rvY$}
    \\&= \rvY\brs{\setn{\circOne,\circTwo,\circThree,\circFour,\circFive,\circSix}}
      && \text{by $\ocscen(\ocsG)$ result}
    \\&= \setn{1,2,3,4,5,6}
      && \text{by definition of $\rvY$}
    \\
  \ocsVar(\rvY;\ocsE)
      &= \ocsVaro(\ocsG) = 0
      && \text{by \exme{spinner outcome subspace} example \xref{ex:spinner}}
    \\
  \ocsE(\rvZ)
    &\eqd \argmin_{x\in\omsH}\max_{y\in\omsH}\metric{x}{y}\psp(y)
    &&\text{by definition of $\ocsE$ \xref{def:ocsE}}
  \\&= \argmin_{x\in\omsH}\max_{y\in\omsH\setd\setn{0}}\metric{x}{y}\psp(y)
    &&\text{because $\psp(0)=0$}
  \\&= \argmin_{x\in\omsH}\max_{y\in\omsH\setd\setn{0}}\metric{x}{y}\frac{1}{6}
    &&\text{by definition of $\ocsG$}
  \\&= \argmin_{x\in\omsH}\max_{y\in\omsH\setd\setn{0}}\metric{x}{y}
    && \text{because $\ff(x)=\frac{1}{6}x$ is \prope{strictly isotone} and by \prefpp{lem:argminmaxphi}}
  \\&=\mathrlap{\argmin_{x\in\omsH}\max_{y\in\omsH}
         \setn{\begin{array}{c@{\,}c@{\,}c}
           \metricn(1,1) & \cdots &\metricn(1,6) \\
           \metricn(2,1) & \cdots &\metricn(2,6) \\
           \metricn(3,1) & \cdots &\metricn(3,6) \\
           \metricn(4,1) & \cdots &\metricn(4,6) \\
           \metricn(5,1) & \cdots &\metricn(5,6) \\
           \metricn(6,1) & \cdots &\metricn(6,6) \\
           \metricn(0,1) & \cdots &\metricn(0,6)
         \end{array}} 
     =\argmin_{x\in\omsH}\max_{y\in\omsH\setd\setn{0}}
         \setn{\begin{array}{*{6}{c@{\,\,}}}
           {0}&{1}&{2}&{3}&{2}&{1}\\
           {1}&{0}&{1}&{2}&{3}&{2}\\
           {2}&{1}&{0}&{1}&{2}&{3}\\
           {3}&{2}&{1}&{0}&{1}&{2}\\
           {2}&{3}&{2}&{1}&{0}&{1}\\
           {1}&{2}&{3}&{2}&{1}&{0}\\
           \frac{3}{2}&\frac{3}{2}&\frac{3}{2}&\frac{3}{2}&\frac{3}{2}&\frac{3}{2}
         \end{array}}
     = \argmin_{x\in\omsH}
         \setn{\begin{array}{c}
           {3}\\
           {3}\\
           {3}\\
           {3}\\
           {3}\\
           {3}\\
           {\frac{3}{2}}
         \end{array}}
     =   \setn{\begin{array}{c}
           \mbox{ }\\
           \mbox{ }\\
           \mbox{ }\\
           \mbox{ }\\
           \mbox{ }\\
           \mbox{ }\\
           0
         \end{array}}}
  %\\
  %\ocsEa(\rvX)
  %  &\eqd \argmin_{x\in\omsH}\sum_{y}\ocsmom(x,y)
  %  &&\text{by definition of $\pE$ \xref{def:ocsE}}
  %\\&=\mathrlap{\argmin_{x\in\omsH}
  %       \setn{\begin{array}{*{13}{@{\,}c}}
  %         {0} &+& {1} &+& {2} &+& {3} &+& {2} &+& {1} &+& 0\\
  %         {1} &+& {0} &+& {1} &+& {2} &+& {3} &+& {2} &+& 0\\
  %         {2} &+& {1} &+& {0} &+& {1} &+& {2} &+& {3} &+& 0\\
  %         {3} &+& {2} &+& {1} &+& {0} &+& {1} &+& {2} &+& 0\\
  %         {2} &+& {3} &+& {2} &+& {1} &+& {0} &+& {1} &+& 0\\
  %         {1} &+& {2} &+& {3} &+& {2} &+& {1} &+& {0} &+& 0\\
  %         \frac{3}{2} &+& \frac{3}{2} &+& \frac{3}{2} &+& \frac{3}{2} &+& \frac{3}{2} &+& \frac{3}{2} &+& 0
  %       \end{array}}
  %\quad= \argmin_{x\in\omsH}
  %       \setn{\begin{array}{c}
  %         9\\
  %         9\\
  %         9\\
  %         9\\
  %         9\\
  %         9\\
  %         3
  %       \end{array}}}
  %\\&= \setn{0}
  \\
  \ocsVar(\rvZ)
    &\eqd \sum_{x\in\omsH}\metricsq{\ocscen(\ocsG)}{x}\psp(x)
    && \text{by definition of $\ocsVar$ \xref{def:ocsVar}}
  \\&= \sum_{x\in\omsH}\metricsq{\setn{0}}{x}\psp(x)
    && \text{by $\ocsE(\rvX)$ result}
  \\&= \mathrlap{
       \sum_{x\in\omsH\setd\setn{0}}\brp{\frac{3}{2}}^2\frac{1}{6}
     = \seto{\omsH\setd\setn{0}}\brp{\frac{3}{2}}^2\frac{1}{6}
     = 6\brp{\frac{3}{2}}^2\frac{1}{6}
     = \frac{9}{4}
     }
\end{align*}
\end{proof}

\begin{figure}[h]%
  \gsize%
  \centering%
  %{%============================================================================
% Daniel J. Greenhoe
% LaTeX file
% spinner non-linear mappings
%============================================================================
\begin{pspicture}(-4.75,-2.35)(4.75,1.35)%
  %---------------------------------
  % options
  %---------------------------------
  \psset{%
    radius=1.25ex,
    %labelsep=2.5mm,
    linecolor=blue,%
    }%
  %---------------------------------
  % spinner graph
  %---------------------------------
  \rput(0,0){%\psset{unit=2\psunit}%
    \rput{ 210}(0,0){\rput(1,0){\Cnode[fillstyle=solid,fillcolor=snode](0,0){D6}}}%
    \rput{ 150}(0,0){\rput(1,0){\Cnode(0,0){D5}}}%
    \rput{  90}(0,0){\rput(1,0){\Cnode(0,0){D4}}}%
    \rput{  30}(0,0){\rput(1,0){\Cnode(0,0){D3}}}%
    \rput{ -30}(0,0){\rput(1,0){\Cnode(0,0){D2}}}%
    \rput{ -90}(0,0){\rput(1,0){\Cnode[fillstyle=solid,fillcolor=snode](0,0){D1}}}%
    \rput(0,0){$\ocsG$}%
    }
  \rput(D6){\circSix}% 
  \rput(D5){\circFive}%
  \rput(D4){\circFour}%
  \rput(D3){\circThree}%
  \rput(D2){\circTwo}% 
  \rput(D1){\circOne}% 
  %
  \ncline{D6}{D1}%
  \ncline{D5}{D6}%
  \ncline{D4}{D5}%
  \ncline{D3}{D4}%
  \ncline{D2}{D3}%
  \ncline{D1}{D2}%
  %
  \uput[-150](D6){$\frac{3}{10}$}
  \uput[ 150](D5){$\frac{1}{10}$}
  \uput[ 158](D4){$\frac{1}{10}$}
  \uput[  22](D3){$\frac{1}{10}$}
  \uput[ -45](D2){$\frac{1}{10}$}
  \uput[ -45](D1){$\frac{3}{10}$}
  %---------------------------------
  % Y mapping to isomorphic structure H
  %---------------------------------
  \rput(-3.5,0){%\psset{unit=2\psunit}%
    \rput{ 210}(0,0){\rput(1,0){\Cnode[fillstyle=solid,fillcolor=snode](0,0){X6}}}%
    \rput{ 150}(0,0){\rput(1,0){\Cnode(0,0){X5}}}%
    \rput{  90}(0,0){\rput(1,0){\Cnode(0,0){X4}}}%
    \rput{  30}(0,0){\rput(1,0){\Cnode(0,0){X3}}}%
    \rput{ -30}(0,0){\rput(1,0){\Cnode(0,0){X2}}}%
    \rput{ -90}(0,0){\rput(1,0){\Cnode[fillstyle=solid,fillcolor=snode](0,0){X1}}}%
    \rput(0,0){$\omsH$}%
    }
  \rput(X6){$6$}%
  \rput(X5){$5$}%
  \rput(X4){$4$}%
  \rput(X3){$3$}%
  \rput(X2){$2$}%
  \rput(X1){$1$}%
  %
  \ncline{X6}{X1}%
  \ncline{X5}{X6}%
  \ncline{X4}{X5}%
  \ncline{X3}{X4}%
  \ncline{X2}{X3}%
  \ncline{X1}{X2}%
  %
  %\uput[ 158](X6){$\frac{3}{10}$}
  %\uput[ 150](X5){$\frac{1}{10}$}
  %\uput[ 210](X4){$\frac{1}{10}$}
  %\uput[ -22](X3){$\frac{1}{10}$}
  %\uput[ -22](X2){$\frac{1}{10}$}
  %\uput[-158](X1){$\frac{3}{10}$}
  %
  \ncarc[arcangle=-22,linewidth=0.75pt,linecolor=red]{->}{D6}{X6}%
  \ncarc[arcangle= 22,linewidth=0.75pt,linecolor=red]{->}{D5}{X5}%
  \ncarc[arcangle= 22,linewidth=0.75pt,linecolor=red]{->}{D4}{X4}%
  \ncarc[arcangle=-22,linewidth=0.75pt,linecolor=red]{->}{D3}{X3}%
  \ncarc[arcangle=-22,linewidth=0.75pt,linecolor=red]{->}{D2}{X2}%
  \ncarc[arcangle= 22,linewidth=0.75pt,linecolor=red]{->}{D1}{X1}%
  %---------------------------------
  % Z mapping to continuous ring K
  %---------------------------------
  \rput(3.5,0){%\psset{unit=2\psunit}%
    \pscircle(0,0){1}%
    \rput{ 210}(0,0){\rput(1,0){\pnode(0,0){Y6}}}%
    \rput{ 150}(0,0){\rput(1,0){\pnode(0,0){Y5}}}%
    \rput{  90}(0,0){\rput(1,0){\pnode(0,0){Y4}}}%
    \rput{  30}(0,0){\rput(1,0){\pnode(0,0){Y3}}}%
    \rput{ -30}(0,0){\rput(1,0){\pnode(0,0){Y2}}}%
    \rput{ -90}(0,0){\rput(1,0){\pnode(0,0){Y1}}}%
    \rput{-120}(0,0){\rput(1,0){\pnode(0,0){Y16}}}%
    \rput(0,0){$\omsK$}%
    }
  \rput{ 210}(Y6){\psline(-0.1,0)(0.1,0)}%
  \rput{ 150}(Y5){\psline(-0.1,0)(0.1,0)}%
  \rput{  90}(Y4){\psline(-0.1,0)(0.1,0)}%
  \rput{  30}(Y3){\psline(-0.1,0)(0.1,0)}%
  \rput{ -30}(Y2){\psline(-0.1,0)(0.1,0)}%
  \rput{ -90}(Y1){\psline(-0.1,0)(0.1,0)}%
  %
  \uput[ 210](Y6){$6$}%
  \uput[ 150](Y5){$5$}%
  \uput[  90](Y4){$4$}%
  \uput[  30](Y3){$3$}%
  \uput[ -30](Y2){$2$}%
  \uput[ -90](Y1){$1$}%
  %
  \rput(Y16){\pscircle[fillstyle=solid,linecolor=snode,fillcolor=snode](0,0){1ex}}%
  %
  %\ncline{Y5}{Y6}%
  %\ncline{Y4}{Y5}
  %\ncline{Y3}{Y5}
  %\ncline{Y2}{Y3}
  %\ncline{Y1}{Y2}
  %\ncline{Y0}{Y1}
  %
  %\uput[  22](Y6){$\frac{3}{10}$}%
  %\uput[ 200](Y5){$\frac{1}{10}$}%
  %\uput[ 210](Y4){$\frac{1}{10}$}%
  %\uput[  22](Y3){$\frac{1}{10}$}%
  %\uput[ -45](Y2){$\frac{1}{10}$}%
  %\uput[ -22](Y1){$\frac{3}{10}$}%
  %
  \ncarc[arcangle= 22,linewidth=0.75pt,linecolor=green]{->}{D6}{Y6}%
  \ncarc[arcangle= 22,linewidth=0.75pt,linecolor=green]{->}{D5}{Y5}%
  \ncarc[arcangle= 22,linewidth=0.75pt,linecolor=green]{->}{D4}{Y4}%
  \ncarc[arcangle=-22,linewidth=0.75pt,linecolor=green]{->}{D3}{Y3}%
  \ncarc[arcangle=-22,linewidth=0.75pt,linecolor=green]{->}{D2}{Y2}%
  \ncarc[arcangle=-22,linewidth=0.75pt,linecolor=green]{->}{D1}{Y1}%
  %---------------------------------
  % real line
  %---------------------------------
  \rput(-3.5,-2){
    \psline{<->}(0.5,0)(6.5,0)%
    \multirput(1,0)(1,0){6}{\psline(0,-0.1)(0,0.1)}%
    \pnode(6,0){L6}%
    \pnode(5,0){L5}%
    \pnode(4,0){L4}%
    \pnode(3,0){L3}%
    \pnode(2,0){L2}%
    \pnode(1,0){L1}%
    \pscircle[fillstyle=solid,linecolor=snode,fillcolor=snode](3.5,0){1ex}%
    \pscircle[fillstyle=none,linecolor=red,fillcolor=red](3.5,0){1ex}%
    \uput[0](6.5,0){$\omsR$}%
    }
  %
  \uput[-90](L6){$6$}%
  \uput[-90](L5){$5$}%
  \uput[-90](L4){$4$}%
  \uput[-90](L3){$3$}%
  \uput[-90](L2){$2$}%
  \uput[-90](L1){$1$}%
  %
  \ncarc[arcangle= 22,linewidth=0.75pt,linecolor=purple]{->}{D6}{L6}%
  \ncarc[arcangle= 22,linewidth=0.75pt,linecolor=purple]{->}{D5}{L5}%
  \ncarc[arcangle= 22,linewidth=0.75pt,linecolor=purple]{->}{D4}{L4}%
  \ncarc[arcangle=-22,linewidth=0.75pt,linecolor=purple]{->}{D3}{L3}%
  \ncarc[arcangle=-22,linewidth=0.75pt,linecolor=purple]{->}{D2}{L2}%
  \ncarc[arcangle=-22,linewidth=0.75pt,linecolor=purple]{->}{D1}{L1}%
  %---------------------------------
  % labels
  %---------------------------------
  \rput(-0.8,-1.7){$\rvX(\cdot)$}%
  \rput(-1.75,0){$\rvY(\cdot)$}%
  \rput(1.75,0){$\rvZ(\cdot)$}%
  %\ncline[linestyle=dotted,nodesep=1pt]{->}{xzlabel}{xz}%
  %\ncline[linestyle=dotted,nodesep=1pt]{->}{ylabel}{y}%
\end{pspicture}%}%
  {\includegraphics{sto/graphics/spinnerXO6Ycircle.pdf}}%
  %\hfill%
  %{%============================================================================
% Daniel J. Greenhoe
% XeLaTeX file
%============================================================================
{\psset{yunit=2\psunit}%
\begin{pspicture}(-0.7,-0.25)(7.25,1)%
  \psset{%
    labelsep=3pt,
    linewidth=1pt,
    }%
  \psaxes[linecolor=axis,yAxis=false]{->}(0,0)(0,0)(7,1)% x axis
  \psaxes[linecolor=axis,xAxis=false]{->}(0,0)(0,0)(7,1)% y axis
  \psline(0,0.3)(0.5,0.25)(1,0.3)(3,0.9)(3.5,0.75)(4,0.9)(6,0.3)(6.5,0.25)(7,0.3)%
  %
  \psline[linestyle=dotted,linecolor=red](0.5,0.25)(0.5,0)%
  \psline[linestyle=dotted,linecolor=red](3,0.9)(3,0)%
  \psline[linestyle=dotted,linecolor=red](3.5,0.75)(3.5,0)%
  \psline[linestyle=dotted,linecolor=red](4,0.9)(4,0)%
  \psline[linestyle=dotted,linecolor=red](6.5,0.25)(6.5,0)%
  %
  \psline[linestyle=dotted,linecolor=red](0,0.9)(4,0.9)%
  \psline[linestyle=dotted,linecolor=red](0,0.75)(3.5,0.75)%
  \psline[linestyle=dotted,linecolor=red](0,0.3)(7,0.3)%
  \psline[linestyle=dotted,linecolor=red](0,0.25)(6.5,0.25)%
  \uput[0]{0}(7,0){$x$}%
  \uput[-90]{0}(0.5,0){$0.5$}%
  \uput[-90]{0}(3.5,0){$3.5$}%
  \uput[-90]{0}(6.5,0){$0.5$}%
  %
  \uput[170]{0}(0,0.9){$0.9$}%
  \uput[190]{0}(0,0.75){$0.75$}%
  \uput[155]{0}(0,0.3){$0.3$}%
  \uput[205]{0}(0,0.25){$0.25$}%
  \rput[b](3.5,0.3){$\ds\ff(x)\eqd\max_{y\in\omsK}\metric{x}{y}\psp(y)$}%
\end{pspicture}}%
}%
  \caption{weighted spinner mappings \xref{ex:wspinner_xyz}\label{fig:spinnerXO6Ycircle}}%
\end{figure}
%---------------------------------------
%\begin{minipage}{\tw-65mm}%
\begin{example}[\exmd{weighted spinner mappings}]
\label{ex:wspinner_xyz}
%---------------------------------------
Let $\ocsG$ be \structe{weighted spinner outcome subspace} \xref{ex:wspinner}
with random variable mappings as illustrated in \prefpp{fig:spinnerXO6Ycircle}.
This yields the following statistics:
\\\indentx$\begin{array}{>{\gsizes}Mrcl lcccl}
  geometry of $\ocsG$:                                          & \ocscen (\ocsG) &=& \mc{4}{l}{\setn{\circOne,\circSix}} \\
  traditional statistics on real line $\omsR$:                  & \pE  (\rvX)     &=& 3.5          & \ocsVar(\rvW;\pE)   &=& \frac{17}{4}&\approx& 4.25\\
  outcome subspace statistics on real line $\omsR$:             & \ocsE(\rvX)     &=& \setn{3.5}   & \ocsVar(\rvW;\ocsE) &=& \frac{17}{4}&\approx& 4.25\\
  outcome subspace statistics on  isomorphic structure $\ocsH$: & \ocsE(\rvY)     &=& \setn{1,\,6} & \ocsVar(\rvY;\ocsE) &=& \frac{5}{3} &\approx& 1.67\\
  outcome subspace statistics on  continuous structure $\ocsK$: & \ocsE(\rvZ)     &=& \setn{0.5}   & \ocsVar(\rvZ;\ocsE) &=& \frac{37}{20}&=&      1.85
\end{array}$\\
Note that based on the variance values, the statistic $\ocsE(\rvZ)$ on the continuous ring $\ocsK$
is arguably a much better statistic than $\ocsE(\rvX)$ on the (continuous) real line $\omsR$.
\end{example}
\begin{proof}
    \begin{align*}
      \ocscen(\ocsG)
      %  &\eqd \argmin_{x\in\omsH}\max_{y\in\omsH}\ocsmom(x,y)
      %  &&\text{by definition of $\ocscen$ \xref{def:ocscen}}
      %\\&\eqd \argmin_{x\in\omsH}\max_{y\in\omsH}\metric{x}{y}\psp(y)
      %  &&\text{by definition of $\ocsmom$ \xref{def:ocsmom}}
        &= \setn{1,6}
        && \text{by \exme{weighted spinner outcome subspace} example \xref{ex:wspinner}}
      %\\
      %\ocscena(\ocsG)
      %%  &\eqd \argmin_{x\in\omsH}\max_{y\in\omsH}\ocsmom(x,y)
      %%  &&\text{by definition of $\ocscena$ \xref{def:ocscena}}
      %%\\&\eqd \argmin_{x\in\omsH}\max_{y\in\omsH}\metric{x}{y}\psp(y)
      %%  &&\text{by definition of $\ocsmom$ \xref{def:ocsmom}}
      %  &= \setn{1,6}
      %  &&\text{by \prefpp{ex:wspinner}}
      %\\
      %\ocsVar(\rvY)
      %%  &\eqd \sum_{x\in\omsH}\ocsmom_2(C,x)
      %%  &&\text{by definition of $\ocsVar$ \xref{def:ocsVar}}
      %  &= \frac{1}{10}\brs{19}
      %   = 1.9
      %  &&\text{by \prefpp{ex:wspinner}}
      \\
      \pE(\rvX) 
        &\eqd \sum_{x\in\Z} x\psp(x)
      \\&= \mathrlap{1\times\frac{3}{10}+2\times\frac{1}{10}+3\times\frac{1}{10}+4\times\frac{1}{10}+5\times\frac{1}{10}+6\times\frac{1}{10}}
      \\&= \mathrlap{\frac{1}{10}\brp{1\times3 + 2 + 3 + 4 + 5 + 6\times3}
         =    \frac{35}{10}
         =    \frac{7}{2}
         =    3.5}
      \\
      \ocsVar(\rvX;\pE)
        &= \pVar(\rvX)
        && \text{by \prefp{thm:ocsVar}}
      \\&\eqd \sum_{x\in\Z} \brs{x-\pE(\rvX)}^2\psp(x)
        && \text{by definition of $\pVar$ \xref{def:pVar}}
      \\&= \mathrlap{\brp{1-\frac{7}{2}}^2\frac{3}{10}+\brp{2-\frac{7}{2}}^2\frac{1}{10}+\brp{3-\frac{7}{2}}^2\frac{1}{10}+\brp{4-\frac{7}{2}}^2\frac{1}{10}+\brp{5-\frac{7}{2}}^2\frac{1}{10}+\brp{6-\frac{7}{2}}^2\frac{3}{10}}
      \\&= \mathrlap{\frac{1}{10}\brs{\brp{-\frac{5}{2}}^2\times3+\brp{-\frac{3}{2}}^2+\brp{-\frac{1}{2}}^2+\brp{\frac{1}{2}}^2+\brp{\frac{3}{2}}^2+\brp{\frac{5}{2}}^2\times3}}
      \\&= \mathrlap{\frac{1}{40}\brs{75+9+1+1+9+75}
         = \frac{170}{40}
         = \frac{17}{4}
         = 4.25}
      \\
      \ocsE(\rvX)
        &= \pE(\rvX)
        && \text{because on \structe{real line}, $\psp$ is \prope{symmetric}, and by \prefp{thm:pEocsE}}
      \\&= \setn{3.5}
        && \text{by $\pE(\rvX)$ result}
      %  &\eqd \argmin_{x\in\R}\max_{y\in\R}\ocsmom(x,y)
      %  && \text{by definition of $\pE$ \xref{def:ocsE}}
      %\\&\eqd \argmin_{x\in\R}\max_{y\in\R}\metric{x}{y}\psp(y)
      %  && \text{by definition of $\pE$ \xref{def:ocsmom}}
      %\\&\eqd \argmin_{x\in\R}\max_{y\in\R}\abs{x-y}\psp(y)
      %  && \text{by definition usual metric on real line}
      %\\&= \argmin_{x\in\R}\brbl{\begin{array}{lM}
      %       \abs{x-1}\psp(1) & for $x\ge3.5$\\
      %       \abs{x-6}\psp(6) & for $x<3.5$
      %     \end{array}}
      %  && \text{\gsize\centering\psset{unit=6mm}{%============================================================================
% Daniel J. Greenhoe
% XeLaTeX file
%============================================================================
{%\psset{yunit=2\psunit}%
\begin{pspicture}(-1,0)(8,2)%
  \psset{%
    labelsep=1pt,
    linewidth=1pt,
    }%
  \psaxes[linecolor=axis,yAxis=false,labels=none]{->}(0,0)(0,0)(7.5,2.5)% x axis
  \psaxes[linecolor=axis,xAxis=false,labels=none]{->}(0,0)(0,0)(7.5,2.5)% y axis
  \psline(0,1.8)(3.5,0.75)(7,1.8)%
  %
  \psline[linestyle=dotted,linecolor=red](3.5,0.75)(3.5,0)%
  \psline[linestyle=dotted,linecolor=red](0,0.75)(3.5,0.75)%
  %
  \uput[0]{0}(8,0){$x$}%
  \uput[-90]{0}(3.5,0){$3.5$}%
  \uput[180]{0}(0,0.75){$0.75$}%
  \uput[180]{0}(0,1.8){$1.8$}%
  \rput[t](3.5,2){$\ds\ff(x)\eqd\max_{y\in\omsR}\abs{x-y}\psp(y)$}%
\end{pspicture}}%
}}
      %%\\&= \argmin_{x\in\R}\brbl{\begin{array}{lM}
      %%       \abs{x-1}\frac{3}{10} & for $x\ge3.5$\\
      %%       \abs{x-6}\frac{3}{10} & for $x<3.5$
      %%     \end{array}}
      %\\&= \setn{3.5}
      %  && \text{because expression is minimized at argument $x=3.5$}
      \\
      \ocsVar(\rvX;\ocsE)
        &= \ocsVar(\rvX;\pE)
        && \text{by $\pE(\rvX)$ and $\ocsE(\rvX)$ results}
      \\&= \frac{17}{4}= 4.25
        && \text{by $\ocsVar(\rvX;\pE)$ result}
      \\
      \ocsE(\rvY)
        &= \rvY\brs{\ocscen(\ocsG)}
        && \text{because $\ocsG$ and $\omsH$ are \prope{isomorphic} under $\rvY$}
      \\&= \rvY\brs{\setn{\circOne,\circSix}}
        && \text{by $\ocscen(\ocsG)$ result}
      \\&= \setn{1,6}
        && \text{by definition of $\rvY$}
      \\
      \ocsVar(\rvY;\ocsE)
        &= \ocsVaro(\ocsG)
        && \text{because $\ocsG$ and $\omsH$ are \prope{isomorphic} under $\rvY$}
      \\&= \frac{5}{3} \approx 1.667
        && \text{by \exme{weighted spinner outcome subspace} example \xref{ex:wspinner}}
      \\
      \ocsE(\rvZ)
        &\eqd \argmin_{x\in\omsK}\max_{y\in\omsK}\metric{x}{y}\psp(y)
        && \text{by definition of $\ocsE$ \xref{def:ocsE}}
      \\&= 0.5
        %&& \text{\gsize\centering\psset{unit=7.5mm}{%============================================================================
% Daniel J. Greenhoe
% XeLaTeX file
%============================================================================
{\psset{yunit=2\psunit}%
\begin{pspicture}(-0.7,-0.25)(7.25,1)%
  \psset{%
    labelsep=3pt,
    linewidth=1pt,
    }%
  \psaxes[linecolor=axis,yAxis=false]{->}(0,0)(0,0)(7,1)% x axis
  \psaxes[linecolor=axis,xAxis=false]{->}(0,0)(0,0)(7,1)% y axis
  \psline(0,0.3)(0.5,0.25)(1,0.3)(3,0.9)(3.5,0.75)(4,0.9)(6,0.3)(6.5,0.25)(7,0.3)%
  %
  \psline[linestyle=dotted,linecolor=red](0.5,0.25)(0.5,0)%
  \psline[linestyle=dotted,linecolor=red](3,0.9)(3,0)%
  \psline[linestyle=dotted,linecolor=red](3.5,0.75)(3.5,0)%
  \psline[linestyle=dotted,linecolor=red](4,0.9)(4,0)%
  \psline[linestyle=dotted,linecolor=red](6.5,0.25)(6.5,0)%
  %
  \psline[linestyle=dotted,linecolor=red](0,0.9)(4,0.9)%
  \psline[linestyle=dotted,linecolor=red](0,0.75)(3.5,0.75)%
  \psline[linestyle=dotted,linecolor=red](0,0.3)(7,0.3)%
  \psline[linestyle=dotted,linecolor=red](0,0.25)(6.5,0.25)%
  \uput[0]{0}(7,0){$x$}%
  \uput[-90]{0}(0.5,0){$0.5$}%
  \uput[-90]{0}(3.5,0){$3.5$}%
  \uput[-90]{0}(6.5,0){$0.5$}%
  %
  \uput[170]{0}(0,0.9){$0.9$}%
  \uput[190]{0}(0,0.75){$0.75$}%
  \uput[155]{0}(0,0.3){$0.3$}%
  \uput[205]{0}(0,0.25){$0.25$}%
  \rput[b](3.5,0.3){$\ds\ff(x)\eqd\max_{y\in\omsK}\metric{x}{y}\psp(y)$}%
\end{pspicture}}%
}}
        && \text{\gsize\centering\psset{unit=7.5mm}{\includegraphics{sto/graphics/wspinnermax.pdf}}}
       %&& \text{(see plot of $\ds\ff(x)\eqd\max_{y\in\omsK}\metric{x}{y}\psp(y)$ in \prefp{fig:spinnerXO6Ycircle}}
      \\
      \ocsVar(\rvZ;\ocsE)
        &\eqd \sum_{x\in\ocsK} \metricsq{\ocsE(\rvZ)}{x}\psp(x)
        && \text{by definition of $\ocsVar$ \xref{def:ocsVar}}
      \\&= \sum_{x\in\ocsK} \metricsq{\frac{1}{2}}{x}\psp(x)
        && \text{by $\ocsE(\rvZ)$ result}
      \\&= \mathrlap{
           \brp{\frac{1}{2}}^2\frac{3}{10}
          +\brp{\frac{3}{2}}^2\frac{1}{10}
          +\brp{\frac{5}{2}}^2\frac{1}{10}
          +\brp{\frac{5}{2}}^2\frac{1}{10}
          +\brp{\frac{3}{2}}^2\frac{1}{10}
          +\brp{\frac{1}{2}}^2\frac{3}{10}}
      \\&= \mathrlap{\frac{1}{40}\brp{3+9+25+25+9+3} = \frac{74}{40} = \frac{37}{20} = 1.85}
    \end{align*}
\end{proof}


%=======================================
\subsubsection{Pseudo-random number generator (PRNG) examples}
%=======================================
\begin{figure}[h]
  \gsize%
  \centering%
  %%============================================================================
% Daniel J. Greenhoe
% LaTeX file
% linear congruential (LCG) pseudo-random number generator (PRNG) mappings
% x_{n+1} = (7x_n+5)mod 9
% y_{n+1} = (y_n+2)mod 5
%============================================================================
\begin{pspicture}(-2.25,-3.8)(11.5,3.8)%
  %---------------------------------
  % options
  %---------------------------------
  \psset{%
    radius=1.25ex,
    labelsep=2.5mm,
    linecolor=blue,%
    }%
  %---------------------------------
  % LCG PRNG graph G
  % x_{n+1} = (7x_n+5)mod 9
  %   n  0   1   2   3   4   5   6   7   8  ;  9
  % x_n  1   3   8   7   0   5   4   6   2  ;  1
  %---------------------------------
  \rput(0,0){\psset{unit=1.5\psunit}%
    \rput{   0}(0,0){\rput(1,0){\Cnode[fillstyle=solid,fillcolor=snode](0,0){G0}}}%
    \rput{  40}(0,0){\rput(1,0){\Cnode[fillstyle=solid,fillcolor=snode](0,0){G1}}}%
    \rput{  80}(0,0){\rput(1,0){\Cnode[fillstyle=solid,fillcolor=snode](0,0){G2}}}%
    \rput{ 120}(0,0){\rput(1,0){\Cnode[fillstyle=solid,fillcolor=snode](0,0){G3}}}%
    \rput{ 160}(0,0){\rput(1,0){\Cnode[fillstyle=solid,fillcolor=snode](0,0){G4}}}%
    \rput{ 200}(0,0){\rput(1,0){\Cnode[fillstyle=solid,fillcolor=snode](0,0){G5}}}%
    \rput{ 240}(0,0){\rput(1,0){\Cnode[fillstyle=solid,fillcolor=snode](0,0){G6}}}%
    \rput{ 280}(0,0){\rput(1,0){\Cnode[fillstyle=solid,fillcolor=snode](0,0){G7}}}%
    \rput{ 320}(0,0){\rput(1,0){\Cnode[fillstyle=solid,fillcolor=snode](0,0){G8}}}%
    \rput(0,0){$\ocsG_9$}%
    }%
  \rput(G8){$8$}%
  \rput(G7){$7$}%
  \rput(G6){$6$}%
  \rput(G5){$5$}%
  \rput(G4){$4$}%
  \rput(G3){$3$}%
  \rput(G2){$2$}%
  \rput(G1){$1$}%
  \rput(G0){$0$}%
  %          
  \ncline{G8}{G0}%
  \ncline{G7}{G8}%
  \ncline{G6}{G7}%
  \ncline{G5}{G6}%
  \ncline{G4}{G5}%
  \ncline{G3}{G4}%
  \ncline{G2}{G3}%
  \ncline{G1}{G2}%
  \ncline{G0}{G1}%
  %
  \uput[  0](G0){$\frac{1}{9}$}
  \uput[ 40](G1){$\frac{1}{9}$}
  \uput[ 80](G2){$\frac{1}{9}$}
  \uput[120](G3){$\frac{1}{9}$}
  \uput[160](G4){$\frac{1}{9}$}
  \uput[200](G5){$\frac{1}{9}$}
  \uput[240](G6){$\frac{1}{9}$}
  \uput[280](G7){$\frac{1}{9}$}
  \uput[320](G8){$\frac{1}{9}$}
  %
  %
  %---------------------------------
  %---------------------------------
  %---------------------------------
  \rput(5,0){% shaped outcomes spaces
  %---------------------------------
  %---------------------------------
  %---------------------------------
  %
  %
  %---------------------------------
  % distribution shaping mapping y_n=s(x_n) to 5 element set
  %   n  0   1   2   3   4   5   6   7   8  ;  9
  % x_n  1   3   8   7   0   5   4   6   2  ;  1
  % y_n  1   3   0   2   4 ; 0   3   4   4  ;  1
  %---------------------------------
  \rput( 0,0){%\psset{unit=2\psunit}%
    \rput{288}(0,0){\rput(1,0){\Cnode[fillstyle=solid,fillcolor=snode](0,0){S4}}}%
    \rput{216}(0,0){\rput(1,0){\Cnode(0,0){S3}}}%
    \rput{144}(0,0){\rput(1,0){\Cnode(0,0){S2}}}%
    \rput{ 72}(0,0){\rput(1,0){\Cnode(0,0){S1}}}%
    \rput{  0}(0,0){\rput(1,0){\Cnode(0,0){S0}}}%
    \rput(0,0){$\ocsG$}
    }
  \rput(S4){$4$}%
  \rput(S3){$3$}%
  \rput(S2){$2$}%
  \rput(S1){$1$}%
  \rput(S0){$0$}%
  %
  \ncline{S4}{S0}%
  \ncline{S3}{S4}%
  \ncline{S2}{S3}%
  \ncline{S1}{S2}%
  \ncline{S0}{S1}%
  %
  \uput[288](S4){$\frac{3}{9}$}
  \uput[216](S3){$\frac{2}{9}$}
  \uput[144](S2){$\frac{1}{9}$}
  \uput[ 72](S1){$\frac{1}{9}$}
  \uput[  0](S0){$\frac{2}{9}$}
  %
  \ncarc[arcangle=-22,linewidth=0.75pt,linecolor=blue]{->}{G8}{S0}%
  \ncarc[arcangle=-45,linewidth=0.75pt,linecolor=blue]{->}{G7}{S2}%
  \ncarc[arcangle=-45,linewidth=0.75pt,linecolor=blue]{->}{G6}{S4}%
  \ncarc[arcangle=-22,linewidth=0.75pt,linecolor=blue]{->}{G5}{S0}%
  \ncarc[arcangle=-67,linewidth=0.75pt,linecolor=blue]{->}{G4}{S3}%
  \ncarc[arcangle= 45,linewidth=0.75pt,linecolor=blue]{->}{G3}{S3}%
  \ncarc[arcangle=-22,linewidth=0.75pt,linecolor=blue]{->}{G2}{S4}%
  \ncarc[arcangle= 22,linewidth=0.75pt,linecolor=blue]{->}{G1}{S1}%
  \ncarc[arcangle= 37,linewidth=0.75pt,linecolor=blue]{->}{G0}{S4}%
  %---------------------------------
  % random variable mapping Y from S to integer line 
  %---------------------------------
  \rput(0,3){%\psset{unit=0.75\psunit}%
    \pnode(3,0){YB}%
    \Cnode(2,0){Y4}%
    \Cnode[fillstyle=solid,fillcolor=snode](1,0){Y3}%
    \Cnode[fillstyle=solid,fillcolor=snode](0,0){Y2}%
    \Cnode(-1,0){Y1}%
    \Cnode(-2,0){Y0}%
    \pnode(-3,0){YA}%
    }
  \rput(Y4){$4$}%
  \rput(Y3){$3$}%
  \rput(Y2){$2$}%
  \rput(Y1){$1$}%
  \rput(Y0){$0$}%
  \uput[0](YB){$\omsZ$}%
  %
  \ncline[linestyle=dotted]{Y4}{YB}%
  \ncline{Y3}{Y4}%
  \ncline{Y2}{Y3}%
  \ncline{Y1}{Y2}%
  \ncline{Y0}{Y1}%
  \ncline[linestyle=dotted]{Y0}{YA}%
  %
  \uput[90](Y4){$\frac{3}{9}$}
  \uput[90](Y3){$\frac{2}{9}$}
  \uput[90](Y2){$\frac{1}{9}$}
  \uput[90](Y1){$\frac{1}{9}$}
  \uput[90](Y0){$\frac{2}{9}$}
  %
  \ncarc[arcangle=-45,linewidth=0.75pt,linecolor=red]{->}{S4}{Y4}%
  \ncarc[arcangle= 22,linewidth=0.75pt,linecolor=red]{->}{S3}{Y3}%
  \ncarc[arcangle= 45,linewidth=0.75pt,linecolor=red]{->}{S2}{Y2}%
  \ncarc[arcangle=-22,linewidth=0.75pt,linecolor=red]{->}{S1}{Y1}%
  \ncarc[arcangle= 22,linewidth=0.75pt,linecolor=red]{->}{S0}{Y0}%
  %---------------------------------
  % random variable mapping Z from S to ring-like structure 
  %---------------------------------
  \rput( 5,0){%\psset{unit=2\psunit}%
    \rput{288}(0,0){\rput(1,0){\Cnode[fillstyle=solid,fillcolor=snode](0,0){Z4}}}%
    \rput{216}(0,0){\rput(1,0){\Cnode(0,0){Z3}}}%
    \rput{144}(0,0){\rput(1,0){\Cnode(0,0){Z2}}}%
    \rput{ 72}(0,0){\rput(1,0){\Cnode(0,0){Z1}}}%
    \rput{  0}(0,0){\rput(1,0){\Cnode(0,0){Z0}}}%
    \rput(0,0){$\omsH$}%
    }
  \rput(Z4){$4$}%
  \rput(Z3){$3$}%
  \rput(Z2){$2$}%
  \rput(Z1){$1$}%
  \rput(Z0){$0$}%
  %
  \ncline{Z4}{Z0}%
  \ncline{Z3}{Z4}%
  \ncline{Z2}{Z3}%
  \ncline{Z1}{Z2}%
  \ncline{Z0}{Z1}%
  %
  \uput[288](Z4){$\frac{3}{9}$}
  \uput[216](Z3){$\frac{2}{9}$}
  \uput[144](Z2){$\frac{1}{9}$}
  \uput[ 72](Z1){$\frac{1}{9}$}
  \uput[  0](Z0){$\frac{2}{9}$}
  %
  \ncarc[arcangle=-22,linewidth=0.75pt,linecolor=green]{->}{S4}{Z4}%
  \ncarc[arcangle=-45,linewidth=0.75pt,linecolor=green]{->}{S3}{Z3}%
  \ncarc[arcangle=-10,linewidth=0.75pt,linecolor=green]{->}{S2}{Z2}%
  \ncarc[arcangle= 10,linewidth=0.75pt,linecolor=green]{->}{S1}{Z1}%
  \ncarc[arcangle=-10,linewidth=0.75pt,linecolor=green]{->}{S0}{Z0}%
  %---------------------------------
  % random variable mapping X from S to real line
  %---------------------------------
  \rput(0,-3){%\psset{unit=0.5\psunit}%
    \psline{<->}(-2.5,0)(2.5,0)%
    \multirput(-2,0)(1,0){5}{\psline(0,-0.1)(0,0.1)}%
    \pnode(2.5,0){XB}%
    \pnode( 2,0){X4}%
    \pnode( 1,0){X3}%
    \pnode(0.4,0){X240}%
    \pnode(0.33,0){X233}%
    \pnode( 0,0){X2}%
    \pnode(-1,0){X1}%
    \pnode(-2,0){X0}%
    \pnode(-2.5,0){XA}%
    }
  %
  \uput[0](XB){$\omsR$}%
  \rput(X240){\pscircle[fillstyle=solid,linecolor=snode,fillcolor=snode](0,0){1ex}}%
  \rput(X233){\pscircle[fillstyle=none,linecolor=red,fillcolor=red](0,0){1ex}}%
  %
  \uput{1.5mm}[-90](X4){$\frac{3}{9}$}%
  \uput{1.5mm}[-90](X3){$\frac{2}{9}$}%
  \uput{1.5mm}[-90](X2){$\frac{1}{9}$}%
  \uput{1.5mm}[-90](X1){$\frac{1}{9}$}%
  \uput{1.5mm}[-90](X0){$\frac{2}{9}$}%
  %
  \uput{1.5mm}[90](X4){$4$}%
  \uput{1.5mm}[90](X3){$3$}%
  \uput{1.5mm}[90](X2){$2$}%
  \uput{1.5mm}[90](X1){$1$}%
  \uput{1.5mm}[90](X0){$0$}%
  %
  \ncarc[arcangle= 22,linewidth=0.75pt,linecolor=purple]{->}{S4}{X4}%
  \ncarc[arcangle=-22,linewidth=0.75pt,linecolor=purple]{->}{S3}{X3}%
  \ncarc[arcangle=-22,linewidth=0.75pt,linecolor=purple]{->}{S2}{X2}%
  \ncarc[arcangle= 45,linewidth=0.75pt,linecolor=purple]{->}{S1}{X1}%
  \ncarc[arcangle=-22,linewidth=0.75pt,linecolor=purple]{->}{S0}{X0}%
  %---------------------------------
  % labels
  %---------------------------------
  \rput(0,1.9){$\rvY(\cdot)$}%
  \rput(0,-1.9){$\rvX(\cdot)$}%
  \rput(2.75,0){$\rvZ(\cdot)$}%
  }%
  \rput(2.75,0){$\fs(x_n)$}%
\end{pspicture}%%
  \includegraphics{sto/graphics/lcg7x1m9_XYZ.pdf}%
  \caption{LCG mappings to \prope{linear} ($\rvX$), non-linear discrete ($\rvY$)
  and non-linear continuous ($\rvZ$) ordered metric spaces \xref{ex:lcg7x1m9_xyz}\label{fig:lcg7x1m9_xyz}}
\end{figure}
%---------------------------------------
%\begin{minipage}{\tw-65mm}%
\begin{example}[\exmd{LCG mappings, standard ordering}]
\label{ex:lcg7x1m9_xyz}\mbox{}\\
%---------------------------------------
The equation $x_{n+1}=(7x_n+5)\mod9$ with $x_0=1$ is a \structe{linear congruential} (LCG) 
\structe{pseudo-random number generator} (PRNG)
that has \prope{full period}\footnote{
  \citeP{hull1962},
  \citerpgc{kennedy1980}{137}{0824768981}{Theorem 6.1},
  \citerpgc{severance2001}{86}{0471496944}{Hull-Dobell Theorem}
  }
of 9 values.
These 9 values can be mapped, using a \prope{surjective} \xref{def:ftypes} function $\fs\in\clF{\ocsG_9}{\ocsG}$ 
to the 5 element set $\setn{0,1,2,3,4}$ to ``shape" the distribution from a \prope{uniform} distribution to 
\prope{non-uniform}:\footnote{The sequence $\seqn{1,3,0,2,4,\,\,1,3,0,2,4,\,\,1,3,\cdots}$
is generated by the equation $y_{n+1}=(y_n+2)\mod5$ with $y_0=1$}.
\\\indentx$\begin{array}{r|*{9}{c}|*{4}{c}}
    n             & 0 & 1 & 2 & 3 & 4 & 5 & 6 & 7 & 8    & 9 & 10 & 11 & \cdots\\\hline 
  x_n             & 1 & 3 & 8 & 7 & 0 & 5 & 4 & 6 & 2    & 1 &  3 &  8 & \cdots\\
  y_n\eqd\fs(x_n) & 1 & 3 & 0 & 2 & 4 & 1 & 3 & 4 & 4    & 1 &  3 &  0 & \cdots%\\\hline
\end{array}$\\
Let $\ocsG$ be the \structe{outcome subspace} and 
$\rvX$, $\rvY$, and $\rvZ$ be the \fncte{outcome random variables} 
illustrated in \prefpp{fig:lcg7x1m9_xyz}.
%Let $\rvX$ be a random variable from $\ocsG$ to the \structe{real line} \xref{def:Rline},
%    $\rvY$    a random variable from $\ocsG$ to the \structe{integer line} \xref{def:Zline},
%and $\rvZ$    a random variable from $\ocsG$ to the 5 element ring, as illustrated in \pref{fig:lcg7x1m9}.
This yields the following statistics:
\\$\quad\begin{array}{>{\gsizes}Mrcc rcccl}
  geometry of $\ocsG_9$:                                & \ocscen (\ocsG) &=& \mc{4}{l}{\setn{0,1,2,3,4,5,6,7,8}}  \\
  geometry of $\ocsG$:                                  & \ocscen (\ocsG) &=& \setn{4}  \\
  traditional statistics on real line:                  & \pE  (\rvX)     &=& \frac{7}{3}\approx2.333  & \ocsVar(\rvX;\pE)   &=& \frac{22}{9}    &\approx& 2.444\\
  outcome subspace statistics on real line:             & \ocsE (\rvX)    &=& \setn{\frac{12}{5}=2.4}  & \ocsVar(\rvX;\ocsE) &=& \frac{551}{225} &\approx& 2.449\\
  outcome subspace statistics on integer line:          & \ocsE (\rvY)    &=& \setn{2,\,3}             & \ocsVar(\rvY;\ocsE) &=& \frac{16}{9}     &\approx& 1.778 \\
  outcome subspace statistics on isomorphic structure:  & \ocsE (\rvZ)    &=& \setn{4}                 & \ocsVar(\rvZ;\ocsE) &=& \frac{4}{3}     &\approx& 1.333
\end{array}$\\
Note that unlike the statistics $\pE(\rvX)$ and $\ocsE(\rvX)$ on the \structe{real line},
the statistic $\ocsE(\rvZ)$ on the \prope{isomorphic} structure $\ocsK$ yields the \prope{maximally likely} result,
and a much smaller variance as well.
\end{example}
%\end{minipage}\hfill%
%\begin{tabular}{c}
%  \gsize%
%  %\psset{unit=5mm}%
%  {%============================================================================
% Daniel J. Greenhoe
% LaTeX file
% discrete metric real dice mapping to linearly ordered O6c
%============================================================================
{%\psset{unit=0.5\psunit}%
\begin{pspicture}(-3.25,-1.5)(3.25,1.5)%
  %---------------------------------
  % options
  %---------------------------------
  \psset{%
    radius=1.25ex,
    labelsep=2.5mm,
    linecolor=blue,%
    }%
  %---------------------------------
  % dice graph
  %---------------------------------
  \rput(-1.75,0){%\psset{unit=2\psunit}%
    \Cnode[fillstyle=solid,fillcolor=snode](-0.8660,-0.5){D4}%
    \Cnode(-0.8660,0.5){D5}%
    \Cnode(0,1){D6}%
    \Cnode(0.8660,0.5){D3}%
    \Cnode(0.8660,-0.5){D2}%
    \Cnode(0,-1){D1}%
    }
  \rput(D6){$\diceF$}%
  \rput(D5){$\diceE$}%
  \rput(D4){$\diceD$}%
  \rput(D3){$\diceC$}%
  \rput(D2){$\diceB$}%
  \rput(D1){$\diceA$}%
  %
  \ncline{D5}{D6}%
  \ncline{D4}{D5}\ncline{D4}{D6}%
  \ncline{D3}{D5}\ncline{D3}{D6}%
  \ncline{D2}{D3}\ncline{D2}{D4}\ncline{D2}{D6}%
  \ncline{D1}{D2}\ncline{D1}{D3}\ncline{D1}{D4}\ncline{D1}{D5}%
  %
  \uput[ 158](D6){$\frac{1}{30}$}
  \uput[ 150](D5){$\frac{1}{50}$}
  \uput[ 210](D4){$\frac{3}{5}$}
  \uput[  22](D3){$\frac{1}{30}$}
  \uput[ -45](D2){$\frac{1}{20}$}
  \uput[-158](D1){$\frac{1}{10}$}
  %---------------------------------
  % range graph
  %---------------------------------
  \rput(1.75,0){%\psset{unit=2\psunit}%
    \Cnode[fillstyle=solid,fillcolor=snode](-0.8660,-0.5){E4}%
    \Cnode(-0.8660,0.5){E5}%
    \Cnode(0,1){E6}%
    \Cnode(0.8660,0.5){E3}%
    \Cnode(0.8660,-0.5){E2}%
    \Cnode(0,-1){E1}%
    \Cnode(0,0){E0}%
    }
  \rput(E6){$6$}%
  \rput(E5){$5$}%
  \rput(E4){$4$}%
  \rput(E3){$3$}%
  \rput(E2){$2$}%
  \rput(E1){$1$}%
  \rput(E0){$0$}%
  %
  \ncline{E5}{E6}%
  \ncline{E4}{E5}\ncline{E4}{E6}%
  \ncline{E3}{E5}\ncline{E3}{E6}%
  \ncline{E2}{E3}\ncline{E2}{E4}\ncline{E2}{E6}%
  \ncline{E1}{E2}\ncline{E1}{E3}\ncline{E1}{E4}\ncline{E1}{E5}%
  \ncline{E0}{E1}\ncline{E0}{E2}\ncline{E0}{E3}\ncline{E0}{E4}\ncline{E0}{E5}\ncline{E0}{E6}%
  %
  \uput[ 158](E6){$\frac{1}{30}$}
  \uput[ 150](E5){$\frac{1}{50}$}
  \uput[ 210](E4){$\frac{3}{5}$}
  \uput[  22](E3){$\frac{1}{30}$}
  \uput[ -45](E2){$\frac{1}{20}$}
  \uput[-158](E1){$\frac{1}{10}$}
  \uput[  0](E0){$\sfrac{0}{6}$}
  %---------------------------------
  % mapping from die to O6c
  %---------------------------------
  \ncarc[arcangle= 22,linewidth=0.75pt,linecolor=red]{->}{D6}{E6}%
  \ncarc[arcangle= 22,linewidth=0.75pt,linecolor=red]{->}{D5}{E5}%
  \ncarc[arcangle= 22,linewidth=0.75pt,linecolor=red]{->}{D4}{E4}%
  \ncarc[arcangle=-22,linewidth=0.75pt,linecolor=red]{->}{D3}{E3}%
  \ncarc[arcangle=-22,linewidth=0.75pt,linecolor=red]{->}{D2}{E2}%
  \ncarc[arcangle=-22,linewidth=0.75pt,linecolor=red]{->}{D1}{E1}%
  %---------------------------------
  % labels
  %---------------------------------
  \rput(0,0){$\rvX(\cdot)$}%
  %\ncline[linestyle=dotted,nodesep=1pt]{->}{xzlabel}{xz}%
  %\ncline[linestyle=dotted,nodesep=1pt]{->}{ylabel}{y}%
\end{pspicture}
}%}%
%\end{tabular}
\begin{proof}
    \begin{align*}
      \ocscen(\ocsG_9)
        &\eqd \argmin_{x\in\ocsG_9}\max_{y\in\ocsG_9}\metric{x}{y}\psp(y)
        &&\text{by definition of $\ocscen$ \xref{def:ocscen}}
      \\&= \argmin_{x\in\ocsG_9}\max_{y\in\ocsG_9}\metric{x}{y}\frac{1}{9}
        &&\text{by definition of $\ocsG_9$}
      \\&= \argmin_{x\in\ocsG_9}\max_{y\in\ocsG_9}\metric{x}{y}
        && \text{because $\fphi(x)=\frac{1}{9}x$ is \prope{strictly isotone} and by \prefp{lem:argminmaxphi}}
      \\&= \argmin_{x\in\ocsG_9}\setn{4,\,4,\,4,\,4,\,4,\,4,\,4,\,4,\,4}
        && \text{because the maximum distance in $\ocsG_9$ from any $x$ is $4$}
      \\&= \setn{0,1,2,\cdots,8}
        && \text{because the distances for values of $x$ in $\ocsG_9$ are the same}
      \\
      \ocscen(\ocsG)
        &= \setn{4} 
        && \text{by \exme{weighted ring outcome subspace} example \xref{ex:wring5}}
      \\
      \pE(\rvX) 
        &\eqd \sum_{x\in\R} x\psp(x)
        &&\text{by definition of $\pE$ \xref{def:pE}}
      \\&=\mathrlap{0\times\frac{2}{9}+1\times\frac{1}{9}+2\times\frac{1}{9}+3\times\frac{2}{9}+4\times\frac{3}{9} 
         = \frac{21}{9} = \frac{7}{3} \approx 2.333}
      %\\&=\mathrlap{
      %        \frac{1}{9}\brp{0 + 1 + 2 + 6 + 12}
      %   =    \frac{21}{9}
      %   =    \frac{7}{3}
      %   =    2\!\frac{1}{3} \approx 2.33}
      \\
      \ocsVar(\rvX;\pE)
        &= \pVar(\rvX)
        && \text{by \prefp{thm:ocsVar}}
      \\&\eqd \sum_{x\in\R} \brs{x-\pE(\rvX)}^2\psp(x)
        &&\text{by definition of $\pVar$ \xref{def:pVar}}
      \\&= \mathrlap{\brp{0-\frac{7}{3}}^2\frac{2}{9}+\brp{1-\frac{7}{3}}^2\frac{1}{9}+\brp{2-\frac{7}{3}}^2\frac{1}{9}+\brp{3-\frac{7}{3}}^2\frac{2}{9}+\brp{4-\frac{7}{3}}^2\frac{3}{9} 
         = \frac{198}{81} = \frac{22}{9} \approx 2.457}
      %\\&= \mathrlap{\frac{1}{9}\brs{\brp{-\frac{7}{3}}^2\times2+\brp{-\frac{4}{3}}^2\times1+\brp{-\frac{1}{3}}^2\times1+\brp{\frac{2}{3}}^2\times2+\brp{\frac{5}{3}}^2\times3}}
      %\\&= \mathrlap{\frac{1}{81}\brs{98+16+1+8+75} = \frac{198}{81} = \frac{22}{9} \approx 2.457}
      \\
      \ocsE(\rvX)
        &\eqd \argmin_{x\in\R}\max_{y\in\R}\metric{x}{y}\psp(y)
        &&\text{by definition of $\ocsE$ \xref{def:ocsE}}
      \\&\eqd \argmin_{x\in\R}\max_{y\in\R}\abs{x-y}\psp(y)
        &&\text{by definition usual metric on real line \xref{def:Rline}}
      \\&= \mathrlap{\argmin_{x\in\R}\brbl{\begin{array}{lM}
             \abs{x-4}\psp(4) & for $x\le\frac{12}{5}$\\
             \abs{x-0}\psp(0) & otherwise
           \end{array}}
        %\qquad\text{\psset{unit=6mm}\gsize%============================================================================
% Daniel J. Greenhoe
% XeLaTeX file
%============================================================================
{%\psset{yunit=2\psunit}%
\begin{pspicture}(-1.5,-0.6)(9,2)%
  \psset{%
    labelsep=3pt,
    linewidth=1pt,
    }%
  \psaxes[linecolor=axis,yAxis=false,labels=none]{<->}(0,0)(-1.5,0)(8.5,2)% x axis
  \psaxes[linecolor=axis,xAxis=false,labels=none]{->}(0,0)(-1.5,0)(8.5,2)% y axis
  \psline(-1,1.6667)(2.4,0.5333)(8,1.3333)%
  %
  \psline[linestyle=dotted,linecolor=red](2.4,0.5333)(2.4,0)%
  \psline[linestyle=dotted,linecolor=red](0,0.5333)(2.4,0.5333)%
  %
  \uput[0]{0}(8.5,0){$x$}%
  \uput[-90]{0}(2.4,0){$\sfrac{12}{5}$}%
  \uput[180]{0}(0,0.533){$\sfrac{8}{15}$}%
  \rput[b](3,1){$\ds\ff(x)\eqd\max_{y\in\omsR}\abs{x-y}\psp(y)$}%
\end{pspicture}}%
}
        \qquad\text{\includegraphics{sto/graphics/lcg7x1m9_max.pdf}}}
      \\&= \setn{\frac{12}{5}} = 2.4
        && \text{because expression is minimized at argument $x=\frac{12}{5}$}
      \\
      \ocsVar(\rvX;\ocsE)
        &\eqd \sum_{x\in\R}\metricsq{\ocsE(\rvX)}{x}\psp(x)
        && \text{by definition of $\ocsVar$ \xref{def:ocsVar}}
      \\&= \sum_{x\in\R}\metricsq{\frac{12}{5}}{x}\psp(x)
        && \text{by $\ocsE(\rvX)$ result}
      \\&= \mathrlap{
           \brp{\frac{12}{5}-0}^2\frac{2}{9} +
           \brp{\frac{12}{5}-1}^2\frac{1}{9} +
           \brp{\frac{12}{5}-2}^2\frac{1}{9} +
           \brp{\frac{12}{5}-3}^2\frac{2}{9} +
           \brp{\frac{12}{5}-4}^2\frac{3}{9} 
           }
      \\&= \mathrlap{\frac{1}{25\times9}
           2\brp{12- 0}^2 +
           1\brp{12- 5}^2 +
           1\brp{12-10}^2 +
           2\brp{12-15}^2 +
           3\brp{12-20}^2 
         = \frac{551}{225} \approx 2.449
           }
      \\
      \ocsE(\rvY)
      \\&\eqd \argmin_{x\in\omsH}\max_{y\in\omsH}\metric{x}{y}\psp(y)
        &&\text{by definition of $\ocsE$ \xref{def:ocsE}}
      \\&= \argmin_{x\in\omsH}\max_{y\in\omsH}\abs{x-y}\psp(y)
        &&\text{by definition of \structe{integer line} \xref{def:Zline}}
      \\&=\mathrlap{\argmin_{x\in\omsH}\max_{y\in\omsH}\frac{1}{9}
             \setn{\begin{array}{ccccc}
               {0}\times2&{1}\times1&{2}\times1&{3}\times2&{4}\times3\\
               {1}\times2&{0}\times1&{1}\times1&{2}\times2&{3}\times3\\
               {2}\times2&{1}\times1&{0}\times1&{1}\times2&{2}\times3\\
               {3}\times2&{2}\times1&{1}\times1&{0}\times2&{1}\times3\\
               {4}\times2&{3}\times1&{2}\times1&{1}\times2&{0}\times3
             \end{array}}
      \quad= \argmin_{x\in\omsH}\frac{1}{9}
             \setn{\begin{array}{c}
               12\\
                9\\
                6\\
                6\\
                8
             \end{array}}
      \quad= \argmin_{x\in\omsH}\frac{1}{9}
             \setn{\begin{array}{c}
               \mbox{ }\\
               \mbox{ }\\
               2\\
               3\\
               \mbox{ }
             \end{array}}}
      \\
      \ocsVar(\rvY;\ocsE)
        &\eqd \sum_{x\in\Z}\metricsq{\ocsE(\rvY)}{x}\psp(x)
        && \text{by definition of $\ocsVar$ \xref{def:ocsVar}}
      \\&= \sum_{x\in\Z}\metricsq{\setn{2,3}}{x}\psp(x)
        && \text{by $\ocsE(\rvY)$ result}
      \\&= \mathrlap{
           2^2\times\frac{2}{9} +
           1^2\times\frac{1}{9} +
           0^2\times\frac{1}{9} +
           0^2\times\frac{2}{9} +
           1^2\times\frac{3}{9} 
         = \frac{16}{9} \approx 1.778}
      \\
      \ocsE(\rvZ)
        &= \rvZ\brs{\ocscen(\ocsG)}
        && \text{because $\ocsG$ and $\omsH$ are \prope{isomorphic} under $\rvZ$}
      \\&= \setn{4}
        &&\text{by $\ocscen(\ocsG)$ result}
      \\
      \ocsVar(\rvZ;\ocsE)
        &\eqd \sum_{x\in\omsH} \metricsq{\ocsE(\rvZ)}{x}\psp(x)
        && \text{by definition of $\ocsVar$ \xref{def:ocsVar}}
      \\&= \sum_{x\in\omsH} \metricsq{\setn{4}}{x}\psp(x)
        && \text{by $\ocsE(\rvZ)$ result}
      \\&= \mathrlap{
           1^2\times\frac{2}{9} +
           2^2\times\frac{1}{9} +
           2^2\times\frac{1}{9} +
           1^2\times\frac{2}{9} +
           0^2\times\frac{3}{9} 
         = \frac{12}{9}= \frac{4}{3} \approx 1.333}
       \\
%      \ocsEM(\rvZ)
%        &\eqd \argmax_{x\in\omsH}\min_{y\in\omsH\setd\setn{x}}\ocsmom(x,y)
%        &&\text{by definition of $\ocsEM$ \xref{def:ocsEM}}
%      \\&\eqd \argmin_{x\in\omsH}\max_{y\in\omsH\setd\setn{x}}\metric{x}{y}\psp(y)
%        &&\text{by definition of $\ocsmom$ \xref{def:ocsmom}}
%      \\&= \argmin_{x\in\omsH}\max_{y\in\omsH\setd\setn{x}}\frac{1}{9}\metric{x}{y}\psp(y)9
%      \\&= \argmin_{x\in\omsH}\max_{y\in\omsH\setd\setn{x}}\metric{x}{y}\psp(y)9
%        && \text{because $\fphi(x)=\frac{1}{6}x$ is \prope{strictly isotone} and by \prefp{lem:argminmaxphi}}
%      %\\&= \argmax_{x\in\omsH}\min_{y\in\omsH\setd\setn{x}}
%      %       \setn{\begin{array}{ccccccccc}
%      %                                 &\metricn(0,1)\frac{1}{9}&\metricn(0,2)\frac{1}{9}&\metricn(0,3)\frac{2}{9}&\metricn(0,4)\frac{3}{9}\\
%      %         \metricn(1,0)\frac{2}{9}&                        &\metricn(1,2)\frac{1}{9}&\metricn(1,3)\frac{2}{9}&\metricn(1,4)\frac{3}{9}\\
%      %         \metricn(2,0)\frac{2}{9}&\metricn(2,1)\frac{1}{9}&                        &\metricn(2,3)\frac{2}{9}&\metricn(2,4)\frac{3}{9}\\
%      %         \metricn(3,0)\frac{2}{9}&\metricn(3,1)\frac{1}{9}&\metricn(3,2)\frac{1}{9}&                        &\metricn(3,4)\frac{3}{9}\\
%      %         \metricn(4,0)\frac{2}{9}&\metricn(4,1)\frac{1}{9}&\metricn(4,2)\frac{1}{9}&\metricn(4,3)\frac{2}{9}&                          
%      %       \end{array}}
%      \\&=\mathrlap{\argmax_{x\in\omsH}\min_{y\in\omsH\setd\setn{x}}%\frac{1}{9}
%             \setn{\begin{array}{ccccccccc}
%                         &{2}\times1&{1}\times1&{1}\times2&{2}\times3\\
%               {2}\times2&          &{2}\times1&{1}\times2&{1}\times3\\
%               {1}\times2&{2}\times1&          &{2}\times2&{1}\times3\\
%               {1}\times2&{1}\times1&{2}\times1&          &{1}\times3\\
%               {2}\times2&{1}\times1&{1}\times1&{1}\times2& 
%             \end{array}}
%      \quad= \argmax_{x\in\omsH}
%             \setn{\begin{array}{c}
%                1\\
%                2\\
%                2\\
%                1\\
%                1
%             \end{array}}
%      \quad= \setn{1,2}}
%      \\\\
%      \ocsEm(\rvZ)
%        &\eqd \argmin_{x\in\omsH}\min_{y\in\omsH\setd\setn{x}}\ocsmom(x,y)
%        &&\text{by definition of $\ocsEm$ \xref{def:ocsEm}}
%      \\&\eqd \argmin_{x\in\omsH}\min_{y\in\omsH\setd\setn{x}}\metric{x}{y}\psp(y)
%        &&\text{by definition of $\ocsmom$ \xref{def:ocsmom}}
%      \\&= \argmin_{x\in\omsH}\min_{y\in\omsH\setd\setn{x}}\frac{1}{9}\metric{x}{y}\psp(y)9
%      \\&= \argmin_{x\in\omsH}\min_{y\in\omsH\setd\setn{x}}\metric{x}{y}\psp(y)9
%        && \text{because $\fphi(x)=\frac{1}{9}x$ is \prope{strictly isotone} and by \prefp{lem:argminmaxphi}}
%      \\&=\mathrlap{\argmin_{x\in\omsH}\min_{y\in\omsH\setd\setn{x}}%\frac{1}{9}
%             \setn{\begin{array}{ccccccccc}
%                         &{2}\times1&{1}\times1&{1}\times2&{2}\times3\\
%               {2}\times2&          &{2}\times1&{1}\times2&{1}\times3\\
%               {1}\times2&{2}\times1&          &{2}\times2&{1}\times3\\
%               {1}\times2&{1}\times1&{2}\times1&          &{1}\times3\\
%               {2}\times2&{1}\times1&{1}\times1&{1}\times2& 
%             \end{array}}
%      \quad= \argmin_{x\in\omsH}
%             \setn{\begin{array}{c}
%                1\\
%                2\\
%                2\\
%                1\\
%                1
%             \end{array}}
%      \quad= \setn{0,3,4}}
%      \\
%      \ocsEa(\rvZ)
%        &\eqd \argmin_{x\in\omsH}\sum_{y\in\omsH}\ocsmom(x,y)
%        &&\text{by definition of $\ocsEa$ \xref{def:ocsEa}}
%      \\&\eqd \argmin_{x\in\omsH}\max_{y\in\omsH}\metric{x}{y}\psp(y)
%        &&\text{by definition of $\ocsmom$ \xref{def:ocsmom}}
%      \\&= \argmin_{x\in\omsH}\max_{y\in\omsH}\frac{1}{9}\metric{x}{y}\psp(y)9
%      \\&= \argmin_{x\in\omsH}\max_{y\in\omsH}\metric{x}{y}\psp(y)9
%        && \text{because $\fphi(x)=\frac{1}{9}x$ is \prope{strictly isotone} and by \prefp{lem:argminmaxphi}}
%      %\\&= \argmin_{x\in\omsH}\max_{y\in\omsH}
%      %       \setn{\begin{array}{ccccccccc}
%      %         \ocsmom(0,0)&+&\ocsmom(0,1)&+&\ocsmom(0,2)&+&\ocsmom(0,3)&+&\ocsmom(0,4)\\
%      %         \ocsmom(1,0)&+&\ocsmom(1,1)&+&\ocsmom(1,2)&+&\ocsmom(1,3)&+&\ocsmom(1,4)\\
%      %         \ocsmom(2,0)&+&\ocsmom(2,1)&+&\ocsmom(2,2)&+&\ocsmom(2,3)&+&\ocsmom(2,4)\\
%      %         \ocsmom(3,0)&+&\ocsmom(3,1)&+&\ocsmom(3,2)&+&\ocsmom(3,3)&+&\ocsmom(3,4)\\
%      %         \ocsmom(4,0)&+&\ocsmom(4,1)&+&\ocsmom(4,2)&+&\ocsmom(4,3)&+&\ocsmom(4,4)
%      %       \end{array}}
%      %\\&= \argmin_{x\in\omsH}\max_{y\in\omsH}
%      %       \setn{\begin{array}{ccccccccc}
%      %         \metricn(0,0)\frac{2}{9}&+&\metricn(0,1)\frac{1}{9}&+&\metricn(0,2)\frac{1}{9}&+&\metricn(0,3)\frac{2}{9}&+&\metricn(0,4)\frac{3}{9}\\
%      %         \metricn(1,0)\frac{2}{9}&+&\metricn(1,1)\frac{1}{9}&+&\metricn(1,2)\frac{1}{9}&+&\metricn(1,3)\frac{2}{9}&+&\metricn(1,4)\frac{3}{9}\\
%      %         \metricn(2,0)\frac{2}{9}&+&\metricn(2,1)\frac{1}{9}&+&\metricn(2,2)\frac{1}{9}&+&\metricn(2,3)\frac{2}{9}&+&\metricn(2,4)\frac{3}{9}\\
%      %         \metricn(3,0)\frac{2}{9}&+&\metricn(3,1)\frac{1}{9}&+&\metricn(3,2)\frac{1}{9}&+&\metricn(3,3)\frac{2}{9}&+&\metricn(3,4)\frac{3}{9}\\
%      %         \metricn(4,0)\frac{2}{9}&+&\metricn(4,1)\frac{1}{9}&+&\metricn(4,2)\frac{1}{9}&+&\metricn(4,3)\frac{2}{9}&+&\metricn(4,4)\frac{3}{9}\\
%      %       \end{array}}
%      \\&=\mathrlap{\argmin_{x\in\omsH}\max_{y\in\omsH}%\frac{1}{9}
%             \setn{\begin{array}{ccccccccc}
%               {0}\times2&+&{2}\times1&+&{1}\times1&+&{1}\times2&+&{2}\times3\\
%               {2}\times2&+&{0}\times1&+&{2}\times1&+&{1}\times2&+&{1}\times3\\
%               {1}\times2&+&{2}\times1&+&{0}\times1&+&{2}\times2&+&{1}\times3\\
%               {1}\times2&+&{1}\times1&+&{2}\times1&+&{0}\times2&+&{1}\times3\\
%               {2}\times2&+&{1}\times1&+&{1}\times1&+&{1}\times2&+&{0}\times3\\
%             \end{array}}
%      \quad= \argmin_{x\in\omsH}%\frac{1}{9}
%             \setn{\begin{array}{ccccc}
%                11\\
%                11\\
%                11\\
%                 8\\
%                 8
%             \end{array}}
%      \quad= \setn{3,4}}
%      \\
%      \ocsEg(\rvZ)
%        &\eqd \argmin_{x\in\omsH}\prod_{y\in\omsH\setd\setn{x}}{\metric{x}{y}^{\psp(y)}}
%        &&\text{by definition of $\ocsEg$ \xref{def:ocsEg}}
%      \\&= \argmin_{x\in\omsH}\prod_{y\in\omsH\setd\setn{x}}{\metric{x}{y}^{9\psp(y)\frac{1}{9}}}
%      \\&= \argmin_{x\in\omsH}\brs{\prod_{y\in\omsH\setd\setn{x}}{\metric{x}{y}^{9\psp(y)}}}^\frac{1}{9}
%      \\&= \argmin_{x\in\omsH}\prod_{y\in\omsH\setd\setn{x}}{\metric{x}{y}^{9\psp(y)}}
%        && \text{because $\fphi(x)=\frac{1}{9}x$ is \prope{strictly isotone} and by \prefp{lem:argminmaxphi}}
%      %\\&= \argmin_{x\in\omsH}
%      %       \setn{\begin{array}{ccccccccc}
%      %                                 &      & \metric{0}{1}^{\psp(1)} &\times& \metric{0}{2}^{\psp(2)} &\times& \metric{0}{3}^{\psp(3)} &\times& \metric{0}{4}^{\psp(4)}\\
%      %         \metric{1}{0}^{\psp(0)} &      &                         &\times& \metric{1}{2}^{\psp(2)} &\times& \metric{1}{3}^{\psp(3)} &\times& \metric{1}{4}^{\psp(4)}\\
%      %         \metric{2}{0}^{\psp(0)} &\times& \metric{2}{1}^{\psp(1)} &      &                         &\times& \metric{2}{3}^{\psp(3)} &\times& \metric{2}{4}^{\psp(4)}\\
%      %         \metric{3}{0}^{\psp(0)} &\times& \metric{3}{1}^{\psp(1)} &\times& \metric{3}{2}^{\psp(2)} &      &                         &\times& \metric{3}{4}^{\psp(4)}\\
%      %         \metric{4}{0}^{\psp(0)} &\times& \metric{4}{1}^{\psp(1)} &\times& \metric{4}{2}^{\psp(2)} &\times& \metric{4}{3}^{\psp(3)} &      &
%      %       \end{array}}
%      %\\&= \argmin_{x\in\omsH}
%      %       \setn{\begin{array}{ccccccccc}
%      %                                   &      & \metricn(0,1)^\frac{1}{9} &\times& \metricn(0,2)^\frac{1}{9} &\times& \metricn(0,3)^\frac{2}{9} &\times& \metricn(0,4)^\frac{3}{9}\\
%      %         \metricn(1,0)^\frac{2}{9} &      &                           &\times& \metricn(1,2)^\frac{1}{9} &\times& \metricn(1,3)^\frac{2}{9} &\times& \metricn(1,4)^\frac{3}{9}\\
%      %         \metricn(2,0)^\frac{2}{9} &\times& \metricn(2,1)^\frac{1}{9} &      &                           &\times& \metricn(2,3)^\frac{2}{9} &\times& \metricn(2,4)^\frac{3}{9}\\
%      %         \metricn(3,0)^\frac{2}{9} &\times& \metricn(3,1)^\frac{1}{9} &\times& \metricn(3,2)^\frac{1}{9} &      &                           &\times& \metricn(3,4)^\frac{3}{9}\\
%      %         \metricn(4,0)^\frac{2}{9} &\times& \metricn(4,1)^\frac{1}{9} &\times& \metricn(4,2)^\frac{1}{9} &\times& \metricn(4,3)^\frac{2}{9} &      &                          
%      %       \end{array}}
%      \\&=\mathrlap{\argmin_{x\in\omsH}
%             \setn{\begin{array}{ccccccccc}
%                     &      & {2}^1 &\times& {1}^1 &\times& {1}^2 &\times& {2}^3\\
%               {2}^2 &\times&       &      & {2}^1 &\times& {1}^2 &\times& {1}^3\\
%               {1}^2 &\times& {2}^1 &\times&       &      & {2}^2 &\times& {1}^3\\
%               {1}^2 &\times& {1}^1 &\times& {2}^1 &\times&       &      & {1}^3\\
%               {2}^2 &\times& {1}^1 &\times& {1}^1 &\times& {1}^2 &      &
%             \end{array}}
%      \quad= \argmin_{x\in\omsH}
%             \setn{\begin{array}{c}
%                16\\
%                 8\\
%                 8\\
%                 2\\
%                 4
%             \end{array}}
%      \quad= \setn{3}}
%      \\
%      \ocsEh(\rvZ)
%        &\eqd \argmin_{x\in\ocso}\brp{\sum_{y\in\ocso}\frac{1}{\metric{x}{y}}\psp(y)}^{-1} 
%        &&\text{by definition of $\ocsEh$ \xref{def:ocsEh}}
%      \\&=\argmax_{x\in\ocso}\brp{\sum_{y\in\ocso}\frac{1}{\metric{x}{y}}\psp(y)}
%      \\&= \argmax_{x\in\ocso}\brp{\frac{1}{9}\sum_{y\in\ocso}\frac{1}{\metric{x}{y}}\psp(y)9}
%      \\&= \argmax_{x\in\ocso}\sum_{y\in\ocso}\frac{9\psp(y)}{\metric{x}{y}}
%        && \text{because $\fphi(x)=\frac{1}{9}x$ is \prope{strictly isotone} and by \prefp{lem:argminmaxphi}}
%      %\\&= \argmin_{x\in\omsH}
%      %       \setn{\begin{array}{ccccc}
%      %         0                                + \frac{1}{\metric{0}{1}}{\psp(1)} + \frac{1}{\metric{0}{2}}{\psp(2)} + \frac{1}{\metric{0}{3}}{\psp(3)} + \frac{1}{\metric{0}{4}}{\psp(4)}\\
%      %         \frac{1}{\metric{1}{0}}{\psp(0)} + 0                                + \frac{1}{\metric{1}{2}}{\psp(2)} + \frac{1}{\metric{1}{3}}{\psp(3)} + \frac{1}{\metric{1}{4}}{\psp(4)}\\
%      %         \frac{1}{\metric{2}{0}}{\psp(0)} + \frac{1}{\metric{2}{1}}{\psp(1)} + 0                                + \frac{1}{\metric{2}{3}}{\psp(3)} + \frac{1}{\metric{2}{4}}{\psp(4)}\\
%      %         \frac{1}{\metric{3}{0}}{\psp(0)} + \frac{1}{\metric{3}{1}}{\psp(1)} + \frac{1}{\metric{3}{2}}{\psp(2)} + 0                                + \frac{1}{\metric{3}{4}}{\psp(4)}\\
%      %         \frac{1}{\metric{4}{0}}{\psp(0)} + \frac{1}{\metric{4}{1}}{\psp(1)} + \frac{1}{\metric{4}{2}}{\psp(2)} + \frac{1}{\metric{4}{3}}{\psp(3)} + 0                         
%      %       \end{array}}
%      %\\&= \argmin_{x\in\omsH}
%      %       \setn{\begin{array}{ccccc}
%      %         \brs{                            + \frac{1}{2}\cdot\frac{1}{9} + \frac{1}{1}\cdot\frac{1}{9} + \frac{1}{1}\cdot\frac{2}{9} + \frac{1}{2}\cdot\frac{3}{9}}^{-1}\\
%      %         \brs{\frac{1}{2}\cdot\frac{2}{9} + 0                           + \frac{1}{2}\cdot\frac{1}{9} + \frac{1}{1}\cdot\frac{2}{9} + \frac{1}{1}\cdot\frac{3}{9}}^{-1}\\
%      %         \brs{\frac{1}{1}\cdot\frac{2}{9} + \frac{1}{2}\cdot\frac{1}{9} + 0                           + \frac{1}{2}\cdot\frac{2}{9} + \frac{1}{1}\cdot\frac{3}{9}}^{-1}\\
%      %         \brs{\frac{1}{1}\cdot\frac{2}{9} + \frac{1}{1}\cdot\frac{1}{9} + \frac{1}{2}\cdot\frac{1}{9} + 0                           + \frac{1}{1}\cdot\frac{3}{9}}^{-1}\\
%      %         \brs{\frac{1}{2}\cdot\frac{2}{9} + \frac{1}{1}\cdot\frac{1}{9} + \frac{1}{1}\cdot\frac{1}{9} + \frac{1}{1}\cdot\frac{2}{9} + 0                          }^{-1}
%      %       \end{array}}
%      %\\&= \argmin_{x\in\omsH}9
%      %       \setn{\begin{array}{ccccc}
%      %         \brs{0           + \frac{1}{2} + \frac{1}{1} + \frac{2}{1} + \frac{3}{2}}^{-1}\\
%      %         \brs{\frac{2}{2} + 0           + \frac{1}{2} + \frac{2}{1} + \frac{3}{1}}^{-1}\\
%      %         \brs{\frac{2}{1} + \frac{1}{2} + 0           + \frac{2}{2} + \frac{3}{1}}^{-1}\\
%      %         \brs{\frac{2}{1} + \frac{1}{1} + \frac{1}{2} + 0           + \frac{3}{1}}^{-1}\\
%      %         \brs{\frac{2}{2} + \frac{1}{1} + \frac{1}{1} + \frac{2}{1} + 0          }^{-1}
%      %       \end{array}}
%      %\\&= \argmin_{x\in\omsH}9
%      %       \setn{\begin{array}{c}
%      %         \brs{\frac{10}{2}}^{-1}\\
%      %         \brs{\frac{13}{2}}^{-1}\\
%      %         \brs{\frac{13}{2}}^{-1}\\
%      %         \brs{\frac{13}{2}}^{-1}\\
%      %         \brs{\frac{11}{2}}^{-1}
%      %       \end{array}}
%      \\&=\mathrlap{\argmax_{x\in\omsH}
%             \setn{\begin{array}{*{4}{cc}c}
%               0           &+& \frac{1}{2} &+& \frac{1}{1} &+& \frac{2}{1} &+& \frac{3}{2}\\
%               \frac{2}{2} &+& 0           &+& \frac{1}{2} &+& \frac{2}{1} &+& \frac{3}{1}\\
%               \frac{2}{1} &+& \frac{1}{2} &+& 0           &+& \frac{2}{2} &+& \frac{3}{1}\\
%               \frac{2}{1} &+& \frac{1}{1} &+& \frac{1}{2} &+& 0           &+& \frac{3}{1}\\
%               \frac{2}{2} &+& \frac{1}{1} &+& \frac{1}{1} &+& \frac{2}{1} &+& 0          
%             \end{array}}
%      \quad= \argmax_{x\in\omsH}\frac{1}{2}
%             \setn{\begin{array}{c}
%               {10}\\
%               {13}\\
%               {13}\\
%               {13}\\
%               {11}
%             \end{array}}
%      \quad= \setn{1,2,3}}
    \end{align*}
\end{proof}
%
\begin{figure}[h]
  \gsize%
  \centering%
  %%============================================================================
% Daniel J. Greenhoe
% LaTeX file
% linear congruential (LCG) pseudo-random number generator (PRNG) mappings
% x_{n+1} = (7x_n+5)mod 9
% y_{n+1} = (y_n+2)mod 5
%============================================================================
\begin{pspicture}(-2.25,-2.25)(11.5,2.25)%
  %---------------------------------
  % options
  %---------------------------------
  \psset{%
    radius=1.25ex,
    labelsep=2.5mm,
    linecolor=blue,%
    }%
  %---------------------------------
  % LCG PRNG graph G
  % x_{n+1} = (7x_n+5)mod 9
  %   n  0   1   2   3   4   5   6   7   8  ;  9
  % x_n  1   3   8   7   0   5   4   6   2  ;  1
  %---------------------------------
  \rput(0,0){\psset{unit=1.5\psunit}%
    \rput{   0}(0,0){\rput(1,0){\Cnode[fillstyle=solid,fillcolor=snode](0,0){G1}}}%
    \rput{  40}(0,0){\rput(1,0){\Cnode[fillstyle=solid,fillcolor=snode](0,0){G3}}}%
    \rput{  80}(0,0){\rput(1,0){\Cnode[fillstyle=solid,fillcolor=snode](0,0){G8}}}%
    \rput{ 120}(0,0){\rput(1,0){\Cnode[fillstyle=solid,fillcolor=snode](0,0){G7}}}%
    \rput{ 160}(0,0){\rput(1,0){\Cnode[fillstyle=solid,fillcolor=snode](0,0){G0}}}%
    \rput{ 200}(0,0){\rput(1,0){\Cnode[fillstyle=solid,fillcolor=snode](0,0){G5}}}%
    \rput{ 240}(0,0){\rput(1,0){\Cnode[fillstyle=solid,fillcolor=snode](0,0){G4}}}%
    \rput{ 280}(0,0){\rput(1,0){\Cnode[fillstyle=solid,fillcolor=snode](0,0){G6}}}%
    \rput{ 320}(0,0){\rput(1,0){\Cnode[fillstyle=solid,fillcolor=snode](0,0){G2}}}%
    \rput(0,0){$\ocsG_9$}%
    }%
  \rput(G8){$8$}%
  \rput(G7){$7$}%
  \rput(G6){$6$}%
  \rput(G5){$5$}%
  \rput(G4){$4$}%
  \rput(G3){$3$}%
  \rput(G2){$2$}%
  \rput(G1){$1$}%
  \rput(G0){$0$}%
  %          
  \ncline{G1}{G3}%
  \ncline{G3}{G8}%
  \ncline{G8}{G7}%
  \ncline{G7}{G0}%
  \ncline{G0}{G5}%
  \ncline{G5}{G4}%
  \ncline{G4}{G6}%
  \ncline{G6}{G2}%
  \ncline{G2}{G1}%
  %
  \uput[  0](G1){$\frac{1}{9}$}
  \uput[ 40](G3){$\frac{1}{9}$}
  \uput[ 80](G8){$\frac{1}{9}$}
  \uput[120](G7){$\frac{1}{9}$}
  \uput[160](G0){$\frac{1}{9}$}
  \uput[200](G5){$\frac{1}{9}$}
  \uput[240](G4){$\frac{1}{9}$}
  \uput[280](G6){$\frac{1}{9}$}
  \uput[320](G2){$\frac{1}{9}$}
  %
  %
  %---------------------------------
  %---------------------------------
  %---------------------------------
  \rput(5,0){% shaped outcomes spaces
  %---------------------------------
  %---------------------------------
  %---------------------------------
  %
  %
  %---------------------------------
  % distribution shaping mapping y_n=s(x_n) to 5 element set
  %   n  0   1   2   3   4   5   6   7   8  ;  9
  % x_n  1   3   8   7   0   5   4   6   2  ;  1
  % y_n  1   3   0   2   4 ; 0   3   4   4  ;  1
  %---------------------------------
  \rput( 0,0){%\psset{unit=2\psunit}%
    \rput{288}(0,0){\rput(1,0){\Cnode(0,0){S4}}}%
    \rput{216}(0,0){\rput(1,0){\Cnode(0,0){S2}}}%
    \rput{144}(0,0){\rput(1,0){\Cnode(0,0){S0}}}%
    \rput{ 72}(0,0){\rput(1,0){\Cnode[fillstyle=solid,fillcolor=snode](0,0){S3}}}%
    \rput{  0}(0,0){\rput(1,0){\Cnode(0,0){S1}}}%
    \rput(0,0){$\ocsG$}%
    }
  \rput(S4){$4$}%
  \rput(S3){$3$}%
  \rput(S2){$2$}%
  \rput(S1){$1$}%
  \rput(S0){$0$}%
  %
  %\ncline{S4}{S0}
  \ncline{S4}{S1}\ncline{S3}{S4}%
  \ncline{S2}{S4}%
  \ncline{S0}{S2}%
  \ncline{S3}{S0}%
  \ncline{S1}{S3}%
  %
  \uput[288](S4){$\frac{3}{9}$}
  \uput[ 72](S3){$\frac{2}{9}$}
  \uput[216](S2){$\frac{1}{9}$}
  \uput[  0](S1){$\frac{1}{9}$}
  \uput[144](S0){$\frac{2}{9}$}
  %
  \ncarc[arcangle= 22,linewidth=0.75pt,linecolor=red]{->}{G8}{S0}%
  \ncarc[arcangle= 67,linewidth=0.75pt,linecolor=red]{->}{G7}{S2}%
  \ncarc[arcangle=-45,linewidth=0.75pt,linecolor=red]{->}{G6}{S4}%
  \ncarc[arcangle=-22,linewidth=0.75pt,linecolor=red]{->}{G5}{S0}%
  \ncarc[arcangle=-22,linewidth=0.75pt,linecolor=red]{->}{G4}{S3}%
  \ncarc[arcangle= 22,linewidth=0.75pt,linecolor=red]{->}{G3}{S3}%
  \ncarc[arcangle=-22,linewidth=0.75pt,linecolor=red]{->}{G2}{S4}%
  \ncarc[arcangle= 10,linewidth=0.75pt,linecolor=red]{->}{G1}{S1}%
  \ncarc[arcangle= 22,linewidth=0.75pt,linecolor=red]{->}{G0}{S4}%
  %%---------------------------------
  %% random variable mapping Y from S to integer line 
  %%---------------------------------
  %\rput(0,3){%\psset{unit=0.75\psunit}%
  %  \pnode(3,0){YB}%
  %  \Cnode(2,0){Y4}%
  %  \Cnode[fillstyle=solid,fillcolor=snode](1,0){Y3}%
  %  \Cnode[fillstyle=solid,fillcolor=snode](0,0){Y2}%
  %  \Cnode(-1,0){Y1}%
  %  \Cnode(-2,0){Y0}%
  %  \pnode(-3,0){YA}%
  %  }
  %\rput(Y4){$4$}%
  %\rput(Y3){$3$}%
  %\rput(Y2){$2$}%
  %\rput(Y1){$1$}%
  %\rput(Y0){$0$}%
  %%
  %\ncline[linestyle=dotted]{Y4}{YB}%
  %\ncline{Y3}{Y4}%
  %\ncline{Y2}{Y3}%
  %\ncline{Y1}{Y2}%
  %\ncline{Y0}{Y1}%
  %\ncline[linestyle=dotted]{Y0}{YA}%
  %%
  %\uput[ 0](YB){$\Z$}
  %\uput[90](Y4){$\frac{3}{9}$}
  %\uput[90](Y3){$\frac{2}{9}$}
  %\uput[90](Y2){$\frac{1}{9}$}
  %\uput[90](Y1){$\frac{1}{9}$}
  %\uput[90](Y0){$\frac{2}{9}$}
  %%
  %\ncarc[arcangle=-45,linewidth=0.75pt,linecolor=red]{->}{S4}{Y4}%
  %\ncarc[arcangle= 22,linewidth=0.75pt,linecolor=red]{->}{S3}{Y3}%
  %\ncarc[arcangle= 45,linewidth=0.75pt,linecolor=red]{->}{S2}{Y2}%
  %\ncarc[arcangle=-22,linewidth=0.75pt,linecolor=red]{->}{S1}{Y1}%
  %\ncarc[arcangle= 22,linewidth=0.75pt,linecolor=red]{->}{S0}{Y0}%
  %---------------------------------
  % random variable mapping Z from S to ring-like structure 
  %---------------------------------
  \rput( 5,0){%\psset{unit=2\psunit}%
    \rput{288}(0,0){\rput(1,0){\Cnode(0,0){Z4}}}%
    \rput{216}(0,0){\rput(1,0){\Cnode(0,0){Z2}}}%
    \rput{144}(0,0){\rput(1,0){\Cnode(0,0){Z0}}}%
    \rput{ 72}(0,0){\rput(1,0){\Cnode[fillstyle=solid,fillcolor=snode](0,0){Z3}}}%
    \rput{  0}(0,0){\rput(1,0){\Cnode(0,0){Z1}}}%
    \rput(0,0){$\omsH$}%
    }
  \rput(Z4){$4$}%
  \rput(Z3){$3$}%
  \rput(Z2){$2$}%
  \rput(Z1){$1$}%
  \rput(Z0){$0$}%
  %
  %\ncline{Z4}{Z0}
  \ncline{Z4}{Z1}\ncline{Z2}{Z4}\ncline{Z3}{Z4}%
  \ncline{Z2}{Z4}%
  \ncline{Z0}{Z2}%
  \ncline{Z3}{Z0}%
  \ncline{Z1}{Z3}%
  %
  \uput[288](Z4){$\frac{3}{9}$}%
  \uput[ 72](Z3){$\frac{2}{9}$}%
  \uput[216](Z2){$\frac{1}{9}$}%
  \uput[  0](Z1){$\frac{1}{9}$}%
  \uput[144](Z0){$\frac{2}{9}$}%
  %
  \ncarc[arcangle=-22,linewidth=0.75pt,linecolor=green]{->}{S4}{Z4}%
  \ncarc[arcangle= 22,linewidth=0.75pt,linecolor=green]{->}{S3}{Z3}%
  \ncarc[arcangle=-10,linewidth=0.75pt,linecolor=green]{->}{S2}{Z2}%
  \ncarc[arcangle= 10,linewidth=0.75pt,linecolor=green]{->}{S1}{Z1}%
  \ncarc[arcangle=-10,linewidth=0.75pt,linecolor=green]{->}{S0}{Z0}%
  %%---------------------------------
  %% random variable mapping X from S to real line
  %%---------------------------------
  %\rput(0,-3){%\psset{unit=0.5\psunit}%
  %  \psline{<->}(-2.5,0)(2.5,0)%
  %  \multirput(-2,0)(1,0){5}{\psline(0,-0.1)(0,0.1)}%
  %  \pnode(2.5,0){XB}%
  %  \pnode( 2,0){X4}%
  %  \pnode( 1,0){X3}%
  %  \pnode(0.4,0){X240}%
  %  \pnode(0.33,0){X233}%
  %  \pnode( 0,0){X2}%
  %  \pnode(-1,0){X1}%
  %  \pnode(-2,0){X0}%
  %  \pnode(-2.5,0){XA}%
  %  }
  %%
  %\rput(X240){\pscircle[fillstyle=solid,linecolor=snode,fillcolor=snode](0,0){1ex}}%
  %\rput(X233){\pscircle[fillstyle=none,linecolor=red,fillcolor=red](0,0){1ex}}%
  %%
  %\uput{1.5mm}[-90](X4){$\frac{3}{9}$}%
  %\uput{1.5mm}[-90](X3){$\frac{2}{9}$}%
  %\uput{1.5mm}[-90](X2){$\frac{1}{9}$}%
  %\uput{1.5mm}[-90](X1){$\frac{1}{9}$}%
  %\uput{1.5mm}[-90](X0){$\frac{2}{9}$}%
  %%
  %\uput[0](XB){$\R$}%
  %\uput{1.5mm}[90](X4){$4$}%
  %\uput{1.5mm}[90](X3){$3$}%
  %\uput{1.5mm}[90](X2){$2$}%
  %\uput{1.5mm}[90](X1){$1$}%
  %\uput{1.5mm}[90](X0){$0$}%
  %%
  %\ncarc[arcangle= 22,linewidth=0.75pt,linecolor=purple]{->}{S4}{X4}%
  %\ncarc[arcangle= 67,linewidth=0.75pt,linecolor=purple]{->}{S3}{X3}%
  %\ncarc[arcangle=-22,linewidth=0.75pt,linecolor=purple]{->}{S2}{X2}%
  %\ncarc[arcangle= 45,linewidth=0.75pt,linecolor=purple]{->}{S1}{X1}%
  %\ncarc[arcangle=-22,linewidth=0.75pt,linecolor=purple]{->}{S0}{X0}%
  %---------------------------------
  % labels
  %---------------------------------
  %\rput(0,1.9){$\rvY(\cdot)$}%
  %\rput(0,-1.9){$\rvX(\cdot)$}%
  \rput(2.75,0){$\rvZ(\cdot)$}%
  %\ncline[linestyle=dotted,nodesep=1pt]{->}{xzlabel}{xz}%
  %\ncline[linestyle=dotted,nodesep=1pt]{->}{ylabel}{y}%
  }%
  \rput(2.75,0){$\fs(x_n)$}%
\end{pspicture}%%
  \includegraphics{sto/graphics/lcg7x1m9_seqorder.pdf}%
  \caption{sequentially ordered LCG mappings \xref{ex:lcg7x1m9_seqorder}\label{fig:lcg7x1m9_seqorder}}
\end{figure}
%---------------------------------------
%\begin{minipage}{\tw-65mm}%
\begin{example}[\exmd{LCG mappings, sequential ordering}]
\label{ex:lcg7x1m9_seqorder}\mbox{}\\
%---------------------------------------
In \prefpp{ex:lcg7x1m9_xyz}, the structures $\ocsG_9$, $\ocsG$, and $\omsH$ were ordered as a standard ring of 
integers ($0<1<2<\cdots<7<8<0$ for $\ocsG_9$).
In this current example, as illustrated in \prefpp{fig:lcg7x1m9_seqorder}, these structures are ordered as they appear in 
the sequences generated by $x_{n+1}=(7x_n+5)\mod9$ and $\fs$ (see \pref{ex:lcg7x1m9_xyz} for sequence description).
This yields the following statistics:
\\$\quad\begin{array}{>{\gsize}Mrcc rcccl}
  geometry of $\ocsG_9$:                                & \ocscen (\ocsG) &=& \mc{4}{l}{\setn{0,1,2,3,4,5,6,7,8}}  \\
  geometry of $\ocsG$:                                  & \ocscen (\ocsG) &=& \setn{3}  \\
  outcome subspace statistics on isomorphic structure:  & \ocsE   (\rvZ)  &=& \setn{3}   & \ocsVar(\rvZ;\ocsE) &=& \frac{10}{9} &\approx& 1.111
\end{array}$\\
Note that a change in ordering structure (from standard ring ordering to sequential ordering) yields a change in statistics
($\ocsE(\rvZ)=\setn{3}$ as opposed to $\ocsE(\rvZ)=\setn{4}$).
Intuitively, the sequential ordering of \pref{ex:lcg7x1m9_seqorder} should yield a better estimate
than that of \pref{ex:lcg7x1m9_xyz}, because it more closely matches the way the PRNG produces a sequence.
This intuition is also supported by the variance values 
($\ocsVar(\rvZ)=\sfrac{12}{9}$ for standard ring ordering, $\ocsVar(\rvZ)=\frac{10}{9}$ for sequential ordering).
However, counterintuitively, the sequential ordering no longer yields the maximally likely result of $\setn{4}$.
\end{example}
%\end{minipage}\hfill%
%\begin{tabular}{c}
%  \gsize%
%  %\psset{unit=5mm}%
%  {%============================================================================
% Daniel J. Greenhoe
% LaTeX file
% discrete metric real dice mapping to linearly ordered O6c
%============================================================================
{%\psset{unit=0.5\psunit}%
\begin{pspicture}(-3.25,-1.5)(3.25,1.5)%
  %---------------------------------
  % options
  %---------------------------------
  \psset{%
    radius=1.25ex,
    labelsep=2.5mm,
    linecolor=blue,%
    }%
  %---------------------------------
  % dice graph
  %---------------------------------
  \rput(-1.75,0){%\psset{unit=2\psunit}%
    \Cnode[fillstyle=solid,fillcolor=snode](-0.8660,-0.5){D4}%
    \Cnode(-0.8660,0.5){D5}%
    \Cnode(0,1){D6}%
    \Cnode(0.8660,0.5){D3}%
    \Cnode(0.8660,-0.5){D2}%
    \Cnode(0,-1){D1}%
    }
  \rput(D6){$\diceF$}%
  \rput(D5){$\diceE$}%
  \rput(D4){$\diceD$}%
  \rput(D3){$\diceC$}%
  \rput(D2){$\diceB$}%
  \rput(D1){$\diceA$}%
  %
  \ncline{D5}{D6}%
  \ncline{D4}{D5}\ncline{D4}{D6}%
  \ncline{D3}{D5}\ncline{D3}{D6}%
  \ncline{D2}{D3}\ncline{D2}{D4}\ncline{D2}{D6}%
  \ncline{D1}{D2}\ncline{D1}{D3}\ncline{D1}{D4}\ncline{D1}{D5}%
  %
  \uput[ 158](D6){$\frac{1}{30}$}
  \uput[ 150](D5){$\frac{1}{50}$}
  \uput[ 210](D4){$\frac{3}{5}$}
  \uput[  22](D3){$\frac{1}{30}$}
  \uput[ -45](D2){$\frac{1}{20}$}
  \uput[-158](D1){$\frac{1}{10}$}
  %---------------------------------
  % range graph
  %---------------------------------
  \rput(1.75,0){%\psset{unit=2\psunit}%
    \Cnode[fillstyle=solid,fillcolor=snode](-0.8660,-0.5){E4}%
    \Cnode(-0.8660,0.5){E5}%
    \Cnode(0,1){E6}%
    \Cnode(0.8660,0.5){E3}%
    \Cnode(0.8660,-0.5){E2}%
    \Cnode(0,-1){E1}%
    \Cnode(0,0){E0}%
    }
  \rput(E6){$6$}%
  \rput(E5){$5$}%
  \rput(E4){$4$}%
  \rput(E3){$3$}%
  \rput(E2){$2$}%
  \rput(E1){$1$}%
  \rput(E0){$0$}%
  %
  \ncline{E5}{E6}%
  \ncline{E4}{E5}\ncline{E4}{E6}%
  \ncline{E3}{E5}\ncline{E3}{E6}%
  \ncline{E2}{E3}\ncline{E2}{E4}\ncline{E2}{E6}%
  \ncline{E1}{E2}\ncline{E1}{E3}\ncline{E1}{E4}\ncline{E1}{E5}%
  \ncline{E0}{E1}\ncline{E0}{E2}\ncline{E0}{E3}\ncline{E0}{E4}\ncline{E0}{E5}\ncline{E0}{E6}%
  %
  \uput[ 158](E6){$\frac{1}{30}$}
  \uput[ 150](E5){$\frac{1}{50}$}
  \uput[ 210](E4){$\frac{3}{5}$}
  \uput[  22](E3){$\frac{1}{30}$}
  \uput[ -45](E2){$\frac{1}{20}$}
  \uput[-158](E1){$\frac{1}{10}$}
  \uput[  0](E0){$\sfrac{0}{6}$}
  %---------------------------------
  % mapping from die to O6c
  %---------------------------------
  \ncarc[arcangle= 22,linewidth=0.75pt,linecolor=red]{->}{D6}{E6}%
  \ncarc[arcangle= 22,linewidth=0.75pt,linecolor=red]{->}{D5}{E5}%
  \ncarc[arcangle= 22,linewidth=0.75pt,linecolor=red]{->}{D4}{E4}%
  \ncarc[arcangle=-22,linewidth=0.75pt,linecolor=red]{->}{D3}{E3}%
  \ncarc[arcangle=-22,linewidth=0.75pt,linecolor=red]{->}{D2}{E2}%
  \ncarc[arcangle=-22,linewidth=0.75pt,linecolor=red]{->}{D1}{E1}%
  %---------------------------------
  % labels
  %---------------------------------
  \rput(0,0){$\rvX(\cdot)$}%
  %\ncline[linestyle=dotted,nodesep=1pt]{->}{xzlabel}{xz}%
  %\ncline[linestyle=dotted,nodesep=1pt]{->}{ylabel}{y}%
\end{pspicture}
}%}%
%\end{tabular}
\begin{proof}
\begin{align*}
  \ocscen(\ocsG_9)
  \\&= \setn{0,1,2,\cdots,8}
    && \text{by \exme{LCG mappings standard ordering} example \xref{ex:lcg7x1m9_xyz}}
  \\
  \ocscen(\ocsG)
    &= \setn{3}
    &&\text{by \prefpp{ex:wring5short}}
  \\
  \ocsE(\rvZ)
    &= \rvZ\brs{\ocscen(\ocsG)}
    && \text{because $\ocsG$ and $\omsH$ are \prope{isomorphic} under $\rvZ$}
  \\&= \rvZ\brs{\setn{3}}
    && \text{by $\ocscen(\ocsG)$ result}
  \\&= \setn{3}
    && \text{by definition of $\rvZ$}
  \\
  \ocsVar(\rvZ;\ocsE)
    &= \ocsVaro(\ocsG)
    && \text{because $\ocsG$ and $\omsH$ are \prope{isomorphic} under $\rvZ$}\
  \\&= \frac{10}{9} \approx 1.111
    &&\text{by \prefpp{ex:wring5short}}
\end{align*}
\end{proof}

\begin{figure}[h]
  \gsize%
  \centering%
  %%============================================================================
% Daniel J. Greenhoe
% LaTeX file
% linear congruential (LCG) pseudo-random number generator (PRNG) mappings
% x_{n+1} = (7x_n+5)mod 9
% y_{n+1} = (y_n+2)mod 5
%============================================================================
\begin{pspicture}(-2.25,-2.3)(11.5,2.3)%
  %---------------------------------
  % options
  %---------------------------------
  \psset{%
    radius=1.25ex,
    labelsep=2.5mm,
    linecolor=blue,%
    }%
  %---------------------------------
  % LCG PRNG graph G
  % x_{n+1} = (7x_n+5)mod 9
  %   n  0   1   2   3   4   5   6   7   8  ;  9
  % x_n  1   3   8   7   0   5   4   6   2  ;  1
  %---------------------------------
  \rput(0,0){\psset{unit=1.5\psunit}%
    \rput{   0}(0,0){\rput(1,0){\Cnode[fillstyle=solid,fillcolor=snode](0,0){G1}}}%
    \rput{  40}(0,0){\rput(1,0){\Cnode[fillstyle=solid,fillcolor=snode](0,0){G3}}}%
    \rput{  80}(0,0){\rput(1,0){\Cnode[fillstyle=solid,fillcolor=snode](0,0){G8}}}%
    \rput{ 120}(0,0){\rput(1,0){\Cnode[fillstyle=solid,fillcolor=snode](0,0){G7}}}%
    \rput{ 160}(0,0){\rput(1,0){\Cnode[fillstyle=solid,fillcolor=snode](0,0){G0}}}%
    \rput{ 200}(0,0){\rput(1,0){\Cnode[fillstyle=solid,fillcolor=snode](0,0){G5}}}%
    \rput{ 240}(0,0){\rput(1,0){\Cnode[fillstyle=solid,fillcolor=snode](0,0){G4}}}%
    \rput{ 280}(0,0){\rput(1,0){\Cnode[fillstyle=solid,fillcolor=snode](0,0){G6}}}%
    \rput{ 320}(0,0){\rput(1,0){\Cnode[fillstyle=solid,fillcolor=snode](0,0){G2}}}%
    \rput(0,0){$\ocsG_9$}%
    }%
  \rput(G8){$8$}%
  \rput(G7){$7$}%
  \rput(G6){$6$}%
  \rput(G5){$5$}%
  \rput(G4){$4$}%
  \rput(G3){$3$}%
  \rput(G2){$2$}%
  \rput(G1){$1$}%
  \rput(G0){$0$}%
  %          
  \ncline{->}{G1}{G3}%
  \ncline{->}{G3}{G8}%
  \ncline{->}{G8}{G7}%
  \ncline{->}{G7}{G0}%
  \ncline{->}{G0}{G5}%
  \ncline{->}{G5}{G4}%
  \ncline{->}{G4}{G6}%
  \ncline{->}{G6}{G2}%
  \ncline{->}{G2}{G1}%
  %
  \uput[  0](G1){$\frac{1}{9}$}
  \uput[ 40](G3){$\frac{1}{9}$}
  \uput[ 80](G8){$\frac{1}{9}$}
  \uput[120](G7){$\frac{1}{9}$}
  \uput[160](G0){$\frac{1}{9}$}
  \uput[200](G5){$\frac{1}{9}$}
  \uput[240](G4){$\frac{1}{9}$}
  \uput[280](G6){$\frac{1}{9}$}
  \uput[320](G2){$\frac{1}{9}$}
  %
  %
  %---------------------------------
  %---------------------------------
  %---------------------------------
  \rput(5,0){% shaped outcomes spaces
  %---------------------------------
  %---------------------------------
  %---------------------------------
  %
  %
  %---------------------------------
  % distribution shaping mapping y_n=s(x_n) to 5 element set
  %   n  0   1   2   3   4   5   6   7   8  ;  9
  % x_n  1   3   8   7   0   5   4   6   2  ;  1
  % y_n  1   3   0   2   4 ; 0   3   4   4  ;  1
  %---------------------------------
  \rput( 0,0){%\psset{unit=2\psunit}%
    \rput{288}(0,0){\rput(1,0){\Cnode(0,0){S4}}}%
    \rput{216}(0,0){\rput(1,0){\Cnode(0,0){S2}}}%
    \rput{144}(0,0){\rput(1,0){\Cnode(0,0){S0}}}%
    \rput{ 72}(0,0){\rput(1,0){\Cnode[fillstyle=solid,fillcolor=snode](0,0){S3}}}%
    \rput{  0}(0,0){\rput(1,0){\Cnode(0,0){S1}}}%
    \rput(0,0){$\ocsG$}%
    }
  \rput(S4){$4$}%
  \rput(S3){$3$}%
  \rput(S2){$2$}%
  \rput(S1){$1$}%
  \rput(S0){$0$}%
  %
  %\ncline{S4}{S0}
  \ncline{->}{S4}{S1}\ncline{->}{S2}{S4}\ncline{->}{S3}{S4}%
  \ncline{->}{S2}{S4}%
  \ncline{->}{S0}{S2}%
  \ncline{->}{S3}{S0}%
  \ncline{->}{S1}{S3}%
  %
  \uput[288](S4){$\frac{3}{9}$}
  \uput[ 72](S3){$\frac{2}{9}$}
  \uput[216](S2){$\frac{1}{9}$}
  \uput[  0](S1){$\frac{1}{9}$}
  \uput[144](S0){$\frac{2}{9}$}
  %
  \ncarc[arcangle= 22,linewidth=0.75pt,linecolor=red]{->}{G8}{S0}%
  \ncarc[arcangle= 67,linewidth=0.75pt,linecolor=red]{->}{G7}{S2}%
  \ncarc[arcangle=-45,linewidth=0.75pt,linecolor=red]{->}{G6}{S4}%
  \ncarc[arcangle=-22,linewidth=0.75pt,linecolor=red]{->}{G5}{S0}%
  \ncarc[arcangle=-22,linewidth=0.75pt,linecolor=red]{->}{G4}{S3}%
  \ncarc[arcangle= 22,linewidth=0.75pt,linecolor=red]{->}{G3}{S3}%
  \ncarc[arcangle=-22,linewidth=0.75pt,linecolor=red]{->}{G2}{S4}%
  \ncarc[arcangle= 10,linewidth=0.75pt,linecolor=red]{->}{G1}{S1}%
  \ncarc[arcangle= 22,linewidth=0.75pt,linecolor=red]{->}{G0}{S4}%
  %%---------------------------------
  %% random variable mapping Y from S to integer line 
  %%---------------------------------
  %\rput(0,3){%\psset{unit=0.75\psunit}%
  %  \pnode(3,0){YB}%
  %  \Cnode(2,0){Y4}%
  %  \Cnode[fillstyle=solid,fillcolor=snode](1,0){Y3}%
  %  \Cnode[fillstyle=solid,fillcolor=snode](0,0){Y2}%
  %  \Cnode(-1,0){Y1}%
  %  \Cnode(-2,0){Y0}%
  %  \pnode(-3,0){YA}%
  %  }
  %\rput(Y4){$4$}%
  %\rput(Y3){$3$}%
  %\rput(Y2){$2$}%
  %\rput(Y1){$1$}%
  %\rput(Y0){$0$}%
  %%
  %\ncline[linestyle=dotted]{Y4}{YB}%
  %\ncline{Y3}{Y4}%
  %\ncline{Y2}{Y3}%
  %\ncline{Y1}{Y2}%
  %\ncline{Y0}{Y1}%
  %\ncline[linestyle=dotted]{Y0}{YA}%
  %%
  %\uput[ 0](YB){$\Z$}
  %\uput[90](Y4){$\frac{3}{9}$}
  %\uput[90](Y3){$\frac{2}{9}$}
  %\uput[90](Y2){$\frac{1}{9}$}
  %\uput[90](Y1){$\frac{1}{9}$}
  %\uput[90](Y0){$\frac{2}{9}$}
  %%
  %\ncarc[arcangle=-45,linewidth=0.75pt,linecolor=red]{->}{S4}{Y4}%
  %\ncarc[arcangle= 22,linewidth=0.75pt,linecolor=red]{->}{S3}{Y3}%
  %\ncarc[arcangle= 45,linewidth=0.75pt,linecolor=red]{->}{S2}{Y2}%
  %\ncarc[arcangle=-22,linewidth=0.75pt,linecolor=red]{->}{S1}{Y1}%
  %\ncarc[arcangle= 22,linewidth=0.75pt,linecolor=red]{->}{S0}{Y0}%
  %---------------------------------
  % random variable mapping Z from S to ring-like structure 
  %---------------------------------
  \rput( 5,0){%\psset{unit=2\psunit}%
    \rput{288}(0,0){\rput(1,0){\Cnode(0,0){Z4}}}%
    \rput{216}(0,0){\rput(1,0){\Cnode(0,0){Z2}}}%
    \rput{144}(0,0){\rput(1,0){\Cnode(0,0){Z0}}}%
    \rput{ 72}(0,0){\rput(1,0){\Cnode[fillstyle=solid,fillcolor=snode](0,0){Z3}}}%
    \rput{  0}(0,0){\rput(1,0){\Cnode(0,0){Z1}}}%
    \rput(0,0){$\omsH$}%
    }
  \rput(Z4){$4$}%
  \rput(Z3){$3$}%
  \rput(Z2){$2$}%
  \rput(Z1){$1$}%
  \rput(Z0){$0$}%
  %
  %\ncline{Z4}{Z0}
  \ncline{->}{Z4}{Z1}\ncline{->}{Z2}{Z4}\ncline{->}{Z3}{Z4}%
  \ncline{->}{Z2}{Z4}%
  \ncline{->}{Z0}{Z2}%
  \ncline{->}{Z3}{Z0}%
  \ncline{->}{Z1}{Z3}%
  %
  \uput[288](Z4){$\frac{3}{9}$}%
  \uput[ 72](Z3){$\frac{2}{9}$}%
  \uput[216](Z2){$\frac{1}{9}$}%
  \uput[  0](Z1){$\frac{1}{9}$}%
  \uput[144](Z0){$\frac{2}{9}$}%
  %
  \ncarc[arcangle=-22,linewidth=0.75pt,linecolor=green]{->}{S4}{Z4}%
  \ncarc[arcangle= 22,linewidth=0.75pt,linecolor=green]{->}{S3}{Z3}%
  \ncarc[arcangle=-10,linewidth=0.75pt,linecolor=green]{->}{S2}{Z2}%
  \ncarc[arcangle= 10,linewidth=0.75pt,linecolor=green]{->}{S1}{Z1}%
  \ncarc[arcangle=-10,linewidth=0.75pt,linecolor=green]{->}{S0}{Z0}%
  %%---------------------------------
  %% random variable mapping X from S to real line
  %%---------------------------------
  %\rput(0,-3){%\psset{unit=0.5\psunit}%
  %  \psline{<->}(-2.5,0)(2.5,0)%
  %  \multirput(-2,0)(1,0){5}{\psline(0,-0.1)(0,0.1)}%
  %  \pnode(2.5,0){XB}%
  %  \pnode( 2,0){X4}%
  %  \pnode( 1,0){X3}%
  %  \pnode(0.4,0){X240}%
  %  \pnode(0.33,0){X233}%
  %  \pnode( 0,0){X2}%
  %  \pnode(-1,0){X1}%
  %  \pnode(-2,0){X0}%
  %  \pnode(-2.5,0){XA}%
  %  }
  %%
  %\rput(X240){\pscircle[fillstyle=solid,linecolor=snode,fillcolor=snode](0,0){1ex}}%
  %\rput(X233){\pscircle[fillstyle=none,linecolor=red,fillcolor=red](0,0){1ex}}%
  %%
  %\uput{1.5mm}[-90](X4){$\frac{3}{9}$}%
  %\uput{1.5mm}[-90](X3){$\frac{2}{9}$}%
  %\uput{1.5mm}[-90](X2){$\frac{1}{9}$}%
  %\uput{1.5mm}[-90](X1){$\frac{1}{9}$}%
  %\uput{1.5mm}[-90](X0){$\frac{2}{9}$}%
  %%
  %\uput[0](XB){$\R$}%
  %\uput{1.5mm}[90](X4){$4$}%
  %\uput{1.5mm}[90](X3){$3$}%
  %\uput{1.5mm}[90](X2){$2$}%
  %\uput{1.5mm}[90](X1){$1$}%
  %\uput{1.5mm}[90](X0){$0$}%
  %%
  %\ncarc[arcangle= 22,linewidth=0.75pt,linecolor=purple]{->}{S4}{X4}%
  %\ncarc[arcangle= 67,linewidth=0.75pt,linecolor=purple]{->}{S3}{X3}%
  %\ncarc[arcangle=-22,linewidth=0.75pt,linecolor=purple]{->}{S2}{X2}%
  %\ncarc[arcangle= 45,linewidth=0.75pt,linecolor=purple]{->}{S1}{X1}%
  %\ncarc[arcangle=-22,linewidth=0.75pt,linecolor=purple]{->}{S0}{X0}%
  %---------------------------------
  % labels
  %---------------------------------
  %\rput(0,1.9){$\rvY(\cdot)$}%
  %\rput(0,-1.9){$\rvX(\cdot)$}%
  \rput(2.75,0){$\rvZ(\cdot)$}%
  %\ncline[linestyle=dotted,nodesep=1pt]{->}{xzlabel}{xz}%
  %\ncline[linestyle=dotted,nodesep=1pt]{->}{ylabel}{y}%
  }%
  \rput(2.75,0){$\fs(x_n)$}%
\end{pspicture}%%
  \includegraphics{sto/graphics/lcg7x1m9_dgraph.pdf}%
  \caption{LCG mappings to \prope{linear} ($\rvX$), non-linear discrete ($\rvY$)
  and non-linear continuous ($\rvZ$) ordered metric spaces \xref{ex:lcg7x1m9_dgraph}\label{fig:lcg7x1m9_dgraph}}
\end{figure}
%---------------------------------------
%\begin{minipage}{\tw-65mm}%
\begin{example}[\exmd{LCG mappings, sequential directed graph}]
\label{ex:lcg7x1m9_dgraph}\mbox{}\\
%---------------------------------------
Let $\ocsG$, $\omsH$ and $\rvZ$ be a illustrated in \prefpp{fig:lcg7x1m9_dgraph}.
In \prefpp{ex:lcg7x1m9_seqorder}, the outcome values were ordered sequentially \emph{like} a PRNG,
but the metrics were \prope{commutative}, which is \emph{unlike} a PRNG.
In this example, the outcomes are assigned \fncte{quasi-metric}s \xxref{def:qmetric}{rem:qmetric} that are \prope{non-commutative}.
For example in the shaped sequence $\fs(x_n)=\seqn{\ldots,3, 4, 4, 1, 3, \ldots}$, 
the ``distance" from $3$ to $4$ is $\metric{3}{4}=1$,
but from $4$ to $3$ is $\metric{4}{3}=2$.
This yields the following statistics:
\\$\quad\begin{array}{>{\gsize}Mrcc rcccl}
  geometry of $\ocsG_9$:                                & \ocscen (\ocsG) &=& \mc{4}{l}{\setn{0,1,2,3,4,5,6,7,8}}  \\
  geometry of $\ocsG$:                                  & \ocscen (\ocsG) &=& \setn{3}  \\
  outcome subspace statistics on isomorphic structure:  & \ocsE   (\rvZ)  &=& \setn{3}   & \ocsVar(\rvZ;\ocsE) &=& \frac{4}{3} &\approx& 1.333
\end{array}$\\
Note that this technique yields the same estimate $\ocsE(\rvZ)=\setn{3}$ as \pref{ex:lcg7x1m9_seqorder}, 
but with a larger variance.
\end{example}
%\end{minipage}\hfill%
%\begin{tabular}{c}
%  \gsize%
%  %\psset{unit=5mm}%
%  {%============================================================================
% Daniel J. Greenhoe
% LaTeX file
% discrete metric real dice mapping to linearly ordered O6c
%============================================================================
{%\psset{unit=0.5\psunit}%
\begin{pspicture}(-3.25,-1.5)(3.25,1.5)%
  %---------------------------------
  % options
  %---------------------------------
  \psset{%
    radius=1.25ex,
    labelsep=2.5mm,
    linecolor=blue,%
    }%
  %---------------------------------
  % dice graph
  %---------------------------------
  \rput(-1.75,0){%\psset{unit=2\psunit}%
    \Cnode[fillstyle=solid,fillcolor=snode](-0.8660,-0.5){D4}%
    \Cnode(-0.8660,0.5){D5}%
    \Cnode(0,1){D6}%
    \Cnode(0.8660,0.5){D3}%
    \Cnode(0.8660,-0.5){D2}%
    \Cnode(0,-1){D1}%
    }
  \rput(D6){$\diceF$}%
  \rput(D5){$\diceE$}%
  \rput(D4){$\diceD$}%
  \rput(D3){$\diceC$}%
  \rput(D2){$\diceB$}%
  \rput(D1){$\diceA$}%
  %
  \ncline{D5}{D6}%
  \ncline{D4}{D5}\ncline{D4}{D6}%
  \ncline{D3}{D5}\ncline{D3}{D6}%
  \ncline{D2}{D3}\ncline{D2}{D4}\ncline{D2}{D6}%
  \ncline{D1}{D2}\ncline{D1}{D3}\ncline{D1}{D4}\ncline{D1}{D5}%
  %
  \uput[ 158](D6){$\frac{1}{30}$}
  \uput[ 150](D5){$\frac{1}{50}$}
  \uput[ 210](D4){$\frac{3}{5}$}
  \uput[  22](D3){$\frac{1}{30}$}
  \uput[ -45](D2){$\frac{1}{20}$}
  \uput[-158](D1){$\frac{1}{10}$}
  %---------------------------------
  % range graph
  %---------------------------------
  \rput(1.75,0){%\psset{unit=2\psunit}%
    \Cnode[fillstyle=solid,fillcolor=snode](-0.8660,-0.5){E4}%
    \Cnode(-0.8660,0.5){E5}%
    \Cnode(0,1){E6}%
    \Cnode(0.8660,0.5){E3}%
    \Cnode(0.8660,-0.5){E2}%
    \Cnode(0,-1){E1}%
    \Cnode(0,0){E0}%
    }
  \rput(E6){$6$}%
  \rput(E5){$5$}%
  \rput(E4){$4$}%
  \rput(E3){$3$}%
  \rput(E2){$2$}%
  \rput(E1){$1$}%
  \rput(E0){$0$}%
  %
  \ncline{E5}{E6}%
  \ncline{E4}{E5}\ncline{E4}{E6}%
  \ncline{E3}{E5}\ncline{E3}{E6}%
  \ncline{E2}{E3}\ncline{E2}{E4}\ncline{E2}{E6}%
  \ncline{E1}{E2}\ncline{E1}{E3}\ncline{E1}{E4}\ncline{E1}{E5}%
  \ncline{E0}{E1}\ncline{E0}{E2}\ncline{E0}{E3}\ncline{E0}{E4}\ncline{E0}{E5}\ncline{E0}{E6}%
  %
  \uput[ 158](E6){$\frac{1}{30}$}
  \uput[ 150](E5){$\frac{1}{50}$}
  \uput[ 210](E4){$\frac{3}{5}$}
  \uput[  22](E3){$\frac{1}{30}$}
  \uput[ -45](E2){$\frac{1}{20}$}
  \uput[-158](E1){$\frac{1}{10}$}
  \uput[  0](E0){$\sfrac{0}{6}$}
  %---------------------------------
  % mapping from die to O6c
  %---------------------------------
  \ncarc[arcangle= 22,linewidth=0.75pt,linecolor=red]{->}{D6}{E6}%
  \ncarc[arcangle= 22,linewidth=0.75pt,linecolor=red]{->}{D5}{E5}%
  \ncarc[arcangle= 22,linewidth=0.75pt,linecolor=red]{->}{D4}{E4}%
  \ncarc[arcangle=-22,linewidth=0.75pt,linecolor=red]{->}{D3}{E3}%
  \ncarc[arcangle=-22,linewidth=0.75pt,linecolor=red]{->}{D2}{E2}%
  \ncarc[arcangle=-22,linewidth=0.75pt,linecolor=red]{->}{D1}{E1}%
  %---------------------------------
  % labels
  %---------------------------------
  \rput(0,0){$\rvX(\cdot)$}%
  %\ncline[linestyle=dotted,nodesep=1pt]{->}{xzlabel}{xz}%
  %\ncline[linestyle=dotted,nodesep=1pt]{->}{ylabel}{y}%
\end{pspicture}
}%}%
%\end{tabular}
\begin{proof}
\begin{align*}
  \ocscen(\ocsG_9)
    &\eqd \argmin_{x\in\ocsG_9}\max_{y\in\ocsG_9}\metric{x}{y}\psp(y)
    &&\text{by definition of $\ocscen$ \xref{def:ocscen}}
  \\&= \argmin_{x\in\ocsG_9}\max_{y\in\ocsG_9}\metric{x}{y}\frac{1}{9}
    &&\text{by definition of $\ocsG_9$}
  \\&= \argmin_{x\in\ocsG_9}\max_{y\in\ocsG_9}\metric{x}{y}
    && \text{because $\fphi(x)=\frac{1}{9}x$ is \prope{strictly isotone} and by \prefp{lem:argminmaxphi}}
  \\&= \argmin_{x\in\ocsG_9}\setn{8,\,8,\,8,\,8,\,8,\,8,\,8,\,8,\,8}
    && \text{because the maximum distance in $\ocsG_9$ from any $x$ is $8$}
  \\&= \setn{0,1,2,\cdots,8}
    && \text{because the distances for values of $x$ in $\ocsG_9$ are the same}
  \\
  \ocscen(\ocsG)
    &= \setn{3}
    && \text{by \prefpp{ex:wring5shortd}}
  \\
  \ocsE(\rvZ)
    &= \rvZ\brs{\ocscen(\ocsG)}
    && \text{because $\ocsG$ and $\omsH$ are \prope{isomorphic} under $\rvZ$}
  \\&= \rvZ\brs{\setn{3}}
    && \text{by $\ocscen(\ocsG)$ result}
  \\&= \setn{3}
    && \text{by definition of $\rvZ$}
  \\
  \ocsVar(\rvZ;\ocsE)
    &= \ocsVaro(\ocscen(\ocsG))
    && \text{because $\ocsG$ and $\omsH$ are \prope{isomorphic} under $\rvZ$}
  \\&= \frac{4}{3} \approx 1.333
    && \text{by \prefpp{ex:wring5shortd}}
\end{align*}
\end{proof}


%=======================================
\subsubsection{Genomic signal processing (GSP) examples}
%=======================================
\begin{figure}[h]
  \gsize%
  \centering%
  %%============================================================================
% Daniel J. Greenhoe
% LaTeX file
% DNA to real line and integer line
%============================================================================
{%\psset{unit=0.5\psunit}%
\begin{pspicture}(-5.4,-1.75)(5.4,1.75)%
  %---------------------------------
  % options
  %---------------------------------
  \psset{%
    linecolor=blue,%
    radius=1.25ex,
    labelsep=2.5mm,
    }%
  %---------------------------------
  % DNA outcome space
  %---------------------------------
  \rput(0,0){%
    \Cnode[fillstyle=solid,fillcolor=snode](-1, 1){Da}%
    \Cnode[fillstyle=solid,fillcolor=snode]( 1, 1){Dt}%
    \Cnode[fillstyle=solid,fillcolor=snode](-1,-1){Dc}%
    \Cnode[fillstyle=solid,fillcolor=snode]( 1,-1){Dg}%
    \uput[-90](0,1){$\ocsG$}
    }%
  \ncline{Dc}{Dg}%
  \ncline{Dt}{Dc}\ncline{Dt}{Dg}%
  \ncline{Da}{Dt}\ncline{Da}{Dc}\ncline{Da}{Dg}%
  %
  \rput(Dg){$\symG$}%
  \rput(Dc){$\symC$}%
  \rput(Dt){$\symT$}%
  \rput(Da){$\symA$}%
  %
  \uput[ -45](Dg){$\frac{1}{4}$}%
  \uput[-135](Dc){$\frac{1}{4}$}%
  \uput[  45](Dt){$\frac{1}{4}$}%
  \uput[ 135](Da){$\frac{1}{4}$}%
  %---------------------------------
  % real line
  %---------------------------------
  \rput(-3.5,0){\psset{unit=0.5\psunit}%
    \pnode(0,2.5){RB}%
    \pnode(0,1.5){R3}%
    \pnode(0,0.5){R2}%
    \pnode(0,0){R12}%
    \pnode(0,-0.5){R1}%
    \pnode(0,-1.5){R0}%
    \pnode(0,-2.5){RA}%
    }%
  \ncline{<->}{RA}{RB}
  \rput(R3){\psline(-0.1,0)(0.1,0)}%
  \rput(R2){\psline(-0.1,0)(0.1,0)}%
  \rput(R1){\psline(-0.1,0)(0.1,0)}%
  \rput(R0){\psline(-0.1,0)(0.1,0)}%
  %
  \rput(R12){\pscircle[fillstyle=solid,linecolor=snode,fillcolor=snode](0,0){1ex}}%
  %\rput(X233){\pscircle[fillstyle=none,linecolor=red,fillcolor=red](0,0){1ex}}%
  %
  \uput[180](R3){$3$}%
  \uput[180](R2){$2$}%
  \uput[180](R1){$1$}%
  \uput[180](R0){$0$}%
  \uput[165](RB){$\omsR$}
  %
  \ncarc[arcangle=-22,linewidth=0.75pt,linecolor=red]{->}{Dg}{R1}%
  \ncarc[arcangle=-22,linewidth=0.75pt,linecolor=red]{->}{Dc}{R0}%
  \ncarc[arcangle=-22,linewidth=0.75pt,linecolor=red]{->}{Dt}{R3}%
  \ncarc[arcangle=-22,linewidth=0.75pt,linecolor=red]{->}{Da}{R2}%
  %---------------------------------
  % integer line
  %---------------------------------
  \rput(3.5,0){\psset{unit=0.5\psunit}%
    \pnode(0,2.5){ZB}%
    \Cnode(0,1.5){Z3}%
    \Cnode[fillstyle=solid,fillcolor=snode](0,0.5){Z2}%
    \Cnode[fillstyle=solid,fillcolor=snode](0,-0.5){Z1}%
    \Cnode(0,-1.5){Z0}%
    \pnode(0,-2.5){ZA}%
    }%
  \ncline[linestyle=dotted]{Z3}{ZB}%
  \ncline{Z2}{Z3}%
  \ncline{Z1}{Z2}%
  \ncline{Z0}{Z1}%
  \ncline[linestyle=dotted]{ZA}{Z0}%
  %
  \rput(Z3){$3$}%
  \rput(Z2){$2$}%
  \rput(Z1){$1$}%
  \rput(Z0){$0$}%
  \uput[15](ZB){$\omsZ$}
  %
  \ncarc[arcangle=-22,linewidth=0.75pt,linecolor=green]{->}{Dg}{Z1}%
  \ncarc[arcangle=-22,linewidth=0.75pt,linecolor=green]{->}{Dc}{Z0}%
  \ncarc[arcangle=-22,linewidth=0.75pt,linecolor=green]{->}{Dt}{Z3}%
  \ncarc[arcangle=-22,linewidth=0.75pt,linecolor=green]{->}{Da}{Z2}%
  %---------------------------------
  % labels
  %---------------------------------
  \rput(-2.5,0){$\rvX(\cdot)$}%
  \rput(2.5,0){$\rvY(\cdot)$}%
\end{pspicture}
}%%
  \includegraphics{sto/graphics/gspRZ.pdf}%
  \caption{DNA random variable mappings to \structe{real line} and \structe{integer line} \xref{ex:gspRZ}\label{fig:gspRZ}}
\end{figure}
%---------------------------------------
\begin{example}[\exmd{DNA to linear structures}]
\label{ex:gspRZ}
%---------------------------------------
\structe{Genomic Signal Processing} (\structe{GSP}) analyzes biological sequences called \structe{genome}s.
These sequences are constructed over a set of 4 symbols that are commonly referred to as 
$\symA$, $\symT$, $\symC$, and $\symG$,
each of which corresponds to a nucleobase (adenine,  thymine, cytosine, and guanine, 
respectively).\footnote{
  \citePc{mendel1853e}{Mendel (1853): gene coding uses discrete symbols},
  \citePpc{watson1953}{737}{Watson and Crick (1953): gene coding symbols are adenine,  thymine, cytosine, and guanine},
  \citePp{watson1953may}{965},
  \citerpg{pommerville2013}{52}{1449647960}
  }
A typical genome sequence contains a large number of symbols 
(about 3 billion for humans, 29751 for the SARS virus).%
\footnote{%
  \citeWuc{genbank}{http://www.ncbi.nlm.nih.gov/genome/guide/human/}{Homo sapiens, NC\_000001--NC\_000022 (22 chromosome pairs), NC\_000023 (X chromosome), NC\_000024 (Y chromosome), NC\_012920 (mitochondria)},
  \citeWuc{genbank}{http://www.ncbi.nlm.nih.gov/nuccore/30271926}{SARS coronavirus, NC\_004718.3}
  \citePc{gregory2006}{homo sapien chromosome 1},
  \citePc{he2004}{SARS coronavirus}
  }
%\\[0.3ex]\begin{minipage}{\tw-45mm}%
Let $\ocsG\eqd\ocs{\setn{\symA,\symT,\symC,\symG}}{\metricn}{\emptyset}{\psp}$ 
be the \structe{outcome subspace} \xref{def:ocs} generated by a \structe{genome}
where $\metricn$ is the \fncte{discrete metric} \xref{def:dmetric},
$\orel=\emptyset$ indicates a completely unordered set \xref{def:order}, and
$\psp(\symA)=\psp(\symT)=\psp(\symC)=\psp(\symG)=\frac{1}{4}$ (uniformly distributed).
Let $\omsH\eqd\omsR$ be the \structe{real line} \xref{def:Rline}. 
This yields the following statistics:
\\$\indentx\begin{array}{>{\gsize}Mrcc rcccl}
  geometry of $\ocsG$:                                  & \ocscen (\ocsG) &=& \mc{4}{l}{\setn{\symA,\symT,\symC,\symG}}  \\
  traditional statistics on real line:                  & \pE  (\rvX)     &=& 1.5          & \ocsVar(\rvX;\pE)   &=& \frac{5}{4} &=& 1.25 \\
  outcome subspace statistics on real line:             & \ocsE (\rvX)    &=& 1.5          & \ocsVar(\rvX;\ocsE) &=& \frac{5}{4} &=& 1.25\\
  outcome subspace statistics on integer line:          & \ocsE (\rvY)    &=& \setn{1,2}   & \ocsVar(\rvY;\ocsE) &=& \frac{1}{2} &=& 0.5
\end{array}$\\
%\\
The symbols $\symA$, $\symT$, $\symC$ and $\symG$ in general again have 
an order structure and a \structe{metric geometry} \xref{rem:mgeo} 
that is fundamentally dissimilar from that mapped to by the random variables $\rvX$ and $\rvY$.
Therefore, statistical inferences made using these random variables will likely lead to results 
that arguably have little relationship with intuition or reality.
%(see figure to the right where genomic struture is has the discrete metric topology illustrated with an undirected graph).
\end{example}
\begin{proof}
    \begin{align*}
      \ocscen(\ocsG)
        &= \setn{\symA,\,\symT,\,\symC,\,\symG}
        && \text{by \prefp{ex:dna}}
      \\
      \pE(\rvX) 
        &\eqd \int_{\R} x\psp(x) \dx
        && \text{by definition of $\pE$ \xref{def:pE}}
      \\&= \sum_{x\in\Z} x\psp(x)
        && \text{by definition of $\psp$ and \prefp{prop:pE}}
      \\&= \frac{1}{4} \sum_{x\in\setn{0,1,2,3}} x
        && \text{by definitions of $\ocsG$, $\omsH$ and $\rvX$}
      \\&= \frac{1}{4}\brp{0+1+2+3}
         = \frac{6}{4}
         = \frac{3}{2}
         = 1.5
      \\
      \ocsVar(\rvX;\pE)
        &= \pVar(\rvX)
        && \text{by \prefp{thm:ocsVar}}
      \\&= \int_{\R} \brs{x-\pE(\rvX)}^2\psp(x)
        && \text{by definition of $\pVar$ \xref{def:pVar}}
      \\&= \sum_{x\in\Z} \brs{x-\pE(\rvX)}^2\psp(x)
        &&\text{by definition of $\psp$ and \prefp{prop:pE}}
      \\&=   \frac{1}{4}\sum_{x\in\omsH} \brs{x-\frac{3}{2}}^2
        &&\text{by $\pE(\rvX)$ result}
      \\&= \mathrlap{\frac{1}{4}\brs{\brp{0-\frac{3}{2}}^2 + \brp{1-\frac{3}{2}}^2 + \brp{2-\frac{3}{2}}^2 + \brp{3-\frac{3}{2}}^2}}
      \\&= \mathrlap{\frac{1}{4\cdot2^2}\brs{\brp{0-3}^2 + \brp{2-3}^2 + \brp{4-3}^2 + \brp{6-3}^2}
         = \frac{20}{16}= \frac{5}{4} = 1.25}
      \\
      \ocsE(\rvX)
        &= \pE(\rvX)
        && \text{because on \structe{real line}, $\psp$ is \prope{symmetric}, and by \prefp{thm:pEocsE}}
      \\&= \frac{3}{2}= 1.5
        && \text{by $\pE(\rvX)$ result}
      \\
      \ocsE(\rvX)
        &\eqd \argmin_{x\in\R}\max_{y\in\R}\metric{x}{y}\psp(y)
        &&\text{by definition of $\ocsE$ \xref{def:ocsE}.\qquad\textbf{(alternate proof)}}
      \\&=    \argmin_{x\in\R}\max_{y\in\R}\abs{x-y}\psp(y)
        &&\text{by definition usual metric on real line}
      \\&=    \argmin_{x\in\R}\max_{y\in\setn{0,1,2,3}}\abs{x-y}\frac{1}{4}
        &&\text{by definition of $\ocsG$}
      \\&=    \argmin_{x\in\R}\max_{y\in\setn{0,1,2,3}}\abs{x-y}
        && \text{because $\ff(x)=\frac{1}{4}x$ is \prope{strictly isotone} and by \prefpp{lem:argminmaxphi}}
      \\&= \mathrlap{%
           \argmin_{x\in\R}\brb{\begin{array}{lM}
             \abs{x-3} & for $x\le\frac{3}{2}$\\
             \abs{x-0} & otherwise
           \end{array}}
           %\qquad\text{\gsize\centering\psset{unit=6mm}{%============================================================================
% Daniel J. Greenhoe
% XeLaTeX file
%============================================================================
\begin{pspicture}(-0.75,-0.5)(7.25,2)%
  \psset{%
    labelsep=1pt,
    linewidth=1pt,
    yunit=0.5\psunit,
    }%
  \psaxes[linecolor=axis,yAxis=false,labels=none]{->}(0,0)(0,0)(3.5,4)% x axis
  \psaxes[linecolor=axis,xAxis=false,labels=none]{->}(0,0)(0,0)(3.5,4)% y axis
  \psline(0,3)(1.5,1.5)(3,3)%
  %
  \psline[linestyle=dotted,linecolor=red](1.5,1.5)(1.5,0)%
  \psline[linestyle=dotted,linecolor=red](0,1.5)(1.5,1.5)%
  %
  \uput[0]{0}(3.5,0){$x$}%
  \uput[-90]{0}(1.5,0){$\sfrac{3}{2}$}%
  \uput{5pt}[180]{0}(0,3){$3$}%
  \uput[180]{0}(0,1.5){$\sfrac{3}{2}$}%
  \rput[l](2.75,1.25){$\ds\ff(x)\eqd\max_{y\in\setn{0,1,2,3}}\abs{x-y}$}%
\end{pspicture}%
}}
           \qquad\begin{array}{l}\includegraphics{sto/graphics/dna_Rmax.pdf}\end{array}%
           }
      \\&= \setn{\frac{3}{2}} = 1\frac{1}{2} = 1.5
        && \text{because expression is minimized at argument $x=\frac{3}{2}$}
      \\
      \ocsVar(\rvX;\ocsE)
        &= \ocsVar(\rvX;\pE)
        && \text{because $\ocsE(\rvX)=\pE(\rvX)$}
      \\&= \frac{5}{4}
        && \text{by $\ocsVar(\rvX;\pE)$ result}
      \\
      \ocsE(\rvY)
        &\eqd \argmin_{x\in\Z}\max_{y\in\Z}\metric{x}{y}\psp(y)
        &&\text{by definition of $\ocsE$ \xref{def:ocsE}}
      \\&= \argmin_{x\in\Z}\max_{y\in\Z}\abs{x-y}\psp(y)
        &&\text{by definition of \structe{integer line} \xref{def:Zline}}
      \\&= \argmin_{x\in\Z}\max_{y\in\setn{0,1,2,3}}\abs{x-y}\frac{1}{4}
      \\&= \argmin_{x\in\Z}\max_{y\in\setn{0,1,2,3}}\abs{x-y}
        && \text{because $\fphi(x)=\frac{1}{4}x$ is \prope{strictly isotone} and by \prefp{lem:argminmaxphi}}
      \\&=\mathrlap{\argmin_{x\in\setn{0,1,2,3}}\max_{y\in\setn{0,1,2,3}}
             \setn{\begin{array}{cccc}
                0 & 1 & 2 & 3\\
                1 & 0 & 1 & 2\\
                2 & 1 & 0 & 1\\
                3 & 2 & 1 & 0
             \end{array}}
      = \argmin_{x\in\setn{0,1,2,3}}
             \setn{\begin{array}{c}
                3\\
                2\\
                2\\
                3
             \end{array}}
      = \argmin_{x\in\setn{0,1,2,3}}
             \setn{\begin{array}{c}
                \mbox{}\\
                1\\
                2\\
                \mbox{}
             \end{array}}}
      \\
      \ocsVar(\rvY;\ocsE)
        &\eqd \sum_{x\in\Z}\metricsq{\ocsE(\rvY)}{x}\psp(x)
        && \text{by definition of $\ocsVar$ \xref{def:ocsVar}}
      \\&= \sum_{x\in\Z}\metricsq{\setn{1,2}}{x}\psp(x)
        && \text{by $\ocsE(\rvY)$ result}
      \\&= \mathrlap{
             \abs{0-1}^2\times\frac{1}{4} + 
             \abs{1-1}^2\times\frac{1}{4} + 
             \abs{2-2}^2\times\frac{1}{4} + 
             \abs{3-2}^2\times\frac{1}{4} 
         = \frac{1}{2} = 0.5}
    \end{align*}
\end{proof}



%---------------------------------------
\begin{example}[\exmd{GSP to complex plane}]
\label{ex:gsp_C}
%---------------------------------------
\mbox{}\\\begin{minipage}{\tw-46mm}%
A possible solution for the GSP problem \xref{ex:gspRZ} is to map $\setn{\symA, \symT, \symC, \symG}$
to the \structe{complex plane} \xref{ex:Cplane} 
rather than the \structe{real line} \xref{def:Rline} such that (see also illustration to the right)
\\\indentx$\begin{array}{rcrcl@{\qquad}rcrcl}
  a&\eqd&\rvX(\symA) &=& -1+i  & t&\eqd&\rvX(\symT) &=& 1+i\\
  c&\eqd&\rvX(\symC) &=& -1-i  & g&\eqd&\rvX(\symG) &=& 1-i .
\end{array}$
\end{minipage}%
\hfill%
{\begin{tabular}{c}%
  \gsize%
  \psset{unit=5mm}%
  %{%============================================================================
% Daniel J. Greenhoe
% LaTeX file
% genomic metric with discrete metric topology mapped to complex plain
%============================================================================
{%\psset{unit=0.5\psunit}%
\begin{pspicture}(-1,-2)(8,2)%
  %---------------------------------
  % options
  %---------------------------------
  \psset{%
    linecolor=blue,%
    radius=1.25ex,
    labelsep=2.5mm,
    }%
  %---------------------------------
  % genome graph
  %---------------------------------
  \rput(1.2,0){%
    \Cnode[fillstyle=solid,fillcolor=snode](-1, 1){A}%
    \Cnode[fillstyle=solid,fillcolor=snode]( 1, 1){T}%
    \Cnode[fillstyle=solid,fillcolor=snode](-1,-1){C}%
    \Cnode[fillstyle=solid,fillcolor=snode]( 1,-1){G}%
    }%
  \ncline{C}{G}%
  \ncline{T}{C}\ncline{T}{G}%
  \ncline{A}{T}\ncline{A}{C}\ncline{A}{G}%
  \rput(G){$\symG$}%
  \rput(C){$\symC$}%
  \rput(T){$\symT$}%
  \rput(A){$\symA$}%
  %---------------------------------
  % complex plain
  %---------------------------------
  \rput(6,0){%
    \pnode(-1, 1){a}%
    \pnode( 1, 1){t}%
    \pnode(-1,-1){c}%
    \pnode( 1,-1){g}%
    \psaxes[linecolor=axis,labels=none,linewidth=0.5pt]{<->}(0,0)(-1.5,-1.5)(1.5,1.5)%
    %\psline{<->}(-1.25,0)(1.25,0)% x axis
    %\psline{<->}(0,-1.25)(0,1.25)% y axis
    }%
  %\ncline{a}{t}%
  %\ncline{g}{a}%
  %\ncline{c}{g}%
  %
  \rput(g){$\bullet$}%
  \rput(c){$\bullet$}%
  \rput(t){$\bullet$}%
  \rput(a){$\bullet$}%
  %
  \uput[ -45](g){$g$}%
  \uput[-135](c){$c$}%
  \uput[  45](t){$t$}%
  \uput[ 135](a){$a$}%
  %---------------------------------
  % mapping from die to L6
  %---------------------------------
  \ncarc[arcangle=-22,linewidth=0.75pt,linecolor=red]{->}{G}{g}%
  \ncarc[arcangle= 22,linewidth=0.75pt,linecolor=red]{->}{C}{c}%
  \ncarc[arcangle=-22,linewidth=0.75pt,linecolor=red]{->}{T}{t}%
  \ncarc[arcangle= 22,linewidth=0.75pt,linecolor=red]{->}{A}{a}%
  %\ncline[linewidth=0.75pt,linecolor=red]{->}{b}{lb}%
  %---------------------------------
  % labels
  %---------------------------------
  \rput(3.5,-0.3){$\rvX(\cdot)$}%
  %\ncline[linestyle=dotted,nodesep=1pt]{->}{xzlabel}{xz}%
  %\ncline[linestyle=dotted,nodesep=1pt]{->}{ylabel}{y}%
\end{pspicture}
}%}%
  {\includegraphics{sto/graphics/dmetricdnatoc.pdf}}%
\end{tabular}}
However, this solution also is arguably unsatisfactory for two reasons:
\begin{enumerate}
  \item The order structures are dissimilar. Note that 
        \\\indentx$c<a$, but $\symC$ and $\symA$ are \prope{incomparable} \xref{def:order}.
  \item The metric geometries are dissimilar. 
        Let $\metricn$ be the \fncte{discrete metric} and $\metrican$ the \fncte{usual metric} in $\C$.
        Note that 
        \\\indentx$\metric{\symA}{\symT} = \metric{\symA}{\symC} = \metric{\symA}{\symG} = 1$, but
        \\\indentx$\metrica{a}{t}=\abs{a-t}=2 \neq 2\sqrt{2}=\abs{a-g}=\metrica{a}{g}$.
\end{enumerate}
\end{example}



%---------------------------------------
\begin{minipage}{\tw-70mm}%
\begin{example}[\exmd{DNA mapping with extended range}]
\label{ex:dnaXO4c}\mbox{}\\
%---------------------------------------
\prefpp{ex:gspRZ} presented a mapping from a DNA structure to a linearly ordered lattices,
but the order and metric geometry was not preserved.
In this example, a different structure is used that does preserve both order and metric geometry
(see illustration to the right). This yields the following statistics:
\\\indentx$\ocsE(\rvX) = \setn{0} \qquad  \ocsVar (\rvX) = \frac{1}{4}$
\end{example}
\end{minipage}\hfill%
\begin{tabular}{c}
  \gsize%
  %\psset{unit=5mm}%
  %{%============================================================================
% Daniel J. Greenhoe
% LaTeX file
% DNA graph to O4 with center
%============================================================================
{%\psset{unit=0.5\psunit}%
\begin{pspicture}(-3.2,-1.75)(3.2,1.75)%
  %---------------------------------
  % options
  %---------------------------------
  \psset{%
    radius=1.25ex,
    labelsep=2.5mm,
    linecolor=blue,%
    }%
  %---------------------------------
  % dice graph
  %---------------------------------
  \rput(-1.75,0){%\psset{unit=2\psunit}%
    \Cnode[fillstyle=solid,fillcolor=snode]( 1,-1){DG}%
    \Cnode[fillstyle=solid,fillcolor=snode](-1,-1){DC}%
    \Cnode[fillstyle=solid,fillcolor=snode]( 1, 1){DT}%
    \Cnode[fillstyle=solid,fillcolor=snode](-1, 1){DA}%
    \uput[-90](0,1){$\ocsG$}%
    }
  \rput(DG){$\symG$}%
  \rput(DC){$\symC$}%
  \rput(DT){$\symT$}%
  \rput(DA){$\symA$}%
  %
  \ncline{DC}{DG}\naput[labelsep=0pt]        {${\scy\metric{\symC}{\symG}=}1$}%
  \ncline{DT}{DG}\naput[labelsep=0pt,nrot=:U]{${\scy\metric{\symT}{\symG}=}1$}%
  \ncline{DA}{DT}\naput[labelsep=0pt]        {${\scy\metric{\symA}{\symT}=}1$}%
  \ncline{DC}{DA}\naput[labelsep=0pt,nrot=:U]{${\scy\metric{\symA}{\symC}=}1$}%
  \ncline{DC}{DT}\naput[labelsep=0pt,nrot=:U,npos=0.25]{${\scy\metric{\symT}{\symC}=}1$}%
  \ncline{DA}{DG}\naput[labelsep=0pt,npos=0.5,nrot=:U]{${\scy\metric{\symA}{\symG}=}1$}%
  %
  \uput[ -90](DG){${\scy\psp(\symG)=}\frac{1}{4}$}
  \uput[ -90](DC){${\scy\psp(\symC)=}\frac{1}{4}$}
  \uput[  90](DT){${\scy\psp(\symT)=}\frac{1}{4}$}
  \uput[  90](DA){${\scy\psp(\symA)=}\frac{1}{4}$}
  %---------------------------------
  % range graph
  %---------------------------------
  \rput(1.75,0){%\psset{unit=2\psunit}%
    \Cnode( 1,-1){Dg}%
    \Cnode(-1,-1){Dc}%
    \Cnode( 1, 1){Dt}%
    \Cnode(-1, 1){Da}%
    \Cnode[fillstyle=solid,fillcolor=snode](0, 0){D0}%
    \rput(0,-0.75){$\omsH$}%
    }
  \rput(Dg){$g$}%
  \rput(Dc){$c$}%
  \rput(Dt){$t$}%
  \rput(Da){$a$}%
  \rput(D0){$0$}%
  %
  \ncline{Da}{D0}\naput[labelsep=0pt,nrot=:U]{${\scy\metric{0}{a}=}\sfrac{1}{2}$}
  \ncline{D0}{Dt}\naput[labelsep=0pt,nrot=:U]{${\scy\metric{0}{t}=}\sfrac{1}{2}$}
  \ncline{Dc}{D0}\naput[labelsep=0pt,nrot=:U]{${\scy\metric{0}{c}=}\sfrac{1}{2}$}
  \ncline{D0}{Dg}\naput[labelsep=0pt,nrot=:U]{${\scy\metric{0}{g}=}\sfrac{1}{2}$}%
  %\ncline{Dc}{Dg}\naput*{$1$}%
  %\ncline{Dt}{Dg}\naput*{$1$}%
  %\ncline{Da}{Dt}\naput*{$1$}
  %\ncline{Da}{Dc}\naput*{$1$}%
  %
  \uput[ -90](Dg){${\scy\psp(g)=}\frac{1}{4}$}
  \uput[ -90](Dc){${\scy\psp(c)=}\frac{1}{4}$}
  \uput[  90](Dt){${\scy\psp(t)=}\frac{1}{4}$}
  \uput[  90](Da){${\scy\psp(a)=}\frac{1}{4}$}
  \uput[20](D0){${\scy\psp(0)=}\frac{0}{4}$}
  %---------------------------------
  % mapping from die to O6c
  %---------------------------------
  \ncarc[arcangle=-22,linewidth=0.75pt,linecolor=red]{->}{DG}{Dg}%
  \ncarc[arcangle= 22,linewidth=0.75pt,linecolor=red]{->}{DC}{Dc}%
  \ncarc[arcangle= 22,linewidth=0.75pt,linecolor=red]{->}{DT}{Dt}%
  \ncarc[arcangle=-22,linewidth=0.75pt,linecolor=red]{->}{DA}{Da}%
  %---------------------------------
  % labels
  %---------------------------------
  \rput(0,0){$\rvX(\cdot)$}%
  %\ncline[linestyle=dotted,nodesep=1pt]{->}{xzlabel}{xz}%
  %\ncline[linestyle=dotted,nodesep=1pt]{->}{ylabel}{y}%
\end{pspicture}
}%}%
  {\includegraphics{sto/graphics/dnaXO4c.pdf}}%
\end{tabular}
\begin{proof}
\begin{align*}
  \ocsE(\omsH)
    &\eqd \argmin_{x\in\omsH}\max_{y\in\omsH}\metric{x}{y}\psp(y)
    &&\text{by definition of $\ocsE$ \xref{def:ocsE}}
  \\&= \argmin_{x\in\omsH}\max_{y\in\omsH\setd\setn{0}}\metric{x}{y}\psp(y)
    &&\text{because $\psp(0)=0$}
  \\&= \argmin_{x\in\omsH\setd\setn{0}}\max_{y\in\omsH\setd\setn{0}}\metric{x}{y}\frac{1}{4}
    && \text{by definition of $\ocsG$}
  \\&= \argmin_{x\in\omsH\setd\setn{0}}\max_{y\in\omsH\setd\setn{0}}\metric{x}{y}
    && \text{because $\fphi(x)=\frac{1}{4}x$ is \prope{strictly isotone} and by \prefp{lem:argminmaxphi}}
  %\\&= \argmin_{x\in\omsH}\max_{y\in\omsH}
  %       \setn{\begin{array}{ccccccc}
  %         \metricn(1,1) &\metricn(1,2) &\metricn(1,3) &\metricn(1,4) &\metricn(1,0)\frac{0}{4}\\
  %         \metricn(2,1) &\metricn(2,2) &\metricn(2,3) &\metricn(2,4) &\metricn(2,0)\frac{0}{4}\\
  %         \metricn(3,1) &\metricn(3,2) &\metricn(3,3) &\metricn(3,4) &\metricn(3,0)\frac{0}{4}\\
  %         \metricn(4,1) &\metricn(4,2) &\metricn(4,3) &\metricn(4,4) &\metricn(4,0)\frac{0}{4}\\
  %         \metricn(0,1) &\metricn(0,2) &\metricn(0,3) &\metricn(0,4) &\metricn(0,0)\frac{0}{4}
  %       \end{array}}
  \\&=\mathrlap{\argmin_{x\in\omsH}\max_{y\in\omsH}
         \setn{\begin{array}{cccccc}
           \metricn(1,1) &\metricn(1,2) &\metricn(1,3) &\metricn(1,4) \\
           \metricn(2,1) &\metricn(2,2) &\metricn(2,3) &\metricn(2,4) \\
           \metricn(3,1) &\metricn(3,2) &\metricn(3,3) &\metricn(3,4) \\
           \metricn(4,1) &\metricn(4,2) &\metricn(4,3) &\metricn(4,4) \\
           \metricn(0,1) &\metricn(0,2) &\metricn(0,3) &\metricn(0,4) 
         \end{array}}
    = \argmin_{x\in\omsH}\max_{y\in\omsH}
         \setn{\begin{array}{cccc}
           {0}&{1}&{1}&{1}\\
           {1}&{0}&{1}&{1}\\
           {1}&{1}&{0}&{1}\\
           {1}&{1}&{1}&{0}\\
           \frac{1}{2}&\frac{1}{2}&\frac{1}{2}&\frac{1}{2}
         \end{array}}
    = \argmin_{x\in\omsH}
         \setn{\begin{array}{c}
           {1}\\
           {1}\\
           {1}\\
           {1}\\
           \frac{1}{2}
         \end{array}}}
  \\&= \setn{0}
    && \text{because expression is minimized at $x=\setn{0}$}
  %\\
  %\ocsEa(\rvX)
  %  &\eqd \argmin_{x\in\omsH}\sum_{y}\ocsmom(x,y)
  %      &&\text{by definition of $\ocsEa$ \xref{def:ocsEa}}
  %\\&\eqd \argmin_{x\in\omsH}\sum_{y}\metric{x}{y}\psp(y)
  %      &&\text{by definition of $\ocsmom$ \xref{def:ocsmom}}
  %\\&= \argmin_{x\in\omsH}\sum_{y\in\omsH\setd\setn{0}}\metric{x}{y}\psp(y)
  %     &&\text{because $\psp(0)=0$}
  %\\&= \argmin_{x\in\omsH}\sum_{y\in\omsH\setd\setn{0}}\metric{x}{y}\frac{1}{4}
  %\\&= \argmin_{x\in\omsH}\sum_{y\in\omsH\setd\setn{0}}\metric{x}{y}
%        && \text{because $\fphi(x)=\frac{1}{4}x$ is \prope{strictly isotone} and by \prefp{lem:argminmaxphi}}
  %\\&=\mathrlap{\argmin_{x\in\omsH}
  %       \setn{\begin{array}{*{7}{c}}
  %         0           &+& 1           &+& 1           &+& 1          \\
  %         1           &+& 0           &+& 1           &+& 1          \\
  %         1           &+& 1           &+& 0           &+& 1          \\
  %         1           &+& 1           &+& 1           &+& 0          \\
  %         \frac{1}{2} &+& \frac{1}{2} &+& \frac{1}{2} &+& \frac{1}{2}
  %       \end{array}}
  %\quad= \argmin_{x\in\omsH}
  %       \setn{\begin{array}{c}
  %         3\\
  %         3\\
  %         3\\
  %         3\\
  %         2
  %       \end{array}}
  %\quad= \setn{0}}
  \\
  \ocsVar(\rvX)
    &\eqd \sum_{x\in\omsH}\metricsq{\ocsE(\rvX)}{x}\psp(x)
    && \text{by definition of $\ocsVar$ \xref{def:ocsVar}}
  \\&= \sum_{x\in\omsH}\metricsq{\setn{0}}{x}\psp(x)
    && \text{by $\ocsE(\rvX)$ result}
  \\&= \mathrlap{
       \sum_{x\in\omsH\setd\setn{0}}\brp{\frac{1}{2}}^2\frac{1}{4}
     = \seto{\omsH\setd\setn{0}}\brp{\frac{1}{2}}^2\frac{1}{4}
     = 4\brp{\frac{1}{2}}^2\frac{1}{4}
     = \frac{1}{4}
     }
\end{align*}
\end{proof}

\begin{figure}[h]
  \gsize%
  \centering%
  %%============================================================================
% Daniel J. Greenhoe
% LaTeX file
% GSP with depth 2 Markov probability model
%============================================================================
\begin{pspicture}(-7.2,-7.2)(7.2,7.2)%
  %---------------------------------
  % options
  %---------------------------------
  \psset{%
    linecolor=blue,%
    radius=1em,
    nodesep=1.5pt,
    labelsep=2.5mm,
    arrowsize=5pt,
    }%
  \rput(0,0){\psset{unit=2\psunit}%
                     \Cnode(0, 3){TT}%
    \Cnode(-1, 2){TC}\Cnode(0, 2){TG}\Cnode(1, 2){TA}%
    %
                     \Cnode(-2, 1){CT}\Cnode( 2, 1){AT}%
    \Cnode(-3, 0){CC}\Cnode(-2, 0){CA}\Cnode( 2, 0){AC}\Cnode( 3, 0){AA}%
                     \Cnode(-2,-1){CG}\Cnode( 2,-1){AG}%
    %
    \Cnode(-1,-2){GC}\Cnode(0,-2){GT}\Cnode(1,-2){GA}%
                     \Cnode(0,-3){GG}%
    }
  %
  \rput(GA){$\symG\symA$}\rput(GT){$\symG\symT$}\rput(GC){$\symG\symC$}\rput(GG){$\symG\symG$}%
  \rput(CA){$\symC\symA$}\rput(CT){$\symC\symT$}\rput(CC){$\symC\symC$}\rput(CG){$\symC\symG$}%
  \rput(TA){$\symT\symA$}\rput(TT){$\symT\symT$}\rput(TC){$\symT\symC$}\rput(TG){$\symT\symG$}%
  \rput(AA){$\symA\symA$}\rput(AT){$\symA\symT$}\rput(AC){$\symA\symC$}\rput(AG){$\symA\symG$}%
  %
  \ncline{->}{AA}{AT}\ncline{->}{AA}{AC}\ncline{->}{AA}{AG}%
  \ncline{->}{GG}{GA}\ncline{->}{GG}{GT}\ncline{->}{GG}{GC}%
  \ncline{->}{CC}{CT}\ncline{->}{CC}{CA}\ncline{->}{CC}{CG}%
  \ncline{->}{TT}{TA}\ncline{->}{TT}{TC}\ncline{->}{TT}{TG}%
  %
  {\psset{doubleline=false,linecolor=red,arcangle=10}%
    \ncarc{->}{AT}{TA}\ncarc{->}{TA}{AT}%
    \ncarc{->}{AC}{CA}\ncarc{->}{CA}{AC}%
    \ncarc{->}{AG}{GA}\ncarc{->}{GA}{AG}%
    \ncarc{->}{CT}{TC}\ncarc{->}{TC}{CT}
    \ncarc{->}{CG}{GC}\ncarc{->}{GC}{CG}%
    \ncarc{->}{GT}{TG}\ncarc{->}{TG}{GT}%
    }%
  %
  \ncline{->}{AT}{TC}\ncline{->}{AT}{TG}\ncarc[arcangle=-45,linecolor=green]{->}{AT}{TT}%
  \ncline{->}{AC}{CT}\ncline{->}{AC}{CG}\ncarc[arcangle= 45,linecolor=green]{->}{AC}{CC}%
  \ncline{->}{AG}{GT}\ncline{->}{AG}{GC}\ncarc[arcangle= 45,linecolor=green]{->}{AG}{GG}%
  \ncline{->}{TA}{AC}\ncline{->}{TA}{AG}\ncarc[arcangle= 45,linecolor=green]{->}{TA}{AA}%
  \ncline{->}{TC}{CA}\ncline{->}{TC}{CG}\ncarc[arcangle=-45,linecolor=green]{->}{TC}{CC}%
  \ncline{->}{TG}{GA}\ncline{->}{TG}{GC}\ncarc[arcangle= 45,linecolor=green]{->}{TG}{GG}%
  \ncline{->}{CA}{AT}\ncline{->}{CA}{AG}\ncarc[arcangle= 45,linecolor=green]{->}{CA}{AA}%
  \ncline{->}{CT}{TA}\ncline{->}{CT}{TG}\ncarc[arcangle= 45,linecolor=green]{->}{CT}{TT}%
  \ncline{->}{CG}{GA}\ncline{->}{CG}{GT}\ncarc[arcangle=-45,linecolor=green]{->}{CG}{GG}%
  \ncline{->}{GA}{AT}\ncline{->}{GA}{AC}\ncarc[arcangle=-45,linecolor=green]{->}{GA}{AA}%
  \ncline{->}{GT}{TA}\ncline{->}{GT}{TC}\ncarc[arcangle= 45,linecolor=green]{->}{GT}{TT}%
  \ncline{->}{GC}{CA}\ncline{->}{GC}{CT}\ncarc[arcangle= 45,linecolor=green]{->}{GC}{CC}%
  %
  \nccircle[linecolor=red,angleA=-90]{->}{AA}{1.5em}%
  \nccircle[linecolor=red,angleA=0]{->}{TT}{1.5em}%
  \nccircle[linecolor=red,angleA=90]{->}{CC}{1.5em}%
  \nccircle[linecolor=red,angleA=180]{->}{GG}{1.5em}%
  %
  %\uput[ 158](D6){$\frac{1}{6}$}
  %\uput[ 150](D5){$\frac{1}{6}$}
  %\uput[ 210](D4){$\frac{1}{6}$}
  %\uput[  22](D3){$\frac{1}{6}$}
  %\uput[ -45](D2){$\frac{1}{6}$}
  %\uput[-158](D1){$\frac{1}{6}$}
\end{pspicture}
  \includegraphics{sto/graphics/gsp_markov2.pdf}
  \caption{DNA with depth-2 Markov modelling \xref{ex:gsp_markov2}\label{fig:gsp_markov2}}
\end{figure}
%---------------------------------------
\begin{example}[\exmd{GSP with Markov model}]
\label{ex:gsp_markov2}
%---------------------------------------
Markov probability models have often been used in genomic signal processing (GSP).
A change in the statistics in the sequence may in some cases mean a change in function of the genomic sequence (DNA code).
Finding such a change in statistics then is very useful in identifying functions of segments of genomic sequences.
Let $\ocsG$ be an \structe{outcome subspace} \xref{def:ocs} representing a 
Markov model of depth 2 for a genomic sequence as illustrated in \prefpp{fig:gsp_markov2},
with joint and conditional probabilities computed over a finite window.
Let $\ocsH$ be an outcome subspace isomorphic to $\ocsG$, and 
$\rvX$ be a random variable mapping $\ocsG$ to $\omsH$.
A change in the value of the statistic $\ocsE(\rvX)$ over the window then may indicate a change in function
within the genomic sequence.
\end{example}



\end{tabstr}

%\fi
