%============================================================================
% NCTU - Hsinchu, Taiwan
% LaTeX File
% Daniel Greenhoe
%============================================================================

%======================================
\newapp{The Structure of Algebra }
\label{app:algebra}
\index{algebra}
%======================================

\begin{figure}[ht]
\begin{center}
\setlength{\unitlength}{0.8cm}
\begin{picture}(16,12)
\begin{footnotesize}
\thicklines

\put(9,11){\makebox ( 7,1)[l]{vector-vector multiplication} }
\put(1,11){\framebox( 7,1){algebra} }

\put( 4.5, 10){\line(0,1){1} }
\put(9,9){\makebox ( 7,1)[l]{vector addition, scalar-vector multiplication} }
\put(1,9){\framebox( 7,1){vector space} }

\put( 4.5, 8){\line(0,1){1} }
\put(9,7){\makebox ( 8,1)[l]{\parbox[c][][c]{8cm}
         {each non-zero element has multiplicative inverse\\ 
         (supports division)}} }
\put(1,7){\framebox( 7,1){field} }

\put( 4.5, 6){\line(0,1){1} }
\put(9,5){\makebox ( 7,1)[l]{scalar addition, subtraction, multiplication} }
\put(1, 5){\framebox( 7,1){ring} }

\put( 6.5, 4){\line(0,1){1} }
\put( 2.5, 4){\line(0,1){1} }
\put( 5  , 3){\framebox( 3,1){group under $+$} }
\put( 1  , 3){\framebox( 3,1){group under $\times$}  }

\put( 6.5, 2){\line(0,1){1} }
\put( 2.5, 2){\line(0,1){1} }
\put( 1  , 1){\framebox(7,1){set} }
\end{footnotesize}
\end{picture}
\end{center}
\caption{
   Vector algebra
   \label{fig:vect_alg}
   }
\end{figure}



%---------------------------------------
\section{Structural components}
%---------------------------------------
A set together with one or more operations forms several standard 
mathematical structures:

group $\supseteq$ ring $\supseteq$ commutative ring $\supseteq$ integral domain $\supseteq$ field

%---------------------------------------
\begin{definition}
\index{group}
%---------------------------------------
Let $X$ be a set and $\Diamond:X\times X\to X$ be an operation on $X$. \\
Then a {\bf group} is the pair $(X,\Diamond)$
such that for all $x,y,z\in X$
\citep{durbin}{29}

\begin{tabular}{lll}
   1. & There exists $e    \in X$ such that $e    \Diamond x = x\Diamond e    = x$  & (identity element) \\
   2. & There exists $(-x) \in X$ such that $(-x) \Diamond x = x\Diamond (-x) = e$  & (inverse) \\
   3. & $x\Diamond(y\Diamond z) = (x\Diamond y)\Diamond z$                          & (associtive).
\end{tabular}
\end{definition}


%---------------------------------------
\begin{definition}
\index{ring}
%---------------------------------------
Let $X$ be a set and $+:X\times X\to X$ and $*:X\times X\to X$ be operations on $X$. 
Furthermore, let the operation $*$ also be represented by juxtapostion as in $a*b=ab$.\\
Then a {\bf ring} is the tripple $(X,+,*)$
such that for all $x,y,z\in X$
\citepp{durbin}{114}{115}

\begin{tabular}{lll}
   1. & $(X,+,*)$ is a group with respect to $(X,+)$ & (additive group)                             \\
   2. & $x(yz)  = (xy)z    $                         & (associative with respect to $*$) \\
   3. & $x(y+z) = (xy)+(xz)$                         & ($*$ is left distributive over $+$)          \\
   4. & $(x+y)z = (xz)+(yz)$                         & ($*$ is right distributive over $+$).
\end{tabular}
\end{definition}

%---------------------------------------
\begin{definition}
\index{ring!characteristic}
%---------------------------------------
   Let $R=(A,+,*)$ be a ring.
Then the {\bf characteristic} of $R$ is defined such that
\citep{durbin}{128}
\[
   \Rchar{R}=
      \left\{
      \begin{array}{ll}
         \min\{n\in\Zp:na=0, \; \forall a\in A\} & \mbox{ if } \{n\in\Zp:na=0\}\not= \emptyset \\
         0                                       & \mbox{ if } \{n\in\Zp:na=0\}    = \emptyset
      \end{array}
      \right.
\]
\end{definition}


%---------------------------------------
\begin{definition}
\index{ring!commutative}
%---------------------------------------
The tripple $(X,+,*)$ is a {\bf commutative ring} if for all $x,y,z\in X$
\citep{durbin}{118}

\begin{tabular}{lll}
   1. & $(X,+,*)$ is a ring  & (ring) \\
   2. & $xy=yx$              & (commutative).
\end{tabular}
\end{definition}

%---------------------------------------
\begin{definition}
\index{integral domain}
%---------------------------------------
Let 
\begin{liste}
   \item $0$ be the identity element with respect to operator $+$
   \item $1$ be the identity element with respect to operator $*$.
\end{liste}
Then the tripple $(X,+,*)$ is an {\bf integral domain} if for all $x,y\in X$
\citep{durbin}{120}

\begin{tabular}{lll}
   1. & $(X,+,*)$ is a ring      & (ring)           \\
   2. & $xy=yx$                  & (commutative with respect to $*$) \\
   3. & $1\in X$                 & (multiplicative identity element)  \\
   4. & for all $x,y\in X-\{0\}$, $xy\ne0$ & (no zero divisors).
\end{tabular}
\end{definition}



%---------------------------------------
\begin{theorem}
%---------------------------------------
Let $(X,+,*)$ be an integral domain and $x,y,z\in X$.
Then \citep{durbin}{120}
\begin{marray}
   xy &=& xz &\implies& y &=& z & \mbox{ (left cancellation)}\\
   xy &=& zy &\implies& x &=& z & \mbox{ (right cancellation)}
\end{marray}
\end{theorem}

\begin{proof}
\begin{marray}
            & xy    &=& xz                                      \\ 
   \implies & xy-xz &=& 0       & \mbox{ by ring property}      \\
   \implies & x(y-z)&=& 0       & \mbox{ by ring property}      \\
   \implies & (y-z) &=& 0       & \mbox{ because integral domain} \\
   \implies & y     &=& z       & \mbox{ }
\end{marray}

\begin{marray}
            & xy    &=& zy                                      \\ 
   \implies & xy-zy &=& 0       & \mbox{ by ring property}      \\
   \implies & (x-z)y&=& 0       & \mbox{ by ring property}      \\
   \implies & (x-z) &=& 0       & \mbox{ because integral domain} \\
   \implies & x &=& z           & \mbox{ }
\end{marray}
\end{proof}





%---------------------------------------
\begin{definition}
\index{field}
%---------------------------------------
The tripple $(X,+,*)$ is a {\bf field} if for all $x,y\in X$
\citep{durbin}{123}

\begin{tabular}{lll}
   1. & $(X,+,*)$ is a ring      & (ring)           \\
   2. & $xy=yx$                  & (commutative with respect to $*$) \\
   3. & $(X-\{0\},*)$ is a group & (group with respect to $*$).
\end{tabular}
\end{definition}


%---------------------------------------
\section{Vector structures}
%---------------------------------------
%---------------------------------------
\begin{definition}
%\label{def:vspace}
\index{vector space}
%---------------------------------------
An $n$-dimensional {\bf vector space} $V$ is the tupple $(F^n,+,\cdot)$
where 
\begin{liste}
   \item $F$ is the ground field,
   \item $+:V\times V\to V$ is a vector-vector addition operator, and 
   \item $\cdot:F\times V\to V$ is scalar-vector multiplication operator 
\end{liste}
such that the following properties are satisfied
for $v,w\in V$ and $\alpha,\beta\in F$: \citep{wicker}{29}

\begin{tabular}{lll}
   1. & commutative:           & $v+w = w+v$                                           \\
   2. & scalar multiplication: & $\alpha\in F$ and $v\in V$ implies $(\alpha v)\in V$  \\
   3. & scalar distributive:   & $\alpha(v+w)=(\alpha v)+(\alpha w)$                   \\
   4. & vector distributive:   & $(\alpha+\beta)v =(\alpha v)+(\beta v)$                \\
   5. & associative:           & $(\alpha\beta)v = \alpha(\beta v)$
\end{tabular}
\end{definition}


%---------------------------------------
\begin{definition}
%\label{def:algebra}
\index{algebra}
%---------------------------------------
An {\bf algebra} is an ordered pair $(V,\otimes)$ where 
\begin{liste}
   \item $V$ is a vector space and 
   \item $\otimes:V\times V\to V$ is a vector-vector multiplication operator
\end{liste}
with the following properties for $u,v,w\in V$,
$\alpha$ in the ground field of $V$,
and $\otimes$ is represented by juxtaposition:
\citep{aa}{3} \citep{michel}{56}

\begin{tabular}{lll}
   1. & associative:        & $ (uv)w = u(vw)            $ \\
   2. & left distributive:  & $ u( v \oplus w) = (uv) \oplus (uw) $ \\
   3. & right distributive: & $ ( u \oplus v)w = (uw) \oplus (vw) $ \\
   4. & scalar commutative: & $ \alpha(vw) = (\alpha v)w = v (\alpha w)                      $
\end{tabular}
\end{definition}








