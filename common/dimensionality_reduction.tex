%============================================================================
% LaTeX File
% Daniel J. Greenhoe
%============================================================================
%=======================================
\chapter{Dimensionality Reduction}
\label{chp:dimreduct}
%=======================================
%=======================================
%\section{Introduction}
%=======================================
A number of statistical estimation algorithms yield subspace structures with 
high dimensionality---one key example is \ope{Kernel Density Estimation} (\ope{KDE})\xref{chp:pdfest}.
Richard Bellman described this problem as the 
``curse of dimensionality".\footnote{
  \citerp{bellman1954}{206},
  \citerpc{bellman1961}{94}{\textsection ``5.16 The Curse of Dimensionality"},
  \citerp{bellman1971}{44},
  \citerppgc{bishop2006}{33}{38}{9780387310732}{\textsection1.4 ``The Curse of Dimensionality"},
  \citerpgc{gramacki2017}{3, 59}{9783319716886}{\textsection3.9 ``The Curse of Dimensionality"}
  }

Here are some references about \ope{dimensionality reduction}:
\\\begin{tabular}{ll}
    \citeP{cunningham2015}: & ``\emph{Linear Dimensionality Reduction: Survey, Insights, and Generalizations}"
  \\\citeP{sorzano2014}:    & ``\emph{A survey of dimensionality reduction techniques}"
  \\\citer{lee2007}:        & ``\emph{Nonlinear Dimensionality Reduction}" 
\end{tabular}

Algorithms for the operation include the following:
\begin{listi}
  \item \ope{Principal Component Analysis} (\ope{PCA})\footnote{
          \citeP{pearson1901}, 
          \citeP{eckart1936},
          \citerc{jolliffe2013}{9781475719048},
          }
\end{listi}