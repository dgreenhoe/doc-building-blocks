%============================================================================
% XeLaTeX File
% Daniel J. Greenhoe
%============================================================================
%=======================================
\chapter{Laplace Transform}
%=======================================
\qboxnqt{Pierre-Simon Laplace\footnotemark}
{La langue de l�analyse, la plus parfaite de toutes les langues,
�tant par elle-m/<eme un puissant instrument de d/'ecouvertes;
ses notations, lorsqu'elles sont n/'ecessaires et heureusement imagin/'ees,
sont des germes de nouveaux calculs.}
{The language of analysis, most perfect of all, being in itself a powerful instrument of discoveries,
its notations, especially when they are necessary and happily imagined, are the seeds of new calculi.}
\footnotetext{
  \citerpcu{laplace1814}{xxxi}{Introduction}{https://archive.org/details/bub_gb_6MRLAAAAMAAJ/page/n35/},
  \citer{laplace1812},
  \citerppu{laplace1902}{48}{49}{https://books.google.com/books?id=DfI7AQAAMAAJ&pg=RA1-PA48},
  \citerpcu{moritz1914}{200}{Quote 1222., but ``conceived" not ``imagined", and ``are so many germs" not ``are the seeds"}{https://archive.org/details/memorabiliamath02morigoog/page/n217/},
  \url{https://todayinsci.com/L/Laplace_Pierre/LaplacePierre-Analysis-Quotations.htm},
  \url{https://translate.google.com/},
  }
%=======================================
\section{Operator Definition}
%=======================================
%--------------------------------------
\begin{definition}
\footnote{
  \citerpgc{bracewell1978}{219}{007007013X}{Chapter 11 The Laplace transform},
  \citerpcu{vanderpol1959}{13}{5. Strip of convergence of the Laplace integral}{https://archive.org/details/in.ernet.dli.2015.141269/page/n23/mode/2up},
  \citerpcu{levy1958}{2}{``two-sided transformation"}{https://archive.org/details/c_cowles_ieee_Levy/page/n5/mode/2up},
  \citerpcu{betten2008L}{295}{(B.1)}{https://link.springer.com/content/pdf/bbm:978-3-540-85051-9/1.pdf}
  }
\label{def:opLT}
%--------------------------------------
Let $\spLLRBu$ be the space of all
\structe{Lebesgue square-integrable functions}.\\
\defboxt{
  The \opd{Laplace Transform} operator $\opLT$ is here defined as
  \\\indentx$\ds
    \brs{\hxs{\opLT} \ff}(s)
    \eqd \Lscale
    \int_{x\in\R} \ff(x) e^{-sx} \dx
    \qquad\scy\forall \ff\in\spLLRBu
  $
  %\\This definition of the Fourier Transform is also called the \opd{unitary Laplace Transform}.
  }
\end{definition}

%--------------------------------------
\begin{remark}
%--------------------------------------
A scaling factor $\frac{1}{\sqrt{2\pi}}$ in front of $\int_\R$ in \pref{def:opLT}
is not typically found in references offering
definitions of the Laplace Transform, and is not included here either.
That is not to say, however, that it's not a good idea.
Including it would make the operator $\opLT$ more directly compatible
with the \ope{Unitary Fourier Transform} operator $\opFT$\ifsxref{harFour}{def:opFT}.
Note also that a $\frac{1}{2\pi}$ scaling factor is included in
[\citerpg{bachman2002}{268}{9780387988993}] in their definition of \ope{convolution}
\xxref{def:conv}{sec:opLT_conv}.
\end{remark}

%=======================================
\section{Operator Inverse}
%=======================================
%--------------------------------------
\begin{theorem}
\footnote{
  \citerpgc{bracewell1978}{220}{007007013X}{Chapter 11 The Laplace transform}
  }
\label{thm:opLTi}
%--------------------------------------
\thmbox{\begin{array}{rc>{\ds}lM}
  \fg(x) &=& \opLTi[\fG(s)] \eqd \frac{1}{2\pi i}\int_{c-i\infty}^{c+i\infty} \fG(s) e^{sx} ds
           & for some $c\in\Rp$
\end{array}}
\end{theorem}

%=======================================
\section{Transversal properties}
%=======================================
%--------------------------------------
\begin{theorem}
\footnote{
  \citerpgc{bracewell1978}{224}{007007013X}{Table 11.1 Theorems for the Laplace Transform},
  \citerpcu{levy1958}{15}{Equation 0.8}{https://archive.org/details/c_cowles_ieee_Levy/page/n15/}
  }
\label{thm:opLT_transversal}
\label{thm:opLT_shift}
\label{thm:opLT_dilation}
%--------------------------------------
Let $\opLT$ be the \ope{Laplace Transform} operator \xref{def:opLT}.
Let $\fG(s)\eqd \brs{\opLT\fg(x)}$
and $\fF(s)\eqd \brs{\opLT\ff(x)}$.
Let the \structe{radius of convergence} of $\fG(s)$ be $A\leq\Real(s)\leq B$
and the \structe{radius of convergence} of $\fF(s)$ be $C\leq\Real(s)\leq D$.
\thmbox{\begin{array}{>{\ds}l        c>{\ds}l                                             D     l              @{\hspace{2pt}}l                                             C                       D}
    \mc{3}{H}{Mapping}                                                                  & \mc{3}{|H}{Radius of Convergence}                                               & \mc{1}{|H}{Domain}    & \mc{1}{|H}{Property}
  \\\hline
    \opLT\brs{\fg(x-\alpha)}        &=& e^{-\alpha s} \;\fG(s)                          & for & \Real(s)                    & \in\intcc{A}{B}                             & \forall x,\alpha\in\C & (\ope{translation})
  \\\opLT\brs{\fg(\alpha x)}        &=& \frac{1}{\abs{\alpha}}\fG\brp{\frac{s}{\alpha}} & for & \Real\brp{\frac{s}{\alpha}} & \in\intcc{A}{B}                             & \forall x,\alpha\in\C & (\ope{dilation})
\end{array}}
\end{theorem}
\begin{proof}
\begin{align*}
  \opLT\brs{\fg(x-\alpha)}
    &\eqd \Lscale \int_{x\in\R} \fg(x-\alpha) e^{-sx} \dx                               && \text{by definition of $\opLT$} && \text{\xref{def:opLT}}
  \\&=    \Lscale \int_{u+\alpha=-\infty}^{u+\alpha=\infty} \fg(u) e^{-s(\alpha+u)} \du && \text{where $u\eqd x-\alpha$}   && \text{$\implies$ $x=\alpha+u$}
  \\&=    e^{-\alpha s} \;\Lscale \int_{u=-\infty}^{u=\infty} \fg(u) e^{-su} \du        && \text{by property of exponents} && \text{$b^{x+\alpha}=b^xb^\alpha$}
  \\&=    e^{-\alpha s} \;\Lscale \int_{x=-\infty}^{x=\infty} \fg(x) e^{-sx} \dx        && \text{by change of variable}    && \text{$u\to x$}
  \\&\eqd e^{-\alpha s} \;\brs{\opLT \fg(x)}                                            && \text{by definition of $\opLT$} && \text{\xref{def:opLT}}
  \\&\eqd e^{-\alpha s} \;\fG(s)                                                        && \forall \Real(s)\in\intcc{A}{B} && \text{by definition of $\fG(s)$}
  \\
  \opLT\brs{\fg(\alpha x)}
    &\eqd \Lscale \int_{x=-\infty}^{x=\infty} \fg(\alpha x) e^{-sx} \dx
    && \text{by definition of $\opLT$}
    && \text{\xref{def:opLT}}
  \\&=    \Lscale \int_{u/\alpha=-\infty}^{u/\alpha=\infty} \fg(u) e^{-s(u/\alpha)} \frac{1}{\alpha}\du && \text{where $u\eqd \alpha x$}   && \text{$\implies$ $x=\frac{u}{\alpha}$}
  \\&=    \Lscale \frac{1}{\alpha}\int_{u/\alpha=-\infty}^{u/\alpha=\infty} \fg(u) e^{-(s/\alpha)u} \du
  \\&=\mathrlap{
          \brbl{\begin{array}{>{\ds}rM}
            \Lscale \frac{1}{\alpha}\int_{u=-\infty}^{u=\infty} \fg(u) e^{-(s/\alpha)u} \du & if $\alpha\ge0$
          \\[2ex]
            \Lscale \frac{1}{\alpha}\int_{u=\infty}^{u=-\infty} \fg(u) e^{-(s/\alpha)u} \du & otherwise
          \end{array}}
          }
  \\&=\mathrlap{
          \brbl{\begin{array}{>{\ds}rM}
             \Lscale \frac{1}{\alpha}\int_{u=-\infty}^{u=\infty} \fg(u) e^{-(s/\alpha)u} \du & if $\alpha\ge0$
          \\[2ex]
            -\Lscale \frac{1}{\alpha}\int_{u=-\infty}^{u=\infty} \fg(u) e^{-(s/\alpha)u} \du & otherwise
          \end{array}}
          }
  \\&=    \frac{1}{\abs{\alpha}} \Lscale \int_{x\in\R} \fg(x) e^{-(s/\alpha)x} \dx   && A \leq\Real\brp{\frac{s}{\alpha}}\leq B && \text{by change of variable $u\to x$}
  \\&\eqd \frac{1}{\abs{\alpha}}\brs{\opLT \fg(x)}\brp{\frac{s}{\alpha}}             && A \leq\Real\brp{\frac{s}{\alpha}}\leq B && \text{by definition of $\opLT$ \xref{def:opLT}}
  \\&\eqd \frac{1}{\abs{\alpha}}\fG\brp{\frac{s}{\alpha}}                            && A \leq\Real\brp{\frac{s}{\alpha}}\leq B && \text{by definition of $\fG(s)$}
\end{align*}
\end{proof}

%--------------------------------------
\begin{corollary}
\footnote{
  \citerpgc{bracewell1978}{224}{007007013X}{Table 11.1 Theorems for the Laplace Transform}
  }
\label{cor:opLT_reversal}
%--------------------------------------
Let $\opLT$, $\fG(s)$, $A$, and $B$ be defined as in \prefpp{thm:opLT_transversal}.
\corbox{\begin{array}{>{\ds}lc>{\ds}l   D       l              @{\hspace{2pt}}l     C                       D}
    \mc{3}{H}{Mapping}                  & \mc{3}{|H}{Radius of Convergence}       & \mc{1}{|H}{Domain}    & \mc{1}{|H}{Property}
  \\\hline%                             &     &              &                    &                       &
    \opLT\brs{\fg(-x)}  &=& \fG\brp{-s} & for & \Real\brp{s} & \in\intcc{-B}{-A}  & \forall x,\alpha\in\C & (\ope{reversal})
\end{array}}
\end{corollary}
\begin{proof}
\begin{align*}
  \opLT\brs{\fg(-x)}
    &= \opLT\brs{\fg([-1]x)}
    && \Real\brp{s}\in\intcc{A}{B}
    && \text{by definition of unary operator $-$}
  \\&= \opLT\brs{\frac{1}{\abs{-1}}\fg\brp{\frac{x}{-1}}}
    && \Real\brp{\frac{s}{-1}} \in\intcc{A}{B}
    && \text{by \prope{dilation} property \xref{thm:opLT_dilation}}
  \\&= \fG\brp{-s}
    && \Real\brp{s} \in\intcc{-B}{-A}
    &&
\end{align*}
\end{proof}

%=======================================
\section{Linear properties}
%=======================================
%--------------------------------------
\begin{theorem}
\footnote{
  \citerpgc{bracewell1978}{224}{007007013X}{Table 11.1 Theorems for the Laplace Transform},
  \citerpcu{betten2008H}{296}{(B.6)}{https://link.springer.com/content/pdf/bbm:978-3-540-85051-9/1.pdf},
  \citerpcu{levy1958}{13}{Equation 0.2}{https://archive.org/details/c_cowles_ieee_Levy/page/n15/}
  }
\label{thm:opLT_linear}
%--------------------------------------
Let $\opLT$ be the \ope{Laplace Transform} operator \xref{def:opLT}.
Let $\fG(s)\eqd \brs{\opLT\fg(x)}$
and $\fF(s)\eqd \brs{\opLT\ff(x)}$.
Let the \structe{radius of convergence} of $\fG(s)$ be $A\leq\Real(s)\leq B$
and the \structe{radius of convergence} of $\fF(s)$ be $C\leq\Real(s)\leq D$.
\thmbox{\begin{array}{>{\ds}l        c>{\ds}l                                             D     l              @{\hspace{2pt}}l                                             C                       D}
    \mc{3}{H}{Mapping}                                                                  & \mc{3}{|H}{Radius of Convergence}                                               & \mc{1}{|H}{Domain}    & \mc{1}{|H}{Property}
  \\\hline
    \opLT\brs{\ff(x)+\fg(x)}        &=& \fF(s) + \fG(s)                                 & for & \Real(s)                    & \in\intcc{A}{B}\seti\intcc{C}{D}            & \forall x,\alpha\in\C & (\prope{additive})
  \\\opLT\brs{\alpha\fg(x)}         &=& \alpha\fG(s)                                    & for & \Real(s)                    & \in\intcc{A}{B}                             & \forall x,\alpha\in\C & (\prope{homogeneous})
\end{array}}
\end{theorem}

%--------------------------------------
\begin{corollary}[Linear Properties]
\label{cor:opLT_linear}
%--------------------------------------
Let $\opLT$ be the \ope{Laplace Transform} operator \xref{def:opLT}.
Let $A$ and $B$ be real numbers such that $\intcc{A}{B}$
is the \structe{radius of convergence} of $\opLT\brs{\fg(x)}$.
Let $C$ and $D$ be real numbers such that $\intcc{C}{D}$
is the \structe{radius of convergence} of $\opLT\brs{\ff(x)}$.
Let $A_n$ and $B_n$ be real numbers such that $\intcc{A_n}{B_n}$
is the \structe{radius of convergence} of $\opLT\brs{\fg_n(x)}$.
\corbox{\begin{array}{>{\ds}l                      c  >{\ds}l                                       D     >{\ds}l                                               C}
    \mc{3}{H}{Mapping}                                                                            & \mc{2}{|H}{Radius of Convergence}                         & \mc{1}{|H}{Domain}
  \\\hline
    \opLT\brs{ 0                               }  &=&  0                                          & for & \Real(s)\in\intcc{A}{B}                             & \forall x\in\C
  \\\opLT\brs{ -\fg(x)                         }  &=& -\opLT\brs{\fg(x)}                          & for & \Real(s)\in\intcc{A}{B}                             & \forall x\in\C
  \\\opLT\brs{ \ff(x)-\fg(x)                   }  &=&  \opLT\brs{\fg(x)} - \opLT{\ff(x)}          & for & \Real(s)\in\intcc{A}{B}\seti\intcc{C}{D}            & \forall x\in\C
  \\\opLT\brs{ \sum_{n=1}^\xN \alpha_n\fg_n(x) }  &=&  \sum_{n=1}^\xN \alpha_n\opLT\brs{\fg_n(x)} & for & \Real(s)\in\setopi_{n=1}^\xN\intcc{A_n}{B_n}        & \forall x,\alpha_n\in\C
\end{array}}
\end{corollary}
\begin{proof}
\begin{enumerate}
  \item By \prefpp{thm:opLT_linear}, the operator \ope{Laplace Transform} operator $\opLT$
        is \prope{additive} and \prope{homogeneous}. \label{item:opLT_linear}
  \item By \pref{item:opLT_linear} and \prefpp{def:linop}, $\opLT$ is \prope{linear}. \label{item:opLT_linear_linear}
  \item By \pref{item:opLT_linear_linear} and \prefpp{thm:L_prop}, the four properties listed follow.
\end{enumerate}
\end{proof}

%=======================================
\section{Modulation properties}
%=======================================
%--------------------------------------
\begin{theorem}
\footnote{
  \citerpgc{bracewell1978}{224}{007007013X}{Table 11.1 Theorems for the Laplace Transform},
  \citerpcu{levy1958}{19}{Equation 1.2}{https://archive.org/details/c_cowles_ieee_Levy/page/n19/}
  }
\label{thm:opLT_mod}
%--------------------------------------
Let $\opLT$ be the \ope{Laplace Transform} operator \xref{def:opLT}.
Let $\fG(s)\eqd \brs{\opLT\fg(x)}$
and $\fF(s)\eqd \brs{\opLT\ff(x)}$.
Let the \structe{radius of convergence} of $\fG(s)$ be $A\leq\Real(s)\leq B$
and the \structe{radius of convergence} of $\fF(s)$ be $C\leq\Real(s)\leq D$.
\thmbox{\begin{array}{>{\ds}l        c>{\ds}l           D     l        @{\hspace{2pt}}l                                C }
    \mc{3}{H}{Mapping}                                & \mc{3}{|H}{Radius of Convergence}                            & \mc{1}{|H}{Domain}    %& \mc{1}{|H}{Property}
  \\\hline
    \opLT\brs{e^{-\alpha x} \fg(x)} &=& \fG(s+\alpha) & for & \Real(s) & \in\intcc{A-\Real(\alpha)}{B-\Real(\alpha)} & \forall x,\alpha\in\C %& (\ope{modulation})
\end{array}}
\end{theorem}
\begin{proof}
\begin{align*}
  \opLT\brs{e^{-\alpha x} \fg(x)}
    &\eqd \Lscale \int_{x\in\R} e^{-\alpha x} \fg(x) e^{-sx} \dx
          %&& \forall \Real(s)\in\intcc{A-\Real(\alpha)}{B-\Real(\alpha)}
          && \text{by definition of $\opLT$}
          && \text{\xref{def:opLT}}
  \\&=    \Lscale \int_{x\in\R} \fg(x) e^{-(s+\alpha)x} \dx        && A\leq\Real(s+\alpha)\leq B                      && \text{$b^{x+y} = b^x b^y$}
  \\&\eqd \brs{\opLT\fg(x)}(s+\alpha)                              && A-\Real(\alpha)\leq\Real(s)\leq B-\Real(\alpha) && \text{\xref{def:opLT}}
  \\&\eqd \fG(s+\alpha)                                            && A-\Real(\alpha)\leq\Real(s)\leq B-\Real(\alpha) && \text{by definition of $\fG(s)$}
\end{align*}
\end{proof}

%--------------------------------------
\begin{corollary}
\footnote{
  \citerpgc{bracewell1978}{224}{007007013X}{Table 11.1 Theorems for the Laplace Transform}
  }
\label{cor:opLT_mod}
%--------------------------------------
Let $\opLT$ be the \ope{Laplace Transform} operator \xref{def:opLT}.
Let $\fG(s)\eqd \brs{\opLT\fg(x)}$
and $\fF(s)\eqd \brs{\opLT\ff(x)}$.
Let the \structe{radius of convergence} of $\fG(s)$ be $A\leq\Real(s)\leq B$
and the \structe{radius of convergence} of $\fF(s)$ be $C\leq\Real(s)\leq D$.
\corbox{\begin{array}{>{\ds}l               c>{\ds}r                                                       D     l        @{\hspace{2pt}}l   C }
    \mc{3}{H}{Mapping}                                                                                   & \mc{3}{|H}{Radius of Convergence}                                               & \mc{1}{|H}{Domain}    %& \mc{1}{|H}{Property}
  \\\hline
    \opLT\brs{\cos\brp{\omega_o x} \fg(x)} &=&  \frac{1}{2}\fG(s-i\omega_o) + \frac{1}{2}\fG(s+i\omega_o) & for & \Real(s) & \in\intcc{A}{B} & \forall x,\alpha\in\C %& (\ope{modulation})
  \\[2ex]
    \opLT\brs{\sin\brp{\omega_o x} \fg(x)} &=& -\frac{i}{2}\fG(s-i\omega_o) + \frac{i}{2}\fG(s+i\omega_o) & for & \Real(s) & \in\intcc{A}{B} & \forall x,\alpha\in\C %& (\ope{modulation})
\end{array}}
\end{corollary}
\begin{proof}
\begin{align*}
  \opLT\brs{\cos\brp{\omega_o x} \fg(x)}
    &=     \opLT\brs{\brp{\frac{e^{i\omega_o x} + e^{-i\omega_o x}}{2}} \fg(x)}
        && \forall \Real(s)\in\intcc{A-\Real(-i\omega_o)}{B-\Real(i\omega_o)}
  \\&=     \frac{1}{2}\opLT\brs{e^{ i\omega_o x} \fg(x)}
         + \frac{1}{2}\opLT\brs{e^{-i\omega_o x} \fg(x)}
        && \forall \Real(s)\in\intcc{A-0}{B-0}
  \\&=   \frac{1}{2}\fG(s-i\omega_o) + \frac{1}{2}\fG(s+i\omega_o)
        && \forall \Real(s)\in\intcc{A}{B}
  \\
  \opLT\brs{\sin\brp{\omega_o x} \fg(x)}
    &=     \opLT\brs{\brp{\frac{e^{i\omega_o x} - e^{-i\omega_o x}}{2i}} \fg(x)}
        && \forall \Real(s)\in\intcc{A-\Real(-i\omega_o)}{B-\Real(i\omega_o)}
  \\&=     \frac{1}{2i}\opLT\brs{e^{ i\omega_o x} \fg(x)}
         - \frac{1}{2i}\opLT\brs{e^{-i\omega_o x} \fg(x)}
        && \forall \Real(s)\in\intcc{A-0}{B-0}
  \\&=   -\frac{i}{2}\fG(s-i\omega_o) + \frac{i}{2}\fG(s+i\omega_o)
        && \forall \Real(s)\in\intcc{A}{B}
\end{align*}
\end{proof}

%=======================================
\section{Causality properties}
%=======================================
%---------------------------------------
\begin{definition}
\footnote{
  \citerpu{betten2008H}{285}{https://link.springer.com/content/pdf/bbm:978-3-540-85051-9/1.pdf}
  }
\label{def:step}
%---------------------------------------
The \fnctd{Heaviside step function} $\step(x)$ or \fnctd{unit step function} is defined as
\defbox{
  \step(x) \eqd \brbl{\begin{array}{cM}
    1 & $\forall x\ge0$\\
    0 & otherwise
  \end{array}}
  }
\end{definition}

%---------------------------------------
\begin{theorem}
\footnote{
  \citerpgc{bracewell1978}{227}{007007013X}{Table 11.2 Some Laplace transforms}
  %\citerpcu{vanderpol1959}{13}{5. Strip of convergence of the Laplace integral}{https://archive.org/details/in.ernet.dli.2015.141269/page/n23/mode/2up},
  %\citerpcu{levy1958}{2}{``two-sided transformation"}{https://archive.org/details/c_cowles_ieee_Levy/page/n5/mode/2up}
  }
\label{thm:opLT_step}
%---------------------------------------
Let $\opLT$ be the \ope{Laplace Transform} operator \xref{def:opLT}
and $\step(x)$ the \fncte{unit step} function \xref{def:step}.
\thmbox{\begin{array}{>{\ds}lc>{\ds}r          Dl                                   C                 }
    \mc{3}{H}{Mapping}                       & \mc{2}{|H}{Radius of Convergence} & \mc{1}{|H}{Domain}
  \\\hline
    \opLT\brs{\step(x)}     &=& \frac{ 1}{s} & for & \Real(s) > 0                  & \forall x\in\R
  \\[2ex]
    \opLT\brs{\step(-x)}    &=& \frac{-1}{s} & for & \Real(s) < 0                  & x\in\R
\end{array}}
\end{theorem}
\begin{proof}
\begin{align*}
  \opLT\brs{\step(x)}
    &\eqd \int_\R \step(x) e^{-sx}\dx
    && \text{by definition of $\opLT$}
    && \text{\xref{def:opLT}}
  \\&= \int_0^{\infty}  e^{-sx}\dx
    && \text{by definition of $\step(x)$}
    && \text{\xref{def:step}}
  \\&= \brlr{\frac{e^{-sx}}{-s}}_0^{\infty}
    && \text{by \thme{Fundamental Theorem of Calculus}}
  \\&= \lim_{x\to\infty}\brs{\frac{e^{-sx}}{-s}} - \brp{\frac{e^0}{-s}}
  \\&= 0 + \frac{1}{s}
    && \forall\Real(s) > 0
  \\&= \frac{1}{s}
    && \forall\Real(s) > 0
  \\\\
  \opLT\brs{\step(-x)}
    &= \frac{-1}{s}
    && \Real(s) < 0
   %&& \text{by \ope{reversal} property \xref{cor:opLT_reversal} and \pref{thm:opLT_step}}
    && \text{by \pref{thm:opLT_step} and \pref{cor:opLT_reversal}}
\end{align*}
\end{proof}

%---------------------------------------
\begin{corollary}
\footnote{
  \citerpgc{bracewell1978}{227}{007007013X}{Table 11.2 Some Laplace transforms}
  }
\label{cor:opLT_step}
%---------------------------------------
Let $\opLT$ be the \ope{Laplace Transform} operator \xref{def:opLT}
and $\step(x)$ the \fncte{unit step} function.
\corbox{\begin{array}{F>{\ds}lc>{\ds}c                                                     D     l                             C }
    &\mc{3}{H}{Mapping}                                                                  & \mc{2}{|H}{Radius of Convergence} & \mc{1}{|H}{Domain}
  \\\hline
         (1).& \opLT\brs{\step(x)\cos(\omega_o x)}  &=& \frac{s}{s^2+\omega_o^2}         & for & \Real(s) > 0 & x,\omega_o\in\R
  \\[2ex](2).& \opLT\brs{\step(x)\sin(\omega_o x)}  &=& \frac{\omega_o}{s^2+\omega_o^2}  & for & \Real(s) > 0 & x,\omega_o\in\R
  \\[2ex](3).& \opLT\brs{\step(-x)\cos(\omega_o x)} &=& \frac{-s}{s^2+\omega_o^2}        & for & \Real(s) < 0 & x,\omega_o\in\R
  \\[2ex](4).& \opLT\brs{\step(-x)\sin(\omega_o x)} &=& \frac{-\omega_o}{s^2+\omega_o^2} & for & \Real(s) < 0 & x,\omega_o\in\R
\end{array}}
\end{corollary}
\begin{proof}
\begin{align*}
  \opLT\brs{\cos(\omega_o x)\step(x)}(s)
    &= \frac{1}{2}\opLT\brs{\step(x)}(s-i\omega_o)
     + \frac{1}{2}\opLT\brs{\step(x)}(s+i\omega_o)
    &&
    && \text{by \prope{modulation}}
    && \text{\xref{cor:opLT_mod}}
  \\&= \frac{1}{2}\brs{\frac{1}{s-i\omega_o}}
     + \frac{1}{2}\brs{\frac{1}{s+i\omega_o}}
    && \Real(s) > 0
    && \text{by \prope{causal} property}
    && \text{\xref{thm:opLT_step}}
  \\&= \mathrlap{
       \frac{1}{2}\brs{\frac{1}{s-i\omega_o}}\brs{\frac{s+i\omega_o}{s+i\omega_o}}
     + \frac{1}{2}\brs{\frac{1}{s+i\omega_o}}\brs{\frac{s-i\omega_o}{s-i\omega_o}}
     \qquad\text{(\ope{Rationalizing the Denominator})}
     }
  \\&= \frac{1}{2}\brs{\frac{(s+i\omega_o) + (s-i\omega_o)}{s^2+\omega_o^2}}
    && \Real(s) > 0
  \\&= \frac{s}{s^2+\omega_o^2}
    && \Real(s) > 0
  \\
  \opLT\brs{\sin(\omega_o x)\step(x)}(s)
    &= -\frac{i}{2}\opLT\brs{\step(x)}(s-i\omega_o)
       +\frac{i}{2}\opLT\brs{\step(x)}(s+i\omega_o)
    &&
    && \text{by \prope{modulation}}
    && \text{\xref{cor:opLT_mod}}
  \\&= -\frac{i}{2}\brs{\frac{1}{s-i\omega_o}}
       +\frac{i}{2}\brs{\frac{1}{s+i\omega_o}}
    && \Real(s) > 0
    && \text{by \prope{causal} property}
    && \text{\xref{thm:opLT_step}}
  \\&= \mathrlap{
     - \frac{i}{2}\brs{\frac{1}{s-i\omega_o}}\brs{\frac{s+i\omega_o}{s+i\omega_o}}
     + \frac{i}{2}\brs{\frac{1}{s+i\omega_o}}\brs{\frac{s-i\omega_o}{s-i\omega_o}}
     \qquad\text{(\ope{Rationalizing the Denominator})}
     }
  \\&= \frac{i}{2}\brs{\frac{-(s+i\omega_o) + (s-i\omega_o)}{s^2+\omega_o^2}}
    && \Real(s) > 0
  \\&= \frac{\omega_o}{s^2+\omega_o^2}
    && \Real(s) > 0
\end{align*}

\begin{align*}
  \opLT\brs{\step(-x)\cos(\omega_o x)}(s)
    &= \opLT\brs{\step(-x)\cos(\omega_o(-x))}(s)
    &&
    && \text{by \prope{even} property of $\cos(x)$}
    && \text{\xref{thm:cos-x}}
  \\&= \opLT\brs{\step(x)\cos(\omega_o x)}(-s)
    &&
    && \text{by \prope{reversal}}
    && \text{\xref{cor:opLT_reversal}}
  \\&= \frac{(-s)}{(-s)^2+\omega_o^2}
    && \Real(s) < 0
    && \text{by (1)}
    &&
  \\&= \frac{-s}{s^2+\omega_o^2}
    && \Real(s) < 0
  \\
  \opLT\brs{\sin(\omega_o x)\step(-x)}(s)
    &= \opLT\brs{-\sin(\omega_o(-x))\step(-x)}(s)
    &&
    && \text{by \prope{odd} property of $\sin(x)$}
    && \text{\xref{thm:sin-x}}
  \\&= -\opLT\brs{\sin(\omega_o(-x))\step(-x)}(s)
    &&
    && \text{by \prope{homogeneous} property}
    && \text{\xref{thm:opLT_linear}}
  \\&= -\opLT\brs{\sin(\omega_o x)\step(x)}(-s)
    &&
    && \text{by \prope{reversal} property}
    && \text{\xref{cor:opLT_reversal}}
  \\&= -\brs{\frac{\omega_o}{(-s)^2+\omega_o^2}}
    && \Real(s) < 0
    && \text{by (2)}
    &&
  \\&= \frac{-\omega_o}{s^2+\omega_o^2}
    && \Real(s) < 0
\end{align*}
\end{proof}

%---------------------------------------
\begin{corollary}
\footnote{
  \citerpgc{bracewell1978}{227}{007007013X}{Table 11.2 Some Laplace transforms}
  }
\label{cor:opLT_stepcosexp}
%---------------------------------------
Let $\opLT$ be the \ope{Laplace Transform} operator \xref{def:opLT}
and $\step(x)$ the \fncte{unit step} function.
\corbox{\begin{array}{F>{\ds}lc>{\ds}c     D     l                             C }
    \mc{3}{H}{Mapping}                  & \mc{2}{|H}{Radius of Convergence} & \mc{1}{|H}{Domain}
  \\\hline
           (1).&\opLT\brs{\step(x)\cos(\omega_o x)e^{-\alpha x}} &=& \frac{s+\alpha}{(s+\alpha)^2+\omega_o^2}  & for & \Real(s) > -\Real(\alpha) & x,\omega_o\in\R;\,\alpha\in\C
    \\[2ex](2).&
    \opLT\brs{\step(x)\sin(\omega_o x)e^{-\alpha x}} &=& \frac{\omega_o}{(s+\alpha)^2+\omega_o^2}  & for & \Real(s) > -\Real(\alpha) & x,\omega_o\in\R;\,\alpha\in\C
    \\[2ex](3).&
    \opLT\brs{\step(-x)\cos(-\omega_o x)e^{\alpha x}} &=& \frac{-s+\alpha}{(s-\alpha)^2+\omega_o^2}  & for & \Real(s) > -\Real(\alpha) & x,\omega_o\in\R;\,\alpha\in\C
    \\[2ex](4).&
    \opLT\brs{\step(-x)\sin(-\omega_o x)e^{\alpha x}} &=& \frac{\omega_o}{(s-\alpha)^2+\omega_o^2}  & for & \Real(s) > -\Real(\alpha) & x,\omega_o\in\R;\,\alpha\in\C
\end{array}}
\end{corollary}
\begin{proof}
\begin{align*}
  \opLT\brs{\step(x)\cos(\omega_o x)e^{-\alpha x}}
    &= \frac{s+\alpha}{(s+\alpha)^2+\omega_o^2}
    && \forall\Real(s) \in\intoo{0-\Real(\alpha)}{\infty-\Real(\alpha)}
    && \text{by \pref{cor:opLT_step} and \pref{thm:opLT_mod}}
  \\&= \frac{s+\alpha}{(s+\alpha)^2+\omega_o^2}
    && \forall\Real(s) > -\Real(\alpha)
  \\
  \opLT\brs{\step(x)\sin(\omega_o x)e^{-\alpha x}}
    &= \frac{\omega_o}{(s+\alpha)^2+\omega_o^2}
    && \forall\Real(s) \in\intoo{0-\Real(\alpha)}{\infty-\Real(\alpha)}
    && \text{by \pref{cor:opLT_step} and \pref{thm:opLT_mod}}
  \\&= \frac{\omega_o}{(s+\alpha)^2+\omega_o^2}
    && \forall\Real(s) > -\Real(\alpha)
  \\
  \opLT\brs{\step(-x)\cos(-\omega_o x)e^{\alpha x}}
    &= \frac{-s+\alpha}{(-s+\alpha)^2+\omega_o^2}
    && \forall\Real(s) < \Real(\alpha)
    && \text{by (1) and \pref{cor:opLT_reversal}}
  \\&= \frac{-s+\alpha}{(s-\alpha)^2+\omega_o^2}
    && \forall\Real(s) < \Real(\alpha)
  \\
  \opLT\brs{\step(-x)\sin(-\omega_o x)e^{\alpha x}}
    &= \frac{\omega_o}{(-s+\alpha)^2+\omega_o^2}
    && \forall\Real(s) < \Real(\alpha)
    && \text{by (2) and \pref{cor:opLT_reversal}}
  \\&= \frac{\omega_o}{(-s+\alpha)^2+\omega_o^2}
    && \forall\Real(s) < \Real(\alpha)
    && \text{by (2) and \pref{cor:opLT_reversal}}
  \\&= \frac{\omega_o}{(s-\alpha)^2+\omega_o^2}
    && \forall\Real(s) < \Real(\alpha)
\end{align*}
\end{proof}

%---------------------------------------
\begin{corollary}
\label{cor:opLT_cos}
\label{cor:opLT_sin}
%---------------------------------------
Let $\opLT$ be the \ope{Laplace Transform} operator \xref{def:opLT}.
\corbox{\begin{array}{>{\ds}lMlC}
    \opLT\brs{\cos(\omega_o x)} & is \propb{divergent} & \forall s\in\C & \forall x,\omega_o\in\R
  \\\opLT\brs{\sin(\omega_o x)} & is \propb{divergent} & \forall s\in\C & \forall x,\omega_o\in\R
\end{array}}
\end{corollary}
\begin{proof}
\begin{align*}
  \opLT\brs{\cos(\omega_o x)}
    &= \mcom{\opLT\brs{\step( x)\cos(\omega_o x)}}{$\forall \Real(s)>0$}
     + \mcom{\opLT\brs{\step(-x)\cos(\omega_o x)}}{$\forall \Real(s)<0$}
    && \text{by \prefp{cor:opLT_step}}
  \\&= \mcom{\frac{s}{s^2+\omega_o^2}}{$\forall \Real(s)>0$}
     + \mcom{\frac{-s}{s^2+\omega_o^2}}{$\forall \Real(s)<0$}
    && \text{by \prefp{cor:opLT_step}}
  \\&= 0
    && \forall\Real(s)\in\intoo{-\infty}{0}\seti\intoo{0}{\infty}=\emptyset
  \\&\text{$\implies$ $\opLT\brs{\cos(\omega_o x)}$ is \propb{divergent} $\forall s\in\C$}
  \\
  \opLT\brs{\sin(\omega_o x)}
    &= \mcom{\opLT\brs{\step( x)\sin(\omega_o x)}}{$\forall \Real(s)>0$}
     + \mcom{\opLT\brs{\step(-x)\sin(\omega_o x)}}{$\forall \Real(s)<0$}
    && \text{by \prefp{cor:opLT_step}}
  \\&= \mcom{\frac{\omega_o}{s^2+\omega_o^2}}{$\forall \Real(s)>0$}
     + \mcom{\frac{-\omega_o}{s^2+\omega_o^2}}{$\forall \Real(s)<0$}
    && \text{by \prefp{cor:opLT_step}}
  \\&= 0
    && \forall\Real(s)\in\intoo{-\infty}{0}\seti\intoo{0}{\infty}=\emptyset
  \\&\text{$\implies$ $\opLT\brs{\sin(\omega_o x)}$ is \propb{divergent} $\forall s\in\C$}
\end{align*}
\end{proof}

%=======================================
\section{Exponential decay properties}
%=======================================
%---------------------------------------
\begin{corollary}
\footnote{
  \citerpgc{bracewell1978}{227}{007007013X}{Table 11.2 Some Laplace transforms},
  %\citerpcu{levy1958}{19}{with $\psi=0$, $\alpha_0=\alpha$, and $\alpha_1=1$}{https://archive.org/details/c_cowles_ieee_Levy/page/n21/},
  %\url{http://ece-research.unm.edu/bsanthan/ece541/table_ME.pdf}
  }
\label{cor:opLT_expabs}
%---------------------------------------
Let $\opLT$ be the \ope{Laplace Transform} operator \xref{def:opLT}
and $\step(x)$ the \fncte{unit step} function.
Let $A\eqd\Real(\alpha)$.
\corbox{\begin{array}{>{\ds}lc>{\ds}l                                    D     l                            C }
    \mc{3}{H}{Mapping}                                                 & \mc{2}{|H}{Radius of Convergence}& \mc{1}{|H}{Domain}
  \\\hline%                                                            &                                  &
    \opLT\brs{e^{-\alpha\abs{x}}}  &=& \frac{2\alpha}{\alpha^2 - s^2}  & for & \Real(s) \in \intoo{-A}{A} & x,\alpha\in\R
\end{array}}
\end{corollary}
\begin{proof}
  \begin{align*}
    \opLT\brs{e^{-\alpha\abs{x}}}
      &\eqd \int_\R e^{-\alpha\abs{x}} e^{-sx} \dx
      && \text{by definition of $\opLT$}
      && \text{\xref{def:opLT}}
    \\&\eqd \int_\R \step( x) e^{-\alpha x} e^{-sx} \dx
       +    \int_\R \step(-x) e^{ \alpha x} e^{-sx} \dx
      && \text{by definition of $\step(x)$}
      && \text{\xref{def:step}}
    \\&= \mathrlap{
         \mcom{\opLT\brs{\step( x) e^{-\alpha x}}}{$\forall\Real(s)\in\intoo{0-\Real(\alpha)}{\infty-\Real(\alpha)}$}
       + \mcom{\opLT\brs{\step(-x) e^{ \alpha x}}}{$\forall\Real(s)\in\intoo{\Real(\alpha)-\infty}{\Real(\alpha)-0}$}
         \qquad\begin{array}{MM}
           by&\prefp{thm:opLT_mod} and\\
             &\prefp{thm:opLT_step}
         \end{array}
         }
     %&& \forall\Real(s)\in\intoo{0-\Real(\alpha)}{\infty-\Real(\alpha)}\seti\intoo{\Real(\alpha)-\infty}{\Real(\alpha)-0}
     %&& \text{by definition of $\opLT$}
     %&& \text{\xref{def:opLT}}
    \\&= \brs{\frac{1}{s+\alpha}} + \brs{\frac{-1}{s-\alpha}}
      && \forall\Real(s)\in\intoo{-\Real(\alpha)}{\Real(\alpha)}
      && \text{by \pref{thm:opLT_step} and \pref{thm:opLT_mod}}
    \\&= \frac{(s-\alpha)-(s+\alpha)}{(s+\alpha)(s-\alpha)}
      && \forall\Real(s)\in\intoo{-\Real(\alpha)}{\Real(\alpha)}
    \\&= \frac{2\alpha}{\alpha^2 - s^2}
      && \forall\Real(s)\in\intoo{-\Real(\alpha)}{\Real(\alpha)}
  \end{align*}
\end{proof}

%---------------------------------------
\begin{corollary}
\footnote{
  \citerpgc{bracewell1978}{227}{007007013X}{Table 11.2 Some Laplace transforms},
  \citerpcu{levy1958}{19}{with $\psi=0$, $\alpha_0=\alpha$, and $\alpha_1=1$}{https://archive.org/details/c_cowles_ieee_Levy/page/n21/},
  \url{http://ece-research.unm.edu/bsanthan/ece541/table_ME.pdf}
  %\url{https://ethz.ch/content/dam/ethz/special-interest/baug/ibk/structural-mechanics-dam/education/identmeth/fourier.pdf}
  }
\label{cor:opLT_cosabs}
%---------------------------------------
Let $\opLT$ be the \ope{Laplace Transform} operator \xref{def:opLT}
and $\step(x)$ the \fncte{unit step} function.
Let $A\eqd\Real(\alpha)$.
\corbox{\begin{array}{F>{\ds}lc>{\ds}l                                    D     l                                                    C }
    &\mc{3}{H}{Mapping}                                                                                            & \mc{2}{|H}{Radius of Convergence}& \mc{1}{|H}{Domain}
  \\\hline%                                                                                                        &                                  &
    (1).& \opLT\brs{\cos(\omega_o x)e^{-\alpha\abs{x}}\step( x)}  &=& \frac{ s+\alpha}{(s+\alpha)^2 + \omega_o^2}  & for & \Real(s) \in \intoo{-A}{A} & x,\alpha\in\R
  \\(2).& \opLT\brs{\cos(\omega_o x)e^{-\alpha\abs{x}}\step(-x)}  &=& \frac{-s+\alpha}{(s-\alpha)^2 + \omega_o^2}  & for & \Real(s) \in \intoo{-A}{A} & x,\alpha\in\R
  \\(3).& \opLT\brs{\cos(\omega_o x)e^{-\alpha\abs{x}}         }  &=& \frac{ s+\alpha}{(s+\alpha)^2 + \omega_o^2} +  \frac{-s+\alpha}{(s-\alpha)^2 + \omega_o^2}  & for & \Real(s) \in \intoo{-A}{A} & x,\alpha\in\R
\end{array}}
\end{corollary}
\begin{proof}
\begin{enumerate}
  \item Proof for (1):
    \begin{align*}
      &\opLT\brs{\cos(\omega_o x)e^{-\alpha\abs{x}}\step(x)}(s)
      \\&= \opLT\brs{\cos( \omega_o x)e^{-\alpha x}\step( x)}(s)
        &&
        && \text{by definition of $\step(x)$}
        && \text{\xref{def:step}}
      \\&= \opLT\brs{\cos( \omega_o x)\step( x)}(s+\alpha)
        && \forall \Real(s) \in \intoo{0-\Real(\alpha)}{\infty-\Real(\alpha)}
        && \text{by \prope{modulation} property}
        && \text{\xref{thm:opLT_mod}}
      \\&= \frac{s+\alpha}{(s+\alpha)^2 + \omega_o^2}
        && \forall \Real(s) \in \intoo{-\Real(\alpha)}{\Real(\alpha)}
        && \text{by \prefp{cor:opLT_step}}
    \end{align*}

  \item Proof for (2):
    \begin{align*}
      &\opLT\brs{\cos(\omega_o x)e^{-\alpha\abs{x}}\step(-x)}
      \\&= \opLT\brs{\cos(\omega_o x)e^{ \alpha x}\step(-x)}
        && \text{by definition of $\step(x)$}
        && \text{\xref{def:step}}
      \\&= \opLT\brs{\cos(-\omega_o x)e^{ \alpha x}\step(-x)}
        && \text{by \prope{even} property of $\cos(x)$}
        && \text{\xref{thm:cos-x}}
      \\&= \opLT\brs{e^{\alpha x}\cos(\omega_o(-x))\step(-x)}
      \\&= \opLT{\mcom{\cos(\omega_o(-x))\step(-x)}{$\fg(x)$}}(s-\alpha)
        && \text{by \prope{modulation} property}
        && \text{\xref{thm:opLT_mod}}
      \\&= \opLT{\mcom{\cos(\omega_o(-x))\step(-x)}{$\fg(x)$}}(s-\alpha)
        && \text{by \prope{modulation} property}
        && \text{\xref{thm:opLT_mod}}
      \\&= \frac{-s+\alpha}{(s-\alpha)^2 + \omega_o^2}
        && \forall \Real(s) \in \intoo{-\Real(\alpha)}{\Real(\alpha)}
        && \text{by \pref{cor:opLT_step} and \prefp{thm:opLT_mod}}
    \end{align*}

  \item Proof for (3):
    \begin{align*}
      &\opLT\brs{\cos(\omega_o x)e^{-\alpha\abs{x}}}
      \\&= \opLT\brs{\cos(\omega_o x)e^{-\alpha\abs{x}}\step(x)}
         + \opLT\brs{\cos(\omega_o x)e^{-\alpha\abs{x}}\step(-x)}
      \\&= \opLT\brs{\cos( \omega_o x)e^{-\alpha x}\step( x)}
         + \opLT\brs{\cos(-\omega_o x)e^{ \alpha x}\step(-x)}
      \\&= \frac{s+\alpha}{(s+\alpha)^2 + \omega_o^2} +  \frac{-s+\alpha}{(s-\alpha)^2 + \omega_o^2}
        && \forall \Real(s) \in \intoo{-\Real(\alpha)}{\Real(\alpha)}
    \end{align*}

  \item lemma: \label{ilem:opLT_cosexpstep}
    \\$\begin{array}{>{\ds}rclM}
         \lim_{x\to\infty}\brs{\frac{e^{(-s -\alpha \pm i\beta)x}}{-s -\alpha \pm i\beta}}
           &=& \brbl{\begin{array}{lMM}
                 0      & if $\Real(s)>-\alpha$ \\
                 \infty & otherwise  & (diverges)
               \end{array}}
         \\
         \lim_{x\to-\infty}\brs{\frac{e^{(-s +\alpha \pm i\beta)x}}{-s +\alpha \pm i\beta}}
           &=& \brbl{\begin{array}{lMM}
                 0      & if $\Real(s)<\alpha$ \\
                 \infty & otherwise  & (diverges)
               \end{array}}
       \end{array}$

  \item Alternate proof for (1):
    \begin{align*}
      &\boxed{\opLT\brs{\cos(\beta x)e^{-\alpha\abs{x}}\step(x)}}
      \\&\eqd \int_{\R} \cos(\beta x)e^{-\alpha\abs{x}} e^{-s x} \step(x)\dx
        && \text{by def. of $\opLT$}&& \text{\xref{def:opLT}}
      \\&= \int_{\R} \brs{\frac{e^{i\beta x}+e^{-i\beta x}}{2}} e^{-\alpha\abs{x}} e^{-s x} \step(x)\dx
        &&  \text{by \thme{Euler's Identity}}&&\text{\xref{thm:eid}}
      \\&= \frac{1}{2}\int_{\R} \brs{e^{-\alpha\abs{x}+i\beta x -sx}+e^{-\alpha\abs{x}-i\beta x-sx}} \step(x)\dx
      \\&= \frac{1}{2}\int_{0}^{\infty}  \brs{e^{(-s - \alpha +i\beta)x} + e^{(-s - \alpha -i\beta)x}} \dx
        && \text{by definition of $\step(x)$}
      \\&= \frac{1}{2} \brs{\frac{e^{(-s - \alpha +i\beta)x}}{-s -\alpha +i\beta} + \frac{e^{(-s -\alpha - i\beta)x}}{-s - \alpha -i\beta}}_{0}^{\infty}
       %&& \text{by Fundamental Theorem of Calculus}
      \\&= \frac{1}{2} \brs{(0+0)-\brp{\frac{1}{-s -\alpha +i\beta} + \frac{1}{-s -\alpha -i\beta}}}
        && \Real(s)>-\alpha
        && \text{by \pref{ilem:opLT_cosexpstep}}
      \\&= \mathrlap{
           \frac{1}{2}\brs{\frac{1}{s +\alpha -i\beta}} \brs{\frac{s +\alpha +i\beta}{s +\alpha +i\beta}} +
           \frac{1}{2}\brs{\frac{1}{s +\alpha +i\beta}} \brs{\frac{s +\alpha -i\beta}{s +\alpha -i\beta}}
           \qquad\Real(s)>-\alpha
           }
      \\&= \frac{1}{2}\brs{\frac{(s +\alpha +i\beta) + (s +\alpha -i\beta)}{(s +\alpha +i\beta)(s +\alpha -i\beta)}}
        && \Real(s)>-\alpha
      \\&= \boxed{\frac{s +\alpha}{(s +\alpha +i\beta)(s +\alpha -i\beta)}}
        && \Real(s)>-\alpha
      \\&= \boxed{\frac{s +\alpha}{(s +\alpha)^2 + \beta^2}}
        && \Real(s)>-\alpha
    \end{align*}

  \item lemma: \label{ilem:opLT_cosexp}
    \\$\begin{array}{>{\ds}rclM}
         \lim_{x\to-\infty}\brs{e^{(-s + \alpha + i\beta)x} + e^{(s + \alpha -i\beta)x}}
           &=& \brbl{\begin{array}{lMD}
                 0      & if $v\eqd\Real(s)\in\intoo{-\alpha}{\alpha}$ & (radius of convergence is $\intoo{-\alpha}{\alpha}$)\\
                 \infty & otherwise                                    & (diverges)
               \end{array}}
         \\
         \lim_{x\to\infty}\brs{e^{(-s -\alpha + i\beta)x} + e^{(s -\alpha -i\beta)x}}
           &=& \brbl{\begin{array}{lMD}
                 0      & if $v\eqd\Real(s)\in\intoo{-\alpha}{\alpha}$ & (radius of convergence is $\intoo{-\alpha}{\alpha}$)\\
                 \infty & otherwise                                    & (diverges)
               \end{array}}
       \end{array}$

  \item Alternate proof for (3):
    \begin{align*}
      \boxed{\opLT\brs{\cos(\beta x)e^{-\alpha\abs{x}}}}
        &\eqd \int_{\R} \cos(\beta x)e^{-\alpha\abs{x}} e^{-s x}\dx
        && \text{by def. of $\opLT$}&& \text{\xref{def:opLT}}
      \\&= \int_{\R} \brs{\frac{e^{i\beta x}+e^{-i\beta x}}{2}} e^{-\alpha\abs{x}} e^{-s x}\dx
        &&  \text{by \thme{Euler's Identity}}&&\text{\xref{thm:eid}}
      \\&= \frac{1}{2}\int_{\R} \brs{e^{-\alpha\abs{x}+i\beta x -sx}+e^{-\alpha\abs{x}-i\beta x-sx}} \dx
      \\&= \mathrlap{
           \frac{1}{2}\int_{-\infty}^{0} \brs{e^{(-s + \alpha +i\beta)x} + e^{(-s + \alpha -i\beta)x}} \dx
         + \frac{1}{2}\int_{0}^{\infty}  \brs{e^{(-s - \alpha +i\beta)x} + e^{(-s - \alpha -i\beta)x}} \dx
           }
      \\&= \mathrlap{
           \frac{1}{2} \brs{\frac{e^{(-s + \alpha +i\beta)x}}{-s +\alpha +i\beta} + \frac{e^{(-s +\alpha - i\beta)x}}{-s + \alpha -i\beta}}_{-\infty}^{0}
         + \frac{1}{2} \brs{\frac{e^{(-s - \alpha +i\beta)x}}{-s -\alpha +i\beta} + \frac{e^{(-s -\alpha - i\beta)x}}{-s - \alpha -i\beta}}_{0}^{\infty}
           }
       %&& \text{by Fundamental Theorem of Calculus}
      \\&= \mathrlap{
           \frac{1}{2} \brs{\frac{1}{-s + \alpha +i\beta} + \frac{1}{ -s+ \alpha -i\beta} - (0+0)}
         + \frac{1}{2} \brs{(0+0)-\brp{\frac{1}{-s -\alpha +i\beta} + \frac{1}{-s -\alpha -i\beta}}}
           }
      \\&\indentx\text{by \pref{ilem:opLT_cosexp}}
      \\&= \brs{\frac{-1}{s -( \alpha + i\beta)}}
         + \brs{\frac{1}{-s -(-\alpha + i\beta)}}
      \\&= \frac{-[s -(-\alpha + i\beta)] + [s -( \alpha + i\beta)]}
                { [s -( \alpha + i\beta)]   [s -(-\alpha + i\beta)]}
      \\&= \boxed{
           \frac{-2\alpha}
                { [s -( \alpha + i\beta)]   [s -(-\alpha + i\beta)]}
           }
      \\&= \frac{-2\alpha}
                { s^2 -2i s\beta + ( \alpha + i\beta)(-\alpha + i\beta)}
      \\&= \frac{-2\alpha}
                { s^2 -2i s\beta + -(\alpha^2-\beta^2)}
      \\&= \boxed{\frac{-2\alpha}
                       { (s -i\beta)^2 -\alpha^2}}
    \end{align*}
\end{enumerate}
\end{proof}

%---------------------------------------
\begin{corollary}
\footnote{
  \citerpgc{bracewell1978}{227}{007007013X}{Table 11.2 Some Laplace transforms},
  \citerpcu{levy1958}{19}{with $\psi=0$, $\alpha_0=\alpha$, and $\alpha_1=1$}{https://archive.org/details/c_cowles_ieee_Levy/page/n21/},
  \url{http://ece-research.unm.edu/bsanthan/ece541/table_ME.pdf}
  %\url{https://ethz.ch/content/dam/ethz/special-interest/baug/ibk/structural-mechanics-dam/education/identmeth/fourier.pdf}
  }
\label{cor:opLT_sinabs}
%---------------------------------------
Let $\opLT$ be the \ope{Laplace Transform} operator \xref{def:opLT}
and $\step(x)$ the \fncte{unit step} function.
Let $A\eqd\Real(\alpha)$.
\corbox{\begin{array}{F>{\ds}lc>{\ds}l                                    D     l                                                    C }
    &\mc{3}{H}{Mapping}                                                                                            & \mc{2}{|H}{Radius of Convergence}& \mc{1}{|H}{Domain}
  \\\hline%                                                                                                        &                                  &
    (1).& \opLT\brs{\sin(\omega_o x)e^{-\alpha\abs{x}}\step( x)}  &=& \frac{ \omega_o}{(s+\alpha)^2 + \omega_o^2}  & for & \Real(s) \in \intoo{-A}{A} & x,\alpha\in\R
  \\(2).& \opLT\brs{\sin(\omega_o x)e^{-\alpha\abs{x}}\step(-x)}  &=& \frac{-\omega_o}{(s-\alpha)^2 + \omega_o^2}  & for & \Real(s) \in \intoo{-A}{A} & x,\alpha\in\R
  \\(3).& \opLT\brs{\sin(\omega_o x)e^{-\alpha\abs{x}}         }  &=& \frac{ \omega_o}{(s+\alpha)^2 + \omega_o^2} +  \frac{-\omega_o}{(s-\alpha)^2 + \omega_o^2}  & for & \Real(s) \in \intoo{-A}{A} & x,\alpha\in\R
\end{array}}
\end{corollary}
\begin{proof}
\begin{enumerate}
  \item Proof for (1):
    \begin{align*}
      &\opLT\brs{\sin(\omega_o x)e^{-\alpha\abs{x}}\step(x)}
      \\&= \opLT\brs{\sin( \omega_o x)e^{-\alpha x}\step( x)}
        && \text{by definition of $\step(x)$}
        && \text{\xref{def:step}}
      \\&= \frac{s+\alpha}{(\omega_o)^2 + \omega_o^2}
        && \forall \Real(s) \in \intoo{-\Real(\alpha)}{\Real(\alpha)}
        && \text{by \prefp{cor:opLT_step} and \prefp{thm:opLT_mod}}
    \end{align*}

  \item Proof for (2):
    \begin{align*}
      &\opLT\brs{\sin(\omega_o x)e^{-\alpha\abs{x}}\step(-x)}
      \\&= \opLT\brs{\sin(-\omega_o x)e^{ \alpha x}\step(-x)}
        && \text{by definition of $\step(x)$}
        && \text{\xref{def:step}}
      \\&= \opLT\brs{-\sin(\omega_o x)e^{ \alpha x}\step(-x)}
        && \text{by \prope{odd} property of $\sin(x)$}
        && \text{\xref{thm:sin-x}}
      \\&= -\opLT\brs{\sin(\omega_o x)e^{ \alpha x}\step(-x)}
        && \text{by \prope{homogeneous} property}
        && \text{\xref{thm:opLT_linear}}
      \\&= \frac{-\omega_o}{(s-\alpha)^2 + \omega_o^2}
        && \forall \Real(s) \in \intoo{-\Real(\alpha)}{\Real(\alpha)}
        && \text{by \prefp{thm:opLT_mod} and \pref{cor:opLT_step}}
    \end{align*}

  \item Proof for (3):
    \begin{align*}
      &\opLT\brs{\sin(\omega_o x)e^{-\alpha\abs{x}}}
      \\&= \opLT\brs{\sin(\omega_o x)e^{-\alpha\abs{x}}\step(x)}
         + \opLT\brs{\sin(\omega_o x)e^{-\alpha\abs{x}}\step(-x)}
      \\&= \opLT\brs{\sin( \omega_o x)e^{-\alpha x}\step( x)}
         + \opLT\brs{\sin(-\omega_o x)e^{ \alpha x}\step(-x)}
      \\&= \frac{\omega_o}{(s+\alpha)^2 + \omega_o^2} +  \frac{-\omega_o}{(s-\alpha)^2 + \omega_o^2}
        && \forall \Real(s) \in \intoo{-\Real(\alpha)}{\Real(\alpha)}
    \end{align*}
%\end{proof}

%%-------------------------------------
%\begin{example}
%\label{ex:opLT_sinexp}
%%-------------------------------------
%Let $\omega\eqd\Imag(s)$
%(let $\omega$ be the \fncte{imaginary part} of the complex variable $s$).
%\exbox{
%  \opLT\brs{\sin(\beta x)e^{-\alpha\abs{x}}}
%    = \brbl{\begin{array}{>{\ds}lMD}
%                0 & for $\Real(s)\in\intoo{-\alpha}{\alpha}$
%              &%(radius of convergence)
%              \\
%              \text{$\pm\infty$ (\prope{divergent})}
%              & otherwise
%              &
%            \end{array}}
% }
%\end{example}
%\begin{proof}
%\begin{enumerate}
  \item lemma: \label{ilem:opLT_sinexp}
    \\$\begin{array}{>{\ds}rclM}
         \lim_{x\to-\infty}\brs{e^{(-s + \alpha + i\beta)x} - e^{(s + \alpha -i\beta)x}}
           &=& \brbl{\begin{array}{lMD}
                 0      & if $v\eqd\Real(s)\in\intoo{-\alpha}{\alpha}$ & (radius of convergence is $\intoo{-\alpha}{\alpha}$)\\
                 \pm\infty & otherwise                                    & (diverges)
               \end{array}}
         \\
         \lim_{x\to\infty}\brs{e^{(-s -\alpha + i\beta)x} - e^{(s -\alpha -i\beta)x}}
           &=& \brbl{\begin{array}{lMD}
                 0      & if $v\eqd\Real(s)\in\intoo{-\alpha}{\alpha}$ & (radius of convergence is $\intoo{-\alpha}{\alpha}$)\\
                 \pm\infty & otherwise                                    & (diverges)
               \end{array}}
       \end{array}$

  \item Main proof:
    \begin{align*}
      \boxed{\opLT\brs{\sin(\beta x)e^{-\alpha\abs{x}}}}
        &\eqd \int_{\R} \sin(\beta x)e^{-\alpha\abs{x}} e^{-s x}\dx
        && \text{by def. of $\opLT$}&& \text{\xref{def:opLT}}
      \\&= \int_{\R} \brs{\frac{e^{i\beta x}-e^{-i\beta x}}{2i}} e^{-\alpha\abs{x}} e^{-s x}\dx
        &&  \text{by \thme{Euler's Identity}}&&\text{\xref{thm:eid}}
      \\&= \frac{1}{2i}\int_{\R} \brs{e^{-\alpha\abs{x}+i\beta x -sx}-e^{-\alpha\abs{x}-i\beta x-sx}} \dx
      \\&= \mathrlap{
           \frac{1}{2i}\int_{-\infty}^{0} \brs{e^{(-s + \alpha +i\beta)x} - e^{(-s + \alpha -i\beta)x}} \dx
         + \frac{1}{2i}\int_{0}^{\infty}  \brs{e^{(-s - \alpha +i\beta)x} - e^{(-s - \alpha -i\beta)x}} \dx
           }
      \\&= \mathrlap{
           \frac{1}{2i} \brs{\frac{e^{(-s + \alpha +i\beta)x}}{-s +\alpha +i\beta} - \frac{e^{(s +\alpha - i\beta)x}}{s + \alpha -i\beta}}_{-\infty}^{0}
         + \frac{1}{2i} \brs{\frac{e^{(-s - \alpha +i\beta)x}}{-s -\alpha +i\beta} - \frac{e^{(s -\alpha - i\beta)x}}{s - \alpha -i\beta}}_{0}^{\infty}
           }
       %&& \text{by Fundamental Theorem of Calculus}
      \\&= \mathrlap{
           \frac{1}{2i} \brs{           \frac{1}{-s + \alpha +i\beta} - \frac{1}{ s+ \alpha -i\beta} - (0+0)}
         + \frac{1}{2i} \brs{(0+0)-\brp{\frac{1}{-s - \alpha +i\beta} - \frac{1}{ s -\alpha -i\beta}}       }
           }
      \\&\indentx\text{by \pref{ilem:opLT_sinexp}}
      \\&= \mathrlap{
           \frac{1}{2i} \brs{ \frac{1}{-s + \alpha +i\beta} + \frac{ 1}{ s -\alpha -i\beta}  }
         + \frac{1}{2i} \brs{\frac{-1}{ s + \alpha -i\beta} + \frac{-1}{-s- \alpha +i\beta}  }
           }
      \\&= 0
    \end{align*}
\end{enumerate}
\end{proof}

%======================================
\section{Product properties}
\label{sec:opLT_conv}
%======================================
\pref{thm:opLT_conv} (next) demonstrates that multiplication in the ``time domain"
is equivalent to convolution in the ``s domain" and
vice-versa.

%--------------------------------------
\begin{theorem}[\thmd{convolution theorem}]
\footnote{
  \citerpg{bracewell1978}{224}{007007013X},
  \citerppg{bachman2002}{268}{270}{9780387988993},
  \citerpg{bachman1964}{8}{9781483267562}
  }
\label{thm:opLT_conv}
%--------------------------------------
Let $\opLT$ be the \ope{Laplace Transform} operator \xref{def:opLT}
and $\conv$ the convolution operator \xref{def:conv}.
Let $A$, $B$, $C$, and $D$ be defined as in \prefpp{cor:opLT_linear}.
\thmbox{
\begin{array}{rcllC}
  \opLT\brs{\ff(x)\conv\fg(x)}(s)
    &=& \brs{\opLT\ff}(s)\, \brs{\opLT\fg}(s)
    & \forall \Real(s)\in\intcc{A}{B}\seti\intcc{C}{D}
    & \forall\ff,\fg\in\spLLRBu
  \\
  \opLT\brs{\ff(x) \fg(x)}(s)
    &=& \Lscale\brs{\opLT\ff}(s) \conv \brs{\opLT\fg}(s)
    & \forall \Real(s)\in\intcc{A+C}{B+D},\,c\in\intoo{A}{B}
    & \forall\ff,\fg\in\spLLRBu.
\end{array}}
\end{theorem}
\begin{proof}
\begin{align*}
   \opLT \brs{\ff(x)\conv\fg(x)}(s)
     &= \opLT \brs{ \int_{u\in\R} \ff(u)\fg(x-u) \du }(s)
     && \text{by definition of $\conv$} &&\text{\xref{def:conv}}
   \\&=  \int_{u\in\R} \ff(u)\brs{\opLT \fg(x-u)}(s) \du
   \\&=  \int_{u\in\R} \ff(u) e^{-su} \; \brs{\opLT\fg(x)}(s) \du
     && \text{by \prope{translation} property}
     && \text{\xref{thm:opLT_shift}}
   \\&= \mcom{\brp{\Lscale\int_{u\in\R} \ff(u) e^{-su} \du}}
             {$\brs{\opLT\ff}(s)$                          } \;
        \brs{\opLT\fg}(s)
   \\&= \brs{\opLT\ff}(s)\;  \brs{\opLT\fg}(s)
     && \Real(s)\in\intcc{A}{B}\seti\intcc{C}{D}
     && \text{by definition of $\opLT$ \xref{def:opLT}}
   \\
   \opLT[\ff(x)\fg(x)](s)
     &= \opLT\brs{\brp{\opLT^{-1}\opLT\ff(x)}\;\fg(x)}(s)
     && \text{by def. of operator inverse} &&\text{\ifxref{operator}{def:op_inv}}
   \\&= \opLT\brs{\brp{ \Lscalei \int_{v\in\R} \brs{\opLT\ff(x)}(v) e^{sxv} \dv }\;\fg(x)}(s)
     && \text{by \prefp{thm:opLTi}}
   \\&= \Lscalei \int_{v\in\R} \brs{\opLT\ff(x)}(v) \brs{\opLT\brp{e^{sxv} \;\fg(x)}}(s,v) \dv
   \\&= \Lscalei \int_{v\in\R} \brs{\opLT\ff(x)}(v) \brs{\opLT\fg(x)}(s-v) \dv
     && \text{by \prefp{thm:opLT_shift}}
   \\&= \Lscalei\brs{\opLT\ff}(s)\conv \brs{\opLT\fg}(s)
     && \text{by definition of $\conv$} &&\text{\xref{def:conv}}
\end{align*}
\end{proof}

%=======================================
\section{Calculus properties}
%=======================================
%---------------------------------------
\begin{theorem}
\footnote{
  \citerpcu{betten2008L}{301}{(B.27)}{https://link.springer.com/content/pdf/bbm:978-3-540-85051-9/1.pdf},
  \citerpcu{levy1958}{15}{Equation 0.7}{https://archive.org/details/c_cowles_ieee_Levy/page/n15/}
  }
\label{thm:opLT_diff}
\label{thm:opLT_int}
%---------------------------------------
Let $\opLT$ be the \ope{Laplace Transform} operator \xref{def:opLT}.
\thmbox{\begin{array}{>{\ds}r c >{\ds}rc>{\ds}l}
  \brb{\lim_{t\to-\infty}\fx(t)=0} &\implies&
     \brb{\opLT \brs{\ddt \fx(t)}} &=& s \brs{\opLT\fx}(s)
    \\&&
    \opLT \int_{u=-\infty}^{u=t} \fx(u) \du &=& \frac{1}{s} \brs{\opLT\fx}(s)
\end{array}}
\end{theorem}
\begin{proof}
\begin{enumerate}
  \item
\begin{align*}
  \boxed{\opLT \brs{\ddt \fx(t)}}
    &\eqd \int_{t\in\R} \mcom{\brs{\ddt \fx(t)}}{$\dv$} \mcom{e^{-st}}{$u$} \dt
    && \text{by definition of $\opLT$}
  \\&= \brlr{\mcom{e^{-st}}{$u$} \mcom{\fx(t)}{$v$}}_{t=-\infty}^{t=+\infty}
      -\int_{t\in\R} \mcom{\fx(t)}{$v$} \mcom{(-s)e^{-s t}}{$\du$} \dt
    && \text{by \thme{Integration by Parts}}
  \\&= \cancelto{0}{e^{-s \infty}}\fx(\infty) - e^{s \infty}\cancelto{0}{\fx(-\infty)}
      -(-s)\mcom{\int_{t\in\R} \fx(t) e^{-st} \dt}{\ope{Laplace Transform} of $\fx(t)$}
    && \text{by left hypothesis}
  \\&= \boxed{s \brs{\opLT\fx}(s)}
\end{align*}

  \item
\begin{enumerate}
  \item Define the \fncte{Heaviside function} $\fh(t)$ as\qquad \label{idef:heaviside}
    $\ds \fh(t)\eqd\brbl{\begin{array}{lM}
                           1 & for $t\ge0$
                         \\0 & otherwise
                         \end{array}}$

  \item Remainder of proof\ldots
    \begin{align*}
      \boxed{\opLT \int_{u=-\infty}^{u=t} \fx(u) \du}
        &\eqd \int_{t=-\infty}^{t=+\infty} \brs{\int_{u=-\infty}^{u=t} \fx(u) \du} e^{-s t} \dt
        && \text{by definition of $\opLT$}
      \\&= \int_{t=-\infty}^{t=+\infty} \brs{\int_{u=-\infty}^{u=+\infty} \fx(u) \fh(t-u) \du} e^{-s t} \dt
        && \brp{\begin{array}{M}
             by definition of \fncte{Heaviside function}\\
             \pref{idef:heaviside}
           \end{array}}
      \\&= \int_{v=-\infty}^{v=+\infty} \int_{u=-\infty}^{u=+\infty} \fx(u) \fh(v)  e^{-s (u+v)} \du \dv
        && \brp{\begin{array}{Mr@{\hspace{2pt}}c@{\hspace{2pt}}l}
             where      & v&\eqd&t-u\\
             $\implies$ & t&=&u+v
           \end{array}}
      \\&= \brs{\int_{v=-\infty}^{v=+\infty} \fh(v) e^{-s v} \dv}
            \mcom{\brs{\int_{u=-\infty}^{u=+\infty} \fx(u)   e^{-s u} \du }}
                 {\ope{Laplace Transform} of $\fx(t)$}
      \\&= \brs{\int_{v=0}^{v=\infty} e^{-s v} \dv} \brs{\opLT\fx}(s)
        && \brp{\begin{array}{M}
             by definition of \fncte{Heaviside function}\\
             \pref{idef:heaviside}
           \end{array}}
      \\&= \brlr{\frac{1}{-s}e^{-s v}}_{v=0}^{v=\infty} \brs{\opLT\fx}(s)
        && \text{by \thme{Fundamental Theorem of Calculus}}
      \\&= \boxed{\frac{1}{s} \brs{\opLT\fx}(s)}
    \end{align*}
\end{enumerate}
\end{enumerate}
\end{proof}

