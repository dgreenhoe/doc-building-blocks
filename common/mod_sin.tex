%============================================================================
% LaTeX File
% Daniel Greenhoe
%============================================================================

%======================================
\chapter{Modulation}
\label{chp:mod_sin}
%======================================
The transmission is performed by allowing the information sequence $\su$ to
affect the behavior of a {\em carrier} signal.
This technique is called {\em modulation} and we say that the 
information sequence {\em modulates} the carrier.
There are two general types of modulation:\\
\index{modulation!memoryless}
\index{modulation!with memory}
\begin{tabular}{lll}
   1. & memoryless modulation:  & only depends on the current signal value   \\
   2. & modulation with memory: & depends on current and past signal values.
\end{tabular}

\index{estimation}
The {\em receiver} generates an estimate\footnote{
   {\em estimation theory}: 
   Section~\ref{sec:awgn_est} \prefpo{sec:awgn_est},
   Appendix~\ref{app:est} \prefpo{app:est}
   }
$\sue$ of the sent information sequence $\su$ from the received signal $r(t)$.


%======================================
\section{Memoryless Modulation}
%======================================
%======================================
\subsection{Definitions}
%======================================
\begin{figure}[ht]\color{figcolor}
\begin{center}
\begin{fsK}
\setlength{\unitlength}{0.17mm}                  
\begin{picture}(900,150)(-100,-50)
  \thinlines                                      
  %\graphpaper[10](0,0)(700,100)                  
  \put(-100,  60 ){\makebox(30,0)[b]{$u$} }
  \put(-100,  50 ){\vector(1,0){50} }

  \put(-70, -50 ){\dashbox{4}( 280,160){} }
  \put(-70, -40 ){\makebox( 280,160)[b]{transmitter} }
  \put(   50,  60 ){\makebox(50,50)[b]{$\theta$} }
  \put(   50,  50 ){\vector(1,0){50} }

  \put(- 50,  00 ){\framebox( 100,100){mapper} }
  \put( 100,  00 ){\framebox( 100,100){modulator} }
  \put( 200,  50 ){\vector(1,0){100} }

  \put( 300,  00 ){\framebox(100,100){} }
  \put( 300,  10 ){\makebox(100, 90)[b]{channel} }
  \put( 210,  60 ){\makebox( 90, 50)[b]{$\fs(t;\theta)$} }
  \put( 400,  60 ){\makebox( 80, 50)[b]{$r(t;\theta)$} }

  \put( 400,  50 ){\vector(1,0){100} }
  \put( 500,  00 ){\framebox(100,100){detector} }
  \put( 600,  60 ){\makebox(50,50)[b]{$\hat{\theta}$} }
  \put( 600,  50 ){\vector(1,0){50}}
  \put( 650,  00 ){\framebox(100,100){} }
  \put( 650,  30 ){\makebox(100,40)[t]{inverse} }
  \put( 650,  30 ){\makebox(100,40)[b]{mapper} }
  \put( 480, -50 ){\dashbox{4}( 280,160){} }
  \put( 480, -40 ){\makebox( 280,160)[b]{receiver} }

  \put( 760,  60 ){\makebox(40,50)[b]{$\ue$} }
  \put( 750,  50 ){\vector(1,0){50}}
\end{picture}                                   
\end{fsK}
\end{center}
\caption{
   Memoryless modulation system model
   \label{fig:mod_model}
   }
\end{figure}


%---------------------------------------
\begin{definition}[Digital modulation]
\index{modulation!sinusoidal carriers}
\label{def:psk}
\label{def:qam}
\label{def:pam}
%---------------------------------------
Let 
\begin{liste}
  \item $a_n\in\{0,1,\ldots,K-1\}$, $f_n\in\{0,1,\ldots,M-1\}$, and $\theta_n\in\{0,1,\ldots,N-1\}$
  \item $a_{\mathsf{offset}}, f_{\mathsf{offset}}, \theta_{\mathsf{offset}}\in\R$
  \item $E,F\in\Rp$
  \item $T\in(0,\infty)$ be the signalling period
  \item $\{u_n\}$ be an information sequence to be sent to a receiver
  \item $\fg$ be a function of the form
        \[ (a_n,f_n,\theta_n) = \fg(u_n). \]
  \item $S$ be a set of modulation waveforms
    \defbox{
      S \eqd \left\{
        \f\fs(t;u_n)=
        \left[a_n - a_{\mathsf{offset}}\left] \sqrt{\frac{2E}{T}} 
        \cos\left[2\pi\left[f_c+F f_n- f_{\mathsf{offset}}\left]t +
        \left[\theta_n \frac{2\pi}{N}-\theta_{\mathsf{offset}} \right]
        \right.\right.\right.\right.
      \right]\right\}
      }
\end{liste}
Then
\begin{liste}
  \item A {\bf memoryless digital modulation using sinusoidal carriers} (MDMSC)
        is the pair $(\fg,S)$.
  \item A {\bf Pulse Amplitude Modulation} (PAM) is MDMSC with 
        \[f_n= f_{\mathsf{offset}} = \theta_n = \theta_{\mathsf{offset}} = 0\]
  \item A {\bf Phase Shift Keying} (PSK) is MDMSC with 
        \[a_n = a_{\mathsf{offset}} = f_n = f_{\mathsf{offset}} = 0\]
  \item A {\bf Frequency Shift Keying} (FSK) is MDMSC with 
        \[a_n = a_{\mathsf{offset}} = \theta_n = \theta_{\mathsf{offset}} = 0\]
  \item A {\bf Quadrature Amplitude Modulation} (QAM) is MDMSC with 
        \[f_n = f_{\mathsf{offset}} = 0\]
\end{liste}
\end{definition}

%---------------------------------------
\begin{theorem}
%---------------------------------------
Let $(\fg,S)$ be an MDMSC.
The energy $\opE \f\fs(t;n)$ of $\f\fs(t;n)\in S$ is
\thmbox{ \opE s_n \eqa a_n^2 E }
\end{theorem}
\begin{proof}
\begin{align*}
  \opE \f\fs(t;n)
    &\eqd \norm{a_n \sqrt{\frac{2E}{T}} \cos(2\pi(f_c+\Delta f f_n)t+\theta_n)}^2
  \\&= a_n^2 \frac{2E}{T}
       \norm{ \cos(2\pi(f_c+\Delta f f_n)t+\theta_n)}^2
  \\&= a_n^2 \frac{2E}{T}
       \int_0^T \cos^2(2\pi(f_c+\Delta f f_n)t+\theta_n) \dt
  \\&= a_n^2 \frac{2E}{T}\frac{1}{2}
       \int_0^T 1 + \cos(4\pi(f_c+\Delta f f_n)t+4\theta_n) \dt
  \\&= a_n^2 \frac{E}{T}\left[
       \int_0^T 1  \dt +
       \int_0^T \cos(4\pi(f_c+\Delta f f_n)t+4\theta_n) \dt
       \right]
  \\&\eqa a_n^2 \frac{E}{T}  \int_0^T 1  \dt
  \\&=    a_n^2 E
\end{align*}
\end{proof}

%======================================
\subsection{Orthogonality}
%======================================

%---------------------------------------
\begin{proposition}
\label{lem:pam_basis}
%---------------------------------------
Let $(V, \inprodn, S)$ be a modulation space and $\fs(t;m)\in S$.
\propbox{
  \brb{\text{$\otriple{V}{\inprodn}{S}$ is \prope{PAM}}}
  \implies
  \brb{\begin{array}{lM}
    \Psi \eqd \left\{ 
      \psi(t) = \frac{\sqrt{2}}{\norm{\sym}} \sym(t) \cos(2\pi f_c t)
      \right\}
    & is an orthonormal basis for $S$.
  \end{array}}
  }
\end{proposition}
\begin{proof}
\begin{enumerate}
  \item  Proof that $\Psi$ spans $S$:
    \begin{align*}
       \fs(t;m) 
         &\eqd a_m\frac{\sqrt{2}}{\norm{\sym}} \sym(t) \cos(2\pi f_c t) 
       \\&=    a_m\psi(t)
    \end{align*}

  \item Proof that $\Psi$ is orthonormal with respect to $\inprodn$.
  \begin{align*}
   \inprod{\psi_c(t)}{\psi_c(t)}
      &= \inprod{\frac{\sqrt{2}}{\norm{\sym}} \sym(t) \cos(2\pi f_c t)}
                {\frac{\sqrt{2}}{\norm{\sym}} \sym(t) \cos(2\pi f_c t)}
    \\&= \frac{\sqrt{2}}{\norm{\sym}}\frac{\sqrt{2}}{\norm{\sym}}
         \inprod{ \sym(t) \cos(2\pi f_c t)}
                { \sym(t) \cos(2\pi f_c t)}
    \\&= \frac{2}{\norm{\sym}^2}
         \opL
         \int_0^T \sym^2(t) \cos^2(2\pi f_c t) \; dt
    \\&= \frac{2}{\norm{\sym}^2}
         \opL
         \int_0^T \sym^2(t)\frac{1}{2}\left[1+ \cos(4\pi f_c t)\right] \; dt
    \\&= \frac{2}{\norm{\sym}^2}\frac{1}{2}
         \int_0^T \sym^2(t)\left[ 1 \right] \; dt
    \\&= \frac{1}{\norm{\sym}^2}
         \inprod{\sym(t)}{\sym(t)}
    \\&= \frac{1}{\norm{\sym}^2}
         \norm{\sym(t)}^2
    \\&= 1
\end{align*}
\end{enumerate}
\end{proof}

%---------------------------------------
\begin{proposition}
\label{lem:psk_basis}
%---------------------------------------
Let $(V, \inprodn, S)$ be a modulation space and $\fs(t;m)\in S$.
\propbox{
  \brb{\text{$\otriple{V}{\inprodn}{S}$ is \prope{PSK}}}
  \implies
  \brb{\begin{array}{lM}
   \Psi \eqd \left\{ \begin{array}{rcl}
      \psi_c(t) &=& \frac{\sqrt{2}}{\norm{\sym}} \sym(t) \cos(2\pi f_c t), \\
      \psi_\fs(t) &=& -\frac{\sqrt{2}}{\norm{\sym}} \sym(t) \sin(2\pi f_c t)
      \end{array}\right\}
    & is an orthonormal basis for $S$.
  \end{array}}
  }
\end{proposition}
\begin{proof}
\begin{enumerate}
  \item  $\Psi$ spans $S$:
    \begin{align*}
       \fs(t;a_m,b_m) 
         &\eqd r\frac{\sqrt{2}}{\norm{\sym}} \sym(t) \cos(2\pi f_c t + \theta_m)
       \\&=    r\frac{\sqrt{2}}{\norm{\sym}} \sym(t) 
               \left[ \cos\theta_m\cos(2\pi f_ct)-\sin\theta_m\sin(2\pi f_ct)\right]  
       \\&=    r\frac{\sqrt{2}}{\norm{\sym}} \sym(t)\cos\theta_m\cos(2\pi f_ct) - 
               r\frac{\sqrt{2}}{\norm{\sym}} \sym(t)\sin\theta_m\sin(2\pi f_ct)
       \\&=    r\cos\theta_m\frac{\sqrt{2}}{\norm{\sym}} \sym(t)\cos(2\pi f_ct) - 
               r\sin\theta_m\frac{\sqrt{2}}{\norm{\sym}} \sym(t)\sin(2\pi f_ct)
       \\&=    r\cos\theta_m \psi_c(t) + r_m\sin\theta_m \psi_\fs(t)
    \end{align*}

\item Proof that $\Psi$ is orthonormal with respect to $\inprodn$:
  See proof of \prefpp{lem:QAM_basis}.
\end{enumerate}
\end{proof}

\begin{figure}[ht]\color{figcolor}
\begin{center}
\begin{fsL}
\setlength{\unitlength}{0.2mm}
\begin{picture}(230,230)(-100,-100)
  %\graphpaper[10](0,0)(200,200)                  
  \thicklines                                      
  \put(-100 ,   0 ){\line(1,0){200} }
  \put(   0 ,-100 ){\line(0,1){200} }
  \put(   0 ,   0 ){\vector(1,1){60} }
  \thinlines
  %\put(   0 ,  60 ){\line(1,0){60} }
  %\put(  60 ,   0 ){\line(0,1){60} }
  \qbezier[12]( 60,  0)( 60, 30)( 60, 60)
  \qbezier[12](  0, 60)( 30, 60)( 60, 60)
  
  %\put(  60 ,-100 ){\makebox(  40, 90)[t]{$r\cos\theta$} }
  %\put(-100 ,   0 ){\makebox(  90, 90)[rt]{$r\sin\theta$} }
  \put( -30,   60 ){$\dot{s}_2(m)$ }
  \put(  50,  -15 ){$\dot{s}_1(m)$ }
  \put( -20 ,   0 ){\makebox(  80, 80){$r$} }
  \put(  28 ,   5 ){\makebox(  50, 50)[bl]{$\theta_m=\frac{2\pi}{8}(m-1)$} }
  \put( 110 ,  -5 ){$\psi_1=\frac{\sqrt{2}}{\norm{\sym}} \sym(t) \cos(2\pi f_c t)$}
  \put(   0 , 110 ){$\psi_2=-\frac{\sqrt{2}}{\norm{\sym}} \sym(t) \sin(2\pi f_c t)$}

  \put(  80 ,   0 ){\circle*{10}}
  \put(  57 ,  57 ){\circle*{10}}
  \put(   0 ,  80 ){\circle*{10}}
  \put( -57 ,  57 ){\circle*{10}}
  \put( -80 ,   0 ){\circle*{10}}
  \put( -57 , -57 ){\circle*{10}}
  \put(   0 , -80 ){\circle*{10}}
  \put(  57 , -57 ){\circle*{10}}

  %\put(   0 ,   0 ){\circle{160}}
  %\qbezier[20](  0, 80)( 80, 80)( 80, 00)
  %\qbezier[20](-80,  0)(-80, 80)(  0, 80)
  %\qbezier[20](-80,  0)(-80,-80)(  0,-80)
  %\qbezier[20](  0,-80)( 80,-80)( 80,  0)
  \setlength{\unitlength}{0.16mm}
  %============================================================================
% NCTU - Hsinchu, Taiwan
% LaTeX File
% Daniel Greenhoe
%
% Unit circle with radius 100
%============================================================================

\qbezier( 100,   0)( 100, 41.421356)(+70.710678,+70.710678) % 0   -->1pi/4
\qbezier(   0, 100)( 41.421356, 100)(+70.710678,+70.710678) % pi/4-->2pi/4
\qbezier(   0, 100)(-41.421356, 100)(-70.710678,+70.710678) %2pi/4-->3pi/4
\qbezier(-100,   0)(-100, 41.421356)(-70.710678,+70.710678) %3pi/4--> pi 
\qbezier(-100,   0)(-100,-41.421356)(-70.710678,-70.710678) % pi  -->5pi/4
\qbezier(   0,-100)(-41.421356,-100)(-70.710678,-70.710678) %5pi/4-->6pi/4
\qbezier(   0,-100)( 41.421356,-100)( 70.710678,-70.710678) %6pi/4-->7pi/4
\qbezier( 100,   0)( 100,-41.421356)( 70.710678,-70.710678) %7pi/4-->2pi



\end{picture}                                   
\end{fsL}
\end{center}
\caption{
   PSK vector representation, $M=8$
   \label{fig:PSK_vector}
   }
\end{figure}



%---------------------------------------
\begin{theorem}[Orthogonality for FSK]
%---------------------------------------
Let $(g,S)$ be an FSK modulation.
\begin{enume}
  \item If $F\in\set{n\frac{1}{2T}}{k\in\Zp}$, 
        then $\fs_m,\fs_n\in S$ are orthogonal for $m\not=n$.
  \item If $\fs_1,\fs_2\in S$ possibly different phases and
        $F\in\set{n\frac{1}{T}}{k\in\Zp}$, 
        then $\fs_m,\fs_n\in S$ are orthogonal for $m\not=n$.
\end{enume}
\end{theorem}
\begin{proof}

\begin{enumerate}
  \item Proof for identical phases:
\begin{align*}
   \inprod{\psi_m(t)}{\psi_n(t)}
     &=   \inprod{\sqrt{\frac{2}{T}}\cos[2\pi (f_c+mf_d)t]}
                 {\sqrt{\frac{2}{T}}\cos[2\pi (f_c+nf_d)t]}
   \\&=   \frac{2}{T}\inprod{\cos[2\pi (f_c+mf_d)t]}{\cos[2\pi (f_c+nf_d)t]}
   \\&=   \frac{2}{T}\int_0^T \cos[2\pi (f_c+mf_d)t] \cos[2\pi (f_c+nf_d)t] \; dt
   \\&=   \frac{1}{2}\frac{2}{T}\int_0^T \cos[2\pi(f_c+mf_d)t-2\pi(f_c+nf_d)t]+  \cos[2\pi(f_c+mf_d)t+2\pi(f_c+nf_d)t] \; dt
   \\&=   \frac{1}{T}\int_0^T \cos[2\pi(m-n)f_dt] + \cos[4\pi(f_ct+2\pi(m+n)f_dt] \; dt
   \\&\eqa \frac{1}{T}\int_0^T \cos[2\pi(m-n)f_dt]  \; dt
   \\&=   \left.\frac{1}{T} \frac{1}{2\pi(m-n)f_d} \sin[2\pi(m-n)f_dt] \right|_0^T
   \\&=   \frac{\sin[2\pi(m-n)f_dT]}{2\pi(m-n)f_d T} 
   \\&=   \left\{
          \begin{tabular}{ll}
             $1$                                           & for $m=n$ \\
             $\frac{\sin[2\pi(m-n)f_dT]}{2\pi(m-n)f_d T}$  & for $m\ne n$.
          \end{tabular}
          \right.
   \\&=   \left\{
          \begin{tabular}{ll}
             $1$                                           & for $m=n$ \\
             $0$  &       for $m\ne n$ and $f_d=\frac{k}{2T}$, $k=1,2,3,\ldots$.
          \end{tabular}
          \right.
\end{align*}

\item Proof for different phase:
  \begin{align*}
     \inprod{\psi_m(t;\phi)}{\psi_n(t)}
        &= \opL \inprod{\cos(2\pi f_m t + \phi)}{\cos(2\pi f_n t)}
      \\&= \opL \int_t^{t+T} \cos(2\pi f_m t + \phi) \cos(2\pi f_n t) \; dt
      \\&= \int_t^{t+T} \cos\left[2\pi (f_m-f_n) t + \phi\right] \dt
      \\&= \left.\frac{\sin[2\pi (f_m-f_n) t + \phi]}
                      {2\pi(f_m-f_n)}
           \right|_t^{t+T}
      \\&= \frac{\sin[2\pi (f_m-f_n)(t+T) + \phi]-\sin[2\pi (f_m-f_n)t + \phi]}{2\pi(f_m-f_n)}
  \end{align*}

\item For orthogonality, this implies
  \begin{align*}
     2\pi (f_m-f_n)(t+T) + \phi &= 2\pi (f_m-f_n)t + \phi + k2\pi, k=1,2,3,\ldots \\
     2\pi (f_m-f_n)T  &= k2\pi \\
     (f_m-f_n)T  &= k \\
     f_m-f_n  &= \frac{k}{T} \\
  \end{align*}
\end{enumerate}
\end{proof}

%---------------------------------------
\begin{proposition}
\label{lem:QAM_basis}
%---------------------------------------
Let $(V, \inprodn, S)$ be a QAM modulation space and 
$\fs(t;a_m,b_m)\in S$.
Then the set
\[
   \Psi \eqd \left\{ \begin{array}{rcl}
      \psi_c(t) &=& \frac{\sqrt{2}}{\norm{\sym}} \sym(t) \cos(2\pi f_c t), \\
      \psi_\fs(t) &=& -\frac{\sqrt{2}}{\norm{\sym}} \sym(t) \sin(2\pi f_c t)
      \end{array}\right\}
\]

is an orthonormal basis for $S$.
\end{proposition}

\begin{proof}

1. $\Psi$ spans $S$:
\begin{eqnarray*}
   \fs(t;a_m,b_m) 
     &\eqd& a_m\frac{\sqrt{2}}{\norm{\sym}} \sym(t) \cos(2\pi f_c t) + 
            b_m\frac{\sqrt{2}}{\norm{\sym}} \sym(t) \sin(2\pi f_c t)
   \\&=&    a_m\psi_c(t) + b_m\psi_\fs(t)
\end{eqnarray*}

2. $\Psi$ is orthonormal with respect to $\inprodn$.

\begin{eqnarray*}
   \inprod{\psi_c(t)}{\psi_c(t)}
      &=& \inprod{\frac{\sqrt{2}}{\norm{\sym}} \sym(t) \cos(2\pi f_c t)}
                 {\frac{\sqrt{2}}{\norm{\sym}} \sym(t) \cos(2\pi f_c t)}
    \\&=& \frac{\sqrt{2}}{\norm{\sym}}\frac{\sqrt{2}}{\norm{\sym}}
          \inprod{ \sym(t) \cos(2\pi f_c t)}
                 { \sym(t) \cos(2\pi f_c t)}
    \\&=& \frac{2}{\norm{\sym}^2}
          \opL
          \int_0^T \sym^2(t) \cos^2(2\pi f_c t) \; dt
    \\&=& \frac{2}{\norm{\sym}^2}
          \opL
          \int_0^T \sym^2(t)\frac{1}{2}\left[1+ \cos(4\pi f_c t)\right] \; dt
    \\&=& \frac{2}{\norm{\sym}^2}\frac{1}{2}
          \int_0^T \sym^2(t)\left[ 1 \right] \; dt
    \\&=& \frac{1}{\norm{\sym}^2}
          \inprod{\sym(t)}{\sym(t)}
    \\&=& \frac{1}{\norm{\sym}^2}
          \norm{\sym(t)}^2
    \\&=& 1
\\ \\
   \inprod{\psi_\fs(t)}{\psi_\fs(t)}
      &=& \inprod{\frac{\sqrt{2}}{\norm{\sym}} \sym(t) \sin(2\pi f_c t)}
                 {\frac{\sqrt{2}}{\norm{\sym}} \sym(t) \sin(2\pi f_c t)}
    \\&=& \frac{\sqrt{2}}{\norm{\sym}}\frac{\sqrt{2}}{\norm{\sym}}
          \inprod{ \sym(t) \sin(2\pi f_c t)}
                 { \sym(t) \sin(2\pi f_c t)}
    \\&=& \frac{2}{\norm{\sym}^2}
          \opL
          \int_0^T \sym^2(t) \sin^2(2\pi f_c t) \; dt
    \\&=& \frac{2}{\norm{\sym}^2}
          \opL
          \int_0^T \sym^2(t)\frac{1}{2}\left[1- \cos(4\pi f_c t)\right] \; dt
    \\&=& \frac{2}{\norm{\sym}^2}\frac{1}{2}
          \int_0^T \sym^2(t)\left[ 1 \right] \; dt
    \\&=& \frac{1}{\norm{\sym}^2}
          \inprod{\sym(t)}{\sym(t)}
    \\&=& \frac{1}{\norm{\sym}^2}
          \norm{\sym(t)}^2
    \\&=& 1
\\ \\
   \inprod{\psi_\fs(t)}{\psi_c(t)}
      &=& \inprod{\psi_c(t)}{\psi_\fs(t)}
    \\&=& \inprod{\frac{\sqrt{2}}{\norm{\sym}} \sym(t) \cos(2\pi f_c t)}
                 {\frac{\sqrt{2}}{\norm{\sym}} \sym(t) \sin(2\pi f_c t)}
    \\&=& \frac{\sqrt{2}}{\norm{\sym}}\frac{\sqrt{2}}{\norm{\sym}}
          \inprod{ \sym(t) \cos(2\pi f_c t)}
                 { \sym(t) \sin(2\pi f_c t)}
    \\&=& \frac{2}{\norm{\sym}^2}
          \opL
          \int_0^T \sym^2(t) \cos(2\pi f_c t)\sin(2\pi f_c t) \; dt
    \\&=& \frac{2}{\norm{\sym}^2}
          \opL
          \int_0^T \sym^2(t)\frac{1}{2}
          \left[\sin(4\pi f_c t)-\sin(0)\right] \; dt
    \\&=& \frac{1}{\norm{\sym}^2}
          \int_0^T \sym^2(t)
          \left[\opL\sin(4\pi f_c t)-0\right] \; dt
    \\&=& \frac{1}{\norm{\sym}^2}
          \int_0^T \sym^2(t)
          \left[0-0\right] \; dt
    \\&=& 0
\end{eqnarray*}
\end{proof}


Definition~\ref{def:qam} represents elements of $S$ in rectangular form
$(a_m,b_m)$.
The elements of $S$ can also be represented in polar form
$(r_m,\theta_m)$ as shown below.
\begin{eqnarray*}
   \fs(t;m) 
     &=& \dot{s}_c(a_m) \psi_c(t) + \dot{s}_c(b_m) \psi_\fs(t)
   \\&=& r_m \frac{\sqrt{2}}{\norm{\sym}} \sym(t) 
         \left[
            a_m \cos(2\pi f_c t) -b_m \sin(2\pi f_c t)
         \right]
   \\&=& r_m \frac{\sqrt{2}}{\norm{\sym}} \sym(t) 
         \left[
            \cos\theta_m \cos(2\pi f_c t) -\sin\theta_m \sin(2\pi f_c t)
         \right]
   \\&=& r_m \frac{\sqrt{2}}{\norm{\sym}} \sym(t)  
         \cos\left[ 2\pi f_c t + \theta_m \right]
\end{eqnarray*}



\begin{figure}[ht]\color{figcolor}
\begin{center}
\begin{fsL}
\setlength{\unitlength}{0.2mm}
\begin{picture}(230,230)(-100,-100)
  %\graphpaper[10](0,0)(200,200)                  
  \thicklines                                      
  \put(-100 ,   0 ){\line(1,0){200} }
  \put(   0 ,-100 ){\line(0,1){200} }
  \put(   0 ,   0 ){\vector(1,1){60} }
  \thinlines
  %\put(   0 ,  60 ){\line(1,0){60} }
  %\put(  60 ,   0 ){\line(0,1){60} }
  \qbezier[12]( 60,  0)( 60, 30)( 60, 60)
  \qbezier[12](  0, 60)( 30, 60)( 60, 60)
  
  \put(  50 , -15 ){$a_m$} 
  \put( -30 ,  55 ){$b_m$} 
  \put(  10 ,  34 ){$r_m$} 
  \put(  20 ,   5 ){$\theta_m$} 
  \put( 110 ,  -5 ){$\psi_c(t)=\frac{\sqrt{2}}{\norm{\sym}} \sym(t) \cos(2\pi f_c t)$}
  \put(   0 , 110 ){$\psi_\fs(t)=-\frac{\sqrt{2}}{\norm{\sym}} \sym(t) \sin(2\pi f_c t)$}

  \put(  60 ,  60 ){\circle*{10}}
  \put( -60 ,  60 ){\circle*{10}}
  \put( -60 , -60 ){\circle*{10}}
  \put(  60 , -60 ){\circle*{10}}
\end{picture}                                   
\end{fsL}
\end{center}
\caption{
   QAM rectangular $(a_m,b_m)$ and polar $(r_m,\theta_m)$ representations
   \label{fig:QAM_vector}
   }
\end{figure}


%=======================================
\subsection{Measures}
%=======================================
%---------------------------------------
\subsubsection{Measures}
%---------------------------------------
%---------------------------------------
\begin{theorem}
\index{energy!Pulse Amplitude Modulation}
\index{distance!Pulse Amplitude Modulation}
%---------------------------------------
\thmboxt{
  The PAM modulation space has {\bf energy} and {\bf distance} measures
  \\\indentx$\begin{array}{rcl}
   \opE{\fs(t;m)}              &=& a_m^2                \\
   \metric{\fs(t;m)}{\fs(t;n)} &=& |a_m - a_n|.
  \end{array}$
  }
\end{theorem}
\begin{proof}
Because PAM is a modulation space,
\begin{liste}
   \item the energy measure follows from \prefp{thm:ms_energy} (\prefpo{thm:ms_energy})
   \item the distance measure from \prefp{thm:ms_dist}  (\prefpo{thm:ms_dist}).
\end{liste}
\end{proof}

%---------------------------------------
\begin{proposition}
\label{thm:ms_energy}
\index{energy!generalized coherent modulation}
%---------------------------------------
Let
\begin{liste}
   \item $(V, \inprodn, S)$ be a modulation space and $\fs(t)\in S$
   \item $\Psi\eqd\{ \psi_n(t): n=1,2,\ldots,N\}$ 
         be a set of orthonormal functions that span $S$
   \item $\dot{s}_n\eqd\inprod{\fs(t)}{\psi_n(t)}$
\end{liste}
\propboxt{
  The {\bf energy} in $\fs(t)$ is
  \\\indentx$\ds\opE \fs(t) = \sum_{n=1}^N \abs{\dot{s}_n}^2$
  }
\end{proposition}
\begin{proof}
\begin{align*}
   \opE \fs(t)
     &\eqd \norm{\fs(t)}^2      
   \\&=    \norm{\sum_{n=1}^N \dot{s}_n \psi_n(t)}^2  
   \\&=    \sum_{n=1}^N |\dot{s}_n|^2   
     %&&    \text{by \prefp{thm:ortho_norm} (\prefpo{thm:ortho_norm}) }
\end{align*}
\end{proof}

%---------------------------------------
\begin{proposition}
\label{thm:ms_dist}
\index{distance!generalized coherent modulation}
%---------------------------------------
Let 
\begin{liste}
   \item $(V, \inprodn, S)$ be a modulation space and $\fs(t;m)\in S$
   \item $\Psi\eqd\{ \psi_n(t): n=1,2,\ldots,N\}$ 
         be a set of orthonormal functions that span $S$
   \item $\dot{s}_n(m)\eqd\inprod{\fs(t;m)}{\psi_n(t)}$
\end{liste}
\propboxt{
  The {\bf distance} between waveforms $\fs(t;m)$ and $\fs(t;k)$ is
  \\\indentx$\ds\metric{\fs(t;m)}{\fs(t;k)} \eqd \sqrt{\sum_{n=1}^N |\dot{s}_n(m)-\dot{s}_n(k)|^2}$
  }
\end{proposition}
\begin{proof}
\begin{align*}
   \distsq{\fs(t;m)}{\fs(t;k)}
     &\eqd \norm{ \fs(t;m)-\fs(t;k)}^2 
   \\&=    \sum_{n=1}^N |\dot{s}_n(m)-\dot{s}_n(k)|^2 
     && \text{by \prefp{thm:ortho_dist} (\prefpo{thm:ortho_dist})} 
\end{align*}
\end{proof}

%---------------------------------------
\begin{theorem}
\index{energy!Phase Shift Keying}
\index{distance!Phase Shift Keying}
%---------------------------------------
\thmboxt{
  The PSK modulation space has {\bf energy} and {\bf distance} measures
  \\\indentx$\begin{array}{rcl}
     \opE{\fs(t;m)}               &=& \ds r^2              
   \\\metric{\fs(t;m)}{\fs(t;n)}  &=& r \sqrt{ 2 - 2\cos\left( \theta_m - \theta_n \right)}.
  \end{array}$
  }
\end{theorem}
\begin{proof}
\begin{align*}
   \opE{\fs(t;m)} 
     &\eqd \norm{\fs(t;m)}^2
   \\&=    \norm{\dot{s}_c(m)\psi_1(t) + \dot{s}_s(m)\psi_2(t)}^2
   \\&=    \dot{s}_c^2(m) + \dot{s}_s^2(m)  
     %&&    \text{by \prefp{thm:ortho_norm} }
   \\&=    \left( r \cos\theta_m \right)^2 +
           \left( r \sin\theta_m \right)^2 
   \\&=    r^2 \left( \cos^2\theta_m + \sin^2\theta_m \right)
   \\&=    r^2
\end{align*}

\begin{align*}
   \distsq{\fs(t;m)}{\fs(t;n)}
     &= \norm{\fs(t;m)-\fs(t;n)}^2
   \\&= \norm{ [\dot{s}_c(m)\psi_1(t)+\dot{s}_s(m)\psi_2(t)]-   
               [\dot{s}_c(n)\psi_1(t)+\dot{s}_s(n)\psi_2(t)]
             }^2
   \\&= \norm{ [\dot{s}_c(m)-\dot{s}_c(n)]\psi_1(t) +
               [\dot{s}_s(m)-\dot{s}_s(n)]\psi_2(t)
             }^2
   \\&= [\dot{s}_c(m)-\dot{s}_c(n)]^2 + [\dot{s}_s(m)-\dot{s}_s(n)]^2
        \hspace{3ex} \mbox{ by \prefp{thm:ortho_dist} }
   \\&= [r\cos \theta_m - r\cos \theta_n]^2 + [r\sin \theta_m + r\sin \theta_n ]^2
   \\&= r^2\left( [\cos \theta_m - \cos \theta_n]^2 + [\sin \theta_m + \sin \theta_n ]^2 \right)
   \\&= r^2\left( [\cos^2\theta_m -2\cos \theta_m\cos \theta_n + \cos^2 \theta_n] + 
                  [\sin^2\theta_m -2\sin \theta_m\sin \theta_n + \sin^2 \theta_n]\right)
   \\&= r^2\left( [\cos^2\theta_m + \sin^2 \theta_m] +
                  [\cos^2\theta_n + \cos^2 \theta_n] -
                  2[\cos \theta_m\cos \theta_n + \sin \theta_m \sin \theta_n ] 
           \right)
   \\&= r^2[ 1 + 1 - 2\cos(\theta_m-\theta_n) ]
   \\&= 2r^2[ 1 - \cos(\theta_m-\theta_n) ]
\end{align*}
\end{proof}

%---------------------------------------
\begin{theorem}
\index{energy!Frequency Shift Keying}
\index{distance!Frequency Shift Keying}
%---------------------------------------
\thmboxt{
  The FSK modulation space has {\bf energy} and {\bf distance} measures equivalent to
  \\\indentx$\begin{array}{rcl}
     \opE{\fs(t;m)}              &=& \dot{s}^2            \\
     \metric{\fs(t;m)}{\fs(t;n)} &=& \sqrt{2}\;\dot{s}
\end{array}$}
\end{theorem}
\begin{proof}
The energy measure is a result of \prefp{thm:ms_energy} (\prefpo{thm:ms_energy}).\\
For distance,
\begin{align*}
   \distsq{\fs(t;m)}{\fs(t;n)}
     &= \sum_{k=1}^N |\dot{s}_k(m)-\dot{s}_{nk}|^2 
     && \text{\prefp{thm:ms_dist}}
   \\&= \sum_{k=1}^N |\dot{s}_k(m)-\dot{s}_{nk}|^2 
   \\&= (\dot{s}-0)^2 + (\dot{s}-0)^2
   \\&= 2\dot{s}^2.
\end{align*}
\end{proof}

%---------------------------------------
\begin{theorem}
\index{energy!Quadrature Amplitude Modulation}
\index{distance!Quadrature Amplitude Modulation}
%---------------------------------------
\thmboxt{
  The QAM modulation space has {\bf energy} and {\bf distance} measures equivalent to
  \\\indentx$\begin{array}{rclcl}
    \opE{\fs(t;m)}    &=& a_m^2 + b_m^2  &=& r_m^2 
  \\\metric{\fs(t;m)}{\fs(t;n)} &= \sqrt{(a_m-a_n)^2 + (b_m-b_n)^2}
  \end{array}$}
\end{theorem}
\begin{proof}
\begin{align*}
   \opE{\fs(t;m)} 
     &\eqd \norm{\fs(t;m)}^2 
   \\&=    \norm{
           a_m\frac{\sqrt{2}}{\norm{\sym}} \sym(t) \cos(2\pi f_c t) + 
           b_m\frac{\sqrt{2}}{\norm{\sym}} \sym(t) \sin(2\pi f_c t) 
           }^2
   \\&=    \norm{ a_m \psi_c(t) + b_m \psi_\fs(t) }^2
   \\&=    a_m^2 + b_m^2  
    %&& \text{by \prefp{thm:ortho_norm} \prefpo{thm:ortho_norm}}
   \\&=    ( r_m \cos\theta_m )^2 + ( r_m \sin\theta_m )^2 
   \\&=    r_m^2 \left( \cos^2\theta_m + \sin^2\theta_m \right)
   \\&=    r_m^2
\end{align*}

\begin{align*}
   \distsq{\fs(t;m)}{\fs(t;n)} &\eqd \norm{\fs(t;m) - \fs(t;n) }^2
   %\\&=& \norm{\left(
   %         a_m\frac{\sqrt{2}}{\norm{\sym}} \sym(t) \cos(2\pi f_c t) + 
   %         b_m\frac{\sqrt{2}}{\norm{\sym}} \sym(t) \sin(2\pi f_c t) 
   %      \right) - \left(
   %         a_n\frac{\sqrt{2}}{\norm{\sym}} \sym(t) \cos(2\pi f_c t) + 
   %         b_n\frac{\sqrt{2}}{\norm{\sym}} \sym(t) \sin(2\pi f_c t) 
   %      \right) }^2
   \\&= \norm{\left(a_m \psi_c(t) + b_m\psi_\fs(t)\right) - 
              \left(a_n \psi_c(t) + b_n\psi_\fs(t)\right) }^2
   \\&= |a_m-a_n|^2 + |b_m-b_n|^2
     && \text{by \prefp{thm:ortho_dist} \prefpo{thm:ortho_dist}}
\end{align*}
\end{proof}

%======================================
\section{Continuous Phase Modulation (CPM)}
\index{Continuous Phase Modulation}
\index{CPM}
\label{sec:CPM}
%======================================
\begin{figure}[ht]\color{figcolor}
\begin{center}
\begin{fsK}
\setlength{\unitlength}{0.17mm}                  
\begin{picture}(900,150)(-100,-50)
  \thinlines                                      
  %\graphpaper[10](0,0)(700,100)                  
  \put(-100,  60 ){\makebox(30,0)[b]{$u_n$} }
  \put(-100,  50 ){\vector(1,0){50} }

  \put(-70, -50 ){\dashbox{4}( 280,160){} }
  \put(-70, -40 ){\makebox( 280,160)[b]{transmitter} }
  \put(   50,  60 ){\makebox(50,50)[b]{$y_n$} }
  \put(   50,  50 ){\vector(1,0){50} }

  \put(- 50,  00 ){\framebox( 100,100){mapper} }
  \put( 100,  00 ){\framebox( 100,100){modulator} }
  \put( 200,  50 ){\vector(1,0){100} }

  \put( 300,  00 ){\framebox(100,100){} }
  \put( 300,  10 ){\makebox(100, 80)[c]{channel} }
  \put( 210,  60 ){\makebox( 90, 50)[b]{$\f\fs(t;\{u_n\})$} }
  \put( 400,  60 ){\makebox( 80, 50)[b]{$\fr(t;\{u_n\})$} }

  \put( 400,  50 ){\vector(1,0){100} }
  \put( 500,  00 ){\framebox(100,100){detector} }
  \put( 600,  60 ){\makebox(50,50)[b]{$\hat{y_n}$} }
  \put( 600,  50 ){\vector(1,0){50}}
  \put( 650,  00 ){\framebox(100,100){} }
  \put( 650,  30 ){\makebox(100,40)[t]{inverse} }
  \put( 650,  30 ){\makebox(100,40)[b]{mapper} }
  \put( 480, -50 ){\dashbox{4}( 280,160){} }
  \put( 480, -40 ){\makebox( 280,160)[b]{receiver} }

  \put( 760,  60 ){\makebox(40,50)[b]{$\{u_n\}$} }
  \put( 750,  50 ){\vector(1,0){50}}
\end{picture}                                   
\end{fsK}
\end{center}
\caption{
   Continuous Phase Modulation system model
   \label{fig:cpm_model}
   }
\end{figure}

\begin{figure}[ht] \color{figcolor}
\begin{center}
\begin{fsL}
\setlength{\unitlength}{0.2mm}
\begin{tabular}{cc}
   LREC & LRC 
\\
\begin{picture}(200,200)(-50,-20)
  %\graphpaper[10](-100,0)(300,150)
  \thinlines                                      
  \put( -20 ,   0 ){\line(1,0){170} }
  \put(   0 , -20 ){\line(0,1){170} }

  \put( -30 , 100 ){$\frac{1}{2}$ }
  \put(  50 , -20 ){\makebox(100,40)[b]{$LT$} }
  \put( 160 , -10 ){$t$ }
  \qbezier[12](100,  0)(100, 50)(100,100)
  \qbezier[12](  0,100)( 50,100)(100,100)
  %\multiput(  0,100)(10,0){10}{\line(1,0){5}}

  \put(  0,  0){\line(1,1){100}}
  \put(100,100){\line(1,0){50}}
\end{picture}                                   
&
\begin{picture}(200,150)(-50,-20)
  %\graphpaper[10](-100,0)(300,150)
  \thinlines                                      
  \put( -20 ,   0 ){\line(1,0){170} }
  \put(   0 , -20 ){\line(0,1){170} }

  \put( -30 , 100 ){$\frac{1}{2}$ }
  \put(  50 , -20 ){\makebox(100,40)[b]{$LT$} }
  \put( 160 , -10 ){$t$ }
  \qbezier[12](100,  0)(100, 50)(100,100)
  \qbezier[12](  0,100)( 50,100)(100,100)

  \qbezier(   0,   0)(  30,   0)(  50,  50)
  \qbezier(  50,  50)(  70, 100)( 100, 100)
  \put(100,100){\line(1,0){50}}
\end{picture}
\end{tabular}
\end{fsL}
\end{center}
\caption{
  CPM phase pulses $\rho(t)$
   \label{fig:CPM_pp}
   }
\end{figure}

Continuous modulation can be realized using {\em phase pulses}
which are illustrated in \prefpp{fig:CPM_pp} and defined in 
\pref{def:CPM_pp} (next).
%---------------------------------------
\begin{definition}
\label{def:CPM_pp}
%---------------------------------------
Let $L\in\Zp$ be the {\bf response length}
and $T$ the {\bf signalling rate}.
The function $\rho:\R\to\R$ is a {\bf phase pulse} if
\begin{enume}
   \item $\rho(t)$ is continuous
   \item $\rho(t)=0$ for $t\le0$
   \item $\rho(t)=\frac{1}{2}$ for $t\ge LT$.
\end{enume}
\end{definition}

%---------------------------------------
\begin{definition}
\label{def:CPM}
%---------------------------------------
Let 
\begin{eqnarray*}
   n   & = & \floor{\frac{t}{T}} \\
   x_n &\in& \{0,1,\ldots, M-1 \} \\
   y_n & = & 2x_n -1 \in \{ \pm1, \pm2,\ldots, \pm(M-1)\}.
\end{eqnarray*}

Then
{\bf Continuous Phase Modulation} (CPM) signalling waveforms are
\begin{eqnarray*}
   \f\fs(t;\ldots,u_{n-1},u_n)
     &=& a \frac{2}{\sqrt{T}} 
           \cos\left[ 
              2\pi f_ct + 
              2\pi\sum_{k=-\infty}^n y_k h_k \rho(t-kT)
           \right]
   \\&=& a \frac{2}{\sqrt{T}} 
           \cos\left( \begin{array}{ccccc}
              \ds \underbrace{2\pi f_ct} &+&
              \ds \underbrace{\pi \sum_{k=-\infty}^{n-L} y_k h_k}  &+&
              \ds \underbrace{2\pi \sum_{k=n-L+1}^n y_k h_k \rho(t-kT)}
              \\
              \mbox{carrier} && \mbox{state} && \mbox{maintains continuous phase}
           \end{array}
           \right)
\end{eqnarray*}
\end{definition}


%---------------------------------------
\subsection{Phase Pulse waveforms}
%---------------------------------------
\[   \rho(t) = \int_t \rho'(t) \; dt \]

\begin{fsL}
\begin{align*}
   \intertext{Rectangular (LREC)} \\
   \rho'(t) &=
      \left\{ 
         \begin{tabular}{ll}
            $\frac{1}{2LT}$   & for $0\le t \le LT$ \\
            $0$               & otherwise
         \end{tabular}
      \right.
&
   \rho(t) &= 
      \left\{ 
         \begin{tabular}{ll}
            $0$               & for $t<0$ \\
            $\frac{1}{2LT}t$  & for $0\le t< LT$ \\
            $\frac{1}{2}$     & for $t\ge LT$ 
         \end{tabular}
      \right.
\\ \\
   \intertext{Raised Cosine (LRC)} \\
   \rho'(t) &= 
      \left\{ 
         \begin{array}{ll}
            \frac{1}{2LT}\left[ 1 - \cos\left(\frac{2\pi}{LT}t\right) \right]  & \mbox{for } 0\le t< LT \\
            0              & \mbox{otherwise}
         \end{array}
      \right.
&
   \rho(t) &=
      \left\{ 
         \begin{array}{ll}
            0             & \mbox{for } t<0 \\
            \frac{1}{2LT}\left[ t - \frac{LT}{2\pi}\sin\left(\frac{2\pi}{LT}t\right) \right]  & \mbox{for } 0\le t< LT \\
            \frac{1}{2}   & \mbox{for } t\ge LT
         \end{array}
      \right.
\\ \\
   \intertext{Gaussian Minimum Shift Keying (GMSK)} \\
   \rho'(t) &= 
      \left\{ 
         \begin{array}{ll}
            \Qb{\frac{2\pi B(t-\frac{T}{2})}{\sqrt{ln2}}} -   
            \Qb{\frac{2\pi B(t+\frac{T}{2})}{\sqrt{ln2}}}
            & \mbox{for } 0\le t< LT \\
            0              & \mbox{otherwise}
         \end{array}
      \right.
&
   \rho(t) &=
      \int_{-\infty}^t \rho'(t) \; dt
\end{align*}
\end{fsL}

%---------------------------------------
\subsection{Special Cases}
%---------------------------------------
%---------------------------------------
\begin{definition}
\index{Full Response Continuous Phase Modulation}
\index{Partial Response Continuous Phase Modulation}
%---------------------------------------
{\bf Full response} CPM has response length $L=1$.
{\bf Partial response} CPM has response length $L\ge2$.
\end{definition}

In the case of Full Response CPM, the signalling waveform simplifies to
\begin{align*}
   \f\fs(t;\ldots,u_{n-1},u_n)
     &= a \frac{2}{\sqrt{T}} 
          \cos\left(
             2\pi f_ct + 
             \pi \sum_{k=-\infty}^{n-L} y_k h_k  + 
             2\pi \sum_{k=n-L+1}^n y_k h_k \rho(t-kT)
          \right)
   \\&= a \frac{2}{\sqrt{T}} 
          \cos\left(
             2\pi f_ct + 
             \pi \sum_{k=-\infty}^{n-1} y_k h_k  + 
             2\pi \sum_{k=n-1+1}^n y_k h_k \rho(t-kT)
          \right)
   \\&= a \frac{2}{\sqrt{T}} 
          \cos\brp{
             \mcom{2\pi f_ct}{carrier} +
             \mcom{\pi \sum_{k=-\infty}^{n-1} y_k h_k}{state}  +
             \mcom{2\pi y_n h_n \rho(t-nT)}{maintains c.p.}
             }
\end{align*}

%---------------------------------------
\begin{definition}
\index{Continuous Phase Frequency Shift Keying}
\index{CPFSK}
%---------------------------------------
\opd{Continuous Phase Frequency Shift Keying} (CPFSK)
is full response CPM ($L=1$) with $h_n=h$ is constant
and LREC phase pulse.
\end{definition}

In CPFSK, the signalling waveform is
\begin{align*}
   \f\fs(t;\ldots,u_{n-1},u_n)
     &=& a \frac{2}{\sqrt{T}} 
           \cos\left(
              2\pi f_ct + 
              \pi \sum_{k=-\infty}^{n-L} y_k h_k  + 
              2\pi \sum_{k=n-L+1}^n y_k h_k \rho(t-kT)
           \right)
   \\&=& a \frac{2}{\sqrt{T}} 
           \cos\left(
              2\pi f_ct + 
              \pi \sum_{k=-\infty}^{n-1} y_k h  + 
              2\pi y_n h 
              \left( \frac{1}{2T}(t-nT)  \right)
           \right)
   \\&=& a \frac{2}{\sqrt{T}} 
           \cos\brp{
             \mcom{2\pi \left(f_c +  \frac{h}{2T}y_n \right)t}{carrier} +
             \mcom{\pi h\sum_{k=-\infty}^{n-1} y_k }{state}  -
             \mcom{\pi hn y_n }{maintains c.p.}
             }
\end{align*}

Two sinusoidal waveforms are \hie{coherent} if their frequency difference is 
$k\frac{1}{2T}$. 
The waveforms of CPFSK are therefore orthogonal if $h=m\frac{1}{2}$.
%---------------------------------------
\begin{definition}
\index{Orthogonal Continuous Phase Frequency Shift Keying}
%---------------------------------------
\opd{Orthogonal Continuous Phase Frequency Shift Keying}
is full response CPM ($L=1$) with $h_n\in\set{m\frac{1}{2}}{m\in\Z}$
and LREC phase pulse.
\end{definition}

For $m\in\Zp$, orthogonal CPFSK signalling waveforms are
\begin{align*}
   \f\fs(t;\ldots,u_{n-1},u_n)
     &= a \frac{2}{\sqrt{T}} 
          \cos\left(
             2\pi f_ct + 
             \pi \sum_{k=-\infty}^{n-L} y_k h_k  + 
             2\pi \sum_{k=n-L+1}^n y_k h_k \rho(t-kT)
          \right)
   \\&= a \frac{2}{\sqrt{T}} 
          \cos\left(
             2\pi \left(f_c +  \frac{h}{2T}y_n \right)t 
             + \pi \sum_{k=-\infty}^{n-1} y_k h  
             -\pi hn y_n  
          \right)
   \\&= a \frac{2}{\sqrt{T}} 
          \cos\brp{
            \mcom{2\pi \left(f_c +  \frac{m}{4T}y_n \right)t}{carrier} +
            \mcom{\frac{m}{2}\pi \sum_{k=-\infty}^{n-1} y_k }{state}  -
            \mcom{\frac{m}{2}\pi n y_n }{maintains c.p.}
            }
\end{align*}

The minimum value of $m$ in orthogonal CPFSK is $1$. 
When $m=1$ (the minimum value for orthogonality), 
the orthogonal CPFSK is also called {\em Minimum Shift Keying}.
%---------------------------------------
\begin{definition}
\index{Minimum Shift Keying}
\index{MSK}
%---------------------------------------
\opd{Minimum Phase Shift Keying} (MSK) is 
is full response CPM ($L=1$) with $h_n=\frac{1}{2}$
and LREC phase pulse.
\end{definition}

In MSK, the signalling waveform is
\begin{eqnarray*}
   \f\fs(t;\ldots,u_{n-1},u_n)
     &=& a \frac{2}{\sqrt{T}} 
           \cos\left( 
              2\pi f_ct + 
              \frac{\pi}{2} \left( \sum_{k=-\infty}^{n-1} y_k  +
              \frac{t-nT}{T}\cdot y_n \right)
           \right).
   \\&=& a \frac{2}{\sqrt{T}} 
           \cos\left(
              2\pi \left(f_c +  \frac{m}{4T}y_n \right)t 
              + \frac{m}{2}\pi \sum_{k=-\infty}^{n-1} y_k   
              -\frac{m}{\pi} n y_n  
           \right)
   \\&=& a \frac{2}{\sqrt{T}} 
           \cos\brp{
              \mcom{2\pi \left(f_c +  \frac{1}{4T}y_n \right)t}{carrier} +
              \mcom{\frac{\pi}{2} \sum_{k=-\infty}^{n-1} y_k }{state}  -
              \mcom{\frac{\pi}{2} n y_n }{maintains c.p.}
           }
\end{eqnarray*}


In summary:

\begin{tabular}{l|lll}
   Technique & $\rho(t)$ & $L$ & $h_k$ \\
\hline
   Continuous Phase Frequency Shift Keying (CPFSK) & LREC & $1$  & $h$ (constant) \\
   Minimum Shift Keying (MSK)                      & LREC & $1$  & $\frac{1}{2}$ 
\end{tabular}

%---------------------------------------
\subsection{Detection}
\index{trellis}
%---------------------------------------
The state of the signalling waveforms at intervals $nT$ can be
described by trellis diagrams.

\begin{figure}[ht]\color{figcolor}
\begin{center}
\begin{fsL}
\setlength{\unitlength}{0.15mm}
\begin{picture}(500,320)(-100,0)
  %\graphpaper[10](0,0)(300,300)
  \put( -90 , 300 ){$\phi_3=\frac{3\pi}{2}$}
  \put( -90 , 200 ){$\phi_2=\pi$}
  \put( -90 , 100 ){$\phi_1=\frac{\pi}{2}$}
  \put( -90 ,   0 ){$\phi_0=0$}

  \thinlines
  \multiput(0,0)(100,0){5}{
     \put(   0 ,   0 ){\circle*{10}}
     \put(   0 , 100 ){\circle*{10}}
     \put(   0 , 200 ){\circle*{10}}
     \put(   0 , 300 ){\circle*{10}}
  }

  \qbezier[40](  0,  0)( 50,150)(100,300)
  \qbezier[40](100,300)(250,150)(400,  0)
  \qbezier[20](100,100)(150, 50)(200,  0)
  \qbezier[40](200,  0)(250,150)(300,300)
  \qbezier[40](300,300)(350,250)(400,200)

  \put(  0,  0){\line( 1, 1){300}}
  \put(100,300){\line( 1,-3){100}}
  \put(200,  0){\line( 1, 1){200}}
  \put(300,300){\line( 1,-3){100}}
\end{picture}                                   
\end{fsL}
\hspace{1cm}
\begin{tabular}{cl}
   $\cdots$ & $x_n=0$ \\
   ---      & $x_n=1$ 
\end{tabular}
\caption{
  CPM $M=2$, $h=1/2$ (MSK-2) trellis diagram
   \label{fig:MSK-2_trellis}
   }
\end{center}
\end{figure}

\begin{figure}[ht]\color{figcolor}
\begin{center}
\begin{fsL}
\setlength{\unitlength}{0.15mm}
\begin{picture}(500,320)(-100,0)
  %\graphpaper[10](0,0)(300,300)
  \put( -90 , 200 ){$\phi_2=\frac{4}{3}\pi$}
  \put( -90 , 100 ){$\phi_1=\frac{2}{3}\pi$}
  \put( -90 ,   0 ){$\phi_0=0$}

  \thinlines
  \multiput(0,0)(100,0){5}{
     \put(   0 ,   0 ){\circle*{10}}
     \put(   0 , 100 ){\circle*{10}}
     \put(   0 , 200 ){\circle*{10}}
  }

  \qbezier[40](  0,  0)( 50,100)(100,200)
  \qbezier[40](100,200)(200,100)(300,  0)
  \qbezier[20](100,100)(150, 50)(200,  0)
  \qbezier[40](200,  0)(250,100)(300,200)
  \qbezier[40](300,  0)(350,100)(400,200)
  \qbezier[40](200,200)(300,100)(400,  0)
  \qbezier[40](300,200)(350,150)(400,100)

  \put(  0,  0){\line( 1, 1){200}}
  \put(200,  0){\line( 1, 1){200}}
  \put(200,100){\line( 1, 1){100}}
  \put(300,  0){\line( 1, 1){100}}
  \put(100,200){\line( 1,-2){100}}
  \put(200,200){\line( 1,-2){100}}
  \put(300,200){\line( 1,-2){100}}
\end{picture}                                   
\end{fsL}
\hspace{1cm}
\begin{tabular}{cl}
   $\cdots$ & $x_n=0$ \\
   ---      & $x_n=1$ 
\end{tabular}
\caption{
  CPM $M=2$, $h=2/3$ trellis diagram
   \label{fig:CPM_M2_h23_trellis}
   }
\end{center}
\end{figure}
