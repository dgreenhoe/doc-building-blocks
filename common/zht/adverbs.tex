%============================================================================
% Daniel J. Greenhoe
% XeLaTeX file
% UTF-8 encodiing
%============================================================================
%=======================================
\chapter{Adverbs}
%=======================================
%=======================================
\section{Degree of Precision}
%=======================================
\begin{tblvv}{Adverbs}{voc:adv}
  \tblx almost & \zhtsd{幾}{乎}{j/-i}{h/-u}{some}{at}{almost}
  \tblc almost & \zhtss{快}{ku/`ai}{almost}
\end{tblvv}

%\begin{minipage}{6\tw/8}%
%  English speakers often say that something ``\emph{almost}" happened.
%  One way to express this in Chinese is with the character \zht{快},
%  as illustrated to the left and with an example in \pref{sen:almost} (next).
%\end{minipage}

%---------------------------------------
\begin{example}[\exmd{Sentences using \emph{almost}}]
\label{ex:almost}
\mbox{}\\
%---------------------------------------
\begin{enumerate}
  \item \zhtsd{屋}{子}{w/-u}{zi}{house}{son}{house}%
        \zhtsd{裡}{面}{l\v{i}}{mi/`an}{inside}{face}{inside}%
        \zhtss{好}{h/>ao}{good}\zhtss{暗}{a\`n}{dark}\zhtsC
        \zhtss{小}{xi/>ao}{little}\zhtsd{章}{魚}{zh/-ang}{y\'u}{stamp}{fish}{octopus}%
        \zhtsde{幾}{乎}{j/-i}{h/-u}{}{}{almost}%
        \zhtss{看}{k/`an}{to look at}\zhtss{不}{b\'u}{not}\zhtss{見}{ji/`an}{to see}%
        \zhtsd{東}{西}{d/-ong}{\=xi}{east}{west}{things}%
        \zhtsP%
        \citetbl{\citerp{lovemind_octopus}{10}}
        \engbox{The house was very dark inside, and Little Octopus almost couldn't see anything.}

  \item %差一點就沒命了﹗
        \zhtss{差}{ch/`a}{difference}
        \zhtsd{一}{點}{y/`i}{di/>an}{one}{drop}{a little}
        \zhtss{就}{ji/`u}{just}
        \zhtsd{沒}{命}{m/'ei}{m/`ing}{no}{life}{lose life}
        \zhtss{了}{le}{(a change of state)}
        \zhtsX
        \citetbl{\citerp{zhang2006catbell}{4}}
        \engbox{[I] almost lost my life!}

  \item \zhtss{哇}{wa}{wow}\zhtsX
        \zhtss{好}{h/>ao}{very}%
        \zhtss{胖}{p/`ang}{fat}\zhtss{呀}{a}{}\zhtsX
        \zhtss{胖}{p/`ang}{fat}%
        \zhtss{到}{d/`ao}{to reach}%
        \zhtss{都}{d/-ou}{all}%
        \zhtss{快}{ku/`ai}{almost}%
        \zhtss{擠}{j\v{i}}{almost}%
        \zhtss{不}{b/`u}{not}%
        \zhtsd{進}{來}{j/`in}{l/'ai}{to enter}{to come}{to come in}%
        \zhtsX%
        \citetbl{\citerp{strange_visitor}{9}}
        \engbox{Wow! So Fat! So fat\\that you almost can't\\squeeze through\\the door!}
\end{enumerate}
\end{example}



%=======================================
\section{Adverbial phrases}
%=======================================
\ftbox{\zhtss{的}{de}{'s (possesses)}}%
\hfill%
\begin{minipage}{\tw-30mm}%
  The character 的 (de) often indicates possession,
  as in ``\zht{我的書}" (w/>o de sh/-u) --- ``my book".
  In this case it is used as a mechanism to introduce an adjective
  (which book? \ldots the book that is mine).
\end{minipage}
\\
The character \zht{的} can also be used to introduce an adverbial phrase---that 
is a phrase that helps modify or describe some other verb or adjective in
the sentence.
In this case the verb is the one ``possessed" by the adverb---and thus the adverb describes the verb.
An adverbial phrase introduced by \zht{的} comes {\em before} the verb it is modifying.
Examples follow in \prefpp{ex:adv_phrases_de}.

%---------------------------------------
\begin{example}[\exmd{Adverbial phrases using \zht{的}}]
\mbox{}\\
\label{ex:adv_phrases_de}
%---------------------------------------
\begin{enumerate}
  \item \ldots
        \zhtss{雞}{j/-i}{chicken}
        \zhtsd{媽}{媽}{m/-a}{m/-a}{mother}{mother}{mother}
        \zhtsd{焦}{急}{ji/-ao}{j/'i}{scorched}{urgent}{anxious}
        \zhtss{的}{de}{-ly}
        \zhtss{問}{w\`en}{to ask}\zhtsP
        \citetbl{\citerp{baobao_gorilla}{5}}
        \engbox{\ldots Mother Hen\\anxiously asked.}

  \item \zhtsd{烏}{鴉}{w/-u}{y/-a}{black}{crow}{crow}
        \zhtssp{把}{b/>a}{}
        \zhtss{水}{sh/>ui}{water}
        \zhtsd{貪}{婪}{t/-an}{l/'an}{greedy}{greedy}{greedy}
        \zhtss{的}{de}{-ly}
        \zhtsd{喝}{了}{h/-e}{le}{drink}{}{drank}
        \zhtsd{起}{來}{q\v{i}}{l/'ai}{to rise}{to come to}{to rise}\zhtsP
        \citetbl{\citerp{10crows}{60}}
        \engbox{The crow greedily\\drank up the water.}

  \item \zhtsd{黃}{牛}{hu/'ang}{ni/'u}{yellow}{cow}{cow}%
        \zhtss{在}{z/`ai}{at}%
        \zhtsd{池}{塘}{ch/'i}{t/'ang}{pond}{pond}{pond}\zhtss{裡}{l\v{i}}{inside}%
        \zhtsd{悠}{閑}{y/-ou}{xi/'an}{leisurely}{leisure}{leisurely}\zhtssp{的}{de}{'s}%
        \zhtss{游}{y/'ou}{to swim}\zhtss{來}{l/'ai}{to come}%
        \zhtss{游}{y/'ou}{to swim}\zhtss{去}{q/`u}{to go}\zhtsP
        \citetbl{\citerp{baobao_cow}{17}}
        \\\engbox{The cow leisurely swam back and forth across the pond.}
\end{enumerate}
\end{example}

\ftbox{\zhtss{得}{de}{(to the extent that)}}
\quad\begin{minipage}{\tw-45mm}%
  One meaning of the character \zht{得} (d/'e) is ``\hie{to the extent that}",
  and is often used to begin an adverbial phrase.
  An adverbial phrase introduced by \zht{得} comes {\em after} the verb it is modifying.
\end{minipage}\\
Note however that ``to the extent that" is not the only meaning of \zht{得};
it also means \hie{to get}.
When \zht{得} means ``to the extent that", it is pronounced with neutral tone (de);
when \zht{得} means ``to get", it is pronounced with rising tone (d/'e).
The \zht{得} character can be neatly used to express the following English sentence structures:
  \\\indentx\begin{tabular}{cl}
    \imark & He was \blank \emph{to the extent that} \blank . \\
    \imark & He was \emph{so} \blank \emph{that} he \blank . \\
    \imark & He was \blank, in fact, he was \emph{very} \blank .
  \end{tabular}\\
%Examples follow in \pref{ex:adv_phrases} (next).

%---------------------------------------
\begin{example}[\exmd{Adverbial phrases}]
\mbox{}\\
\label{ex:adv_phrases}
%---------------------------------------
\begin{enumerate}
  \item \zhtsd{國}{王}{g/'uo}{w/'ang}{country}{king}{king}%
        \zhtsd{高}{興}{g/-ao}{x/`ing}{high}{cheerful}{glad}%
        \zhtssp{得}{de}{(to the extent of)}%
        \tcom{\zhtss{不}{b/`u}{not}\zhtss{得}{d/'e}{to get}\zhtss{了}{li/>ao}{to understand}}
             {beyond understanding}\zhtsP%
        \citetbl{\citerp{cape}{17}}
        \engbox{The king was so happy that\\he was beside himself with joy.}

  \item \zhtsd{冬}{天}{d/-ong}{ti/-an}{winter}{sky}{winter}%
        \zhtss{還}{h/'ai}{still}%
        \zhtss{早}{z/>ao}{early}%
        \zhtssp{得}{de}{to the extent that}%
        \zhtss{很}{h/>en}{early}\zhtsP%
        \citetbl{\citerp{baobao_cow}{14}}%
        \engbox{It is still early for\\winter---very early.}

  \item \zhtsd{小}{鴨}{xi/>ao}{y/-a}{little}{duck}{duckling}%
        \zhtss{凍}{d\`ong}{to freeze}%
        \zhtssp{得}{de}{(to the extent that)}%
        \zhtsd{渾}{身}{h/'un}{sh/-en}{entire}{body}{entire body}%
        \zhtsd{發}{抖}{f/-a}{d/>ou}{to send out}{to tremble}{to shiver}\zhtsP%
        \citetbl{\citerp{ugly_duckling}{20}}%
        \engbox{The duckling was so freezing\\cold that his whole body shook.}

  \item \zhtss{狼}{l/'ang}{wolf}%
        \zhtss{嚇}{xi/`a}{to scare}%
        \zhtssp{得}{de}{(to the extent that)}%
        \zhtss{全}{q/'uan}{entire}\zhtss{身}{sh/-en}{body}%
        \zhtsd{發}{抖}{f/-a}{d/>ou}{to send out}{to tremble}{to shake}\zhtsP%
        \citetbl{\citerp{baobao_gorilla}{11}}%
        \engbox{The wolf was so scared\\that his whole body shook.}

  \item \zhtss{象}{xi/`ang}{elephant}%
        \zhtss{嚇}{xi/`a}{to scare}%
        \zhtssp{得}{de}{to the extent}%
        \zhtsd{立}{刻}{l\`i}{k\`e}{to stand}{to carve}{immediately}%
        \zhtss{拔}{b/'a}{to pull out}\zhtss{腿}{t/>ui}{leg}%
        \zhtss{就}{ji/`u}{just}\zhtss{跑}{p/>ao}{run}\zhtsP%
        \citetbl{\citerp{baobao_elephant}{23}}%
        \engbox{The elephant was so scared\\that he immediately picked\\up his feet and ran.}

  \item \zhtss{我}{w/>o}{I}%
        \zhtsd{實}{在}{sh/'i}{z/`ai}{real}{to exist}{really}%
        \zhtss{渴}{k/>e}{thirsty}%
        \zhtssp{得}{de}{to the extent that}%
        \tcom{\zhtss{受}{sh\`ou}{to receive}\zhtss{不}{b/`u}{not}\zhtss{了}{li/>ao}{understand}}
             {can't stand it}\zhtsP%
        \citetbl{\citerp{10crows}{59}}%
        \engbox{I am really so thirsty\\that I can't stand it.}

  \item \zhtss{牛}{ni/'u}{cow}\zhtss{的}{de}{'s}%
        \zhtsd{肚}{子}{d/`u}{zi}{stomach}{son}{stomach}%
        \zhtss{餓}{\`e}{hungry}%
        \zhtssp{得}{de}{to the extent of}%
        \zhtsd{咕}{嚕}{g/-u}{l/-u}{(a sound)}{(a sound)}{``gurgle"}%
        \zhtsd{咕}{嚕}{g/-u}{l/-u}{(a sound)}{(a sound)}{``gurgle"}%
        \zhtss{叫}{ji/`ao}{to call}\zhtsP%
        \citetbl{\citerp{baobao_cow}{22}}%
        \\\engbox{The cow's stomach was so hungry that with a ``gurgle gurgle" it called out.}

  \item \zhtsd{萊}{特}{L/'ai}{t\`e}{wild weeds}{special}{Wright}%
        \zhtsd{兄}{弟}{xi/-ong}{d\`i}{elder brother}{younger brother}{brothers}%
        \zhtsd{曾}{經}{c/'eng}{j/-ing}{once}{after}{once}%
        \zhtss{摔}{sh/-uai}{to fall}%
        \zhtssp{得}{de}{to the extent that}%
        \zhtss{鼻}{b/'i}{nose}%
        \zhtss{青}{q/-ing}{blue}%
        \zhtss{臉}{li/>an}{face}%
        \zhtss{腫}{zh/>ong}{swollen}\zhtsP%
        \citetbl{\citerp{wright2038}{5}}%
        \\\engbox{Once the Wright brothers took such a tumble that their faces were bruised and swollen.}
\end{enumerate}
\end{example}

\ftbox{\zhtss{個}{ge}{one that}}%
\hfill%
\begin{minipage}{\tw-48mm}%
  Another way to introduce an adverbial phrase
  is with the use of the character \zht{個} (ge).
  \zht{個} is more commonly used as a general purpose measure word
  (e.g. \zht{兩個蘋果}---two (ge) apples).
  In the case of introducing an adverbial phrase,
  it roughly means ``\hie{one that}".
  An adverbial phrase introduced by \zht{個} comes {\em after} the verb it is modifying.
  Often \zht{個} introduces the adverb \zht{不停}.
  %Examples follow in \prefpp{ex:adv_ge}.
\end{minipage}\hfill%
\ftbox{\zhtsd{不}{停}{b/`u}{t/'ing}{not}{stop}{nonstop}}

%---------------------------------------
\begin{example}[\exmd{Adverb introduced using \zht{個}}]
\mbox{}\\
\label{ex:adv_ge}
%---------------------------------------
\begin{enumerate}
  \item %這幾天大雨下個不停。
        \zhtss{這}{zh/`e}{these}%
        \zhtsd{幾}{天}{j/>i}{ti/-an}{several}{sky}{several days}%
        \zhtsd{大}{雨}{d/`a}{y/>u}{big}{rain}{heavy rain}%
        \zhtss{下}{xi/`a}{down}%
        \zhtss{個}{ge}{one that}
        \zhtsd{不}{停}{b/`u}{t/'ing}{not}{stop}{nonstop}%
        \zhtsP%
        \citetbl{\citerp{firedragon}{8}}
        \engbox{For several days now,\\heavy rain has been\\falling nonstop.}

  \item \zhtss{狼}{l/'ang}{wolf}\zhtsd{叔}{叔}{sh/'u}{shu}{uncle}{uncle}{uncle}%
        \zhtsd{還}{是}{h/'ai}{sh\`i}{still}{is}{still}%
        \zhtss{抖}{d/>ou}{to tremble}%
        \zhtss{個}{ge}{one that}%
        \zhtsd{不}{停}{b/`u}{t/'ing}{not}{stop}{nonstop}%
        \zhtsP
        \citetbl{\citerp{baobao_gorilla}{11}}
        \engbox{Uncle Wolf was still\\trembling nonstop.}

  \item \zhtss{小}{xi/>ao}{little}\zhtsd{山}{羊}{sh/-an}{y/'ang}{mountain}{sheep}{goat}%
        \zhtsd{不}{禁}{b/`u}{j/-in}{not}{to endure}{cannot refrain}%
        \zhtss{呵}{h/-e}{Ha}\zhtss{呵}{h/-e}{Ha}\zhtss{呵}{h/-e}{Ha}%
        \zhtss{的}{de}{'s}\zhtss{笑}{xi/`ao}{to laugh}\zhtss{個}{ge}{one that}%
        \zhtsd{不}{停}{b/`u}{t/'ing}{not}{stop}{nonstop}%
        \zhtsP%
        \citetbl{\citerp{baobao_elephant}{19}}
        \\\engbox{The little goat just couldn't keep himself from laughing ``Ha-Ha-Ha" nonstop.}
\end{enumerate}
\end{example}

\ftbox{\zhtss{連}{li/'an}{even} \ldots 
       \zhtss{都}{d/-ou}{all}
       \zhtsd{沒}{有}{m/'ei}{y/>ou}{not}{have}{not have}}%
\hfill%
\begin{minipage}{\tw-53mm}%
  The pattern \zht{連}\ldots\zht{都沒有} (li/'an \ldots d/-ou m/'ei y/>ou)
  ``even \ldots all not have"
  is often used in Chinese.
  \pref{ex:lian_dou} (next) illustrates this pattern.
\end{minipage}

%---------------------------------------
\begin{example}[\exmd{Adverbial phrase using \zht{連}\ldots\zht{都沒有}}]
\mbox{}\\
\label{ex:lian_dou}
%---------------------------------------
\exboxt{%
  \zhtss{狼}{l/'ang}{wolf}%
  \zhtsd{叔}{叔}{sh/'u}{shu}{uncle}{uncle}{uncle}%
  \zhtss{連}{li/'an}{even}%
  \zhtss{黑}{h/-ei}{black}%
  \zhtsd{猩}{猩}{x/-ing}{x/-ing}{orangutan}{orangutan}{gorilla}%
  \zhtss{的}{de}{'s}%
  \zhtsd{影}{子}{y/>ing}{zi}{shadow}{son}{shadow}%
  \zhtss{都}{d/-ou}{all}%
  \zhtss{沒}{m/'ei}{not}%
  \zhtsd{看}{到}{k/`an}{d/`ao}{see}{to reach}{see}%
  \zhtsP\footnotemark%
  \\\engbox{Uncle Wolf didn't even see so much as the Black Gorilla's shadow.}
  }
  \citetblt{\citerp{baobao_gorilla}{9}}
\end{example}


\ftbox{\zhtsd{拼}{命}{p/-in}{m/`ing}{go all out}{life}{desperately}}%
\hfill%
\begin{minipage}{\tw-40mm}
  In English, when wanting to emphasize the desperation of a situation,
  one may add ``\ldots for your life!" after the verb.
  In Chinese, the ``for your life" is \zht{拼命}
  and may be used with most any verb to express
  the use of extreme exertion.
  \pref{ex:forlife} (next) illustrates this structure.
\end{minipage}

%---------------------------------------
\begin{example}[\exmd{\blank for your life!}]
\mbox{}\\
\label{ex:forlife}
%---------------------------------------
\begin{enumerate}
  \item 
    \zhtsd{拼}{命}{p/-in}{m/`ing}{go all out}{life}{desperately}%
    \zhtss{跑}{p/>ao}{run}\zhtsX
    \engbox{Run for your life!}
  %\tblh Climb for your life!
  %  & \zhtsd{拼}{命}{p/-in}{m/`ing}{go all out}{life}{desperately}%
  %    \zhtss{爬}{p/'a}{climb}\zhtsX
  %\tblh Scrub for your life!
  %  & \zhtsd{拼}{命}{p/-in}{m/`ing}{go all out}{life}{desperately}%
  %    \zhtss{刷}{sh/-ua}{scrub}\zhtsX
  %\tblh Study for your life!
  %  & \zhtsd{拼}{命}{p/-in}{m/`ing}{go all out}{life}{desperately}%
  %    \zhtss{學}{x/'ue}{study}\zhtsX
  \item 
    \zhtsd{烏}{鴉}{w/-u}{y/-a}{black}{crow}{crow}%
    \zhtsd{拼}{命}{p/-in}{m/`ing}{go all out}{life}{desperately}%
    \zhtss{的}{de}{'s}%
    \zhtss{喝}{h/-e}{drank}\zhtsP%
    \citetbl{\citerp{10crows}{61}}
    \engbox{The crow drank for his life.}

  \item 
    \zhtss{小}{xi/>ao}{little} \zhtss{白}{b/'ai}{white} \zhtss{兔}{t/`u}{rabbit}%
    \zhtsd{拼}{命}{p/-in}{m/`ing}{go all out}{life}{desperately}%
    \zhtss{洗}{x/>i}{wash}\zhtsP%
    %\zhtsd{可}{是}{k/>e}{sh\`i}{to approve}{is}{however}%
    %\zhtsd{怎}{麼}{z/>en}{me}{who}{}{how}%
    %\zhtss{洗}{x\v{i}}{wash},%
    %\zhtss{也}{y/>e}{also}%
    %\zhtss{洗}{x\v{i}}{wash},%
    %\zhtss{不}{b/`u}{not}%
    %\zhtss{白}{b/'ai}{white}\zhtsP%
    \citetbl{\citerp{cloud_pillow}{5}}
    \engbox{Little White Rabbit\\washed for his life.} %; but no matter how much he washed, he didn't wash white.
\end{enumerate}
\end{example}

\begin{minipage}{\tw-106mm}
In English there is a sentence pattern
``No matter how he (verb), he still cannot (verb) enough."
A similar mechanism in Chinese is illustrated to the right and
in \pref{ex:no_matter} (next).
\end{minipage}
\hfill\ftbox{%
\zhtsd{可}{是}{k/>e}{sh/`i}{to approve}{that}{but}%
\zhtsd{怎}{麼}{z/>e}{me}{who}{(?)}{how}%
\charsBlank{(verb)}%
\zhtss{也}{y/>e}{also}%
%\charsBlank{(same verb)}%
%\zhtss{不}{b/`u}{not}%
\charsBlank{(failed action)}%
\zhtsP%
}

%---------------------------------------
\begin{example}[\exmd{No matter how}]
\mbox{}\\
\label{ex:no_matter}
%---------------------------------------
\begin{enumerate}
  \item \zhtsd{可}{是}{k/>e}{sh/`i}{to approve}{that}{but}%
        \zhtsd{怎}{麼}{z/>e}{me}{who}{}{how}%
        \zhtss{洗}{x/>i}{wash}%
        \zhtss{也}{y/>e}{also}%
        \zhtss{洗}{x/>i}{wash}\zhtss{不}{b/`u}{not}%
        \zhtsd{乾}{淨}{g/-an}{j/`ing}{dry}{clean}{clean}\zhtsP%
        \citetbl{\citerp{cloud_pillow}{5}}%
        \engbox{But no matter how much\\he washed, he couldn't\\wash himself clean.}

  \item \zhtsd{可}{是}{k/>e}{sh/`i}{to approve}{that}{but}%
        \zhtsd{公}{雞}{g/-ong}{j/-i}{male}{chicken}{rooster}%
        \zhtss{再}{z/`ai}{further}%
        \zhtsd{怎}{麼}{z/>e}{me}{who}{(?)}{how}%
        \zhtsd{賣}{力}{m/`ai}{l\`i}{to sell}{force}{to exert}\zhtsC
        \zhtss{也}{y/>e}{also}%
        \zhtsd{沒}{有}{m/'ei}{y/>ou}{no}{have}{not have}%
        \zhtsd{鬧}{鐘}{n/`ao}{zh/-ong}{noisy}{clock}{alarm clock}%
        \zhtss{叫}{ji/`ao}{to crow}\zhtssp{得}{de}{(to the extent that)}%
        \zhtss{久}{ji/>u}{duration}\zhtsC
        \zhtss{叫}{ji/`ao}{to crow}\zhtssp{得}{de}{(to the extent that)}%
        \zhtsd{宏}{亮}{h/'ong}{li/`ang}{magnificent}{loud and clear}{magnificent and clear}%
        \zhtsP%
        \citetbl{\citerp{cd_roosterking}{10}}%
        \engbox{But no matter how much the rooster\\
                 further exerted himself, he didn't\\ 
                 have the alarm clock's crowing duration\\ 
                 or brilliant and magnificent sound.}
\end{enumerate}
\end{example}

