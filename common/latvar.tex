%============================================================================
% Daniel J. Greenhoe
% LaTeX file
%============================================================================


%%=======================================
%\chapter{Lattice Varieties}
%\label{chp:latvar}
%%=======================================
%
%\begin{figure}[th]
%  \begin{center}
%    \psset{yunit=12mm}%============================================================================
% Daniel J. Greenhoe
% LaTeX file
% lattice ({factors of 30}, |)
% nominal unit = 15mm
%============================================================================
  \begin{pspicture}(-6,-0.3)(6,5.5)%
     \footnotesize
     \psset{%
       cornersize=relative,
       framearc=0.25,
       subgriddiv=1,
       gridlabels=4pt,
       gridwidth=0.2pt,
       }%
     \begin{tabstr}{0.75}
     \rput(3,5){\rnode{bounded}  {\psframebox{\begin{tabular}{c}\prop{bounded}\ifnxref{lattice}{def:latb}\end{tabular}}}}%
     %
     \rput(-3,3){\rnode{modular}  {\psframebox{\begin{tabular}{c}\prop{modular}\ifnxref{latm}{def:latm}\end{tabular}}}}%
     \rput(-3,2){\rnode{distributive}{\psframebox{\begin{tabular}{c}\prop{distributive}\ifnxref{latd}{def:latd}\end{tabular}}}}%
     %
     \rput(3,4){\rnode{complemented}  {\psframebox{\begin{tabular}{c}\prop{complemented}\ifnxref{latc}{def:latc}\end{tabular}}}}%
     \rput(3,3){\rnode{ortholat}  {\psframebox{\begin{tabular}{c}\prop{orthocomplemented}\ifnxref{ortholat}{def:latoc}\end{tabular}}}}%
     \rput(3,2){\rnode{orthomod}  {\psframebox{\begin{tabular}{c}\prop{orthomodular}\ifnxref{ortholat}{def:latoc_omod}\end{tabular}}}}%
     \rput(3,1){\rnode{modortho}  {\psframebox{\begin{tabular}{c}\prop{modular orthocomplemented}\ifnxref{ortholat}{def:latmoc}\end{tabular}}}}%
     %
     \rput(0,0){\rnode{boolean}   {\psframebox{\begin{tabular}{c}\prop{boolean}\ifnxref{boolean}{def:boolean}\end{tabular}}}}%
     \end{tabstr}
     %
     \psset{doubleline=true}%
     %\ncline{<-}{bounded}{modular}%
     \ncline{<-}{bounded}{complemented}%
     \ncline{<-}{complemented}{ortholat}%
     \ncline{<-}{modular}{distributive}%
     \ncline{<-}{distributive}{boolean}%
     \ncline{<-}{ortholat}{orthomod}%
     \ncline{<-}{modular}{modortho}%
     \ncline{<-}{orthomod}{modortho}%
     \ncline{<-}{modortho}{boolean}%
     %
     %\psgrid[unit=10mm](-8,-1)(8,9)%
  \end{pspicture}%
%  \end{center}
%  \caption{lattice of orthocomplemented lattices\label{fig:latvar_latortholat}}
%\end{figure}

%%=======================================
%\section{Atomic lattices}
%%=======================================
%%---------------------------------------
%\begin{definition}
%\citetbl{
%  \citer{larson1975}\\
%  \citorpc{birkhoff1938}{800}{see footnote \ddag}
%  }
%%---------------------------------------
%Let $\latL\eqd\latticed$ be a lattice
%with least upper bound $1$ $\brp{\joinop\setX=1}$
%and  greatest lower bound $0$ $\brp{\meetop\setX=0}$.
%\defboxt{
%  \begin{tabular}{>{$\imark$\hspace{1ex}}l}
%    $x$ is an  \hid{atom}      of $\latL$ if $x$ \prope{covers} \xref{def:cover} 0.\\
%    $x$ is an  \hid{anti-atom} of $\latL$ if $x$ is \prope{covered by} 1.\\
%    $\latL$ is \hid{atomic} if every $x\in\setX\setd 1$      can be represented as joins of atoms of $\latL$. \\
%    $\latL$ is \hid{anti-atomic} if every $x\in\setX\setd 0$ can be represented as meets of anti-atoms of $\latL$.
%  \end{tabular}}
%\end{definition}
%
%%---------------------------------------
%\begin{example}
%\label{ex:lat_atomic}
%%---------------------------------------
%Here are some examples of lattices that are \prope{atomic}, \prope{anti-atomic},
%both, and neither\ldots\\
%\\
%{\psset{unit=\latunit}%
%\begin{tabular}{|cccccc|}%
%\hline%
%\mc{2}{|G|}{atomic lattices}&%
%\mc{2}{|G|}{anti-atomic}&%
%\mc{2}{|G|}{atomic and anti-atomic}%
%\\\hline%
%{%============================================================================
% Daniel J. Greenhoe
% LaTeX file
% nominal unit = 5mm
%============================================================================
\begin{pspicture}(-1.8,-\latbot)(1.8,3.3)
  %---------------------------------
  % nodes
  %---------------------------------
  \Cnode(0,3){t}%
  \Cnode(0,2){xy}%
  \Cnode(-1.5,1){w}\Cnode(-0.5,1){x}\Cnode(0.5,1){y}\Cnode(1.5,1){z}%
  \Cnode(0,0){b}%
  %---------------------------------
  % node connections
  %---------------------------------
  \ncline{t}{w}\ncline{t}{xy}\ncline{t}{z}%
  \ncline{xy}{x}\ncline{xy}{y}%
  \ncline{b}{w}\ncline{b}{x}\ncline{b}{y}\ncline{b}{z}%
  %---------------------------------
  % node labels
  %---------------------------------
  %\uput[ 90](t) {$\setn{x,y,z}$}%
  %\uput[-90](b) {$\szero$}%
\end{pspicture}%}&%
%{%============================================================================
% Daniel J. Greenhoe
% LaTeX file
% nominal unit = 5mm
%============================================================================
\begin{pspicture}(-1.8,-\latbot)(1.8,3.3)%
  %---------------------------------
  % nodes
  %---------------------------------
  \Cnode(0,3){t}%
  \Cnode(-0.5,2){xy}%
  \Cnode(-1.5,1){w}\Cnode(-0.5,1){x}\Cnode(0.5,1){y}\Cnode(1.5,1){z}%
  \Cnode(0,0){b}%
  %---------------------------------
  % node connections
  %---------------------------------
  \ncline{t}{xy}\ncline{t}{z}%
  \ncline{xy}{w}\ncline{xy}{x}\ncline{xy}{y}%
  \ncline{b}{w}\ncline{b}{x}\ncline{b}{y}\ncline{b}{z}%
  %---------------------------------
  % node labels
  %---------------------------------
  %\uput[ 90](t) {$\setn{x,y,z}$}%
  %\uput[180](xy){$\setn{x,y}$}%   
  %\uput[0](yz){$\setn{y,z}$}%
  %\uput[180](x) {$\setn{x}$}%     
  %\uput[0](z) {$\setn{z}$}%
  %\uput[-90](b) {$\szero$}%
  %\uput[0](100,300){\rnode{xzlabel}{$\setn{x,z}$}}% 
  %\uput[0](100,  0){\rnode{ylabel}{$\setn{y}$}}%
  %\ncline[linestyle=dotted,nodesep=1pt]{->}{xzlabel}{xz}%
  %\ncline[linestyle=dotted,nodesep=1pt]{->}{ylabel}{y}%
\end{pspicture}%}\vline&%
%{%============================================================================
% Daniel J. Greenhoe
% LaTeX file
% nominal unit = 5mm
%============================================================================
\begin{pspicture}(-1.8,-\latbot)(1.8,3.3)
  %---------------------------------
  % nodes
  %---------------------------------
  \Cnode(0,3){t}
  \Cnode(-1.5,2){w}\Cnode(-0.5,2){x}\Cnode(0.5,2){y}\Cnode(1.5,2){z}%
  \Cnode(0,1){xy}%
  \Cnode(0,0){b}
  %---------------------------------
  % node connections
  %---------------------------------
  \ncline{t}{w}\ncline{t}{x}\ncline{t}{y}\ncline{t}{z}%
  \ncline{xy}{x}\ncline{xy}{y}%
  \ncline{b}{w}\ncline{b}{xy}\ncline{b}{z}
  %---------------------------------
  % node labels
  %---------------------------------
  %\uput[ 90](t) {$\setn{x,y,z}$}%
  %\uput[180](xy){$\setn{x,y}$}%   
  %\uput[0](yz){$\setn{y,z}$}%
  %\uput[180](x) {$\setn{x}$}%     
  %\uput[0](z) {$\setn{z}$}%
  %\uput[-90](b) {$\szero$}%
  %\uput[0](100,300){\rnode{xzlabel}{$\setn{x,z}$}}% 
  %\uput[0](100,  0){\rnode{ylabel}{$\setn{y}$}}%
  %\ncline[linestyle=dotted,nodesep=1pt]{->}{xzlabel}{xz}%
  %\ncline[linestyle=dotted,nodesep=1pt]{->}{ylabel}{y}%
\end{pspicture}%}&%
%{%============================================================================
% Daniel J. Greenhoe
% LaTeX file
% nominal unit = 5mm
%============================================================================
\begin{pspicture}(-1.8,-\latbot)(1.8,3.3)
  %---------------------------------
  % settings
  %---------------------------------
  %---------------------------------
  % nodes
  %---------------------------------
  \Cnode(0,3){t}%
  \Cnode(-1.5,2){w}\Cnode(-0.5,2){x}\Cnode(0.5,2){y}\Cnode(1.5,2){z}%
  \Cnode(-0.5,1){xy}%
  \Cnode(0,0){b}%
  %---------------------------------
  % node connections
  %---------------------------------
  \ncline{t}{w}\ncline{t}{x}\ncline{t}{y}\ncline{t}{z}%
  \ncline{xy}{w}\ncline{xy}{x}\ncline{xy}{y}%
  \ncline{b}{xy}\ncline{b}{z}%
  %---------------------------------
  % node labels
  %---------------------------------
  %\uput[ 90](t) {$\setn{x,y,z}$}%
  %\uput[180](xy){$\setn{x,y}$}%   
  %\uput[0](yz){$\setn{y,z}$}%
  %\uput[180](x) {$\setn{x}$}%     
  %\uput[0](z) {$\setn{z}$}%
  %\uput[-90](b) {$\szero$}%
  %\uput[0](100,300){\rnode{xzlabel}{$\setn{x,z}$}}% 
  %\uput[0](100,  0){\rnode{ylabel}{$\setn{y}$}}%
  %\ncline[linestyle=dotted,nodesep=1pt]{->}{xzlabel}{xz}%
  %\ncline[linestyle=dotted,nodesep=1pt]{->}{ylabel}{y}%
\end{pspicture}%}\vline&%
%{\psset{yunit=1.5\latunit}%============================================================================
% Daniel J. Greenhoe
% LaTeX file
% lattice M5
% nominal unit = 5mm
%============================================================================
\begin{pspicture}(-2.4,-\latbot)(2.4,2.2)%
  %---------------------------------
  % nodes
  %---------------------------------
                                \Cnode(0,2){t}%
  \Cnode(-2,1){v}\Cnode(-1,1){w}\Cnode(0,1){x}\Cnode(1,1){y}\Cnode(2,1){z}%
                                \Cnode(0,0){b}%
  %---------------------------------
  % node connections
  %---------------------------------
  \ncline{t}{v}\ncline{t}{w}\ncline{t}{x}\ncline{t}{y}\ncline{t}{z}%
  \ncline{b}{v}\ncline{b}{w}\ncline{b}{x}\ncline{b}{y}\ncline{b}{z}%
  %---------------------------------
  % node labels
  %---------------------------------
  %\uput[ 90](t) {$1$}%
\end{pspicture}%}&%
%{%============================================================================
% Daniel J. Greenhoe
% LaTeX file
% recommended unit = 5mm
%============================================================================
\begin{pspicture}(-1.3,-\latbot)(1.3,3.3)%
  %---------------------------------
  % nodes
  %---------------------------------
  \Cnode(0,3){t}%
  \Cnode(-1,2){xy} \Cnode(0,2){xz} \Cnode(1,2){yz}%
  \Cnode(-1,1){x}  \Cnode(0,1){y}  \Cnode(1,1){z}%
  \Cnode(0,0){b}%
  %---------------------------------
  % node connections
  %---------------------------------
  \ncline{t}{xy}\ncline{t}{xz}\ncline{t}{yz}%
  \ncline{x}{xy}\ncline{x}{xz}%
  \ncline{y}{xy}\ncline{y}{yz}%
  \ncline{z}{xz}\ncline{z}{yz}%
  \ncline{b}{x} \ncline{b}{y} \ncline{b}{z}%
\end{pspicture}%}%
%\\
%\mc{6}{|G|}{neither atomic nor anti-atomic}%
%\\
%{\psset{yunit=0.75\latunit}%============================================================================
% Daniel J. Greenhoe
% LaTeX file
% lattice M2 on M2
% nominal unit = 10mm
%============================================================================
\begin{pspicture}(-1.4,-\latbot)(1.4,4.4)
  %---------------------------------
  % nodes
  %---------------------------------
           \Cnode(0,4){t}
  \Cnode(-1,3){c}\Cnode(1,3){d}%
           \Cnode(0,2){m}%
  \Cnode(-1,1){x}\Cnode(1,1){y}%
           \Cnode(0,0){b}
  %---------------------------------
  % node connections
  %---------------------------------
  \ncline{t}{c}\ncline{t}{d}%
  \ncline{m}{c}\ncline{m}{d}%
  \ncline{m}{x}\ncline{m}{y}%
  \ncline{b}{x}\ncline{b}{y}%
  %---------------------------------
  % node labels
  %---------------------------------
  %\uput[ 90](t) {$1$}%
\end{pspicture}%}&%
%{\psset{yunit=1.00\latunit}%============================================================================
% Daniel J. Greenhoe
% LaTeX file
% nominal unit = 5mm
%============================================================================
\begin{pspicture}(-1.8,-\latbot)(1.8,3.3)
  %---------------------------------
  % nodes
  %---------------------------------
  \Cnode(0,3){t}%
  \Cnode(-1.5,2){d}%
  \Cnode(-1.5,1){c}%
  \Cnode(-0.5,1.5){x}\Cnode(0.5,1.5){y}\Cnode(1.5,1.5){z}%
  \Cnode(0,0){b}%
  %---------------------------------
  % node connections
  %---------------------------------
  \ncline{t}{d}\ncline{t}{x}\ncline{t}{y}\ncline{t}{z}%
  \ncline{c}{d}%
  \ncline{b}{c}\ncline{b}{x}\ncline{b}{y}\ncline{b}{z}%
  %---------------------------------
  % node labels
  %---------------------------------
  %\uput[ 90](t) {$\setn{x,y,z}$}%
\end{pspicture}%}&%
%{\psset{yunit=0.75\latunit}%============================================================================
% Daniel J. Greenhoe
% LaTeX file
%============================================================================
\begin{pspicture}(-1.3,-\latbot)(1.3,4.4)
  %---------------------------------
  % nodes
  %---------------------------------
                 \Cnode(0,4){t}%
                 \Cnode(0,3){e}%
  \Cnode(-1,2){x}\Cnode(0,2){d}\Cnode(1,2){y}%
                 \Cnode(0,1){c}%
                 \Cnode(0,0){b}%
  %---------------------------------
  % node connections
  %---------------------------------
  \ncline{t}{x}\ncline{t}{y}%                            top    half of m2
  \ncline{b}{x}\ncline{b}{y}%                            bottom half of m2
  \ncline{t}{e}\ncline{e}{d}\ncline{d}{c}\ncline{c}{b}%  middle linear 5 element component
  %---------------------------------
  % node labels
  %---------------------------------
  %\uput[0](t) {$1$}%
\end{pspicture}%}&%
%{\psset{yunit=1.00\latunit}%============================================================================
% Daniel J. Greenhoe
% LaTeX file
% nominal unit = 10mm
%============================================================================
\begin{pspicture}(-1.3,-\latbot)(1.3,3.3)%
  %---------------------------------
  % nodes
  %---------------------------------
  \Cnode(0,3){t}
  \Cnode(-1,2){c}\Cnode(1,2){d}%
  \Cnode(0,1.5){m}%
  \Cnode(-1,1){x}\Cnode(1,1){y}%
  \Cnode(0,0){b}
  %---------------------------------
  % node connections
  %---------------------------------
  \ncline{t}{c}\ncline{t}{m}\ncline{t}{d}%
  \ncline{c}{x}\ncline{d}{y}%
  \ncline{b}{x}\ncline{b}{m}\ncline{b}{y}%
  %---------------------------------
  % node labels
  %---------------------------------
  %\uput[ 90](t) {$1$}%
\end{pspicture}%}&%
%{\psset{yunit=1.00\latunit}%============================================================================
% Daniel J. Greenhoe
% LaTeX file
% lattice L3 in O6
% nominal unit = 5mm
%============================================================================
\begin{pspicture}(-1.3,-\latbot)(1.3,3.3)%
  %---------------------------------
  % nodes
  %---------------------------------
  \Cnode(0,3){t}%
  \Cnode(-1,2){c}\Cnode(1,2){d}%
  \Cnode(0,1.5){m}%
  \Cnode(-1,1){x}\Cnode(1,1){y}%
  \Cnode(0,0){b}%
  %---------------------------------
  % node connections
  %---------------------------------
  \ncline{t}{c}\ncline{t}{d}\ncline{t}{m}%
  \ncline{x}{c}\ncline{y}{d}%
  \ncline{x}{m}%
  \ncline{b}{x}\ncline{b}{y}%
  %---------------------------------
  % node labels
  %---------------------------------
  %\uput[ 90](t) {$\setn{x,y,z}$}%
  %\uput[180](xy){$\setn{x,y}$}%   
  %\uput[0](yz){$\setn{y,z}$}%
  %\uput[180](x) {$\setn{x}$}%     
  %\uput[0](z) {$\setn{z}$}%
  %\uput[-90](b) {$\szero$}%
  %\uput[0](100,300){\rnode{xzlabel}{$\setn{x,z}$}}% 
  %\uput[0](100,  0){\rnode{ylabel}{$\setn{y}$}}%
  %\ncline[linestyle=dotted,nodesep=1pt]{->}{xzlabel}{xz}%
  %\ncline[linestyle=dotted,nodesep=1pt]{->}{ylabel}{y}%
\end{pspicture}%}&%
%{\psset{yunit=1.00\latunit}%============================================================================
% Daniel J. Greenhoe
% LaTeX file
% lattice M2 on M2
% nominal unit = 10mm
%============================================================================
{%
\begin{pspicture}(-0.5,-\latbot)(0.5,3.5)%
  %---------------------------------
  % nodes
  %---------------------------------
  \Cnode(0,3){t}%
  \Cnode(0,2){d}%
  \Cnode(0,1){c}%
  \Cnode(0,0){b}%
  %---------------------------------
  % node connections
  %---------------------------------
  \ncline{d}{t}%
  \ncline{c}{d}%
  \ncline{b}{c}%
\end{pspicture}
}%}%
%\\\hline%
%\end{tabular}
%}
%\end{example}

%lab.tex%%=======================================
%lab.tex%\section{Bounded lattices}
%lab.tex%%=======================================
%lab.tex%Let $\latL\eqd\latticed$ be a lattice.
%lab.tex%By the definition of a \structe{lattice} \xref{def:lattice},
%lab.tex%the \structe{upper bound} ($x \join y$) and \structe{lower bound} ($x\meet y$) of any two elements in $\setX$
%lab.tex%is also in $\setX$.
%lab.tex%But what about the upper and lower bounds
%lab.tex%of the entire set $\setX$ ($\joinop\setX$ and $\meetop\setX$)%
%lab.tex%\footnote{$\joinop\setX$: \prefp{def:join}, $\meetop\setX$:\prefpp{def:meet}}?
%lab.tex%If both of these are in $\setX$, then the lattice $\latL$ is said to be
%lab.tex%\prope{bounded} (next definition).
%lab.tex%All \prope{finite} lattices are bounded (next proposition).
%lab.tex%However, not all lattices are bounded---%
%lab.tex%for example, the lattice $\opair{\Z}{\le}$ (the lattice of integers 
%lab.tex%with the standard integer ordering relation) is \prope{unbounded}.
%lab.tex%%see \prefpp{ex:latb_Z} for a counter example of a lattice that is \prope{unbounded}.
%lab.tex%\prefpp{prop:latb_prop} gives two properties of bounded lattices.
%lab.tex%Boundedness is one of the ``\hie{classic 10}" properties\ifsxref{boolean}{thm:boo_prop}
%lab.tex%of \hie{Boolean algebras}\ifsxref{boolean}{def:booalg}.
%lab.tex%Conversely, a bounded and complemented lattice that satisfies the conditions
%lab.tex%$1'=0$ and \thme{Elkan's law} \emph{is} a \structe{Boolean algebra}\ifsxref{boolean}{prop:boo_char_elkan}.
%lab.tex%%---------------------------------------
%lab.tex%\begin{definition}
%lab.tex%\label{def:latb}
%lab.tex%%---------------------------------------
%lab.tex%Let $\latL\eqd\latticed$ be a lattice.
%lab.tex%Let $\joinop\setX$ be the least upper bound of $\opair{\setX}{\orel}$ and
%lab.tex%let $\meetop\setX$ be the greatest lower bound of $\opair{\setX}{\orel}$.
%lab.tex%\defboxt{
%lab.tex%  %\begin{tabular}{>{$\imark$\quad}ll}
%lab.tex%  \indentx\begin{tabular}{ll}
%lab.tex%    $\latL$ is \hid{upper bounded} & if $\brp{\joinop\setX}\in\setX.$ \\
%lab.tex%    $\latL$ is \hid{lower bounded} & if $\brp{\meetop\setX}\in\setX.$ \\
%lab.tex%    $\latL$ is \hid{bounded}       & if $\latL$ is both upper and lower bounded.
%lab.tex%  \end{tabular}
%lab.tex%  \\
%lab.tex%  A \prope{bounded} lattice is optionally denoted $\latbd$,
%lab.tex%  where $\bzero\eqd\meetop\setX$ and $\bid\eqd\joinop\setX$.
%lab.tex%  }
%lab.tex%\end{definition}
%lab.tex%
%lab.tex%%---------------------------------------
%lab.tex%\begin{proposition}
%lab.tex%\label{prop:latb_finite}
%lab.tex%%---------------------------------------
%lab.tex%Let $\latL\eqd\latticed$ be a lattice.
%lab.tex%\propbox{
%lab.tex%  \text{$\latL$ is \prope{finite}}
%lab.tex%  \qquad\implies\qquad
%lab.tex%  \text{$\latL$ is \prope{bounded}}
%lab.tex%  }
%lab.tex%\end{proposition}
%lab.tex%
%lab.tex%
%lab.tex%%---------------------------------------
%lab.tex%\begin{proposition}
%lab.tex%\label{prop:latb_prop}
%lab.tex%%---------------------------------------
%lab.tex%Let $\latL\eqd\latticed$ be a lattice with $\joinop\setX\eqd\bid$ and
%lab.tex%$\meetop\setX\eqd\bzero$.
%lab.tex%\propbox{%
%lab.tex%  \text{$\latL$ is \prope{bounded}}
%lab.tex%  \qquad\implies\qquad
%lab.tex%  \brbl{\begin{array}{lclCDD}
%lab.tex%    x \join \bid   &=& \bid   & \forall x\in\setX & (\prop{upper bounded}) & and \\
%lab.tex%    x \meet \bzero &=& \bzero & \forall x\in\setX & (\prop{lower bounded}) & and \\
%lab.tex%    x \join \bzero &=& x      & \forall x\in\setX & (\prop{join-identity}) & and \\
%lab.tex%    x \meet \bid   &=& x      & \forall x\in\setX & (\prop{meet-identity}) & 
%lab.tex%  \end{array}}%
%lab.tex%  }
%lab.tex%\end{proposition}
%lab.tex%\begin{proofns}
%lab.tex%\begin{align*}
%lab.tex%  x \join \bid
%lab.tex%    &= x \join \brp{\joinop\setX}
%lab.tex%    && \text{by definition of $\bid$ \xref{def:latb}}
%lab.tex%  \\&= \joinop\setX
%lab.tex%    && \text{because $x\in\setX$}
%lab.tex%  \\&= \bid
%lab.tex%    && \text{by definition of $\bid$ \xref{def:latb}}
%lab.tex%  \\
%lab.tex%  x \meet \bzero
%lab.tex%    &= x \meet \brp{\meetop\setX}
%lab.tex%    && \text{by definition of $\bzero$ \xref{def:latb}}
%lab.tex%  \\&= \meetop\setX
%lab.tex%    && \text{because $x\in\setX$}
%lab.tex%  \\&= \bzero
%lab.tex%    && \text{by definition of $\bzero$ \xref{def:latb}}
%lab.tex%  \\
%lab.tex%  \boxed{x}
%lab.tex%    &= \joinop\setn{x}
%lab.tex%  \\&\orel \joinop\setn{x,\lzero}
%lab.tex%    && \text{because $\setn{x}\subseteq\setn{\lzero,x}$ and \prope{isotone} property \xref{prop:orel_isotone}}
%lab.tex%  \\&= \boxed{x \join \lzero}
%lab.tex%    && \text{by definition of $\join$ \xref{def:join}}
%lab.tex%  \\&= x \join \brp{\meetop\setX}
%lab.tex%    && \text{by definition of $\lzero$ \xref{def:latb}}
%lab.tex%  \\&\orel x\join \brp{\meetop\setn{x}}
%lab.tex%    && \text{because $\setn{x}\subseteq\setX$ and \prope{isotone} property \xref{prop:orel_isotone}}
%lab.tex%  \\&\orel x\join \brp{\meetop\setn{x,x}}
%lab.tex%    && \text{by definition of $\setn{\cdot}$}
%lab.tex%  \\&= x\join \brp{x\meet x}
%lab.tex%    && \text{by definition of $\meet$ \xref{def:meet}}
%lab.tex%  \\&= \boxed{x}
%lab.tex%    && \text{by \prope{absorptive} property of lattices \xref{thm:lattice}}
%lab.tex%  %\\
%lab.tex%  %\boxed{x}
%lab.tex%  \\&= x\meet \brp{x\join x}
%lab.tex%    && \text{by \prope{absorptive} property of lattices \xref{thm:lattice}}
%lab.tex%  \\&\eqd x\meet \brp{\joinop\setn{x,x}}
%lab.tex%    && \text{by definition of $\join$ \xref{def:join}}
%lab.tex%  \\&\eqd x\meet \brp{\joinop\setn{x}}
%lab.tex%    && \text{by definition of set $\setn{\cdot}$}
%lab.tex%  \\&\orel x \meet \brp{\joinop\setX}
%lab.tex%    && \text{because $\setn{x}\subseteq\setn{x,\lid}$ and by \prope{isotone} property of $\meetop$ \xref{prop:orel_isotone}}
%lab.tex%  \\&= \boxed{x \meet \lid}
%lab.tex%    && \text{by definition of $\lid$ \xref{def:latb}}
%lab.tex%  \\&= \meetop\setn{x,\lid}
%lab.tex%    && \text{by definition of $\meet$ \xref{def:meet}}
%lab.tex%  \\&\orel \meetop\setn{x}
%lab.tex%    && \text{because $\setn{x}\subseteq\setn{x,\lid}$ and by \prope{isotone} property of $\meetop$ \xref{prop:orel_isotone}}
%lab.tex%  \\&= \boxed{x}
%lab.tex%\end{align*}
%lab.tex%\end{proofns}
%lab.tex%
%lab.tex%%---------------------------------------
%lab.tex%\begin{definition}
%lab.tex%\citetbl{
%lab.tex%  \citerpg{birkhoff1967}{5}{0821810251}
%lab.tex%  }
%lab.tex%\label{def:height}
%lab.tex%%---------------------------------------
%lab.tex%Let $\latL\eqd\latbX$ be a \structe{bounded lattice} \xref{def:latb}.
%lab.tex%\defboxp{
%lab.tex%  The \fnctd{height} $\height(x)$ of a point $x\in\latL$ is the 
%lab.tex%  \vale{least upper bound} of the \fncte{length}s \xref{def:length} of all the \structe{chain}s 
%lab.tex%  that have $\lzero$ and in which $x$ is the \vale{least upper bound}.
%lab.tex%  The \fnctd{height} $\height(\latL)$ of the lattice $\latL$ is defined as
%lab.tex%  \\\indentx$\height(\latL)\eqd\height(\lid)$ .
%lab.tex%  }
%lab.tex%\end{definition}

