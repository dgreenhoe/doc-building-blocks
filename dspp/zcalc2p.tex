%============================================================================
% Daniel J. Greenhoe
% GNU Ocatave file
% 1st order low pass filter design
%============================================================================
%=======================================
%\chapter{Coefficient Calculation}
%=======================================
%----------------------------------------------------------------------------
\section{2nd Order \propb{low-pass} calculation---polar form}
%----------------------------------------------------------------------------
%---------------------------------------
% problem statement
%---------------------------------------
\mbox{}\vfill
{\begin{align*}
  \boxed{\Fh(\omega)}
    &= \brlr{\Zh(z)}_{z=e^{i\omega}}
  \\&= G\brs{\frac{\brp{z+1}^2}
                  {\brp{z-p}\brp{z-p^\ast}}}_{z=e^{i\omega}}
  \\&= G\brs{\frac{\brp{z+1}^2}
                  {\brp{z-re^{i\phi}}\brp{z-(re^{i\phi})^\ast}}}_{z=e^{i\omega}}
  \\&= G\brs{\frac{z^2 + 2z + 1}
                  {z^2 - 2r\cos(\phi)z + r^2}
            }_{z=e^{i\omega}}
\end{align*}}
\vfill
We need 3 equations to solve for the 3 unkowns $G$, $r$, and $\phi$
\vfill\mbox{}
%---------------------------------------
\newpage
Equation 1: Gain=1 at DC
%---------------------------------------
\vfill
\begin{align*}
  1 &= \abs{\Fh(0)}
  \\&= G\brs{\frac{z^2 + 2z            + 1}
                  {z^2 - 2r\cos(\phi)z + r^2}}_{z=e^{i\omega},\,\omega=0}
  \\&= G\brs{\frac{1   + 2             + 1}
                  {1   - 2r\cos(\phi)  + r^2}}
    && \text{(see \prefpo{equ:trickdc})}
  \\
  \\&\implies \boxed{G = \frac{r^2 - 2r\cos(\phi) + 1}{4}}
\end{align*}
\vfill\mbox{}

%---------------------------------------
\newpage
Equation 2: Gain = $\sfrac{1}{2}$ at corner frequency
%---------------------------------------
{\Large\begin{align*}
  \boxed{\abs{\Fh(\omega)}^2}
    &= \abs{\Zh(z)}^2_{z=e^{i\omega}}
     = G^2\brs{\frac{z^2 + 2z            + 1}
                    {z^2 - 2r\cos(\phi)z + r^2}}^2_{z=e^{i\omega}}
  \\&= G^2\brlr{
         \brs{\frac{z^2 + 2z + 1} {z^2 - 2r\cos(\phi)z + r^2}}
         \brs{\frac{z^2 + 2z + 1} {z^2 - 2r\cos(\phi)z + r^2}}^\ast
         }_{z=e^{i\omega}}
  \\&= G^2\brlr{
         \brs{\frac{z^2       + 2z      + 1} {z^2       - 2r\cos(\phi)z      + r^2}}
         \brs{\frac{z^{2\ast} + 2z^\ast + 1} {z^{2\ast} - 2r\cos(\phi)z^\ast + r^2}}
         }_{z=e^{i\omega}}
  \\&= G^2\brlr{
         \frac{
         \brs{ \abs{z}^4                    + 2 \abs{z}^2z               +  z^2} +
         \brs{2\abs{z}^2z^\ast              + 4 \abs{z}^2                + 2z  } +
         \brs{z^{2\ast}                     + 2      z^\ast              + 1   }
         }{\begin{array}{cl} 
            &\brs{\abs{z}^4 - 2r\cos(\phi)z\abs{z}^2 + r^2z^2}
         \\+&\brs{-2r\cos(\phi)\abs{z}^2z^\ast +4r^2\cos^2(\phi)\abs{z}^2 -2r^3\cos(\phi)z}
         \\+&\brs{r^2z^{2\ast} - 2r^3\cos(\phi)z^\ast + r^4}
         \end{array}
         }}_{z=e^{i\omega}}
  \\&= G^2\brlr{
         \frac{\abs{z}^4 + 2           \abs{z}^2(z+z^\ast) +    (z^2+z^{2\ast}) + 4               \abs{z}^2 + 2             (z+z^\ast) + 1  }  
              {\abs{z}^4 - 2r\cos(\phi)\abs{z}^2(z+z^\ast) + r^2(z^2+z^{2\ast}) + 4r^2\cos^2(\phi)\abs{z}^2 - 2r^3\cos(\phi)(z+z^\ast) + r^4}
         }_{z=e^{i\omega}}
  \\&= G^2\brs {
         \frac{1         + 4           \cos(\omega)        + 2   \cos(2\omega)  + 4                         +4             \cos(\omega) + 1  }  
              {1         - 4r\cos(\phi)\cos(\omega)        + 2r^2\cos(2\omega)  + 4r^2\cos^2(\phi)          -4r^3\cos(\phi)\cos(\omega) + r^4}
             }
  \\&= G^2\brs {
         \frac{2   \boxed{\cos(2\omega)}  + 8                  \boxed{\cos(\omega)} +  \boxed{6}}  
              {2r^2\boxed{\cos(2\omega)}  - 4r\cos(\phi)[1+r^2]\boxed{\cos(\omega)} +  \boxed{r^4 +  4r^2\cos^2(\phi) + 1}}
             }
\end{align*}


$\ds\boxed{
  G^2\brs {
         \frac{2   \boxed{\cos(2\omega_c)}  + 8                  \boxed{\cos(\omega_c)} +  \boxed{6}}  
              {2r^2\boxed{\cos(2\omega_c)}  - 4r\cos(\phi)[1+r^2]\boxed{\cos(\omega_c)} +  \boxed{r^4 +  4r^2\cos^2(\phi) + 1}}
             }
  = \frac{1}{2}
  }$}
\vfill\mbox{}

%---------------------------------------
\newpage
Equation 3: For more smoothness in passband, set 2nd derivative to 0.
%---------------------------------------
\newpage
Aside: Who cares about the second derivative?
\vfill
In math, \propd{smoothness} of $\ff(x)$ is the number of derivatives $\dndxn\ff(x)$ that are continuous.
\vfill
Example: Hermite interpolation.

  $\begin{array}{r|@{\qquad}*{13}{r}}
     p & \mc{13}{l}{\ds(1-y)^p\fP_m(y)= (1-y)^p \sum_{k=0}^{p-1} {p-1+k\choose k} y^k}
  \\ \hline
     1 & 1 &-&   y
  \\ 2 & 1 &-&  3y^2 &+&  2y^3
  \\ 3 & 1 &-& 10y^3 &+&  15y^4 &-&   6y^5
  \\ 4 & 1 &-&  35y^4 &+&  84y^5 &-&   70y^6  &+&    20y^7
  \\ 5 & 1 &-& 126y^5 &+&  420y^6 &-&  540y^7 &+&   315y^8 &-&    70y^9
  \\ 6 & 1 &-& 462y^6 &+& 1980y^7 &-& 3465y^8 &+&  3080y^9 &-&  1386y^{10} &+& 252y^{11}
  \end{array}$

{\Large
%\begin{figure}[h]
%\begin{scriptsize}
%\color[rgb]{0.2,0.2,0.2}
%\begin{fsL}
\setlength{\unitlength}{0.8mm}
\begin{picture}(335,150)(-10,-30)
  %\graphpaper[10](-100,0)(300,150)
  \thicklines
  \put(   0,   0 ){\line(1,0){240} }
  \put(   0,   0 ){\line(0,1){120} }

  \qbezier[10](100,50)(100, 25)( 100,0)
  \qbezier[40](0,100)(100,100)( 200,100)

  {\color{blue}
    \qbezier(   0, 100)(  60, 100)( 100,  50)
    \qbezier( 100,  50)( 140,   0)( 200,   0)
    \put( 130,  25 ){\makebox(0,0)[l]{$\leftarrow (1-y)^p \fP(y)$} }
    \put(   5, 105 ){\makebox(0,0)[lb]{$\downarrow$ first $p-1$ derivatives are zero at $y=0$} }
    \put( 200,   5 ){\makebox(0,0)[lb]{$\downarrow$ first $p-1$ derivatives are zero at $y=1$} }
    }

  {\color{red}
    \qbezier(   0,   0)(  60,   0)( 100,  50)
    \qbezier( 100,  50)( 140, 100)( 200, 100)
    \put( 130,  75 ){\makebox(0,0)[l]{$\leftarrow y^p \fP(1-y)$} }
    \put(   5,  -3 ){\makebox(0,0)[lt]{$\uparrow$ first $p-1$ derivatives are zero at $y=0$} }
    \put( 200,  95 ){\makebox(0,0)[lt]{$\uparrow$ first $p-1$ derivatives are zero at $y=1$} }
    }

  \put( 100,   3 ){\makebox(0,0)[b]{$\frac{\pi}{2}$} }
  \put( 260,   0 ){\makebox(0,0)[l]{$\omega$} }
  \put( -10, 100 ){\makebox(0,0)[r]{$1$} }
\end{picture}
%\end{fsL}
%\end{center}
%\end{scriptsize}
%\caption{
%  Polynomial quadrature condition low-pass and high-pass terms
%  \label{fig:lphp}
%  }
%\end{figure}
}

%\prefpp{lem:quadcon_y} expresses the quadrature condition
%as a 
polynomial in $y=\sin^2\brp{\frac{\omega}{2}}$.
%The first term in this polynomial quadrature condition is a
%low-pass response and the second term is a high pass;
%and they meet in the middle at $\omega=\frac{\pi}{2}$.
    \[
      \mcom{(1-y)^p \fP(y)}{low-pass} + \mcom{y^p \fP(1-y)}{high-pass} = 1
    \]
The low-pass and high-pass terms are especially smooth at
$\omega=0$ ($y=0$) and $\omega=\pi$ ($y=1$)
%in that the first $p-1$ derivatives at both points are zero
%for both terms.

%This is illustrated in \prefpp{fig:lphp}.
\newpage
%--------------------------------------
\begin{theorem}[\thme{Hermite Interpolation}]
\footnote{
  \citerc{greenhoe2013wsd}{\url{https://www.researchgate.net/publication/322978679}}
  }
\label{thm:interpo_hermite}
%--------------------------------------
\thmbox{\begin{array}{*{3}{>{\ds}l}D}
  \left.\deriv{^n}{y^n} \left[
        (1-y)^p \sum_{k=0}^{p-1}{p+k-1\choose k} y^k
        \right]\right|_{y=0}
    &=& \kdelta_n
    &   for $n=0,1,2,\ldots,p-1$
  \\
  \left.\deriv{^n}{y^n} \left[
        (1-y)^p \sum_{k=0}^{p-1} {p+k-1\choose k} y^k
        \right]\right|_{y=1}
    &=& 0
    &   for $n=0,1,2,\ldots,p-1$
  \\
  \left.\deriv{^n}{y^n} \left[
        y^p \sum_{k=0}^{p-1}{p+k-1\choose k} (1-y)^k
        \right]\right|_{y=0}
    &=& 0
    &   for $n=0,1,2,\ldots,p-1$
  \\
  \left.\deriv{^n}{y^n} \left[
        y^p \sum_{k=0}^{p-1} {p+k-1\choose k} (1-y)^k
        \right]\right|_{y=1}
    &=& \kdelta_n
    &   for $n=0,1,2,\ldots,p-1$
  \end{array}}
\end{theorem}





\newpage
Back to filtering\ldots
\vfill
{\begin{align*}
  0 &= \ddwddw\abs{\Fh(\omega)}^2_{\omega=0}
  \\&= \ddwddw G^2 \brs{\frac{\ff(\omega)}
                             {\fg(\omega)}
                       }_{\omega=0}
  \\&= \ddw    G^2 \brs{\frac{\ff'\fg + \ff \fg'}
                             {\fg^2}
                       }_{\omega=0}
    && \text{by \thme{product rule}}
  \\&=         G^2 \brs{\frac{(\ff''\fg + \ff'\fg' - \ff'\fg' -\ff\fg'')\fg^2 - (\ff'\fg - \ff\fg')(2\fg\fg')}
                             {\fg^4}
                       }_{\omega=0}
  \\&=         G^2 \brs{\frac{\ff''\fg -\ff\fg''}
                             {\fg^2}
                       }_{\omega=0}
     \qquad \implies \boxed{\brlr{\ff''\fg = \ff\fg''}_{\omega=0}}
\end{align*}
\ldots because $\ff'(0) = \fg'(0) = 0$ (see \prefpo{equ:mrflat})
\vfill\mbox{}

%---------------------------------------
\newpage
%---------------------------------------
(Another) Aside: Feel the need to differentiate a product multiple times?
\\\vfill
Leibniz has you covered\ldots
\\\vfill
%This is simplified thanks to the \hie{Leibniz rule}, also called the 
%\hie{generalized product rule} (\hie{GPR}, next lemma).
%The Leibniz rule is remarkably similar in form to the \hie{binomial theorem}.
%--------------------------------------
\begin{lemma}[\thmd{Leibniz rule} / \thmd{generalized product rule} / \thmd{GPR}]
%\footnote{\url{http://en.wikipedia.org/wiki/Leibniz_rule_(generalized_product_rule)}}
\index{Leibniz rule}
\index{generalized product rule}
\index{binomial coefficient}
\footnote{
  \citerpg{benisrael2002}{154}{3211829245},
  \citor{leibniz1710}
  }
\label{lem:LGPR}
%--------------------------------------
%Let $\ff(x),\fg(x)\in\spLLR$ with derivatives
%$\ff^{(n)}(x)\eqd\deriv{^n}{x^n}\ff(x)$ and
%$\fg^{(n)}(x)\eqd\deriv{^n}{x^n}\fg(x)$ for $n=0,1,2,\ldots$,
%and ${n\choose k}\eqd\frac{n!}{(n-k)!k!}$ (binomial coefficient).
%Then
\lembox{
  \deriv{^n}{x^n}[\ff(x)\fg(x)] =
  \sum_{k=0}^n {n\choose k} \ff^{(k)}(x) \fg^{(n-k)}(x)
  }
\end{lemma}
\vfill\mbox{}
Example:
\qquad
$\ds\deriv{^3}{x^3}\brs{\ff(x)\fg(x)} = \ff'''(x)\fg(x) + 3\ff''(x)\fg'(x) + 3\ff'(x)\fg''(x) + \ff(x)\fg'''(x)$
\vfill\mbox{}


%--------------------------------------
\newpage
%--------------------------------------
\vfill
{\LARGE\begin{align*}
  &\implies \mcom{\brs{-8\cos(2\omega)  - 8\cos(\omega)}}
                 {\Large$\ff''$}
            \mcom{\brs{2r^2\cos(2\omega)  - 4r\cos(\phi)[1+r^2]\cos(\omega) +  r^4 +  4r^2\cos^2(\phi) + 1}}
                 {\Large$\fg$}
 \\&\qquad= \brs{\mcom{\brs{2\cos(2\omega)  + 8 \cos(\omega) +  6}}
                      {\Large$\ff$}
                 \mcom{\brs{-8r^2\cos(2\omega)  + 4r\cos(\phi)[1+r^2]\cos(\omega) }}
                      {\Large$\fg''$}
                   }_{\omega=0} 
  \\&\implies \brs{-8  - 8}      \brs{ 2r^2 - 4r\cos(\phi)[1+r^2] +  r^4 +  4r^2\cos^2(\phi) + 1}
       \\&\qquad= \brs{2  + 8  +  6} \brs{-8r^2 + 4r\cos(\phi)[1+r^2]}
  \\&\implies\boxed{r^4 + \brs{4\cos^2(\phi)-6} r^2 + 1 = 0 }
  \\
\end{align*}}

\newpage
\vfill
{\LARGE$\begin{array}{rc>{\ds}l}
  r^2 &=& \frac{-\brs{4\cos^2(\phi)-6} \pm \sqrt{\brs{4\cos^2\phi-6}^2 - 4}}{2}
    \\&=& \frac{-\brs{4\cos^2(\phi)-6} \pm \sqrt{\brs{16\cos^4\phi-48\cos^2\phi+36} - 4}}{2}
    \\&=& \frac{-2\brs{2\cos^2(\phi)-3} \pm 2\sqrt{4\cos^4\phi-12\cos^2\phi+8}}{2}
    \\&=& \mcom{\brs{3-2\cos^2(\phi)}}            {\Large$1\rightarrow      3        \rightarrow1$} 
     \pm \mcom{\sqrt{4\cos^4\phi-12\cos^2\phi+8}}{\Large$0\rightarrow\sqrt{8}\eqa2.8\rightarrow0$}
    \\\\
   r &=&\sqrt{\brs{3-2\cos^2(\phi)} \pm \sqrt{4\cos^4\phi-12\cos^2\phi+8}}
\end{array}$}
\hfill
\tbox{\includegraphics[width=90mm]{graphics/iir2rphi.pdf}}
\\
\vfill\mbox{}

%---------------------------------------
\newpage
Example: 2nd order \propb{low-pass} with corner frequency $\omega_c=\frac{2}{3}\pi$
%---------------------------------------
\begin{align*}
  1 &= \abs{\Fh(0)}
  \\&= G\brs{\frac{z^2 + 2z            + 1}
                  {z^2 - 2r\cos(\phi)z + r^2}}_{z=e^{i\omega},\,\omega=0}
  \\&= G\brs{\frac{1   + 2             + 1}
                  {1   - 2r\cos(\phi)  + r^2}}
    && \text{(see \prefpo{equ:trickdc})}
  \\
  \\&\implies \boxed{G = \frac{r^2 - 2r\cos(\phi) + 1}{4}}
\end{align*}



\begin{align*}
  \frac{1}{2}
  &= G^2\brs {
         \frac{2   \boxed{\cos(2\omega_c)}  + 8                  \boxed{\cos(\omega_c)} +  \boxed{6}}  
              {2r^2\boxed{\cos(2\omega_c)}  - 4r\cos(\phi)[1+r^2]\boxed{\cos(\omega_c)} +  \boxed{r^4 +  4r^2\cos^2(\phi) + 1}}
             }
\\&= G^2\brs {
         \frac{2-\sqrt{3}}  
              {r^4 + 2\cos(\phi)r^3  + [1-\sqrt{3}]r^2  + 2r\cos(\phi) + 1}
              }
\end{align*}


{\LARGE We can combine the previous two \fbox{boxed} equations to eliminate $G$


\begin{align*}
1/2 
  &= \brs{\frac{r^2 - 2r\cos(\phi) + 1}{4}}^2
     \brs{\frac{2-\sqrt{3}}  
               {r^4 + 2\cos(\phi)r^3  + [1-\sqrt{3}]r^2  + 2r\cos(\phi) + 1}}
\\&= \brs{\frac{2-\sqrt{3}}{16}}
     \frac{\brs{r^2 - 2r\cos(\phi) + 1}^2}
          {r^4 + 2\cos(\phi)r^3  + [1-\sqrt{3}]r^2  + 2\cos(\phi)r + 1}
\end{align*}
{\Large
$\ds{
8\brs{r^4 + 2\cos(\phi)r^3  + [1-\sqrt{3}]r^2  + 2\cos(\phi)r + 1}
   = \brs{2-\sqrt{3}} \brs{r^2 - 2r\cos(\phi) + 1}^2
}$          
          
$\ds{
8\brs{r^4 + 2\cos(\phi)r^3  + [1-\sqrt{3}]r^2  + 2\cos(\phi)r + 1}
   = \brs{2-\sqrt{3}} \brs{r^4 - 4\cos(\phi)r^3 + 6\cos^2(\phi)r^2 -4\cos(\phi)r  + 1}
}$          
}

{\Large
$\ds\boxed{
    \brs{6+\sqrt{3}}r^4 
 + 4\brs{6-\sqrt{3}}\cos(\phi)r^3 
 + \brs{8-8\sqrt{3}- 6\brp{2-\sqrt{3}}\cos^2(\phi)}r^2  
 + 4\brs{18-\sqrt{3}}\cos(\phi)r  
 +\brs{6+\sqrt{3}}
 =0
}$     }     
          
          
          
\newpage\mbox{}\vfill
{\normalsize
\lstinputlisting[numbers=left, numberstyle=\tiny, language=C++]{../common/cpp/df2_order2.cpp}
}
\mbox{}\vfill

\newpage
%--------------------------------------
Bonus!!! Using {\LaTeX} / {\XeLaTeX} / PostScript for plotting!!!
%--------------------------------------
\\
{\normalsize
\lstinputlisting[numbers=left, numberstyle=\tiny, language=Octave]{../common/math/graphics/dsp/iir2rphi.tex}
}


