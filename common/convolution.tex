%============================================================================
% Daniel J. Greenhoe
% LaTeX File
%============================================================================
%======================================
\chapter{Convolution}
%======================================
%======================================
\section{Definition}
%======================================
%--------------------------------------
\begin{definition}
\footnote{
  \citerpgc{bachman2002}{268}{9780387988993}{Definition 5.2.1, but with $1/2\pi$ scaling factor},
  \citerpg{bachman1964}{6}{9781483267562},
  \citerpgc{bracewell1978}{224}{007007013X}{Table 11.1 Theorems for the Laplace Transform}
  }
\label{def:conv}
%--------------------------------------
\defbox{\indxs{\conv}\begin{array}{M}
  The \opd{convolution operation} is defined as % $\conv\in\clF{\R^2}{\R}$ is defined as
  \\\indentx$\ds
    \brs{\ff\conv\fg}(x)
    \eqd \ff(x)\conv \fg(x)
    \eqd \int_{u\in\R} \ff(u)\fg(x-u) \du
    \qquad\scy\forall\ff,\fg\in\spLLRBu
  $
\end{array}}
\end{definition}

%======================================
\section{Properties}
%======================================
%--------------------------------------
\begin{theorem}
\footnote{
  \citerppg{bachman2002}{268}{270}{9780387988993}
  }
\label{thm:conv_algebra}
%--------------------------------------
\thmbox{\begin{array}{>{\ds}rc>{\ds}lCD}
    \ff\conv\fg &=& \fg\conv\ff & (\prope{commutative})
  \\\ff\conv\brp{\fg\conv\fh} &=& \brp{\fg\conv\fg}\conv\fh & & (\prope{associateve})
  \\\brp{\alpha\ff}\conv\fg   &=& \alpha\brp{\ff\conv\fg} = \ff\conv\brp{\alpha\fg} & \forall \alpha\in\C
  \\\ff\conv\brp{\fg+\fh} &=& \brp{\ff\conv\fg}+\brp{\ff\conv\fh}                   &                      & (\prope{distributive})
\end{array}}
\end{theorem}

%--------------------------------------
\begin{theorem}[\thmd{convolution theorem}]
\footnote{
  \citerppg{bachman2002}{268}{270}{9780387988993},
  \citerpg{bachman1964}{8}{9781483267562}
  }
\label{thm:conv}
%--------------------------------------
Let $\opLT$ be the \ope{Laplace Transform} operator \xref{def:opLT}
and $\conv$ the convolution operator \xref{def:conv}.
\thmbox{
\begin{array}{rclC}
  \mcom{\opLT\brs{\ff(x)\conv\fg(x)}(s)}{convolution    in ``time domain"}
       &=& \mcom{\sqrt{2\pi}\brs{\opLT\ff}(s)\, \brs{\opLT\fg}(s)}  {multiplication in ``s domain"}
       & \forall\ff,\fg\in\spLLRBu\\
  \mcom{\opLT\brs{\ff(x) \fg(x)}(s)} {multiplication in ``time domain"}
       &=& \mcom{\Lscale\brs{\opLT\ff}(s) \conv \brs{\opLT\fg}(s)}{convolution    in ``s domain"}
       & \forall\ff,\fg\in\spLLRBu.
\end{array}
}
\end{theorem}
\begin{proof}
\begin{align*}
   \opLT \brs{\ff(x)\conv\fg(x)}(s)
     &= \opLT \brs{ \int_{u\in\R} \ff(u)\fg(x-u) \du }(s)
     && \text{by definition of $\conv$} &&\text{\xref{def:conv}}
   \\&=  \int_{u\in\R} \ff(u)\brs{\opLT \fg(x-u)}(s) \du
   \\&=  \int_{u\in\R} \ff(u) e^{-su} \; \brs{\opLT\fg(x)}(s) \du
     && \text{by \prope{translation} property} 
     && \text{\xref{thm:opLT_shift}}
   \\&= \mcom{\brp{\Lscale\int_{u\in\R} \ff(u) e^{-su} \du}}
             {$\brs{\opLT\ff}(s)$                          } \;
        \brs{\opLT\fg}(s)
   \\&= \brs{\opLT\ff}(s)\;  \brs{\opLT\fg}(s)
     && \text{by definition of $\opLT$} &&\text{\xref{def:opLT}}
   \\
   \opLT[\ff(x)\fg(x)](s)
     &= \opLT\brs{\brp{\opLT^{-1}\opLT\ff(x)}\;\fg(x)}(s)
     && \text{by def. of operator inverse} &&\text{\ifxref{operator}{def:op_inv}}
   \\&= \opLT\brs{\brp{ \Lscalei \int_{v\in\R} \brs{\opLT\ff(x)}(v) e^{sxv} \dv }\;\fg(x)}(s)
     && \text{by \prefp{thm:opFTi}}
   \\&= \Lscalei \int_{v\in\R} \brs{\opLT\ff(x)}(v) \brs{\opLT\brp{e^{sxv} \;\fg(x)}}(s,v) \dv
   \\&= \Lscalei \int_{v\in\R} \brs{\opLT\ff(x)}(v) \brs{\opLT\fg(x)}(s-v) \dv
     && \text{by \prefp{thm:ft_shift}}
   \\&= \Lscalei\brs{\opLT\ff}(s)\conv \brs{\opLT\fg}(s)
     && \text{by definition of $\conv$} &&\text{\xref{def:conv}}
\end{align*}
\end{proof}


