%============================================================================
% LaTeX File
% Daniel J. Greenhoe
%============================================================================

%======================================
\chapter{Probability Space}
\label{chp:prbspace}
%======================================

\begin{figure}[th]
  \centering
  %============================================================================
% Daniel J. Greenhoe
% LaTeX file
%============================================================================
\begin{pspicture}(-7,0.5)(5,9)%
  %-------------------------------------
  % settings
  %-------------------------------------
   \psset{%
    arrowsize=4mm,
    arrowlength=0.6,
    arrowinset=0.1,
    %linecolor=blue,
    %linewidth=1pt,
     cornersize=relative,
     framearc=0.25,
    % gridcolor=graph,
    % subgriddiv=1,
    % gridlabels=4pt,
    % gridwidth=0.2pt,
    xunit=1\latunit,
    yunit=1.25\latunit,
     }%
  %-------------------------------------
  % nodes
  %-------------------------------------
   \begin{tabstr}{0.75}%
     \rput(-1, 8){\rnode{spaces}    {\psframebox{\begin{tabular}{c}abstract spaces\end{tabular}}}}%
     \rput(-4, 7){\rnode{lin}       {\psframebox{\begin{tabular}{c}linear spaces\ifnxref{vector}{def:vspace}\end{tabular}}}}%
     \rput( 2, 7){\rnode{top}       {\psframebox{\begin{tabular}{c}topological spaces\ifnxref{vstopo}{def:toplinspace}\end{tabular}}}}%
     \rput( 2, 6){\rnode{metric}    {\psframebox{\begin{tabular}{c}metric spaces\ifnxref{metric}{def:metric}\end{tabular}}}}%
     \rput( 2, 4.5){\rnode{commetric} {\psframebox{\begin{tabular}{c}complete metric spaces\ifnxref{seq}{def:ms_complete}\end{tabular}}}}%
     \rput(-4, 5){\rnode{metriclin} {\psframebox{\begin{tabular}{c}metric linear spaces\end{tabular}}}}%
     \rput(-4, 4){\rnode{normlin}   {\psframebox{\begin{tabular}{c}normed linear spaces\ifnxref{vsnorm}{def:norm}\end{tabular}}}}%
     \rput( 2, 3){\rnode{banach}    {\psframebox{\begin{tabular}{c}Banach spaces\ifnxref{seq}{def:banach}\end{tabular}}}}%
     \rput(-4, 3){\rnode{inprod}    {\psframebox{\begin{tabular}{c}inner-product spaces\ifnxref{vsinprod}{def:inprod}\end{tabular}}}}%
     \rput( 2, 2){\rnode{hilbert}   {\psframebox{\begin{tabular}{c}Hilbert spaces\ifnxref{seq}{def:hilbert}\end{tabular}}}}%
     \rput( 2, 1){\rnode{zero}      {\psframebox{\begin{tabular}{c}$\spZero$\ifnxref{subspace}{prop:subspace_0X}\end{tabular}}}}%
   \end{tabstr}%
  %-------------------------------------
  % connecting lines/arrows
  %-------------------------------------
   %\ncline[doubleline=true]{->}{lin}{spaces} double arrow seems to cause trouble for xdvipdfmx
   \ncline{lin}{spaces}
   \ncline{top}{spaces}%
   \ncline{com}{spaces}%
   \ncline{metriclin}{lin}%
   \ncline{normlin}{metriclin}%
   \ncline{inprod}{normlin}
   \ncline{banach}{normlin}%
   \ncline{hilbert}{inprod}%
   \ncline{hilbert}{banach}%
   \ncline{zero}{hilbert}%
   \ncline{metric}{top}%
   \ncline{metriclin}{metric}
   \ncline{commetric}{metric}%
   \ncline{banach}{commetric}%
  %-------------------------------------
  % labeling
  %-------------------------------------
   %\psccurve[linestyle=dashed,linecolor=red,fillstyle=none]%
   %  (0,-5)(20,4)(70,60)(20,60)(15,55)(-5,38)(-20,35)(-10,20)(-25,5)%
   %\psline[linecolor=red]{->}(60,75)(60,68)%
   %\uput[135](60,75){complete spaces}%
   %\psline[linecolor=red]{->}(26,80)(15,74)%
   %\psline[linecolor=red]{->}(30,80)(30,62.5)%
   %\uput[90](28,80){analytic spaces}%
   %
   %\psgrid[unit=10mm](-8,-1)(8,9)%
\end{pspicture}%

  \caption{Lattice of mathematical spaces\label{fig:vector_spaces}}
\end{figure}%


%used in problatn.tex 20170218%\qboxnps
%used in problatn.tex 20170218%  {
%used in problatn.tex 20170218%    Blaise Pascal (1623--1662), mathematician
%used in problatn.tex 20170218%    \index{Pascal, Blaise}
%used in problatn.tex 20170218%    \index{quotes!Pascal, Blaise}
%used in problatn.tex 20170218%    \footnotemark
%used in problatn.tex 20170218%  }
%used in problatn.tex 20170218%  {../common/people/pascal.jpg}
%used in problatn.tex 20170218%  {It is not certain that everything is certain.}
%used in problatn.tex 20170218%  \footnotetext{\begin{tabular}[t]{ll}
%used in problatn.tex 20170218%    quote: & \url{http://en.wikiquote.org/wiki/Blaise_Pascal} \\
%used in problatn.tex 20170218%    image: & \url{http://en.wikipedia.org/wiki/Image:Blaise_pascal.jpg}
%used in problatn.tex 20170218%  \end{tabular}}


%======================================
\section{Probability Space}
%======================================
In mathematics, a \hie{space} is simply a set and in the most general definition,
nothing else.
However, normally for a space to actually be useful, some additional structure is added.
One of the most general additional structures is a \hie{topology};
and a space together with a topology is called a
\hie{topological space}.\footnote{\hie{topological space}: \prefp{def:top_space}}
A topological space imposes additional structure on a space  in the form of subsets
and guarantees that these subsets are closed under such fundamental operations as
set \hie{union} and set \hie{intersection}.
With the additional structure available in a topological space, we are able to
analyze such basic concepts as
  \begin{liste}
    \item continuity
    \item convergence
    \item connectivity
  \end{liste}

However for a great number of mathematical applications,
we need to \hie{measure} mathematical objects---
the most general measurement being measures on subsets of some set.
Examples of measurement in mathematics include
  \begin{liste}
    \item integration
    \item probability
  \end{liste}
Before measurement can be effectively performed on a set,
the set must be provided with a subset structure.
However, a simple topology does not provide sufficient structure
for effective measurement.
For example, often we would not only like to measure some subset $\setA$,
but also its complement $\cmpA$.
A topology is not closed under the complement operation.
Rather instead we equip the space with a more powerful (and thus less general) structure
called a \hie{sigma-algebra}.
A sigma-algebra is a subset structure that is closed under set complement.
A set together with a sigma-algebra is called a \hie{measurable space}.
And a set together with a sigma-algebra and a \hie{measure} on that sigma-algebra
is called a \hie{measure space}.
\footnote{\begin{tabular}[t]{>{\em}l<{:}>{Definition }l>{page }l}
  measure           & \ref{def:measure}       & \pageref{def:measure}       \\
  $\sigma$-algebra  & \ref{def:set_sigma-algebra} & \pageref{def:set_sigma-algebra} \\
  measure space     & \ref{def:mspace}        & \pageref{def:mspace}
  \end{tabular}}

%The next definition presents a very important measure space---the \hie{probability space}.
%%---------------------------------------
%\begin{definition}
%\label{def:prob_space}
%\index{probability space}
%\index{outcomes} \index{events} \index{probability measure}
%%---------------------------------------
%\defbox{\begin{tabular}{l@{\qquad}l}
%  \mc{2}{l}{The triple $\ps$ is a \hid{probability space} over $\pso$ where}
%  \\
%  $\imark$ $\pso$ is the set of all possible {\bf outcomes} of an experiment \\
%  $\imark$ $\pse$ is a $\sigma$-algebra on $\pso$  \\
%  $\imark$ $\psp:\pse\to[0,1]$ is a measure on $\pse$.
%\end{tabular}}
%\end{definition}

%---------------------------------------
\begin{theorem}
%---------------------------------------
\thmbox{
  \mcom{
    \text{$\ps$ is a probability space}
    \quad\implies\quad
    \text{$\ps$ is a measure space}
  }{(every probability space is a measure space)}
  }
\end{theorem}


%---------------------------------------
\begin{example}[Single coin flip]
%---------------------------------------
Let $\coinhead$ represent ``heads" and $\cointail$ represent ``tails"
in a coin toss.
The probabilities related to the single flip of a ``fair" coin can be evaluated
on probability space $\ps$ defined as follows:
\exbox{\begin{array}{rcl@{\qquad}D}
  \pso &=&   \setn{\coinhead,\, \cointail}
       &     (set of outcomes)
    \\
  \pse &=&   \setn{\mcom{\emptyset}{neither},\,
                   \mcom{\setn{\coinhead}}{heads},\,
                   \mcom{\setn{\cointail}}{tails},\,
                   \mcom{\setn{\coinhead,\,\cointail}}{heads or tails}
                  }
       &     (sigma-algebra on $\pso$)
    \\
  \psp x &=& \left\{\begin{array}{l>{\text{for }}l@{\qquad}D}
      0           & x=\emptyset                  & (neither heads nor tails) \\
      \frac{1}{2} & x=\setn{\coinhead}           & (heads)                   \\
      \frac{1}{2} & x=\setn{\text{\cointail}}    & (tails)                   \\
      1           & x=\setn{\coinhead,\cointail} & (either heads or tails)
    \end{array}\right.
       &      (probability measure on $\pse$)
\end{array}}
\end{example}

%---------------------------------------
\begin{example}[Two coin flips]
%---------------------------------------
Let $\seto{\setA}$ be the number of elements in a set $\setA$.
Let $\coinhead$ represent ``heads" and $\cointail$ represent ``tails"
in a coin toss.
Suppose we flip a coin two consecutive time and want to
know how many ``heads" we will have.
The following is a valid probability space $\ps$:
\exbox{\begin{array}{rcl}
  \pso &=& \setn{ (\cointail\cointail),\, (\cointail\coinhead),\, (\coinhead\cointail),\, (\coinhead\coinhead) }  \\
  \pse &=& \setn{\begin{array}{l}
    \mcom{\emptyset}{nothing},\,
    \mcom{\setn{(\cointail\cointail)}}{0 heads},\,
    \mcom{\setn{(\cointail\coinhead),\,(\coinhead\cointail)}}{1 heads},\,
    \mcom{\setn{(\coinhead\coinhead)}}{2 heads}
    \mcom{\setn{(\cointail\cointail),\,(\cointail\coinhead),\,(\coinhead\cointail)}}{0 heads or 1 heads},\, \\
    \mcom{\setn{(\cointail\cointail),\,(\coinhead\coinhead)}}{0 heads or 2 heads},\,
    \mcom{\setn{(\cointail\coinhead),\,(\coinhead\cointail),\,(\coinhead\coinhead)}}{1 heads or 2 heads},\,
    \pso
    %\mcom{\setn{(\cointail\cointail),\, (\cointail\coinhead),\, (\coinhead\cointail),\, (\coinhead\coinhead)} }{0 or 1 or 2 heads ($\pso$)}
    \end{array}}
  \\
  \psp x &=& \frac{1}{4} \seto{x}
%  \psp x &=& \left\{\begin{array}{*{2}{l>{\text{for }}l@{\qquad\qquad}}}
%    0           & x=\emptyset         & \frac{3}{4} & x=\text{ 0 heads or 1 heads}            \\
%    \frac{1}{4} & x=\text{ 0 heads}   & \frac{1}{2} & x=\text{ 0 heads or 2 heads}            \\
%    \frac{1}{2} & x=\text{ 1 heads}   & \frac{3}{4} & x=\text{ 1 heads or 2 heads}            \\
%    \frac{1}{4} & x=\text{ 2 heads}   & 1           & x=\text{ 0 or 1 or 2 heads}
%    \end{array}\right.
\end{array}}

Note however that
\[   \pse = \left\{ \emptyset, \mbox{0 heads}, \mbox{1 head}, \mbox{2 heads}, \pso \right\} \]
is {\bf not} a valid event set because it is {\bf not} a sigma-algebra.
For example, $\mbox{1 head}\cup\mbox{2 heads}$ is not in $\pse$.
\end{example}


%---------------------------------------
\begin{example}[Even/odd dice measure space]
\label{ex:prob_even_odd_dice}
%---------------------------------------
Suppose we have an ``unfair" dice and we want to know whether
the result of rolling the dice one time will be ``even" or ``odd".
We can construct the following measure space (probability space) $\ps$:

\exbox{\begin{array}{ll}
  \pso &= \left\{ \text{\diceA,\diceB,\diceC,\diceD,\diceE,\diceF} \right\}
  \\
  \pse &= \left\{ \mcom{\left\{\hspace{1ex} \right\}}{$\emptyset$},
                  \mcom{\left\{\text{\diceA,\diceC,\diceE}\right\}}{odd},
                  \mcom{\left\{\text{\diceB,\diceD,\diceF}\right\}}{even},
                  \mcom{\left\{ \text{\diceA,\diceB,\diceC,\diceD,\diceE,\diceF} \right\}}{$\pso$}
          \right\}
  \\
  \psp(e) &=
    \left\{\begin{array}{l@{\qquad} >{\text{for }e=}l @{\qquad}D}
      0           & \{\hspace{1ex}\}
                  & ($\emptyset$)
                  \\
      1           & \left\{ \text{\diceA,\diceB,\diceC,\diceD,\diceE,\diceF} \right\}
                  & ($\pso$)
                  \\
      \frac{1}{3} & \left\{\text{\diceA,\diceC,\diceE}\right\}
                  & (odd)
                  \\
      \frac{2}{3} & \left\{\text{\diceB,\diceD,\diceF}\right\}
                  & (even)
    \end{array}\right.
\end{array}}
\end{example}

Example~\ref{ex:prob_even_odd_dice} (previous) illustrated a
measure space in which the events (ignorning $\emptyset$ and $\pso$)
are {\em mutually exclusive}.
Example~\ref{ex:prob_dice} (next) illustrates a measure space
where events are {\em not} mutually exclusive.
%---------------------------------------
\begin{example}
\label{ex:prob_dice}
%---------------------------------------
Suppose we have a ``fair" dice and we are primarily interested in the
events of the first four
$\left(\setn{\text{\diceA,\diceB,\diceC,\diceD}}\right)$
(that is, whether one roll of the dice will produce
a value in the set $\{1,2,3,4\}$)
and the last three
$\left(\setn{\text{\diceD,\diceE,\diceF}}\right)$
However, these events do not by themselves form a $\sigma$-algebra.
Rather under the $\cap$ and $\cup$ operations, these two events generate
a total of eight possible events that together form a $\sigma$-algebra.
The resulting measure space $\ps$ is
\exbox{\begin{array}{ll}
  \pso &= \left\{ \text{\diceA,\diceB,\diceC,\diceD,\diceE,\diceF} \right\}
  \\
  \pse &= \left\{ \mcom{\setn{\quad}}{$\emptyset$},\;
                  \mcom{\setn{\text{\diceA,\diceB,\diceC,\diceD,\diceE,\diceF}}}{$\pso$},\;
                  \mcom{\setn{\text{\diceA,\diceB,\diceC,\diceD}}}{first four},\;
                  \mcom{\setn{\text{\diceD,\diceE,\diceF}}}{last three},\;
                  \right.
                  \\&\qquad
                  \left.
                  \mcom{\setn{\text{\diceD}}}{$\setn{1234}\cap\setn{456}$},\;
                  \mcom{\setn{\text{\diceA,\diceB,\diceC,\diceE,\diceF}}}{$\setn{4}^c$},\;
                  \mcom{\setn{\text{\diceE,\diceF}}}{$\setn{4}^c\cap\setn{456}$},\;
                  \mcom{\setn{\text{\diceA,\diceB,\diceC}}}{$\setn{1234}\cap\setn{4}^c$},\;
          \right\}
  \\
  \psp(e) &=
    \left\{\begin{array}{l@{\qquad} >{\text{for }e=}l @{\qquad}D}
      0           & \setn{\quad}
                  & ($\emptyset$)\\
      1           & \setn{\text{\diceA,\diceB,\diceC,\diceD,\diceE,\diceF}}
                  & ($\pso$)\\
      \frac{2}{3} & \setn{\text{\diceA,\diceB,\diceC,\diceD}}
                  & (first four)\\
      \frac{1}{2} & \setn{\text{\diceD,\diceE,\diceF}}
                  & (last three)\\
      \frac{1}{6} & \setn{\text{\diceD}}
                  & ($\setn{1234}\cap\setn{456}$) \\
      \frac{5}{6} & \setn{\text{\diceA,\diceB,\diceC,\diceE,\diceF}}
                  & ($\setn{4}^c$) \\
      \frac{1}{3} & \setn{\text{\diceE,\diceF}}
                  & ($\setn{4}^c\cap\setn{456}$) \\
      \frac{1}{2} & \setn{\text{\diceA,\diceB,\diceC}}
                  & ($\setn{1234}\cap\setn{4}^c$)
    \end{array}\right.
\end{array}}
\end{example}

Why go through all the trouble of requiring a $\sigma$-algebra?
Having a $\sigma$-algebra in place ensures that anything we might possibly
want to measure {\em can} be measured.
It makes sure all possible combinations are taken into account.
And why go through the additional trouble of requiring a measure space?
With a measure space available, expressing the measure over a complex
set is often greatly simplified because the measure space provides nice
algebraic properties (namely the $\sigma$-additive property.
\pref{ex:prob_123456} (next) illustrates how a rather complex
$\sigma$-algebra (64 elements) can be compactly represented in a measure space.
%---------------------------------------
\begin{example}
\label{ex:prob_123456}
%---------------------------------------
Suppose we have a ``fair" dice and we are interested in measuring over the
power set of events (largest possible algebra---$2^6=64$ events).
This leads to the measure space $\ps$ where
\exbox{\begin{array}{ll@{\qquad}D}
  \pso    &= \setn{ \text{\diceA,\diceB,\diceC,\diceD,\diceE,\diceF} }
          \\
  \pse    &= \mathcal{P}(\pso) & (the power-set of $\pso$)
          \\
  \psp(e) &= \frac{1}{6} |e|
          & ($\frac{1}{6}$ times the number of possible outcomes in event $e$)
\end{array}}
\end{example}

%---------------------------------------
\begin{example}[Gaussian noise]
%---------------------------------------
Let $X\sim\pN{0}{\sigma^2}$ be a random variable with Gaussian distribution.
We can construct the following probability space $\ps$:

\exbox{\begin{array}{rcl}
  \pso &=& \R \\
  \pse &=& \setn{ \emptyset, \pso } \setu \set{(a,b)}{a,b\in\R,a<b} \\
  \psp x &=& \left\{\begin{array}{ll}
    0      & \mbox{ for } x=\emptyset \\
    1      & \mbox{ for } x=\pso      \\
    \frac{1}{\sqrt{2\pi\sigma^2}}\int_a^b e^{\frac{x^2}{2\sigma^2}}\dx & \mbox{ otherwise}
    \end{array}\right.
\end{array}}
\end{example}

%---------------------------------------
\begin{example}
\label{ex:prob_1011}
%---------------------------------------
The set of outcomes $\pso$ can also be a set of waveforms:
\exbox{\begin{array}{ll}
  \pso    &= \setn{\begin{tabular}{llll}
               \textifsym{|H|LL|HHH|L} &
               \textifsym{L|H|LL|HHH}  &
               \textifsym{H|L|H|LL|HH} &
               \textifsym{HH|L|H|LL|H} \\
               \textifsym{HHH|L|H|LL}  &
               \textifsym{L|HHH|L|H|L} &
               \textifsym{LL|HHH|L|H}  &
             \end{tabular}}
          \\
  \pse    &= \mathcal{P}(\pso)
          \\
  \psp(e) &= \frac{1}{7} |e|
\end{array}}
\end{example}





%=======================================
\section{Probability subspaces}
%=======================================

%---------------------------------------
% Euclidean 3-space partitioned by power lattice
%---------------------------------------
\begin{minipage}[c]{\tw/2}
%\begin{figure}[th]
\begin{center}
\footnotesize
\setlength{\unitlength}{\tw/440}%
\begin{picture}(440,520)(-220,0)%
  \thicklines
  %{\color{graphpaper}\graphpaper[10](-200,0)(400,520)}%
  {\color{picbox}%
    %\put( -50,400){\framebox(100,100){}}%
    \put( 100,250){\framebox(100,100){}}%
    \put( -50,250){\framebox(100,100){}}%
    \put(-200,250){\framebox(100,100){}}%
    \put( 100,100){\framebox(100,100){}}%
    \put( -50,100){\framebox(100,100){}}%
    \put(-200,100){\framebox(100,100){}}%
    \put( -25,  0){\framebox( 50, 50){}}%
    }%
  {\color{black}%
    \put(   0,400){\line( 0,-1){ 50}}%
    \put(   0,400){\line(-3,-1){150}}%
    \put(   0,400){\line( 3,-1){150}}%
    \put(   0,200){\line(-3, 1){150}}%
    \put( 150,200){\line(-3, 1){150}}%
    \put(-150,200){\line( 3, 1){150}}%
    \put( 150,200){\line( 0, 1){ 50}}%
    \put(-150,200){\line( 0, 1){ 50}}%
    \put(   0,200){\line( 3, 1){150}}%
    \put(   0, 50){\line(-3, 1){150}}%
    \put(   0, 50){\line( 0, 1){ 50}}%
    \put(   0, 50){\line( 3, 1){150}}%
    }%
  \put(0,450){%
    \setlength{\unitlength}{1\tw/(450*2)}%
    \begin{picture}(0,0)(0,0)%
      %{\color{graphpaper}\graphpaper[10](-150,-150)(300,300)}%
      {\color{red}%
        \put( -50,  50){\line( 1, 1){100} }%
        \put(   0,   0){\line(-1,-1){ 50} }%
        \put(   0,-100){\line(-1,-1){ 50} }%
        \put( -50, -50){\line( 1, 1){ 50} }%
        \put(-150, -50){\line( 1, 1){ 50} }%
        \put(  50, -50){\line( 1, 1){100} }%
        }%
      {\color{green}%
        \put(-150, -50){\line( 1, 0){200} }%
       %\put( -75,  25){\line( 1, 0){ 50} }%
       %\put(  25,  25){\line( 1, 0){100} }%
        \put(   0,   0){\line( 1, 0){100} }%
        \put(-100,   0){\line( 1, 0){ 50} }%
        \put(-100, 100){\line( 1, 0){200} }%
        \put(-100,-100){\line( 1, 0){ 50} }%
        \put(   0,-100){\line( 1, 0){100} }%
        \put( 150,  50){\line(-1, 0){ 50} }%
        }%
      {\color{blue}%
        \put( -50,-150){\line( 0, 1){200} }%
       %\put(  25,  25){\line( 0, 1){100} }%
        \put(  50, 150){\line( 0,-1){ 50} }%
       %\put(  25, -75){\line( 0, 1){ 50} }%
        \put(-100,-100){\line( 0, 1){ 50} }%
        \put(-100,   0){\line( 0, 1){100} }%
        \put( 100,-100){\line( 0, 1){200} }%
        \put(   0,-100){\line( 0, 1){ 50} }%
        \put(   0,   0){\line( 0, 1){100} }%
        }%
    \end{picture}%
  }
  \put(-150,300){%
    \setlength{\unitlength}{1\tw/(400*3)}%
    \begin{picture}(0,0)(0,0)%
      %{\color{graphpaper}\graphpaper[10](-150,-150)(300,300)}%
      {\color{red}%
        \put(-150, -50){\line( 1, 1){100} }%
        \put(  50, -50){\line( 1, 1){100} }%
        }%
      {\color{green}%
        \put(-150, -50){\line( 1, 0){200} }%
        \put( 150,  50){\line(-1, 0){200} }%
        }%
    \end{picture}%
  }
  \put(0,300){%
    \setlength{\unitlength}{1\tw/(400*3)}%
    \begin{picture}(0,0)(0,0)%
      %{\color{graphpaper}\graphpaper[10](-150,-150)(300,300)}%
      {\color{red}%
        \put( -50,  50){\line( 1, 1){100} }%
        \put(  50, -50){\line(-1,-1){100} }%
        }%
      {\color{blue}%
        \put( -50,-150){\line( 0, 1){200} }%
        \put(  50, 150){\line( 0,-1){200} }%
        }%
    \end{picture}%
  }
  \put(150,300){%
    \setlength{\unitlength}{1\tw/(400*3)}%
    \begin{picture}(0,0)(0,0)%
      %{\color{graphpaper}\graphpaper[10](-150,-150)(300,300)}%
      {\color{green}%
        \put(-100, 100){\line( 1, 0){200} }%
        \put(-100,-100){\line( 1, 0){200} }%
        }%
      {\color{blue}%
        \put(-100,-100){\line( 0, 1){200} }%
        \put( 100,-100){\line( 0, 1){200} }%
        }%
    \end{picture}%
  }
  \put(-150,150){%
    \setlength{\unitlength}{1\tw/(400*3)}%
    \begin{picture}(0,0)(0,0)%
      %{\color{graphpaper}\graphpaper[10](-150,-150)(300,300)}%
      {\color{red}%
        \put(0,0){\vector( 1, 1){100} }%
        \put(0,0){\vector(-1,-1){100} }%
        }%
    \end{picture}%
  }
  \put(0,150){%
    \setlength{\unitlength}{1\tw/(400*3)}%
    \begin{picture}(0,0)(0,0)%
      %{\color{graphpaper}\graphpaper[10](-150,-150)(300,300)}%
      {\color{green}%
        \put(0,0){\vector( 1, 0){100} }%
        \put(0,0){\vector(-1, 0){100} }%
        }%
    \end{picture}%
  }
  \put(150,150){%
    \setlength{\unitlength}{1\tw/(400*3)}%
    \begin{picture}(0,0)(0,0)%
      %{\color{graphpaper}\graphpaper[10](-150,-150)(300,300)}%
      {\color{blue}%
        \put(0,0){\vector( 0, 1){100} }%
        \put(0,0){\vector( 0,-1){100} }%
        }%
    \end{picture}%
  }
  \put(0,25){%
    \setlength{\unitlength}{1\tw/(400*3)}%
    \begin{picture}(0,0)(0,0)%
      %{\color{graphpaper}\graphpaper[10](-150,-150)(300,300)}%
      {\color{black}%
        \put(0,0){\circle*{15}}%
        }%
    \end{picture}%
  }
\end{picture}
\end{center}
%\caption{
%  Euclidean 3-dimensional space partitioned as a power lattice
%  \label{fig:lat_E3d_power}
%  }
%\end{figure}
\end{minipage}
\begin{minipage}[c]{\tw/2}
  %---------------------------------------
  \begin{example}
  \label{ex:lat_E3d_power}
  %---------------------------------------
  The figure to the left illustrates the subspace lattice
  \[ (\powerset(\text{$x$-axis, $y$-axis, $z$-axis}),\, \subseteq,\, +,\, \seti). \]
  The least upper bound of the lattice is the 3-dimensional Euclidean space.
  The architecture of this subspace lattice is called the
  \hie{power architecture}. (\prefp{def:arch_power}).
  A power architecture is so named because it is isomorphic to the \hie{power set}
  of a set.
  \end{example}
\end{minipage}







%---------------------------------------
% Euclidean 3-space partitioned by power lattice
%---------------------------------------
\begin{figure}[th]
\begin{minipage}[c]{8\tw/16}
\begin{center}
\footnotesize
\setlength{\unitlength}{\tw/440}%
\begin{picture}(440,520)(-220,0)%
  \thicklines
  %{\color{graphpaper}\graphpaper[10](-200,0)(400,520)}%
  \put(-150,300){\makebox(0,0){%
    \includegraphics*[width=4\tw/16, height=4\tw/16, clip=true]{../common/normxy_00.eps}}}%
  \put(   0,300){\makebox(0,0){%
    \includegraphics*[width=4\tw/16, height=4\tw/16, clip=true]{../common/normxy_80.eps}}}%
  \put( 150,300){\makebox(0,0){%
    \includegraphics*[width=4\tw/16, height=4\tw/16, clip=true]{../common/normxy_95.eps}}}%
  \put(-150,180){\makebox(0,0){%
    \begin{picture}(200,120)(-100,0)
      \setlength{\unitlength}{2\tw/(8*200)}%
      %{\color{graphpaper}\graphpaper[10](-100,0)(200,120)}%
      \thicklines%
      \color{axis}%
        \put(-100,   0 ){\line(1,0){200} }%
        \qbezier[30](  0,0)(  0, 60)(  0,120)%
      \color{red}%
        \qbezier( -40,  60)(   0, 180)(  40,  60)%
        \qbezier(-100,   0)( -60,   0)( -40,  60)%
        \qbezier(  40,  60)(  60,   0)( 100,   0)%
    \end{picture}
  }}%
  \put(0,180){\makebox(0,0){%
    \begin{picture}(200,120)(-100,0)
      \setlength{\unitlength}{2\tw/(8*200)}%
      %{\color{graphpaper}\graphpaper[10](-100,0)(200,120)}%
      \thicklines%
      \color{axis}%
        \put(-100,   0 ){\line(1,0){200} }%
        \qbezier[30](  0,0)(  0, 60)(  0,120)%
      \color{green}%
        \qbezier( -40,  60)(   0, 180)(  40,  60)%
        \qbezier(-100,   0)( -60,   0)( -40,  60)%
        \qbezier(  40,  60)(  60,   0)( 100,   0)%
    \end{picture}
  }}%
  \put(150,180){\makebox(0,0){%
    \begin{picture}(200,120)(-100,0)
      \setlength{\unitlength}{2\tw/(8*200)}%
      %{\color{graphpaper}\graphpaper[10](-100,0)(200,120)}%
      \thicklines%
      \color{axis}%
        \put(-100,   0 ){\line(1,0){200} }%
        \qbezier[30](  0,0)(  0, 60)(  0,120)%
      \color{blue}%
        \qbezier( -40,  60)(   0, 180)(  40,  60)%
        \qbezier(-100,   0)( -60,   0)( -40,  60)%
        \qbezier(  40,  60)(  60,   0)( 100,   0)%
    \end{picture}
  }}%
  {\color{picbox}%
    %\put( -50,400){\framebox(100,100){}}%
    \put( 100,250){\framebox(100,100){}}%
    \put( -50,250){\framebox(100,100){}}%
    \put(-200,250){\framebox(100,100){}}%
    \put( 100,100){\framebox(100,100){}}%
    \put( -50,100){\framebox(100,100){}}%
    \put(-200,100){\framebox(100,100){}}%
    \put( -25,  0){\framebox( 50, 50){}}%
    }%
  {\color{black}%
    \put(   0,400){\line( 0,-1){ 50}}%
    \put(   0,400){\line(-3,-1){150}}%
    \put(   0,400){\line( 3,-1){150}}%
    \put(   0,200){\line(-3, 1){150}}%
    \put( 150,200){\line(-3, 1){150}}%
    \put(-150,200){\line( 3, 1){150}}%
    \put( 150,200){\line( 0, 1){ 50}}%
    \put(-150,200){\line( 0, 1){ 50}}%
    \put(   0,200){\line( 3, 1){150}}%
    \put(   0, 50){\line(-3, 1){150}}%
    \put(   0, 50){\line( 0, 1){ 50}}%
    \put(   0, 50){\line( 3, 1){150}}%
    }%
  \put(0,450){%
    \setlength{\unitlength}{1\tw/(450*2)}%
    \begin{picture}(0,0)(0,0)%
      %{\color{graphpaper}\graphpaper[10](-150,-150)(300,300)}%
      {\color{red}%
        \put( -50,  50){\line( 1, 1){100} }%
        \put(   0,   0){\line(-1,-1){ 50} }%
        \put(   0,-100){\line(-1,-1){ 50} }%
        \put( -50, -50){\line( 1, 1){ 50} }%
        \put(-150, -50){\line( 1, 1){ 50} }%
        \put(  50, -50){\line( 1, 1){100} }%
        }%
      {\color{green}%
        \put(-150, -50){\line( 1, 0){200} }%
       %\put( -75,  25){\line( 1, 0){ 50} }%
       %\put(  25,  25){\line( 1, 0){100} }%
        \put(   0,   0){\line( 1, 0){100} }%
        \put(-100,   0){\line( 1, 0){ 50} }%
        \put(-100, 100){\line( 1, 0){200} }%
        \put(-100,-100){\line( 1, 0){ 50} }%
        \put(   0,-100){\line( 1, 0){100} }%
        \put( 150,  50){\line(-1, 0){ 50} }%
        }%
      {\color{blue}%
        \put( -50,-150){\line( 0, 1){200} }%
       %\put(  25,  25){\line( 0, 1){100} }%
        \put(  50, 150){\line( 0,-1){ 50} }%
       %\put(  25, -75){\line( 0, 1){ 50} }%
        \put(-100,-100){\line( 0, 1){ 50} }%
        \put(-100,   0){\line( 0, 1){100} }%
        \put( 100,-100){\line( 0, 1){200} }%
        \put(   0,-100){\line( 0, 1){ 50} }%
        \put(   0,   0){\line( 0, 1){100} }%
        }%
    \end{picture}%
  }
  \put(-150,300){%
    \setlength{\unitlength}{1\tw/(400*3)}%
    \begin{picture}(0,0)(0,0)%
      %{\color{graphpaper}\graphpaper[10](-150,-150)(300,300)}%
      {\color{red}%
        \put(-150, -50){\line( 1, 1){100} }%
        \put(  50, -50){\line( 1, 1){100} }%
        }%
      {\color{green}%
        \put(-150, -50){\line( 1, 0){200} }%
        \put( 150,  50){\line(-1, 0){200} }%
        }%
    \end{picture}%
  }
  \put(0,300){%
    \setlength{\unitlength}{1\tw/(400*3)}%
    \begin{picture}(0,0)(0,0)%
      %{\color{graphpaper}\graphpaper[10](-150,-150)(300,300)}%
      {\color{red}%
        \put( -50,  50){\line( 1, 1){100} }%
        \put(  50, -50){\line(-1,-1){100} }%
        }%
      {\color{blue}%
        \put( -50,-150){\line( 0, 1){200} }%
        \put(  50, 150){\line( 0,-1){200} }%
        }%
    \end{picture}%
  }
  \put(150,300){%
    \setlength{\unitlength}{1\tw/(400*3)}%
    \begin{picture}(0,0)(0,0)%
      %{\color{graphpaper}\graphpaper[10](-150,-150)(300,300)}%
      {\color{green}%
        \put(-100, 100){\line( 1, 0){200} }%
        \put(-100,-100){\line( 1, 0){200} }%
        }%
      {\color{blue}%
        \put(-100,-100){\line( 0, 1){200} }%
        \put( 100,-100){\line( 0, 1){200} }%
        }%
    \end{picture}%
  }
  \put(-150,150){%
    \setlength{\unitlength}{1\tw/(400*3)}%
    \begin{picture}(0,0)(0,0)%
      %{\color{graphpaper}\graphpaper[10](-150,-150)(300,300)}%
      {\color{red}%
        \put(0,0){\vector( 1, 1){100} }%
        \put(0,0){\vector(-1,-1){100} }%
        }%
    \end{picture}%
  }
  \put(0,150){%
    \setlength{\unitlength}{1\tw/(400*3)}%
    \begin{picture}(0,0)(0,0)%
      %{\color{graphpaper}\graphpaper[10](-150,-150)(300,300)}%
      {\color{green}%
        \put(0,0){\vector( 1, 0){100} }%
        \put(0,0){\vector(-1, 0){100} }%
        }%
    \end{picture}%
  }
  \put(150,150){%
    \setlength{\unitlength}{1\tw/(400*3)}%
    \begin{picture}(0,0)(0,0)%
      %{\color{graphpaper}\graphpaper[10](-150,-150)(300,300)}%
      {\color{blue}%
        \put(0,0){\vector( 0, 1){100} }%
        \put(0,0){\vector( 0,-1){100} }%
        }%
    \end{picture}%
  }
  \put(0,25){%
    \setlength{\unitlength}{1\tw/(400*3)}%
    \begin{picture}(0,0)(0,0)%
      %{\color{graphpaper}\graphpaper[10](-150,-150)(300,300)}%
      {\color{black}%
        \put(0,0){\circle*{15}}%
        }%
    \end{picture}%
  }
\end{picture}
\end{center}
\end{minipage}
\caption{
  Euclidean 3-dimensional space partitioned as a power lattice
  \label{fig:psub_E3d_power_P}
  }
\end{figure}








%---------------------------------------
% Euclidean 3-d subspace lattice with progressive architecture
%---------------------------------------
\begin{minipage}[c]{\tw/2}
  %\begin{figure}[th]
  \begin{center}
  \footnotesize
  \setlength{\unitlength}{\tw/400}%
  \begin{picture}(400,520)(-200,0)%
    \thicklines
    %{\color{graphpaper}\graphpaper[50](-200,0)(400,520)}%
    {\color{picbox}%
      %\put( -50,400){\framebox(100,100){}}%
      \put(-125,250){\framebox(100,100){}}%
      \put( 100,100){\framebox(100,100){}}%
      \put( -50,100){\framebox(100,100){}}%
      \put(-200,100){\framebox(100,100){}}%
      \put( -25,  0){\framebox(50,50){}}%
      }%
    {\color{black}%
      \put(   0,400){\line( 3,-4){150}}%
      \put(   0,400){\line(-3,-2){ 75}}%
      \put( -75,250){\line( 3,-2){ 75}}%
      \put( -75,250){\line(-3,-2){ 75}}%
      \put(   0, 50){\line(-3, 1){150}}%
      \put(   0, 50){\line( 0, 1){ 50}}%
      \put(   0, 50){\line( 3, 1){150}}%
      }%
    \put(0,450){%
      \setlength{\unitlength}{\tw/(330*3)}%
      \begin{picture}(0,0)(0,0)%
        %{\color{graphpaper}\graphpaper[10](-150,-150)(300,300)}%
        {\color{red}%
          \put( -50,  50){\line( 1, 1){100} }%
          \put(   0,   0){\line(-1,-1){ 50} }%
          \put(   0,-100){\line(-1,-1){ 50} }%
          \put( -50, -50){\line( 1, 1){ 50} }%
          \put(-150, -50){\line( 1, 1){ 50} }%
          \put(  50, -50){\line( 1, 1){100} }%
          }%
        {\color{green}%
          \put(-150, -50){\line( 1, 0){200} }%
         %\put( -75,  25){\line( 1, 0){ 50} }%
         %\put(  25,  25){\line( 1, 0){100} }%
          \put(   0,   0){\line( 1, 0){100} }%
          \put(-100,   0){\line( 1, 0){ 50} }%
          \put(-100, 100){\line( 1, 0){200} }%
          \put(-100,-100){\line( 1, 0){ 50} }%
          \put(   0,-100){\line( 1, 0){100} }%
          \put( 150,  50){\line(-1, 0){ 50} }%
          }%
        {\color{blue}%
          \put( -50,-150){\line( 0, 1){200} }%
         %\put(  25,  25){\line( 0, 1){100} }%
          \put(  50, 150){\line( 0,-1){ 50} }%
         %\put(  25, -75){\line( 0, 1){ 50} }%
          \put(-100,-100){\line( 0, 1){ 50} }%
          \put(-100,   0){\line( 0, 1){100} }%
          \put( 100,-100){\line( 0, 1){200} }%
          \put(   0,-100){\line( 0, 1){ 50} }%
          \put(   0,   0){\line( 0, 1){100} }%
          }%
      \end{picture}%
    }
    \put(-75,300){%
      \setlength{\unitlength}{1\tw/(400*3)}%
      \begin{picture}(0,0)(0,0)%
        %{\color{graphpaper}\graphpaper[10](-150,-150)(300,300)}%
        {\color{red}%
          \put(-150, -50){\line( 1, 1){100} }%
          \put(  50, -50){\line( 1, 1){100} }%
          }%
        {\color{green}%
          \put(-150, -50){\line( 1, 0){200} }%
          \put( 150,  50){\line(-1, 0){200} }%
          }%
      \end{picture}%
    }
    \put(-150,150){%
      \setlength{\unitlength}{1\tw/(400*3)}%
      \begin{picture}(0,0)(0,0)%
        %{\color{graphpaper}\graphpaper[10](-150,-150)(300,300)}%
        {\color{red}%
          \put(0,0){\vector( 1, 1){100} }%
          \put(0,0){\vector(-1,-1){100} }%
          }%
      \end{picture}%
    }
    \put(0,150){%
      \setlength{\unitlength}{1\tw/(400*3)}%
      \begin{picture}(0,0)(0,0)%
        %{\color{graphpaper}\graphpaper[10](-150,-150)(300,300)}%
        {\color{green}%
          \put(0,0){\vector( 1, 0){100} }%
          \put(0,0){\vector(-1, 0){100} }%
          }%
      \end{picture}%
    }
    \put(150,150){%
      \setlength{\unitlength}{1\tw/(400*3)}%
      \begin{picture}(0,0)(0,0)%
        %{\color{graphpaper}\graphpaper[10](-150,-150)(300,300)}%
        {\color{blue}%
          \put(0,0){\vector( 0, 1){100} }%
          \put(0,0){\vector( 0,-1){100} }%
          }%
      \end{picture}%
    }
    \put(0,25){%
      \setlength{\unitlength}{1\tw/(400*3)}%
      \begin{picture}(0,0)(0,0)%
        %{\color{graphpaper}\graphpaper[10](-150,-150)(300,300)}%
        {\color{black}%
          \put(0,0){\circle*{15}}%
          }%
      \end{picture}%
    }
  \end{picture}
  \end{center}
  %\caption{
  %  Euclidean 3-dimensional space partitioned as a progressive lattice
  %  \label{fig:lat_E3d_power_progressive}
  %  }
  %\end{figure}
\end{minipage}
\begin{minipage}[c]{\tw/2}
  %---------------------------------------
  \begin{example}
  \label{ex:lat_E3d_progressive}
  %---------------------------------------
  The figure to the left illustrates the subspace lattice
  \[ (\setn{\text{0, $x$-axis, $y$-axis, $z$-axis, $xy$-plane, $xyz$-space}},\, \subseteq,\, +,\, \seti). \]
  The least upper bound of the lattice is the 3-dimensional Euclidean space.
  The architecture of this subspace lattice is called the \hie{progressive architecture}
  (\prefp{def:arch_progressive}).
  A progressive architecture is so named because it {\em progressively} adds
  one dimension for each step up the linearly ordered portion
  (left hand side) of the lattice.
\end{example}
\end{minipage}

%---------------------------------------
% Euclidean 3-d subspace with primitive architecture
%---------------------------------------
\begin{minipage}[c]{\tw/2}
  %\begin{figure}[th]
  \begin{center}
  \footnotesize
  \setlength{\unitlength}{\tw/400}%
  \begin{picture}(400,370)(-200,0)%
    \thicklines
    %{\color{graphpaper}\graphpaper[50](-200,0)(400,370)}%
    {\color{picbox}%
      %\put( -50,400){\framebox(100,100){}}%
      %\put(-125,250){\framebox(100,100){}}%
      \put( 100,100){\framebox(100,100){}}%
      \put( -50,100){\framebox(100,100){}}%
      \put(-200,100){\framebox(100,100){}}%
      \put( -25,  0){\framebox(50,50){}}%
      }%
    {\color{black}%
      \put(   0,250){\line(-3,-1){150}}%
      \put(   0,250){\line( 0,-1){ 50}}%
      \put(   0,250){\line( 3,-1){150}}%
      \put(   0, 50){\line(-3, 1){150}}%
      \put(   0, 50){\line( 0, 1){ 50}}%
      \put(   0, 50){\line( 3, 1){150}}%
      }%
    \put(0,300){%
      \setlength{\unitlength}{\tw/(330*3)}%
      \begin{picture}(0,0)(0,0)%
        %{\color{graphpaper}\graphpaper[10](-150,-150)(300,300)}%
        {\color{red}%
          \put( -50,  50){\line( 1, 1){100} }%
          \put(   0,   0){\line(-1,-1){ 50} }%
          \put(   0,-100){\line(-1,-1){ 50} }%
          \put( -50, -50){\line( 1, 1){ 50} }%
          \put(-150, -50){\line( 1, 1){ 50} }%
          \put(  50, -50){\line( 1, 1){100} }%
          }%
        {\color{green}%
          \put(-150, -50){\line( 1, 0){200} }%
         %\put( -75,  25){\line( 1, 0){ 50} }%
         %\put(  25,  25){\line( 1, 0){100} }%
          \put(   0,   0){\line( 1, 0){100} }%
          \put(-100,   0){\line( 1, 0){ 50} }%
          \put(-100, 100){\line( 1, 0){200} }%
          \put(-100,-100){\line( 1, 0){ 50} }%
          \put(   0,-100){\line( 1, 0){100} }%
          \put( 150,  50){\line(-1, 0){ 50} }%
          }%
        {\color{blue}%
          \put( -50,-150){\line( 0, 1){200} }%
         %\put(  25,  25){\line( 0, 1){100} }%
          \put(  50, 150){\line( 0,-1){ 50} }%
         %\put(  25, -75){\line( 0, 1){ 50} }%
          \put(-100,-100){\line( 0, 1){ 50} }%
          \put(-100,   0){\line( 0, 1){100} }%
          \put( 100,-100){\line( 0, 1){200} }%
          \put(   0,-100){\line( 0, 1){ 50} }%
          \put(   0,   0){\line( 0, 1){100} }%
          }%
      \end{picture}%
    }
    \put(-150,150){%
      \setlength{\unitlength}{1\tw/(400*3)}%
      \begin{picture}(0,0)(0,0)%
        %{\color{graphpaper}\graphpaper[10](-150,-150)(300,300)}%
        {\color{red}%
          \put(0,0){\vector( 1, 1){100} }%
          \put(0,0){\vector(-1,-1){100} }%
          }%
      \end{picture}%
    }
    \put(0,150){%
      \setlength{\unitlength}{1\tw/(400*3)}%
      \begin{picture}(0,0)(0,0)%
        %{\color{graphpaper}\graphpaper[10](-150,-150)(300,300)}%
        {\color{green}%
          \put(0,0){\vector( 1, 0){100} }%
          \put(0,0){\vector(-1, 0){100} }%
          }%
      \end{picture}%
    }
    \put(150,150){%
      \setlength{\unitlength}{1\tw/(400*3)}%
      \begin{picture}(0,0)(0,0)%
        %{\color{graphpaper}\graphpaper[10](-150,-150)(300,300)}%
        {\color{blue}%
          \put(0,0){\vector( 0, 1){100} }%
          \put(0,0){\vector( 0,-1){100} }%
          }%
      \end{picture}%
    }
    \put(0,25){%
      \setlength{\unitlength}{1\tw/(400*3)}%
      \begin{picture}(0,0)(0,0)%
        %{\color{graphpaper}\graphpaper[10](-150,-150)(300,300)}%
        {\color{black}%
          \put(0,0){\circle*{15}}%
          }%
      \end{picture}%
    }
  \end{picture}
  \end{center}
  %\caption{
  %  Euclidean 3-dimensional space partitioned as a progressive lattice
  %  \label{fig:lat_E3d_power_progressive}
  %  }
  %\end{figure}
\end{minipage}
\begin{minipage}[c]{\tw/2}
  %---------------------------------------
  \begin{example}
  \label{ex:lat_E3d_primitive}
  %---------------------------------------
  The figure to the left illustrates the subspace lattice
  \[ (\setn{\text{0, $x$-axis, $y$-axis, $z$-axis, $xyz$-space}},\, \subseteq,\, +,\, \seti). \]
  The least upper bound of the lattice is the 3-dimensional Euclidean space.
  The architecture of this subspace lattice is called the
  \hie{primitive architecture} (\prefp{def:arch_primitive}).
\end{example}
\end{minipage}


%---------------------------------------
% Euclidean 3-space partitioned by power lattice
%---------------------------------------
\begin{minipage}[c]{\tw/2}
%\begin{figure}[th]
\begin{center}
\footnotesize
\setlength{\unitlength}{\tw/440}%
\begin{picture}(100,520)(-50,0)%
  \thicklines
  %{\color[gray]{0.4}\graphpaper[50](-50,0)(100,520)}%
  {\color{picbox}%
    %\put( -50,400){\framebox(100,100){}}%
    \put( -50,250){\framebox(100,100){}}%
    \put( -50,100){\framebox(100,100){}}%
    \put( -25,  0){\framebox( 50, 50){}}%
    }%
  {\color{black}%
    \put(   0,400){\line( 0,-1){ 50}}%
    \put(   0,200){\line( 0, 1){ 50}}%
    \put(   0, 50){\line( 0, 1){ 50}}%
    }%
  \put(0,450){%
    \setlength{\unitlength}{1\tw/(450*2)}%
    \begin{picture}(0,0)(0,0)%
      %{\color{graphpaper}\graphpaper[10](-150,-150)(300,300)}%
      {\color{red}%
        \put( -50,  50){\line( 1, 1){100} }%
        \put(   0,   0){\line(-1,-1){ 50} }%
        \put(   0,-100){\line(-1,-1){ 50} }%
        \put( -50, -50){\line( 1, 1){ 50} }%
        \put(-150, -50){\line( 1, 1){ 50} }%
        \put(  50, -50){\line( 1, 1){100} }%
        }%
      {\color{green}%
        \put(-150, -50){\line( 1, 0){200} }%
       %\put( -75,  25){\line( 1, 0){ 50} }%
       %\put(  25,  25){\line( 1, 0){100} }%
        \put(   0,   0){\line( 1, 0){100} }%
        \put(-100,   0){\line( 1, 0){ 50} }%
        \put(-100, 100){\line( 1, 0){200} }%
        \put(-100,-100){\line( 1, 0){ 50} }%
        \put(   0,-100){\line( 1, 0){100} }%
        \put( 150,  50){\line(-1, 0){ 50} }%
        }%
      {\color{blue}%
        \put( -50,-150){\line( 0, 1){200} }%
       %\put(  25,  25){\line( 0, 1){100} }%
        \put(  50, 150){\line( 0,-1){ 50} }%
       %\put(  25, -75){\line( 0, 1){ 50} }%
        \put(-100,-100){\line( 0, 1){ 50} }%
        \put(-100,   0){\line( 0, 1){100} }%
        \put( 100,-100){\line( 0, 1){200} }%
        \put(   0,-100){\line( 0, 1){ 50} }%
        \put(   0,   0){\line( 0, 1){100} }%
        }%
    \end{picture}%
  }
    \put(0,300){%
      \setlength{\unitlength}{1\tw/(400*3)}%
      \begin{picture}(0,0)(0,0)%
        %{\color{graphpaper}\graphpaper[10](-150,-150)(300,300)}%
        {\color{red}%
          \put(-150, -50){\line( 1, 1){100} }%
          \put(  50, -50){\line( 1, 1){100} }%
          }%
        {\color{green}%
          \put(-150, -50){\line( 1, 0){200} }%
          \put( 150,  50){\line(-1, 0){200} }%
          }%
      \end{picture}%
    }
  \put(0,150){%
    \setlength{\unitlength}{1\tw/(400*3)}%
    \begin{picture}(0,0)(0,0)%
      %{\color{graphpaper}\graphpaper[10](-150,-150)(300,300)}%
      {\color{red}%
        \put(0,0){\vector( 1, 1){100} }%
        \put(0,0){\vector(-1,-1){100} }%
        }%
    \end{picture}%
  }
  \put(0,25){%
    \setlength{\unitlength}{1\tw/(400*3)}%
    \begin{picture}(0,0)(0,0)%
      %{\color{graphpaper}\graphpaper[10](-150,-150)(300,300)}%
      {\color{black}%
        \put(0,0){\circle*{15}}%
        }%
    \end{picture}%
  }
\end{picture}
\end{center}
%\caption{
%  Euclidean 3-dimensional space partitioned as a power lattice
%  \label{fig:lat_E3d_power}
%  }
%\end{figure}
\end{minipage}
\begin{minipage}[c]{\tw/2}
  %---------------------------------------
  \begin{example}
  \label{ex:lat_E3d_linear}
  %---------------------------------------
  The figure to the left illustrates the subspace lattice
  \[ (\setn{\text{$0$, $x$-axis, $xy$-plane, $xyz$-space}},\, \subseteq,\, \setu,\, \seti). \]
  The least upper bound of the lattice is the 3-dimensional Euclidean space.
  The architecture of this subspace lattice is called the
  \hie{linear architecture} \xref{def:arch_linear}.
  \end{example}
\end{minipage}




%---------------------------------------
\begin{example}
\label{ex:psub_lat_alg_xyz}
%---------------------------------------
Suppose a random process is capable of producing three values
$\pso\eqd\setn{x,y,z}$.
There are five \hie{algebras of sets} on $\pso$
and therefore five probability spaces $(\pso,\,\pse_n,\, \psp)$ on $\pso$
with the five values of $\pse_n$ listed below:
\footnote{\hie{algebra of sets}: \prefp{def:set_algebra}} \\
\begin{minipage}[c]{\tw/3}
  %\begin{figure}[th]
  \begin{center}
  \footnotesize
  \setlength{\unitlength}{3\tw/900}%
  \begin{picture}(300,400)(-150,-50)%
    \thicklines
    %{\color{graphpaper}\graphpaper[50](-150,-50)(300,400)}%
    \color{red}%
      \put( 0,250){\line(-2,-1){100}}%
      \put( 0,250){\line( 0,-1){50}}%
      \put( 0,250){\line( 2,-1){100}}%
      \put( 0, 50){\line(-2,1){100}}%
      \put( 0, 50){\line( 0,1){50}}%
      \put( 0, 50){\line( 2,1){100}}%
  %
    \put(0,300){%
      \setlength{\unitlength}{3\tw/(3*1000)}%
      \begin{picture}(0,0)(0,150)%
      %{\color{graphpaper}\graphpaper[50](-100,0)(200,300)}%
      \thicklines%
      \color{black}%
        \put( 100, 300){\makebox(0,0)[tl]{$(\pso,\pse_5,\psp)$}}%
        \put(  15, 300){\makebox(0,0)[l]{$1$}}%
        \put(-115, 200){\makebox(0,0)[r]{$\frac{2}{3}$}}%
        \put(  15, 200){\makebox(0,0)[l]{$\frac{2}{3}$}}%
        \put( 115, 200){\makebox(0,0)[l]{$\frac{2}{3}$}}%
        \put(-115, 100){\makebox(0,0)[r]{$\frac{1}{3}$}}%
        \put(  15, 100){\makebox(0,0)[l]{$\frac{1}{3}$}}%
        \put( 115, 100){\makebox(0,0)[l]{$\frac{1}{3}$}}%
        \put(  15,   0){\makebox(0,0)[l]{$0$}}%
      \color{latline}%
        \put(   0, 300){\line(-1,-1){100} }%
        \put(   0, 300){\line( 0,-1){100} }%
        \put(   0, 300){\line( 1,-1){100} }%
        \put( 100, 100){\line( 0, 1){100} }%
        \put( 100, 100){\line(-1, 1){100} }%
        \put(   0, 100){\line(-1, 1){100} }%
        \put(   0, 100){\line( 1, 1){100} }%
        \put(-100, 100){\line( 0, 1){100} }%
        \put(-100, 100){\line( 1, 1){100} }%
        \put(   0,   0){\line(-1, 1){100} }%
        \put(   0,   0){\line( 0, 1){100} }%
        \put(   0,   0){\line( 1, 1){100} }%
      \color{latdot}%
        \put(   0, 300){\circle*{40}}%
        \put( 100, 200){\circle*{40}}%
        \put(   0, 200){\circle*{40}}%
        \put(-100, 200){\circle*{40}}%
        \put( 100, 100){\circle*{40}}%
        \put(   0, 100){\circle*{40}}%
        \put(-100, 100){\circle*{40}}%
        \put(   0,   0){\circle*{40}}%
      \end{picture}%
    }
    \put(-100,150){%
      \setlength{\unitlength}{3\tw/(3*1000)}%
      \begin{picture}(0,0)(0,150)%
      %{\color{graphpaper}\graphpaper[50](-100,0)(200,300)}%
      \thicklines%
      \color{black}%
        \put(   0, 330){\makebox(0,0)[b]{$(\pso,\pse_2,\psp)$}}%
        \put(  15, 300){\makebox(0,0)[l]{$1$}}%
        \put( 115, 200){\makebox(0,0)[l]{$\frac{2}{3}$}}%
        \put(-115, 100){\makebox(0,0)[r]{$\frac{1}{3}$}}%
        \put(  15,   0){\makebox(0,0)[l]{$0$}}%
      \color{latline}%
        \put(   0, 300){\line(-1,-1){100} }%
        \put(   0, 300){\line( 0,-1){100} }%
        \put(   0, 300){\line( 1,-1){100} }%
        \put( 100, 100){\line( 0, 1){100} }%
        \put( 100, 100){\line(-1, 1){100} }%
        \put(   0, 100){\line(-1, 1){100} }%
        \put(   0, 100){\line( 1, 1){100} }%
        \put(-100, 100){\line( 0, 1){100} }%
        \put(-100, 100){\line( 1, 1){100} }%
        \put(   0,   0){\line(-1, 1){100} }%
        \put(   0,   0){\line( 0, 1){100} }%
        \put(   0,   0){\line( 1, 1){100} }%
      \color{latdot}%
        \put(   0, 300){\circle*{40}}%
        \put( 100, 200){\circle*{40}}%
        \put(-100, 100){\circle*{40}}%
        \put(   0,   0){\circle*{40}}%
      \end{picture}%
    }
    \put(0,150){%
      \setlength{\unitlength}{3\tw/(3*1000)}%
      \begin{picture}(0,0)(0,150)%
      %{\color{graphpaper}\graphpaper[50](-100,0)(200,300)}%
      \thicklines%
      \color{black}%
        \put(   0, 330){\makebox(0,0)[b]{$(\pso,\pse_3,\psp)$}}%
        \put(  15, 300){\makebox(0,0)[l]{$1$}}%
        \put(  15, 200){\makebox(0,0)[l]{$\frac{2}{3}$}}%
        \put(  15, 100){\makebox(0,0)[l]{$\frac{1}{3}$}}%
        \put(  15,   0){\makebox(0,0)[l]{$0$}}%
      \color{latline}%
        \put(   0, 300){\line(-1,-1){100} }%
        \put(   0, 300){\line( 0,-1){100} }%
        \put(   0, 300){\line( 1,-1){100} }%
        \put( 100, 100){\line( 0, 1){100} }%
        \put( 100, 100){\line(-1, 1){100} }%
        \put(   0, 100){\line(-1, 1){100} }%
        \put(   0, 100){\line( 1, 1){100} }%
        \put(-100, 100){\line( 0, 1){100} }%
        \put(-100, 100){\line( 1, 1){100} }%
        \put(   0,   0){\line(-1, 1){100} }%
        \put(   0,   0){\line( 0, 1){100} }%
        \put(   0,   0){\line( 1, 1){100} }%
      \color{latdot}%
        \put(   0, 300){\circle*{40}}%
        \put(   0, 200){\circle*{40}}%
        \put(   0, 100){\circle*{40}}%
        \put(   0,   0){\circle*{40}}%
      \end{picture}%
    }
    \put(100,150){%
      \setlength{\unitlength}{3\tw/(3*1000)}%
      \begin{picture}(0,0)(0,150)%
      %{\color{graphpaper}\graphpaper[50](-100,0)(200,300)}%
      \thicklines%
      \color{black}%
        \put(   0, 330){\makebox(0,0)[b]{$(\pso,\pse_4,\psp)$}}%
        \put(  15, 300){\makebox(0,0)[l]{$1$}}%
        \put(-115, 200){\makebox(0,0)[r]{$\frac{2}{3}$}}%
        \put( 115, 100){\makebox(0,0)[l]{$\frac{1}{3}$}}%
        \put(  15,   0){\makebox(0,0)[l]{$0$}}%
      \color{latline}%
        \put(   0, 300){\line(-1,-1){100} }%
        \put(   0, 300){\line( 0,-1){100} }%
        \put(   0, 300){\line( 1,-1){100} }%
        \put( 100, 100){\line( 0, 1){100} }%
        \put( 100, 100){\line(-1, 1){100} }%
        \put(   0, 100){\line(-1, 1){100} }%
        \put(   0, 100){\line( 1, 1){100} }%
        \put(-100, 100){\line( 0, 1){100} }%
        \put(-100, 100){\line( 1, 1){100} }%
        \put(   0,   0){\line(-1, 1){100} }%
        \put(   0,   0){\line( 0, 1){100} }%
        \put(   0,   0){\line( 1, 1){100} }%
      \color{latdot}%
        \put(   0, 300){\circle*{40}}%
        \put(-100, 200){\circle*{40}}%
        \put( 100, 100){\circle*{40}}%
        \put(   0,   0){\circle*{40}}%
      \end{picture}%
    }
    \put(0,0){%
      \setlength{\unitlength}{3\tw/(3*1000)}%
      \begin{picture}(0,0)(0,150)%
      %{\color{graphpaper}\graphpaper[50](-100,0)(200,300)}%
      \thicklines%
      \color{black}%
        \put( 130, 150){\makebox(0,0)[l]{$(\pso,\pse_1,\psp)$}}%
        \put(  15, 300){\makebox(0,0)[l]{$1$}}%
        \put(  15,   0){\makebox(0,0)[l]{$0$}}%
      \color{latline}%
        \put(   0, 300){\line(-1,-1){100} }%
        \put(   0, 300){\line( 0,-1){100} }%
        \put(   0, 300){\line( 1,-1){100} }%
        \put( 100, 100){\line( 0, 1){100} }%
        \put( 100, 100){\line(-1, 1){100} }%
        \put(   0, 100){\line(-1, 1){100} }%
        \put(   0, 100){\line( 1, 1){100} }%
        \put(-100, 100){\line( 0, 1){100} }%
        \put(-100, 100){\line( 1, 1){100} }%
        \put(   0,   0){\line(-1, 1){100} }%
        \put(   0,   0){\line( 0, 1){100} }%
        \put(   0,   0){\line( 1, 1){100} }%
      \color{latdot}%
        \put(   0, 300){\circle*{40}}%
        \put(   0,   0){\circle*{40}}%
      \end{picture}%
    }
  \end{picture}
  \end{center}
\end{minipage}
\begin{minipage}[c]{2\tw/3}
  \begin{align*}
    \pse_{ 1} &=\{ && \emptyset, &&           &&           &&           &&             &&             &&             && \setX && \}\\
    \pse_{ 2} &=\{ && \emptyset, && \setn{x}, &&           &&           &&             &&             && \setn{y,z}, && \setX && \}\\
    \pse_{ 3} &=\{ && \emptyset, &&           && \setn{y}, &&           &&             && \setn{x,z}, &&             && \setX && \}\\
    \pse_{ 4} &=\{ && \emptyset, &&           &&           && \setn{z}, && \setn{x,y}, &&             &&             && \setX && \}\\
    \pse_{ 5} &=\{ && \emptyset, && \setn{x}, && \setn{y}, && \setn{z}, && \setn{x,y}, && \setn{x,z}, && \setn{y,z}, && \setX && \}\\
  \end{align*}
Suppose the samples of $\pso$ are generated by a physical process such that they are all
equally likely to occur.
Then by varying the sigma-algebra $\pse_n$ effectively varies the probability
distribution of the probability space $(\pso,\, \pse_n,\, \psp)$.
This is illustrated by the figure to the left.
\end{minipage}
\end{example}




\begin{figure}
%\begin{minipage}[c]{10\tw/16}
  \begin{center}
  \begin{fsL}
  \setlength{\unitlength}{14\tw/(16*550)}%
  \begin{picture}(400,475)(-50,-100)%
    %{\color{graphpaper}\graphpaper[50](-50,-100)(400,450)}%
    \thicklines%
    \color{black}%
      \put(210,300){\makebox(0,0)[l]{$(\pso_5,\powerset(\pso_5),\psp_{\pso_5})$}}%
      \put(135,150){\makebox(0,0)[l]{$(\pso_4,\powerset(\pso_4),\psp_{\pso_4})$}}%
      \put(0,60){\makebox(0,0)[b]{$(\pso_3,\powerset(\pso_3),\psp_{\pso_3})$}}%
      \put(150,60){\makebox(0,0)[b]{$(\pso_2,\powerset(\pso_2),\psp_{\pso_2})$}}%
      \put(275,70){\makebox(0,0)[b]{$(\pso_1,\powerset(\pso_1),\psp_{\pso_1})$}}%
      \put(165,-75){\makebox(0,0)[r]{$(\pso_0,\powerset(\pso_0),\psp_{\pso_0})$}}%
    \color{black}%
      \put(150,250){\line( 3,-4){150}}%
      \put(150,250){\line(-3,-2){75}}%
      \put(75,100){\line(-3,-2){75}}%
      \put(75,100){\line(3,-2){75}}%
      \qbezier(225,-50)(112.5,-25)(  0,-0)%
      \qbezier(225,-50)(185.5,-25)(150,-0)%
      \qbezier(225,-50)(263.5,-25)(300,-0)%
    \color{blue}%
      \put(100,300){\begin{boxedminipage}[lt]{25mm}\scriptsize%
        \textifsym{|LLLLLLLL|}\\
        \textifsym{L|H|L|H|L|H|L|H}\\
        \textifsym{LL|HH|LL|HH}\\
        \textifsym{L|HH|LL|HH|L}\\
        \textifsym{LLLL|HHHH}\\
        \textifsym{L|H|L|HH|L|H|L}\\
        \textifsym{LL|HHHH|LL}\\
        \textifsym{L|HH|L|H|LL|H}
        \end{boxedminipage}}%
      \put(25,150){\begin{boxedminipage}[lt]{25mm}\scriptsize%
        \hspace{1pt}\\[1ex]
        \textifsym{|LLLLLLLL|}\\
        \textifsym{LL|HH|LL|HH}\\
        \textifsym{LLLL|HHHH}\\
        \textifsym{LL|HHHH|LL}\\
        \\
        \end{boxedminipage}}%
      \put(-50,25){\begin{boxedminipage}[lt]{25mm}\scriptsize%
        \textifsym{|LLLLLLLL|}\\
        \textifsym{LLLL|HHHH}\\
        \end{boxedminipage}}%
    \color{red}%
      \put(100,25){\begin{boxedminipage}[lt]{25mm}\scriptsize%
        \textifsym{LL|HH|LL|HH}\\
        \textifsym{LL|HHHH|LL}\\
        \end{boxedminipage}}%
      \put(225,25){\begin{boxedminipage}[lt]{25mm}\scriptsize%
        \textifsym{L|H|L|H|L|H|L|H}\\
        \textifsym{L|HH|LL|HH|L}\\
        \textifsym{L|H|L|HH|L|H|L}\\
        \textifsym{L|HH|L|H|LL|H}
        \end{boxedminipage}}%
    \color{black}%
      \put(175,-75){\begin{boxedminipage}[lt]{25mm}\scriptsize%
        \hspace{1pt}\\[1ex]
        \textifsym{|LLLLLLLL|}
        \\
        \end{boxedminipage}}%
  \end{picture}%
  \end{fsL}
  \end{center}
%\end{minipage}
\caption{
  Probability spaces using Hadamard sample spaces in progressive architecture
  \label{fig:psub_hadamard}
  }
\end{figure}


%---------------------------------------
\begin{example}
%---------------------------------------
\prefpp{fig:psub_hadamard} illustrates a probability subspace architecture
where partitioning is performed with respect to the sample space $\pso$.
Here, \hie{Hadamard} waveforms are partitioned across the probability subspaces.
As a process selects from different subspaces, the statistical properties of that
process will also vary.
\end{example}









