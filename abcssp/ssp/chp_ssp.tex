%============================================================================
% Daniel J. Greenhoe
%============================================================================
%\chapter{Order and metric compatible symbolic sequence processing}\label{chp:ssp}
%\chapter{Stochastic processing using mapping to R\raisebox{0.5ex}{N}}\label{chp:ssp}
%\chapter{Stochastic processing on R\raisebox{0.5ex}{N}}\label{chp:ssp}
%\chapter[SSP on R\raisebox{0.65ex}{N}]{Symbolic sequence processing on R\raisebox{0.65ex}{N}}\label{chp:ssp}
\chapter{Symbolic sequence processing on R\raisebox{0.65ex}{N}}\label{chp:ssp}
\qboxnps
  {\href{http://en.wikipedia.org/wiki/Aristotle}{Aristotle}
   \href{http://www-history.mcs.st-andrews.ac.uk/Timelines/TimelineA.html}{384 BC -- 322 BC},
   \href{http://www-history.mcs.st-andrews.ac.uk/BirthplaceMaps/Places/Greece.html}{Greek philosopher}
   \index{Aristotle}
   \index{quotes!Aristotle}
   \footnotemark
  }
  {../common/people/small/aristotle.jpg}
  {\ldots those who assert that the mathematical sciences
     say nothing of the beautiful or the good are in error.
     For these sciences say and prove a great deal about them;
     if they do not expressly mention them, but prove attributes
     which are their results or definitions, it is not true that they tell
     us nothing about them.
     The chief forms of beauty are order and symmetry and definiteness,
     which the mathematical sciences demonstrate in a special degree.}
  \citetblt{
    %quote:  \citerc{aristotle}{paragraphs 34--35?} \\
    quote:  \citerc{aristotle_metaphysics}{Book XIII Part 3} \\
    %        \url{http://en.wikiquote.org/wiki/Aristotle} \\
    %image:  \url{http://en.wikipedia.org/wiki/Aristotle}
    image:  \url{http://upload.wikimedia.org/wikipedia/commons/9/98/Sanzio_01_Plato_Aristotle.jpg}
    }

 %%============================================================================
% LaTeX File
% Daniel J. Greenhoe
%============================================================================

%=======================================
%\chapter{Introduction to the structure and design series}
\chapter{Introduction}
%\addcontentsline{toc}{section}{Introduction to the structure and design series}
%=======================================

%=======================================
\section*{Mathematics as art}
%=======================================
\qboxnps
  {\href{http://en.wikipedia.org/wiki/Aristotle}{Aristotle}
   \href{http://www-history.mcs.st-andrews.ac.uk/Timelines/TimelineA.html}{384 BC -- 322 BC},
   \href{http://www-history.mcs.st-andrews.ac.uk/BirthplaceMaps/Places/Greece.html}{Greek philosopher}
   \index{Aristotle}
   \index{quotes!Aristotle}
   \footnotemark
  }
  {../common/people/small/aristotle.jpg}
  {\ldots those who assert that the mathematical sciences
     say nothing of the beautiful or the good are in error.
     For these sciences say and prove a great deal about them;
     if they do not expressly mention them, but prove attributes
     which are their results or definitions, it is not true that they tell
     us nothing about them.
     The chief forms of beauty are order and symmetry and definiteness,
     which the mathematical sciences demonstrate in a special degree.}
  \citetblt{
    %quote: & \citerc{aristotle}{paragraphs 34--35?} \\
    quote: & \citerc{aristotle_metaphysics}{Book XIII Part 3} \\
    %       & \url{http://en.wikiquote.org/wiki/Aristotle} \\
    %image: & \url{http://en.wikipedia.org/wiki/Aristotle}
    image: & \url{http://upload.wikimedia.org/wikipedia/commons/9/98/Sanzio_01_Plato_Aristotle.jpg}
    }

\qboxnpq
  {\href{http://en.wikipedia.org/wiki/Norbert_Wiener}{Norbert Wiener}
   \href{http://www-history.mcs.st-andrews.ac.uk/Timelines/TimelineG.html}{(1894--1964)},
   \href{http://www-history.mcs.st-andrews.ac.uk/BirthplaceMaps/Places/USA.html}{American mathematician}
    \index{Wiener, Norbert}
    \index{quotes!Wiener, Norbert}
    \footnotemark
  }
  {../common/people/small/wiener.jpg}
  {The musician regarded me as heavy-handed and Philistine.
   This was partly because of my actual social ineptitude and bad manners,
   but it was also due to the fact that he considered that mathematics
   by its own nature stood in direct opposition to the arts.
   On the other hand, I maintained the thesis of this book:
   that mathematics is essentially one of the arts;}
   %and I dingdonged on this theme far too much for the patience of a man
   %initially disposed to hate mathematics for its own sake.
   %Later on we got into an explicit quarrel,
   %in which we really said the unpleasant things we thought of one another,
   %and this finally cleared up into a certain degree of understanding
   %and even of a limited friendship.}
  \citetblt{
    quote: & \citerp{wiener}{65} \\
    image: & \url{http://www-history.mcs.st-andrews.ac.uk/PictDisplay/Wiener_Norbert.html}
    }


\qboxnps
  {\href{http://en.wikipedia.org/wiki/G._H._Hardy}{G.H. Hardy}
   \href{http://www-history.mcs.st-andrews.ac.uk/Timelines/TimelineG.html}{(1877--1947)},
   \href{http://www-history.mcs.st-andrews.ac.uk/BirthplaceMaps/Places/UK.html}{English mathematician}
    \index{Hardy, G.H.}
    \index{quotes!Hardy, G.H.}
    \footnotemark
  }
  {../common/people/small/hardy.jpg}
  {A mathematician, like a painter or a poet, is a maker of patterns. 
   If his patterns are more permanent than theirs, 
   it is because they are made with \emph{ideas}.}
  \citetblt{
    quote: & \citerc{hardy1940}{section 10} \\
    image: & \url{http://www-history.mcs.st-andrews.ac.uk/PictDisplay/Hardy.html}
    }




Just as Plato pointed out that rhetoric is
one of the arts\citetblt{\citerp{plato1875}{320}},
so Norbert Wiener pointed out that mathematics is also
one of the arts.
And as in all the arts, value is not primarily measured by
the measure of utility in real world applications,
but by the measure of beauty that it helps create.



%=======================================
\section*{Structure of mathematics}
%=======================================
\qboxnpqt
  {Ren\'e Descartes, philosopher and mathematician (1596--1650)
   \index{Descartes, Ren\'e}
   \index{quotes!Descartes, Ren\'e}
   \footnotemark}
  {../common/people/small/descart.jpg}
  {Je me plaisois surtout aux math\'ematiques,
    \`a cause de la certitude et de l'\'evidence de leurs raisons:
    mais je ne remarquois point encore leur vrai usage;
    et, pensant qu'elles ne servoient qu'aux arts m\'ecaniques,
    je m'\'etonnois de ce que leurs fondements \'etant si fermes et si solides,
    on n'avoit rien b\^ati dessus de plus relev\'e:}
  {I was especially delighted with the mathematics,
    on account of the certitude and evidence of their reasonings;
    but I had not as yet a precise knowledge of their true use;
    and thinking that they but contributed to the advancement of the mechanical arts,
    I was astonished that foundations, so strong and solid,
    should have had no loftier superstructure reared on them.}
  \citetblt{
    quote: & \citer{descartes_method} \\
    translation: & \citerc{descartes_method_eng}{part I, paragraph 10} \\
    image: & \url{http://en.wikipedia.org/wiki/Image:Descartes_Discourse_on_Method.png}
    }

\qboxnpqt
  { Jules Henri Poincar\'e (1854-1912), physicist and mathematician
    \index{Poincar\'e, Jules Henri}
    \index{quotes!Poincar\'e, Jules Henri}
    \footnotemark
  }
  {../common/people/small/poincare.jpg}
  {\ldots on fait la science avec des faits comme une maison avec des pierres ; 
   mais une accumulation de faits n'est pas plus une science qu'un tas de 
   pierres n'est une maison.}
  {Science is built up of facts, as a house is built of stones;
   but an accumulation of facts is no more a science than a heap of stones is a house.}
  \citetblt{
    quote:       & \citerc{poincare_sah}{Chapter IX, paragraph 7} \\
    translation: & \citerp{poincare_sah_eng}{141} \\
    image:       & \url{http://www-groups.dcs.st-and.ac.uk/~history/PictDisplay/Poincare.html}
    }


\qboxnps
  {
    Freeman Dyson (1923--2020), physicist and mathematician  %(January 1994)
    \index{Dyson, Freeman}
    \index{quotes!Dyson, Freeman}
    \footnotemark
  }
  {../common/people/small/dyson.jpg}
  {The bottom line for mathematicians is that the architecture has to be right.
    In all the mathematics that I did, the essential point was to find
    the right architecture.
    It's like building a bridge.
    Once the main lines of the structure are right,
    then the details miraculously fit.
    The problem is the overall design.}
  \citetblt{
    quote: & \citerp{dyson1994}{20}  \\
    image: & \url{http://en.wikipedia.org/wiki/Image:FreemanDysonOSCON2004.jpg}
    }

Just as in paintings and architectural works, 
the art that is mathematics is often demonstrated in the structure inherent in it.
This text tries to show some of that structure.

%=======================================
\section*{Connected concepts}
%=======================================

\qboxnpqt
  { \href{http://en.wikipedia.org/wiki/Joseph_Louis_Lagrange}{Joseph-Louis Lagrange}
    (\href{http://www-history.mcs.st-andrews.ac.uk/Timelines/TimelineD.html}{1736--1813},
     \href{http://www-history.mcs.st-andrews.ac.uk/BirthplaceMaps/Places/Italy.html}{Italian-French mathematician and astronomer}
    \index{Langrange, Joseph-Louis}
    \index{quotes!Langrange, Joseph-Louis}
    \footnotemark
  }
  {../common/people/small/lagrange.jpg}
  {Tant que l'Alg\`ebre et la G\'eom\'etrie ont \'et\'e s\'epar\'ees,
   leurs progr\`es ont \'et\'e lents et leurs usages born\'es;
   mais lorsque ces deux sciences se sont r\'eunies, elles vers la perfection.}
  {As long as algebra and geometry have been separated,
   their progress have been slow and their uses limited;
   but when these two sciences have been united, they have lent each other mutual forces,
   and have marched together with a rapid step towards perfection.
  }
  \citetblt{
    quote:       & \citerp{lagrange1795}{271} \\
    translation: & \citerpg{grattan1990}{254}{3764322373} \\
%                 & \url{http://www-groups.dcs.st-and.ac.uk/~history/Quotations/Lagrange.html} \\
    image:       & \url{http://en.wikipedia.org/wiki/Joseph_Louis_Lagrange}
    }


\qboxnps
  {\href{http://en.wikipedia.org/wiki/G._H._Hardy}{G.H. Hardy}
   \href{http://www-history.mcs.st-andrews.ac.uk/Timelines/TimelineG.html}{(1877--1947)},
   \href{http://www-history.mcs.st-andrews.ac.uk/BirthplaceMaps/Places/UK.html}{English mathematician}
    \index{Hardy, G.H.}
    \index{quotes!Hardy, G.H.}
    \footnotemark
  }
  {../common/people/small/hardy.jpg}
  {The ``seriousness" of a mathematical theorem lies,
    not in its practical consequences,
    which are usually negligible,
    but in the {\em significance} of the mathematical ideas which it connects.
    We may say, roughly, that a mathematical idea is ``significant" if it can be
    connected, in a natural illuminating way,
    with a large complex of other mathematical ideas.}
  \citetblt{
    quote: & \citerc{hardy1940}{section 11} \\
    image: & \url{http://www-history.mcs.st-andrews.ac.uk/PictDisplay/Hardy.html}
    }


Mathematics is not simply a collection of definitions and equations.
Rather it is a carefully built \emph{structure} of \emph{connected} concepts.
And the purpose of this text is to present the foundations of mathematics
in a {\em structured} and {\em connected} manner.

That is, we start with the most basic and fundamental concepts,
and build upon them other structures.
But critical to any structure is the connections between components.
As pointed out by Hardy, the value of a mathematical concept is not primarily in
it's applicability to real world problems,
but rather in how well it is connected too and helps connect other mathematical concepts.

%=======================================
\section*{Abstraction}
%=======================================
\qboxnps
  { attributed to Karl Gustav Jakob Jacobi (1804--1851), mathematician
    \index{Jacobi, Karl Gustav Jakob}
    \index{quotes!Jacobi, Karl Gustav Jakob}
    \footnotemark
  }
  {../common/people/small/jacobi.jpg}
  {Man muss immer generalisieren.\quotec \\
   \quoteo One should always generalize.}
  \citetblt{
    %quote: & \url{http://en.wikiquote.org/wiki/Gustav_Jacobi} \\
     quote: & \citerpg{davis1999}{134}{0395929687} \\
     image: & \url{http://en.wikipedia.org/wiki/Carl_Gustav_Jacobi}
    }

\qboxnpqt
  { Jules Henri Poincar\'e (1854-1912), physicist and mathematician
    \index{Poincar\'e, Jules Henri}
    \index{quotes!Poincar\'e, Jules Henri}
    \footnotemark
  }
  {../common/people/small/poincare.jpg}
  {Les math\'ematiciens n'\'etudient pas des objets, 
   mais des relations entre les objets; 
   il leur est donc indiff\'erent de remplacer ces objets par d'autres, 
   pourvu que les relations ne changent pas. 
   La mati\`ere ne leur importe pas, la forme seule les int\'eresse.}
  {Mathematicians do not study objects, but the relations between objects;
   to them it is a matter of indifference if these objects are replaced by others,
   provided that the relations do not change.
   Matter does not engage their attention,
   they are interested in form alone.}
  \citetblt{
    quote:       & \citerc{poincare_sah}{Chapter 2} \\
    translation: & \citerp{poincare_sah_eng}{20} \\
    %image:       & \url{http://en.wikipedia.org/wiki/Image:Poincare_jh.jpg}
    image:       & \url{http://www-groups.dcs.st-and.ac.uk/~history/PictDisplay/Poincare.html}
    }

\qboxnqt
  {
    Jules Henri Poincar\'e (1854-1912), physicist and mathematician
    \index{Poincar\'e, Jules Henri}
    \index{quotes!Poincar\'e, Jules Henri}
    \footnotemark
  }
  %{../common/people/small/poincare.jpg}
  {Je ne sais si je n'ai d\'ej\`a` dit quelque part que la Math\'ematique est 
  l'art de donner le m\^eme nom \`a des choses diff\'erentes. 
  Il convient que ces choses, diff\'erentes par la mati\`ere, 
  soient semblables par la forme, qu'elles puissent, 
  pour ainsi dire, se couler dans le m\^eme moule. 
  Quand le langage a \'et\'e bien choisi, on est tout \'etonn\'e 
  de voir que toutes les d\'emonstrations, faites pour un objet connu, 
  s'appliquent imm\'ediatement \`a beaucoup d'objets nouveaux ; 
  on n'a rien \`a y changer, pas m\^eme les mots, puisque les noms sont devenus les m\^emes.}
  %
  {I think I have already said somewhere that mathematics is the art
   of giving the same name to different things. 
   It is enough that these things, though differing in matter, 
   should be similar in form, to permit of their being, so to speak,
   run in the same mould.
   When language has been well chosen, one is astonished to find that all
   demonstrations made for a known object apply immediately to many new objects:
   nothing requires to be changed, not even the terms,
   since the names have become the same.}
  \citetblt{
    quote:   & \citerc{poincare_sam}{book 1, chapter 2, paragraph 20} \\
             & \url{http://fr.wikisource.org/wiki/Science_et_m\%C3\%A9thode_-_Livre_premier\%2C_\%C2\%A7_II} \\
    trans.:  & \citerp{poincare_sam_eng}{34} \\
   %image:        & \url{http://en.wikipedia.org/wiki/Image:Poincare_jh.jpg}
    }

The twentieth century was the century of abstraction in mathematics.%
\footnote{This concept of generalization so prevelant in 
twentieth century mathematics was well described by Van de Vel:
``It is typical of an axiomatic approach not to emphasize what an object \emph{\bf represents},
but rather how it \emph{\bf behaves}."---\citerp{vel1993}{3}
}
Here are some examples:\citetbl{
  \citor{peano1888} \\
  \citor{weber1893} \\
  \citor{dedekind1900} \\
  \citor{frechet1906} \\
  \citor{hausdorff1914} \\
  \citor{banach1922} \\
  \citor{vonNeumann1929} \\
  \citor{birkhoff1948}
  }
\begin{liste}
  \item In 1888, Giuseppe Peano introduced the \hie{vector space}, 
        a generalization of real functions.

  \item In 1893, Heinrich Weber introduced the algebraic \hie{field}, 
        a generalization of the real numbers and associated arithmetic.

  \item In 1900, Richard Dedekind introduced \hie{modularity}, a generalization of
        \prop{distributivity}.

  \item In 1906, Maurice Ren\'e Fr\'echet
        introduced the concepts of the \hie{abstract space} as well as the 
        \hie{metric space}, which generalize concepts in analysis involving 
        convergence.

  \item In 1914, Felix Hausdorff introduced the modern concept of the 
        \hie{topological space}, a further generalization of the 
        already very general metric space.

  \item In 1922, Stephen Banach introduced the \hie{normed linear space},
        a generalization of operations involving integration.

  \item In 1929, John von Neumann introduced the \hie{Hilbert space}, a 
        generalization of earlier work by Hilbert involving integral operations.

  \item In 1948, Garrett Birkhoff introduced the \hie{lattice},
        a generalization involving partially ordered sets.
\end{liste}

This text presents the most general and abstract structures first,
followed by progressively more specific structures.
The idea is, that we try to prove as much as possible in the most general setting;
and then we only need to prove that a specific structure is a special case of the general
structure and hence can simply \hie{inherit} the properties of that general structure
without having to prove everything all over again.


%=======================================
\section*{Concrete and specific cases}
%=======================================

\qboxnps
  {\href{http://en.wikipedia.org/wiki/Paul_Halmos}{Paul R. Halmos}
   \href{http://www-history.mcs.st-andrews.ac.uk/Timelines/TimelineG.html}{(1916--2006)},
   \href{http://www-history.mcs.st-andrews.ac.uk/BirthplaceMaps/Places/Germany.html}{Hungarian-born American mathematician}
   \index{Halmos, Paul R.}
   \index{quotes!Halmos, Paul R.}
   \footnotemark
  }
  {../common/people/small/halmos.jpg}
  {\ldots the source of all great mathematics is the special case,
    the concrete example.
    It is frequent in mathematics that every instance of a concept of seemingly
    great generality is in essence the same as a small and concrete special case.}
  \citetblt{
    quote: & \citer{halmos1985} \\
    image: & \url{http://en.wikipedia.org/wiki/Image:Paul_Halmos.jpeg}
    }



\qboxnps
  {attributed to \href{http://en.wikipedia.org/wiki/Hilbert}{David Hilbert}
   \href{http://www-history.mcs.st-andrews.ac.uk/Timelines/TimelineF.html}{(1862--1943)},
   \href{http://www-history.mcs.st-andrews.ac.uk/BirthplaceMaps/Places/Germany.html}{German mathematician}
    \index{Hilbert, David}
    \index{quotes!Hilbert, David}
    \footnotemark
  }
  {../common/people/small/hilbert.jpg}
  {The art of doing mathematics consists in finding
    that special case which contains all the germs of generality.}
  \citetblt{
    quote: & \citer{rose1988} \\
    image: & \url{http://en.wikipedia.org/wiki/Image:Hilbert.JPG}
    }

%2014jan05sunday
%Hermann Weyl:
%    ``Nevertheless I should not pass over in silence the fact that today the feeling among mathematicians is beginning to spread that the fertility of these abstracting methods is approaching exhaustion. The case is this: that all these nice general concepts do not fall into our laps by themselves. But definite concrete problems were first conquered in their undivided complexity, singlehanded by brute force, so to speak. Only afterwards the axiomaticians came along and stated: Instead of breaking the door with all your might and bruising your hands, you should have constructed such and such a key of skill, and by it you would have been able to open the door quite smoothly. But they can construct the key only because they are able, after the breaking in was successful, to study the lock from within and without. Before you can generalize, formalize, and axiomatize, there must be a mathematical substance."
%Pioneers of Representation Theory: Frobenius, Burnside, Schur, and Brauer, by Charles W. Curtis, pg 210.
%http://www.plambeck.org/archives/001272.html

\qboxnpq
  {
    Hermann Weyl (1885--1955); mathematician, theoretical physicist, and philosopher
    \index{Weyl, Hermann}
    \index{quotes!Weyl, Hermann}
    \footnotemark
  }
  {../common/people/weylhermann_wkp_free.jpg}
  {Nevertheless I should not pass over in silence the fact that today the 
   feeling among mathematicians is beginning to spread that the fertility 
   of these abstracting methods is approaching exhaustion. 
   The case is this: that all these nice general concepts do not fall into our laps by themselves. 
   But definite concrete problems were first conquered in their undivided complexity, 
   singlehanded by brute force, so to speak. Only afterwards the axiomaticians came along and stated: 
   Instead of breaking the door with all your might and bruising your hands, 
   you should have constructed such and such a key of skill, 
   and by it you would have been able to open the door quite smoothly. 
   But they can construct the key only because they are able, after the breaking in was successful, 
   to study the lock from within and without. 
   Before you can generalize, formalize, and axiomatize, there must be a mathematical substance.}
  \footnotetext{
    quote: \citePpc{weyl1935}{14}{H. Weyl, quoting himself in ``a conference on topology and abstract algebra as two ways of mathematical understanding, in 1931"}. 
    image: \url{https://en.wikipedia.org/wiki/File:Hermann_Weyl_ETH-Bib_Portr_00890.jpg}: ``This work is free and may be used by anyone for any purpose."
    }

It is general abstract concepts that allows us to easily see the connectivity
between multiple mathematical ideas and to more easily develop new ideas.
However, it is concrete specific examples that often make a general abstract concept clear
and more easily remembered in the future; and it may be argued that it is the concrete
examples that give the general abstract concepts meaning and significance.
This text provides many concrete examples that will hopefully greatly clarify
the more general concepts;
and will hopefully also help give value to the abstract concepts.

%=======================================
\section*{Applications}
%=======================================

\qboxnpq
  {
    Joseph Louis Lagrange (1736--1813), mathematician
    \index{Lagrange, Joseph Louis}
    \index{quotes!Lagrange, Joseph Louis}
    \footnotemark
  }
  {../common/people/small/lagrange.jpg}
  {I regard as quite useless the reading of large treatises of pure analysis:
    too large a number of methods pass at once before the eyes.
    It is in the works of applications that one must study them;
    one judges their ability there and one apprises the manner of making use of them.}
  \citetblt{
    quote: &  \citerp{stopple2003}{xi} \\
          %&  \url{http://www.math.okstate.edu/~wli/teach/fmq.html} \\
          %&  \url{http://www-groups.dcs.st-and.ac.uk/~history/Quotations/Lagrange.html} \\
    image: & \url{http://en.wikipedia.org/wiki/Image:Langrange_portrait.jpg}
    }

\qboxnps
  {
    \href{http://en.wikipedia.org/wiki/Eric_temple_bell}{Eric Temple Bell}
    (1883--1960), Scotish-American mathematician
    \index{Bell, Eric Temple}
    \index{quotes!Bell, Eric Temple}
    \footnotemark
  }
  {../common/people/small/bell.jpg}
  {The pursuit of pretty formulas and neat theorems
    can no doubt quickly degenerate into a silly vice,
    but so can the quest for austere generalities which are so very general indeed
    that they are incapable of application to any particular.}
  \citetblt{
    %quote: & \citer{eves1972} \\
    quote: & \citerp{bell1986}{488} \\
    image: & \url{http://www-history.mcs.st-andrews.ac.uk/PictDisplay/Bell.html}
    }



\begin{minipage}{4\tw/16}%
  \includegraphics*[width=\tw, keepaspectratio=true, clip=true]
  {../common/people/small/plimpton322.jpg}\footnotemark
\end{minipage}%
\footnotetext{\hie{Plimpton 322}: One of the most famous mathematical tablets from Ancient Babylon.
  Image source: \url{http://en.wikipedia.org/wiki/Plimpton_322}
  }%
\hfill
\begin{minipage}{10\tw/16}%
  Mathematics is a very old art form.
  Archeologists have found thousands of mathematical tablets among the ruins of the ancient
  Babylonians, buried in the sand for 4000 years but still readable (by some) today.
  Mathematics has for centuries been held in esteem because of its ability to
  solve practical problems.
\end{minipage}
Before the century of generalization---the twentieth century---and the 
century of precision---the nineteenth century---came several centuries of 
applications.
In fact, mathematics came to enjoy one of its finest hours during the
European Renaissance.
During this time, mathematics was brought to bear by so many to solve so many
extremely practical problems.
One of the most famous examples is Isaac \prop{Newton},
who demonstrated that the motion of objects
as small as an apple to as large as a planet could be accurately expressed and
(more importantly) accurately predicted by the analytical power of the calculus.
\hie{Peter the Great} of Russia, even though he himself did not
know too much about mathematics,
kept \prop{Euler} as a member of his court to work on the solution of very
practical problems. Peter the Great even referred to Euler,
at least in the beginning, as {\em My Professor}
(later, as the relationship became strained, he referred to Euler as
{\em My Cyclops}---Euler only had one good eye at the time).


\begin{figure}
\color{figcolor}
\begin{center}
\begin{fsL}
%============================================================================
% LaTeX File
% Daniel J. Greenhoe
% graphic of number of noteable mathematicians over time
%============================================================================
\psset{xunit=0.005\psxunit}%
\psset{yunit=0.003\psyunit}%
\begin{pspicture}(-1100,-200)(2100,900)
  %-------------------------------------
  % axes
  %-------------------------------------
  \psaxes[linecolor=axis,linewidth=0.75pt,yAxis=false,ticks=none,labelsep=2pt,labels=none]{<->}(0,0)(-1100,-10)(2100,750)% x-axis
  \psaxes[linecolor=axis,linewidth=0.75pt,xAxis=false,ticks=none,labelsep=2pt,labels=none]{->}(0,0)(-1100,-10)(2100,750)% y-axis
  %-------------------------------------
  % data plot
  %-------------------------------------
  \psset{linecolor=blue}%
  \rput(-1000, 0){\psline{-o}(0,0)(0,   0)}%
  \rput( -950, 0){\psline{-o}(0,0)(0,   0)}%
  \rput( -900, 0){\psline{-o}(0,0)(0,   0)}%
  \rput( -850, 0){\psline{-o}(0,0)(0,   0)}%
  \rput( -800, 0){\psline{-o}(0,0)(0,   1)}%
  \rput( -750, 0){\psline{-o}(0,0)(0,   2)}%
  \rput( -700, 0){\psline{-o}(0,0)(0,   1)}%
  \rput( -650, 0){\psline{-o}(0,0)(0,   0)}%
  \rput( -600, 0){\psline{-o}(0,0)(0,   2)}%
  \rput( -550, 0){\psline{-o}(0,0)(0,   3)}%
  \rput( -500, 0){\psline{-o}(0,0)(0,   2)}%
  \rput( -450, 0){\psline{-o}(0,0)(0,  11)}%
  \rput( -400, 0){\psline{-o}(0,0)(0,   9)}%
  \rput( -350, 0){\psline{-o}(0,0)(0,  12)}%
  \rput( -300, 0){\psline{-o}(0,0)(0,   5)}%
  \rput( -250, 0){\psline{-o}(0,0)(0,   9)}%
  \rput( -200, 0){\psline{-o}(0,0)(0,   6)}%
  \rput( -150, 0){\psline{-o}(0,0)(0,   7)}%
  \rput( -100, 0){\psline{-o}(0,0)(0,   3)}%
  \rput(  -50, 0){\psline{-o}(0,0)(0,   0)}%
  \rput(    0, 0){\psline{-o}(0,0)(0,   1)}%
  \rput(   50, 0){\psline{-o}(0,0)(0,   3)}%
  \rput(  100, 0){\psline{-o}(0,0)(0,   5)}%
  \rput(  150, 0){\psline{-o}(0,0)(0,   3)}%
  \rput(  200, 0){\psline{-o}(0,0)(0,   3)}%
  \rput(  250, 0){\psline{-o}(0,0)(0,   4)}%
  \rput(  300, 0){\psline{-o}(0,0)(0,   4)}%
  \rput(  350, 0){\psline{-o}(0,0)(0,   3)}%
  \rput(  400, 0){\psline{-o}(0,0)(0,   4)}%
  \rput(  450, 0){\psline{-o}(0,0)(0,   8)}%
  \rput(  500, 0){\psline{-o}(0,0)(0,   9)}%
  \rput(  550, 0){\psline{-o}(0,0)(0,   4)}%
  \rput(  600, 0){\psline{-o}(0,0)(0,   3)}%
  \rput(  650, 0){\psline{-o}(0,0)(0,   3)}%
  \rput(  700, 0){\psline{-o}(0,0)(0,   0)}%
  \rput(  750, 0){\psline{-o}(0,0)(0,   2)}%
  \rput(  800, 0){\psline{-o}(0,0)(0,   7)}%
  \rput(  850, 0){\psline{-o}(0,0)(0,  15)}%
  \rput(  900, 0){\psline{-o}(0,0)(0,   9)}%
  \rput(  950, 0){\psline{-o}(0,0)(0,   9)}%
  \rput( 1000, 0){\psline{-o}(0,0)(0,  14)}%
  \rput( 1050, 0){\psline{-o}(0,0)(0,   7)}%
  \rput( 1100, 0){\psline{-o}(0,0)(0,   7)}%
  \rput( 1150, 0){\psline{-o}(0,0)(0,   8)}%
  \rput( 1200, 0){\psline{-o}(0,0)(0,   6)}%
  \rput( 1250, 0){\psline{-o}(0,0)(0,  16)}%
  \rput( 1300, 0){\psline{-o}(0,0)(0,  10)}%
  \rput( 1350, 0){\psline{-o}(0,0)(0,   9)}%
  \rput( 1400, 0){\psline{-o}(0,0)(0,  13)}%
  \rput( 1450, 0){\psline{-o}(0,0)(0,  21)}%
  \rput( 1500, 0){\psline{-o}(0,0)(0,  40)}%
  \rput( 1550, 0){\psline{-o}(0,0)(0,  40)}%
  \rput( 1600, 0){\psline{-o}(0,0)(0,  59)}%
  \rput( 1650, 0){\psline{-o}(0,0)(0,  81)}%
  \rput( 1700, 0){\psline{-o}(0,0)(0,  74)}%
  \rput( 1750, 0){\psline{-o}(0,0)(0,  99)}%
  \rput( 1800, 0){\psline{-o}(0,0)(0, 144)}%
  \rput( 1850, 0){\psline{-o}(0,0)(0, 302)}%
  \rput( 1900, 0){\psline{-o}(0,0)(0, 614)}%
  \rput( 1950, 0){\psline{-o}(0,0)(0, 653)}%
  \rput( 1975, 0){\psline{-o}(0,0)(0, 435)}%
  \rput( 2000, 0){\psline{-o}(0,0)(0, 154)}%
  \rput( 2006, 0){\psline{-o}(0,0)(0, 112)}%
       %    |                       |_____ number of notable mathematicians
       %    |_____________________________ year
  %-------------------------------------
  % number of mathematician labels (above lollipops)
  %-------------------------------------
  \scriptsize%
  \uput[90](-1000,   0){$  0$}%
  \uput[90]( -950,   0){$  0$}%
  \uput[90]( -900,   0){$  0$}%
  \uput[90]( -850,   0){$  0$}%
  \uput[90]( -800,   1){$  1$}%
  \uput[90]( -750,   2){$  2$}%
  \uput[90]( -700,   1){$  1$}%
  \uput[90]( -650,   0){$  0$}%
  \uput[90]( -600,   2){$  2$}%
  \uput[90]( -550,   3){$  3$}%
  \uput[90]( -500,   2){$  2$}%
  \uput[90]( -450,  11){$ 11$}%
  \uput[90]( -400,   9){$  9$}%
  \uput[90]( -350,  12){$ 12$}%
  \uput[90]( -300,   5){$  5$}%
  \uput[90]( -250,   9){$  9$}%
  \uput[90]( -200,   6){$  6$}%
  \uput[90]( -150,   7){$  7$}%
  \uput[90]( -100,   3){$  3$}%
  \uput[90](  -50,   0){$  0$}%
  \uput[90](    0,   1){$  1$}%
  \uput[90](   50,   3){$  3$}%
  \uput[90](  100,   5){$  5$}%
  \uput[90](  150,   3){$  3$}%
  \uput[90](  200,   3){$  3$}%
  \uput[90](  250,   4){$  4$}%
  \uput[90](  300,   4){$  4$}%
  \uput[90](  350,   3){$  3$}%
  \uput[90](  400,   4){$  4$}%
  \uput[90](  450,   8){$  8$}%
  \uput[90](  500,   9){$  9$}%
  \uput[90](  550,   4){$  4$}%
  \uput[90](  600,   3){$  3$}%
  \uput[90](  650,   3){$  3$}%
  \uput[90](  700,   0){$  0$}%
  \uput[90](  750,   2){$  2$}%
  \uput[90](  800,   7){$  7$}%
  \uput[90](  850,  15){$ 15$}%
  \uput[90](  900,   9){$  9$}%
  \uput[90](  950,   9){$  9$}%
  \uput[90]( 1000,  14){$ 14$}%
  \uput[90]( 1050,   7){$  7$}%
  \uput[90]( 1100,   7){$  7$}%
  \uput[90]( 1150,   8){$  8$}%
  \uput[90]( 1200,   6){$  6$}%
  \uput[90]( 1250,  16){$ 16$}%
  \uput[90]( 1300,  10){$ 10$}%
  \uput[90]( 1350,   9){$  9$}%
  \uput[90]( 1400,  13){$ 13$}%
  \uput[90]( 1450,  21){$ 21$}%
  \uput[90]( 1500,  40){$ 40$}%
  \uput[90]( 1550,  40){$ 40$}%
  \uput[90]( 1600,  59){$ 59$}%
  \uput[90]( 1650,  81){$ 81$}%
  \uput[90]( 1700,  74){$ 74$}%
  \uput[90]( 1750,  99){$ 99$}%
  \uput[90]( 1800, 144){$144$}%
  \uput[90]( 1850, 302){$302$}%
  \uput[135]( 1900, 614){$614$}%
  \uput[45]( 1950, 653){$653$}%
  \uput[45]( 1975, 435){$435$}%
  \uput[60]( 2000, 154){$154$}%
  \uput[45]( 2006, 112){$112$}%
       % |      |    |     |____ number of notable mathematicians
       % |      |    |__________ number of notable mathematicians
       % |      |_______________ year
       % |______________________ place label above lollipops
  %-------------------------------------
  % year labels (under y-axis)
  %-------------------------------------
  \uput[-90](-1000,0){$-1000$}%
  \uput[-90]( -350,0){$ -350$}%
  \uput[-90](    0,0){$    0$}%
  \uput[-90]( 1000,0){$ 1000$}%
  \uput[-90]( 1450,0){$ 1450$}%
  \uput[-90]( 1600,0){$ 1600$}%
  \uput[-90]( 1800,0){$ 1800$}%
  \uput[-90]( 2000,0){$ 2000$}%
  \uput[-90]( 1600,0){$ 1600$}%
    %
%    \rput(-1000,30){\makebox(0,0)[b] {0}}%
%    \rput(-350,42){\makebox(0,0)[b] {12}}%
%    \rput(0,31){\makebox(0,0)[b] {1}}%
%    \rput(1000,44){\makebox(0,0)[b] {14}}%
%    \rput(1450,51){\makebox(0,0)[b] {21}}%
%    \rput(1600,89){\makebox(0,0)[b] {59}}%
%    \rput(1800,174){\makebox(0,0)[b] {144}}%
%    \rput(1850,342){\makebox(0,0)[b] {302}}%
%    \rput(1950,683){\makebox(0,0)[b] {653}}%
%    \rput(1975,465){\makebox(0,0)[bl] {435}}%
%    \rput(2000,184){\makebox(0,0)[bl] {154}}%
%    \rput(2006,122){\makebox(0,0)[bl] {112}}%
%  %
  %-------------------------------------
  % inventions
  %-------------------------------------
  \color{red}%
  \psset{linecolor=red,linewidth=0.75pt}%
 %\rput[r]( 980,800){\blank[10mm] available}%
  \rput[r]( 980,800){IBM PC}%
 %\rput[r]( 980,700){\blank[10mm] available}%
  \rput[r]( 980,700){calculator}%
  \rput[r]( 980,600){second industrial revolution}%
  \rput[r]( 980,500){industrial revolution}%
  \rput[r]( 980,400){slide rule invented}%
 %\rput[r]( 980,400){\blank[10mm] invented}%
 %\rput[r]( 980,300){\blank[10mm] invented}%
  \rput[r]( 980,300){Gutenberg printing press invented}%
%
  \psline[linestyle=dashed](980,800)(1980,800)(1980,500)%
  \psline[linestyle=dashed](980,700)(1970,700)(1970,600)%
  \psline[linestyle=dashed](980,600)(1850,600)(1850,500)%
  \psline[linestyle=dashed](980,500)(1800,500)(1800,400)%
  \psline[linestyle=dashed](980,400)(1625,400)(1625,300)%
  \psline[linestyle=dashed](980,300)(1450,300)(1450,200)%
%
%  \rput( 1980,800){\psline(0,-1){50}} \rput(1000,800){\psline(1,0){980} }  \rput( 980,800){\makebox(0,0)[r] {IBM PC}}%
%  \rput( 1970,700){\psline(0,-1){50}} \rput(1000,700){\psline(1,0){970} }  \rput( 980,700){\makebox(0,0)[r] {calculator}}%
%  \rput( 1850,600){\psline(0,-1){50}} \rput(1000,600){\psline(1,0){850} }  \rput( 980,600){\makebox(0,0)[r] {second industrial revolution}}%
%  \rput( 1800,500){\psline(0,-1){50}} \rput(1000,500){\psline(1,0){800} }  \rput( 980,500){\makebox(0,0)[r] {industrial revolution}}%
%  \rput( 1625,400){\psline(0,-1){50}} \rput(1000,400){\psline(1,0){625} }  \rput( 980,400){\makebox(0,0)[r] {slide rule invented}}%
%  \rput( 1450,300){\psline(0,-1){50}} \rput(1000,300){\psline(1,0){450} }  \rput( 980,300){\makebox(0,0)[r] {Gutenberg printing press invented}}%
  %-------------------------------------
  % time periods
  %-------------------------------------
  \rput( 400, -50){\psline(0,0)(0,-200)}% 
  \rput(1300, -50){\psline(0,0)(0,-200)}% 
  \rput(1650, -50){\psline(0,0)(0,-200)}% 
  \rput( 850,-150){Middle Ages / ``Dark Ages"}%
  \rput(1475,-150){Renaissance}%
\end{pspicture}%

\end{fsL}
\end{center}
\caption{
   Number of notable mathematicians alive over time
   \label{fig:intro_timeline}
   }
\end{figure}

But it is not that applications are no longer important to mathematics,
but rather applications drive mathematics forward.
Evidence of this hypothesis is given in 
\prefpp{fig:intro_timeline}.\footnote{Data for \prefpp{fig:intro_timeline} extracted from \\
  \url{http://www-history.mcs.st-andrews.ac.uk/Timelines/WhoWasThere.html}}
This graph shows the number of ``notable" mathematicians alive during the last 3000 years;
And a ``notable mathematician" is here defined as one whose name appears
in \hie{Saint Andrew's University's} \hie{Who Was There} 
website.\footnote{\url{http://www-history.mcs.st-andrews.ac.uk/Timelines/WhoWasThere.html}}
Note the following:
\begin{liste}

\item The number of mathematicians starts to exponentially increase at 
about the time of Gutenberg's invention of the printing press---
that is, when information of discoveries and results could be widely and economically 
circulated.\footnote{This point is also made by Resnikoff and Wells in\\
\citerp{resnikoff}{9}.}

\item There is another increase after the invention of the slide rule in the early 1600s---
that is, when computational power increased.

\item There are huge increases around the time of the first and second industrial revolutions---
that is, when there were many {\bf applications} that called for mathematical solutions.

\item After the invention of the pocket scientific calculator in 1972 
and home IBM PC in 1981---machines that could 
often make hard-core mathmatical analysis unnecessary in real-world applications---
there was a huge drop in the number of mathematicians.

\end{liste}

The point here is, that however little some mathematicians may think of 
real-world applications,
historically it would seem that applications are critical to 
how well mathematics thrives and develops. 
When mathematics is needed for the development of applications, 
mathematics prospers.
But when mathematics is not viewed as critical to those applications
(such as after the introduction of the personal computer), 
mathematics withers.


%=======================================
\section*{Writing style}
%=======================================
\qboxnpqt
  { %\href{http://www-history.mcs.st-andrews.ac.uk/Biographies/Poincare.html}{Jules Henri Poincar\'e} 
    \href{http://en.wikipedia.org/wiki/Henri_Poincar\%C3\%A9}{Jules Henri Poincar\'e} 
    \href{http://www-history.mcs.st-andrews.ac.uk/Timelines/TimelineF.html}{(1854--1912)}, 
    \href{http://www-history.mcs.st-andrews.ac.uk/BirthplaceMaps/Places/France.html}{French physicist and mathematician}
    \index{Poincar\'e, Jules Henri}
    \index{quotes!Poincar\'e, Jules Henri}
    \footnotemark
  }
  {../common/people/small/poincare.jpg}
  {Ainsi, la logique et l'intuition ont chacune leur r\^ole n\'ecessaire.
    Toutes deux sont indispensables.
    La logique qui peut seule donner la certitude est l'instrument de la d\'emonstration:
    l'intuition est l'instrument de l'invention.}
  {Thus, logic and the intuition each have their necessary r\^ole.
    Each is indispensable.
    Logic, which alone can give certainty, is the instrument of demonstration;
    intuition is the instrument of invention.}
  \citetblt{
    quote:       & \citerc{poincare_vos}{chapter 1 \S V}  \\
    translation: & \citerp{poincare_vos_e}{23} \\
    image:       & \url{http://en.wikipedia.org/wiki/Image:Poincare_jh.jpg}
    }

  \paragraph{Dual structure writing style.}
  Mathematician \href{http://en.wikipedia.org/wiki/Steenrod}{Norman E. Steenrod}
  proposed a mathematical style of writing
  in which there is a distinction between
  ``the {\em formal} or {\em logical} structure consisting of definitions, theorems,
  and proofs, and the complementary
  {\em informal} or {\em introductory} material consisting of motivations, analogies,
  examples, and metamathematical explanations."\citep{steenrod}{1}
  This is the style that largely characterizes this text.
  The overall development is in the definition-lemma-theorem style,
  which seems to be a reliable way to impose rigor on the development.
  The motivation and general discussion are in the more informal style,
  which helps give an intuitive understanding.

\qboxnps
  {
    \href{http://en.wikipedia.org/wiki/Carl_Friedrich_Gauss}{Karl Friedrich Gauss}
    (1777--1855), German mathematician
    \index{Gauss, Karl Friedrich}
    \index{quotes!Gauss, Karl Friedrich}
    \footnotemark
  }
  {../common/people/small/gauss.jpg}
  {You know that I write slowly.
    This is chiefly because I am never satisfied until I have said as much as
    possible in a few words,
    and writing briefly takes far more time than writing at length.}
  \citetblt{
    quote: & \citerp{simmons2007}{177} \\
    image: & \url{http://en.wikipedia.org/wiki/Karl_Friedrich_Gauss}
    }


  \paragraph{Informal structure.}
  More specifically, my guidelines for the informal portions of text are as follows:
  \begin{dingautolist}{"C0}
    \item The text should be terse rather than verbose.
    %\item The text should not be cluttered with supporting background
    %      mathematical material.
    %      Most all such material is placed in appendices at the end of the text.
    \item Key points are often indicated by enclosure in boxes---
          red double boxes for definitions, blue single boxes for theorems,
          curved-corner boxes for examples.
          This makes parsing key points faster and more efficient---
          allowing you to easily distinguish between key results, detailed proofs,
          and general discussion.
    \item Examples will be explicitly worked out, giving readers
          confidence that they really understand the material.
  \end{dingautolist}


\qboxnps
  {
    \href{http://en.wikipedia.org/wiki/Carl_Friedrich_Gauss}{Karl Friedrich Gauss}
    (1777--1855), German mathematician
    \index{Gauss, Karl Friedrich}
    \index{quotes!Gauss, Karl Friedrich}
    \footnotemark
  }
  %{../common/people/small/gauss1828.jpg}
  {../common/people/gauss.jpg}
  {I mean the word proof not in the sense of lawyers,
    who set two half proofs equal to a whole one,
    but in the sense of the mathematician, where \textonehalf proof = 0
    and it is demanded for proof that every doubt becomes impossible.}
  \citetblt{
    quote: & \citerp{simmons2007}{177} \\
    %image: & \url{http://www-history.mcs.st-andrews.ac.uk/PictDisplay/Gauss.html}
    image: & \url{http://en.wikipedia.org/wiki/Karl_Friedrich_Gauss}
    }

\qboxnpq
  {
    \href{http://en.wikipedia.org/wiki/Carl_Gustav_Jakob_Jacobi}{Carl Gustav Jacob Jacobi}
    (1804--1851), Jewish-German mathematician
    \index{Jacobi, Carl Gustav Jacob}
    \index{quotes!Jacobi, Carl Gustav Jacob}
    \footnotemark
  }
  {../common/people/small/jacobic.jpg}
  {Dirichlet alone, not I, nor Cauchy, nor Gauss knows what a completely rigorous proof is.
   Rather we learn it first from him.
   When Gauss says he has proved something it is clear;
   when Cauchy says it, one can wager as much pro as con;
   when Dirichlet says it, it is certain.}
  %{Dirichlet alone, not I, nor Cauchy, nor Gauss knows what a completely rigorous proof is,
  % and we are learning it from him.
  % When Gauss says he has proved something, it is very probable to me;
  % when Cauchy says it, it is more likely than not,
  % when Dirichlet says it, it is \emph{proved}.}
  \citetblt{
    quote: & \url{http://lagrange.math.trinity.edu/aholder/misc/quotes.shtml} \\
           & \citerp{schubring2005}{558} \\
          %& \citerp{biermann1988}{46} \\
    image: & \url{http://en.wikipedia.org/wiki/Carl_Gustav_Jakob_Jacobi}
    }

\qboxnps
  {
    \href{http://en.wikipedia.org/wiki/Enrico_Fermi}{Enrico Fermi}
    \href{http://www-history.mcs.st-andrews.ac.uk/Timelines/TimelineG.html}{(1777--1855)},
    \href{http://www-history.mcs.st-andrews.ac.uk/BirthplaceMaps/Places/Italy.html}{Italian physicist}
    \index{Fermi, Enrico}
    \index{quotes!Fermi, Enrico}
    \footnotemark
  }
  {../common/people/small/fermi.jpg}
  {If it is true, it can be proved.}
  \citetblt{
    quote: & \citerp{benedetto}{85} \\
    image: & \url{http://www-history.mcs.st-andrews.ac.uk/Mathematicians/Fermi.html}
    }


%Aubrey, John (1626-1697)
%[About Thomas Hobbes:]
%He was 40 years old before he looked on geometry; which happened accidentally. Being in a gentleman's library, Euclid's Elements lay open, and "twas the 47 El. libri I" [Pythagoras' Theorem]. He read the proposition . "By God", sayd he, "this is impossible:" So he reads the demonstration of it, which referred him back to such a proposition; which proposition he read. That referred him back to another, which he also read. Et sic deinceps, that at last he was demonstratively convinced of that trueth. This made him in love with geometry.
%In O. L. Dick (ed.) Brief Lives, Oxford: Oxford University Press, 1960, p. 604.
%http://math.furman.edu/~mwoodard/ascquota.html

  \paragraph{Formal structure.}
  Guidelines in writing the proofs include:
  \begin{dingautolist}{"C0}
    \item Proofs should be very detailed (verbose rather than terse).
          This could tend to obscure the points the proofs are proving
          (``can't see the forest for all of the trees");
          but placing these points in boxes helps remedy this situation
          (see previous discussion).
    \item Every step that relies on another definition or theorem
          should be justified immediately to the right of the step,
          with cross referencing information.
          Many times these references include page numbers for more convenient access.
          If you happen to be viewing the pdf version of this text,
          then you can simply click on a particular reference and your pdf viewer will
          take you immediately to that location.
    \item Proofs, whenever possible, should be {\em direct proofs}
          clearly linking statement to statement with nothing more than
          the equality relation ($=$).
          They should not be cluttered with extensive explanations
          between steps. Let the mathematics speak for itself instead
          of me constantly jumping in waving my hands about in an effort to
          make clear what the mathematical equations make clear
          themselves.
    \item The development of proofs should rely fundamentally on
          propositional logic(logical AND $\land$, logical OR $\lor$, and logical NOT $\lnot$)
          and predicate logic (``for all" universal quantifier $\forall$ and
          ``there exists" existential quantifier $\exists$.)
  \end{dingautolist}

%Although countless practical problems were solved in the eighteenth century
%by the mathematical framework available,
%some began to see that this framework, howbeit sufficiently useful,
%was not sufficiently rigorous to withstand the scrutiny of rigorous logic
%and not sufficiently general to  support the most general functions.
%The nineteenth century was the century of analysis for mathematics.
%Thus with the arrival of the nineteenth century came also the arrival of
%the analytic mathematical samurais such as Cantor, Dirichlet, Lebesgue, and Weirstrass.
%These were men given to detail and who established mathematics on a detailed
%and solid mathematical
%footing through intense and thorough analysis.
%In the end Cantor lost his mind; but the world gained a mindset
%of a solid mathematical
%foundation on which new mathematical structures could be built in the
%twentieth century.



%---------------------------------------
\section*{External reference support}
%---------------------------------------
\qboxnps
  {
    \href{http://en.wikipedia.org/wiki/Niels_Henrik_Abel}{Niels Henrik Abel}
   (1802--1829),
   \href{http://www-history.mcs.st-andrews.ac.uk/BirthplaceMaps/Places/Russia.html}{Nowegian mathematician}
    \index{Abel, Niels Henrik}
    \index{quotes!Abel, Niels Henrik}
    \footnotemark
  }
  {../common/people/small/abel.jpg}
  {It appears to me that if one wants to make progress in mathematics,
    one should study the masters and not the pupils.}
  \citetblt{
    quote: & \citerp{simmons2007}{187} \\
    image: & \url{http://en.wikipedia.org/wiki/Image:Niels_Henrik_Abel.jpg}
    }
%
I have tried to include extensive reference information throughout the text.
These refences appear as brief footnotes at the bottom of the page where
the reference is relevant.
More information about a reference is given in the \hie{bibliography} at the
end of this text.
In most cases, each reference has an associated web link.
That link contains more information about the reference---in some
cases this means a full text download, in other cases a partial viewing,
in some cases the contents of the refence can be searched online,
and in some cases the location where that reference can be found
in libraries scattered throughout the world.
If you happen to be viewing the pdf version of this text,
then you can simply click on a particular footnote number or
footnote reference and your pdf viewer will
immediately take you to that location.
In providing easier access to references,
you the reader can ``study the masters and not the pupils."


%---------------------------------------
\section*{Historical viewpoint}
%---------------------------------------
\qboxnps
  {
    \href{http://en.wikipedia.org/wiki/Yoshida_Kenko}{Yoshida Kenko (Urabe Kaneyoshi)}
    (1283? -- 1350?),
    Japanese author and Buddhist monk
    \index{Kenko, Yoshida}  \index{Kaneyoshi, Urabe}
    \index{quotes!Kenko, Yoshida}  \index{quotes!Kaneyoshi, Urabe}
    \footnotemark
  }
  {../common/people/small/kenko.jpg}
  %{The pleasantest of all diversions is to sit alone under the lamp,
  % a book spread out before you,
  % and to make friends with people of a distant past you have never known. (Keene translation page 12)
  %}
  {To sit alone in the lamplight with a book spread out before you,
   and hold intimate converse with men of unseen generations---
   such is a pleasure beyond compare.}
  \citetblt{
    quote: & \citer{kenko_sansom} \\
    image: & \url{http://en.wikipedia.org/wiki/Yoshida_Kenko}
    }


\qboxnpq
  {
    \href{http://en.wikipedia.org/wiki/Niccol\%C3\%B2_Machiavelli}{Niccol\`o Machiavelli}
    (1469--1527), Italian political philosopher,
    in a 1513 letter to friend Francesco Vettori.
    \index{Machiavelli, Niccol\`o}
    \index{quotes!Machiavelli, Niccol\`o}
    \footnotemark
  }
  {../common/people/small/mach.jpg}
  {When evening comes, I return home and go to my study.
    On the threshold I strip naked, taking off my muddy, sweaty workaday clothes,
    and put on the robes of court and palace,
    and in this graver dress I enter the courts of the ancients and am welcomed by them,
    and there I taste the food that alone is mine, and for which I was born.
    And there I make bold to speak to them and ask the motives of their actions,
    and they, in their humanity reply to me.
    And for the space of four hours I forget the world, remember no vexation,
    fear poverty no more, tremble no more at death;
    I pass indeed into their world.}
  \citetblt{
    quote: & \citerp{machiavelli}{139?} \\
    image: & \url{http://en.wikipedia.org/wiki/Niccol\%C3\%B2_Machiavelli}
    }

\qboxnps
  {
    \href{http://www-history.mcs.st-andrews.ac.uk/Biographies/Poincare.html}
         {Henri Poincar\'e}
    \index{Henri Poincar\'e}
    \index{quotes!Henri Poincar\'e}
    (\href{http://www-history.mcs.st-andrews.ac.uk/Timelines/TimelineF.html}{1854--1912}),
    \href{http://www-history.mcs.st-andrews.ac.uk/BirthplaceMaps/Places/France.html}{French}
    mathematician and physicist,
    in an address to the Fourth International Congress of Mathematicians at Rome, 1908
    \footnotemark
  }
  {../common/people/small/poincare.jpg}
  {The true method of foreseeing the future of mathematics is to study its history
   and its actual state.}
  \citetblt{
    quote: & \citerp{bottazzini}{1} \\
    %quote: & \citer{poincare_sam} ??????  maybe not  \\
    image: & \url{http://www-groups.dcs.st-and.ac.uk/~history/PictDisplay/Poincare.html}
    }

I have tried to include information that gives readers an understanding of where
concepts came from in the history of mathematics.
In some cases, the original source is given as a reference.
In some cases, you can download the source text for free if you have an internet
connection. Web link addresses are provided in the bibliography.
Besides references,
sometimes I have also included quotes from famous mathematicians that
influenced mathematical thinking at the time when a mathematical idea was developing.
These quotes normally appear as
\shadowbox{``shadow boxes"} in the text.

In providing such information from notable mathematicians, you the reader can
``make bold to speak to them and ask the motives of their actions,
        and they, in their humanity reply"---and all this without having to change clothes.

%---------------------------------------
\section*{Hyper-link support}
%---------------------------------------
As already mentioned, most of the references in the bibliography feature web links
for further information and in some cases full text download capability.
Also web links are given for most of the images of
famous mathematicians appearing in this book.
%Lastly, this book itself is available at \\
%  \url{http://banyan.cm.nctu.edu.tw/~dgreenhoe/msd/index.html}\\

The pdf (portable document format) version of this text has
been embedded with an extensive number of hyper-links.
These hyperlinks are highlighted by a yellow box.
For example, in the table of contents,
you can click on a chapter or section title and immediately jump to that chapter or section.
In a proof statement, you can click on a reference to a previous result or definition
and jump to that result.
In a footnote reference, you can click on that reference and immediately jump
to the bibliography for more information about that reference.
In the bibliography, most of the references have links to the world wide web.
If your computer is online and you click on one of those links,
your default browser will display that web page after securing permission from you to do so.




  %============================================================================
% Daniel J. Greenhoe
% XeLaTeX file
%============================================================================
%=======================================
\section{Outcome subspace sequences}
%=======================================
%=======================================
%\subsection{Correlation operation on outcome subspace sequences}
%=======================================
%=======================================
\subsection{Definitions}
%=======================================
%---------------------------------------
\begin{definition}
\label{def:ocsseqmetric}
%---------------------------------------
Let $\Dom_1$ and $\Dom_2$ be \structe{convex subset}s \xref{def:convex} of $\Z$.\\
Let $\Dom\eqd\intoo{\meetop\Dom_1-\joinop\Dom_2-1}{\joinop\Dom_1-\meetop\Dom_2+1}.$
Let $\seq{x_n}{\Dom_1}$ and $\seq{y_n}{\Dom_2}$ be \structe{sequence}s over an \structe{outcome subspace} $\ocsD$.
\defboxp{
  The \fnctd{outcome subspace sequence metric} $\metrica{\seqn{x_n}}{\seqn{y_n}}$ is defined as
  \\\indentx
  $\ds\metrica{\seqn{x_n}}{\seqn{y_n}} \eqd \sum_{n\in\Dom} \ff(n)$
  \quad where \quad
  $
  \ff(n) \eqd \brb{\begin{array}{lM}
    \ocsd(x_n,y_n) & if $n\in   \Dom_1$ and $n\in   \Dom_2$\\
    1              & if $n\in   \Dom_1$ but $n\notin\Dom_2$\\
    1              & if $n\notin\Dom_1$ but $n\in   \Dom_2$\\
    0              & otherwise
  \end{array}}\qquad{\scy\forall n\in\Dom}
  $
  }
\end{definition}

%---------------------------------------
\begin{proposition}
%---------------------------------------
Let $\seq{x_n}{\Dom_1}$ and $\seq{y_n}{\Dom_2}$ be \structe{sequence}s over an \structe{outcome subspace} $\ocsD$.\\
Let $\metrican$ be the \fncte{outcome subspace sequence metric}.
\propbox{
  \Dom_1\seti\Dom_2\neq\emptyset \quad\implies\quad \text{$\metrican$ is a \fncte{metric}}
  }
\end{proposition}
\begin{proof}
This follows from the \thme{Fr{\'e}chet product metric} \xref{prop:fpm}.
In particular, $\metrican$ is a sum of metrics that include
the metrics $\ocsd(x_n,y_n)$ and the \fncte{discrete metric} \xref{def:dmetric}.
%This follows directly from either of the following two \fncte{product metric}s:
%\begin{liste}
%  \item Proof using \thme{Fr{\'e}chet product metric}:
%    \begin{align*}
%      \metrica{\seqn{x_n}}{\seqn{y_n}} 
%        &\eqd \sum_{n\in\Dom} \ocsd(x_n,y_n)
%        && \text{by \prefp{def:ocsseqmetric}}
%      \\&= \sum_{n\in\Dom} \alpha_n\ocsd(x_n,y_n)
%        && \text{for $\alpha_1=\alpha_2=\cdots=\alpha_\xN=1$}
%      \\&= \sum_{n\in\Dom} \alpha_n\ocsd_n(x_n,y_n)
%        && \text{for $\ocsd_1=\ocsd_2=\cdots=\ocsd_\xN=\ocsd$}
%      \\&\implies\text{$\metrican$ is a \fncte{metric}}
%        && \text{by \prope{Fr{\'e}chet product metric} \xref{prop:fpm}}
%    \end{align*}
%
%  \item Proof using \thme{Power mean metric}:
%    \begin{align*}
%      \metrica{\seqn{x_n}}{\seqn{y_n}} 
%        &\eqd \sum_{n\in\Dom} \ocsd(x_n,y_n)
%        && \text{by \prefp{def:ocsseqmetric}}
%      \\&= \xN\sum_{n\in\Dom} \lambda_n\ocsd(x_n,y_n)
%        && \text{for $\lambda_1=\lambda_2=\cdots=\lambda_\xN=\frac{1}{\xN}$}
%      \\&\implies\text{$\metrican$ is a \fncte{metric}}
%        && \text{by \prope{Power mean metric} \xref{prop:pmm}}
%      \\&&&\text{\qquad and \prope{$\alpha$-scaled metric} \xref{prop:asm}}
%    \end{align*}
%
%\end{liste}
\end{proof}

In standard signal processing, the \ope{autocorrelation} of a \structe{sequence} $\seqnD{x_n}$ 
is another \structe{sequence} $\seqnD{y_n}$ defined as 
$y_n\eqd\sum_{m\in\Z}x_m x_{m-n}$.
However, this definition requires that the sequence $\seqnD{x_n}$ be constructed over a \structe{field}.
In an \structe{outcome subspace sequence}, we in general do not have a \structe{field};
for example, in a \structe{die outcome subspace}, the expressions $\dieA+\dieB$ and $\dieA\times\dieB$ are undefined.
This paper offers an alternative definition (next) for \ope{autocorrelation} that uses the \fncte{distance} $\ocsd$ 
and that does not require a \structe{field}. 
%The autocorrelation function is used in \prefpp{ex:rdie_lp} and illustrated in \prefpp{fig:rdie_auto}.
%---------------------------------------
\begin{definition}
\label{def:ocsRxx}
%---------------------------------------
Let $\seqnD{x_n}$ and $\seqnD{y_n}$ be \fncte{sequence}s 
%\structe{outcome subspace sequence}s with \prope{finite support} \xref{def:support}
over the \structe{outcome subspace} $\ocsD$.
%Let $\xN$ be any whole number such that $\xN\ge\max(\seto{\support\seqn{x_n}},\seto{\support\seqn{y_n}})$, 
%where $\support\seqn{x_n}$ is the \ope{support} of $\seqn{x_n}$
%and   $\seto{\setX}$ is the \fncte{order} of a finite set $\setX$ \xref{def:seto}.
%Let $\metricbn$ be the \fncte{discrete metric} \xref{def:dmetric}.
%Let $\xN\eqd\sum_{n\in\Dom}\metricb{x_n}{\ocsz}$.
\\Let $\metrica{\seqn{x_n}}{\seqn{y_n}}$ be the \fnctd{outcome subspace sequence metric} \xref{def:ocsseqmetric}.
\\The \fnctd{cross-correlation} $\Rxy(n)$ of $\seqn{x_n}$ and $\seqn{y_n}$ and 
the \fnctd{autocorrelation}  $\Rxx(n)$ of $\seqn{x_n}$ 
are defined as
\defbox{\begin{array}{rclD}
  \Rxy(n) &\eqd& \ds - \sum_{m\in\Z}\metrica{\seqn{x_{m-n}}}{\seqn{y_m}} &\qquad (\fnctd{cross-correlation} $\Rxy(n)$ of $\seqn{x_n}$ and $\seqn{y_n}$)\\
  \Rxx(n) &\eqd& \ds - \sum_{m\in\Z}\metrica{\seqn{x_{m-n}}}{\seqn{x_m}} &\qquad (\fnctd{autocorrelation}  $\Rxx(n)$ of $\seqn{x_n}$)
\end{array}}\\
Moreover, the \opd{$\xM$-offset autocorrelation} of $\seqn{x_n}$ and $\seqn{y_n}$ is here defined 
as $\Rxx(n)+\xM$ \xref{def:axn}.
\end{definition}

%=======================================
\subsection{Examples of symbolic sequence statistics}
%=======================================
%---------------------------------------
\begin{example}[\exmd{fair die sequence}]
\label{ex:fdie_sha}
%---------------------------------------
Consider the pseudo-uniformly distributed \structe{fair die} \xref{def:fdie} 
sequence generated by the C code\footnotemark 
%\begin{lstlisting}
%\\\indentx\lstinline{#include<stdlib.h>} \ldots \lstinline!srand(0x5EED); for(n=0; n<N; n++){x[n] = 'A' + rand()%6;}!\\
%
\\\begin{minipage}{85mm}%
\begin{lstlisting}
#include<stdlib.h>
...
srand(0x5EED);
for(n=0; n<N; n++){x[n] = 'A' + rand()%6;}
\end{lstlisting}
\end{minipage}%
\hspace{10mm}%
\begin{tabular}{lclcl}
  where & \lstinline!'A'! &represents& $\dieA$ &,
     \\ & \lstinline!'B'! &represents& $\dieB$ &,
     %\\ & \lstinline!'C'! &represents& $\dieC$ &,
     %\\ & \lstinline!'D'! &represents& $\dieD$ &,
     %\\ & \lstinline!'E'! &represents& $\dieE$ &, and
     \\ & $\vdots$
     \\ & \lstinline!'F'! &represents& $\dieF$ &.
\end{tabular}
\\
%where \lstinline{'A'} represents $\dieA$, \lstinline{'B'} represents $\dieB$, etc.
The resulting sequence is partially displayed here:
   \\\includegraphics{ssp/plots/fdie_5eed_51_seq.pdf}\\
\footnotetext{For a more complete source code listing, see \prefpp{sec:src_die}}
This sequence constrained to a length of $\xN=2667\times6=16002$ elements is approximately \prope{uniformly distributed} 
and \prope{uncorrelated}, as illustrated next:
%calculated over %$(16000-16)-(16-1)+1=15970$ elements:
\\\begin{tabular}{|>{\scs}c|>{\scs}c|}
     \hline
     \includegraphics{ssp/plots/fdie_5eed_16002_histo.pdf}%
    &\includegraphics{ssp/plots/fdie_5eed_16002_auto.pdf}
    \\histogram & \ope{$2\xN$-offset autocorrelation}
   \\\hline
\end{tabular}
\end{example}

%---------------------------------------
\begin{example}[\exmd{real die sequence}]
\label{ex:rdie_sha}
%---------------------------------------
Consider the pseudo-uniformly distributed \structe{real die} \xref{def:rdie}
sequence generated as in \prefpp{ex:fdie_sha},
but with the real die metric rather than the fair die metric.
This change will not affect the distribution of the sequence, but it does affect 
the autocorrelation, as illustrated next:
\\\begin{tabular}{|>{\scs}c|>{\scs}c|}
     \hline
     \includegraphics{ssp/plots/rdie_5eed_16002_histo.pdf}%
     &\includegraphics{ssp/plots/rdie_5eed_16002_auto.pdf}
   \\\hline
\end{tabular}
\end{example}


%---------------------------------------
\begin{example}[\exmd{spinner sequence}]
\label{ex:spinner_sha}
%---------------------------------------
Consider the pseudo-uniformly distributed \structe{spinner} \xref{def:spinner} 
sequence generated by the C code\footnotemark 
%\begin{lstlisting}
%\\\indentx\lstinline{#include<stdlib.h>} \ldots \lstinline!srand(0x5EED); for(n=0; n<N; n++){x[n] = 'A' + rand()%6;}!\\
%
\\\begin{minipage}{85mm}%
\begin{lstlisting}
#include<stdlib.h>
...
srand(0x5EED);
for(n=0; n<N; n++){x[n] = 'A' + rand()%6;}
\end{lstlisting}
\end{minipage}%
\hspace{10mm}%
\begin{tabular}{lclcl}
  where & \lstinline!'A'! &represents& $\spinA$ &,
     \\ & \lstinline!'B'! &represents& $\spinB$ &,
     %\\ & \lstinline!'C'! &represents& $\spinC$ &,
     %\\ & \lstinline!'D'! &represents& $\spinD$ &,
     %\\ & \lstinline!'E'! &represents& $\spinE$ &, and
     \\ & $\vdots$
     \\ & \lstinline!'F'! &represents& $\spinF$ &.
\end{tabular}
\\
%where \lstinline{'A'} represents $\dieA$, \lstinline{'B'} represents $\dieB$, etc.
The resulting sequence is partially displayed here:
   \\\includegraphics{ssp/plots/spin_5eed_51_seq.pdf}\\
\footnotetext{For a more complete source code listing, see \prefpp{sec:src_spinner}}
This sequence is in essence identical to the fair die sequence \xref{ex:fdie_sha} and real die sequence \xref{ex:rdie_sha}
and thus yields what is essentially an identical histogram.
But because the metric is different, the autocorrelation  is also different.
In particular, because the nodes of the spinner metric are on average farther apart with respect to the spinner metric,
the sequence is less correlated (with respect to the metric), as illustrated next:
\\\begin{tabular}{|>{\scs}c|>{\scs}c|}
     \hline
     \includegraphics{ssp/plots/spin_5eed_16002_histo.pdf}%
    &\includegraphics{ssp/plots/spin_5eed_16002_auto.pdf}
   \\\hline
\end{tabular}
\end{example}


%---------------------------------------
\begin{example}[\exmd{weighted real die sequence}]
\label{ex:wrdie_sha}
%---------------------------------------
Consider the non-uniformly distributed \structe{weighted real die} \xref{def:wrdie} sequence with 
    \\\indentx$\psp(\dieE)=0.75$ and $\psp(\dieA)=\psp(\dieB)=\psp(\dieC)=\psp(\dieD)=\psp(\dieF)=0.05$,\\
    generated by the C code\footnote{For a more complete source code listing, see \prefpp{sec:src_rdie}}
\\\begin{minipage}{85mm}%
\begin{lstlisting}
srand(0x5EED);
for(n=0; n<N; n++){ u=rand()%100;
  if     (u< 5) x[n]='A'; /* 00-04 */ 
  else if(u<10) x[n]='B'; /* 05-09 */ 
  else if(u<15) x[n]='C'; /* 10-14 */ 
  else if(u<20) x[n]='D'; /* 15-19 */ 
  else if(u<95) x[n]='E'; /* 20-94 */ 
  else          x[n]='F'; /* 95-99 */ }
\end{lstlisting}
\end{minipage}%
\hspace{10mm}%
\begin{tabular}{lclcl}
  where & \lstinline!'A'! &represents& $\dieA$ &,
     \\ & \lstinline!'B'! &represents& $\dieB$ &,
     \\ & \lstinline!'C'! &represents& $\dieC$ &,
     \\ & \lstinline!'D'! &represents& $\dieD$ &,
     \\ & \lstinline!'E'! &represents& $\dieE$ &, and
     \\ & \lstinline!'F'! &represents& $\dieF$ &.
\end{tabular}
\\
     The resulting sequence is partially displayed here:
  \\\includegraphics{ssp/plots/wrdie_5eed_51_seq.pdf}\\
  Of course the resulting histogram, as illustrated below on the left, reflects the non-uniform distribution.
  Also note, as illustrated below on the right, that the weighted sequence is much more correlated 
  (as defined by \prefp{def:ocsRxx})
  as compared to the uniformly distributed 
  \structe{real die} sequence of \prefpp{ex:rdie_sha}.
  %The following two illustrations, calculated over $(160000/50)-1=3199$ elements, 
  %are based on the above sequence downsampled by a factor of 50
  %(making them better comparable to the results obtained in \pref{item:rdie_rhp_R1map} and \pref{item:rdie_hp_R3map} below.
     \\\begin{tabular}{|>{\scs}c|>{\scs}c|}
          \hline
          \includegraphics{ssp/plots/wrdie_5eed_16002_histo.pdf}%
         &\includegraphics{ssp/plots/wrdie_5eed_16002_auto.pdf}
         % \includegraphics{ssp/graphics/wdie_5eed_160000_down50_histo.pdf}%
         %&\includegraphics{ssp/graphics/wdie_5eed_160000_down50_auto.pdf}
         %\\histogram & autocorrelation
         %\\\mc{2}{|>{\scs}c|}{calculated over $(160000/50)-1=3199$ elements}
        \\\hline
     \end{tabular}
\end{example}

%---------------------------------------
\begin{example}[\exmd{weighted die sequence}]
\label{ex:wdie_sha}
%---------------------------------------
Consider the non-uniformly distributed \structe{weighted die} \xref{def:wdie} sequence 
generated as in \pref{ex:wrdie_sha}.
Of course the resulting histogram is identical to that of \pref{ex:wrdie_sha},
but because the distance function is different, the autocorrelation sequence is also different.
     \\\begin{tabular}{|>{\scs}c|>{\scs}c|}
          \hline
          \includegraphics{ssp/plots/wdie_5eed_16002_histo.pdf}%
         &\includegraphics{ssp/plots/wdie_5eed_16002_auto.pdf}
        \\\hline
     \end{tabular}
\end{example}

%---------------------------------------
\begin{example}[\exmd{weighted spinner sequence}]
\label{ex:wspin_sha}
%---------------------------------------
Consider the non-uniformly distributed die sequence with 
    \\\indentx$\psp(\spinE)=0.75$ and $\psp(\spinA)=\psp(\spinB)=\psp(\spinC)=\psp(\spinD)=\psp(\spinF)=0.05$,\\
    generated by the C code\footnote{For a more complete source code listing, see \prefpp{sec:src_spinner}}
\\\begin{minipage}{85mm}%
\begin{lstlisting}
srand(0x5EED);
for(n=0; n<N; n++){ u=rand()%100;
  if     (u< 5) x[n]='A'; /* 00-04 */ 
  else if(u<10) x[n]='B'; /* 05-09 */ 
  else if(u<15) x[n]='C'; /* 10-14 */ 
  else if(u<20) x[n]='D'; /* 15-19 */ 
  else if(u<95) x[n]='E'; /* 20-94 */ 
  else          x[n]='F'; /* 95-99 */ }
\end{lstlisting}
\end{minipage}%
\hspace{10mm}%
\begin{tabular}{lclcl}
  where & \lstinline!'A'! &represents& $\spinA$ &,
     \\ & \lstinline!'B'! &represents& $\spinB$ &,
     \\ & \lstinline!'C'! &represents& $\spinC$ &,
     \\ & \lstinline!'D'! &represents& $\spinD$ &,
     \\ & \lstinline!'E'! &represents& $\spinE$ &, and
     \\ & \lstinline!'F'! &represents& $\spinF$ &.
\end{tabular}
\\
  The resulting sequence and histogram is in essence the same as in %\prefpp{ex:wrdie_sha}.
  the \exme{weighted die sequence} example \xref{ex:wrdie_sha}.
  %\\\includegraphics{ssp/plots/wspin_5eed_51_seq.pdf}\\
  %Of course the resulting histogram, as illustrated below on the left, is in essence identical to that of the 
  %\exme{weighted die sequence} example \xref{ex:wrdie_sha}.
  But note, as illustrated below on the right, that the \exme{weighted spinner sequence} of this example is 
  significantly less correlated than the \exme{weighted die sequence} of \prefpp{ex:wrdie_sha},
  presumably due to the larger \structe{range} \xref{def:range} of the spinner metric ($\setn{0,1,2,3}$)
  as compared to the \structe{range} of the \structe{weighted real die} metric ($\setn{0,1,2}$) and 
  the \structe{weighted die} metric ($\setn{0,1}$).
     \\\begin{tabular}{|>{\scs}c|>{\scs}c|}
          \hline
          \includegraphics{ssp/plots/wspin_5eed_16002_histo.pdf}%
         &\includegraphics{ssp/plots/wspin_5eed_16002_auto.pdf}
        \\\hline
     \end{tabular}
\end{example}


%---------------------------------------
\begin{example}[\exmd{Random DNA sequence}]
\label{ex:dna_5eed}
%---------------------------------------
Consider the pseudo-uniformly distributed DNA sequence generated by the C code\footnotemark 
\\\begin{minipage}{85mm}%
\begin{lstlisting}
srand(0x5EED);
for(n=0; n<N; n++){  r=rand()%4;
  switch(r){ case 0: x[n]='A'; break;
             case 1: x[n]='T'; break;
             case 2: x[n]='C'; break;
             case 3: x[n]='G'; break; }}
\end{lstlisting}
\end{minipage}%
\hspace{10mm}%
\begin{tabular}{lclcl}
  where & \lstinline!'A'! &represents& $\symA$ &,
     \\ & \lstinline!'T'! &represents& $\symT$ &,
     \\ & \lstinline!'C'! &represents& $\symC$ &, and
     \\ & \lstinline!'G'! &represents& $\symG$ &.
\end{tabular}
\footnotetext{For a more complete source code listing, see \prefpp{sec:src_dna}}
  \\\includegraphics{ssp/plots/dna_5eed_51_seq.pdf}\\
This sequence constrained to a length of $4000\times4=16000$ elements is approximately \prope{uniformly distributed} and \prope{uncorrelated}, 
as illustrated next:
  \\\begin{tabular}{|>{\scs}c|>{\scs}c|}
       \hline
       \includegraphics{ssp/plots/dna_5eed_16000_histo.pdf}%
      &\includegraphics{ssp/plots/dna_5eed_16000_auto.pdf}%
      %\\histogram & autocorrelation
     \\\hline
  \end{tabular}
\end{example}

%---------------------------------------
\begin{example}[\exmd{SARS coronavirus DNA sequence}]
\label{ex:dna_sars}
%---------------------------------------
Consider the genome sequence (DNA sequence) for the SARS coronavirus with \hie{GenBank} accession number NC\_004718.3.\footnote{\citer{ncbiSars}} %identifier gi30271926.
This sequence is of length $\xN=29751$ and is partially displayed here, followed by its histogram and 
\ope{$2\xN$-offset autocorrelation} plots.
  \\\includegraphics{ssp/plots/dna_sars_51_seq.pdf}\\
     \\\begin{tabular}{|>{\scs}c|>{\scs}c|}
          \hline
          \includegraphics{ssp/plots/dna_sars_histo.pdf}%
         &\includegraphics{ssp/plots/dna_sars_auto.pdf}
         %\\histogram & autocorrelation
        \\\hline
     \end{tabular}
\end{example}

%---------------------------------------
\begin{example}[\exmd{Ebola virus DNA sequence}]
\label{ex:dna_ebola}
%---------------------------------------
Consider the genome sequence (DNA sequence) for the Ebola virus with GenBank accession \lstinline{AF086833.2}.\footnote{\citer{ncbiEbola}}
This sequence is of length 18959 and is partially displayed here:
  \\\includegraphics{ssp/plots/dna_ebola_51_seq.pdf}\\
     \\\begin{tabular}{|>{\scs}c|>{\scs}c|}
          \hline
          \includegraphics{ssp/plots/dna_ebola_histo.pdf}%
         &\includegraphics{ssp/plots/dna_ebola_auto.pdf}
         %\\histogram & autocorrelation
        \\\hline
     \end{tabular}
\end{example}

%---------------------------------------
\begin{example}[\exmd{Bacterium DNA sequence}]
\label{ex:dna_mpbacterium}
%---------------------------------------
Consider the genome sequence (DNA sequence) for the bacterium 
\hie{Melissococcus plutonius strain 49.3 plasmid pMP19}
with GenBank accession \lstinline{NZ_CM003360.1}.\footnote{\citer{ncbiMP}}
This sequence is of length 19430 and is partially displayed here:
  \\\includegraphics{ssp/plots/dna_mpbacterium_51_seq.pdf}\\
     \\\begin{tabular}{|>{\scs}c|>{\scs}c|}
          \hline
          \includegraphics{ssp/plots/dna_mpbacterium_histo.pdf}%
         &\includegraphics{ssp/plots/dna_mpbacterium_auto.pdf}
         %\\histogram & autocorrelation
        \\\hline
     \end{tabular}
\end{example}

%---------------------------------------
\begin{example}[\exmd{Papaya DNA sequence}]
\footnote{
  \citer{ncbiPapaya}
  }
\label{ex:dna_papaya}
%---------------------------------------
Consider the genome sequence segment for the fruit
\hie{carica papaya} with GenBank accession \lstinline{DS982815.1}.
This sequence is not a complete genome, rather it is a ``genomic scaffold" \xref{def:dnan}. 
As such, there are some elements for which the content is not known.
For these locations, the symbol $\symN$ is used.
In this particular sequence, there are 144 $\symN$ symbols.
This sequence is of length 15495 and is partially displayed here:
  \\\includegraphics{ssp/plots/dna_papaya1446_51_seq.pdf}\\
     \\\begin{tabular}{|>{\scs}c|>{\scs}c|}
          \hline
          \includegraphics{ssp/plots/dna_papaya1446_histo.pdf}%
         &\includegraphics{ssp/plots/dna_papaya1446_auto.pdf}
         %\\histogram & autocorrelation
        \\\hline
     \end{tabular}
\end{example}

%=======================================
\section{Extending to distance linear spaces}
%=======================================
%=======================================
\subsection{Motivation}
%=======================================
\prefpp{sec:ocs} demonstrated how a stochastic process could be defined as an \structe{outcome subspace}
with \prope{order} and \prope{metric} structures.
\prefpp{ex:realdieXRYZ} reviewed an example of a \structe{real die outcome subspace} 
that was mapped through 4 different \fncte{random variable}s to 4 different \structe{weighted graph}s.
Two of these random variables ($\rvY$ and $\rvZ$) mapped to structures (\structe{weighted graphs}) that are very similar 
to the \structe{real die} with respect to order and metric geometry.
Two other random variables ($\rvW$ and $\rvX$) mapped to structures 
(the \structe{real line} and the \structe{integer line}) that are very dissimilar.
The implication of this example is that if we want statistics that closely model the underlying stochastic process,
then we should map to a structure that that an order structure and distance geometry similar to 
that of the underlying stochastic process, and not simply the one that is the most convenient.
Ideally, we would like to map to a structure that is \prope{isomorphic} \xref{def:isomorphic}
and \prope{isometric} \xref{def:isometric} to the structure of the stochastic process.

However, for sequence processing using very basic methods such as FIR filtering, Fourier analysis,
or wavelet analysis,
we would very much like to map into the \structe{real line} $\R^1$ or possibly some higher dimensional space
$\R^n$.
Because the real line is often very dissimilar to the stochastic process,
we are motivated to find structures in $\R^n$ that \emph{are} similar.
And that is what this section presents---mapping from a stochastic process $\ocsD$ 
into an \structe{ordered distance linear space} $\omlsRnD$ in which $\omlo$ is an extension of $\ocso$ and
$\omld$ is an extension of $\ocsd$.

Thus, for sequence processing on an \structe{outcome subspace} $\ocsD$, we would like to 
define a \fncte{random variable} $\rvX$ and an \structe{ordered distance linear space} $\omlsRnD$ 
that satisfy the following constraints: 
\begin{enumerate}
  \item The random variable maps the elements of $\ocso$ into $\R^n$ and
  \item the order relation $\omlr$ is an \prope{extension} to $\omlo$ of the order relation $\ocsr$ on $\ocso$ and
  \item the \fncte{distance} function $\omld$ is an \prope{extension} to $\omlo$ of the distance function $\ocsd$ on $\ocso$.
\end{enumerate}

%=======================================
\subsection{Some random variables}
%=======================================
In this section, we first define some \fncte{random variable}s \xref{def:ocsrv} 
that are used later in this paper.
%---------------------------------------
\begin{definition}
\label{def:rv_dietrad}
%---------------------------------------
The \opd{traditional die random variable} $\rvX$ maps from the set %\xref{def:rvt} 
\\$\setn{\dieA,\dieB,\dieC,\dieD,\dieE,\dieF}$ into the set $\R^1$
and is defined as\footnotemark
\\\indentx
  $\rvX(\dieA)\eqd 1$, 
  $\rvX(\dieB)\eqd 2$, 
  $\rvX(\dieC)\eqd 3$, 
  $\rvX(\dieD)\eqd 4$, 
  $\rvX(\dieE)\eqd 5$, and 
  $\rvX(\dieF)\eqd 6$.
\end{definition}

%---------------------------------------
\begin{definition}
\label{def:rv_diepam}
%---------------------------------------
The \opd{PAM die random variable} $\rvX$ maps from the set $\setn{\dieA,\dieB,\dieC,\dieD,\dieE,\dieF}$ into the set $\R^1$
and is defined as\footnotemark
\\\indentx
  $\rvX(\dieA)\eqd-2.5$, 
  $\rvX(\dieB)\eqd-1.5$, 
  $\rvX(\dieC)\eqd-0.5$, 
  $\rvX(\dieD)\eqd+0.5$, 
  $\rvX(\dieE)\eqd+1.5$, and 
  $\rvX(\dieF)\eqd+2.5$.
\end{definition}
\footnotetext{PAM is an acronym for \ope{pulse amplitude modulation} and is a standard technique in 
the field of digital communications.}

%---------------------------------------
\begin{definition}
\label{def:rv_dieqpsk}
%---------------------------------------
The \opd{QPSK die random variable} $\rvX$ maps from the set $\setn{\dieA,\dieB,\dieC,\dieD,\dieE,\dieF}$ into the set $\C^1$
and is defined as\footnotemark 
\\\indentx$\begin{array}{rclrclrcl}
    \rvX(\dieA) &\eqd& \exp\brp{ 30\times\frac{\pi}{180}i}, 
   &\rvX(\dieB) &\eqd& \exp\brp{ 90\times\frac{\pi}{180}i}, 
   &\rvX(\dieC) &\eqd& \exp\brp{150\times\frac{\pi}{180}i}, 
  \\\rvX(\dieD) &\eqd& \exp\brp{210\times\frac{\pi}{180}i}, 
   &\rvX(\dieE) &\eqd& \exp\brp{270\times\frac{\pi}{180}i},
   &\rvX(\dieF) &\eqd& \exp\brp{330\times\frac{\pi}{180}i}.
\end{array}$
\end{definition}
\footnotetext{QPSK is an acronym for \ope{quadrature phase shift keying} and is a standard technique in 
the field of digital communications.}

%---------------------------------------
\begin{definition}
\label{def:rv_dieR3}
%---------------------------------------
The \opd{$\R^3$ die random variable} $\rvX$ maps from the set $\setn{\dieA,\dieB,\dieC,\dieD,\dieE,\dieF}$ into the set $\R^3$
and is defined as 
\\\indentx$\begin{array}{rclrclrcl}
    \rvX(\dieA) &\eqd& \otriple{+1}{ 0}{ 0}, 
   &\rvX(\dieB) &\eqd& \otriple{ 0}{+1}{ 0}, 
   &\rvX(\dieC) &\eqd& \otriple{ 0}{ 0}{+1}, 
  \\\rvX(\dieD) &\eqd& \otriple{ 0}{ 0}{-1}, 
   &\rvX(\dieE) &\eqd& \otriple{ 0}{-1}{ 0},
   &\rvX(\dieF) &\eqd& \otriple{-1}{ 0}{ 0}.
\end{array}$
\end{definition}

%---------------------------------------
\begin{definition}
\label{def:rv_dieR6}
%---------------------------------------
The \opd{$\R^6$ die random variable} $\rvX$ maps from the set $\setn{\dieA,\dieB,\dieC,\dieD,\dieE,\dieF}$ into the set $\R^6$
and is defined as 
\\\indentx$\begin{array}{rclrclrcl}
    \rvX(\dieA) &\eqd& \osix{1}{0}{0}{0}{0}{0}, 
   &\rvX(\dieB) &\eqd& \osix{0}{1}{0}{0}{0}{0}, 
   &\rvX(\dieC) &\eqd& \osix{0}{0}{1}{0}{0}{0}, 
  \\\rvX(\dieD) &\eqd& \osix{0}{0}{0}{1}{0}{0}, 
   &\rvX(\dieE) &\eqd& \osix{0}{0}{0}{0}{1}{0},
   &\rvX(\dieF) &\eqd& \osix{0}{0}{0}{0}{0}{1}.
\end{array}$
\end{definition}

%---------------------------------------
\begin{definition}
\label{def:rv_spinR1}
%---------------------------------------
The \opd{$\R^1$ spinner random variable} $\rvX$ maps from the set $\setn{\spinA,\spinB,\spinC,\spinD,\spinE,\spinF}$ into the set $\R^1$
and is defined as\footnotemark
\\\indentx
  $\rvX(\spinA)\eqd 1$, 
  $\rvX(\spinB)\eqd 2$, 
  $\rvX(\spinC)\eqd 3$, 
  $\rvX(\spinD)\eqd 4$, 
  $\rvX(\spinE)\eqd 5$, and 
  $\rvX(\spinF)\eqd 6$.
\end{definition}

%---------------------------------------
\begin{definition}
\label{def:rv_spinqpsk}
%---------------------------------------
The \opd{QPSK spinner random variable} $\rvX$ maps from the set $\setn{\spinA,\spinB,\spinC,\spinD,\spinE,\spinF}$ into the set $\C^1$
and is defined as 
\\\indentx$\begin{array}{rclrclrcl}
    \rvX(\spinA) &\eqd& \exp\brp{-90\times\frac{\pi}{180}i}, 
   &\rvX(\spinB) &\eqd& \exp\brp{-30\times\frac{\pi}{180}i}, 
   &\rvX(\spinC) &\eqd& \exp\brp{ 30\times\frac{\pi}{180}i}, 
  \\\rvX(\spinD) &\eqd& \exp\brp{ 90\times\frac{\pi}{180}i}, 
   &\rvX(\spinE) &\eqd& \exp\brp{150\times\frac{\pi}{180}i},
   &\rvX(\spinF) &\eqd& \exp\brp{210\times\frac{\pi}{180}i}.
\end{array}$
\end{definition}

%---------------------------------------
\begin{definition}
\footnote{
  \citePp{galleani2010}{772}
  }
\label{def:rv_dnapam}
%---------------------------------------
The \opd{PAM DNA random variable} $\rvX$ maps from the set $\setn{\dnaA,\dnaC,\dnaG,\dnaT}$ into the set $\R^1$
and is defined as\footnotemark
\\\indentx
  $\rvX(\dnaA)\eqd-1.5$, 
  $\rvX(\dnaC)\eqd-0.5$, 
  $\rvX(\dnaT)\eqd+0.5$, 
  $\rvX(\dnaG)\eqd+1.5$. 
\end{definition}

%---------------------------------------
\begin{definition}
\footnote{
  \citePp{galleani2010}{772}
  }
\label{def:rv_dnaqpsk}
%---------------------------------------
The \opd{QPSK DNA random variable} $\rvX$ maps from the set $\setn{\dnaA,\dnaC,\dnaG,\dnaT}$ into the set $\C^1$
and is defined as 
\\\indentx$\begin{array}{rclrcl}
     \rvX(\dnaA) &\eqd& \exp\brp{ 45\times\frac{\pi}{180}i}, 
    &\rvX(\dnaC) &\eqd& \exp\brp{135\times\frac{\pi}{180}i}, 
   \\\rvX(\dnaG) &\eqd& \exp\brp{225\times\frac{\pi}{180}i}, 
    &\rvX(\dnaT) &\eqd& \exp\brp{315\times\frac{\pi}{180}i}.
\end{array}$
\end{definition}
\footnotetext{QPSK is an acronym for \ope{quadrature phase shift keying} and is a standard technique in 
the field of digital communications.}

%---------------------------------------
\begin{definition}
\label{def:rv_dnaR4}
%---------------------------------------
The \opd{$\R^4$ DNA random variable} $\rvX$ maps from the set $\setn{\dnaA,\dnaC,\dnaG,\dnaT}$ into the set $\R^4$
and is defined as\footnotemark
\\\indentx$\begin{array}{rclrclrclrcl}
     \rvX(\dnaA) &\eqd& \oquad{1}{0}{0}{0}, 
    &\rvX(\dnaC) &\eqd& \oquad{0}{1}{0}{0}, 
    &\rvX(\dnaT) &\eqd& \oquad{0}{0}{1}{0}, 
    &\rvX(\dnaG) &\eqd& \oquad{0}{0}{0}{1}
\end{array}$
\end{definition}
\footnotetext{This type of mapping has previously been used by \citeP{voss1992} in calculating the \ope{Voss Spectrum}, 
  (a kind of Fourier analysis) of DNA sequences.
  See also \citePp{galleani2010}{772}.
  }


%=======================================
\subsection{Some ordered distance linear spaces}
%=======================================
%---------------------------------------
\begin{definition}
\label{def:dieR1oml}
%---------------------------------------
The structure $\oml{\R^1}{\omlr}{\omld}$ is the \structd{$\R^1$ die distance linear space}
if $\omlr$ is the \rele{standard ordering relation} on $\R$, 
and $\omld(x,y)\eqd\abs{x-y}$ (the \fncte{Euclidean metric} on $\R$, \xrefnp{def:emetric}).
\end{definition}

%---------------------------------------
\begin{definition}
\label{def:dieR3oml}
%---------------------------------------
%Let $\rvX$ be the \opd{$\R^3$ die random variable} \xref{def:dieR3rv}.
The structure $\oml{\R^3}{\omlr}{\omld}$ is the \structd{$\R^3$ die distance linear space}
if $\omlr=\emptyset$, and $\omld$ is the 
$2$-scaled \fncte{Lagrange arc distance} $\omld$ defined as follows:
\quad$\omld(p,q) \eqd 2\metrican(p,q)$\\
where $\metrican$ is the \fncte{Lagrange arc distance} \xref{def:larc}.
\end{definition}
%\begin{figure}[h]
%  \centering%
%  \gsize%
\mbox{}\hfill%
  \begin{tabular}{c}\includegraphics{ssp/graphics/rdie_r3met.pdf}\end{tabular}%
  \hspace{15mm}%
  $\begin{tabstr}{0.75}\begin{array}{|c|*{6}{@{\hspace{2pt}}c}|}
    \hline
    \ocsd(x,y) & \dieA &\dieB &\dieC &\dieD &\dieE &\dieF
    \\\hline
      \dieA &    0   &   1   &   1   &   1   &   1   &   2
    \\\dieB &    1   &   0   &   1   &   1   &   2   &   1
    \\\dieC &    1   &   1   &   0   &   2   &   1   &   1
    \\\dieD &    1   &   1   &   2   &   0   &   1   &   1
    \\\dieE &    1   &   2   &   1   &   1   &   0   &   1
    \\\dieF &    2   &   1   &   1   &   1   &   1   &   0
    \\\hline
  \end{array}\end{tabstr}$
\hfill\mbox{}\\
%  \caption{\structe{distance linear space} $\opair{\R^3}{\ocsd'}$ for \structe{real die}\label{fig:rdie3dmet}}
%\end{figure}
Used together with the \fncte{$\R^3$ die random variable} $\rvX$ \xref{def:rv_dieR3},
the distance $\omld$ in the \structe{$\R^3$ die distance linear space} \xrefr{def:dieR3oml}
is an extension of $\ocsd$ in the \structe{real die outcome subspace} 
\\$\ocsG\eqd\ocsrdie$ \xref{def:rdie}.
We can also say that $\rvX$ is an \fncte{isometry} \xref{def:isometry} 
and that the two structures are \prope{isometric}.
For example, %in $\ocsG$, $\ocsd(\dieA,\dieB)=1$ and $\ocsd(\dieA,\dieF)=2$; likewise in $\omlD$, 
\\\indentx$\begin{array}{lclclclD}
  \omld\brs{\rvX(\dieA),\rvX(\dieB)} &=& \omld\brs{\otriple{1}{0}{0},\otriple{ 0}{1}{0}} &=& 1 &=& \ocsd(\dieA,\dieB) & and\\
  \omld\brs{\rvX(\dieA),\rvX(\dieF)} &=& \omld\brs{\otriple{1}{0}{0},\otriple{-1}{0}{0}} &=& 2 &=& \ocsd(\dieA,\dieF) & .
\end{array}$
\\
As for order, the mapping $\rvX$ is also \prope{order preserving} \xref{def:opreserve},
but trivially, because the \structe{real die outcome subspace} is \prope{unordered} \xref{def:unordered}.
But if we still honor the standard ordering on each dimension $\R$ in $\R^3$,
then the two structures are \prope{not isomorphic} \xref{def:isomorphic} because\footnote{%
  Note that while $\rvX^{-1}$ \xref{def:inverse} does not exist as a \structe{function}, it does exist as a \structe{relation}.
  }
the inverse $\rvX^{-1}$ 
is \prope{not order preserving} \xref{thm:latiso}---%
for example, $\rvX(\dieD)=\otriple{0}{0}{-1}\orel\otriple{0}{0}{1}=\rvX(\dieC)$, but $\dieD$ and $\dieC$
are \prope{incomparable} \xref{def:incomparable} in $\ocsG$.

%---------------------------------------
\begin{definition}
\label{def:spinR2oml}
%---------------------------------------
The structure $\oml{\R^2}{\omlr}{\omld}$ is the \structd{$\R^2$ spinner distance linear space}
if $\omlr=\emptyset$, and $\omld$ is the 
$3$-scaled \fncte{Lagrange arc distance} $\omld$ defined as follows:
\quad$\omld(p,q) \eqd 3\metrican(p,q)$\\
where $\metrican$ is the \fncte{Lagrange arc distance} \xref{def:larc}.
\end{definition}
%\begin{figure}[h]
%  \centering%
%  \gsize%
\mbox{}\hfill%
  \begin{tabular}{c}\includegraphics{ssp/graphics/spinner_r2.pdf}\end{tabular}%
  \hspace{15mm}%
  $\begin{tabstr}{0.75}\begin{array}{|c|*{6}{@{\hspace{5pt}}c}|}
    \hline
    \ocsd(x,y)    & \spinA & \spinB & \spinC & \spinD & \spinE & \spinF
    \\\hline
      \spinA      &    0   &   1    &    2   &    3   &    2   &    1
    \\\spinB      &    1   &   0    &    1   &    2   &    3   &    2
    \\\spinC      &    2   &   1    &    0   &    1   &    2   &    3
    \\\spinD      &    3   &   2    &    1   &    0   &    1   &    2
    \\\spinE      &    2   &   3    &    2   &    1   &    0   &    1
    \\\spinF      &    1   &   2    &    3   &    2   &    1   &    0
    \\\hline
  \end{array}\end{tabstr}$
\hfill\mbox{}\\
%  \caption{\structe{distance linear space} $\opair{\R^2}{\ocsd'}$ for \structe{spinner}\label{fig:rdie3dmet}}
%\end{figure}
Used together with the \fncte{QPSK spinner random variable} $\rvX$ \xref{def:rv_spinqpsk},
the distance $\omld$ in the \structe{$\R^2$ spinner distance linear space} \xref{def:spinR2oml}
is an extension of $\ocsd$ in the \structe{spinner outcome subspace} 
$\ocsG\eqd\ocsspin$ \xref{def:spinner}.
We can again say that $\rvX$ is an \fncte{isometry} %\xref{def:isometry} 
and that the two structures are \prope{isometric}.
For example, %in $\ocsG$, $\ocsd(\spinA,\spinB)=1$, $\ocsd(\spinA,\spinC)=2$, and $\ocsd(\spinA,\spinD)=3$; likewise in $\omlD$, 
\\\indentx$\begin{array}{lclclclD}
  \omld\brs{\rvX(\spinA),\rvX(\spinB)} &=& \omld\brs{\opair{0}{-1},\opair{\sfrac{\sqrt{3}}{2}}{-\sfrac{1}{2}}} &=& 1 &=& \ocsd(\spinA,\spinB) & and\\
  \omld\brs{\rvX(\spinA),\rvX(\spinC)} &=& \omld\brs{\opair{0}{-1},\opair{\sfrac{\sqrt{3}}{2}}{+\sfrac{1}{2}}} &=& 2 &=& \ocsd(\spinA,\spinC) & and\\
  \omld\brs{\rvX(\spinA),\rvX(\spinD)} &=& \omld\brs{\opair{0}{-1},\opair{0}{1}}                               &=& 3 &=& \ocsd(\spinA,\spinD) & .
\end{array}$
\\
The mapping $\rvX$ is again trivally \prope{order preserving}.
And if we again honor the standard ordering on each dimension $\R$ in $\R^2$,
then the two structures are \prope{not isomorphic} \xref{def:isomorphic} because
the inverse $\rvX^{-1}$ 
is \prope{not order preserving}---%
for example, $\rvX(\spinA)=\opair{0}{-1}\orel\opair{0}{1}=\rvX(\spinD)$, but $\spinA$ and $\spinD$
are \prope{incomparable} in $\ocsG$.

%The next definition introduces a \structe{metric linear space} for \ope{fair die sequence processing}
%that unlike in the \structe{real die} structure, the random variable mapping 
%is fully \prope{order preserving} (trivally).
%---------------------------------------
\begin{definition}
\label{def:fdieR6oml}
%---------------------------------------
The structure $\oml{\R^6}{\omlr}{\omld}$ is the \structd{$\R^6$ die distance linear space}
if $\omlr=\emptyset$, and $\omld$ is defined as 
$\ds\omld(p,q) \eqd \sfrac{\sqrt{2}}{2}\metrica{p}{q}$,
where $\metrican$ is the \fncte{Euclidean metric} on $\R^6$ \xref{def:emetric}.
\end{definition}

Used together with the \fncte{$\R^6$ die random variable} $\rvX$ \xref{def:rv_dieR6},
the distance $\omld$ in the \structe{$\R^6$ fair die distance linear space} \xref{def:fdieR6oml}
is an extension of $\ocsd$ in the \structe{fair die outcome subspace} 
$\ocsG\eqd\ocsrdie$ \xref{def:rdie}.
We can again say that $\rvX$ is an \fncte{isometry} %\xref{def:isometry} 
and that the two structures are \prope{isometric}.
For example, %in $\ocsG$, $\ocsd(\dieA,\dieB)=1$, $\ocsd(\dieA,\dieC)=1$, and $\ocsd(\dieA,\dieF)=1$; likewise in $\omlD$, 
\\\indentx$\begin{array}{lclclclD}
  \omld\brs{\rvX(\dieA),\rvX(\dieB)} &=& \omld\brs{\osix{1}{0}{0}{0}{0}{0},\osix{0}{1}{0}{0}{0}{0}} &=& 1 &=& \ocsd(\dieA,\dieB) & and\\
  \omld\brs{\rvX(\dieA),\rvX(\dieC)} &=& \omld\brs{\osix{1}{0}{0}{0}{0}{0},\osix{0}{0}{1}{0}{0}{0}} &=& 1 &=& \ocsd(\dieA,\dieC) & and\\
  \omld\brs{\rvX(\dieA),\rvX(\dieF)} &=& \omld\brs{\osix{1}{0}{0}{0}{0}{0},\osix{0}{0}{0}{0}{0}{1}} &=& 1 &=& \ocsd(\dieA,\dieF) & .
\end{array}$
\\
The mapping $\rvX$ is again trivally \prope{order preserving},
and the inverse $\rvX^{-1}$ is trivally \prope{order preserving} as well.
And so unlike the \structe{$\R^3$ die distance linear space} \xrefr{def:dieR3oml}
and the \structe{$\R^2$ spinner distance linear space} \xrefr{def:spinR2oml},
this pair of structures is \prope{isomorphic}.




  %-------------------------------------
  \section{Symbolic sequence processing applications}
  %-------------------------------------
\qboxnpq
  {
    Joseph Louis Lagrange (1736-1813), mathematician
    \index{Lagrange, Joseph Louis}
    \index{quotes!Lagrange, Joseph Louis}
    \footnotemark
  }
  {../common/people/small/lagrange.jpg}
  {I regard as quite useless the reading of large treatises of pure analysis:
    too large a number of methods pass at once before the eyes.
    It is in the works of applications that one must study them;
    one judges their ability there and one apprises the manner of making use of them.}
  \citetblt{
    quote: &  \citerp{stopple2003}{xi} \\
          %&  \url{http://www.math.okstate.edu/~wli/teach/fmq.html} \\
          %&  \url{http://www-groups.dcs.st-and.ac.uk/~history/Quotations/Lagrange.html} \\
    image: & \url{http://en.wikipedia.org/wiki/Image:Langrange_portrait.jpg}
    }

  %============================================================================
% Daniel J. Greenhoe
% LaTeX file
%============================================================================
%=======================================
\subsection{Low pass filtering/Smoothing}
%=======================================
%=======================================
%\section{Real die sequence processing}
%\label{sec:rdie}
%=======================================

%=======================================
%\subsubsection{Low pass filtering of real die sequences}
%=======================================
%---------------------------------------
\begin{example}[\exmd{low pass filtering of real die sequence}]
\label{ex:rdie_lp}
\mbox{}\\
\addcontentsline{toc}{subsubsection}{* low pass filtering of real die sequence}
%---------------------------------------
\begin{enumerate}
  \item \label{item:rdie_lp_seq}
     Consider the pseudo-uniformly distributed die sequence presented in \prefpp{ex:rdie_sha}.
     Suppose we want to \ope{filter} this sequence with a
     \fncte{low pass sequence} in order to ``smooth out" the sequence. 
     But to perform the actual filtering, note that the die sequence
     must first be mapped into a \structe{linear space} $\R^\xN$.
  
  \item \label{item:rdie_lp_R1_rect16_euclid}
        Suppose we first use the \ope{traditional die random variable} \xref{def:rv_dietrad} 
        to map the die sequence into $\R^1$.
        \ope{Filter}ing \xref{def:filter} this $\R$-valued sequence using the 
        \fncte{length 16 rectangular low pass sequence} \xref{ex:lp_rect} 
        in the \structe{$\R^1$ die distance linear space} \xref{def:dieR1oml} 
        and then mapping the result back to a \fncte{die sequence} %\xref{def:rdie}
        using the \fncte{Euclidean metric} \xref{def:emetric}, 
        produces the result partially dispayed here:
        \\\includegraphics{../common/math/sspplots/rdie_lp_12000_R1_rect16_euclid_seq.pdf}\\
        Note that the die sequence has indeed been smoothed out, but it's uniform distribution has been destroyed---almost all 
        of its values are around the ``expected value" 3.5, 
        as illustrated below on the left.
        Of course such filtering also introduces correlation, giving the \ope{autocorrelation} sequence 
        a slightly wider center lobe as illustrated below on the right.
        Both diagrams are calculated over a length $12000$ sequence.
     \\\begin{tabular}{|>{\scs}c|>{\scs}c|}
          \hline
          \includegraphics{../common/math/sspplots/rdie_lp_12000_R1_rect16_euclid_histo.pdf}%
         &\includegraphics{../common/math/sspplots/rdie_lp_12000_R1_rect16_euclid_auto.pdf}
         \\histogram & \ope{$2\xN$-offset autocorrelation} \xref{def:ocsRxx}
        \\\hline
     \end{tabular}

  \item \label{item:rdie_lp_R3_rect16_larc}
        Alternatively, suppose we next try
        using the \ope{$\R^3$ die random variable} \xref{def:rv_dieR3}
        to map the die sequence into $\R^3$.
        \ope{Filter}ing %\xref{def:filtern} 
        this new sequence using the \ope{length 16 rectangular low pass sequence}
        in the \structe{$\R^3$ distance linear space} \xref{def:dieR3oml} 
        and then mapping back to a \fncte{die sequence} %over the \structe{real die outcome subspace} 
        using the \fncte{Lagrange arc distance} yields the result partially displayed here:
        \\\includegraphics{../common/math/sspplots/rdie_lp_12000_R3_rect16_larc_seq.pdf}\\
        Note that the \fncte{die sequence} does appear to be ``smoothed out", 
        but this time the distribution is much more uniform, as illustrated below on the left;
        and is slightly less correlated (12795 compared to 17658), as illustrated below on the right.
     \\\begin{tabular}{|>{\scs}c|>{\scs}c|}
          \hline
          \includegraphics{../common/math/sspplots/rdie_lp_12000_R3_rect16_larc_histo.pdf}%
         &\includegraphics{../common/math/sspplots/rdie_lp_12000_R3_rect16_larc_auto.pdf}
        \\\hline
     \end{tabular}
        

  \item \label{item:rdie_lp_R3_hann16_larc}
        Using a \ope{length 16 Hanning low pass sequence} \xref{def:lp_hann}
        rather than the \fncte{length 16 rectangular low pass sequence} as in \pref{item:rdie_lp_R3_rect16_larc}
        results in a distribution that is more uniform and in a sequence that is very slightly less correlated:
        \\\includegraphics{../common/math/sspplots/rdie_lp_12000_R3_hann16_larc_seq.pdf}
     \\\begin{tabular}{|>{\scs}c|>{\scs}c|}
          \hline
          \includegraphics{../common/math/sspplots/rdie_lp_12000_R3_hann16_larc_histo.pdf}%
         &\includegraphics{../common/math/sspplots/rdie_lp_12000_R3_hann16_larc_auto.pdf}
        \\\hline
     \end{tabular}

  \item \label{item:rdie_lp_R3_hann50_larc}
        Using a \ope{length 50 Hanning low pass sequence} \xref{ex:lp_hann} rather than the 
        \fncte{length 16 Hanning low pass sequence} as in \pref{item:rdie_lp_R3_hann16_larc}
        results in about the same uniformity of distribution, about 1.8\% lower side lobes in the autocorrelation sequence
        ($\frac{12733-12505}{12733}\times100\approx1.8$),
        but a wider main lobe (presumably due to the longer filter width):
        %\\\includegraphics{../common/math/sspplots/rdie_lp_12000_R3_hann50_larc_seq.pdf}
     \\\begin{tabular}{|>{\scs}c|>{\scs}c|}
          \hline
          \includegraphics{../common/math/sspplots/rdie_lp_12000_R3_hann50_larc_histo.pdf}%
         &\includegraphics{../common/math/sspplots/rdie_lp_12000_R3_hann50_larc_auto.pdf}
        \\\hline
     \end{tabular}

  \item \label{item:rdie_lp_R3_rect50_larc}
        Using a \ope{length 50 rectangular low pass sequence} rather than the 
        \fncte{length 50 Hanning low pass sequence} as in \pref{item:rdie_lp_R3_hann50_larc}
        results in a distribution that is a little less uniform and about 3.3\% more correlated
        ($\abs{\frac{12505-12916}{12505}\times100}\approx3.3$):
        %\\\includegraphics{../common/math/sspplots/rdie_lp_12000_R3_rect50_larc_seq.pdf}
     \\\begin{tabular}{|>{\scs}c|>{\scs}c|}
          \hline
          \includegraphics{../common/math/sspplots/rdie_lp_12000_R3_rect50_larc_histo.pdf}%
         &\includegraphics{../common/math/sspplots/rdie_lp_12000_R3_rect50_larc_auto.pdf}
        \\\hline
     \end{tabular}

  \item \label{item:rdie_lp_R3_euclid}
        Replacing the \fncte{Lagrange arc distance} by the \fncte{Euclidean metric} in this example
        has very little effect.
        More details follow:
    \begin{enumerate}
      \item Using the \fncte{Euclidean metric} in $\R^3$ rather than the \fncte{Lagrange arc distance} in \pref{item:rdie_lp_R3_rect16_larc}
            yields sequences that are \textbf{identical}.\footnote{\seessp{rdie_lp_12000m16.xlg}}

      \item Using the \fncte{Euclidean metric} in \pref{item:rdie_lp_R3_hann16_larc} rather than the \fncte{Lagrange arc distance} 
            yields sequences that \textbf{differ} at 6 locations out of $N+M+M-1=12000+16+16-1=12031$ locations
            (differ at approximately 0.05\% of the locations):\footnote{\seessp{rdie_lp_12000m16.xlg}}
            \\\indentx\begin{tabstr}{0.75}\begin{tabular}{|c|c|c|}
                         \hline% 2016 June 08 Wednesday 11:03:38 PM UTC
                           n   & Euclidean & Lagrange
                         \\\hline
                             281 & $\dieF$ & $\dieD$ 
                         \\ 1630 & $\dieA$ & $\dieC$ 
                         \\11888 & $\dieF$ & $\dieB$ 
                         \\\hline
                       \end{tabular}\end{tabstr}

      \item Using the \fncte{Euclidean metric} in $\R^3$ rather than the \fncte{Lagrange arc distance} as in 
            \pref{item:rdie_lp_R3_hann50_larc} (length 50 Hanning filter)
            yields sequences that are \emph{identical}.\footnote{\seessp{rdie_lp_12000m50.xlg}} % 2016 June 08 Wednesday 11:07:28 PM UTC

      \item Using the \fncte{Euclidean metric} in $\R^3$ rather than the \fncte{Lagrange arc distance} as 
            in \pref{item:rdie_lp_R3_rect50_larc} (length 50 rectangular filter)
            \textbf{differ} at 85 locations out of $N+M+M-1=12000+50+50-1=12099$ locations
            (differ at approximately 0.7\% of the locations).\footnote{\seessp{rdie_lp_12000m50.xlg}} % 2016 June 08 Wednesday 11:07:28 PM UTC
    \end{enumerate}

\end{enumerate}
\end{example}


%=======================================
%\subsubsection{Low pass filtering of spinner sequences}
%=======================================
%---------------------------------------
\begin{example}[\exmd{low pass filtering of spinner sequence}]
\label{ex:spin_lp}
\addcontentsline{toc}{subsubsection}{* low pass filtering of spinner sequence}
\mbox{}\\
%---------------------------------------
\begin{enumerate}
  \item \label{item:spin_lp_seq}
     Consider the pseudo-uniformly distributed spinner sequence presented in \prefpp{ex:spinner_sha}.
     As in \prefpp{ex:rdie_lp}, suppose we want to \ope{filter} this sequence with a
     \fncte{low pass rectangular sequence} in order to ``smooth out" the sequence.
     %But to perform the actual filtering, note that the die sequence
     %must first be mapped into a \structe{linear space} $\R^\xN$.

  \item \label{item:spin_lp_R1_rect16_euclid}
        Suppose we first use the \ope{$\R^1$ spinner random variable} \xref{def:rv_spinR1} 
        to map the spinner sequence into $\R^1$.
        \ope{Filter}ing this mapped sequence using the \fncte{length 16 rectangular low pass sequence} 
        and then mapping the result back to a \fncte{spinner sequence} %over the \structe{spinner outcome subspace} \xref{def:rdie}
        using the \fncte{Euclidean metric}, produces the result partially dispayed here 
        (in essense the same as in \prefp{ex:rdie_lp}):
        \\\includegraphics{../common/math/sspplots/spin_lp_12000_R1_rect16_euclid_seq.pdf}\\
        Again, it's uniform distribution has been essentially destroyed.
        %---almost all  of its values are around the ``expected value" 3.5.
        \\\begin{tabular}{|>{\scs}c|>{\scs}c|}
             \hline
             \includegraphics{../common/math/sspplots/spin_lp_12000_R1_rect16_euclid_histo.pdf}%
            &\includegraphics{../common/math/sspplots/spin_lp_12000_R1_rect16_euclid_auto.pdf}
           \\\hline
        \end{tabular}


  \item \label{item:spin_lp_R2_rect16_larc}
        Alternatively, suppose we next try
        using the \ope{QPSK spinner random variable} \xref{def:rv_spinqpsk}
        to map the spinner sequence into $\Cnum\eqd\R^2$.
        \ope{Filter}ing this new sequence using the \ope{length 16 rectangular low pass sequence}
        in the \structe{$\R^2$ spinner distance linear space} \xref{def:spinR2oml}
        and then mapping back to a \fncte{sequence} over the \structe{spinner outcome subspace} 
        using the \fncte{Lagrange arc distance} yields the result partially displayed here:
        \\\includegraphics{../common/math/sspplots/spin_lp_12000_R2_rect16_larc_seq.pdf}\\
        Note that the sequence does appear to be ``smoothed out", 
        but this time the distribution is much more uniform and about 
        69\% less correlated than the $\R^1$ method of \pref{item:spin_lp_R1_rect16_euclid}:
        \\\begin{tabular}{|>{\scs}c|>{\scs}c|}
             \hline
             \includegraphics{../common/math/sspplots/spin_lp_12000_R2_rect16_larc_histo.pdf}%
            &\includegraphics{../common/math/sspplots/spin_lp_12000_R2_rect16_larc_auto.pdf}
           \\\hline
        \end{tabular}\\
        Furthermore, it is about 58\% less correlated than the $\R^3$ filtering for the die sequence
        used in \pref{item:rdie_lp_R3_rect16_larc} of \prefpp{ex:rdie_lp}.

  \item Using the \fncte{Euclidean metric} rather than the \fncte{Lagrange arc distance} as in \pref{item:spin_lp_R2_rect16_larc}
        results in a sequence that differs at 99 different locations out of $N+M+M-1=12000+16+15=12031$ locations
        (approximately 0.8\% of the locations differ).\footnote{\seessp{spin_lp_12000m16.xlg}} % 2016 June 08 Wednesday 11:11:16 PM UTC
        \\\begin{tabular}{|>{\scs}c|>{\scs}c|}
             \hline
             \includegraphics{../common/math/sspplots/spin_lp_12000_R2_rect16_euclid_histo.pdf}%
            &\includegraphics{../common/math/sspplots/spin_lp_12000_R2_rect16_euclid_auto.pdf}
           \\\hline
        \end{tabular}

  \item \label{item:spin_lp_R2_hann16_larc}
        Using a \ope{length 16 Hanning low pass sequence} rather than the 
        \fncte{length 16 Rectangular low pass sequence} as in \pref{item:spin_lp_R2_rect16_larc}
        results in a distribution that is more uniform and about 5.3\% less correlated:
        \\\includegraphics{../common/math/sspplots/spin_lp_12000_R2_hann16_larc_seq.pdf}
     \\\begin{tabular}{|>{\scs}c|>{\scs}c|}
          \hline
          \includegraphics{../common/math/sspplots/spin_lp_12000_R2_hann16_larc_histo.pdf}%
         &\includegraphics{../common/math/sspplots/spin_lp_12000_R2_hann16_larc_auto.pdf}
        \\\hline
     \end{tabular}

  \item \begin{minipage}[t]{\tw-70mm}%
          Using the \fncte{Euclidean metric} rather than the \fncte{Lagrange arc distance} as in \pref{item:spin_lp_R2_hann16_larc}
          results in a sequence that differs at exactly 2 locations (approximately 0.017\%) 
          out of $12031$ locations:\footnotemark
        \end{minipage}%
        \hfill\citetblt{\seessp{spin_lp_12000m16.xlg}} % 2016 June 08 Wednesday 11:11:16 PM UTC
        %\begin{minipage}{60mm}%
          \begin{tabstr}{0.75}\begin{tabular}[t]{|c|c|c|}
            \hline
              n   & Euclidean & Lagrange
            \\\hline
               4149 & $\dieD$   & $\dieE$
            \\ 5594 & $\dieB$   & $\dieA$
            \\\hline
          \end{tabular}\end{tabstr}%

  \item \label{item:spin_lp_R2_hann50_larc}
        Using a \ope{length 50 Hanning low pass sequence} rather than the 
        \fncte{length 16 Hanning low pass sequence} as in \pref{item:spin_lp_R2_hann16_larc}
        results in the following:%about the same uniformity of distribution, slightly lower side lobes in the autocorrelation function,
        %but a wider main lobe (due to the longer filter width):
        %\\\includegraphics{../common/math/sspplots/spin_lp_12000_R2_hann50_larc_seq.pdf}
     \\\begin{tabular}{|>{\scs}c|>{\scs}c|}
          \hline
          \includegraphics{../common/math/sspplots/spin_lp_12000_R2_hann50_larc_histo.pdf}%
         &\includegraphics{../common/math/sspplots/spin_lp_12000_R2_hann50_larc_auto.pdf}
        \\\hline
     \end{tabular}

  \item \label{item:spin_lp_R2_rect50_larc}
        Using a \ope{length 50 Rectangular low pass sequence} rather than the 
        \fncte{length 50 Hanning low pass sequence} as in \pref{item:spin_lp_R2_hann50_larc}
        results in the following: %a little less uniform:
        %\\\includegraphics{../common/math/sspplots/spin_lp_12000_R2_rect50_larc_seq.pdf}
     \\\begin{tabular}{|>{\scs}c|>{\scs}c|}
          \hline
          \includegraphics{../common/math/sspplots/spin_lp_12000_R2_rect50_larc_histo.pdf}%
         &\includegraphics{../common/math/sspplots/spin_lp_12000_R2_rect50_larc_auto.pdf}
        \\\hline
     \end{tabular}
\end{enumerate}
\end{example}



%=======================================
%\subsubsection{Low pass filtering of fair die sequences}
%=======================================
%---------------------------------------
\begin{example}[\exmd{low pass filtering of fair die sequence}]
\label{ex:fdie_lp}\mbox{}\\
\addcontentsline{toc}{subsubsection}{* low pass filtering of fair die sequence}
%---------------------------------------
\begin{enumerate}
  \item \label{item:fdie_lp_seq}
     Consider the pseudo-uniformly distributed die sequence presented in \prefpp{ex:fdie_sha}.
     Suppose we want to \ope{filter} %\xref{def:filter} 
     this sequence with a
     \fncte{low pass sequence} %\xref{ex:lp_rect} 
     in order to ``smooth out" the sequence, just as in \prefpp{ex:rdie_lp}.
     %But to perform the actual filtering, note that the die sequence
     %must first be mapped into a \structe{linear space} $\R^\xN$.

  \item \label{item:fdie_lp_R1_rect16_euclid}
        Suppose we first use the \ope{traditional die random variable} \xref{def:rv_dietrad} 
        to map the die sequence into $\R^1$.
        \ope{Filter}ing this mapped sequence using the \fncte{length 16 rectangular low pass sequence} 
        and then mapping the result back to a \fncte{die sequence} % over the \structe{real die outcome subspace} %\xref{def:rdie}
        using the \fncte{Euclidean metric}, produces a result identical to that of 
        \prefpp{item:rdie_lp_R1_rect16_euclid} of \pref{ex:rdie_lp}.
  

  \item \label{item:fdie_lp_R6_rect16_euclid}
        Alternatively, suppose we next 
        use the \ope{$\R^6$ die random variable} \xref{def:rv_dieR6}
        to map the die sequence into $\R^6$.
        \ope{Filter}ing this new sequence using the \ope{length 16 rectangular low pass sequence}
        in the \structe{$\R^6$ die distance linear space} \xref{def:fdieR6oml} 
        and then mapping back to a \fncte{die sequence} %over the \structe{fair die outcome subspace} 
        using the \fncte{Euclidean metric} yields a much more uniform distribution and 
        a sequence that is about 28\% less correlated.
        \\\includegraphics{../common/math/sspplots/fdie_lp_12000_R6_rect16_euclid_seq.pdf}%
     \\\begin{tabular}{|>{\scs}c|>{\scs}c|}
          \hline
          \includegraphics{../common/math/sspplots/fdie_lp_12000_R6_rect16_euclid_histo.pdf}%
         &\includegraphics{../common/math/sspplots/fdie_lp_12000_R6_rect16_euclid_auto.pdf}
        \\\hline
     \end{tabular}\\
     Note further that this $\R^6$ technique yeilds a sequence that is about 9.9\% more correlated than 
     yielded by the $\R^3$ technique
     used in \pref{item:rdie_lp_R3_rect16_larc} of \prefpp{ex:rdie_lp}.

  \item \label{item:fdie_lp_R6_hann16_euclid}
        Using a \ope{length 16 Hanning low pass sequence} rather than the 
        \fncte{length 16 Rectangular low pass sequence} as in \pref{item:fdie_lp_R6_rect16_euclid}
        results in a distribution that is more uniform and a sequence that is about 0.12\% less correlated:
        \\\includegraphics{../common/math/sspplots/fdie_lp_12000_R6_hann16_euclid_seq.pdf}
     \\\begin{tabular}{|>{\scs}c|>{\scs}c|}
          \hline
          \includegraphics{../common/math/sspplots/fdie_lp_12000_R6_hann16_euclid_histo.pdf}%
         &\includegraphics{../common/math/sspplots/fdie_lp_12000_R6_hann16_euclid_auto.pdf}
        \\\hline
     \end{tabular}\\
     Note further that this $\R^6$ technique yields a sequence that is about 10\% more correlated than 
     yielded by the $\R^3$ technique
     used in \pref{item:rdie_lp_R3_hann16_larc} of \prefpp{ex:rdie_lp}.

  \item \label{item:fdie_lp_R6_hann50_euclid}
        Using a \ope{length 50 Hanning low pass sequence} rather than the 
        \fncte{length 16 Hanning low pass sequence} as in \pref{item:fdie_lp_R6_hann16_euclid}
        results in the following: %about the same uniformity of distribution, slightly lower side lobes in the autocorrelation function,
        %but a wider main lobe (due to the longer filter width):
     \\\begin{tabular}{|>{\scs}c|>{\scs}c|}
          \hline
          \includegraphics{../common/math/sspplots/fdie_lp_12000_R6_hann50_euclid_histo.pdf}%
         &\includegraphics{../common/math/sspplots/fdie_lp_12000_R6_hann50_euclid_auto.pdf}
        \\\hline
     \end{tabular}\\
     Note further that this $\R^6$ technique yields a sequence that is about 11\% more correlated than 
     yielded by the $\R^3$ technique
     used in \pref{item:rdie_lp_R3_hann50_larc} of \prefpp{ex:rdie_lp}.

  \item \label{item:fdie_lp_R6_rect50_euclid}
        Using a \ope{length 50 Rectangular low pass sequence} rather than the 
        \fncte{length 50 Hanning low pass sequence} as in \pref{item:fdie_lp_R6_hann50_euclid}
        results in the following: %a distribution that is a little less uniform:
     \\\begin{tabular}{|>{\scs}c|>{\scs}c|}
          \hline
          \includegraphics{../common/math/sspplots/fdie_lp_12000_R6_rect50_euclid_histo.pdf}%
         &\includegraphics{../common/math/sspplots/fdie_lp_12000_R6_rect50_euclid_auto.pdf}
        \\\hline
     \end{tabular}\\
     Note further that this $\R^6$ technique yeilds a sequence that is about 8.8\% more correlated than yeilded by the $\R^3$ technique
     used in \pref{item:rdie_lp_R3_rect50_larc} of \prefpp{ex:rdie_lp}.

  \item In the \structe{fair die outcome space}, the \fncte{Lagrange arc distance} does not seem so appropriate.
        That being said however, \ldots 
    \begin{enumerate}
      \item using the \fncte{Lagrange arc distance} rather than the \fncte{Euclidean metric} in 
            \prefp{item:fdie_lp_R6_rect16_euclid} yields results that are 
            \textbf{identical}\footnote{\seessp{fdie_lp_12000m16.xlg}} %2016 June 08 Wednesday 11:18:43 PM UTC

      \item using the \fncte{Lagrange arc distance} rather than the \fncte{Euclidean metric} in 
            \prefp{item:fdie_lp_R6_hann16_euclid} yields results that \textbf{differ} at 5 locations
            (differ at approximately 0.04\% of the total possible 
            $N+M+M-1=12000+16+16-1=12031$ locations):\footnote{\seessp{fdie_lp_12000m16.xlg}} %2016 June 08 Wednesday 11:18:43 PM UTC
            \\\indentx\begin{tabstr}{0.75}\begin{tabular}[t]{|c|c|c|}
                         \hline
                           n   & Euclidean & Lagrange
                         \\\hline
                              430 & $\dieA$ & $\dieE$
                         \\  2181 & $\dieE$ & $\dieC$
                         \\  5055 & $\dieB$ & $\dieA$
                         \\\hline
                       \end{tabular}\end{tabstr}
              \indentx\begin{tabstr}{0.75}\begin{tabular}[t]{|c|c|c|}
                         \hline
                           n   & Euclidean & Lagrange
                         \\\hline
                             8688 & $\dieD$ & $\dieC$
                         \\ 10866 & $\dieD$ & $\dieE$
                         \\\hline
                       \end{tabular}\end{tabstr}

      \item using the \fncte{Lagrange arc distance} rather than the \fncte{Euclidean metric} in 
            \prefp{item:fdie_lp_R6_hann50_euclid} yields results that are 
            \textbf{identical}\footnote{\seessp{fdie_lp_12000m50.xlg}} %2016 June 08 Wednesday 11:23:15 PM UTC

      \item using the \fncte{Lagrange arc distance} rather than the \fncte{Euclidean metric} in 
            \prefp{item:fdie_lp_R6_rect50_euclid} yields results that \textbf{differ} at 289 locations
            (differ at approximately 2.4\% of the total possible 
            $12031$ locations):\footnote{\seessp{fdie_lp_12000m50.xlg}} %2016 June 08 Wednesday 11:23:15 PM UTC
    \end{enumerate}

  \item Empirical evidence observed in items
                    \ref{item:fdie_lp_R6_rect16_euclid},
                    \ref{item:fdie_lp_R6_hann16_euclid},  
                    \ref{item:fdie_lp_R6_hann50_euclid}, and
                    \ref{item:fdie_lp_R6_rect50_euclid},
        suggests that the $\R^6$ technique of this example leads to about 10\% more correlation 
        than the $\R^3$ technique of \prefpp{ex:rdie_lp}.
\end{enumerate}
\end{example}






  %============================================================================
% Daniel J. Greenhoe
% LaTeX file
%============================================================================
%=======================================
\subsection{High pass filtering}
\label{sec:hp}
%=======================================
%---------------------------------------
\begin{example}[\exmd{high pass filtering of weighted real die sequence}]
\label{ex:wrdie_hp}
\addcontentsline{toc}{subsubsection}{* high pass filtering of weighted real die sequence}
\mbox{}\\
%---------------------------------------
\begin{enumerate}
  \item \label{item:wrdie_hp_R1_rect50}
        Consider a length $50(1200+2)-(50-1)=60051$ non-uniformly distributed die sequence generated as
        described in \prefpp{ex:wrdie_sha}.
        To remove the strong $\dieE$ bais, we could map and \ope{filter} \xref{def:filter} the sequence with the 
        \ope{length 50 high pass rectangular sequence} \xref{def:hp_rect}.
        Such filtering will obviously introduce correlation into the die sequence. 
        The low pass filtering of \prefp{ex:rdie_lp} (``smoothing") also introduced correlation,
        but wanting a ``smooth" sequence informally implies a willingness to accept a highly correlated sequence.
        However in this current example, we would prefer to have an \prope{uncorrelated} sequence.
        To negate the correlation introduced by filtering, 
        we \ope{down sample} \xref{def:downsample} the filtered sequence by a factor of 50 and
        remove the first and last element, leaving a sequence of length $1200$.

  \item \label{item:wrdie_hp_R1_rect50_euclid}
        If the filtering and downsampling described in \pref{item:wrdie_hp_R1_rect50} 
        is performed in the traditional $\R^1$ space,
        then after mapping back to a \structe{die sequence} using the \fncte{Euclidean metric},
        we obtain the result partially dispayed here\ldots
          \\\includegraphics{ssp/plots/wrdie_hp_1200_R1_rect50_euclid_seq.pdf}\\
        where the bias at $\dieE$ has been replaced by a new bias at $\dieA$, 
        as illustrated quantitatively below on the left, 
        calculated over $\xN=1200$ elements. 
          \\\begin{tabular}{|>{\scs}c|>{\scs}c|}
               \hline
               \includegraphics{ssp/plots/wrdie_hp_1200_R1_rect50_euclid_histo.pdf}
              &\includegraphics{ssp/plots/wrdie_hp_1200_R1_rect50_euclid_auto.pdf}
             \\\hline
          \end{tabular}

  \item \label{item:wrdie_hp_R3_rect50_larc}
        Alternatively, suppose we next use the \ope{$\R^3$ die random variable} \xref{def:rv_dieR3}
        to map the die sequence into $\R^3$.
        \ope{Filter}ing this new sequence using the \ope{length 50 rectangular high pass sequence}
        in the \structe{$\R^3$ distance linear space} \xref{def:dieR3oml} 
        and then mapping back to a \fncte{die sequence} %over the \structe{real die outcome subspace} 
        using the \fncte{Lagrange arc distance} yields the following results:
        \\\includegraphics{ssp/plots/wrdie_hp_1200_R3_rect50_larc_seq.pdf}
        \\\begin{tabular}{|>{\scs}c|>{\scs}c|}
             \hline
             \includegraphics{ssp/plots/wrdie_hp_1200_R3_rect50_larc_histo.pdf}
            &\includegraphics{ssp/plots/wrdie_hp_1200_R3_rect50_larc_auto.pdf}
           \\\hline
        \end{tabular}\\
        Note that neither the $\R^1$ method of \pref{item:wrdie_hp_R1_rect50_euclid}
        nor the $\R^3$ method of \pref{item:wrdie_hp_R3_rect50_larc}
        yields a uniformly distributed sequence; 
        but the $\R^3$ method at least comes significantly closer to this end.
        Moreover, the $\R^3$ method also yields a sequence that is less correlated.

  \item \label{item:wrdie_hp_R3_hann50_larc}
        Replacing the \ope{length 50 rectangular high pass filter} in \pref{item:wrdie_hp_R3_rect50_larc}
        with the \ope{length 50 Hanning high pass filter} \xref{def:hp_hann} yields a different sequence
        with similar distribution but is slightly more correlated:
        \\\includegraphics{ssp/plots/wrdie_hp_1200_R3_hann50_larc_seq.pdf}
        \\\begin{tabular}{|>{\scs}c|>{\scs}c|}
             \hline
             \includegraphics{ssp/plots/wrdie_hp_1200_R3_hann50_larc_histo.pdf}
            &\includegraphics{ssp/plots/wrdie_hp_1200_R3_hann50_larc_auto.pdf}
           \\\hline
        \end{tabular}

  \item \label{item:wrdie_hp_R3_euclid}
        Replacing the \fncte{Lagrange arc distance} by the \fncte{Euclidean metric} in this example
        has very little effect, even before downsampling.
        Before downsampling, the length of each sequence is $M(N+2)=50(1202)=60100$ elements.
        More details follow:
        \begin{enumerate}
          \item Using the \fncte{Euclidean metric} rather than the \fncte{Lagrange arc distance} 
                in \pref{item:wrdie_hp_R3_rect50_larc} yields results that are 
                \textbf{identical}.\citetbl{\seessp{wrdie_hp_1200m50.xlg}} %2016 June 08 Wednesday 11:27:40 PM UTC

          \item \label{item:wrdie_hp_R3_hann50_euclid}
                Using the \fncte{Euclidean metric} rather than the \fncte{Lagrange arc distance} 
                in \pref{item:wrdie_hp_R3_hann50_larc}
                yields results that \textbf{differ} at 4 locations 
                (approximately 0.007\% of all the locations).\citetbl{\seessp{wrdie_hp_1200m50.xlg}} %2016 June 08 Wednesday 11:27:40 PM UTC

        \end{enumerate}

  \item For the type of sequence processing described in this example, 
         \pref{item:wrdie_hp_R3_euclid} very informally \emph{suggests} the following:
          \begin{enumerate}
            \item The processing is not highly sensitive to the choice of distance function.
            \item The processing is not heavily dependent on the \prope{triangle inequality}.
            %\item The sensitivity to the choice of distance function, at least in the Hanning case, 
            %      decreases with filter length.
          \end{enumerate}
\end{enumerate}
\end{example}

%=======================================
%\subsubsection{High pass filtering of spinner sequences}
%=======================================
%---------------------------------------
\begin{example}[\exmd{high pass filtering of weighted spinner sequence}]
\label{ex:spin_rhp}
\addcontentsline{toc}{subsubsection}{* high pass filtering of weighted spinner sequence}
\mbox{}\\
%---------------------------------------
\begin{enumerate}
  \item \label{item:wspin_hp_seq} \label{item:wspin_hp_rect50}
    Consider a length $50(1200+2)-(50-1)=60051$  non-uniformly distributed \fncte{spinner sequence} 
    generated as described in \prefpp{ex:wspin_sha}.
    To remove the strong $\spinE$ bais, we could \ope{filter} the \fncte{sequence} with 
    the \fncte{length 50 high pass rectangular sequence} %\xref{def:hp_rect}. 
    and down sample the filtered sequence by a factor of 50, as described in \prefpp{ex:wrdie_hp}.

  \item \label{item:wspin_hp_rect50_R1}
        If the filtering described in \pref{item:wspin_hp_rect50} is performed in the traditional $\R^1$ space,
        then after mapping back to a \structe{spinner sequence}
        using the \fncte{Euclidean metric},
        we obtain the result partially dispayed here\ldots
          \\\includegraphics{ssp/plots/wspin_hp_1200_R1_rect50_euclid_seq.pdf}\\
        where the bias at $\spinE$ has been replaced by a new bias at $\spinA$, 
        as illustrated quantitatively below on the left, 
        calculated over 1200 elements. %$\floor{(160000+50-1)/50}-1=3199$ elements.
          \\\begin{tabular}{|>{\scs}c|>{\scs}c|}
               \hline
               \includegraphics{ssp/plots/wspin_hp_1200_R1_rect50_euclid_histo.pdf}
              &\includegraphics{ssp/plots/wspin_hp_1200_R1_rect50_euclid_auto.pdf}
             %\\\mc{2}{|>{\scs}c|}{calculated over $\floor{(160000+50-1)/50}-1=3199$ elements}
             \\\hline
          \end{tabular}


  \item \label{item:wspin_hp_hann50}\label{item:wspin_hp_hann50_R1}
        If we replace the \ope{length 50 rectangular high pass filter} of \pref{item:wspin_hp_rect50_R1}
        with a \ope{length 50 Hanning high pass filter}
        then we obtain the result partially dispayed here\ldots
          \\\includegraphics{ssp/plots/wspin_hp_1200_R1_hann50_euclid_seq.pdf}\\
        \ldots where the bias at $\spinE$ again has been replaced by a new bias at $\spinA$:
          \\\begin{tabular}{|>{\scs}c|>{\scs}c|}
               \hline
               \includegraphics{ssp/plots/wspin_hp_1200_R1_hann50_euclid_histo.pdf}
              &\includegraphics{ssp/plots/wspin_hp_1200_R1_hann50_euclid_auto.pdf}
             \\\hline
          \end{tabular}


  \item \label{item:wspin_hp_rect50_R2_larc}
        If the rectangular filtering in $\R^1$ of \pref{item:wspin_hp_rect50_R1}
        is instead performed in $\R^2$ 
        and mapped back to a \structe{spinner sequence} using the \fncte{Lagrange arc distance},
        then we obtain the result partially dispayed here:
        \\\includegraphics{ssp/plots/wspin_hp_1200_R2_rect50_larc_seq.pdf}
        \\\begin{tabular}{|>{\scs}c|>{\scs}c|}
             \hline
             \includegraphics{ssp/plots/wspin_hp_1200_R2_rect50_larc_histo.pdf}
            &\includegraphics{ssp/plots/wspin_hp_1200_R2_rect50_larc_auto.pdf}
           \\\hline
        \end{tabular}\\
        %\\\includegraphics{ssp/plots/wspin_hp_1200_R2_rect50_euclid_seq.pdf}
        %\\\begin{tabular}{|>{\scs}c|>{\scs}c|}
        %     \hline
        %     \includegraphics{ssp/plots/wspin_hp_1200_R2_rect50_euclid_histo.pdf}
        %    &\includegraphics{ssp/plots/wspin_hp_1200_R2_rect50_euclid_auto.pdf}
        %   \\\hline
        %\end{tabular}\\
        Note that neither the $\R^1$ methods 
        (described in \pref{item:wspin_hp_rect50_R1} and \pref{item:wspin_hp_hann50_R1})
        nor the $\R^2$ method (described in \pref{item:wspin_hp_rect50_R2_larc})
        yields a uniformly distributed sequence; 
        but the $\R^2$ method at least comes significantly closer to this end.

  \item \label{item:wspin_hp_rect50_R2_euclid}
        Replacing the \fncte{Lagrange arc distance} by the \fncte{Euclidean metric} as in \pref{item:wspin_hp_rect50_R2_larc}
        yields a sequence that differs at a total of 272 locations 
        (approximately 0.5\% of the locations).\citetbl{\seessp{wspin_hp_1200m50.xlg}} %2016 June 08 Wednesday 11:31:09 PM UTC

  \item \label{item:wspin_hp_hann50_R2_larc}
        If instead of using the rectangular filtering (as in \pref{item:wspin_hp_rect50_R2_larc}),
        we use the Hanning filtering of \pref{item:wspin_hp_hann50}
        in $\R^2$ and map back to a \structe{spinner sequence} using the \fncte{Lagrange arc distance},
        then we obtain the result partially dispayed here:
        \\\includegraphics{ssp/plots/wspin_hp_1200_R2_hann50_larc_seq.pdf}
        \\\begin{tabular}{|>{\scs}c|>{\scs}c|}
             \hline
             \includegraphics{ssp/plots/wspin_hp_1200_R2_hann50_larc_histo.pdf}
            &\includegraphics{ssp/plots/wspin_hp_1200_R2_hann50_larc_auto.pdf}
           \\\hline
        \end{tabular}
        %\\\includegraphics{ssp/plots/wspin_hp_1200_R2_hann50_euclid_seq.pdf}
        %\\\begin{tabular}{|>{\scs}c|>{\scs}c|}
        %     \hline
        %     \includegraphics{ssp/plots/wspin_hp_1200_R2_hann50_euclid_histo.pdf}
        %    &\includegraphics{ssp/plots/wspin_hp_1200_R2_hann50_euclid_auto.pdf}
        %   \\\hline
        %\end{tabular}

  \item \label{item:wspin_hp_hann50_R2_euclid}
        Replacing the \fncte{Lagrange arc distance} by the \fncte{Euclidean metric} in \pref{item:wspin_hp_hann50_R2_larc}
        %yields a very similar result: % but different result, as illustrated next:
        yields a sequence that differs at a total of 3 locations 
        (approximately 0.005\% of the locations).\citetbl{\seessp{wspin_hp_1200m50.xlg}} %2016 June 08 Wednesday 11:31:09 PM UTC
        %\\\begin{tabular}{|>{\scs}c|>{\scs}c|}
        %     \hline
        %     \includegraphics{ssp/plots/wspin_hp_1200_R2_hann50_euclid_histo.pdf}
        %    &\includegraphics{ssp/plots/wspin_hp_1200_R2_hann50_euclid_auto.pdf}
        %   \\\hline
        %\end{tabular}  
\end{enumerate}
\end{example}



%=======================================
%\subsubsection{High pass filtering of fair die sequences}
%=======================================
%---------------------------------------
\begin{example}[\exmd{high pass filtering of weighted die sequence}]
\label{ex:wdie_hp}
\addcontentsline{toc}{subsubsection}{* high pass filtering of weighted die sequence}
\mbox{}\\
%---------------------------------------
\begin{enumerate}
  \item \label{item:wdie_hp_seq}
        Consider a length $50(1200+2)-(50-1)=60051$  \fncte{weighted die sequence} generated as 
        described in \prefpp{ex:wdie_sha}.
        To remove the strong $\dieE$ bais, we could map and \ope{filter} the sequence with the 
        \ope{length 16 high pass rectangular sequence} \xref{ex:hp_rect}.
        To negate the correlation introduced by filtering, 
        we \ope{down sample} the filtered sequence by a factor of 16.

  \item \label{item:wdie_hp_R1_rect16_euclid}
        If the die sequence of \pref{item:wdie_hp_seq} is mapped into $\R^1$ using the
        \fncte{traditional die random variable} \xref{def:rv_dietrad}, \ope{filter}ed,
        \ope{down sample}d, and mapped back to a die sequence using the \fncte{Euclidean metric},
        we obtain the result partially dispayed here\ldots
          \\\includegraphics{ssp/plots/wrdie_hp_1200_R1_rect50_euclid_seq.pdf}\\
        where the bias at $\dieE$ has been replaced by a new bias at $\dieA$, 
        as illustrated quantitatively below on the left, 
        calculated over 1200 elements. 
          \\\begin{tabular}{|>{\scs}c|>{\scs}c|}
               \hline
               \includegraphics{ssp/plots/wdie_hp_1200_R1_rect16_euclid_histo.pdf}
              &\includegraphics{ssp/plots/wdie_hp_1200_R1_rect16_euclid_auto.pdf}
             \\\hline
          \end{tabular}

  \item \label{item:wdie_hp_R6_rect16_euclid}
        But if instead of processing the die sequence in $\R^1$ as in \pref{item:wdie_hp_R1_rect16_euclid},
        processing is performed in $\R^6$ 
        and mapped back to a die sequence using the \fncte{Euclidean metric},
        then we obtain the result partially dispayed here:
        \\\includegraphics{ssp/plots/wdie_hp_1200_R6_rect16_euclid_seq.pdf}
        \\\begin{tabular}{|>{\scs}c|>{\scs}c|}
             \hline
             \includegraphics{ssp/plots/wdie_hp_1200_R6_rect16_euclid_histo.pdf}
            &\includegraphics{ssp/plots/wdie_hp_1200_R6_rect16_euclid_auto.pdf}
           \\\hline
        \end{tabular}

  \item \label{item:wdie_hp_R6_hann16_euclid}
        Replacing the \ope{length 16 rectangular sequence} in \pref{item:wdie_hp_R6_rect16_euclid}
        with a \ope{length 16 Hanning sequence} in $\R^6$ yields the following results:
        \\\includegraphics{ssp/plots/wdie_hp_1200_R6_hann16_euclid_seq.pdf}
        \\\begin{tabular}{|>{\scs}c|>{\scs}c|}
             \hline
             \includegraphics{ssp/plots/wdie_hp_1200_R6_hann16_euclid_histo.pdf}
            &\includegraphics{ssp/plots/wdie_hp_1200_R6_hann16_euclid_auto.pdf}
           \\\hline
        \end{tabular}

  \item \label{item:wdie_hp_R6_hann50_euclid}
        Replacing the \ope{length 16 Hanning sequence} in \pref{item:wdie_hp_R6_hann16_euclid}
        with a \ope{length 50 Hanning sequence} in $\R^6$ yields the following results:
        \\\includegraphics{ssp/plots/wdie_hp_1200_R6_hann50_euclid_seq.pdf}
        \\\begin{tabular}{|>{\scs}c|>{\scs}c|}
             \hline
             \includegraphics{ssp/plots/wdie_hp_1200_R6_hann50_euclid_histo.pdf}
            &\includegraphics{ssp/plots/wdie_hp_1200_R6_hann50_euclid_auto.pdf}
           \\\hline
        \end{tabular}\\
     Note that this $\R^6$ technique yields a sequence that is about 8.7\% more correlated than yielded by the $\R^3$ technique
     used in \pref{item:wrdie_hp_R3_hann50_larc} of \prefpp{ex:wrdie_hp}.

  \item \label{item:wdie_hp_R6_rect50_euclid}
        Replacing the \ope{length 50 Hanning sequence} in \pref{item:wdie_hp_R6_hann50_euclid}
        with a \ope{length 50 rectangular sequence} in $\R^6$ yields the following results:
        \\\includegraphics{ssp/plots/wdie_hp_1200_R6_rect50_euclid_seq.pdf}
        \\\begin{tabular}{|>{\scs}c|>{\scs}c|}
             \hline
             \includegraphics{ssp/plots/wdie_hp_1200_R6_rect50_euclid_histo.pdf}
            &\includegraphics{ssp/plots/wdie_hp_1200_R6_rect50_euclid_auto.pdf}
           \\\hline
        \end{tabular}\\
     Note that this $\R^6$ technique yields a sequence that is about 7.3\% more correlated than yielded by the $\R^3$ technique
     used in \pref{item:wrdie_hp_R3_rect50_larc} of \prefpp{ex:wrdie_hp}.

  \item As in \prefpp{ex:fdie_lp}, here again the \fncte{Lagrange arc distance} does not seem so appropriate.
        That again being said however, \ldots 
    \begin{enumerate}
      \item using the \fncte{Lagrange arc distance} rather than the \fncte{Euclidean metric} in 
            \prefp{item:wdie_hp_R6_rect16_euclid} yields results that are 
            \textbf{identical}.\citetbl{\seessp{wdie_hp_1200m16.xlg}} %2016 June 08 Wednesday 11:34:23 PM UTC
      \item using the \fncte{Lagrange arc distance} rather than the \fncte{Euclidean metric} in 
            \prefp{item:wdie_hp_R6_hann16_euclid} yields results that \textbf{differ} at 17 locations
            (differ at approximately 0.09\% of the total possible $M(N+2)=16(1200+2)=19232$ 
            locations).\citetbl{\seessp{wdie_hp_1200m16.xlg}} %2016 June 08 Wednesday 11:34:23 PM UTC
    \end{enumerate}

  \item Empirical evidence observed in items
                    \pref{item:wdie_hp_R6_hann50_euclid} and
                    \pref{item:wdie_hp_R6_rect50_euclid}
        suggests that the $\R^6$ technique of this example leads to about 8\% more correlation 
        than the $\R^3$ technique of \prefpp{ex:wrdie_hp}.

\end{enumerate}
\end{example}




  %============================================================================
% Daniel J. Greenhoe
% LaTeX file
%============================================================================
%=======================================
\subsection{Fourier Analysis}
%=======================================
%=======================================
%\subsubsection{Fourier analysis of die sequences}
%=======================================
%---------------------------------------
\begin{example}[\exmd{length 1200 non-stationary die sequence with 10Hz oscillating mean}]
\label{ex:nonstat34}
\addcontentsline{toc}{subsubsection}{* length 1200 non-stationary die sequence with 10Hz oscillating mean}
\mbox{}\\
%---------------------------------------
\begin{enumerate}
  \item \label{item:nonstat34_psp}
     Suppose we have a length $\xN\eqd1200$ die sequence $\seqn{x_n}$ with the following distribution:
     \\\indentx$\begin{array}{lDlD}
       \psp(\dieA)=\psp(\dieB)=\psp(\dieD)=\psp(\dieE)=\psp(\dieF)=0.15 &and& \psp(\dieC)=0.25            &     \\
       \mc{3}{M}{\qquad for $n\in\set{p+ (2m)\frac{\xM}{2}}{p=0,1,\ldots,\frac{\xM}{2}-1,\,m=0,1,2,\ldots,9}$} & and  \\
       \psp(\dieA)=\psp(\dieB)=\psp(\dieC)=\psp(\dieE)=\psp(\dieF)=0.15 &and& \psp(\dieD)=0.25 &                \\
       \mc{3}{M}{\qquad for $n\in\set{p+(2m+1)\frac{\xM}{2}}{p=0,1,\ldots,\frac{\xM}{2}-1,\,m=0,1,2,\ldots,9}$} & 
     \end{array}$\\
     where $\xM\eqd120$.
     That is, the distribution of the sequence oscillates every $\sfrac{\xM}{2}=60$ samples between one that favors $\dieC$ 
     and one that favors $\dieD$.
     Moreover, if we were to evaluate the sequence using a \ope{Discrete Fourier Transform} operator $\dft$ \xref{def:dft}, 
     we might expect to see a strong component at $\frac{\xN}{\xM}=10$ 
     (or 10 Hz---the distribution goes through 10 cycles during the course of the sequence).
  
  \item \label{item:nonstat34_R1pam}
        Suppose we first use the \fncte{PAM die random variable} \xref{def:rv_diepam} to map
        the sequence of \pref{item:nonstat34_psp} into $\R^1$.
        The magnitude of the $\dft:\R^1\to\C^1$ of the mapped sequence is as follows:
     \\\begin{tabular}{|>{\scs}c|}
          \hline
          \includegraphics{ssp/plots/diedftR1_1525_1200m120.pdf}%
        \\\hline
     \end{tabular}\\
     Looking at the above result, it would be next to impossible to discern that the distribution had a significantly strong
     oscillation of 10 cycles.
     In fact, the magnitude of the DFT at 10Hz is only $0.895699$, or $10\log_{10}(0.895699)=-0.478377$ dB.
     There are exactly 456 out of a total $\sfrac{\xN}{2}=600$ values that are greater than 
     the DFT magnitude at 10Hz.\footnote{\seessp{diedft_1525_1200m120.xlg}}
     % 
     That is, to either a human observer or a machine algorithm, the 10Hz component is effectively lost in the noise.
     
  \item \label{item:nonstat34_C1qpsk}
    Suppose we next use the \fncte{QPSK die random variable} \xref{def:rv_dieqpsk} to map
    the sequence into the complex plane.
    The magnitude of the $\dft:\C^1\to\C^1$ operation on the mapped sequence is as follows:
    \\\begin{tabular}{|>{\scs}c|}
         \hline
         \includegraphics{ssp/plots/diedftC1_1525_1200m120.pdf}%
       \\\hline
    \end{tabular}\\
    %Again, the 10Hz component is effectively lost in the noise.
    The magnitude of the DFT at 10Hz is $0.589990$, or $10\log_{10}(0.589990)=-2.291552$ dB.
    There are exactly 831 out of a total $\xN=1200$ values that are greater than 
    the DFT magnitude at 10Hz.\footnote{\seessp{diedft_1525_1200m120.xlg}}
    Again, the 10Hz component is effectively lost in the noise.
     
  \item \label{item:nonstat34_R6}
    Suppose we next use the \fncte{$\R^6$ die random variable} \xref{def:rv_dieR6} to map
    the sequence into $\R^6$.
    The magnitude of $\dft:\R^6\to\C^6$ of the mapped sequence is as follows:
    \\\begin{tabular}{|>{\scs}c|}
         \hline
         \includegraphics{ssp/plots/diedftR6_1525_1200m120.pdf}%
       \\\hline
    \end{tabular}\\
    The magnitude at 10Hz is $1.556295$, or $10\log_{10}(1.556295)=1.920920$ dB.
    Besides the DC component (0Hz component), this is the uniquely greatest value of the 600 samples.
    And in fact, there are only 5 out of a total $\sfrac{\xN}{2}=600$ samples
    that are $0.90\times1.556295$ or greater.\footnote{\seessp{diedft_1525_1200m120.xlg} 
      The 5 largest values are the points
      $\opair{0}{14.251433}$, $\opair{10}{1.556295}$, $\opair{344}{1.513501}$, $\opair{456}{1.405843}$ and $\opair{557}{1.468970}$.
      } 
    %And, besides the DC component, the 10Hz component is the uniquely greatest value of the 600 samples.
    % dieC1dft_1200m120_20160312_044601.log
    Thus, using the $\R^6$ mapping technique of this example, 
    it is much simpler to detect the 10Hz oscillating distribution.
\end{enumerate}
\end{example}

%---------------------------------------
\begin{example}[\exmd{length 12000 non-stationary die sequence with 10Hz oscillating mean}]
\label{ex:nonstat34_12000m1200}
\addcontentsline{toc}{subsubsection}{* length 12000 non-stationary die sequence with 10Hz oscillating mean}
\mbox{}\\
%---------------------------------------
\begin{enumerate}
  \item \label{item:nonstat34_12000m1200_psp}
    Suppose we have a length $\xN\eqd12000$ die sequence $\seqn{x_n}$ with the following distribution:
    \\\indentx$\begin{array}{lDlD}
      \psp(\dieA)=\psp(\dieB)=\psp(\dieD)=\psp(\dieE)=\psp(\dieF)=0.16 &and& \psp(\dieC)=0.20            &     \\
      \mc{3}{M}{\qquad for $n\in\set{p+ (2m)\frac{\xM}{2}}{p=0,1,\ldots,\frac{\xM}{2}-1,\,m=0,1,2,\ldots,9}$} & and  \\
      \psp(\dieA)=\psp(\dieB)=\psp(\dieC)=\psp(\dieE)=\psp(\dieF)=0.16 &and& \psp(\dieD)=0.20 &                \\
      \mc{3}{M}{\qquad for $n\in\set{p+(2m+1)\frac{\xM}{2}}{p=0,1,\ldots,\frac{\xM}{2}-1,\,m=0,1,2,\ldots,9}$} & 
    \end{array}$\\
    where $\xM\eqd1200$.
    That is, the distribution of the sequence oscillates every $\sfrac{\xM}{2}=600$ samples between one that favors $\dieC$ 
    and one that favors $\dieD$.
    If we were to evaluate the sequence using the \ope{Discrete Fourier Transform} operator, 
    we again might expect to see a strong component at $\frac{\xN}{\xM}=10$ 
    (or 10 Hz---the distribution goes through 10 cycles during the course of the sequence).
  
  \item \label{item:nonstat34_12000m1200_R1pam}
    Suppose we first use the \fncte{PAM die random variable} \xref{def:rv_diepam} to map
    the sequence of \pref{item:nonstat34_12000m1200_psp} into $\R^1$.
    In the magnitude of $\dft:\R^1\to\C^1$ there are 1130 values out of a possible $\sfrac{\xN}{2}=6000$ 
    values greater than the value at 10Hz (that value being $2.174512$).\footnote{\seessp{diedft_1620_12000m1200.xlg}}
    As in \prefpp{ex:nonstat34}, the subtle 10Hz component is effectively lost in the noise.
     
  \item \label{item:nonstat34_12000m1200_C1qpsk}
    Suppose we next use the \fncte{QPSK die random variable} \xref{def:rv_dieqpsk} to map
    the sequence into the complex plane.
    There are exactly 1932 out of a total $\xN=12000$ values that are greater than the DFT value at 10Hz
    (that value being $1.348693$).\footnote{\seessp{diedft_1620_12000m1200.xlg}}
    As in \prefpp{ex:nonstat34}, the subtle 10Hz component is effectively lost in the noise.
     
  \item \label{item:nonstat34_12000m1200_R6}
    Suppose we next use the \fncte{$\R^6$ die random variable} \xref{def:rv_dieR6} to map
    the sequence into $\R^6$.
    The magnitude of $\dft:\R^6\to\C^6$ of the mapped sequence is as follows:
    \\\begin{tabular}{|>{\scs}c|}
         \hline
         \includegraphics{ssp/plots/diedftR6_1620_12000m1200.pdf}%
       \\\hline
    \end{tabular}\\
    Besides the DC component, the value at 100Hz (that value being ($1.965018$) 
    is the uniquely greatest value of the $\sfrac{\xN}{2}=6000$ samples;
    and it is $10\log_{10}(1.965018/1.660189)=0.699\cdots$dB larger than the next largest value.\footnote{
    \seessp{diedft_1620_12000m1200.xlg}
    The 10 largest values are
    $\opair{    0}{44.763194}$, $\opair{   10}{ 1.965018}$, $\opair{   90}{ 1.602474}$, $\opair{ 1223}{ 1.660189}$,
    $\opair{ 1313}{ 1.555349}$, $\opair{ 2385}{ 1.551028}$, $\opair{ 3039}{ 1.550918}$, $\opair{ 4154}{ 1.563756}$,
    $\opair{ 4187}{ 1.586362}$, and $\opair{ 5147}{ 1.623052}$.}
    Thus, even though the oscillating distribution is very subtle
    (even more subtle than that of \prefpp{ex:nonstat34}), the $\R^6$ mapping technique and subsequent analysis
    are able to detect it.
\end{enumerate}
\end{example}

%---------------------------------------
\begin{example}[\exmd{length 12000 non-stationary die sequence with 100Hz oscillating mean}]
\label{ex:nonstat34_12000m120}
\addcontentsline{toc}{subsubsection}{* length 12000 non-stationary die sequence with 100Hz oscillating mean}
\mbox{}\\
%---------------------------------------
\begin{enumerate}
  \item \label{item:nonstat34_12000m120_psp}
     Suppose we have a length $\xN\eqd12000$ die sequence $\seqn{x_n}$ with the following distribution:
     \\\indentx$\begin{array}{lDlD}
       \psp(\dieA)=\psp(\dieB)=\psp(\dieD)=\psp(\dieE)=\psp(\dieF)=0.16 &and& \psp(\dieC)=0.20            &     \\
       \mc{3}{M}{\qquad for $n\in\set{p+ (2m)\frac{\xM}{2}}{p=0,1,\ldots,\frac{\xM}{2}-1,\,m=0,1,2,\ldots,9}$} & and  \\
       \psp(\dieA)=\psp(\dieB)=\psp(\dieC)=\psp(\dieE)=\psp(\dieF)=0.16 &and& \psp(\dieD)=0.20 &                \\
       \mc{3}{M}{\qquad for $n\in\set{p+(2m+1)\frac{\xM}{2}}{p=0,1,\ldots,\frac{\xM}{2}-1,\,m=0,1,2,\ldots,9}$} & 
     \end{array}$\\
     where $\xM\eqd120$.
     That is, the distribution of the sequence oscillates every $\sfrac{\xM}{2}=60$ samples between one that favors $\dieC$ 
     and one that favors $\dieD$.
     If we were to evaluate the sequence using the \ope{Discrete Fourier Transform} operator, 
     we might expect to see a strong component at $\frac{\xN}{\xM}=100$
     (or 100 Hz---the distribution goes through 100 cycles during the course of the sequence).
  

  \item \label{item:nonstat34_12000m120_R1pam}
    Suppose we first use the \fncte{PAM die random variable} \xref{def:rv_diepam} to map
    the sequence of \pref{item:nonstat34_12000m120_psp} into $\R^1$.
    In the magnitude $\dft:\R^1\to\C^1$ there are 1320 values out of a possible $\sfrac{\xN}{2}=6000$ 
    values greater than the value at 100Hz
    that value being $2.081469$).\footnote{\seessp{diedft_1620_12000m120.xlg}}
    The subtle 100Hz component is effectively lost in the noise.
     
  \item \label{item:nonstat34_12000m120_C1qpsk}
    Suppose we next use the \fncte{QPSK die random variable} \xref{def:rv_dieqpsk} to map
    the sequence into the complex plane.
    There are exactly 1555 out of a total $\xN12000$ values that are greater than the DFT value at 100Hz
    that value being $1.425427$).\footnote{\seessp{diedft_1620_12000m120.xlg}}
    The subtle 100Hz component is effectively lost in the noise.
     
  \item \label{item:nonstat34_12000m120_R6}
    Suppose we next use the \fncte{$\R^6$ die random variable} \xref{def:rv_dieR6} to map
    the sequence into $\R^6$.
    The magnitude of $\dft:\R^6\to\C^6$ of the mapped sequence is as follows:
    \\\begin{tabular}{|>{\scs}c|}
         \hline
         \includegraphics{ssp/plots/diedftR6_1620_12000m120.pdf}%
       \\\hline
    \end{tabular}\\
     Besides the DC component, the value at 100Hz (that value being $2.256927$) 
     is the uniquely greatest value of the $\sfrac{\xN}{2}=6000$.
     and it is $10\log_{10}(2.256927/1.599335)=1.495\cdots$dB larger than the next largest value.\footnote{
     \seessp{diedft_1620_12000m120.xlg}
     The 10 largest values are
       $\opair{   0}{44.763060}$, 
       $\opair{  90}{ 1.577597}$, 
       $\opair{ 100}{ 2.256927}$, 
       $\opair{ 486}{ 1.599335}$, 
       $\opair{1223}{ 1.585154}$, 
       $\opair{1313}{ 1.547956}$,
       $\opair{3039}{ 1.553522}$, 
       $\opair{3162}{ 1.561863}$,
       $\opair{5147}{ 1.558487}$, and
       $\opair{5567}{ 1.533659}$.
     }
     Thus, even though the oscillating distribution is very subtle, the $\R^6$ mapping technique
     and subsequent analysis are able to detect it.
\end{enumerate}
\end{example}

%=======================================
%\subsubsection{Fourier analysis of DNA sequences}
%=======================================
\begin{figure}%
  \centering%
  \gsize%
  $\begin{array}{|r|rMl|c|c|c|c|}
   \hline
   \mc{1}{|M|}{cycle}   & \mc{3}{|M|}{domain} & \psp(\dnaA) & \psp(\dnaC)  & \psp(\dnaG) & \psp(\dnaT)  \\\hline\hline
     0 &     0 &--&   599 & 0.24  & \cellcolor{bgp}  0.28  & 0.24  & 0.24                  \\
       &   600 &--&  1199 & 0.24  &                  0.24  & 0.24  & \cellcolor{bgp}  0.28 \\\hline
     1 &  1200 &--&  1799 & 0.24  & \cellcolor{bgp}  0.28  & 0.24  & 0.24                  \\
       &  1800 &--&  2399 & 0.24  &                  0.24  & 0.24  & \cellcolor{bgp}  0.28 \\\hline
     2 &  2400 &--&  2999 & 0.24  & \cellcolor{bgp}  0.28  & 0.24  & 0.24                  \\
       &  3000 &--&  3599 & 0.24  &                  0.24  & 0.24  & \cellcolor{bgp}  0.28 \\\hline
     3 &  3600 &--&  4199 & 0.24  & \cellcolor{bgp}  0.28  & 0.24  & 0.24                  \\
       &  4200 &--&  4799 & 0.24  &                  0.24  & 0.24  & \cellcolor{bgp}  0.28 \\\hline
     4 &  4800 &--&  5399 & 0.24  & \cellcolor{bgp}  0.28  & 0.24  & 0.24                  \\
       &  5400 &--&  5999 & 0.24  &                  0.24  & 0.24  & \cellcolor{bgp}  0.28 \\\hline
  \end{array}$%
  \hfill%
  $\begin{array}{|r|rMl|c|c|c|c|}%
   \hline%
   \mc{1}{|M|}{cycle}   & \mc{3}{|M|}{domain} & \psp(\dnaA) & \psp(\dnaC)  & \psp(\dnaG) & \psp(\dnaT)  \\\hline\hline
     5 &  6000 &--&  6599 & 0.24  & \cellcolor{bgp}  0.28  & 0.24  & 0.24                  \\
       &  6600 &--&  7199 & 0.24  &                  0.24  & 0.24  & \cellcolor{bgp}  0.28 \\\hline
     6 &  7200 &--&  7799 & 0.24  & \cellcolor{bgp}  0.28  & 0.24  & 0.24                  \\
       &  7800 &--&  8399 & 0.24  &                  0.24  & 0.24  & \cellcolor{bgp}  0.28 \\\hline
     7 &  8400 &--&  8999 & 0.24  & \cellcolor{bgp}  0.28  & 0.24  & 0.24                  \\
       &  9000 &--&  9599 & 0.24  &                  0.24  & 0.24  & \cellcolor{bgp}  0.28 \\\hline
     8 &  9600 &--& 10199 & 0.24  & \cellcolor{bgp}  0.28  & 0.24  & 0.24                  \\
       & 10200 &--& 10799 & 0.24  &                  0.24  & 0.24  & \cellcolor{bgp}  0.28 \\\hline
     9 & 10800 &--& 11399 & 0.24  & \cellcolor{bgp}  0.28  & 0.24  & 0.24                  \\
       & 11400 &--& 11999 & 0.24  &                  0.24  & 0.24  & \cellcolor{bgp}  0.28 \\\hline
  \end{array}$
  \caption{Distribution used in \prefpp{ex:nonstatdna}\label{fig:nonstatdna}}
\end{figure}
%---------------------------------------
\begin{example}[\exmd{length 12000 non-stationary artificial DNA sequence with 10Hz oscillating mean}]
\label{ex:nonstatdna}
\addcontentsline{toc}{subsubsection}{* length 12000 non-stationary artificial dna sequence with 10Hz oscillating mean}
\mbox{}\\
%---------------------------------------
\begin{enumerate}
  \item \label{item:nonstatdna_psp}
     Suppose we have a length $\xN\eqd12000$ die sequence $\seqn{x_n}$ with the following distribution (see also \prefp{fig:nonstatdna}):
     \\\indentx$\begin{array}{lDlD}
       \psp(\dnaA)=\psp(\dnaT)=\psp(\dnaG)=0.24 &and& \psp(\dnaC)=0.28            &     \\
       \mc{3}{M}{\qquad for $n\in\set{p+ 2m   \frac{\xM}{2}}{p=0,1,\ldots,\frac{\xM}{2}-1,\,m=0,1,2,\ldots,9}$} & and  \\
       \psp(\dnaA)=\psp(\dnaC)=\psp(\dnaG)=0.24 &and& \psp(\dnaT)=0.28            &     \\
       \mc{3}{M}{\qquad for $n\in\set{p+(2m+1)\frac{\xM}{2}}{p=0,1,\ldots,\frac{\xM}{2}-1,\,m=0,1,2,\ldots,9}$} & 
     \end{array}$\\
     where $\xM\eqd1200$.
     That is, the distribution of the sequence oscillates every $\sfrac{\xM}{2}=600$ samples between one that favors $\dnaC$ 
     and one that favors $\dnaT$.
     Moreover, if we were to evaluate the sequence using a \ope{Discrete Fourier Transform} (DFT) operator, 
     we might expect to see a strong component at $\frac{\xN}{\xM}=10$ 
     (or 10 Hz---the distribution goes through 10 cycles during the course of the sequence).
  
  \item \label{item:nonstatdna_R1pam}
    Suppose we first use the \fncte{PAM DNA random variable} \xref{def:rv_dnapam} to map
    the DNA sequence into $\R^1$.
    The magnitude of $\dft:\R^1\to\C^1$ of the sequence after applying this mapping is as follows:
    \\\begin{tabular}{|>{\scs}c|}
         \hline
         \includegraphics{ssp/plots/dnadftR1.pdf}%
       \\\hline
    \end{tabular}\\
    %Looking at the above result, it would be next to impossible to discern that the distribution had a significantly strong
    %oscillation of 10 cycles.
    The magnitude of the DFT at 10Hz is only $1.163575$ ($10\log_{10}(1.163575)=0.657944$ dB).
    There are exactly 2023 out of a total $\sfrac{\xN}{2}=6000$ values that are greater than the DFT value at 10Hz
    (that value being $1.163575$).\footnote{\seessp{dnadft_12000m1200.xlg}}
    Here again, the 10Hz component is effectively lost in the noise.
    
  \item \label{item:nonstatdna_C1qpsk}
    Suppose we next use the \fncte{QPSK DNA random variable} \xref{def:rv_dieqpsk} to map
    the DNA sequence into the complex plane.
    The magnitude of $\dft:\C^1\to\C^1$ of the sequence after applying this mapping is as follows:
    \\\begin{tabular}{|>{\scs}c|}
         \hline
         \includegraphics{ssp/plots/dnadftC1.pdf}%
       \\\hline
    \end{tabular}\\
    The DFT at 10Hz is $1.888671$, or $10\log_{10}(1.888671)=2.761563$ dB.
    There are exactly 343 out of a total $\xN=6000$ values that are greater than the DFT value at 10Hz.
    (that value being $1.888671$).\footnote{\seessp{dnadft_12000m1200.xlg}}
    Using this mapping it would be difficult to detect the subtle but significant 10Hz component.
     
  \item \label{item:nonstatdna_R4}
    Suppose we next use the \fncte{$\R^4$ DNA random variable} \xref{def:rv_dnaR4} to map
    the sequence into $\R^4$.
    The magnitude of $\dft:\R^4\to\C^4$ of the mapped sequence is as follows:
     \\\begin{tabular}{|>{\scs}c|}
          \hline
          \includegraphics{ssp/plots/dnadftR4.pdf}%
        \\\hline
     \end{tabular}\\
     The magnitude of the DFT at 10Hz is $1.932042$ ($10\log_{10}(1.932042)=2.860166$ dB).
     Besides itself and the DC component, there are only two out of a total $\sfrac{\xN}{2}=6000$ samples
     that are greater or equal to this value.\footnote{
      \seessp{dnadft_12000m1200.xlg}\\
       The 4 largest values are at
         $\opair{   0}{54.791926}$, 
         $\opair{  10}{ 1.932042}$, 
         $\opair{4187}{ 1.962836}$, and 
         $\opair{5147}{ 2.057553}$.
       } 
     Thus, using the $\R^4$ mapping technique and subsequent analysis of this example, 
     it is much simpler to detect the 10Hz oscillation.
\end{enumerate}
\end{example}

%---------------------------------------
\begin{example}[\exmd{Fourier analysis of Ebola DNA sequence}]
\label{ex:dftebola}
\addcontentsline{toc}{subsubsection}{* Fourier analysis of Ebola DNA sequence}
\mbox{}\\
%---------------------------------------
\begin{enumerate}
  \item \label{item:dftebola_psp}
     Consider the Ebola DNA sequence described in \prefpp{ex:dna_ebola}.
     DNA sequences commonly exhibit a strong DFT harmonic component at $\sfrac{2\pi}{3}$ 
     radians.\footnote{\citePp{galleani2010}{771}}
  
  \item \label{item:dftebola_R1pam}
    Suppose we first use the \fncte{PAM DNA random variable} \xref{def:rv_dnapam} to map
    the DNA sequence into $\R^1$.
    The magnitude of $\dft:\R^1\to\C^1$ of the sequence after applying this mapping is as follows:
    \\\begin{tabular}{|>{\scs}c|}
         \hline
         \includegraphics{ssp/plots/dna_AF086833_ebola_dftR1.pdf}%
       \\\hline
    \end{tabular}\\
    The component at $\sfrac{2\pi}{3}$ is easy to pick out with a signal to noise ratio (SNR) of\\ 
    $10\log_{10}(4.290296/1.123163)\approx5.8$ dB.\footnote{\seessp{dna_AF086833_ebola_dft.xlg}}
    Here, the noise value 1.123163 is the \fncte{RMS} (\fncte{root mean square}) of the DFT magnitude 
    sequence from $n=1$ to $n=\xN/2-1$ computed as follows:
    \\\indentx$\ds\sqrt{\frac{1}{\xN/2-1}\sum_{n=1}^{n=\xN/2-1} x_n^2}$.
     
  \item \label{item:dftebola_C1qpsk}
    Suppose we next use the \fncte{QPSK DNA random variable} \xref{def:rv_dieqpsk} to map
    the dna sequence into the complex plane.
    The magnitude of $\dft:\C^1\to\C^1$ of the sequence after applying this mapping is as follows:
     \\\begin{tabular}{|>{\scs}c|}
          \hline
          \includegraphics{ssp/plots/dna_AF086833_ebola_dftC1.pdf}%
        \\\hline
     \end{tabular}\\
    The component at $\sfrac{2\pi}{3}$ is again easy to pick out with a signal to noise ratio (SNR) of\\
    $10\log_{10}(6.412578/0.998659)\approx8.1$ dB.\footnote{\seessp{dna_AF086833_ebola_dft.xlg}}
    Here, the noise value 0.998659 is the \fncte{RMS} of the DFT magnitude 
    sequence from $n=1$ to $n=\xN-1$.
    %C^1 rms = 0.998685, max=6.412578   8.0760413380372487836054657324457 dB

  \item \label{item:dftebola_R4}
    Suppose we next use the \fncte{$\R^4$ DNA random variable} \xref{def:rv_dnaR4} to map
    the sequence into $\R^4$.
    The magnitude of $\dft:\R^4\to\C^4$ of the mapped sequence is as follows:
    \\\begin{tabular}{|>{\scs}c|}
         \hline
         \includegraphics{ssp/plots/dna_AF086833_ebola_dftR4.pdf}%
       \\\hline
    \end{tabular}\\
    The component at $\sfrac{2\pi}{3}$ is again easy to pick out with a signal to noise ratio (SNR) of\\ 
    $10\log_{10}(3.944811/0.860665)\approx 6.6$ dB.\footnote{\seessp{dna_AF086833_ebola_dft.xlg}}
    Here, the noise value 0.860665 is the \fncte{RMS} of the DFT magnitude 
    sequence from $n=1$ to $n=\xN/2-1$.
    %R^4 rms = 0.860710, max=3.944811    6.6116935177820583786215400221832 dB

  \item In conclusion, for this application, there is only a small advantage to using 
        the $\R^4$ mapping \xrefr{item:dftebola_R4}
        versus the $\R^1$ mapping \xrefr{item:dftebola_R1pam}, 
        and even a demonstrable disadvantage when compared to the $\C^1$ mapping \xrefr{item:dftebola_C1qpsk}.
\end{enumerate}
\end{example}



  %============================================================================
% Daniel J. Greenhoe
% LaTeX file
%============================================================================
%=======================================
\subsection{Wavelet Analysis}
%=======================================
In this section, we use what is in \emph{essense} wavelet analysis,
but yet is not truly wavelet analysis in the strict sense:
\begin{enumerate}
  \item For starters, standard \fncte{wavelet}s and their associated \fncte{scaling function}s are not sequences \xref{def:sequence},
        but rather are functions with domain $\R$ (not $\Z$ or some convex subset of $\Z$).
  \item While it is true that the celebrated \ope{Fast Wavelet Transform} (FWT) does work \emph{internally} with sequences
        (using \ope{filter bank}s),\footnote{\citerppgc{greenhoe2013wsd}{360}{364}{0983801134}{J.6 Filter Banks}} 
        the FWT is actually defined to work on functions with domain $\R$;
        and so the function to be analyzed by the FWT must first be \ope{sampled} by 
        a \fncte{scaling function}, which yields a sequence that can be processed by the 
        \ope{filter bank}s.\footnote{\citerppgc{greenhoe2013wsd}{369}{372}{0983801134}{Appendix L}}
  \item Wavelet analysis is typically performed by translating the wavelet or scaling function 
        by fixed amounts depending on the ``scale" of the given wavelet.
        For example, a Haar wavelet of length 4000 would typically ``jump" in offsets of 4000:
        0, 4000, 8000, 12000, \ldots.\footnote{\citerppgc{greenhoe2013wsd}{27}{62}{0983801134}{Chapter 2. The Structure of Wavelets}}
        As one might imagine, this may be reason for concern if you are using this wavelet to 
        perform edge detection (you might jump over and miss detecting the edge).
        In this section, wavelet sequences are translated by offsets of 1, making an edge harder to miss.
\end{enumerate}
%=======================================
%\subsubsection{Wavelet analysis of die sequences}
%=======================================
%---------------------------------------
\begin{example}[\exmd{statistical edge detection using Haar wavelet on non-stationary die sequence}]
\label{ex:nonstat48}
\addcontentsline{toc}{subsubsection}{* statistical edge detection using Haar wavelet on non-stationary die sequence}
\mbox{}\\
%---------------------------------------
\begin{enumerate}
  \item \label{item:nonstat48_psp}
     Suppose we have a length $\xN\eqd12000$ die sequence $\seqn{x_n}$ with the following distribution:
     \\\indentx$\begin{array}{lD}
       \psp(\dieA)=\psp(\dieB)=\psp(\dieC)=\psp(\dieD)=\psp(\dieE)=\psp(\dieF)=\frac{1}{6}                             & for $n\in\intcc{0}{3999}\setu\intcc{8000}{11999}$\\
       \psp(\dieA)=\psp(\dieB)=\psp(\dieC)=\psp(\dieE)=\psp(\dieF)=\frac{1}{10} \;\text{and}\; \psp(\dieD)=\frac{1}{2} & for $n\in\intcc{4000}{7999}$
     \end{array}$\\
     That is, the distribution of the sequence is uniformly distributed in the first and last thirds, 
     but biased towards $\dieC$ in the middle third.
     In this example we use a simple statistical edge detector to try to find the statistical ``edges" at 4000 and 8000.
     The edge detector here is a 
     \ope{filter} operation $\opW$ \xref{def:filter} using a \fncte{length 200 Haar wavelet sequence} \xref{def:whaar}.\footnote{
     Empirical evidence due to \citeP{singh1997} suggests that the Haar wavelet performs better
     than several other common wavelets as an edge detector.}
  
  \item \label{item:nonstat48_R1pam}
    Suppose we first use the \fncte{PAM die random variable} \xref{def:rv_diepam} to map
    the sequence of \pref{item:nonstat48_psp} into $\R^1$.
    The magnitude of $\opW:\R^1\to\C^1$ of the mapped sequence is as follows:\footnote{
    Note that the plot in \pref{item:nonstat48_R1pam} 
    has been down sampled by a factor of 10 for practical reasons of displaying the very large data set.}
    \\\begin{tabular}{|>{\scs}c|}
         \hline
         \includegraphics{ssp/plots/diehaarR1_12000m4000_h200_1050_D10.pdf}%
       \\\hline
    \end{tabular}\\
     We might expect to see strongest evidence of the edges at $4000+200/2=4100$ and $8100$.
     But looking at the above result, this is not apparent. 
     In fact, there are a total of 10646 values that are greater than or equal to the value at location 4100 
     (that value being $0.015$).\footnote{\seessp{diehaar_12000m4000_h200_1050.xlg}} 

  \item \label{item:nonstat48_C1qpsk}
    Suppose we next use the \fncte{QPSK die random variable} \xref{def:rv_dieqpsk} to map
    the die sequence into the complex plane.
    The magnitude of $\opW:\C^1\to\C^1$ of the mapped sequence is as follows:
    \\\begin{tabular}{|>{\scs}c|}
         \hline
         \includegraphics{ssp/plots/diehaarC1_12000m4000_h200_1050_D10.pdf}%
       \\\hline
    \end{tabular}\\
     Using this method, the edges are apparent.   
     And the value of the peak at $n=4083$ (with value $0.223383$) is about
     %$10\log_{10}(0.223383/0.072097)\approx4.9$ dB above the noise floor.\footnote{\seessp{diehaar_12000m4000_h200_1050.xlg}} 
     $10\log_{10}(0.223383/0.072741)\approx4.9$ dB above the noise 
     floor.\footnote{\seessp{diehaar_12000m4000_h200_1050.xlg}} 
    Here, the noise value 0.072741 is the \fncte{RMS} (see \pref{item:dftebola_R1pam} of \prefp{ex:dftebola})
    of the DFT magnitude sequence computed over the domain $n=200\ldots\xN-1$.
     %4.9113292531261044618521061907862 dB
     %above the next highest peak, which occurs at $n=521$ (with value $0.183576$).
%max|W(C^1 sequence)| from n=4050..4150 is 0.223383:
%rms|W(C^1 sequence)| from n=200..11999 is 0.072744:
%4.8725294107136569466927663732011 dB

     
  \item \label{item:nonstat48_R4}
    Suppose we next use the \fncte{$\R^6$ die random variable} \xref{def:rv_dieR6} to map
    the sequence into $\R^6$.
    The magnitude of $\opW:\R^6\to\R^6$ of the mapped sequence is as follows:
     \\\begin{tabular}{|>{\scs}c|}
          \hline
          \includegraphics{ssp/plots/diehaarR6_12000m4000_h200_1050_D10.pdf}%
        \\\hline
     \end{tabular}\\
     Using this method, the edges are also apparent.
     And the value of the peak at $n=4102$ (with value $0.209165$) is about 
     %$10\log_{10}(0.209165/0.068596)\approx4.8$ dB above the noise floor.
     $10\log_{10}(0.209165/0.065768)\approx5.0$ dB above the noise 
     floor.\footnote{Here the RMS noise value is computed over the domain $n=200\ldots\xN-1$.\\ 
       \seessp{diehaar_12000m4000_h200_1050.xlg}
       } 
     This is only a slight improvement over \pref{item:nonstat48_C1qpsk}.

%max|W(R^6 sequence)| from n=4050..4150 is 0.209165:
%rms|W(R^6 sequence)| from n=0..11999 is 0.068596:
%4.8419022323079719296397098112867 dB
%max|W(R^6 sequence)| from n=4050..4150 is 0.209165:
%rms|W(R^6 sequence)| from n=200..11999 is 0.065771:
%5.0245456986608778348245061173019 dB
\end{enumerate}
\end{example}


%=======================================
%\subsubsection{Wavelet analysis of DNA sequences}
%=======================================
%---------------------------------------
\begin{example}[\exmd{statistical edge detection using Haar wavelet on non-stationary artificial DNA sequence}]
\addcontentsline{toc}{subsubsection}{* statistical edge detection using Haar wavelet on non-stationary artificial DNA sequence}
\mbox{}\\
\label{ex:}
%---------------------------------------
\begin{enumerate}
  \item 
     Suppose we have a length $\xN\eqd12000$ dna sequence $\seqn{x_n}$ with the following distribution:
     \\\indentx$\begin{array}{lD}
       \psp(\dnaA)=\psp(\dnaC)=\psp(\dnaG)=\psp(\dnaT)=\frac{1}{4}                             & for $n\in\intcc{0}{3999}\setu\intcc{8000}{11999}$\\
       \psp(\dnaA)=\psp(\dnaG)=\psp(\dnaT)=\frac{17}{100} \;\text{and}\; \psp(\dnaC)=\frac{49}{100} & for $n\in\intcc{4000}{7999}$
     \end{array}$\\
     That is, the distribution of the sequence is uniformly distributed in the first and last thirds, 
     but biased towards $\dnaC$ in the middle third.
     Just as in \pref{ex:nonstat48}, we again use 
     a filter operation $\opW$ with \fncte{length 200 Haar wavelet sequence} as 
     a simple statistical edge detector to try to locate the statistical ``edges" at 4000 and 8000.
  
  \item \label{item:dna_haar_R1pam}
    Suppose we first use the \fncte{PAM DNA random variable} \xref{def:rv_dnapam} to map
    the DNA sequence into $\R^1$.
    The magnitude of $\opW:\R^1\to\C^1$ of the mapped sequence is as follows:\footnote{
    Note that the plot in \pref{item:dna_haar_R1pam} 
    has been down sampled by a factor of 10 for practical reasons of displaying the very large data set.}
     \\\begin{tabular}{|>{\scs}c|}
          \hline
          \includegraphics{ssp/plots/dnahaarR1_12000m4000_h200_1749_D5.pdf}%
        \\\hline
     \end{tabular}\\
     We might expect to see strongest evidence of the edges at or near $4000+200/2=4100$ and $8100$.
     In fact, the sequence does have peaks at 4087 (with value 0.185) and at 8087 (with value 0.230).
     The peak at 4087 is about $10\log_{10}(0.185000/0.071355)\approx 4.1$ dB above the noise floor.
     %max|W(R^1 sequence)| from n=4050..4150 is 0.185000:
     %max|W(R^1 sequence)| from n=8050..8150 is 0.230000:
     %rms|W(R^1 sequence)| from n=200..11999 is 0.071355:
     However, there are 103 other values not around the $n=4087$ and $n=8087$ peaks that are 
  0.185 or greater.
  These 102 values represent roughly 11 other peaks, each of which could trigger a 
  ``false positive" decision.\footnote{\seessp{dnahaar_12000m4000_h200_1749.xlg}}
     
  \item \label{item:dna_haar_C1}
    Suppose we next use the \fncte{QPSK DNA random variable} \xref{def:rv_dieqpsk} to map
    the dna sequence into the complex plane.
    The magnitude of $\opW:\C^1\to\C^1$ of the mapped sequence is as follows:
     \\\begin{tabular}{|>{\scs}c|}
          \hline
          \includegraphics{ssp/plots/dnahaarC1_12000m4000_h200_1749_D5.pdf}%
        \\\hline
     \end{tabular}\\
     Using this method, the edges are apparent.
     And the value of the peak at $n=4086$ (with value $0.215870$) is 
     $10\log_{10}(0.215870/0.070068)\approx 4.9$ dB above the noise floor.\footnote{\seessp{dnahaar_12000m4000_h200_1749.xlg}}
     %max|W(C^1 sequence)| from n=4050..4150 is 0.215870:
     %rms|W(C^1 sequence)| from n=200..11999 is 0.070068:
     %only 
     %$10\log_{10}(0.215870/0.190263)=0.548380$dB
     %above the next highest peak, which occurs at $n=10405$ (with value $0.190263$).\footnote{\seessp{dnahaar_12000m4000_h200_1749.xlg}}
     %However it does produce two false edges at 9815 and 10405---the value at 
     %$n=4086$ is $0.215870$.
     %$n=4100$ is $0.181108$,
     %but it is greater than or equal to this value at 9815, 10405, 10407, and 10408.\footnote{
     %The values are $\opair{9815}{0.181108}$, $\opair{10405}{0.190263}$, $\opair{10407}{0.185607}$,
     %and $\opair{10408}{0.186682}$.}
     % dnahaar_12000m4000_h200_1749_20160402_095901.log


  \item \label{item:dna_haar_R4}
    Suppose we next use the \fncte{$\R^4$ DNA random variable} \xref{def:rv_dnaR4} to map
    the sequence into $\R^4$.
    The magnitude of $\opW:\R^4\to\C^4$ of the mapped sequence is as follows:
     \\\begin{tabular}{|>{\scs}c|}
          \hline
          \includegraphics{ssp/plots/dnahaarR4_12000m4000_h200_1749_D5.pdf}%
        \\\hline
     \end{tabular}\\
     Using this method, the edges are also apparent.
     And the value of the peak at $n=4096$ (with value $0.181246$) is 
     $10\log_{10}(0.181246/0.059594)\approx4.8$ dB
     above the noise floor.\footnote{\seessp{dnahaar_12000m4000_h200_1749.xlg}}
     Note that this is a slight decrease in performance as compared to \pref{item:dna_haar_C1}.

     %max|W(R^4 sequence)| from n=4050..4150 is 0.181246:
     %rms|W(R^4 sequence)| from n=200..11999 is 0.059594:
     %
     %$10\log_{10}(0.181246/0.141421)=1.077545$dB
     %above the next non-starting highest peak, which occurs at $n=9815$ (with value $0.141421$).
     %Note that this is a small improvement over \pref{item:dna_haar_C1}.
     %Unlike the complex mapping in \pref{item:dna_haar_C1},
     %it shows an edge near the start of the sequence.
     %But also unlike the method of \pref{item:dna_haar_C1}, it contains no false edges in that 
     %there are no other values greater than that around the 4100 peak, 8100 peak, and starting peak.
     %In fact, the 4100 peak is $10\log_{10}(0.181246/0.141421)=1.077545$dB above the next highest peak.\footnote{
     %Compare $\opair{4096}{0.181246}$, $\opair{4100}{0.168819}$, and $\opair{9815}{0.141421}$.
     %} 


\end{enumerate}
\end{example}

%---------------------------------------
\begin{example}[\exmd{Wavelet analysis of Phage Lambda DNA sequence}]
\label{ex:dnapl}
\addcontentsline{toc}{subsubsection}{* Wavelet analysis of Phage Lambda DNA sequence}
\mbox{}\\
%---------------------------------------
\begin{enumerate}
  \item \label{item:dnapl_psp}
     Consider the Phage Lambda DNA sequence.
     It has a strong $\dnaC\dnaG$ bias before $n=20000$ and 
            a strong $\dnaA\dnaT$ bias after,\footnote{\citerpg{cristianini2007}{14}{1139460153}}
     as demonstrated next by mapping 
     \\\indentx$\dnaA\to+1$\qquad$\dnaT\to+1$\qquad$\dnaC\to-1$\qquad$\dnaG\to-1$\\
     and filtering the resulting sequence in $\R^1$
     with a \fncte{length 1600 Haar scaling sequence} \xref{def:shaar}---such
     a filtering operation acts as a kind of ``sliding window" histogram of the DNA sequence.
     \\\begin{tabular}{|>{\scs}c|}
          \hline
          \includegraphics{ssp/plots/dna_NC001416_phagelambdaR1b_hs1600.pdf}%
        \\\hline
     \end{tabular}\\
  
  \item \label{item:dnapl_R1pam}
    Suppose we first use the \fncte{PAM DNA random variable} \xref{def:rv_dnapam} to map
    the DNA sequence into $\R^1$.
    The magnitude of the length 1600 Haar wavelet operation on the mapped sequence is as follows:
     \\\begin{tabular}{|>{\scs}c|}
          \hline
          \includegraphics{ssp/plots/dna_NC001416_phagelambdaR1_h1600.pdf}%
        \\\hline
     \end{tabular}\\
    Note that it is very difficult to pick out the edge at 20000.

  \item \label{item:dnapl_C1qpsk}
    Suppose we next use the \fncte{QPSK DNA random variable} \xref{def:rv_dnaqpsk} to map
    the DNA sequence into the complex plane.
    The length 1600 Haar wavelet operation on the mapped sequence is as follows:
     \\\begin{tabular}{|>{\scs}c|}
          \hline
          \includegraphics{ssp/plots/dna_NC001416_phagelambdaC1_h1600.pdf}%
        \\\hline
     \end{tabular}\\
     If one did not know apriori that there was an edge at 20000, it would still be difficult to identify.
     
  \item \label{item:dnapl_R6}
    Suppose we next use the \fncte{$\R^4$ DNA random variable} \xref{def:rv_dnaR4} to map
    the DNA sequence into $\R^4$.
    Filtering the mapped sequence with a length 1600 Haar wavelet sequence results in the following:
     \\\begin{tabular}{|>{\scs}c|}
          \hline
          \includegraphics{ssp/plots/dna_NC001416_phagelambdaR4_h1600.pdf}%
        \\\hline
     \end{tabular}\\
     Here there is a clear peak near $20000$.

  \item And here is the same analysis as used in \pref{item:dnapl_R6}, but at scale 4000
        (using a length 4000 Haar wavelet filter):
     \\\begin{tabular}{|>{\scs}c|}
          \hline
          \includegraphics{ssp/plots/dna_NC001416_phagelambdaR4_h4000.pdf}%
        \\\hline
     \end{tabular}\\
     Again, the peak near 20000 is quite pronounced.
     However, at the low resolution scale (of 4000), it would be difficult to determine precisely where the
     statistical edge actually was.
\end{enumerate}
\end{example}





