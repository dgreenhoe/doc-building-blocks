%============================================================================
% Daniel J. Greenhoe
% LaTeX file
% Dirichlet function D_10(x), tau=1 (period 1)
% nominal unit = 10mm
%           1  sin[(pi/tau)(2n+1)x]
% Dn(x) = --- ---------------------
%         tau      sin[(pi/tau)x]
%============================================================================
\begin{pspicture}(-3.5,-1.2)(3.5,3.5)%
  %-------------------------------------
  % scaling
  %-------------------------------------
  \psset{yunit=0.142857\psyunit}% scale by 3/21
  %-------------------------------------
  % axes
  %-------------------------------------
  \psaxes[linewidth=0.75pt,linecolor=axis,Dy=5]{<->}(0,0)(-3.5,-8)(3.5,23)%
  %-------------------------------------
  % annotation
  %-------------------------------------
  \psline[linestyle=dashed,linewidth=0.75pt,linecolor=red](-3.5,21)(3.5,21)%
  \uput[30](0,21){$\frac{1}{\tau}(2n+1)=21$}%
  %\psline[linestyle=dashed,linewidth=0.75pt,linecolor=red](-0.333333,0)(-0.333333,-1.25)%
  %\psline[linestyle=dashed,linewidth=0.75pt,linecolor=red]( 0.333333,0)( 0.333333,-1.25)%
  %\uput[-45](0.3333,-1){$+\frac{\tau}{2\xN+1}=\frac{1}{3}$}%
  %\uput[225](-0.3333,-1){$-\frac{\tau}{2\xN+1}=-\frac{1}{3}$}%
  %-------------------------------------
  % D1(x)
  %-------------------------------------
  %                                                                   n=10
  \psplot[plotpoints=1024,linewidth=0.75pt]{-3.25}{3.25}{x 180 mul 2 10 mul 1 add mul sin x 180 mul sin div}%
  \psplot[plotpoints=1024,linewidth=0.75pt,linestyle=dotted]{-3.25}{-3.5}{x 180 mul 2 10 mul 1 add mul sin x 180 mul sin div}%
  \psplot[plotpoints=1024,linewidth=0.75pt,linestyle=dotted]{3.25}{3.5}{x 180 mul 2 10 mul 1 add mul sin x 180 mul sin div}%
\end{pspicture}%
