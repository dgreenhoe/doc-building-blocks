%============================================================================
% LaTeX File
% Daniel J. Greenhoe
%============================================================================

%======================================
\chapter{Operations on Sequences}
\label{app:dsp}
\label{app:ztrans}
%======================================





%A digital filter is an operator on a sequence 
%For a wide class of digital filters, this operator is linear.
%This operation can often be more clearly 
%understood by the use of a special transform called the {\em Z-transform} (\prefp{def:opZ}).
%The Z-transform represents linear filters by ratios of polynomials
%(a polynomial divided by a polynomial) in a free variable $z$.
%The roots of the numerator polynomial are called \structe{zeros}.
%The roots of the denominator polynomial are called \structe{poles}.
%The location in the $z$-plane of these poles and zeros 
%determine the behavior of the filter operation.




%======================================
%\section {Operations}
%======================================
%======================================
\section {Z-transform}
%======================================
\ifdochasnot{seq}{
%--------------------------------------
\begin{definition}
\label{def:sequence}
\label{def:seq}
\footnote{
  \citerp{bromwich1908}{1},
  \citerpgc{thomson2008}{23}{143484367X}{Definition 2.1},
  \citerpg{joshi1997}{31}{8122408265}
  }
\label{def:tuple}
%--------------------------------------
Let $\clFyx$ be the set of all functions from a set $\setY$ to a set $\setX$.
Let $\Z$ be the set of integers.
\defbox{\begin{array}{l}
  \text{A function $\ff$ in $\clFyx$ is a \hid{sequence} over $\setX$ if $\setY=\Z$.}\\
  \text{A sequence may be denoted in the form $\ds\seqxZ{x_n}$ or simply as $\ds\seqn{x_n}$.}\\
  %\text{A function $\ff$ in $\clFyx$ is an \hid{n-tuple} over $\setX$ if $\setY=\setn{1,2,\ldots,\xN}$.}\\
  %\text{An n-tuple may be denoted in the form $\ds\tuplexn{x_n}$ or simply as $\ds\tuplen{x_n}$.}
\end{array}}
\end{definition}
  }

%--------------------------------------
\begin{definition}
\footnote{
  \citerpgc{kubrusly2011}{347}{0817649972}{Example 5.K}
  }
\label{def:spllR}
\label{def:spllF}
%--------------------------------------
Let $\fieldF$ be a \structe{field} \xref{def:field}.
\defboxt{
  The \structd{space of all absolutely square summable sequences} $\spllF$ over $\F$ is defined as
  \\\indentx$\ds\spllF\eqd\set{\seqxZ{x_n}}{\sum_{n\in\Z}\abs{x_n}^2 < \infty}$
  }
\end{definition}

The space $\spllR$ is an example of a \structe{separable Hilbert space}. 
In fact, $\spllR$ is the \emph{only} separable Hilbert space in the sense that all separable Hilbert spaces
are isomorphically equivalent. 
For example, $\spllR$ is isomorphic to $\spLLR$, the \structe{space of all absolutely square Lebesgue integrable functions}.
%That is, their topological structure is the same.
%Differences occur in the nature of operators on the spaces.

%--------------------------------------
\begin{definition}
\label{def:dsp_conv}
\label{def:convd}
\index{convolution}
%--------------------------------------
%Let $\seq{x_n}{n\in\Z}$ and $\seq{y_n}{n\in\Z}$ be sequences \xref{def:seq} in the space $\spllR$ \xref{def:spllR}. 
\defbox{\begin{array}{M}
  The \hid{convolution} operation $\hxs{\convd}$ is defined as 
  \\\indentx
  $\ds{\seqn{x_n}\convd\seqn{y_n}} \eqd \seq{\sum_{m\in\Z} x_{m} y_{n-m}}{n\in\Z}\qquad\scy\forall\seqxZ{x_n},\seqxZ{y_n}\in\spllR$
\end{array}}
\end{definition}

%--------------------------------------
\begin{definition}
\label{def:opZ}
\footnote{
  \structe{Laurent series}: \citerpg{aa}{49}{0821821466}
  }
%--------------------------------------
%Let $\seq{x_n}{n\in\Z}$ be a sequence in the space $\spllR$. %over a ring $\ring$. 
\defbox{\begin{array}{M}
  The \hid{z-transform} $\opZ$ of $\seq{x_n}{n\in\Z}$ is defined as
  \\\indentx
  $\ds\brs{\hxs{\opZ}\seqn{x_n}}(z) \eqd \mcom{\sum_{n\in\Z} x_n z^{-n}}{Laurent series}\qquad\scy\forall\seqn{x_n}\in\spllR$
\end{array}}
\end{definition}


%%--------------------------------------
%\begin{theorem}
%%--------------------------------------
%\thmbox{
%  \opZ\opT^n\seqn{x_n} = \
%  }
%\end{theorem}
%

%--------------------------------------
\begin{proposition}
%--------------------------------------
Let $\convd$ be the \structe{convolution operator} \xref{def:dsp_conv}.
\propbox{
  \seqn{x_n}\convd\seqn{y_n} = \seqn{y_n}\convd\seqn{x_n}
  \qquad{\scy\forall\seqxZ{x_n},\seqxZ{y_n}\in\spllR}
  \qquad\text{($\convd$ is \prope{commutative})}
  }
\end{proposition}
\begin{proof}
\begin{align*}
  [x\convd y](n)
    &\eqd \sum_{m\in\Z} x_m y_{n-m}
    &&    \text{by \prefp{def:dsp_conv}}
  \\&=    \sum_{k\in\Z} x_{n-k} y(k)
    &&    \text{where $k=n-m \iff m=n-k$}
  \\&=    \sum_{k\in\Z} x_{n-k} y(k)
    &&    \text{by change commutivity of addition}
  \\&=    \sum_{m\in\Z} x_{n-m} y_m
    &&    \text{by change of variables}
  \\&=    \sum_{m\in\Z} y_m x_{n-m} 
    &&    \text{by \prop{commutative} property of the field over $\C$}
  \\&\eqd \brp{y\convd x}_n
    &&    \text{by \prefp{def:dsp_conv}}
\end{align*}
\end{proof}

%--------------------------------------
\begin{theorem}
%--------------------------------------
Let $\opZ\fx[n]$ be the \ope{z-transform} of $\fx[n]$.
\thmbox{\opZ\seqn{a\fx[n]} = a X(z)}
\end{theorem}
\begin{proof}
\begin{align*}
  a X(z)
    &\eqd a \opZ \seqn{\fx[n]}                && \text{by definition of $X(z)$}
  \\&\eqd a \sum_{n\in\Z} \fx[n] z^{-n}       && \text{by definition of $\opZ$ operator}
  \\&\eqd \sum_{n\in\Z} \brp{a \fx[n]} z^{-n} && \text{by \prope{distributive} property}
  \\&\eqd \opZ\seqn{a\fx[n]}                  && \text{by definition of $\opZ$ operator}
\end{align*}
\end{proof}

%--------------------------------------
\begin{theorem}
%--------------------------------------
\thmbox{
  \opZ\seqn{\fx[n-k]} = z^{-k}X(z)
  }
\end{theorem}
\begin{proof}
\begin{align*}
  z^{-k}X(z) 
    &= z^{-k} \opZ\seqn{\fx[n]}
   &&\eqd z^{-k}\sum_{n=-\infty}^{n=+\infty} \fx[n] z^{-n}
  \\&=          \sum_{n=-\infty}^{n=+\infty} \fx[n] z^{-n-k}
  \\&=          \sum_{m-k=-\infty}^{m-k=+\infty} \fx[m-k] z^{-m}
    && \text{where $m\eqd n+k$ $\implies$ $n=m-k$}
  \\&=          \sum_{m=-\infty}^{m=+\infty} \fx[m-k] z^{-m}
  \\&=          \sum_{n=-\infty}^{n=+\infty} \fx[n-k] z^{-n}
    && \text{by change of free variable $m\rightarrow n$}
  \\&\eqd \opZ\seqn{\fx[n-k]}
    && \text{by definition of $\opZ$ operator}
\end{align*}
\end{proof}

%--------------------------------------
\begin{theorem}
%--------------------------------------
Let $\convd$ be the convolution operator \xref{def:dsp_conv}.
%$\seq{x_n}{n\in\Z}$ and $\seq{y_n}{n\in\Z}$ be sequences in the space $\spllR$. %be sequences over a ring $\ring$.
\thmbox{
  \opZ\mcom{\brp{\seqn{x_n}\convd\seqn{y_n}}}{sequence convolution} = \mcom{\brp{\opZ\seqn{x_n}}\;\brp{\opZ\seqn{y_n}}}{series multiplication}
  \qquad{\scy\forall\seqxZ{x_n},\seqxZ{y_n}\in\spllR}
  }
\end{theorem}
\begin{proof}
\begin{align*}
  [\opZ(x\convd y)](z)
    &\eqd \opZ {\left(\sum_{m\in\Z} x_m y_{n-m}\right)}
    &&    \text{by \prefp{def:dsp_conv}}
  \\&\eqd \sum_{n\in\Z} \sum_{m\in\Z} x_m y_{n-m} z^{-n}
    &&    \text{by \prefp{def:opZ}}
  \\&=    \sum_{n\in\Z} \sum_{m\in\Z} x_m y_{n-m} z^{-n}
  \\&=    \sum_{m\in\Z} \sum_{n\in\Z} x_m y_{n-m} z^{-n}
  \\&=    \sum_{m\in\Z} \sum_{k\in\Z} x_m y_k z^{-(m+k)}
    &&    \text{where $k=n-m \iff n=m+k$}
  \\&=    {\left[\sum_{m\in\Z} x_m z^{-m}\right]} 
          {\left[\sum_{k\in\Z} y_k z^{-k}\right]}
  \\&\eqd \brp{\opZ\seqn{x_n}}\;\brp{\opZ\seqn{y_n}}
    &&    \text{by \prefp{def:opZ}}
\end{align*}
\end{proof}





%=======================================
\section{Zero Locations}
%=======================================

%=======================================
%\subsection{Zeros inside the unit circle}
%=======================================
The system property of \prope{minimum phase} is defined in \pref{def:ztr_minphase} (next) and 
illustrated in \prefpp{fig:pz_minphase}.
%--------------------------------------
\begin{definition}
\footnote{
  \citerpg{farina2000}{91}{0471384569},
  \citerpg{dumitrescu2007}{36}{1402051247}
  }
\label{def:ztr_minphase}
\index{minimum phase}
%--------------------------------------
Let $\Zx(z)\eqd\opZ\seqn{x_n}$ be the \fncte{z transform} \xref{def:opZ} of a sequence $\seqxZ{x_n}$ in $\spllR$.
Let $\seqxZ{z_n}$ be the \structe{zeros} of $\Zx(z)$.
\defbox{\begin{array}{M}
  The sequence $\seqn{x_n}$ is \hid{minimum phase} if
  \\\indentx$\ds\mcom{\abs{z_n}<1\qquad \forall n\in\Z}{$\Zx(z)$ has all its \structe{zeros} inside the unit circle}$
\end{array}}
\end{definition}


\begin{figure}[ht]
  \centering%
  \includegraphics{graphics/pz_minphase.pdf}%
\caption{
   Minimum Phase filter
   \label{fig:pz_minphase}
   }
\end{figure}


The impulse response of a minimum phase filter has most of its energy concentrated
near the beginning of its support, as demonstrated next.
%--------------------------------------
\begin{theorem}[\thmd{Robinson's Energy Delay Theorem}]
\footnote{
  \citerpg{dumitrescu2007}{36}{1402051247},
  \citor{robinson1962},  % referenced by claerbout1976
  \citorc{robinson1966}{???},  % referenced by online thesis
  \citerpp{claerbout1976}{52}{53}
  %\citerp{os}{291}\\
  %\citerp{mallat}{253}
  }
\label{thm:ztr_redp}
\index{minimum phase!energy}
\index{energy}
%--------------------------------------
Let $\fp(z)\eqd\sum_{n=0}^\xN a_n z^{-n}$ 
and $\fq(z)\eqd\sum_{n=0}^\xN b_n z^{-n}$ 
be polynomials.
\thmbox{
  \brb{\begin{array}{lMD}
    \fp & is \prope{minimum phase} & and\\
    \fq & is \emph{not} minimum phase & 
  \end{array}}
  \implies
  \mcom{\sum_{n=0}^{m-1} \abs{a_n}^2}{\parbox{20mm}{``energy" of the first $m$ coefficients of $\fp(z)$}} \ge 
  \mcom{\sum_{n=0}^{m-1} \abs{b_n}^2}{\parbox{20mm}{``energy" of the first $m$ coefficients of $\fq(z)$}} 
  \qquad \forall 0\le m\le\xN
  }
\end{theorem}


%--------------------------------------
\begin{example}
%--------------------------------------
An example of a minimum phase polynomial is the Daubechies-4 scaling function.
%This function is generated by a minimum phase filter.
The minimum phase polynomial causes most of the energy to be concentrated near the origin, making it very \hie{asymmetric}.
In contrast, the Symlet-4 has a design very similar to that of Daubechies-4, 
but the selected zeros are not all within the unit circle in the complex $z$ plane.
This results in a scaling function that is more symmetric and less contrated near the origin.
Both scaling functions are illustrated in \prefpp{fig:pz_d4}.
%The Daubechies-2 scaling function can be specified by the rational expression 
%\begin{eqnarray*}
%   H(z) &=& \frac{\sqrt{2}}{8}
%            \left[
%               1-\sqrt{3}) + (3-\sqrt{3})z^{-1} + (3+\sqrt{3})z^{-2} + (1+\sqrt{3})z^{-3} 
%            \right]
%   \\   &=& \frac{\sqrt{2}}{8}
%            \left[
%            \frac{(1-\sqrt{3})z^3 + (3-\sqrt{3})z^{2} + (3+\sqrt{3})z + (1+\sqrt{3})  }
%                 {z^3}
%            \right]
%   \\   &=& \frac{\sqrt{2}(1-\sqrt{3})}{8}
%            \left[
%            \frac{z^3 -\sqrt{3}z^{2} + (-3-2\sqrt{3})z + (-2-\sqrt{3})  }
%                 {z^3}
%            \right]
%   \\   &=& \frac{\sqrt{2}(1-\sqrt{3})}{8}
%            \left[
%            \frac{(z+1)^2 [z-(2-\sqrt{3}]  }
%                 {z^3}
%            \right]
%\end{eqnarray*}
\end{example}
\begin{figure}[h]
\begin{center}
\footnotesize
\setlength{\unitlength}{0.08mm}
\begin{tabular}{|c|c|}
  \hline
  Daubechies-4 & Symlets-4
  \\\hline
  \includegraphics{graphics/D4_pz.pdf}&\includegraphics{graphics/S4_pz.pdf} \\
  \includegraphics{graphics/d4_phi_h.pdf}&\includegraphics{graphics/s4_phi_h.pdf}\\ 
  \hline
  \prope{minimum phase} & \emph{not} minimum phase
  \\\hline
\end{tabular}
\caption{
   Daubechies-4 and Symlet-4 scaling functions pole-zero plots
   \label{fig:pz_d4}
   }
\end{center}
\end{figure}




%=======================================
%\section{Digital Filters}
%\index{digital filter}
%=======================================

%=======================================
\section{Rational polynomial operators}
%=======================================
A digital filter is simply an operator on $\spllR$.
If the digital filter is a causal LTI system, then it can be expressed as 
a rational polynomial in $z$ as shown next.

%=======================================
\begin{lemma}
%=======================================
A causal LTI operator $\mathcal{H}$ can be expressed as a rational expression $H(z)$.
\begin{eqnarray*}
   H(z) &=& \frac{b_0 + b_1z^{-1} + b_2z^{-2} + \cdots + b_Nz^{-N}}
                 {1   + a_1z^{-1} + a_2z^{-2} + \cdots + a_Nz^{-N}}
   \\   &=& \frac{\sum\limits_{n=0}^{N} b_n z^{-n}}
                 {1   + \sum\limits_{n=1}^{N} a_n z^{-n}}
\end{eqnarray*}
\end{lemma}


%=======================================
%\subsection{Poles and zeros}
%=======================================
A filter operation $H(z)$ can be expressed as a product of its roots (poles and zeros).
\begin{eqnarray*}
   H(z) &=& \frac{b_0 + b_1z^{-1} + b_2z^{-2} + \cdots + b_\xN z^{-\xN}}
                 {1   + a_1z^{-1} + a_2z^{-2} + \cdots + a_\xN z^{-\xN}}
   \\   &=& \alpha\frac{(z-z_1)(z-z_2)\cdots(z-z_\xN)}
                 {(z-p_1)(z-p_2)\cdots(z-p_\xN)}
\end{eqnarray*}
where $\alpha$ is a constant, $z_i$ are the zeros, and $p_i$ are the poles.
The poles and zeros of such a rational expression are often plotted in the z-plane with a unit circle
about the origin (representing $z=e^{i\omega}$).
Poles are marked with $\times$ and zeros with $\bigcirc$.
An example is shown in \prefp{fig:pz}.  
Notice that in this figure the zeros and poles are either real or occur in 
complex conjugate pairs.

\begin{figure}[ht]
  \centering
  \includegraphics{graphics/pz_realcoefs.pdf}
  \caption{
     Pole-zero plot for rational expression with real coefficients
     \label{fig:pz}
     }
\end{figure}


%--------------------------------------
\begin{theorem}
\index{stability}
%--------------------------------------
A causal LTI filter is {\bf stable} if all of its poles are {\bf inside} the unit circle.
\end{theorem}
Stable/unstable filters are illustrated in \prefpp{fig:pz_unstable}.

\begin{figure}[ht]
  \centering%
  \begin{tabular}{c@{\hspace{1cm}}c}
     \includegraphics{graphics/pz_stable.pdf}
    &\includegraphics{graphics/pz_unstable.pdf}
%
%\begin{picture}(300,300)(-150,-130)
%  \thicklines%
%  \color{axis}
%    \put(-130,   0){\line(1,0){260} }
%    \put(   0,-130){\line(0,1){260} }
%    \put( 140,   0){\makebox(0,0)[l]{$\Reb{z}$}}
%    \put(   0, 140){\makebox(0,0)[b]{$\Imb{z}$}}
%
%  \color{circle}
%    \qbezier( 100,   0)( 100, 41.421356)(+70.710678,+70.710678) % 0   -->1pi/4
%    \qbezier(   0, 100)( 41.421356, 100)(+70.710678,+70.710678) % pi/4-->2pi/4
%    \qbezier(   0, 100)(-41.421356, 100)(-70.710678,+70.710678) %2pi/4-->3pi/4
%    \qbezier(-100,   0)(-100, 41.421356)(-70.710678,+70.710678) %3pi/4--> pi 
%    \qbezier(-100,   0)(-100,-41.421356)(-70.710678,-70.710678) % pi  -->5pi/4
%    \qbezier(   0,-100)(-41.421356,-100)(-70.710678,-70.710678) %5pi/4-->6pi/4
%    \qbezier(   0,-100)( 41.421356,-100)( 70.710678,-70.710678) %6pi/4-->7pi/4
%    \qbezier( 100,   0)( 100,-41.421356)( 70.710678,-70.710678) %7pi/4-->2pi
%
%  \color{pole}
%    \put(  30,  60){\makebox(0,0){$\times$}}
%    \put(  30,- 60){\makebox(0,0){$\times$}}
%    \put(- 20,  20){\makebox(0,0){$\times$}}
%    \put(- 20,- 20){\makebox(0,0){$\times$}}
%
%  \color{zero}
%    \put( 120,  80){\circle{10}}
%    \put( 120,- 80){\circle{10}}
%    \put(- 80,  40){\circle{10}}
%    \put(- 80,- 40){\circle{10}}
%\end{picture}                                   
%&
%\begin{picture}(300,300)(-150,-130)
%  \thicklines%
%  \color{axis}
%    \put(-130,   0){\line(1,0){260} }
%    \put(   0,-130){\line(0,1){260} }
%    \put( 140,   0){\makebox(0,0)[l]{$\Reb{z}$}}
%    \put(   0, 140){\makebox(0,0)[b]{$\Imb{z}$}}
%
%  \color{circle}
%    \qbezier( 100,   0)( 100, 41.421356)(+70.710678,+70.710678) % 0   -->1pi/4
%    \qbezier(   0, 100)( 41.421356, 100)(+70.710678,+70.710678) % pi/4-->2pi/4
%    \qbezier(   0, 100)(-41.421356, 100)(-70.710678,+70.710678) %2pi/4-->3pi/4
%    \qbezier(-100,   0)(-100, 41.421356)(-70.710678,+70.710678) %3pi/4--> pi 
%    \qbezier(-100,   0)(-100,-41.421356)(-70.710678,-70.710678) % pi  -->5pi/4
%    \qbezier(   0,-100)(-41.421356,-100)(-70.710678,-70.710678) %5pi/4-->6pi/4
%    \qbezier(   0,-100)( 41.421356,-100)( 70.710678,-70.710678) %6pi/4-->7pi/4
%    \qbezier( 100,   0)( 100,-41.421356)( 70.710678,-70.710678) %7pi/4-->2pi
%
%  \color{pole}
%    \put(  67, 133){\makebox(0,0){$\times$}}
%    \put(  67,-133){\makebox(0,0){$\times$}}
%    \put(- 20,  20){\makebox(0,0){$\times$}}
%    \put(- 20,- 20){\makebox(0,0){$\times$}}
%
%  \color{zero}
%    \put( 120,  80){\circle{10}}
%    \put( 120,- 80){\circle{10}}
%    \put(- 80,  40){\circle{10}}
%    \put(- 80,- 40){\circle{10}}
%\end{picture}                                   
\\
stable & unstable
\end{tabular}
  \caption{
     Pole-zero plot stable/unstable causal LTI filters
     \label{fig:pz_unstable}
     }
\end{figure}


%=======================================
\section{Sample rate conversion}
%=======================================
%---------------------------------------
\begin{theorem}[\thmd{upsampling}]
%\footnote{
%  \citerpgc{strang1996}{101}{0961408871}{Second Nobel Identity}
%  }
\label{thm:upsample}
%---------------------------------------
Let $\seqxZ{x_n}$ and $\seqxZ{y_n}$ be \structe{sequences} \xref{def:seq} 
in $\spllF$ \xref{def:spllF} over a \structe{field} $\F$.
\thmbox{
  y_n = 
  \brb{\begin{array}{lM}
    x_{(n/\xN)} & for $n\mod \xN=0$  \\
    0         & otherwise
  \end{array}}
  \qquad\implies\qquad
  {\Zy(z) = \Zx\brp{z^\xN}}
  }
\end{theorem}
\begin{proof}
\begin{align*}
  \Zy(z)
    &\eqd \sum_{n\in\Z} y_n z^{-n}
    &&    \text{by definition of \fncte{z-transform}}
    &&    \text{\xref{def:opZ}}
  \\&=    \sum_{n\mod\xN=0}   y_n z^{-n} +
          \sum_{n\mod\xN\ne0} y_n z^{-n}
  \\&=    \sum_{n\mod\xN=0} x_{n/\xN} z^{-n} +
          \cancelto{0}{\sum_{n\mod \xN\ne 0} 0 z^{-n}}
    &&    \text{by definition of $\seqn{y_n}$}
  \\&=    \sum_{m\in\Z} x_m z^{-m\xN}
    &&    \text{where $m\eqd n/\xN\,\implies\,n=m\xN$}
  \\&=    \sum_{m\in\Z} x_m \brp{z^\xN}^{-m}
  \\&\eqd \Zx\brp{z^\xN}
    &&    \text{by definition of \fncte{z-transform}}
    &&    \text{\xref{def:opZ}}
\end{align*}
\end{proof}



%---------------------------------------
\begin{theorem}[\thmd{downsampling}]
\index{decimation}
\label{thm:downsample}
%---------------------------------------
Let $\seqxZ{x_n}$ and $\seqxZ{y_n}$ be \structe{sequences} \xref{def:seq} 
in $\spllF$ \xref{def:spllF} over a \structe{field} $\F$.
\thmbox{
  \brb{y_n = x_{(\xN n)}}
  \qquad\implies\qquad
  \brb{\Zy(z)=\frac{1}{\xN}\sum_{m=0}^{\xN-1}\Zx\left( e^{i\frac{2\pi m}{\xN}} z^{\frac{1}{\xN}} \right)}
  }
\end{theorem}
\begin{proof}
\begin{align*}
  \Zy(z)
    &\eqd \sum_{n\in\Z} y_n z^{-n}
    &&    \text{by definition of \fncte{z-transform}}
    &&    \text{\xref{def:opZ}}
  \\&=    \sum_{n\in\Z} x_{(n\xN)} z^{-n}
    &&    \text{by definition of $\seqn{y_n}$}
  \\&=    \sum_{n\in\Z} x_n \Big[ \kdelta_{(n\mod \xN)} \Big] z^{-\frac{n}{\xN}}
  \\&=    \sum_{n\in\Z} x_n \brs{\frac{1}{\xN}\sum_{m=0}^{\xN-1} e^{-i\frac{2\pi nm}{\xN}}} z^{-\frac{n}{\xN}}
    &&    \text{by \thme{Summation around unit circle}}
    &&    \text{\xref{cor:trig_circle}}
  \\&=    \frac{1}{\xN}\sum_{m=0}^{\xN-1} \sum_{n\in\Z} x_n 
                 \brp{e^{i\frac{2\pi m}{\xN}}}^{-n} 
                 \brp{ z^\frac{1}{\xN}}^{-n}
  \\&=    \frac{1}{\xN}\sum_{m=0}^{\xN-1} \sum_{n\in\Z} x_n 
                 \brp{e^{i\frac{2\pi m}{\xN}} z^\frac{1}{\xN}}^{-n} 
  \\&\eqd \frac{1}{\xN}\sum_{m=0}^{\xN-1}\Zx\brp{e^{i\frac{2\pi m}{\xN}} z^\frac{1}{\xN}}
    &&    \text{by definition of \fncte{z-transform}}
    &&    \text{\xref{def:opZ}}
\end{align*}
\end{proof}





%======================================
\section {Filter Banks}
%======================================
\structe{Conjugate quadrature filters} (next definition) are used in \structe{filter banks}.
If $\Zx(z)$ is a \structe{low-pass filter}, then the conjugate quadrature filter of $\Zy(z)$ is a \structe{high-pass filter}.

%--------------------------------------
\begin{definition}
\footnote{
  \citerpg{strang1996}{109}{0961408871},
  \citerppgc{haddad1992}{256}{259}{0323138365}{section 4.5},
  \citerpgc{vaidyanathan1993}{342}{0136057187}{(7.2.7), (7.2.8)},
  \citor{smith1984},
  \citor{smith1986},
  \citor{mintzer1985}
  }
\label{def:cqf}
%--------------------------------------
Let $\seqxZ{x_n}$ and $\seqxZ{y_n}$ be \structe{sequences} \xref{def:seq} in $\spllR$ \xref{def:spllR}.
\defboxt{
  The sequence $\seqn{y_n}$ is a \fnctd{conjugate quadrature filter} with shift $\xN$ with respect to $\seqn{x_n}$ if
  \\\indentx$\ds y_n = \pm(-1)^n x^\ast_{\xN-n}$\\
  A \structe{conjugate quadrature filter} is also called a \hid{CQF} or a \hid{Smith-Barnwell filter}.
  \\
  Any triple $\otriple{\seqn{x_n}}{\seqn{y_n}}{\xN}$ in this form is said to satisfy the
  \\\indentx\propd{conjugate quadrature filter condition} or
  the \propd{CQF condition}.
  }
\end{definition}

%--------------------------------------
\begin{theorem}[\thmd{CQF theorem}]
%\begin{theorem} % [\thmd{conjugate quadrature filters}/\thmd{CQF}/\thmd{Smith-Barnwell filters}]
\label{thm:cqf}
\footnote{
  %\citerpgc{dau}{135}{0898712742}{(5.1.34)},
  %\citerpgc{vidakovic}{59}{0471293652}{(3.34)},
  \citerpg{strang1996}{109}{0961408871},
  \citerppgc{mallat}{236}{238}{012466606X}{(7.58),(7.73)},
  \citerppgc{haddad1992}{256}{259}{0323138365}{section 4.5},
  \citerpgc{vaidyanathan1993}{342}{0136057187}{(7.2.7), (7.2.8)}
  }
%--------------------------------------
Let $\Dy(\omega)$ and $\Dx(\omega)$ be the \ope{DTFT}s \xref{def:dtft} of the sequences $\seqxZ{y_n}$ and $\seqxZ{x_n}$, respectively, in $\spllR$ \xref{def:spllR}.
\thmboxt{
  $\ds\begin{array}{>{\ds}r c rc>{\ds}l DD}
    \mcom{y_{n} = \pm (-1)^n x^\ast_{\xN-n}}
         {(1) \structe{CQF} in ``time"}
      &\iff& \Zy(z)      &=& \pm (-1)^\xN z^{-\xN} \Zx^\ast\brp{\frac{-1}{z^\ast}}   & (2)& \structe{CQF} in ``z-domain"
    \\&\iff& \Dy(\omega) &=& \pm (-1)^\xN e^{-i\omega\xN} \Dx^\ast(\omega+\pi)       & (3)& \structe{CQF} in ``frequency"
    \\&\iff& x_{n}       &=& \pm (-1)^\xN (-1)^n y^\ast_{\xN-n}                      & (4)& ``reversed" \structe{CQF} in ``time"
    \\&\iff& \Zx(z)      &=& \pm z^{-\xN} \Zy^\ast\brp{\frac{-1}{z^\ast}}            & (5)& ``reversed" \structe{CQF} in ``z-domain"
    \\&\iff& \Dx(\omega) &=& \pm e^{-i\omega\xN} \Dy^\ast(\omega+\pi)                & (6)& ``reversed" \structe{CQF} in ``frequency"
  \end{array}$
  \\$\indentx\scy\forall\xN\in\Z$
  }
\end{theorem}
\begin{proof}
\begin{enumerate}
  \item Proof that $(1)\implies(2)$:
    \begin{align*}
      \Zy(z)
        &= \sum_{n\in\Z}  y_{n}  z^{-n}
        && \text{by definition of \fncte{z-transform} \xref{def:opZ}}
      \\&= \sum_{n\in\Z} \mcom{(\pm) (-1)^n x^\ast_{\xN-n}}{\structe{CQF}} z^{-n}
        && \text{by (1)}
      \\&= \pm\sum_{m\in\Z} (-1)^{\xN-m} x_m^\ast z^{-(\xN-m)}
        && \text{where } m\eqd\xN-n \implies n=\xN-m
      \\&= \pm (-1)^\xN z^{-\xN}
           \sum_{m\in\Z}(-1)^{-m} x_m^\ast \brp{z^{-1}}^{-m}
      \\&= \pm (-1)^\xN z^{-\xN}
           \sum_{m\in\Z} x^\ast_m \brp{-\frac{1}{z}}^{-m}
      \\&= \pm (-1)^\xN z^{-\xN}
           \brs{\sum_{m\in\Z} x_m \brp{-\frac{1}{z^\ast}}^{-m}}^\ast
      \\&= \pm (-1)^\xN z^{-\xN}\Zx^\ast\brp{\frac{-1}{z^\ast}}
        && \text{by definition of \fncte{z-transform} \xref{def:opZ}}
    \end{align*}

  \item Proof that $(1)\impliedby(2)$:
    \begin{align*}
      \Zy(z)
        &= \pm (-1)^\xN z^{-\xN} \Zx^\ast\brp{\frac{-1}{z^\ast}}
        && \text{by (2)}
      \\&= \pm (-1)^\xN z^{-\xN} \brs{\sum_{m\in\Z} x_m \brp{\frac{-1}{z^\ast}}^{-m}}^\ast
        && \text{by definition of \fncte{z-transform} \xref{def:opZ}}
      \\&= \pm (-1)^\xN z^{-\xN} \brs{\sum_{m\in\Z} x^\ast_m \brp{-z^{-1}}^{-m}}
        && \text{by definition of \fncte{z-transform} \xref{def:opZ}}
      \\&= \sum_{m\in\Z}(\pm) (-1)^{\xN-m} x^\ast_m z^{-(\xN-m)}
      \\&= \sum_{m\in\Z}(\pm) (-1)^{n} x^\ast_{\xN-n} z^{-n}
        && \text{where $n=\xN-m$ $\implies$ $m\eqd\xN-n$}
      \\&\implies x_n = \pm (-1)^n x^\ast_{\xN-n}
    \end{align*}

  \item Proof that $(1)\implies(3)$:
    \begin{align*}
      \Dy(\omega)
        &\eqd \Zx(z)\Big|_{z=e^{i\omega}}
        &&    \text{by definition of \fncte{DTFT} \xref{def:dtft}}
      \\&=    \brs{\pm (-1)^\xN z^{-\xN}\Zx\brp{\frac{-1}{z^\ast}}}_{z=e^{i\omega}}
        &&    \text{by (2)}
      \\&=    \pm (-1)^\xN e^{-i\omega\xN}\Zx\brp{e^{i\pi}e^{\i\omega}}
      \\&=    \pm (-1)^\xN e^{-i\omega\xN}\Zx\brp{e^{i(\omega+\pi)}}
      \\&=    \pm (-1)^\xN e^{-i\omega\xN}\Dx\brp{\omega+\pi}
        &&    \text{by definition of \fncte{DTFT} \xref{def:dtft}}
    \end{align*}

  \item Proof that $(1)\implies(6)$:
    \begin{align*}
      \Dx(\omega) 
        &= \sum_{n\in\Z}  y_{n}  e^{-i\omega n}
        && \text{by definition of \fncte{DTFT} \xref{def:dtft}}
      \\&= \sum_{n\in\Z} \mcom{\pm (-1)^n x_{\xN-n}^\ast}{\structe{CQF}} e^{-i\omega n}
        && \text{by (1)}
      \\&= \sum_{m\in\Z}\pm (-1)^{\xN-m} x_m^\ast e^{-i\omega (\xN-m)}
        && \text{where } m\eqd\xN-n \implies n=\xN-m
      \\&= \pm (-1)^\xN e^{-i\omega\xN}
           \sum_{m\in\Z}(-1)^m x_m^\ast e^{i\omega m}
      \\&= \pm (-1)^\xN e^{-i\omega\xN}
           \sum_{m\in\Z}e^{i\pi m} x_m^\ast e^{i\omega m}
      \\&= \pm (-1)^\xN e^{-i\omega\xN}
           \sum_{m\in\Z}x_m^\ast e^{i (\omega+\pi )m}
      \\&= \pm (-1)^\xN e^{-i\omega\xN}
           \left[ \sum_{m\in\Z}x_m e^{-i(\omega+\pi )m} \right]^\ast
      \\&= \pm (-1)^\xN e^{-i\omega\xN}\Dx^\ast\left(\omega+\pi \right)
        && \text{by definition of \fncte{DTFT} \xref{def:dtft}}
    \end{align*}

  \item Proof that $(1)\impliedby(3)$:
    \begin{align*}
      y_n
        &= \frac{1}{2\pi}\int_{-\pi}^{+\pi} \Dy(\omega) e^{i\omega n} \dw
        && \text{by \prefp{thm:idtft}}
      \\&= \frac{1}{2\pi}\int_{-\pi}^{+\pi} \mcom{\pm (-1)^\xN e^{-i\xN\omega} \Dx^\ast(\omega+\pi)}{right hypothesis} e^{i\omega n} \dw
        && \text{by right hypothesis}
      \\&= \pm (-1)^\xN\frac{1}{2\pi}\int_{-\pi}^{+\pi}  \Dx^\ast(\omega+\pi) e^{i\omega(n-\xN)} \dw
        && \text{by right hypothesis}
      \\&= \pm (-1)^\xN\frac{1}{2\pi}\int_{0}^{2\pi}  \Dx^\ast(v) e^{i(v-\pi)(n-\xN)} \dv
        && \text{where $v\eqd\omega+\pi$ $\implies$ $\omega=v-\pi$}
      \\&= \pm (-1)^\xN e^{-i\pi(n-\xN)} \frac{1}{2\pi}\int_{0}^{2\pi}  \Dx^\ast(v) e^{iv(n-\xN)} \dv
      \\&= \pm (-1)^\xN \mcom{(-1)^\xN}{$e^{i\pi N}$} \mcom{(-1)^n}{$e^{-i\pi n}$} 
           \brs{\frac{1}{2\pi}\int_{0}^{2\pi}  \Dx(v) e^{iv(\xN-n)} \dv}^\ast
      \\&= \pm (-1)^n x_{\xN-n}^\ast
        && \text{by \prefp{thm:idtft}}
    \end{align*}

  \item Proof that (1)$\iff$(4): % 2013jun14fri
    \begin{align*}
      y_{n} = \pm (-1)^n x^\ast_{\xN-n}
        &\iff (\pm)(-1)^n y_{n} = (\pm)(\pm) (-1)^n (-1)^n x^\ast_{\xN-n}
      \\&\iff \pm(-1)^n y_{n} = x^\ast_{\xN-n}
      \\&\iff \brp{\pm(-1)^n y_{n}}^\ast = \brp{x^\ast_{\xN-n}}^\ast
      \\&\iff \pm(-1)^n y^\ast_{n} = x_{\xN-n}
      \\&\iff x_{\xN-n} = \pm(-1)^n y^\ast_{n}
      \\&\iff x_m = \pm(-1)^{\xN-m} y^\ast_{\xN-m}
        &&    \text{where $m\eqd \xN-n\implies n=\xN-m$}
      \\&\iff x_m = \pm(-1)^{\xN-m} y^\ast_{\xN-m}
      \\&\iff x_m = \pm(-1)^\xN (-1)^m y^\ast_{\xN-m}
      \\&\iff x_n = \pm(-1)^\xN (-1)^n y^\ast_{\xN-n}
        &&    \text{by change of free variables}
    \end{align*} 

  \item Proofs for (5) and (6): not included. See proofs for (2) and (3).
\end{enumerate}
\end{proof}


%--------------------------------------
\begin{theorem}
\footnote{
  \citerpp{vidakovic}{82}{83},
  \citerpp{mallat}{241}{242}
  }
\label{thm:cqf_ddw}
%--------------------------------------
Let $\Dy(\omega)$ and $\Dx(\omega)$ be the \ope{DTFT}s \xref{def:dtft} of the sequences $\seqxZ{y_n}$ and $\seqxZ{x_n}$, respectively, in $\spllR$ \xref{def:spllR}.
\thmboxt{
  Let $y_n = \pm(-1)^n x^\ast_{N-n}$ (\prope{CQF condition}, {\scs\prefp{def:cqf}}). Then\\
  $\brb{\begin{array}{D@{\qquad}>{\ds}rc>{\ds}lcl@{\qquad}D}
       (A)&\left.\opddwn  \Dy(\omega)\right|_{\omega=0} = 0
          &\iff  & \left.\opddwn  \Dx(\omega)\right|_{\omega=\pi}   &=& 0 & (B)
       \\&&\iff  & \sum_{k\in\Z} (-1)^k k^n  x_k                    &=& 0 & (C)
       \\&&\iff  & \sum_{k\in\Z} k^n  y_k                           &=& 0 & (D)
  \end{array}}
  \quad\scy\forall n\in\Znn$
  }
\end{theorem}
\begin{proof}
\begin{enumerate}
  \item Proof that (A)$\implies$(B): 
    \begin{align*}
      0
        &= \opddwn\Dy(\omega)\Big|_{\omega=0}
        && \text{by (A)}
      \\&= \opddwn(\pm)(-1)^\xN e^{-i\omega\xN}\Dx^\ast(\omega+\pi)\Big|_{\omega=0}
        && \text{by \prefp{thm:cqf}}
      \\&= \brlr{(\pm)(-1)^\xN \sum_{\ell=0}^n \bcoef{n}{\ell} \brs{\opddw}^\ell\brs{e^{-i\omega\xN}}\cdot\brs{\opddw}^{n-\ell}\brs{\Dx^\ast(\omega+\pi)}}_{\omega=0}
        && \text{by \thme{Leibnitz GPR} \xref{lem:LGPR}}
      \\&= \brlr{(\pm)(-1)^\xN \sum_{\ell=0}^n \bcoef{n}{\ell} {-i\xN}^\ell e^{-i\omega\xN}\brs{\opddw}^{n-\ell}\brs{\Dx^\ast(\omega+\pi)}}_{\omega=0}
      \\&= \brlr{(\pm)(-1)^\xN  \cancelto{1}{e^{-i0\xN}} \sum_{\ell=0}^n \bcoef{n}{\ell} {-i\xN}^\ell \brs{\opddw}^{n-\ell}\brs{\Dx^\ast(\omega+\pi)}}_{\omega=0}
    \end{align*}
  \[\begin{array}{r rcl l}
    \implies & \Dx^{(0)}(\pi) &=& 0 \\
    \implies & \Dx^{(1)}(\pi) &=& 0 \\
    \implies & \Dx^{(2)}(\pi) &=& 0 \\
    \implies & \Dx^{(3)}(\pi) &=& 0 \\
    \implies & \Dx^{(4)}(\pi) &=& 0 \\
    \vdots   & \mc{1}{c}{\vdots}    \\
    \implies & \Dx^{(n)}(\pi) &=& 0 & \text{for $n=0,1,2,\ldots$}
  \end{array}\]

  \item Proof that (A)$\impliedby$(B): 
    \begin{align*}
      0
        &= \opddwn\Dx(\omega)\Big|_{\omega=\pi}
        && \text{by (B)}
      \\&= \opddwn(\pm) e^{-i\omega\xN} \Dy^\ast(\omega+\pi)\Big|_{\omega=\pi}
        && \text{by \prefp{thm:cqf}}
      \\&= \brlr{(\pm) \sum_{\ell=0}^n \bcoef{n}{\ell} \brs{\opddw}^\ell\brs{e^{-i\omega\xN}}\cdot\brs{\opddw}^{n-\ell}\brs{\Dy^\ast(\omega+\pi)}}_{\omega=\pi}
        && \text{by \thme{Leibnitz GPR} \xref{lem:LGPR}}
      \\&= \brlr{(\pm) \sum_{\ell=0}^n \bcoef{n}{\ell} (-i\xN)^\ell e^{-i\omega\xN}\brs{\opddw}^{n-\ell}\brs{\Dy^\ast(\omega+\pi)}}_{\omega=\pi}
      \\&= \brlr{(\pm) \cancelto{1}{e^{-i\pi\xN}} \sum_{\ell=0}^n \bcoef{n}{\ell} {-i\xN}^\ell \brs{\opddw}^{n-\ell}\brs{\Dy^\ast(\omega+\pi)}}_{\omega=\pi}
      \\&= \brlr{(\pm) (-1)^\xN \sum_{\ell=0}^n \bcoef{n}{\ell} {-i\xN}^\ell \brs{\opddw}^{n-\ell}\brs{\Dy^\ast(\omega+\pi)}}_{\omega=\pi}
    \end{align*}
  \[\begin{array}{r rcl l}
    \implies & \Dy^{(0)}(0) &=& 0 \\
    \implies & \Dy^{(1)}(0) &=& 0 \\
    \implies & \Dy^{(2)}(0) &=& 0 \\
    \implies & \Dy^{(3)}(0) &=& 0 \\
    \implies & \Dy^{(4)}(0) &=& 0 \\
    \vdots   & \mc{1}{c}{\vdots}    \\
    \implies & \Dy^{(n)}(0) &=& 0 \\
    \implies & \Dy^{(n)}(0) &=& 0 & \text{for $n=0,1,2,\ldots$}
  \end{array}\]


  \item Proof that (B)$\iff$(C): by \prefp{thm:dtft_ddw}
  \item Proof that (A)$\iff$(D): by \prefp{thm:dtft_ddw}
  \item Proof that (CQF)$\notimpliedby$(A): Here is a counterexample: $\Dy(\omega)=0$.

\end{enumerate}
\end{proof}





%\begin{figure}[h]
%\begin{center}
%\begin{tabular}{cc}
%  %\includegraphics*[width=5\tw/16, clip=true]{../common/wavelets/d2sc_x3125.eps} &
%  \psset{unit=10mm}%============================================================================
% Daniel J. Greenhoe
% LaTeX file
% Daubechies-p2 scaling function and coefficients
% nominal unit = 8mm
% nominal fontsize = \scriptsize
%============================================================================
\begin{pspicture}(-1,-1.5)(3.5,2.2)%
  %-------------------------------------
  % nodes
  %-------------------------------------
  \pnode(0, 0.4829629131){h0}%  
  \pnode(1, 0.8365163037){h1}%  
  \pnode(2, 0.2241438680){h2}%  
  \pnode(3,-0.1294095226){h3}%  
  %-------------------------------------
  % plot
  %-------------------------------------
  \psaxes[linecolor=axis,yAxis=false,labels=none,linewidth=0.75pt]{->}(0,0)(0,-1.5)(3.5,2.2)%  x-axis
  \psaxes[linecolor=axis,xAxis=false,linewidth=0.75pt]{<->}(0,0)(0,-1.5)(3.5,2.2)% y-axis
  \psline[linecolor=red]{-o}(0,0)(h0)%  
  \psline[linecolor=red]{-o}(1,0)(h1)%  
  \psline[linecolor=red]{-o}(2,0)(h2)%  
  \psline[linecolor=red]{-o}(3,0)(h3)%  
  \fileplot{../../common/wavelets/graphics/d2_phi.dat}%
  %-------------------------------------
  % labels
  %-------------------------------------
  \uput{2mm}[-90](1,0){$1$}%
 %\uput{2mm}[-90](2,0){$2$}%
  \uput{2mm}[ 90](3,0){$3$}%
\end{pspicture}% &
%  \setlength{\unitlength}{0.10mm}%============================================================================
% LaTeX File
% Daniel J. Greenhoe
% Daubechies-2 Pole Zero plot
% nominal font size = \gsize
% nominal unit = 10mm
%============================================================================
\begin{pspicture}(-1.5,-2)(5.1,2)%
  %-------------------------------------
  % settings
  %-------------------------------------
  \psset{radius=1mm,linewidth=0.75pt}%
  %-------------------------------------
  % nodes
  %-------------------------------------
  \pnode(0,0){origin}%          z-plane origin
  \pnode(-1,0){pzeroes}%        p zeroes location
  \pnode(0,0){poles}%           2p-1 poles location
  %-------------------------------------
  % asymptotic regions
  %   |z-1| = sqrt(2) (arc with radius sqrt(2) centered at z=1)
  %   |z+1| = sqrt(2) (arc with radius sqrt(2) centered at z=-1)
  % sqrt(2) ~= 1.4142135623730950488016887242097
  %-------------------------------------
  \pscustom[fillstyle=solid,fillcolor=pzasymshade,linestyle=solid,linecolor=pzasym]{%
    \psarc(-1,0){1.414214}{-45}{45}%
    \psarcn( 1,0){1.414214}{135}{-135}% 
    }%
  %\pnode(0.2,0.73){zm1P}%
  %\pnode(2,1){zp1P}%
  %\rput[bl](-1.5,0.433){\rnode[r]{zm1L}{$\scy\abs{z-1}=\sqrt{2}$}}%
  %\rput[br](2.9,1.6){\rnode[b]{zp1L}{$\scy\abs{z+1}=\sqrt{2}$}}%
  %\ncline[linecolor=pzasym]{->}{zm1L}{zm1P}%
  %\ncline[linecolor=pzasym,nodesepA=2pt]{->}{zp1L}{zp1P}%
  %-------------------------------------
  % axes
  %-------------------------------------
  \psaxes[linecolor=axis,labels=none]{<->}(0,0)(-1.5,-1.5)(4.5,1.5)%
  \uput{1pt}[0](4.5,0){\color{axis}$\scy\Reb{z}$}%
  \uput{2pt}[210](0,1.5){\color{axis}$\scy\Imb{z}$}%
  %-------------------------------------
  % unit circle
  %-------------------------------------
  \pscircle[linecolor=unitcircle](origin){1}%            unit circle
  \rput[tr](-1.5mm,-1.5mm){\rnode[b]{circleL}{$\scy\abs{z}=1$}}% circle label
  \pnode(-0.707,-0.707){circleP}% circle point
  \ncline[linecolor=unitcircle,linestyle=dotted,nodesepA=0pt]{->}{circleL}{circleP}% circle pointer
  %-------------------------------------
  % B-spline zeroes (at z=-1)
  %-------------------------------------
  \Cnode[linewidth=1pt](-1,0){zp}%
  \rput[bl](-1.45,-1.9){\rnode[t]{zpL}{$p$}$=2$ zeroes}%
  \ncline[linestyle=dotted,linecolor=zero,nodesepA=1pt]{->}{zpL}{zp}%
  \uput{1mm}[30](-1,0){\color{zero}$\scy2$}%
  %-------------------------------------
  % orthogonality zeroes
  %-------------------------------------
  \Cnode [linecolor=zero,linewidth=1pt](0.2679491924, 0 ){z1}% inner zero
  \Cnode*[linecolor=zero,linewidth=1pt](3.7320508076, 0 ){z2}% outer zero
  %-------------------------------------
  % radial lines
  %-------------------------------------
  \ncline[linestyle=dashed]{origin}{z2}%
  %-------------------------------------
  % poles
  %-------------------------------------
  %\rput(poles){\color{pole}$\mathbf\times$}%
  \rput(poles){%
    \psline[linecolor=pole,linewidth=1pt](-0.1,+0.1)(+0.1,-0.1)%
    \psline[linecolor=pole,linewidth=1pt](-0.1,-0.1)(+0.1,+0.1)%
    }%
  \rput[tl](-1.45,1.9){$2$\rnode[br]{polesL}{$p$}$-1=3$ poles}%
  \ncline[linestyle=dotted,linecolor=pole,nodesepA=0pt,nodesepB=5pt]{->}{polesL}{poles}%
  \uput{1mm}[30](0,0){\color{pole}$\scy3$}%
  %-------------------------------------
  % discarded zeroes region
  %-------------------------------------
 %\psframe[linestyle=dotted](0.8,-1.9)(3,1.9)%
  \psframe[linestyle=dotted](3,-1.9)(4,1.9)%
  \rput[b]{-90}(3,0){\color{zero}\footnotesize$p-1=1$ discarded zero}%
\end{pspicture}%
%
%\begin{picture}(750,400)(-150,-200)
%  %\graphpaper[10](0,0)(200,200)
%  \thicklines%
%  \color{axis}%
%    \color{axis}%
%    \put(-130,   0){\line(1,0){550} }
%    \put(   0,-130){\line(0,1){260} }
%    \put( 430,   0){\makebox(0,0)[l]{$\Reb{z}$}}
%    \put(   0, 140){\makebox(0,0)[b]{$\Imb{z}$}}
%  \color{circle}%
%    %\input{circ512.inc}
%    \color{circle}%
%    \qbezier( 100,   0)( 100, 41.421356)(+70.710678,+70.710678)% 0   -->1pi/4
%    \qbezier(   0, 100)( 41.421356, 100)(+70.710678,+70.710678)% pi/4-->2pi/4
%    \qbezier(   0, 100)(-41.421356, 100)(-70.710678,+70.710678)%2pi/4-->3pi/4
%    \qbezier(-100,   0)(-100, 41.421356)(-70.710678,+70.710678)%3pi/4--> pi
%    \qbezier(-100,   0)(-100,-41.421356)(-70.710678,-70.710678)% pi  -->5pi/4
%    \qbezier(   0,-100)(-41.421356,-100)(-70.710678,-70.710678)%5pi/4-->6pi/4
%    \qbezier(   0,-100)( 41.421356,-100)( 70.710678,-70.710678)%6pi/4-->7pi/4
%    \qbezier( 100,   0)( 100,-41.421356)( 70.710678,-70.710678)%7pi/4-->2pi
%    %\put( 110, 110){\makebox(0,0)[lb]{$z=e^{i\omega}$}}
%    %\put( 105, 105){\vector(-1,-1){33}}
%  \color{pzasym}%
%    \qbezier(41.4214,0)(41.4214, 58.5786)(0, 100)%      % inner upper arc
%    \qbezier(41.4214,0)(41.4214,-58.5786)(0,-100)%      % inner lower arc
%    \qbezier(0, 100)(41.421, 141.421)(100, 141.421)    %outer upper arc
%    \qbezier(100, 141.421)(158.579,141.421)(200,100)
%    \qbezier(200,100)(241.421,58.5786)(241.421,0)
%    \qbezier(0,-100)(41.421,-141.421)(100,-141.421)    %outer lower arc
%    \qbezier(100,-141.421)(158.579,-141.421)(200,-100)
%    \qbezier(200,-100)(241.421,-58.5786)(241.421,0)
%  \color{pole}%
%    \color{pole}%
%    \put(-110,-110){\makebox(0,0)[tl]{3 poles}}%
%    \put(-100,-100){\vector(1, 1){91}}%
%    \put(   0,    0){\makebox(0,0)[c]{$\times$}}%
%    \put(   0,    0){\makebox(0,0)[c]{$\times$}}%
%    \put(   0,    0){\makebox(0,0)[c]{\hspace{1em}$^3$}}%
%  \color{zero}%
%    \color{zero}%
%    \put(-10,110){\makebox(0,0)[br]{p=2 zero}}%
%    \put(-100,100){\vector( 0,-1){60}}%
%    \put(-100,    0){\circle{15}}%
%    \put(-100,    0){\circle{15}$^2$}%
%    \put( 100,76){\makebox(0,0)[bl]{1 zero}}%
%    \put( 98,71){\vector(-1,-1){64}}%
%    \put(373.2,-50){\vector(0,1){40}}%
%    \put(373.2,-60){\makebox(0,0)[ct]{discarded zero from $\fQ(z^{-1})$}}%
%    \put(  26.79491924, 0 ){\circle {15}}% inner zero
%    \put( 373.20508076, 0 ){\circle*{15}}% outer zero
%\end{picture}
  \\
%   Scaling function & Pole-zero plot
%\end{tabular}
%\caption{
%   Daubechies-2 scaling function and scaling filter pole-zero plot
%   \label{fig:pz_d2}
%   }
%\end{center}
%\end{figure}





%%--------------------------------------
%\begin{definition}
%\index{causal}
%%--------------------------------------
%A filter (or system or operator) $\mathcal{H}$ is {\bf causal} 
%if its current output does not depend on future inputs.
%\end{definition}
%
%%--------------------------------------
%\begin{definition}
%\index{time-invariant}
%%--------------------------------------
%A filter (or system or operator) $\mathcal{H}$ is {\bf time-invariant} 
%if the mapping it performs does not change with time.
%\end{definition}
%
%%--------------------------------------
%\begin{definition}
%\index{linear}
%%--------------------------------------
%An operation $\mathcal{H}$ is {\bf linear} if any output $y_n$ can be described
%as a linear combination of inputs $x_n$ as in 
%\begin{eqnarray*}
%   y_n &=& \sum\limits_m h(m) x(n-m).
%\end{eqnarray*}
%\end{definition}

