%============================================================================
% Daniel J. Greenhoe
% LaTeX file
%============================================================================
%======================================
\chapter{Probability Space}
\label{chp:prbspace}
%======================================
%=======================================
\section{A generalized probability function}
%=======================================
The traditional probability function $\psp$ is defined on a \structe{Boolean lattice}\ifsxref{boolean}{def:boolean}.
This text introduces a new definition of $\psp$ which can be defined not only on 
\structe{Boolean lattice}s, but on some \prope{non-Boolean} lattices as well.
%This new freedom is exploited in \prefpp{sec:apps_prob}.

%=======================================
\subsection{Definitions}
%=======================================
%---------------------------------------
\begin{definition}
%\footnote{
%  \citerppgc{pap1995}{8}{9}{0792336585}{Definition 2.3(13)}
%  }
\label{def:prob}
\label{def:ps}
\label{def:psp}
%---------------------------------------
%Let $\intcc{0}{1}$ be a \structe{real interval}.
%Let $\latL\eqd\logicX$ be a \structe{logic} \xref{def:logic}.
Let $\latL\eqd\latnX$ be a \structe{lattice with negation}\ifsxref{negation}{def:latn}.
Let $\distrib$ be the \rele{distributivity relation}\ifsxref{latd}{def:Drel}.
\defboxt{
  A function $\psp$ in $\clFxr$ is a \fnctd{probability} function on $\latL$ if
  \\$\begin{array}{>{\qquad}Flc lcl CDD}
    1. &                   &        & \psp(\lzero)   &=&      0               &                     & (\prope{nondegenerate}) & and \\
    2. &                   &        & \psp(\lid)     &=&      1               &                     & (\prope{normalized})    & and \\
    3. & x\orel y          &\implies& \psp(x)        &\orel&  \psp(y)         & \forall x,y\in\setX & (\prope{monotone})      & and \\
   %3. & x\meet y = \lzero &\implies& \psp(x\join y) &=&      \psp(x)+\psp(y) & \forall x,y\in\setX & (\prope{additive})      & .      
    4. & \brb{\begin{array}{lD}
           x\meet y = \lzero            & and\\
           \otriple{z}{x}{y}\in\distrib & $\forall z\in\setX$
         \end{array}} 
       &\implies& \psp(x\join y) &=&      \psp(x)+\psp(y) & \forall x,y\in\setX & (\prope{additive}).   &       
  \end{array}$
  \\
  If $\psp$ is a \fncte{probability} on a \structe{lattice with negation} $\latL$, 
  then $\opair{\latL}{\psp}$ is a \structd{probability space}. % over $\pso$,
  %$\pso$ is the \structd{outcome space}, and $\pse$ is the \structd{event space}.
  }
\end{definition}

%---------------------------------------
\hrule
\begin{remark}
\label{rem:prob}
%---------------------------------------
\pref{def:prob} (previous) is not any standard definition of the \fncte{probability function}.
On a \structe{Boolean lattice}, the \fnctd{measure-theoretic probability} function, due to A. N. Kolmogorov, is defined as%
  \footnote{
    \citerppgc{billingsley1995}{22}{23}{0471007102}{Probability Measures},
    \citer{kolmogorov1933},
    \citerpc{kolmogorov1933e2}{16}{\structe{field of probability}},
    \citerppgc{pap1995}{8}{9}{0792336585}{Definition 2.3(13)}, %{,\prope{\txsigma-additive}},
    \citerpg{kalmbach1986}{27}{9971500094}%{\prope{\txsigma-additive},\prope{\txsigma-complete}}
    }
  \\\indentx$\begin{array}{Flc lcl CDD}
    (1). &                   &        & \psp(\lid)     &=&      1               &                     & (\prope{normalized})     & and \\
    (2). &                   &        & \psp(x)        &\oreld& 0               & \forall x\in\setX   & (\prope{nonnegative})& and \\
    (3). & \ds\meetop_{n=1}^\infty x_n=\lzero &\implies& \ds\psp\brp{\joinop_{n=1}^\infty x_n} &=&  \ds\sum_{n=1}^\infty\psp(x_n) & \forall x_n\in\setX & (\prope{\txsigma-additive})   & .      
  \end{array}$\\
The advantage of this definition is that $\psp$ is a \fncte{measure}, and hence all the power of measure theory 
is subsequently at one's disposal in using $\psp$.
However, it has often been argued that the requirement of \prope{\txsigma-additivity} is unnecessary for a probability function.
Even as early as 1930, de Finetti argued against it, in what became a kind of polite running debate with Fr/'echet.\footnote{
  \citeP{definetti1930a},
  \citeP{frechet1930a},
  \citeP{definetti1930b},
  \citeP{frechet1930b},
  \citeP{definetti1930c},
  \citePpp{cifarelli1996}{258}{260}
  }
In fact, Kolmogorov himself provided some argument against \prope{\txsigma-additivity} when referring to the closely related 
\hie{Axiom of Continuity} saying,
``Since the new axiom is essential for infinite fields of probability only,
it is almost impossible to elucidate its empirical meaning\ldots
For, in describing any observable random process we can obtain only finite fields of probability.\ldots"
But in its support he added, ``This limitation has been found expedient in researches of the most diverse sort."\footnote{
  \citerp{kolmogorov1933e2}{15}
  }

There are several other definitions of probability that only require \prope{additivity} rather than \prope{\txsigma-additivity}.
On a \structe{Boolean lattice}, the \fnctd{traditional probability} function is defined as%
  \footnote{
    \citerppg{papoulis}{21}{22}{0070484775},
    \citerpc{kolmogorov1933e2}{2}{\textsection 1. Axioms I--V}
    }
  \\\indentx$\begin{array}{Flc lcl CDD}
    (1). &                   &        & \psp(\lid)     &=&      1               &                     & (\prope{normalized})     & and \\
    (2). &                   &        & \psp(x)        &\oreld& 0               & \forall x\in\setX   & (\prope{nonnegative})& and \\
    (3). & x\meet y = \lzero &\implies& \psp(x\join y) &=&      \psp(x)+\psp(y) & \forall x,y\in\setX & (\prope{additive})   & .      
  \end{array}$\\
This definition implies (on a \structe{Boolean lattice}) that 
  \\\indentx$\begin{array}{FrclCDD}
      (a). & \psp(\lzero)  &=&     0                              &                     & (\prope{nondegenerate})      & and
    \\(b). & \psp(x)       &\orel& 1                              & \forall x\in\setX   & (\prope{upper bounded})      & and 
    \\(c). & \psp(x)       &=&     1-\psp(\negat{x})              & \forall x\in\setX   &                              & and 
    \\(d). & \psp(x\join y)&\orel& \psp(x)+\psp(y)                & \forall x,y\in\setX & (\prope{subadditive})        & and 
    \\(e). & \psp(x\join y)&=&     \psp(x)+\psp(y)-\psp(x\meet y) & \forall x,y\in\setX &                              & and
    \\(f). & x\orel y \implies \psp(x)&\orel&\psp(y)              & \forall x,y\in\setX & (\prope{monotone})           & . 
  \end{array}$
  \\
On a \structe{distributive pseudocomplemented lattice}, the \fnctd{generalized probability} function has been defined as%
  \footnote{
    \citePp{narens2014}{118},
    \citerg{narens2007}{9812708014}
    }
  \\$\begin{array}{>{\qquad}Frcl CDD}
    (1). &  \psp(\lzero)    &=&      0                                              & (\prope{nondegenerate}) & and \\
    (2). &  \psp(\lid)      &=&      1                                              & (\prope{normalized}) & and \\
    (3). &  0\orel\psp(\lid)&\orel&  1                                              &                & and \\
    (4). &  \psp(x\join y)  &=&\psp(x)+\psp(y)-\psp(x\meet y) & \forall x,y\in\setX &                &  .      
  \end{array}$
  \\
On an \structe{orthomodular lattice}, or a \structe{finite modular lattice}, 
the \fnctd{quantum probability} function is defined as%
  \footnote{
    \citePpc{greechie1971}{126}{{\scshape Definitions}},
    \citePp{narens2014}{118}
    }
  \\$\begin{array}{>{\qquad}Flc lcl CDD}
    (1). &                   &        & \psp(\lzero)   &=&      0               &                     & (\prope{nondegenerate})& and \\
    (2). &                   &        & \psp(\lid)     &=&      1               &                     & (\prope{normalized})  & and \\
    (3). & x\ocop y          &\implies& \psp(x\join y) &=&      \psp(x)+\psp(y) & \forall x,y\in\setX & (\prope{additive})& .      
  \end{array}$
  \\
However, for lattices that are not \prope{distributive}, \prope{modular}, or \prope{orthomodular}, 
none of these definitions work out so well.
Take for example the \structe{O$_6$ lattice} with the ``very reasonable" probability function given in \prefpp{ex:ps_o6}.
This probability space $\opair{\text{O$_6$}}{\psp}$ fails to be any of the 4 probability functions defined in this Remark.
It fails to be a \fncte{measure-theoretic} or \fncte{traditional probability} function because 
\\\indentx$a\meet b=0$\qquad{but}\qquad$\psp(a\join b)=\psp(\lid)=1\neq\frac{1}{3}+\frac{1}{2}=\psp(a)+\psp(b)$ .\\
It fails to be a \fncte{generalized probability} function because 
\\\indentx$\psp(a\join b)=\psp(\lid)=1\neq\frac{1}{3}+\frac{1}{2}-0=\psp(a)+\psp(b)-\psp(0)=\psp(a)+\psp(b)-\psp(a\meet b)$ .\\
It fails to be an \fncte{quantum probability} function because 
\\\indentx$a\ocop b=0$\qquad{but}\qquad$\psp(a\join b)=\psp(\lid)=1\neq\frac{1}{3}+\frac{1}{2}=\psp(a)+\psp(b)$ .\\
In each of these cases, the function $\psp$ fails to be \prope{additive}.
The solution of \prefpp{def:prob} is simply to ``switch off" \prope{additivity} when the lattice is not \prope{distributive}.
This method is a little ``crude", but at least it allows us to define probability on a very wide class of lattices,
while retaining compatibility with the \prope{Boolean} case \xxxref{prop:ps_01}{prop:ps_ortho_xy}{prop:ps_boa_xy}.
\end{remark}
(end \pref{rem:prob})
\hrule

%=======================================
\subsection{Properties}
%=======================================
%---------------------------------------
\begin{proposition}
\footnote{
  \citerpgc{papoulis}{21}{0070484775}{(2-11)}
  }
\label{prop:ps_01}
%---------------------------------------
%Let $\opair{\latL}{\psp}$ be a \structe{probability space} \xref{def:ps}.
\propbox{
  \brb{\begin{array}{M}
    $\ps$ is a \structe{probability space}\\ 
    \xref{def:ps}
  \end{array}}
  \qquad\implies\qquad
  \brb{\begin{array}{rc rclCDD}
   %  &     & \psp(\lzero) &=& 0     &                   & \\
    0 &\orel& \psp(x)      &\orel& 1 & \forall x\in\setX & 
  \end{array}}
  }
\end{proposition}
\begin{proof}
  \begin{align*}
   % \psp(\lzero)
   %   &= \brlr{\psp(x)}_{x=\lzero}
   % \\&= \brlr{\psp(x\join y)-\psp(y)}_{x=\lzero}
   %   && \text{because $x\meet\lzero=\lzero$ and by \prope{additive} property of $\psp$ \xref{def:ps}}
   % \\&= \psp(\lzero\join y)-\psp(y)
   % \\&= \psp(y)-\psp(y)
   %   && \text{by definition of $\lzero$}
   % \\&= 0
   % \\
    0
      &= \psp(\lzero)
      && \text{by by \prope{nondegenerate} property of $\psp$ \xref{def:ps}}
     %&& \text{by previous result}
    \\&\orel \psp(x)
      && \text{because $\lzero\orel x$ and \prope{monotone} property of $\psp$ \xref{def:ps}}
    \\
    \psp(x)
      &\orel \psp(\lid)
      && \text{because $x\orel\lid$ and \prope{monotone} property of $\psp$ \xref{def:ps}}
    \\&= \lid
      && \text{by \prope{normalized} property of $\psp$ \xref{def:ps}}
  \end{align*}
\end{proof}

%---------------------------------------
\begin{proposition}
%\footnote{
%  \citerpgc{papoulis}{21}{0070484775}{(2-12)}
%  }
\label{prop:ps_ortho_xy}
\label{prop:ps_negat}
%---------------------------------------
Let $\ps$ be a \structe{probability space} \xref{def:ps}.
\propbox{
   \brb{\begin{array}{@{}MD}
     $\latL$ is \prope{orthocomplemented} & \xref{def:latoc}
   \end{array}}
   \implies
   \brb{\begin{array}{rclC}
     \psp(x) &=& 1 - \psp(\negat{x}) & \forall x\in\setX
   \end{array}}
   }
\end{proposition}
\begin{proof}
    \begin{align*}
      1-\psp(\negat{x})
        &= \psp(\lid)-\psp(\negat{x})
        && \text{by \prope{normalized} property of $\psp$ \xref{def:ps}}
      \\&= \psp(x\join\negat{x})-\psp(\negat{x})
        && \text{by \prope{excluded middle} property of \fncte{ortho negation} \xref{def:negor}}
      \\&= \psp(x)+\psp(\negat{x})-\psp(\negat{x})
        && \text{because $(x)(\negat{x})=\lzero$ and \prope{additive} property \xref{def:ps}}
      \\&= \psp(x)
    \end{align*}
\end{proof}

%---------------------------------------
\begin{proposition}
\footnote{
  \citerpgc{papoulis}{21}{0070484775}{(2-13)},
  \citerppgc{feller1970}{22}{23}{0471257087}{(7.4),(7.6)}
  }
\label{prop:ps_boa_xy}
\label{prop:ps_add}
%---------------------------------------
Let $\ps$ be a \structe{probability space} \xref{def:ps}.
\propbox{
   \brb{\begin{array}{M}
     $\latL$ is \prope{Boolean}\\
     \xref{def:boolean}
   \end{array}}
   \implies
   \brbl{\begin{array}{FrclCD}
     1. & \psp(x\join y) &=&     \psp(x) + \psp(y) - \psp(x\meet y) & \forall x,y\in\setX & and\\
     2. & \psp(x\join y) &\orel& \psp(x) + \psp(y)                  & \forall x,y\in\setX & (\prope{Boole's inequality})
   \end{array}}
   }
\end{proposition}
\begin{proof}
\begin{enumerate}
  \item lemma: Proof that $\psp((\negat x)\meet y)=\psp(y)-\psp(x\meet y)$: \label{item:ps_boa_xy}
    \begin{align*}
      \psp(y)-\psp(xy)
        &= \psp(\lid\meet{y})-\psp(xy)
        && \text{by definition of $\lid$ and $\meet$ \xref{def:meet}}
      \\&= \psp\brs{(x\join\negat{x})y}-\psp(xy)
        && \text{by \prope{excluded middle} property of \structe{Boolean lattice}s}
      \\&= \psp(xy\join\negat{x}y)-\psp(xy)
        && \text{by \prope{distributive} property of \structe{Boolean lattice}s}
      \\&= \psp(xy)+\psp(\negat{x}y)-\psp(xy)
        %&& \text{because $(xy)(\negat{x}y)=\lzero$ and by \prope{additive} prop. \xref{def:ps}}
        && \text{because $(xy)(\negat{x}y)=\lzero$ and by \prope{additive} property}
      \\&= \psp(\negat{x}y)
    \end{align*}

  \item Proof that $\psp(x\join y) = \psp(x) + \psp(y) - \psp(x\meet y)$:
    \begin{align*}
      \psp(x\join y) 
        &= \psp(x\join \negat{x}y)
        && \text{by property of \prope{Boolean lattice}s} 
      \\&= \psp(x) + \psp(\negat{x}y)
        %&& \text{because $(x)(\negat{x}y)=\lzero$ and by \prope{additive} property \xref{def:ps}}
        && \text{because $(x)(\negat{x}y)=\lzero$ and by \prope{additive} property}
      \\&= \psp(x) + \psp(y) - \psp(x\meet y)
        && \text{by \prefpp{item:ps_boa_xy}}
    \end{align*}
\end{enumerate}
\end{proof}


%=======================================
\section{Examples}
%=======================================
%---------------------------------------
\exboxt{
  \begin{tabular}{m{\tw-52.5mm}}
%\begin{example}[\exmd{/-Lukasiewicz 3-valued probability space}/\exm{Kleene 3-valued probability space}/\exm{RM$_3$ probability space}]
\begin{example}[\exmd{Kleene 3-valued probability space}]
\label{ex:prob_L3_kl}
%---------------------------------------
    The function $\negat$ on the lattice $\latL$
    as illustrated to the right is a \fncte{Kleene negation}\ifsxref{negation}{def:negkl}. %{ex:negat_L3_kl}. %{def:negkl}.
    Together with the probability function $\psp$, also illustrated to the right, the structure $\ps$
    is a \structe{probability space} \xref{def:ps}.
\end{example}
  \end{tabular}%
  \tbox{\includegraphics{graphics/lat3_L3_1x0_-0-x-1_prob.pdf}}
}

%---------------------------------------
\exboxt{
  \begin{tabular}{m{\tw-73.5mm}}
\begin{example}
\label{ex:limpx_m2}
%---------------------------------------
    The \structe{lattice with negation} $\latL$ \xref{def:latn} illustrated to the right is 
    \prope{Boolean} (and thus also \prope{distributive}---\xrefnp{thm:boolean}).
    The structure $\ps$ is a \structe{probability space} \xref{def:ps}.
\end{example}
  \end{tabular}%
  \tbox{\includegraphics{graphics/lat4_m2_boa_prob.pdf}}
  }

%---------------------------------------
\exboxt{
  \begin{tabular}{m{\tw-82.343mm}}
\begin{example}
\label{ex:ps_o6}
%---------------------------------------
    The \structe{lattice with negation} $\latL$ illustrated to the right is an
    \structe{orthocomplemented O$_6$ lattice} \xref{def:o6}.
    %Together with the probability function $\psp$, also illustrated to the right, 
    The structure $\ps$ is a \structe{probability space} \xref{def:ps}.
\end{example}
  \end{tabular}%
  \tbox{\includegraphics{graphics/lat6_o6_prob_10adbc.pdf}}
  }

%---------------------------------------
\exboxt{
  \begin{tabular}{m{\tw-83.1mm}}
\begin{example}[Single coin toss]
\label{ex:ps_coin}
%---------------------------------------
    Let $\coinhead$ represent ``heads" and $\cointail$ represent ``tails"
    in a coin toss.
    Let $0<p<1$ be the probability of a head.
    A \structe{probability space} $\ps$ for this single coin toss is as illustrated to the right.
\end{example}
  \end{tabular}%
  \tbox{\includegraphics{graphics/lat4_m2_psE_coin.pdf}}
  }


%%---------------------------------------
%\begin{example}[Single coin flip]
%%---------------------------------------
%\exbox{\begin{array}{rcl@{\qquad}D}
%  \pso &=&   \setn{\coinhead,\, \cointail}
%       &     (set of outcomes)
%    \\
%  \pse &=&   \setn{\mcom{\emptyset}{neither},\,
%                   \mcom{\setn{\coinhead}}{heads},\,
%                   \mcom{\setn{\cointail}}{tails},\,
%                   \mcom{\setn{\coinhead,\,\cointail}}{heads or tails}
%                  }
%       &     (sigma-algebra on $\pso$)
%    \\
%  \psp x &=& \left\{\begin{array}{l>{\text{for }}l@{\qquad}D}
%      0           & x=\emptyset                  & (neither heads nor tails) \\
%      \frac{1}{2} & x=\setn{\coinhead}           & (heads)                   \\
%      \frac{1}{2} & x=\setn{\text{\cointail}}    & (tails)                   \\
%      1           & x=\setn{\coinhead,\cointail} & (either heads or tails)
%    \end{array}\right.
%       &      (probability measure on $\pse$)
%\end{array}}
%\end{example}

%%---------------------------------------
%\begin{example}[Two coin flips]
%%---------------------------------------
%Let $\seto{\setA}$ be the number of elements in a set $\setA$.
%Let $\coinhead$ represent ``heads" and $\cointail$ represent ``tails"
%in a coin toss.
%Suppose we flip a coin two consecutive time and want to
%know how many ``heads" we will have.
%The following is a valid probability space $\ps$:
%\exbox{\begin{array}{rcl}
%  \pso &=& \setn{ (\cointail\cointail),\, (\cointail\coinhead),\, (\coinhead\cointail),\, (\coinhead\coinhead) }  \\
%  \pse &=& \setn{\begin{array}{l}
%    \mcom{\emptyset}{nothing},\,
%    \mcom{\setn{(\cointail\cointail)}}{0 heads},\,
%    \mcom{\setn{(\cointail\coinhead),\,(\coinhead\cointail)}}{1 heads},\,
%    \mcom{\setn{(\coinhead\coinhead)}}{2 heads},\,
%    \mcom{\setn{(\cointail\cointail),\,(\cointail\coinhead),\,(\coinhead\cointail)}}{0 heads or 1 heads},\, \\
%    \mcom{\setn{(\cointail\cointail),\,(\coinhead\coinhead)}}{0 heads or 2 heads},\,
%    \mcom{\setn{(\cointail\coinhead),\,(\coinhead\cointail),\,(\coinhead\coinhead)}}{1 heads or 2 heads},\,
%    \pso
%    %\mcom{\setn{(\cointail\cointail),\, (\cointail\coinhead),\, (\coinhead\cointail),\, (\coinhead\coinhead)} }{0 or 1 or 2 heads ($\pso$)}
%    \end{array}}
%  \\
%  \psp x &=& \frac{1}{4} \seto{x}
%%  \psp x &=& \left\{\begin{array}{*{2}{l>{\text{for }}l@{\qquad\qquad}}}
%%    0           & x=\emptyset         & \frac{3}{4} & x=\text{ 0 heads or 1 heads}            \\
%%    \frac{1}{4} & x=\text{ 0 heads}   & \frac{1}{2} & x=\text{ 0 heads or 2 heads}            \\
%%    \frac{1}{2} & x=\text{ 1 heads}   & \frac{3}{4} & x=\text{ 1 heads or 2 heads}            \\
%%    \frac{1}{4} & x=\text{ 2 heads}   & 1           & x=\text{ 0 or 1 or 2 heads}
%%    \end{array}\right.
%\end{array}
%}
%
%Note however that
%\[   \pse = \left\{ \emptyset, \mbox{0 heads}, \mbox{1 head}, \mbox{2 heads}, \pso \right\} \]
%is {\bf not} a valid event set because it is {\bf not} a sigma-algebra.
%For example, $\mbox{1 head}\cup\mbox{2 heads}$ is not in $\pse$.
%\end{example}

%---------------------------------------
\begin{figure}% Double coin toss
%---------------------------------------
  \centering
  \exbox{
    \tbox{\includegraphics{graphics/lat8_l2e3_prob_coin.pdf}}
    \begin{array}{rclrclrcl}
      \psp(\emptyset) &=&0        & \psp(\setO)          &=& 1               \\
      \psp(\setA)     &=&(1-p)^2  & \psp(\setA\setu\setB)&=&(1-p)^2+2p(1-p)  \\
      \psp(\setB)     &=&2p(1-p)  & \psp(\setA\setu\setC)&=& 2p^2-2p+1       \\
      \psp(\setC)     &=&p^2      & \psp(\setB\setu\setC)&=&-p^2+2p 
    \end{array}
    }
  \caption{\structe{Double coin toss} \xref{ex:ps_doublecoin} \label{fig:ps_doublecoin}}
\end{figure}
%---------------------------------------
\begin{example}[\exmd{Double coin toss}]
\label{ex:ps_doublecoin}
%---------------------------------------
    Let $\coinhead$ represent ``heads" and $\cointail$ represent ``tails"
    in a double coin toss in which each toss is \prope{independent} \xref{def:indepen} of the other.
    Let $0<p<1$ be the probability of a head.
    The \structe{probability space} $\ps$ is illustrated in \prefpp{fig:ps_doublecoin}.
\end{example}
\begin{proof}
\begin{align*}
  \psp(\setO)
    &= 1
    && \text{by \prope{normalized} property of $\psp$ \xref{def:psp}}
  \\
  \psp(\setC)
    &= \psp\setn{\coinhead\coinhead}
    && \text{by definition of $\setC$}
  \\&= \psp(\coinhead)\psp(\coinhead)
    && \text{by definition of \prope{independence} \xref{def:indepen}}
  \\&= p^2
    && \text{by definition of $p$}
  \\
  \psp(\setA)
    &= \psp\setn{\cointail\cointail}
    && \text{by definition of $\setA$}
  \\&= \psp(\cointail)\psp(\cointail)
    && \text{by definition of \prope{independence} \xref{def:indepen}}
  \\&= \brb{1-\psp(\coinhead)}\brb{1-\psp(\coinhead)}
    && \text{by \pref{prop:ps_negat}}
  \\&= (1-p)^2
    && \text{by definition of $p$}
  \\
  \psp(\setB)
    &= \psp\setn{(\cointail\coinhead),(\coinhead\cointail)}
    && \text{by definition of $\setB$}
  \\&= \psp\setn{\cointail\coinhead} + \psp\setn{\coinhead\cointail}
    && \text{by \prope{additive} property of $\psp$ \xref{def:psp}}
  \\&= \psp\setn{\cointail}\psp\setn{\coinhead} + \psp\setn{\coinhead}\psp\setn{\cointail}
    && \text{by definition of \prope{independence} \xref{def:indepen}}
  \\&= (1-p)p + p(1-p)
    && \text{by \pref{prop:ps_negat} and definition of $p$}
  \\&= -2p^2 + p + 1
  \\
  \psp(\setA\setu\setB)
    &= \psp(\setA) + \psp(\setB) -\psp(\setA\seti\setB)
    && \text{by \pref{prop:ps_add}}
  \\&= \psp(\setA) + \psp(\setB) -\psp(\emptyset)
  \\&= (1-p)^2 + (-2p^2 + p + 1) + 0
    && \text{by previous results}
  \\&= -p^2-p+1
  \\
  \psp(\emptyset)
    &= 0
    && \text{by \prope{nondegenerate} property of $\psp$ \xref{def:psp}}
\end{align*}
\end{proof}

%%---------------------------------------
%\begin{example}[Even/odd dice measure space]
%\label{ex:prob_even_odd_dice}
%%---------------------------------------
%Suppose we have an ``unfair" dice and we want to know whether
%the result of rolling the dice one time will be ``even" or ``odd".
%We can construct the following measure space (probability space) $\ps$:
%
%\exbox{\begin{array}{ll}
%  \pso &= \left\{ \text{\diceA,\diceB,\diceC,\diceD,\diceE,\diceF} \right\}
%  \\
%  \pse &= \left\{ \mcom{\left\{\hspace{1ex} \right\}}{$\emptyset$},
%                  \mcom{\left\{\text{\diceA,\diceC,\diceE}\right\}}{odd},
%                  \mcom{\left\{\text{\diceB,\diceD,\diceF}\right\}}{even},
%                  \mcom{\left\{ \text{\diceA,\diceB,\diceC,\diceD,\diceE,\diceF} \right\}}{$\pso$}
%          \right\}
%  \\
%  \psp(e) &=
%    \left\{\begin{array}{l@{\qquad} >{\text{for }e=}l @{\qquad}D}
%      0           & \{\hspace{1ex}\}
%                  & ($\emptyset$)
%                  \\
%      1           & \left\{ \text{\diceA,\diceB,\diceC,\diceD,\diceE,\diceF} \right\}
%                  & ($\pso$)
%                  \\
%      \frac{1}{3} & \left\{\text{\diceA,\diceC,\diceE}\right\}
%                  & (odd)
%                  \\
%      \frac{2}{3} & \left\{\text{\diceB,\diceD,\diceF}\right\}
%                  & (even)
%    \end{array}\right.
%\end{array}}
%\end{example}
%
%Example~\ref{ex:prob_even_odd_dice} (previous) illustrated a
%measure space in which the events (ignorning $\emptyset$ and $\pso$)
%are {\em mutually exclusive}.
%Example~\ref{ex:prob_dice} (next) illustrates a measure space
%where events are {\em not} mutually exclusive.

%---------------------------------------
\exboxt{
\parbox{\tw-72.7mm}{
\begin{example}[even/odd die]
\label{ex:prob_dice}
%---------------------------------------
Let $0<p<1$ be the probability of a die toss being odd.
Then 
\\\indentx$\psp(\emptyset)\!=\!0$, $\psp(\setA)\!=\!p$, $\psp(\setB)\!=\!1-p$, and $\psp(\setO)\!=\!1$ .\\
The order structure of this \structe{probability space} is illustrated to the right.
\end{example}
}%
\tbox{\includegraphics{graphics/lat4_m2_prob_die.pdf}}
}
\begin{proof}
\begin{align*}
  \psp(\setO)
    &= 1
    && \text{by \prope{normalized} property of $\psp$ \xref{def:psp}}
  \\
  \psp(\setC)
    &= \psp\setn{\coinhead\coinhead}
    && \text{by definition of $\setC$}
  \\&= \psp(\coinhead)\psp(\coinhead)
    && \text{by definition of \prope{independence} \xref{def:indepen}}
  \\&= p^2
    && \text{by definition of $p$}
  \\
  \psp(\setA)
    &= \psp\setn{\dieA,\dieC,\dieE}
    && \text{by definition of $\setA$}
  \\&= p
    && \text{by definition of $p$}
  \\
  \psp(\setB)
    &= \psp\setn{\dieB,\dieD,\dieF}
    && \text{by definition of $\setB$}
  \\&= \psp\setn{\dieA,\dieC,\dieE}^\setopc
    && \text{by definition of set complement $\setopc$}
  \\&= \psp\setA^\setopc
    && \text{by definition of $\setA$}
  \\&= \psp(\negat\setA)
    && \text{by definition of $\negat$}
  \\&= 1-\psp(\setA)
    && \text{by \prefp{prop:ps_negat}}
  \\&= 1-p
    && \text{by definition of $p$}
  \\
  \psp(\emptyset)
    &= 0
    && \text{by \prope{nondegenerate} property of $\psp$ \xref{def:psp}}
\end{align*}
\end{proof}

%---------------------------------------
\begin{figure}% Single die
%---------------------------------------
  \centering
  \exbox{
    \tbox{\includegraphics{graphics/lat8_l2e3_prob_die.pdf}}
    \begin{array}{rclrclrcl}
      \psp(\emptyset) &=&0        & \psp(\setO)          &=& 1               \\
      \psp(\setA)     &=&(1-p)^2  & \psp(\setA\setu\setB)&=&(1-p)^2+2p(1-p)  \\
      \psp(\setB)     &=&2p(1-p)  & \psp(\setA\setu\setC)&=& 2p^2-2p+1       \\
      \psp(\setC)     &=&p^2      & \psp(\setB\setu\setC)&=&-p^2+2p 
    \end{array}
    }
  \caption{\structe{Die toss} \xref{ex:prob_dice} \label{fig:prob_dice}}
\end{figure}
%---------------------------------------
\begin{example}
\label{ex:prob_dice}
%---------------------------------------
Let $p\eqd\frac{1}{6}$ be the probability of any face of a fair die.
The \structe{probability space} $\ps$ is illustrated in \prefpp{fig:prob_dice}.
\end{example}
\begin{proof}
The proof is in essence the same as for \prefpp{ex:ps_doublecoin}.
\end{proof}

%%---------------------------------------
%\begin{example}
%\label{ex:prob_dice}
%%---------------------------------------
%Suppose we have a ``fair" dice and we are primarily interested in the
%events of the first four
%$\left(\setn{\text{\diceA,\diceB,\diceC,\diceD}}\right)$
%(that is, whether one roll of the dice will produce
%a value in the set $\{1,2,3,4\}$)
%and the last three
%$\left(\setn{\text{\diceD,\diceE,\diceF}}\right)$
%However, these events do not by themselves form a $\sigma$-algebra.
%Rather under the $\cap$ and $\cup$ operations, these two events generate
%a total of eight possible events that together form a $\sigma$-algebra.
%The resulting measure space $\ps$ is
%\exbox{\begin{array}{ll}
%  \pso &= \left\{ \text{\diceA,\diceB,\diceC,\diceD,\diceE,\diceF} \right\}
%  \\
%  \pse &= \left\{ \mcom{\setn{\quad}}{$\emptyset$},\;
%                  \mcom{\setn{\text{\diceA,\diceB,\diceC,\diceD,\diceE,\diceF}}}{$\pso$},\;
%                  \mcom{\setn{\text{\diceA,\diceB,\diceC,\diceD}}}{first four},\;
%                  \mcom{\setn{\text{\diceD,\diceE,\diceF}}}{last three},\;
%                  \right.
%                  \\&\qquad
%                  \left.
%                  \mcom{\setn{\text{\diceD}}}{$\setn{1234}\cap\setn{456}$},\;
%                  \mcom{\setn{\text{\diceA,\diceB,\diceC,\diceE,\diceF}}}{$\setn{4}^c$},\;
%                  \mcom{\setn{\text{\diceE,\diceF}}}{$\setn{4}^c\cap\setn{456}$},\;
%                  \mcom{\setn{\text{\diceA,\diceB,\diceC}}}{$\setn{1234}\cap\setn{4}^c$},\;
%          \right\}
%  \\
%  \psp(e) &=
%    \left\{\begin{array}{l@{\qquad} >{\text{for }e=}l @{\qquad}D}
%      0           & \setn{\quad}
%                  & ($\emptyset$)\\
%      1           & \setn{\text{\diceA,\diceB,\diceC,\diceD,\diceE,\diceF}}
%                  & ($\pso$)\\
%      \frac{2}{3} & \setn{\text{\diceA,\diceB,\diceC,\diceD}}
%                  & (first four)\\
%      \frac{1}{2} & \setn{\text{\diceD,\diceE,\diceF}}
%                  & (last three)\\
%      \frac{1}{6} & \setn{\text{\diceD}}
%                  & ($\setn{1234}\cap\setn{456}$) \\
%      \frac{5}{6} & \setn{\text{\diceA,\diceB,\diceC,\diceE,\diceF}}
%                  & ($\setn{4}^c$) \\
%      \frac{1}{3} & \setn{\text{\diceE,\diceF}}
%                  & ($\setn{4}^c\cap\setn{456}$) \\
%      \frac{1}{2} & \setn{\text{\diceA,\diceB,\diceC}}
%                  & ($\setn{1234}\cap\setn{4}^c$)
%    \end{array}\right.
%\end{array}
%\tbox{\includegraphics{graphics/lat8_l2e3_prob_die.pdf}}}
%\end{example}

%Why go through all the trouble of requiring a $\sigma$-algebra?
%Having a $\sigma$-algebra in place ensures that anything we might possibly
%want to measure {\em can} be measured.
%It makes sure all possible combinations are taken into account.
%And why go through the additional trouble of requiring a measure space?
%With a measure space available, expressing the measure over a complex
%set is often greatly simplified because the measure space provides nice
%algebraic properties (namely the $\sigma$-additive property.
%\pref{ex:prob_123456} (next) illustrates how a rather complex
%$\sigma$-algebra (64 elements) can be compactly represented in a measure space.
%%---------------------------------------
%\begin{example}
%\label{ex:prob_123456}
%%---------------------------------------
%Suppose we have a ``fair" dice and we are interested in measuring over the
%power set of events (largest possible algebra---$2^6=64$ events).
%This leads to the measure space $\ps$ where
%\exbox{\begin{array}{ll@{\qquad}D}
%  \pso    &= \setn{ \text{\diceA,\diceB,\diceC,\diceD,\diceE,\diceF} }
%          \\
%  \pse    &= \mathcal{P}(\pso) & (the power-set of $\pso$)
%          \\
%  \psp(e) &= \frac{1}{6} |e|
%          & ($\frac{1}{6}$ times the number of possible outcomes in event $e$)
%\end{array}}
%\end{example}

%---------------------------------------
\begin{example}[Gaussian distribution on $\R$]
%---------------------------------------
Let $\ssB$ be the \structe{Borel algebra on $\R$}.
Let $\latL\eqd\opair{\ssB}{\subseteq}$ be the lattice formed by the elements of $\ssB$---this
lattice is a \structe{Boolean algebra}.
Let 
\\\indentx$\ds\psp(\setA)\eqd\frac{1}{\sqrt{2\pi\sigma^2}}\int_\setA e^{\frac{x^2}{2\sigma^2}}\dx$ for $\setA\subseteq\ssB$\\
and where $\int$ is the \ope{Lebesgue integral} \xref{def:intL}.
Then $\opair{\latL}{\psp}$ is a \structd{probability space}. 
\end{example}

%%---------------------------------------
%\begin{example}[Gaussian noise]
%%---------------------------------------
%Let $X\sim\pN{0}{\sigma^2}$ be a random variable with Gaussian distribution.
%We can construct the following probability space $\ps$:
%
%\exbox{\begin{array}{rcl}
%  \pso &=& \R \\
%  \pse &=& \setn{ \emptyset, \pso } \setu \set{(a,b)}{a,b\in\R,a<b} \\
%  \psp x &=& \left\{\begin{array}{ll}
%    0      & \mbox{ for } x=\emptyset \\
%    1      & \mbox{ for } x=\pso      \\
%    \frac{1}{\sqrt{2\pi\sigma^2}}\int_a^b e^{\frac{x^2}{2\sigma^2}}\dx & \mbox{ otherwise}
%    \end{array}\right.
%\end{array}}
%\end{example}

%%---------------------------------------
%\begin{example}
%\label{ex:prob_1011}
%%---------------------------------------
%The set of outcomes $\pso$ can also be a set of waveforms:
%\exbox{\begin{array}{ll}
%  \pso    &= \setn{\begin{tabular}{llll}
%               \textifsym{|H|LL|HHH|L} &
%               \textifsym{L|H|LL|HHH}  &
%               \textifsym{H|L|H|LL|HH} &
%               \textifsym{HH|L|H|LL|H} \\
%               \textifsym{HHH|L|H|LL}  &
%               \textifsym{L|HHH|L|H|L} &
%               \textifsym{LL|HHH|L|H}  &
%             \end{tabular}}
%          \\
%  \pse    &= \mathcal{P}(\pso)
%          \\
%  \psp(e) &= \frac{1}{7} |e|
%\end{array}}
%\end{example}



