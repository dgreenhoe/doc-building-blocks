%============================================================================
% LaTeX File
% Daniel J. Greenhoe
%============================================================================





%======================================
\chapter{Covenants}
%======================================
\qboxnpq
  {\href{http://en.wikipedia.org/wiki/Giuseppe_Peano}{Giuseppe Peano}
   \href{http://www-history.mcs.st-andrews.ac.uk/Timelines/TimelineF.html}{(1858--1932)},
   \href{http://www-history.mcs.st-andrews.ac.uk/BirthplaceMaps/Places/Italy.html}{Italian} mathematician,
   credited with being one of the first to introduce the concept of the \hie{linear space} (\hie{vector space}).}
  {../common/people/peano.jpg}
  {The geometric calculus, in general, 
   consists in a system of operations on geometric entities, and their consequences, 
   analogous to those that algebra has on the numbers.
   It permits the expression in formulas of the results of geometric constructions,
   the representation with equations of propositions of geometry,
   and the substitution of a transformation of equations for a verbal argument.}
  \citetblt{
    quote: & \citerp{peano1888e}{ix} \\
    image  & \url{http://en.wikipedia.org/wiki/Giuseppe_Peano}
    }
%\qboxnpqt
%  {
%    Jules Henri Poincar\'e (1854-1912), physicist and mathematician
%    \index{Poincar\'e, Jules Henri}
%    \index{quotes!Poincar\'e, Jules Henri}
%    \footnotemark
%  }
%  {../common/people/poincare.jpg}
%  {
%    \ldots et que nous nommons le temps et l'espace. \ldots
%    ce n'est pas la nature qui nous les impose,
%    c'est nous qui les imposons \`a la nature parce que nous les trouvons
%    commodes, \ldots
%  }
%  {
%    \ldots and which are called time and space. \ldots
%    it is not nature which imposes them upon us,
%    it is we who impose them upon nature because
%    we find them convenient.
%  }
%\citetblt{
%  quote:       & \citorc{poincare_vos}{Introduction, paragraph 10}  \\
%  translation: & \citorp{poincare_vos_e}{13}  \\
%  image:       & \url{http://en.wikipedia.org/wiki/Image:Poincare_jh.jpg}
%  }


%\qboxnpq
%  {
%    Karl Friedrich Gauss, mathematician (1777--1855)
%    in an November 8, 1824 letter to Franz Adolph Taurinus
%    \index{Gauss, Karl Friedrich}
%    \footnotemark
%  }
%  {../common/people/gauss.jpg}
%  {But it seems to me that in spite of the word-mastery of the metaphysicians,
%   we know really too little, or even nothing at all,
%   about the true nature of space to be able to confuse something that seems
%   unnatural with absolutely impossible.
%   If non-Euclidean geometry is the real one and the constant is incomparable
%   to the magnitudes that we encounter on earth or in the heavens then it can be
%   determined aposteriori.
%   I have therefore occasionally for fun expressed the wish that Euclidean geometry
%   not be the real one, for then we would have a priori an absolute measure.}
%  \footnotetext{\begin{tabular}[t]{ll}
%    quote: & \url{http://www.ltn.lv/~podnieks/} \\
%           & Burris(2003), \url{http://www.math.uwaterloo.ca/~snburris/htdocs/noneucl.pdf}, page 12 \\
%           & \url{http://resolver.sub.uni-goettingen.de/purl?PPN236010751} \\
%    image: & \url{http://en.wikipedia.org/wiki/Karl_Friedrich_Gauss}
%  \end{tabular}}






%\begin{figure}
%\scriptsize
%$\begin{array}{*{5}{>{\ds}c}}
%     && \fcolorbox{blue}{bg_blue}{\parbox[c]{3\tw/16}{\centering topological space}}
%  \\ && \Uparrow & \Nwarrow
%  \\ \fcolorbox{red}{bg_red}{\parbox[c]{3\tw/16}{\centering linear space}}
%     && \fcolorbox{blue}{bg_blue}{\parbox[c]{3\tw/16}{\centering metric space}}
%     && \fcolorbox{blue}{bg_blue}{\parbox[c]{3\tw/16}{\centering measure space}}
%  \\ & \Nwarrow & \Uparrow && \Uparrow
%  \\ && \fcolorbox{blue}{bg_blue}{\parbox[c]{3\tw/16}{\centering normed linear space}}
%     && \fcolorbox{blue}{bg_blue}{\parbox[c]{3\tw/16}{\centering probability space}}
%  \\ &  \Nearrow  & & \Nwarrow
%  \\ \fcolorbox{blue}{bg_blue}{\parbox[c]{3\tw/16}{\centering inner-product space}}
%     &&&&
%     \fcolorbox{blue}{bg_blue}{\parbox[c]{3\tw/16}{\centering Banach space}}
%  \\ & \Nwarrow && \Nearrow
%  \\ && \fcolorbox{blue}{bg_blue}{\parbox[c]{3\tw/16}{\centering Hilbert space}}
%  \\ &\Nearrow & & \Nwarrow
%  \\ \fcolorbox{blue}{bg_blue}{\parbox[c]{3\tw/16}{\centering$\spII$}}
%     &&&&
%     \fcolorbox{blue}{bg_blue}{\parbox[c]{3\tw/16}{\centering$\spLL$}}
%  \\ & \Nwarrow && \Nearrow
%  \\ && \fcolorbox{blue}{bg_blue}{\parbox[c]{3\tw/16}{\centering$0$}}
%\end{array}$
%\end{figure}

%======================================
\section{General linear spaces}
%======================================
