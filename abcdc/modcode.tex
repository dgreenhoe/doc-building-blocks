%============================================================================
% LaTeX File
% Daniel Greenhoe
%============================================================================

%======================================
\chapter{Line Coding}
\label{chp:modulation_codes}
\index{modulation codes}
\index{line codes}
\index{baseband modulation}
%======================================
\begin{figure}[ht]
\begin{center}
\begin{fsK}
\setlength{\unitlength}{0.20mm}                  
\begin{picture}(800,100)(50,-50)  
  \thinlines                                      
  %\graphpaper[10](0,0)(700,100)                  

  \put(   50,   0 ){\vector  (   1,  0)   {50}           }
  \put( 100,  -50){\framebox( 100,100)    {}             }
  \put( 100,  -30){\makebox( 100,  60)[t] {modulation}   }
  \put( 100,  -30){\makebox( 100,  60)[c] {coder}        }
 %\put( 100,  -30){\makebox( 100,  60)[b] {(state machine)}       }

  \put( 200,    0 ){\vector  (   1,  0)   {50}           }
  \put( 250,  -50 ){\framebox( 100,100)   {}             }
  \put( 250,  -30 ){\makebox ( 100, 60)[t]{mapping}      }
  \put( 250,  -30 ){\makebox ( 100, 60)[c]{$0\to-1$}     }
  \put( 250,  -30 ){\makebox ( 100, 60)[b]{$1\to+1$}     }
  \put( 350,    0 ){\vector  (   1,  0)   {50}           }
  \put( 400,  -50 ){\framebox( 100,100)   {modulator}    }

  \put( 550,  -50 ){\dashbox{4}( 100,100)   {$+$}          }
  \put( 550,  -40 ){\makebox ( 100,100)[b]{channel}      }
  \put( 550,  -50 ){\makebox ( 100, 95)[t]{$n(t)$}       }
  \put( 600,   00 ){\circle{20}                          }
  \put( 500,    0 ){\vector  (   1,  0)   {90}           }
  \put( 610,    0 ){\vector  (   1,  0)   {90}           }
  \put( 600,   30 ){\vector  (   0, -1)   {20}           }

  \put(  50,   10 ){\makebox (  50, 50)[b]{$\su$}        }
  \put( 200,   10 ){\makebox (  50, 50)[b]{$\sy$}        }
  \put( 350,   10 ){\makebox (  50, 50)[b]{$\sa$}        }
  \put( 500,   10 ){\makebox (  50, 50)[b]{$s(t;\vu)$}       }
  \put( 650,   10 ){\makebox (  50, 50)[b]{$r(t;\vu)$}       }

  \put( 700,  -50 ){\framebox( 100,100)   {receiver}    }
  \put( 800,   10 ){\makebox( 50,0)[b]   {$\sue$}    }
  \put( 800,    0 ){\vector  (   1,  0)   {50}           }
\end{picture}                                   
\end{fsK}
\end{center}
\caption{
   Modulation coding system model
   \label{fig:modcode_model}
   }
\end{figure}


This chapter discusses {\em modulation coding}. 
Modulation codes are also called {\em line codes} or 
{\em data translation codes}. \citep{proakis}{579}
Modulation coding is a transform $T:\su\to\sy$
from an input sequence $\su$ to an encoded sequence $\sy$
(see Figure~\ref{fig:modcode_model}).
Modulation codes typically seek to accomplish two objectives:

\begin{tabular}{llp{10cm}}
   1.& time shaping: & 
     eliminate long strings of ones or zeros to improve synchronization
     or make media access more reliable.
\\
   2.& spectral shaping: &
     modify spectral characteristics such as reducing the DC component.
\end{tabular}

A particular modulation code may be specified using several 
methods including

\begin{tabular}{llp{10cm}}
   1.& state machine \\
   2.& transition matrix \\
   3.& algebraic equations.
\end{tabular}

%---------------------------------------
\section{Channel model}
%---------------------------------------
The modulation coding system model is illustrated in 
Figure~\ref{fig:modcode_model}.

The {\em modulation coding state machine} is a transform
$T:(u_n)\to(y_n)$.
Modulation coding can be modeled as a {\em state-space} with
input $u_n$, output $y_n$, state $\vx_n$ and state equations 
\footnote{\citerc{smithIII}{\url{\ldots/State\_Space\_Models.html}}}
\begin{eqnarray*}
   \vx_{n+1} &=& f_1(\vx_n, u_n) \\
   y_n         &=& f_2(\vx_n, u_n).
\end{eqnarray*}

Other quantities appearing in Figure~\ref{fig:modcode_model}
can be expressed as
\[
\begin{array}{lrcl}
   \mbox{mapping output:}   & a_n &=& 2y_n-1  \\
   \mbox{channel signal:}   & s(t)&=& \sum_n a_n \lambda(t-nT) \\
   \mbox{receive signal:}   & r(t)&=& s(t) + n(t).
\end{array}
\]
The signaling waveform $\lambda(t)$ can be any of a number of waveforms.
A common choice is the simple pulse function illustrated in 
Figure~\ref{fig:rpulse}.
But this assumes the channel supports an infinitely wide bandwidth signal.
Bandlimited choices of signaling waveforms are described in 
Chapter~\ref{chp:isi} (page~\pageref{chp:isi}).

\begin{figure}[ht]
\begin{center}
\begin{fsL}
\setlength{\unitlength}{0.1mm}
\begin{picture}(360,100)(-100,0)
  %\graphpaper[10](0,0)(300,200)                  
  \thicklines                                      
  \put(-100 ,   0 ){\line(1,0){300} }
  \thinlines
  \put(    0 ,   0 ){\line(0,1){100} }
  \put(    0 , 100 ){\line(1,0){100} }
  \put(  100 ,   0 ){\line(0,1){100} }
  \put(  -10 , -40 ){$0$ }
  \put(   90 , -40 ){$T$ }
  \put(  -60 ,  90 ){$A$ }
  \put(  210 , -12 ){$t$ }
\end{picture}                                   
\end{fsL}
\end{center}
\caption{
   Pulse signaling waveform
   \label{fig:rpulse}
   }
\end{figure}



%=======================================
\section{Non-Return to Zero Modulation (NRZ)}
\label{sec:NRZ}
\index{Non-Return to Zero}
\index{NRZ}
%=======================================
%---------------------------------------
\subsection{Description}
%---------------------------------------
\begin{figure}[ht]
\begin{center}
\begin{fsL}
\setlength{\unitlength}{0.1mm}
\begin{picture}(1400,300)(-100,-130)
  %\graphpaper[10](0,0)(300,300)
  \thicklines
  \put        (-100,   0){\line( 1, 0){1400}} 

  \thinlines
  \put        (   0, 100){\line( 1, 0){ 100}}   
  \put        (   0,   0){\line( 0, 1){ 100}} 
  \put        ( 100,   0){\line( 0, 1){ 100}} 

  \put        ( 100,   0){\line( 0,-1){ 100}} 
  \put        ( 100,-100){\line( 1, 0){ 200}}   
  \put        ( 300,-100){\line( 0, 1){ 200}} 
  \put        ( 300, 100){\line( 1, 0){ 100}}   
  \put        ( 400, 100){\line( 0,-1){ 200}} 
  \put        ( 400,-100){\line( 1, 0){ 100}} 
  \put        ( 500,-100){\line( 0, 1){ 200}} 
  \put        ( 500, 100){\line( 1, 0){ 300}} 
  \put        ( 800, 100){\line( 0,-1){ 200}} 
  \put        ( 800,-100){\line( 1, 0){ 300}} 
  \put        (1100,-100){\line( 0, 1){ 200}} 
  \put        (1100, 100){\line( 1, 0){ 100}} 
  \put        (1200, 100){\line( 0,-1){ 100}} 

  %\multiput(0,-10)(100,  0){13}{\line(0,1){20}} 
  \multiput(0,0)(100,  0){13}{\qbezier[30](  0,-130)(  0,  0)(  0, 160)}

  \put        (   0, 130 ){\makebox (100,0){1}     }
  \put        ( 100, 130 ){\makebox (100,0){0}     }
  \put        ( 200, 130 ){\makebox (100,0){0}     }
  \put        ( 300, 130 ){\makebox (100,0){1}     }
  \put        ( 400, 130 ){\makebox (100,0){0}     }
  \put        ( 500, 130 ){\makebox (100,0){1}     }
  \put        ( 600, 130 ){\makebox (100,0){1}     }
  \put        ( 700, 130 ){\makebox (100,0){1}     }
  \put        ( 800, 130 ){\makebox (100,0){0}     }
  \put        ( 900, 130 ){\makebox (100,0){0}     }
  \put        (1000, 130 ){\makebox (100,0){0}     }
  \put        (1100, 130 ){\makebox (100,0){1}     }

  \put        ( -10, 100 ){\makebox (  0, 0)[r]{$+A$}     }
  \put        ( -10,-100 ){\makebox (  0, 0)[r]{$-A$}     }
  \put        ( -100, 130 ){\makebox (  0, 0)[r]{$u_n:$}     }
  \put        ( -100,  50 ){\makebox (  0, 0)[r]{$s(t):$}     }

\end{picture}                                   
\end{fsL}
\caption{
  NRZ modulated waveform
   \label{fig:NRZ_wave}
   }
\end{center}
\end{figure}

The non-return to zero (NRZ) waveform is illustrated in Figure~\ref{fig:NRZ_wave}.

%---------------------------------------
\subsection{Statistics}
%---------------------------------------
Note that even if the data sequence $u_n$ is an IID and WSS 
\footnote{IID: independently and identically distributed. 
WSS: wide sense stationary}
sequence, 
the channel signal $s(t)$ is {\bf not} WSS.
Specifically, the autocorrelation 
$\Rss(t+\tau,t)$ of $s(t)$ is not just a function of the time difference $\tau$,
but also a function of time $t$.
This is due to the fact that within a bit period, if one point is known
then all the points in that bit period are known.
Thus the points in a single bit period are 
certainly not independent and their autocorrelation is a function of time.

However, it is still possible to compute the time average of the 
autocorrelation and the Fourier transform of this average
(similar to the spectral density).
This is described in Theorem~\ref{thm:NRZ_stat} and 
illustrated in Figure~\ref{fig:NRZ_stat}.


\begin{center}
\begin{fsL}
\setlength{\unitlength}{0.075mm}
\begin{figure}[ht]

\begin{tabular}{ccc}


\begin{picture}(800,300)(-400,-150)
  %\graphpaper[10](0,0)(300,300)
  \thicklines
  \put        (-400,   0){\line( 1, 0){ 750}} 
  \put        (-150,-150){\line( 1, 1){ 300}}

  \thinlines
  \multiput(-400,0)(100,0){6}{
  \put        (   0,-100){\line( 0, 1){  50}}
  \put        (   0,- 50){\line( 1, 1){ 174}}
  \put        (   0,-100){\line( 2, 1){ 200}}
  \put        (   0,- 50){\line( 2, 1){ 200}}
  \put        ( 200,   0){\line( 0, 1){  50}}
  %\put        ( 150,  50){\line( 0, 1){  50}}
  \put        ( 174, 125){\line( 1,-3){  25}}
}
  \put        ( 310, 50 ){\makebox (  0, 0)[l]{$A^2$}     }
  \put        ( 360,  0 ){\makebox (  0, 0)[l]{$t$}     }
  \put        (-150,-160 ){\makebox (  0, 0)[t]{$\tau$}     }
  \qbezier[20]( 100,-130)( 100,-70)( 100,  0)
  \put        ( 100,-130 ){\makebox (  0, 0)[t]{$T$}    } 
\end{picture}
                                   
&

\begin{picture}(350,300)(-150,-150)
  %\graphpaper[10](0,0)(300,300)
  \thinlines

  \put        (-150,   0){\line( 1, 0){ 300}} 
  \put        (   0, -50){\line( 0, 1){ 180}}
  \put        (-100,   0){\line( 1, 1){ 100}}
  \put        ( 100,   0){\line(-1, 1){ 100}}

  \put        ( -20, 100 ){\makebox (  0, 0)[r]{$A^2$}     }
  \put        (-100, -10 ){\makebox (0, 0)[t]{$-T$}     }
  \put        ( 100, -10 ){\makebox (0, 0)[t]{$+T$}     }
  \put        ( 160,  0 ){\makebox (  0, 0)[l]{$\tau$}     }
\end{picture}

&

\begin{picture}(750,300)(-350,-150)
  %\graphpaper[10](0,0)(300,300)
  \thinlines

  \put        (-350,   0){\line( 1, 0){ 700}} 
  \put        (   0, -50){\line( 0, 1){ 180}}
  \qbezier    (-100,   0)(   0, 200)( 100,   0)
  \qbezier    ( 100,   0)( 150,-100)( 200,   0)
  \qbezier    ( 200,   0)( 250,  70)( 300,   0)
  \qbezier    (-100,   0)(-150,-100)(-200,   0)
  \qbezier    (-200,   0)(-250,  70)(-300,   0)

  \put        ( -20, 100 ){\makebox (  0, 0)[r]{$A^2$}     }

  \multiput(-300,0)(100,  0){7}{\qbezier[20](  0,-50)(  0,-25)(  0, 0)}
  \put        (-300, -60 ){\makebox (0, 0)[t]{$-\frac{3}{T}$}     }
  \put        (-200, -60 ){\makebox (0, 0)[t]{$-\frac{2}{T}$}     }
  \put        (-100, -60 ){\makebox (0, 0)[t]{$-\frac{1}{T}$}     }
  \put        (   0, -60 ){\makebox (0, 0)[t]{$0$}     }
  \put        ( 100, -60 ){\makebox (0, 0)[t]{$ \frac{1}{T}$}     }
  \put        ( 200, -60 ){\makebox (0, 0)[t]{$ \frac{2}{T}$}     }
  \put        ( 300, -60 ){\makebox (0, 0)[t]{$ \frac{3}{T}$}     }
  \put        ( 360,  0 ){\makebox (  0, 0)[l]{$f$}     }

\end{picture}
                             
\\
$\Rss(t+\tau,t)$ & $\avgRss(\tau)$  & $\avgSss(f)$
\end{tabular}
\caption{
  Statistics of NRZ modulated waveform
  \label{fig:NRZ_stat}
}
\end{figure}
\end{fsL}
\end{center}



%\begin{figure}[ht]
%\begin{center}
%\input{"sinc_sqt.lpl"}
%\end{center}
%\caption{
%  $A^2 T \frac{\sin(\pi fT)}{\pi fT}$
%  \label{fig:sinc_sqt}
%}
%\end{figure}

%--------------------------------------
\begin{theorem}
\label{thm:NRZ_stat}
%--------------------------------------
Let
\begin{liste}
   \item $u_n:\Z\to\{0,1\}$ be an IID WSS random process with probabilities 
         \[ \psp\{u_n=0\}=\psp\{u_n=1\}= \frac{1}{2} \hspace{1cm}\mbox{for all $n$}\]
   \item $s(t)$ be the waveform NRZ modulated by $u_n$
   \item $\Rss(t+\tau,t)$ be the autocorrelation of $s(t)$ such that
         \[ \Rss(t+\tau,t) \eqd \Eb{s(t+\tau)s(t)}\]
   \item $\avgRss(\tau)$ be the time average of $\Rss(t+\tau,t)$.
         \[ \avgRss(\tau) \eqd \frac{1}{T}\int_0^T \Rss(t+\tau,t) \dt \]
   \item $\avgSss(f)$ be the Fourier transform of $\avgRss(\tau)$ such that
         \[ \avgSss(f) \eqd \int_\tau \avgRss(\tau) e^{-i2\pi f\tau} \dtau.\]
\end{liste}

Then 
\begin{eqnarray*}
   \Rss(t+\tau,t) 
     &=& \left\{
         \begin{array}{ll}
            A^2 &:\tau \le (t\mod[T])\le T \\
            0 & :\mbox{otherwise}
         \end{array}
         \right.
\\
   \avgRss(\tau) 
     &=& \left\{
         \begin{array}{ll}
            A^2\left( 1- \frac{|\tau|}{T}\right) &:|\tau|\le T \\
            0 & :|\tau|>T.
         \end{array}
         \right.
\\
   \avgSxx(f) 
     &=& A^2 \left[ \frac{\sin\left(\pi fT\right)}{\pi fT}\right]^2.
\end{eqnarray*}
\end{theorem}


\begin{proof}
For time intervals $\tau \le (t\mod[T])\le T$, 
identical portions of $s(t+\tau)$ and $s(t)$
overlap and the resulting autocorrelation is
\begin{eqnarray*}
   \Rss(t+\tau,t) 
     &=& \Eb{s(t+\tau)s(t)} 
   \\&=& (-A)(-A)\pP{[s(t+\tau)=-A]\land [s(t)=-A]} + 
         (-A)(+A)\pP{[s(t+\tau)=-A]\land [s(t)=-A]} + \\&&
         (+A)(-A)\pP{[s(t+\tau)=-A]\land [s(t)=-A]} + 
         (+A)(+A)\pP{[s(t+\tau)=-A]\land [s(t)=-A]} 
   \\&=& (-A)(-A)\frac{1}{2} + 
         (-A)(+A)\cdot 0 +
         (+A)(-A)\cdot 0 +
         (+A)(+A)\frac{1}{2} 
   \\&=& A^2
\end{eqnarray*}

For all other time intervals, especially $|\tau|>T$,
$s(t+\tau)$ and $s(t)$ are statistically independent and hence
 \[ \Rss(\tau) = \Eb{s(t+\tau)s(t)} = \Eb{s(t+\tau)}\Eb{s(t)} = 0\cdot0 = 0.\]

Alternatively,
\begin{eqnarray*}
   \Rss(t+\tau,t) 
     &=& \Eb{s(t+\tau)s(t)} 
   \\&=& (-A)(-A)\pP{[s(t+\tau)=-A]\land [s(t)=-A]} + 
         (-A)(+A)\pP{[s(t+\tau)=-A]\land [s(t)=-A]} + \\&&
         (+A)(-A)\pP{[s(t+\tau)=-A]\land [s(t)=-A]} + 
         (+A)(+A)\pP{[s(t+\tau)=-A]\land [s(t)=-A]} 
   \\&=& (-A)(-A)\frac{1}{4} + 
         (-A)(+A)\frac{1}{4} +
         (+A)(-A)\frac{1}{4} +
         (+A)(+A)\frac{1}{4} 
   \\&=& A^2 -A^2 -A^2 + A^2
   \\&=& 0.
\end{eqnarray*}

%For the transform needed to compute $\avgSxx(f)$, 
%see Theorem~\ref{thm:ft_pairs} (page~\pageref{thm:ft_pairs}).
\end{proof}


%---------------------------------------
\subsection{Detection}
%---------------------------------------
\begin{figure}[ht]
\begin{center}
\begin{fsL}
\setlength{\unitlength}{0.1mm}
\begin{tabular}{ccc}
\begin{picture}(360,300)(-100,-150)
  %\graphpaper[10](0,0)(300,200)                  
  \thicklines                                      
  \put(-100 ,   0 ){\line(1,0){300} }
  \thinlines
  \put(    0 ,   0 ){\line(0,1){100} }
  \put(    0 , 100 ){\line(1,0){100} }
  \put(  100 ,   0 ){\line(0,1){100} }
  \put(  -10 , -40 ){$0$ }
  \put(   90 , -40 ){$T$ }
  \put(  -60 ,  90 ){$\frac{1}{\sqrt{T}}$ }
  \put(  210 , -12 ){$t$ }
\end{picture}                                   
&
\begin{picture}(360,300)(-100,-150)
  %\graphpaper[10](0,0)(300,200)                  
  \thicklines                                      
  \put(-100 ,   0 ){\line(1,0){300} }
  \thinlines
  \put(    0 ,   0 ){\line(0,-1){100} }
  \put(    0 ,-100 ){\line(1, 0){100} }
  \put(  100 ,   0 ){\line(0,-1){100} }
  \put(  -10 ,  20 ){$0$ }
  \put(   90 ,  20 ){$T$ }
  \put(  -60 ,-100 ){$-A$ }
  \put(  210 , -12 ){$t$ }
\end{picture}                                   
&
\begin{picture}(360,300)(-100,-150)
  %\graphpaper[10](0,0)(300,200)                  
  \thicklines                                      
  \put(-100 ,   0 ){\line(1,0){300} }
  \thinlines
  \put(    0 ,   0 ){\line(0,1){100} }
  \put(    0 , 100 ){\line(1,0){100} }
  \put(  100 ,   0 ){\line(0,1){100} }
  \put(  -10 , -40 ){$0$ }
  \put(   90 , -40 ){$T$ }
  \put(  -30 ,  90 ){$A$ }
  \put(  210 , -12 ){$t$ }
\end{picture}
\\
$\psi(t)$ & $s(t;a=-1)=-A\psi(t)$ & $s(t;a=+1)=+A\psi(t)$
\end{tabular}
\end{fsL}
\end{center}
\caption{
   NRZ critical functions
   \label{fig:NRZ_psi}
   }
\end{figure}



%---------------------------------------
\begin{proposition}
\label{prop:NRZ_basis}
%---------------------------------------
The function
\begin{eqnarray*}
   \psi(t) 
     &=& \left\{
         \begin{tabular}{ll}
            $\frac{1}{\sqrt{T}}$ & for $0\le t < T$ \\
            $0$                  & otherwise.
         \end{tabular}
         \right.
\end{eqnarray*}
forms an {\bf orthonormal basis} for the NRZ signaling waveforms such that

\begin{eqnarray*}
   s(t;a=-1) &=& -A \psi(t)  \\
   s(t;a=+1) &=& +A \psi(t) .
\end{eqnarray*}

\end{proposition}

\begin{proof}
\begin{eqnarray*}
   \inprod{\psi(t)}{\psi(t)}
     &=& \inprod{\frac{1}{\sqrt{T}}} {\frac{1}{\sqrt{T}}}
   \\&=& \frac{1}{\sqrt{T}} \frac{1}{\sqrt{T}} \inprod{1}{1}
   \\&=& \frac{1}{T} \int_0^T 1\cdot1 \; dt
   \\&=& \frac{1}{T} t|_0^T
   \\&=& \frac{1}{T} (T-0)
    \\&=& 1
\end{eqnarray*}
\end{proof}



\begin{figure}[ht]
\begin{center}
\begin{fsL}
\setlength{\unitlength}{0.2mm}
\begin{picture}(300,200)(-100,-50)
  %\graphpaper[10](-100,0)(300,150)
  \thinlines                                      
  \put(-100 ,   0 ){\line(1,0){300} }

  \qbezier[20](   0,  0)(   0, 60)(   0,120)
  \qbezier[20](  50,-50)(  50, -5)(  50, 40)
  \qbezier[20]( 100,  0)( 100, 20)( 100,120)

  \put(  20 , 130 ){\makebox(0,0)[rb]{$(\dot{r}|a_n=-1)\sim\pN{-A}{\sigma^2}$}}
  \put(  80 , 130 ){\makebox(0,0)[lb]{$(\dot{r}|a_n=+1)\sim\pN{+A}{\sigma^2}$}}

  \put(   0 , -10 ){\makebox(0,0)[t]{$-A$}}
  \put(  50 , -10 ){\makebox(0,0)[t]{$0$}}
  \put( 100 , -10 ){\makebox(0,0)[t]{$+A$}}

  \put(  30 , -30 ){\makebox(0,0)[rt]{$D_0=(-\infty,0]$}}
  \put(  70 , -30 ){\makebox(0,0)[lt]{$D_1=(0,\infty)$}}

  \qbezier( -40,  60)(   0, 180)(  40,  60)
  \qbezier(-100,   0)( -60,   0)( -40,  60)
  \qbezier(  40,  60)(  60,   0)( 100,   0)

  \qbezier(  60,  60)( 100, 180)( 140,  60)
  \qbezier(   0,   0)(  40,   0)(  60,  60)
  \qbezier( 140,  60)( 160,   0)( 200,   0)
\end{picture}                                   
\end{fsL}
\end{center}
\caption{
  Decision statistics for NRZ modulation in AWGN channel
   \label{fig:NRZ_norm}
   }
\end{figure}


%---------------------------------------
\begin{proposition}
%---------------------------------------
Let
\begin{eqnarray*}
   \dot{r}(-1) &\eqd& \inprod{r(t)|s(t;a=-1)\mbox{ was transmitted }}{\psi(t)} \\
   \dot{r}(+1) &\eqd& \inprod{r(t)|s(t;a=+1)\mbox{ was transmitted }}{\psi(t)}.
\end{eqnarray*}

Then $\dot{r}(-1)$ and $\dot{r}(+1)$ are {\bf independent} random variables
with marginal distributions
\begin{eqnarray*}
  \dot{r}(-1) &\sim&  \pN{-A}{\sigma^2} \\
  \dot{r}(+1) &\sim&  \pN{+A}{\sigma^2}
\end{eqnarray*}
\end{proposition}

\begin{proof}
This follows directly from 
Theorem~\ref{thm:ms_stats} (page~\pageref{thm:ms_stats}).
\end{proof}


%---------------------------------------
\begin{proposition}
%---------------------------------------
The value
\[  \dot{r} \eqd \inprod{r(t)}{\psi(t)} \]
is a sufficient statistic for optimal ML detection of the transmitted 
symbol $a$.

The optimal estimate $\estML[a]$ of $a$ is
\[
   \hat{a} = \left\{
     \begin{array}{ll}
        -1 : \dot{r}\le 0 \\
        +1 : \dot{r}>   0.
     \end{array}
     \right.
\]
\end{proposition}

\begin{proof}
   This is a result of Theorem~\ref{thm:ml_est_det} (page~\pageref{thm:ml_est_det}).
\end{proof}

%---------------------------------------
\begin{proposition}
%---------------------------------------
The probability of detection error in an NRZ modulation system
   \[ \pP{\mbox{error}} = \Qb{\frac{a}{N_o}}.\]
\end{proposition}

\begin{proof}
\begin{eqnarray*}
   \pP{error}
     &=& \pP{s_0(t) \mbox{ sent } \land \dot{r}>0 } +
         \pP{s_1(t) \mbox{ sent } \land \dot{r}<0 } 
   \\&=& \pP{\dot{r}>0 | s_0(t) \mbox{ sent }}\pP{s_0(t) \mbox{ sent } } +
         \pP{\dot{r}<0 | s_1(t) \mbox{ sent }}\pP{s_1(t) \mbox{ sent } } 
   \\&=& 2\pP{\dot{r}>0 | s_0(t) \mbox{ sent }}\frac{1}{2}
   \\&=& \Qb{ \frac{\E\dot{r}}{\sqrt{\var{\dot{r}}}}}
   \\&=& \Qb{\frac{a}{N_o}}
\end{eqnarray*}
\end{proof}





%=======================================
\section{Return to Zero Modulation (RZ)}
\label{sec:RZ}
\index{Return to Zero}
\index{RZ}
%=======================================
\begin{figure}[ht]
\begin{center}
\begin{fsK}
\setlength{\unitlength}{0.2mm}                  
\begin{picture}(450,200)(-200,-150)  
  \thinlines                                      
  %\graphpaper[10](0,0)(700,100)                  

  \put(-200,  10 ){\makebox (  50, 50)[b]{$u_n$}        }
  \put(-200,   0 ){\vector  (   1,  0)   {50}           }
  \put(-150, -50 ){\framebox( 100,100)   {$\uparrow2$}  }
  \put(-150, -40 ){\makebox ( 100, 80)[b]{upsampler}    }
  \put(- 50,   0 ){\vector  (   1,  0)   {50}           }
  \put( 200,  10 ){\makebox (  50, 50)[b]{$y_n$}        }

  \put( 200,   0 ){\vector  (   1,  0)   {50}         }
  \put(   0,-150 ){\dashbox ( 200,200)   {  }           }
  \put(   0,-140 ){\makebox ( 200,200)[b]{RZ coder}         }
  \put(   0,-140 ){\makebox ( 200,200)[ ]{
        \begin{fsK}
        \begin{math}
        \begin{array}{c|cc}
           u_{2n}' & y_n & y_{n+1} \\
           \hline
           0 &   0 & 0   \\
           1 &   1 & 0  
        \end{array}
        \end{math}
        \end{fsK}
     }}

  \put(- 50,   10 ){\makebox (  50, 50)[b]{$u_n'$}        }
\end{picture}                                   
\end{fsK}
\end{center}
\caption{
   RZ modulation coder
   \label{fig:RZ_model}
   }
\end{figure}


The non-return to zero (RZ) modulation coder is illustrated in Figure~\ref{fig:RZ_model}.
An example RZ modulated waveform is illustrated in Figure~\ref{fig:RZ_wave}.
An RZ modulated waveform $s(t)$ can be decomposed into 
a deterministic periodic waveform $d(t)$ and 
a random waveform $r(t)$ such that $s(t)=d(t)+r(t)$
(see \prefp{fig:RZ_decomp}).\footnote{\citer{kao}}


\begin{figure}[ht]
\begin{center}
\begin{fsL}
\setlength{\unitlength}{0.1mm}
\begin{picture}(1400,300)(-100,-130)
  %\graphpaper[10](0,0)(300,300)
  \thicklines
  \put        (-100,   0){\line( 1, 0){1400}} 

  \thinlines
  \put        (   0, 100){\line( 1, 0){  50}}   
  \put        (   0,   0){\line( 0, 1){ 100}} 

  \put        (  50, 100){\line( 0,-1){ 200}} 
  \put        (  50,-100){\line( 1, 0){ 250}}   
  \put        ( 300,-100){\line( 0, 1){ 200}} 
  \put        ( 300, 100){\line( 1, 0){  50}}   
  \put        ( 350, 100){\line( 0,-1){ 200}} 
  \put        ( 350,-100){\line( 1, 0){ 150}} 
  \put        ( 500,-100){\line( 0, 1){ 200}}
 
  \put        ( 500, 100){\line( 1, 0){  50}} 
  \put        ( 550, 100){\line( 0,-1){ 200}} 
  \put        ( 550,-100){\line( 1, 0){  50}} 
  \put        ( 600,-100){\line( 0, 1){ 200}} 

  \put        ( 600, 100){\line( 1, 0){  50}} 
  \put        ( 650, 100){\line( 0,-1){ 200}} 
  \put        ( 650,-100){\line( 1, 0){  50}} 
  \put        ( 700,-100){\line( 0, 1){ 200}} 

  \put        ( 700, 100){\line( 1, 0){  50}} 
  \put        ( 750, 100){\line( 0,-1){ 200}} 
  \put        ( 750,-100){\line( 1, 0){  50}} 

  \put        ( 800,-100){\line( 1, 0){ 300}} 
  \put        (1100,-100){\line( 0, 1){ 200}} 
  \put        (1100, 100){\line( 1, 0){  50}} 
  \put        (1150, 100){\line( 0,-1){ 100}} 
  \put        (1150,-100){\line( 0, 1){ 200}} 
  \put        (1150,-100){\line( 1, 0){  50}} 

  %\multiput(0,-10)(100,  0){13}{\line(0,1){20}} 
  \multiput(0,0)(100,  0){13}{\qbezier[30](  0,-130)(  0,  0)(  0, 160)}

  \put        (   0, 130 ){\makebox (100, 0){1}     }
  \put        ( 100, 130 ){\makebox (100, 0){0}     }
  \put        ( 200, 130 ){\makebox (100, 0){0}     }
  \put        ( 300, 130 ){\makebox (100, 0){1}     }
  \put        ( 400, 130 ){\makebox (100, 0){0}     }
  \put        ( 500, 130 ){\makebox (100, 0){1}     }
  \put        ( 600, 130 ){\makebox (100, 0){1}     }
  \put        ( 700, 130 ){\makebox (100, 0){1}     }
  \put        ( 800, 130 ){\makebox (100, 0){0}     }
  \put        ( 900, 130 ){\makebox (100, 0){0}     }
  \put        (1000, 130 ){\makebox (100, 0){0}     }
  \put        (1100, 130 ){\makebox (100, 0){1}     }

  \put        ( -10, 100 ){\makebox (  0, 0)[r]{+A}     }
  \put        ( -10,-100 ){\makebox (  0, 0)[r]{-A}     }
  \put        ( -100, 130 ){\makebox (  0, 0)[r]{$u_n:$}     }
  \put        ( -100,  50 ){\makebox (  0, 0)[r]{$s(t):$}     }

\end{picture}                                   
\end{fsL}
\caption{
  RZ waveform
   \label{fig:RZ_wave}
   }
\end{center}
\end{figure}








\begin{figure}[ht]
\begin{center}
\begin{fsL}
\setlength{\unitlength}{0.075mm}
\begin{tabular}{c}

\begin{picture}(1400,300)(-100,-130)
  %\graphpaper[10](0,0)(300,300)
  \thicklines
  \put        (-100,   0){\line( 1, 0){1400}} 

  \thinlines
  \put        (   0, 100){\line( 1, 0){  50}}   
  \put        (   0,   0){\line( 0, 1){ 100}} 

  \put        (  50, 100){\line( 0,-1){ 200}} 
  \put        (  50,-100){\line( 1, 0){ 250}}   
  \put        ( 300,-100){\line( 0, 1){ 200}} 
  \put        ( 300, 100){\line( 1, 0){  50}}   
  \put        ( 350, 100){\line( 0,-1){ 200}} 
  \put        ( 350,-100){\line( 1, 0){ 150}} 
  \put        ( 500,-100){\line( 0, 1){ 200}}
 
  \put        ( 500, 100){\line( 1, 0){  50}} 
  \put        ( 550, 100){\line( 0,-1){ 200}} 
  \put        ( 550,-100){\line( 1, 0){  50}} 
  \put        ( 600,-100){\line( 0, 1){ 200}} 

  \put        ( 600, 100){\line( 1, 0){  50}} 
  \put        ( 650, 100){\line( 0,-1){ 200}} 
  \put        ( 650,-100){\line( 1, 0){  50}} 
  \put        ( 700,-100){\line( 0, 1){ 200}} 

  \put        ( 700, 100){\line( 1, 0){  50}} 
  \put        ( 750, 100){\line( 0,-1){ 200}} 
  \put        ( 750,-100){\line( 1, 0){  50}} 

  \put        ( 800,-100){\line( 1, 0){ 300}} 
  \put        (1100,-100){\line( 0, 1){ 200}} 
  \put        (1100, 100){\line( 1, 0){  50}} 
  \put        (1150, 100){\line( 0,-1){ 100}} 
  \put        (1150,-100){\line( 0, 1){ 200}} 
  \put        (1150,-100){\line( 1, 0){  50}} 

  %\multiput(0,-10)(100,  0){13}{\line(0,1){20}} 
  \multiput(0,0)(100,  0){13}{\qbezier[30](  0,-130)(  0,  0)(  0, 160)}

  \put        (   0, 130 ){\makebox (100, 0){1}     }
  \put        ( 100, 130 ){\makebox (100, 0){0}     }
  \put        ( 200, 130 ){\makebox (100, 0){0}     }
  \put        ( 300, 130 ){\makebox (100, 0){1}     }
  \put        ( 400, 130 ){\makebox (100, 0){0}     }
  \put        ( 500, 130 ){\makebox (100, 0){1}     }
  \put        ( 600, 130 ){\makebox (100, 0){1}     }
  \put        ( 700, 130 ){\makebox (100, 0){1}     }
  \put        ( 800, 130 ){\makebox (100, 0){0}     }
  \put        ( 900, 130 ){\makebox (100, 0){0}     }
  \put        (1000, 130 ){\makebox (100, 0){0}     }
  \put        (1100, 130 ){\makebox (100, 0){1}     }

  \put        ( -10, 100 ){\makebox (  0, 0)[r]{+A}     }
  \put        ( -10,-100 ){\makebox (  0, 0)[r]{-A}     }
  \put        ( -100, 130 ){\makebox (  0, 0)[r]{$u_n:$}     }
  \put        ( -100,  50 ){\makebox (  0, 0)[r]{$s(t):$}     }

\end{picture}

\\ = \\

\begin{picture}(1400,300)(-100,-130)
  %\graphpaper[10](0,0)(300,300)
  \thicklines
  %\put        (-100,   0){\line( 1, 0){1400}} 

  \thinlines
  \multiput(0,0)(100,  0){12}{
     \put        (   0,-100){\line( 0, 1){ 100}}   
     \put        (   0,   0){\line( 1, 0){  50}}   
     \put        (  50,   0){\line( 0,-1){ 100}}   
     \put        (  50,-100){\line( 1, 0){  50}}   
}
  \multiput(0,0)(100,  0){13}{\qbezier[30](  0,-130)(  0,  0)(  0, 160)}


  \put        ( -10, 100 ){\makebox (  0, 0)[r]{+A}     }
  \put        ( -10,-100 ){\makebox (  0, 0)[r]{-A}     }
  \put        ( -100,  50 ){\makebox (  0, 0)[r]{$d(t):$}     }

\end{picture}

\\ + \\ 

\begin{picture}(1400,300)(-100,-130)
  %\graphpaper[10](0,0)(300,300)
  \thicklines
  %\put        (-100,   0){\line( 1, 0){1400}} 

  \thinlines
  \put        (   0,   0){\line( 0, 1){ 100}} % 1
  \put        (   0, 100){\line( 1, 0){  50}}   
  \put        (  50, 100){\line( 0,-1){ 100}} 
  \put        (  50,   0){\line( 1, 0){  50}}   

  \put        ( 100,   0){\line( 0,-1){ 100}} % 0
  \put        ( 100,-100){\line( 1, 0){  50}}   
  \put        ( 150,-100){\line( 0, 1){ 100}} 
  \put        ( 150,   0){\line( 1, 0){  50}}   

  \put        ( 200,   0){\line( 0,-1){ 100}} % 0
  \put        ( 200,-100){\line( 1, 0){  50}}   
  \put        ( 250,-100){\line( 0, 1){ 100}} 
  \put        ( 250,   0){\line( 1, 0){  50}}   

  \put        ( 300,   0){\line( 0, 1){ 100}} % 1
  \put        ( 300, 100){\line( 1, 0){  50}}   
  \put        ( 350, 100){\line( 0,-1){ 100}} 
  \put        ( 350,   0){\line( 1, 0){  50}}   

  \put        ( 400,   0){\line( 0,-1){ 100}} % 0
  \put        ( 400,-100){\line( 1, 0){  50}}   
  \put        ( 450,-100){\line( 0, 1){ 100}} 
  \put        ( 450,   0){\line( 1, 0){  50}}   

  \put        ( 500,   0){\line( 0, 1){ 100}} % 1
  \put        ( 500, 100){\line( 1, 0){  50}}   
  \put        ( 550, 100){\line( 0,-1){ 100}} 
  \put        ( 550,   0){\line( 1, 0){  50}}   

  \put        ( 600,   0){\line( 0, 1){ 100}} % 1
  \put        ( 600, 100){\line( 1, 0){  50}}   
  \put        ( 650, 100){\line( 0,-1){ 100}} 
  \put        ( 650,   0){\line( 1, 0){  50}}   

  \put        ( 700,   0){\line( 0, 1){ 100}} % 1
  \put        ( 700, 100){\line( 1, 0){  50}}   
  \put        ( 750, 100){\line( 0,-1){ 100}} 
  \put        ( 750,   0){\line( 1, 0){  50}}   

  \put        ( 800,   0){\line( 0,-1){ 100}} % 0
  \put        ( 800,-100){\line( 1, 0){  50}}   
  \put        ( 850,-100){\line( 0, 1){ 100}} 
  \put        ( 850,   0){\line( 1, 0){  50}}   

  \put        ( 900,   0){\line( 0,-1){ 100}} % 0
  \put        ( 900,-100){\line( 1, 0){  50}}   
  \put        ( 950,-100){\line( 0, 1){ 100}} 
  \put        ( 950,   0){\line( 1, 0){  50}}   

  \put        (1000,   0){\line( 0,-1){ 100}} % 0
  \put        (1000,-100){\line( 1, 0){  50}}   
  \put        (1050,-100){\line( 0, 1){ 100}} 
  \put        (1050,   0){\line( 1, 0){  50}}   

  \put        (1100,   0){\line( 0, 1){ 100}} % 1
  \put        (1100, 100){\line( 1, 0){  50}}   
  \put        (1150, 100){\line( 0,-1){ 100}} 
  \put        (1150,   0){\line( 1, 0){  50}}   


  \multiput(0,0)(100,  0){13}{\qbezier[30](  0,-130)(  0,  0)(  0, 160)}


  \put        ( -10, 100 ){\makebox (  0, 0)[r]{+A}     }
  \put        ( -10,-100 ){\makebox (  0, 0)[r]{-A}     }
  \put        ( -100,  50 ){\makebox (  0, 0)[r]{$r(t):$}     }

\end{picture}
\end{tabular}

\end{fsL}
\caption{
  Decomposition of RZ modulated waveform
   \label{fig:RZ_decomp}
   }
\end{center}
\end{figure}

%--------------------------------------
\begin{theorem}
\label{thm:RZ_stat}
%--------------------------------------
Let
\begin{liste}
   \item $u_n:\Z\to\{0,1\}$ be an IID WSS random process with probabilities 
         \[ \psp\{u_n=0\}=\psp\{u_n=1\}= \frac{1}{2} \hspace{1cm}\mbox{for all $n$}\]
   \item $s(t)$ be the waveform RZ modulated by $u_n$
   \item $d(t)$ be the deterministic periodic waveform illustrated in 
         Figure~\ref{fig:RZ_decomp}
   \item $\Rss(t+\tau,t)$ be the autocorrelation of $s(t)$ such that
         \[ \Rss(t+\tau,t) \eqd \Eb{s(t+\tau)s(t)}\]
   \item $\avgRss(\tau)$ be the time average of $\Rss(t+\tau,t)$.
         \[ \avgRss(\tau) \eqd \frac{1}{T}\int_0^T \Rss(t+\tau,t) \dt \]
   \item $\avgSss(f)$ be the Fourier transform of $\avgRss(\tau)$ such that
         \[ \avgSss(f) \eqd \int_\tau \avgRss(\tau) e^{-i2\pi f\tau} \dtau.\]
\end{liste}

Then 
\begin{eqnarray*}
   \Rss(t+\tau,t) 
     &=& 
         \left\{
         \begin{array}{ll}
            A^2 + d(t+\tau)d(t) & :\tau \le (t\;\mod[T])\le \frac{T}{2} \\
            d(t+\tau)d(t)       & :\mbox{otherwise}
         \end{array}
         \right.
\end{eqnarray*}

\begin{eqnarray*}
   \avgRss(\tau) 
     &=& \frac{A^2}{2}\left( 1- \frac{2|\tau|}{T}\right) \chi_{[-T/2,T/2]}(\tau) + 
         \frac{A^2}{2} \sum_n \left( 1 - \frac{2|\tau-nT|}{T} \right)\chi_{[-T/2,T/2]}(\tau-nT)
\end{eqnarray*}

\begin{eqnarray*}
   \avgSxx(f) 
     &=& \frac{A^2 T}{4} \left[ \frac{\sin\left(\pi f\frac{T}{2}\right)}{\pi f\frac{T}{2}}\right]^2 + 
         \frac{A^2 T}{4} \sum_k \left[ \frac{\sin\left(\pi k\frac{1}{2}\right)}{\pi k\frac{1}{2}}\right]^2 \delta\left(f-\frac{k}{T}\right) 
\end{eqnarray*}
\end{theorem}


\begin{proof}
\begin{eqnarray*}
   \Rss(t+\tau,t) 
     &=& \Eb{s(t+\tau)s(t)} 
   \\&=& \Eb{[d(t+\tau)r(t+\tau)][d(t)+r(t)]} 
   \\&=& \Eb{d(t+\tau)d(t) + d(t+\tau)r(t) + r(t+\tau)d(t)+ r(t+\tau)r(t)} 
   \\&=& d(t+\tau)d(t) + d(t+\tau)\Eb{r(t)} + d(t)\Eb{r(t+\tau)}+ \Eb{r(t+\tau)r(t)} 
   \\&=& \Rxx[rr](t+\tau,t) + d(t+\tau)d(t) + d(t+\tau)\cdot0 + d(t)\cdot0 
   \\&=& \Rxx[rr](t+\tau,t) + d(t+\tau)d(t) 
\end{eqnarray*}

For time intervals $\tau \le (t\mod[T])\le T/2$, 
identical portions of $r(t+\tau)$ and $r(t)$
overlap and the resulting autocorrelation is
\begin{eqnarray*}
   \Rxx[rr](t+\tau,t) 
     &=& (-A)(-A)\pP{[s(t+\tau)=-A]\land [s(t)=-A]} + 
         (-A)(+A)\pP{[s(t+\tau)=-A]\land [s(t)=-A]} + \\&&
         (+A)(-A)\pP{[s(t+\tau)=-A]\land [s(t)=-A]} + 
         (+A)(+A)\pP{[s(t+\tau)=-A]\land [s(t)=-A]} 
   \\&=& (-A)(-A)\frac{1}{2} + 
         (-A)(+A)\cdot 0 +
         (+A)(-A)\cdot 0 +
         (+A)(+A)\frac{1}{2} 
   \\&=& A^2
\end{eqnarray*}

For all other time intervals, especially $|\tau|>T$,
$r(t+\tau)$ and $r(t)$ are statistically independent and hence
 \[ \Rxx[rr](\tau) = \Eb{r(t+\tau)r(t)} = \Eb{r(t+\tau)}\Eb{r(t)} = 0\cdot0 = 0.\]

%For the transform needed to compute $\avgSss(f)$, 
%see Theorem~\ref{thm:ft_pairs} page~\pageref{thm:ft_pairs}.

To compute the time average $\avgRss(\tau)$,
we need to find the average of both $\Rxx[rr](t+\tau,t)$ and $d(t+\tau)d(t)$.
\begin{eqnarray*}
   \frac{1}{T}\int_0^T d(t+\tau)d(t) \dt
     &=& \frac{1}{T} \frac{A^2 T}{2} \sum_n \left( 1 - \frac{|\tau-nT|}{T/2} \right)\chi_{[-T/2,T/2]}(\tau-nT)
   \\&=& \frac{A^2}{2} \sum_n \left( 1 - \frac{2|\tau-nT|}{T} \right)\chi_{[-T/2,T/2]}(\tau-nT)
\end{eqnarray*}

\begin{eqnarray*}
   \frac{1}{T}\int_0^T  \Rxx[rr](t+\tau,t) \dt
     &=& \left\{
         \begin{array}{ll}
            \frac{A^2}{2}\left( 1- \frac{2|\tau|}{T}\right) &:|\tau|\le \frac{T}{2} \\
            0 & :|\tau|>\frac{T}{2}.
         \end{array}
         \right.
\end{eqnarray*}

\begin{eqnarray*}
   \avgRss(\tau) 
     &=& \frac{A^2}{2}\left( 1- \frac{2|\tau|}{T}\right) \chi_{[-T/2,T/2]}(\tau) + 
         \frac{A^2}{2} \sum_n \left( 1 - \frac{2|\tau-nT|}{T} \right)\chi_{[-T/2,T/2]}(\tau-nT)
\end{eqnarray*}

\begin{eqnarray*}
   \avgSxx(f)
     &=& \frac{A^2 T}{4} \left[ \frac{\sin\left(\pi f\frac{T}{2}\right)}{\pi f\frac{T}{2}}\right]^2 + 
         \frac{A^2 T}{4} \sum_k \left[ \frac{\sin\left(\pi \frac{k}{T}\frac{T}{2}\right)}{\pi \frac{k}{T}\frac{T}{2}}\right]^2 \delta\left(f-\frac{k}{T}\right) 
   \\&=& \frac{A^2 T}{4} \left[ \frac{\sin\left(\pi f\frac{T}{2}\right)}{\pi f\frac{T}{2}}\right]^2 + 
         \frac{A^2 T}{4} \sum_k \left[ \frac{\sin\left(\pi k\frac{1}{2}\right)}{\pi k\frac{1}{2}}\right]^2 \delta\left(f-\frac{k}{T}\right)
\end{eqnarray*}

\end{proof}


%=======================================
\section{Manchester Modulation}
\label{sec:manchester}
\index{Manchester Modulation}
%=======================================
\begin{figure}[ht]
\begin{center}
\begin{fsK}
\setlength{\unitlength}{0.2mm}                  
\begin{picture}(450,200)(-200,-150)  
  \thinlines                                      
  %\graphpaper[10](0,0)(700,100)                  

  \put(-200,  10 ){\makebox (  50, 50)[b]{$u_n$}        }
  \put(-200,   0 ){\vector  (   1,  0)   {50}           }
  \put(-150, -50 ){\framebox( 100,100)   {$\uparrow2$}  }
  \put(-150, -40 ){\makebox ( 100, 80)[b]{upsampler}    }
  \put(- 50,   0 ){\vector  (   1,  0)   {50}           }
  \put( 200,  10 ){\makebox (  50, 50)[b]{$y_n$}        }

  \put( 200,   0 ){\vector  (   1,  0)   {50}         }
  \put(   0,-150 ){\dashbox ( 200,200)   {  }           }
  \put(   0,-140 ){\makebox ( 200,200)[b]{Manchester coder}         }
  \put(   0,-140 ){\makebox ( 200,200)[ ]{
        \begin{fsK}
        \begin{math}
        \begin{array}{c|cc}
           u_{2n}' & y_n & y_{n+1} \\
           \hline
           0 &   0 & 1   \\
           1 &   1 & 0  
        \end{array}
        \end{math}
        \end{fsK}
     }}

  \put(- 50,   10 ){\makebox (  50, 50)[b]{$u_n'$}        }
\end{picture}                                   
\end{fsK}
\end{center}
\caption{
   Manchester modulation coder
   \label{fig:manchester_model}
   }
\end{figure}


The Manchester modulation coder is illustrated in Figure~\ref{fig:manchester_model}.
An example RZ modulated waveform is illustrated in Figure~\ref{fig:manchester_wave}.


\begin{figure}[ht]
\begin{center}
\begin{fsL}
\setlength{\unitlength}{0.1mm}
\begin{picture}(1400,300)(-100,-130)
  %\graphpaper[10](0,0)(300,300)
  \thicklines
  \put        (-100,   0){\line( 1, 0){1400}} 

  \thinlines
  %\put        (   0,   0){\line( 0, 1){ 100}} 
  \put        (   0, 100){\line( 1, 0){  50}}   
  \put        (  50,-100){\line( 0, 1){ 200}} 
  \put        (  50,-100){\line( 1, 0){ 100}}   
  \put        ( 150,-100){\line( 0, 1){ 200}} 
  \put        ( 150, 100){\line( 1, 0){  50}}   
  \put        ( 200,-100){\line( 0, 1){ 200}} 
  \put        ( 200,-100){\line( 1, 0){  50}}   
  \put        ( 250,-100){\line( 0, 1){ 200}} 
  \put        ( 250, 100){\line( 1, 0){ 100}}   
  \put        ( 350,-100){\line( 0, 1){ 200}} 
  \put        ( 350,-100){\line( 1, 0){ 100}}   
  \put        ( 450,-100){\line( 0, 1){ 200}} 
  \put        ( 450, 100){\line( 1, 0){ 100}}   
  \put        ( 550,-100){\line( 0, 1){ 200}} 
  \put        ( 550,-100){\line( 1, 0){  50}}   
  \put        ( 600,-100){\line( 0, 1){ 200}} 
  \put        ( 600, 100){\line( 1, 0){  50}}   
  \put        ( 650,-100){\line( 0, 1){ 200}} 
  \put        ( 650,-100){\line( 1, 0){  50}}   
  \put        ( 700,-100){\line( 0, 1){ 200}} 
  \put        ( 700, 100){\line( 1, 0){  50}}   
  \put        ( 750,-100){\line( 0, 1){ 200}} 
  \put        ( 750,-100){\line( 1, 0){ 100}}   
  \put        ( 850,-100){\line( 0, 1){ 200}} 
  \put        ( 850, 100){\line( 1, 0){  50}}   
  \put        ( 900,-100){\line( 0, 1){ 200}} 
  \put        ( 900,-100){\line( 1, 0){  50}}   
  \put        ( 950,-100){\line( 0, 1){ 200}} 
  \put        ( 950, 100){\line( 1, 0){  50}}   
  \put        (1000,-100){\line( 0, 1){ 200}} 
  \put        (1000,-100){\line( 1, 0){  50}}   
  \put        (1050,-100){\line( 0, 1){ 200}} 
  \put        (1050, 100){\line( 1, 0){ 100}}   
  \put        (1150,-100){\line( 0, 1){ 200}} 
  \put        (1150,-100){\line( 1, 0){  50}}   

  %\multiput(0,-10)(100,  0){13}{\line(0,1){20}} 
  \multiput(0,0)(100,  0){13}{\qbezier[30](  0,-130)(  0,  0)(  0, 160)}

  \put        (   0, 130 ){\makebox (100,0){1}     }
  \put        ( 100, 130 ){\makebox (100,0){0}     }
  \put        ( 200, 130 ){\makebox (100,0){0}     }
  \put        ( 300, 130 ){\makebox (100,0){1}     }
  \put        ( 400, 130 ){\makebox (100,0){0}     }
  \put        ( 500, 130 ){\makebox (100,0){1}     }
  \put        ( 600, 130 ){\makebox (100,0){1}     }
  \put        ( 700, 130 ){\makebox (100,0){1}     }
  \put        ( 800, 130 ){\makebox (100,0){0}     }
  \put        ( 900, 130 ){\makebox (100,0){0}     }
  \put        (1000, 130 ){\makebox (100,0){0}     }
  \put        (1100, 130 ){\makebox (100,0){1}     }

  \put        ( -10, 100 ){\makebox (  0, 0)[r]{$+A$}     }
  \put        ( -10,-100 ){\makebox (  0, 0)[r]{$-A$}     }
  \put        ( -100, 130 ){\makebox (  0, 0)[r]{$u_n:$}     }
  \put        ( -100,  50 ){\makebox (  0, 0)[r]{$s(t):$}     }

\end{picture}                                   
\end{fsL}
\caption{
  Manchester modulated waveform
   \label{fig:manchester_wave}
   }
\end{center}
\end{figure}

%--------------------------------------
\begin{theorem}
\label{thm:manchester_stat}
%--------------------------------------
Let
\begin{liste}
   \item $u_n:\Z\to\{0,1\}$ be an IID WSS random process with probabilities 
         \[ \psp\{u_n=0\}=\psp\{u_n=1\}= \frac{1}{2} \hspace{1cm}\mbox{for all $n$}\]
   \item $s(t)$ be the waveform Manchester modulated by $u_n$
   \item $\Rss(t+\tau,t)$ be the autocorrelation of $s(t)$ such that
         \[ \Rss(t+\tau,t) \eqd \Eb{s(t+\tau)s(t)}\]
   \item $\avgRss(\tau)$ be the time average of $\Rss(t+\tau,t)$.
         \[ \avgRss(\tau) \eqd \frac{1}{T}\int_0^T \Rss(t+\tau,t) \dt \]
   \item $\avgSss(f)$ be the Fourier transform of $\avgRss(\tau)$ such that
         \[ \avgSss(f) \eqd \int_\tau \avgRss(\tau) e^{-i2\pi f\tau} \dtau.\]
\end{liste}

Then 
\begin{eqnarray*}
   \Rss(t+\tau,t) 
     &=& \left\{
         \begin{array}{rl}
             0    & : 0 \le (t\;\mod[T]) < \tau \\
            +A^2  & : \tau \le (t\;\mod[T]) < \frac{T}{2} \\
            -A^2  & : \frac{T}{2} \le (t\;\mod[T]) < \tau + \frac{T}{2} \\
            +A^2  & : \tau + \frac{T}{2} \le (t\;\mod[T]) < T
         \end{array}
         \right.
\end{eqnarray*}

\begin{eqnarray*}
   \avgRss(\tau) 
     &=& \left\{
         \begin{array}{rl}
             A^2\left(1 - 3\frac{|\tau|}{T} \right)             & : 0 \le |\tau| < \frac{T}{2} \\
             -\frac{A^2}{2} \left(1 - \frac{|\tau|}{T} \right)  & : \frac{T}{2} \le |\tau| < T 
         \end{array}
         \right.
\end{eqnarray*}
\attention
\begin{eqnarray*}
   \avgSxx(f) 
    &\eqq& A^2 T\frac{\sin^4\pi f T/2}{\pi f T/2}
\end{eqnarray*}
\end{theorem}

\begin{proof}
\begin{eqnarray*}
  \avgSss(f)
    &=& \int_\tau \avgRss(\tau) e^{-i2\pi f\tau}\dtau
  \\&=& \int_\tau \avgRss(\tau) \cos(2\pi f\tau)\dtau
       -i\int_\tau \avgRss(\tau) \sin(2\pi f\tau)\dtau
  \\&=& 2\int_0^T \avgRss(\tau) \cos(2\pi f\tau)\dtau + 0
  \\&=& 2\int_0^{T/2} A^2\left(1-3\frac{\tau}{T}\right) \cos(2\pi f\tau)\dtau 
       -2\int_{T/2}^T \frac{A^2}{2}\left(1-\frac{\tau}{T}\right) \cos(2\pi f\tau)\dtau 
  \\&=& 2A^2\int_0^{T/2} \cos(2\pi f\tau)\dtau 
       -A^2\int_{T/2}^T  \cos(2\pi f\tau)\dtau 
       -\frac{6A^2}{T}\int_0^{T/2} \tau \cos(2\pi f\tau)\dtau 
       +\frac{A^2}{T}\int_{T/2}^T  \tau \cos(2\pi f\tau)\dtau 
  \\&=& A^2T\left( \frac{\sin\pi fT}{\pi fT}\right)
       -A^2T\left( \frac{\sin2\pi fT}{2\pi fT}\right)
       +\frac{A^2T}{2}\left( \frac{\sin\pi fT}{\pi fT}\right)
       -\frac{6A^2T}{4}\frac{\sin\pi fT}{\pi fT}
       -\frac{6A^2}{4\pi f}\frac{\cos\pi fT}{\pi fT}
\\&&
       + \frac{6A^2}{(2\pi f)^2T}
       + A^2T\frac{\sin2\pi fT}{2\pi fT} 
       + \frac{A^2}{2\pi f}\frac{\cos2\pi fT}{2\pi fT}
       - \frac{A^2T}{4}\frac{\sin\pi fT}{\pi fT}
       - \frac{T}{4\pi f}\frac{\cos\pi fT}{\pi fT}
  \\&\eqq& A^2 T \frac{\sin^4\pi f T/2}{\pi f T/2}
\end{eqnarray*}
\attention
I have not been able to solve this well yet.
The last line is taken from reference
\cite{kao}.
\end{proof}

%=======================================
\section{Non-Return to Zero Modulation Inverted (NRZI)}
\label{sec:NRZI}
\index{Non-Return to Zero Inverted}
\index{NRZI}
%=======================================
NRZI is a modulation code, however it is {\em not} a runlength-limited code.
NRZI has memory and is therefore a kind of state machine.


\begin{figure}[ht]
\begin{center}
\begin{fsL}
\setlength{\unitlength}{0.1mm}
\begin{picture}(1400,300)(-100,-130)
  %\graphpaper[10](0,0)(300,300)
  \thicklines
  \put        (-100,   0){\line( 1, 0){1400}} 

  \thinlines
  \put        (   0,   0){\line( 0, 1){ 100}} 
  \put        (   0, 100){\line( 1, 0){ 300}}   
  \put        ( 300, 100){\line( 0,-1){ 200}}
  \put        ( 300,-100){\line( 1, 0){ 200}}
  \put        ( 500,-100){\line( 0, 1){ 200}} 
  \put        ( 500, 100){\line( 1, 0){ 100}}   
  \put        ( 600, 100){\line( 0,-1){ 200}}   
  \put        ( 600,-100){\line( 1, 0){ 100}}   
  \put        ( 700,-100){\line( 0, 1){ 200}} 
  \put        ( 700, 100){\line( 1, 0){ 400}}   
  \put        (1100, 100){\line( 0,-1){ 200}}   
  \put        (1100,-100){\line( 1, 0){ 100}}

  \multiput(0,0)(100,  0){13}{\qbezier[30](  0,-130)(  0,  0)(  0, 160)}

  \put        (   0, 130 ){\makebox (100,30){1}     }
  \put        ( 100, 130 ){\makebox (100,30){0}     }
  \put        ( 200, 130 ){\makebox (100,30){0}     }
  \put        ( 300, 130 ){\makebox (100,30){1}     }
  \put        ( 400, 130 ){\makebox (100,30){0}     }
  \put        ( 500, 130 ){\makebox (100,30){1}     }
  \put        ( 600, 130 ){\makebox (100,30){1}     }
  \put        ( 700, 130 ){\makebox (100,30){1}     }
  \put        ( 800, 130 ){\makebox (100,30){0}     }
  \put        ( 900, 130 ){\makebox (100,30){0}     }
  \put        (1000, 130 ){\makebox (100,30){0}     }
  \put        (1100, 130 ){\makebox (100,30){1}     }

  \put        ( -10, 100 ){\makebox (  0, 0)[r]{+A}     }
  \put        ( -10,-100 ){\makebox (  0, 0)[r]{-A}     }

\end{picture}                                   
\end{fsL}
\caption{
  NRZI waveform
   \label{fig:NRZI_wave}
   }
\end{center}
\end{figure}





%---------------------------------------
\begin{definition}
\label{def:NRZI}
%---------------------------------------
{\bf Non-Return to Zero Inverted (NRZI)} is a modulation code
with input sequence $u_n$ and output sequence $y_n$ such that
(see Figure~\ref{fig:NRZI_model})
\[ y_n  = (y_{n-1} + u_n)\mod[2].  \]
\end{definition}

\begin{figure}[ht]
\begin{center}
\begin{fsK}
\setlength{\unitlength}{0.2mm}                  
\begin{picture}(300,200)(-50,-150)  
  \thinlines                                      
  %\graphpaper[10](0,0)(700,100)                  

  \put(- 50,   0 ){\vector  (   1,  0)   {90}          }
  \put(   0,-150 ){\dashbox ( 200,200)   {  }           }
  \put(   0,-140 ){\makebox ( 200,200)[b]{NRZI modulation coder}     }
  \put(   0, -50 ){\makebox ( 100,100)   {$+$}          }
  \put(  50,   0 ){\circle{20}                          }
  \put(  50,-100 ){\vector  (   0,  1)   { 90}          }
  \put(  60,   0 ){\line    (   1,  0)   { 10}          }
  \put(  90,   0 ){\line    (   1,  0)   {110}          }
  \put(  70, -10 ){\framebox(  20, 20)   {   }          }
  \put(   0,   0 ){\makebox ( 160, 50)   {$\mod[2]$}    }
  \put(  80, -70 ){\framebox(  40, 40)   {$z^{-1}$}     }
  \put( 100,-100 ){\line    (   0,  1)   { 30}          }
  \put( 100, -30 ){\line    (   0,  1)   { 30}          }
  \put(  50,-100 ){\line    (   1,  0)   { 50}          }
  \put( 200,   0 ){\vector  (   1,  0)   { 50}          }

  \put(- 50,   10 ){\makebox (  50, 50)[b]{$u_n$}        }
  \put( 200,   10 ){\makebox (  50, 50)[b]{$y_n$}        }
\end{picture}                                   
\end{fsK}
\end{center}
\caption{
   NRZI modulation coder
   \label{fig:NRZI_model}
   }
\end{figure}


\paragraph{Detection.}
Detection in an AWGN channel can be performed using a trellis 
(see Figure~\ref{fig:NRZI_trellis}) or 
single statistic decision regions. 
A very clean decision region approach is the {\em duobinary ISI solution}
described in Section~\ref{sec:duobinary} (page \pageref{sec:duobinary}).

\begin{figure}[ht]
\begin{center}
\begin{fsL}
\setlength{\unitlength}{0.30mm}
\begin{picture}(400,120)(-100,0)
  %\graphpaper[10](0,0)(300,300)
  \put( -80 , 100 ){state 1}
  \put( -80 ,   0 ){state 0}
  \thinlines
  \put(   0 , 100 ){\circle*{10}}
  \put(   0 ,   0 ){\circle*{10}}

\multiput(0,0)(100,0){3}{
  \put        (  0,100){\line( 1,-1){100}} % state0 path1
  \put        (  0,  0){\line( 1, 1){100}} % state1 path1
  \multiput   (  0,100)(10,0){10}{\line( 1, 0){5}} % state1 path1
  \multiput   (  0,  0)(10,0){10}{\line( 1, 0){5}} % state0 path1
  \put( 100 , 100 ){\circle*{10}}
  \put( 100 ,   0 ){\circle*{10}}
}
\end{picture}                                   
\end{fsL}
\hspace{1cm}
\begin{tabular}{cl}
   $\cdots$ & $y_n=0$ \\
   ---      & $y_n=1$ 
\end{tabular}
\caption{
  NRZI trellis diagram
   \label{fig:NRZI_trellis}
   }
\end{center}
\end{figure}



%=======================================
\section{Runlength-limited modulation codes}
\index{Runlength-limited modulation codes}
%=======================================
%---------------------------------------
\subsubsection{Definitions}
%---------------------------------------
%---------------------------------------
\begin{definition}
\index{(d,k)}
\index{(d,k;n)}
%---------------------------------------
A {\bf $(d,k)$-coded sequence} is any binary sequence such that
\[
   d \le \mbox{(the number of 0s between any two consecutive 1s)} \le k. 
\]

A {\bf $(d,k;n)$-coded sequence} is a $(d,k)$-coded sequence of length $n$.
\end{definition}

%---------------------------------------
\begin{definition}{\bf Fixed length code set, $X(d,k;n)$.}\\
\index{(d,k)!fixed length code set}
%---------------------------------------
The set $X(d,k;n)$ is a set of $(d,k;n)$-coded sequences such that
if
\[ (a_1,a_2,\ldots,a_n),(b_1,b_2,\ldots,b_n)\in X(d,k;n)\]
then
\[ (a_1,a_2,\ldots,a_n,b_1,b_2,\ldots,b_n)\]
is also a $(d,k)$-coded sequence.
\end{definition}

%---------------------------------------
\begin{definition}{\bf Variable length code set, $\bar{X}(d,k;n)$.}\\
\index{(d,k)!variable length code set}
%---------------------------------------
The set $\bar{X}(d,k;n)$ is a set of $(d,k;m)$-coded sequences such that
$m\le n$ and if
   \[ (a_1,a_2,\ldots,a_k),(b_1,b_2,\ldots,b_m)\in \bar{X}(d,k;n)\]
then
   \[ (a_1,a_2,\ldots,a_k,b_1,b_2,\ldots,b_m)\]
is also a $(d,k)$-coded sequence.
\end{definition}


\paragraph{State diagram.}
A $(d,k)$ code can be modeled as a state diagram with $k+1$ states
such that the output $y_n$ is
\begin{eqnarray*}
   y_n &=& \left\{
              \begin{tabular}{ll}
                 $1$ & : state$=0$ \\
                 $0$ & : state$\ne 0$.
              \end{tabular}
           \right.
\end{eqnarray*}
and transitions between states are as shown in Figure~\ref{fig:(d,k)_sm}.

\begin{figure}[ht]
\begin{center}
\begin{fsK}
\setlength{\unitlength}{0.15mm}
\begin{picture}(960,130)(-30,-100)  
  \thinlines                                      
  %\graphpaper[10](0,0)(700,100)                  

  \put(   0,   0 ){\makebox (   0,  0)   {$0$}          }
  \put(   0,   0 ){\circle               { 60}          }
  \put(   0,  10 ){\makebox ( 100, 50)[b]{$0,1$}          }
  \put(  30,   0 ){\vector  (   1,  0)   { 40}          }
  \put(   0,-100 ){\vector  (   0,  1)   { 70}          }

  \put( 100,   0 ){\makebox (   0,  0)   {$1$}          }
  \put( 100,   0 ){\circle               { 60}          }
  \put( 100,  10 ){\makebox ( 100, 50)[b]{$0,1$}          }
  \put( 130,   0 ){\vector  (   1,  0)   { 40}          }
 %\put( 100, -30 ){\vector  (   0, -1)   { 70}          }
 %\put( 110,-100 ){\makebox (  90,100)[l]{$u_n=1$}          }

  \put( 200,   0 ){\makebox (   0,  0)   {$2$}          }
  \put( 200,   0 ){\circle               { 60}          }
  \put( 200,  10 ){\makebox ( 100, 50)[b]{$0,1$}          }
  \put( 230,   0 ){\vector  (   1,  0)   { 40}          }
 %\put( 200, -30 ){\vector  (   0, -1)   { 70}          }
 %\put( 210,-100 ){\makebox (  90,100)[l]{$u_n=1$}          }

  \put( 300,   0 ){\makebox (   0,  0)   {$\cdots$}     }
 %\put( 300,   0 ){\circle               { 60}          }
  \put( 300,  10 ){\makebox ( 100, 50)[b]{$0,1$}          }
  \put( 330,   0 ){\vector  (   1,  0)   { 40}          }
 %\put( 300, -30 ){\vector  (   0, -1)   { 70}          }
  \put( 200,-100 ){\makebox ( 200,100)[ ]{$\cdots$}     }

  \put( 400,   0 ){\makebox (   0,  0)   {$d-1$}        }
  \put( 400,   0 ){\circle               { 60}          }
  \put( 400,  10 ){\makebox ( 100, 50)[b]{$0,1$}          }
  \put( 430,   0 ){\vector  (   1,  0)   { 40}          }
 %\put( 400, -30 ){\vector  (   0, -1)   { 70}          }
 %\put( 410,-100 ){\makebox (  90,100)[l]{$1$}          }

  \put( 500,   0 ){\makebox (   0,  0)   {$d$}          }
  \put( 500,   0 ){\circle               { 60}          }
  \put( 500,  10 ){\makebox ( 100, 50)[b]{$0$}          }
  \put( 530,   0 ){\vector  (   1,  0)   { 40}          }
  \put( 500, -30 ){\vector  (   0, -1)   { 70}          }
  \put( 510,-100 ){\makebox (  90,100)[l]{$u_n=1$}          }

  \put( 600,   0 ){\makebox (   0,  0)   {$d+1$}        }
  \put( 600,   0 ){\circle               { 60}          }
  \put( 600,  10 ){\makebox ( 100, 50)[b]{$0$}          }
  \put( 630,   0 ){\vector  (   1,  0)   { 40}          }
  \put( 600, -30 ){\vector  (   0, -1)   { 70}          }
  \put( 610,-100 ){\makebox (  90,100)[l]{$u_n=1$}          }

  \put( 700,   0 ){\makebox (   0,  0)   {$\cdots$}     }
 %\put( 700,   0 ){\circle               { 60}          }
  \put( 700,  10 ){\makebox ( 100, 50)[b]{$0$}          }
  \put( 730,   0 ){\vector  (   1,  0)   { 40}          }
 %\put( 700, -30 ){\vector  (   0, -1)   { 70}          }
  \put( 600,-100 ){\makebox ( 200,100)[ ]{$\cdots$}     }

  \put( 800,   0 ){\makebox (   0,  0)   {$k-1$}        }
  \put( 800,   0 ){\circle               { 60}          }
  \put( 800,  10 ){\makebox ( 100, 50)[b]{$0$}          }
  \put( 830,   0 ){\vector  (   1,  0)   { 40}          }
  \put( 800, -30 ){\vector  (   0, -1)   { 70}          }
  \put( 810,-100 ){\makebox (  90,100)[l]{$u_n=1$}          }

  \put( 900,   0 ){\makebox (   0,  0)   {$k$}          }
  \put( 900,   0 ){\circle               { 60}          }
 %\put( 900,  10 ){\makebox ( 100, 50)[b]{$u_n=0$}          }
 %\put( 930,   0 ){\vector  (   1,  0)   { 40}          }
  \put( 900, -30 ){\vector  (   0, -1)   { 70}          }
  \put( 910,-100 ){\makebox (  90,100)[l]{$u_n=0,1$}          }

  \put( 900,-100 ){\vector  (  -1,  0)   {900}          }
\end{picture}                                   
\end{fsK}
\end{center}
\caption{
   $(d,k)$-coded sequence state diagram 
   \label{fig:(d,k)_sm}
   }
\end{figure}

%---------------------------------------
\begin{definition}
\index{transition matrix}
%---------------------------------------
The {\bf transition matrix} $\opD_0$ is the $N\times N$
square matrix with elements  $a_{mn}$ such that
\[
   a_{mn} =
      \left\{
         \begin{tabular}{ll}
            1 &: coding state changes from $m$ to $n$ when input is $0$.\\
            0 &: coding state does not change when input is $0$.
         \end{tabular}
      \right.
\]

The {\bf transition matrix} $\opD_1$ is the $N\times N$
square matrix with elements  $b_{mn}$ such that
\[
   b_{mn} =
      \left\{
         \begin{tabular}{ll}
            1 &: coding state changes from $m$ to $n$ when input is $1$.\\
            0 &: coding state does not change when input is $1$.
         \end{tabular}
      \right.
\]

The {\bf transition matrix} $\opD$ is the $N\times N$
square matrix with elements  $d_{mn}$ such that
\[
   d_{mn} = a_{mn} \lor b_{mn}
\]
where $\lor$ is the {\em inclusive-OR operation}.
\end{definition}

\paragraph{Transition matrices.}
The transition matrices for a $(d,k)$ code are as follows:
\begin{fsK}
\[
   \opD_0 = 
      \left[
         \begin{array}{*{8}{c}}
            0      & 1 & 0 & 0 & \cdots & 0  \\
            0      & 0 & 1 & 0 & \cdots & 0  \\
            0      & 0 & 0 & 1 & \cdots & 0  \\
            \vdots &   &   &   & \ddots & 0  \\
            0      & 0 & 0 & 0 & \cdots & 1  \\
            1      & 0 & 0 & 0 & \cdots & 0  
         \end{array}
      \right]
\hspace{3ex}
   \opD_1 = 
      \left[
         \begin{array}{*{6}{c}}
                   &   &   &   &        &    \\
            0      & 1 & 0 & 0 & \cdots & 0  \\
            0      & 0 & 0 & 0 & \cdots & 0  \\
            \vdots &   &   &   & \ddots & 0  \\
            0      & 0 & 0 & 0 & \cdots & 0  \\
            1      & 0 & 0 & 0 & \cdots & 0  \\
            1      & 0 & 0 & 0 & \cdots & 0  \\
            \vdots &   &   &   & \ddots & 0  \\
            1      & 0 & 0 & 0 & \cdots & 0 
         \end{array}
      \right]
         \begin{array}{l}
            \mbox{row} \\
            \hline
            0          \\
            1          \\
            \vdots     \\
            d-1        \\
            d          \\
            d+1        \\
            \vdots     \\
            k-1
         \end{array}
\hspace{3ex}
   \opD   = 
      \left[
         \begin{array}{*{8}{c}}
            0      & 1 & 0 & 0 & \cdots & 0  \\
            1      & 0 & 1 & 0 & \cdots & 0  \\
            1      & 0 & 0 & 1 & \cdots & 0  \\
            \vdots &   &   &   & \ddots & 0  \\
            1      & 0 & 0 & 0 & \cdots & 1  \\
            1      & 0 & 0 & 0 & \cdots & 0  
         \end{array}
      \right]
\]
\end{fsK}
%---------------------------------------
\subsubsection{Characteristics}
%---------------------------------------
\paragraph{Symbol mapping.}
The symbols to be transmitted are mapped into the elements of 
$X(d,k;n)$.
The maximum number of symbols that can be mapped is
   \[ \floor{\log_2\left| X(d,k;n)\right|}, \]
where $|\cdot|:X\to\Z$ represents the order of a set X.

%---------------------------------------
\begin{definition}
\index{channel capacity}
%---------------------------------------
The {\bf capacity} of a $(d,k)$-coded sequence is
\[
   C(d,k) \eqd \lim_{n\to\infty}\frac{1}{n}
               \floor{\log_2\left| X(d,k;n)\right|}.
\]
\end{definition}

%---------------------------------------
\begin{theorem}
%---------------------------------------
Let 
\begin{liste}
   \item $\opD$ be the transition matrix of $(d,k)$
   \item $\lambda_{\mathrm{max}}$ be the largest eigenvalue of $\opD$.
\end{liste}

Then the capacity $C(d,k)$ is
\[
   C(d,k) = \log_2 \lambda_{\mathrm{max}}.
\]
\end{theorem}

The capacities for several $X(d,k)$-coded sequences are given in 
Table~\ref{tbl:C(d,k)}. \citep{proakis}{582}

\begin{table*}[t]
\begin{fsL}
\[
\begin{array}{r|*{7}{l}}
    k     &    d=0 &    d=1 &    d=2 &    d=3 &    d=4 &    d=5 &    d=6\\
\hline      
    2     & 0.8791 & 0.4057  \\ 
    3     & 0.9468 & 0.5515 & 0.2878 \\
    4     & 0.9752 & 0.6174 & 0.4057 & 0.2232 \\ 
    5     & 0.9881 & 0.6509 & 0.4650 & 0.3218 & 0.1823 \\
    6     & 0.9942 & 0.6690 & 0.4979 & 0.3746 & 0.2669 & 0.1542  \\
    7     & 0.9971 & 0.6793 & 0.5174 & 0.4057 & 0.3142 & 0.2281 & 0.1335 \\
    8     & 0.9986 & 0.6853 & 0.5293 & 0.4251 & 0.3432 & 0.2709 & 0.1993 \\
    9     & 0.9993 & 0.6888 & 0.5369 & 0.4376 & 0.3630 & 0.2979 & 0.2382 \\
   10     & 0.9996 & 0.6909 & 0.5418 & 0.4460 & 0.3746 & 0.3158 & 0.2633 \\
   11     & 0.9998 & 0.6922 & 0.5450 & 0.4516 & 0.3833 & 0.3282 & 0.2804 \\
   12     & 0.9999 & 0.6930 & 0.5471 & 0.4555 & 0.3894 & 0.3369 & 0.2924 \\
   13     & 0.9999 & 0.6935 & 0.5485 & 0.4583 & 0.3937 & 0.3432 & 0.3011 \\
   14     & 0.9999 & 0.6938 & 0.5495 & 0.4602 & 0.3968 & 0.3478 & 0.3074 \\
   15     & 0.9999 & 0.6939 & 0.5501 & 0.4615 & 0.3991 & 0.3513 & 0.3122 \\
   \infty & 1.0000 & 0.6942 & 0.5515 & 0.4650 & 0.4057 & 0.3620 & 0.3282
\end{array}
\]
\end{fsL}
\caption{
   $C(d,k)$: Capacities of $(d,k)$-coded sequences
   \label{tbl:C(d,k)}
   }
\end{table*}


%---------------------------------------
\begin{definition}
\index{efficiency}
%---------------------------------------
The {\bf efficiency} of the $X(d,k;n)$ code set is
\[
    \mathrm{efficiency} \eqd \frac{\mbox{code rate of } X(d,k;n)}{C(d,k)}.
\]

The {\bf efficiency} of the $\bar{X}(d,k;n)$ code set is
\[
    \mathrm{efficiency} \eqd \frac{\mbox{average code rate of } X(d,k;n)}{C(d,k)}.
\]
\end{definition}


%---------------------------------------
\subsubsection{Examples: fixed-length, no memory}
%---------------------------------------
%---------------------------------------
\begin{example}{\bf Code set $X(1,3;4)$:}\\
\label{ex:(1,3,4)}
%---------------------------------------
\begin{figure}[ht]
\begin{center}
\begin{fsK}
\setlength{\unitlength}{0.15mm}
\begin{picture}(360,130)(-30,-100)  
  \thinlines                                      
  %\graphpaper[10](0,0)(700,100)                  

  \put(   0,   0 ){\makebox (   0,  0)   {$0$}          }
  \put(   0,   0 ){\circle               { 60}          }
  \put(   0,  10 ){\makebox ( 100, 50)[b]{$0,1$}          }
  \put(  30,   0 ){\vector  (   1,  0)   { 40}          }
  \put(   0,-100 ){\vector  (   0,  1)   { 70}          }

  \put( 100,   0 ){\makebox (   0,  0)   {$1$}          }
  \put( 100,   0 ){\circle               { 60}          }
  \put( 100,  10 ){\makebox ( 100, 50)[b]{$0$}          }
  \put( 130,   0 ){\vector  (   1,  0)   { 40}          }
  \put( 100, -30 ){\vector  (   0, -1)   { 70}          }
  \put( 110,-100 ){\makebox (  90,100)[l]{$1$}          }

  \put( 200,   0 ){\makebox (   0,  0)   {$2$}          }
  \put( 200,   0 ){\circle               { 60}          }
  \put( 200,  10 ){\makebox ( 100, 50)[b]{$0$}        }
  \put( 230,   0 ){\vector  (   1,  0)   { 40}          }
  \put( 200, -30 ){\vector  (   0, -1)   { 70}          }
  \put( 210,-100 ){\makebox (  90,100)[l]{$1$}          }

  \put( 300,   0 ){\makebox (   0,  0)   {$3$}          }
  \put( 300,   0 ){\circle               { 60}          }
 %\put( 300,  10 ){\makebox ( 100, 50)[b]{$0,1$}        }
 %\put( 330,   0 ){\vector  (   1,  0)   { 40}          }
  \put( 300, -30 ){\vector  (   0, -1)   { 70}          }
  \put( 310,-100 ){\makebox (  90,100)[l]{$0,1$}          }

  \put( 300,-100 ){\vector  (  -1,  0)   {300}          }
\end{picture}                                   
\end{fsK}
\end{center}
\caption{
   $(1,3)$-coded sequence state diagram
   \label{fig:(1,3)_sm}
   }
\end{figure}

Transition matrices:
\[
   \opD_0 = 
      \left[
         \begin{array}{*{8}{c}}
            0 & 1 & 0 & 0  \\
            0 & 0 & 1 & 0  \\
            0 & 0 & 0 & 1  \\
            1 & 0 & 0 & 0  
         \end{array}
      \right]
\hfill
   \opD_1 = 
      \left[
         \begin{array}{*{8}{c}}
            0 & 1 & 0 & 0  \\
            1 & 0 & 0 & 0  \\
            1 & 0 & 0 & 0  \\
            1 & 0 & 0 & 0  
         \end{array}
      \right]
\hfill
   \opD   = 
      \left[
         \begin{array}{*{8}{c}}
            0 & 1 & 0 & 0 \\
            1 & 0 & 1 & 0 \\
            1 & 0 & 0 & 1 \\
            1 & 0 & 0 & 0 
         \end{array}
      \right]
\]

Capacity:
\begin{eqnarray*}
   |\opD-\lambda \opI | 
     &=& \left|
            \begin{array}{*{8}{c}}
               -\lambda & 1        & 0        & 0 \\
               1        & -\lambda & 1        & 0 \\
               1        & 0        & -\lambda & 1 \\
               1        & 0        & 0        & -\lambda 
            \end{array}
         \right|
   \\&=& -\lambda \left|
            \begin{array}{*{8}{c}}
               -\lambda & 1        & 0 \\
               0        & -\lambda & 1 \\
               0        & 0        & -\lambda 
            \end{array}
         \right|
        -1\cdot\left|
            \begin{array}{*{8}{c}}
               1        & 1        & 0 \\
               1        & -\lambda & 1 \\
               1        & 0        & -\lambda 
            \end{array}
         \right|
   \\&=& -\lambda(-\lambda^3) -1(\lambda^2+\lambda+1)
   \\&=& \lambda^4 - \lambda^2 - \lambda - 1
\end{eqnarray*}

\begin{eqnarray*}
   C(d,k) 
     = \log_2(\lambda_{\mathrm{max}})
     = \log_2(1.46557123)
     = 0.551463
\end{eqnarray*}

There are multiple sets that are $X(1,3;4)$ code sets:
\[
\begin{tabular}{c|ccc}
                 & \mc{3}{c}{$X(1,3;4)$ code sets}  \\
   $u_n$         & set1 & set2 & set3 \\
   \hline
   0 & 0010 & 1000 & 0100 \\
   1 & 0101 & 1010 & 0101
\end{tabular}
\]

The efficiency for each of these sets is
\[
    \mathrm{efficiency} 
    = \frac{\mbox{code rate}}{C(d,k)} 
    = \frac{1/4}{0.5515} 
    = 0.4533
\]
\end{example}


%---------------------------------------
\begin{example}{\bf Code set $X(2,\infty,4)$:}\\
%---------------------------------------

\begin{figure}[ht]
\begin{center}
\begin{fsK}
\setlength{\unitlength}{0.15mm}
\begin{picture}(560,130)(-30,-100)  
  \thinlines                                      
  %\graphpaper[10](0,0)(700,100)                  

  \put(   0,   0 ){\makebox (   0,  0)   {$0$}          }
  \put(   0,   0 ){\circle               { 60}          }
  \put(   0,  10 ){\makebox ( 100, 50)[b]{$0,1$}          }
  \put(  30,   0 ){\vector  (   1,  0)   { 40}          }
  \put(   0,-100 ){\vector  (   0,  1)   { 70}          }

  \put( 100,   0 ){\makebox (   0,  0)   {$1$}          }
  \put( 100,   0 ){\circle               { 60}          }
  \put( 100,  10 ){\makebox ( 100, 50)[b]{$0,1$}          }
  \put( 130,   0 ){\vector  (   1,  0)   { 40}          }
 %\put( 100, -30 ){\vector  (   0, -1)   { 70}          }
 %\put( 110,-100 ){\makebox (  90,100)[l]{$1$}          }

  \put( 200,   0 ){\makebox (   0,  0)   {$2$}          }
  \put( 200,   0 ){\circle               { 60}          }
  \put( 200,  10 ){\makebox ( 100, 50)[b]{$0$}          }
  \put( 230,   0 ){\vector  (   1,  0)   { 40}          }
  \put( 200, -30 ){\vector  (   0, -1)   { 70}          }
  \put( 210,-100 ){\makebox (  90,100)[l]{$1$}          }

  \put( 300,   0 ){\makebox (   0,  0)   {$3$}     }
  \put( 300,   0 ){\circle               { 60}          }
  \put( 300,  10 ){\makebox ( 100, 50)[b]{$0$}          }
  \put( 330,   0 ){\vector  (   1,  0)   { 40}          }
  \put( 300, -30 ){\vector  (   0, -1)   { 70}          }
  \put( 310,-100 ){\makebox ( 200,100)[l]{$1$}     }

  \put( 400,   0 ){\makebox (   0,  0)   {$4$}        }
  \put( 400,   0 ){\circle               { 60}          }
  \put( 400,  10 ){\makebox ( 100, 50)[b]{$0$}          }
  \put( 430,   0 ){\vector  (   1,  0)   { 40}          }
  \put( 400, -30 ){\vector  (   0, -1)   { 70}          }
  \put( 410,-100 ){\makebox (  90,100)[l]{$1$}          }

  \put( 500,   0 ){\makebox (   0,  0)   {$\cdots$}          }
 %\put( 500,   0 ){\circle               { 60}          }
 %\put( 500,  10 ){\makebox ( 100, 50)[b]{$0$}          }
 %\put( 530,   0 ){\vector  (   1,  0)   { 40}          }
 %\put( 500, -30 ){\vector  (   0, -1)   { 70}          }
  \put( 510,-100 ){\makebox (  90,100)[c]{$\cdots$}          }

  \put( 400,-100 ){\vector  (  -1,  0)   {400}          }
\end{picture}                                   
\end{fsK}
\end{center}
\caption{
   $(2,\infty)$-coded sequence state diagram
   \label{fig:(2,inf)_sm}
   }
\end{figure}


\[
\begin{array}{c|c}
   u_n         & \mbox{code} \\
   \hline
   0 & 0001 \\
   1 & 0010
\end{array}
\]


\[
    \mathrm{efficiency} 
    = \frac{\mbox{code rate}}{C(d,k)} 
    = \frac{1/4}{0.5515} 
    = 0.4533
\]


\end{example}



%---------------------------------------
\begin{example}{\bf Code set $X(0,3,4)$:}\\
%---------------------------------------

\begin{figure}[ht]
\begin{center}
\begin{fsK}
\setlength{\unitlength}{0.15mm}
\begin{picture}(430,130)(0,-100)  
  \thinlines                                      
  %\graphpaper[10](0,0)(700,100)                  

  \put(   0,   0 ){\vector  (   1,  0)   { 70}          }
 %\put(   0,   0 ){\makebox (   0,  0)   {$0$}          }
 %\put(   0,   0 ){\circle               { 60}          }
 %\put(   0,  10 ){\makebox ( 100, 50)[b]{$0,1$}          }
 %\put(  30,   0 ){\vector  (   1,  0)   { 40}          }
  \put(   0,-100 ){\line    (   0,  1)   {100}          }

  \put( 100,   0 ){\makebox (   0,  0)   {$0$}          }
  \put( 100,   0 ){\circle               { 60}          }
  \put( 100,  10 ){\makebox ( 100, 50)[b]{$0$}          }
  \put( 130,   0 ){\vector  (   1,  0)   { 40}          }
  \put( 100, -30 ){\vector  (   0, -1)   { 70}          }
  \put( 110,-100 ){\makebox (  90,100)[l]{$1$}          }

  \put( 200,   0 ){\makebox (   0,  0)   {$1$}          }
  \put( 200,   0 ){\circle               { 60}          }
  \put( 200,  10 ){\makebox ( 100, 50)[b]{$0$}          }
  \put( 230,   0 ){\vector  (   1,  0)   { 40}          }
  \put( 200, -30 ){\vector  (   0, -1)   { 70}          }
  \put( 210,-100 ){\makebox (  90,100)[l]{$1$}          }

  \put( 300,   0 ){\makebox (   0,  0)   {$2$}     }
  \put( 300,   0 ){\circle               { 60}          }
  \put( 300,  10 ){\makebox ( 100, 50)[b]{$0$}          }
  \put( 330,   0 ){\vector  (   1,  0)   { 40}          }
  \put( 300, -30 ){\vector  (   0, -1)   { 70}          }
  \put( 310,-100 ){\makebox ( 200,100)[l]{$1$}     }

  \put( 400,   0 ){\makebox (   0,  0)   {$3$}        }
  \put( 400,   0 ){\circle               { 60}          }
 %\put( 400,  10 ){\makebox ( 100, 50)[b]{$0$}          }
 %\put( 430,   0 ){\vector  (   1,  0)   { 40}          }
  \put( 400, -30 ){\vector  (   0, -1)   { 70}          }
  \put( 410,-100 ){\makebox (  90,100)[l]{$0,1$}          }

  \put( 400,-100 ){\vector  (  -1,  0)   {400}          }
\end{picture}                                   
\end{fsK}
\end{center}
\caption{
   $(0,3)$-coded sequence state diagram
   \label{fig:(0,3)_sm}
   }
\end{figure}

The state diagram is shown in Figure~\ref{fig:(0,3)_sm}.

The transition matrices are
\[
   \opD_0 = 
      \left[
         \begin{array}{*{8}{c}}
            0 & 1 & 0 & 0  \\
            0 & 0 & 1 & 0  \\
            0 & 0 & 0 & 1  \\
            1 & 0 & 0 & 0  
         \end{array}
      \right]
\hfill
   \opD_1 = 
      \left[
         \begin{array}{*{8}{c}}
            1 & 0 & 0 & 0  \\
            1 & 0 & 0 & 0  \\
            1 & 0 & 0 & 0  \\
            1 & 0 & 0 & 0  
         \end{array}
      \right]
\hfill
   \opD   = 
      \left[
         \begin{array}{*{8}{c}}
            1 & 1 & 0 & 0 \\
            1 & 0 & 1 & 0 \\
            1 & 0 & 0 & 1 \\
            1 & 0 & 0 & 0 
         \end{array}
      \right]
\]

To find the channel capacity:

\begin{eqnarray*}
   |\opD-\lambda \opI | 
     &=& \left|
            \begin{array}{*{8}{c}}
               1-\lambda & 1        & 0        & 0 \\
               1        & -\lambda & 1        & 0 \\
               1        & 0        & -\lambda & 1 \\
               1        & 0        & 0        & -\lambda 
            \end{array}
         \right|
   \\&=& (1-\lambda) \left|
            \begin{array}{*{8}{c}}
               -\lambda & 1        & 0 \\
               0        & -\lambda & 1 \\
               0        & 0        & -\lambda 
            \end{array}
         \right|
        -1\cdot\left|
            \begin{array}{*{8}{c}}
               1        & 1        & 0 \\
               1        & -\lambda & 1 \\
               1        & 0        & -\lambda 
            \end{array}
         \right|
   \\&=& (1-\lambda)(-\lambda^3) -1(\lambda^2-(-\lambda-1))
   \\&=& \lambda^4 - \lambda^3 - \lambda^2 - \lambda - 1
\end{eqnarray*}

\begin{eqnarray*}
   C(d,k) 
     &=& \log_2(\lambda_{\mathrm{max}})
   \\&=& \log_2(1.927562)
   \\&=& 0.946777
\end{eqnarray*}



\[
\begin{array}{c|c}
   u_n & \mbox{code} \\
   \hline
   000 & 0100  \\
   001 & 0101  \\
   010 & 0110  \\
   011 & 1001  \\
   100 & 1010  \\
   101 & 1011  \\
   110 & 1100  \\
   111 & 1101
\end{array}
\]

\[
    \mathrm{efficiency} 
      = \frac{\mbox{code rate}}{C(d,k)} 
    = \frac{3/4}{0.9468} 
    = 0.7921
\]


\end{example}


%---------------------------------------
\subsubsection{Example: fixed-length, with memory}
%---------------------------------------
%---------------------------------------
\begin{example}{\bf Code set $X(1,3;2)$ (Miller code):}\\
\label{ex:miller}
%---------------------------------------
\begin{figure}[ht]
\begin{center}
\begin{fsK}
\setlength{\unitlength}{0.2mm}                  
\begin{picture}(450,200)(-200,-150)  
  \thinlines                                      
  %\graphpaper[10](0,0)(700,100)                  

  \put(-200,  10 ){\makebox (  50, 50)[b]{$u_n$}        }
  \put(-200,   0 ){\vector  (   1,  0)   {50}           }
  \put(-150, -50 ){\framebox( 100,100)   {$\uparrow2$}  }
  \put(-150, -40 ){\makebox ( 100, 80)[b]{upsampler}    }
  \put(- 50,   0 ){\vector  (   1,  0)   {50}           }
  \put( 200,  10 ){\makebox (  50, 50)[b]{$y_n$}        }

  \put( 200,   0 ){\vector  (   1,  0)   {50}         }
  \put(   0,-150 ){\dashbox ( 200,200)   {  }           }
  \put(   0,-140 ){\makebox ( 200,200)[b]{Miller coder}         }
  \put(   0,-140 ){\makebox ( 200,200)[ ]{
        \begin{fsK}
        \begin{math}
        \begin{array}{cc|cc}
           u_{2n}' & y_{n-1} & y_n & y_{n+1} \\
           \hline
           0 & 0 & 1 & 0   \\
           0 & 1 & 0 & 0   \\
           1 & 0 & 0 & 1   \\
           1 & 1 & 0 & 1  
        \end{array}
        \end{math}
        \end{fsK}
     }}

  \put(- 50,   10 ){\makebox (  50, 50)[b]{$u_n'$}        }
\end{picture}                                   
\end{fsK}
\end{center}
\caption{
   Miller modulation coder
   \label{fig:Miller_model}
   }
\end{figure}

The state diagram, transition matrices, and capacity for 
$(1,3)$-coded sequences is shown in 
Example~\ref{ex:(1,3,4)} (page \pageref{ex:(1,3,4)}).
The operation is illustrated in Figure~\ref{fig:Miller_model} and 
described in the following table:
\[
   \begin{array}{cc|cc}
      u_{2n}' & y_{n-1} & y_n & y_{n+1} \\
      \hline
      0 & 0 & 1 & 0   \\
      0 & 1 & 0 & 0   \\
      1 & 0 & 0 & 1   \\
      1 & 1 & 0 & 1  
   \end{array}
\]

\[
    \mathrm{efficiency} 
    = \frac{\mbox{code rate}}{C(d,k)} 
    = \frac{1/2}{0.5515} 
    = 0.9066
\]
\end{example}

Compare this to the memoryless $X(1,3,4)$ code 
which has efficiency $0.4533$
(Example~\ref{ex:(1,3,4)} page \pageref{ex:(1,3,4)}).
In this case, allowing the code to have memory
has doubled the efficiency.



%---------------------------------------
\subsubsection{Example: variable-length, no memory}
%---------------------------------------
%---------------------------------------
\begin{example}{\bf Code set $\bar{X}(2,7)$:}\\
%---------------------------------------
This code has both variable length input and variable length output.
Many disk storage devices designed by IBM use this code.

\begin{figure}[ht]
\begin{center}
\begin{fsK}
\setlength{\unitlength}{0.15mm}
\begin{picture}(760,130)(-30,-100)  
  \thinlines                                      
  %\graphpaper[10](0,0)(700,100)                  

  \put(   0,   0 ){\makebox (   0,  0)   {$0$}          }
  \put(   0,   0 ){\circle               { 60}          }
  \put(   0,  10 ){\makebox ( 100, 50)[b]{$0,1$}          }
  \put(  30,   0 ){\vector  (   1,  0)   { 40}          }
  \put(   0,-100 ){\vector  (   0,  1)   { 70}          }

  \put( 100,   0 ){\makebox (   0,  0)   {$1$}          }
  \put( 100,   0 ){\circle               { 60}          }
  \put( 100,  10 ){\makebox ( 100, 50)[b]{$0,1$}          }
  \put( 130,   0 ){\vector  (   1,  0)   { 40}          }
 %\put( 100, -30 ){\vector  (   0, -1)   { 70}          }
 %\put( 110,-100 ){\makebox (  90,100)[l]{$1$}          }

  \put( 200,   0 ){\makebox (   0,  0)   {$2$}          }
  \put( 200,   0 ){\circle               { 60}          }
  \put( 200,  10 ){\makebox ( 100, 50)[b]{$0$}        }
  \put( 230,   0 ){\vector  (   1,  0)   { 40}          }
  \put( 200, -30 ){\vector  (   0, -1)   { 70}          }
  \put( 210,-100 ){\makebox (  90,100)[l]{$1$}          }

  \put( 300,   0 ){\makebox (   0,  0)   {$3$}          }
  \put( 300,   0 ){\circle               { 60}          }
  \put( 300,  10 ){\makebox ( 100, 50)[b]{$0$}        }
  \put( 330,   0 ){\vector  (   1,  0)   { 40}          }
  \put( 300, -30 ){\vector  (   0, -1)   { 70}          }
  \put( 310,-100 ){\makebox (  90,100)[l]{$1$}          }

  \put( 400,   0 ){\makebox (   0,  0)   {$4$}          }
  \put( 400,   0 ){\circle               { 60}          }
  \put( 400,  10 ){\makebox ( 100, 50)[b]{$0$}        }
  \put( 430,   0 ){\vector  (   1,  0)   { 40}          }
  \put( 400, -30 ){\vector  (   0, -1)   { 70}          }
  \put( 410,-100 ){\makebox (  90,100)[l]{$1$}          }

  \put( 500,   0 ){\makebox (   0,  0)   {$5$}          }
  \put( 500,   0 ){\circle               { 60}          }
  \put( 500,  10 ){\makebox ( 100, 50)[b]{$0$}        }
  \put( 530,   0 ){\vector  (   1,  0)   { 40}          }
  \put( 500, -30 ){\vector  (   0, -1)   { 70}          }
  \put( 510,-100 ){\makebox (  90,100)[l]{$1$}          }

  \put( 600,   0 ){\makebox (   0,  0)   {$6$}          }
  \put( 600,   0 ){\circle               { 60}          }
  \put( 600,  10 ){\makebox ( 100, 50)[b]{$0$}        }
  \put( 630,   0 ){\vector  (   1,  0)   { 40}          }
  \put( 600, -30 ){\vector  (   0, -1)   { 70}          }
  \put( 610,-100 ){\makebox (  90,100)[l]{$1$}          }

  \put( 700,   0 ){\makebox (   0,  0)   {$7$}          }
  \put( 700,   0 ){\circle               { 60}          }
 %\put( 700,  10 ){\makebox ( 100, 50)[b]{$0,1$}        }
 %\put( 730,   0 ){\vector  (   1,  0)   { 40}          }
  \put( 700, -30 ){\vector  (   0, -1)   { 70}          }
  \put( 710,-100 ){\makebox (  90,100)[l]{$0,1$}          }

  \put( 700,-100 ){\vector  (  -1,  0)   {700}          }
\end{picture}                                   
\end{fsK}
\end{center}
\caption{
   $(2,7)$-coded sequence state diagram
   %\label{fig:(2,7)_sm}
   }
\end{figure}

\begin{fsK}
\[
   \opD_0 = 
      \left[
         \begin{array}{*{8}{c}}
            0 & 1 & 0 & 0 & 0 & 0 & 0 & 0  \\
            0 & 0 & 1 & 0 & 0 & 0 & 0 & 0  \\
            0 & 0 & 0 & 1 & 0 & 0 & 0 & 0  \\
            0 & 0 & 0 & 0 & 1 & 0 & 0 & 0  \\
            0 & 0 & 0 & 0 & 0 & 1 & 0 & 0  \\
            0 & 0 & 0 & 0 & 0 & 0 & 1 & 0  \\
            0 & 0 & 0 & 0 & 0 & 0 & 0 & 1  \\
            1 & 0 & 0 & 0 & 0 & 0 & 0 & 0  
         \end{array}
      \right]
\hspace{2ex}
   \opD_1 = 
      \left[
         \begin{array}{*{8}{c}}
            0 & 1 & 0 & 0 & 0 & 0 & 0 & 0  \\
            0 & 0 & 1 & 0 & 0 & 0 & 0 & 0  \\
            1 & 0 & 0 & 0 & 0 & 0 & 0 & 0  \\
            1 & 0 & 0 & 0 & 0 & 0 & 0 & 0  \\
            1 & 0 & 0 & 0 & 0 & 0 & 0 & 0  \\
            1 & 0 & 0 & 0 & 0 & 0 & 0 & 0  \\
            1 & 0 & 0 & 0 & 0 & 0 & 0 & 0  \\
            1 & 0 & 0 & 0 & 0 & 0 & 0 & 0  
         \end{array}
      \right]
\hspace{2ex}
   \opD   = 
      \left[
         \begin{array}{*{8}{c}}
            0 & 1 & 0 & 0 & 0 & 0 & 0 & 0  \\
            0 & 0 & 1 & 0 & 0 & 0 & 0 & 0  \\
            1 & 0 & 0 & 1 & 0 & 0 & 0 & 0  \\
            1 & 0 & 0 & 0 & 1 & 0 & 0 & 0  \\
            1 & 0 & 0 & 0 & 0 & 1 & 0 & 0  \\
            1 & 0 & 0 & 0 & 0 & 0 & 1 & 0  \\
            1 & 0 & 0 & 0 & 0 & 0 & 0 & 1  \\
            1 & 0 & 0 & 0 & 0 & 0 & 0 & 0  
         \end{array}
      \right]
\]
\end{fsK}

\begin{eqnarray*}
   C(d,k) 
     = \log_2(\lambda_{\mathrm{max}})
     = \log_2(1.431343)
     = 0.517370
\end{eqnarray*}

The code words are \citep{proakis}{584}
\[
\begin{array}{l|l}
   u_n & code \\
   \hline
   10   & 1000     \\
   11   & 0100     \\
   011  & 00100    \\
   010  & 001000   \\
   000  & 100100   \\
   0011 & 00100100 \\
   0010 & 00001000.  
\end{array}
\]

\[
    \mathrm{efficiency} 
    = \frac{\mbox{code rate}}{C(d,k)} 
    = \frac{1/2}{0.517370} 
    = 0.9664
\]
\end{example}






\clearpage
%---------------------------------------
\section{Miller-NRZI modulation code}
\index{Miller-NRZI}
\index{Delay modulation}
%---------------------------------------
\begin{figure}[ht]
\begin{center}
\begin{fsK}
\setlength{\unitlength}{0.2mm}                  
\begin{picture}(750,200)(-200,-150)  
  \thinlines                                      
  %\graphpaper[10](0,0)(700,100)                  

  \put(-200,  10 ){\makebox (  50, 50)[b]{$u_n$}        }
  \put(-200,   0 ){\vector  (   1,  0)   {50}           }
  \put(-150, -50 ){\framebox( 100,100)   {$\uparrow2$}  }
  \put(-150, -40 ){\makebox ( 100, 80)[b]{upsampler}    }
  \put(- 50,   0 ){\vector  (   1,  0)   {50}           }
  \put( 200,  10 ){\makebox ( 100, 50)[b]{$w_n$}        }

  \put( 300,-150 ){\dashbox ( 200,200)   {  }           }
  \put( 300,-140 ){\makebox ( 200,200)[b]{NRZI coder}   }
  \put( 300, -50 ){\makebox ( 100,100)   {$+$}          }
  \put( 350,   0 ){\circle{20}                          }
  \put( 350,-100 ){\vector  (   0,  1)   { 90}          }
  \put( 360,   0 ){\line    (   1,  0)   { 10}          }
  \put( 390,   0 ){\line    (   1,  0)   {110}          }
  \put( 370, -10 ){\framebox(  20, 20)   {   }          }
  \put( 300,   0 ){\makebox ( 160, 50)   {$\mod[2]$}    }
  \put( 380, -70 ){\framebox(  40, 40)   {$z^{-1/2}$}     }
  \put( 400,-100 ){\line    (   0,  1)   { 30}          }
  \put( 400, -30 ){\line    (   0,  1)   { 30}          }
  \put( 350,-100 ){\line    (   1,  0)   { 50}          }
  \put( 500,   0 ){\vector  (   1,  0)   { 50}          }

  \put( 200,   0 ){\vector  (   1,  0)   {140}         }
  \put(   0,-150 ){\dashbox ( 200,200)   {  }           }
  \put(   0,-140 ){\makebox ( 200,200)[b]{Miller coder}         }
  \put(   0,-140 ){\makebox ( 200,200)[ ]{
        \begin{fsK}
        \begin{math}
        \begin{array}{cc|cc}
           u_{2n}' & w_{n-1} & w_n & w_{n+1} \\
           \hline
           0 & 0 & 1 & 0   \\
           0 & 1 & 0 & 0   \\
           1 & 0 & 0 & 1   \\
           1 & 1 & 0 & 1  
        \end{array}
        \end{math}
        \end{fsK}
     }}

  \put(- 50,   10 ){\makebox (  50, 50)[b]{$u_n'$}        }
  \put( 500,   10 ){\makebox (  50, 50)[b]{$y_n$}        }
\end{picture}                                   
\end{fsK}
\end{center}
\caption{
   Miller-NRZI modulation coder
   \label{fig:Miller-NRZI_model}
   }
\end{figure}


Miller-NRZI modulation coding is commonly called
\begin{liste}
   \item Miller coding
   \item Miller with precoding
   \item Delay modulation.
\end{liste}

Miller-NRZI is a concatenation of a {\em Miller coder}
(Example~\ref{ex:miller}) and an NRZI coder (Section~\ref{sec:NRZI}).
Equations governing the operation of the coder include
\begin{eqnarray*}
   y_n &=& y_{n-1} \lxor w_n \\
   y_{n+1} &=& y_n \lxor w_{n+1}.
\end{eqnarray*}

The composition of the Miller and NRZI operations produces 
the following state table:
\[
\begin{array}{cccc|ccccc}
   \mbox{input} && \mc{2}{c}{state}& & && \mc{2}{c}{output} \\
   u_{2n}'      &\vline& w_{n-1} & y_{n-1} & w_n & w_{n+1} &\vline& y_n & y_{n+1} \\
   \hline
   \hline
   0 && 0 & 0    & 1 & 0 && 1 & 1  \\
   0 && 0 & 1    & 1 & 0 && 0 & 0  \\
   0 && 1 & 0    & 0 & 0 && 0 & 0  \\
   0 && 1 & 1    & 0 & 0 && 1 & 1  \\
   \hline
   1 && 0 & 0    & 0 & 1 && 0 & 1  \\
   1 && 0 & 1    & 0 & 1 && 1 & 0  \\
   1 && 1 & 0    & 0 & 1 && 0 & 1  \\
   1 && 1 & 1    & 0 & 1 && 1 & 0  
\end{array}
\]

For each input bit $u_n$, there are two new output bits $(y_n, y_{n+1})$
and two new state bits $(w_{n+1}, y_{n+1})$.
Notice that because  
\[
\begin{array}{clrclcl}
   \mbox{old state}     &\equiv& (w_{n-1}, y_{n-1}) &=& (y_{n-1}\lxor y_{n-2}, y_{n-1}) &\equiv& f(\mbox{old output}) \\
   \mbox{current state} &\equiv& (w_{n+1}, y_{n+1}) &=& (y_{n+1}\lxor y_n, y_{n+1})     &\equiv& f(\mbox{current output})
\end{array}
\]
the output pair $(y_n, y_{n+1})$ also contains the state information 
and can therefore also be used as the labels for the state of the system.
This can be viewed as more convenient because then the output pair and
the state pair are identical.
In this case, state diagrams and trellises are easier to 
illustrate since we only have to label the states, 
while the outputs do not have to be labeled because the
output pair $(y_n, y_{n+1})$ is identical to the state pair $(y_n, y_{n+1})$.

Conversion from the state pairs to the equivalent output pairs are as follows:
\[
\begin{array}{cc|cc}
   w_{n+1} & y_{n+1} & y_n & y_{n+1} \\
   \hline
   0 & 0 & 0 & 0 \\
   0 & 1 & 1 & 1 \\
   1 & 0 & 1 & 0 \\
   1 & 1 & 0 & 1   
\end{array}
\hspace{1cm}
\begin{array}{cc|cc}
   w_{n-1} & y_{n-1} & y_{n-2} & y_{n-1} \\
   \hline
   0 & 0 & 0 & 0 \\
   0 & 1 & 1 & 1 \\
   1 & 0 & 1 & 0 \\
   1 & 1 & 0 & 1   
\end{array}
\]

Using these conversions, a new equivalent state table is as follows:

\[
\begin{array}{cccc|cccc}
   \mbox{input} &      & \mc{2}{c}{\mbox{old output}} & \mc{2}{c}{\mbox{new output}} \\
   u_{2n}'      &\vline& y_{n-2} & y_{n-1}     & y_n & y_{n+1} \\
   \hline
   \hline
   0 && 0 & 0    & 1 & 1  \\
   0 && 0 & 1    & 1 & 1  \\
   0 && 1 & 0    & 0 & 0  \\
   0 && 1 & 1    & 0 & 0  \\
   \hline
   1 && 0 & 0    & 0 & 1  \\
   1 && 0 & 1    & 1 & 0  \\
   1 && 1 & 0    & 0 & 1  \\
   1 && 1 & 1    & 1 & 0  
\end{array}
\]


A trellis diagram equivalent to this state table can be found in 
Figure~\ref{fig:Miller_trellis}. 
Notice the symmetry of the trellis.
In particular, if we flip the trellis about an imaginary center axis
while leaving the state labels undisturbed,
the same trellis results.



\begin{figure}[ht]
\begin{center}
\begin{fsL}
\setlength{\unitlength}{0.2mm}
\begin{picture}(550,320)(-100,0)
  %\graphpaper[10](0,0)(300,300)
  \put( -10 , 300 ){\makebox(0,0)[r]{state $11$}}
  \put( -10 , 200 ){\makebox(0,0)[r]{state $10$}}
  \put( -10 , 100 ){\makebox(0,0)[r]{state $01$}}
  \put( -10 ,   0 ){\makebox(0,0)[r]{state $00$}}

  \thinlines
  \put(   0 ,   0 ){\circle*{10}}
  \put(   0 , 100 ){\circle*{10}}
  \put(   0 , 200 ){\circle*{10}}
  \put(   0 , 300 ){\circle*{10}}

\multiput(0,0)(100,0){5}{
  %\thicklines
  \linethickness{1mm}
  \put        (  0,300){\line( 1,-1){100}} % state0 path1
  \put        (  0,200){\line( 1,-1){100}} % state1 path1
  \put        (  0,100){\line( 1, 1){100}} % state2 path1
  \put        (  0,  0){\line( 1, 1){100}} % state3 path1
  
  %\thinlines
  \linethickness{0.1mm}
  %\put        (  0,300){\line( 1,-3){100}} % state0 path0
  %\put        (  0,200){\line( 1,-2){100}} % state1 path0
  %\put        (  0,100){\line( 1, 2){100}} % state2 path0
  %\put        (  0,  0){\line( 1, 3){100}} % state3 path0

  \qbezier[50](  0,300)(  0,300)(100,  0)  % state0 path0
  \qbezier[50](  0,200)(  0,200)(100,  0)  % state1 path0
  \qbezier[50](  0,100)(  0,100)(100,300)  % state2 path0
  \qbezier[50](  0,  0)(  0,  0)(100,300)  % state3 path0

  \put( 100 ,   0 ){\circle*{10}}
  \put( 100 , 100 ){\circle*{10}}
  \put( 100 , 200 ){\circle*{10}}
  \put( 100 , 300 ){\circle*{10}}
}
\end{picture}                                   
\end{fsL}
\hspace{1cm}
\begin{tabular}{cl}
   $\cdots$ & $y_n=0$ \\
   ---      & $y_n=1$ 
\end{tabular}
\caption{
  Miller-NRZI trellis diagram
   \label{fig:Miller_trellis}
   }
\end{center}
\end{figure}













\if 0
\begin{figure}[ht]
\begin{center}
\begin{fsK}
\setlength{\unitlength}{0.1mm}
\begin{tabular}{cccc}
 \mc{2}{c}{$\psi_1(t)$} & \mc{2}{c}{$\psi_2(t)$} 
\\
\mc{2}{c}{
\begin{picture}(360,230)(-100,-100)
  %\graphpaper[10](0,0)(300,200)                  
  \thicklines                                      
  \put(-100 ,   0 ){\line(1,0){300} }
  \thinlines
  \put(    0 ,   0 ){\line(0,1){100} }
  \put(    0 , 100 ){\line(1,0){100} }
  \put(  100 ,   0 ){\line(0,1){100} }
  \put(  -10 , -40 ){$0$ }
  \put(   90 , -40 ){$T$ }
  \put(  -60 ,  90 ){$\frac{1}{\sqrt{T}}$ }
  \put(  210 , -12 ){$t$ }
\end{picture}                                   
}
&
\mc{2}{c}{
\begin{picture}(360,230)(-100,-100)
  %\graphpaper[10](0,0)(300,200)                  
  \thicklines                                      
  \put(-100 ,   0 ){\line(1,0){300} }
  \thinlines
  \put(    0 ,   0 ){\line(0, 1){100} }
  \put(    0 , 100 ){\line(1, 0){ 50} }
  \put(   50 , 100 ){\line(0,-1){200} }
  \put(   50 ,-100 ){\line(1, 0){ 50} }
  \put(  100 ,-100 ){\line(0,1){100} }
  \put(  -10 , -40 ){$0$ }
  \put(   40 , -40 ){$\frac{T}{2}$ }
  \put(   90 , -40 ){$T$ }
  \put(  -60 ,  90 ){$\frac{1}{\sqrt{T}}$ }
  \put(  210 , -12 ){$t$ }
\end{picture}
}                                   
\\
\\
\\
$s(t;u_n=0,y_n=3)$ & 
$s(t;u_n=0,y_n=0)$ & 
$s(t;u_n=1,y_n=1)$ & 
$s(t;u_n=1,y_n=2)$
\\
\begin{picture}(360,230)(-100,-100)
  %\graphpaper[10](0,0)(300,200)                  
  \thicklines                                      
  \put(-100 ,   0 ){\line(1,0){300} }
  \thinlines
  \put(    0 ,   0 ){\line(0,1){100} }
  \put(    0 , 100 ){\line(1,0){100} }
  \put(  100 ,   0 ){\line(0,1){100} }
  \put(  -10 , -40 ){$0$ }
  \put(   90 , -40 ){$T$ }
  \put(  -50 ,  90 ){$A$ }
  \put(  210 , -12 ){$t$ }
\end{picture}                                   
&
\begin{picture}(360,230)(-100,-100)
  %\graphpaper[10](0,0)(300,200)                  
  \thicklines                                      
  \put(-100 ,   0 ){\line(1,0){300} }
  \thinlines
  \put(    0 ,   0 ){\line(0,-1){100} }
  \put(    0 ,-100 ){\line(1, 0){100} }
  \put(  100 ,   0 ){\line(0,-1){100} }
  \put(  -10 ,  20 ){$0$ }
  \put(   90 ,  20 ){$T$ }
  \put(  -90 ,-100 ){$-A$ }
  \put(  210 , -12 ){$t$ }
\end{picture}                                   
&
\begin{picture}(360,230)(-100,-100)
  %\graphpaper[10](0,0)(300,200)                  
  \thicklines                                      
  \put(-100 ,   0 ){\line(1,0){300} }
  \thinlines
  \put(    0 ,   0 ){\line(0, 1){100} }
  \put(    0 , 100 ){\line(1, 0){ 50} }
  \put(   50 , 100 ){\line(0,-1){200} }
  \put(   50 ,-100 ){\line(1, 0){ 50} }
  \put(  100 ,-100 ){\line(0,1){100} }
  \put(  -10 , -40 ){$0$ }
  \put(   40 , -40 ){$\frac{T}{2}$ }
  \put(   90 , -40 ){$T$ }
  \put(  -60 ,  90 ){$A$ }
  \put(  210 , -12 ){$t$ }
\end{picture}                                   
&
\begin{picture}(360,230)(-100,-100)
  %\graphpaper[10](0,0)(300,200)                  
  \thicklines                                      
  \put(-100 ,   0 ){\line(1,0){300} }
  \thinlines
  \put(    0 ,   0 ){\line(0,-1){100} }
  \put(    0 ,-100 ){\line(1, 0){ 50} }
  \put(   50 ,-100 ){\line(0, 1){200} }
  \put(   50 , 100 ){\line(1, 0){ 50} }
  \put(  100 , 100 ){\line(0,-1){100} }
  \put(  -10 , -40 ){$0$ }
  \put(   40 , -40 ){$\frac{T}{2}$ }
  \put(   90 , -40 ){$T$ }
  \put(  -60 ,  90 ){$A$ }
  \put(  210 , -12 ){$t$ }
\end{picture}                                   
\end{tabular}
\end{fsK}
\end{center}
\caption{
   Miller signaling waveforms using rectangular pulse.
   \label{fig:Miller_signal}
   }
\end{figure}
\fi

