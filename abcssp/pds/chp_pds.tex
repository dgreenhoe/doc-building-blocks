%============================================================================
% Daniel J. Greenhoe
%============================================================================
%=======================================
%\chapter{Power Distance Spaces for Symbolic Sequence Analysis}
\chapter{Distance spaces}
\label{app:distance}
\label{app:dspace}
%=======================================
\qboxns
  {
    \href{http://en.wikipedia.org/wiki/Proclus}{Proclus Lycaeus}
    (\href{http://www-history.mcs.st-andrews.ac.uk/Timelines/TimelineA.html}{412 -- 485 AD}),
    Greek philosopher,
    commenting on the 
    \href{http://en.wikipedia.org/wiki/Epicureans}{Epicureans} 
    view of the \prope{triangle inequality} property.
    \index{Proclus}
    \index{quotes!Proclus}
    \footnotemark
  }
  {
   The Epicureans are wont to ridicule this theorem, 
   saying it is evident even to an ass and needs no proof;
   it is as much the mark of an ignorant man,
   they say, to require persuasion of evident truths as to believe
   what is obscure without question.\ldots
   That the present theorem is known to an ass they make out from the
   observation that, if straw is placed at one extremity of the sides,
   an ass in quest of provender will make his way along the one side
   and not by way of the two others.
  }
  \citetblt{
    \citerp{proclus}{251}
    }
%
%\qboxnps
%  {
%    \href{http://en.wikipedia.org/wiki/Eric_Temple_Bell}{Eric Temple Bell}
%    (\href{http://www-history.mcs.st-andrews.ac.uk/Timelines/TimelineG.html}{1883--1960}),
%    mathematician and author
%    \index{Bell, Eric Temple}
%    \footnotemark
%  }
%  {../common/people/bell.jpg}
%  {The cowboys have a way of trussing up a steer or a pugnacious bronco
%    which fixes the brute so that it can neither move nor think.
%    This is the hog-tie, and it is what Euclid did to geometry.}
%  \footnotetext{\begin{tabular}[t]{ll}
%    quote: & \citerp{crayshaw}{191} \\
%           & \url{http://www-groups.dcs.st-and.ac.uk/~history/Quotations/Bell.html} \\
%    image: & \url{http://www-history.mcs.st-andrews.ac.uk/PictDisplay/Bell.html}
%  \end{tabular}}
%\qboxnpqt
%  { \href{http://en.wikipedia.org/wiki/Joseph_Louis_Lagrange}{Joseph-Louis Lagrange}
%    (\href{http://www-history.mcs.st-andrews.ac.uk/Timelines/TimelineD.html}{1736--1813},
%     \href{http://www-history.mcs.st-andrews.ac.uk/BirthplaceMaps/Places/Italy.html}{Italian-French mathematician and astronomer}
%    \index{Langrange, Joseph-Louis}
%    \index{quotes!Langrange, Joseph-Louis}
%    \footnotemark
%  }
%  {../common/people/small/lagrange.jpg}
%  {Tant que l'Alg\`ebre et la G\'eom\'etrie ont \'et\'e s\'epar\'ees,
%   leurs progr\`es ont \'et\'e lents et leurs usages born\'es;
%   mais lorsque ces deux sciences se sont r\'eunies, elles vers la perfection.}
%  {As long as algebra and geometry have been separated,
%   their progress have been slow and their uses limited;
%   but when these two sciences have been united, they have lent each other mutual forces,
%   and have marched together with a rapid step towards perfection.
%  }
%  \citetblt{
%    quote:       & \citerp{lagrange1795}{271} \\
%    translation: & \citerpg{grattan1990}{254}{3764322373} \\
%%                 & \url{http://www-groups.dcs.st-and.ac.uk/~history/Quotations/Lagrange.html} \\
%    image:       & \url{http://en.wikipedia.org/wiki/Joseph_Louis_Lagrange}
%    }
%\qboxnps
%  {
%    Freeman Dyson (1923--), physicist and mathematician  %(January 1994)
%    \index{Dyson, Freeman}
%    \index{quotes!Dyson, Freeman}
%    \footnotemark
%  }
%  {../common/people/small/dyson.jpg}
%  {The bottom line for mathematicians is that the architecture has to be right.
%    In all the mathematics that I did, the essential point was to find
%    the right architecture.
%    It's like building a bridge.
%    Once the main lines of the structure are right,
%    then the details miraculously fit.
%    The problem is the overall design.}
%  \citetblt{
%    quote: & \citerp{dyson1994}{20}  \\
%    image: & \url{http://en.wikipedia.org/wiki/Image:FreemanDysonOSCON2004.jpg}
%    }

  %============================================================================
% LaTeX File
% Daniel J. Greenhoe
%============================================================================

%=======================================
%\chapter{Introduction to the structure and design series}
\chapter{Introduction}
%\addcontentsline{toc}{section}{Introduction to the structure and design series}
%=======================================

%=======================================
\section*{Mathematics as art}
%=======================================
\qboxnps
  {\href{http://en.wikipedia.org/wiki/Aristotle}{Aristotle}
   \href{http://www-history.mcs.st-andrews.ac.uk/Timelines/TimelineA.html}{384 BC -- 322 BC},
   \href{http://www-history.mcs.st-andrews.ac.uk/BirthplaceMaps/Places/Greece.html}{Greek philosopher}
   \index{Aristotle}
   \index{quotes!Aristotle}
   \footnotemark
  }
  {../common/people/small/aristotle.jpg}
  {\ldots those who assert that the mathematical sciences
     say nothing of the beautiful or the good are in error.
     For these sciences say and prove a great deal about them;
     if they do not expressly mention them, but prove attributes
     which are their results or definitions, it is not true that they tell
     us nothing about them.
     The chief forms of beauty are order and symmetry and definiteness,
     which the mathematical sciences demonstrate in a special degree.}
  \citetblt{
    %quote: & \citerc{aristotle}{paragraphs 34--35?} \\
    quote: & \citerc{aristotle_metaphysics}{Book XIII Part 3} \\
    %       & \url{http://en.wikiquote.org/wiki/Aristotle} \\
    %image: & \url{http://en.wikipedia.org/wiki/Aristotle}
    image: & \url{http://upload.wikimedia.org/wikipedia/commons/9/98/Sanzio_01_Plato_Aristotle.jpg}
    }

\qboxnpq
  {\href{http://en.wikipedia.org/wiki/Norbert_Wiener}{Norbert Wiener}
   \href{http://www-history.mcs.st-andrews.ac.uk/Timelines/TimelineG.html}{(1894--1964)},
   \href{http://www-history.mcs.st-andrews.ac.uk/BirthplaceMaps/Places/USA.html}{American mathematician}
    \index{Wiener, Norbert}
    \index{quotes!Wiener, Norbert}
    \footnotemark
  }
  {../common/people/small/wiener.jpg}
  {The musician regarded me as heavy-handed and Philistine.
   This was partly because of my actual social ineptitude and bad manners,
   but it was also due to the fact that he considered that mathematics
   by its own nature stood in direct opposition to the arts.
   On the other hand, I maintained the thesis of this book:
   that mathematics is essentially one of the arts;}
   %and I dingdonged on this theme far too much for the patience of a man
   %initially disposed to hate mathematics for its own sake.
   %Later on we got into an explicit quarrel,
   %in which we really said the unpleasant things we thought of one another,
   %and this finally cleared up into a certain degree of understanding
   %and even of a limited friendship.}
  \citetblt{
    quote: & \citerp{wiener}{65} \\
    image: & \url{http://www-history.mcs.st-andrews.ac.uk/PictDisplay/Wiener_Norbert.html}
    }


\qboxnps
  {\href{http://en.wikipedia.org/wiki/G._H._Hardy}{G.H. Hardy}
   \href{http://www-history.mcs.st-andrews.ac.uk/Timelines/TimelineG.html}{(1877--1947)},
   \href{http://www-history.mcs.st-andrews.ac.uk/BirthplaceMaps/Places/UK.html}{English mathematician}
    \index{Hardy, G.H.}
    \index{quotes!Hardy, G.H.}
    \footnotemark
  }
  {../common/people/small/hardy.jpg}
  {A mathematician, like a painter or a poet, is a maker of patterns. 
   If his patterns are more permanent than theirs, 
   it is because they are made with \emph{ideas}.}
  \citetblt{
    quote: & \citerc{hardy1940}{section 10} \\
    image: & \url{http://www-history.mcs.st-andrews.ac.uk/PictDisplay/Hardy.html}
    }




Just as Plato pointed out that rhetoric is
one of the arts\citetblt{\citerp{plato1875}{320}},
so Norbert Wiener pointed out that mathematics is also
one of the arts.
And as in all the arts, value is not primarily measured by
the measure of utility in real world applications,
but by the measure of beauty that it helps create.



%=======================================
\section*{Structure of mathematics}
%=======================================
\qboxnpqt
  {Ren\'e Descartes, philosopher and mathematician (1596--1650)
   \index{Descartes, Ren\'e}
   \index{quotes!Descartes, Ren\'e}
   \footnotemark}
  {../common/people/small/descart.jpg}
  {Je me plaisois surtout aux math\'ematiques,
    \`a cause de la certitude et de l'\'evidence de leurs raisons:
    mais je ne remarquois point encore leur vrai usage;
    et, pensant qu'elles ne servoient qu'aux arts m\'ecaniques,
    je m'\'etonnois de ce que leurs fondements \'etant si fermes et si solides,
    on n'avoit rien b\^ati dessus de plus relev\'e:}
  {I was especially delighted with the mathematics,
    on account of the certitude and evidence of their reasonings;
    but I had not as yet a precise knowledge of their true use;
    and thinking that they but contributed to the advancement of the mechanical arts,
    I was astonished that foundations, so strong and solid,
    should have had no loftier superstructure reared on them.}
  \citetblt{
    quote: & \citer{descartes_method} \\
    translation: & \citerc{descartes_method_eng}{part I, paragraph 10} \\
    image: & \url{http://en.wikipedia.org/wiki/Image:Descartes_Discourse_on_Method.png}
    }

\qboxnpqt
  { Jules Henri Poincar\'e (1854-1912), physicist and mathematician
    \index{Poincar\'e, Jules Henri}
    \index{quotes!Poincar\'e, Jules Henri}
    \footnotemark
  }
  {../common/people/small/poincare.jpg}
  {\ldots on fait la science avec des faits comme une maison avec des pierres ; 
   mais une accumulation de faits n'est pas plus une science qu'un tas de 
   pierres n'est une maison.}
  {Science is built up of facts, as a house is built of stones;
   but an accumulation of facts is no more a science than a heap of stones is a house.}
  \citetblt{
    quote:       & \citerc{poincare_sah}{Chapter IX, paragraph 7} \\
    translation: & \citerp{poincare_sah_eng}{141} \\
    image:       & \url{http://www-groups.dcs.st-and.ac.uk/~history/PictDisplay/Poincare.html}
    }


\qboxnps
  {
    Freeman Dyson (1923--2020), physicist and mathematician  %(January 1994)
    \index{Dyson, Freeman}
    \index{quotes!Dyson, Freeman}
    \footnotemark
  }
  {../common/people/small/dyson.jpg}
  {The bottom line for mathematicians is that the architecture has to be right.
    In all the mathematics that I did, the essential point was to find
    the right architecture.
    It's like building a bridge.
    Once the main lines of the structure are right,
    then the details miraculously fit.
    The problem is the overall design.}
  \citetblt{
    quote: & \citerp{dyson1994}{20}  \\
    image: & \url{http://en.wikipedia.org/wiki/Image:FreemanDysonOSCON2004.jpg}
    }

Just as in paintings and architectural works, 
the art that is mathematics is often demonstrated in the structure inherent in it.
This text tries to show some of that structure.

%=======================================
\section*{Connected concepts}
%=======================================

\qboxnpqt
  { \href{http://en.wikipedia.org/wiki/Joseph_Louis_Lagrange}{Joseph-Louis Lagrange}
    (\href{http://www-history.mcs.st-andrews.ac.uk/Timelines/TimelineD.html}{1736--1813},
     \href{http://www-history.mcs.st-andrews.ac.uk/BirthplaceMaps/Places/Italy.html}{Italian-French mathematician and astronomer}
    \index{Langrange, Joseph-Louis}
    \index{quotes!Langrange, Joseph-Louis}
    \footnotemark
  }
  {../common/people/small/lagrange.jpg}
  {Tant que l'Alg\`ebre et la G\'eom\'etrie ont \'et\'e s\'epar\'ees,
   leurs progr\`es ont \'et\'e lents et leurs usages born\'es;
   mais lorsque ces deux sciences se sont r\'eunies, elles vers la perfection.}
  {As long as algebra and geometry have been separated,
   their progress have been slow and their uses limited;
   but when these two sciences have been united, they have lent each other mutual forces,
   and have marched together with a rapid step towards perfection.
  }
  \citetblt{
    quote:       & \citerp{lagrange1795}{271} \\
    translation: & \citerpg{grattan1990}{254}{3764322373} \\
%                 & \url{http://www-groups.dcs.st-and.ac.uk/~history/Quotations/Lagrange.html} \\
    image:       & \url{http://en.wikipedia.org/wiki/Joseph_Louis_Lagrange}
    }


\qboxnps
  {\href{http://en.wikipedia.org/wiki/G._H._Hardy}{G.H. Hardy}
   \href{http://www-history.mcs.st-andrews.ac.uk/Timelines/TimelineG.html}{(1877--1947)},
   \href{http://www-history.mcs.st-andrews.ac.uk/BirthplaceMaps/Places/UK.html}{English mathematician}
    \index{Hardy, G.H.}
    \index{quotes!Hardy, G.H.}
    \footnotemark
  }
  {../common/people/small/hardy.jpg}
  {The ``seriousness" of a mathematical theorem lies,
    not in its practical consequences,
    which are usually negligible,
    but in the {\em significance} of the mathematical ideas which it connects.
    We may say, roughly, that a mathematical idea is ``significant" if it can be
    connected, in a natural illuminating way,
    with a large complex of other mathematical ideas.}
  \citetblt{
    quote: & \citerc{hardy1940}{section 11} \\
    image: & \url{http://www-history.mcs.st-andrews.ac.uk/PictDisplay/Hardy.html}
    }


Mathematics is not simply a collection of definitions and equations.
Rather it is a carefully built \emph{structure} of \emph{connected} concepts.
And the purpose of this text is to present the foundations of mathematics
in a {\em structured} and {\em connected} manner.

That is, we start with the most basic and fundamental concepts,
and build upon them other structures.
But critical to any structure is the connections between components.
As pointed out by Hardy, the value of a mathematical concept is not primarily in
it's applicability to real world problems,
but rather in how well it is connected too and helps connect other mathematical concepts.

%=======================================
\section*{Abstraction}
%=======================================
\qboxnps
  { attributed to Karl Gustav Jakob Jacobi (1804--1851), mathematician
    \index{Jacobi, Karl Gustav Jakob}
    \index{quotes!Jacobi, Karl Gustav Jakob}
    \footnotemark
  }
  {../common/people/small/jacobi.jpg}
  {Man muss immer generalisieren.\quotec \\
   \quoteo One should always generalize.}
  \citetblt{
    %quote: & \url{http://en.wikiquote.org/wiki/Gustav_Jacobi} \\
     quote: & \citerpg{davis1999}{134}{0395929687} \\
     image: & \url{http://en.wikipedia.org/wiki/Carl_Gustav_Jacobi}
    }

\qboxnpqt
  { Jules Henri Poincar\'e (1854-1912), physicist and mathematician
    \index{Poincar\'e, Jules Henri}
    \index{quotes!Poincar\'e, Jules Henri}
    \footnotemark
  }
  {../common/people/small/poincare.jpg}
  {Les math\'ematiciens n'\'etudient pas des objets, 
   mais des relations entre les objets; 
   il leur est donc indiff\'erent de remplacer ces objets par d'autres, 
   pourvu que les relations ne changent pas. 
   La mati\`ere ne leur importe pas, la forme seule les int\'eresse.}
  {Mathematicians do not study objects, but the relations between objects;
   to them it is a matter of indifference if these objects are replaced by others,
   provided that the relations do not change.
   Matter does not engage their attention,
   they are interested in form alone.}
  \citetblt{
    quote:       & \citerc{poincare_sah}{Chapter 2} \\
    translation: & \citerp{poincare_sah_eng}{20} \\
    %image:       & \url{http://en.wikipedia.org/wiki/Image:Poincare_jh.jpg}
    image:       & \url{http://www-groups.dcs.st-and.ac.uk/~history/PictDisplay/Poincare.html}
    }

\qboxnqt
  {
    Jules Henri Poincar\'e (1854-1912), physicist and mathematician
    \index{Poincar\'e, Jules Henri}
    \index{quotes!Poincar\'e, Jules Henri}
    \footnotemark
  }
  %{../common/people/small/poincare.jpg}
  {Je ne sais si je n'ai d\'ej\`a` dit quelque part que la Math\'ematique est 
  l'art de donner le m\^eme nom \`a des choses diff\'erentes. 
  Il convient que ces choses, diff\'erentes par la mati\`ere, 
  soient semblables par la forme, qu'elles puissent, 
  pour ainsi dire, se couler dans le m\^eme moule. 
  Quand le langage a \'et\'e bien choisi, on est tout \'etonn\'e 
  de voir que toutes les d\'emonstrations, faites pour un objet connu, 
  s'appliquent imm\'ediatement \`a beaucoup d'objets nouveaux ; 
  on n'a rien \`a y changer, pas m\^eme les mots, puisque les noms sont devenus les m\^emes.}
  %
  {I think I have already said somewhere that mathematics is the art
   of giving the same name to different things. 
   It is enough that these things, though differing in matter, 
   should be similar in form, to permit of their being, so to speak,
   run in the same mould.
   When language has been well chosen, one is astonished to find that all
   demonstrations made for a known object apply immediately to many new objects:
   nothing requires to be changed, not even the terms,
   since the names have become the same.}
  \citetblt{
    quote:   & \citerc{poincare_sam}{book 1, chapter 2, paragraph 20} \\
             & \url{http://fr.wikisource.org/wiki/Science_et_m\%C3\%A9thode_-_Livre_premier\%2C_\%C2\%A7_II} \\
    trans.:  & \citerp{poincare_sam_eng}{34} \\
   %image:        & \url{http://en.wikipedia.org/wiki/Image:Poincare_jh.jpg}
    }

The twentieth century was the century of abstraction in mathematics.%
\footnote{This concept of generalization so prevelant in 
twentieth century mathematics was well described by Van de Vel:
``It is typical of an axiomatic approach not to emphasize what an object \emph{\bf represents},
but rather how it \emph{\bf behaves}."---\citerp{vel1993}{3}
}
Here are some examples:\citetbl{
  \citor{peano1888} \\
  \citor{weber1893} \\
  \citor{dedekind1900} \\
  \citor{frechet1906} \\
  \citor{hausdorff1914} \\
  \citor{banach1922} \\
  \citor{vonNeumann1929} \\
  \citor{birkhoff1948}
  }
\begin{liste}
  \item In 1888, Giuseppe Peano introduced the \hie{vector space}, 
        a generalization of real functions.

  \item In 1893, Heinrich Weber introduced the algebraic \hie{field}, 
        a generalization of the real numbers and associated arithmetic.

  \item In 1900, Richard Dedekind introduced \hie{modularity}, a generalization of
        \prop{distributivity}.

  \item In 1906, Maurice Ren\'e Fr\'echet
        introduced the concepts of the \hie{abstract space} as well as the 
        \hie{metric space}, which generalize concepts in analysis involving 
        convergence.

  \item In 1914, Felix Hausdorff introduced the modern concept of the 
        \hie{topological space}, a further generalization of the 
        already very general metric space.

  \item In 1922, Stephen Banach introduced the \hie{normed linear space},
        a generalization of operations involving integration.

  \item In 1929, John von Neumann introduced the \hie{Hilbert space}, a 
        generalization of earlier work by Hilbert involving integral operations.

  \item In 1948, Garrett Birkhoff introduced the \hie{lattice},
        a generalization involving partially ordered sets.
\end{liste}

This text presents the most general and abstract structures first,
followed by progressively more specific structures.
The idea is, that we try to prove as much as possible in the most general setting;
and then we only need to prove that a specific structure is a special case of the general
structure and hence can simply \hie{inherit} the properties of that general structure
without having to prove everything all over again.


%=======================================
\section*{Concrete and specific cases}
%=======================================

\qboxnps
  {\href{http://en.wikipedia.org/wiki/Paul_Halmos}{Paul R. Halmos}
   \href{http://www-history.mcs.st-andrews.ac.uk/Timelines/TimelineG.html}{(1916--2006)},
   \href{http://www-history.mcs.st-andrews.ac.uk/BirthplaceMaps/Places/Germany.html}{Hungarian-born American mathematician}
   \index{Halmos, Paul R.}
   \index{quotes!Halmos, Paul R.}
   \footnotemark
  }
  {../common/people/small/halmos.jpg}
  {\ldots the source of all great mathematics is the special case,
    the concrete example.
    It is frequent in mathematics that every instance of a concept of seemingly
    great generality is in essence the same as a small and concrete special case.}
  \citetblt{
    quote: & \citer{halmos1985} \\
    image: & \url{http://en.wikipedia.org/wiki/Image:Paul_Halmos.jpeg}
    }



\qboxnps
  {attributed to \href{http://en.wikipedia.org/wiki/Hilbert}{David Hilbert}
   \href{http://www-history.mcs.st-andrews.ac.uk/Timelines/TimelineF.html}{(1862--1943)},
   \href{http://www-history.mcs.st-andrews.ac.uk/BirthplaceMaps/Places/Germany.html}{German mathematician}
    \index{Hilbert, David}
    \index{quotes!Hilbert, David}
    \footnotemark
  }
  {../common/people/small/hilbert.jpg}
  {The art of doing mathematics consists in finding
    that special case which contains all the germs of generality.}
  \citetblt{
    quote: & \citer{rose1988} \\
    image: & \url{http://en.wikipedia.org/wiki/Image:Hilbert.JPG}
    }

%2014jan05sunday
%Hermann Weyl:
%    ``Nevertheless I should not pass over in silence the fact that today the feeling among mathematicians is beginning to spread that the fertility of these abstracting methods is approaching exhaustion. The case is this: that all these nice general concepts do not fall into our laps by themselves. But definite concrete problems were first conquered in their undivided complexity, singlehanded by brute force, so to speak. Only afterwards the axiomaticians came along and stated: Instead of breaking the door with all your might and bruising your hands, you should have constructed such and such a key of skill, and by it you would have been able to open the door quite smoothly. But they can construct the key only because they are able, after the breaking in was successful, to study the lock from within and without. Before you can generalize, formalize, and axiomatize, there must be a mathematical substance."
%Pioneers of Representation Theory: Frobenius, Burnside, Schur, and Brauer, by Charles W. Curtis, pg 210.
%http://www.plambeck.org/archives/001272.html

\qboxnpq
  {
    Hermann Weyl (1885--1955); mathematician, theoretical physicist, and philosopher
    \index{Weyl, Hermann}
    \index{quotes!Weyl, Hermann}
    \footnotemark
  }
  {../common/people/weylhermann_wkp_free.jpg}
  {Nevertheless I should not pass over in silence the fact that today the 
   feeling among mathematicians is beginning to spread that the fertility 
   of these abstracting methods is approaching exhaustion. 
   The case is this: that all these nice general concepts do not fall into our laps by themselves. 
   But definite concrete problems were first conquered in their undivided complexity, 
   singlehanded by brute force, so to speak. Only afterwards the axiomaticians came along and stated: 
   Instead of breaking the door with all your might and bruising your hands, 
   you should have constructed such and such a key of skill, 
   and by it you would have been able to open the door quite smoothly. 
   But they can construct the key only because they are able, after the breaking in was successful, 
   to study the lock from within and without. 
   Before you can generalize, formalize, and axiomatize, there must be a mathematical substance.}
  \footnotetext{
    quote: \citePpc{weyl1935}{14}{H. Weyl, quoting himself in ``a conference on topology and abstract algebra as two ways of mathematical understanding, in 1931"}. 
    image: \url{https://en.wikipedia.org/wiki/File:Hermann_Weyl_ETH-Bib_Portr_00890.jpg}: ``This work is free and may be used by anyone for any purpose."
    }

It is general abstract concepts that allows us to easily see the connectivity
between multiple mathematical ideas and to more easily develop new ideas.
However, it is concrete specific examples that often make a general abstract concept clear
and more easily remembered in the future; and it may be argued that it is the concrete
examples that give the general abstract concepts meaning and significance.
This text provides many concrete examples that will hopefully greatly clarify
the more general concepts;
and will hopefully also help give value to the abstract concepts.

%=======================================
\section*{Applications}
%=======================================

\qboxnpq
  {
    Joseph Louis Lagrange (1736--1813), mathematician
    \index{Lagrange, Joseph Louis}
    \index{quotes!Lagrange, Joseph Louis}
    \footnotemark
  }
  {../common/people/small/lagrange.jpg}
  {I regard as quite useless the reading of large treatises of pure analysis:
    too large a number of methods pass at once before the eyes.
    It is in the works of applications that one must study them;
    one judges their ability there and one apprises the manner of making use of them.}
  \citetblt{
    quote: &  \citerp{stopple2003}{xi} \\
          %&  \url{http://www.math.okstate.edu/~wli/teach/fmq.html} \\
          %&  \url{http://www-groups.dcs.st-and.ac.uk/~history/Quotations/Lagrange.html} \\
    image: & \url{http://en.wikipedia.org/wiki/Image:Langrange_portrait.jpg}
    }

\qboxnps
  {
    \href{http://en.wikipedia.org/wiki/Eric_temple_bell}{Eric Temple Bell}
    (1883--1960), Scotish-American mathematician
    \index{Bell, Eric Temple}
    \index{quotes!Bell, Eric Temple}
    \footnotemark
  }
  {../common/people/small/bell.jpg}
  {The pursuit of pretty formulas and neat theorems
    can no doubt quickly degenerate into a silly vice,
    but so can the quest for austere generalities which are so very general indeed
    that they are incapable of application to any particular.}
  \citetblt{
    %quote: & \citer{eves1972} \\
    quote: & \citerp{bell1986}{488} \\
    image: & \url{http://www-history.mcs.st-andrews.ac.uk/PictDisplay/Bell.html}
    }



\begin{minipage}{4\tw/16}%
  \includegraphics*[width=\tw, keepaspectratio=true, clip=true]
  {../common/people/small/plimpton322.jpg}\footnotemark
\end{minipage}%
\footnotetext{\hie{Plimpton 322}: One of the most famous mathematical tablets from Ancient Babylon.
  Image source: \url{http://en.wikipedia.org/wiki/Plimpton_322}
  }%
\hfill
\begin{minipage}{10\tw/16}%
  Mathematics is a very old art form.
  Archeologists have found thousands of mathematical tablets among the ruins of the ancient
  Babylonians, buried in the sand for 4000 years but still readable (by some) today.
  Mathematics has for centuries been held in esteem because of its ability to
  solve practical problems.
\end{minipage}
Before the century of generalization---the twentieth century---and the 
century of precision---the nineteenth century---came several centuries of 
applications.
In fact, mathematics came to enjoy one of its finest hours during the
European Renaissance.
During this time, mathematics was brought to bear by so many to solve so many
extremely practical problems.
One of the most famous examples is Isaac \prop{Newton},
who demonstrated that the motion of objects
as small as an apple to as large as a planet could be accurately expressed and
(more importantly) accurately predicted by the analytical power of the calculus.
\hie{Peter the Great} of Russia, even though he himself did not
know too much about mathematics,
kept \prop{Euler} as a member of his court to work on the solution of very
practical problems. Peter the Great even referred to Euler,
at least in the beginning, as {\em My Professor}
(later, as the relationship became strained, he referred to Euler as
{\em My Cyclops}---Euler only had one good eye at the time).


\begin{figure}
\color{figcolor}
\begin{center}
\begin{fsL}
%============================================================================
% LaTeX File
% Daniel J. Greenhoe
% graphic of number of noteable mathematicians over time
%============================================================================
\psset{xunit=0.005\psxunit}%
\psset{yunit=0.003\psyunit}%
\begin{pspicture}(-1100,-200)(2100,900)
  %-------------------------------------
  % axes
  %-------------------------------------
  \psaxes[linecolor=axis,linewidth=0.75pt,yAxis=false,ticks=none,labelsep=2pt,labels=none]{<->}(0,0)(-1100,-10)(2100,750)% x-axis
  \psaxes[linecolor=axis,linewidth=0.75pt,xAxis=false,ticks=none,labelsep=2pt,labels=none]{->}(0,0)(-1100,-10)(2100,750)% y-axis
  %-------------------------------------
  % data plot
  %-------------------------------------
  \psset{linecolor=blue}%
  \rput(-1000, 0){\psline{-o}(0,0)(0,   0)}%
  \rput( -950, 0){\psline{-o}(0,0)(0,   0)}%
  \rput( -900, 0){\psline{-o}(0,0)(0,   0)}%
  \rput( -850, 0){\psline{-o}(0,0)(0,   0)}%
  \rput( -800, 0){\psline{-o}(0,0)(0,   1)}%
  \rput( -750, 0){\psline{-o}(0,0)(0,   2)}%
  \rput( -700, 0){\psline{-o}(0,0)(0,   1)}%
  \rput( -650, 0){\psline{-o}(0,0)(0,   0)}%
  \rput( -600, 0){\psline{-o}(0,0)(0,   2)}%
  \rput( -550, 0){\psline{-o}(0,0)(0,   3)}%
  \rput( -500, 0){\psline{-o}(0,0)(0,   2)}%
  \rput( -450, 0){\psline{-o}(0,0)(0,  11)}%
  \rput( -400, 0){\psline{-o}(0,0)(0,   9)}%
  \rput( -350, 0){\psline{-o}(0,0)(0,  12)}%
  \rput( -300, 0){\psline{-o}(0,0)(0,   5)}%
  \rput( -250, 0){\psline{-o}(0,0)(0,   9)}%
  \rput( -200, 0){\psline{-o}(0,0)(0,   6)}%
  \rput( -150, 0){\psline{-o}(0,0)(0,   7)}%
  \rput( -100, 0){\psline{-o}(0,0)(0,   3)}%
  \rput(  -50, 0){\psline{-o}(0,0)(0,   0)}%
  \rput(    0, 0){\psline{-o}(0,0)(0,   1)}%
  \rput(   50, 0){\psline{-o}(0,0)(0,   3)}%
  \rput(  100, 0){\psline{-o}(0,0)(0,   5)}%
  \rput(  150, 0){\psline{-o}(0,0)(0,   3)}%
  \rput(  200, 0){\psline{-o}(0,0)(0,   3)}%
  \rput(  250, 0){\psline{-o}(0,0)(0,   4)}%
  \rput(  300, 0){\psline{-o}(0,0)(0,   4)}%
  \rput(  350, 0){\psline{-o}(0,0)(0,   3)}%
  \rput(  400, 0){\psline{-o}(0,0)(0,   4)}%
  \rput(  450, 0){\psline{-o}(0,0)(0,   8)}%
  \rput(  500, 0){\psline{-o}(0,0)(0,   9)}%
  \rput(  550, 0){\psline{-o}(0,0)(0,   4)}%
  \rput(  600, 0){\psline{-o}(0,0)(0,   3)}%
  \rput(  650, 0){\psline{-o}(0,0)(0,   3)}%
  \rput(  700, 0){\psline{-o}(0,0)(0,   0)}%
  \rput(  750, 0){\psline{-o}(0,0)(0,   2)}%
  \rput(  800, 0){\psline{-o}(0,0)(0,   7)}%
  \rput(  850, 0){\psline{-o}(0,0)(0,  15)}%
  \rput(  900, 0){\psline{-o}(0,0)(0,   9)}%
  \rput(  950, 0){\psline{-o}(0,0)(0,   9)}%
  \rput( 1000, 0){\psline{-o}(0,0)(0,  14)}%
  \rput( 1050, 0){\psline{-o}(0,0)(0,   7)}%
  \rput( 1100, 0){\psline{-o}(0,0)(0,   7)}%
  \rput( 1150, 0){\psline{-o}(0,0)(0,   8)}%
  \rput( 1200, 0){\psline{-o}(0,0)(0,   6)}%
  \rput( 1250, 0){\psline{-o}(0,0)(0,  16)}%
  \rput( 1300, 0){\psline{-o}(0,0)(0,  10)}%
  \rput( 1350, 0){\psline{-o}(0,0)(0,   9)}%
  \rput( 1400, 0){\psline{-o}(0,0)(0,  13)}%
  \rput( 1450, 0){\psline{-o}(0,0)(0,  21)}%
  \rput( 1500, 0){\psline{-o}(0,0)(0,  40)}%
  \rput( 1550, 0){\psline{-o}(0,0)(0,  40)}%
  \rput( 1600, 0){\psline{-o}(0,0)(0,  59)}%
  \rput( 1650, 0){\psline{-o}(0,0)(0,  81)}%
  \rput( 1700, 0){\psline{-o}(0,0)(0,  74)}%
  \rput( 1750, 0){\psline{-o}(0,0)(0,  99)}%
  \rput( 1800, 0){\psline{-o}(0,0)(0, 144)}%
  \rput( 1850, 0){\psline{-o}(0,0)(0, 302)}%
  \rput( 1900, 0){\psline{-o}(0,0)(0, 614)}%
  \rput( 1950, 0){\psline{-o}(0,0)(0, 653)}%
  \rput( 1975, 0){\psline{-o}(0,0)(0, 435)}%
  \rput( 2000, 0){\psline{-o}(0,0)(0, 154)}%
  \rput( 2006, 0){\psline{-o}(0,0)(0, 112)}%
       %    |                       |_____ number of notable mathematicians
       %    |_____________________________ year
  %-------------------------------------
  % number of mathematician labels (above lollipops)
  %-------------------------------------
  \scriptsize%
  \uput[90](-1000,   0){$  0$}%
  \uput[90]( -950,   0){$  0$}%
  \uput[90]( -900,   0){$  0$}%
  \uput[90]( -850,   0){$  0$}%
  \uput[90]( -800,   1){$  1$}%
  \uput[90]( -750,   2){$  2$}%
  \uput[90]( -700,   1){$  1$}%
  \uput[90]( -650,   0){$  0$}%
  \uput[90]( -600,   2){$  2$}%
  \uput[90]( -550,   3){$  3$}%
  \uput[90]( -500,   2){$  2$}%
  \uput[90]( -450,  11){$ 11$}%
  \uput[90]( -400,   9){$  9$}%
  \uput[90]( -350,  12){$ 12$}%
  \uput[90]( -300,   5){$  5$}%
  \uput[90]( -250,   9){$  9$}%
  \uput[90]( -200,   6){$  6$}%
  \uput[90]( -150,   7){$  7$}%
  \uput[90]( -100,   3){$  3$}%
  \uput[90](  -50,   0){$  0$}%
  \uput[90](    0,   1){$  1$}%
  \uput[90](   50,   3){$  3$}%
  \uput[90](  100,   5){$  5$}%
  \uput[90](  150,   3){$  3$}%
  \uput[90](  200,   3){$  3$}%
  \uput[90](  250,   4){$  4$}%
  \uput[90](  300,   4){$  4$}%
  \uput[90](  350,   3){$  3$}%
  \uput[90](  400,   4){$  4$}%
  \uput[90](  450,   8){$  8$}%
  \uput[90](  500,   9){$  9$}%
  \uput[90](  550,   4){$  4$}%
  \uput[90](  600,   3){$  3$}%
  \uput[90](  650,   3){$  3$}%
  \uput[90](  700,   0){$  0$}%
  \uput[90](  750,   2){$  2$}%
  \uput[90](  800,   7){$  7$}%
  \uput[90](  850,  15){$ 15$}%
  \uput[90](  900,   9){$  9$}%
  \uput[90](  950,   9){$  9$}%
  \uput[90]( 1000,  14){$ 14$}%
  \uput[90]( 1050,   7){$  7$}%
  \uput[90]( 1100,   7){$  7$}%
  \uput[90]( 1150,   8){$  8$}%
  \uput[90]( 1200,   6){$  6$}%
  \uput[90]( 1250,  16){$ 16$}%
  \uput[90]( 1300,  10){$ 10$}%
  \uput[90]( 1350,   9){$  9$}%
  \uput[90]( 1400,  13){$ 13$}%
  \uput[90]( 1450,  21){$ 21$}%
  \uput[90]( 1500,  40){$ 40$}%
  \uput[90]( 1550,  40){$ 40$}%
  \uput[90]( 1600,  59){$ 59$}%
  \uput[90]( 1650,  81){$ 81$}%
  \uput[90]( 1700,  74){$ 74$}%
  \uput[90]( 1750,  99){$ 99$}%
  \uput[90]( 1800, 144){$144$}%
  \uput[90]( 1850, 302){$302$}%
  \uput[135]( 1900, 614){$614$}%
  \uput[45]( 1950, 653){$653$}%
  \uput[45]( 1975, 435){$435$}%
  \uput[60]( 2000, 154){$154$}%
  \uput[45]( 2006, 112){$112$}%
       % |      |    |     |____ number of notable mathematicians
       % |      |    |__________ number of notable mathematicians
       % |      |_______________ year
       % |______________________ place label above lollipops
  %-------------------------------------
  % year labels (under y-axis)
  %-------------------------------------
  \uput[-90](-1000,0){$-1000$}%
  \uput[-90]( -350,0){$ -350$}%
  \uput[-90](    0,0){$    0$}%
  \uput[-90]( 1000,0){$ 1000$}%
  \uput[-90]( 1450,0){$ 1450$}%
  \uput[-90]( 1600,0){$ 1600$}%
  \uput[-90]( 1800,0){$ 1800$}%
  \uput[-90]( 2000,0){$ 2000$}%
  \uput[-90]( 1600,0){$ 1600$}%
    %
%    \rput(-1000,30){\makebox(0,0)[b] {0}}%
%    \rput(-350,42){\makebox(0,0)[b] {12}}%
%    \rput(0,31){\makebox(0,0)[b] {1}}%
%    \rput(1000,44){\makebox(0,0)[b] {14}}%
%    \rput(1450,51){\makebox(0,0)[b] {21}}%
%    \rput(1600,89){\makebox(0,0)[b] {59}}%
%    \rput(1800,174){\makebox(0,0)[b] {144}}%
%    \rput(1850,342){\makebox(0,0)[b] {302}}%
%    \rput(1950,683){\makebox(0,0)[b] {653}}%
%    \rput(1975,465){\makebox(0,0)[bl] {435}}%
%    \rput(2000,184){\makebox(0,0)[bl] {154}}%
%    \rput(2006,122){\makebox(0,0)[bl] {112}}%
%  %
  %-------------------------------------
  % inventions
  %-------------------------------------
  \color{red}%
  \psset{linecolor=red,linewidth=0.75pt}%
 %\rput[r]( 980,800){\blank[10mm] available}%
  \rput[r]( 980,800){IBM PC}%
 %\rput[r]( 980,700){\blank[10mm] available}%
  \rput[r]( 980,700){calculator}%
  \rput[r]( 980,600){second industrial revolution}%
  \rput[r]( 980,500){industrial revolution}%
  \rput[r]( 980,400){slide rule invented}%
 %\rput[r]( 980,400){\blank[10mm] invented}%
 %\rput[r]( 980,300){\blank[10mm] invented}%
  \rput[r]( 980,300){Gutenberg printing press invented}%
%
  \psline[linestyle=dashed](980,800)(1980,800)(1980,500)%
  \psline[linestyle=dashed](980,700)(1970,700)(1970,600)%
  \psline[linestyle=dashed](980,600)(1850,600)(1850,500)%
  \psline[linestyle=dashed](980,500)(1800,500)(1800,400)%
  \psline[linestyle=dashed](980,400)(1625,400)(1625,300)%
  \psline[linestyle=dashed](980,300)(1450,300)(1450,200)%
%
%  \rput( 1980,800){\psline(0,-1){50}} \rput(1000,800){\psline(1,0){980} }  \rput( 980,800){\makebox(0,0)[r] {IBM PC}}%
%  \rput( 1970,700){\psline(0,-1){50}} \rput(1000,700){\psline(1,0){970} }  \rput( 980,700){\makebox(0,0)[r] {calculator}}%
%  \rput( 1850,600){\psline(0,-1){50}} \rput(1000,600){\psline(1,0){850} }  \rput( 980,600){\makebox(0,0)[r] {second industrial revolution}}%
%  \rput( 1800,500){\psline(0,-1){50}} \rput(1000,500){\psline(1,0){800} }  \rput( 980,500){\makebox(0,0)[r] {industrial revolution}}%
%  \rput( 1625,400){\psline(0,-1){50}} \rput(1000,400){\psline(1,0){625} }  \rput( 980,400){\makebox(0,0)[r] {slide rule invented}}%
%  \rput( 1450,300){\psline(0,-1){50}} \rput(1000,300){\psline(1,0){450} }  \rput( 980,300){\makebox(0,0)[r] {Gutenberg printing press invented}}%
  %-------------------------------------
  % time periods
  %-------------------------------------
  \rput( 400, -50){\psline(0,0)(0,-200)}% 
  \rput(1300, -50){\psline(0,0)(0,-200)}% 
  \rput(1650, -50){\psline(0,0)(0,-200)}% 
  \rput( 850,-150){Middle Ages / ``Dark Ages"}%
  \rput(1475,-150){Renaissance}%
\end{pspicture}%

\end{fsL}
\end{center}
\caption{
   Number of notable mathematicians alive over time
   \label{fig:intro_timeline}
   }
\end{figure}

But it is not that applications are no longer important to mathematics,
but rather applications drive mathematics forward.
Evidence of this hypothesis is given in 
\prefpp{fig:intro_timeline}.\footnote{Data for \prefpp{fig:intro_timeline} extracted from \\
  \url{http://www-history.mcs.st-andrews.ac.uk/Timelines/WhoWasThere.html}}
This graph shows the number of ``notable" mathematicians alive during the last 3000 years;
And a ``notable mathematician" is here defined as one whose name appears
in \hie{Saint Andrew's University's} \hie{Who Was There} 
website.\footnote{\url{http://www-history.mcs.st-andrews.ac.uk/Timelines/WhoWasThere.html}}
Note the following:
\begin{liste}

\item The number of mathematicians starts to exponentially increase at 
about the time of Gutenberg's invention of the printing press---
that is, when information of discoveries and results could be widely and economically 
circulated.\footnote{This point is also made by Resnikoff and Wells in\\
\citerp{resnikoff}{9}.}

\item There is another increase after the invention of the slide rule in the early 1600s---
that is, when computational power increased.

\item There are huge increases around the time of the first and second industrial revolutions---
that is, when there were many {\bf applications} that called for mathematical solutions.

\item After the invention of the pocket scientific calculator in 1972 
and home IBM PC in 1981---machines that could 
often make hard-core mathmatical analysis unnecessary in real-world applications---
there was a huge drop in the number of mathematicians.

\end{liste}

The point here is, that however little some mathematicians may think of 
real-world applications,
historically it would seem that applications are critical to 
how well mathematics thrives and develops. 
When mathematics is needed for the development of applications, 
mathematics prospers.
But when mathematics is not viewed as critical to those applications
(such as after the introduction of the personal computer), 
mathematics withers.


%=======================================
\section*{Writing style}
%=======================================
\qboxnpqt
  { %\href{http://www-history.mcs.st-andrews.ac.uk/Biographies/Poincare.html}{Jules Henri Poincar\'e} 
    \href{http://en.wikipedia.org/wiki/Henri_Poincar\%C3\%A9}{Jules Henri Poincar\'e} 
    \href{http://www-history.mcs.st-andrews.ac.uk/Timelines/TimelineF.html}{(1854--1912)}, 
    \href{http://www-history.mcs.st-andrews.ac.uk/BirthplaceMaps/Places/France.html}{French physicist and mathematician}
    \index{Poincar\'e, Jules Henri}
    \index{quotes!Poincar\'e, Jules Henri}
    \footnotemark
  }
  {../common/people/small/poincare.jpg}
  {Ainsi, la logique et l'intuition ont chacune leur r\^ole n\'ecessaire.
    Toutes deux sont indispensables.
    La logique qui peut seule donner la certitude est l'instrument de la d\'emonstration:
    l'intuition est l'instrument de l'invention.}
  {Thus, logic and the intuition each have their necessary r\^ole.
    Each is indispensable.
    Logic, which alone can give certainty, is the instrument of demonstration;
    intuition is the instrument of invention.}
  \citetblt{
    quote:       & \citerc{poincare_vos}{chapter 1 \S V}  \\
    translation: & \citerp{poincare_vos_e}{23} \\
    image:       & \url{http://en.wikipedia.org/wiki/Image:Poincare_jh.jpg}
    }

  \paragraph{Dual structure writing style.}
  Mathematician \href{http://en.wikipedia.org/wiki/Steenrod}{Norman E. Steenrod}
  proposed a mathematical style of writing
  in which there is a distinction between
  ``the {\em formal} or {\em logical} structure consisting of definitions, theorems,
  and proofs, and the complementary
  {\em informal} or {\em introductory} material consisting of motivations, analogies,
  examples, and metamathematical explanations."\citep{steenrod}{1}
  This is the style that largely characterizes this text.
  The overall development is in the definition-lemma-theorem style,
  which seems to be a reliable way to impose rigor on the development.
  The motivation and general discussion are in the more informal style,
  which helps give an intuitive understanding.

\qboxnps
  {
    \href{http://en.wikipedia.org/wiki/Carl_Friedrich_Gauss}{Karl Friedrich Gauss}
    (1777--1855), German mathematician
    \index{Gauss, Karl Friedrich}
    \index{quotes!Gauss, Karl Friedrich}
    \footnotemark
  }
  {../common/people/small/gauss.jpg}
  {You know that I write slowly.
    This is chiefly because I am never satisfied until I have said as much as
    possible in a few words,
    and writing briefly takes far more time than writing at length.}
  \citetblt{
    quote: & \citerp{simmons2007}{177} \\
    image: & \url{http://en.wikipedia.org/wiki/Karl_Friedrich_Gauss}
    }


  \paragraph{Informal structure.}
  More specifically, my guidelines for the informal portions of text are as follows:
  \begin{dingautolist}{"C0}
    \item The text should be terse rather than verbose.
    %\item The text should not be cluttered with supporting background
    %      mathematical material.
    %      Most all such material is placed in appendices at the end of the text.
    \item Key points are often indicated by enclosure in boxes---
          red double boxes for definitions, blue single boxes for theorems,
          curved-corner boxes for examples.
          This makes parsing key points faster and more efficient---
          allowing you to easily distinguish between key results, detailed proofs,
          and general discussion.
    \item Examples will be explicitly worked out, giving readers
          confidence that they really understand the material.
  \end{dingautolist}


\qboxnps
  {
    \href{http://en.wikipedia.org/wiki/Carl_Friedrich_Gauss}{Karl Friedrich Gauss}
    (1777--1855), German mathematician
    \index{Gauss, Karl Friedrich}
    \index{quotes!Gauss, Karl Friedrich}
    \footnotemark
  }
  %{../common/people/small/gauss1828.jpg}
  {../common/people/gauss.jpg}
  {I mean the word proof not in the sense of lawyers,
    who set two half proofs equal to a whole one,
    but in the sense of the mathematician, where \textonehalf proof = 0
    and it is demanded for proof that every doubt becomes impossible.}
  \citetblt{
    quote: & \citerp{simmons2007}{177} \\
    %image: & \url{http://www-history.mcs.st-andrews.ac.uk/PictDisplay/Gauss.html}
    image: & \url{http://en.wikipedia.org/wiki/Karl_Friedrich_Gauss}
    }

\qboxnpq
  {
    \href{http://en.wikipedia.org/wiki/Carl_Gustav_Jakob_Jacobi}{Carl Gustav Jacob Jacobi}
    (1804--1851), Jewish-German mathematician
    \index{Jacobi, Carl Gustav Jacob}
    \index{quotes!Jacobi, Carl Gustav Jacob}
    \footnotemark
  }
  {../common/people/small/jacobic.jpg}
  {Dirichlet alone, not I, nor Cauchy, nor Gauss knows what a completely rigorous proof is.
   Rather we learn it first from him.
   When Gauss says he has proved something it is clear;
   when Cauchy says it, one can wager as much pro as con;
   when Dirichlet says it, it is certain.}
  %{Dirichlet alone, not I, nor Cauchy, nor Gauss knows what a completely rigorous proof is,
  % and we are learning it from him.
  % When Gauss says he has proved something, it is very probable to me;
  % when Cauchy says it, it is more likely than not,
  % when Dirichlet says it, it is \emph{proved}.}
  \citetblt{
    quote: & \url{http://lagrange.math.trinity.edu/aholder/misc/quotes.shtml} \\
           & \citerp{schubring2005}{558} \\
          %& \citerp{biermann1988}{46} \\
    image: & \url{http://en.wikipedia.org/wiki/Carl_Gustav_Jakob_Jacobi}
    }

\qboxnps
  {
    \href{http://en.wikipedia.org/wiki/Enrico_Fermi}{Enrico Fermi}
    \href{http://www-history.mcs.st-andrews.ac.uk/Timelines/TimelineG.html}{(1777--1855)},
    \href{http://www-history.mcs.st-andrews.ac.uk/BirthplaceMaps/Places/Italy.html}{Italian physicist}
    \index{Fermi, Enrico}
    \index{quotes!Fermi, Enrico}
    \footnotemark
  }
  {../common/people/small/fermi.jpg}
  {If it is true, it can be proved.}
  \citetblt{
    quote: & \citerp{benedetto}{85} \\
    image: & \url{http://www-history.mcs.st-andrews.ac.uk/Mathematicians/Fermi.html}
    }


%Aubrey, John (1626-1697)
%[About Thomas Hobbes:]
%He was 40 years old before he looked on geometry; which happened accidentally. Being in a gentleman's library, Euclid's Elements lay open, and "twas the 47 El. libri I" [Pythagoras' Theorem]. He read the proposition . "By God", sayd he, "this is impossible:" So he reads the demonstration of it, which referred him back to such a proposition; which proposition he read. That referred him back to another, which he also read. Et sic deinceps, that at last he was demonstratively convinced of that trueth. This made him in love with geometry.
%In O. L. Dick (ed.) Brief Lives, Oxford: Oxford University Press, 1960, p. 604.
%http://math.furman.edu/~mwoodard/ascquota.html

  \paragraph{Formal structure.}
  Guidelines in writing the proofs include:
  \begin{dingautolist}{"C0}
    \item Proofs should be very detailed (verbose rather than terse).
          This could tend to obscure the points the proofs are proving
          (``can't see the forest for all of the trees");
          but placing these points in boxes helps remedy this situation
          (see previous discussion).
    \item Every step that relies on another definition or theorem
          should be justified immediately to the right of the step,
          with cross referencing information.
          Many times these references include page numbers for more convenient access.
          If you happen to be viewing the pdf version of this text,
          then you can simply click on a particular reference and your pdf viewer will
          take you immediately to that location.
    \item Proofs, whenever possible, should be {\em direct proofs}
          clearly linking statement to statement with nothing more than
          the equality relation ($=$).
          They should not be cluttered with extensive explanations
          between steps. Let the mathematics speak for itself instead
          of me constantly jumping in waving my hands about in an effort to
          make clear what the mathematical equations make clear
          themselves.
    \item The development of proofs should rely fundamentally on
          propositional logic(logical AND $\land$, logical OR $\lor$, and logical NOT $\lnot$)
          and predicate logic (``for all" universal quantifier $\forall$ and
          ``there exists" existential quantifier $\exists$.)
  \end{dingautolist}

%Although countless practical problems were solved in the eighteenth century
%by the mathematical framework available,
%some began to see that this framework, howbeit sufficiently useful,
%was not sufficiently rigorous to withstand the scrutiny of rigorous logic
%and not sufficiently general to  support the most general functions.
%The nineteenth century was the century of analysis for mathematics.
%Thus with the arrival of the nineteenth century came also the arrival of
%the analytic mathematical samurais such as Cantor, Dirichlet, Lebesgue, and Weirstrass.
%These were men given to detail and who established mathematics on a detailed
%and solid mathematical
%footing through intense and thorough analysis.
%In the end Cantor lost his mind; but the world gained a mindset
%of a solid mathematical
%foundation on which new mathematical structures could be built in the
%twentieth century.



%---------------------------------------
\section*{External reference support}
%---------------------------------------
\qboxnps
  {
    \href{http://en.wikipedia.org/wiki/Niels_Henrik_Abel}{Niels Henrik Abel}
   (1802--1829),
   \href{http://www-history.mcs.st-andrews.ac.uk/BirthplaceMaps/Places/Russia.html}{Nowegian mathematician}
    \index{Abel, Niels Henrik}
    \index{quotes!Abel, Niels Henrik}
    \footnotemark
  }
  {../common/people/small/abel.jpg}
  {It appears to me that if one wants to make progress in mathematics,
    one should study the masters and not the pupils.}
  \citetblt{
    quote: & \citerp{simmons2007}{187} \\
    image: & \url{http://en.wikipedia.org/wiki/Image:Niels_Henrik_Abel.jpg}
    }
%
I have tried to include extensive reference information throughout the text.
These refences appear as brief footnotes at the bottom of the page where
the reference is relevant.
More information about a reference is given in the \hie{bibliography} at the
end of this text.
In most cases, each reference has an associated web link.
That link contains more information about the reference---in some
cases this means a full text download, in other cases a partial viewing,
in some cases the contents of the refence can be searched online,
and in some cases the location where that reference can be found
in libraries scattered throughout the world.
If you happen to be viewing the pdf version of this text,
then you can simply click on a particular footnote number or
footnote reference and your pdf viewer will
immediately take you to that location.
In providing easier access to references,
you the reader can ``study the masters and not the pupils."


%---------------------------------------
\section*{Historical viewpoint}
%---------------------------------------
\qboxnps
  {
    \href{http://en.wikipedia.org/wiki/Yoshida_Kenko}{Yoshida Kenko (Urabe Kaneyoshi)}
    (1283? -- 1350?),
    Japanese author and Buddhist monk
    \index{Kenko, Yoshida}  \index{Kaneyoshi, Urabe}
    \index{quotes!Kenko, Yoshida}  \index{quotes!Kaneyoshi, Urabe}
    \footnotemark
  }
  {../common/people/small/kenko.jpg}
  %{The pleasantest of all diversions is to sit alone under the lamp,
  % a book spread out before you,
  % and to make friends with people of a distant past you have never known. (Keene translation page 12)
  %}
  {To sit alone in the lamplight with a book spread out before you,
   and hold intimate converse with men of unseen generations---
   such is a pleasure beyond compare.}
  \citetblt{
    quote: & \citer{kenko_sansom} \\
    image: & \url{http://en.wikipedia.org/wiki/Yoshida_Kenko}
    }


\qboxnpq
  {
    \href{http://en.wikipedia.org/wiki/Niccol\%C3\%B2_Machiavelli}{Niccol\`o Machiavelli}
    (1469--1527), Italian political philosopher,
    in a 1513 letter to friend Francesco Vettori.
    \index{Machiavelli, Niccol\`o}
    \index{quotes!Machiavelli, Niccol\`o}
    \footnotemark
  }
  {../common/people/small/mach.jpg}
  {When evening comes, I return home and go to my study.
    On the threshold I strip naked, taking off my muddy, sweaty workaday clothes,
    and put on the robes of court and palace,
    and in this graver dress I enter the courts of the ancients and am welcomed by them,
    and there I taste the food that alone is mine, and for which I was born.
    And there I make bold to speak to them and ask the motives of their actions,
    and they, in their humanity reply to me.
    And for the space of four hours I forget the world, remember no vexation,
    fear poverty no more, tremble no more at death;
    I pass indeed into their world.}
  \citetblt{
    quote: & \citerp{machiavelli}{139?} \\
    image: & \url{http://en.wikipedia.org/wiki/Niccol\%C3\%B2_Machiavelli}
    }

\qboxnps
  {
    \href{http://www-history.mcs.st-andrews.ac.uk/Biographies/Poincare.html}
         {Henri Poincar\'e}
    \index{Henri Poincar\'e}
    \index{quotes!Henri Poincar\'e}
    (\href{http://www-history.mcs.st-andrews.ac.uk/Timelines/TimelineF.html}{1854--1912}),
    \href{http://www-history.mcs.st-andrews.ac.uk/BirthplaceMaps/Places/France.html}{French}
    mathematician and physicist,
    in an address to the Fourth International Congress of Mathematicians at Rome, 1908
    \footnotemark
  }
  {../common/people/small/poincare.jpg}
  {The true method of foreseeing the future of mathematics is to study its history
   and its actual state.}
  \citetblt{
    quote: & \citerp{bottazzini}{1} \\
    %quote: & \citer{poincare_sam} ??????  maybe not  \\
    image: & \url{http://www-groups.dcs.st-and.ac.uk/~history/PictDisplay/Poincare.html}
    }

I have tried to include information that gives readers an understanding of where
concepts came from in the history of mathematics.
In some cases, the original source is given as a reference.
In some cases, you can download the source text for free if you have an internet
connection. Web link addresses are provided in the bibliography.
Besides references,
sometimes I have also included quotes from famous mathematicians that
influenced mathematical thinking at the time when a mathematical idea was developing.
These quotes normally appear as
\shadowbox{``shadow boxes"} in the text.

In providing such information from notable mathematicians, you the reader can
``make bold to speak to them and ask the motives of their actions,
        and they, in their humanity reply"---and all this without having to change clothes.

%---------------------------------------
\section*{Hyper-link support}
%---------------------------------------
As already mentioned, most of the references in the bibliography feature web links
for further information and in some cases full text download capability.
Also web links are given for most of the images of
famous mathematicians appearing in this book.
%Lastly, this book itself is available at \\
%  \url{http://banyan.cm.nctu.edu.tw/~dgreenhoe/msd/index.html}\\

The pdf (portable document format) version of this text has
been embedded with an extensive number of hyper-links.
These hyperlinks are highlighted by a yellow box.
For example, in the table of contents,
you can click on a chapter or section title and immediately jump to that chapter or section.
In a proof statement, you can click on a reference to a previous result or definition
and jump to that result.
In a footnote reference, you can click on that reference and immediately jump
to the bibliography for more information about that reference.
In the bibliography, most of the references have links to the world wide web.
If your computer is online and you click on one of those links,
your default browser will display that web page after securing permission from you to do so.




  %============================================================================
% LaTeX File
% Daniel J. Greenhoe
%============================================================================

%======================================
\chapter{Distance Spaces}
\label{chp:dspace}
%======================================

%  \begin{figure}[t]
%  \[\begin{array}{*{5}{>{\ds}c}}
%       && \fcolorbox{blue}{bg_blue}{\parbox[c]{3\tw/16}{\centering topological space}}
%    \\ && \\setUparrow & \Nwarrow
%    \\ \fcolorbox{blue}{bg_blue}{\parbox[c]{3\tw/16}{\centering vector space}}
%       && \fcolorbox{red}{bg_red}{\parbox[c]{3\tw/16}{\centering metric space}}
%       && \fcolorbox{blue}{bg_blue}{\parbox[c]{3\tw/16}{\centering measure space}}
%    \\ & \Nwarrow & \\setUparrow && \\setUparrow
%    \\ && \fcolorbox{blue}{bg_blue}{\parbox[c]{3\tw/16}{\centering normed linear space}}
%       && \fcolorbox{blue}{bg_blue}{\parbox[c]{3\tw/16}{\centering probability space}}
%    \\ &  \Nearrow  & & \Nwarrow
%    \\ \fcolorbox{blue}{bg_blue}{\parbox[c]{3\tw/16}{\centering inner-product space}}
%       &&&&
%       \fcolorbox{blue}{bg_blue}{\parbox[c]{3\tw/16}{\centering Banach space}}
%    \\ & \Nwarrow && \Nearrow
%    \\ && \fcolorbox{blue}{bg_blue}{\parbox[c]{3\tw/16}{\centering Hilbert space}}
%    \\ &\Nearrow & & \Nwarrow
%    \\ \fcolorbox{blue}{bg_blue}{\parbox[c]{3\tw/16}{\centering$\spII$}}
%       &&&&
%       \fcolorbox{blue}{bg_blue}{\parbox[c]{3\tw/16}{\centering$\spLL$}}
%    \\ & \Nwarrow && \Nearrow
%    \\ && \fcolorbox{blue}{bg_blue}{\parbox[c]{3\tw/16}{\centering$0$}}
%  \end{array}\]
%  \end{figure}


%\qboxnps
%  {
%    \href{http://en.wikipedia.org/wiki/Eric_Temple_Bell}{Eric Temple Bell}
%    (\href{http://www-history.mcs.st-andrews.ac.uk/Timelines/TimelineG.html}{1883--1960}),
%    mathematician and author
%    \index{Bell, Eric Temple}
%    \footnotemark
%  }
%  {../common/people/bell.jpg}
%  {The cowboys have a way of trussing up a steer or a pugnacious bronco
%    which fixes the brute so that it can neither move nor think.
%    This is the hog-tie, and it is what Euclid did to geometry.}
%  \footnotetext{\begin{tabular}[t]{ll}
%    quote: & \citerp{crayshaw}{191} \\
%           & \url{http://www-groups.dcs.st-and.ac.uk/~history/Quotations/Bell.html} \\
%    image: & \url{http://www-history.mcs.st-andrews.ac.uk/PictDisplay/Bell.html}
%  \end{tabular}}

A \structe{distance space} \xref{def:dspace} can be defined as a \structe{metric space} \xref{def:metric}
without the \prope{triangle inequality} constraint.
Much of the material in this section about \structe{distance space}s is
standard in \structe{metric space}s.
However, this paper works through this material again to demonstrate ``how far we can go", and can't go, 
without the \prope{triangle inequality}.

%======================================
\section{Fundamental structure of distance spaces}
%======================================
%======================================
\subsection{Definitions}
%======================================
%---------------------------------------
\begin{definition}
\footnote{
  \citePpc{menger1928}{76}{``Abstand $a$ $b$ definiert ist\ldots" (distance from $a$ to $b$ is defined as\ldots")},
  \citePpc{wilson1931a}{361}{\textsection1., ``distance", ``semi-metric space"},
  \citePp{blumenthal1938}{38},
  \citerpgc{blumenthal1953}{7}{0828402426}{``{\scshape Definition 5.1.} A distance space is called semimetric provided\ldots"},
  \citePpc{galvin1984}{67}{``distance function"},
  \citerpgc{laos1998}{118}{9810231806}{``distance space"},
  \citerpgc{khamsi2001}{13}{0471418250}{``semimetric space"},
  \citePpc{bessenyei2014}{2}{``semimetric space"}, %{{\scshape Conventions And Basic Notions}},
  \citerpgc{deza2014}{3}{3662443422}{``\textbf{distance} (or \textbf{dissimilarity})"}
  }
\label{def:distance}
\label{def:dspace}
%---------------------------------------
%Let $\setX$ be a set and $\Rnn$ the set of non-negative real numbers.
\defboxt{
  A function $\hxs{\distancen}$ in the set $\clF{\setX\times\setX}{\R}$\ifsxref{relation}{def:clFxy} is a \fnctd{distance} if
  \\\indentx$\begin{array}{F rcl CDD}
        1. & \distance{x}{y} &\ge& 0               & \forall x,y   \in\setX & (\prope{non-negative})   & and 
      \\2. & \distance{x}{y} &=  & 0  \iff x=y     & \forall x,y   \in\setX & (\prope{nondegenerate})  & and 
      \\3. & \distance{x}{y} &=  & \distance{y}{x} & \forall x,y   \in\setX & (\prope{symmetric})      & 
    \end{array}$
  \\The pair $\dspaceX$ is a \structd{distance space} if $\distancen$ is a \fncte{distance} on a set $\setX$.
  \\A \fncte{distance} is also called a \fnctd{dissimilarity}.
  }
\end{definition}

%In a \structe{metric space} \xref{def:metric}\index{space!metric},
%it is sometimes useful to know the maximum distance between any two points in the set.
%This maximum distance is called the \fncte{diameter} of the set 
%(\pref{def:diam}, next definition).
%Some \structe{distance space}s \xref{def:distance} and all \structe{metric space}s \xref{def:metric} induce 
%\structe{topological space}s \xref{def:topology}.
%However the \fncte{set function} \xref{def:setf}
%\fncte{set diameter} (next definition) and the related property of \prope{boundedness} \xref{def:bounded} 
%are fundamentally \structe{distance space} concepts, 
%not topological ones.\footnote{in \structe{metric space}: \citerpg{munkres2000}{121}{0131816292}}
%---------------------------------------
\begin{definition}
\label{def:diam}
\footnote{
  in \structe{metric space}:
  %\citerpg{davis2005}{28}{0071243399} \\
  \citerp{hausdorff1937e}{166},
  \citerp{copson1968}{23},
  \citerp{michel1993}{267},
  \citerpg{molchanov2005}{389}{185233892X}
  }
%---------------------------------------
Let $\dspaceX$ be a \structe{distance space} and $\psetX$ be the \structe{power set} of $\setX$ \xref{def:pset}.
\defboxt{
  The \vald{diameter} in $\dspaceX$ of a set $\setA\in\psetX$ is
  \quad$\ds
  \diam\setA \eqd
    \brp{\begin{array}{ll}
      0                            & \text{for } \setA=\emptyset \\
      \sup\set{\distance{x}{y}}{x,y\in\setA} & \text{otherwise}
    \end{array}}$
  }
\end{definition}

%The \fncte{diameter} of a set is a special case of a  
%\prefpp{sec:topmetric} presents some topological properties of metric spaces.
%---------------------------------------
\begin{definition}
\footnote{
  in \structe{metric space}:
  \citerpc{thron1966}{154}{definition 19.5},
  \citerpg{bruckner1997}{356}{013458886X}
  }
\label{def:bounded}
%---------------------------------------
Let $\dspaceX$ be a \structe{distance space}.
Let $\psetX$ be the \structe{power set} \xref{def:pset} of $\setX$.
\defboxt{
  A set $\setA$ is \propd{bounded} in $\dspaceX$ if
  \\\indentx$\setA\in\psetX$ and $\diam\setA<\infty$.
  }
\end{definition}

%%---------------------------------------
%\begin{remark}
%%---------------------------------------
%Although the function $\diam$ is a \fncte{set function}, it is not a \fncte{measure}.
%\end{remark}
%\begin{proof}
%Let $\setA$ and $\setB$ be sets that are \prope{disjoint} and \prope{disconnected}
%with respect to each other.
%\begin{align*}
%  \diam{\setA \setu \setB}
%    &= \diam\setA + \diam\setB + \inf\set{\distance{x}{y}}{x,y\in\setA}
%  \\&\ge \diam\setA + \diam\setB
%  \\\implies & \text{$\diam$ is not a measure}
%    && \text{by definition of measure \ifdochas{measure}{\prefpo{def:measure}}}
%\end{align*}
%\end{proof}


%%---------------------------------------
%\begin{definition}
%\label{def:ods}
%%---------------------------------------
%\defboxp{
%  The triple $\omsD$ is an \structd{ordered distance space} if
%  $\dspaceX$ is a \structe{distance space} \xref{def:dspace}
%  and $\osetX$ is an \structe{ordered set} \xref{def:poset}.
%  }
%\end{definition}

%=======================================
\subsection{Properties}
%=======================================
%%---------------------------------------
%\begin{remark}
%\footnote{
%  \citePpc{heath1961}{810}{{\scshape Theorem}},
%  \citePpc{galvin1984}{71}{{\scshape 2.3 Lemma}}
%  }
%%---------------------------------------
%Although every \structe{metric space} \xref{def:metric} induces a \structe{topology} \xref{def:topology},
%a \structe{distance space} does \textbf{not} in general induce a topology.
%\end{remark}
%
%---------------------------------------
\begin{remark}
%---------------------------------------
%Let $\dspaceX$ be a \structb{distance space} \xref{def:dspace}.
Let $\seqxZ{x_n}$ be a \fncte{sequence} in a \structe{distance space} $\dspaceX$.
The \structe{distance space} $\dspaceX$ does not necessarily have all the nice properties that a 
\structe{metric space} \xref{def:metric} has.
In particular, note the following:
\\\indentx\remboxp{$\begin{array}{FMcMM}
  1. & $\distancen$   is a \fncte{distance} in $\dspaceX$ &\notimplies& $\distancen$ is \prope{continuous} in $\dspaceX$                  & \xref{ex:dspace_21}.\\
  2. & $\balln$ is an \structe{open ball} in $\dspaceX$ &\notimplies& $\balln$ is \prope{open} in $\dspaceX$                              & \xref{ex:dspace_1n}.\\
  3. & $\baseB$ is the set of all                                    &\notimplies& $\baseB$ is a \structe{base} for a                     & \xref{ex:dspace_1n}.\footnotemark\\
     &                            \structe{open ball}s in $\dspaceX$ &           &                                    topology on $\setX$ & \\
  4. & $\seqn{x_n}$ is \prope{convergent} in $\dspaceX$ &\notimplies& limit is \prope{unique}                                             & \xref{ex:dspace_01}.\\
  5. & $\seqn{x_n}$ is \prope{convergent} in $\dspaceX$ &\notimplies& $\seqn{x_n}$ is \prope{Cauchy} in $\dspaceX$                        & \xref{ex:dspace_1n}.\\
\end{array}$}
%In a \structe{metric space} \xref{def:metric}, if a \fncte{sequence} has a limit, that limit is unique.
%This is not true in general for \structe{distance space}s \xref{def:distance}.
%See \prefpp{ex:dspace_01} for a distance space which converges to two distinct limits.
\footnotetext{
  \citePpc{heath1961}{810}{{\scshape Theorem}},
  \citePpc{galvin1984}{71}{{\scshape 2.3 Lemma}}
  }
\end{remark}


%=======================================
\section{Open sets in distance spaces}
%=======================================
%=======================================
\subsection{Definitions}
%=======================================
%\pref{def:ball} (next) defines the \structe{open ball}.
%In a \structe{metric space} \xref{def:metric}\index{space!metric}, 
%sets are often specified in terms of an \prope{open ball};
%and an open ball is specified in terms of a metric.
%---------------------------------------
\begin{definition}
\footnote{
  in \structe{metric space}:
  \citerp{ab}{35}
  }
\label{def:ball}
\label{def:ballc}
%---------------------------------------
Let $\dspaceX$ be a \structe{distance space} \xref{def:distance}.
Let $\Rp$ be the \structe{set of positive real numbers} \xref{def:Rx}.
\\\defboxt{$\begin{array}{Mrcl}
  An \structd{open ball}   centered at $x$ with radius $r$ is the set & \ball{x}{r}  &\eqd& \set{y\in\setX}{\distance{x}{y}<r}.\\
  A  \structd{closed ball} centered at $x$ with radius $r$ is the set & \ballc{x}{r} &\eqd& \set{y\in\setX}{\distance{x}{y}\le r}.
\end{array}$}
\end{definition}

%Open balls will often ``appear" different in different metric spaces. 
%Some examples include the following \xref{ex:taxicab}:\\
%\setlength{\unitlength}{\tw/6000}
%\begin{tabular}{cll>{\footnotesize(}l<{\footnotesize)}}
%%\begin{tabular}{cl}%<{:}l>{\footnotesize(}l<{\footnotesize)}}
%  \begin{picture}(300,350)(-130,-130)
%    \thicklines
%    \color{axis}%
%      \put(-130,   0){\line(1,0){260} }%
%      \put(   0,-130){\line(0,1){260} }%
%    \color{blue}%
%      \qbezier( 100,0)( 50, 50)(0, 100)%
%      \qbezier(-100,0)(-50, 50)(0, 100)%
%      \qbezier(-100,0)(-50,-50)(0,-100)%
%      \qbezier( 100,0)( 50,-50)(0,-100)%
%  \end{picture}
%  & \prope{taxi-cab metric}%  & \pref{ex:ms_taxi} & \prefpo{ex:ms_taxi} 
%  \\
%  \begin{picture}(300,300)(-130,-130)
%    \thicklines
%    \color{axis}%
%      \put(-130,   0){\line(1,0){260} }%
%      \put(   0,-130){\line(0,1){260} }%
%    \color{blue}%============================================================================
% NCTU - Hsinchu, Taiwan
% LaTeX File
% Daniel Greenhoe
%
% Unit circle with radius 100
%============================================================================

\qbezier( 100,   0)( 100, 41.421356)(+70.710678,+70.710678) % 0   -->1pi/4
\qbezier(   0, 100)( 41.421356, 100)(+70.710678,+70.710678) % pi/4-->2pi/4
\qbezier(   0, 100)(-41.421356, 100)(-70.710678,+70.710678) %2pi/4-->3pi/4
\qbezier(-100,   0)(-100, 41.421356)(-70.710678,+70.710678) %3pi/4--> pi 
\qbezier(-100,   0)(-100,-41.421356)(-70.710678,-70.710678) % pi  -->5pi/4
\qbezier(   0,-100)(-41.421356,-100)(-70.710678,-70.710678) %5pi/4-->6pi/4
\qbezier(   0,-100)( 41.421356,-100)( 70.710678,-70.710678) %6pi/4-->7pi/4
\qbezier( 100,   0)( 100,-41.421356)( 70.710678,-70.710678) %7pi/4-->2pi


%
%  \end{picture}
%  & \prope{Euclidean metric}% & \pref{ex:ms_euclidean} & \prefpo{ex:ms_euclidean} 
%  \\
%  \begin{picture}(300,300)(-130,-130)
%    \thicklines
%    \color{axis}%
%      \put(-130,   0){\line(1,0){260} }%
%      \put(   0,-130){\line(0,1){260} }%
%    \color{blue}%
%      \put(-100,-100){\line( 1, 0){200} }%
%      \put(-100,-100){\line( 0, 1){200} }%
%      \put( 100, 100){\line(-1, 0){200} }%
%      \put( 100, 100){\line( 0,-1){200} }%
%  \end{picture}
%  & \prope{sup metric}% & \pref{ex:ms_sup} & \prefpo{ex:ms_sup} 
%%  \\
%%  \begin{picture}(300,300)(-130,-130)%
%%    %{\color{graphpaper}\graphpaper[10](-150,-150)(300,300)}%
%%    \thicklines%
%%    \color{axis}%
%%      \put(-130,   0){\line(1,0){260} }%
%%      \put(   0,-130){\line(0,1){260} }%
%%    \color{blue}%
%%      \qbezier( 100,0)(0,0)(0, 100)%
%%      \qbezier( 100,0)(0,0)(0,-100)%
%%      \qbezier(-100,0)(0,0)(0,-100)%
%%      \qbezier(-100,0)(0,0)(0, 100)%
%%  \end{picture}
%%  & \prope{parabolic metric} & \pref{ex:ms_parabolic} & \prefpo{ex:ms_parabolic}
%%  \\ 
%%  \begin{picture}(400,400)(-200,-200)%
%%    \thicklines%
%%    \color{axis}%
%%      \put(-200,   0){\line(1,0){400} }%
%%      \put(   0,-200){\line(0,1){400} }%
%%    \color{blue}%
%%      \qbezier( 100,0)(100,100)(0, 100)%
%%      \qbezier(-171,0)(-50,50)(0, 100)%
%%      \qbezier(-171,0)(-50,-50)(0,-171)%
%%      \qbezier( 100,0)(50,-50)(0,-171)%
%%  \end{picture}
%%  & \prope{exponential metric} & \pref{ex:ms_32x} & \prefpo{ex:ms_32x}
%%  \\ 
%%  \begin{picture}(300,300)(-150,-150)%
%%    \thicklines%
%%    \color{axis}%
%%      \put(-130,   0){\line(1,0){260} }%
%%      \put(   0,-130){\line(0,1){260} }%
%%    \color{blue}%
%%      \qbezier( 100,0)(70,70)(0, 100)%
%%      \qbezier(-100,0)(-70,70)(0, 100)%
%%      \qbezier(-100,0)(-70,-70)(0,-100)%
%%      \qbezier( 100,0)(70,-70)(0,-100)%
%%  \end{picture}
%%  & \prope{tangential metric} & \pref{ex:ms_tan} & \prefpo{ex:ms_tan}
%\end{tabular}

%%---------------------------------------
%\begin{theorem}
%\footnote{
%  \citerpp{isham1999}{10}{11} \\
%  \citor{birkhoff1933}
%  }
%%---------------------------------------
%\thmbox{\begin{array}{rc>{\ds}l}
%  \distancesub{1}{x}{y} \join \distancesub{2}{x}{y} 
%    &\eqd& \max\brb{\distancesub{1}{x}{y},\, \distancesub{2}{x}{y}} 
%    \\
%  \distancesub{1}{x}{y} \meet \distancesub{2}{x}{y} 
%    &\eqd& \inf_{\seqn{x=x_1,x_2,\ldots,x_n=y}} \sum_{i=1}^{n-1} \brb{\distancesub{1}{x_i}{x_{i+1}},\, \distancesub{2}{x_i}{x_{i+1}}}
%\end{array}}
%\end{theorem}

%\begin{figure}[h]
%  \centering%
%  \psset{unit=6mm}%
%  %============================================================================
% Daniel J. Greenhoe
% LaTeX file
%============================================================================
{\psset{unit=0.30mm}
\begin{pspicture}(-152,-80)(152,90)
  %-------------------------------------
  % options
  %-------------------------------------
  \psset{
    dotsize=5pt,
    linestyle=dashed,
    fillstyle=none,
    %labelsep=5pt,
    }
  %-------------------------------------
  % design support
  %-------------------------------------
  %\psgrid[unit=100\psunit](-1,-1)(1,1)
  %-------------------------------------
  % axes
  %-------------------------------------
  %\psline[linecolor=axis]{<->}(-120,0)(120,0)% x-axis
  %\psline[linecolor=axis]{<->}(0,-120)(0,120)% y-axis
  %-------------------------------------
  % nodes
  %-------------------------------------
  \pnode(  0,  7){o}% origin (and center of outer ball)
  \pnode( 41, 25){p1}% a point p (and center of an inner ball)
  \pnode(  0, 55){p2}% a point p (and center of an inner ball)
  \pnode( 83, 19){p3}% a point p (and center of an inner ball)
  \pnode( 67,-37){p4}% a point p (and center of an inner ball)
  \pnode(-23,-29){p5}% a point p (and center of an inner ball)
  \pnode(-59, 17){p6}% a point p (and center of an inner ball)
  \pnode(-13, 11){p7}% a point p (and center of an inner ball)
  \pnode( 20,-53){p8}% a point p (and center of an inner ball)
  %-------------------------------------
  % objects
  %-------------------------------------
  \psccurve[linecolor=blue](110,0)(80,80)(0,70)(-110,0)(-50,-30)(0,-70)% open set
  \psdot(p1)\pscircle[linecolor=red](p1){33}\uput{4pt}[0](p1){$p_1$}%
  \psdot(p2)\pscircle[linecolor=red](p2){15}\uput{4pt}[90](p2){$p_2$}%
  \psdot(p3)\pscircle[linecolor=red](p3){20}\uput{4pt}[0](p3){$p_3$}%
  \psdot(p4)\pscircle[linecolor=red](p4){11}\uput{4pt}[45](p4){$p_4$}%
  \psdot(p5)\pscircle[linecolor=red](p5){19}\uput{4pt}[0](p5){$p_5$}%
  \psdot(p6)\pscircle[linecolor=red](p6){30}\uput{4pt}[90](p6){$p_6$}%
  \psdot(p7)\pscircle[linecolor=red](p7){31}\uput{4pt}[0](p7){$p_7$}%
  \psdot(p8)\pscircle[linecolor=red](p8){15}\uput{4pt}[90](p8){$p_8$}%
\end{pspicture}
}%
%}%


%  \caption{\structe{open set} \xref{def:dspace_open} \label{fig:ms_open}}
%\end{figure}
%---------------------------------------
\begin{definition}
\label{def:dspace_open}
%\label{def:dspace_closed}
%---------------------------------------
Let $\dspaceX$ be a \structe{distance space}. % \xref{def:dspace}.
Let $\setX\setd\setA$ be the \ope{set difference} of $\setX$ and a set $\setA$.
\defboxp{
  A set $\setU$ is \propd{open} in $\dspaceX$ if $\setU\in\psetX$ and 
  %\\\indentx$x\in\setU \qquad\implies\qquad$there exists $r\in\Rp$ such that $\ball{x}{r}\subseteq \setU$.
  for every $x$ in $\setU$ there exists $r\in\Rp$ such that $\ball{x}{r}\subseteq \setU$.
  A set $\setU$ is an \structd{open set} in $\dspaceX$ if $\setU$ is \prope{open} in $\dspaceX$. 
  A set $\setD$ is \propd{closed} in $\dspaceX$ if $\brp{\setX\setd\setD}$ is \prope{open}.
  A set $\setD$ is a \structd{closed set} in $\dspaceX$ if $\setD$ is \prope{closed} in $\dspaceX$. 
  }
\end{definition}

%=======================================
\subsection{Properties}
%=======================================
%\pref{thm:dspace_open} (next) identifies four fundamental properties of open sets in
%distance spaces.
%These properties are the same as those defining a topology\ifsxref{topology}{def:topology}.
%---------------------------------------
\begin{theorem}
\footnote{
  in \structe{metric space}:
  \citerpp{dieudonne1969}{33}{34},
  \citerpg{rosenlicht}{39}{0486650383}
  %\citerp{giles1987}{215} \\
  %\citerpg{davis2005}{19}{0071243399}
  %\citerpg{ab}{35}{0120502577}
  }
\label{thm:dspace_open}
\index{space!metric}
%---------------------------------------
Let $\dspaceX$ be a \structe{distance space}. % \xref{def:dspace}.
Let $\xN$ be any (finite) positive integer.
Let $\Gamma$ be a \structe{set} possibly with an uncountable number of elements.
\thmbox{\begin{array}{F Mll>{\ds}l l}
    1. &                                                                &                         &          & \setX                                   & \text{is \prope{open}.}\\
    2. &                                                                &                         &          & \emptyset                               & \text{is \prope{open}.}\\
    3. & each element in $\setxn{\setU_n}$                              & \text{is \prope{open}} & \implies & \setopi_{n=1}^\xN \setU_n               & \text{is \prope{open}.}\\
    4. & each element in $\set{\setU_\gamma\in\psetX}{\gamma\in\Gamma}$ & \text{is \prope{open}} & \implies & \setopu_{\gamma\in\Gamma} \setU_\gamma  & \text{is \prope{open}.} 
  \end{array}}
\end{theorem}
\begin{proof}
\begin{enumerate}
  \item Proof that $\setX$ is \prope{open} in $\dspaceX$:
    \begin{enumerate}
      \item By definition of \structe{open set} \xref{def:dspace_open}, 
            $\setX$ is \prope{open} $\iff \forall x\in\setX \quad\exists r \st \ball{x}{r}\subseteq \setX$.
      \item By definition of \structe{open ball} \xref{def:ball}, it is always true that $\ball{x}{r}\subseteq\setX$ in $\dspaceX$.
      \item Therefore, $\setX$ is \prope{open} in $\dspaceX$.
    \end{enumerate}

  \item Proof that $\emptyset$ is \prope{open} in $\dspaceX$:
    \begin{enumerate}
      \item By definition of \structe{open set} \xref{def:dspace_open}, 
            $\emptyset$ is \prope{open} $\iff \forall x\in\setX \quad\exists r \st \ball{x}{r}\subseteq \emptyset$.
      \item By definition of \structe{empty set} $\emptyset$ \xref{def:emptyset}, this is always true because no $x$ is in $\emptyset$.
      \item Therefore, $\emptyset$ is \prope{open} in $\dspaceX$.
    \end{enumerate}

  \item Proof that $\setopu\setU_\gamma$ is \prope{open} in $\dspaceX$:
    \begin{enumerate}
      \item By definition of \structe{open set} \xref{def:dspace_open}, 
            $\setopu\setU_\gamma$ is \prope{open} $\iff \forall x\in\setopu\setU_\gamma \quad\exists r \st \ball{x}{r}\subseteq\setopu\setU_\gamma$.
      \item If $x\in\setopu\setU_\gamma$, then there is at least one $\setU\in\setopu\setU_\gamma$ that contains $x$.
      \item By the left hypothesis in (4), that set $\setU$ is open and so for that $x$,
            $\exists r \st \ball{x}{r}\subseteq\setU\subseteq\setopu\setU_\gamma$.
      \item Therefore, $\setopu\setU_\gamma$ is \prope{open} in $\dspaceX$.
    \end{enumerate}

  \item Proof that $\setU_1$ and $\setU_2$ are \prope{open} $\implies$ $\setU_1\seti\setU_2$ is \prope{open}: \label{ilem:ms_open_seti}
    \begin{enumerate}
      \item By definition of \structe{open set} \xref{def:dspace_open}, 
            $\setU_1\seti\setU_2$ is \prope{open} $\iff \forall x\in\setU_1\seti\setU_2\quad\exists r \st \ball{x}{r}\subseteq\setU_1\seti\setU_2$.
      \item By the left hypothesis above, $\setU_1$ and $\setU_2$ are \prope{open};
            and by the definition of \structe{open set}s \xref{def:dspace_open}, 
            there exists $r_1$ and $r_2$ such that
            $\ball{x}{r_1}\subseteq\setU_1$ and $\ball{x}{r_2}\subseteq\setU_2$.
      \item Let $r\eqd\min\setn{r_1,r_2}$. Then $\ball{x}{r}\subseteq\setU_1$ and $\ball{x}{r}\subseteq\setU_2$.
      \item By definition of \ope{set intersection} $\seti$ then, $\ball{x}{r}\subseteq\setU_1\seti\setU_2$.
      \item By definition of \ope{open set} \xref{def:dspace_open}, $\setU_1\seti\setU_2$ is \prope{open}.
    \end{enumerate}

  \item Proof that $\setopi_{n=1}^\xN\setU_n$ is \prope{open} (by induction):
    \begin{enumerate}
      \item Proof for $\xN=1$ case: $\setopi_{n=1}^\xN\setU_n=\setopi_{n=1}^1\setU_n=\setU_1$ is \prope{open} by hypothesis.
      \item Proof that $\xN$ case $\implies$ $\xN+1$ case: 
        \begin{align*} 
          \setopi_{n=1}^{\xN+1}\setU_n
            &= \brp{\setopi_{n=1}^{\xN}\setU_n}\seti\setU_{\xN+1}
            && \text{by property of $\setopi$}
          \\&\implies \text{\prope{open}} 
            && \text{by ``$\xN$ case" hypothesis and \prefp{ilem:ms_open_seti}}
        \end{align*}
    \end{enumerate}
\end{enumerate}
\end{proof}

%---------------------------------------
\begin{corollary}
\label{cor:dspace_open}
%---------------------------------------
Let $\dspaceX$ be a \structe{distance space}.
%Let an $\prope{open}$ set in $\psetX$ be defined as in \prefpp{def:dspace_open}.
\corboxt{
  The set 
  \quad$\topT\eqd\set{\setU\in\psetX}{\text{$\setU$ is \prope{open} in $\dspaceX$}}$\quad
  is a \structe{topology} on $\setX$, 
  \\and $\topspaceX$ is a \structe{topologogical space}.
  }
\end{corollary}
\begin{proof}
This follows directly from the definition of an \prope{open} set \xref{def:dspace_open},
\prefpp{thm:dspace_open}, and the definition of \structe{topology} \xref{def:topology}.
\end{proof}

%%---------------------------------------
%\begin{corollary}
%\label{cor:dspace_open}
%%---------------------------------------
%Let $\dspaceX$ be a \structe{distance space} \xref{def:dspace}.
%Let $\Gamma$ be a \structe{set} possibly with an uncountable number of elements.
%Let $\baseU\eqd\set{\setU_\gamma\in\psetX}{\gamma\in\Gamma}$.
%Let $\baseB$ be a family of sets such that $\baseB\subseteq\psetX$.
%\corbox{
%  \brb{\begin{array}{FMD}
%    1. & each set in $\baseU$ is \prope{open} in $\dspaceX$ & and
%  \\2. & $\baseU\subseteq\baseB$ &and 
%  \\3. & $\setX\in\baseB$ & and
%  \\4. & $\setU_1,\setU_2\in\baseU\implies\setU_1\seti\setU_2\in\baseB$ & and
%  \end{array}}
%  \quad\implies\quad
%  \brb{\begin{array}{M}
%    $\baseB$ is a \structe{base} for a\\\structe{topology} \xref{def:topology} on $\setX$
%  \end{array}}
%  }
%\end{corollary}
%\begin{proof}
%This follows directly from \prefpp{thm:dspace_open}, the definition of \structe{topology} \xref{def:topology},
%and the definition of \structe{base} \xref{def:base}.
%\end{proof}

Of course it is possible to define a very large number of topologies even on a finite set with just a handful of elements;\footnote{
  For a finite set $\setX$ with $n$ elements, there are 
  29 topologies on $\setX$ if $n=3$, 
  6942 topologies on $\setX$ if $n=5$, and
  and 8,977,053,873,043 (almost 9 trillion) topologies on $\setX$ if $n=10$.
  References:  \citeoeis{A000798},
  \citerp{brown1996}{31},
  \citerpg{comtet1974}{229}{9027704414},
  \citer{comtet1966},
  \citerp{chatterji1967}{7},
  \citer{evans1967},
  \citerp{krishnamurthy1966}{157}
  }
and it is possible to define an infinite number of topologies even on a \prope{linearly ordered} infinite set
like the \structe{real line} $\osetR$.\footnote{%
  For examples of topologies on the real line, see the following:
      \citerpgc{adams2008}{31}{0131848690}{"six topologies on the real line"},
      \citerppgc{salzmann2007}{64}{70}{0521865166}{Weird topologies on the real line},
      \citerpgc{murdeshwar1990}{53}{8122402461}{``often used topologies on the real line"},
      \citerppgc{joshi1983}{85}{91}{0852264445}{\textsection4.2 Examples of Topological Spaces}
  }
Be that as it may, \pref{def:dspacetop} (next definition) defines a single but convenient 
\structe{topological space} in terms of a \structe{distance space}.
Note that every \structe{metric space} conveniently and naturally induces a \structe{topological space}
because the \structe{open ball}s of the metric space form a \structe{base} for the \structe{topology}.
This is not the case for all {distance space}s.
But if the open balls of a \structe{distance space} are all \prope{open}, 
then those open balls induce a topology (next theorem).\footnote{
  \structe{metric space}: \prefp{def:metric}; 
  \structe{open ball}: \prefp{def:ball};
  \structe{base}: \prefp{def:base};  
  \structe{topology}: \prefp{def:topology};
  not all open balls are open in a distance space: \prefpp{ex:dspace_01} and \prefpp{ex:dspace_1n};
  }
%---------------------------------------
\begin{definition}
\label{def:dspacetop}
%---------------------------------------
Let $\dspaceX$ be a \structe{distance space}. % \xref{def:dspace}.
%Let $\psetX$ be the \structe{power set} \xref{def:psetx} of $\setX$.
\defboxt{
  The set 
  \quad$\topT\eqd\set{\setU\in\psetX}{\text{$\setU$ is \prope{open} in $\dspaceX$}}$\quad
  is the \structd{topology induced by $\dspaceX$ on $\setX$}. 
  \\The pair $\topspaceX$ is called the \structd{topological space induced by $\dspaceX$}. 
  }
\end{definition}

For any \structe{distance space} $\dspaceX$, no matter how strange, 
there is guaranteed to be at least one \structe{topological space induced by $\dspaceX$}---and that 
is the \structe{indiscrete topological space} \xref{ex:idts} because for any distance space $\dspaceX$, 
$\emptyset$ and $\setX$ are \prope{open set}s in $\dspaceX$ \xref{thm:dspace_open}.



%In a \structe{metric space}, the set of all \structe{open ball}s 
%is a \structe{base} for a \structe{topology}.
%This is largely due to the fact that in a \structe{metric space}, all \structe{open ball}s are \prope{open}.
%However, in a \structe{distance space}, \structe{open ball}s are not always \prope{open} \xref{ex:dspace_1n}.
%But if they all are, then the open balls induce a topology (next theorem).\footnote{
%  \structe{metric space}: \prefp{def:metric}; 
%  \structe{open ball}: \prefp{def:ball};
%  \structe{base}: \prefp{def:base};  
%  \structe{topology}: \prefp{def:topology};
%  not all \structe{open ball}s are \prope{open} in a \structe{distance space}: \prefp{ex:dspace_1n};
%  }
%---------------------------------------
\begin{theorem}
\label{thm:baseoball}
%---------------------------------------
Let $\baseB$ be the set of all \structe{open ball}s in a \structe{distance space}
$\dspaceX$. % \xref{def:dspace}.
\thmbox{
  \brb{\text{every \structe{open ball} in $\baseB$ is \prope{open}}}
  \quad\iff\quad
  \brb{\text{$\baseB$ is a \structe{base} for a \structe{topology}}}
  }
\end{theorem}
\begin{proof}
    \begin{align*}
      \mathrlap{\boxed{\text{every \structe{open ball} in $\baseB$ is \prope{open}}}}
      \\&\implies\text{for every $x$ in $\setB_y\in\baseB$ there exists $r\in\Rp$ such that $\ball{x}{r}\subseteq \setB_y$}
        && \text{by definition of \prope{open} \xref{def:dspace_open}}
      \\&\implies \brb{\begin{array}{M}
                    for every $x\in\setX$ and for every $\setB_y\in\baseB$ containing $x$,\\
                    there exists $\setB_x\in\baseB$ such that\qquad $x\in\setB_x\subseteq\setB_y$.
                  \end{array}}
        && \text{because $\forall\opair{x}{r}\in\setX\times\Rp$, $\ball{x}{r}\subseteq\setX$}
      \\&\implies \boxed{\text{$\baseB$ is a \structe{base} for $\topT$}}
        && \text{by \prefp{thm:basex}}
      %
      \\&\implies \brb{\begin{array}{M}
                    for every $x\in\setX$ and for every $\setU\subseteq\topT$ containing $x$,\\
                    there exists $\setB_x\in\baseB$ such that\qquad $x\in\setB_x\subseteq\setU$.
                  \end{array}}
        && \text{by \prefp{thm:basex}}
      \\&\implies \brb{\begin{array}{M}
                    for every $x\in\setX$ and for every $\setB_y\in\baseB\subseteq\topT$ containing $x$,\\
                    there exists $\setB_x\in\baseB$ such that\qquad $x\in\setB_x\subseteq\setB_y$.
                  \end{array}}
        && \text{by definition of \structe{base} \xref{def:base}}
      \\&\implies \brb{\begin{array}{M}
                    for every $x\in\setB_y\in\baseB\subseteq\topT$,\\
                    there exists $\setB_x\in\baseB$ such that\qquad $x\in\setB_x\subseteq\setB_y$.
                  \end{array}}
      \\&\implies \boxed{\text{every \structe{open ball} in $\baseB$ is \prope{open}}}
        && \text{by definition of \prope{open} \xref{def:dspace_open}}
    \end{align*}
\end{proof}

%%---------------------------------------
%\begin{remark}
%\label{rem:dspace_tspace}
%%---------------------------------------
%In a \structe{metric space} \xref{def:metric}, the set of all \structe{open ball}s \xref{def:ball} 
%is a \structe{base} \xref{def:base} for a \structe{topology} \xref{def:topology}.
%That is, the open balls induce a topology on the base set.
%In a \structe{distance space} \xref{def:dspace}, however, this is \emph{not} the case \xxref{thm:baseoball}{ex:dspace_1n}.
%But this by no means implies that all the beautiful structure of topological spaces is unavailable to us in a distance space.
%Rather, any set of \structe{open set}s \xref{def:dspace_open} in a \structe{distance space} $\dspaceX$ that
%satifies the four relations in \prefpp{thm:dspace_open} \emph{is}, by \prefpp{def:topology}, a \structe{topology} on $\setX$.
%For example, regardless of how strange the \fncte{distance function} $\distancen$ may be, 
%of course the \structe{discrete topology} $\psetX$
%and \structe{indiscrete topology} $\setn{\emptyset,\setX}$ \xref{ex:discretetop}
%are still legitimate topologies on $\setX$.
%And in any \structe{topological space} we may construct on a distance space, 
%all the structure of topological space, such as \structe{derived sets} \xxref{def:clsA}{def:bndA},
%and fundamental properties, such as the \thme{closed set theorem} \xref{thm:cst}, 
%is still available to us.
%\end{remark}

%=======================================
\section{Sequences in distance spaces}
%=======================================
%=======================================
\subsection{Definitions}
%=======================================
%---------------------------------------
\begin{definition}
\footnote{
  in \structe{metric space}:
  \citerpg{rosenlicht}{45}{0486650383},
  \citerpgc{giles1987}{37}{0521359287}{3.2 Definition},
  \citerpgc{khamsi2001}{13}{0471418250}{Definition 2.1}
  %\citerpgc{thomson2008}{30}{143484367X}{Definition 2.1}
  %\citor{cauchy1821}
  ``$\to$" symbol: \citorpc{leathem1905}{13}{section III.11}  % referenced by bromwich1955 page 3
  }
\label{def:dspace_converge}
\label{def:dspace_limit}
%---------------------------------------
%Let $\topspaceX$ be the \structe{topological space} induced by a \structe{distance space} $\dspaceX$ \xref{def:dspace}.
Let $\seqxZ{x_n\in\setX}$ be a  \fncte{sequence} in a \structe{distance space} $\dspaceX$. % \xref{def:dspace}.
\defboxp{
  The sequence $\seqn{x_n}$ \propd{converge}s to a \propd{limit} $x$ if
    for any $\varepsilon\in\Rp$, there exists $\xN\in\Z$
    such that for all $n>\xN$,
    $\distance{x_n}{x}<\varepsilon$.
  \\This condition can be expressed in any of the following forms:
  \\$\indentx\begin{array}{>{\scy}rM@{\qquad}>{\scy}rM}
      1. & The \opd{limit} of the sequence $\seqn{x_n}$ is $x$.             & 3. & $\ds\lim_{n\to\infty} \seqn{x_n} = x$.
    \\2. & The sequence $\seqn{x_n}$ is \propd{convergent} with limit $x$.    & 4. & $\ds\seqn{x_n} \to x$.                
  \end{array}$
  \\A \fncte{sequence} that converges is \propd{convergent}.
    %a \fncte{sequence} that does not converge is said to \propd{diverge}, or is \propd{divergent}.
    %An element $x\in\setA$ is a \vald{limit point} of $\setA$ if it is the limit of some $\setA$-valued sequence $\seqn{x_n\in\setA}$.
  }
\end{definition}

%---------------------------------------
\begin{definition}
\footnote{
  in \structe{metric space}:
  \citerpgc{apostol1975}{73}{0201002884}{4.7},
  \citerpg{rosenlicht}{51}{0486650383}
  }
\label{def:cauchy}
\index{Cauchy sequences}
\index{sequences!Cauchy}
%---------------------------------------
Let $\seqxZ{x_n\in\setX}$ be a \fncte{sequence} in a \structe{distance space} $\dspaceX$. % \xref{def:distance}.
\\\defboxp{
  The sequence $\seqn{x_n}$ is a \structd{Cauchy sequence} in $\dspaceX$ if
  \\\indentx for every $\varepsilon\in\Rp$, there exists $\xN\in\Z$ such that $\forall n,m>\xN,\; \distance{x_n}{x_m}<\varepsilon$\qquad{\scs(\prope{Cauchy condition})}.
  }
\end{definition}


%%---------------------------------------
%\begin{definition}
%\footnote{
%  \citerpgc{blumenthal1953}{9}{0828402426}{{\scshape Definition 6.3}}
%  %\citerpgc{berberian1961}{27}{0821819127}{Theorem~II.4.1}
%  %\citerp{pedersen2000}{4}
%  }
%\label{def:dspace_continuous}
%%---------------------------------------
%%Let $\omsR$ be an \prope{ordered distance space} \xref{def:ods}
%%on the \structe{set of real numbers} $\R$ with the \fncte{usual metric} $\distancea{x}{y}\eqd\abs{x-y}$.
%Let $\dspaceX$ be a \structe{distance space} \xref{def:dspace}.
%\\\defboxp{
%  The \fncte{distance function} $\distancen$ is \propd{continuous} at $\opair{x}{y}$ if
%  \\\indentx$
%  \mcom{\seqn{x_n} \to x \text{ and } \seqn{y_n} \to y}{convergence in $\dspaceX$}
%  \quad \implies \quad
%  \mcom{\seqn{\distance{x_n}{y_n}} \to \distance{x}{y}}{convergence in $\opair{\R}{\distancean}$}
%  $\\
%  The \fncte{distance function} $\distancen$ is \propd{continuous} if $\distancen$ is \prope{continuous} at each $\opair{x}{y}$ in $\setX^2$.
%  }
%\end{definition}


%A \fncte{sequence} is said to be \prope{complete} in a \structe{distance space} $\dspaceX$
%if every \prope{Cauchy sequence} in $\dspaceX$ \prope{converges} to a \struct{limit} in $\dspaceX$ (next definition).
%---------------------------------------
\begin{definition}
\footnote{
  in \structe{metric space}:
  \citerpg{rosenlicht}{52}{0486650383}
  }
\label{def:complete}
%---------------------------------------
Let $\seqxZ{x_n\in\setX}$ be a \fncte{sequence} in a \structe{distance space} $\dspaceX$.
\\\defboxt{
  The sequence $\seqxZ{x_n\in\setX}$ is \propd{complete} in $\dspaceX$ if
  \\\indentx$\text{$\seqn{x_n}$ is \prope{Cauchy} in $\dspaceX$}
    \quad \implies \quad
    \text{$\seqn{x_n}$ is \prope{convergent} in $\dspaceX$.}$
  }
\end{definition}


%=======================================
\subsection{Properties}
%=======================================
%---------------------------------------
\begin{proposition}
\footnote{
  in \structe{metric space}:
  \citerpgc{giles1987}{49}{0521359287}{Theorem 3.30}
  }
\label{prop:cauchy==>bounded}
%---------------------------------------
Let $\seqxZ{x_n\in\setX}$ be a \fncte{sequence} in a \structe{distance space} $\dspaceX$. % \xref{def:dspace}.
\propbox{
  \brb{\begin{array}{M}
    $\seqn{x_n}$ is \prope{Cauchy} 
    in $\dspaceX$
  \end{array}}
  \qquad\implies\qquad
  \brb{\begin{array}{M}
    $\seqn{x_n}$ is \prope{bounded} 
    in $\dspaceX$
  \end{array}}
}
\end{proposition}
\begin{proof}
\begin{align*}
  \text{$\seqn{x_n}$ is \prope{Cauchy}}
    &\implies \text{for every $\varepsilon\in\Rp$},\; \exists\xN\in\Z \st \forall n,m>\xN,\; \distance{x_n}{x_m}<\varepsilon
    \quad \text{(by \pref{def:cauchy})}
  \\&\implies \exists\xN\in\Z \st \forall n,m>\xN,\; \distance{x_n}{x_m}<1
    \qquad \text{(arbitrarily choose $\varepsilon\eqd1$)}
  \\&\implies \exists\xN\in\Z \st \forall n,m\in\Z,\; \distance{x_n}{x_{m+1}}<\max\setn{\setn{1}\setu\set{\distance{x_p}{x_q}}{p,q\ngtr N}}
  \\&\implies \text{$\seqn{x_n}$ is \prope{bounded}}
    \qquad \text{(by \prefp{def:bounded})}
\end{align*}
\end{proof}


%---------------------------------------
\begin{proposition}
\footnote{
  in \structe{metric space}:
  \citerpg{rosenlicht}{52}{0486650383}
  }
\label{prop:cauchy_subseq}
%---------------------------------------
Let $\seqxZ{x_n\in\setX}$ be a \fncte{sequence} in a \structe{distance space} $\dspaceX$. % \xref{def:dspace}.
Let $\ff\in\clFzz$ \xref{def:clFxy} be a \prope{strictly monotone} function such that $\ff(n)<\ff(n+1)$.
\propbox{
  \mcom{\text{$\seq{x_n}{n\in\Z}$ is \prope{Cauchy}}}{sequence is \prope{Cauchy}}
  \qquad\implies\qquad
  \mcom{\text{$\seq{x_{\ff(n)}}{n\in\Z}$ is \prope{Cauchy}}}{subsequence is also \prope{Cauchy}}
  }
\end{proposition}
\begin{proof}
\begin{align*}
  &\text{$\seq{x_n}{n\in\Z}$ is \prope{Cauchy}}
  \\&\implies \text{for any given } \varepsilon>0,\; \exists\xN \st \forall n,m>\xN,\; \distance{x_n}{x_m}<\varepsilon
    &&        \text{by \prefp{def:cauchy}}
  \\&\implies \text{for any given } \varepsilon>0,\; \exists\xN' \st \forall \ff(n),\ff(m)>\xN',\; \distance{x_{\ff(n)}}{x_{\ff(m)}}<\varepsilon
  \\&\implies \seq{x_{\ff(n)}}{n\in\Z} \text{ is \prope{Cauchy}}
    &&        \text{by \prefp{def:cauchy}}
\end{align*}
\end{proof}



%%---------------------------------------
%\begin{proposition}
%\footnote{
%  \citerppg{deza2014}{6}{7}{3662443422}
%  }
%%---------------------------------------
%\propboxt{
%  If $\fq$ is a \fncte{quasi-metric} \xref{def:qmetric}, then 
%  \\\indentx$\begin{array}{rcl MD}
%    \distancen_1(x,y) &\eqd& \min\setn{\fq(x,y),\fq(y,x)}                    & is a \fncte{metric} & (\fncte{bi-distance}).\\
%    \distancen_2(x,y) &\eqd& \brp{\fq^r(x,y),\fq^r(y,x)}^\frac{1}{r},\,r\ge1 & is a \fncte{metric}.
%  \end{array}$
%  }
%\end{proposition}

%---------------------------------------
\begin{theorem}
\footnote{
  in \structe{metric space}:
  \citerpgc{kubrusly2001}{128}{0817641742}{Theorem 3.40},
  \citerpgc{haaser1991}{75}{0486665097}{6$\cdot$10, 6$\cdot$11 Propositions},
  \citerpgc{bryant1985}{40}{0521318971}{Theorem 3.6, 3.7},
  \citerppg{sutherland1975}{123}{124}{0198531613} %{Proposition 9.23}
  %\citerpgc{rosenlicht}{52}{0486650383}{Proposition}\\
  }
\label{thm:comcls}
%\label{prop:comcls_com}
%---------------------------------------
%Let $\dspaceX$ be a \structe{metric space}.
%Let $\dspaceA$ be a \structe{subspace} of a \structe{distance space} $\dspaceX$ \xref{def:dspace}.
Let $\dspaceX$ be a \structe{distance space}. % \xref{def:dspace}.
%Let $\setA$ be a subset of $\setX$.
Let $\clsA$ be the \structe{closure} \xref{def:clsA} of a $\setA$ in a 
\structe{topological space induced by $\dspaceX$}. % \xref{def:dspacetop}.
\thmbox{
  \brb{\begin{array}{FMMD}
    1. & \vale{limit}s are \prope{unique} in $\dspaceX$ & \xref{def:dspace_converge} & and\\
    2. & $\dspaceA$ is \prope{complete} in $\dspaceX$ & \xref{def:complete}        &
  \end{array}}
  \quad\implies\quad
  \mcom{\text{$\setA$ is \prope{closed} in $\dspaceX$}}{$\setA=\clsA$}
  %\\
  %\brb{\begin{array}{FMD}
  %  1. & $\dspaceX$ is \prope{complete} in $\dspaceX$ &  and \\
  %  2. &  $\setA$ is \prope{closed} in $\dspaceX$       &($\setA=\clsA$)
  %\end{array}}
  %&\implies&
  %\brb{\text{$\dspaceA$ is \prope{complete} in $\dspaceX$}}
  }
\end{theorem}
\begin{proof}
\begin{enumerate}
  \item Proof that $\setA\subseteq\clsA$: by \prefp{lem:intAAclsA}
  \item Proof that $\clsA\subseteq\setA$ (proof that $x\in\clsA$ $\implies$ $x\in\setA$):
    \begin{enumerate}
      \item Let $x$ be a point in $\clsA$ ($x\in\clsA$).
      \item Define a \fncte{sequence} of open balls $\seqn{\ball{x}{\frac{1}{1}},\,\ball{x}{\frac{1}{2}},\,\ball{x}{\frac{1}{3}},\,\ldots}$.
      \item Define a \fncte{sequence} of points $\seqn{x_1,\,x_2,\,x_3,\,\ldots}$ such that $x_n\in\ball{x_n}{\frac{1}{n}}\seti\setA$.
      \item Then $\seqn{x_n}$ is \prope{convergent} in $\setX$ with limit $x$ by \prefp{def:dspace_converge}
      \item and  $\seqn{x_n}$ is \prope{Cauchy} in $\setA$ by \prefp{def:cauchy}.
      %\item By the left hypothesis ($\dspaceA$ is \prope{complete}), $\seqn{x_n}$ is therefore also \prope{convergent} in $\setA$.\\
      \item By the hypothesis 2, $\seqn{x_n}$ is therefore also \prope{convergent} in $\setA$.\\
            Let this limit be $y$. Note that $y\in\setA$.\label{item:com_cls_yY}
      %\item By \prefp{prop:xn_to_xy}, limits are \prope{unique}, so $y=x$. \label{item:com_cls_yx}
      \item By hypothesis 1, limits are \prope{unique}, so $y=x$. \label{item:com_cls_yx}
      \item Because $y\in\setA$ (\pref{item:com_cls_yY}) and $y=x$ (\pref{item:com_cls_yx}), so $x\in\setA$.
      \item Therefore, $x\in\clsA\implies x\in\setA$ and $\clsA\subseteq\setA$.
    \end{enumerate}
\end{enumerate}
\end{proof}

%---------------------------------------
\begin{proposition}
\footnote{
  in \structe{metric space}:
  \citerpg{rosenlicht}{46}{0486650383}
  }
%---------------------------------------
Let $\seq{x_n}{n\in\Z}$ be a sequence in a \structe{distance space} $\dspaceX$.
Let $\ff:\Z\to\Z$ be a strictly increasing function such that $\ff(n)<\ff(n+1)$.
\propbox{
  \mcomr{\seq{x_n}{n\in\Z} \to x}{sequence converges to limit $x$}
  \qquad\implies\qquad
  \mcoml{\seq{x_{\ff(n)}}{n\in\Z} \to x}{subsequence converges to the same limit $x$}
  }
\end{proposition}
\begin{proof}
\begin{align*}
  \seq{x_n}{n\in\Z} \to x
    &\implies \forall \varepsilon>0,\; \exists\xN \st \forall n>\xN,\; \distance{x_n}{x}<\varepsilon
    &&        \text{by \prefp{thm:ms_converge}}
  \\&\implies \forall \varepsilon>0,\; \exists \ff(N) \st \forall \ff(n)>\ff(N),\; \distance{x_{\ff(n)}}{x}<\varepsilon
  \\&\implies \seq{x_{\ff(n)}}{n\in\Z} \to x
    &&        \text{by \prefp{thm:ms_converge}}
\end{align*}
\end{proof}



%%---------------------------------------
%\begin{theorem}
%\footnote{
%  \citerpgc{kubrusly2001}{128}{0817641742}{Corollary 3.41}
%  }
%\label{thm:comcomcls}
%%---------------------------------------
%%Let $\dspaceA$ be a \structe{subspace} of a metric space $\dspaceX$ ($\setA\subseteq\setX$).
%Let $\dspaceX$ be a \structe{metric space} \xref{def:metric}.
%Let $\setA$ be a subset of $\setX$.
%Let $\clsA$ be the \structe{closure} \xref{def:clsA} of $\setA$ in $\dspaceX$.
%\thmboxp{
%If $\dspaceX$ is \prope{complete} \xref{def:complete}, then
%  \\\indentx$
%  \brb{\text{$\dspaceA$ is \prope{complete}}}
%  \qquad\iff\qquad
%  \mcom{\text{$\setA$ is \prope{closed} in $\dspaceX$}}{$\setA=\clsA$}
%  $
%  }
%\end{theorem}
%\begin{proof}
%Note that in this corollary, the metric space $\dspaceX$ is assumed to be \prope{complete}.
%\begin{enumerate}
%  \item Proof that \prope{complete} $\implies$ \prope{closed}: by \pref{thm:comcls} (1).
%  \item Proof that \prope{complete} $\impliedby$ \prope{closed}: by \prope{complete} hypothesis and \pref{thm:comcls} (2).
%    \begin{enumerate}
%      \item By left hypothesis 2, $\setA$ is closed in $\dspaceX$.
%      \item By \prefpp{thm:insubset_closed} and because $\setA$ is closed in $\dspaceX$,
%            sequences converge in $\setA$.
%      \item Therefore by \prefpp{def:complete}, $\dspaceA$ is complete.
%    \end{enumerate}
%\end{enumerate}
%\end{proof}

%%---------------------------------------
%\begin{example}
%%---------------------------------------
%Let $\Q$ be the set of \sete{rational numbers}.
%\exbox{%\begin{array}{ll}
%  \text{The metric space $(\Q,\distance{x}{y}=\abs{x-y})$ is {\em not} \prope{complete}.}
%  %2. & \text{The metric space $(\R,\distance{x}{y}=\abs{x-y})$ {\em is} complete.    }\\
%}%\end{array}}
%\end{example}
%\begin{proofns}
%Let $\seq{x_n}{n\in\Znn}$ be the sequence of values approximating $\pi$ truncated to $n$
%decimal points:
%  \[ \seq{x_n}{n\in\Znn} \eqd \seqn{3,\, 3.1,\, 3.14,\, 3.141,\, 3.1415,\, 3.14159,\, 3.141592,\,\ldots} \]
%This is a Cauchy sequence.
%However, this sequence (and all sequences converging to an irrational number)
%does not converge to a rational number ($\Q$) and thus is not in
%the metric space $(\Q,\distancen)$ and thus $(\Q,\distancen)$ is {\em not complete}.
%%But of course this sequence (and all other like sequences) do converge to real numbers and thus
%%$(\R,\distancen)$ is a complete metric space.
%\end{proofns}

%%---------------------------------------
%\begin{example}[\exmd{Cauchy's convergence criterion}/\exmd{Cauchy's criterion}]
%\footnote{
%  \citerpgc{sohrab2003}{54}{0817642110}{Theorem 2.2.5}
%  }
%%---------------------------------------
%Let $\seqxZ{r_n\in\R}$ be a \textbf{real} sequence.
%\exboxp{
%  The metric space $\opair{\seq{r_n}}{\abs{r_n-r_m}}$ is \prope{complete}.
%  }
%\end{example}

%---------------------------------------
\begin{theorem}[\thmd{Cantor intersection theorem}]
\footnote{
  in \structe{metric space}:
  \citerp{davis2005}{28},
  \citerp{hausdorff1937e}{150}
  }
\label{thm:cantor_int}
\label{thm:cit}
\index{Cantor intersection theorem}
\index{theorems!Cantor intersection}
%---------------------------------------
Let $\dspaceX$ be a \prope{distance space} \xref{def:dspace},
$\seqnZ{\setA_n}$ a \fncte{sequence} with each $\setA_n\in\psetX$, and $\seto{\setA}$ the number of elements in $\setA$.
\thmbox{
  \brb{\begin{array}{FMCMD}
    1. & $\dspaceX$ is \prope{complete}                &                 & \xref{def:complete}       & and \\
    2. & $\setA_n$ is \prope{closed}                     & \forall n\in\Zp & \xref{def:closedset}    & and \\
    3. & $\diam \setA_{n} \ge \diam \setA_{n+1}$         & \forall n\in\Zp & \xref{def:diam}         & and \\
    4. & $\diam \seqxZ{\setA_n} \to 0$                   &                 & \xref{def:dspace_limit}
  \end{array}}
  \qquad\implies\qquad
  \brb{\seto{\ds\setopi_{n\in\Zp} \setA_n } = 1}
  }
\end{theorem}
\begin{proof}
\begin{enumerate}
\item Proof that $\seto{\setopi_{n\in\Z} \setA_n}<2$:
  \begin{enumerate}
    \item Let $\setA\eqd\seti \setA_n$.
    \item $x\ne y$ and $\{x,y\}\in \setA \implies \distance{x}{y}>0$ and $\{x,y\}\subseteq \setA_n \forall n$
    \item $\exists n \st \diam \setA_n < \distance{x}{y}$ by left hypothesis 4
    \item $\implies \exists n \st \sup\set{\distance{x}{y}}{x,y\in \setA_n}<\distance{x}{y}$
    \item This is a contradiction, so $\{x,y\}\notin \setA$ and $\seto{\setopi \setA_n}<2$.
  \end{enumerate}
                                                 
\item Proof that $\seto{\seti \setA_n}\ge1$:
  \begin{enumerate}
    \item Let $x_n\in \setA_n$ and $x_m\in \setA_m$
    \item $\forall \varepsilon,\; \exists\xN\in\Zp \st \setA_N < \varepsilon$
    \item $\forall m,n>\xN,\; x_n\in \setA_n\subseteq \setA_N$ and $x_m\in \setA_m\subseteq \setA_N$
    \item $\distance{x_n}{x_m}\le\diam \setA_N < \varepsilon \implies \{x_n\}$ is a Cauchy sequence
    \item Because $\{x_n\}$ is complete, $x_n\to x$.
    \item $\implies x\in \cls{\brp{\setA_n}} = \setA_n$
    \item $\implies \seto{\setA_n}\ge1$
  \end{enumerate}
\end{enumerate}
\end{proof}

%---------------------------------------
\begin{definition}
\footnote{
  \citerpgc{blumenthal1953}{9}{0828402426}{{\scshape Definition 6.3}}
  }
\label{def:dspace_cont}
%---------------------------------------
Let $\dspaceX$ be a \structe{distance space}.
Let $\setC$ be the set of all \prope{convergent} sequences in $\dspaceX$.
\thmboxp{
  The \fncte{distance function} $\distancen$ is \propd{continuous} in $\dspaceX$ if
  \\\indentx$\ds
    \seqn{x_n},\seqn{y_n}\in\setC
    \quad\implies\quad
    \lim_{n\to\infty}\seqn{\distance{x_n}{y_n}}=\distance{\lim_{n\to\infty}\seqn{x_n}}{\lim_{n\to\infty}\seqn{y_n}}
    $.\\
  A \fncte{distance function} is \propd{discontinuous} if it is not \prope{continuous}.
  }
\end{definition}

%---------------------------------------
\begin{remark}
%---------------------------------------
Rather than defining \prope{continuity} of a \structe{distance function} in terms of 
the \thme{sequential characterization of continuity} as in \pref{def:dspace_cont} (previous),
we could define continuity using an \thme{inverse image characterization of continuity}" \xref{def:dspacetop}.
Assuming an equivalent \structe{topological space} is used for both characterizations, the 
two characterizations are equivalent \xref{thm:limcont}.
In fact, one could construct an equivalence such as the following:
%%---------------------------------------
%\begin{corollary}
%\label{cor:limcont}
%%---------------------------------------
%Let $\dspaceX$ be a \structe{distance space}.
%Let $\topspaceX$ be the \structe{topological space induced by $\dspaceX$} \xref{def:dspacetop}.
%Let $\topspace{\R}{\topS}$ be the \structe{usual topological space over $\R$}.
%Let $\setC$ be the set of all \prope{convergent} sequences in $\dspaceX$.
\rembox{
  \brb{\begin{tabstr}{0.75}\begin{array}{M}
    \\
    $\distancen$ is \prope{continuous}  in $\clF{\setX^2}{\R}$\\
    \xref{def:continuous}\\
    {\scs(\thme{inverse image characterization of continuity})}
  \end{array}\end{tabstr}}
  \quad\iff\quad
  \brb{\begin{tabstr}{0.75}\begin{array}{M}
    $\ds\seqn{x_n},\seqn{y_n}\in\setC\quad\implies$\\
    $\ds\lim_{n\to\infty}\seqn{\distance{x_n}{y_n}}=\distance{\lim_{n\to\infty}\seqn{x_n}}{\lim_{n\to\infty}\seqn{y_n}}$\\
    \xref{def:converge}\\
    {\scs(\thme{sequential characterization of continuity})}
  \end{array}\end{tabstr}}
  }\\
%\end{corollary}
%\begin{proof}
Note that just as $\seqn{x_n}$ is a sequence in $\setX$, so the ordered pair $\opair{\seqn{x_n}}{\seqn{y_n}}$
is a sequence in $\setX^2$.
The remainder %of this proof 
follows from \prefpp{thm:limcont}.
%\end{proof}
However, use of the \thme{inverse image characterization} is somewhat troublesome
because we would need a topology on $\setX^2$, and we don't immediately have one defined and ready to use.
In fact, we don't even immediately have a distance space on $\setX^2$ defined or even open balls in such a distance space.
The result is, for the scope of this paper, it is arguably not worthwhile constructing the extra structure, 
but rather instead this paper
uses the \thme{sequential characterization} as a definition (as in \pref{def:dspace_cont}).
\end{remark}



%=======================================
\section{Examples}
%=======================================
Similar distance functions and several of the observations for the examples 
in this section can be found in \citerppg{blumenthal1953}{8}{13}{0828402426}.

In a \structe{metric space}, all \structe{open ball}s are \prope{open},
the \structe{open ball}s form a \structe{base} for a \structe{topology}, 
the limits of \prope{convergent} sequences are \prope{unique}, 
and the \fncte{metric function} is \prope{continuous}.
In the \structe{distance space} of the next example, none of these properties hold.
%---------------------------------------
\begin{example}
\footnote{
  A similar distance function $\distancen$ and \prefp{item:dspace_01} 
  can in essence be found in \citerpg{blumenthal1953}{8}{0828402426}.
  %Definitions for \pref{ex:dspace_01}:
  %$\opair{x}{y}$: \prefpp{def:opair};
  %$\intoo{a}{b}$ and $\intoc{a}{b}$: \prefpp{def:intxx};
  %$\abs{x}$: \prefpp{def:abs};
  %$\clF{\R\times\R}{\R}$: \prefpp{def:clFxy};
  %\fncte{distance}: \prefpp{def:distance};
  %\structe{open ball}: \prefpp{def:ball};
  %\prope{open}: \prefpp{def:dspace_open};
  %\structe{base}: \prefpp{def:base};
  %\structe{topology}: \prefpp{def:topology};
  %\structe{open set}: \prefpp{def:dspace_open};
  %\structe{topological space induced by $\dspaceRd$}: \prefpp{def:dspacetop};
  %\prope{discontinuous}: \prefpp{def:dspace_cont};
  }
\label{ex:dspace_01}
%---------------------------------------
%Let $\setX\eqd\R$.
Let $\opair{x}{y}$ be an \structe{ordered pair} in $\R^2$.
Let $\intoo{a}{b}$ be an \structe{open interval} and $\intoc{a}{b}$ a \structe{half-open interval} in $\R$.
Let $\abs{x}$ be the \fncte{absolute value} of $x\in\R$.
The function $\distance{x}{y}\in\clF{\R\times\R}{\R}$ such that
\\\indentx$\distance{x}{y} \eqd \brb{\begin{array}{lMD}
      y         & $\forall \opair{x}{y}\in\setn{4}\times\intoc{0}{2}$ & (\structe{vertical half-open interval})\\
      x         & $\forall \opair{x}{y}\in\intoc{0}{2}\times\setn{4}$ & (\structe{horizontal half-open interval})\\
      \abs{x-y} & otherwise                                           & (\prope{Euclidean})
    \end{array}}$\quad is a \fncte{distance} on $\R$.
\\
Note some characteristics of the \structe{distance space} $\dspaceRd$:
\begin{enumerate}
  \item $\dspaceRd$ is not a \structe{metric space} because $\distancen$ does not satisfy the \prope{triangle inequality}:
    \\\indentx$\distance{0}{4}
        \eqd \abs{0-4} = 4
        \nleq 2
        = \abs{0-1}     + 1
        \eqd \distance{0}{1} + \distance{1}{4}
       $

  \item Not every \structe{open ball} in $\dspaceRd$ is \prope{open}.\label{item:dspace_01_oballo}\\
        For example, the \structe{open ball} $\ball{3}{2}$ is \prope{not open} 
        because $4\in\ball{3}{2}$ \emph{but} for all $0<\varepsilon<1$
        \\\indentx$\ball{4}{\varepsilon}=\intoo{4-\varepsilon}{4+\varepsilon}\setu\intoo{0}{\varepsilon}
           \nsubseteq\intoo{1}{5}
           = \ball{3}{2}$

  \item The \structe{open balls} of $\dspaceRd$ do not form a \structe{base} for a \structe{topology} on $\R$.\\
        This follows directly from \pref{item:dspace_01_oballo} and \prefpp{thm:baseoball}.

  \item In the \structe{distance space} $\dspaceRd$, limits are \prope{not unique};\\
        For example, the sequence $\seqn{\sfrac{1}{n}}_1^\infty$ converges both to the limit $0$ and the limit $4$ in $\dspaceRd$:
        \label{item:dspace_01}
    \\\indentx$\begin{array}{*{5}{>{\ds}l}}
      \lim_{n\to\infty}\distance{x_n}{0} 
        &\eqd \lim_{n\to\infty}\distance{\sfrac{1}{n}}{0}
        &\eqd \lim_{n\to\infty}\abs{\sfrac{1}{n}-0}
        &= 0
       %&\implies \lim_{n\to\infty}\seqn{\sfrac{1}{n}}=0
        &\quad\implies\quad \seqn{\sfrac{1}{n}}\to0
        \\
      \lim_{n\to\infty}\distance{x_n}{4} 
        &\eqd \lim_{n\to\infty}\distance{\sfrac{1}{n}}{4}
        &\eqd \lim_{n\to\infty}\seqn{\sfrac{1}{n}}
        &= 0
       %&\implies \lim_{n\to\infty}\seqn{\sfrac{1}{n}}=4
        &\quad\implies\quad \seqn{\sfrac{1}{n}}\to4
    \end{array}$

  \item The \structe{topological space $\topspaceX$ induced by $\dspaceRd$} also 
        yields limits of $0$ and $4$ for the sequence $\seqn{\sfrac{1}{n}}_1^\infty$, just 
        as it does in \pref{item:dspace_01}.
        This is largely due to the fact that, for small $\varepsilon$, 
        the open balls $\ball{0}{\varepsilon}$ and $\ball{4}{\varepsilon}$ are \prope{open}.
    \begin{align*}
        \text{$\ball{0}{\varepsilon}$ is \prope{open}}
          &\implies \text{for each $\setU\in\topT$ that contains $0$, $\exists\xN\in\Zp \st \sfrac{1}{n}\in\setU\quad\forall n>\xN$}
        \\&\iff \seqn{\sfrac{1}{n}}\to0
          \qquad\text{by definition of \prope{convergence} \xref{def:converge}}
        \\
        \text{$\ball{4}{\varepsilon}$ is \prope{open}}
          &\implies \text{for each $\setU\in\topT$ that contains $4$, $\exists\xN\in\Zp \st \sfrac{1}{n}\in\setU\quad\forall n>\xN$}
        \\&\iff \seqn{\sfrac{1}{n}}\to4
          \qquad\text{by definition of \prope{convergence} \xref{def:converge}}
    \end{align*}  

  %\item The distance function $\distancen\in\clF{\R\times\R}{\R}$ is \prope{discontinuous} \label{item:dspace_01_discon}
  %      with respect to the \structe{Euclidean topologies of $\R$ and $\R^2$}.
  %      Consider the following: 
  %      Let $\opair{x}{y}\in\R^2$ be an \structe{ordered pair}.
  %      Let $\ball{\opair{x}{y}}{\varepsilon}$ be an \structe{open ball} in the \structe{Euclidean metric space}
  %      $\metspace{\R^2}{\metrican}$ where
  %      \\$\metrica{\opair{x_1}{x_2}}{\opair{y_1}{y_2}}\eqd\sqrt{(x_1-x_2)^2+(y_1-y_2)^2}$\qquad%
  %        ${\scy\forall \opair{x_1}{x_2},\opair{y_1}{y_2}\in\R^2}$.\footnote{
  %      Note that $\C\eqd\R^2$, where $\C$ is the set of \structe{complex numbers}, and thus $\dspaceRd$ 
  %      can be viewed as a \structe{distance space} over $\C$ and the  
  %      \fncte{distance function} $\metrican$ somewhat like a \fncte{projection} 
  %      from $\C^2=\R^4$ onto a ``real axis" $\R$ in $\C$.
  %      It is like a \fncte{projection} in the sense that it maps from $\C^2=\R^4$ to $\R$ and also because 
  %      it has a property similar to the \prope{idempotent} property in the sense that 
  %      \\\indentx$\metrica{\opair{\metrica{\opair{x_1}{x_2}}{\opair{y_1}{y_2}}}{0}}{\opair{0}{0}}=
  %      \metrica{\opair{x_1}{x_2}}{\opair{y_1}{y_2}}$ .
  %      } 
  %      \qquad Then
  %      \begin{align*}
  %        \distancen^{-1}\brs{\intoo{0}{2}}
  %          &\supseteq \setn{4}\times\intoo{0}{2}
  %          && \text{by definition of \fncte{distance} $\distancen$}
  %        \\&\nsupseteq\ball{\opair{4}{1}}{\varepsilon}\quad\forall \varepsilon>0
  %          && \text{by definition of \structe{open ball} $\balln$}
  %        \\&\implies \text{$\distancen^{-1}\brs{\intoo{0}{2}}$ is \prope{not open}}
  %          &&\text{by \prefpp{def:dspace_open}}
  %        \\&\implies \text{$\distancen$ is \prope{discontinuous}}
  %          &&\text{by \prefpp{def:continuous}}
  %      \end{align*}

  \item The distance function $\distancen$ is \prope{discontinuous} \xref{def:dspace_cont}:
  %\item As an alternative to the \pref{item:dspace_01_discon} method of using \structe{open set}s to prove that $\distancen$ is \prope{discontinuous}, we could use convergent sequences to prove the same:
    \begin{align*}
      \lim_{n\to\infty}\seqn{\distance{1-\sfrac{1}{n}}{4-\sfrac{1}{n}}}
        &=\lim_{n\to\infty}\seqn{\abs{\brp{1-\sfrac{1}{n}}-\brp{4-\sfrac{1}{n}}}}
         = \abs{1-4} = 3 \neq 4 = \distance{0}{4}
      \\&= \distance{\lim_{n\to\infty}\seqn{1-\sfrac{1}{n}}}{\lim_{n\to\infty}\seqn{4-\sfrac{1}{n}}}
    \end{align*}
   %which by \prefpp{cor:limcont} implies that $\distancen$ is \prope{discontinuous}.
   %   \item Therefore, by \prefpp{cor:limcont}, $\distancen$ is \prope{discontinuous}.
   % \end{enumerate}
\end{enumerate}
\end{example}

In a \structe{metric space}, all \prope{convergent} sequences are also \prope{Cauchy}.
However, this is not the case for all \structe{distance space}s, as demonstrated next:
%---------------------------------------
\begin{example}
\footnote{
  The distance function $\distancen$ and \prefp{item:dspace_1n_cauchy} 
  can in essence be found in \citerpg{blumenthal1953}{9}{0828402426}
  }
\label{ex:dspace_1n}
%---------------------------------------
%Let $\setX\eqd\set{\frac{1}{n}}{n=1,2,3,\ldots}\setu\setn{0}$ be a \structe{set}.
%The function $\distance{x}{y}\in\clF{\setX^2}{\R}$ such that
The function $\distance{x}{y}\in\clF{\R\times\R}{\R}$ such that
\\\indentx$\distance{x}{y} \eqd \brb{\begin{array}{lMD}
      \abs{x-y} & for $x=0$ or $y=0$ or $x=y$ & (\prope{Euclidean})\\
      1         & otherwise                   & (\prope{discrete})
    \end{array}}$\quad is a \fncte{distance} on $\R$. 
\\
Note some characteristics of the \structe{distance space} $\dspaceRd$:
\begin{enumerate}
  \item $\dspaceX$ is not a \structe{metric space} because the \prope{triangle inequality} does not hold:
      \\\indentx
        $\distance{\frac{1}{4}}{\frac{1}{2}}
        = 1
        \nleq \frac{3}{4}
        = \abs{\frac{1}{4}-0}     + \abs{0-\frac{1}{2}}
        = \distance{\frac{1}{4}}{0} + \distance{0}{\frac{1}{2}}
        $

  \item The \structe{open ball} $\ball{\frac{1}{4}}{\frac{1}{2}}$ is \prope{not open}
        because for any $\varepsilon\in\Rp$, no matter how small,  \label{item:dspace_1n_oballo}
        \\\indentx
         $\ball{0}{\varepsilon} = \intoo{-\varepsilon}{+\varepsilon}
            \nsubseteq \setn{0,\,\frac{1}{4}}
            =\set{x\in\setX}{\distance{\frac{1}{4}}{x}<\frac{1}{2}}
            \eqd\ball{\frac{1}{4}}{\frac{1}{2}}$
  
  \item Even though not all the \structe{open ball}s are \prope{open}, 
        it is still possible to have an \structe{open set} in $\dspaceX$. 
        For example, the set $\setU\eqd\setn{1,\,2}$ is \prope{open}:
        \\\indentx$\begin{array}{rclclclcl}
            \ball{1}{1}
            &\eqd& \set{x\in\setX}{\distance{1}{x}<1}
            &=& \setn{1}
            &\subseteq& \setn{1,\,2}
            &\eqd& \setU
            \\
            \ball{2}{1}
            &\eqd& \set{x\in\setX}{\distance{2}{x}<1}
            &=& \setn{2}
            &\subseteq& \setn{1,\,2}
            &\eqd& \setU
          \end{array}$

  \item By \pref{item:dspace_1n_oballo} and \prefpp{thm:baseoball}, 
        the \structe{open ball}s of $\dspaceRd$ do not form a \structe{base} for a \structe{topology} on $\R$.

  \item Even though the open balls in $\dspaceRd$ do not induce a topology on $\setX$, it is still possible to 
        find a set of \structe{open set}s in $\dspaceX$ that \emph{is} a topology. 
        For example, the set
          $\setn{\emptyset,\,\setn{1,2},\,\R}$
        is a topology on $\R$.

  \item In $\dspaceRd$, limits of \prope{convergent} sequences are \prope{unique}:
    \\\indentx$\ds
      \seqn{x_n}\to x \quad\implies\quad
      \lim_{n\to\infty}\distance{x_n}{x}=\brb{\begin{array}{rclMD}
        \lim\abs{x_n-0} &=& 0 & for $x=0$ & OR\\
            \abs{x-x}   &=& 0 & for constant $\seqn{x_n}$ for $n>\xN$ & OR\\
                    1   &\neq& 0 & otherwise
      \end{array}}$\\
     which says that there are only two ways for a sequence to converge: either $x=0$ or the sequence eventually becomes constant
     (or both). Any other sequence will \prope{diverge}. Therefore we can say the following:
    \begin{enumerate}
      \item If $x=0$ and the sequence is not constant, then the limit is \prope{unique} and $0$.
      \item If $x=0$ and the sequence is constant, then the limit is \prope{unique} and $0$.
      \item If $x\neq0$ and the sequence is constant, then the limit is \prope{unique} and $x$.
      \item If $x\neq0$ and the sequence is not constant, then the sequence diverges and there is no limit.
    \end{enumerate}

  \item In $\dspaceRd$, a \prope{convergent} sequence is not necessarily \prope{Cauchy}.\label{item:dspace_1n_cauchy}
    For example,
    \begin{enumerate}
      \item the sequence $\seqnZp{\sfrac{1}{n}}$ is \prope{convergent} with limit $0$:
       $\ds\lim_{n\to\infty}\distance{\sfrac{1}{n}}{0}
        = \lim_{n\to\infty}\sfrac{1}{n}
        = 0$
      \item However, even though $\seqn{\sfrac{1}{n}}$ is \prope{convergent}, it is \prope{not Cauchy}:
       $\ds\lim_{n,m\to\infty}\distance{\sfrac{1}{n}}{\sfrac{1}{m}}
       = 1
       \neq 0
      $
    \end{enumerate}

  \item The \fncte{distance function} $\distancen$ is \prope{discontinuous} in $\dspaceX$:
    \begin{align*}
        \lim_{n\to\infty}\seqn{\distance{\sfrac{1}{n}}{2-\sfrac{1}{n}}} 
        &= 1 
      \\&\neq 2 = \distance{0}{2} = \distance{\lim_{n\to\infty}\seqn{\sfrac{1}{n}}}{\lim_{n\to\infty}\seqn{2-\sfrac{1}{n}}}
    \end{align*}
    %which by \prefpp{cor:limcont} implies that $\distancen$ is \prope{discontinuous}.
\end{enumerate}
\end{example}

%In any \structe{distance space} in which the \prope{triangle inequality} holds 
%(a \structe{metric space}),
%the \fncte{distance} function is always \prope{continuous}. % \xref{prop:metric_continuous}.
%This is even true in the case of the \fncte{discrete metric} % \xref{ex:dmetric}
%which is induced from any other metric using the \prope{non-continuous} 
%\fncte{discrete metric preserving function}. % \xref{ex:mpf_discrete}.
%However, distance functions are not all \prope{continuous}, as demonstrated next.
%---------------------------------------
\begin{example}
\footnote{
  The distance function $\distancen$ and \prefp{item:dspace_21_cont}
  can in essence be found in \citerpg{blumenthal1953}{9}{0828402426}
  }
\label{ex:dspace_21}
%---------------------------------------
%Let $\setX\eqd\set{\frac{1}{n}}{n=1,2,3,\ldots}\setu\setn{0}$ be a \structe{set}.
The function $\distance{x}{y}\in\clF{\R\times\R}{\R}$ such that
\\\indentx$\distance{x}{y} \eqd \brb{\begin{array}{rMD}
      2\abs{x-y} & $\forall \opair{x}{y}\in\setn{\opair{0}{1},\,\opair{1}{0}}$ & (\prope{dilated Euclidean})\\
       \abs{x-y} & otherwise                                                   & (\prope{Euclidean})
    \end{array}}$\quad is a \fncte{distance} on $\R$.
\\
Note some characteristics of the \structe{distance space} $\dspaceRd$:
\begin{enumerate}
  \item $\dspaceRd$ is \emph{not} a \structe{metric space} because $\distancen$ does \emph{not} 
        satisfy the \prope{triangle inequality}:
      \\\indentx$\ds\distance{0}{1}
        \eqd 2\abs{0-1}
        = 2
        \nleq 1
        = \abs{0-\sfrac{1}{2}}  + \abs{\sfrac{1}{2}-1}
        \eqd \distance{0}{\sfrac{1}{2}} + \distance{\sfrac{1}{2}}{1}
        $

  \item The function $\distancen$ is \prope{discontinuous}: \label{item:dspace_21_cont}
        %In particular, it is \prope{not continous} at the point $\opair{\lim\seqn{1-\frac{1}{n}}}{\lim\seqn{\frac{1}{n}}}$.
    \begin{align*}
      &\lim_{n\to\infty}\seqn{\distance{1-\sfrac{1}{n}}{\sfrac{1}{n}}}
        \eqd\lim_{n\to\infty}\seqn{\abs{1-\sfrac{1}{n}-\sfrac{1}{n}}}
        =1
      \\&\qquad\neq 2 
        =2\abs{0-1}
        \eqd\distance{0}{1}
        =\distance{\lim_{n\to\infty}\seqn{1-\sfrac{1}{n}}}{\lim_{n\to\infty}\seqn{\sfrac{1}{n}}}
    \end{align*}
    %which by \prefpp{cor:limcont} implies that $\distancen$ is \prope{discontinuous}.

  \item In $\dspaceX$, \structe{open ball}s are \prope{open}: \label{item:dspace_21_oballo}
    \begin{enumerate}
      \item $\metrica{x}{y}\eqd\abs{x-y}$ is a \fncte{metric} and thus all open balls in that do not contain both $0$ and $1$ are \prope{open}.
      \item By \prefpp{ex:mpf_ascaled}, $\metricb{x}{y}\eqd2\abs{x-y}$ is also a \fncte{metric} and thus all open balls containing $0$ and $1$ only are \prope{open}.
      \item The only question remaining is with regards to open balls that contain $0$, $1$ and some other element(s) in $\R$.
            But even in this case, open balls are still open. For example:
        \\\indentx$\ball{-1}{2} = \intoo{-1}{2} = \intoo{-1}{1}\setu\intoo{1}{2}$\\
        Note that both $\intoo{-1}{1}$ and $\intoo{1}{2}$ are \prope{open}, and thus by \prefpp{thm:dspace_open},
        $\ball{-1}{2}$ is \prope{open} as well.
    \end{enumerate}

  \item By \pref{item:dspace_21_oballo} and \prefpp{thm:baseoball}, 
        the \structe{open ball}s of $\dspaceRd$ \emph{do} form a \structe{base} for a \structe{topology} on $\R$.

  \item In $\dspaceX$, the limits of \prope{convergent} sequences are \prope{unique}.
        This is demonstrated in \prefpp{ex:pdspace_21} using additional structure developed in \pref{chp:pdspace}.

  \item In $\dspaceX$, \prope{convergent} sequences are \prope{Cauchy}.\\
        This is also demonstrated in \prefpp{ex:pdspace_21}.

\end{enumerate}
\end{example}

The \fncte{distance function}s in \prefpp{ex:dspace_01}--\prefpp{ex:dspace_21} were all \prope{discontinuous}.
In the absence of the \prope{triangle inequality} and in light of these examples, 
one might try replacing the \prope{triangle inequality} with the weaker requirement of \prope{continuity}.
However, as demonstrated by the next example, this also leads to an arguably disastrous result.
%---------------------------------------
\begin{example}
\footnote{
  \citerppg{blumenthal1953}{12}{13}{0828402426},
  \citerppg{laos1998}{118}{119}{9810231806}
  }
\label{ex:dspace_xy2}
%---------------------------------------
The function $\distancen\in\clF{\R\times\R}{\R}$ such that $\distance{x}{y}\eqd(x-y)^2$ is a \fncte{distance} on $\R$.
\\Note some characteristics of the \structe{distance space} $\dspaceRd$:
\begin{enumerate}
  \item $\dspaceRd$ is \emph{not} a \structe{metric space} because the \prope{triangle inequality} does not hold:
    \\\indentx$\ds\distance{0}{2} \eqd \brp{0-2}^2 = 4 \nleq 2 = \brp{0-1}^2 + \brp{1-2}^2 \eqd \distance{0}{1} + \distance{1}{2} $

  \item The \fncte{distance function} $\distancen$ is \prope{continuous} in $\dspaceX$.
        This is demonstrated in the more general setting of \pref{chp:pdspace} in \prefpp{ex:pdspace_xy2}.

  \item Calculating the length of curves in $\dspaceX$ leads to a paradox:\footnote{
        This is the method of ``inscribed polygons" for calculating the length of a curve and goes back to Archimedes:
        \citerpg{brunschwig2003}{26}{0674021568},
        \citerpc{walmsley1920}{200}{\textsection158},
        }
    \begin{enumerate}
      \item Partition $\intcc{0}{1}$ into $2^\xN$ consecutive line segments connected at the points 
            \\\indentx$\seqn{0,\,\frac{1}{2^\xN},\,\frac{2}{2^\xN},\,\frac{3}{2^\xN},\,\ldots,\,\frac{2^{\xN-1}1}{2^\xN},\,1}$
      \item Then the distance, as measured by $\distancen$, between any two consecutive points is
            \\\indentx$\distance{p_n}{p_{n+1}}\eqd\brp{p_n-p_{n+1}}^2=\brp{\frac{1}{2^\xN}}^2=\frac{1}{2^{2\xN}}$
      \item But this leads to the paradox that the total length of $\intcc{0}{1}$ is 0:
            \\\indentx$\ds %1=\brp{0-1}^2\eqd\distance{0}{1}
              \lim_{\xN\to\infty}\sum_{n=0}^{2^{\xN}-1}\frac{1}{2^{2\xN}}
              =\lim_{\xN\to\infty}\frac{2^\xN}{2^{2\xN}}
              =\lim_{\xN\to\infty}\frac{1}{2^{\xN}}
              =0
             $
    \end{enumerate}
\end{enumerate}
\end{example}

  %============================================================================
% LaTeX File
% Daniel J. Greenhoe
%============================================================================

%======================================
\chapter{Distance Spaces with Power Triangle Inequalities}
\label{chp:trirel}
\label{chp:pdspace}
%======================================
%======================================
\section{Definitions}
\label{sec:pdspace_def}
%======================================
%%---------------------------------------
%\begin{definition}
%\footnote{
%
%  }
%\label{def:ftri}
%%---------------------------------------
%%Let $\dspaceX$ be a \structe{distance space} \xref{def:dspace}.
%Let $\Rp$ be the set of all \structe{positive real numbers} 
%and $\Rx$ be the set of \structe{extended real numbers} \xref{def:Rx}.
%\defboxp{
%  A function $\ptf$ is a \fnctd{triangle function} if
%  \\\indentx$\begin{array}{FrclCD}
%    1. & \ftri(\ff,\fg) &=&     \ftri(\fg,\ff)  & \forall \ff,\fg\in  & (\prope{symmetric})
%  \\2. & \ftri(\ff,\fg) &\orel& \ftri(\fh,\fk)\quad\text{whenever $\ff\orel\fh$ and $\fg\orel\fk$} & \forall \ff,\fg,\fh,\fk\in & (\prope{order preserving})
%  \\3. & 
%  \end{array}$
%  }
%\end{definition}

This chapter introduces a relation called the \rele{power triangle inequality} \xref{def:trirel}.
It is a generalization of other common relations, including the \rele{triangle inequality} \xref{def:trirels}.
The \rele{power triangle inequality} is defined in terms of a function herein called the \fncte{power triangle function} (next definition).
This function is a special case of the \fncte{power mean} with $\xN=2$ and $\lambda_1=\lambda_2=\frac{1}{2}$ \xref{def:pmean}.
\fncte{Power mean}s have the attractive properties of being \prope{continuous} and \prope{strictly monontone} with respect
to a free parameter $p\in\Rx$ \xref{thm:pmean_continuous}.
This fact is inherited and exploited by the \rele{power triangle inequality} \xref{cor:tri_mono}.
%---------------------------------------
\begin{definition}
\footnote{
  \citeP{greenhoe2015pds}
  }
\label{def:ptf}
%---------------------------------------
Let $\dspaceX$ be a \structe{distance space} \xref{def:dspace}.
Let $\Rp$ be the set of all \structe{positive real numbers} 
and $\Rx$ be the set of \structe{extended real numbers} $\brp{\Rx\eqd\R\setu\setn{-\infty,\,\infty}}$. %\xref{def:Rx}.
\defboxp{
  The \fnctd{power triangle function} $\ptf$ on $\dspaceX$ is %, for some $\otriple{x}{y}{z}\in\setX^3$, 
  defined as
  \\\indentx$\begin{array}{rc>{\ds}lCC}
    \ptfD
      &\eqd& 2\sigma\brs{\frac{1}{2}\distancep{p}{x}{z} + \frac{1}{2}\distancep{p}{z}{y}}^\frac{1}{p} 
      &      \forall \opair{p}{\sigma}\in\Rx\times\Rnn, 
      &      x,y,z\in\setX
  \end{array}$
  }
\end{definition}

%---------------------------------------
\begin{remark}
\footnote{
  \citerp{sherstnev1962}{4},
  %\citer{sherstnev1964},
  \citerpgc{schweizer1983}{9}{0486143759}{(1.6.1)--(1.6.4)},
  \citePp{bessenyei2014}{2}
  }
%---------------------------------------
In the field of \hie{probabilistic metric spaces}, 
a function called he \fncte{triangle function} was introduced by Sherstnev in 1962.
However, the \fncte{power triangle function} as defined in this present paper is \emph{not} a special case of
(is not compatible with) the \fncte{triangle function} of Sherstnev.
Another definition of \fncte{triangle function} has been offered by Bessenyei in 2014
with special cases of $\Phi(u,v)\eqd c(u+v)$ and $\Phi(u,v)\eqd(u^p+v^p)^\frac{1}{p}$,
which \emph{are} similar to the definition of \fncte{power triangle function} offered in this present paper.
\end{remark}

%---------------------------------------
\begin{definition}
\label{def:trirel}
\label{def:pdspace}
\label{def:ptineq}
%---------------------------------------
Let $\dspaceX$ be a \structe{distance space}.
Let $\clRxxx$ be the set of all trinomial \structe{relation}s\ifsxref{relation}{def:clRxy} on $\setX$.
\defboxp{
  A relation $\trirelD$ in $\clRxxx$
  is a \reld{power triangle inequality} on $\dspaceX$ if
  \\\indentx$\ds\trirelD\eqd\set{\otriple{x}{y}{z}\in\setX^3}{\distance{x}{y}\le\ptfD}$
    \qquad for some $\opair{p}{\sigma}\in\Rx\times\Rp$.
  \\
  The tupple $\pdspaceX$ is a \structd{power distance space} and $\distancen$ a \fnctd{power distance} or \fnctd{power distance function} if 
  $\dspaceX$ is a \structe{distance space} in which the 
  \structe{triangle relation} $\trirelD$ holds.
 %\\\indentx$\brb{x,y,z\in\setX} \quad\implies\quad \brb{\distance{x}{y}\le\ptfD}$
  %The notation $\trirel(p,\sigma;x,y,z)$ indicates that the triple $\otriple{x}{y}{z}$ is in 
  %the \rele{power triangle inequality} $\trirelD$ ($\otriple{x}{y}{z}\in\trirelD$).
  %
  }
\end{definition}

%This paper introduces a new relation called the \rele{power triangle inequality} (next).
%%---------------------------------------
%\begin{definition}
%\label{def:trirel}
%%---------------------------------------
%Let $\clRxxx$ be the set of all trinomial \structe{relation}s \xref{def:clRxy} on a 
%\structe{set} $\setX$.
%Let $\dspaceX$ be a \structe{distance space} \xref{def:dspace} on $\setX$.
%\defboxp{
%  A relation $\trirel(x,y,z)$ in $\clRxxx$ is a \reld{power triangle inequality} on $\dspaceX$ if
%  $\otriple{x}{x}{z}\in\trirel\quad\forall x,z\in\setX$.
%  }
%\end{definition}
%
%%---------------------------------------
%\begin{definition}
%\footnote{
%  \citePp{bessenyei2014}{2}
%  \citePpc{czerwik1993}{5}{\structe{b-metric}; (1),(2),(5)},
%  \citeP{fagin2003p},
%  \citePc{fagin2003}{Definition 4.2 (Relaxed metrics)},
%  \citePpc{xia2009}{453}{Definition 2.1},
%  \citerpgc{heinonen2001}{109}{0387951040}{14.1 Quasimetric spaces.},
%  \citerpgc{kirk2014}{113}{3319109278}{Definition 12.1},
%  \citerpg{deza2014}{7}{3662443422}
%  }
%%---------------------------------------
%Special cases of \structe{power triangle inequality}s \xref{def:trirel} on a \structe{distance space} $\dspaceX$ 
%are defined as follows:
%\\\defboxp{$\begin{array}{Fl@{\hspace{2pt}}lM}
% %1. &\trirel\eqd\setX^3                                                                                                              &\big\  & is the \reld{trivial power triangle inequality}.\\
%  1. &\trirel\eqd\big\{\otriple{x}{y}{z}\in\setX^3\,|\,\distance{x}{y} \orel \distance{x}{z}+\distance{z}{y}                                   &\big\} & is the \reld{triangle inequality}.\\
%  2. &\trirel\eqd\big\{\otriple{x}{y}{z}\in\setX^3\,|\,\distance{x}{y} \orel \sigma\brs{\distance{x}{z}+\distance{z}{y}}            ,\,\sigma>0&\big\} & is the \reld{relaxed triangle inequality}.\\
%  3. &\trirel\eqd\big\{\otriple{x}{y}{z}\in\setX^3\,|\,\distance{x}{y} \orel \max\setn{\distance{x}{z},\,\distance{z}{y}}                      &\big\} & is the \reld{inframetric inequality}.\\
%  4. &\trirel\eqd\big\{\otriple{x}{y}{z}\in\setX^3\,|\,\distance{x}{y} \orel \sigma\max\setn{\distance{x}{z},\,\distance{z}{y}}     ,\,\sigma>0&\big\} & is the \reld{\txsigma-inframetric inequality}.\\
%  5. &\trirel\eqd\big\{\otriple{x}{y}{z}\in\setX^3\,|\,\distance{x}{y} \orel \brs{\distancen^p(x,z) + \distancen^p(z,y)}^\frac{1}{p},\,p>0     &\big\} & is the \reld{$p^{th}$ order triangle inequality}.
%\end{array}$}
%\end{definition}

The \fncte{power triangle function} can be used to define some standard inequalities (next definition).
See \prefpp{cor:ftri_means} for some justification of the definitions.
%---------------------------------------
\begin{definition}
\footnote{
  \citePp{bessenyei2014}{2},
  \citePpc{czerwik1993}{5}{\structe{b-metric}; (1),(2),(5)},
  \citeP{fagin2003p},
  \citePc{fagin2003}{Definition 4.2 (Relaxed metrics)},
  \citePpc{xia2009}{453}{Definition 2.1},
  \citerpgc{heinonen2001}{109}{0387951040}{14.1 Quasimetric spaces.},
  \citerpgc{kirk2014}{113}{3319109278}{Definition 12.1},
  \citerpg{deza2014}{7}{3662443422},
  \citePp{hoehn1985}{151},
  \citerpgc{gibbons1977}{51}{1611971101}{\ope{square-mean-root} (\ope{SMR}) (2.4.1)},
  \citerc{euclid}{triangle inequality---Book I Proposition 20}
  }
\label{def:trirels}
%---------------------------------------
Let $\trirelD$ be a \rele{power triangle inequality} on 
a \structe{distance space} $\dspaceX$.
\\\defboxp{$\begin{array}{*{2}{Fl@{\,}r@{\,}@{\,}c@{\,}@{\,}l@{\,}M}}
  1. & \trirel(& \infty      ,&\sfrac{\sigma}{2}  ;\distancen&)&{\scs is the} \reld{\txsigma-inframetric inequality} &  6. & \trirel(& 1           ,&1          ;\distancen &)&{\scs is the} \reld{triangle inequality}
\\2. & \trirel(& \infty      ,&\frac{1}{2}        ;\distancen&)&{\scs is the} \reld{inframetric inequality}          &  7. & \trirel(& 0           ,&\frac{1}{2};\distancen &)&{\scs is the} \reld{geometric inequality}           
\\3. & \trirel(& 2           ,&\sfrac{\sqrt{2}}{2};\distancen&)&{\scs is the} \reld{quadratic inequality}            &  8. & \trirel(&-1           ,&\frac{1}{4};\distancen &)&{\scs is the} \reld{harmonic inequality}            
\\4. & \trirel(& 1           ,&\sigma             ;\distancen&)&{\scs is the} \reld{relaxed triangle inequality}     &  9. & \trirel(&-\infty      ,&\frac{1}{2};\distancen &)&{\scs is the} \reld{minimal  inequality}            
\\5. & \trirel(& \sfrac{1}{2},&2                  ;\distancen&)&{\scs is the} \reld{square mean root inequality}     &     &         &              &                       & &                                      
%\\5. & \trirel(& \sfrac{1}{2},&2                  ;\distancen&)&\mc{6}{@{\,}M}{{\scs is the} \reld{square mean root inequality}}    
\end{array}$}
\end{definition}

%%---------------------------------------
%\begin{definition}
%\footnote{
%  \citePp{bessenyei2014}{2},
%  \citePpc{czerwik1993}{5}{\structe{b-metric}; (1),(2),(5)},
%  \citeP{fagin2003p},
%  \citePc{fagin2003}{Definition 4.2 (Relaxed metrics)},
%  \citePpc{xia2009}{453}{Definition 2.1},
%  \citerpgc{heinonen2001}{109}{0387951040}{14.1 Quasimetric spaces.},
%  \citerpgc{kirk2014}{113}{3319109278}{Definition 12.1},
%  \citerpg{deza2014}{7}{3662443422}
%  }
%\label{def:trirels}
%%---------------------------------------
%Let $\dspaceX$ be a \structe{distance space} \xref{def:dspace}.
%\\\defboxp{$\begin{array}{FrcllM}
%  1. &\distance{x}{y} &\orel& \distance{x}{z}+\distance{z}{y}                         & \forall x,y,z\in\setX,\,          & is the \reld{triangle inequality}.\\
%  2. &\distance{x}{y} &\orel& \sigma\brs{\distance{x}{z}+\distance{z}{y}}             & \forall x,y,z\in\setX,\, \sigma>0 & is the \reld{relaxed triangle inequality}.\\
%  3. &\distance{x}{y} &\orel& \max\setn{\distance{x}{z},\,\distance{z}{y}}            & \forall x,y,z\in\setX,\,          & is the \reld{inframetric inequality}.\\
%  4. &\distance{x}{y} &\orel& \sigma\max\setn{\distance{x}{z},\,\distance{z}{y}}      & \forall x,y,z\in\setX,\, \sigma>0 & is the \reld{\txsigma-inframetric inequality}.\\
%  5. &\distance{x}{y} &\orel& \brs{\distancen^p(x,z) + \distancen^p(z,y)}^\frac{1}{p} & \forall x,y,z\in\setX,\, p>0      & is the \reld{$p^{th}$ order triangle inequality}.
%\end{array}$}
%\end{definition}

%---------------------------------------
\begin{definition}
\footnote{
  \structb{metric space}:
  \citerp{dieudonne1969}{28},
  \citerp{copson1968}{21},
  \citorp{hausdorff1937e}{109},
  \citer{frechet1928},
  \citePp{frechet1906}{30} 
  %\citor{hausdorff1914}\\
  %\cithrpg{ab}{34}{0120502577} 
  %\citerc{euclid}{triangle inequality---Book I Proposition 20}
  \structb{near metric space}:
  \citePpc{czerwik1993}{5}{\structe{b-metric}; (1),(2),(5)},
  \citeP{fagin2003p},
  \citePc{fagin2003}{Definition 4.2 (Relaxed metrics)},
  \citePpc{xia2009}{453}{Definition 2.1},
  \citerpgc{heinonen2001}{109}{0387951040}{14.1 Quasimetric spaces.},
  \citerpgc{kirk2014}{113}{3319109278}{Definition 12.1},
  \citerpg{deza2014}{7}{3662443422}
  }
%\label{def:metric}
\label{def:mspace}
\label{def:nmetric}
\label{def:pdspace_spaces}
%---------------------------------------
Let $\dspaceX$ be a \structe{distance space} \xref{def:dspace}.
\defbox{\begin{array}{FMMMM}
    1. & $\dspaceX$ is a &\structd{metric space}               & if the \rele{triangle inequality}             & holds in $\setX$.
  \\2. & $\dspaceX$ is a &\structd{near metric space}          & if the \rele{relaxed triangle inequality}     & holds in $\setX$.
  \\3. & $\dspaceX$ is an&\structd{inframetric space}          & if the \rele{inframetric inequality}          & holds in $\setX$.
  \\4. & $\dspaceX$ is a &\structd{\txsigma-inframetric space} & if the \rele{\txsigma-inframetric inequality} & holds in $\setX$.
\end{array}}
\end{definition}

%%---------------------------------------
%\begin{definition}
%\footnote{
%  \citerp{dieudonne1969}{28},
%  \citerp{copson1968}{21},
%  \citorp{hausdorff1937e}{109},
%  \citer{frechet1928},
%  \citePp{frechet1906}{30} 
%  %\citor{hausdorff1914}\\
%  %\cithrpg{ab}{34}{0120502577} 
%  \citerc{euclid}{triangle inequality---Book I Proposition 20}
%  }
%\label{def:metric}
%%---------------------------------------
%%Let $\setX$ be a set.
%\defboxp{A \fncte{distance space} \xref{def:dspace} $\dspaceX$ is a \structd{metric space} if
%the \prope{triangle inequality} \xref{def:trirels} holds for all of $\setX$.
%In this case, $\distancen$ is a \fncte{metric}.}
%%  \\\indentx$\begin{array}{rcl CDDr@{}l@{\,}D}
%%        \distance{x}{y} &\le& \distance{x}{z}+\distance{z}{y}       & \forall x,y,z \in\setX & (\prope{subadditive}/\prope{triangle inequality}).\footnotemark
%%    \end{array}$\\
%%  The ordered pair $\dspaceX$ is a \structd{metric space} if $\distancen$ is a \fncte{metric}.
%\end{definition}


%%---------------------------------------
%\begin{definition}
%\label{def:metric}
%\label{def:(X,d)}
%\footnote{
%  \citerp{dieudonne1969}{28},
%  \citerp{copson1968}{21},
%  \citorp{hausdorff1937e}{109},
%  \citer{frechet1928},
%  \citePp{frechet1906}{30} 
%  %\citor{hausdorff1914}\\
%  %\cithrpg{ab}{34}{0120502577} 
%  \citerc{euclid}{triangle inequality---Book I Proposition 20}
%  }
%\index{space!metric}
%%---------------------------------------
%Let $\setX$ be a set.
%\defboxp{
%  A function $\distancen$ in the set $\clFxr$ is a \fnctd{metric} if 
%  \\\indentx$\begin{array}{F rcl CDDr@{}l@{\,}D}
%        \cline{8-8}
%        1. & \distance{x}{y} &\ge& 0                                 & \forall x,y   \in\setX & (\prope{non-negative})   & and &&\vline
%      \\2. & \distance{x}{y} &=  & 0  \iff x=y                       & \forall x,y   \in\setX & (\prope{nondegenerate})  & and &&\vline & \fncte{distance}
%      \\3. & \distance{x}{y} &=  & \distancen(y,x)                     & \forall x,y   \in\setX & (\prope{symmetric})      & and &&\vline
%      \\\cline{8-8}
%        4. & \distance{x}{y} &\le& \distance{x}{z}+\distance{z}{y}       & \forall x,y,z \in\setX & (\prope{subadditive}/\prope{triangle inequality}).\footnotemark
%    \end{array}$\\
%  The ordered pair $\dspaceX$ is a \structd{metric space} if $\distancen$ is a \fncte{metric}.
%  }
%\end{definition}
%%\footnotetext{\citorc{euclid}{Book I Proposition 20}}



%%---------------------------------------
%\begin{definition}
%\label{def:qmetric}
%\footnote{
%  \citerppg{deza2014}{6}{7}{3662443422},
%  \citerpg{deza2006}{4}{0444520872},
%  \citePpc{wilson1931}{675}{\textsection 1.},
%  \citeP{ribeiro1943},
%  \citePpc{kelly1963}{71}{Introduction},
%  \citeP{patty1967},
%  \citePp{stoltenberg1969}{65},
%  \citerpgc{grabiec2006}{3}{1594549176}{Introduction},
%  }
%\index{space!metric}
%%---------------------------------------
%%Let $\setX$ be a set and $\Rnn$ the set of non-negative real numbers.
%\defboxp{
%  A function $\hxs{\distancen}\in\clF{\setX\times\setX}{\R}$ \xref{def:clFxy} is an \fnctd{asymmetric metric} on a \structe{set} $\setX$ if
%  \\\indentx$\begin{array}{F rcl CDD}
%        1. & \distance{x}{y} &\ge& 0                                 & \forall x,y   \in\setX & (\prope{non-negative})   & and 
%      \\2. & \distance{x}{y} &=  & 0  \iff x=y                       & \forall x,y   \in\setX & (\prope{nondegenerate})  & and 
%      \\3. & \distance{x}{y} &\le& \distance{x}{z}+\distance{z}{y}       & \forall x,y,z \in\setX & (\prope{subadditive} / \prope{triangle inequality}).
%    \end{array}$
%  \\The ordered pair $\dspaceX$ is an \structd{asymmetric metric space} if $\distancen$ is an \fncte{asymmetric metric} on $\setX$.\\
%  An \fncte{asymmetric metric} is also called a \fnctd{directed metric}.\footnotemark
%  }
%\footnotetext{
%  In the literature, the term \fnctd{quasi-metric} can refer to either what is here called a 
%  \fncte{asymmetric metric} \xref{def:qmetric} or a \fncte{near metric} \xref{def:nmetric}.
%  This paper avoids the terminology conflict by not further using the term \fncte{quasi-metric}.
%  }
%\end{definition}
%
%%---------------------------------------
%\begin{definition}
%\footnote{
%  \citePpc{czerwik1993}{5}{\structe{b-metric}; (1),(2),(5)},
%  \citeP{fagin2003p},
%  \citePc{fagin2003}{Definition 4.2 (Relaxed metrics)},
%  \citePpc{xia2009}{453}{Definition 2.1},
%  \citerpgc{heinonen2001}{109}{0387951040}{14.1 Quasimetric spaces.},
%  \citerpgc{kirk2014}{113}{3319109278}{Definition 12.1},
%  \citerpg{deza2014}{7}{3662443422}
%  }
%\label{def:nmetric}
%%---------------------------------------
%%Let $\setX$ be a set.
%\defboxp{A \fncte{distance space} \xref{def:dspace} $\dspaceX$ is a \structd{near metric space} if
%the \prope{relaxed triangle inequality} \xref{def:trirels} holds for all of $\setX$.
%In this case, $\distancen$ is a \fncte{near metric}.}
%%\defboxp{
%%  A function $\distancen$ in the set $\clFxr$ is a \fnctd{near metric} if for some $\sigma\ge1$
%%  \\\indentx$\begin{array}{F rcl CDDr@{}l@{\,}D}
%%        \cline{8-8}
%%        1. & \distance{x}{y} &\ge& 0                                 & \forall x,y   \in\setX & (\prope{non-negative})   & and &&\vline
%%      \\2. & \distance{x}{y} &=  & 0  \iff x=y                       & \forall x,y   \in\setX & (\prope{nondegenerate})  & and &&\vline & \fncte{distance}
%%      \\3. & \distance{x}{y} &=  & \distancen(y,x)                     & \forall x,y   \in\setX & (\prope{symmetric})      & and &&\vline
%%      \\\cline{8-8}
%%        4. & \distance{x}{y} &\le& \sigma\brs{\distance{x}{z}+\distance{z}{y}} & \forall x,y,z \in\setX & (\prope{relaxed subadditive} / \prope{relaxed triangle inequality}).
%%    \end{array}$\\
%%  The pair $\dspaceX$ is a \structd{near metric space} if $\distancen$ is a \fncte{near metric}.
%%  }
%\end{definition}

%=======================================
\section{Properties}
\label{sec:pdspace_prop}
%=======================================
%=======================================
\subsection{Relationships of the power triangle function}
\label{sec:pdspace_propr}
%=======================================
%%% We start with a proposition (next), that while it may be somewhat interesting, 
%%% it is debatable whether it is significant in the sense of Hardy's seriousness criterion.\footnote{
%%%   ``The ``seriousness" of a mathematical theorem lies,
%%%     not in its practical consequences, which are usually negligible,
%%%     but in the \emph{significance} of the mathematical ideas which it connects.
%%%     We may say, roughly, that a mathematical idea is ``significant" if it can be
%%%     connected, in a natural illuminating way,
%%%     with a large complex of other mathematical ideas.
%%%   --- G. H. Hardy (1940). Reference: \citerc{hardy1940}{section 11}
%%% }
%%% %---------------------------------------
%%% \begin{proposition}
%%% \label{prop:tri_mpf}
%%% %---------------------------------------
%%% Let $\pdspaceX$ be a \fncte{power distance space} \xref{def:pdspace}
%%% with \fncte{power triangle function} $\ptf$ \xref{def:ptf}
%%% \propbox{
%%%   \brb{\text{$\pwdfn$ is a \structe{metric} on $\setX$}} 
%%%   \quad\implies\quad
%%%   \brb{\begin{array}{M}
%%%     $\pwdfa{\opair{x}{z}}{\opair{z}{y}}\eqd\ptfD$\\
%%%     is a \fncte{power distance function} on $\setX^2$
%%%   \end{array}}
%%%   }
%%% \end{proposition}
%%% \begin{proof}
%%% %{thm:met_power}
%%% \begin{enumerate}
%%%   %\item Proof that $\opair{x}{z}=\opair{z}{y}\implies\pwdf{\opair{x}{z}}{\opair{z}{y}}=0$ for $r\in\intco{1}{\infty}$:
%%%   %  \begin{align*}
%%%   %    \pwdfn(\opair{x}{z},\opair{z}{y})
%%%   %      &\eqd \brp{\sum_{n=1}^\xN \lambda_n \pwdfan^r(x_n,y_n)}^\frac{1}{r}
%%%   %      &&    \text{by definition of $\pwdfn$}
%%%   %    \\&=    \brp{\sum_{n=1}^\xN \lambda_n \pwdfan^r(x_n,x_n)}^\frac{1}{r}
%%%   %      &&    \text{by $\opair{x}{z}=\opair{z}{y}$ hypothesis}
%%%   %    \\&=    \brp{\sum_{n=1}^\xN 0}^\frac{1}{r}
%%%   %      &&    \text{because $\pwdfan$ is \prope{nondegenerate}}
%%%   %    \\&=    0
%%%   %  \end{align*}
%%%   %
%%%   \item Proof that $\pwdfa{\opair{x}{z}}{\opair{z}{y}}=0\implies\opair{x}{z}=\opair{z}{y}$ for $r\in\intco{1}{\infty}$:
%%%      \begin{align*}
%%%       0 &=    \pwdfa{\opair{x}{z}}{\opair{z}{y}}
%%%         &&    \forall x,y,z\in\setX
%%%         &&    \text{by $\pwdfa{\opair{x}{z}}{\opair{z}{y}}=0$ hypothesis}
%%%       \\&\eqd \ptfD
%%%         &&    \forall x,y,z\in\setX
%%%         &&    \text{by definition of $\pwdfan$}
%%%       \\&\eqd 2\sigma\brs{\sfrac{1}{2}\pwdfp{p}{x}{z} + \sfrac{1}{2}\pwdfp{p}{z}{y}}^\frac{1}{p}
%%%         &&    \forall x,y,z\in\setX
%%%         &&    \text{by definition of $\ptf$ \xref{def:ptf}}
%%%       \\&\implies \pwdf{x}{y}= 0 
%%%         &&    \forall x,y\in\setX
%%%         &&    \text{because $\pwdfn$ is \prope{non-negative}}
%%%       \\&\implies \opair{x}{z}=\opair{z}{y}
%%%         &&    \forall x,y,z\in\setX
%%%         &&    \text{because $\pwdfn$ is \prope{nondegenerate}}
%%%     \end{align*}
%%%   
%%%   \item Proof that $\pwdfa{\opair{x}{z}}{\opair{z}{y}}=0\impliedby\opair{x}{z}=\opair{z}{y}$ for $r\in\intco{1}{\infty}$:
%%%      \begin{align*}
%%%       \pwdfa{\opair{x}{z}}{\opair{z}{y}}
%%%         &= \pwdfa{\opair{x}{z}}{\opair{x}{z}}
%%%         && \text{by right hypothesis}
%%%       \\&\eqd \ptf(p,\sigma,x,x,z;\pwdfn)
%%%         && \text{by definition of $\pwdfan$}
%%%       \\&\eqd 2\sigma\brs{\sfrac{1}{2}\pwdfp{p}{x}{z} + \sfrac{1}{2}\pwdfp{p}{z}{x}}^\frac{1}{p}
%%%         &&    \text{by definition of $\ptf$ \xref{def:ptf}}
%%%       \\&= 2\sigma\brs{\sfrac{1}{2}\pwdfp{p}{x}{z} + \sfrac{1}{2}\pwdfp{p}{x}{z}}^\frac{1}{p}
%%%         &&    \text{by \prope{symmetric} property of $\pwdfn$ \xref{def:pwdf}}
%%%       \\&= 2\sigma\pwdf{x}{z}
%%%     \end{align*}
%%%   
%%%   \item Proof that $\pwdfn$ satisfies the triangle inequality property  for $r=1$:
%%%     \begin{align*}
%%%       \pwdf{\opair{x}{z}}{\opair{z}{y}}
%%%         &\eqd \brp{\sum_{n=1}^\xN \lambda_n \pwdfan^r(x_n,y_n)}^\frac{1}{r}
%%%         &&    \text{by definition of $\pwdfn$}
%%%       \\&=    \sum_{n=1}^\xN \lambda_n \pwdfa{x_n}{y_n}
%%%         &&    \text{by $r=1$ hypothesis}
%%%       \\&\leq \sum_{n=1}^\xN \lambda_n \brs{\pwdfa{z_n}{x_n}+\pwdfa{z_n}{y_n}}
%%%         &&    \text{by \prope{triangle inequality}}
%%%       \\&=    \sum_{n=1}^\xN \lambda_n \pwdfa{z_n}{x_n} + \sum_{n=1}^\xN \lambda_n \pwdfa{z_n}{y_n}
%%%       \\&=    \brp{\sum_{n=1}^\xN \lambda_n \pwdfan^r(z_n,x_n)}^\frac{1}{r} + \brp{\sum_{n=1}^\xN \lambda_n \pwdfan^r(z_n,y_n)}^\frac{1}{r}
%%%         &&    \text{by $r=1$ hypothesis}
%%%       \\&\eqd \pwdf{\tuplen{z_n}}{\opair{x}{z}} + \pwdf{\tuplen{z_n}}{\opair{z}{y}}
%%%         &&    \text{by definition of $\pwdfn$}
%%%     \end{align*}
%%% 
%%%   \item Proof that $\pwdfn$ satisfies the triangle inequality property  for $r\in\intoo{1}{\infty}$:
%%%     \begin{align*}
%%%       &\pwdf{\opair{x}{z}}{\opair{z}{y}}
%%%       \\&\eqd \brp{\sum_{n=1}^\xN \lambda_n \pwdfan^r(x_n,y_n)}^\frac{1}{r}
%%%         &&    \text{by definition of $\pwdfn$}
%%%       \\&\leq \brp{\sum_{n=1}^\xN \lambda_n \brs{\pwdfan(z_n,x_n)+\pwdfan(z_n,y_n)}^r}^\frac{1}{r}
%%%         &&    \text{by \prope{subaddtive} property \xref{def:metric}}
%%%       \\&=    \brp{\sum_{n=1}^\xN \brs{\lambda_n^\frac{1}{r} \pwdfan(z_n,x_n)+\lambda_n^\frac{1}{r}\pwdfan(z_n,y_n)}^r}^\frac{1}{r}
%%%         &&    \text{by \prope{subaddtive} property \xref{def:metric}}
%%%       \\&\leq \brp{\sum_{n=1}^\xN \brs{\lambda_n^\frac{1}{r}\pwdfan(z_n,x_n)}^r}^\frac{1}{r} 
%%%             + \brp{\sum_{n=1}^\xN \brs{\lambda_n^\frac{1}{r}\pwdfan(z_n,y_n)}^r}^\frac{1}{r}
%%%         &&    \text{by \thme{Minkowski's inequality}} %{ \xref{thm:lp_minkowski}}
%%%       \\&\leq \brp{\sum_{n=1}^\xN \lambda_n \pwdfan^r(z_n,x_n)}^\frac{1}{r} 
%%%             + \brp{\sum_{n=1}^\xN \lambda_n \pwdfan^r(z_n,y_n)}^\frac{1}{r}
%%%       \\&\eqd \pwdf{\tuplen{z_n}}{\opair{x}{z}} + \pwdf{\tuplen{z_n}}{\opair{z}{y}}
%%%         &&    \text{by definition of $\pwdfn$}
%%%     \end{align*}
%%% 
%%%   \item Proof for the $r=\infty$ case:
%%%     \begin{enumerate}
%%%       \item Proof that $\pwdf{\opair{x}{z}}{\opair{z}{y}} = \max\opair{x}{z}$: by \prefp{thm:seq_Mr}
%%%       
%%%   \item Proof that $\opair{x}{z}=\opair{z}{y}\implies\pwdf{\opair{x}{z}}{\opair{z}{y}}=0$:
%%%     \begin{align*}
%%%       \pwdf{\opair{x}{z}}{\opair{z}{y}}
%%%         &\eqd \max\set{\pwdfa{x_n}{y_n}}{n=1,2,\ldots,\xN}
%%%         &&    \text{by definition of $\pwdfn$}
%%%       \\&=    \max\set{\pwdfa{x_n}{x_n}}{n=1,2,\ldots,\xN}
%%%         &&    \text{by $\opair{x}{z}=\opair{z}{y}$ hypothesis}
%%%       \\&=    0
%%%         &&    \text{because $\pwdfan$ is \prope{nondegenerate}}
%%%       \\
%%%     \end{align*}
%%%   \item Proof that $\opair{x}{z}=\opair{z}{y}\impliedby\pwdf{\opair{x}{z}}{\opair{z}{y}}=0$:
%%%     \begin{align*}
%%%       0
%%%         &=    \pwdf{\opair{x}{z}}{\opair{z}{y}}
%%%         &&    \text{by $\pwdf{\opair{x}{z}}{\opair{z}{y}}=0$ hypothesis}
%%%       \\&\eqd \max\set{\pwdfa{x_n}{y_n}}{n=1,2,\ldots,\xN}
%%%         &&    \text{by definition of $\pwdfn$}
%%%       \\\implies \pwdfa{x_n}{y_n}&=0 \text{ for } n=1,2,\ldots,\xN
%%%       \\\implies \opair{x}{z}&=\opair{z}{y}
%%%         &&    \text{because $\pwdfan$ is \prope{nondegenerate}}
%%%       \\
%%%     \end{align*}
%%%   \item Proof that $\pwdfn$ satisfies the triangle inequality property:
%%%     \begin{align*}
%%%       &\pwdf{\opair{x}{z}}{\opair{z}{y}}
%%%       \\&\eqd \max\set{\pwdfa{x_n}{y_n}}{n=1,2,\ldots,\xN}
%%%         &&    \text{by definition of $\pwdfn$}
%%%       \\&\le  \max\set{\pwdfa{x_n}{z_n}+\pwdfa{z_n}{y_n}}{n=1,2,\ldots,\xN}
%%%         %&&    \text{because $\pwdfan$ satisfies the triangle inequality property}
%%%         &&    \text{by \prope{subadditive} property}
%%%       \\&\le  \max\set{\pwdfa{x_n}{z_n}}{n=1,2,\ldots,\xN}
%%%           +   \max\set{\pwdfa{z_n}{y_n}}{n=1,2,\ldots,\xN}
%%%         &&    \text{by \prope{non-negative} property}
%%%       \\&=    \max\set{\pwdfa{z_n}{x_n}}{n=1,2,\ldots,\xN}
%%%           +   \max\set{\pwdfa{z_n}{y_n}}{n=1,2,\ldots,\xN}
%%%         &&    \text{by \prope{symmetry} property}
%%%       \\&\eqd \pwdfn(\tuplen{z_n},\opair{x}{z}) + \pwdfn(\tuplen{z_n},\opair{z}{y})
%%%         &&    \text{by definition of $\pwdfn$}
%%%     \end{align*}
%%%   \end{enumerate}
%%% 
%%%   \item And so by \prefpp{thm:metric_equiv}, $\pwdfn$ is a \fncte{metric} for $r\in\intcc{1}{\infty}$.
%%% \end{enumerate}
%%% 
%%% 
%%% 
%%% \end{proof}
%---------------------------------------
\begin{corollary}
\label{cor:tri_mono}
%---------------------------------------
Let $\ptfD$ be the \fncte{power triangle function} \xref{def:ptf} in the \structe{distance space} \xref{def:dspace} $\dspaceX$.
Let $\omsR$ be the \structe{ordered metric space} with the usual ordering relation $\le$ and usual metric $\absn$ on $\R$.
\corboxp{
  The function $\ptfD$ 
  is \propb{continuous} and \propb{strictly monotone} in $\omsR$ with respect to both the variables $p$ and $\sigma$.
  }
\end{corollary}
\begin{proof}
\begin{enumerate}
  \item Proof that $\ptfD$ is \prope{continuous} and \prope{strictly monotone} with respect to $p$: This follows directly from \prefpp{thm:seq_Mr}.
  \item Proof that $\ptfD$ is \prope{continuous} and \prope{strictly monotone} with respect to $\sigma$:
    \begin{align*}
      \ptfD
        &\eqd 2\sigma\mcom{\brs{\frac{1}{2}\distancep{p}{x}{z} + \frac{1}{2}\distancep{p}{z}{y}}^\frac{1}{p}}{$\ff(p,x,y,z)$}
        && \text{by definition of $\ptf$ \xref{def:ptf}}
      \\&=  2\sigma\ff(p,x,y,z)
        && \text{where $\ff$ is defined as above}
      \\&\implies\text{$\ptf$ is \prope{affine} with respect to $\sigma$}
      \\&\mathrlap{\implies\text{$\ptf$ is \prope{continuous} and \prope{strictly monotone} with respect to $\sigma$:}}
    \end{align*} 
\end{enumerate}
\end{proof}

%---------------------------------------
\begin{corollary}
\label{cor:ftri_means}
%---------------------------------------
Let $\ptfD$ be the \fncte{power triangle function} \xrefh{def:ptf} in the \structe{distance space} \xrefh{def:dspace} $\dspaceX$.
\\\corboxp{$
   \ptfD =
    \brbl{\begin{array}{>{\ds}lDlD}
      2\sigma\max\setn{\distancen(x,z),\,\distancen(z,y)}                                 &for& p=\infty,  & (\fncte{maximum}, {\scs corresponds to} \structe{inframetric space})\\
      2\sigma\brs{\frac{1}{2}\distancep{2}{x}{z} + \frac{1}{2}\distancep{2}{z}{y}}^\frac{1}{2} &for& p=2,       & (\fncte{quadratic mean})\\
       \sigma\brs{\distancen(x,z) + \distancen(z,y)}                                      &for& p=1,       & (\fncte{arithmetic mean}, corresponds to \structe{near metric space})\\
      2\sigma     \sqrt{\distancen(x,z)}\,\sqrt{\distancen(z,y)}                          &for& p=0        & (\fncte{geometric mean})  \\
      4\sigma     \brs{\frac{1}{\distancen(x,z)} + \frac{1}{\distancen(z,y)}}^{-1}        &for& p=-1       & (\fncte{harmonic mean})   \\
      2\sigma\min\setn{\distancen(x,z),\, \distancen(z,y)}                                &for& p=-\infty, & (\fncte{minimum})         \\
    \end{array}}
  $}
\end{corollary}
\begin{proof}
These follow directly from \prefpp{thm:seq_Mr}.
\end{proof}

%---------------------------------------
\begin{corollary}
%---------------------------------------
Let $\dspaceX$ be a \structe{distance space}.
% with \rele{power triangle inequality} \xrefh{def:trirel} $\trirelD$
%defined in terms a \fncte{power triangle function} \xrefh{def:ptf} $\ptfD$ on $\dspaceX$.
\corbox{\begin{array}{rclclcl}
        \ds 2\sigma\min\setn{\distancen(x,z),\, \distancen(z,y)}                         
   &\le&\ds 4\sigma     \brs{\frac{1}{\distancen(x,z)} + \frac{1}{\distancen(z,y)}}^{-1} 
   &\le&\ds 2\sigma     \sqrt{\distancen(x,z)}\,\sqrt{\distancen(z,y)}                   
 \\&\le&\ds  \sigma\brs{\distancen(x,z) + \distancen(z,y)}                               
   &\le&\ds 2\sigma\max\setn{\distancen(x,z),\,\distancen(z,y)}                          
\end{array}}
\end{corollary}
\begin{proof}
These follow directly from \prefpp{cor:means}.
\end{proof}

%%---------------------------------------
%\begin{proposition}
%%---------------------------------------
%Let $\dspaceX$ be a \structe{distance space} \xref{def:dspace}.
%\\\propboxp{
%  $\sigma\brs{\lambda_1\distancen^p(x,z) + \lambda_2\distancen^p(z,y)}^\frac{1}{p} =$
%  \\\indentx$\brb{\begin{array}{lDllD}
%     %\brs{\distancen^p(x,z) + \distancen^p(z,y)}^\frac{1}{p}          &for&            & \lambda=1            \sigma=1, & (right side of {\rele{$p^{th}$ order triangle inequality}})\\
%     %\max\setn{\distancen(x,z) + \distancen(z,y)}                     &for& p=\infty,  & \lambda=\frac{1}{2}  \sigma=1, & (right side of \rele{inframetric inequality})\\
%      \sigma\max\setn{\distancen(x,z),\,\distancen(z,y)}               &for& p=\infty,  &                                & (right side of \rele{\txsigma-inframetric inequality})\\
%     %          \distancen(x,z) + \distancen(z,y)                      &for& p=1,       & \lambda=1            \sigma=1, & (right side of \rele{triangle    inequality})\\
%      \frac{1}{2}\sigma\brs{\distancen(x,z) + \distancen(z,y)}                       &for& p=1,       & \lambda_1=\lambda_2=\frac{1}{2}& (right side of \rele{relaxed triangle inequality})\\
%      \sigma     \brs{\distancen(x,z)}^{\lambda_1} \brs{\distancen(z,y)}^{\lambda_2} &for& p=0        &                                & \\
%      \sigma     \brs{\frac{\lambda_1}{\distancen(x,z)} + \frac{\lambda_2}{\distancen(z,y)}}^{-1}&for& p=-1 &                 & \\
%      \sigma\min\setn{\distancen(x,z),\, \distancen(z,y)}              &for& p=-\infty, &                                & 
%    \end{array}}$
%  }
%\end{proposition}
%\begin{proof}
%These follow from direct parameter ($\sigma$, $p$) substitution and/or \prefpp{thm:seq_Mr}.
%\begin{align*}
%  \brb{\sigma\brs{\lambda\distancen^p(x,z) + \lambda\distancen^p(z,y)}^\frac{1}{p}}_{p=\infty}
%    &= \sigma\brb{\brs{2\lambda}^\frac{1}{p}\brs{\frac{1}{2}\distancen^p(x,z) + \frac{1}{2}\distancen^p(z,y)}^\frac{1}{p}}_{p=\infty}
%   &&\ocom{= \sigma\max\setn{\brs{\distancen(x,z),\,\distancen(z,y)}}}{by \prefp{thm:seq_Mr}}
%  %&& \text{by \prefpp{thm:seq_Mr}}
%  \\
%  \brb{\sigma\brs{\lambda\distancen^p(x,z) + \lambda\distancen^p(z,y)}^\frac{1}{p}}_{p=0,\lambda=\frac{1}{2}}
%    &= \sigma\brb{\brs{\frac{1}{2}\distancen^p(x,z) + \frac{1}{2}\distancen^p(z,y)}^\frac{1}{p}}_{p=0}
%   &&= \sigma\distancen^\frac{1}{2}(x,z)\distancen^\frac{1}{2}(z,y)
%   %&& \text{by \prefpp{thm:seq_Mr}}
%  \\
%  \brb{\sigma\brs{\lambda\distancen^p(x,z) + \lambda\distancen^p(z,y)}^\frac{1}{p}}_{p=-1}
%    &= \sigma\brb{\brs{\lambda}^\frac{1}{p}\brs{\distancen^p(x,z) + \distancen^p(z,y)}^\frac{1}{p}}_{p=-1}
%   &&= \frac{\sigma}{\lambda}\brs{\frac{1}{\distancen(x,z)} + \frac{1}{\distancen(z,y)}}^{-1}
%   %&& \text{by \prefpp{thm:seq_Mr}}
%  \\
%  \brb{\sigma\brs{\lambda\distancen^p(x,z) + \lambda\distancen^p(z,y)}^\frac{1}{p}}_{p=-\infty}
%    &= \sigma\brb{\brs{2\lambda}^\frac{1}{p}\brs{\frac{1}{2}\distancen^p(x,z) + \frac{1}{2}\distancen^p(z,y)}^\frac{1}{p}}_{p=-\infty}
%   &&= \sigma\min\setn{\brs{\distancen(x,z),\,\distancen(z,y)}}
%   %&& \text{by \prefpp{thm:seq_Mr}}
%\end{align*}
%\end{proof}

%=======================================
\subsection{Properties of power distance spaces}
\label{sec:pdspace_propd}
%=======================================
The \prope{power triangle inequality} property of a \structe{power distance space}
axiomatically endows a metric with an upper bound. 
\pref{lem:pdspace_ineq} (next) demonstrates that there is a complementary lower bound 
somewhat similar in form to the \prope{power triangle inequality} upper bound.
In the special case where $2\sigma=2^\frac{1}{p}$, 
%For the case $\opair{p}{\sigma}=\opair{1}{1}$, 
the lower bound helps provide a simple proof of the \prope{continuity} of 
a large class of \fncte{power distance function}s \xref{thm:pdspace_continuous}.
The inequality $2\sigma\le2^\frac{1}{p}$ is a special relation in this paper and appears repeatedly in this paper;
it appears as an inequality in \prefpp{lem:tri_open}, \prefpp{cor:tspace_base} and \prefpp{cor:oball_open},
and as an equality in \pref{lem:pdspace_ineq} (next) and \prefpp{thm:pdspace_continuous}.
It is plotted in \prefpp{fig:sigmap}.
\begin{figure}[t]
  %\gsize%
  \footnotesize%
  \centering%
  \includegraphics{../common/math/graphics/pdfs/trirel_sigmap.pdf}
  \caption{$\sigma = \frac{1}{2}(2^{\frac{1}{p}}) = 2^{\frac{1}{p}-1}$ or $p=\frac{\ln2}{\ln(2\sigma)}$ 
  %(see  \prefpp{thm:openball_openset})
  \label{fig:sigmap}
  \scs(see \prefp{lem:pdspace_ineq}, \prefp{lem:tri_open}, \prefp{cor:oball_open}, \prefp{cor:tspace_base}, and \prefp{thm:pdspace_continuous}).
  }
\end{figure}
%---------------------------------------
\begin{lemma}
\footnote{
  in \structe{metric space} ($\opair{p}{\sigma}=\opair{1}{1}$):
  \citerpg{dieudonne1969}{28}{1406727911},
  \citerpg{michel1993}{266}{048667598X},
  \citerpgc{berberian1961}{37}{0821819127}{Theorem~II.4.1}
  }
\label{lem:pdspace_ineq}
%---------------------------------------
Let $\pdspaceX$ be a \structe{power triangle triangle space} \xref{def:pdspace}. 
Let $\absn$ be the \fncte{absolute value} function \xref{def:abs}.
Let $\max\setn{x,y}$ be the maximum and $\min\setn{x,y}$ the minimum of any $x,y\in\Rx$.
Then, for all $\opair{p}{\sigma}\in\Rx\times\Rp$,
%Then, for all $x,y,z\in\setX$
\lembox{\begin{array}{Frc>{\ds}lCD}
  %\opair{p}{\sigma}\in\R\times\Rp \quad\implies\quad
  1.&
  \distancep{p}{x}{y} &\ge& 
    \max\brb{0,\,
      \frac{2}{(2\sigma)^p}\distancep{p}{x}{z}-\distancep{p}{z}{y},\,
      \frac{2}{(2\sigma)^p}\distancep{p}{y}{z}-\distancep{p}{z}{x}
      }
  &\forall x,y,z\in\setX
  & and
  \\
  2.&
  %\opair{p}{\sigma}\in\set{\R\times\Rp}{p\neq0,\,2\sigma=2^\frac{1}{p}}\setu\setn{\opair{\infty}{\sfrac{1}{2}},\,\opair{-\infty}{\sfrac{1}{2}}} 
  %\quad\implies\quad
  \distance{x}{y}&\ge&\abs{\distance{x}{z} - \distance{z}{y}} 
  \quad\text{if \;$p\neq0$\; and \;$2\sigma=2^\frac{1}{p}$}
  &\forall x,y,z\in\setX .
\end{array}}
\end{lemma}
\begin{proof}
\begin{enumerate}
  \item lemma: $\frac{2}{(2\sigma)^p}\distancep{p}{x}{z}-\distancep{p}{z}{y} \le \distancep{p}{x}{y}\quad\forall\opair{p}{\sigma}\in\Rx\times\Rp$: 
        Proof:\label{ilem:pdspace_ineq}
    %\begin{enumerate}
      %\item Proof for $\opair{p}{\sigma}\in(\Rx\setd\setn{0})\times\Rp$: 
        \begin{align*}
          \frac{2}{(2\sigma)^p}\distancep{p}{x}{z}-\distancep{p}{z}{y}
            &\le \frac{2}{(2\sigma)^p}\brs{2\sigma\brs{\sfrac{1}{2}\distancep{p}{x}{y}+\sfrac{1}{2}\distancep{p}{y}{z}}^\frac{1}{p}}^p - \distancep{p}{z}{y}
            && \text{by \prope{power triangle inequality}}
          \\&=   \frac{2(2\sigma)^p}{(2\sigma)^p}\brs{\sfrac{1}{2}\distancep{p}{x}{y}+\sfrac{1}{2}\distancep{p}{y}{z}} - \distancep{p}{z}{y}
          \\&=   \brs{\distancep{p}{x}{y}+\distancep{p}{y}{z}} - \distancep{p}{y}{z}
            && \text{by \prope{symmetric} property of $\distancen$}
          \\&=   \distancep{p}{x}{y}
        \end{align*}

      %\item Proof for $\opair{p}{\sigma}\in\setn{0}\times\Rp$: For this case the lemma trivially holds because 
      %  \\\indentx$
      %      \brs{\frac{2}{(2\sigma)^p}\distancep{p}{x}{z}-\distancep{p}{z}{y}}_{p=0} 
      %         = \frac{2}{1}-1 = 1 \le 1 =   \brs{\distancep{p}{x}{y}}_{p=0}
      %      $

      %\item Proof for $\opair{p}{\sigma}=\opair{-\infty}{\frac{1}{2}}$:
      %  \begin{align*}
      %    \brs{\frac{2}{(2\sigma)^p}\distancep{p}{x}{z}-\distancep{p}{z}{y}}_{\opair{p}{\sigma}=\opair{-\infty}{\frac{1}{2}}}
      %      &= \brs{\frac{2}{1}\distancep{p}{x}{z}-\distancep{p}{z}{y}}_{\opair{p}{\sigma}=\opair{-\infty}{\frac{1}{2}}}
      %    \\&\le \brs{2\brs{2\sigma\brs{\sfrac{1}{2}\distancep{p}{x}{y}+\sfrac{1}{2}\distancep{p}{y}{z}}^\frac{1}{p}}^p - \distancep{p}{z}{y}}_{\opair{p}{\sigma}=\opair{-\infty}{\frac{1}{2}}}
      %      && \text{by \prope{power triangle inequality}}
      %    \\&= \brs{\brs{\distancep{p}{x}{y}+\distancep{p}{y}{z}} - \distancep{p}{z}{y}}_{p=-\infty}
      %    \\&= \brs{\brs{\distancep{p}{x}{y}+\distancep{p}{y}{z}} - \distancep{p}{y}{z}}_{p=-\infty}
      %      && \text{by \prope{symmetric} property of $\distancen$}
      %    \\&=   \distancep{p}{x}{y}
      %  \end{align*}
    %\end{enumerate}

  \item Proof for $\opair{p}{\sigma}\in\Rx\times\Rp$ case:  \label{item:pdspace_max}
    \\\indentx$\begin{array}{rclclM}
      \distancep{p}{x}{y} & &                    &\ge& \frac{2}{(2\sigma)^p}\distancep{p}{x}{z}-\distancep{p}{z}{y} & \text{by \pref{ilem:pdspace_ineq}}
    \\\distancep{p}{x}{y} &=&\distancep{p}{y}{x} &\ge& \frac{2}{(2\sigma)^p}\distancep{p}{y}{z}-\distancep{p}{z}{x} & \text{by \prope{commutative} property of $\distancen$ and \pref{ilem:pdspace_ineq}}
    \\\distancep{p}{x}{y} & &                    &\ge& 0                                                            & \text{by \prope{non-negative} property of $\distancen$ \xref{def:dspace}}
    \end{array}$\\
  The rest follows because $\fg(x)\eqd x^\frac{1}{p}$ is \prope{strictly monotone} in $\clFrr$.

  \item Proof for $2\sigma=2^\frac{1}{p}$ case:
        \begin{align*}
          \distance{x}{y} 
            &\ge  \max\brb{0,\,
                    \frac{2}{(2\sigma)^p}\distancep{p}{x}{z}-\distancep{p}{z}{y},\,
                    \frac{2}{(2\sigma)^p}\distancep{p}{y}{z}-\distancep{p}{z}{x}
                    }^{\frac{1}{p}}
            && \text{by \prefpp{item:pdspace_max}}
          \\&=    \max\brb{0,\,
                    \distance{x}{z}-\distance{z}{y},\,
                    \distance{y}{z}-\distance{z}{x}
                    }
            && \text{by $2\sigma=2^\frac{1}{p}$ hyp. $\iff\frac{2}{(2\sigma)^p}=1$}
          \\&=    \max\brb{0,\,
                    \brp{\distance{x}{z}-\distance{z}{y}},\,
                   -\brp{\distance{x}{z}-\distance{z}{y}}
                    }
            && \text{by \prope{symmetric} property of $\distancen$}
          \\&= \abs{\brp{\distance{x}{z}-\distance{z}{y}}}
        \end{align*}
\end{enumerate}
\end{proof}


%\begin{figure}[h]
%  \gsize%
%  \centering%
%  %%============================================================================
% Daniel J. Greenhoe
% LaTeX file
%============================================================================
  \begin{pspicture}(-1,-3.5)(5.5,4)%
    \psaxes[linecolor=axis,labels=all,ticks=all,showorigin=true]{<->}(0,0)(-1.5,-3.5)(5,3.5)%
    \psplot[plotpoints=128,linecolor=blue]{0.63}{5}{2 log x log 2 log add div}% 
    %\psplot[plotpoints=1024,linecolor=blue,linewidth=0.75pt]{0.00000001}{4}{2 ln x 0.01 mul ln 2 ln add div}% 
    \psplot[plotpoints=256,linecolor=blue]{0.000001}{0.4}{2 ln x 2 mul ln div}% 
    %\psplot[plotpoints=128,linecolor=blue]{0.01}{0.4}{2 log x log 2 log add div}% 
    %\psline(0,0)
    \pnode(0.06766764,-0.34657359){ipnt}% inflection point at (ln2/(-2), (1/2)e^{-2})
    \pnode(1,-0.7){lpnt}% label for inflection point
    \psline[linestyle=dotted,dotsep=2pt,linecolor=red](0,1)(1,1)%
    \psline[linestyle=dotted,dotsep=2pt,linecolor=red](1,0)(1,1)%
    \psline[linestyle=dotted,dotsep=2pt,linecolor=red](0.5,-3.5)(0.5,3.5)%
    \psline[linecolor=red,linewidth=0.75pt]{->}(lpnt)(ipnt)%
    \rput[tl]{0}(lpnt){\begin{tabular}{@{}c}%
      second order inflection point at\\% 
      $\opair{\sigma}{p}=\opair{\frac{1}{2}e^{-2}}{-\frac{1}{2}\ln2}$%
      \end{tabular}}%
    \uput[0]{0}(0,3.5){$p$}%
    \uput[0]{0}(5,0){$\sigma$}%
    \uput[-60]{0}(0.5,0){$\frac{1}{2}$}%
  \end{pspicture}%

%  %============================================================================
% Daniel J. Greenhoe
% LaTeX file
%============================================================================
  \begin{pspicture}(-8,-0.65)(8,5.2)%
    %-----------------------------------
    % plots
    %-----------------------------------
    \psaxes[linecolor=axis,labels=all,ticks=all,showorigin=true,xAxis=false,subticks=2]{->}(0,0)(-3.5,0)(3.5,5)%
    \psaxes[linecolor=axis,labels=all,ticks=all,showorigin=true,yAxis=false,subticks=2]{<->}(0,0)(-3.5,0)(3.5,5)%
    \psplot[plotpoints=128,linecolor=blue,arrows=<->]{0.3}{3.5}{2 1 x div exp 2 div}%
    \psplot[plotpoints=256,linecolor=blue,arrows=<-o]{-3.5}{-0.001}{2 1 x div exp 2 div}%
    %-----------------------------------
    % label points
    %-----------------------------------
    \pnode(-1,4){Lgeo}\pnode(0,0.5){Pgeo}% % geometric inequality
    \pnode(-1,3){Lip}\pnode(-0.34657359,0.06766764){Pip}% inflection point at (ln2/(-2), (1/2)e^{-2})
    \pnode(-1.5,2){Lhar}\pnode(-1,0.25){Phar}% % harmonic inequality
    %
    \pnode(1,4){Lsmr}\pnode(0.5,2){Psmr}% % square mean root inequality
    \pnode(1.4,3){Lms}\pnode(1,1){Pms}% % triangle inequality
    \pnode(1.5,2){Lqm}\pnode(2,0.7071){Pqm}% % quadratic mean inequality
    %-----------------------------------
    % label arrows
    %-----------------------------------
    \psline[linestyle=dotted,dotsep=2pt,linecolor=red](0,1)(1,1)%
    \psline[linestyle=dotted,dotsep=2pt,linecolor=red](1,0)(1,1)%
    \psline[linestyle=dotted,dotsep=2pt,linecolor=red](0.5,0)(0.5,2)%
    \psline[linestyle=dotted,dotsep=2pt,linecolor=red](0,2)(0.5,2)%
    \psline[linestyle=dotted,dotsep=2pt,linecolor=red](2,0)(2,0.707)%
    \psline[linestyle=dotted,dotsep=2pt,linecolor=red](0,0.707)(2,0.707)%
    \psline[linestyle=dotted,dotsep=2pt,linecolor=red](-3.5,0.5)(3.5,0.5)%
    \psline[linecolor=red,linewidth=0.75pt]{->}(Lip)(Pip)%
    \psline[linecolor=red,linewidth=0.75pt]{->}(Lms)(Pms)%
    \psline[linecolor=red,linewidth=0.75pt]{->}(Lsmr)(Psmr)%
    \psline[linecolor=red,linewidth=0.75pt]{->}(Lgeo)(Pgeo)%
    %\psline[linecolor=red,linewidth=0.75pt]{->}(Lqm)(Pqm)%
    \psline[linecolor=red,linewidth=0.75pt]{->}(2.2,1.89)(Pqm)%
    \psline[linecolor=red,linewidth=0.75pt]{->}(Lhar)(Phar)%
    %-----------------------------------
    % labels
    %-----------------------------------
    \rput[r]{0}(Lip){$2^\mathrm{nd}$ order inflection point at $\opair{-\frac{1}{2}\ln2}{\frac{1}{2}e^{-2}}$}%
    %\rput[r]{0}(Lip){\begin{tabular}{r@{}}%
    %  second order inflection point at\\%
    %  $\opair{p}{\sigma}=\opair{-\frac{1}{2}\ln2}{\frac{1}{2}e^{-2}}\approx\opair{-0.347}{0.0677}$
    %  \end{tabular}}%
    \rput[l]{0}(Lms){\structe{triangle inequality} at $\opair{p}{\sigma}=\opair{1}{1}$}%
    \rput[l]{0}(Lsmr){\structe{square mean root (SMR) inequality} at $\opair{\frac{1}{2}}{2}$}%
    \rput[tl]{0}(3.6,0.82){\begin{tabular}{@{}r@{}}%
      to \structe{inframetric inequality} at\\%
      $\opair{p}{\sigma}=\opair{\infty}{\frac{1}{2}}$
      \end{tabular}}%
    \rput[tr]{0}(-3.6,0.65){\begin{tabular}{@{}r@{}}%
      to \structe{minimal inequality} at\\%
      $\opair{p}{\sigma}=\opair{\infty}{\frac{1}{2}}$
      \end{tabular}}%
    \rput[l]{0}(Lqm){\structe{quadratic mean inequality} at $\opair{2}{\frac{\sqrt{2}}{2}}$}%
    \rput[r]{0}(Lgeo){\structe{geometric inequality} at $\opair{p}{\sigma}=\opair{0}{\frac{1}{2}}$}%
    \rput[r]{0}(Lhar){\structe{harmonic inequality} at $\opair{p}{\sigma}=\opair{-1}{\frac{1}{4}}$}%
    \uput[0]{0}(3.5,0){$p$}%
    \uput[180]{0}(0,5){$\sigma$}%
    \uput[0]{0}(0,0.5){$\frac{1}{2}$}%
  \end{pspicture}%

%  \caption{$\sigma = 2^{\frac{1}{p}-1}$ or $p=\frac{\ln2}{\ln(2\sigma)}$ for \prefpp{thm:pdspace_continuous}\label{fig:sigmap}}
%\end{figure}
%\begin{figure}[h]
%  \centering%
%  \gsize%
%  \psset{unit=2mm}%============================================================================
% Daniel J. Greenhoe
% LaTeX file
% nominal unit = 30mm
%============================================================================
\begin{pspicture}(-1.1,-1.1)(1.1,1.1)%
  %-------------------------------------
  % options
  %-------------------------------------
  \psset{
    dotsize=5pt,
    %labelsep=5pt,
    }
  %-------------------------------------
  % axes
  %-------------------------------------
  %\psline[linecolor=axis]{<->}(-120,0)(120,0)% x-axis
  %\psline[linecolor=axis]{<->}(0,-120)(0,120)% y-axis
  %-------------------------------------
  % nodes
  %-------------------------------------
  \pnode(  0,  0){o}% origin (and center of outer ball)
  \pnode(  0, .55){p}% point p (and center of inner ball)
  \pnode(-.25, .45){q}% point q
  \pnode(  0, 1.00){otop}% top of outer ball
  \pnode(  0,-1.00){bottom}% top of outer ball
  \pnode(.71,-.71){obr}% bottom right corner of outer ball
  \pnode(-.71,-.71){obl}% bottom right corner of outer ball
  \pnode( .2828,.8328){itr}% bottom right corner of inner ball
  \pnode( .2828,.2672){ibr}% bottom right corner of inner ball
  \pnode(-.2828,.2672){ibl}% bottom right corner of inner ball
  \pnode(1.00,   0){right}% right side of outer ball
  %-------------------------------------
  % objects
  %-------------------------------------
  \pscircle[linecolor=blue,linestyle=dashed](o){1}% B(o,r) (outer ball)
  \pscircle[linecolor=red,linestyle=dashed](p){0.40}%   B(p,r_p) (ball centered at p with radius r_p)
  %\ncline{->}{o}{q}     \naput{$\metric{\theta}{q}$}%
  \ncline{->}{p}{itr}   \nbput{$r_p\le r-\metric{\theta}{p}$}%
  \ncline{->}{o}{obr}   \ncput*{$r$}% radius line
  \ncline{->}{o}{p}     \nbput{$\metric{\theta}{p}$}% 
  \psdot(o)             \uput{5pt}[225](o){$\theta$}% origin (and center of outer ball)
  \psdot(p)             \uput{5pt}[135](p){$p$}% point p
  %\psdot(q)             \uput{5pt}[ 90](q){$q$}% point q
  \uput{5pt}[45](obl){$\ball{\theta}{r}$}
  \uput[-135](ibl){$\ball{p}{r_p}$}
\end{pspicture}%
%
%  %\input{ballinball1.tex}
%  \caption{
%  %[Every point in an \sete{open ball} \xref{def:ball} is contained in an {open ball} that is contained in the original open ball]
%  %Every point in an open ball is contained in an open ball that is contained in the original open ball
%  \structe{open ball} that is \prope{open} (see \prefp{thm:openball_openset})
%  \label{fig:ms_openball_openset}
%  }
%\end{figure}
%---------------------------------------
\begin{theorem}
\label{thm:openball_openset}
%---------------------------------------
Let $\pdspaceX$ be a \structe{power distance space} \xref{def:pdspace}.
Let $\balln$ be an \structe{open ball} \xref{def:ball} on $\dspaceX$. 
Then for all $\opair{p}{\sigma}\in(\Rx\setd\setn{0})\times\Rp$,
\thmbox{
  \brb{\begin{array}{FlD}
    A. & 2\sigma \le 2^{\frac{1}{p}} & and\\
    B. & q\in\ball{\theta}{r}  
  \end{array}}
  \quad\implies\quad
  \brb{\begin{array}{Fl}
    1. & \exists r_q\in\Rp \st\\ 
       & \ball{q}{r_q}\subseteq\ball{\theta}{r}
  \end{array}}
  \quad\implies\quad
  \brb{\begin{array}{Fl}
       &\\
    B. & q\in\ball{\theta}{r}  
  \end{array}}
  }
\end{theorem}
\begin{proof}
\begin{enumerate}
  \item lemma: \label{ilem:openball_openset_lem}
    \begin{align*}
      q\in\ball{\theta}{r}
        &\iff \distance{\theta}{q} < r
       && \text{by definition of \structe{open ball} \xref{def:ball}}
      \\&\iff 0< r-\distance{\theta}{q} 
        && \text{by field property of real numbers}
      \\&\iff \exists r_q\in\Rp \st 0< r_q< r-\distance{\theta}{q}
        &&\text{by \thme{The Archimedean Property}\footnotemark}
    \end{align*}
    %\begin{align*}
    %  q\in\ball{\theta}{r}
    %    &\iff \distance{\theta}{q} < r
    %    && \text{by definition of \structe{open ball}}
    %   %&& \text{by definition of \structe{open ball} \xref{def:ball}}
    %  \\&\iff (2\sigma)^p\distance{\theta}{q} < (2\sigma)^p r
    %    && \text{because $(2\sigma)^p>0$}
    %  \\&\iff 0<(2\sigma)^p r-(2\sigma)^p\distance{\theta}{q} 
    %    && \text{by field property of real numbers}
    %  \\&\iff \exists r_q\in\Rp \st 0<(2\sigma)^p r_q<(2\sigma)^p r-(2\sigma)^p\distance{\theta}{q}
    %    &&\text{by \thme{The Archimedean Property}\footnotemark}
    %\end{align*}
\footnotetext{
  \citerpgc{ab}{17}{0120502577}{Theorem 3.3 (``\thme{The Archimedean Property}") and Theorem 3.4},
  \citerpgc{zorich2004}{53}{3540403868}{$6^\circ$ (``\thme{The principle of Archimedes}") and $7^\circ$}
  }

  \item Proof that $(A),(B)\implies(1)$:
    \begin{align*}
      \ball{q}{r_q}
        &\eqd \set{x\in\setX}{\distance{q}{x}<r_q}
        &&    \text{by definition of \structe{open ball} \xref{def:ball}}
      \\&=    \set{x\in\setX}{\distancep{p}{q}{x}< r_q^p\in\Rp}
        &&    \text{because $\ff(x)\eqd x^p$ is \prope{monotone}}
      \\&\subseteq \set{x\in\setX}{\distancep{p}{q}{x}< r^p-\distancep{p}{\theta}{q}}
        &&    \text{by hypothesis B and \prefp{ilem:openball_openset_lem}}
      \\&=    \set{x\in\setX}{\distancep{p}{\theta}{q}+\distancep{p}{q}{x}< r^p}
        &&    \text{by field property of real numbers}
      \\&=    \set{x\in\setX}{\brs{\distancep{p}{\theta}{q}+\distancep{p}{q}{x}}^{\frac{1}{p}}<r}
        &&    \text{because $\ff(x)\eqd x^{\frac{1}{p}}$ is \prope{monotone}}
      \\&\subseteq \set{x\in\setX}{2^{1-\sfrac{1}{p}}\sigma\brs{\distancep{p}{\theta}{q}+\distancep{p}{q}{x}}^{\frac{1}{p}}<r}
        &&    \text{by hypothesis A which implies $2^{1-\sfrac{1}{p}}\sigma\le1$}
      \\&= \set{x\in\setX}{2\sigma\brs{\sfrac{1}{2}\distance{\theta}{q}{x}+\sfrac{1}{2}\distancep{p}{q}{x}}^{\frac{1}{p}}<r}
        &&    \text{because $2^{1-\sfrac{1}{p}}\sigma=2\sigma(\sfrac{1}{2})^{\frac{1}{p}}$}
      \\&\eqd \set{x\in\setX}{\ptf(p,\sigma,\theta,x,q)<r}
        && \text{by definition of $\ptf$ \xref{def:ptf}}
      \\&\subseteq \set{x\in\setX}{\distance{\theta}{x}<r}
        && \text{by definition of $\pdspaceX$ \xref{def:pdspace}}
      \\&\eqd \ball{\theta}{r}
        &&    \text{by definition of \structe{open ball} \xref{def:ball}}
    \end{align*}
    %\begin{align*}
    %  \ball{q}{r_q}
    %    &\eqd \set{x\in\setX}{\distance{q}{x}<r_q\in\Rp}
    %    &&    \text{by definition of \structe{open ball} \xref{def:ball}}
    %  \\&=    \set{x\in\setX}{(2\sigma)^p\distancep{p}{q}{x}<(2\sigma)^p r_q^p\in\Rp}
    %    &&    \text{because $\ff(x)\eqd x^p$ is \prope{monotone}}
    %  \\&\subseteq \set{x\in\setX}{(2\sigma)^p\distancep{p}{q}{x}<(2\sigma)^p r^p-(2\sigma)^p\distancep{p}{\theta}{q}}
    %    &&    \text{by left hypothesis and \prefp{ilem:openball_openset_lem}}
    %  \\&=    \set{x\in\setX}{(2\sigma)^p\distance{\theta}{q}{x}+(2\sigma)^p\distancep{p}{q}{x}<(2\sigma)^p r^p}
    %    &&    \text{by property of real numbers}
    %  \\&=    \set{x\in\setX}{\distance{\theta}{q}{x}+\distancep{p}{q}{x}<r^p}
    %    &&    \text{by property of real numbers\footnotemark}
    %  \\&=    \set{x\in\setX}{\brs{\distance{\theta}{q}{x}+\distancep{p}{q}{x}}^{\frac{1}{p}}<r}
    %    &&    \text{by property of real numbers}
    %  \\&\subseteq \set{x\in\setX}{\brs{\sfrac{1}{2}\distance{\theta}{q}{x}+\sfrac{1}{2}\distancep{p}{q}{x}}^{\frac{1}{p}}<r}
    %    && \text{by property of real numbers}
    %  \\&\subseteq \set{x\in\setX}{2\sigma\brs{\sfrac{1}{2}\distance{\theta}{q}{x}+\sfrac{1}{2}\distancep{p}{q}{x}}^{\frac{1}{p}}<r}
    %    && \text{by $\sigma\in\intoc{0}{\sfrac{1}{2}}$ hypothesis}
    %  \\&\eqd \set{x\in\setX}{\ptf(p,\sigma,\theta,x,q)<r}
    %    && \text{by definition of $\ptf$ \xref{def:ptf}}
    %  \\&\subseteq \set{x\in\setX}{\distance{\theta}{x}<r}
    %    && \text{by definition of $\pdspaceX$ \xref{def:pdspace}}
    %  \\&\eqd \ball{\theta}{r}
    %    &&    \text{by definition of \structe{open ball} \xref{def:ball}}
    %\end{align*}
    %\footnotetext{
    %  Here the $(2\sigma)^2$ factor disappears, and with it some hope of relaxing the $\sigma\in\intoc{0}{\sfrac{1}{2}}$ 
    %  constraint.
    %  }

  \item Proof that $(B)\impliedby(1)$:
    \begin{align*}
      q &\in    \set{x\in\setX}{\distance{q}{x}=0}
        &&    \text{by \prope{nondegenerate} property \xref{def:dspace}}
      \\&\subseteq \set{x\in\setX}{\distance{q}{x}<r_q}
        &&    \text{because $r_q>0$}
      \\&\eqd \ball{q}{r_q}
        &&    \text{by definition of \structe{open ball} \xref{def:ball}}
      \\&\subseteq \ball{\theta}{r}
        &&    \text{by hypothesis 2}
    \end{align*}
\end{enumerate}
\end{proof}

%---------------------------------------
\begin{corollary}
\label{cor:tspace_openball}
%---------------------------------------
Let $\pdspaceX$ be a \structe{power distance space}.
Then for all $\opair{p}{\sigma}\in(\Rx\setd\setn{0})\times\Rp$,
\thmbox{
  %\brb{2^{1-\sfrac{1}{p}}\sigma \le 1}
  \brb{2\sigma \le 2^{\frac{1}{p}}}
  \quad\implies\quad
  \brb{\text{every \structe{open ball} in $\dspaceX$ is \prope{open}}}
  }
\end{corollary}
\begin{proof}
This follows from \prefpp{thm:openball_openset} and \prefpp{thm:baseoball}.
\end{proof}

%\prefpp{cor:tspace_base} shows that in a \structe{distance space} \xref{def:dspace} $\dspaceX$ with 
%the metric $\distancen$ always induces a topology $\topT$ on $\setX$.
%The set $\setX$ together with  topology $\topT$ is a \structe{topological space}.
%More specifically, the set of \structe{open balls} in a metric space form a \structe{base} for a \structe{topological space}.
%Therefore, {\em every} \structe{metric space} \xref{def:metric}\index{space!metric} {\em is} a topological space,
%and everything that is true of a topological space is also true for all \structe{metric space}\index{space!metric}s.
%---------------------------------------
\begin{corollary}
\label{cor:tspace_base}
%\label{thm:(X,d)->(X,t)}
%---------------------------------------
Let $\pdspaceX$ be a \structe{power distance space}.
Let $\baseB$ be the set of all \structe{open ball}s in $\dspaceX$.
Then for all $\opair{p}{\sigma}\in(\Rx\setd\setn{0})\times\Rp$,
%Let $\ball{x}{r}$ be an \structe{open ball} \xrefh{def:ball} on $\dspaceX$.
%\propboxp{
%  If for some $p\in\Rx$ and $\sigma\in\Rp$, $\ptfD\le\ptf(1,1;x,y,z)$ ${\scy\forall x,y,z\in\setX}$ then
%  the set of all \structe{open balls} in $\dspaceX$ is a \structe{base} for the topological space $\topspaceX$ where
%  \\\indentx
%  $\ds\topT \eqd \set{\setU\in\psetX}{\text{$\setU$ is the union of balls in $\dspaceX$}}$.
%  }
%\\Let $\baseB\eqd\set{\ball{x}{r}}{x\in\setX,\,r\in\Rp}$.
%Let $\topT \eqd \set{\setU\in\psetX}{\text{$\setU$ is the union of elements in $\baseB$}}$.
\corbox{
  \brb{2\sigma \le 2^{\frac{1}{p}}}
  \quad\implies\quad
  \brb{\begin{array}{M}
    $\baseB$ is a \structe{base} for $\topspaceX$
    %2. & $\topspaceX$ is a \structe{topological space} & and \\
  \end{array}}
  }
\end{corollary}
\begin{proof}
\begin{enumerate}
  \item The set of all \structe{open balls} in $\dspaceX$ is a \structe{base} for $\topspaceX$ by 
        \prefpp{cor:tspace_openball} and \prefpp{thm:basex}.
        
  \item $\topT$ is a topology on $\setX$ by \prefpp{def:baseB}.
\end{enumerate}
\end{proof}

\begin{figure}[h]
  \center
  \includegraphics{../common/math/graphics/pdfs/ballinopen.pdf}
  \caption{\structe{open set} (see \prefp{lem:tri_open}) \label{fig:ms_open} }
\end{figure}
\pref{lem:tri_open} (next) demonstrates that every point in an open set is contained in an open ball that is 
contained in the original open set (see also \prefp{fig:ms_open}).
%---------------------------------------
\begin{lemma}
\label{lem:tri_open}
%---------------------------------------
Let $\pdspaceX$ be a \structe{power distance space}.
Let $\balln$ be an \structe{open ball} on $\dspaceX$.
Then for all $\opair{p}{\sigma}\in(\Rx\setd\setn{0})\times\Rp$,
\\\lemboxp{$%
  \brb{\begin{array}{FlD}
    A. & 2\sigma \le 2^{\frac{1}{p}} & and\\
    B. & \mc{2}{M}{$\setU$ is \prope{open} in $\dspaceX$}
  \end{array}}
  \implies
  \brb{\begin{array}{Fl}
    1. & \forall x\in\setU,\; \exists r\in\Rp \st\\ 
       & \ball{x}{r}\subseteq \setU
  \end{array}}
  \implies
  \brb{\begin{array}{FM}
    B. & $\setU$ is\\ 
       & \prope{open} in $\dspaceX$
  \end{array}}
  $}
\end{lemma}
\begin{proof}
\begin{enumerate}
  \item Proof that for ($(A),(B)\implies(1)$):
    \begin{align*}
      \setU
        &= \Setu\set{\ball{x_\gamma}{r_\gamma}}{\ball{x_\gamma}{r_\gamma}\subseteq\setU}
        && \text{by left hypothesis and \prefp{cor:tspace_base}}
      \\&\supseteq \ball{x}{r}
        && \text{because $x$ must be in one of those balls in $\setU$}
        %&& \text{by \prefp{thm:openball_openset}}
    \end{align*}

  \item Proof that ($(B)\impliedby(1)$) case:
    \begin{align*}
      \setU 
        &= \Setu\set{x\in\setX}{x\in\setU}
        && \text{by definition of union operation $\Setu$}
      \\&= \Setu\set{\ball{x}{r}}{x\in\setU\text{ and }\ball{x}{r}\subseteq\setU}
        && \text{by hypothesis (1)}
      \\&\implies \text{$\setU$ is \prope{open}}
        && \text{by \prefp{cor:tspace_base} and \prefp{cor:dspace_open}}
    \end{align*}
\end{enumerate}
\end{proof}

%---------------------------------------
\begin{corollary}
\footnote{
  in \structe{metric space} ($\opair{p}{\sigma}=\opair{1}{1}$):
  \citerppg{rosenlicht}{40}{41}{0486650383},
  \citerpg{ab}{35}{0120502577}
  }
\label{cor:oball_open}
%\label{prop:cball_closed}
%---------------------------------------
Let $\pdspaceX$ be a \structe{power distance space}.
Let $\balln$ be an \structe{open ball} \xrefh{def:ball} on $\dspaceX$.
Then for all $\opair{p}{\sigma}\in(\Rx\setd\setn{0})\times\Rp$,
\corbox{
  \brb{2\sigma \le 2^{\frac{1}{p}}}
  \quad\implies\quad
  \brb{\begin{array}{M}
    every \structe{open ball} $\ball{x}{r}$  in $\dspaceX$ is \prope{open}
  \end{array}}
  }
\end{corollary}
\begin{proof}
    \begin{align*}
        &\text{The union of any set of open balls is open}    && \text{by \prefp{cor:tspace_base}}
      \\&\qquad\text{$\implies$ the union of a set of just one open ball is open}
      \\&\qquad\text{$\implies$ every open ball is open.}
    \end{align*}
\end{proof}


%---------------------------------------
\begin{theorem}
\footnote{
  in \structe{metric space}:
  \citerpg{rosenlicht}{45}{0486650383},
  \citerpgc{giles1987}{37}{0521359287}{3.2 Definition}
  %\citerpgc{khamsi2001}{13}{0471418250}{Definition 2.1}
  %\citerpgc{thomson2008}{30}{143484367X}{Definition 2.1}
  %\citor{cauchy1821}
%  ``$\to$" symbol: \citorpc{leathem1905}{13}{section III.11}  % referenced by bromwich1955 page 3
  }
\label{thm:ms_converge}
\index{convergence!metric space}
%---------------------------------------
Let $\pdspaceX$ be a \structe{power distance space}.
Let $\topspaceX$ be a \structe{topological space induced by $\dspaceX$}. % \xref{def:dspacetop}.
Let $\seqxZ{x_n\in\setX}$ be a  sequence in $\dspaceX$.
\thmbox{
  \mcom{\text{$\seqn{x_n}$ converges to a limit $x$}}{\xref{def:converge}}
  \quad\iff\quad
  \brb{\begin{array}{M}
    for any $\varepsilon\in\Rp$, there exists $\xN\in\Z$\\ 
    such that for all $n>\xN$,
    \quad
    $\distance{x_n}{x}<\varepsilon$
  \end{array}}
  }
\end{theorem}
\begin{proof}
%\begin{enumerate}
  %\item Proof that $\seqn{x_n}\to x$ $\implies$   $\distance{x_n}{x}<\varepsilon$:
    \begin{align*}
      \seqn{x_n}\to x
        &\iff x_n\in\setU \quad\forall\setU\in\setN_x,\,n>\xN
        && \text{by \prefp{def:converge}}
      \\&\iff \exists \ball{x}{\varepsilon} \st x_n\in\ball{x}{\varepsilon} \forall n>\xN
        %&& \text{by \prefpp{def:dspace_open}}
        && \text{by \prefp{lem:tri_open}}
      \\&\iff \distance{x_n}{x}<\varepsilon
        && \text{by \prefp{def:ball}}
    \end{align*}

%  \item Proof that $\seqn{x_n}\to x$ $\impliedby$ $\distance{x_n}{x}<\varepsilon$:
%    \begin{align*}
%      \distance{x_n}{x}<\varepsilon
%      \\&\implies \seqn{x_n}\to x
%    \end{align*}
%\end{enumerate}
\end{proof}


In \structe{distance space}s \xref{def:dspace}, not all \prope{convergent} sequences are \prope{Cauchy} \xref{ex:dspace_1n}.
However in a distance space with any \rele{power triangle inequality} \xref{def:trirel}, all \prope{convergent} sequences are
\prope{Cauchy} (next theorem).
%A sequence that is \prope{convergent} is always \prope{Cauchy} (next theorem).
%However, in a metric space, the converse is not true---a sequence that is \prope{convergent} is not in general \prope{Cauchy}.
%This is in contrast to the special case of a real sequence in the metric space $\dspace{\R}{\abs{x-y}}$. %with the usual metric ($\distance{x}{y}\eqd\abs{x-y}$).
%In this case, all Cauchy sequences are convergent and the Cauchy property is referred to as the 
%\hie{Cauchy condition}.\footnote{
%  \citerppc{whittaker1915}{13}{15}{2.22}
%  }
%---------------------------------------
\begin{theorem}
\footnote{
  in \structe{metric space}:
  \citerpgc{giles1987}{49}{0521359287}{Theorem 3.30},
  \citerpg{rosenlicht}{51}{0486650383},
  \citerppgc{apostol1975}{72}{73}{0201002884}{Theorem 4.6}
  }
\label{thm:convergent==>cauchy}
%---------------------------------------
Let $\pdspaceX$ be a \structe{power distance space}.
%Let $\dspaceX$ be a \structe{distance space} \xrefh{def:dspace} with \rele{power triangle inequality} \xrefh{def:trirel} $\trirelD$
%defined in terms a \fncte{power triangle function} \xrefh{def:ptf} $\ptfD$ on $\dspaceX$.
Let $\balln$ be an \structe{open ball} \xrefh{def:ball} on $\dspaceX$.
\thmboxp{
  For any $\opair{p}{\sigma}\in\Rx\times\Rp$,
  \\\indentx$
  \brb{\begin{array}{M}
    $\seqn{x_n}$ is \prope{convergent}\\ 
    in $\dspaceX$
  \end{array}}
  \implies
  \brb{\begin{array}{M}
    $\seqn{x_n}$ is \prope{Cauchy}\\ 
    in $\dspaceX$
  \end{array}}
  \implies
  \brb{\begin{array}{M}
    $\seqn{x_n}$ is \prope{bounded}\\ 
    in $\dspaceX$
  \end{array}}
$}
\end{theorem}
\begin{proof}
\begin{enumerate}
  \item Proof that \prope{convergent} $\implies$ \prope{Cauchy}:
    \begin{align*}
      \distance{x_n}{x_m}
        &\le \ptf(p,\sigma;x_n,x_m,x)
        &&   \text{by definition of \rele{power triangle inequality} \xref{def:trirel}}
      \\&\eqd 2\sigma\brs{\frac{1}{2}\distancep{p}{x_n}{x} + \frac{1}{2}\distancep{p}{x_m}{x}}^\frac{1}{p}
        && \text{by definition of \fncte{power triangle function} \xref{def:ptf}}
      \\&<    2\sigma\brs{\frac{1}{2}\varepsilon^p + \frac{1}{2}\varepsilon^p}^\frac{1}{p}
        && \text{by \prope{convergence} hypothesis \xref{def:converge}}
      \\&=    2\sigma\varepsilon
        && \text{by definition of \prope{convergence} \xref{def:converge}}
      \\&\implies \text{\prope{Cauchy}}
        && \text{by definition of \prope{Cauchy} \xref{def:cauchy}}
      \\
      \distance{x_n}{x_m}
        &\le \ptf(\infty,\sigma;x_n,x_m,x)
        &&   \text{by definition of \rele{power triangle inequality} at $p=\infty$}
      \\&=  2\sigma\max\setn{\distancen(x_n,x),\,\distancen(x_m,x)}
        && \text{by \prefpp{cor:ftri_means}}
      \\&=  2\sigma\max\setn{\varepsilon,\,\varepsilon}
        && \text{by \prope{convergent} hypothesis \xref{def:converge}}
      \\&=  2\sigma\varepsilon
        && \text{by definition of $\max$}
      \\
      \distance{x_n}{x_m}
        &\le \ptf(-\infty,\sigma;x_n,x_m,x)
        &&   \text{by definition of \rele{power triangle inequality} at $p=-\infty$}
      \\&=  2\sigma\min\setn{\distancen(x_n,x),\,\distancen(x_m,x)}
        && \text{by \prefpp{cor:ftri_means}}
      \\&=  2\sigma\min\setn{\varepsilon,\,\varepsilon}
        && \text{by \prope{convergent} hypothesis \xref{def:converge}}
      \\&=  2\sigma\varepsilon
        && \text{by definition of $\min$}
      \\
      \distance{x_n}{x_m}
        &\le \ptf(0,\sigma;x_n,x_m,x)
        &&   \text{by definition of \rele{power triangle inequality} at $p=0$}
      \\&=  2\sigma\sqrt{\distancen(x_n,x)}\,\sqrt{\distancen(x_m,x)}
        && \text{by \prefpp{cor:ftri_means}}
      \\&=  2\sigma\sqrt{\varepsilon}\,\sqrt{\varepsilon}
        && \text{by \prope{convergent} hypothesis \xref{def:converge}}
      \\&=  2\sigma\varepsilon
        && \text{by property of $\R$}
    \end{align*}

  \item Proof that \prope{Cauchy} $\implies$ \prope{bounded}: by \prefpp{prop:cauchy==>bounded}.

\end{enumerate}
\end{proof}

%---------------------------------------
\begin{theorem}
\footnote{
  in \structe{metric space}:
  \citerpg{rosenlicht}{52}{0486650383}
  }
%---------------------------------------
Let $\pdspaceX$ be a \structe{power distance space}.
Let $\ff\in\clFzz$ be a \prope{strictly monotone} function such that $\ff(n)<\ff(n+1)$.
\thmboxt{
For any $\opair{p}{\sigma}\in\Rx\times\Rp$
\\\indentx$
  \brb{\begin{array}{FMD}
    1. & $\seq{x_n}{n\in\Z}$ is \prope{Cauchy}     & and \\
    2. & $\seq{x_{\ff(n)}}{n\in\Z}$ is \prope{convergent}
  \end{array}}
  \qquad\implies\qquad
  \brb{\seq{x_n}{n\in\Z} \text{ is \prope{convergent}.}}
  $}
\end{theorem}
\begin{proof}
\begin{align*}
  \distance{x_n}{x}
    &= \distance{x}{x_n}
    && \text{by \prope{symmetric} property of $\distancen$}
  \\&\le \ptf(p,\sigma;x,x_n,x_{\ff(n)})
    &&   \text{by definition of \rele{power triangle inequality} \xref{def:trirel}}
  \\&\eqd 2\sigma\brs{\frac{1}{2}\distancep{p}{x}{x_{\ff(n)}} + \frac{1}{2}\distancep{p}{x_{\ff(n)}}{x_n}}^\frac{1}{p}
    && \text{by definition of \fncte{power triangle function} \xref{def:ptf}}
  \\&=  2\sigma\brs{\frac{1}{2}\varepsilon  + \frac{1}{2}\distancep{p}{x_{\ff(n)}}{x_n}}^\frac{1}{p}
    && \text{by left hypothesis 2}
  \\&=  2\sigma\brs{\frac{1}{2}\varepsilon^p  + \frac{1}{2}\varepsilon^p}^\frac{1}{p}
    && \text{by left hypothesis 1}
  \\&= 2\sigma\varepsilon
  \\&\implies \text{\prope{convergent}}
    && \text{by definition of \prope{convergent} \xref{def:converge}}
\end{align*}
\end{proof}

%---------------------------------------
\begin{theorem}
\footnote{
  in \structe{metric space} ($\opair{p}{\sigma}=\opair{1}{1}$ case):
  \citerpgc{berberian1961}{37}{0821819127}{Theorem~II.4.1}
  %\citerp{pedersen2000}{4}
  }
\label{thm:pdspace_continuous}
%---------------------------------------
Let $\pdspaceX$ be a \structe{power distance space}.
Let $\opair{\R}{\distancean}$ be a metric space of real numbers with the usual metric 
$\distancea{x}{y}\eqd\abs{x-y}$.
%Let 
%\\\indentx$\ff(p,\sigma,x,y,z)\eqd \brp{\max\brb{
%      \frac{2}{(2\sigma)^p}\distancep{p}{x}{z}-\distancep{p}{z}{y},\,
%      \frac{2}{(2\sigma)^p}\distancep{p}{y}{z}-\distancep{p}{z}{x}
%      }}^\frac{1}{p}$.\\
%\\\indentx$\ff(p,\sigma,x,y,z)\eqd \brp{\min\brb{0,\,
%    \max\brb{
%      \frac{2}{(2\sigma)^p}\distancep{p}{x}{z}-\distancep{p}{z}{y},\,
%      \frac{2}{(2\sigma)^p}\distancep{p}{y}{z}-\distancep{p}{z}{x}
%      }}}^{\frac{1}{p}}$.\\
Then
\thmbox{
%For all $p\in\Rx$, $\sigma\in\Rp$\\
%  \brb{\begin{array}{lC}
%    \ff(p,\sigma,x,y,z) \ge \abs{\distance{x}{z}-\distance{z}{y}} & \forall x,y,z\in\setX
%  \end{array}}
  \brb{2\sigma=2^\frac{1}{p}}
  \quad\implies\quad
  \brb{\begin{array}{M}
    $\distancen$ is \prope{continuous} in $\opair{\R}{\distancean}$
  \end{array}}
  }
\end{theorem}
\begin{proof}
\begin{align*}
  \abs{\distance{x}{y}-\distance{x_n}{y_n}}
    &\le \abs{\distance{x}{y}-\distance{x_n}{y}} + \abs{\distance{x_n}{y}-\distance{x_n}{y_n}}
    && \text{by \prope{triangle inequality} of $\metspace{\R}{\abs{x-y}}$}
  \\&=   \abs{\distance{x}{y}-\distance{y}{x_n}} + \abs{\distance{y}{x_n}-\distance{x_n}{y_n}}
    && \text{by \prope{commutative} property of $\distancen$ \xref{def:dspace}}
  \\&\le \distance{x}{x_n} + \distance{y}{y_n}
    && \text{by $2\sigma=2^\frac{1}{p}$ and \prefpp{lem:pdspace_ineq}}
  \\&=   0\qquad\text{as $n\to\infty$}
\end{align*}
\end{proof}

%%---------------------------------------
%\begin{theorem}
%\footnote{
%  in \structe{metric space} ($p=\sigma=1$ case):
%  \citerpgc{berberian1961}{37}{0821819127}{Theorem~II.4.1}
%  %\citerp{pedersen2000}{4}
%  }
%\label{thm:pdspace_continuous}
%%---------------------------------------
%Let $\pdspaceX$ be a \structe{power distance space}.
%Let $\opair{\R}{\distancean}$ be a metric space of real numbers with the usual metric 
%$\distancea{x}{y}\eqd\abs{x-y}$.
%\thmbox{
%%For all $p\in\Rx$, $\sigma\in\Rp$\\
%  \brb{\begin{array}{FlD}
%    1. & \opair{\seqn{x_n}}{\seqn{y_n}} \to \opair{x}{y}         & and \\
%    2. & \opair{p}{\sigma}\in(\R\setd\setn{-\infty,0})\times\Rp  & and \\
%    3. & 2\sigma = 2^{\frac{1}{p}}                             &
%  \end{array}}
%  \implies
%  \brb{\begin{array}{FMD}
%    1. & $\seqn{\distance{x_n}{y_n}} \to \distance{x}{y}$ & and \\
%    2. & $\distancen$ is \prope{continuous} in $\opair{\R}{\distancean}$ & and\\
%    3. & $p=\infty \quad\implies\quad \sigma=\frac{1}{2}$
%  \end{array}}
%  \quad%
%  {\begin{array}{C}
%    \forall x,y,x_n,y_n\in\setX 
%    %\forall p\in\Rx & and\\
%    %\forall \sigma\in\Rp
%  \end{array}}
%  }
%\end{theorem}
%\begin{proof}
%\begin{enumerate}
%  \item lemma: \label{ilem:metric_continuous}
%\begin{align*}
%  &\distancea{\distance{x}{y}}{\distance{x_n}{y_n}}
%  \\&\eqd \abs{\distance{x}{y}-\distance{x_n}{y_n}}
%    && \text{by definition of $\distancean$}
%  \\&\le \abs{                         % p(x-y) = |x-y| (usual metric)
%           \ptf(p,\sigma;x,y,x_n)   % d(x,y) <= \ptf(p,\sigma;x,y,x_n)
%          -\distance{x_n}{y_n}         %
%           }
%    &&   \text{by \prope{power triangle inequality} \xref{def:trirel}}
%  \\&=   \abs{
%           2\sigma\brs{\frac{1}{2}\distancen^p(x,x_n)+\frac{1}{2}\distancen^p(x_n,y)}^\frac{1}{p}  % \ptf(p,\sigma;x,y,x_n)
%          -\distance{x_n}{y_n}
%           }
%    &&   \text{by definition of $\ptf$ \xref{def:ptf}}
%  \\&\le \abs{
%           2\sigma\brs{\frac{1}{2}\distancen^p(x,x_n)+\frac{1}{2}
%           \ptf^p(p,\sigma;x_n,y,y_n)                           % \distancen^p(x_n,y) <=\ptf(p,\sigma;x_n,y,y_n)
%           }^\frac{1}{p}  % \ptf(p,\sigma;x,y,x_n)
%          -\distance{x_n}{y_n}
%           }
%    &&   \text{by \prope{power triangle inequality} \xref{def:trirel}}
%\end{align*}
%
%  \item Proof for the $p\in\Rx\setd\setn{-\infty,0,\infty}$ case:
%    \begin{align*}
%      &\distancea{\distance{x}{y}}{\distance{x_n}{y_n}}
%    %\\&\eqd \abs{\distance{x}{y}-\distance{x_n}{y_n}}
%    \\&\le \lim_{\varepsilon\to0}\abs{
%             2\sigma\brs{\frac{1}{2}\distancen^p(x,x_n)
%            +\frac{1}{2}\ptf^p(p,\sigma;x_n,y,y_n)                           % \distancen^p(x_n,y) <=\ptf(p,\sigma;x_n,y,y_n)
%             }^\frac{1}{p}  % \ptf(p,\sigma;x,y,x_n)
%            -\distance{x_n}{y_n}
%             }
%      &&   \text{by \prefp{ilem:metric_continuous}}
%    \\&\eqd \mathrlap{\lim_{\varepsilon\to0}\abs{
%             2\sigma\brs{\frac{1}{2}\distancen^p(x,x_n)+\frac{1}{2}
%             \brp{2\sigma\brs{\frac{1}{2}\distancen^p(x_n,y_n)+\frac{1}{2}\distancen^p(y_n,y)}^\frac{1}{p}}^p  % \ptf(p,\sigma;x_n,y,y_n)
%             }^\frac{1}{p}  % \ptf(p,\sigma;x,y,x_n)
%            -\distance{x_n}{y_n}
%             }
%      \quad\text{by def. of $\ptf$ \xref{def:ptf}}}
%    \\&=  \lim_{\varepsilon\to0}\abs{
%             2\sigma\brs{\frac{1}{2}\varepsilon+\frac{1}{2}
%             \brp{2\sigma\brs{\frac{1}{2}\distancen^p(x_n,y_n)+\frac{1}{2}\varepsilon}^\frac{1}{p}}^p  % \ptf(p,\sigma;x_n,y,y_n)
%             }^\frac{1}{p}  % \ptf(p,\sigma;x,y,x_n)
%            -\distance{x_n}{y_n}
%             }
%      &&   \text{by $\opair{\seqn{x_n}}{\seqn{y_n}}\to\opair{x}{y}$ hypothesis}
%    \\&=  \abs{
%             2\sigma\brs{\frac{1}{2}
%             \brp{2\sigma\brs{\frac{1}{2}\distancen^p(x_n,y_n)}^\frac{1}{p}}^p  % \ptf(p,\sigma;x_n,y,y_n)
%             }^\frac{1}{p}  % \ptf(p,\sigma;x,y,x_n)
%            -\distance{x_n}{y_n}
%             }
%    \\&=  \abs{
%             2\sigma\brs{\brp{\frac{1}{2}}^\frac{1}{p}
%             \brp{2\sigma\brs{\frac{1}{2}\distancen^p(x_n,y_n)}^\frac{1}{p}} 
%             }
%            -\distance{x_n}{y_n}
%             }
%    \\&=  \abs{
%             2\sigma\brs{\brp{\frac{1}{2}}^\frac{1}{p}
%             \brp{2\sigma\brs{\brp{\frac{1}{2}}^\frac{1}{p}\distancen(x_n,y_n)}} 
%             }
%            -\distance{x_n}{y_n}
%             }
%    \\&=  \abs{4\sigma^2{\brp{\frac{1}{2}}^\frac{2}{p}{{\distancen(x_n,y_n)}} }-\distance{x_n}{y_n}  }
%    \\&=  \abs{4\sigma^2\brp{\frac{1}{2}}^\frac{2}{p}-1}\distance{x_n}{y_n}
%    \\&\implies \text{convergence when $4\sigma^2 = 2^\frac{2}{p}$}
%     &&\text{(see \prefp{fig:sigmap})}
%    \\&\implies \text{convergence when $p=\frac{\ln2}{\ln(2\sigma)}$} 
%     &&\text{(see \prefp{fig:sigmap})}
%    \end{align*}
%
%  \item Proof for $\opair{p}{\sigma}=\opair{\infty}{\frac{1}{2}}$ case:
%    \begin{align*}
%      &\distancea{\distance{x}{y}}{\distance{x_n}{y_n}}
%  \\&\le \abs{
%           2\sigma\brs{\frac{1}{2}\distancen^p(x,x_n)
%          +\frac{1}{2}\ptf^p(p,\sigma;x_n,y,y_n)                           % \distancen^p(x_n,y) <=\ptf(p,\sigma;x_n,y,y_n)
%           }^\frac{1}{p}  % \ptf(p,\sigma;x,y,x_n)
%          -\distance{x_n}{y_n}
%           }
%    &&   \text{by \prefp{ilem:metric_continuous}}
%  \\&= \mathrlap{\abs{
%           2\sigma\brp{\frac{1}{2}\distancen^p(x,x_n)
%          +\frac{1}{2}\brs{2\sigma\max\setn{\distancen(x_n,y_n),\,\distancen(y_n,y)}}^p
%           }^\frac{1}{p}  % \ptf(p,\sigma;x,y,x_n)
%          -\distance{x_n}{y_n}
%           }
%    \quad\text{by \prefp{cor:ftri_means}}}
%  \\&= \abs{
%           2\sigma\brs{\frac{1}{2}\distancen^p(x,x_n)
%          +\frac{1}{2}\brs{2\sigma\distancen(x_n,y_n)}^p
%           }^\frac{1}{p}  % \ptf(p,\sigma;x,y,x_n)
%          -\distance{x_n}{y_n}
%           }
%    &&   \text{by $\opair{\seqn{x_n}}{\seqn{y_n}}\to\opair{x}{y}$ hypothesis}
%  \\&= \abs{2\sigma\max\setn{\distancen(x,x_n),\,2\sigma\distancen(x_n,y_n)}
%          -\distance{x_n}{y_n}
%           }
%    && \text{by \prefp{thm:seq_Mr}}
%  \\&= \abs{4\sigma^2\distancen(x_n,y_n)-\distance{x_n}{y_n}}
%    && \text{by $\opair{\seqn{x_n}}{\seqn{y_n}}\to\opair{x}{y}$ hypothesis}
%  \\&= \abs{4\sigma^2-1}\distancen(x_n,y_n)
%    && \text{by \prope{non-negative} property of $\distancen$ \xref{def:distance}}
%  \\&= 0\quad\text{for}\quad\sigma=\frac{1}{2}
%    \end{align*}
%
%  \item Failed proof for $p=-\infty$ case:
%    \begin{align*}
%      &\distancea{\distance{x}{y}}{\distance{x_n}{y_n}}
%  \\&\le \abs{
%           2\sigma\brs{\frac{1}{2}\distancen^p(x,x_n)
%          +\frac{1}{2}\ptf^p(p,\sigma;x_n,y,y_n)                           % \distancen^p(x_n,y) <=\ptf(p,\sigma;x_n,y,y_n)
%           }^\frac{1}{p}  % \ptf(p,\sigma;x,y,x_n)
%          -\distance{x_n}{y_n}
%           }
%    &&   \text{by \prefp{ilem:metric_continuous}}
%  \\&= \mathrlap{\abs{
%           2\sigma\brp{\frac{1}{2}\distancen^p(x,x_n)
%          +\frac{1}{2}\brs{2\sigma\min\setn{\distancen(x_n,y_n),\,\distancen(y_n,y)}}^p
%           }^\frac{1}{p}  % \ptf(p,\sigma;x,y,x_n)
%          -\distance{x_n}{y_n}
%           }
%    \quad\text{by \prefp{cor:ftri_means}}}
%  \\&= \abs{
%           2\sigma\brs{\frac{1}{2}\distancen^p(x,x_n)
%          +\frac{1}{2}\brs{2\sigma\distancen(y_n,y)}^p
%           }^\frac{1}{p}
%          -\distance{x_n}{y_n}
%           }
%    && \text{by $\opair{\seqn{x_n}}{\seqn{y_n}}\to\opair{x}{y}$ hypothesis}
%  \\&= \abs{2\sigma\min\setn{\distancen(x,x_n),\,2\sigma\distancen(y_n,y)} -\distance{x_n}{y_n} }
%    && \text{by \prefp{thm:seq_Mr}}
%  \\&= \abs{\distancen(x_n,y_n)}
%    && \text{by $\opair{\seqn{x_n}}{\seqn{y_n}}\to\opair{x}{y}$ hypothesis}
%  \\&= \distancen(x_n,y_n)
%    && \text{by \prope{non-negative} property of $\distancen$ \xref{def:distance}}
%  \\&\neq 0
%    && \text{$\neq0$ for any value of $\sigma$}
%    \end{align*}
%
%  \item Failed proof for the $p=0$ case:
%    \begin{enumerate}
%      \item demonstration of failed limit:
%    \begin{align*}
%      &\distancea{\distance{x}{y}}{\distance{x_n}{y_n}}
%  \\&\le \abs{
%           2\sigma\brs{\frac{1}{2}\distancen^p(x,x_n)
%          +\frac{1}{2}\ptf^p(p,\sigma;x_n,y,y_n)                           % \distancen^p(x_n,y) <=\ptf(p,\sigma;x_n,y,y_n)
%           }^\frac{1}{p}  % \ptf(p,\sigma;x,y,x_n)
%          -\distance{x_n}{y_n}
%           }
%    &&   \text{by \prefp{ilem:metric_continuous}}
%  \\&= \mathrlap{\abs{
%           2\sigma\brp{\frac{1}{2}\distancen^p(x,x_n)
%          +\frac{1}{2}\brs{2\sigma\sqrt{\distancen(x_n,y_n)}\sqrt{\distancen(y_n,y)}}^p
%           }^\frac{1}{p}  % \ptf(p,\sigma;x,y,x_n)
%          -\distance{x_n}{y_n}
%           }
%    \quad\text{by \prefp{cor:ftri_means}}}
%  \\&= \abs{
%           2\sigma\sqrt{\distancen(x,x_n)}
%          \sqrt{2\sigma\sqrt{\distancen(x_n,y_n)}\sqrt{\distancen(y_n,y)}}
%          -\distance{x_n}{y_n}
%           }
%    &&   \text{by \prefp{cor:ftri_means}}
%  \\&= \abs{
%           2\sigma\sqrt{\varepsilon}
%          \sqrt{2\sigma\sqrt{\distancen(x_n,y_n)}\sqrt{\varepsilon}}
%          -\distance{x_n}{y_n}
%           }
%    &&   \text{by \prefp{cor:ftri_means}}
%  \\&= \mathrlap{\distancen(x_n,y_n)
%    \qquad\text{by \prope{non-negative} property of $\distancen$ \xref{def:distance}}}
%  \\&\neq 0
%    && \text{$\neq0$ for any value of $\sigma$}
%    \end{align*}
%
%      \item For whatever (if anything) it's worth, note that the constraint $4\sigma^2=2^{\frac{2}{p}}$ is \prope{discontinuous} at $p=0$:
%        \\\indentx$\ds
%          \lim_{p\to0^+}\sigma
%            = \lim_{p\to0^+}\sqrt{\frac{1}{2}2^{\frac{2}{p}}} 
%            = \infty
%            \neq 0
%            = \lim_{p\to0^-}\sqrt{\frac{1}{2}2^{\frac{2}{p}}} 
%            = \lim_{p\to0^-}\sigma
%          $ 
%    \end{enumerate}
%
%  \item Proof that $\distancen$ is \prope{continuous}: this follows directly from \prefpp{thm:limcont}.
%\end{enumerate}
%\end{proof}

In \structe{distance space}s and \structe{topological space}s, limits of convergent sequences are in general \prope{not unique} 
\xxref{ex:dspace_01}{ex:seq_xt31}.
However \pref{thm:xn_to_xy} (next) demonstrates that, in a \structe{power distance space},
limits \emph{are} unique.
%---------------------------------------
\begin{theorem}[\thmd{Uniqueness of limit}]
\footnote{
  in \structe{metric space}:
  \citerpg{rosenlicht}{46}{0486650383},
  \citerpgc{thomson2008}{32}{143484367X}{Theorem 2.8}
  }
\label{thm:xn_to_xy}
%---------------------------------------
Let $\pdspaceX$ be a \structe{power distance space}.
Let $x,y,\in \setX$ and let $\seqn{x_n\in\setX}$ be an $\setX$-valued sequence.
%Then for all $\opair{p}{\sigma}\in(\Rx\setd\setn{0})\times\Rp$,
\thmbox{
  \brb{\begin{array}{FlD}
    1. & \brb{\opair{\seqn{x_n}}{\seqn{y_n}}\to\opair{x}{y}} & and\\
    2. & \opair{p}{\sigma}\in\Rx\times\Rp                    &
  \end{array}}
  \quad\implies\quad
  \brb{x=y}%{the \structe{limit} of a \prope{convergent} sequence is \prope{unique}}
  }
\end{theorem}
\begin{proof}
  \begin{enumerate}
    \item lemma: Proof that for all $\opair{p}{\sigma}\in\Rx\times\Rp$ and for any $\varepsilon\in\Rp$, 
          there exists $\xN$ such that $\distance{x}{y} <2\sigma\varepsilon$\label{ilem:xn_to_xy_2e}:
      \begin{align*}
        \distance{x}{y} 
          &\le  \ptf(p,\sigma;x,y,x_n)
          &&    \text{by definition of \ineqe{power triangle inequality} \xref{def:trirel}}
        \\&\eqd 2\sigma\brs{\frac{1}{2}\distancep{p}{x}{x_n} + \frac{1}{2}\distancep{p}{x_n}{y}}^\frac{1}{p}
          &&    \text{by definition of \fncte{power triangle function} \xref{def:ptf}}
        \\&<    2\sigma\brs{\frac{1}{2}\varepsilon^p + \frac{1}{2}\varepsilon^p}^\frac{1}{p}
          &&    \text{by left hypothesis and for $p\in\Rx\setd\setn{-\infty,0,\infty}$}
        \\&=    2\sigma\varepsilon
        \\
        \distance{x}{y} 
          &\le  \ptf(\infty,\sigma;x,y,x_n)
          &&    \text{by definition of \ineqe{power triangle inequality} at $p=\infty$}
        \\&=    2\sigma\max\setn{\distancen(x,x_n),\,\distancen(x_n,y)}
          &&    \text{by \prefpp{cor:ftri_means}}
        \\&<    2\sigma\varepsilon
          &&    \text{by left hypothesis}
        \\
        \distance{x}{y} 
          &\le  \ptf(-\infty,\sigma;x,y,x_n)
          &&    \text{by definition of \ineqe{power triangle inequality} at $p=-\infty$}
        \\&=    2\sigma\min\setn{\distancen(x,x_n),\,\distancen(x_n,y)}
          &&    \text{by \prefpp{cor:ftri_means}}
        \\&<    2\sigma\varepsilon
          &&    \text{by left hypothesis}
        \\
        \distance{x}{y} 
          &\le  \ptf(0,\sigma;x,y,x_n)
          &&    \text{by definition of \ineqe{power triangle inequality} at $p=0$}
        \\&=    2\sigma\sqrt{\distancen(x,x_n)}\,\sqrt{\distancen(x_n,y)}
          &&    \text{by \prefpp{cor:ftri_means}}
        \\&=    2\sigma\sqrt{\varepsilon}\,\sqrt{\varepsilon}
          &&    \text{by left hypothesis}
        \\&<    2\sigma\varepsilon
          &&    \text{by property of real numbers}
      \end{align*}
      %\begin{align*}
      %  &\seqn{x_n} \to x \text{ and }  \seqn{x_n} \to y
      %  \\&\implies \exists\xN \st \forall n>\xN,\;  \distance{x}{x_n}<\varepsilon \text{ and }  \distance{x_n}{y}<\varepsilon
      %  \\&\implies \exists\xN \st \forall n>\xN,\;  \distance{x}{y} \le \ptf(p,\sigma;x,y,x_n)
      %    && \text{by def. of p.t.ineq. \xref{def:trirel}}
      %  \\&\implies \exists\xN \st \forall n>\xN,\;  \distance{x}{y} \le 2\sigma\brs{\frac{1}{2}\distancen^p(x,x_n) + \frac{1}{2}\distancen^p(x_n,y)}^\frac{1}{p}
      %    && \text{by def. of p.t.ineq. \xref{def:trirel}}
      %  \\&\implies \exists\xN \st \forall n>\xN,\;  \distance{x}{y} \le 2\sigma\brs{\frac{1}{2}\varepsilon^p + \frac{1}{2}\varepsilon^p}^\frac{1}{p}
      %    && \text{by left hypothesis}
      %  \\&\implies \exists\xN \st \forall n>\xN,\;  \distance{x}{y} \le 2\sigma\varepsilon
      %\end{align*}
  
    \item Proof that $x=y$ (proof by contradiction): %$\distance{x}{y}=0$:
      \begin{align*}
        x\neq y
          &\implies \distance{x}{y}\neq0
          && \text{by the \prope{nondegenerate} property of $\distancen$ \xref{def:distance}}
        \\&\implies \distance{x}{y}>0
          && \text{by \prope{non-negative} property of $\distancen$ \xref{def:distance}}
        \\&\implies \exists \varepsilon \st \distance{x}{y}>2\sigma\varepsilon
        \\&\implies \text{\vale{contradiction} to \prefp{ilem:xn_to_xy_2e}}
        \\&\implies \distance{x}{y}=0
        \\&\implies x=y
      \end{align*}
      %\begin{enumerate}
      %  \item If $\distance{x}{y}>0$, then we could choose an arbitrarily small $\varepsilon$ such that
      %    \quad$\distance{x}{y} >2\varepsilon$
      %  \item But this would contradict the earlier result of $\distance{x}{y} <2\varepsilon$.
      %  \item Therefore, $\distance{x}{y}=0$ (proof by contradiction).
      %\end{enumerate}
  
    %\item Proof that $x=y$:
    %  \begin{align*}
    %    \distance{x}{y}=0
    %      &\implies x=y
    %      && \text{by the \prope{nondegenerate} property of $\distancen$ \xref{def:distance}}
    %  \end{align*}
  \end{enumerate}
\end{proof}


%%=======================================
%\subsection{Properties of selected distance spaces}
%%=======================================
%A \structe{distance space} does not necessarily have all the nice properties (next theorem)
%that a \structe{metric space} has \xref{thm:tri_metric}.
%%---------------------------------------
%\begin{theorem}
%%---------------------------------------
%Let $\seqxZ{x_n}$ be a \structe{sequence} in a \structe{distance space} \xref{def:dspace} $\dspaceX$.
%\\\thmboxp{$\begin{array}{FMcMM}
%  1. & $\distancen$   is a \fncte{distance} in $\dspaceX$ &\notimplies& $\distancen$ is \prope{continuous} in $\dspaceX$  & \xref{ex:dspace_21}.\\
%  2. & $\balln$ is an \structe{open ball} in $\dspaceX$ &\notimplies& $\balln$ is \prope{open} in $\dspaceX$          & \xref{ex:dspace_1n}.\\
%  3. & $\baseB$ is the set of all \structe{open ball}s in $\dspaceX$ &\notimplies& $\baseB$ is a \structe{base} for a topology on $\setX$ & \xref{ex:dspace_1n}.\footnotemark\\
%  4. & $\seqn{x_n}$ is \prope{convergent} in $\dspaceX$ &\notimplies& limit is \prope{unique}                           & \xref{ex:dspace_01}.\\
%  5. & $\seqn{x_n}$ is \prope{convergent} in $\dspaceX$ &\notimplies& $\seqn{x_n}$ is \prope{Cauchy} in $\dspaceX$    & \xref{ex:dspace_1n}.\\
%\end{array}$}
%%In a \structe{metric space} \xref{def:metric}, if a \fncte{sequence} has a limit, that limit is unique.
%%This is not true in general for \structe{distance space}s \xref{def:distance}.
%%See \prefpp{ex:dspace_01} for a distance space which converges to two distinct limits.
%\footnotetext{
%  \citePpc{heath1961}{810}{{\scshape Theorem}},
%  \citePpc{galvin1984}{71}{{\scshape 2.3 Lemma}}
%  }
%\end{theorem}
%
%%---------------------------------------
%\begin{theorem}
%\label{thm:tri_metric}
%%---------------------------------------
%Let $\seqxZ{x_n}$ be a \structe{sequence} in a \structe{distance space} \xref{def:dspace} $\dspaceX$.
%\\If $\dspaceX$ is a \structd{metric space} \xref{def:metric} then
%\\\thmboxp{$\begin{array}{FMcMM}
%  1. & \mc{3}{M}{$\distancen$ is \prope{continuous} in $\dspaceX$      }                                            & \xref{thm:pdspace_continuous}.\\
%  2. & $\balln$ is an \structe{open ball} in $\dspaceX$ &\implies& $\balln$ is \prope{open} in $\dspaceX$& \xref{cor:oball_open}.\\
% %3. & \mc{3}{M}{$\baseB$ is a \structe{base} for a topology on $\setX$}                                            & \xref{ex:dspace_1n}.\footnotemark\\
%  3. & $\baseB$ is the set of all \structe{open ball}s in $\dspaceX$ &\implies& $\baseB$ is a \structe{base} for a topology on $\setX$ & \xref{cor:tspace_base}.\\
%  4. & $\seqn{x_n}$ is \prope{convergent} in $\dspaceX$ &\implies& limit is \prope{unique}                        & \xref{thm:xn_to_xy}.\\
%  5. & $\seqn{x_n}$ is \prope{convergent} in $\dspaceX$ &\implies& $\seqn{x_n}$ is \prope{Cauchy} in $\dspaceX$ & \xref{prop:convergent==>cauchy}.\\
%\end{array}$}
%\end{theorem}
%
%%---------------------------------------
%\begin{theorem}
%\label{thm:tri_nmetric}
%%---------------------------------------
%Let $\seqxZ{x_n}$ be a \structe{sequence} in a \structe{distance space} \xref{def:dspace} $\dspaceX$.
%\\If $\dspaceX$ is a \structd{near metric space} \xref{def:nmetric} then
%\\\thmboxp{$\begin{array}{FMcMM}
%  1. & \mc{3}{M}{$\distancen$ is \prope{continuous} in $\dspaceX$      }                                            & \xref{prop:nmetric_metric_continuous}.\\
%  2. & $\balln$ is an \structe{open ball} in $\dspaceX$ &\implies& $\balln$ is \prope{open} in $\dspaceX$& \xref{prop:nmetric_oball_open}.\\
%  3. & $\baseB$ is the set of all \structe{open ball}s in $\dspaceX$ &\implies& $\baseB$ is a \structe{base} for a topology on $\setX$ & \xref{prop:nmetric_(X,d)->(X,t)}.\\
%  4. & $\seqn{x_n}$ is \prope{convergent} in $\dspaceX$ &\implies& limit is \prope{unique}                        & \xref{prop:nmetric_xn_to_xy}.\\
%  5. & $\seqn{x_n}$ is \prope{convergent} in $\dspaceX$ &\implies& $\seqn{x_n}$ is \prope{Cauchy} in $\dspaceX$ & \xref{prop:nmetric_convergent==>cauchy}.\\
%\end{array}$}
%\end{theorem}
%

%=======================================
\section{Examples}
%=======================================
It is not always possible to find a \rele{triangle relation} \xref{def:trirel} $\trirelD$
that holds in every \structe{distance space} \xref{def:dspace}, as demonstrated by 
\pref{ex:pdspace_01} and \pref{ex:pdspace_1n} (next two examples).
%---------------------------------------
\begin{example}
\label{ex:pdspace_01}
%---------------------------------------
Let $\distance{x}{y}\in\clF{\R\times\R}{\R}$ be defined such that
\\\indentx$\distance{x}{y} \eqd \brb{\begin{array}{lMD}
      y         & $\forall \opair{x}{y}\in\setn{4}\times\intoc{0}{2}$ & (\structe{vertical half-open interval})\\
      x         & $\forall \opair{x}{y}\in\intoc{0}{2}\times\setn{4}$ & (\structe{horizontal half-open interval})\\
      \abs{x-y} & otherwise                                           & (\prope{Euclidean})
    \end{array}}$.
\\
Note the following about the pair $\dspaceRd$:
\begin{enumerate}
  \item By \prefpp{ex:dspace_01}, $\dspaceRd$ is a \structe{distance space}, but not a \structe{metric space}---that is,
        the \rele{triangle relation} $\trirel(1,1;\distancen)$ does not hold in $\dspaceRd$.
  \item Observe further that $\dspaceRd$ is \emph{not} a \structe{power distance space}.
        In particular, the \rele{triangle relation} $\trirelD$ does not hold in $\dspaceRd$
        for any finite value of $\sigma$ (does not hold for any $\sigma\in\Rp$):
       \begin{align*}
         \distance{0}{4} 
           = 4 
            \nle 0
            = \lim_{\varepsilon\to0}2\sigma\varepsilon
           &= \lim_{\varepsilon\to0}2\sigma\brs{\sfrac{1}{2}\abs{0-\varepsilon}^p + \sfrac{1}{2}{\varepsilon}^p}^\frac{1}{p}
         \\&\eqd \lim_{\varepsilon\to0}2\sigma\brs{\sfrac{1}{2}\distancep{p}{0}{\varepsilon} + \sfrac{1}{2}\distancep{p}{\varepsilon}{4}}^\frac{1}{p}
            \eqd \lim_{\varepsilon\to0}\trirel(p,\sigma;0,4,\varepsilon;\distancen)
       \end{align*}
\end{enumerate}
\end{example}

%---------------------------------------
\begin{example}
\label{ex:pdspace_1n}
%---------------------------------------
Let $\distance{x}{y}\in\clF{\R\times\R}{\R}$ be defined such that
\\\indentx$\distance{x}{y} \eqd \brb{\begin{array}{lMD}
      \abs{x-y} & for $x=0$ or $y=0$ or $x=y$ & (\prope{Euclidean})\\
      1         & otherwise                   & (\prope{discrete})
    \end{array}}$.\\
Note the following about the pair $\dspaceRd$:
\begin{enumerate}
  \item By \prefpp{ex:dspace_1n}, $\dspaceRd$ is a \structe{distance space}, but not a \structe{metric space}---that is,
        the \rele{triangle relation} $\trirel(1,1;\distancen)$ does not hold in $\dspaceRd$.
  \item Observe further that $\dspaceRd$ is \emph{not} a \structe{power distance space}---that is,
        the \rele{triangle relation} $\trirelD$ does not hold in $\dspaceRd$
        for any value of $\opair{p}{\sigma}\in\Rx\times\Rp$.
    \begin{enumerate}
      \item Proof that $\trirelD$ does not hold for any $\opair{p}{\sigma}\in\setn{\infty}\times\Rp$:\label{item:pdspace_1n_infty}
        \begin{align*}
          \lim_{n,m\to\infty} \distance{\sfrac{1}{n}}{\sfrac{1}{m}}
            &\eqd 1 \nleq 0 = 2\sigma\max\setn{0,0}
            && \text{by definition of $\distancen$}
         %\\&\nleq0
          \\&=2\sigma\lim_{n,m\to\infty} \max\setn{\distance{\sfrac{1}{n}}{0},\,\distance{0}{\sfrac{1}{m}}}
            && \text{by \prefpp{cor:ftri_means}}
          \\&\ge\lim_{n,m\to\infty}2\sigma \brs{\sfrac{1}{2}\distancep{p}{\sfrac{1}{n}}{0}+\sfrac{1}{2}\distancep{p}{0}{\sfrac{1}{m}}}^{\frac{1}{p}}
           %&&\text{by \prope{continuous} and \prope{strictly monotone} properties of $\ptf$ \xref{cor:tri_mono}}
            &&\text{by \prefpp{cor:tri_mono}}
          \\&\eqd\lim_{n,m\to\infty}\ptf(p,\sigma,\sfrac{1}{n},\sfrac{1}{m},0)
            &&\text{by definition of $\ptf$ \xref{def:ptf}}
        \end{align*}
      \item Proof that $\trirelD$ does not hold for any $\opair{p}{\sigma}\in\Rx\times\Rp$:
            By \prefpp{cor:tri_mono}, the \fncte{triangle function} \xref{def:ptf} $\ptfD$ 
            is \prope{continuous} and \prope{strictly monotone} in $\omsR$ with respect to the variable $p$.
            Item~\ref{item:pdspace_1n_infty} demonstrates that $\trirelD$ fails to hold at the best case of $p=\infty$, 
            and so by \pref{cor:tri_mono}, it doesn't hold for any other value of $p\in\Rx$ either.
  \end{enumerate}
\end{enumerate}
\end{example}

%---------------------------------------
\begin{example}
\label{ex:pdspace_21}
%---------------------------------------
Let $\distancen$ be a function in $\clF{\R\times\R}{\R}$ such that
\\\indentx$\distance{x}{y} \eqd \brb{\begin{array}{rMD}
      2\abs{x-y} & $\forall \opair{x}{y}\in\setn{\opair{0}{1},\,\opair{1}{0}}$ & (\prope{dilated Euclidean})\\
       \abs{x-y} & otherwise                                                   & (\prope{Euclidean})
    \end{array}}$.
\\
Note the following about the pair $\dspaceRd$:
\begin{enumerate}
  \item By \prefpp{ex:dspace_21}, $\dspaceRd$ is a \structe{distance space}, but not a \structe{metric space}---that is,
        the \rele{triangle relation} $\trirel(1,1;\distancen)$ does not hold in $\dspaceRd$.
  \item But observe further that $\pdspace{\R}{\distancen}{1}{2}$ \emph{is} a \structe{power distance space}:\label{item:pdspace_21}
    \begin{enumerate}
      \item Proof that $\trirel(1,2;\distancen)$ \xref{def:trirel} holds for all $\opair{x}{y}\in\setn{\opair{0}{1},\,\opair{1}{0}}$:
        \begin{align*}
          \distance{1}{0}
            &= \distance{0}{1}
             \eqd 2\abs{0-1}
             = 2
            && \text{by definition of $\distancen$}
            %&& \text{by definition of $\absn$ \xref{def:abs}}
          \\&\le 2
             \le 2\brp{\abs{0-z} + \abs{z-1}}\quad\scy\forall z\in\R
            && \text{by definition of $\absn$ \xref{def:abs}}
          \\&=   2\sigma\brp{\sfrac{1}{2}\abs{0-z}^p + \sfrac{1}{2}\abs{z-1}^p}^\frac{1}{p}\quad\scy\forall z\in\R
            && \text{for $\opair{p}{\sigma}=\opair{1}{2}$}
          \\&\eqd 2\sigma\brp{\sfrac{1}{2}\distancep{p}{0}{z} + \distancep{p}{z}{1}}^\frac{1}{p}\quad\scy\forall z\in\R%,\,\opair{p}{\sigma}=\opair{1}{2}
            && \text{for $\opair{p}{\sigma}=\opair{1}{2}$ and by definition of $\distancen$}
          \\&\eqd \ptf(1,2;0,1,z)
            && \text{by definition of $\ptf$ \xref{def:ptf}}
        \end{align*}

      \item Proof that $\trirel(1,2;\distancen)$ holds for all other $\opair{x}{y}\in\Rx\times\Rp$:
        \begin{align*}
          \distance{x}{y}
            &\eqd 2\abs{x-y}
            && \text{by definition of $\distancen$}
          \\&\le \brp{\abs{x-z} + \abs{z-y}}
            && \text{by property of \structe{Euclidean metric space}s}
          \\&=   2\sigma\brp{\sfrac{1}{2}\abs{0-z}^p + \sfrac{1}{2}\abs{z-1}^p}^\frac{1}{p}
            && \text{for $\opair{p}{\sigma}=\opair{1}{1}$}
          \\&\eqd \ptf(1,1;x,y,z)
            && \text{by definition of $\ptf$ \xref{def:ptf}}
          \\&\le  \ptf(1,2;x,y,z)
            && \text{by \prefpp{cor:tri_mono}}
        \end{align*}
    \end{enumerate}

  \item In $\dspaceX$, the limits of \prope{convergent} sequences are \prope{unique}.
        This follows directly from the fact that $\pdspace{\R}{\distancen}{1}{2}$ is a \structe{power distance space} \xref{item:pdspace_21}
        and by \prefp{thm:xn_to_xy}.

  \item In $\dspaceX$, \prope{convergent} sequences are \prope{Cauchy}.
        This follows directly from the fact that $\pdspace{\R}{\distancen}{1}{2}$ is a \structe{power distance space} \xref{item:pdspace_21}
        and by \prefp{thm:convergent==>cauchy}.
\end{enumerate}
\end{example}

%---------------------------------------
\begin{example}
\label{ex:pdspace_xy2}
%---------------------------------------
Let $\distancen$ be a function in $\clF{\R\times\R}{\R}$ such that $\distance{x}{y}\eqd(x-y)^2$.
Note the following about the pair $\dspaceRd$:
\begin{enumerate}
  \item It was demonstrated in \prefpp{ex:dspace_xy2} that $\dspaceRd$ is a \structe{distance space},
        but that it is \emph{not} a \structe{metric space} because the \prope{triangle inequality} does not hold.

  \item However, the tuple $\pdspace{\R}{\distancen}{p}{\sigma}$ \emph{is} a 
        \structe{power distance space} \xref{def:pdspace} for any $\opair{p}{\sigma}\in\Rx\times\intco{2}{\infty}$:
        In particular, for all $x,y,z\in\R$, the \rele{power triangle inequality} \xref{def:ptineq} must hold.
        The ``worst case" for this is when a third point $z$ is exactly ``halfway between" $x$ and $y$ in $\distance{x}{y}$; 
        that is, when $z=\frac{x+y}{2}$:
        \begin{align*}
          \brp{x-y}^2
            &\eqd \distance{x}{y}
            && \text{by definition of $\distancen$}
          \\&\le  \ptfD
            && \text{by definition \prope{power triangle inequality}}
          \\&\eqd 2\sigma\brs{\sfrac{1}{2}\distancep{p}{x}{z}+\sfrac{1}{2}\distancep{p}{z}{y}}^\frac{1}{p}
            && \text{by definition $\ptf$ \xref{def:ptf}}
          \\&\eqd  2\sigma\brs{\sfrac{1}{2}\brp{x-z}^{2p}+\sfrac{1}{2}\brp{z-y}^{2p}}^\frac{1}{p}
            && \text{by definition of $\distancen$}
          \\&=     2\sigma\brs{\sfrac{1}{2}\abs{x-z}^{2p}+\sfrac{1}{2}\abs{z-y}^{2p}}^\frac{1}{p}
            && \text{because $(x)^2 = \abs{x}^2$ for all $x\in\R$}
          \\&=     2\sigma\brs{\sfrac{1}{2}\abs{x-\frac{x+y}{2}}^{2p}+\sfrac{1}{2}\abs{\frac{x+y}{2}-y}^{2p}}^\frac{1}{p}
            && \text{because $z=\frac{x+y}{2}$ is the ``worst case" scenario}
          \\&=     2\sigma\brs{\sfrac{1}{2}\abs{\frac{y-x}{2}}^{2p}+\sfrac{1}{2}\abs{\frac{x-y}{2}}^{2p}}^\frac{1}{p}
          \\&=     2\sigma\brs{\abs{\frac{x-y}{2}}^{2p}}^\frac{1}{p}
             =     \frac{2\sigma}{4}\abs{x-y}^{2}
          %\\&\implies\text{$\sigma\ge2$ and $p\in\Rx$}
          \\&\implies \opair{p}{\sigma}\in\Rx\times\intco{2}{\infty}
        \end{align*}

  \item The \fncte{power distance function} $\distancen$ is \prope{continuous} in 
        $\pdspace{\R}{\distancen}{p}{\sigma}$
        for any $\opair{p}{\sigma}$ such that $\sigma\ge2$ and $2\sigma=p^\frac{1}{p}$.
        This follows directly from \prefpp{thm:pdspace_continuous}.
\end{enumerate}
\end{example}








