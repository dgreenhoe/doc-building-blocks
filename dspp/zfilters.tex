%============================================================================
% XeLaTeX File
% Daniel J. Greenhoe
%============================================================================

%=======================================
\chapter{z-domain Filters}
%=======================================
%---------------------------------------
\section*{Linear Time-Invariant (LTI) Systems}
%---------------------------------------
\newpage
\includegraphics[width=\tw-50mm]{graphics/cosLcos.pdf}

2 things can change: 
\\\indentx\begin{tabular}{clc}
    \imark&magnitue& $A$
  \\\imark&phase   & $\phi$
\end{tabular}

These can be written as an \hib{ordered pair} $\opair{A}{\phi}$.

\newpage
The ordered pair is one of the most fundamental concepts in mathematics.

$\begin{array}{|c|c|}
   \hline
   \includegraphics[height=80mm]{graphics/123le12.pdf}
  &\includegraphics[height=80mm]{graphics/fnctsq.pdf}
  \\\le = \setn{\opair{1}{1},\,\opair{1}{2},\,\opair{2}{2}}
  &\ff(x)\eqd x^2 = \setn{\opair{1}{1},\,\opair{-1}{1},\,\opair{2}{4},\,\opair{-2}{4}}
  \\\hline
\end{array}$

\newpage



  $\ds\begin{array}{rcl cl C}
   \opair{a}{b} &\hxsd{+}&     \opair{c}{d} &\eqd& \opair{a+c}{b+d}                   & \\
   \opair{a}{b} &\hxsd{\cdot}& \opair{c}{d} &\eqd& \opair{a c-b d}{a d + b c} &    \\
   i            &            &              &\eqd& \opair{0}{1}
  \end{array}$


\propbox{\begin{array}{rcl@{\qquad}C@{\qquad}D}
     i^2   &=& -1     &                     &
\end{array}}
\begin{proof}
\begin{align*}
  i^2
    &= \opair{0}{1}\opair{0}{1}
    %&& \text{by \prefp{def:i}}
  \\&= (0\cdot0 -1\cdot1, 0\cdot1+1\cdot0)
    %&& \text{by \prefp{thm:C+}}
  \\&= (-1, 0)
  \\&= -1
\end{align*}
\end{proof}

\thmbox{\begin{array}{rcl@{\qquad}C@{\qquad}D}
  \opair{a}{b} &=& a + ib & \forall \opair{a}{b}\in\C  & (\hie{rectangular coordinates})
\end{array}}
\begin{proof}
\begin{align*}
  \opair{a}{b}
    &= a(1,0) + b\opair{0}{1}
  \\&= a1 + bi
    %&& \text{by \prefp{def:i}}
  \\&= a + ib
\end{align*}
\end{proof}

%\cite{landau1966}


\qboxnpqt
  {
    Leonhard Euler (1707--1783), mathematician
    \index{Euler, Leonhard}
    \index{quotes!Euler, Leonhard}
    \footnotemark
  }
  %{../common/people/euler.jpg}
  {../common/people/euler1753bw_handmann_wkp_pdomain.jpg}
  {Weil nun alle m\"oglichen Zahlen, die man sich nur immer vorstellen mag,
    entweder gr\"o{\ss}er oder kleiner als 0, oder etwa 0 selbst sind, so ist klar, da{\ss}
    die Quadratwurzeln von Negativzahlen nicht einmal zu den m\"oglichen Zahlen
    gerechnet werden k\"onnen.
    Folglich m\"ussen wir sagen, da{\ss} dies unm\"ogliche Zahlen sind.
    Und dieser Umstand leitet uns auf den Begri{\ss} von solchen Zahlen,
    welche ihrer Natur nach unm\"oglich sind, und gew\"ohnlich imagin\"are
    oder eingebildete Zahlen genannt werden, weil sie blo\ss in der Einbildung
    vorhanden sind.}
  {And, since all numbers which it is possible to conceive,
    are either greater or less than 0, or are 0 itself,
    it is evident that we cannot rank the square root of a negative
    number amongst possible numbers, and we must therefore say that
    it is an \prop{impossible quantity}.
    In this manner we are led to the idea of numbers,
    which from their nature are impossible;
    and therefore they are usually called \hie{imaginary quantities},
    because they exist merely in the imagination.}
  \footnotetext{\begin{tabular}[t]{ll}
    quote: & \citorp{euler1770}{60} \\
    %www.math.uni-konstanz.de/~hoffmann/Funktionentheorie/kap1.pdf
    %www.springer.com/cda/content/document/cda_downloaddocument/9783540435549-c1.pdf?SGWID=0-0-45-130631-p2253211
    translation: & \citorp{euler1770e}{43} \\
    %image: & \url{http://en.wikipedia.org/wiki/Image:Leonhard_Euler.jpg}
    image: & \url{http://en.wikipedia.org/wiki/File:Leonhard_Euler.jpg}, public domain
  \end{tabular}}


\includegraphics[height=60mm]{graphics/complex.pdf}

$\begin{array}{rclM}
  z 
    &=& a + ib                       & \structe{rectangular coordinates}
  \\&=& A\cos\phi + iA\sin\phi
  \\&=& A\brp{\cos\phi + i\sin\phi}
  \\&=& Ae^{i\phi}                   & \text{by \thme{Euler's Identity}}
  \\&=& A \angle \phi                & \text{\structe{polar coordinates}}
\end{array}$

\newpage
%=======================================
\section*{Inner product spaces}
%=======================================
\defbox{\begin{array}{>{\qquad}F >{\ds}rc>{\ds}l C M D}
  \mc{7}{M}{%A \structe{functional}\ifsxref{functionals}{def:functional} 
  $\inprod{\vx}{\vy}$ is an \fnctd{inner product} if}
  \\
   1. & \inprod{\alpha\vx}{\vy}    &=& \alpha\inprod{\vx}{{\vy}}
      & %\forall \vx,\vy\in X,\;\forall\alpha\in\C
      & (\prope{homogeneous})
      & and
      \\
   2. & \inprod{\vx+\vy}{\vu} &=& \inprod{\vx}{{\vu}} + \inprod{\vy}{{\vu}}
      & %\forall \vx,\vy,\vu\in X
      & (\prope{additive})
      & and
      \\
   3. & \inprod{\vx    }{\vy} &=& \inprod{\vy}{\vx}^\ast
      & %\forall \vx,\vy\in X
      & (\prope{conjugate symmetric})
      & and
      \\
   4. & \inprod{\vx    }{\vx} &\ge& 0
      & %\forall \vx\in X
      & (\prope{non-negative})
      & and
      \\
   5. & \inprod{\vx    }{\vx} &=& 0 \iff \vx=\vzero
      & %\forall \vx\in X
      & (\prope{non-isotropic})
  %\\
  %\mc{7}{M}{An inner product is also called a \fnctd{scalar product}.}\\
  %\mc{7}{M}{The tuple $\inprodspaceX$ is called an \structd{inner product space}.}
  %A function $\inprodn:X\times X\to\C$ is a \hid{hermitian form} on $\spO$ if it satisfies conditions 1--3.
  %A function $\inprodn:X\times X\to\C$ is a \hid{pre-inner product} on $\spO$ if it satisfies conditions 1--4.
\end{array}}

\newpage
Examples of Inner-products

$\begin{array}{M>{\ds}rc>{\ds}lc>{\ds}l}
    \opd{Fourier Transform}: & \opFT   \fx(t) &\eqd& \int_{t\in\R} \fx(t) e^{-i\omega t} \dt &=& \inprod{\fx(t)}{e^{i\omega t}}_1
  \\\\\opd{Laplace Transform}: & \opL    \fx(t) &\eqd& \int_{t\in\R} \fx(t) e^{-s t}       \dt &=& \inprod{\fx(t)}{\brp{e^{-s t}}^\ast}_2
  \\\\\opd{DTFT}:              & \opDTFT \fx[n] &\eqd& \sum_{n\in\Z} \fx[n] e^{-i\omega n}     &=& \inprod{\fx[n]}{e^{i\omega n}}_3
  \\\\\opd{Z-Transform}:       & \opZ    \fx[n] &\eqd& \sum_{n\in\Z} \fx[n] z^{-n}             &=& \inprod{\fx[n]}{\brp{z^{-n}}^\ast}_4
  \\\\\opd{Expected value}:    & \pE     \rvx\rvy^\ast &\eqd& \sum\sum \rvx\rvy^\ast \psp(\rvx,\rvy) &=& \inprod{\vx}{\vy}_5
\end{array}$
\footnote{
  Expected value inner product: \citerpp{moon2000}{105}{106}
  }




\newpage
What can we do with the inner product?
\footnote{
  \citerpg{berberian1961}{27}{0821819127},
  \citerpg{haaser1991}{277}{0486665097}
  }

\thmbox{\begin{array}{Frcl@{\quad}Cl}
    1. & \inprod{\vx}{\vy+\vz}   &=& \inprod{\vx}{\vy} + \inprod{\vx}{\vz} & \forall \vx,\vy,\vz\in\setX   
  \\2. & \inprod{\vx}{\alpha\vy} &=& \alpha^\ast \inprod{\vx}{\vy}         & \forall \vx,\vy\in\setX,\; \alpha\in\F  
  \\3. & \inprod{\vx}{\vzero}    &=& \inprod{\vzero}{\vx} = 0              & \forall \vx\in\setX
  \\4. & \inprod{\vx-\vy}{\vz}   &=& \inprod{\vx}{\vz}-\inprod{\vy}{\vz}   & \forall \vx,\vy,\vz\in\setX   
  \\5. & \inprod{\vx}{\vy-\vz}   &=& \inprod{\vx}{\vy}-\inprod{\vx}{\vz}   & \forall \vx,\vy,\vz\in\setX   
  \\6. & \inprod{\vx}{\vz}       &=& \inprod{\vy}{\vz}                     & \forall \vz\in\setX\neq\setn{\vzero}  & \iff \vx=\vy
  \\7. & \inprod{\vx}{\vy}       &=& 0                                     & \forall \vx\in\setX                   &\iff \vy=\vzero
\end{array}}

\newpage
$\norm{\vx} \eqd \sqrt{\inprod{\vx}{\vx}}$

%--------------------------------------
\thmd{Cauchy-Schwarz Inequality}
\footnote{
  \citerpg{haaser1991}{278}{0486665097},
  \citerp{ab}{278},
  %\citerpp{pedersen2000}{34}{35},
  \citorp{cauchy1821}{455},
  \citorp{bunyakovsky}{6},
  \citor{schwarz}
  }
%--------------------------------------
\thmbox{\begin{array}{rclcl@{\qquad}C}
  \abs{\inprod{\vx}{\vy}}^2 &\le& \inprod{\vx}{\vx} \; \inprod{\vy}{\vy}
    &&
    %& \forall \vx,\vy\in\setX
    \\
  \abs{\inprod{\vx}{\vy}}^2 &=& \inprod{\vx}{\vx} \; \inprod{\vy}{\vy}
    &\iff& \exists \alpha\in\F \st \vy=\alpha\vx
    %& \forall \vx,\vy\in\setX
    \\
  \abs{\inprod{\vx}{\vy}}   &\le& \norm{\vx}\,\norm{\vy}
    &&
    %& \forall \vx,\vy\in\setX
    \\
  \abs{\inprod{\vx}{\vy}}   &=& \norm{\vx}\,\norm{\vy}
    &\iff& \exists \alpha\in\F \st \vy=\alpha\vx
    %& \forall \vx,\vy\in\setX
\end{array}
}

%--------------------------------------
\thmd{Minkowski's inequality}]
\footnote{
  \citerppgc{ab}{278}{279}{0120502577}{Theorem 32.3},
  \citer{maligranda1995},
  \citorp{minkowski1910}{115} 
  }
\label{thm:vsi_minkowski}
\label{thm:minkowineq}
\index{inequalities!Minkowski}
%--------------------------------------
%Let $\inprodspaceX$ be an \structe{inner product space}.
%Let $\normn$ be a function in $\clFfr$ such that $\norm{\vx}\eqd\sqrt{\inprod{\vx}{\vx}}$.\footnote{
%The function $\normn$ is a \structe{norm} \xrefP{thm:norm=inprod} and is called the \structe{normed induced by the inner product $\inprodn$}
%\xrefP{def:norm=inprod}.}
%$\norm{\vx} \eqd \sqrt{\inprod{\vx}{\vx}}$.
\thmbox{
  \norm{\vx+\vy} \le \norm{\vx}+\norm{\vy}
%  \quad\sst
%  \forall \vx,\vy\in\setX
  }

%--------------------------------------
\thmd{Polarization Identities}
\label{thm:polar_id}
\index{inner product!Polarization Identity}
\index{norm!Polarization Identity}
\footnote{
  \citerppgc{berberian1961}{29}{30}{0821819127}{Theorem~II.3.3},
  \citerpgc{istratescu1987}{110}{9027721823}{Proposition 4.1.5},
  \citerp{bollobas1999}{132},
  \citorp{jordan1935}{721} 
  %\cithrp{ab}{280} 
  }
%--------------------------------------
%Let $\spO\eqd\linearspaceX$ be a linear space,
%$\inprodn\in\clFxxf$ a function, and $\norm{\vx}\eqd\sqrt{\inprod{\vx}{\vx}}$.
\thmbox{%\begin{array}{M}
  %$\inprodspaceX$ is an inner product space  $\implies$
  %\\\quad
  4\inprod{\vx}{\vy} =
    \brb{\begin{array}{llC}
      \norm{\vx+ \vy}^2 -\norm{\vx- \vy}^2 +i\norm{\vx+i\vy}^2 -i\norm{\vx-i\vy}^2 & \text{for } \F=\C & \\%\forall \vx,\vy\in\setX \\
      \norm{\vx+ \vy}^2 -\norm{\vx- \vy}^2                                         & \text{for } \F=\R & %\forall \vx,\vy\in\setX 
    \end{array}}
    }

\newpage
%--------------------------------------
\thmd{Best Approximation Theorem}
\label{thm:bat}
\footnote{
  \citerpp{walter}{3}{4},
  \citerp{pedersen2000}{39},
  \citerpp{edwards1995}{94}{100},
  \citor{weyl1940}
  }
%--------------------------------------
%Let $\setxn{\vx_n\in\setX}$ be a set of vectors in an \structe{inner product space} \xrefP{def:inprod} $\inprodspaceX$ and with
%$\norm{\vx}\eqd\sqrt{\inprod{\vx}{\vx}}$\ifsxref{vsinprod}{def:norm=inprod}.

\thmbox{
  \brb{\begin{array}{N}
  $\setn{\vx_n}$ \large is\\
  \Large\prope{orthonormal}\\
  \xref{def:orthog}
  \end{array}}
  \implies
  \brb{\begin{array}{F>{\ds}l@{\quad}CD}
    1. & \arg\min_{\seq{\alpha_n}{n=1}^\xN} \norm{ \vx - \sum_{n=1}^\xN \alpha_{n} \vx_{n} }
         = \mcoml{\seq{\inprod{\vx}{\vx_n}}{n=1}^\xN}{best $\alpha_n=\inprod{\vx}{\vx_n}$}
       & %\forall\vx\in\setX
       & and
    \\[1ex]
    2. & \mcom{\ds\left(     \sum_{n=1}^\xN \inprod{\vx}{\vx_n}\vx_{n} \right)}{approximation}
         \orthog
         \mcom{\ds\left(\vx- \sum_{n=1}^\xN \inprod{\vx}{\vx_n}\vx_{n} \right)}{approximation error}
       & %\forall\vx\in\setX
       & 
  \end{array}}
  }



\newpage
%--------------------------------------
\section{Processing in another domain}
%--------------------------------------
Choose a basis, and project your signal onto it

\includegraphics[width=\tw-50mm]{graphics/baslat_cosh.pdf}

This is done using a transform. 

\newpage
One transform is the DTFT:


\[{\brb{\opDTFT \fx[n]}(\omega) \eqd \sum_{n\in\Z} \fx[n] e^{-i\omega n}}\]

\newpage

But the DTFT is a bit restrictive\ldots\\ like a horse forced to run around a track.

\includegraphics[height=130mm]{graphics/unitcircle.pdf}

\newpage

Let the horses run free\ldots\\ use the \opb{Z-Transform}

\[{\brb{\opZ    \fx[n]}(z) \eqd \sum_{n\in\Z} \fx[n] z^{-n}} \]

Note that \[ z=e^{i\omega} \]


\newpage


\includegraphics[height=150mm]{graphics/vecres.pdf}

\newpage
%--------------------------------------
\thmd{Geometric Series}
\footnote{
  \citerpc{hall1894}{39}{article 55}
  }
\label{thm:series_geometric}
%--------------------------------------
\thmbox{
  \sum_{k=0}^{n-1} r^k = \frac{1-r^n}{1-r}
  \qquad
  \forall r\in\C\setd0
  }
\\
\begin{proof}
\begin{align*}
  \sum_{k=0}^{n-1} r^k
    &= \brp{\frac{1}{1-r}}\brs{\brp{1-r}\sum_{k=0}^{n-1} r^k}
  \\&= \brp{\frac{1}{1-r}}\brs{\sum_{k=0}^{n-1} r^k - r\sum_{k=0}^{n-1} r^k}
  \\&= \brp{\frac{1}{1-r}}\brs{\sum_{k=0}^{n-1} r^k - \brp{\sum_{k=0}^{n-1} r^k -1 + r^n}}
  \\&= \brp{\frac{1}{1-r}}\brs{1 - r^n}
  \\&= \frac{1-r^n}{1-r}
\end{align*}
\end{proof}


\newpage
%---------------------------------------
\section{Daubechies Wavelets}
%---------------------------------------
\qboxnps
  {\href{http://en.wikipedia.org/wiki/G._H._Hardy}{G.H. Hardy}
   \href{http://www-history.mcs.st-andrews.ac.uk/Timelines/TimelineG.html}{(1877--1947)},
   \href{http://www-history.mcs.st-andrews.ac.uk/BirthplaceMaps/Places/UK.html}{English mathematician}
    \index{Hardy, G.H.}
    \index{quotes!Hardy, G.H.}
    \footnotemark
  }
  {../common/people/small/hardy.jpg}
  {The ``seriousness" of a mathematical theorem lies,
    not in its practical consequences,
    which are usually negligible,
    but in the {\em significance} of the mathematical ideas which it connects.
    We may say, roughly, that a mathematical idea is ``significant" if it can be
    connected, in a natural illuminating way,
    with a large complex of other mathematical ideas.}
  \citetblt{
    quote: & \citerc{hardy1940}{section 11} \\
    image: & \url{http://www-history.mcs.st-andrews.ac.uk/PictDisplay/Hardy.html}
    }
\newpage

\includegraphics[height=70mm]{graphics/D8_pz.pdf}

\begin{tabular}{cc}
  \includegraphics[height=70mm]{graphics/d8_phi_h.pdf}&%\includegraphics[height=70mm]{graphics/d8_psi_g.pdf}
\end{tabular}

\newpage
  \begin{tabular}{cc}%
    \includegraphics[height=70mm]{graphics/bspline_pz.pdf}&\includegraphics[height=70mm]{graphics/Dp_pz.pdf}\\%
    Cardinal B-spline&Daubechies-p%
  \end{tabular}%

\includegraphics[height=70mm]{graphics/n3_h.pdf}





\newpage
\section*{Minimum phase filters}
o  \includegraphics[height=50mm]{graphics/pz_minphase.pdf}%

The impulse response of a minimum phase filter has most of its energy concentrated
near the beginning of its support

%--------------------------------------
\thmd{Robinson's Energy Delay Theorem}
\footnote{
  \citerpg{dumitrescu2007}{36}{1402051247},
  \citor{robinson1962},  % referenced by claerbout1976
  \citorc{robinson1966}{???},  % referenced by online thesis
  \citerpp{claerbout1976}{52}{53}
  %\citerp{os}{291}\\
  %\citerp{mallat}{253}
  }
\label{thm:ztr_redp}
\index{minimum phase!energy}
\index{energy}
%--------------------------------------
\\
%Let $\fp(z)\eqd\sum_{n=0}^\xN a_n z^{-n}$ 
%and $\fq(z)\eqd\sum_{n=0}^\xN b_n z^{-n}$ 
%be polynomials.
\thmbox{
  \brb{\begin{array}{lMD}
    \fp & is \prope{minimum phase} & and\\
    \fq & is \emph{not} minimum phase & 
  \end{array}}
  \implies
  \mcom{\sum_{n=0}^{m-1} \abs{a_n}^2}{\parbox{20mm}{``energy" of the first $m$ coefficients of $\fp(z)$}} \ge 
  \mcom{\sum_{n=0}^{m-1} \abs{b_n}^2}{\parbox{20mm}{``energy" of the first $m$ coefficients of $\fq(z)$}} 
  %\qquad \forall 0\le m\le\xN
  }




\newpage

  \exboxt{\begin{tabular}{c|c}
    Daubechies-8 & Symlet-8
    \\\hline
    \includegraphics[height=50mm]{graphics/D8_pz.pdf}&\includegraphics[height=60mm]{graphics/S8_pz.pdf}\\
    \includegraphics[height=50mm]{graphics/d8_phi_h.pdf}&\includegraphics[height=50mm]{graphics/s8_phi_h.pdf}
    %\includegraphics{graphics/d8_psi_g.pdf}&\includegraphics{graphics/s8_psi_g.pdf}
  \end{tabular}}
