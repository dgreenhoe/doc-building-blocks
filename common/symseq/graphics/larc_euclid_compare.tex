%============================================================================
% Daniel J. Greenhoe
% LaTeX file
% Lagrange arc metric examples in R^2
%         y
%         |   o p        Let (rp,tp) be the polar location of point p.
%         |  /           where rp is the Euclidean distance from (0,0) to p 
%         | /            and tp is radian measure from the x-axis to p.
%         |/tp           Let (rq,tq) be the polar location of point q.
% --------|---------- x  The "Lagrange arc" r(theta) is defined here as
%         |\tq                          theta -tq        theta -tp
%         | \            r(theta) = rp ----------- + rq -----------
%         |  o q                          tp-tq            tq-tp
%         |              
%============================================================================
\begin{pspicture}(-1.5,-0.6)(1.5,1.5)%
  %\psset{plotstyle=line, linecolor=blue, linewidth=1pt, dotsize=3pt}%
  \psset{dotsize=3pt}%
  \psaxes[linecolor=axis]{<->}(0,0)(-1.5,-0.6)(1.5,1.5)%
  \pnode(1,0){p}%
  \pnode(-0.5,0){q1}%
  \pnode(-0.5,0.75){q2}%
  %\pnode(-0.6,0.25){q2}%
  %
  \psplot[polarplot=true,linecolor=blue]{180}{0}{1 x 180 sub 0 180 sub div mul 0.5 x 0 sub 180 0 sub div mul add}%
  \psplot[polarplot=true,linecolor=red]{123.690}{0}{1 x 123.690 sub 0 123.690 sub div mul 0.9014 x 0 sub 123.690 0 sub div mul add}%
  %\psplot[polarplot=true,linecolor=red]{157.380}{0}{1 x 157.380 sub 0 157.380 sub div mul 0.65 x 0 sub 157.380 0 sub div mul add}%
  \psdot[linecolor=purple](p)%
  \psdot[linecolor=blue](q1)%
  \psdot[linecolor=red](q2)%
  %
  \uput[45]{0}(p){$p$}%
  \uput[-45]{0}(q1){$q_1$}%
  \uput[135]{0}(q2){$q_2$}%
\end{pspicture}%
