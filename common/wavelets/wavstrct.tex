%============================================================================
% Daniel J. Greenhoe
% LaTeX File
%============================================================================
%======================================
\chapter{Wavelet Structures}
%======================================
\qboxnpqt
  { Jules Henri Poincar\'e (1854-1912), physicist and mathematician
    \index{Poincar\'e, Jules Henri}
    \index{quotes!Poincar\'e, Jules Henri}
    \footnotemark
  }
  {../common/people/small/poincare.jpg}
  {\ldots on fait la science avec des faits comme une maison avec des pierres ; 
   mais une accumulation de faits n'est pas plus une science qu'un tas de 
   pierres n'est une maison.}
  {Science is built up of facts, as a house is built of stones;
   but an accumulation of facts is no more a science than a heap of stones is a house.}
  \citetblt{
    quote:       & \citerc{poincare_sah}{Chapter IX, paragraph 7} \\
    translation: & \citerp{poincare_sah_eng}{141} \\
    image:       & \url{http://www-groups.dcs.st-and.ac.uk/~history/PictDisplay/Poincare.html}
    }

\qboxnps
  {
    Freeman Dyson (1923--), physicist and mathematician  %(January 1994)
    \index{Dyson, Freeman}
    \index{quotes!Dyson, Freeman}
    \footnotemark
  }
  %{../common/people/dyson_.flickr8168451.jpg}
  {../common/people/dyson_isepp-org_95-96.jpg}  %http://www.isepp.org/Media/Speaker%20Images/95-96%20Images/dyson.jpg
  %{../common/people/small/dyson.jpg}
  {The bottom line for mathematicians is that the architecture has to be right.
    In all the mathematics that I did, the essential point was to find
    the right architecture.
    It's like building a bridge.
    Once the main lines of the structure are right,
    then the details miraculously fit.
    The problem is the overall design.}
  \citetblt{
    quote: & \citerp{dyson1994}{20}  \\
    %image: & \url{http://www.flickr.com/photos/russnelson/8168451/}
    image: & \scs\url{http://www.isepp.org/Media/Speaker\%20Images/95-96\%20Images/dyson.jpg}
    }

%=======================================
\section{Introduction}
%=======================================
%=======================================
\subsection{What are wavelets?}
%=======================================
In Fourier analysis, \prope{continuous} {dilations} \xref{def:opD} of the \fncte{complex exponential} \xref{def:exp}
form a  \structe{basis} \xref{def:basis_schauder} for the \structe{space of square integrable functions} $\spLLR$ \xref{def:spLLR} 
such that
  \\\indentx$\ds\spLLR=\linspan\set{\opDil_\omega e^{ix}}{\scy\omega\in\R}$.

In Fourier series analysis \xref{thm:opFSi}, \prope{discrete} dilations of the complex exponential 
form a  basis for $\spLL{\intoo{0}{2\pi}}$ such that
  \\\indentx$\ds\spLL{\intoo{0}{2\pi}}=\linspan\setjZ{\opDil_j e^{ix}}$.

In Wavelet analysis, for some \fncte{mother wavelet} \xref{def:wavelet} $\fpsi(x)$,
  \\\indentx$\ds\spLLR=\linspan\set{\opDil_\omega\opTrn_\tau \fpsi(x)}{\omega,\tau\in\R}$.

However, the ranges of parameters $\omega$ and $\tau$ can be much reduced to the countable set $\Z$ resulting in
a \prope{dyadic} wavelet basis such that for some mother wavelet $\fpsi(x)$,
  \\\indentx$\ds\spLLR=\linspan\set{\opDil^j\opTrn^n \fpsi(x)}{j,n\in\Z}$.\\
This text deals almost exclusively with dyadic wavelets. 
Wavelets that are both \prope{dyadic} and \prope{compactly supported} have the attractive feature 
that they can be easily implemented in hardware or software by use of the 
\structe{Fast Wavelet Transform} \xref{fig:fwt}.

%=======================================
\subsection{Analyses}
%=======================================
    \begin{minipage}{\tw-65mm}%
      An analysis can be partially characterized by its order structure with respect
      to an order relation such as the set inclusion relation $\subseteq$.
      Most transforms have a very simple M-$n$ order structure,
      as illustrated to the right.
      The M-$n$ lattices for $n\ge3$ are \prope{modular} but not \prope{distributive}.
      Analyses typically have one subspace that is a \hie{scaling} subspace;
      and this subspace is often simply a family of constants
      (as is the case with \hi{Fourier Analysis}).
    \end{minipage}%
    \hfill\tbox{\includegraphics{../common/math/graphics/pdfs/latmn.pdf}}%

    \begin{minipage}{\tw-55mm}%
      A special characteristic of wavelet analysis is that there is not just one
      scaling subspace,
      but an entire sequence of scaling subspaces.
      These scaling subspaces are \prope{linearly ordered} with respect to the
      ordering relation $\subseteq$. In wavelet theory, this structure is called a \structe{multiresolution analysis},
      or \structe{MRA} \xref{def:mra}.

     The MRA was introduced by St{/'e}phane G. Mallat in 1989.
     The concept of a scaling space was perhaps first introduced by Taizo Iijima in 1959 in Japan,
    and later as the \structe{Gaussian Pyramid} by Burt and Adelson in the 1980s in the West.\footnotemark
    \end{minipage}%
    \footnotetext{%
      \citorp{mallat89}{70},
      \citor{iijima1959},
      \citor{burt1983},
      \citor{adelson1981},
      \citer{lindeberg1993},
      \citer{alvertez1993},
      \citer{guichard2012},
      \citerc{weickert1999}{historical survey}
      }
    \hfill\tbox{\includegraphics{../common/math/graphics/pdfs/latmra.pdf}}%

    \begin{minipage}{\tw-65mm}%
      A second special characteristic of wavelet analysis is that it's order structure
      with respect to the $\subseteq$ relation is not a simple M-$n$ lattice 
     (as is with the case of Fourier and other analyses).
      Rather, it is a lattice of the form illustrated to the right.
      This lattice is \prope{non-complemented}, \prope{non-distributive},
      \prope{non-modular}, and \prope{non-Boolean} \xref{prop:order_wavstrct}.
    \end{minipage}%
    \hfill\tbox{\includegraphics{../common/math/graphics/pdfs/latwav.pdf}}%

    \begin{minipage}{\tw-65mm}%
      The wavelet subspace structure is similar in form to that of the \structe{Primorial numbers},\footnotemark
      illustrated to the right by a \hie{Hasse diagram}.
      %In the world of mathematical structures,
      %there is circumstantial evidence that the order structure of wavelet analyses is quite rare,
      %if not outright unique.
      %For example, suppose we replace the wavelet subspaces with prime numbers
      %and the scaling subspaces with their products as illustrated to the right.
      %The resulting sequence $\seqn{1,\,2,\,6,\,30,\,210}$ as of 2011 July 30
      %has no matches in Neil J.A. Sloane's  \emph{Online Encyclopedia of Integer Sequences}
      %(hosted by \emph{AT\&T Research}).\footnotemark
    \end{minipage}%
    \citetblt{%
      \citeoeis{A002110}%
      }%
    \hfill\tbox{\includegraphics{../common/math/graphics/pdfs/latp_1235711.pdf}}%

  An analysis can be represented using three different structures:
    %\paragraph{Equivalence of lattice representations.}
    %So far we have discussed representing a wavelet analysis using three different structures:
\\\begin{tabular}{@{\qquad}ll}
  \circOne    & sequence of subspaces             \\
  \circTwo    & sequence of basis coefficients         \\
  \circThree  & sequence of basis vectors
\end{tabular}\\
These structures are isomorphic to each other, and can therefore be used interchangeably.
%(see \prefp{thm:VPb_isomorphic}).
%(see \prefp{fig:wav_VPb_isomorphic}).
%That is, a ``\hie{wavelet analysis}" can be described using any of these structures.
%However, sometimes when introducing theorems about wavelets,
%it is convenient to use elements from not just one, but from multiple lattices;
%and so it is convenient to have a ``collection" of wavelet analysis elements
%all assembled together into one formally defined tuple.
%\pref{def:wavsys} (next) does just that---it defines a \hie{wavelet analysis} in terms of a tuple with elements
%extracted from the four wavelet structures.

%---------------------------------------
% isomorphic lattices
%---------------------------------------
\mbox{}\hfill
  \tbox{\includegraphics{../common/math/graphics/pdfs/latp_VW.pdf}}\qquad{\Large$\thapprox$}\qquad
  \tbox{\includegraphics{../common/math/graphics/pdfs/latp_hg.pdf}}\qquad{\Large$\thapprox$}\qquad
  \tbox{\includegraphics{../common/math/graphics/pdfs/latp_pp.pdf}}
\hfill\mbox{}

\prefpp{fig:analyses} illustrate the order structures of some analyses,
        including two wavelet analyses:
\begin{figure}[th]
  \centering%
  \begin{tabular}{|c|c|}%
    \hline%
    \mc{1}{B}{Cosine analysis  (even Fourier series)}&\mc{1}{B}{Cosine polynomial analysis}%
    \\\includegraphics{../common/math/graphics/pdfs/baslat_cosh.pdf}&\includegraphics{../common/math/graphics/pdfs/baslat_cose.pdf}%
    \\\hline%
    \mc{1}{|B|}{Chebyshev polynomial analysis\cittrp{rivlin1974}{4}}&\mc{1}{|B|}{Hadamard-3 analysis}%
    \\\includegraphics{../common/math/graphics/pdfs/baslat_cheby.pdf}&\includegraphics{../common/math/graphics/pdfs/baslat_h3.pdf}%
    \\\hline
    \mc{1}{|B|}{Haar/Daubechies-$p1$ wavelet analysis} & \mc{1}{B|}{Daubechies-$p2$ wavelet analysis}%
    \\\includegraphics{../common/math/graphics/pdfs/baslat_d1.pdf}&\includegraphics{../common/math/graphics/pdfs/baslat_d2.pdf}%
    \\\hline%
  \end{tabular}%
  \caption{examples of the order structures of some analyses\label{fig:analyses}}
\end{figure}

%=======================================
%\section{Wavelet analysis}
%=======================================
%=======================================
\section{Definition}
%=======================================
The term ``wavelet" comes from the French word ``\hie{ondelette}", meaning ``small wave". 
And in essence, wavelets are ``small waves" (as opposed to the ``long waves" of Fourier analysis) 
that form a basis for the Hilbert space $\spLLR$.\citetbl{
  \citerpg{strang1996}{ix}{0961408871},
  \citerpg{atkinson2009}{191}{1441904581}
  }
%---------------------------------------
\begin{definition}
\citetbl{
  \citerpgc{wojtaszczyk1997}{17}{0521578949}{Definition 2.1}
  }
\label{def:wavelet}
\label{def:seqWn}
\label{def:wavstrct_psi}
%---------------------------------------
%Let $\MRAspaceLLRV$ be an \structe{multiresolution space} \xref{def:mra}.
Let $\opTrn$ and $\opDil$ be as defined in \prefp{def:opT}.
\defboxp{
  A function $\fpsi(x)$ in $\spLLR$ is a \fnctd{wavelet function} for $\spLLR$ if
  \\\indentx$\set{\opDil^j\opTrn^n\fpsi}{\scy j,n\in\Z}$ is a \structe{Riesz basis} for $\spLLR$.\\
  In this case, $\fpsi$ is also called the \fnctd{mother wavelet} of the basis $\set{\opDil^j\opTrn^n\fpsi}{\scy j,n\in\Z}$.
  The sequence of subspaces $\seqjZ{\spW_j}$ is the \structd{wavelet analysis} induced by $\fpsi$, 
  where each subspace $\spW_j$ is defined as
  \\\indentx$\spW_j\eqd\linspan\setnZ{\opDil^j\opTrn^n\fpsi}$ .
  }
\end{definition}

%---------------------------------------
%\begin{definition}
%---------------------------------------
%Let $\MRAspaceLLRV$ be an \structe{multiresolution space} \xref{def:mra}.
%Let the operation $\adds$ represent \hie{Minkowski addition} on $\spLLR$\ifsxref{subspace}{def:sub_add}.%
%\defboxt{
%  %The \hid{wavelet subspace} $\spW_j$ is the orthogonal complement of $\spV_j$ in $\spV_{j+1}$ such that
%  %  \\\indentx$\spV_j \adds \spW_j = \spV_{j+1}$
%  %  \\
%  The sequence $\seqjZ{\spW_j}$ is a \hid{wavelet analysis} on $\MRAspaceLLRV$ if
%  \\\indentx$\begin{array}{FMCD}
%    1. & $\spV_{j+1} = \mcom{\spV_j \adds \spW_j}{\hi{Minkowski addition}}$
%       & \forall \spW_j \in \seqxZ{\spW_j}\text{ and }\spV_j\in\seqjZ{\spV_j}
%       & and
%     %\qquad\text{\scriptsize ($\spW_j$ is the complement of $\spV_j$ in $\spV_{j+1}$)}
%    \\
%    2. & \mc{2}{M}{There exists $\fpsi\in\spLLR$ such that $\seqxZ{\opTrn^n\fpsi}$ is a \structe{Riesz basis} for $\spW_0$.}
%  \end{array}$
%  }
%\end{definition}

%%---------------------------------------
%\begin{proposition}[complemented subspaces]
%%---------------------------------------
%Let $\MRAspaceLLRV$ be an \structe{MRA space}.
%Let $\spV_j^\orthog$ be the \structe{orthogonal complement} \xref{def:sub_ocomp} of $\spV_j$.
%\propbox{
%  \spV_j^{\orthog\orthog}=\spV_j \qquad\scy\forall n\in\Z \qquad\scs\text{\prope{involutory}}
%  }
%\end{proposition}
%\begin{proof}
%\begin{enume}
%  \item By \pref{def:mra}, $\spV_j$ is \prope{closed} in $\spLLR$ ($\spV_j=\cls\spV_j$).
%  \item By \prefpp{thm:inprod_orthog}, $\spV_j^{\orthog\orthog}=\spV_j$.
%\end{enume}
%\end{proof}

A \structe{wavelet analysis} $\seqn{\spW_j}$ is often constructed from a \structe{multiresolution anaysis} \xref{def:mra}
$\seqn{\spV_j}$ under the relationship
\\\indentx$\ds\spV_{j+1} = \spV_j \adds \spW_j$,\qquad where $\adds$ is subspace addition (\ope{Minkowski addition}).\\
By this relationship alone, $\seqn{\spW_j}$ is in no way uniquely defined 
in terms of a multiresolution analysis $\seqn{\spV_j}$.
In general there are many possible complements of a subspace $\spV_j$.
To uniquely define such a wavelet subspace, one or more additional constraints are required.
One of the most common additional constraints is \hie{orthogonality}, such that
$\spV_j$ and $\spW_j$ are orthogonal to each other\ifdochas{ortho}{ (see \prefp{chp:ortho})}.




%=======================================
\section{Dilation equation}
%=======================================
Suppose $\seqxZ{\opTrn^n\fpsi}$ is a basis for $\spW_0$.
By \prefp{def:seqWn}, the wavelet subspace $\spW_0$ is contained in the 
scaling subspace $\spV_1$.
By \prefp{def:mra}, the sequence $\seqxZ{\opDil\opTrn^n\fphi}$ is a basis for $\spV_1$.
Because $\spW_0$ is contained in $\spV_1$,
the sequence $\seqxZ{\opDil\opTrn^n\fphi}$ is also a basis for $\spW_0$.

%--------------------------------------
\begin{theorem}[\thmd{wavelet dilation equation}]
\label{thm:g->psi}
%--------------------------------------
Let $\mrasys$ be a \structe{multiresolution system} \xref{def:mrasys}
and $\seqjZ{\spW_j}$ be a \structe{wavelet analysis} \xref{def:seqWn} 
with respect to\\
$\mrasys$ and with \fncte{wavelet function} $\fpsi$ \xref{def:wavelet}.
\thmbox{
  \begin{array}{l rc>{\ds}l @{\qquad}D}
    \exists \seqxZ{g_n} \st
      & \fpsi &=& \sum_{n\in\Z}  g_n \opDil \opTrn^n \fphi
      & 
  \end{array}}
\end{theorem}
\begin{proof}
\begin{align*}
  \fpsi &\in \spW_0
        && \text{by \prefp{def:seqWn}}
      \\&\subseteq \spV_1
        && \text{by \prefp{def:seqWn}}
      \\&= \Span\seqxZ{\opDil\opTrn^n\fphi(x)}
        && \text{by \prefp{def:mra} (MRA)}
      \\&\implies 
         \exists \seqxZ{g_n} \st
            \fpsi = \sum_{n\in\Z}  g_n \opDil \opTrn^n \fphi
\end{align*}

%\item Proof that $\fpsi(x) = \sqrt{2} \sum_{n\in\Z}  g_n  \fphi(2x-n)$:
%\begin{align*}
%              &  \set{\fphi(2x-n)}{n\in\Z} \text{ spans } \spV_1
%              && \text{by (1)}
%  \\
%              &  \set{\fpsi(x-n)}{n\in\Z} \text{ spans } \spW_0
%  \\ \implies & \fpsi(x)\in \spW_0 \subset \spV_1
%  \\ \implies & \text{$\fpsi(x)$ can be represented as a linear combination of $\fphi(2x-n)$}.
%\end{align*}
%\end{enumerate}
\end{proof}

A \structe{wavelet system} (next definition) consists of two subspace sequences: 
\begin{liste}
  \item A \structb{multiresolution analysis} $\seqn{\spV_j}$ \xref{def:mra}
     provides ``coarse" approximations of a function in $\spLLR$ at different ``scales" or resolutions.
  \item A \structb{wavelet analysis} $\seqn{\spW_j}$
     provides the ``detail" of the function missing from the approximation provided by a given scaling subspace
     \xref{def:seqWn}.
\end{liste}

%--------------------------------------
\begin{definition}
\label{def:wavsys}
\label{def:gn}
%--------------------------------------
Let $\mrasys$ be a \structe{multiresolution system} \xref{def:mra}
and $\seqjZ{\spW_j}$ a wavelet analysis \xref{def:seqWn}
with respect to $\seqjZ{\spV_j}$.
Let $\seqxZ{g_n}$ be a sequence of coefficients.
\defbox{\begin{array}{M}
  A \structd{wavelet system} is the tuple \quad$\ds\wavsys$\\
  and the sequence $\seqxZ{g_n}$ that satisfies the equation
  $\ds\fpsi = \sum_{n\in\Z}  g_n \opDil \opTrn^n \fphi$\\
  is the \structd{wavelet coefficient sequence}.
\end{array}}
\end{definition}

%--------------------------------------
\begin{remark}
%--------------------------------------
The pair of coefficient sequences $\opair{\seqn{h_n}}{\seqn{g_n}}$ generates 
the scaling function $\fphi(x)$ \xref{def:wavstrct_phi} 
and the wavelet function $\fpsi(x)$ \xref{def:wavstrct_psi}.
These functions in turn generate 
the multiresolution analysis $\seqn{\spV_j}$ \xref{def:seqVn}
and the wavelet analysis $\seqn{\spW_j}$ \xref{def:seqWn}.
Therefore, the coefficient sequence pair $\opair{\seqn{h_n}}{\seqn{g_n}}$ 
totally defines a wavelet system\\$\wavsys$ \xref{def:wavsys}.

Furthermore, especially in the case of orthonormal wavelets, the wavelet coefficient
sequence $\seqxZ{g_n}$ is often defined in terms of the 
scaling coefficient sequence $\seqxZ{h_n}$
in a very simple and straightforward manner.
Therefore, in the case of an orthonormal wavelet system, the coefficient
scaling sequence $\seqxZ{h_n}$ often totally defines the entire wavelet system.
And in this case, designing a wavelet system is only a matter of finding a handful of
scaling coefficients $\seqn{h_1,\,h_2,\,\ldots,\,h_n}$\ldots because once you have these,
you can generate everything else.
\end{remark}


%%---------------------------------------
%\begin{definition}
%\label{def:wav_lat_coef}
%%---------------------------------------
%Let $\seqxZ{h_n}$ be a sequence of scaling coefficients and
%    $\seqxZ{g_n}$ be the associated sequence of wavelet coefficients.
%%    $\subseteq$ the set inclusion relation,
%%    $\setu$ the set union operation, and
%%    $\seti$ the set intersection operation.
%\defbox{\begin{array}{l}
%  \text{The tupple }
%  \qquad \wavlatcoef \\
%  \text{is called the \hid{lattice of wavelet bases coefficients}.}
%\end{array}}
%\end{definition}



%\begin{figure}[t]
%\setlength{\unitlength}{8mm}
%\begin{center}
%\begin{tabular}{cc}
%   \includegraphics*[width=6\tw/16, height=6\tw/16, clip=true]{../common/wavelets/haar_sj.eps} &
%   \includegraphics*[width=6\tw/16, height=6\tw/16, clip=true]{../common/wavelets/haar_sk.eps} \\
%   \footnotesize varying dilation, constant translation   &
%   \footnotesize varying translation, constant dilation
%\end{tabular}
%\caption{
%   \label{fig:haar-jn}
%   Haar scaling function at varying dilations and translations.
%   }
%\end{center}
%\end{figure}



%=======================================
\section{Order structure}
%=======================================
%The axioms of \prefp{def:mra} generate a subspace architecture.
%These transform representation sequences may be \hie{ordered} with \hie{ordering relations}
%as follows:%
%\footnote{\hie{projection operator ordering}: \prefp{def:operator_lattice}}

\begin{minipage}{\tw-70mm}\raggedright
The \structe{wavelet system} $\wavsys$ \xref{def:wavsys} together with the 
set inclusion relation $\subseteq$ 
forms an \structe{ordered set}\ifsxref{order}{def:poset}, 
illustrated to the right by a \hie{Hasse diagram}\ifsxref{order}{def:hasse}.%\ifdochas{order}{\footnotemark}
%Under these three order relations, wavelet system generate three isomorphic lattices such 
%as are illustrated in \prefp{fig:wav_VPb_isomorphic} 
%and in the figure to the right.
\end{minipage}%
\hfill\tbox{\includegraphics{../common/math/graphics/pdfs/latwav.pdf}}%

%---------------------------------------
\begin{proposition}
\label{prop:order_wavstrct}
%---------------------------------------
Let $\wavsys$ be a wavelet system with order relation $\subseteq$.
The lattice $\latL\eqd\lattice{\seqn{\spV_j},\seqn{\spW_j}}{\subseteq}{\join}{\meet}$ has 
the following properties:
\propbox{\begin{array}{FM}
      1.  & $\latL$ is \prope{nondistributive}.
    \\2.  & $\latL$ is \prope{nonmodular}.
    %\cntn & $\latL$ is \prope{complemented}.
    %\cntn & $\latL$ is \prope{not uniquely complemented}.
    %\cntn & $\latL$ is \prope{nonorthocomplemented}.
    \\3.  & $\latL$ is \prope{noncomplemented}.
    \\4.  & $\latL$ is \prope{nonBoolean}.
\end{array}}
\end{proposition}
\begin{proof}
%\mbox{}\hspace{20mm}
%  \latmatlw{4}{0.5}
%    {
%           &       & \null                 \\  
%           & \null                         \\  
%     \null &       & \null &       & \null \\  
%           &       & \null                   
%    }
%    {\ncline{1,3}{2,2}\ncline{2,2}{3,1}
%     \ncline{1,3}{3,5}
%     \ncline{2,2}{3,3}
%     \ncline{4,3}{3,1}\ncline{4,3}{3,3}\ncline{4,3}{3,5}
%    }
%    {\nput{ 90}{1,3}{$1$}
%     \nput{135}{2,2}{$v$}
%     \nput{0}{3,1}{$x$}
%     \nput{ 67}{3,3}{$y$}
%     \nput{  0}{3,5}{$z$}
%     \nput{-90}{4,3}{$0$}
%    }
\begin{enumerate}
  \item Proof that $\latL$ is \prope{nondistributive}: \label{item:wavprop_nondistrib}
    \begin{enumerate}
      \item $\latL$ contains the $N5$ lattice\ifsxref{latm}{def:lat_N5}.
      \item Because $\latL$ contains the $N5$ lattice, $\latL$ is \prope{nondistributive}\ifdochas{latm}{ by \prefp{thm:latd_char_n5m3}}.
    \end{enumerate}

  \item Proof that $\latL$ is \prope{nonmodular} and \prope{nondistributive}: 
    \begin{enumerate}
      \item $\latL$ contains the $N5$ lattice\ifsxref{latm}{def:lat_N5}.
      \item Because $\latL$ contains the $N5$ lattice, $\latL$ is \prope{nonmodular}\ifdochas{latm}{ by \prefp{thm:lat_mod_iff_N5}}.
    \end{enumerate}

  \item Proof that $\latL$ is \prope{noncomplemented}:
    \begin{minipage}{50mm}
      \begin{align*}
          x' &= y' = v' = z
        \\z' &= \setn{x,y,v}
        \\x''&= \brp{x'}'
           \\&= z'
           \\&= \setn{x,y,v}
           \\&\ne  x
      \end{align*}
    \end{minipage}\qquad\tbox{\includegraphics{../common/math/graphics/pdfs/lat6_plat_xyzv10.pdf}}%

  %\item Proof that $\latL$ is \prope{not uniquely complemented}:\\
  %   For example, subspace $\spW_2$ in \prefp{fig:wav_VPb_isomorphic} is complemented
  %   by $\spV_1$, $\spV_2$, and $\spW_1$.
  %\item Proof that $\latL$ is \prope{orthomodular}:
  %  \begin{enumerate}
  %    \item $\latL$ does \emph{not} contain the $O_6$ lattice\ifdochas{ortholat}{ \xref{def:latoc_omod}}.
  %    \item Because $\latL$ does not contain the $O_6$ lattice, $\latL$ is \prope{orthomodular}\ifdochas{ortholat}{ by \prefp{thm:latoc_omod}}.
  %  \end{enumerate}

  \item Proof that $\latL$ is \prope{nonBoolean}:
    \begin{enumerate}
      \item $\latL$ is \prope{nondistributive} (\pref{item:wavprop_nondistrib}).
      \item Because $\latL$ is \prope{nondistributive}, it is \prope{nonBoolean}\ifdochas{boolean}{ by \prefp{def:booalg}}.
    \end{enumerate}
\end{enumerate}
\end{proof}




%=======================================
\section{Subspace algebraic structure}
%=======================================
%--------------------------------------
\begin{theorem}
\label{thm:mra_subalg}
%--------------------------------------
Let $\wavsys$ be a \structe{wavelet system} \xref{def:wavsys}.
Let $\spV_1 \adds \spV_2$ represent \fncte{Minkowski addition} of two subspaces $\spV_1$ and $\spV_2$ of a Hilbert space $\spH$.
\thmbox{\begin{array}{rc>{\ds}l D}
    \spLLR &=& \lim_{j\to\infty}\spV_j                 
             & ($\spLLR$ is equivalent to one very large scaling subspace)\\
           &=& \spV_j \adds \spW_j \adds \spW_{j+1} \adds \spW_{j+2} \adds\, \cdots 
             & $\brp{\begin{array}{D}$\spLLR$ is equivalent to one scaling space\\
                                       and a sequence of wavelet subspaces\end{array}} $\\
           &=& \cdots\,\adds \spW_{-2} \adds \spW_{-1} \adds \spW_0 \adds \spW_1 \adds \spW_2 \adds\,\cdots        
             & ($\spLLR$ is equivalent to a sequence of wavelet subspaces)
  \end{array}}
\end{theorem}
\begin{proof}
\begin{enumerate}
  \item Proof for (1):
    \begin{align*}
      \spLLR 
        &= \lim_{j\to\infty}\spV_j                 
        && \text{by \prefp{def:mra}}
    \end{align*}

  \item Proof for (2):
    \begin{align*}
      \mcom{\spV_j \adds \spW_j}{$\spV_{j+1}$} \adds \spW_{j+1} \adds \spW_{j+2} \adds \cdots
        &= \mcom{\spV_{j+1} \adds \spW_{j+1}}{$\spV_{j+2}$} \adds \spW_{j+2} \adds \spW_{j+3} \adds \cdots
      \\&= \mcom{\spV_{j+2} \adds \spW_{j+2}}{$\spV_{j+3}$} \adds \spW_{j+3} \adds \spW_{j+4} \adds \cdots
      \\&= \mcom{\spV_{j+3} \adds \spW_{j+3}}{$\spV_{j+4}$} \adds \spW_{j+4} \adds \spW_{j+5} \adds \cdots
      \\&= \mcom{\spV_{j+5} \adds \spW_{j+5}}{$\spV_{j+5}$} \adds \spW_{j+6} \adds \spW_{j+6} \adds \cdots
      \\&= \lim_{j\to\infty}\spV_{j+5} \adds \spW_{j+5} \adds \spW_{j+6} \adds \spW_{j+6} \adds \cdots
      \\&= \spLLR
    \end{align*}

  \item Proof for (3):
    \begin{align*}
      \spLLR &= \mcom{\spV_0}{$\spV_{-1}\adds\spW_{-1}$} \adds \spW_0 \adds \spW_1 \adds \spW_2 \adds \spW_3 \adds \cdots
             && \text{by (2)}
           \\&= \mcom{\spV_{-1}}{$\spV_{-2}\adds\spW_{-2}$} \spW_{-1} \adds \spW_0 \adds \spW_1 \adds \spW_2 \adds \spW_3 \adds \cdots
           \\&= \mcom{\spV_{-2}}{$\spV_{-3}\adds\spW_{-3}$} \spW_{-2} \adds \spW_{-1} \adds \spW_0 \adds \spW_1 \adds \spW_2 \adds \spW_3 \adds \cdots
           \\&= \mcom{\spV_{-3}}{$\spV_{-4}\adds\spW_{-4}$} \spW_{-3} \adds \spW_{-2} \adds \spW_{-1} \adds \spW_0 \adds \spW_1 \adds \spW_2 \adds \spW_3 \adds \cdots
           \\&\vdots
           \\&= \cdots \adds \spW_{-3} \adds \spW_{-2} \adds \spW_{-1} \adds \spW_0 \adds \spW_1 \adds \spW_2 \adds \spW_3 \adds \cdots
    \end{align*}
\end{enumerate}
\end{proof}

%--------------------------------------
\begin{remark}
%--------------------------------------
In the special case that two subspaces $\spW_1$ and $\spW_2$ are \prope{orthogonal} to each other, then 
the \fncte{subspace addition} operation $\spW_1\adds\spW_2$ is frequently expressed as
$\spW_1\oplus\spW_2$.
In the case of an \structe{orthonormal wavelet system}\ifsxref{ortho}{def:ows}, 
the expressions in \prefpp{thm:mra_subalg} could be expressed as
\\\indentx$\begin{array}{rc>{\ds}l}
    \spLLR &=& \lim_{j\to\infty}\spV_j                 \\
           &=& \spV_j \oplus \spW_j \oplus \spW_{j+1} \oplus \spW_{j+2} \oplus\, \cdots \\
           &=& \cdots\,\oplus \spW_{-2} \oplus \spW_{-1} \oplus \spW_0 \oplus \spW_1 \oplus \spW_2 \oplus\,\cdots .       
  \end{array}$
\end{remark}.


%=======================================
\section{Necessary conditions}
%=======================================

%--------------------------------------
\begin{theorem}[\thmd{quadrature condition}s in ``time"]
\label{thm:wavsys_quadcon}
%--------------------------------------
Let $\wavsys$ be a wavelet system \xref{def:wavsys}.
\thmbox{\begin{array}{F>{\ds}rc>{\ds}lC}
  1. & \sum_{m\in\Z} h_m \sum_{k\in\Z} h_k^\ast \inprod{\fphi}{\opTrn^{2n-m+k} \fphi} &=& \inprod{\fphi}{\opTrn^n \fphi} & \forall n\in\Z\\
  2. & \sum_{m\in\Z} g_m \sum_{k\in\Z} g_k^\ast \inprod{\fphi}{\opTrn^{2n-m+k} \fphi} &=& \inprod{\fpsi}{\opTrn^n \fpsi} & \forall n\in\Z\\
  3. & \sum_{m\in\Z} h_m \sum_{k\in\Z} g_k^\ast \inprod{\fphi}{\opTrn^{2n-m+k} \fphi} &=& \inprod{\fphi}{\opTrn^n \fpsi} & \forall n\in\Z
\end{array}}
\end{theorem}
\begin{proof}
\begin{enumerate}
  \item Proof for (1): by \prefp{thm:wav_quadcon}.
  \item Proof for (2): 
    \begin{align*}
      \inprod{\fpsi}{\opTrn^n \fpsi}
        &= \inprod{\sum_{m\in\Z} g_m \opDil \opTrn^m \fphi }{\opTrn^n \sum_{k\in\Z} g_k \opDil \opTrn^k \fphi}
        && \text{by \thme{wavelet dilation equation}}
        && \text{\xref{thm:g->psi}}
      \\&= \sum_{m\in\Z} g_m \sum_{k\in\Z} g_k^\ast \inprod{\opDil \opTrn^m \fphi }{\opTrn^n \opDil \opTrn^k \fphi}
        && \text{by properties of $\inprodn$}
        && \text{\xref{def:inprod}}
      \\&= \sum_{m\in\Z} g_m \sum_{k\in\Z} g_k^\ast \inprod{\fphi }{\left(\opDil \opTrn^m \right)^\ast \opTrn^n \opDil \opTrn^k \fphi}
        && \text{by def. of operator adjoint}
        && \text{\ifxref{operator}{prop:op_adjoint}}
      \\&= \sum_{m\in\Z} g_m \sum_{k\in\Z} g_k^\ast \inprod{\fphi }{\left(\opDil \opTrn^m \right)^\ast \opDil \opTrn^{2n} \opTrn^k \fphi}
        && \text{by \prefp{prop:DTTD}}
      \\&= \sum_{m\in\Z} g_m \sum_{k\in\Z} g_k^\ast \inprod{\fphi }{\opTrna^m \opDila \opDil \opTrn^{2n} \opTrn^k \fphi}
        && \text{by operator star-algebra prop.}
        && \text{\ifxref{operator}{thm:op_star}}
      \\&= \sum_{m\in\Z} g_m \sum_{k\in\Z} g_k^\ast \inprod{\fphi }{\opTrn^{-m} \opDil^{-1} \opDil \opTrn^{2n} \opTrn^k \fphi}
        && \text{by \prefp{prop:TD_unitary}}
      \\&= \sum_{m\in\Z} g_m \sum_{k\in\Z} g_k^\ast \inprod{\fphi }{\opTrn^{2n-m+k} \fphi}
    \end{align*}

  \item Proof for (3): 
    \begin{align*}
      &\inprod{\fphi}{\opTrn^n \fpsi}
      \\&= \inprod{\sum_{m\in\Z} h_m \opDil \opTrn^m \fphi }{\opTrn^n \sum_{k\in\Z} g_k \opDil \opTrn^k \fphi}
        && \text{by \prefp{thm:dilation_eq}}
        && \text{and \prefp{thm:g->psi}}
      \\&= \sum_{m\in\Z} h_m \sum_{k\in\Z} g_k^\ast \inprod{\opDil \opTrn^m \fphi }{\opTrn^n \opDil \opTrn^k \fphi}
        && \text{by properties of $\inprodn$}
        && \text{\xref{def:inprod}}
      \\&= \sum_{m\in\Z} h_m \sum_{k\in\Z} g_k^\ast \inprod{\fphi }{\left(\opDil \opTrn^m \right)^\ast \opTrn^n \opDil \opTrn^k \fphi}
        && \text{by definition of operator adjoint}
        && \text{\ifxref{operator}{prop:op_adjoint}}
      \\&= \sum_{m\in\Z} h_m \sum_{k\in\Z} g_k^\ast \inprod{\fphi }{\left(\opDil \opTrn^m \right)^\ast \opDil \opTrn^{2n} \opTrn^k \fphi}
        && \text{by \prefp{prop:DTTD}}
      \\&= \sum_{m\in\Z} h_m \sum_{k\in\Z} g_k^\ast \inprod{\fphi }{\opTrna^m \opDila \opDil \opTrn^{2n} \opTrn^k \fphi}
        && \text{by operator star-algebra properties}
        && \text{\ifxref{operator}{thm:op_star}}
      \\&= \sum_{m\in\Z} h_m \sum_{k\in\Z} g_k^\ast \inprod{\fphi }{\opTrn^{-m} \opDil^{-1} \opDil \opTrn^{2n} \opTrn^k \fphi}
        && \text{by \prefp{prop:TD_unitary}}
      \\&= \sum_{m\in\Z} h_m \sum_{k\in\Z} g_k^\ast \inprod{\fphi }{\opTrn^{2n-m+k} \fphi}
    \end{align*}
\end{enumerate}
\end{proof}



%=======================================
%\section{Fourier properties}
%=======================================

%--------------------------------------
\begin{proposition}
\label{prop:vsmra_real_Fpsi}
\label{prop:psi_g_phi}
%--------------------------------------
Let $\wavsys$ be a wavelet system.
Let $\Fphi(\omega)$ and $\Fpsi(\omega)$ be the \fncte{Fourier transform}s\ifsxrefs{harFour}{def:ft}of $\fphi(x)$ and $\fpsi(x)$, respectively.
Let $\Dg(\omega)$ be the \fncte{Discrete time Fourier transform}\ifsxrefs{dsp}{def:dtft}of $\seqn{g_n}$.
%  $\begin{array}[t]{rc>{\ds}l c>{\ds}l D}
%    \Fpsi\brp{\omega}
%      &\eqd& \opFT\fpsi
%      &\eqd& \frac{1}{\sqrt{2\pi}}\int_t \fpsi(x) e^{-i\omega t} \dx
%      &      (\structe{Fourier transform}, \prefp{def:ft})
%      \\
%    \Dg(\omega)
%      &\eqd& \opDTFT\seqn{g_n}
%      &\eqd& \sum_{n\in\Z} g_n e^{-i\omega n}
%      &      (\structe{Discrete-time Fourier Transform}).
%  \end{array}$
\propbox{
  \Fpsi\brp{\omega}
    %\eqd
    %\mcom{\opFT\fpsi = \brp{\opDili \opDTFT\seqn{g_n}} \; \brp{\opDili \opFT\fphi}}
    %     {operator notation}
    =
    {\cwt \: \Dg\brp{\frac{\omega}{2}}\: \Fphi\brp{\frac{\omega}{2}}}
    %     {traditional notation}
  }
\end{proposition}
\begin{proof}
\begin{align*}
  \Fpsi\brp{\omega}
    &\eqd \opFT\fpsi
  \\&= \opFT\sum_{n\in\Z} g_n \opDil \opTrn^n \fphi
    && \text{by \thme{wavelet dilation equation}}
    && \text{\xref{thm:g->psi}}
  \\&= \sum_{n\in\Z} g_n \opFT\opDil \opTrn^n \fphi
  \\&= \sum_{n\in\Z} g_n \opDili \opFT\opTrn^n \fphi
    && \text{by \prefp{cor:wavstrct_FTD}}
  \\&= \sum_{n\in\Z} g_n \opDili e^{-i\omega n} \opFT\fphi
    && \text{by \prefp{cor:wavstrct_FTD}}
  \\&= \sum_{n\in\Z} g_n \sqrt{2}\brp{\opDili e^{-i\omega n}} \brp{\opDili\opFT\fphi}
    && \text{by \prefp{prop:DjTnfg}}
  \\&= \sqrt{2}\brp{\opDili \sum_{n\in\Z} g_n e^{-i\omega n}} \; \brp{\opDili \opFT\fphi}
  \\&= \sqrt{2}\brp{\opDili \opDTFT\seqn{g_n}} \; \brp{\opDili \opFT\fphi}
    && \text{by definition of $\opDTFT$}
    && \text{\ifxref{dsp}{def:dtft}}
  \\&= \sqrt{2}\cwt \: \Dg\brp{\frac{\omega}{2}}\: \cwt \Fphi\brp{\frac{\omega}{2}}
    && \text{by property of $\opDil$} && \text{\xref{prop:opDi}}
  \\&= \cwt \: \Dg\brp{\frac{\omega}{2}}\: \Fphi\brp{\frac{\omega}{2}}
\end{align*}
%
%\begin{align*}
%  \Fpsi\brp{\omega}
%    &= \opF\fpsi
%  \\&= \opFT\sum_{n\in\Z} g_n \opDil \opTrn^n \fphi
%    && %\text{by \thme{dilation equation} \xref{thm:dilation_eq}}
%  \\&= \sum_{n\in\Z} g_n \opFT\opDil \opTrn^n \fphi
%  \\&= \sum_{n\in\Z} g_n \opDili \opFT\opTrn^n \fphi
%    && \text{by \prefp{prop:vsmra_real_FD}}
%  \\&= \sum_{n\in\Z} g_n \opDili e^{-i\omega n} \opFT\fphi
%    && \text{by \prefp{prop:vsmra_real_FT}}
%  \\&= \brp{\opDili \sum_{n\in\Z} g_n e^{-i\omega n}} \; \brp{\opDili \opFT\fphi}
%  \\&= \brp{\opDili \opDTFT\seqn{g_n}} \; \brp{\opDili \opFT\fphi}
%  \\&= \fscale \: \Dg\brp{\frac{\omega}{2}}\: \fscale \Fphi\brp{\frac{\omega}{2}}
%    && \text{by property of $\opDil$} && \text{\xref{prop:opDi}}
%  \\&= \frac{1}{2}\: \Dh\brp{\frac{\omega}{2}}\: \Fphi\brp{\frac{\omega}{2}}
%\end{align*}
\end{proof}

%=======================================
%\section{Immediate results}
%=======================================



%=======================================
%\subsection{Power Spectrum}
%=======================================
%%--------------------------------------
%\begin{definition}
%\citep{chui}{134}
%\label{def:wav_S}
%\index{scaling power spectrum function         }
%\index{wavelet power spectrum function         }
%\index{scaling wavelet power spectrum function }
%\index{Laurent polynomial}
%%--------------------------------------
%Let $\wavsys$ be a \hi{wavelet system}.
%Let $\Szfg(z)$ be the \fncte{complex cross-power spectrum} of $\ff$ and $\fg$ \xref{def:Szfg} in $\spLLR$
%and $\Swfg(\omega)$ be the \fncte{cross-power spectrum} of $\ff$ and $\fg$ \xref{def:Swfg} in $\spLLR$.
%\defbox{\begin{array}{>{\ds}rc>{\ds}lM}
%    \Szphi(z)  &\eqd&  \left.\Szfg(z)\right|_{\ff=\fg=\fphi} &is the \hid{scaling power spectrum function}.
%    \Szpsi(z)  &\eqd&  \left.\Szfg(z)\right|_{\ff=\fg=\fpsi} &is the \hid{wavelet power spectrum function}.
%    \Szpsi(z)  &\eqd&  \Szfg(z) &is the \hid{scaling power spectrum function}.
%    \\
%    \Spsi(\omega) &\eqd&  \sum_{n\in\Z} \Rpsi(n) \fkernea{n}{\omega}
%    &is the \hid{wavelet power spectrum function}.
%    \\
%    \Shs (\omega) &\eqd&  \sum_{n\in\Z} \Rhs (n) \fkernea{n}{\omega}
%    &is the \hid{scaling wavelet power spectrum function}.
%\end{array}}
%%\\The Laurent polynomial $\Sphi(\omega)$ is also called the \hid{Euler-Frobenius polynomial}.
%\end{definition}


%In this chapter, we don't assume the special case of orthonormality.
%But good things happen if we do happen to have orthonormality.
%One of them is that the power spectrum equations in \prefp{lem:SSS}
%simplify to constants \xref{lem:SSSo}.

\pref{thm:gen_quadcon} (next) presents the \structe{quadrature} necessary conditions of a \hi{wavelet system}.
These relations simplify dramatically in the special case of an
\structe{orthonormal wavelet system} \xref{thm:oquadcon}.
%--------------------------------------
\begin{theorem}[\thmd{Quadrature conditions} in ``frequency"]
\citetbl{
  \citerp{chui}{135},
  \citerp{goswami}{110}
  }
\label{thm:gen_quadcon}
%--------------------------------------
Let $\wavsys$ be a \hi{wavelet system}.
Let $\Fx(\omega)$ be the \fncte{discrete time Fourier transform}\ifsxrefs{dsp}{def:dtft}for a sequence $\seqxZ{x_n}$ in $\spllR$.
Let $\Swphi(\omega)$ be the \fncte{auto-power spectrum} \xref{def:Swfg} of $\fphi$,
    $\Swpsi(\omega)$ be the \fncte{auto-power spectrum} of $\fpsi$,
and $\Shs(\omega)$ be the \fncte{cross-power spectrum} of $\fphi$ and $\fpsi$.
\thmbox{\begin{array}{F>{\ds}lc>{\ds}l}
   1. & \abs{\Dh\left(\omega     \right)}^2 \Sphi(\omega) + \abs{\Dh\brp{\omega+\pi }}^2 \Sphi(\omega+\pi) &=& 2\Sphi(2\omega)
\\ 2. & \abs{\Dg\left(\omega     \right)}^2 \Sphi(\omega) + \abs{\Dg\brp{\omega+\pi }}^2 \Sphi(\omega+\pi) &=& 2\Spsi(2\omega)
\\ 3. & \Dh(\omega)\Dg^\ast(\omega)         \Sphi(\omega) + \Dh(\omega +\pi)\Dg^\ast(\omega +\pi)\Sphi(\omega+\pi) &=& 2\Shs(2\omega)
\end{array}}
\end{theorem}
\begin{proof}
\begin{enumerate}
  %\item First note that $\Dh(\omega)$ and $\Dg(\omega)$ are periodic with period $2\pi$ such that\ifsxrefs{dsp}{prop:dtft_2pi}
  %  \\$\begin{array}{rclC}
  %       \Dh(\omega+2\pi n) &=& \Dh(\omega) & \forall n\in\Z   \\
  %       \Dg(\omega+2\pi n) &=& \Dg(\omega) & \forall n\in\Z   
  %     \end{array}$

\item Proof for (1): by \prefp{thm:Sphi}.

\item Proof for (2):
\begin{align*}
   2\Spsi(2\omega)
     &\eqd 2\brp{2\pi}\sum_{n\in\Z} \left|\Fpsi(2\omega+2\pi n)\right|^2
   \\&= 2\brp{2\pi}\sum_{n\in\Z} \left|\cwt \Dg\left(\frac{2\omega+2\pi n}{2}\right)\Fphi\left(\frac{2\omega+2\pi n}{2}\right)\right|^2
        \indentx\text{by \prefp{lem:Fphi}}
   \\&= 2\pi
        \sum_{n\in\Ze} \left|\Dg\left(\frac{2\omega+2\pi n}{2}\right)\right|^2\left|\Fphi\left(\frac{2\omega+2\pi n}{2}\right)\right|^2 +
     \\&\qquad 2\pi
        \sum_{n\in\Zo} \left|\Dg\left(\frac{2\omega+2\pi n}{2}\right)\right|^2\left|\Fphi\left(\frac{2\omega+2\pi n}{2}\right)\right|^2
   \\&= 2\pi\sum_{n\in\Z} \abs{\Dg\brp{\omega+2\pi n     }}^2 \abs{\Fphi\brp{\omega+2\pi n       }}^2 +
        2\pi\sum_{n\in\Z} \abs{\Dg\brp{\omega+2\pi n+\pi }}^2 \abs{\Fphi\brp{\omega+2\pi n + \pi }}^2  
   \\&= 2\pi\sum_{n\in\Z} \abs{\Dg\brp{\omega            }}^2 \abs{\Fphi\brp{\omega+2\pi n       }}^2 +
        2\pi\sum_{n\in\Z} \abs{\Dg\brp{\omega+\pi        }}^2 \abs{\Fphi\brp{\omega+2\pi n + \pi }}^2  
   \\&= \abs{\Dg\brp{\omega     }}^2 \brp{2\pi\sum_{n\in\Z} \abs{\Fphi\brp{\omega+2\pi n       }}^2 +}
        \abs{\Dg\brp{\omega+\pi }}^2 \brp{2\pi\sum_{n\in\Z} \abs{\Fphi\brp{\omega+\pi+2\pi n   }}^2  }
   \\&= \abs{\Dg\brp{\omega     }}^2\Sphi(\omega) +
        \abs{\Dg\brp{\omega+\pi }}^2 \Sphi(\omega+\pi)
        \indentx\text{by \prefp{thm:Swfg}}
\end{align*}


\item Proof for (3):
\begin{align*}
  2\Shs(2\omega)
    &=  2\brp{2\pi}\sum_{n\in\Z} \Fphi(2\omega+2\pi n) \Fpsi^\ast(2\omega+2\pi n)
  \\&=  2\brp{2\pi}\sum_{n\in\Z}
        \cwt 
        \Dh  \left(\omega +\pi n \right)
        \Fphi\left(\omega +\pi n \right)
        \cwt 
        \Dg^\ast  \left(\omega +\pi n \right)
        \Fphi^\ast\left(\omega +\pi n \right)
        \quad\text{by \prefp{lem:Fphi}}
  \\&=  2\pi
        \sum_{n\in\Z}
        \Dh  \left(\omega +\pi n \right)
        \Dg^\ast  \left(\omega +\pi n \right)
        \left| \Fphi\left(\omega +\pi n \right) \right|^2
  \\&=  2\pi
        \sum_{n\in\Zo}
        \Dh  \left(\omega +\pi n \right)
        \Dg^\ast  \left(\omega +\pi n \right)
        \left| \Fphi\left(\omega +\pi n \right) \right|^2
      \\&\qquad+ 2\pi\sum_{n\in\Ze}
        \Dh  \left(\omega +\pi n \right)
        \Dg^\ast  \left(\omega +\pi n \right)
        \left| \Fphi\left(\omega +\pi n \right) \right|^2
  \\&=  2\pi\sum_{n\in\Z}
        \Dh  \left(\omega +2\pi n+\pi \right)
        \Dg^\ast  \left(\omega +2\pi n+\pi \right)
        \left| \Fphi\left(\omega +2\pi n+\pi \right) \right|^2
      \\&\qquad+ 2\pi\sum_{n\in\Z}
        \Dh  \left(\omega +2\pi n\right)
        \Dg^\ast  \left(\omega +2\pi n\right)
        \left| \Fphi\left(\omega +2\pi n\right) \right|^2
  \\&=  2\pi
        \sum_{n\in\Z}
        \Dh  \left(\omega +\pi \right)
        \Dg^\ast  \left(\omega +\pi \right)
        \left| \Fphi\left(\omega +2\pi n+\pi \right) \right|^2
      + 2\pi\sum_{n\in\Z}
        \Dh  \left(\omega \right)
        \Dg^\ast  \left(\omega \right)
        \left| \Fphi\left(\omega +2\pi n\right) \right|^2
  \\&=  \Dh  \left(\omega \right)
        \Dg^\ast  \left(\omega \right)
        \brp{2\pi\sum_{n\in\Z} \left| \Fphi\left(\omega +2\pi n\right) \right|^2}
      \\&\qquad+ \Dh  \left(\omega +\pi \right)
        \Dg^\ast  \left(\omega +\pi \right)
        \brp{2\pi\sum_{n\in\Z}\left| \Fphi\left(\omega +\pi+2\pi n\right) \right|^2}
  \\&=  \Dh(\omega)
        \Dg^\ast(\omega)
        \brp{2\pi\sum_{n\in\Z} \left| \Fphi(\omega +2\pi n) \right|^2}
      + \Dh  (\omega +\pi)
        \Dg^\ast (\omega +\pi)
        \brp{2\pi\sum_{n\in\Z}\left| \Fphi(\omega +\pi+2\pi n) \right|^2}
  \\&=  \Dh(\omega     ) \Dg^\ast(\omega     )\Sphi(\omega)
      + \Dh(\omega +\pi) \Dg^\ast(\omega +\pi)\Sphi(\omega+\pi)
        \indentx\text{by \prefp{thm:Swfg}}
\end{align*}

\end{enumerate}
\end{proof}

%%=======================================
%\begin{definition}
%%=======================================
%Let $\wavsys$ be a wavelet system.
%Let $\oppS\seqn{\fphi}_m$ be the \hid{span} of the basis vectors $\seqn{\fphi}_m$.
%We define the following order relations. 
%\defbox{\begin{array}{rcll}
%%  \opP_m         &\orel&  \opP_n         & \quad\text{if}\quad \opP_m\opP_n=\opP_n\opP_m=\opP_m \\
%  \spV_m         &\orela& \spV_j         & \quad\text{if}\quad \spV_m \subseteq \spV_j \\
%  \seqn{\fphi}_m &\orelb& \seqn{\fphi}_n & \quad\text{if}\quad \oppS\seqn{\fphi}_m \subseteq \oppS\seqn{\fphi}_n %\qquad\text{where $\oppS$ is the span}
%\end{array}}
%\end{definition}
%



%The sequences of subspaces discussed in this section together with
%set relations $\subseteq$, $\setu$, and $\seti$,
%for a lattice.
%This ``\hie{lattice of wavelet subspaces}" is defined next.
%%---------------------------------------
%\begin{definition}
%\label{def:wav_lat_subspace}
%%---------------------------------------
%Let $\seqjZ{\spV_j}$ be a sequence of scaling subspaces and
%    $\seqjZ{\spW_j}$ be a sequence of wavelet subspaces.
%%    $\subseteq$ the set inclusion relation,
%%    $\odot$ the set union operation, and
%%    $\seti$ the set intersection operation.
%\defbox{\begin{array}{l}
%  \text{The tupple }
%  \qquad \wavlatsubs \\
%  \text{is called the \hid{lattice of wavelet subspaces}.}
%\end{array}}
%\end{definition}


%=======================================
\section{Sufficient condition}
%=======================================
In this text, an often used sufficient condition for designing the \structe{wavelet coefficient sequence} 
$\seqn{g_n}$ \xref{def:gn} is the \prope{conjugate quadrature filter condition} \xref{def:cqf}. 
It expresses the sequence $\seqn{g_n}$ in terms of the \structe{scaling coefficient sequence} \xref{def:hn}
and a ``shift" integer $\xN$ as $g_n = \pm(-1)^n h^\ast_{\xN-n}$.
The \structe{CQF condition} has the following ``nice" properties:
\\\indentx\begin{tabular}{>{\scs}rp{\tw-30mm}}
    1. & Given a \structe{scaling coefficient sequence} $\seqn{h_n}$ \xref{def:hn}, 
         it is extremely simple to compute the \structe{wavelet coefficient sequence} $\seqn{g_n}$ \xref{def:gn}.
  \\2. & If $\setn{\opTrn\fphi}$ of a \structe{wavelet system} $\wavsys$ \xref{def:wavsys} is \prope{orthonormal} and 
         $\otriple{\seqn{g_n}}{\seqn{h_n}}{\xN}$ satisfies the \prope{CQF condition}, 
         then $\setn{\opTrn^n\fpsi}$ is also \prope{orthnormal}\ifsxref{ortho}{thm:ortho_qmr}.
  \\3. & If $\setn{\opTrn\fphi}$ of a \structe{wavelet system} $\wavsys$ \xref{def:wavsys} is \prope{orthonormal} and 
         $\otriple{\seqn{g_n}}{\seqn{h_n}}{\xN}$ satisfies the \prope{CQF condition}, 
         then the \structe{wavelet subspace} $\spW_0$ is 
         \prope{orthnormal} to the \structe{scaling subspace} $\spV_0$ ($\spW_0\orthog\spV_0$)\ifsxref{ortho}{thm:ortho_qmr}.
\end{tabular}

%---------------------------------------
\begin{theorem}
\label{thm:wavstrct_cqf}
%---------------------------------------
Let $\wavsys$ be a \structe{wavelet system} \xref{def:wavsys}.
Let $\Dg(\omega)$ be the \ope{DTFT} \xref{def:dtft} and $\Zg(z)$ the \ope{Z-transform} \xref{def:opZ} of $\seqn{g_n}$.
\thmbox{
  \begin{array}{>{\ds}rc >{\ds}rcl @{\qquad}D}
  \mcom{g_n = \pm(-1)^n h^\ast_{\xN-n},\,{\scy\xN\in\Z}}{\structe{conjugate quadrature filter}}
      &\iff&     \Dg(\omega)                   &=& \pm (-1)^\xN e^{-i\omega\xN} \Dh^\ast(\omega+\pi)\Big|_{\omega=\pi}   & (1)
    \\&\implies& \sum_{n\in\Z} (-1)^n g_n      &=& \sqrt{2}                                                              & (2)
    \\&\iff&     \Zg(z)\Big|_{z=-1}            &=& \sqrt{2}                                                              & (3)
    \\&\iff&     \Dg(\omega)\Big|_{\omega=\pi} &=& \sqrt{2}                                                              & (4)
  \end{array}}
\end{theorem}
\begin{proof}
  \begin{enumerate}
    \item Proof that CQF$\iff$(1): by \prefp{thm:cqf}

    \item Proof that CQF$\implies$(4):
      \begin{align*}
        \Dg(\pi)
          &= \Dg(\omega)\Big|_{\omega=\pi}
        \\&= \pm (-1)^\xN e^{-i\omega\xN} \Dh^\ast(\omega+\pi)\Big|_{\omega=\pi}
          && \text{by \thme{CQF theorem}}
          && \text{\xref{thm:cqf}}
        \\&= \pm (-1)^\xN e^{-i\pi\xN} \Dh^\ast(2\pi)
        \\&= \pm (-1)^\xN (-1)^\xN \Dh^\ast(0)
          && \text{by \thme{DTFT periodicity}}
          && \text{\xref{prop:dtft_2pi}}
        \\&= \sqrt{2}
          && \text{by \thme{admissibility condition}}
          && \text{\xref{thm:admiss}}
      \end{align*}

    \item Proof that (2)$\iff$(3)$\iff$(4): by \prefp{prop:dsp_zminone}
  \end{enumerate}
\end{proof}

%=======================================
\section{Support size}
%=======================================
%--------------------------------------
\begin{theorem}[\thmd{support size}]
\citetbl{
  \citerppg{mallat}{243}{244}{012466606X}
  }
\label{thm:support}
%--------------------------------------
Let $\wavsys$ be a \structe{wavelet system} \xref{def:wavsys}
induced by the \thme{CQF conditions} \xref{thm:wavstrct_cqf}.
Let $\support\ff$ be the support of a function $\ff$ \xref{def:support}.
\thmbox{
  \begin{array}{>{\ds}rc>{\ds}l}
  \support\fphi &=& \support\fh
  \\
  %\xN\in\Zo \text{ and }  g_n  = \pm(-1)^\xN\fh(\xN-n) \quad\implies\quad
  \support\fpsi &=& \intcc{\frac{\xN-(n_2-n_1)}{2}}{\frac{\xN+(n_2-n_1)}{2}}
  \end{array}
  }
\end{theorem}
\begin{proof}
\begin{enumerate}
  \item Proof that $\support\fphi = \support\fh$: by \prefpp{thm:mra_support}

  %\item Definitions: \label{item:wavstrct_support_def}
  %  \\$\begin{array}{rcl}
  %    \support\fphi &\eqd& [a,b] \\
  %    \support\fh   &\eqd& [k,m].
  %   \end{array}$

  %\item lemma: \label{ilem:mra_support_lemma}
  %  \\$\support\fphi(x)=\brs{a,b} \quad\iff\quad \support\fphi(2x)=\brs{\frac{a}{2},\frac{b}{2}}$
  %
  %\item lemma: \label{ilem:mra_support_mpy}
  %  \\$\support\brs{\lambda\fphi(x)}=\support\brs{\fphi(x)}\quad\forall\lambda\in\R\setd0$.

  \item Proof that $\support \psi=\intcc{\frac{\xN-(n_2-n_1)}{2}}{\frac{\xN+(n_2-n_1)}{2}}$:
      \begin{align*}
        \support\fpsi(x)
          &= \support \brs{ \sum_{n\in\Z} g_n \opDil\opTrn^n\fphi(x) }
          && \text{by \thme{wavelet dilation equation}} 
          && \text{\xref{thm:g->psi}}
        \\&= \support \brs{\sqrt{2}\sum_{n\in\Z}  g_n \fphi(2x-n)}
          && \text{by definition of $\opTrn$ and $\opDil$} 
          && \text{\xref{def:opTD}}
        \\&= \support \brs{\sqrt{2}\sum_{n\in\Z} \pm (-1)^\xN \fh(\xN-n)\fphi(2x-n)}
          && \text{by \thme{CQF conditions}} && \text{\xref{thm:wavstrct_cqf}}
        \\&= \support \brs{\sum_{n\in\Z} \fh(\xN-n)\fphi(2x-n)}
          && \text{by \prefpp{ilem:mra_support_mpy}}
        \\&= \cls{\set{x\in\R}{\sum_{n\in\Z} \fh(\xN-n)\fphi(2x-n)\ne 0}}
          && \text{by definition of $\support$} && \text{\xref{def:support}}
        \\&= \intcc{\frac{n_1}{2}+\frac{\xN-n_2}{2}} {\frac{n_2}{2}+\frac{\xN-n_1}{2}}
        \\&= \intcc{\frac{\xN-(n_2-n_1)}{2}}         {\frac{\xN+(n_2-n_1)}{2}}
      \end{align*}
\end{enumerate}
\end{proof}

%--------------------------------------
\begin{example}
%--------------------------------------
Here are some examples using \fncte{Daubechies wavelet function}s.
\exbox{\begin{array}{NNN}
%    \includegraphics{../common/math/graphics/pdfs/b3_g.pdf}
    \includegraphics{../common/math/graphics/pdfs/d1_psi_g.pdf}
   &\includegraphics{../common/math/graphics/pdfs/d2_psi_g.pdf}
   &\includegraphics{../common/math/graphics/pdfs/d3_psi_g.pdf}
   %&\includegraphics{../common/math/graphics/pdfs/d4_psi_g.pdf}
  %\\\fncte{B-spline of order 2}
  \\\fncte{Daubechies-1}
   &\fncte{Daubechies-2}
   &\fncte{Daubechies-3}
   %&\fncte{Daubechies-4 wavelet function}
  \\\ifxref{compactp}{ex:dau-p1}&\ifxref{compactp}{ex:dau-p2}&\ifxref{compactp}{ex:dau-p3}
\end{array}}
\end{example}

%=======================================
\section{Examples}
%=======================================
%Under the very general constraints of this chapter, I know of no wavelet examples.
No further examples of wavelets are presented in this section. 
Examples begin in the next chapter which is about a property called the \prope{partition of unity}.
%The very minimal of requirements, it seems is a \prope{partition of unity} \xref{chp:pounity}.
Other design constraints leading to wavelets with more ``powerful" properties include 
\prope{vanishing moments} \xref{chp:vanish}, \prope{orthonormality}\ifsxref{ortho}{chp:ortho},
\prope{compact support}\ifsxref{compactp}{chp:compactp}, and \prope{minimum phase} \xref{def:minphase}.

%Here are some examples of \structe{wavelet systems} \xref{def:wavsys}:
%\begin{longtable}{|l||ll||c|c|c|c|c|}
%  \hline
%  Name   & \mc{2}{||c|}{Reference} & \rotatebox{75}{\prope{partition of unity}} 
%                                   & \rotatebox{75}{\prope{vanishing moments}} 
%                                   & \rotatebox{75}{\prope{orthonormality}} 
%                                   & \rotatebox{75}{\prope{compact support}}
%                                   & \rotatebox{75}{\prope{minimum phase}}
%%         &          &       & \xref{chp:pounity}         & \xref{chp:vanish}         & \xref{chp:ortho}       & \xref{chp:compactp}
%  \\\hline
%  %Haar             & \scs\pref{ex:pun_n=2}       & \scs\prefpo{ex:pun_n=2}       & $\checkmark$ & 1 & $\checkmark$ & $\checkmark$ & $\checkmark$ \\
%  %order 1 B-spline & \scs\pref{ex:sw_gh_tent}    & \scs\prefpo{ex:sw_gh_tent}    & $\checkmark$ & 2 & $          $ & $          $ & $          $ \\
%  %order 3 B-spline & \scs\pref{ex:sw_gh_bspline} & \scs\prefpo{ex:sw_gh_bspline} & $\checkmark$ & 4 & $          $ & $          $ & $          $ \\
%  order 0 B-spline & \pref{ex:N0_hg}                          & \prefpo{ex:N0_hg} & $\checkmark$ & 1 & $\checkmark$ & $\checkmark$ & $\checkmark$ \\
%  order 1 B-spline & \pref{ex:N1_hg}                          & \prefpo{ex:N1_hg} & $\checkmark$ & 2 & $          $ & $          $ & $          $ \\
%  order 2 B-spline & \pref{ex:N2_hg}                          & \prefpo{ex:N2_hg} & $\checkmark$ & 4 & $          $ & $          $ & $          $ \\
%  Daubechies-p2    & \ifdochas{compactp}{\pref{ex:dau-p2}}    & \ifdochas{compactp}{\prefpo{ex:dau-p2}}        & $\checkmark$ & 2 & $\checkmark$ & $\checkmark$ & $\checkmark$ \\
%  Daubechies-p3    & \ifdochas{compactp}{\pref{ex:dau-p3}}    & \ifdochas{compactp}{\prefpo{ex:dau-p3}}        & $\checkmark$ & 3 & $\checkmark$ & $\checkmark$ & $\checkmark$ \\
%  Symlet-p4        & \ifdochas{compactp}{\pref{ex:symlet_p4}} & \ifdochas{compactp}{\prefpo{ex:symlet_p4}}     & $\checkmark$ & 4 & $\checkmark$ & $\checkmark$ & $          $ \\
%  \hline
%\end{longtable}




%================================================================================
%================================================================================
%================================================================================
%================================================================================
%================================================================================

%\paragraph{Fourier Transform.}
%One of the most widely used transforms is the Fourier transform.
%The Fourier Transform is an integral operator with an exponential kernel.
%And what is so special about exponential kernels?
%Is it just that they were discovered sooner than other kernels with other transforms?
%The answer in general is ``no".
%The exponential has two properties that makes it extremely special:
%  \begin{liste}
%    \item The exponential is an eigenvalue of any LTI operator \xref{thm:Le=he}
%    \item The exponential generates a continuous point spectrum for the differential operator
%          \xref{thm:spec_D}
%  \end{liste}
%
%\thmbox{
%  \left.\begin{array}{ll}
%    1. & \text{$\opL$ is linear and} \\
%    2. & \text{$\opL$ is time-invariant}
%  \end{array}\right\}
%  \qquad\implies\qquad
%    \opL \lkerne{t}{s}
%    =
%    \mcomr{\Lh^\ast(-s)}{eigenvalue} \mcoml{\lkerne{t}{s}}{eigenvector}
%  }
%
%
%What makes wavelet system unique from other analysis systems
%(such as Fourier analysis) is its \hie{subspace architecture}.
%In this section, we present this architecture using four representations:
%
%\begin{tabular}{lp{\tw/2}ll}
%  \circOne   & lattice of subspaces             \dotfill&\pref{sec:wav_lat_subspace} & \xref{sec:wav_lat_subspace} \\
%  \circTwo   & lattice of projection operators  \dotfill&\pref{sec:wav_lat_op}       & \xref{sec:wav_lat_op} \\
%  \circThree & lattice of bases vectors         \dotfill&\pref{sec:wav_lat_bases}    & \xref{sec:wav_lat_bases} \\
%  \circFour  & lattice of bases coefficients    \dotfill&\pref{sec:wav_lat_coef}     & \xref{sec:wav_lat_coef} \\
%\end{tabular}
%
%\prefp{thm:wav_lat_iso} will show that all four of these representations are
%essentially equivalent (are isomorphisms).
%The wavelet system itself is simply a collection of the projection operators
%\xref{def:wav_transform} found in the wavelet operator lattice.


