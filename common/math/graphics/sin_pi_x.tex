%============================================================================
% Daniel J. Greenhoe
% LaTeX file
% sin(t)
%============================================================================
%  \psset{unit=1mm}
\begin{pspicture}(-40,-15)(40,15)%
  \footnotesize
  \psset{linecolor=blue}%
  %\rput(0,0){% axis
  %  \psset{linecolor=axis}
  %  \multirput(-30,0)(10,0){7}{\psline(0,-1)(0,1)}% markers on x axis
  %  \psline{<->}(-35,0)(35,0)% x axis
  %  \psline{<->}(0,-15)(0,15)%    y axis
  %  \psline(-1,10)(1,10)%
  %  \psline(-1,-10)(1,-10)%
  %  \uput[180](0,10){$1$}% y=1
  %  \uput[0](0,-10){$-1$}% y=-1
  %  \multido{\ival=-3+1,\ipos=-30+10}{7}{%
  %    \uput[-90](\ipos,0){$\ival$}% x=
  %    }%
  %  \uput[0](40,0){$t$}%
  %  }%
  \psaxes[linecolor=axis,unit=10]{<->}(0,0)(-3.5,-1.5)(3.5,1.5)%
  \psplot[plotpoints=100]{-30}{30}{x 18 mul sin 10 mul}%
  \psplot[plotpoints=10,linestyle=dotted,linewidth=2pt]{30}{35}{x 18 mul sin 10 mul}%
  \psplot[plotpoints=10,linestyle=dotted,linewidth=2pt]{-35}{-30}{x 18 mul sin 10 mul}%
  \rput[b](17.5,11){$\ff(t)\eqd\sin(\pi x)$}%
  \rput[0](35,0){$x$}%
\end{pspicture}
