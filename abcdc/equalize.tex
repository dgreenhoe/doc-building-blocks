%============================================================================
% LaTeX File
% Daniel Greenhoe
%============================================================================

%======================================
\chapter{Distorted Frequency Response Channel}
\index{channel!distorted frequency response}
\label{chp:eq}
%======================================
\begin{figure}[ht] \color{figcolor}
\begin{center}
\begin{fsK}
\setlength{\unitlength}{0.15mm}                  
\begin{picture}(900,150)(-200,-100)  
  \thinlines                                      
  %\graphpaper[10](0,0)(700,100)                  

  \put(-170,  -80 ){\dashbox{5}( 280,160)   {}             }
  \put(-170,  -70 ){\makebox( 280,160)[b]   {transmitter}             }
  \put(-180,   10 ){\makebox (  0, 40)[br]{$\su$}        }
  \put(-200,    0 ){\vector  (   1,  0)   { 50}          }
  \put(-150,  -50 ){\framebox( 100,100)   {mapper}    }

  \put(- 50,   10 ){\makebox (  50, 40)[b]{$\sa$}        }
  \put(- 50,    0 ){\vector  (   1,  0)   { 50}          }
  \put(   0,  -50 ){\framebox( 100,100)   {modulator}    }
  \put( 110,   10 ){\makebox (  90,  0)[b]{$s(t;\va)$}       }
  \put( 110,  -90 ){\makebox (   0,  0)[lt]{$\ds s(t;\va)=\sum_n a_n \lambda(t-nT)$}       }
  \put( 100,    0 ){\vector  (   1,  0)   {100}         }
  \put( 200,  -50 ){\framebox( 100,100)   {$\conv h_c(t)$} }
  \put( 200,  -30 ){\makebox ( 100, 80)[b]{frequency}    }
  \put( 200,  -40 ){\makebox ( 100, 80)[b]{distortion}   }

  \put( 300,   10 ){\makebox (  50, 40)[b]{$w(t)$}       }
  \put( 350,    0 ){\usebox{\picAWGN}}
  \put( 460,   10 ){\makebox ( 90, 50)[b]{$r(t)$}       }
  \put( 460,  -90 ){\makebox (   0,  0)[lt]{$\ds r(t)=w(t)+n(t)=\sum_n a_n \fh(t-nT) + n(t)$}       }

  \put( 190,  -80 ){\dashbox{5}( 270,160)   {}             }
  \put( 190,  -70 ){\makebox ( 270,160)[b]{channel}      }

  \put( 500,    0 ){\line     (   1,  0)   {50}         }
  \put( 550,  -50 ){\framebox ( 100,100){receiver}   }
  \put( 660,   10 ){\makebox ( 40, 50)[b]{$\sue$}       }
  \put( 650,    0 ){\vector  (   1,  0)   {50}         }
\end{picture}                                   
\end{fsK}
\end{center}
\caption{
   Equalization system model
   \label{fig:eq_system_model}
   }
\end{figure}


%======================================
\section{Channel Model}
%======================================
In this chapter, the channel model includes both deterministic and random 
distortion.

\begin{tabular}{lll}
   \circOne & linear deterministic distortion  & (convolution with $h_c(t)$) \\
   \circTwo & linear stochastic distortion     & (additive white Gaussian noise).
\end{tabular}

Let 

\begin{tabular}{ll}
   $\su$        & be the information sequence \\
   $\sa$        & be a mapped sequence under a one to one function $a_n=f(u_n)$ \\
   $\lambda(t)$ & be the {\em modulation waveform} \\
   $s(t;\va)$   & be the {\em transmitted waveform} \\
   $h_c(t)$     & be the {\em channel impulse response} \\
   $n(t)$       & be the {\em channel noise} with distribution $n(t)\sim \pN{0}{\sigma^2}$.
\end{tabular}

The following definitions apply throughout this chapter:
\defbox{
\begin{array}{rcl}
   \fs(t;\sa) &\eqd& \sum_n a_n \lambda(t-nT) \\
   \fh(t)     &\eqd& \lambda(t)\conv\fh_c(t) = \int_\tau \fh(\tau)\lambda(t-\tau) \dtau \\
   w(t)       &\eqd& \int_\tau h_c(\tau)s(t-\tau) \; d\tau \\
   r(t)       &\eqd& w(t) + n(t). \\
\end{array}}

Under these definitions the received signal can be expressed as follows:
\begin{eqnarray*}
   r(t)       
     &=&    w(t) + n(t)
   \\&=&    \int_\tau h_c(\tau)s(t-\tau) \dtau + n(t)
   \\&=&    \int_\tau h_c(\tau)\sum_n a_n \lambda(t-\tau-nT) \dtau + n(t)
   \\&=&    \sum_n a_n \int_\tau \fh_c(\tau)\lambda(t-\tau-nT) \dtau + n(t)
   \\&=&    \sum_n a_n \fh(t-nT) + n(t)
\end{eqnarray*}

%======================================
\section{Sufficient statistic sequence}
\label{sec:fd_ml}
%======================================
%======================================
\subsection{Receiver statistics}
%======================================
Define the innerproduct quantities as
\defbox{\begin{array}{rcl}
   \dot{r}_n    &\eqd& \inprod{r(t)}{\psi_n(t)}  \\ 
   \dot{n}_n    &\eqd& \inprod{n(t)}{\psi_n(t)}  \\
   \dot{h}_n(m) &\eqd& \inprod{\fh(t-mT)}{\psi_n(t)} 
\end{array}}

The quantity $\dot{r}_n$ is a random variable with form
\begin{eqnarray*}
   \dot{r}_n
     &\eqd& \inprod{r(t)}{\psi_n(t)} 
   \\&=&    \inprod{w(t)+n(t)}{\psi_n(t)} 
   \\&=&    \inprod{w(t)}{\psi_n(t)} + \inprod{n(t)}{\psi_n(t)} 
   \\&=&    \inprod{\sum_m a_m \fh(t-mT)}{\psi_n(t)} + \inprod{n(t)}{\psi_n(t)} 
   \\&=&    \sum_m a_m \inprod{\fh(t-mT)}{\psi_n(t)} + \inprod{n(t)}{\psi_n(t)} 
   \\&=&    \sum_m a_m \dot{h}_n(m) + \dot{n}_n.
\end{eqnarray*}

By \prefpp{thm:ms_stats},
the quantity $\dot{r}_n$ given $\va $ has Gaussian distribution
\[
   (\dot{r}_n|\va ) \sim \pN{\sum_m a_m \dot{h}_n(m)}{\sigma^2}
\]
and $\dot{r}_n|\va $ and $\dot{r}_m|\va $ are independent for $n\ne m$.


%======================================
\subsection{ML estimate and sufficient statistic}
%======================================
\begin{figure}[ht] \color{figcolor}
\begin{center}
\begin{fsK}
\setlength{\unitlength}{0.15mm}                  
\begin{picture}(1000,200)(0,-80)  
  \thinlines                                      
  %\graphpaper[10](0,0)(700,100)                  

  \put(   0,   10 ){\makebox (  50, 40)[b]{$\sa$}        }
  \put(   0,    0 ){\vector  (   1,  0)   { 50}          }
  \put(  50,  -50 ){\framebox( 100,100)   {modulator}    }
  \put( 150,   10 ){\makebox (  50, 40)[b]{$\fs(t;\sa)$}       }
  \put( 150,    0 ){\vector  (   1,  0)   { 50}         }
  \put( 200,  -50 ){\framebox( 100,100)   {$\conv \fh(t)$} }
  \put( 200,  -40 ){\makebox ( 100, 80)[t]{$\Lh(s)$} }
  \put( 200,  -80 ){\makebox ( 100, 30)[ ]{$\Zh(z)$} }
  \put( 200,  -40 ){\makebox ( 100, 30)[t]{frequency}    }
  \put( 200,  -40 ){\makebox ( 100, 30)[b]{distortion}   }
  \put( 300,    0 ){\line    (   1,  0)   { 50}         }

  \put( 300,   10 ){\makebox (  50, 40)[b]{$\fw(t;\sa)$}       }
  \put( 350,    0 ){\usebox{\picAWGN}}
  \put( 450,   10 ){\makebox (  50, 50)[b]{$r(t)$}       }

  \put( 500,  -50 ){\framebox( 100,100)   {$\conv \fh(-t)$} }
  \put( 500,  -40 ){\makebox ( 100, 80)[t]{$\Lh(-s)$} }
  \put( 500,  -80 ){\makebox ( 100, 30)[ ]{$\Zh(1/z)$} }
  \put( 500,  -40 ){\makebox ( 100, 30)[t]{matched}      }
  \put( 500,  -40 ){\makebox ( 100, 30)[b]{filter}       }
  \put( 600,    0 ){\vector  (   1,  0)   { 50}         }

  \put( 650,    0 ){\usebox{\picSampler}}
  \put( 750,    0 ){\vector  (   1,  0)   {100}         }

  \put(  40,  -80 ){\dashbox{5}( 720,160)   {}             }
  \put(  40,   90 ){\makebox ( 720, 30)[b]{equivalent digital system}}

  \put( 780,   10 ){\makebox ( 100, 40)[lb]{$\ds \fdotr_n=\sum_m a_{n-m}\Rhh(m)+\fdotn_n$}}

\end{picture}                                   
\end{fsK}
\end{center}
\caption{
   Sufficient statistic sequence $(\fdotr_n)$ for ML estimation
   \label{fig:sstat_ML}
   }
\end{figure}


%--------------------------------------
\begin{definition}
\label{def:eq_stats}
%--------------------------------------
\defbox{\begin{array}{rclcll}
   \Rhh(m) &\eqd& \inprod{\fh(t+mT)}{\fh(t)} &\eqd& \ds\int_t \fh(t+mT)h^\ast(t)\dt & \mbox{(autocorrelation)} 
\\
   \fdotr_n &\eqd& \inprod{r(t)}{\fh(t-nT)} &\eqd& \ds\int_t r(t)h^\ast(t-nT)\dt & \mbox{(receive statistic)} 
\\
   \fdotn_n &\eqd& \inprod{n(t)}{\fh(t-nT)} &\eqd& \ds\int_t n(t)h^\ast(t-nT)\dt & \mbox{(noise statistic)} 
\end{array}}
\end{definition}

Under these definitions, the receive statistic can be represented as follows
(see \prefp{fig:sstat_ML}):
\begin{eqnarray*}
   \fdotr_n
     &\eqd& \inprod{r(t)}{\fh(t-nT)} 
   \\&=&    \inprod{\sum_m a_n \fh(t-mT) + n(t)}{\fh(t-nT)} 
   \\&=&    \inprod{\sum_m a_n \fh(t-mT)}{\fh(t-nT)} + \inprod{n(t)}{\fh(t-nT)} 
   \\&=&    \sum_m a_m \inprod{\fh(t-mT)}{\fh(t-nT)} + \inprod{n(t)}{\fh(t-nT)} 
   \\&=&    \sum_m a_m \Rhh(n-m) + \fdotn_n
   \\&=&    \sum_k a_{n-k} \Rhh(k) + \fdotn_n
            \hspace{3ex}k=n-m \iff m=n-k
   \\&=&    \sum_m a_{n-m} \Rhh(m) + \fdotn_n
            \hspace{3ex}\mbox{by change of free variable}
\end{eqnarray*}
%--------------------------------------
\begin{theorem}
\label{thm:eq-ml}
\index{maximum likelihood}
%--------------------------------------
Under Definitions~\ref{def:eq_stats},
\begin{enume}
   \item The sequence $(\fdotr_n)$ is a {\bf sufficient statistic} 
         for determining the maximum likelihood (ML) estimate of $\va$.
   \item The ML estimate of $\va$ is
\formbox{
  \estML[\va ] = \arg\max_\va  \left( 
    2\sum_n a_n \fdotr_n -\sum_n\sum_m a_n a_{m+n} \Rhh(m)
    \right).
}
\end{enume}
\end{theorem}



\begin{proof}
\begin{eqnarray*}
   \estML[\va ]
     &\eqd& \arg\max_\va  \pP{r(t)|\fs(t;\sa)}
   \\&=&    \arg\max_\va   
            \left[ 
            2\int_t r(t)\fw(t;\sae) -\int_t \fw^2(t;\sae) \dt 
            \right]
            \hspace{8ex}\mbox{by \prefpp{thm:estML_general} page~\pageref{thm:estML_general}}
   \\&=&    \arg\max_\va   
            \left[ 
            2\int_t r(t)\sum_n a_n \fh(t-nT) \dt 
            -\int_t \sum_n a_n \fh(t-nT)\sum_m a_m \fh(t-mT) \dt 
            \right]
   \\&=&    \arg\max_\va   
            \left[ 
             2\sum_n a_n \int_t r(t)\fh(t-nT) \dt 
            -\sum_n \sum_m a_n a_m  \int_t \fh(t-nT)\fh(t-mT) \dt 
            \right]
   \\&=&    \arg\max_\va   
            \left[ 
             2\sum_n a_n \int_t r(t)\fh(t-nT) \dt
            -\sum_n \sum_m a_n a_m \Rhh(m-n)
            \right]
   \\&=&    \arg\max_\va   
            \left[ 
             2\sum_n a_n \int_t r(t)\fh(t-nT) \dt
            -\sum_n\sum_k a_n a_k \Rhh(k-n)
            \right]
   \\&=&    \arg\max_\va   
            \left[ 
             2\sum_n a_n \int_t r(t)\fh(t-nT) \dt
            -\sum_n\sum_m a_n a_{m+n} \Rhh(m)
            \right]
   \\&=&    \arg\max_\va   
            \left[ 
             2\sum_n a_n \fdotr_n
            -\sum_n\sum_m a_n a_{m+n} \Rhh(m)
            \right]
\end{eqnarray*}
\end{proof}

If the autocorrelation is zero for $|n|>L$, then
\prefpp{thm:eq-ml} reduces to the simpler form 
stated in \pref{cor:eq-ml} (next).
%--------------------------------------
\begin{corollary}
\label{cor:eq-ml}
%--------------------------------------
If 
   \[ \Rhh(n) = 0 \mbox{ for } |n|>L \]
then 

\corbox{
   \estML[\va ]
     =    \arg\max_\va   
            \left( 
             2\sum_n a_n \fdotr_n
            -\sum_n a_n \left[ a_n \Rhh(0) + 2\sum_{m=1}^L a_{m+n} \Rhh(m) \right]
            \right)
  }
\end{corollary}
\begin{proof}
First note that 
   \[ \sum_n \sum_{m=-L}^L a_{m+n} \Rhh(m) \]
is maximized when $a_{m+n}$ is symmetric about $n$ (??????). \attention
Then

\begin{eqnarray*}
   \estML[\va ]
     &=&    \arg\max_\va   
            \left( 
             2\sum_n a_n \fdotr_n
            -\sum_n\sum_m a_n a_{m+n} \Rhh(m)
            \right)
   \\&=&    \arg\max_\va   
            \left( 
             2\sum_n a_n \fdotr_n
            -\sum_n a_n \sum_{m=-L}^L a_{m+n} \Rhh(m)
            \right)
   \\&=&    \arg\max_\va   
            \left( 
             2\sum_n a_n \fdotr_n -\sum_n a_n 
             \left[ 
                a_n \Rhh(0) +
                \sum_{m=-L}^1 a_{m+n} \Rhh(m) +
                \sum_{m=1}^L a_{m+n} \Rhh(m)
            \right]
            \right)
   \\&=&    \arg\max_\va   
            \left( 
             2\sum_n a_n \fdotr_n -\sum_n a_n 
             \left[ 
                a_n \Rhh(0) +
                2\sum_{m=1}^L a_{m+n} \Rhh(m)
            \right]
            \right)
\end{eqnarray*}
\end{proof}


%======================================
\subsection{Statistics of sufficient statistic sequence $(\fdotr_n)$}
\index{noise!colored}
%======================================
The elements of the ML sufficient sequence $(\fdotr_n|\va )$ have
Gaussian distribution, however the sequence is {\bf colored}.
That is $\fdotr_n$ is correlated 
with $\fdotr_m$ (and therefore also not independent).
To whiten the sequence $(\fdotr_n)$, a whitening filter may be used. 
Whitening filters can be implemented in 
analog (\prefp{sec:whiten}) or digitally (\prefp{sec:d-whiten}).

%--------------------------------------
\begin{theorem}
%--------------------------------------
\thmbox{\begin{array}{rcl}
   \E{\fdotn_n}               &=&    \ds 0                                           \\
   \cov{\fdotn_n}{\fdotn_m}   &=&    \ds N_o \Rhh(n-m)                               \\
   \E\fdotr_n|\va             &=&    \ds \sum_m a_{n-m} \Rhh(m)                      \\
   \fdotr_n|\va               &\sim& \ds \pN{\sum_m a_{n-m} \Rhh(m)}{N_o \Rhh(0)} \\
   \cov{\fdotr_n|\va }{\fdotr_m|\va } &=& N_o \Rhh(n-m) 
\end{array}}
\end{theorem}
\begin{proof}
\begin{eqnarray*}
   \E{\fdotn_n}
     &=&    \E \inprod{n(t)}{\fh(t-nT}  
   \\&=&    \inprod{\E n(t)}{\fh(t-nT}  
   \\&=&    \inprod{0}{\fh(t-nT}  
   \\&=&    0
   \\
   \\
   \cov{\fdotn_n}{\fdotn_m}
     &=& \Eb{\fdotn_n \fdotn_m} - \Eb{\fdotn_n}\Eb{\fdotn_m}
   \\&=& \Eb{\fdotn_n \fdotn_m} - 0\cdot0
   \\&=& \Eb{\inprod{n(t)}{\fh(t-nT)} \inprod{n(t)}{\fh(t-mT)}}
   \\&=& \Eb{\inprod{n(t)}{\fh(t-nT)} \inprod{n(u)}{\fh(u-mT)}}
   \\&=& \Eb{\inprod{n(t)\inprod{n(u)}{\fh(u-mT)}}{\fh(t-nT)} }
   \\&=& \Eb{\inprod{\inprod{n(t)n(u)}{\fh(u-mT)}}{\fh(t-nT)} }
   \\&=& \inprod{\inprod{\Eb{n(t)n(u)}}{\fh(u-mT)}}{\fh(t-nT)}
   \\&=& \inprod{\inprod{N_o\delta(t-u)}{\fh(u-mT)}}{\fh(t-nT)}
   \\&=& N_o \inprod{\fh(t-mT)}{\fh(t-nT)}
   \\&=& N_o \Rhh(n-m)
   \\
   \\
   \E\fdotr_n 
     &\eqd& \E\inprod{r(t)}{\fh(t-nT)}  
   \\&=&    \E\inprod{\sum_k a_k \fh(t-kT) + n(t)}{\fh(t-nT)}  
   \\&=&    \inprod{\sum_k a_k \fh(t-kT) + \E n(t)}{\fh(t-nT)}  
   \\&=&    \inprod{\sum_k a_k \fh(t-kT) + 0}{\fh(t-nT)}  
   \\&=&    \sum_k a_k \inprod{\fh(t-kT)}{\fh(t-nT)}  
   \\&=&    \sum_k a_k \Rhh(n-k)
   \\&=&    \sum_m a_{n-m} \Rhh(m)  \hspace{3ex}m=n-k \iff k=n-m
   \\
   \\
   \cov{\fdotr_n}{\fdotr_m}
     &=& \Eb{\left(\fdotr_n-\E{\fdotr_n}\right)\left(\fdotr_m-\E{\fdotr_m}\right)}
   \\&=& \Eb{\fdotn_n \fdotn_m }
   \\&=& \cov{\fdotn_n}{\fdotn_m}
   \\&=& N_o \Rhh(n-m)
\end{eqnarray*}
\end{proof}


%======================================
\subsection{Spectrum of sufficient statistic sequence $(\fdotr_n)$}
%======================================
The Fourier Transform cannot be used to evaluate the spectrum of the 
sequences $(\fdotr_n)$, $\Rhh(m)$, and $(\fdotn_n)$ directly
because the sequences are not functions of a continuous variable.
Instead we compute the spectral content of their sampled 
continuous equivalents as defined next:
\defbox{\begin{array}{rcl}
   R_s(t)        &\eqd& \inprod{\fh(u+t)}{\fh(u)}\sum_n\delta(t-nT) \\
   \fdotr_s(t) &\eqd& \inprod{r(u)}{\fh(u-t)}\sum_n\delta(t-nT) \\
   \fdotn_s(t) &\eqd& \inprod{n(u)}{\fh(u-t)}\sum_n\delta(t-nT) \\
   a_s(t) &\eqd& a(t)\sum_n\delta(t-nT).
\end{array}}

Note that under these definitions

\propbox{\begin{array}{rcl}
   \Rhh(m)     &=& R_s(t)|_{t=mT}         \\
   \fdotr_n  &=& \fdotr_s(t)|_{t=nT}  \\
   \fdotn_n  &=& \fdotn_s(t)|_{t=nT}  \\
   a_n         &=& a_s(t)|_{t=nT}.        
\end{array}}



\begin{eqnarray*}
     S_s(f) 
     &\eqd& [\opFT R_s](f)
   \\&=& \left[\opFT{\inprod{\fh(u+t)}{\fh(u)}\sum_n\delta(t-nT)}\right](f)
   \\&=& \frac{1}{T} \sum_n \left[\opFT\inprod{\fh(u+t)}{\fh(u)}\right]\left(f-\frac{n}{T}\right)
         \hspace{3ex}\mbox{by \prefpp{thm:f_sampling} page~\pageref{thm:f_sampling}}
   \\&=& \frac{1}{T} \sum_n \int_t\inprod{\fh(u+t)}{\fh(u)} e^{-i2\pi\left(f-\frac{n}{T}\right)t}\dt
   \\&=& \frac{1}{T} \sum_n \int_t \int_u \fh(u+t) h^\ast(u) e^{-i2\pi\left(f-\frac{n}{T}\right)t}\dt
   \\&=& \frac{1}{T} \sum_n \int_t \int_u \fh(u+t) h^\ast(u) e^{-i2\pi\left(f-\frac{n}{T}\right)t}\dt
         \hspace{3ex}v=u+t\iff t=v-u
   \\&=& \frac{1}{T} \sum_n \int_v \int_u \fh(v) \fh^\ast(u) e^{-i2\pi\left(f-\frac{n}{T}\right)(v-u)}\du\dv
   \\&=& \frac{1}{T} \sum_n \int_u \fh^\ast(u) e^{i2\pi\left(f-\frac{n}{T}\right)u} \du
                            \int_v \fh(v)      e^{-i2\pi\left(f-\frac{n}{T}\right)v}\dv
   \\&=& \frac{1}{T} \sum_n \left(\int_u \fh(u) e^{-i2\pi\left(f-\frac{n}{T}\right)u}\du \right)^\ast
                            \int_v \fh(v)      e^{-i2\pi\left(f-\frac{n}{T}\right)v}\dv
   \\&=& \frac{1}{T} \sum_n \Fh^\ast\left(f-\frac{n}{T}\right)
                            \Fh     \left(f-\frac{n}{T}\right)
   \\&=& \frac{1}{T} \sum_n \left|\Fh\left(f-\frac{n}{T}\right)\right|^2
\\
\\
   \left[\opFT \fdotn_s\right](f)
     &=& \left[\opFT{\inprod{n(u)}{\fh(u-t)}\sum_n\delta(t-nT)}\right](f)
   \\&=& \frac{1}{T}\sum_n[\opFT{\inprod{n(u)}{\fh(u-t)}}]\left(f-\frac{n}{T}\right)
   \\&=& \frac{1}{T}\sum_n\int_t \inprod{n(u)}{\fh(u-t)} e^{-i2\pi\left(f-\frac{n}{T}\right)t}\dt
   \\&=& \frac{1}{T}\sum_n\int_t \int_u n(u)\fh^\ast(u-t) e^{-i2\pi\left(f-\frac{n}{T}\right)t}\du\dt
   \\&=& \frac{1}{T}\sum_n\int_v \int_u n(u)\fh^\ast(v) e^{-i2\pi\left(f-\frac{n}{T}\right)(u-v)}\du\dv
         \hspace{3ex}v=u-t\iff t=u-v
   \\&=& \frac{1}{T}\sum_n\int_u n(u) e^{-i2\pi\left(f-\frac{n}{T}\right)u}\du
                          \int_v \fh^\ast(v) e^{i2\pi\left(f-\frac{n}{T}\right)v}\dv
   \\&=& \frac{1}{T}\sum_n\int_u n(u) e^{-i2\pi\left(f-\frac{n}{T}\right)u}\du
                          \left[\int_v \fh(v) e^{i2\pi\left(f-\frac{n}{T}\right)v}\dv\right]^\ast
   \\&=& \frac{1}{T}\sum_n \ft{n}     \left(f-\frac{n}{T}\right) 
                           \Fh^\ast\left(f-\frac{n}{T}\right)
\\
\\
   \left[\opFT{\fdotr}\right](f)
     &=& \ft{a}_s(f) S_s(f) + \ft{\fdotn}_s(f)
   \\&=& \ft{a}_s(f) S_s(f) + \ft{\fdotn}_s(f)
   \\&=& \ft{a}_s(f) 
         \frac{1}{T} \sum_n \left|\Fh\left(f-\frac{n}{T}\right)\right|^2
         +\frac{1}{T}\sum_n \ft{n}     \left(f-\frac{n}{T}\right) 
                           \Fh^\ast\left(f-\frac{n}{T}\right)
\end{eqnarray*}

Note that the Fourier Transform $\ft{n}(f)$ only exists if it has finite energy 
(such as with most bandlimited noise). 
Thus, if $n(t)$ is a true white noise process,
$\ft{n}(f)$ does not exist.



%======================================
\section{Implementations}
%======================================
%======================================
\subsection{Trellis}
\label{eq_trellis}
%======================================
\begin{figure}[ht] \color{figcolor}
\begin{center}
\begin{fsK}
\setlength{\unitlength}{0.15mm}                  
\begin{picture}(1000,200)(0,-80)  
  \thinlines                                      
  %\graphpaper[10](0,0)(700,100)                  

  \put(   0,   10 ){\makebox (  50, 40)[b]{$a_n$}        }
  \put(   0,    0 ){\vector  (   1,  0)   { 50}          }
  \put(  50,  -50 ){\framebox( 100,100)   {modulator}    }
  \put( 150,   10 ){\makebox (  50, 40)[b]{$\fs(t;\sa)$}       }
  \put( 150,    0 ){\vector  (   1,  0)   { 50}         }
  \put( 200,  -50 ){\framebox( 100,100)   {$\conv \fh(t)$} }
  \put( 200,  -40 ){\makebox ( 100, 80)[t]{$\Lh(s)$} }
  \put( 200,  -80 ){\makebox ( 100, 30)[ ]{$\Zh(z)$} }
  \put( 200,  -40 ){\makebox ( 100, 30)[t]{frequency}    }
  \put( 200,  -40 ){\makebox ( 100, 30)[b]{distortion}   }
  \put( 300,    0 ){\line    (   1,  0)   { 50}         }

  \put( 300,   10 ){\makebox (  50, 40)[b]{$w(t)$}       }
  \put( 350,    0 ){\vector  (   1,  0)   { 40}          }
  \put( 350,  -50 ){\dashbox{4}( 100,100)   {$+$}          }
  \put( 350,  -40 ){\makebox ( 100, 80)[b]{noise}        }
  \put( 350,  -50 ){\makebox ( 100, 95)[t]{$n(t)$}       }
  \put( 400,    0 ){\circle{20}                          }
  \put( 410,    0 ){\line    (   1,  0)   { 40}          }
  \put( 400,   30 ){\vector  (   0, -1)   { 20}          }
  \put( 450,    0 ){\line    (   1,  0)   { 50}         }

  \put( 500,  -50 ){\framebox( 100,100)   {$\conv \fh(-t)$} }
  \put( 500,  -40 ){\makebox ( 100, 80)[t]{$\hat{h}(-s)$} }
  \put( 500,  -80 ){\makebox ( 100, 30)[ ]{$\hat{h}(1/z)$} }
  \put( 500,  -40 ){\makebox ( 100, 30)[t]{matched}      }
  \put( 500,  -40 ){\makebox ( 100, 30)[b]{filter}       }
  \put( 600,    0 ){\vector  (   1,  0)   { 50}         }

  \put( 650,  -50 ){\dashbox{4}( 100,100)   {} }
  \put( 650,  -40 ){\makebox ( 100, 80)[b]{sampler}      }
  \put( 650,    0 ){\line    (   1,  0)   { 15}          }
  \put( 650,    0 ){\line    (   1,  0)   { 40}          }
  \put( 690,    0 ){\line    (   1,  1)   { 25}          }
  \put( 710,    0 ){\line    (   1,  0)   { 40}          }
  \put( 750,    0 ){\vector  (   1,  0)   {100}          }

  \put( 450,   10 ){\makebox (  50, 50)[b]{$r(t)$}       }
  \put(  40,  -80 ){\dashbox{5}( 720,160)   {}             }
  \put(  40,   90 ){\makebox ( 720, 30)[b]{equivalent digital system}}

  \put( 780,   10 ){\makebox ( 70, 40)[lb]{$\fdotr_n$}}
  \put( 850,  -50 ){\framebox( 100,100)   {} }
  \put( 850,  -40 ){\makebox( 100,80)[c]{trellis} }
  \put( 850,  -40 ){\makebox( 100,80)[b]{estimate} }
  \put( 950,    0 ){\vector  (   1,  0)   {50}          }
\end{picture}                                   
\end{fsK}
\end{center}
\caption{
   Trellis implementation
   \label{fig:eq_trellis}
   }
\end{figure}

The ML estimate can be computed by the use of a trellis. 
The distance metrics $\mu(n;\va ,L)$ for the trellis can be computed 
recursively.

%--------------------------------------
\begin{theorem}
%--------------------------------------
Let a metric $\mu(n;\va ,L)$ be defined such that
\begin{eqnarray*}
   \Rhh(n) &=& 0 \mbox{ for } |n|>L.
\\
   \mu(n;\va ,L) 
      &\eqd& 2\sum_{k=-\infty}^n a_k \fdotr_k -\sum_{k=-\infty}^n a_k 
             \left[ 
                a_k \Rhh(0) +
                2\sum_{m=1}^L a_{m+k} \Rhh(m)
            \right]
\end{eqnarray*}

Then 
\formbox{
   \mu(n;\va ,L) = \mu(n-1;\va ,L) +
      2a_n \fdotr_n -a_n^2 \Rhh(0) - 2a_n\sum_{m=1}^L a_{m+n} \Rhh(m)
  }
\end{theorem}
\begin{proof}
\begin{eqnarray*}
   &&\mu(n;\va ,L) - \mu(n-1;\va ,L)
      \\&=& 
         \left(
             2\sum_{k=-\infty}^n a_k \fdotr_k -\sum_{k=-\infty}^n a_k 
             \left[ 
                a_k \Rhh(0) +
                2\sum_{m=1}^L a_{m+k} \Rhh(m)
            \right]
         \right) -
         \\&&
         \left(
             2\sum_{k=-\infty}^{n-1} a_k \fdotr_k -\sum_{k=-\infty}^{n-1} a_k 
             \left[ 
                a_k \Rhh(0) +
                2\sum_{m=1}^L a_{m+k} \Rhh(m)
            \right]
         \right)
      \\&=& 
             2a_n \fdotr_n 
             -a_n \left[ a_n \Rhh(0) + 2\sum_{m=1}^L a_{m+n} \Rhh(m)\right]
      \\&=& 2a_n \fdotr_n -a_n^2 \Rhh(0) - 2a_n\sum_{m=1}^L a_{m+n} \Rhh(m)
\end{eqnarray*}
\end{proof}

%--------------------------------------
\begin{example}
%--------------------------------------
Let $L=2$ in a binary $(M=2)$ communications channel.
Then 
\begin{eqnarray*}
   \mu(n;\va ,L) 
     &=& \mu(n-1;\va ,L) +
         2a_n \fdotr_n -a_n^2 \Rhh(0) - 2a_n\sum_{m=1}^L a_{m+n} \Rhh(m)
   \\&=& \mu(n-1;\va ,2) +
         2a_n \fdotr_n -a_n^2 \Rhh(0) 
       - 2a_n a_{n+1} \Rhh(1)
       - 2a_n a_{n+2} \Rhh(2)
\end{eqnarray*}


The metric $\mu(n;\va ,1)$ is controlled by three binary variables
$(a_{n-1},a_n,a_{n+1})$ and therefore the can be represented with 
an $2^{3-1}=4$ state trellis.
At each time interval $n$, each of the 8 path metrics in the set
\[ 
   \left\{ \mu(n;(a_n,a_{n+1},a_{n+2}),2): a_i\in\{-1,+1\} \right\}
\]
are computed and the ``shortest path" through the trellis is selected.
\end{example}





%======================================
\subsection{Minimum mean square estimate}
\label{sec:eq_mmse}
%======================================

\begin{figure}[ht] \color{figcolor}
\begin{center}
\begin{fsK}
\setlength{\unitlength}{0.15mm}                  
\begin{picture}(1000,200)(0,-80)  
  \thinlines                                      
  %\graphpaper[10](0,0)(700,100)                  

  \put(   0,   10 ){\makebox (  50, 40)[b]{$a_n$}        }
  \put(   0,    0 ){\vector  (   1,  0)   { 50}          }
  \put(  50,  -50 ){\framebox( 100,100)   {modulator}    }
  \put( 150,   10 ){\makebox (  50, 40)[b]{$\fs(t;\sa)$}       }
  \put( 150,    0 ){\vector  (   1,  0)   { 50}         }
  \put( 200,  -50 ){\framebox( 100,100)   {$\conv \fh(t)$} }
  \put( 200,  -40 ){\makebox ( 100, 80)[t]{$\Lh(s)$} }
  \put( 200,  -80 ){\makebox ( 100, 30)[ ]{$\Zh(z)$} }
  \put( 200,  -40 ){\makebox ( 100, 30)[t]{frequency}    }
  \put( 200,  -40 ){\makebox ( 100, 30)[b]{distortion}   }
  \put( 300,    0 ){\line    (   1,  0)   { 50}         }

  \put( 300,   10 ){\makebox (  50, 40)[b]{$w(t)$}       }
  \put( 350,    0 ){\vector  (   1,  0)   { 40}          }
  \put( 350,  -50 ){\dashbox{4}( 100,100)   {$+$}          }
  \put( 350,  -40 ){\makebox ( 100, 80)[b]{noise}        }
  \put( 350,  -50 ){\makebox ( 100, 95)[t]{$n(t)$}       }
  \put( 400,    0 ){\circle{20}                          }
  \put( 410,    0 ){\line    (   1,  0)   { 40}          }
  \put( 400,   30 ){\vector  (   0, -1)   { 20}          }
  \put( 450,    0 ){\line    (   1,  0)   { 50}         }

  \put( 500,  -50 ){\framebox( 100,100)   {$\conv \fh(-t)$} }
  \put( 500,  -40 ){\makebox ( 100, 80)[t]{$\hat{h}(-s)$} }
  \put( 500,  -80 ){\makebox ( 100, 30)[ ]{$\hat{h}(1/z)$} }
  \put( 500,  -40 ){\makebox ( 100, 30)[t]{matched}      }
  \put( 500,  -40 ){\makebox ( 100, 30)[b]{filter}       }
  \put( 600,    0 ){\vector  (   1,  0)   { 50}         }

  \put( 650,  -50 ){\dashbox{4}( 100,100)   {} }
  \put( 650,  -40 ){\makebox ( 100, 80)[b]{sampler}      }
  \put( 650,    0 ){\line    (   1,  0)   { 15}          }
  \put( 650,    0 ){\line    (   1,  0)   { 40}          }
  \put( 690,    0 ){\line    (   1,  1)   { 25}          }
  \put( 710,    0 ){\line    (   1,  0)   { 40}          }
  \put( 750,    0 ){\vector  (   1,  0)   {100}          }

  \put( 450,   10 ){\makebox (  50, 50)[b]{$r(t)$}       }
  \put(  40,  -80 ){\dashbox{5}( 720,160)   {}             }
  \put(  40,   90 ){\makebox ( 720, 30)[b]{equivalent digital system}}

  \put( 780,   10 ){\makebox ( 70, 40)[lb]{$\fdotr_n$}}
  \put( 850,  -50 ){\framebox( 100,100)   {} }
  \put( 850,  -40 ){\makebox( 100,80)[c]{MMSE} }
  \put( 850,  -40 ){\makebox( 100,80)[b]{estimate} }
  \put( 950,    0 ){\vector  (   1,  0)   {50}          }
\end{picture}                                   
\end{fsK}
\end{center}
\caption{
   Minimum Mean Square Estimate Implementation
   \label{fig:eq_mmse}
   }
\end{figure}

\prefpp{thm:eq-ml} guarantees that the sequence $(\fdotr_n)$
is a sufficient statistic for computing the ML estimate of information
sequence $(a_n)$.
Using $(\fdotr_n)$, Section~\ref{eq_trellis} shows that the ML estimate
can be computed using a trellis.
However, the trellis calculations can be very computationally demanding.
A simpler approach is to use minimum mean square estimation (MMSE).
MMSE can be computationally less demanding, but yields an estimate 
that is not equal to the ML estimate (MMSE is suboptimal).
Minimum mean square estimation is presented in 
Section~\ref{sec:est_mms} (page~\pageref{sec:est_mms}).
Let

\begin{tabular}{ll}
   $M$: estimate order  & ($M$ is odd)\\
   $N$: parameter order & ($N$ is odd). 
\end{tabular}

Then an estimate $\hat{\va}$ of the transmitted symbols can be calculated 
as follows.

\[
   \hat{\va} \eqd
   \left[\begin{array}{l}
      \hat{a}_{n-\frac{M-1}{2}} \\ 
      \vdots                    \\ 
      \hat{a}_{n-1}             \\
      \hat{a}_{n}               \\
      \hat{a}_{n+1}             \\
      \vdots                    \\ 
      \hat{a}_{n+\frac{M-1}{2}}  
   \end{array}\right]
   = U^H \vp
%
   \hspace{3cm}
%
   \vp \eqd
   \left[\begin{array}{l}
      \vp_{n-\frac{N-1}{2}} \\ 
      \vdots                    \\ 
      \vp_{n-1}             \\
      \vp_{n}               \\
      \vp_{n+1}             \\
      \vdots                    \\ 
      \vp_{n+\frac{N-1}{2}}  
   \end{array}\right]
\]

\[
   U^H \eqd
   \left[\begin{array}{llll}
      \fdotr_{n-\left(\frac{M-1}{2}\right)+\left(\frac{N-1}{2}\right)}    & \fdotr_{n-\left(\frac{M-1}{2}\right)+\left(\frac{N-1}{2}-1\right)}  & \cdots & \fdotr_{n-\left(\frac{M-1}{2}\right)-\left(\frac{N-1}{2}\right)}  \\
      \vdots                                                                & \vdots                                                                & \ddots & \vdots                                                                \\
      \fdotr_{n-(1)+\left(\frac{N-1}{2}\right)}                           & \fdotr_{n-(1)+\left(\frac{N-1}{2}-1\right)}                         & \cdots & \fdotr_{n-(1)-\left(\frac{N-1}{2}\right)}                         \\
      \fdotr_{n+(0)+\left(\frac{N-1}{2}\right)}                           & \fdotr_{n+(0)+\left(\frac{N-1}{2}-1\right)}                         & \cdots & \fdotr_{n+(0)-\left(\frac{N-1}{2}\right)}                         \\
      \fdotr_{n+(1)+\left(\frac{N-1}{2}\right)}                           & \fdotr_{n+(1)+\left(\frac{N-1}{2}-1\right)}                         & \cdots & \fdotr_{n+(1)-\left(\frac{N-1}{2}\right)}                         \\
      \vdots                                                                & \vdots                                                                & \ddots & \vdots                                                                \\
      \fdotr_{n+\left(\frac{M-1}{2}\right)+\left(\frac{N-1}{2}\right)}    & \fdotr_{n+\left(\frac{M-1}{2}\right)+\left(\frac{N-1}{2}-1\right)}  & \cdots & \fdotr_{n+\left(\frac{M-1}{2}\right)-\left(\frac{N-1}{2}\right)} 
   \end{array}\right]
\]


Let
\begin{eqnarray*}
   \hat{\va}(\vp)   &\eqd& U^H\vp    \\
   \ve(\vp)    &\eqd& \hat{\va}-\va \\
   \fCost(\vp) &\eqd& \E\norm{\ve}^2 \eqd \Eb{\ve^T\ve} \\
   \estMS      &\eqd& \arg\min_\vp \fCost(\vp)  \\
   R           &\eqd& \Eb{UU^H} \\
   W           &\eqd& \Eb{ U\vy}.
\end{eqnarray*}

Then 
\begin{eqnarray*}
   \fCost(\vp)           &=& \vp^H R \vp - (W^H\vp)^\ast -W^H\vp + \Eb{\va^H\va} \\
   \grad_\vp \fCost(\vp) &=& 2\Reb{R}\vp - 2\Re{W}  \\
   \estMS                  &=& (\Re{R})^{-1}(\Re{W})  \\
   \fCost(\estMS)        &=&    (\Re{W^H})(\Re{R})^{-1} R (\Re{R})^{-1}(\Re{W}) - 2(\Re{W^H})(\Re{R})^{-1}(\Re{W}) + \Eb{\va^H\va} \\
   \fCost(\estMS)|_{R\mbox{ real}} &=&    \Eb{\va^H\va} - (\Re{W^H})R^{-1}(\Re{W}).
\end{eqnarray*}



%======================================
\subsection{Minimum peak distortion estimate}
\label{sec:eq_pd}
%======================================
\begin{figure}[ht] \color{figcolor}
\begin{center}
\begin{fsK}
\setlength{\unitlength}{0.15mm}                  
\begin{picture}(1000,200)(0,-80)  
  \thinlines                                      
  %\graphpaper[10](0,0)(700,100)                  

  \put(   0,   10 ){\makebox (  50, 40)[b]{$a_n$}        }
  \put(   0,    0 ){\vector  (   1,  0)   { 50}          }
  \put(  50,  -50 ){\framebox( 100,100)   {modulator}    }
  \put( 150,   10 ){\makebox (  50, 40)[b]{$\fs(t;\sa)$}       }
  \put( 150,    0 ){\vector  (   1,  0)   { 50}         }
  \put( 200,  -50 ){\framebox( 100,100)   {$\conv \fh(t)$} }
  \put( 200,  -40 ){\makebox ( 100, 80)[t]{$\Lh(s)$} }
  \put( 200,  -80 ){\makebox ( 100, 30)[ ]{$\Zh(z)$} }
  \put( 200,  -40 ){\makebox ( 100, 30)[t]{frequency}    }
  \put( 200,  -40 ){\makebox ( 100, 30)[b]{distortion}   }
  \put( 300,    0 ){\line    (   1,  0)   { 50}         }

  \put( 300,   10 ){\makebox (  50, 40)[b]{$w(t)$}       }
  \put( 350,    0 ){\vector  (   1,  0)   { 40}          }
  \put( 350,  -50 ){\dashbox{4}( 100,100)   {$+$}          }
  \put( 350,  -40 ){\makebox ( 100, 80)[b]{noise}        }
  \put( 350,  -50 ){\makebox ( 100, 95)[t]{$n(t)$}       }
  \put( 400,    0 ){\circle{20}                          }
  \put( 410,    0 ){\line    (   1,  0)   { 40}          }
  \put( 400,   30 ){\vector  (   0, -1)   { 20}          }
  \put( 450,    0 ){\line    (   1,  0)   { 50}         }

  \put( 500,  -50 ){\framebox( 100,100)   {$\conv \fh(-t)$} }
  \put( 500,  -40 ){\makebox ( 100, 80)[t]{$\hat{h}(-s)$} }
  \put( 500,  -80 ){\makebox ( 100, 30)[ ]{$\hat{h}(1/z)$} }
  \put( 500,  -40 ){\makebox ( 100, 30)[t]{matched}      }
  \put( 500,  -40 ){\makebox ( 100, 30)[b]{filter}       }
  \put( 600,    0 ){\vector  (   1,  0)   { 50}         }

  \put( 650,  -50 ){\dashbox{4}( 100,100)   {} }
  \put( 650,  -40 ){\makebox ( 100, 80)[b]{sampler}      }
  \put( 650,    0 ){\line    (   1,  0)   { 15}          }
  \put( 650,    0 ){\line    (   1,  0)   { 40}          }
  \put( 690,    0 ){\line    (   1,  1)   { 25}          }
  \put( 710,    0 ){\line    (   1,  0)   { 40}          }
  \put( 750,    0 ){\vector  (   1,  0)   {100}          }

  \put( 450,   10 ){\makebox (  50, 50)[b]{$r(t)$}       }
  \put(  40,  -80 ){\dashbox{5}( 720,160)   {}             }
  \put(  40,   90 ){\makebox ( 720, 30)[b]{equivalent digital system}}

  \put( 780,   10 ){\makebox ( 70, 40)[lb]{$\fdotr_n$}}
  \put( 850,  -50 ){\framebox( 100,100)   {$\frac{1}{\Zh(z)\hat{h}(1/z)}$} }
  \put( 950,    0 ){\vector  (   1,  0)   {50}          }

\end{picture}                                   
\end{fsK}
\end{center}
\caption{
   Peak distortion estimation
   \label{fig:eq_pd}
   }
\end{figure}

Peak distortion is achieved when there is {\bf no} ISI.
This means that the impulse response of the channel and post-channel processing
must be only an impulse.
Ideally this can be achieved by filtering $\fdotr_n$ with the inverse of the
equivalent system digital filters.
See \prefpp{fig:eq_pd}.


