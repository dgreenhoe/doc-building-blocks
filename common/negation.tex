%============================================================================
% Daniel J. Greenhoe
% XeLaTeX file
%============================================================================

%=======================================
\chapter{Negation}
\label{chp:latn}
%=======================================

%\qboxnpq{%
%  \href{http://en.wikipedia.org/wiki/William_Stanley_Jevons}{William Stanley Jevons} 
%  \href{http://www-history.mcs.st-andrews.ac.uk/Timelines/TimelineF.html}{(1835--1882)}, 
%  \href{http://www-history.mcs.st-andrews.ac.uk/BirthplaceMaps/UK.html}{English} 
%  \href{http://en.wikipedia.org/wiki/List_of_economists}{economist} and 
%  \href{http://en.wikipedia.org/wiki/List_of_logicians}{logician}
%  \index{quotes!Jevons, William Stanley}
%  \footnotemark}
%  {../common/people/jevons.jpg}
%  {I cannot forget or omit to record this day last week. 
%    I was sleeping as usual for the night at St. Michael's Hamlet. 
%    As I awoke in the morning, the sun was shining brightly into my room. 
%    There was a consciousness on my mind that I was the discoverer 
%    of the true logic of the future. 
%    For a few minutes I felt a delight such as one can seldom hope to feel. 
%    But it would not last long---
%    I remembered only too soon how unworthy and weak an instrument 
%    I was for accomplishing so great a work, 
%    and how hardly could I expect to do it.}
%\citetblt{
%  image: & \url{http://www-history.mcs.st-andrews.ac.uk/PictDisplay/Jevons.html}\\
%  quote: & \citerpc{jevons1886}{219}{1866 March 28 entry}
%  }



\qboxns{Plato's the \hie{Sophist} (circa 360 B.C.) \footnotemark}{
  When we say not-being, we speak, I think, not of something that is the opposite of being, but only of something different.
  \ldots
  Then when we are told that the negative signifies the opposite, we shall not admit it; 
  we shall admit only that the particle ``not" indicates something different 
  from the words to which it is prefixed, 
  or rather from the things denoted by the words that follow the negative.
  }
\citetblt{
  \citerc{platosophist}{257b--257c},
  \citerp{horn2001}{5}
  }

%\qboxnpq
\qboxns
  {\href{http://en.wikipedia.org/wiki/Aristotle}{Aristotle} (384BC--322BC),
   Greek philosopher
   \footnotemark
  }
  %{../common/people/aristotle.jpg}
  {Clearly, then, it is a principle of this kind that is the most certain of all principles.\ldots 
   Let us next state what this principle is.
   ``It is impossible for the same attribute at once to belong and not to belong 
     to the same thing and in the same relation"; \ldots
   %and we must add any further qualifications that may be necessary to meet logical objections. 
   This is the most certain of all principles,\ldots
   %since it possesses the required definition;
   for it is impossible for anyone to suppose that the same thing is and is not,\ldots 
   %as some imagine that Heraclitus says1 --for what a man says does not necessarily represent what he believes.
   %And if it is impossible for contrary attributes to belong at the same time to the same subject\ldots 
   %(the usual qualifications must be added to this premiss also), 
   %and an opinion which contradicts another is contrary to it, 
   %then clearly it is impossible for the same man to suppose at the same time 
   %that the same thing is and is not;\ldots 
   %for the man who made this error would entertain two contrary opinions at the same time.
   %Hence all men who are demonstrating anything refer back to this as an ultimate belief; 
   %for 
   it is by nature the starting-point of all the other axioms as well.}
  \citetblt{
    \citorpu{aristotleMeta4}{4.1005b}{http://www.perseus.tufts.edu/cgi-bin/ptext?lookup=Aristot.+Met.+4.1005b}
    %image: & \url{http://upload.wikimedia.org/wikipedia/commons/9/98/Sanzio_01_Plato_Aristotle.jpg}
    }

%\qboxns{Josiah Royce (1913) \footnotemark}{
%The not-relation is one of the simplest and most 
%fundamental relations known to the human mind. For 
%the study of logic no more important and fruitful relation is known. 
%And none has a wider range of exemplifications in 
%the whole realm of the experience of the rational being.
%}
%\citetblt{
%  \citer{royce1913}\\
%  \citerp{royce1917}{265}\\%{\url{https://archive.org/stream/encyclopaediaofr09hastuoft\#page/264/}}\\
%  \citerp{royce1951}{182}\\
%  \citerp{horn2001}{1}
%  }
%
%\qboxns{F. H. Heinemann (1944) \footnotemark}{
%  There is first of all no agreement 
%  %in the different schools
%  \ldots
%  about the relation of negation to reality. % page 128
%  \ldots
%  In the face of the divergent theories
%  %of these four or five schools it must be stated that,
%  %notwithstanding a certain correspondence of views of all those who depend on Aristotle,
%  \ldots
%  no agreement exists as to the possibility of defining negation, as to its
%  logical status, function and meaning, 
%  as to its field of applicability, \ldots, 
%  and as to the interpretation of the negative judgment.% page 135
%  }
%\citetblt{
%  \citePp{heinemann1944}{128,135}\\
%  \citerp{horn2001}{1}
%  }

\begin{figure}
  \begin{center}
    \psset{xunit=12mm,yunit=18mm}\gsize%============================================================================
% Daniel J. Greenhoe
% LaTeX file
% lattice ({factors of 30}, |)
%============================================================================
  \begin{pspicture}(-6,-1.3)(6,4.5)%
     \footnotesize
     \psset{%
       cornersize=relative,
       framearc=0.25,
       subgriddiv=1,
       gridlabels=4pt,
       gridwidth=0.2pt,
       }%
     \begin{tabstr}{0.75}
     \rput( 0,4){\rnode{negsub}    {\psframebox{\begin{tabular}{c}\struct{subminimal negation}\ifnxref{negation}{def:negsub}\ifnxref{negation}{ex:negat_L3_subminimal}\end{tabular}}}}%
     \rput( 0,3){\rnode{negmin}    {\psframebox{\begin{tabular}{c}\struct{minimal negation}\ifnxref{negation}{def:negmin}\ifnxref{negation}{ex:negat_L3_minimal}\end{tabular}}}}%
     \rput( 2,2){\rnode{negfuz}  {\psframebox{\begin{tabular}{c}\struct{fuzzy negation}\ifnxref{negation}{def:negfuz}\ifnxref{negation}{ex:negat_m2_fuzzy}\end{tabular}}}}%
    %\rput( 4,1){\rnode{ucomp}     {\psframebox{\begin{tabular}{c}\struct{unique complement}\ifnxref{negation}{def:ucomp}\end{tabular}}}}%
    %\rput( 4,2){\rnode{complement}{\psframebox{\begin{tabular}{c}\struct{complement}\ifnxref{negation}{def:complement}\end{tabular}}}}%
    %\rput( 2,1){\rnode{negint}  {\psframebox{\begin{tabular}{c}\struct{intuitionalistic negation}\ifnxref{negation}{def:negint}\ifnxref{negation}{ex:negat_L3_int}\end{tabular}}}}%
     \rput( 2,1){\rnode{negint}  {\psframebox{\begin{tabular}{c}\struct{intuitionalistic negation}\ifnxref{negation}{def:negint}\ifnxref{negation}{ex:neg_discrete}\end{tabular}}}}%
     \rput(-2,2){\rnode{negdm}     {\psframebox{\begin{tabular}{c}\struct{de Morgan negation}\ifnxref{negation}{def:negdm}\ifnxref{negation}{ex:negat_m2_bn4}\end{tabular}}}}%
     \rput(-2,1){\rnode{negkl}     {\psframebox{\begin{tabular}{c}\struct{Kleene negation}\ifnxref{negation}{def:negkl}\ifnxref{negation}{ex:negat_L3_kl}\end{tabular}}}}%
     \rput( 0,0){\rnode{negor}     {\psframebox{\begin{tabular}{c}\struct{ortho negation}\ifnxref{negation}{def:negor}\ifnxref{negation}{ex:negat_m2_ortho}\end{tabular}}}}%
     \rput( 0,-1){\rnode{negorm}    {\psframebox{\begin{tabular}{c}\struct{orthomodular negation}\ifnxref{negation}{def:negorm}\end{tabular}}}}%
    %\rput( 0,0){\rnode{negorm}    {\psframebox{\begin{tabular}{c}\struct{orthomodular negation}\ifnxref{negation}{def:negorm}\ifnxref{negation}{ex:negat_m2_ortho}\end{tabular}}}}%
     \end{tabstr}
     %
     \psset{doubleline=true}%
     \ncline{<-}{negsub}{negmin}%
     \ncline{<-}{negmin}{negdm}%
     \ncline{<-}{negdm}{negkl}%
     \ncline{<-}{negkl}{negor}%
     %
     \ncline{<-}{negmin}{negfuz}%
     \ncline{<-}{negfuz}{negint}%
     \ncline{<-}{negint}{negor}%
     %
     \ncline{<-}{negor}{negorm}%
  \end{pspicture}

   %\includegraphics{graphics/latnegations.pdf}% warning: crosslinks (Def... page...) lost when compiling separately
  \end{center}
  \caption{lattice of negations\label{fig:latnegations}}
\end{figure}

%=======================================
\section{Definitions}
%=======================================
%---------------------------------------
\begin{definition}
\footnote{
  \citerppg{dunn1996}{4}{6}{3110876809},
  \citerppgc{dunn1999}{24}{26}{0792355695}{2 \scshape The Kite of Negations}
  }
\label{def:negsub}
%---------------------------------------
%Let $\intcc{\lzero}{\lid}$ be an \structe{interval} \xref{def:intcc} in $\opair{\R}{\orela}$.
Let $\latL\eqd\latbX$ be a \structe{bounded lattice} \xref{def:latb}.
\defbox{
  \begin{array}{>{\indentx}lclCD}
    \mc{5}{M}{A \structe{function} $\negat\in\clFxx$ is a \fnctd{subminimal negation} on $\latL$ if}\\
      x\orel y       &\implies& \negat{y} \orel \negat{x}   & \forall x,y\in\setX & (\prope{antitone}) \footnotemark
  \end{array}
  }
\end{definition}
\footnotetext{%
  The \prope{antitone} property may also be referred to as \prope{antitonic}, 
  \prope{order-reversing}, or \prope{contrapositive}.
  }

%---------------------------------------
\begin{remark}
\footnote{
  \citerpg{devidi2010}{511}{0080965016},
  \citerpg{devidi2006}{568}{0080442994}
  }
\label{rem:negsub}
%---------------------------------------
In the context of natural language, D. Devidi argues that, \fncte{subminimal negation} \xref{def:negsub} 
is ``difficult to take seriously as" a negation.
He essentially gives this example:
Let $x\eqd$``$p$ is a fish" and $y\eqd$``$p$ has gills".
Suppose ``$p$ is a fish" implies ``$p$ has gills" ($x\orel y$).
Now let $p\eqd$ ``many dogs".
Then the \prope{antitone} property and $x\orel y$ tells us ($\implies$) that
``Not many dogs have gills" implies that ``Not many dogs are fish".
\end{remark}


%---------------------------------------
\begin{definition}
\citetblp{
  \citerppg{dunn1996}{4}{6}{3110876809},
  \citerppgc{dunn1999}{24}{26}{0792355695}{2 \scshape The Kite of Negations},
 %\citerpg{devidi2010}{511}{0080965016}\\
 %\citerpg{devidi2006}{568}{0080442994}\\
  \citerpgc{troelstra1988}{4}{0080570887}{1.6 Intuitionism. (b)},
  \citePpc{devries2007}{11}{Definition 16},
  \citerpgc{gottwald1999}{21}{0792383885}{Definition 3.3},
  \citerpgc{novak1999}{50}{0792385950}{{\scshape Definition 2.26}},
  \citerppgc{nguyen2006}{98}{99}{1584885262}{5.4 Negations},
  \citePc{hohle1978}{???},
  \citePppc{bellman1973}{155}{156}{(N1) $\negat{0}=1$ and $\negat{1}=0$, (N3) $\negat\negat{x}=x$}
  }
\label{def:negat}
\label{def:negmin}
\label{def:negint}
\label{def:negfuz}
%---------------------------------------
%Let $\intcc{\lzero}{\lid}$ be an \structe{interval} \xref{def:intcc} in $\opair{\R}{\orela}$.
Let $\latL\eqd\latbX$ be a \structe{bounded lattice} \xref{def:latb}.
\defbox{
  \begin{array}{>{\indentx}FlclCDD}
    \mc{7}{M}{A \structe{function} $\negat\in\clFxx$ is a \fnctd{negation}, or \fnctd{minimal negation}, on $\latL$ if}
      \\1. & x\orel y       &\implies& \negat{y} \orel \negat{x}   & \forall x,y\in\setX & (\prope{antitone}) & and
      \\2. & x &\orel& \negat\negat{x}                     & \forall x\in\setX   & (\prope{weak double negation}). & %/\prope{constructive double negation}. & 
      \\
    \mc{7}{M}{A \fncte{minimal negation} $\negat$ is an \fnctd{intuitionistic negation} if}
      \\3. & x\meet \negat{x} &=& \lzero &\forall x,y\in\setX & (\prope{non-contradiction}).
      \\
    \mc{7}{M}{A \fncte{minimal negation} $\negat$ is a \fnctd{fuzzy negation} if}
     %\\\cntn & \negat\lzero  &=&        \lid                      &                     & (\prope{boundary condition}) & and
      \\4. & \negat\lid    &=&        \lzero                    &                     & (\prope{boundary condition}). &
  \end{array}
  }
\end{definition}
%\footnotetext{%
%  The \prope{antitone} property may also be referred to as \prope{antitonic}, 
%  \prope{order-reversing}, or \prope{contrapositive}.
%  }

%---------------------------------------
\begin{definition}
\footnote{
  %\citerppg{dunn1996}{4}{6}{3110876809}\\
  \citerppgc{dunn1999}{24}{26}{0792355695}{2 \scshape The Kite of Negations},
  \citerpg{jenei2003}{283}{1402015151},
  %\citerpg{stern1999}{?}{0521461057}\\
  %\citerpg{beran1985}{?}{902771715X}\\
  \citerpg{kalmbach1983}{22}{0123945801},
  \citerpg{lidl1998}{90}{0387982906},
  \citeP{husimi1937}
  }
\label{def:negdm}
\label{def:negkl}
\label{def:negor}
\label{def:negorm}
%---------------------------------------
%Let $\intcc{\lzero}{\lid}$ be an \structe{interval} \xref{def:intcc} in $\opair{\R}{\orela}$.
Let $\latL\eqd\latbX$ be a \structe{bounded lattice} \xref{def:latb}.
\defbox{
  \begin{array}{>{\indentx}FlclCDD}
    \mc{7}{M}{A \fncte{minimal negation} $\negat$ is a \fnctd{de Morgan negation} if}
      \\5. & x &=& \negat\negat{x}                         & \forall x\in\setX   & (\prope{involutory}). & 
      \\
    \mc{7}{M}{A \fncte{de Morgan negation} $\negat$ is a \fnctd{Kleene negation} if}
      \\6. & x \meet \negat{x}  &\orel& y\join \negat{y}           & \forall x,y\in\setX & (\prope{Kleene condition}).     & 
      \\
    \mc{7}{M}{A \fncte{de Morgan negation} $\negat$ is an \fnctd{ortho negation} if}
      \\7. & x\meet \negat{x} &=& \lzero &\forall x,y\in\setX & (\prope{non-contradiction}).
      \\
    \mc{7}{M}{A \fncte{de Morgan negation} $\negat$ is an \fnctd{orthomodular negation} if}
      \\8. & x\meet \negat{x} &=& \lzero                                      &\forall x,y\in\setX & (\prope{non-contradiction}) & and
      \\9. & x \orel y &\implies& x \join \brp{y\meet\negat{x}} = y           &\forall x,y\in\setX & (\prope{orthomodular}). &
    %\\
    %\mc{7}{M}{A \fncte{negation} $\negat$ is a \fnctd{strong negation} if}
    %  \cntn & x &=& \negat\negat{x}                         & \forall x\in\setX   & (\prope{involutive}) & 
    %\\
    %\mc{7}{M}{If $\negat$ is a \structe{minimal negation}, then $\latnX$ is a \structd{lattice with negation}.}
  \end{array}
  }
\end{definition}

%---------------------------------------
\begin{remark}
\footnote{
  \citerpg{cattaneo2009}{78}{3642032818}
  }
%---------------------------------------
The \prope{Kleene condition} is basically a weakened form of the \prope{non-contradiction} and 
\prope{excluded middle} properties because
\indentx$ \mcom{x\meet \negat{x}=\lzero}{\prope{non-contradiction}} \orel \mcom{\lid= y\join \negat{y}}{\prope{excluded middle}}$.
\end{remark}

%%---------------------------------------
%\begin{definition}
%\label{def:negorm}
%\label{def:negmorm}
%%---------------------------------------
%Let $\latL\eqd\latbX$ be a \structe{bounded lattice} \xref{def:latb}.
%\defbox{
%  \begin{array}{>{\indentx}FlclCDD}
%    \mc{7}{M}{An \fncte{ortho negation} $\negat$ is an \fnctd{orthomodular negation} if}
%      \cntn & x \orel y \quad\implies\quad x \join \brp{x^\ocop \meet y} = y &\forall x,y\in\setX & (\prope{orthomodular identity}). &
%    \\
%    \mc{7}{M}{A \fncte{an ortho negation} $\negat$ is a \fnctd{modular orthomodular negation} if}
%      \cntn & x \orel y \quad\implies\quad x \join \brp{x^\ocop \meet y} = y       &\forall x,y\in\setX    & (\prope{orthomodular}) & and
%      \cntn & a \orel y \quad\implies\quad y \meet(x\join a) = (y\meet x) \join a  & \forall x,y,a\in\setX & (\prope{modular}).     &
%  \end{array}
%  }
%\end{definition}


%---------------------------------------
\begin{definition}
\citetblp{
  \citeIppg{fodor2000}{127}{128}{079237732X},
  \citeP{bellman1973}
  }
\label{def:neg_strong}
%---------------------------------------
\defbox{
  \begin{array}{>{\indentx}FlclCDD}
    \mc{7}{M}{A \structe{minimal negation} $\negat\in\clFxx$ is \propd{strict} ($\negat$ is a \fnctd{strict negation}) if}
      \\1. & x\lneq y       &\implies& \negat{y} \lneq \negat{x}   & \forall x,y\in\setX & (\prope{strictly antitone}) & and
      \\2. & \mc{5}{M}{$\negat$ is \prope{continuous}}
      \\
    \mc{7}{M}{A \fncte{strict negation} $\negat$ is \propd{strong} ($\negat$ is a \fnctd{strong negation}) if}
      \\3. & \negat\negat{x} &=& x &\forall x\in\setX & (\prope{involutory}).
  \end{array}
  }
\end{definition}

%---------------------------------------
\begin{definition}
\label{def:latn}
%\label{def:latfuz}
%\label{def:latdem}
%\label{def:latkl}
%\label{def:lator}
%\label{def:latorm}
%---------------------------------------
Let $\latL\eqd\latnX$ be a \structe{bounded lattice} \xref{def:latb} with a function $\negat$ in $\clFxx$.
%Let $\negat$ be a function on a \structe{bounded lattice} $\latbX$ \xref{def:latb}.
%Let $\latL\eqd\latnX$.
\defboxt{
  %\begin{array}{MM}
    If $\negat$ is a  \fncte{minimal negation},         then $\latL$ is a  \structd{lattice with negation}.
    %\cntn & If $\negat$ is an \fncte{intuitionistic negation}, & then $\latL$ is an \structd{intuitionistic lattice}.
    %\cntn & If $\negat$ is an \fncte{fuzzy negation},          & then $\latL$ is a  \structd{fuzzy lattice}.
    %\cntn & If $\negat$ is a  \fncte{de Morgan negation},      & then $\latL$ is a  \structd{de Morgan lattice}.
    %\cntn & If $\negat$ is a  \fncte{Kleene negation},         & then $\latL$ is a  \structd{Kleene lattice}.
    %\cntn & If $\negat$ is an \fncte{ortho negation},          & then $\latL$ is an \structd{ortho lattice}, 
    %\\    &                                                    & \qquad or an \structd{orthocomplemented lattice}.\footnotemark
    %\cntn & If $\negat$ is an \fncte{orthomodular negation},   & then $\latL$ is an \structd{orthomodular lattice}.
  %\end{array}
  }
\end{definition}
%\footnotetext{
%  In the literature, the standard name for this structure is \structe{orthocomplemented lattice}, and 
%  \structe{ortho lattice} is sometimes used as an abbreviation.
%  However, it can be argued that the more natural name for the structure is \structe{ortho lattice},
%  because of its direct connection to the \fncte{ortho negation} function \xref{def:negor} and because 
%  the concepts of a \structe{complemented lattice} and a \structe{lattice with negation} are 
%  fundamentally different \xref{rem:cmpneg}.
%  }



%=======================================
\section{Properties of negations}
%=======================================
%---------------------------------------
\begin{lemma}
\footnote{
  \citerpgc{beran1985}{31}{902771715X}{Theorem 1.2 Proof},
  \citePpc{fay1967}{268}{Lemma 1 Proof},
  \citePpc{devries2007}{12}{Theorem 18}
  }
\label{lem:latn_demorgan_ineq}
%---------------------------------------
Let $\negat\in\clFxx$ be a function on a \structe{bounded lattice} $\latL\eqd\latbX$ \xref{def:latb}.
\lembox{
  \mcom{\brbr{x\orel y \implies \negat{y}\orel \negat{x}}}{\prope{antitone}}
  \implies
  {\brbl{\begin{array}{rclCDD}
    \negat{x} \join \negat{y} &\orel& \negat(x\meet y)        & \forall x,y\in\setX & (\prope{conjunctive de Morgan ineq.}) & and \\
    \negat(x\join y)          &\orel&  \negat{x} \meet \negat{y} & \forall x,y\in\setX & (\prope{disjunctive de Morgan ineq.}) & and \\
  \end{array}}}%{\prope{de Morgan inequalities}}
  }
\end{lemma}
\begin{proof}
\begin{enumerate}
  %\item Proof that \prope{antitone} $\implies$ $\brb{\negat\lzero=\lid}$: \label{item:latn_demorgan_0n1}
  %\item Proof that \prope{antitone} $\implies$ $\brb{\negat\lid=\lzero}$: \label{item:latn_demorgan_1n0}
  \item Proof that \prope{antitone} $\implies$ \prope{conjunctive de Morgan}: \label{item:latn_demorgan_xmeety}
    \begin{align*}
      &\text{$x\meet y\le x$ and $x\meet y\le y$}
        && \text{by definition of $\meet$}
      \\&\implies \text{$\negat(x\meet y)\ge \negat{x}$ and $\negat(x\meet y)\ge \negat{y}$}
        && \text{by \prope{antitone}}
      \\&\implies \negat(x\meet y)\ge \negat{x} \join \negat{y}
        && \text{by definition of $\join$}
    \end{align*}

  \item Proof that \prope{antitone} $\implies$ \prope{disjunctive de Morgan}: \label{item:latn_demorgan_xjoiny}
    \begin{align*}
      &\text{$x\le x\join y$ and $y\le x\join y$}
        && \text{by definition of $\join$}
      \\&\implies \text{$\negat{x}\ge\negat(x\join y)$ and $\negat{y}\ge\negat(x\join y)$}
        && \text{by \prope{antitone}}
      \\&\implies \negat{x} \meet \negat{y} \ge \negat(x\join y)
        && \text{by definition of $\meet$}
      \\&\implies \negat(x\join y) \le \negat{x} \meet \negat{y}
    \end{align*}
\end{enumerate}
\end{proof}    

%---------------------------------------
\begin{lemma}
\footnote{
  \citerppgc{beran1985}{30}{31}{902771715X}{Theorem 1.2},
  \citePpc{fay1967}{268}{Lemma 1},
  \citePc{nakano1971}{cf Beran 1985} %{cf Beran 1985 page 30}
  }
\label{lem:latn_demorgan}
%---------------------------------------
Let $\negat\in\clFxx$ be a function on a \structe{lattice} $\latL\eqd\latX$ \xref{def:lattice}.
\lemboxt{
  If $x=(\negat\negat{x})$ for all $x\in\setX$ (\prope{involutory}), then
  \\%\indentx
  $\mcom{\brbr{x\orel y \implies \negat{y}\orel \negat{x}}}{\prope{antitone}}
  \iff
  \mcom{\brbl{\begin{array}{rclCDD}
    \negat(x\join y) &=& \negat{x} \meet \negat{y} & \forall x,y\in\setX & (\prope{disjunctive de Morgan}) & and \\
    \negat(x\meet y) &=& \negat{x} \join \negat{y} & \forall x,y\in\setX & (\prope{conjunctive de Morgan})
  \end{array}}}{\prope{de Morgan}}$
  }
\end{lemma}
\begin{proof}
\begin{enumerate}
  \item Proof that \prope{antitone} $\implies$ \prope{de Morgan} equalities: 
    \begin{enumerate}
      %\item Proof that $\negat(x\meet y) \ge \negat{x} \join \negat{y}$: \label{item:latn_demorgan_xmeety}
      %  \begin{align*}
      %    &\text{$x\meet y\le x$ and $x\meet y\le y$}
      %      && \text{by definition of $\meet$}
      %    \\&\implies \text{$\negat(x\meet y)\ge \negat{x}$ and $\negat(x\meet y)\ge \negat{y}$}
      %      && \text{by \prope{antitone}}
      %    \\&\implies \negat(x\meet y)\ge \negat{x} \join \negat{y}
      %      && \text{by definition of $\join$}
      %  \end{align*}
      %
      %\item Proof that $\negat(x\join y) \le \negat{x} \meet \negat{y}$: \label{item:latn_demorgan_xjoiny}
      %  \begin{align*}
      %    &\text{$x\le x\join y$ and $y\le x\join y$}
      %      && \text{by definition of $\join$}
      %    \\&\implies \text{$\negat{x}\ge\negat(x\join y) $ and $\negat{y}\ge\negat(x\join y)$}
      %      && \text{by \prope{antitone}}
      %    \\&\implies \negat{x} \meet \negat{y} \ge \negat(x\join y)
      %      && \text{by definition of $\meet$}
      %    \\&\implies \negat(x\join y) \le \negat{x} \meet \negat{y}
      %  \end{align*}
    
      \item Proof that $\negat(\negat{x}\meet \negat{y}) \ge x \join y$: \label{item:latn_demorgan_xomeetyo}
        \begin{align*}
          \negat(\negat{x}\meet \negat{y})
            &\ge \negat\negat{x} \join \negat\negat{y}
            &&   \text{by \pref{lem:latn_demorgan_ineq}}
          \\&=   x \join y
            &&   \text{by \prope{involutory} property \xref{def:latn}}
        \end{align*}
    
      \item Proof that $\negat(\negat{x}\join \negat{y}) \le x \meet y$: \label{item:latn_demorgan_xojoinyo}
        \begin{align*}
          \negat(\negat{x}\join \negat{y})
            &\le \negat\negat{x} \meet \negat\negat{y}
            &&   \text{by \pref{lem:latn_demorgan_ineq}}
          \\&=   x \meet y
            &&   \text{by \prope{involutory} property \xref{def:latn}}
        \end{align*}
    
      \item Proof that $\negat(x\meet y) = \negat{x} \join \negat{y}$:
        \begin{align*}
          \negat(x\meet y)
            &\ge \negat{x} \join \negat{y}
            &&   \text{by \pref{lem:latn_demorgan_ineq}}
            \\
          \negat(x\meet y)
            &=   \negat\brs{\negat\negat{x} \meet \negat\negat{y}}
            &&   \text{by \prope{involutory} property \xref{def:latn}}
          \\&\le \negat{x} \join \negat{y}
            &&   \text{by \pref{item:latn_demorgan_xojoinyo}}
        \end{align*}
    
      \item Proof that $\negat(x\join y) = \negat{x} \meet \negat{y}$:
        \begin{align*}
          \negat(x\join y)
            &\ge \negat{x} \meet \negat{y}
            &&   \text{by \pref{lem:latn_demorgan_ineq}}
            \\
          \negat(x\join y)
            &=   \negat\brs{\negat\negat{x} \join \negat\negat{y}}
            &&   \text{by \prope{involutory} property \xref{def:latn}}
          \\&\le \negat{x} \meet \negat{y}
            &&   \text{by \pref{item:latn_demorgan_xomeetyo}}
        \end{align*}
    
    \end{enumerate}

  \item Proof that \prope{antitone} $\impliedby$ \prope{de Morgan}:
    \begin{align*}
      x\le y \implies \negat{y}
        &= \negat\brp{x \join y}
        && \text{because $x\le y$}
      \\&= \negat{x} \meet \negat{y}
        && \text{by \prope{de Morgan}}
      \\&\le \negat{x}
        && \text{by definition of $\meet$}
    \end{align*}
\end{enumerate}
\end{proof}

%---------------------------------------
\begin{lemma}
\label{lem:latn_boundaries}
%---------------------------------------
Let $\negat\in\clFxx$ be a function on a \structe{lattice} $\latL\eqd\latX$ \xref{def:lattice}.
\lembox{
  \brb{\begin{array}{FlCDD}
    1. & x\orel \negat\negat{x}  & \forall x\in\setX & (\prope{weak double negation}) & and \\
    2. & \negat\lid = \lzero     &                   & (\prope{boundary condition})    & 
  \end{array}}
  \implies
  \brb{\begin{array}{rclD}
    \negat\lzero &=& \lid & (\prope{boundary condition})
  \end{array}}
  }
\end{lemma}
\begin{proof}
  \begin{align*}
    \negat\lzero
      &= \negat\negat\lid
      && \text{by \prope{boundary condition} hypothesis (2)}
    \\&\oreld \lid
      && \text{by \prope{weak double negation} hypothesis (1)}
    \\&\implies \negat\lzero=\lid
      && \text{by \prope{upper bound} property \xref{def:latb}}
  \end{align*}
\end{proof}


%---------------------------------------
\begin{lemma}
\label{lem:latn_noncon}
%---------------------------------------
Let $\negat\in\clFxx$ be a function on a \structe{lattice} $\latL\eqd\latX$ \xref{def:lattice}.
\lembox{
  \brb{\begin{array}{rclCDD}
    (x\meet \negat{x} &=& \lzero & \forall x\in\setX & (\prope{non-contradiction}) & 
  \end{array}}
  \implies
  \brb{\begin{array}{rclCDD}
    \negat\lid &=& \lzero  &  & (\prope{boundary condition}) & 
  \end{array}}
  }
\end{lemma}
\begin{proof}
  \begin{align*}
    \lzero
      &= \lid \meet \negat\lid
      && \text{by \prope{non-contradiction} hypothesis}
    \\&= \negat\lid
      && \text{by definition of g.u.b. $\lid$ and $\meet$}
  \end{align*}
\end{proof}

%---------------------------------------
\begin{lemma}
\footnote{
  \citerpg{varadarajan1985}{42}{0387493867}
  }
\label{lem:latn_biant_bnd}
%---------------------------------------
Let $\negat\in\clFxx$ be a function on a \structe{bounded lattice} $\latL\eqd\latbX$ \xref{def:latb}.
\lembox{
  \brb{\begin{array}{FMCD}
    (A). & $\negat$ is \prope{bijective}  & & and \\
    (B). & $x\orel y \implies \negat{y}\orel \negat{x}$ &\forall x,y\in\setX & (\prope{antitone})
  \end{array}}
  \implies
  \mcom{\brb{\begin{array}{FrclD}
    (1). & \negat\lzero           &=&     \lid     & and \\
    (2). & \negat\lid             &=&     \lzero   & 
  \end{array}}}{\prope{boundary conditions}}
  }
\end{lemma}
\begin{proof}
\begin{enumerate}
  \item Proof that $\negat\lzero=\lid$: 
    \begin{align*}
        &  x \orel \lid 
        && \forall x\in\setX 
        && \text{by definition of l.u.b. $\lid$}
      \\\implies 
        & \negat\lid \orel \negat{x} 
        && \forall x\in\setX 
        && \text{by \prope{antitone} hypothesis}
      \\\implies 
        &  \negat\lid \orel y 
        && \forall y\in\setX 
        && \text{by \prope{bijective} hypothesis}
      \\\implies 
        &  \negat\lid = \lzero
        && 
        && \text{by definition of g.l.b. $\lzero$}
    \end{align*}

  \item Proof that $\negat\lzero=\lid$: 
    \begin{align*}
        &  \lzero \orel x 
        && \forall x\in\setX 
        && \text{by definition of g.l.b. $\lzero$}
      \\\implies 
        &  \negat{x} \orel \negat\lzero 
        && \forall x\in\setX 
        && \text{by \prope{antitone} hypothesis}
      \\\implies 
        &  \negat{x} \orel y 
        && \forall y\in\setX 
        && \text{by \prope{bijective} hypothesis}
      \\\implies 
        &  \negat\lzero = \lid
        && 
        && \text{by definition of l.u.b. $\lid$}
    \end{align*}
\end{enumerate}
\end{proof}



%---------------------------------------
\begin{theorem}
\label{thm:latn_intuitionistic}
%---------------------------------------
Let $\negat\in\clFxx$ be a function on a \structe{bounded lattice} $\latL\eqd\latbX$ \xref{def:latb}.
\thmbox{
  \brb{\begin{array}{M}
    $\negat$ is an\\
    \prope{intuitionistic negation}
  \end{array}}
  \implies
  \brb{\begin{array}{rclCDD}
    \negat\lid &=& \lzero  &  & (\prope{boundary condition}) & 
  \end{array}}
  }
\end{theorem}
\begin{proofns}
This follows directly from \prefpp{def:latn} and \prefpp{lem:latn_noncon}.
\end{proofns}

%---------------------------------------
\begin{theorem}
\label{thm:latn_fuzzy}
%---------------------------------------
Let $\negat\in\clFxx$ be a function on a \structe{bounded lattice} $\latL\eqd\latbX$ \xref{def:latb}.
\thmbox{
  \brb{\begin{array}{M}
    $\negat$ is a\\
    \fncte{fuzzy negation}
  \end{array}}
  \implies
  {\brb{\begin{array}{rclD}
    \negat\lzero &=& \lid     & (\prope{boundary condition}) 
  \end{array}}}
  }
\end{theorem}
\begin{proofns}
This follows directly from \prefpp{def:negfuz} and \prefpp{lem:latn_boundaries}.
\end{proofns}

%---------------------------------------
\begin{theorem}
\label{thm:latn_demorganineq}
%---------------------------------------
%Let $\latL\eqd\latnX$ be a \structe{lattice with negation} \xref{def:latn}.
Let $\negat\in\clFxx$ be a function on a \structe{bounded lattice} $\latL\eqd\latbX$ \xref{def:latb}.
\thmbox{
  \brb{\begin{array}{M}
    $\negat$ is a\\
    minimal\\ 
    negation
  \end{array}}
  \implies
  {\brb{\begin{array}{rclCDD}
    \negat{x} \join \negat{y} &\orel& \negat(x\meet y)        & \forall x,y\in\setX & (\prope{conjunctive de Morgan inequality}) & and \\
    \negat(x\join y)       &\orel&  \negat{x} \meet \negat{y} & \forall x,y\in\setX & (\prope{disjunctive de Morgan inequality}) & 
  \end{array}}}%{\prope{de Morgan inequalities}}
  }
\end{theorem}
\begin{proofns}
This follows directly from \prefpp{def:latn} and \prefpp{lem:latn_demorgan_ineq}.
\end{proofns}

%---------------------------------------
\begin{theorem}
\label{thm:latn_demorgan}
%---------------------------------------
%Let $\latL\eqd\latnX$ be a \structe{lattice with negation} \xref{def:latn}.
Let $\negat\in\clFxx$ be a function on a \structe{bounded lattice} $\latL\eqd\latbX$ \xref{def:latb}.
\thmbox{%
  \brbr{\begin{array}{M}
    $\negat$ is a\\
    de Morgan 
    negation
  \end{array}}
  \implies
  \brbl{\begin{array}{rclCDD}
    \negat(x\join y) &=&     \negat{x} \meet \negat{y} & \forall x,y\in\setX & (\prope{disjunctive de Morgan}) & and \\
    \negat(x\meet y) &=&     \negat{x} \join \negat{y} & \forall x,y\in\setX & (\prope{conjunctive de Morgan})
  \end{array}}
  }
\end{theorem}
\begin{proofns}
  This follows directly from \prefpp{def:latn} and \prefpp{lem:latn_demorgan}.
\end{proofns}

%---------------------------------------
\begin{theorem}
\footnote{
  \citerppg{beran1985}{30}{31}{902771715X},
  \citePpc{birkhoffjvn1936}{830}{L74},
  \citerpgc{cohen1989}{37}{1461388430}{3B.13. Theorem}
  }
\label{thm:latn_ortho}
%---------------------------------------
Let $\negat\in\clFxx$ be a function on a \structe{bounded lattice} $\latL\eqd\latbX$ \xref{def:latb}.
\thmbox{
  \brb{\begin{array}{M}
    $\negat$ is an\\
    ortho
    negation
  \end{array}}
  \implies
  \brb{\begin{array}{FrclCDD}
    1. & \negat\lzero     &=& \lid                    &                     & (\prope{boundary condition})    & and \\
    2. & \negat\lid       &=& \lzero                  &                     & (\prope{boundary condition})    & and \\
    3. & \negat(x\join y) &=& \negat{x} \meet \negat{y} & \forall x,y\in\setX & (\prope{disjunctive de Morgan}) & and \\
    4. & \negat(x\meet y) &=& \negat{x} \join \negat{y} & \forall x,y\in\setX & (\prope{conjunctive de Morgan}) & and \\
    5. & x \join \negat{x}  &=& \lid                    & \forall x\in\setX   & (\prope{excluded middle})       & and \\
    6. & x \meet \negat{x}  &\orel& y\join \negat{y}     & \forall x,y\in\setX & (\prope{Kleene condition}).     & 
  \end{array}}
  }
\end{theorem}
\begin{proof}
\begin{enumerate}
  \item Proof for $\lzero = \negat\lid$ boundary condition: by \prefpp{lem:latn_noncon}\label{item:latn_ortho_bc0}
      
  \item Proof for boundary conditions:\label{item:latn_ortho_bc1}
    \begin{align*}
      \lid
        &= \negat\negat\lid
        && \text{by \prope{involutory} property}
      \\&= \negat\lzero
        && \text{by previous result}
    \end{align*}

  \item Proof for \prope{de Morgan} properties: 
    \begin{enumerate}
      \item By \prefpp{def:latn}, \fncte{ortho negation} is \prope{involutory} and \prope{antitone}.
      \item Therefore by \prefpp{lem:latn_demorgan}, \prope{de Morgan} properties hold.
    \end{enumerate}

  \item Proof for \prope{excluded middle} property: \label{item:latn_ortho_exm}
    \begin{align*}
      x \join \negat{x}
        &= \brp{x \join \negat{x}}^{\negat\negat}
        && \text{by \prope{involutory} property of \fncte{ortho negation} \xref{def:latn}}
      \\&= \negat\brp{x\negat \meet x^{\negat\negat}}
        && \text{by \prope{disjunctive de Morgan} property}
      \\&= \negat\brp{\negat{x} \meet x}
        && \text{by \prope{involutory} property of \fncte{ortho negation} \xref{def:latn}}
      \\&= \negat\brp{x \meet \negat{x}}
        && \text{by \prope{commutative} property of \structe{lattice}s \xref{def:lattice}}
      \\&= \negat\lzero
        && \text{by \prope{non-contradiction} property of \fncte{ortho negation} \xref{def:latn}}
      \\&= \lid
        && \text{by \prope{boundary condition} \xref{item:latn_ortho_bc1} of \fncte{minimal negation}}
    \end{align*}

  \item Proof for \prope{Kleene condition}:
    \begin{align*}
      x\meet \negat{x}
        &= \lzero
        && \text{by \prope{non-contradiction} property \xref{def:latn}}
      \\&\orel \lid
        && \text{by definition of $\lzero$ and $\lid$}
      \\&= y \join \negat{y}
        && \text{by \prope{excluded middle} property \xref{item:latn_ortho_exm}}
    \end{align*}
\end{enumerate}
\end{proof}


%\ifexclude{mssa}{
%=======================================
\section{Examples}
%=======================================
%---------------------------------------
\begin{example}[\exmd{discrete negation}]
\footnote{
  \citerpg{fodor2000}{128}{079237732X},
  \citePpp{yager1980}{256}{257},
  \citePc{yager1979}{cf Fodor}
  }
\label{ex:neg_discrete}
\label{ex:dneg}
%---------------------------------------
Let $\latL\eqd\latnX$ be a \structe{bounded lattice} \xref{def:latb} with a function $\negat\in\clFxx$.
\exboxt{
  The function $\negat{x}$ defined as
  \\\indentx$\negat{x} \eqd \brbl{\begin{array}{cM}
    \lid   & for $x=\lzero$ \\
    \lzero & otherwise
  \end{array}}$\\
is an \fncte{intuitionistic negation} \xref{def:negint} and a \fncte{fuzzy negation} \xref{def:negfuz}.
  }
\end{example}
\begin{proof}
To be an \fncte{intuitionistic negation}, $\negat{x}$ must be \prope{antitone}, have \prope{weak double negation}, 
and have the \prope{non-contradiction} property \xref{def:negint}.
To be a \fncte{fuzzy negation}, $\negat{x}$ must be \prope{antitone}, have \prope{weak double negation}, 
and have the \prope{boundary condition} $\negat\lid=\lzero$.
\begin{align*}
    \brb{\begin{array}{lclMM}
      \negat{y} \le \negat{x} &\iff& \lid\le\lid      & for $0=    x=     y$\\     
      \negat{y} \le \negat{x} &\iff& \lzero\le\lid    & for $0=    x\lneq y$\\ 
      \negat{y} \le \negat{x} &\iff& \lzero\le\lzero  & for $0\neq x\orel y$
    \end{array}}
  &\implies \text{$\negat{x}$ is \prope{antitone}}
  \\
  \brb{\begin{array}{lclclclclM}
      \negat\negat{x} &=& \negat\lid   &=& \lzero &\oreld& \lzero &=& x & for $x=\lzero$    \\
      \negat\negat{x} &=& \negat\lzero &=& \lid   &\oreld& x      &=& x & for $x\neq\lzero$ 
    \end{array}}
  &\implies \text{$\negat{x}$ has \prope{weak double negation}}
  \\
  \brb{\begin{array}{lclclclM}
      x\meet\negat{x} &=& x\meet\lid   &=& \lzero\meet\lzero &=& \lzero & for $x=\lzero$    \\
      x\meet\negat{x} &=& x\meet\lzero &=& x     \meet\lzero &=& \lzero & for $x\neq\lzero$ 
    \end{array}}
  &\implies \text{$\negat{x}$ has \prope{non-contradiction} property}
  \\
  \negat\lid = \lzero
  &\implies \text{$\negat{x}$ has the \prope{boundary condition} property}
\end{align*}
\end{proof}

%---------------------------------------
\begin{example}[\exmd{dual discrete negation}]
\footnote{
  \citerpg{fodor2000}{128}{079237732X},
  \citePpc{ovchinnikov1983}{235}{Example 4}
  }
\label{ex:neg_dualdiscrete}
%---------------------------------------
Let $\latL\eqd\latnX$ be a \structe{bounded lattice} \xref{def:latb} with a function $\negat\in\clFxx$.
\exboxp{
  The function $\negat{x}$ defined as
  \\\indentx$\negat{x} \eqd \brbl{\begin{array}{cM}
    \lzero   & for $x=\lid$ \\
    \lid     & otherwise
  \end{array}}$\\
is a \fncte{subminimal negation} \xref{def:negsub} but it is \emph{not} a \fncte{minimal negation} \xref{def:negat}
(and not any other negation defined here).
  }
\end{example}
\begin{proof}
To be an \fncte{subminimal negation}, $\negat{x}$ must be \prope{antitone} \xref{def:negsub}.
To be a \fncte{minimal negation}, $\negat{x}$ must be \prope{antitone} and have \prope{weak double negation} \xref{def:negat}. 
\begin{align*}
    \brb{\begin{array}{lclM}
      \negat{y} \le \negat{x} &\iff& \lzero\le\lzero  & for $x=     y=1$\\     
      \negat{y} \le \negat{x} &\iff& \lzero\le\lid    & for $x\lneq y=1$\\ 
      \negat{y} \le \negat{x} &\iff& \lid\le\lid      & for $x\orel y\neq1$
    \end{array}}
  &\implies \text{$\negat{x}$ is \prope{antitone}}
  \\
  \brb{\begin{array}{lclclclM}
      \negat\negat{x} &=& \negat\lzero   &=& \lid   &\oreld & x  & for $x=\lid$    \\
      \negat\negat{x} &=& \negat\lid     &=& \lzero &\orel  & x  & for $x\neq\lid$ 
    \end{array}}
  &\implies \text{$\negat{x}$ does \emph{not} have \prope{weak double negation}}
\end{align*}
\end{proof}

%---------------------------------------
\begin{example}
\footnote{
  \citerpg{fodor2000}{128}{079237732X}
  }
%---------------------------------------
Let $\latL\eqd\latbX$ be a \structe{bounded lattice}
\exboxt{
  The function $\negat{x}$ is an \fncte{intuitionistic negation} \xref{def:negint} if
  \\\indentx$\negat{x} \eqd \brbl{\begin{array}{cM}
    \lid   & for $x=\lzero$ \\
    \lzero & otherwise
  \end{array}}$
  }
\end{example}

%---------------------------------------
\begin{example}
\label{ex:negat_L2}
%---------------------------------------
\exbox{
  \begin{array}{m{\tw-36mm}}
    The function $\negat$ illustrated to the right is an \fncte{ortho negation} \xref{def:negor}. 
  \end{array}
  \begin{array}{M}
  \psset{yunit=5mm}%
  \includegraphics{graphics/lat2_L2_10_-0-1.pdf}
  \end{array}
  }
\end{example}
\begin{proof}
\begin{enumerate}
  \item Proof that $\negat$ is \prope{antitone}:
    $\begin{array}[t]{rcl c lcl c lclc M}
      \lzero &\orel& \lid &\implies& \negat\lid &=& \lzero &\orel& x      &=& \negat\lzero &\implies& {$\negat$ is \prope{antitone} over $\opair{\lzero}{\lid}$}\\
    \end{array}$

  \item Proof that $\negat$ is \prope{involutory}:
    $\lid = \negat\lzero = \negat\negat\lid$

  \item Proof that $\negat$ has the \prope{non-contradiction} property:
    $\begin{array}[t]{rcl c lcl c lclc M}
      \lid   &\meet& \negat\lid   &=& \lid   &\meet& \lzero &=& \lzero \\
      \lzero &\meet& \negat\lzero &=& \lzero &\meet& \lid   &=& \lzero 
    \end{array}$
\end{enumerate}
\end{proof}

%---------------------------------------
\begin{example}
\label{ex:negat_L3_nonanitone}
%---------------------------------------
\exbox{
  \begin{array}{m{\tw-90mm}}
    The functions $\negat$ illustrated to the right are \emph{not} any negation defined here.
    In particular, they are \prope{not antitone}.
  \end{array}
  \begin{tabular}{*{3}{>{\psset{yunit=5mm}}c}}
  % \includegraphics{graphics/lat3_L3_1=-1_x=-x_0=-0.pdf}
  %&\includegraphics{graphics/lat3_L3_1=-0_x=-1_0=-x.pdf}
  %&\includegraphics{graphics/lat3_L3_1=-x_x=-0_0=-1.pdf}
   \includegraphics{graphics/lat3_L3_1-1_x-x_0-0.pdf}
  &\includegraphics{graphics/lat3_L3_1-0_x-1_0-x.pdf}
  &\includegraphics{graphics/lat3_L3_1-x_x-0_0-1.pdf}
  \\\mbox{}\quad\scs(a)&\mbox{}\quad\scs(b)&\mbox{}\quad\scs(c)
  \end{tabular}
  }
\end{example}
\begin{proof}
\begin{enume}
  \item Proof that (a) is \prope{not antitone}:
      $a \orel \lid \implies \negat\lid = \lid \nleq a = \negat a$

  \item Proof that (b) is \prope{not antitone}:
      $a \orel \lid \implies \negat\lid = a \nleq \lzero = \negat a$

  \item Proof that (c) is \prope{not antitone}:
      $\lzero \orel a \implies \negat a = \lid \nleq a = \negat\lzero$
\end{enume}
\end{proof}

%---------------------------------------
\begin{example}
\label{ex:negat_o6_nonanitone}
%---------------------------------------
\exbox{
  \begin{array}{m{\tw-55mm}}
    The function $\negat$ as illustrated to the right is \emph{not} a \fncte{subminimal negation}
   (it is \prope{not antitone}) and so is \emph{not} any negation defined here.
   Note however that the problem is \emph{not} the \structe{O$_6$ lattice}---it is possible to define a negation on 
   an \structe{O$_6$ lattice} \xref{ex:negat_o6_dm_or}.
  \end{array}
  \begin{array}{M}
   \psset{unit=5mm}
   %\includegraphics{graphics/lat6_o6_1=-0_a=-b_b=-a_c=-d_d=-c_0=-1.pdf}
   \includegraphics{graphics/lat6_o6_1-0_a-b_b-a_c-d_d-c_0-1.pdf}
  \end{array}
  }
\end{example}
\begin{proof}
Proof that $\negat$ is \prope{not antitone}:
  $a\orel c \implies \negat c=d\nleq b=\negat a$
\end{proof}

%---------------------------------------
\begin{remark}
\label{rem:cmpneg}
%---------------------------------------
The concept of a \rele{complement}\ifsxref{latc}{def:latc} and the concept of a \fncte{negation} are fundamentally different.
A \structe{complement} is a \rele{relation}\ifsxref{relation}{def:relation} on a lattice $\latL$ and 
a \fncte{negation} is a \rele{function}\ifsxref{relation}{def:function}.
In \prefpp{ex:negat_o6_nonanitone}, $b$ and $d$ are both complements of $a$, but yet $\negat$ is \emph{not} a negation.
In the right side lattice of \prefpp{ex:negat_o6_dm_or}, both $b$ and $d$ are complements of $a$
(and so the lattice is \prope{multipy complemented}), but yet 
only $d$ is equal to the negation of $a$ ($d=\negat a$).
It can also be said that complementation is a property \emph{of} a lattice,
whereas negation is a function defined \emph{on} a lattice.
\end{remark}

%---------------------------------------
\begin{example}
\label{ex:negat_L3_subminimal}
%---------------------------------------
\exbox{
  \begin{array}{m{\tw-114mm}}
    Each of the functions $\negat$ illustrated to the right is a \fncte{subminimal negation} \xref{def:negsub};
    \emph{none} of them is a \structe{minimal negation} (each fails to have \prope{weak double negation}).
  \end{array}
  \begin{array}{*{3}{>{\psset{yunit=5mm}}M}}
     \includegraphics{graphics/lat3_L3_1x0_-0-1-x.pdf}
    &\includegraphics{graphics/lat3_L3_1x0_-1x0.pdf}
    &\includegraphics{graphics/lat3_L3_1-0-a_a_0-1.pdf}
  \\\mbox{}\quad\scs(a)&\mbox{}\quad\scs(b)&\mbox{}\quad\scs(c)
  \end{array}
  }
\end{example}
\begin{proof}
\begin{enumerate}
  \item Proof that (a) $\negat$ is \prope{antitone}:
    $\begin{array}[t]{rcl c lcl c lclc M}
      a      &\orel& \lid &\implies& \negat\lid &=& \lzero &\orel& \lzero &=&  \negat a     &\implies& {$\negat$ is \prope{antitone} over $\opair{a}{\lid}$}\\
      \lzero &\orel& \lid &\implies& \negat\lid &=& \lzero &\orel& a      &=& \negat\lzero &\implies& {$\negat$ is \prope{antitone} over $\opair{\lzero}{\lid}$}\\
      \lzero &\orel& a    &\implies&  \negat a   &=& \lzero &\orel& a      &=& \negat\lzero &\implies& {$\negat$ is \prope{antitone} over $\opair{\lzero}{a}$}
    \end{array}$

  \item Proof that (a) $\negat$ \emph{fails} to have \prope{weak double negation}:\\
    $\lid \nle a = \negat\lzero = \negat\negat\lid$

  \item Proof that (b) $\negat$ is \prope{antitone}:
    $\begin{array}[t]{rcl c lcl c lclc M}
      a      &\orel& \lid &\implies& \negat\lid &=& a &\orel& a &=&  \negat a     &\implies& {$\negat$ is \prope{antitone} over $\opair{a}{\lid}$}\\
      \lzero &\orel& \lid &\implies& \negat\lid &=& a &\orel& a &=& \negat\lzero &\implies& {$\negat$ is \prope{antitone} over $\opair{\lzero}{\lid}$}\\
      \lzero &\orel& a    &\implies&  \negat a   &=& a &\orel& a &=& \negat\lzero &\implies& {$\negat$ is \prope{antitone} over $\opair{\lzero}{a}$}
    \end{array}$

  \item Proof that (b) $\negat$ \emph{fails} to have \prope{weak double negation}:
    $\lid \nle a = \negat a = \negat\negat\lid$

  %\item Proof that (c) $\negat$ is \prope{antitone}:
  %  $\begin{array}[t]{rcl c lcl c lclc M}
  %    a      &\orel& \lid &\implies& \negat\lid &=& \lzero &\orel& \lid &=&  \negat a     &\implies& {$\negat$ is \prope{antitone} over $\opair{a}{\lid}$}\\
  %    \lzero &\orel& \lid &\implies& \negat\lid &=& \lzero &\orel& \lid &=& \negat\lzero &\implies& {$\negat$ is \prope{antitone} over $\opair{\lzero}{\lid}$}\\
  %    \lzero &\orel& a    &\implies&  \negat a   &=& \lid   &\orel& \lid &=& \negat\lzero &\implies& {$\negat$ is \prope{antitone} over $\opair{\lzero}{a}$}
  %  \end{array}$
  %
  %\item Proof that (c) $\negat$ \emph{fails} to have \prope{weak double negation}:
  %  $a \nle \lzero = \negat\lid = \negat\negat{a}$

  \item (c) is a special case of the \exme{dual discrete negation} \xref{ex:neg_dualdiscrete}.
\end{enumerate}
\end{proof}

\begin{figure}%
  \centering%
  \psset{unit=5mm}%
  \begin{tabstr}{0.75}
    \begin{tabular}{*{4}{>{\scs}c}}
       \includegraphics{graphics/lat3_L3_1-a-0_a-1_0.pdf}
      &\includegraphics{graphics/lat3_L3_1x0_-0-x-1.pdf}
      &\includegraphics{graphics/lat3_L3_1_a-0_0-1-a.pdf}
      &\includegraphics{graphics/lat3_L3_1-0_a_0-1-x.pdf}
      \\
      (A) \fncte{minimal negation} & (B) \fncte{Kleene negation} & (C) \fncte{intuitionistic negation} & (D) \prope{fuzzy} and \prope{intuitionistic} negation\\
      \xref{ex:negat_L3_minimal}   & \xref{ex:negat_L3_kl}       & \xref{ex:negat_L3_int}              & \xref{ex:negat_L3_intfuz}
    \end{tabular}%
  \end{tabstr}
  \caption{%
    negations on $\latL_3$%
    \label{fig:negat_L3}%
    }%
\end{figure}
%---------------------------------------
\begin{example}
\label{ex:negat_L3_minimal}
%---------------------------------------
%\exbox{
%  \begin{array}{m{\tw-45mm}}
    The function $\negat$ illustrated  in \prefp{fig:negat_L3} (A) is a \fnctb{minimal negation} \xref{def:negmin};
    it is \emph{not} an \fncte{intuitionistic negation} (it does not have the \prope{non-contradiction} property),
    it is \emph{not} a  \fncte{de Morgan negation} (it is not \prope{involutory}),
    and it is \emph{not} a \fncte{fuzzy negation} ($\negat\lid\neq\lzero$).
%  \end{array}
%  \begin{array}{M}
%  \psset{yunit=5mm}%
%  \includegraphics{graphics/lat3_L3_1=-a=-0_a=-1_0.pdf}
%  \end{array}
%  }
\end{example}
\begin{proof}
\begin{enumerate}
  \item Proof that $\negat$ is \prope{antitone}:
    $\begin{array}[t]{rcl c lcl c lclc M}
      a      &\orel& \lid &\implies& \negat\lid &=& a      &\orel& \lid &=& \negat a      &\implies& {$\negat$ is \prope{antitone} over $\opair{a}{\lid}$}\\
      \lzero &\orel& \lid &\implies& \negat\lid &=& a      &\orel& \lid &=& \negat\lzero &\implies& {$\negat$ is \prope{antitone} over $\opair{\lzero}{\lid}$}\\
      \lzero &\orel& a    &\implies&  \negat a   &=& \lid   &\orel& \lid &=& \negat\lzero &\implies& {$\negat$ is \prope{antitone} over $\opair{\lzero}{a}$}
    \end{array}$

  \item Proof that $\negat$ is a \prope{weak double negation} (and so is a \fncte{minimal negation}, but is \emph{not} a \fncte{de Morgan negation}):
    \\$\begin{array}[t]{rcl cl cl cM}
      \lid   &=& \lid          &=& \negat a    &=& \negat\negat\lid   &\implies& $\negat$ is \prope{involutory} at $\lid$\\
      a      &=& a             &=& \negat\lid &=& \negat\negat{a}      &\implies& $\negat$ is \prope{involutory} at $a$\\
      \lzero &\orel& a         &=& \negat\lid &=& \lzero^{\negat\negat} &\implies& $\negat$ is a \prope{weak double negation} at $\lzero$
    \end{array}$

  \item Proof that $\negat$ does \emph{not} have the \prope{non-contradiction} property (and so is not an \structe{intuitionistic negation}):
        \\\indentx$\lid\meet \negat\lid = \lid\meet a = a \neq \lzero$

  \item Proof that $\negat$ is not a \fncte{fuzzy negation}: $\negat\lid=a\neq\lzero$
\end{enumerate}
\end{proof}

%---------------------------------------
\begin{example}[\exm{/-Lukasiewicz 3-valued logic}/\exm{Kleene 3-valued logic}/\exm{RM$_3$ logic}]
\footnote{
  \citeP{lukasiewicz1920},
  \citePpp{avron1991}{277}{278},
  \citePp{kleene1938}{153},
  \citerppc{kleene1952}{332}{339}{\textsection 64. The 3-valued logic},
  \citeP{sobocinski1952}
  }
\label{ex:negat_L3_kl}
%---------------------------------------
%\exbox{
%  \begin{array}{m{\tw-55mm}}
    The function $\negat$ illustrated  in \prefp{fig:negat_L3} (B) is a \fnctb{Kleene negation} \xref{def:negkl},
    and is also a \fncte{fuzzy negation} \xref{def:negfuz}; 
    but it is \emph{not} an \fncte{ortho negation}
    and is \emph{not} an \fncte{intuitionistic negation} (it does not have the \prope{non-contradiction} property).
%  \end{array}
%  \begin{array}{M}
%  \psset{yunit=5mm}%
%  \includegraphics{graphics/lat3_L3_1x0_-0-x-1.pdf}
%  \end{array}
%  }
\end{example}
\begin{proof}
\begin{enumerate}
  \item Proof that $\negat$ is \prope{antitone}:
    $\begin{array}[t]{rcl c lcl c lclc M}
      a      &\orel& \lid &\implies& \negat\lid &=& \lzero &\orel& a    &=&  \negat a     &\implies& {$\negat$ is \prope{antitone} over $\opair{a}{\lid}$}\\
      \lzero &\orel& \lid &\implies& \negat\lid &=& \lzero &\orel& \lid &=& \negat\lzero &\implies& {$\negat$ is \prope{antitone} over $\opair{\lzero}{\lid}$}\\
      \lzero &\orel& a    &\implies&  \negat a   &=& a      &\orel& \lid &=& \negat\lzero &\implies& {$\negat$ is \prope{antitone} over $\opair{\lzero}{a}$}
    \end{array}$

  \item Proof that $\negat$ is \prope{involutory} (and so is a \fncte{de Morgan negation}):
    \\$\begin{array}[t]{rcl cl cM}
      \lid   &=& \negat\lzero &=& \negat\negat\lid   &\implies& $\negat$ is \prope{involutory} at $\lid$\\
      a      &=& \negat a      &=& \negat\negat{a}      &\implies& $\negat$ is \prope{involutory} at $a$\\
      \lzero &=& \negat\lzero &=& \lzero^{\negat\negat} &\implies& $\negat$ is \prope{involutory} at $\lzero$
    \end{array}$

  \item Proof that $\negat$ does \emph{not} have the \prope{non-contradiction} property (and so is not an \structe{ortho negation}):
        \\\indentx$x\meet \negat{x} = x\meet x = x \neq \lzero$

  \item Proof that $\negat$ satisfies the \prope{Kleene condition} (and so is a \fncte{Kleene negation}):
    \\\indentx$\begin{array}{lcl c lcl c c c l c lcl c lcl}
       \lid   &\meet&  \negat\lid   &=& \lid   &\meet& \lzero &=& \lzero &\orel& a    &=& a     &\join&a      &=& a     &\join&\negat a      \\
       \lid   &\meet&  \negat\lid   &=& \lid   &\meet& \lzero &=& \lzero &\orel& \lid &=& \lzero&\join&\lid   &=& \lzero&\join&\negat\lzero \\
       a      &\meet&  \negat a      &=& \lid   &\meet& a      &=& a      &\orel& \lid &=& \lid  &\join&\lzero &=& \lid  &\join&\negat\lid   \\
       a      &\meet&  \negat a      &=& \lid   &\meet& a      &=& a      &\orel& \lid &=& \lzero&\join&\lid   &=& \lzero&\join&\negat\lzero \\
       \lzero &\meet&  \negat\lzero &=& \lzero &\meet& \lid   &=& \lzero &\orel& \lid &=& \lid  &\join&\lzero &=& \lid  &\join&\negat\lid   \\
       \lzero &\meet&  \negat\lzero &=& \lzero &\meet& \lid   &=& \lzero &\orel& a    &=& a     &\join&a      &=& a     &\join&\negat a
     \end{array}$
\end{enumerate}
\end{proof}

%---------------------------------------
\begin{example}
\label{ex:negat_L3_int}
%---------------------------------------
%\exbox{
%  \begin{array}{m{\tw-55mm}}
    The function $\negat$ illustrated  in \prefp{fig:negat_L3} (C) an \fnctb{intuitionistic negation} \xref{def:negint};
    but it is \emph{not} a \fncte{fuzzy negation} ($\lid\neq\negat\lzero$),
    and it is \emph{not} a \structe{de Morgan negation} (it is not \prope{involutory}).
%  \end{array}
%  \begin{array}{M}
%  \psset{yunit=5mm}%
%  \includegraphics{graphics/lat3_L3_1_a=-0_0=-1=-a.pdf}
%  \end{array}
%  }
\end{example}
\begin{proof}
\begin{enumerate}
  \item Proof that $\negat$ is \prope{antitone}:
    $\begin{array}[t]{rcl c lcl c lclc M}
      a      &\orel& \lid &\implies& \negat\lid &=& \lzero &\orel& \lzero &=&  \negat a     &\implies& {$\negat$ is \prope{antitone} at $\opair{a}{\lid}$}\\
      \lzero &\orel& \lid &\implies& \negat\lid &=& \lzero &\orel& a      &=& \negat\lzero &\implies& {$\negat$ is \prope{antitone} at $\opair{\lzero}{\lid}$}\\
      \lzero &\orel& a    &\implies&  \negat a   &=& \lzero &\orel& a      &=& \negat\lzero &\implies& {$\negat$ is \prope{antitone} at $\opair{\lzero}{a}$}
    \end{array}$

  \item Proof that $\negat$ has \prope{weak double negation} property (and so is a \fncte{minimal negation}, but \emph{not} a \fncte{de Morgan negation}):
    \\$\begin{array}[t]{rcl cl cl cM}
      \lid   &\orel& a             &=& \negat\lzero         &=& \negat\negat\lid &\implies& $\negat$ has \prope{weak double negation} at $\lid$\\
      a      &=&     \negat\lzero &=& \negat\negat{a}      &&                      &\implies& $\negat$ has \prope{weak double negation} at $a$\\
      \lzero &=&     \negat a      &=& \lzero^{\negat\negat} &&                      &\implies& $\negat$ is \prope{involutory} at $\lzero$
    \end{array}$

  \item Proof that $\negat$ has the \prope{non-contradiction} property (and so is an \fncte{intuitionistic negation}):
    \\$\begin{array}[t]{rcl c rcl c c}
      \lid   &\meet& \negat\lid   &=& \lid   &\meet&\lzero   &=& \lzero\\
      a      &\meet& \negat a      &=& a      &\meet&\lzero   &=& \lzero\\
      \lzero &\meet& \negat\lzero &=& \lzero &\meet&a        &=& \lzero
    \end{array}$

  \item Proof that $\negat$ is \emph{not} a \fncte{fuzzy negation}: $\negat\lid\neq\lzero$
\end{enumerate}
\end{proof}

%---------------------------------------
\begin{example}[\exmd{Heyting 3-valued logic}/\exmd{Ja/'skowski's first matrix}]
\footnote{
  \citerpg{karpenko2006}{45}{0955117038},
  \citerpgc{johnstone1982}{9}{0521337798}{\textsection 1.12},
  \citeP{heyting1930a},
  \citeP{heyting1930b},
  \citeP{heyting1930c},
  \citeP{heyting1930d},
  \citeP{jaskowski1936},
  \citer{mancosu1998}
  }
\label{ex:negat_L3_intfuz}
%---------------------------------------
%\exbox{
%  \begin{array}{m{\tw-55mm}}
    The function $\negat$ illustrated  in \prefp{fig:negat_L3} (D) is an \fnctb{intuitionistic negation} \xref{def:negint},
    and is also a \fnctb{fuzzy negation} \xref{def:negfuz},
    but it is \emph{not} a \structe{de Morgan negation} (it is not \prope{involutory}).
%  \end{array}
%  \begin{array}{M}
%  \psset{yunit=5mm}%
%  \includegraphics{graphics/lat3_L3_1=-0_a_0=-1=-x.pdf}
%  \end{array}
%  }
\end{example}
\begin{proofns}
This is simply a special case of the \exme{discrete negation} \xref{ex:neg_discrete}.
\end{proofns}
%\begin{proof}
%\begin{enumerate}
%  \item Proof that $\negat$ is \prope{antitone}:
%    $\begin{array}[t]{rcl c lcl c lclc M}
%      a      &\orel& \lid &\implies& \negat\lid &=& \lzero &\orel& \lzero &=&  \negat a     &\implies& {$\negat$ is \prope{antitone} at $\opair{a}{\lid}$}\\
%      \lzero &\orel& \lid &\implies& \negat\lid &=& \lzero &\orel& \lid   &=& \negat\lzero &\implies& {$\negat$ is \prope{antitone} at $\opair{\lzero}{\lid}$}\\
%      \lzero &\orel& a    &\implies&  \negat a   &=& \lzero &\orel& \lid   &=& \negat\lzero &\implies& {$\negat$ is \prope{antitone} at $\opair{\lzero}{a}$}
%    \end{array}$
%
%  \item Proof that $\negat$ has \prope{weak double negation} property (and so is a \fncte{minimal negation}, but \emph{not} a \fncte{de Morgan negation}):
%    \\$\begin{array}[t]{rcl cl cl cM}
%      \lid   &=&     \negat\lzero &=& \negat\negat\lid   &&                   &\implies& $\negat$ is \prope{involutory} at $\lid$\\
%      a      &\orel& \lid          &=& \negat\lzero         &=& \negat\negat{a} &\implies& $\negat$ has \prope{weak double negation} at $a$\\
%      \lzero &=&     \negat\lid   &=& \lzero^{\negat\negat} &&                   &\implies& $\negat$ is \prope{involutory} at $\lzero$
%    \end{array}$
%
%  \item Proof that $\negat$ has the \prope{non-contradiction} property (and so is an \fncte{intuitionistic negation}):
%    \\$\begin{array}[t]{rcl c rcl c c}
%      \lid   &\meet& \negat\lid   &=& \lid   &\meet&\lzero   &=& \lzero\\
%      a      &\meet& \negat a      &=& a      &\meet&\lzero   &=& \lzero\\
%      \lzero &\meet& \negat\lzero &=& \lzero &\meet&\lid     &=& \lzero
%    \end{array}$
%
%  \item Proof that $\negat$ is a \fncte{fuzzy negation}: $\negat\lid=\lzero$,\quad$\negat\lzero=\lid$
%\end{enumerate}
%\end{proof}

%---------------------------------------
\begin{remark}
%---------------------------------------
There is only one linearly ordered \xref{def:toset} 3-element lattice ($\latL_3$) that is a \fncte{fuzzy negation} \xref{ex:negat_L3_intfuz}.
However, this lattice is also an \fncte{intuitionistic negation}.
There are no $\latL_3$ lattices that are \prope{fuzzy} but yet not \prope{intuitionistic}.
In fact, there are only three linearly ordered 3-element lattices with with $\lid=\negat\lzero$ and $\lzero=\negat\lid$.
Of these three, only one is both \prope{fuzzy} and \prope{intuitionistic} \xref{ex:negat_L3_intfuz},
one is \prope{Kleene} but not \prope{fuzzy} \xref{ex:negat_L3_kl}, 
and one is \prope{subminimal} but not \prope{fuzzy} \xref{ex:negat_L3_subminimal}.
It can be claimed that the ``simplist" \fncte{fuzzy negation} that is not \prope{de Morgan} and 
\emph{not} \prope{intuitionistic} is the $\latM_2$ lattice of \pref{ex:negat_m2_fuzzy} (next).
\end{remark}

\begin{figure}
  \centering%
  \psset{unit=10mm}%
  \begin{tabstr}{0.75}%
    \begin{tabular}{*{3}{>{\scs}c}}
       \includegraphics{graphics/lat4_m2_1-0_a_b-b_0-1-a.pdf}
      &\includegraphics{graphics/lat4_m2_a-b_b-a.pdf}
      &\includegraphics{graphics/lat4_bn4_a-a_b-b.pdf}
      \\
      (A) \fncte{fuzzy negation} & (B) \fncte{ortho negation} & (C) \fncte{de Morgan negation}\\ 
      \xref{ex:negat_m2_fuzzy}   & \xref{ex:negat_m2_ortho}   & \xref{ex:negat_m2_bn4}              
    \end{tabular}%
  \end{tabstr}%
  \caption{%
    negations on $\latM_2$%
    \label{fig:negat_m2}%
    }%
\end{figure}
%---------------------------------------
\begin{example}
\label{ex:negat_m2_fuzzy}
%---------------------------------------
%\exbox{
%  \begin{array}{m{\tw-45mm}}
    The function $\negat$ illustrated in \prefp{fig:negat_m2} (A) is a \fnctb{fuzzy negation} \xref{def:negfuz}.
    It is not an \fncte{intuitionistic negation} (it does not have the \prope{non-contradiction} property)
    and it is \emph{not} a \structe{de Morgan negation} (it is not \prope{involutory}).
%  \end{array}
%  \begin{array}{M}
%  \psset{unit=6.5mm}%
%  \includegraphics{graphics/lat4_m2_1=-0_a_b=-b_0=-1=-a.pdf}
%  \end{array}
%  }
\end{example}
\begin{proofns}
Note that 
$\begin{array}{NcNcN}
  \psset{unit=5mm}\includegraphics{graphics/lat4_m2_1-0_a_b-b_0-1-a.pdf}
  &\raisebox{5mm}{=}&
  \psset{unit=5mm}\includegraphics{graphics/lat3_L3_1-0_a_0-1-x.pdf}
  &\raisebox{5mm}{+}&
  \psset{unit=5mm}\includegraphics{graphics/lat3_L3_1x0_-0-x-1.pdf}
  \\                         && \prope{fuzzy} and \prope{intuitionistic} && \fncte{Kleene negation}
  \\\xref{ex:negat_m2_fuzzy} && \xref{ex:negat_L3_intfuz}                && \xref{ex:negat_L3_kl}
\end{array}$


\begin{enumerate}
  \item Proof that $\negat$ is \prope{antitone}:
    $\begin{array}[t]{rcl c lcl c lclc M}
      a      &\orel& \lid &\implies& \negat\lid &=& \lzero &\orel& \lzero &=&  \negat a     &\implies& {$\negat$ is \prope{antitone} at $\opair{a}{\lid}$}\\
      \lzero &\orel& \lid &\implies& \negat\lid &=& \lzero &\orel& \lid   &=& \negat\lzero &\implies& {$\negat$ is \prope{antitone} at $\opair{\lzero}{\lid}$}\\
      \lzero &\orel& a    &\implies&  \negat a   &=& \lzero &\orel& \lid   &=& \negat\lzero &\implies& {$\negat$ is \prope{antitone} at $\opair{\lzero}{a}$}\\
      b      &\orel& \lid &\implies& \negat\lid &=& \lzero &\orel& b      &=&  \negat b     &\implies& {$\negat$ is \prope{antitone} at $\opair{b}{\lid}$}\\
      \lzero &\orel& b    &\implies&  \negat b   &=& b      &\orel& \lid   &=& \negat\lzero &\implies& {$\negat$ is \prope{antitone} at $\opair{\lzero}{b}$}
    \end{array}$

  \item Proof that $\negat$ has \prope{weak double negation} property (and so is a \fncte{minimal negation}, but \emph{not} a \fncte{de Morgan negation}):
    \\$\begin{array}[t]{rcl cl cl cM}
      \lid   &=&     \negat\lzero &=& \negat\negat\lid   &&                   &\implies& $\negat$ is \prope{involutory} at $\lid$\\
      a      &\orel& \lid          &=& \negat\lzero         &=& \negat\negat{a} &\implies& $\negat$ has \prope{weak double negation} at $a$\\
      \lzero &=&     \negat\lid   &=& \lzero^{\negat\negat} &&                   &\implies& $\negat$ is \prope{involutory} at $\lzero$\\
      b      &=    & \negat b      &=& \negat\negat{b}      &=&                  &\implies& $\negat$ is \prope{involutory} at $b$  
    \end{array}$

  \item Proof that $\negat$ does \emph{not} have the \prope{non-contradiction} property (and so is \emph{not} an \fncte{intuitionistic negation}):
    \\$b\meet \negat b=b\meet b = b \neq \lzero$

  \item Proof that $\negat$ is has \prope{boundary conditions} (and so is a \fncte{fuzzy negation}): 
    $\negat\lid=\lzero$,\quad$\negat\lzero=\lid$
\end{enumerate}
\end{proofns}


%---------------------------------------
\begin{example}
\footnote{
  \citePp{belnap1977}{13}
  \citerpgc{restall2000}{177}{041521534X}{Example 8.44},
  \citePpgc{pavicic2008}{28}{0080931669}{Definition 2, \rele{classical implication}}
  %\citePpgc{pavicic2008}{28}{0080931669}{Definition 2, $a\imp_0 b\eqd \negat a \setu b$ (\rele{classical implication})}
  }
\label{ex:negat_m2_ortho}
%---------------------------------------
%\exbox{
%  \begin{array}{m{\tw-50mm}}
    The function $\negat$ illustrated in \prefp{fig:negat_m2} (B) is an \fncte{ortho negation} \xref{def:negor}.
%  \end{array}
%  \begin{array}{M}%
%  \psset{unit=7.5mm}%
%  \includegraphics{graphics/lat4_m2_a=-b_b=-a.pdf}%
%  \end{array}%
%  }
\end{example}
\begin{proofns}
\begin{enumerate}
  \item Proof that $\negat$ is \prope{antitone}:
    $\begin{array}[t]{rcl c lcl c lclc}
      a      &\orel& \lid &\implies& \negat\lid &=& \lzero &\orel& b    &=&  \negat a     \\
      \lzero &\orel& \lid &\implies& \negat\lid &=& \lzero &\orel& \lid &=& \negat\lzero \\
      \lzero &\orel& a    &\implies&  \negat a   &=& b      &\orel& \lid &=& \negat\lzero \\
      b      &\orel& \lid &\implies& \negat\lid &=& \lzero &\orel& a    &=&  \negat b     \\
      \lzero &\orel& b    &\implies&  \negat b   &=& a      &\orel& \lid &=& \negat\lzero 
    \end{array}$

  \item Proof that $\negat$ is \prope{involutory} (and so is a \fncte{de Morgan negation}):
    $\begin{array}[t]{rcl cl}
      \lid   &=& \negat\lzero &=& \negat\negat\lid   \\
      a      &=& \negat a      &=& \negat\negat{a}      \\
      b      &=& \negat b      &=& \negat\negat{b}      \\
      \lzero &=& \negat\lzero &=& \lzero^{\negat\negat} \\
    \end{array}$

  \item Proof that $\negat$ is has the \prope{non-contradiction} property (and so is an \structe{ortho negation}):
    \\$\begin{array}[t]{rcl clclcl}
      \lid   &\meet& \negat\lid   &=& \lid   &\meet& \lzero &=& \lzero \\
      a      &\meet& \negat a      &=& a      &\meet& b      &=& \lzero \\
      b      &\meet& \negat b      &=& b      &\meet& a      &=& \lzero \\
      \lzero &\meet& \negat\lzero &=& \lzero &\meet& \lid   &=& \lzero 
    \end{array}$

%  \item Proof that $\negat$ satisfies the \prope{Kleene condition} (and so is a \fncte{Kleene negation}):
%    \\\indentx$\begin{array}{lcl c lcl c c c l c lcl c lcl}
%       \lid   &\meet&  \negat\lid   &=& \lid   &\meet& \lzero &=& \lzero &\orel& x    &=& x     &\join&x      &=& x     &\join&\negat{x}      \\
%       \lid   &\meet&  \negat\lid   &=& \lid   &\meet& \lzero &=& \lzero &\orel& \lid &=& \lzero&\join&\lid   &=& \lzero&\join&\negat\lzero \\
%       x      &\meet&  \negat{x}      &=& \lid   &\meet& x      &=& x      &\orel& \lid &=& \lid  &\join&\lzero &=& \lid  &\join&\negat\lid   \\
%       x      &\meet&  \negat{x}      &=& \lid   &\meet& x      &=& x      &\orel& \lid &=& \lzero&\join&\lid   &=& \lzero&\join&\negat\lzero \\
%       \lzero &\meet&  \negat\lzero &=& \lzero &\meet& \lid   &=& \lzero &\orel& \lid &=& \lid  &\join&\lzero &=& \lid  &\join&\negat\lid   \\
%       \lzero &\meet&  \negat\lzero &=& \lzero &\meet& \lid   &=& \lzero &\orel& x    &=& x     &\join&x      &=& x     &\join&\negat{x}
%     \end{array}$
\end{enumerate}
\end{proofns}

%---------------------------------------
\begin{example}[\exmd{BN$_4$}]
\footnote{
  \citePp{cignoli1975}{270},
  \citerpgc{restall2000}{171}{041521534X}{Example 8.39},
  \citePppc{devries2007}{15}{16}{Example 26},
  \citeP{dunn1976},
  \citeP{belnap1977}
  }
\label{ex:negat_m2_bn4}
%---------------------------------------
%\exbox{
%  \begin{array}{m{\tw-50mm}}
    The function $\negat$ illustrated in \prefp{fig:negat_m2} (C) is a \fnctb{de Morgan negation} \xref{def:negdm},
    but it is \emph{not} a \fncte{Kleene negation} and not an \fncte{ortho negation} 
    (it does \emph{not} satisfy the \prope{Kleene condition}).
%  \end{array}
%  \begin{array}{M}%
%  \psset{unit=7.5mm}%
%  \includegraphics{graphics/lat4_bn4_a=-a_b=-b.pdf}%
%  \end{array}%
%  }
\end{example}
\begin{proofns}
\begin{enumerate}
  \item Proof that $\negat$ is \prope{antitone}:
    $\begin{array}[t]{rcl c lcl c lclc}
      a      &\orel& \lid &\implies& \negat\lid &=& \lzero &\orel& b    &=&  \negat a     \\
      \lzero &\orel& \lid &\implies& \negat\lid &=& \lzero &\orel& \lid &=& \negat\lzero \\
      \lzero &\orel& a    &\implies&  \negat a   &=& a      &\orel& \lid &=& \negat\lzero \\
      b      &\orel& \lid &\implies& \negat\lid &=& \lzero &\orel& b    &=&  \negat b     \\
      \lzero &\orel& b    &\implies&  \negat b   &=& b      &\orel& \lid &=& \negat\lzero 
    \end{array}$

  \item Proof that $\negat$ is \prope{involutory} (and so is a \fncte{de Morgan negation}):
    $\begin{array}[t]{rcl cl}
      \lid   &=& \negat\lzero &=& \negat\negat\lid   \\
      a      &=& \negat a      &=& \negat\negat{a}      \\
      b      &=& \negat b      &=& \negat\negat{b}      \\
      \lzero &=& \negat\lzero &=& \lzero^{\negat\negat} \\
    \end{array}$

  \item Proof that $\negat$ does \emph{not} have the \prope{non-contradiction} property (and so is \emph{not} an \structe{ortho negation}):
    \\$\begin{array}[t]{rcl clclclcl}
      a      &\meet& \negat a      &=& a      &\meet& a      &=& a &\neq& \lzero \\
      b      &\meet& \negat b      &=& b      &\meet& b      &=& b &\neq& \lzero 
    \end{array}$

  \item Proof that $\negat$ does \emph{not} satisfy the \prope{Kleene condition} (and so is a \fncte{de Morgan negation}):
    \\\indentx$\begin{array}{lcl c lcl c c c l c lcl c lcl}
       %\lid   &\meet&  \negat\lid   &=& \lid   &\meet& \lzero &=& \lzero &\orel& a    &=& a     &\join&a      &=& a     &\join&\negat a      \\
       %\lid   &\meet&  \negat\lid   &=& \lid   &\meet& \lzero &=& \lzero &\orel& \lid &=& \lzero&\join&\lid   &=& \lzero&\join&\negat\lzero \\
       a      &\meet&  \negat a      &=& a      &\meet& a      &=& a      &\nleq& b &\meet&  \negat b &=& b 
       %\lzero &\meet&  \negat\lzero &=& \lzero &\meet& \lid   &=& \lzero &\orel& \lid &=& \lid  &\join&\lzero &=& \lid  &\join&\negat\lid   \\
       %\lzero &\meet&  \negat\lzero &=& \lzero &\meet& \lid   &=& \lzero &\orel& a    &=& a     &\join&a      &=& a     &\join&\negat a
     \end{array}$
\end{enumerate}
\end{proofns}

%---------------------------------------
\begin{example}
%---------------------------------------
\exbox{
  \begin{array}{M}%
  \psset{unit=7.5mm}%
  \includegraphics{graphics/lat5_m3_negdm.pdf}%
  \end{array}%
  \begin{array}{m{\tw-96mm}}
    The function $\negat$ illustrated to the left is a \prope{de Morgan negation} \xref{def:negdm},
    but it is \emph{not} a \fncte{Kleene negation} and not an \fncte{ortho negation} 
    (it does \emph{not} satisfy the \prope{Kleene condition}).
    The \fncte{negation} illustrated to the right is a \prope{Kleene negation} \xref{def:negkl},
    but it is \emph{not} an \fncte{ortho negation} 
    (it does \emph{not} have the \prope{non-contradiction} property).
  \end{array}
  \begin{array}{M}%
  \psset{unit=7.5mm}%
  \includegraphics{graphics/lat5_m3_negkl.pdf}%
  \end{array}%
  }
\end{example}

%---------------------------------------
\begin{example}
\label{ex:negat_o6_dm_or}
%---------------------------------------
\exbox{
  \begin{array}{M}
   \psset{unit=5mm}
   \includegraphics{graphics/lat6_o6_negdm.pdf}
  \end{array}
  \begin{array}{m{\tw-83mm}}
    The function $\negat$ illustrated to the left is a \fncte{de Morgan negation} \xref{def:negdm};
    it is \emph{not} a \fncte{Kleene negation} (it does not satisfy the Kleene condition).
    The \structe{negation} illustrated to the right is an \fncte{ortho negation} \xref{def:negor}.
  \end{array}
  \begin{array}{M}
   \psset{unit=5mm}
   \includegraphics{graphics/lat6_o6_negor.pdf}
  \end{array}
  }
\end{example}

%---------------------------------------
\begin{example}
\label{ex:negat_o6slash}
%---------------------------------------
\exbox{
  \begin{array}{M}
   \psset{unit=5mm}
   \includegraphics{graphics/lat6_o6slash_nonantitone.pdf}
  \end{array}
  \begin{array}{m{\tw-83mm}}
    The function $\negat$ illustrated to the left is \prope{not antitone} and therefore is not a \fncte{negation} \xref{def:negat}.
    The function $\negat$ illustrated to the right is a \fncte{Kleene negation} \xref{def:negkl};
    it is \emph{not} an \fncte{ortho negation} (it does not have the \prope{non-contradiction} property).
  \end{array}
  \begin{array}{M}
   \psset{unit=5mm}
   \includegraphics{graphics/lat6_o6slash_kleene.pdf}
  \end{array}
  }
\end{example}
\begin{proofns}
\begin{enumerate}
  \item Proof that left $\negat$ is \prope{not antitone}:
     $a\orel c$ but $\negat{c}\nle\negat{a}$.
  \item Proof that right $\negat$ satisfies the \prope{Kleene condition}:\\
    $x\meet\negat{x}=\brbl{\begin{array}{cM}b&for $x=b$\\\lzero&otherwise\end{array}}\quad{\scy \forall x\in\setX}$
    \qquad and\qquad
    $y\meet\negat{y}=\brbl{\begin{array}{cM}c&for $y=c$\\\lzero&otherwise\end{array}}\quad{\scy \forall y\in\setX}$
    \\\qquad$\implies$\qquad
    $x\meet\negat{x} \orel y\join \negat{y} \qquad{\scy\forall x,y\in\setX}$

  \item Proof that right $\negat$ does not have the \prope{non-contradiction} property:
    $b\meet \negat{b} = b\meet c = b \neq \lzero$
\end{enumerate}
\end{proofns}

%---------------------------------------
\begin{example}
%---------------------------------------
\exbox{
  \begin{array}{M}%
  \psset{unit=7.5mm}%
  \includegraphics{graphics/lat8_l2e3_abc_dm.pdf}%
  \end{array}%
  \begin{array}{m{\tw-87mm}}
    The lattices illustrated to the left and right are \prope{Boolean}\ifsxref{boolean}{def:boolean}.
    The function $\negat$ illustrated to the left is a \prope{Kleene negation} \xref{def:negkl},
    but it is \emph{not} an \fncte{ortho negation} 
    (it does \emph{not} have the \prope{non-contradiction} property).
    The \fncte{negation} illustrated to the right is an \prope{ortho negation} \xref{def:negor}.
  \end{array}
  \begin{array}{M}%
  \psset{unit=7.5mm}%
  \includegraphics{graphics/lat8_l2e3_abc_ortho.pdf}%
  \end{array}%
  }
\end{example}
\begin{proofns}
\begin{enumerate}
  \item Proof that left side negation does \emph{not} have \prope{non-contradiction} property (and so is \emph{not} an \structe{ortho negation}):
        \\$a\meet \negat a = a\meet d = a \neq \lzero$

  \item Proof that left side negation does \emph{not} satisfy \prope{Kleene condition} (and so is \emph{not} a \structe{Kleene negation}):
        \\$a\meet \negat a = a\meet d = a \nleq f = c\join f = c\join \negat c$

\end{enumerate}
\end{proofns}
%}%end exclude mssa



%2014jun14sat% %---------------------------------------
%2014jun14sat% \begin{lemma}
%2014jun14sat% \label{lem:numDL}
%2014jun14sat% %---------------------------------------
%2014jun14sat% Let $\latL_2^\xN$ be a \structe{Boolean lattice} of order $\xN$.
%2014jun14sat% \\\lemboxt{
%2014jun14sat%   The number of pairwise orthogonal elements in $\latL_2^\xN$ is
%2014jun14sat%   \\\indentx
%2014jun14sat%     $\brbl{\begin{array}{lD}
%2014jun14sat%       0                        & for $\xN<2$\\
%2014jun14sat%       1                        & for $\xN=2$\\
%2014jun14sat%       3                        & for $\xN=3$\\
%2014jun14sat%       \xN + \bcoef{\xN}{\xN-2} & for $\xN>3$
%2014jun14sat%     \end{array}}$
%2014jun14sat%   }
%2014jun14sat% \end{lemma}
%2014jun14sat% \begin{proof}
%2014jun14sat% 
%2014jun14sat% \end{proof}


