%============================================================================
% LaTeX File
% Daniel J. Greenhoe
%============================================================================
%=======================================
\chapter{Spectral Estimation}
%=======================================

Quality of spectral estimators\footnote{
  \citerppgc{proakis2002}{452}{457}{9780130622198}{\textsection ``8.2.4 Performance Characteristics of Nonparametric Power Spectrum Estimators"},
  \citerppgc{proakis1996}{916}{919}{9780133737622}{\textsection ``12.2.4 Performance Characteristics of Nonparametric Power Spectrum Estimators"},
  \citerpgc{rao2018}{731}{9789811080814}{``Table 12.1 Comparison of performance of classical methods"},
  \citerpgc{salivahanan2001}{606}{9780074639962}{\textsection ``12.5 Power Spectrum Estimation: Non-Parametric Methods"},
 %\citerppgc{ifeachor1993}{577}{610}{020154413X}{Chapter 10 ``Spectrum estimation and analysis"},
  \citerppgc{ifeachor2002}{706}{707}{0201596199}{\textsection ``11.3.7 Comparison of the power spectral density estimation methods"},
  \citerpg{chitode2009dsp}{P-100}{9788184316469},
  \citerp{abdaheer2009}{204}
  }

\thmbox{\begin{array}{|Mrcl|}
    \ope{Periodogram}:               & Q &=& 1
  \\\ope{Welch Method 0\% overlap}:  & Q &=& 0.78\xN\Delta f
  \\\ope{Welch Method 50\% overlap}: & Q &=& 1.39\xN\Delta f
  \\\ope{Bartlett Method}:           & Q &=& 1.11\xN\Delta f
  \\\ope{Blackman-Tukey Method}:     & Q &=& 2.34\xN\Delta f
\end{array}}


BT-product references:
\citerppgc{haykin2014}{25}{28}{9781118476772}{\textsection ``2.4 The Inverse Relationship between Time-Domain and Frequency-Domain Representations"},
\citerppgc{marple1987}{144}{146}{9780132141499}{{\textsection ``\scshape5.4 Resolution and the Stability-Time-Bandwidth Product"}}

