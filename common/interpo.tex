%============================================================================
% Daniel J. Greenhoe
% XeLaTeX file
%============================================================================




%=======================================
\chapter{Interpolation}
%=======================================
%=======================================
\section{Polynomial interpolation}
%=======================================
%--------------------------------------
\begin{definition}
\label{def:Ln(x)}
\citetbl{
  \citerp{matthews}{206}
  }
\index{polynomial!Lagrange}
%--------------------------------------
The \hid{Lagrange polynomial} $L_{P,n}(x)$ with respect to the $n+1$ points \\
$P=\set{(x_k,y_k)}{k=0,1,2,\ldots,n}$ is defined as
\defbox{
  L_{P,n}(x) \eqd \sum_{k=0}^n y_k \prod_{m\ne n} \frac{x-x_m}{x_k-x_m}
  }
\end{definition}

%--------------------------------------
\begin{proposition}
%--------------------------------------
Let $L_{P,n}(x)$ be the Lagrange polynomial with respect to the points\\
$P=\set{(x_k,y_k)}{k=0,1,2,\ldots,n}$.
\propbox{\begin{tabular}{ll}
  1.& $L_{P,n}(x)$ is an $n$th order polynomial. \\
  2.& $L_{P,n}(x)$ intersects all $n+1$ points in $P$.
\end{tabular}}
\end{proposition}

%--------------------------------------
\begin{example}[Lagrange interpolation]
%--------------------------------------
The {Lagrange polynomial} $L_{P,3}(x)$ with respect to the $4$ points\\
$P=\{(-2,1),\; (-1,3),\; (3,2),\; (5,4)\}$ is
\exbox{
  L_{P,3}(x) = \frac{   79}{840} x^3 
             + \frac{- 378}{840} x^2
             + \frac{-   7}{840} x
             + \frac{ 2970}{840}
  }
\end{example}
\begin{proof}
\begin{align*}
  L_{P,3}(x)
    &= \sum_{k=0}^n y_k \prod_{m\ne n} \frac{x-x_m}{x_k-x_m}
    \qquad\text{by Definition~\ref{def:Ln(x)}}
  \\&=  y_0\frac{(x+1)(x-3)(x-5)}{(x_0-x_1)(x_0-x_2)(x_0-x_3)} +
        y_1\frac{(x+2)(x-3)(x-5)}{(x_1-x_0)(x_1-x_2)(x_1-x_3)} 
  \\&\quad+
        y_2\frac{(x+2)(x+1)(x-5)}{(x_2-x_0)(x_2-x_1)(x_2-x_3)} +
        y_3\frac{(x+2)(x+1)(x-3)}{(x_3-x_0)(x_3-x_1)(x_3-x_2)}
%
  \\&=  1\frac{(x+1)(x-3)(x-5)}{(-2+1)(-2-3)(-2-5)} +
        3\frac{(x+2)(x-3)(x-5)}{(-1+2)(-1-3)(-1-5)} 
  \\&\quad+
        2\frac{(x+2)(x+1)(x-5)}{(3+2)(3+1)(3-5)}    +
        4\frac{(x+2)(x+1)(x-3)}{(5+2)(5+1)(5-3)}
  \\&=  1\mcom{\frac{x^3-7x^2+7x+15}{-35} }{roots=$-1,3,5$}   +
        3\mcom{\frac{x^3-6x^2-x+30}{24}  }{roots=$-2,3,5$}   +
        2\mcom{\frac{x^3-2x^2-13x-10}{-40}}{roots=$-2,-1,5$}  +
        4\mcom{\frac{x^3-7x-6}{84}        }{roots=$-2,-1,3$}
  \\&= -  \frac{x^3-7x^2+7x+15}{35}  
       +  \frac{x^3-6x^2-x+30}{8}   
       -  \frac{x^3-2x^2-13x-10}{20}  
       +  \frac{x^3-7x-6}{21}       
  \\&=   x^3 \left(
         \frac{-8\cdot20\cdot21 +35\cdot20\cdot21 - 35\cdot8\cdot21 + 35\cdot8\cdot20}
              {35 \cdot 8 \cdot 20 \cdot 21}
             \right)
  \\&\quad
         +x^2 \left(
         \frac{7\cdot8\cdot20\cdot21 -6\cdot35\cdot20\cdot21+ 2\cdot35\cdot8\cdot21+0\cdot35\cdot8\cdot20}
              {35 \cdot 8 \cdot 20 \cdot 21}
             \right)
  \\&\quad
         +x \left(
         \frac{-7\cdot8\cdot20\cdot21 -35\cdot20\cdot21+13\cdot35\cdot8\cdot21-7\cdot35\cdot8\cdot20}
              {35 \cdot 8 \cdot 20 \cdot 21}
             \right)
  \\&\quad
         + \left(
         \frac{-15\cdot8\cdot20\cdot21 +30\cdot35\cdot20\cdot21+10\cdot35\cdot8\cdot21-6\cdot35\cdot8\cdot20}
              {35 \cdot 8 \cdot 20 \cdot 21}
             \right)
  \\&=   \frac{ 11060}{117600} x^3
       + \frac{-52920}{117600} x^2
       + \frac{-  980}{117600} x
       + \frac{415800}{117600}
  \\&=   \frac{   79}{840} x^3
       + \frac{- 378}{840} x^2
       + \frac{-   7}{840} x
       + \frac{ 2970}{840}
\end{align*}
\end{proof}


%--------------------------------------
\begin{definition}
\label{def:Nn(x)}
\citetbl{
  \citerp{matthews}{220}
  }
\index{polynomial!Newton}
%--------------------------------------
The \hid{Newton polynomial} $N_{P,n}(x)$ with respect to the $n+1$ points\\
$P=\set{(x_k,y_k)}{k=0,1,2,\ldots,n}$ is defined as
\defbox{
  N_{P,n}(x) \eqd \sum_{k=0}^n \alpha_k \prod_{m=0}^k (x-x_m)
  }
\end{definition}


%--------------------------------------
\begin{proposition}
%--------------------------------------
Let $N_{P,n}(x)$ be the Newton polynomial with respect to the points\\
$P=\set{(x_k,y_k)}{k=0,1,2,\ldots,n}$.
\propbox{\begin{tabular}{ll}
  1.& $N_{P,n}(x)$ is an $n$th order polynomial. \\
  2.& $N_{P,n}(x)$ intersects all $n+1$ points in $P$.
\end{tabular}}
\end{proposition}

%--------------------------------------
\begin{example}[Newton polynomial interpolation]
\index{polynomial!Newton}
%--------------------------------------
The {Newton polynomial} $N_{P,3}(x)$ with respect to the $4$ points\\
$P=\{(-2,1),\; (-1,3),\; (3,2),\; (5,4)\}$ is
\exbox{
  N_{P,3}(x) = \frac{   79}{840} x^3
             + \frac{- 378}{840} x^2
             + \frac{-   7}{840} x
             + \frac{ 2970}{840}
  }
\end{example}
\begin{proof}
\begin{align*}
  N_{P,3}(x)
    &= \sum_{k=0}^n \alpha_k \prod_{m=1}^k (x-x_m)
  \\&= \alpha_0 + \alpha_1(x-x_0) + \alpha_2(x-x_0)(x-x_1) + \alpha_3(x-x_0)(x-x_1)(x-x_2) 
  \\&= \alpha_0 + \alpha_1(x+2) + \alpha_2(x+2)(x+1) + \alpha_3(x+2)(x+1)(x-3) 
  \\&= \alpha_0 + \alpha_1(x+2) + \alpha_2(x^2+3x+2) + \alpha_3(x^3-7x-6)
  \\&= x^3(\alpha_3) + x^2(\alpha_2) + x(-7\alpha_3+3\alpha_2+\alpha_1)
      +(-6\alpha_3 + 2\alpha_2 + 2\alpha_1 + \alpha_0)
  \\&= \Big[ \alpha_0 \; \alpha_1 \; \alpha_2 \; \alpha_3 \Big] 
       \left[\begin{array}{rrrr}
          1  &   0  & 0  & 0  \\
          2  &   1  & 0  & 0  \\
          2  &   3  & 1  & 0  \\
         -6  &  -7  & 0  & 1
       \end{array}\right]
       \left[\begin{array}{l}
         1   \\
         x   \\
         x^2 \\
         x^3
       \end{array}\right]
  \\
  \\
  \left[\begin{array}{l}
     1  \\
     3  \\
     2  \\
     4  
  \end{array}\right]
    &= \left[\begin{array}{l}
         y_0 \\
         y_1 \\
         y_2 \\
         y_3 
       \end{array}\right]
  \\&= \left[\begin{array}{llll}
         1 & 0         & 0                  & 0                           \\
         1 & (x_1-x_0) & 0                  & 0                           \\
         1 & (x_2-x_0) & (x_2-x_0)(x_2-x_1) & 0                           \\
         1 & (x_3-x_0) & (x_3-x_0)(x_3-x_1) & (x_3-x_0)(x_3-x_1)(x_3-x_2) 
       \end{array}\right]
       \left[\begin{array}{l}
         \alpha_0 \\
         \alpha_1 \\
         \alpha_2 \\
         \alpha_3 
       \end{array}\right]
  \\&= \left[\begin{array}{llll}
         1 & 0     & 0          & 0                           \\
         1 & (-1+2)& 0          & 0                           \\
         1 & (3+2) & (3+2)(3+1) & 0                           \\
         1 & (5+2) & (5+2)(5+1) & (5+2)(5+1)(5-3) 
       \end{array}\right]
       \left[\begin{array}{l}
         \alpha_0 \\
         \alpha_1 \\
         \alpha_2 \\
         \alpha_3 
       \end{array}\right]
  \\&= \left[\begin{array}{llll}
         1 & 0     & 0          & 0                           \\
         1 & 1     & 0          & 0                           \\
         1 & 5     & 20         & 0                           \\
         1 & 7     & 42         & 84                
       \end{array}\right]
       \left[\begin{array}{l}
         \alpha_0 \\
         \alpha_1 \\
         \alpha_2 \\
         \alpha_3 
       \end{array}\right]
  \end{align*}

  \begin{align*}
  \left[\begin{array}{*{8}{r}}
    1 & 0 & 0  & 0  & 1 & 0 & 0 & 0 \\
    1 & 1 & 0  & 0  & 0 & 1 & 0 & 0 \\
    1 & 5 & 20 & 0  & 0 & 0 & 1 & 0 \\
    1 & 7 & 42 & 84 & 0 & 0 & 0 & 1
  \end{array}\right]
    &= \left[\begin{array}{*{8}{r}}
         1 & 0 & 0  & 0  & 1 & 0 & 0 & 0 \\
         0 & 1 & 0  & 0  &-1 & 1 & 0 & 0 \\
         0 & 5 & 20 & 0  &-1 & 0 & 1 & 0 \\
         0 & 7 & 42 & 84 &-1 & 0 & 0 & 1
       \end{array}\right]
  \\&= \left[\begin{array}{*{8}{r}}
         1 & 0 & 0  & 0  & 1 & 0 & 0 & 0 \\
         0 & 1 & 0  & 0  &-1 & 1 & 0 & 0 \\
         0 & 0 & 20 & 0  & 4 &-5 & 1 & 0 \\
         0 & 0 & 42 & 84 & 6 &-7 & 0 & 1
       \end{array}\right]
  \\&= \left[\begin{array}{*{8}{r}}
         1 & 0 & 0  & 0  & 1 & 0 & 0 & 0 \\
         0 & 1 & 0  & 0  &-1 & 1 & 0 & 0 \\
         0 & 0 & 1  & 0  &\frac{1}{5} &-\frac{1}{4} & \frac{1}{20} & 0 \\
         0 & 0 & 42 & 84 & 6 &-7 & 0 & 1
       \end{array}\right]
  \\&= \left[\begin{array}{*{8}{r}}
         1 & 0 & 0  & 0  & 1 & 0 & 0 & 0 \\
         0 & 1 & 0  & 0  &-1 & 1 & 0 & 0 \\
         0 & 0 & 1  & 0  &\frac{1}{5} &-\frac{1}{4} & \frac{1}{20} & 0 \\
         0 & 0 & 0  & 84 & 6-\frac{42}{5} &-7+\frac{42}{4} & -\frac{42}{20} & 1
       \end{array}\right]
  \\&= \left[\begin{array}{*{8}{r}}
         1 & 0 & 0  & 0  & 1 & 0 & 0 & 0 \\
         0 & 1 & 0  & 0  &-1 & 1 & 0 & 0 \\
         0 & 0 & 1  & 0  &\frac{1}{5} &-\frac{1}{4} & \frac{1}{20} & 0 \\
         0 & 0 & 0  & 84 & -\frac{12}{5} &\frac{14}{4} & -\frac{42}{20} & 1
       \end{array}\right]
  \\&= \left[\begin{array}{*{8}{r}}
         1 & 0 & 0  & 0  & 1 & 0 & 0 & 0 \\
         0 & 1 & 0  & 0  &-1 & 1 & 0 & 0 \\
         0 & 0 & 1  & 0  &\frac{4}{20} &-\frac{5}{20} & \frac{1}{20} & 0 \\
         0 & 0 & 0  & 84 & -\frac{24}{10} &\frac{35}{10} & -\frac{21}{10} & \frac{10}{10}
       \end{array}\right]
  \\&= \left[\begin{array}{*{8}{r}}
         1 & 0 & 0  & 0  & 1 & 0 & 0 & 0 \\
         0 & 1 & 0  & 0  &-1 & 1 & 0 & 0 \\
         0 & 0 & 1  & 0  &\frac{4}{20} &-\frac{5}{20} & \frac{1}{20} & 0 \\
         0 & 0 & 0  & 1  & -\frac{24}{840} &\frac{35}{840} & -\frac{21}{840} & \frac{10}{840}
       \end{array}\right]
  \end{align*}

  \begin{align*}
  \left[\begin{array}{l}
    \alpha_0 \\
    \alpha_1 \\
    \alpha_2 \\
    \alpha_3 
  \end{array}\right]
    &= \left[\begin{array}{*{4}{r}}
          1 & 0 & 0 & 0 \\
         -1 & 1 & 0 & 0 \\
         \frac{4}{20} &-\frac{5}{20} & \frac{1}{20} & 0 \\
          -\frac{24}{840} &\frac{35}{840} & -\frac{21}{840} & \frac{10}{840}
       \end{array}\right]
       \left[\begin{array}{l}
         1 \\
         3 \\
         2 \\
         4 
       \end{array}\right]
  \\&= \left[\begin{array}{l}
         1 \\
         2 \\
         -\frac{9}{20} \\
         \frac{79}{840} 
       \end{array}\right]
  \\
  \\
  \\
  N_{P,3}(x)
    &= \Big[ \alpha_0 \; \alpha_1 \; \alpha_2 \; \alpha_3 \Big] 
       \left[\begin{array}{rrrr}
          1  &   0  & 0  & 0  \\
          2  &   1  & 0  & 0  \\
          2  &   3  & 1  & 0  \\
         -6  &  -7  & 0  & 1
       \end{array}\right]
       \left[\begin{array}{l}
         1   \\
         x   \\
         x^2 \\
         x^3
       \end{array}\right]
  \\&= \left[\begin{array}{c|c|c|c} 
         1 & 2 & -\frac{9}{20} & \frac{79}{840} 
       \end{array}\right]
       \left[\begin{array}{rrrr}
          1  &   0  & 0  & 0  \\
          2  &   1  & 0  & 0  \\
          2  &   3  & 1  & 0  \\
         -6  &  -7  & 0  & 1
       \end{array}\right]
       \left[\begin{array}{l}
         1   \\
         x   \\
         x^2 \\
         x^3
       \end{array}\right]
  \\&= \left[\begin{array}{c|c|c|c}
          1 + 4 - \frac{9}{10} - \frac{79}{140}  &
          2 - \frac{27}{20} -\frac{79}{120}      &
          -\frac{9}{20}                          &
          \frac{79}{840}
       \end{array}\right]
       \left[\begin{array}{l}
         1   \\
         x   \\
         x^2 \\
         x^3
       \end{array}\right]
  \\&= \frac{79}{840}x^3 - \frac{378}{840}x^2 - \frac{7}{840}x + \frac{2970}{840}
\end{align*}
\end{proof}




%---------------------------------------
\begin{example}[\exmd{Least squares polynomial interpolation}]
\index{polynomial!least squares}
\index{least squares}
\label{ex:ls}
%---------------------------------------
The best 3rd order polynomial in the \hib{least squares} $S_{P,3}(x)$ sense
with respect to the $4$ points\\
$P=\{(-2,1),\; (-1,3),\; (3,2),\; (5,4)\}$ is
\exbox{
  S_{P,3}(x) = \frac{   79}{840} x^3
             + \frac{- 378}{840} x^2
             + \frac{-   7}{840} x
             + \frac{ 2970}{840}
  }
\end{example}
\begin{proof}
%The {\em least squares} technique is presented in 
%Section~\ref{sec:ls} (page~\pageref{sec:ls}).
\[
  \vx
    =
    \left[\begin{array}{l}
       x_0  \\
       x_1  \\
       x_2  \\
       x_3  
    \end{array}\right] 
    =
    \left[\begin{array}{r}
       -2   \\
       -1   \\
        3   \\
        5   
    \end{array}\right] 
   \qquad\qquad
   \vy 
     =
     \left[\begin{array}{r}
         1   \\
         3   \\
         2   \\
         4   \\
     \end{array}\right] 
     =
     \left[\begin{array}{l}
         1   \\
         3   \\
         2   \\
         4   \\
     \end{array}\right] 
\]

We want to find a third order polynomial 
  \[ d x^3 + c x^2 + bx + a \]
that best approximates
the 4 points in the least squares sense.
We define the matrix $U$ (known) and vector $\estn$ (to be computed)
as follows:
\[
   U^H \eqd 
   \left[\begin{array}{llll}
      1  & x_0 & x_0^2 & x_0^3  \\
      1  & x_1 & x_1^2 & x_1^3  \\
      1  & x_2 & x_2^2 & x_2^3  \\
      1  & x_3 & x_3^2 & x_3^3  
   \end{array}\right]
   \qquad\qquad
   \estn \eqd 
   \left[\begin{array}{l}
      a  \\
      b  \\
      c  \\
      d
   \end{array}\right] 
\]
Then, using \thme{Least squares}\ifsxref{estimate}{thm:ls}, the best coefficients for the polynomial are
\begin{align*}
  \estn
    &= \left[\begin{array}{l}
          a  \\
          b  \\
          c  \\
          d
       \end{array}\right] 
  \\&= R^{-1}W
  \\&= (UU^H)^{-1}\; (U\vy)
  \\&= \left(
   \left[\begin{array}{llll}
      1  & x_0 & x_0^2 & x_0^3  \\
      1  & x_1 & x_1^2 & x_1^3  \\
      1  & x_2 & x_2^2 & x_2^3  \\
      1  & x_3 & x_3^2 & x_3^3  
   \end{array}\right]^H
   \left[\begin{array}{llll}
      1  & x_0 & x_0^2 & x_0^3  \\
      1  & x_1 & x_1^2 & x_1^3  \\
      1  & x_2 & x_2^2 & x_2^3  \\
      1  & x_3 & x_3^2 & x_3^3  
   \end{array}\right]
       \right)^{-1}
       \left(
   \left[\begin{array}{llll}
      1  & x_0 & x_0^2 & x_0^3  \\
      1  & x_1 & x_1^2 & x_1^3  \\
      1  & x_2 & x_2^2 & x_2^3  \\
      1  & x_3 & x_3^2 & x_3^3  
   \end{array}\right]^H
       \left[\begin{array}{l}
          y_0  \\
          y_1  \\
          y_2  \\
          y_3  
       \end{array}\right] 
       \right)
  \\&= \left(
   \left[\begin{array}{rrrr}
      1  & (-2) & (-2)^2 & (-2)^3  \\
      1  & (-1) & (-1)^2 & (-1)^3  \\
      1  & (3) & (3)^2 & (3)^3  \\
      1  & (5) & (5)^2 & (5)^3  
   \end{array}\right]^H
   \left[\begin{array}{rrrr}
      1  & (-2) & (-2)^2 & (-2)^3  \\
      1  & (-1) & (-1)^2 & (-1)^3  \\
      1  & (3) & (3)^2 & (3)^3  \\
      1  & (5) & (5)^2 & (5)^3  
   \end{array}\right]
       \right)^{-1}
       \left(
   \left[\begin{array}{rrrr}
      1  & (-2) & (-2)^2 & (-2)^3  \\
      1  & (-1) & (-1)^2 & (-1)^3  \\
      1  & (3) & (3)^2 & (3)^3  \\
      1  & (5) & (5)^2 & (5)^3  
   \end{array}\right]^H
       \left[\begin{array}{l}
          1  \\
          3  \\
          2  \\
          4  
       \end{array}\right] 
       \right)
  \\&= \left(
   \left[\begin{array}{rrrr}
      1  & -2 &  4 &  -8  \\
      1  & -1 &  1 &  -1  \\
      1  &  3 &  9 &  27  \\
      1  &  5 & 25 & 125  
   \end{array}\right]^H
   \left[\begin{array}{rrrr}
      1  & -2 &  4 &  -8  \\
      1  & -1 &  1 &  -1  \\
      1  &  3 &  9 &  27  \\
      1  &  5 & 25 & 125  
   \end{array}\right]
       \right)^{-1}
       \left(
   \left[\begin{array}{rrrr}
      1  & -2 &  4 &  -8  \\
      1  & -1 &  1 &  -1  \\
      1  &  3 &  9 &  27  \\
      1  &  5 & 25 & 125  
   \end{array}\right]^H
       \left[\begin{array}{l}
          1  \\
          3  \\
          2  \\
          4  
       \end{array}\right] 
       \right)
\\&= \left[\begin{array}{r}
       2970   \\
         -7   \\
       -378   \\
         79
     \end{array}\right]
\end{align*}
\end{proof}







%=======================================
\section{Hermite interpolation}
\label{sec:interpo_hermite}
%=======================================

\begin{table}
\footnotesize
\exbox{
  \renewcommand{\arraystretch}{1}
  \renewcommand{\arraycolsep}{0.4ex}
  \begin{array}{r|@{\qquad}*{13}{r}}
     p & \mc{13}{l}{\ds(1-y)^p\fP_m(y)= (1-y)^p \sum_{k=0}^{p-1} {p-1+k\choose k} y^k}
  %    p & \mc{20}{l}{(1-y)^p\fP_m(y)}
  \\ \hline
     1 & 1 &-&   y
  \\ 2 & 1 &-&  3y^2 &+&  2y^3
  \\ 3 & 1 &-& 10y^3 &+&  15y^4 &-&   6y^5
  \\ 4 & 1 &-&  35y^4 &+&  84y^5 &-&   70y^6  &+&    20y^7
  \\ 5 & 1 &-& 126y^5 &+&  420y^6 &-&  540y^7 &+&   315y^8 &-&    70y^9
  \\ 6 & 1 &-& 462y^6 &+& 1980y^7 &-& 3465y^8 &+&  3080y^9 &-&  1386y^{10} &+& 252y^{11}
  \end{array}}
\caption{
  Low-pass term $(1-y)^p\fP_m(y)$
  \label{tbl:P_lp}
  }
\end{table}




\begin{figure}[h]
\begin{scriptsize}
\begin{center}
\color[rgb]{0.2,0.2,0.2}
\begin{fsL}
\setlength{\unitlength}{0.3mm}
\begin{picture}(335,150)(-10,-30)
  %\graphpaper[10](-100,0)(300,150)
  \thicklines
  \put(   0,   0 ){\line(1,0){240} }
  \put(   0,   0 ){\line(0,1){120} }

  \qbezier[10](100,50)(100, 25)( 100,0)
  \qbezier[40](0,100)(100,100)( 200,100)

  {\color{blue}
    \qbezier(   0, 100)(  60, 100)( 100,  50)
    \qbezier( 100,  50)( 140,   0)( 200,   0)
    \put( 130,  25 ){\makebox(0,0)[l]{$\leftarrow (1-y)^p \fP(y)$} }
    \put(   5, 105 ){\makebox(0,0)[lb]{$\downarrow$ first $p-1$ derivatives are zero at $y=0$} }
    \put( 200,   5 ){\makebox(0,0)[lb]{$\downarrow$ first $p-1$ derivatives are zero at $y=1$} }
    }

  {\color{red}
    \qbezier(   0,   0)(  60,   0)( 100,  50)
    \qbezier( 100,  50)( 140, 100)( 200, 100)
    \put( 130,  75 ){\makebox(0,0)[l]{$\leftarrow y^p \fP(1-y)$} }
    \put(   5,  -3 ){\makebox(0,0)[lt]{$\uparrow$ first $p-1$ derivatives are zero at $y=0$} }
    \put( 200,  95 ){\makebox(0,0)[lt]{$\uparrow$ first $p-1$ derivatives are zero at $y=1$} }
    }

  \put( 100,   3 ){\makebox(0,0)[b]{$\frac{\pi}{2}$} }
  \put( 260,   0 ){\makebox(0,0)[l]{$\omega$} }
  \put( -10, 100 ){\makebox(0,0)[r]{$1$} }
\end{picture}
\end{fsL}
\end{center}
\end{scriptsize}
\caption{
  Polynomial quadrature condition low-pass and high-pass terms
  \label{fig:lphp}
  }
\end{figure}
The  quadrature condition can be expressed as a polynomial 
in $y=\sin^2\frac{\omega}{2}$\ifsxref{compactp}{lem:quadcon_y}.
The first term in this polynomial quadrature condition is a
low-pass response and the second term is a high pass;
and they meet in the middle at $\omega=\frac{\pi}{2}$.
    \[
      \mcom{(1-y)^p \fP(y)}{low-pass} + \mcom{y^p \fP(1-y)}{high-pass} = 1
    \]
The low-pass and high-pass terms are especially smooth at
$\omega=0$ ($y=0$) and $\omega=\pi$ ($y=1$)
in that the first $p-1$ derivatives at both points are zero
for both terms.
This is illustrated in \prefpp{fig:lphp}.
%--------------------------------------
\begin{theorem}[\thme{Hermite Interpolation}]
\label{thm:interpo_hermite}
%--------------------------------------
\thmbox{\begin{array}{*{3}{>{\ds}l}D}
  \left.\deriv{^n}{y^n} \left[
        (1-y)^p \sum_{k=0}^{p-1}{p+k-1\choose k} y^k
        \right]\right|_{y=0}
    &=& \kdelta_n
    &   for $n=0,1,2,\ldots,p-1$
  \\
  \left.\deriv{^n}{y^n} \left[
        (1-y)^p \sum_{k=0}^{p-1} {p+k-1\choose k} y^k
        \right]\right|_{y=1}
    &=& 0
    &   for $n=0,1,2,\ldots,p-1$
  \\
  \left.\deriv{^n}{y^n} \left[
        y^p \sum_{k=0}^{p-1}{p+k-1\choose k} (1-y)^k
        \right]\right|_{y=0}
    &=& 0
    &   for $n=0,1,2,\ldots,p-1$
  \\
  \left.\deriv{^n}{y^n} \left[
        y^p \sum_{k=0}^{p-1} {p+k-1\choose k} (1-y)^k
        \right]\right|_{y=1}
    &=& \kdelta_n
    &   for $n=0,1,2,\ldots,p-1$
  \end{array}}
\end{theorem}
\begin{proof}
Let
\begin{align*}
  \ff(y) &\eqd (1-y)^p \sum_{n=0}^{p-1} {p-1+n \choose n} y^n      \\
  \fg(y) &\eqd y^p     \sum_{n=0}^{p-1} {p-1+n \choose n} (1-y)^n  \\
  q      &\eqd p-1
\end{align*}
%\begin{enumerate}
%  \item Proofs for $\ff^{(n)}(0)$:
  \begin{enumerate}
  \item Proof that $\ff(0)=1$:
  \begin{align*}
  \ff(0)
    &= \left.(1-y)^p \sum_{m=0}^{p-1} {p-1+m\choose m} y^m \right|_{y=0}
  \\&= \left.(1-y)^p \left[{p-1 \choose 0} + \sum_{m=1}^{p-1} {p-1+m\choose m} y^m\right]\right|_{y=0}
  \\&= 1
  \end{align*}

  \item Proof that
      $\ds\ff(y)=
       p \sum_{n=0}^{2p-1}
       \left[\sum_{k=\max(0,n-q)}^{\min(n,p)} (-1)^k \frac{(p+n-k-1)!}{(p-k)!(n-k)!\,k!}\right]
       y^n$:
  \begin{align*}
  (1-y)^p \fP_m(y)
    &= \sum_{n=0}^p     {p\choose n} (-1)^n y^n
       \sum_{m=0}^{p-1} {p-1+m\choose m} y^m
  \\&= \sum_{n=0}^{2p-1} \sum_{k=\max(0,n-q)}^{\min(n,p)}
       {p\choose k} (-1)^k {p-1+n-k\choose n-k} y^n
    && \text{by \pref{thm:polymult} \prefpo{thm:polymult}}
  \\&= \sum_{n=0}^{2p-1} \sum_{k=\max(0,n-q)}^{\min(n,p)}
       (-1)^k \frac{p!}{(p-k)!k!} \frac{(p-1+n-k)!}{(p-1)!(n-k)!} y^n
  \\&= p \sum_{n=0}^{2p-1}
       \left[\sum_{k=\max(0,n-q)}^{\min(n,p)} (-1)^k \frac{(p+n-k-1)!}{(p-k)!(n-k)!\,k!}\right]
       y^n
  \end{align*}

  \item Proof that $\ff^{(n)}(0)=\kdelta_n$ for $n=0,1,2,\ldots,p-1$:
  \begin{align*}
  \left.\deriv{^n}{y^n}\left[ (1-y)^p \fP_m(y) \right]\right|_{y=0}
    &= \left.
       \deriv{^n}{y^n}\left[
         p \sum_{m=0}^{2p-1}
         \left[\sum_{k=\max(0,m-q)}^{\min(m,p)} (-1)^k \frac{(p+m-k-1)!}{(p-k)!(m-k)!\,k!}\right]
         y^m
       \right]
       \right|_{y=0}
    && \text{by 1.}
  \\&= \left. p
       \sum_{m=n}^{2p-1} \sum_{k=\max(0,m-q)}^{\min(m,p)}
       (-1)^k \frac{(p-1+m-k)!}{(p-k)!(m-k)!k!} \frac{m!}{(m-n)!}y^{m-n}
       \right|_{y=0}
  \\&= p \sum_{k=\max(0,n-q)}^{\min(n,p)}
       (-1)^k \frac{(p-1+n-k)!}{(p-k)!} \frac{n!}{(n-k)!k!}
  \\&= p \sum_{k=0}^n
       (-1)^k {n\choose k} \frac{(p+n-k-1)!}{(p-k)!}
  \\&\eqq \kdelta_n \qquad\text{for } n=0,1,2,\ldots,p-1
  \end{align*}

  \item Proof that $\ff^{(n)}(0)=\kdelta_n$ for $n=0,1,2,\ldots,p-1$:
  \begin{align*}
  &\left.\deriv{^n}{y^n}\left[ (1-y)^p \fP_m(y) \right]\right|_{y=0}
  \\&= \left.
       \sum_{k=0}^n {n\choose k}
       \left[\deriv{^{n-k}}{y^{n-k}} (1-y)^p \right]
       \left[\deriv{^k}{y^k}\fP_m(y) \right]
       \right|_{y=0}
    \qquad\text{by \pref{lem:LGPR} (Leibnitz rule)}
  \\&= \left.
       \sum_{k=0}^n {n\choose k}
       \left[\deriv{^{n-k}}{y^{n-k}} (1-y)^p \right]
       \left[\deriv{^k}{y^k} \sum_{m=0}^{p-1}{p-1+m \choose m} y^m \right]
       \right|_{y=0}
    \quad\text{by definition of $\fP_m(y)$}
  \\&= \left.
       \sum_{k=0}^n {n\choose k}
       \left[(-1)^{n-k}\frac{p!}{(p-n+k)!} (1-y)^{(p-n+k)} \right]
       \left[\sum_{m=k}^{p-1}{p-1+m \choose m} \frac{m!}{(m-k)!}y^{m-k} \right]
       \right|_{y=0}
  \\&= \sum_{k=0}^n {n\choose k}
       \left[(-1)^{n-k}\frac{p!}{(p-n+k)!}\right]
       \left[{p-1+k \choose k} k! \right]
  \\&= \sum_{k=0}^n {n\choose k}
       \left[(-1)^{n-k}\frac{p!}{(p-n+k)!}\right]
       \left[\frac{(p-1+k)!}{(p-1)!k!}\; k! \right]
  \\&= (-1)^n p
       \sum_{k=0}^n {n\choose k}
       (-1)^k\frac{(p+k-1)!}{(p+k-n)!}
  \\&\eqq \kdelta_n \qquad \text{for } k=0,1,2,\ldots,p-1
  \end{align*}

  \item Proof that $\ff^{(n)}(1)=0$ for $n=0,1,2,\ldots,p-1$:
  \begin{align*}
  \left.\deriv{^n}{y^n}\left[ (1-y)^p \fP_m(y) \right]\right|_{y=1}
    &= \left.
       \sum_{k=0}^n {n\choose k}
       \left[\deriv{^k}{y^k} (1-y)^p \right]
       \fP_m^{(n-k)}(y)
       \right|_{y=1}
    && \text{by \pref{lem:LGPR} (Leibnitz rule)}
  \\&= \left.
       \sum_{k=0}^n {n\choose k}
       \left[(-1)^k\frac{p!}{(p-k)!} (1-y)^{p-k} \right]
       \fP_m^{(n-k)}(y)
       \right|_{y=1}
  \\&= \left.
       \sum_{k=0}^n {n\choose k} \; 0\cdot
       \fP_m^{(n-k)}(y)
       \right|_{y=1}
    && \text{by \pref{lem:LGPR}}
  \\&= 0 \qquad \text{for } k=0,1,2,\ldots,p-1
  \end{align*}
\end{enumerate}
%\end{enumerate}
\end{proof}


\ifdochasnot{sampling}{%============================================================================
% Daniel J. Greenhoe
% LaTeX/XeLaTeX file
%============================================================================
%=======================================
%\chapter{Cardinal Sequence and Series}
%=======================================

%=======================================
\section{Cardinal Series and Sampling}
%=======================================
%======================================
\subsection{Cardinal series basis}
\label{sec:cardinal}
%======================================
The \prope{Paley-Wiener} class of functions (next definition) are those with a bandlimited Fourier transform.
The cardinal series forms an orthogonal basis for such a space \xrefP{thm:cardinalSeries}.
In a \structe{frame} $\seqnZ{\vx_n}$ with \ope{frame operator} $\opS$ on a \structe{Hilbert Space} $\spH$
with \fncte{inner product} $\inprodn$, 
a function $\ff(x)$ in the space spanned by the frame can be represented by
\\\indentx$\ds\ff(x) = \sum_{n\in\Z} \mcom{\inprod{\ff}{\opSi\vx_n}}{``\fncte{Fourier coefficient}"}\vx_n$.\\
If the frame is \prope{orthonormal} (giving an \structe{orthonormal basis}), then $\opS=\opSi=\opI$ and 
\\\indentx$\ds\ff(x) = \sum_{n\in\Z} \inprod{\ff}{\vx_n}\vx_n$.\\
In the case of the cardinal series, 
the \fncte{Fourier coefficients}\ifsxref{frames}{def:fcoef} are particularly 
simple---these coefficients are samples of $\ff$ taken at regular intervals \xrefP{thm:sampling}.
In fact, one could represent the coefficients using inner product notation with the 
\structe{Dirac delta distribution} $\delta$ \ifxref{relation}{def:dirac} as
follows:
\\\indentx$\ds\inprod{\ff(x)}{\delta(x-n\tau)} \eqd \int_{\R} \ff(x)\delta(x-n\tau) \dt \eqd \ff(n\tau)$

%--------------------------------------
\begin{definition}
\footnote{
  %\citerp{higgins1985}{56}\\
  \citerpgc{higgins1996}{52}{0198596995}{Definition 6.15}
  %\citerp{hardy1941}{332}
  }
\label{def:PW}
%--------------------------------------
\defbox{\begin{array}{M}
  A function $\ff\in\clFcc$ is in the \hid{Paley-Wiener} class of functions $\ds\spPW^p_\sigma$ if\\
  there exists $\fF\in\spL^p\intoo{-\sigma}{\sigma}$ such that 
  \\\qquad$\ds\ff(x) = \int_{-\sigma}^{\sigma} \fF(\omega)e^{ix\omega}\dw$
  \qquad({\scs $\ff$ has a \prope{bandlimited} Fourier transform $\fF$  with bandwidth $\sigma$})
  \\for $p\in\intco{1}{\infty}$ and $\sigma\in\intoo{0}{\infty}$.
\end{array}}
\end{definition}

%--------------------------------------
\begin{theorem}[\thmd{Paley-Wiener Theorem for Functions}]
\footnote{
  \citerpgc{boas1954}{103}{0123745829}{6.8.1 Theorem of Paley and Wiener},
  \citerpgc{katznelson2004}{212}{0521543592}{7.4 Theorem},
  \citerppgc{zygmund1968v2}{272}{273}{0521890535}{(7$\cdot$2) \scshape Theorem of Paley-Wiener},
  %\citerpg{yosida1971}{161}{3540055061},
  \citerpg{yosida1980}{161}{3540586547},
  \citerpgc{rudinr}{375}{0070542341}{19.3 Theorem},
  \citerpgc{young2001}{85}{0127729550}{Theorem 18}
  }
%--------------------------------------
Let $\ff$ be an \structe{entire function} (the domain of $\ff$ is the entire complex plane $\C$).
Let $\sigma\in\Rp$.
\thmbox{
  \brb{\ff\in\spPW^2_\sigma}
  \iff
  \brb{\begin{array}{F>{\ds}lDD}
    1. & \exists C\in\Rp\st\abs{\ff(z)} \le Ce^{\sigma\abs{z}} & (\prope{exponential type}) & and\\
    2. & \ff\in\spLLR
    %2. & \int_{\R} \abs{\ff(x)}^2 \ds < \infty 
  \end{array}}
  }
\end{theorem}

\includegraphics{graphics/sinc.pdf}\\
%--------------------------------------
\begin{theorem}[\thmd{Cardinal sequence}]
%\begin{theorem}
\footnote{
  \citerpgc{higgins1996}{52}{0198596995}{Definition 6.15},
  \citerc{hardy1941}{\prope{orthonormality}},
  \citerpc{higgins1985}{56}{H1.; historical notes}
  %
  %\citerp{higgins1985}{47}\\
  }
\label{thm:cardinalSeries}
%--------------------------------------
%Let $\ff\in\spLLR$ and $\Ff(\omega)$ have bandwidth $W$ such that
%$\Ff(\omega)=0$ for $\abs{\frac{\omega}{2\pi}}>W$. \\
%If the sample rate $$ then
\thmbox{
  %\mcom{\frac{1}{\tau}\ge 2\sigma}{sample rate $\frac{1}{\tau}$ $\ge$ $2\times$ bandwidth $\sigma$}
  \brb{\frac{1}{\tau}\ge 2\sigma}
  \implies
  \text{The sequence}\quad
  %\setbigleft{\frac{\sin\left[\frac{\pi}{\tau}(x-n\tau)\right]}{\frac{\pi}{\tau}(x-n\tau)}}{n\in\Z}
  \seqnZ{\frac{\sin\left[\frac{\pi}{\tau}(x-n\tau)\right]}{\frac{\pi}{\tau}(x-n\tau)}}
  \text{is an \prope{orthonormal} \structe{basis} for $\spPW_\sigma^2$.}
  }
\end{theorem}


%--------------------------------------
\begin{theorem}[\thmd{Sampling Theorem}]
%\begin{theorem}
\footnote{
  %\citerpc{higgins1985}{56}{H1.; historical notes}\\
  %\citerp{higgins1985}{47}\\
  \citor{etwhittaker1915},
  \citor{kotelnikov1933e},
  \citor{jmwhittaker1935},
  \citePc{shannon1948}{Theorem 13},
  \citePp{shannon1949}{11}
  \citerpg{marks1991}{1}{0387973915},
  \citer{nashed1991},
  \citerpg{higgins1996}{5}{0198596995},
  \citerppgc{young2001}{90}{91}{0127729550}{\scshape The Paley-Wiener Space}.
  The {\em sampling theorem} was ``discovered" and published by multiple people: 
  Nyquist in 1928 (DSP?), 
  Whittaker in 1935 (interpolation theory),
  and Shannon in 1949 (communication theory).
  references: \citerp{mallat}{43},
  \citerp{os}{143}.
  }
\label{thm:t_sampling}
\label{thm:sampling}
%--------------------------------------
%Let $\ff\in\spLLR$ and $\Ff(\omega)$ have bandwidth $W$ such that
%$\Ff(\omega)=0$ for $\abs{\frac{\omega}{2\pi}}>W$. \\
%If the sample rate  then
\thmbox{
  \brb{\begin{array}{F>{\ds}lD}
    1. & \ff\in\spPW_\sigma^2 & and\\
    2. & \frac{1}{\tau}\ge 2\sigma
  \end{array}}
  \qquad\implies\qquad
  \ff(x) = \mcom{\ds\sum_{n=1}^\infty \ff(n\tau) \frac{\sin\brs{\frac{\pi}{\tau}(x-n\tau)}}{\frac{\pi}{\tau}(x-n\tau)}}
                {\hie{Cardinal series}}.
  }
\end{theorem}
\begin{proof}
\[ \mbox{Let }\hspace{3ex}
   \fs(x) \eqd \frac{\sin\left[\frac{\pi}{\tau}x\right]}{\frac{\pi}{\tau}x}
   \iff
   \Fs(\omega) = \left\{\begin{array}{ll}
      \tau & : |f|\le \frac{1}{2\tau} \\
      0 & : \mbox{otherwise}
      \end{array}\right.
\]

\begin{enumerate}
  \item Proof that the set is \prope{orthonormal}: see \citer{hardy1941}

  \item Proof that the set is a \prope{basis}:
    \begin{align*}
      \ff(x)
        &= \int_\omega \Ff(\omega) e^{i\omega t} \dw
        && \text{by \thme{inverse Fourier transform}}
        && \text{\xref{thm:opFTi}}
      \\&= \int_\omega \opT \Ffd(\omega)\Fs(\omega) e^{i\omega t} \dw
        && \text{if $W\le\frac{1}{2T}$}
      \\&= \opT \ffd(x) \conv \fs(x) 
        && \text{by \thme{Convolution theorem}}
        && \text{\xref{thm:conv}}
      \\&= \opT \int_u [\ffd(u)] \fs(x-u) \du
        && \text{by \ope{convolution} definition}
        && \text{\xref{def:conv}} 
      \\&= \opT \int_u \left[ \sum_{n\in\Z}\ff(u)\delta(u-n\tau) \right] \fs(x-u) \du
        && \text{by \ope{sampling} definition}
        && \text{\xref{thm:f_sampling}}
      \\&= \opT \sum_{n\in\Z} \int_u \ff(u)\fs(x-u)\delta(u-n\tau) \du
      \\&= \opT \sum_{n\in\Z} \ff(n\tau)\fs(x-n\tau)
        && \text{by prop. of \fncte{Dirac delta}}
      \\&= \opT \sum_{n\in\Z} \ff(n\tau) 
             \frac{\sin\left[\frac{\pi}{\tau}(x-n\tau)\right]}
                  {\frac{\pi}{\tau}(x-n\tau)       }
        && \text{by definition of $\fs(x)$}
    \end{align*}
\end{enumerate}
\end{proof}


%======================================
\subsection{Sampling}
\label{sec:sampling}
\index{sampling}
%======================================
If $\ffd(x)$ is the function $\ff(x)$ sampled at rate $1/T$, 
then $\Ffd(\omega)$ is simply $\Ff(\omega)$
{\em replicated} every $1/T$ Hertz and {\em scaled} by $1/T$.
This is proven in Theorem~\ref{thm:f_sampling} (next) and 
illustrated in \prefpp{fig:f_sampling}.


\begin{figure}[ht]
\index{nyquist sampling rate}
\setlength{\unitlength}{0.1mm}
\begin{center}
\begin{fsL}
\begin{tabular}{c}
%
\begin{picture}(1550,200)(-750,-50)
  %\graphpaper[10](0,0)(600,200)
  \thicklines
  \put(  40,  70){\makebox(0,0)[lb]{$\Ff(\omega)$} }
  \put(-750,   0){\line( 1,0){1500} }
  \put(   0, -50){\line( 0,1){ 200} }
  \put(-100,   0){\line( 1,1){ 100} }
  \put( 100,   0){\line(-1,1){ 100} }
  \put(- 10, 100){\line( 1,0){  20} }
  \put(- 20, 100){\makebox(0,0)[r]{$A$} }
  \put(-100, -10){\line( 0,1){  20} }
  \put( 100, -10){\line( 0,1){  20} }
  \put(-100, -20){\makebox(0,0)[t]{$-W$} }
  \put( 100, -20){\makebox(0,0)[t]{$+W$} }
  \put( 770,   0){\makebox(0,0)[l]{$f$} }
\end{picture}
\\
\begin{picture}(1550,200)(-750,-50)
  %\graphpaper[10](0,0)(600,200)
  \thicklines
  \put(  20, 140){\makebox(0,0)[lt]{$\Ffd(\omega)$ at sample rate $\frac{1}{\tau}=3W$ (``oversampling")} }
  \multiput(-600,0)(+300,0){5}{
     \put(-100,   0){\line( 1,1){ 100} }
     \put( 100,   0){\line(-1,1){ 100} }
     }
  \put(-750,   0){\line( 1,0){1500} }
  \put(   0, -50){\line( 0,1){ 200} }
  \put(- 10, 100){\line( 1,0){  20} }
  \put(- 20, 100){\makebox(0,0)[r]{$\frac{A}{\tau}$} }
  \put(-100, -10){\line( 0,1){  20} }
  \put( 100, -10){\line( 0,1){  20} }
  \put(-100, -20){\makebox(0,0)[t]{$-W$} }
  \put( 100, -20){\makebox(0,0)[t]{$+W$} }
  \put( 770,   0){\makebox(0,0)[l]{$f$} }

  \put(-600, -10){\line( 0,1){  20} }
  \put(-600, -20){\makebox(0,0)[t]{$\frac{-2}{\tau}$} }
  \put(-300, -10){\line( 0,1){  20} }
  \put(-300, -20){\makebox(0,0)[t]{$\frac{-1}{\tau}$} }
  \put( 300, -10){\line( 0,1){  20} }
  \put( 290, -20){\makebox(0,0)[lt]{$\frac{1}{\tau}=3W$} }
  \put( 600, -10){\line( 0,1){  20} }
  \put( 600, -20){\makebox(0,0)[t]{$\frac{2}{\tau}$} }
\end{picture}
\\
\begin{picture}(1550,200)(-750,-50)
  %\graphpaper[10](0,0)(600,200)
  \thicklines
  \put(  20, 80){\makebox(0,0)[lb]{$\Ffd(\omega)$ at sample rate $\frac{1}{\tau}=2W$ (at Nyquist rate)} }
  \multiput(-600,0)(+200,0){7}{
     \put(-100,   0){\line( 3,2){ 100} }
     \put( 100,   0){\line(-3,2){ 100} }
     }
  \put(-750,   0){\line( 1,0){1500} }
  \put(   0, -50){\line( 0,1){ 200} }
  \put(- 10,  67){\line( 1,0){  20} }
  \put(- 20,  67){\makebox(0,0)[r]{$\frac{A}{\tau}$} }
  \put(-100, -10){\line( 0,1){  20} }
  \put( 100, -10){\line( 0,1){  20} }
  \put(-100, -20){\makebox(0,0)[t]{$-W$} }
  \put( 100, -20){\makebox(0,0)[t]{$+W$} }
  \put( 770,   0){\makebox(0,0)[l]{$f$} }

  \put(-600, -10){\line( 0,1){  20} }
  \put(-600, -20){\makebox(0,0)[t]{$\frac{-3}{\tau}$} }
  \put(-400, -10){\line( 0,1){  20} }
  \put(-400, -20){\makebox(0,0)[t]{$\frac{-2}{\tau}$} }
  \put(-200, -10){\line( 0,1){  20} }
  \put(-200, -20){\makebox(0,0)[t]{$\frac{-1}{\tau}$} }
  \put( 200, -10){\line( 0,1){  20} }
  \put( 190, -20){\makebox(0,0)[lt]{$\frac{1}{\tau}=2W$} }
  \put( 400, -10){\line( 0,1){  20} }
  \put( 400, -20){\makebox(0,0)[t]{$\frac{2}{\tau}$} }
  \put( 600, -10){\line( 0,1){  20} }
  \put( 600, -20){\makebox(0,0)[t]{$\frac{3}{\tau}$} }
\end{picture}
\\
\begin{picture}(1550,200)(-750,-50)
  %\graphpaper[10](0,0)(600,200)
  \thicklines
  \put(  20, 43){\makebox(0,0)[lb]{$\Ffd(\omega)$ at sample rate $\frac{1}{\tau}=W$ (``undersampling")} }
  \multiput(-600,0)(+100,0){13}{
     \put(-100,   0){\line( 3,1){ 100} }
     \put( 100,   0){\line(-3,1){ 100} }
     }
  \put(-750,   0){\line( 1,0){1500} }
  \put(   0, -50){\line( 0,1){ 200} }
  \put(- 10,  33){\line( 1,0){  20} }
  \put(- 20,  33){\makebox(0,0)[r]{$\frac{A}{\tau}$} }
  \put(-100, -10){\line( 0,1){  20} }
  \put( 100, -10){\line( 0,1){  20} }
  %\put(-100, -20){\makebox(0,0)[t]{$-W$} }
  %\put( 100, -20){\makebox(0,0)[t]{$+W$} }
  \put( 770,   0){\makebox(0,0)[l]{$f$} }

  \put(-600, -10){\line( 0,1){  20} }
  \put(-600, -20){\makebox(0,0)[t]{$\frac{-6}{\tau}$} }
  \put(-500, -10){\line( 0,1){  20} }
  \put(-500, -20){\makebox(0,0)[t]{$\frac{-5}{\tau}$} }
  \put(-400, -10){\line( 0,1){  20} }
  \put(-400, -20){\makebox(0,0)[t]{$\frac{-4}{\tau}$} }
  \put(-300, -10){\line( 0,1){  20} }
  \put(-300, -20){\makebox(0,0)[t]{$\frac{-3}{\tau}$} }
  \put(-200, -10){\line( 0,1){  20} }
  \put(-200, -20){\makebox(0,0)[t]{$\frac{-2}{\tau}$} }
  \put(-100, -10){\line( 0,1){  20} }
  \put(-100, -20){\makebox(0,0)[t]{$\frac{-1}{\tau}$} }
  \put( 100, -10){\line( 0,1){  20} }
  \put( 100, -20){\makebox(0,0)[ct]{$\frac{1}{\tau}=W$} }
  \put( 200, -10){\line( 0,1){  20} }
  \put( 200, -20){\makebox(0,0)[t]{$\frac{2}{\tau}$} }
  \put( 300, -10){\line( 0,1){  20} }
  \put( 300, -20){\makebox(0,0)[t]{$\frac{3}{\tau}$} }
  \put( 400, -10){\line( 0,1){  20} }
  \put( 400, -20){\makebox(0,0)[t]{$\frac{4}{\tau}$} }
  \put( 500, -10){\line( 0,1){  20} }
  \put( 500, -20){\makebox(0,0)[t]{$\frac{5}{\tau}$} }
  \put( 600, -10){\line( 0,1){  20} }
  \put( 600, -20){\makebox(0,0)[t]{$\frac{6}{\tau}$} }
\end{picture}
\end{tabular}
\end{fsL}
\end{center}
\caption{
   Sampling in frequency domain
   \label{fig:f_sampling}
   }
\end{figure}

%--------------------------------------
\begin{theorem}
\label{thm:f_sampling}
%--------------------------------------
Let $\ff,\ffd\in\spLLR$ and $\Ff,\Ffd\in\spLLR$ 
be their respective fourier transforms.
Let $\ffd(x)$ be the {\bf sampled} $\ff(x)$ such that
  \[ \ffd(x) \eqd \sum_{n\in\Z} \ff(x) \delta(x-n\tau). \]
\thmbox{
  \mcom{\brb{\ffd(x) \eqd \sum_{n\in\Z} \ff(x) \delta(x-n\tau)}}{\fnctd{Shah function}}
  \quad\implies\quad
  \brb{\Ffd(\omega) = \frac{2\pi}{\tau}\sum_{n\in\Z} \Ff\brp{\omega -\frac{2\pi}{\tau}n}}
   }
\end{theorem}
\begin{proof}
\begin{align*}
   \Ffd(\omega) 
     &\eqd \int_t \ffd(x) e^{-i\omega t} \dt
   \\&=    \int_t \left[\sum_{n\in\Z} \ff(x) \delta(x-n\tau)\right] e^{-i\omega t} \dt
   \\&=    \sum_{n\in\Z} \int_t \ff(x) \delta(x-n\tau) e^{-i\omega t} \dt
   \\&=    \sum_{n\in\Z} \ff(n\tau) e^{-i\omega n\tau}
     &&    \text{by definition of $\delta$}
     &&    \text{\ifxref{relation}{def:dirac}}
   \\&=    \frac{2\pi}{\tau} \sum_{n\in\Z} \Ff\left(\omega + \frac{2\pi}{\tau}n \right)
     &&    \text{by \thme{IPSF}}
     &&    \text{\xref{thm:ipsf}}
   \\&=    \frac{2\pi}{\tau} \sum_{n\in\Z} \Ff\left(\omega-\frac{2\pi}{\tau}n\right)
\end{align*}
\end{proof}

Suppose a waveform $\ff(x)$ is sampled at every time $T$
generating a sequence of sampled values $\ff(n\tau)$.
Then in general, we can {\em approximate} $\ff(x)$ by 
using interpolation between the points $\ff(n\tau)$.
Interpolation can be performed using several interpolation techniques. %:\\
%\begin{tabular}{lll}
%  $\imark$ & 0th order interpolation: & convolution with rectangular functions \\
%  $\imark$ & 1st order interpolation: & convolution with triangular functions  \\
%  $\imark$ & 3rd order interpolation: & cubic splines \footnotemark            \\
%  $\imark$ & lowpass smoothing        & convolution with the ``sinc" function 
%\end{tabular}
%\footnotetext{
%  A {\em cubic spline} is a third order polynomial $ax^3+bx^2+cx+d$.
%  Such a polynomial has 4 {\em degrees of freedom}. 
%  Two of these degrees of freedom are used to make the polymomial match
%  at the interpolation interval endpoints.
%  The other two degrees of freedom are used to make the slopes of the 
%  polynomial match at the interval endpoints.
%  }

In general all techniques lead only to an approximation of $\ff(x)$.
However, if $\ff(x)$ is \prope{bandlimited} with bandwidth 
$W\le\frac{1}{2T}$,
then $\ff(x)$ is {\em perfectly reconstructed} (not just approximated)
from the sampled values $\ff(n\tau)$ \xref{thm:sampling}.
 %using ``\hie{lowpass smoothing}" (the last technique in the list).
%This perfect reconstruction is demonstrated in \prefp{thm:sampling}.









}




