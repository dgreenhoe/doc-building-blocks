%============================================================================
% Daniel J. Greenhoe
% LaTeX File
%============================================================================

%=======================================
\chapter{Pseudo Random generators}
%=======================================




\index{m-sequence} \index{Gold sequence}
The most basic binary pn-sequence is the {\em m-sequence}
(maximal length sequence).
From this basic sequence, other sequences can be constructed
such as {\em Gold} sequences.

%=======================================
\section{Generating m-sequences mathematically}
\label{sec:math}
%=======================================
%=======================================
\subsection{Definitions}
\label{sec:def}
%=======================================
An m-sequence can be represented as the coefficients of a \hie{polynomial}
over a \hie{finite field}.
Any \hie{field} is defined by the triplet $(S,+,\cdot)$,
where

\begin{tabular}{clrlll}
   $S$:     & a set \\
   $+$:     & addition   operation in the form       &$+:$     &$S\times S\to S$ \\
   $\cdot$: & multiplication   operation in the form &$\cdot:$ &$S\times S\to S $
\end{tabular}


%--------------------------------------
\begin{definition} {\bf Galois Field 2, GF(2)} \\
\label{def:gf2}
\index{Galois field}
\index{GF(2)}
%--------------------------------------
GF(2) is the field $(S,+,\cdot)$ with members of the triplet defined as

\begin{tabular}{ccccc}
\begin{math}
\begin{array}{ccl}
   S     &=& \{0,1\}                        \\
   +     &:& \{0,1\}\times\{0,1\}\to\{0,1\} \\
   \cdot &:& \{0,1\}\times\{0,1\}\to\{0,1\}
\end{array}
\end{math}
&
 such that
&
\begin{math}
\begin{array}{cc|c}
   a & b & a+b \\
   \hline
   0 & 0 & 0 \\
   0 & 1 & 1 \\
   1 & 0 & 1 \\
   1 & 1 & 0 \\
\end{array}
\end{math}
&and&
\begin{math}
\begin{array}{cc|c}
   a & b & a\cdot b \\
   \hline
   0 & 0 & 0 \\
   0 & 1 & 0 \\
   1 & 0 & 0 \\
   1 & 1 & 1 \\
\end{array}
\end{math}
\end{tabular}

\end{definition}

\index{GF(2)!polynomials over}
M-sequences can be generated and represented as
{\em polynomials over GF(2)}.
A polynomial over GF(2) is a polynomial with coefficients selected from GF(2).
An example of a polynomial over GF(2) is
   \[ 1 + x^2 + x^5 + x^6 + x^7 + x^9. \]

The generation of an m-sequence is equivalent to polynomial division,
which is very similar to integer division.

%--------------------------------------
\begin{definition} {\bf Polynomial division} \\
%--------------------------------------
The quantities of polynomial division are identified as follows:

\begin{center}
\begin{tabular}{ccc}
$\ds \frac{d(x)}{p(x)} = q(x) + \frac{r(x)}{p(x)}$
&where&
\begin{tabular}{llll}
   d(x) & is the dividend \\
   p(x) & is the divisor  \\
   q(x) & is the quotient \\
   r(x) & is the remainder.
\end{tabular}
\end{tabular}
\end{center}
\end{definition}

The ring of integers $\Z$ contains some special elements called {\em primes}
which can only be divided\footnote{
   The expression ``$a$ divides $b$" means that $b/a$ has remainder 0.
}
 by themselves or 1.
Rings of polynomials have a similar elements called {\em primitive polynomials}.

%--------------------------------------
\begin{definition}[\hid{Primitive polynomial}]
\label{def:pn_primitive}
%--------------------------------------
A primitive polynomial $p(x)$ of order $n$ has the properties
\begin{enumerate}
\setlength{\itemsep}{0ex}
   \item $p(x)$ cannot be factored
   \item the smallest order polynomial that $p(x)$ can divide is $x^{2^n-1}+1=0$.
\end{enumerate}
\end{definition}

%--------------------------------------
\begin{example}
\label{ex:pn_primitive}
%--------------------------------------
Some examples\citepp{wicker}{465}{475} of primitive polynomials over $GF(2)$ are
primitive polynomials: \url{http://www.ece.cmu.edu/~koopman/lfsr/index.html}

\[\begin{array}{l|l}
   \text{primitive polynomial} & \text{recipricol}
   \\\hline
   \\\hline
   x + 1                        &
                                \\
   \hline
   x^2 + x   + 1                &
                                \\
   \hline
   x^3 + x   + 1                &
   x^3 + x^2 + 1                \\
   \hline
   x^4 + x   + 1                &
   x^4 + x^3 + 1                \\
   \hline
   x^5+x^4+x^2+x+1              &
   x^5+x^4+x^3+x+1              \\
   x^5+x^3+x^2+x+1              &
   x^5+x^4+x^3+x^2+1            \\
   x^5+x^2+1                    &
   x^5+x^3+1                    \\
   \hline
   x^6 + x + 1                  &
   x^6 + x^5 + 1                \\
   x^6 + x^4 + x^3 + x + 1      &
   x^6 + x^5 + x^3 + x^2 + 1    \\
   x^6 + x^5 + x^2 + x + 1      &
   x^6 + x^5 + x^4 + x + 1      \\\hline
\end{array}\]


\footnote{\begin{tabular}[t]{>{$}l<{$}l}
  \frac{1}{x^3+x+1}:           & \url{http://www.research.att.com/~njas/sequences/A011657} \\
  \frac{1}{x^3+x^2+1}:         & \url{http://www.research.att.com/~njas/sequences/A011656} \\
  \frac{1}{x^4+x+1}:           & \url{http://www.research.att.com/~njas/sequences/A011659} \\
  \frac{1}{x^5+x^4+x^2+x+1}    & \url{http://www.research.att.com/~njas/sequences/A011660} \\
  \frac{1}{x^5+x^3+x^2+x+1}    & \url{http://www.research.att.com/~njas/sequences/A011661} \\
  \frac{1}{x^5+x^2+1}          & \url{http://www.research.att.com/~njas/sequences/A011662} \\
  \frac{1}{x^5+x^4+x^3+x+1}    & \url{http://www.research.att.com/~njas/sequences/A011663} \\
  \frac{1}{x^5+x^3+1}          & \url{http://www.research.att.com/~njas/sequences/A011664} \\
  \frac{1}{x^5+x^4+x^3+x^2+1}  & \url{http://www.research.att.com/~njas/sequences/A011665} \\
  \frac{1}{x^6+x^5+x^4+x+1}    & \url{http://www.research.att.com/~njas/sequences/A011666} \\
  \frac{1}{x^16+x^5+x^3+x^2+1} & \url{http://www.research.att.com/~njas/sequences/A011729} \\
  \frac{1}{x^31+x^3+1}         & \url{http://www.research.att.com/~njas/sequences/A011744} \\
\end{tabular}}

Number of primitive polynomials:
\url{http://www.research.att.com/~njas/sequences/A011260}
\end{example}

An m-sequence is the remainder when dividing any non-zero polynomial by a primitive
polynomial.
We can define an
\hie{equivalence relation}\footnote{\hie{equivalence relation}: \prefp{def:eq_rel}}
on polynomials which defines two polynomials as
{\em equivalent with respect to $p(x)$}
when their remainders are equal.

%--------------------------------------
\begin{definition}[\hid{Equivalence relation}]
\label{def:equiv}
%--------------------------------------
\defbox{\begin{array}{lll}
  \mc{3}{l}{\text{$a_1(x)$ and $a_2(x)$ are equivalent with respect to $p(x)$ if}}
  \\& 1. & a_1(x) = q_1(x) p(x) + r(x)
  \\&    & \text{and}
  \\& 2. & a_2(x) = q_2(x) p(x) + r(x)
\end{array}}
In symbols, this is expressed as $a_1(x)\equiv a_2(x)$.
\end{definition}

Using the equivalence relation of \pref{def:equiv},
we can develop two very useful equivalent representations of
polynomials over GF(2).
We will call these two representations the {\em exponential} representation
and the {\em polynomial} representation.

%--------------------------------------
\begin{example}
\label{ex:representations}
%--------------------------------------
By \prefpp{def:equiv}, and
under $p(x)=x^3+x+1$, we have the following equivalent representations:
\exbox{\begin{array}{cccc}
   \text{exponential} & \text{polynomial} & \text{vector} & \text{decimal} \\
   x^0 & 1       &[001] & 1  \\
   x^1 & x       &[010] & 2  \\
   x^2 & x^2     &[100] & 4  \\
   x^3 & x+1     &[011] & 3  \\
   x^4 & x^2+x   &[110] & 6  \\
   x^5 & x^2+x+1 &[111] & 7  \\
   x^6 & x^2+1   &[101] & 5
\end{array}}
\end{example}
\begin{proof}
\begin{align*}
   \frac{x^0}{x^3+x+1} &=& 0     &+& \frac{1  }{x^3+x+1}     &\implies& x^0 &\equiv& 1   \\
   \frac{x^1}{x^3+x+1} &=& 0     &+& \frac{x  }{x^3+x+1}     &\implies& x^1 &\equiv& x   \\
   \frac{x^2}{x^3+x+1} &=& 0     &+& \frac{x^2}{x^3+x+1}     &\implies& x^2 &\equiv& x^2 \\
   \frac{x^3}{x^3+x+1} &=& 1     &+& \frac{x+1}{x^3+x+1}     &\implies& x^3 &\equiv& x+1 \\
   \frac{x^4}{x^3+x+1} &=& x     &+& \frac{x^2+x}{x^3+x+1}   &\implies& x^4 &\equiv& x^2+x \\
   \frac{x^5}{x^3+x+1} &=& x^2+1 &+& \frac{x^2+x+1}{x^3+x+1} &\implies& x^5 &\equiv& x^2+x+1 \\
   \frac{x^6}{x^3+x+1} &=& x^3+x+1 &+& \frac{x^2+1}{x^3+x+1} &\implies& x^6 &\equiv& x^2+1 \\
   \frac{x^7}{x^3+x+1} &=& x^4+x^2+x+1 &+& \frac{1}{x^3+x+1} &\implies& x^7 &\equiv& 1
\end{align*}

Notice that $x^7\equiv x^0$, and so a cycle is formed with
$2^3-1=7$ elements in the cycle.
The monomials to the left of the $\equiv$ are the {\em exponential}
representation and the polynomials to the right are the {\em polynomial}
representation.
Additionally, the polynomial representation may be put in a vector form giving a
{\em vector} representation.
The vectors may be interpreted as a binary number and represented as a decimal numeral.

\[\begin{array}{cccc}
   \text{exponential} & \text{polynomial} & \text{vector} & \text{decimal} \\
   x^0 & 1       &[001] & 1  \\
   x^1 & x       &[010] & 2  \\
   x^2 & x^2     &[100] & 4  \\
   x^3 & x+1     &[011] & 3  \\
   x^4 & x^2+x   &[110] & 6  \\
   x^5 & x^2+x+1 &[111] & 7  \\
   x^6 & x^2+1   &[101] & 5
\end{array}\]
\end{proof}


%=======================================
\subsection{Generating m-sequences using polynomial division}
%=======================================
An m-sequence is generated by dividing any non-zero polynomial of order less than $m$
by a primitive polynomial of order $m$.
The m-sequence is the coefficients of the resulting polynomial.
M-sequences will repeat every $2^m-1$ values.
This is the maximum sequence length possible when the
sequence is generated by division in polynomials over GF(2).

%--------------------------------------
\begin{example}
\label{ex:1/p(x)}
%--------------------------------------
We can generate an m-sequence of length
$2^3-1=7$ by dividing 1 by the primitive polynomial $x^3+x+1$.

\begin{fsL}
\[
\begin{array}{rrll}
           &   & x^{-3} + x^{-5} + x^{-6} +& x^{-7} + x^{-10} + x^{-12} + x^{-13} + x^{-14} + x^{-17} + \cdots \\
   \cline{3-4}
   x^3+x+1 & \vline & 1 \\
           &        & 1 + x^{-2} + x^{-3} \\
   \cline{3-3}
           &        & x^{-2} + x^{-3} \\
           &        & x^{-2} + x^{-4} + x^{-5} \\
   \cline{3-3}
           &        & x^{-3} + x^{-4} + x^{-5} \\
           &        & x^{-3} + x^{-5} + x^{-6} \\
   \cline{3-3}
           &        & x^{-4} + x^{-6}          \\
           &        & x^{-4} + x^{-6} + x^{-7} \\
   \cline{3-3}
           &        & x^{-7}                   \\
           &        & x^{-7} + x^{-9} + x^{-10} \\
   \cline{3-3}
           &        & x^{-9} + x^{-10}         \\
           &        & x^{-9} + x^{-11} + x^{-12} \\
   \cline{3-3}
           &        & x^{-10} + x^{-11} + x^{-12} \\
           &        & x^{-10} + x^{-12} + x^{-13} \\
   \cline{3-3}
           &        & x^{-11} + x^{-13}          \\
           &        & x^{-11} + x^{-13} + x^{-14} \\
   \cline{3-3}
           &        & x^{-14}           \\
           &        & \vdots
\end{array}
\]
\end{fsL}
The coefficients, starting with the $x^{-1}$ term,
of the resulting polynomial form the m-sequence
\[ 0010111 \; 0010111 \; \cdots \]
which repeats every $2^3-1=7$ elements.
\end{example}


Note that the division operation in Example \ref{ex:1/p(x)}
can be performed using vector notation rather than polynomial notation.

%--------------------------------------
\begin{example}
%--------------------------------------
Generate an m-sequence of length $2^3-1=7$ by dividing 1 by the primitive polynomial
$x^3+x+1$ using vector notation.

\begin{fsL}
\[
\begin{array}{rrllllllllllllllllllllllllllllllllllll}
           &   &   &.&0&0&1&0&1&1&1 &0&0&1&0&1&1&1 &0&\cdots \\
   \cline{3-20}
   1011    &\vline   & 1 &.&0&0&0&0&0&0&0 &0&0&0&0&0&0&0 &0 \\
           &   & 1 & &0&1&1 \\
   \cline{3-7}
           &   & 0 & &0&1&1&0 \\
           &   &   & &0&0&0&0 \\
   \cline{5-8}
           &   &   & &0&1&1&0&0 \\
           &   &   & & &1&0&1&1 \\
   \cline{6-9}
           &   &   & & &0&1&1&1&0 \\
           &   &   & & & &1&0&1&1 \\
   \cline{7-10}
           &   &   & & & &0&1&0&1&0 \\
           &   &   & & & & &1&0&1&1 \\
   \cline{8-11}
           &   &   & & & & &0&0&0&1&0 \\
           &   &   & & & & & &0&0&0&0 \\
   \cline{9-12}
           &   &   & & & & & &0&0&1&0&0 \\
           &   &   & & & & & & &0&0&0&0 \\
   \cline{10-13}
           &   &   & & & & & & &0&1&0&0&0 \\
           &   &   & & & & & & & &1&0&1&1 \\
   \cline{11-14}
           &   &   & & & & & & & &0&0&1&1&0 \\
           &   &   & & & & & & & & &0&0&0&0 \\
   \cline{12-15}
           &   &   & & & & & & & & &0&1&1&0&0 \\
           &   &   & & & & & & & & & &1&0&1&1 \\
   \cline{13-16}
           &   &   & & & & & & & & & &0&1&1&1&0 \\
           &   &   & & & & & & & & & & &0&0&0&0 \\
           &   &   & & & & & & & & & & & &\vdots& &  \\
\end{array}
\]
\end{fsL}

The coefficients, starting to the right of the binary point,
is again the sequence
\[ 0010111 \; 0010111 \; \cdots. \]
\end{example}


%=======================================
\subsection{Multiplication modulo a primitive polynomial}
\label{sec:math-xmod}
%=======================================
If $p(x)$ is a primitive polynomial,
by Definition \ref{def:equiv}
the product of two polynomials is equivalent (with respect to $p(x)$)
of the product {\em modulo} $p(x)$.
The ability to multiplying two polynomials modulo a primitive polynomial
is very useful for manipulating m-sequences.

In general, the product of two polynomials can be evaluated as follows.
Let
\begin{eqnarray*}
   a(x) &\eqd& a_mx^m + a_{m-1}x^{m-1} + \cdots + a_2x^2 + a_1x + a_0 \\
   b(x) &\eqd& b_mx^m + b_{m-1}x^{m-1} + \cdots + b_2x^2 + b_1x + b_0
\end{eqnarray*}

Then
\begin{scriptsize}
\begin{eqnarray*}
   a(x)b(x)
        &=& \left( a_mx^m + a_{m-1}x^{m-1} + \cdots + a_2x^2 + a_1x + a_0 \right)
            \left( b_mx^m + b_{m-1}x^{m-1} + \cdots + b_2x^2 + b_1x + b_0 \right)\\
        &=& a_0b_0 + (a_0b_1+a_1b_0)x + (a_0b_2+a_1b_1+a_2b_0)x^2 + \cdots + a_mb_mx^{2m}\\
        &=& \left( \sum\limits_{i=0}^{m-1}x^i \sum\limits_{j=0}^i a_jb_{i-j} \right) +
            \left( \sum\limits_{i=m}^{2m}x^i \sum\limits_{j=0}^{2m-i} a_{i-m+j}b_{m-j} \right)
\end{eqnarray*}
\end{scriptsize}

The product modulo $p(x)$ is obtained when the terms involving
$x^m$, $x^{m+1}$, \ldots, $x^{2m}$ are replaced by their
equivalent polynomial representations
(see Section \ref{sec:def}).

%--------------------------------------
\begin{example}
\label{ex:math-xmod}
%--------------------------------------
Suppose we want to find $(a_2x^2 + a_1x + a_0)(b_2x^2+b_1x+b_0)$ modulo $x^3+x+1$.
\begin{scriptsize}
\begin{eqnarray*}
   a(x)b(x) &=& (a_2x^2+a_1x+a_0)(b_2x^2+b_1x+b_0)
      \\&=& a_0b_0 + (a_0b_1 + a_1b_0)x + (a_0b_2 + a_1b_1 + a_2b_0)x^2 +
            (a_1b_2 + a_2b_1)x^3 + a_2b_2x^4
      \\&=& a_0b_0 + (a_0b_1 + a_1b_0)x + (a_0b_2 + a_1b_1 + a_2b_0)x^2 +
            (a_1b_2 + a_2b_1)(x+1) + a_2b_2(x^2+x)
      \\&=& (a_0b_0 + a_1b_2 + a_2b_1) +
            (a_0b_1 + a_1b_0 + a_1b_2 + a_2b_1 + a_2b_2)x +
            (a_0b_2 + a_1b_1 + a_2b_0 + a_2b_2)x^2
\end{eqnarray*}
\end{scriptsize}
Notice that if the $a_i$ and $b_i$ coefficients are known,
the resulting product has only three terms.
\end{example}


