%============================================================================
% LaTeX File
% Daniel Greenhoe
%============================================================================

%======================================
\chapter{Filter Topologies}
\label{chp:filtertop}
%======================================
Suppose we want to construct a circuit to compute the 
rational expression $f(x)\frac{b(x)}{a(x)}$.
This is a common problem in {\em Digital Signal Processing (DSP)};
we can borrow results from there.
DSP is generally concerned with polynomials over the field of real or complex numbers.
However, a field is a field, and all fields (whether, real, complex, or GF(2))
support both addition and multiplication;\footnote{
   {\bf Fields:} Roughly speaking, a {\em group} is a set together with an operation 
   on that set.  
   An {\em additive group} is a set $S$ 
   with an addition operation $+:S\times S\to S$.
   A {\em multiplicative group} is a set $S$ 
   with a multiplication operation $\cdot:S\times S\to S$.
   A {\em field} is constructed using two groups: 
   An addition group and a multiplication group.
   See Appendix \ref{app:inprod} page \pageref{app:inprod}.\\
   Reference: \cite[p.123]{durbin}.
   }
the rules change somewhat, but the basic structure is the same regardless.
Alternatively, just as a typical digital filter operates over the real 
or complex field, 
{\bf the m-sequence generator} described in this section
{\bf is a digital filter which operates over the field GF(2)}.

A sequential hardware multiplier-divider for polynomials is simple.
\begin{liste}
   \item Each $x$ in $f(x)$, $b(x)$, and $a(x)$ represents a delay of one clock cycle.
In DSP terminology, a delay of one clock cycle is represented by $z^{-1}$.
Thus, $x=z^{-1}$.
   \item Let $f(x)=f_0 + f_1x + f_2x^2 + \cdots$.\\
         Then let $\dot{f}(n)$ be the sequence $\dot{f}(i)=f_i$, with $i\in\Z$.
   \item Let $b(x)= b_0 + b_1x + b_2x^2 + \cdots + b_mx^m$.\\
         Let $\dot{b}(n)$ be the sequence $\dot{b}(i)=b_i$, with $i\in\Z$.
   \item Let $a(x)=1 + a_1x + a_2x^2 + \cdots + a_mx^m$.\\
         Let $\dot{a}(n)$ be the sequence $\dot{a}(i)=a_i$, with $i\in\Z$.
\end{liste}
Then the multiplier-divider (for any mathematical field) can be implemented as shown
in \prefp{fig:df1}.  
This structure is called the \structe{Direct Form I} implementation;\footnote{
  \citerp{oppenheim1999}{344},
  \citerppgc{schlichtharle2000}{96}{98}{9783540668411}{5.3.2 Direct Form Filters},
  \citerppgc{schlichtharle2011}{147}{150}{9783642143250}{5.3.2 Direct Form Filters}
  }
it implements the rational expression
\[
   f(x) \frac{b_mx^m + b_{m-1}x^{m-1} + \cdots + b_2x^2 + b_1x + b_0}
             {a_mx^m + a_{m-1}x^{m-1} + \cdots + a_2x^2 + a_1x + 1  }
\]

\begin{figure}[ht]
\center{\epsfig{file=df1.eps, height=6cm, clip=}}
%\center{\includegraphics[0,0][4in,4in]{df1.eps}}
\caption{
   Direct Form I Implementation for $\ds f(x)\frac{b(x)}{a(x)}$
   \label{fig:df1}
   }
\end{figure}

In GF(2), the blocks in the figure can be implemented very simply:
\begin{liste}
   \item Each $x=z^{-1}$ element can be implemented as a simple D flip-flop.
   \item An $a_i=1$ or $b_i=1$ coefficient is implemented as a wire (closed circuit).
   \item An $a_i=0$ or $b_i=0$ coefficient is implemented as a no-connect (open circuit).
\end{liste}

%--------------------------------------
\begin{example}
\label{ex:df1}
%--------------------------------------
Suppose we want to build a hardware circuit to 
generate an m-sequence specified by the rational expression
\[ \frac{x^2+x}{x^3+x+1}. \]
To do this we can set $f(x)=1$, $b(x)=x^2+x$ and $a(x)=x^3+x+1$.
The resulting structure is shown in Figure \ref{fig:ex-df1}.
\begin{figure}[ht]
\center{\epsfig{file=ex-df1.eps, height=6cm, clip=}}
\caption{
   Direct Form I Implementation for Example \ref{ex:df1}
   \label{fig:ex-df1}
   }
\end{figure}
Notice that the two flip-flops on the left are for initialization 
only and are not used in the steady state operation of the m-sequence
generator.
In fact, they can be eliminated altogether by proper initialization of 
the flip-flops on the right.
\end{example}

%--------------------------------------
%\subsection{Polynomial multiplication and division using DF2}
%\label{sec:df2}
%\index{direct form 2}
%--------------------------------------
The Direct Form I structure shown in Figure \ref{fig:df1} can be transformed
\footnote{
  \citerpgc{oppenheim1999}{347}{9780137549207}{Figure 6.5 Combination of Delays in Figure 6.4}
  }
to a new structure by transformation rules based on 
\ope{Mason's Gain Formula}.\footnote{
   The transformation rules are as follows:
   \begin{enumerate}
   \setlength{\itemsep}{0ex}
   \item Reverse the direction of all signal paths.
   \item Replace all nodes with addition operators.
   \item Replace all addition operators with nodes.
   \end{enumerate}
   References:
   \citer{mason1951},
   \citer{mason1953},
   \citer{mason1953p},
   \citer{mason1955},
   \citer{mason1956},
   \citer{mason1960},
   \citer{chow1962},
   \citer{phillips1995},
   \citerpgc{oppenheim1999}{363}{9780137549207}{footnote 2}
   }
The resulting structure is known as \structe{Direct Form II}
and is illustrated in \prefp{fig:df2}.
\begin{figure}[ht]
\center{\epsfig{file=df2.eps, height=6cm, clip=}}
\caption{
   Direct Form II Implementation
   \label{fig:df2}
   }
\end{figure}
Again, when using the DF2 structure for m-sequence generation,
the $\dot{f}(n)$ sequence can be eliminated by proper initialization of
the delay elements (flip flops).
