%============================================================================
% Daniel J. Greenhoe
% XeLaTeX file
%============================================================================

%=======================================
\chapter{Logic}
\label{chp:logic}
%=======================================
\qboxnps
  {\href{http://en.wikipedia.org/wiki/Gottfried_Leibniz}{Gottfried Leibniz}
   \href{http://www-history.mcs.st-andrews.ac.uk/Timelines/TimelineC.html}{(1646--1716)},
   \href{http://www-history.mcs.st-andrews.ac.uk/BirthplaceMaps/Places/Germany.html}{German mathematician},
   sharing his thoughts regarding \emph{mathematical logic}.
   \index{Leibniz, Gottfried}
   \index{quotes!Leibniz, Gottfried}
   \footnotemark
  }
  {../common/people/leibniz_wkp_pdomain_bw.jpg}
  {I dare say that this is the last effort of the human mind, and
    when this project shall have been carried out,
    all that men will have to do will be to be happy,
    since they will have an instrument that will serve to exalt the intellect
    not less than the telescope serves to perfect their vision.}
  \citetblt{
    quote: & \citerp{padoa1912}{21} \\
           & \citerc{cajori2}{paragraph 541} \\
    image: & \url{http://en.wikipedia.org/wiki/Gottfried_Leibniz}, public domain
    }

\qboxnpq{%
  \href{http://en.wikipedia.org/wiki/William_Stanley_Jevons}{William Stanley Jevons} 
  \href{http://www-history.mcs.st-andrews.ac.uk/Timelines/TimelineF.html}{(1835--1882)}, 
  \href{http://www-history.mcs.st-andrews.ac.uk/BirthplaceMaps/UK.html}{English} 
  \href{http://en.wikipedia.org/wiki/List_of_economists}{economist} and 
  \href{http://en.wikipedia.org/wiki/List_of_logicians}{logician}
  \index{quotes!Jevons, William Stanley}
  \footnotemark}
  {../common/people/jevons.jpg}
  {I cannot forget or omit to record this day last week. 
    I was sleeping as usual for the night at St. Michael's Hamlet. 
    As I awoke in the morning, the sun was shining brightly into my room. 
    There was a consciousness on my mind that I was the discoverer 
    of the true logic of the future. 
    For a few minutes I felt a delight such as one can seldom hope to feel. 
    But it would not last long---
    I remembered only too soon how unworthy and weak an instrument 
    I was for accomplishing so great a work, 
    and how hardly could I expect to do it.}
\citetblt{
  image: & \url{http://www-history.mcs.st-andrews.ac.uk/PictDisplay/Jevons.html}\\
  quote: & \citerpc{jevons1886}{219}{1866 March 28 entry}
  }


%=======================================
\section{Implications}
%=======================================
Arguably a logic is not a logic without the inclusion of an \fncte{implication} function $\limp$.
The mathematical structure \structe{logic} is formally defined in \prefpp{def:logic}.
But before defining a logic, this text offers a very general definition 
(a ``weak" definition) of implication that can be used in defining a very wide class
of logics---including \prope{non-Boolean} ones.
For \prope{Boolean} logics, the \fncte{classical implication} function 
$x\limpc y$ \xref{ex:limpx} is arguably adequate.
Two key properties of \fncte{classical implication} on a \prope{Boolean} logic
are \prope{entailment} and \prope{modus ponens}. % (next next definition). 
The following definition exploits weakened versions of these two properties
to define implication.
Note that the definition is at this time probably not standard in the literature. 
But without it, it is difficult to offer a complete definition of a logic.

%---------------------------------------
\begin{definition}
\label{def:limp}
%---------------------------------------
%Let $\latL\eqd\latnX$ be a \structe{lattice with negation} \xref{def:latn}.
Let $\latL\eqd\latbX$ be a \structe{bounded lattice} \xref{def:latb}.
\defboxt{
  The function $\hxs{\limp}$ in $\clFxx$ is an \reld{implication} on $\latL$ if
  \\\indentx$\begin{array}{FrclCDD}
    1. & \brb{x\orel y} \implies x\limp y &\oreld& x\join y & \forall x,y\in\setX & (\prope{weak entailment})   & and\\
    2. & x\meet(x\limp y)           &\orel&  \negat{x}\join y        & \forall x,y\in\setX & (\prope{weak modus ponens}) & 
  \end{array}$
  }
\end{definition}

%---------------------------------------
\begin{proposition}
\label{prop:limp_iff}
%---------------------------------------
%Let $\limp$ be an \fncte{implication} \xref{def:limp} on a \structe{lattice with negation} $\latL\eqd\latnX$ \xref{def:latn}.
Let $\limp$ be an \fncte{implication} \xref{def:limp} on a \structe{bounded lattice} $\latL\eqd\latbX$ \xref{def:latb}.
\propbox{
    \brb{x\orel y} \quad\iff\quad \brb{x\limp y \oreld x\join y} \qquad\scy\forall x,y\in\setX
  }
\end{proposition}
\begin{proof}
\begin{enumerate}
  \item Proof for $\implies$ case: by \prope{weak entailment} property of \fncte{implication}s \xref{def:limp}.
  \item Proof for $\impliedby$ case: 
    \begin{align*}
      y
        &\oreld x\meet(x\limp y)
        && \text{by right hypothesis}
      \\&\oreld x\meet(x\join y)
        && \text{by \prope{modus ponens} property of $\limp$ \xref{def:limp}}
      \\&= x
        && \text{by \prope{absorptive} property of \structe{lattice}s \xref{def:lattice}}
    \end{align*}
\end{enumerate}
\end{proof}

%---------------------------------------
\begin{remark}
\citetblp{
  \citePpgc{hardegree1979}{59}{9027707073}{(E),(MP),(E*)},
  \citePp{kalmbach1973}{498},
  \citerppgc{kalmbach1983}{238}{239}{0123945801}{Chapter 4 \textsection 15},
  \citePpg{pavicic2008}{24}{0080931669},
  \citerpgc{xu2003}{27}{354040175X}{Definition 2.1.1},
  \citePp{xu1999}{25},
  \citePp{jun1998}{54}
  }
\label{rem:limp_ortho}
%---------------------------------------
Let $\latL\eqd\latbX$ be a \structe{bounded lattice} \xref{def:latb}.
In the context of \structe{ortho lattice}s, a more common (and stronger) definition of \fncte{implication} $\limp$ might be
  \\\indentx$\begin{array}{FrclCDD}
    1. & x\orel y \implies x\limp y &=&     \lid & \forall x,y\in\setX & (\prope{entailment} / \prope{strong entailment})   & and\\
    2. & x\meet(x\limp y)           &\orel& y    & \forall x,y\in\setX & (\prope{modus ponens} / \prope{strong modus ponens}) & 
  \end{array}$
  \\
This definition yields a result stronger than that of \prefpp{prop:limp_iff}:
\\\indentx$
    \brb{x\orel y} \quad\iff\quad \brb{x\limp y = \lid} \qquad\scy\forall x,y\in\setX
$\\
The \exme{Heyting 3-valued logic} \xref{ex:logic_heyting3} and \exme{Sasaki hook logic} \xref{ex:logic_sasaki}
have both \prope{strong entailment} and \prope{strong modus ponens}.
However, for non-ortho logics in general, these two properties seem inappropriate to serve as a definition for \fncte{implication}.
For example, the \exme{Kleene 3-valued logic} \xref{ex:logic_kleene3}, 
\exme{RM$_3$ logic} \xref{ex:logic_rm3}, and \exme{BN$_4$ logic} \xref{ex:logic_bn4} do not have the \prope{strong entailment} property;
and the \exme{Kleene 3-valued logic}, \exme{/-Lukasiewicz 3-valued logic} \xref{ex:logic_lukas3}, and 
\exme{BN$_4$ logic} do not have the \prope{strong modus ponens} property.
\end{remark}
\begin{proof}
\begin{enumerate}
  \item Proof for $\implies$ case: by \prope{entailment} property of \fncte{implication}s \xref{def:limp}.
  \item Proof for $\impliedby$ case: 
    \begin{align*}
      x\limp y = \lid 
        &\implies x\meet\lid \orel y
        && \text{by \prope{modus ponens} property \xref{def:limp}}
      \\&\implies x \orel y
        && \text{by definition of $\lid$ (\vale{least upper bound}) \xref{def:lub}}
    \end{align*}
\end{enumerate}
\end{proof}

%%---------------------------------------
%\begin{definition}
%\footnote{
%  \citePpg{pavicic2008}{24}{0080931669}
%  }
%\label{def:limp}
%%---------------------------------------
%Let $\latL\eqd\latnX$ be a \structe{lattice with negation} \xref{def:latn}.
%\defboxt{
%  The function $\hxs{\limp}$ in $\clFxx$ is an \reld{implication} on $\latL$ if
%  \\\indentx$\brb{\begin{array}{M}$\latL$ is\\\prope{Boolean}\end{array}} 
%  \qquad\implies\qquad
%  \brb{\begin{array}{rclC}
%    x\limp y &=& \negat{x}\meet y & \forall x,y\in\setX
%  \end{array}}$
%  }
%\end{definition}

%---------------------------------------
\begin{example}
\citetblp{
  \citePp{kalmbach1973}{499},
  \citeP{kalmbach1974},
  \citePc{mittelstaedt1970}{Sasaki hook},
  \citePpc{finch1970}{102}{Sasaki hook (1.1)}, %{ $\brp{x\limp y}=\brp{y\meet x}\join Nx$}\\
  \citerpgc{kalmbach1983}{239}{0123945801}{Chapter 4 \textsection 15, 3. {\scshape Theorem}}
  %\citePpg{pavicic2008}{24}{0080931669}
  }
\label{ex:limpx}
%---------------------------------------
Let $\latL\eqd\latnX$ be a \structe{lattice with negation} \xref{def:latn}.
\exboxp{
  If $\latL$ is an \structb{orthomodular lattice} \xref{def:negor}, then
  the functions listed below are all examples of valid \fncte{implication} functions \xref{def:limp} on $\latL$.
  If $\latL$ is an \structb{ortho lattice}, then 1--5 are \fncte{implication} relations.
  \\$\begin{array}{FrclCD}
    1. & x\limpc y &\eqd& \mc{3}{l}{\negat{x}\join y\qquad\scy\forall x,y\in\setX\qquad\text{\scs(\rele{classical implication}/\rele{material implication}/\rele{horseshoe})}}\\
    2. & x\limps y &\eqd& \negat{x}\join \brp{x\meet y}                                                            & \forall x,y\in\setX & (\rele{Sasaki hook} / \rele{quantum implication})\\
    3. & x\limpd y &\eqd& y \join (\negat{x} \meet \negat{y})                                                       & \forall x,y\in\setX & (\rele{Dishkant implication})\\
    4. & x\limpk y &\eqd& (\negat{x} \meet y) \join (\negat{x} \meet \negat{y}) \join (x \meet ( \negat{x} \join y )) & \forall x,y\in\setX & (\rele{Kalmbach implication}) \\
    5. & x\limpn y &\eqd& (\negat{x} \meet y ) \join (x \meet y ) \join ((\negat{x} \join y ) \meet \negat{y})       & \forall x,y\in\setX & (\rele{non-tollens implication})\\
    6. & x\limpr y &\eqd& ( \negat{x} \meet y ) \join (x \meet y ) \join ( \negat{x} \meet \negat{y})                & \forall x,y\in\setX & (\rele{relevance implication}) 
   %6. & x\imp_6 y &\eqd& y^{\negat\negat}\join \brp{\negat{y}\meet \negat{x}}                                      & \forall x,y\in\setX & (\rele{Dishkant implication})
  \end{array}$
  \\
  Moreover, if $\latL$ is a \structb{Boolean lattice}, then all of these implications are equivalent to $\limpc$,
  and all of them have \prope{strong entailment} and \prope{strong modus ponens}.
  }
\\
Note that ${\scy\forall x,y\in\setX}$,\qquad $x\limpd y=\negat{y}\limps \negat{x}$\qquad and\qquad
$x\limpn y = \negat{y}\limpk \negat{x}$.
The values for the 6 implications on an \structe{orthocomplemented O$_6$ lattice} \xref{def:o6}
are listed in \prefpp{ex:limpx_o6}.
\end{example}
\begin{proof}
\begin{enumerate}
  \item Proofs for the \fncte{classical implication} $\limpc$:
    \begin{enumerate}
      \item Proof that on an \structe{ortho lattice}, $\limpc$ is an \fncte{implication}: %, proof that $\limpc$ has \prope{strong entailment} and \prope{weak modus ponens}:
        \begin{align*}
          x\orel y\implies x\limpc y
            &\eqd   \negat{x} \join y
            &&      \text{by definition of $\limpc$}
          \\&\oreld \negat{y} \join y
            &&      \text{by $x\orel y$ and \prope{antitone} prop. of $\negat$ \xref{def:negor}}
          \\&=      \lid
            &&      \text{by \prope{excluded middle} prop. of $\negat$ \xref{thm:latn_ortho}}
          \\&\implies\text{\prope{strong entailment}}
            && \text{by definition of \prope{strong entailment}}
          \\
          x\meet(\negat{x}\join y)
            &\orel \negat{x}\join y
            &&  \text{by definition of $\meet$ \xref{def:meet}}
          \\&\implies\text{\prope{weak modus ponens}}
            && \text{by definition of \prope{weak modus ponens}}
        \end{align*}
        \begin{minipage}{\tw-50mm}
          Note that in general for an \structe{ortho lattice}, the bound cannot be tightened to \prope{strong modus ponens} because,
          for example in the \structe{O$_6$ lattice} \xref{def:o6} illustrated to the right
          \\\indentx$x\meet(\negat{x}\join y)=x\meet\lid=x\nle y\implies\text{\prope{not strong modus ponens}}$
        \end{minipage}%
        \hfill\tbox{\includegraphics{graphics/lat6_o6_10xy-x.pdf}}%
        %\begin{minipage}{25mm}
        %  \psset{unit=5mm}\gsize%
        %  %============================================================================
% Daniel J. Greenhoe
% LaTeX file
% lattice O6
%============================================================================
\begin{pspicture}(-2.2,-\latbot)(2.2,3.3)%
  %---------------------------------
  % nodes
  %---------------------------------
  \Cnode(0,3){t}%
  \Cnode(-1,2){c}\Cnode(1,2){d}%
  \Cnode(-1,1){x}\Cnode(1,1){y}%
  \Cnode(0,0){b}%
  %---------------------------------
  % node connections
  %---------------------------------
  \ncline{t}{c}\ncline{t}{d}%
  \ncline{c}{x}\ncline{d}{y}%
  \ncline{b}{x}\ncline{b}{y}%
  %---------------------------------
  % node labels
  %---------------------------------
  \uput[0](t) {$\lid$}%
  \uput[0](d) {$\negat{y}$}%
  \uput[180](c) {$x$}%
  \uput[0](y) {$\negat{x}$}%
  \uput[180](x) {$y$}%
  \uput[0](b) {$\lzero$}%
\end{pspicture}%%
        %\end{minipage}
      \item Proof that on a \structe{Boolean lattice}, $\limpc$ is an \fncte{implication}:
        \begin{align*}
          x\meet(\negat{x}\join y)
            &= (x\meet\negat{x})\join (x\meet y)
            &&  \text{by \prope{distributive} prop. of Boolean lat. \ifxref{boolean}{def:boolean}}
          \\&= \lid\join (x\meet y)
            &&  \text{by \prope{excluded middle} property of \structe{Boolean lattice}s}
          \\&= x\meet y
            &&  \text{by definition of $\lid$}
          \\&\orel y
            && \text{by definition of $\meet$ \xref{def:meet}}
          \\&\implies\text{\prope{strong modus ponens}}
            && \text{by definition of \prope{strong modus ponens}}
        \end{align*}
    \end{enumerate}

  \item Proofs for \fncte{Sasaki implication} $\limps$:
    \begin{enumerate}
      \item Proof that on an \structe{ortho lattice}, $\limps$ is an \fncte{implication}:
        \begin{align*}
          x\orel y
            &\implies x\limps y
          \\&\eqd   \negat{x} \join (x \meet y) 
            &&      \text{by definition of $\limpk$}
          \\&=      \negat{x} \join x
            &&      \text{by $x\orel y$ hypothesis}
          \\&=      \lid
            &&      \text{by \prope{excluded middle} prop. of ortho neg. \xref{thm:latn_ortho}}
          \\&\implies \text{\prope{strong entailment}}
            &&      \text{by definition of \prope{strong entailment}}
          \\
          x \meet(x\limps y)
            &\eqd x \meet \brs{\negat{x}\join (x\meet y)}
            && \text{by definition of $\limps$}
          \\&\orel \brs{\negat{x}\join (x\meet y)}
            && \text{by definition of $\meet$ \xref{def:meet}}
          \\&\orel \negat{x}\join y
            && \text{by definition of $\meet$ \xref{def:meet}}
          \\&\implies \text{\prope{weak modus ponens}}
        \end{align*}

      \item Proof that on a \structe{Boolean lattice}, $\limps=\limpc$:
        \begin{align*}
          x\limps y
            &\eqd \negat{x}\join (x\meet y)
            && \text{by definition of $\limps$}
          \\&= \negat{x}\join y
            && \text{\ifdochas{boolean}{by \prefpp{lem:boa_xcxy}}}
          \\&= x\limpc y
            && \text{by definition of $\limpc$}
        \end{align*} 
    \end{enumerate}

  \item Proofs for \fncte{Dishkant implication} $\limpd$:
    \begin{enumerate}
      \item Proof that $x\limpd y\equiv\negat{y}\limps\negat{x}$: \label{item:limpd_limps}
        \begin{align*}
          x\limpd y
            &\eqd  y \join (\negat{x} \meet \negat{y})
            &&      \text{by definition of $\limpd$}
          \\&=     y \join (\negat{y} \meet \negat{x}) 
            &&     \text{by \prope{commutative} property of \structe{lattice}s \xref{thm:lattice}}
          \\&=     \negat\negat{y} \join (\negat{y} \meet \negat{x}) 
            &&     \text{by \prope{involutory} property of \structe{ortho negation}s \xref{def:negor}}
          \\&\eqd  \negat{y}\limps \negat{x}
            &&      \text{by definition of $\limps$}
        \end{align*}

      \item Proof that on an \structe{ortho lattice}, $\limpd$ is an \fncte{implication}:
        \begin{align*}
          x\orel y
            &\implies x\limpd y
          \\&\eqd   y \join (\negat{x} \meet \negat{y}) 
            &&      \text{by definition of $\limpd$}
          \\&=      y \join \negat{y}
            &&      \text{by $x\orel y$ hypoth. and \prope{antitone} prop. \xref{def:negor}}
          \\&=      \lid
            &&      \text{by \prope{excluded middle} prop. of ortho neg. \xref{thm:latn_ortho}}
          \\&\implies \text{\prope{strong entailment}}
            &&      \text{by definition of \prope{strong entailment}}
          \\
          x \meet(x\limpd y)
            &\eqd   y \join (\negat{x} \meet \negat{y}) 
            && \text{by definition of $\limpd$}
          \\&=  y \join \negat{x}
            && \text{by definition of $\meet$ \xref{def:meet}}
          \\&\implies \text{\prope{weak modus ponens}}
        \end{align*}

      \item Proof that on a \structe{Boolean lattice}, $\limpd=\limpc$:
        \begin{align*}
          x\limpd y
            &\eqd   y \join (\negat{x} \meet \negat{y}) 
            && \text{by definition of $\limpd$}
          \\&= \negat{x}\join y
            && \text{\ifdochas{boolean}{by \prefpp{lem:boa_xcxy}}}
          \\&= x\limpc y
            && \text{by definition of $\limpc$}
        \end{align*} 
    \end{enumerate}

  \item Proofs for the \fncte{Kalmbach implication} $\limpk$:
    \begin{enumerate}
      \item Proof that on an \structe{ortho lattice}, $\limpk$ is an \fncte{implication}: \label{item:limpk_ortho}
        \begin{align*}
          x\orel y
            &\implies x\limpk y
          \\&\eqd   \brp{\negat{x} \meet y} \join \brp{\negat{x} \meet \negat{y}} \join\brs{x\meet\brp{\negat{x} \join y}}
            &&      \text{by definition of $\limpk$}
          \\&=      \brp{\negat{x} \meet y} \join \brp{\negat{y}} \join\brs{x\meet\brp{\negat{x} \join y}}
            &&      \text{by \prope{antitone} property \xref{def:negor}}
          \\&=      \brp{\negat{x} \meet y} \join \negat{y} \join\brs{x\meet\brp{\lid}}
            &&      %\text{by \prefp{prop:latoc_x_orel_y}}
          \\&=      \brp{\negat{x} \meet y} \join \brp{x\join \negat{y}}
            &&      \text{by definition of $\lid$ \xref{def:lub}}
          \\&=      \negat\negat\brp{\negat{x} \meet y} \join \brp{x\join \negat{y}}
            &&      \text{by \prope{involutory} property \xref{def:negor}}
          \\&=      \negat\brp{\negat\negat{x} \join \negat{y}} \join \brp{x\join \negat{y}}
            &&      \text{by \prope{de Morgan} property \xref{thm:latn_ortho}}
          \\&=      \negat\brp{x \join \negat{y}} \join \brp{x\join \negat{y}}
            &&      \text{by \prope{involutory} property \xref{def:negor}}
          \\&=      \lid
            &&      \text{by \prope{excluded middle} property \xref{thm:latn_ortho}}
          \\&\implies \text{\prope{strong entailment}}
        \end{align*}
        \begin{align*}
          x\meet(x\limpk y)
            &\eqd   x\meet\brs{\brp{\negat{x} \meet y} \join \brp{\negat{x} \meet \negat{y}} \join\brs{x\meet\brp{\negat{x} \join y}}}
            &&      \text{by definition of $\limpk$}
          \\&\orel  \brp{\negat{x} \meet y} \join \brp{\negat{x} \meet \negat{y}} \join\brs{x\meet\brp{\negat{x} \join y}}
            &&      \text{by definition of $\meet$ \xref{def:meet}}
          \\&\orel  \brp{\negat{x} \meet y} \join \brp{\negat{x} \meet \negat{y}} \join\brp{\negat{x} \join y}
            &&      \text{by definition of $\meet$ \xref{def:meet}}
          \\&\orel  y \join \brp{\negat{x} \meet \negat{y}} \join \negat{x} \join y
            &&      \text{by definition of $\meet$ \xref{def:meet}}
          \\&=      y \join \negat{x}  \join \brp{\negat{x}\meet\negat{y}}
            &&      \text{by \prope{idempotent} p. \xref{thm:lattice}}
          \\&\orel  y \join \negat{x}  \join \negat{x}
            &&      \text{by definition of $\meet$ \xref{def:meet}}
          \\&=      \negat{x}  \join y
            &&      \text{by \prope{idempotent} p. \xref{thm:lattice}}
          \\&\implies \text{\prope{weak modus ponens}}
        \end{align*}

      \item Proof that on a \structe{Boolean lattice}, $\limpk=\limpc$: \label{item:limpk_boolean}
        \begin{align*}
          x\limpk y
            &\eqd \brp{\negat{x} \meet y} \join \brp{\negat{x} \meet \negat{y}} \join\brs{x\meet\brp{\negat{x} \join y}}
            &&  \text{by definition of $\limpk$}
          \\&=  \brp{\negat{x} \meet y} \join \brp{\negat{x} \meet \negat{y}} \join\brs{\brp{x\meet\negat{x}} \join \brp{x\meet y}}
            &&  \text{by \prope{distributive} property \ifxref{boolean}{def:boolean}}
          \\&=  \brp{\negat{x} \meet y} \join \brp{\negat{x} \meet \negat{y}} \join\brs{\brp{\lzero} \join \brp{x\meet y}}
            &&  \text{by \prope{non-contradiction} property}%{ \xref{def:boolean}}
          \\&=  \brp{\negat{x} \meet y} \join \brp{\negat{x} \meet \negat{y}} \join \brp{x\meet y}
            &&  \text{by \prope{bounded} property \xref{def:latb}}
          \\&=  \negat{x} \meet \brp{y \join \negat{y}} \join \brp{x\meet y}
            &&  \text{by \prope{distributive} property \ifxref{boolean}{def:boolean}}
          \\&=  \negat{x} \meet \lid \join \brp{x\meet y}
            &&  \text{by \prope{excluded middle} property} %{ \xref{def:boolean}}
          \\&=  \negat{x} \join \brp{x\meet y}
            &&  \text{by definition of $\lid$ \xref{def:lub}}
          \\&=  \negat{x} \join y
            && \text{\ifdochas{boolean}{by \prefpp{lem:boa_xcxy}}}
          \\&\eqd x\limpc y
            &&  \text{by definition of $\limpc$}
        \end{align*}
    \end{enumerate}

  \item Proofs for the \fncte{non-tollens implication} $\limpn$:
    \begin{enumerate}
      \item Proof that $x\limpn y\equiv\negat{y}\limpk\negat{x}$: \label{item:limpn_limpk}
        \begin{align*}
          x\limpn y
            &\eqd  (\negat{x} \meet y ) \join (x \meet y ) \join \brs{(\negat{x} \join y ) \meet \negat{y}}
            &&      \text{by definition of $\limpn$}
          \\&=     (y \meet \negat{x}) \join (y \meet x) \join\brs{\negat{y}\meet(y \join \negat{x})}
          \\&=     (\negat{\negat{y}} \meet \negat{x}) \join (\negat{\negat{y}} \meet \negat{\negat{x}}) \join\brs{\negat{y}\meet(\negat{\negat{y}} \join \negat{x})}
          \\&\eqd  \negat{y}\limpk \negat{x}
            &&      \text{by definition of $\limpk$}
        \end{align*}

      \item Proof that on an \structe{ortho lattice}, $\limpn$ is an \fncte{implication}: %, proof that $\limpc$ has \prope{strong entailment} and \prope{weak modus ponens}:
        \begin{align*}
          x\orel y
            &\implies x\limpn y
          \\&\equiv \negat{y}\limpk \negat{x}
            &&      \text{by \prefp{item:limpn_limpk}}
          \\&=1     
            &&      \text{by \prefp{item:limpk_ortho}}
          \\&\implies \text{\prope{strong entailment}}
          \\
          x\meet(x\limpn y)
            &=      x\meet(\negat{y}\limpk\negat{x})
            &&      \text{by \prefp{item:limpn_limpk}}
          \\&\orel  \negat\negat{y} \join \negat{x}
            &&      \text{by \prefp{item:limpk_ortho}}
          \\&=      y \join \negat{x}
            &&      \text{by \prope{involutory} property of $\negat$ \xref{def:negor}}
          \\&=      \negat{x} \join y
            &&      \text{by \prope{commutative} property of \structe{lattices} \xref{def:lattice}}
          \\&\implies \text{\prope{weak modus ponens}}
        \end{align*}

      \item Proof that on a \structe{Boolean lattice}, $\limpn=\limpc$:
        \begin{align*}
          x\limpn y
            &=      \negat{y}\limpk\negat{x}
            &&      \text{by \prefp{item:limpn_limpk}}
          \\&=      \negat\negat{y} \join \negat{x}
            &&      \text{by \prefp{item:limpk_boolean}}
          \\&=      y \join \negat{x}
            &&      \text{by \prope{involutory} property of $\negat$ \xref{def:negor}}
          \\&=      \negat{x} \join y
            &&      \text{by \prope{commutative} property of \structe{lattices} \xref{def:lattice}}
          \\&\eqd   x\limpc y
            &&      \text{by definition of $\limpc$}
          %  &\eqd   (\negat{x} \meet y ) \join (x \meet y ) \join ((\negat{x} \join y ) \meet \negat{y})
          %  &&      \text{by definition of $\limpn$}
          %\\&=      (\negat{x} \meet y ) \join (x \meet y ) \join \brs{(\negat{x}\meet \negat{y}) \join (y\meet \negat{y})}
          %  &&      \text{by \prope{distributive} property \xref{def:boolean}}
          %\\&=      (\negat{x} \meet y ) \join (x \meet y ) \join (\negat{x}\meet \negat{y})
          %  &&      \text{by \prope{non-contradiction} prop. \xref{def:boolean}}
          %\\&=      \brs{\negat{x} \meet (y \join \negat{y})} \join (x \meet y ) 
          %  &&      \text{by \prope{distributive} property \xref{def:boolean}}
          %\\&=      \negat{x} \join (x \meet y ) 
          %  &&      \text{by \prope{exlusive middle} prop. \xref{def:boolean}}
          %\\&=      \negat{x} \join y
          %  && \text{\ifdochas{boolean}{by \prefpp{lem:boa_xcxy}}}
          %\\&\eqd   x\limpc y
          %  &&      \text{by definition of $\limpc$}
        \end{align*} 
  \end{enumerate}

  \item Proofs for the \fncte{relevance implication} $\limpr$:
    \begin{enumerate}
      \item Proof that on an \structe{ortho lattice}, $\limpr$ does \emph{not} have \prope{weak entailment}:
        \\\begin{minipage}{\tw-50mm}
          In the \structe{ortho lattice} to the right\ldots
          \begin{align*}
            x\orel y
              &\implies x\limpr y
            \\&\eqd ( \negat{x} \meet y ) \join (x \meet y ) \join ( \negat{x} \meet \negat{y})
              &&    \text{by definition of $\limpr$}
            \\&=    \lzero \join x \join \negat{y}
            \\&=    x \join \negat{y}
            \\&\neq  x \join y
          \end{align*}
        \end{minipage}\hfill%
        \hfill\tbox{\includegraphics{graphics/lat6_o6_10xy-x.pdf}}%
        %\begin{minipage}{25mm}
        %  \psset{unit=5mm}\gsize%
        %  %============================================================================
% Daniel J. Greenhoe
% LaTeX file
% lattice O6
%============================================================================
\begin{pspicture}(-2.2,-\latbot)(2.2,3.3)%
  %---------------------------------
  % nodes
  %---------------------------------
  \Cnode(0,3){t}%
  \Cnode(-1,2){c}\Cnode(1,2){d}%
  \Cnode(-1,1){x}\Cnode(1,1){y}%
  \Cnode(0,0){b}%
  %---------------------------------
  % node connections
  %---------------------------------
  \ncline{t}{c}\ncline{t}{d}%
  \ncline{c}{x}\ncline{d}{y}%
  \ncline{b}{x}\ncline{b}{y}%
  %---------------------------------
  % node labels
  %---------------------------------
  \uput[0](t) {$\lid$}%
  \uput[0](d) {$\negat{y}$}%
  \uput[180](c) {$x$}%
  \uput[0](y) {$\negat{x}$}%
  \uput[180](x) {$y$}%
  \uput[0](b) {$\lzero$}%
\end{pspicture}%%
        %\end{minipage}

      \item Proof that on an \structe{orthomodular lattice}, $\limpr$ \emph{does} have \prope{strong entailment}:
        \begin{align*}
          x\orel y
            &\implies x\limpr y 
          \\&\eqd ( \negat{x} \meet y ) \join (x \meet y ) \join ( \negat{x} \meet \negat{y})  
            &&    \text{by definition of $\limpr$}
          \\&=    ( \negat{x} \meet y ) \join x \join ( \negat{x} \meet \negat{y})  
            &&    \text{by $x\orel y$ hypothesis}
          \\&=    ( \negat{x} \meet y ) \join x \join \negat{y}    
            &&    \text{by $x\orel y$ and \prope{antitone} property \xref{def:negor}}
          \\&=    y \join \negat{y}    
            &&    \text{by \fncte{orthomodular identity} \xref{def:latom}}
          \\&=    \lid
            &&    \text{by \prope{excluded middle} property of $\negat$ \xref{thm:latn_ortho}}
        \end{align*}

      \item Proof that on an \structe{ortho lattice}, $\limpr$ \emph{does} have \prope{weak modus ponens}:
        \begin{align*}
          x\meet\brp{x\limpr y}
            &\eqd x\meet\brs{( \negat{x} \meet y ) \join (x \meet y ) \join ( \negat{x} \meet \negat{y})}  
            &&    \text{by definition of $\limpr$}
          \\&\orel \brs{( \negat{x} \meet y ) \join (x \meet y ) \join ( \negat{x} \meet \negat{y})}  
            &&    \text{by definition of $\meet$ \xref{def:meet}}
          \\&\orel \negat{x} \join (x \meet y ) \join ( \negat{x} \meet \negat{y})
            &&    \text{by definition of $\meet$ \xref{def:meet}}
          \\&\orel \negat{x} \join y  \join ( \negat{x} \meet \negat{y})
            &&    \text{by definition of $\meet$ \xref{def:meet}}
          \\&\orel \negat{x} \join y  
           %&&    \text{by \prope{absorption} property of \structe{lattice}s \xref{thm:lattice}}
            &&    \text{by \prope{absorption} property \xref{thm:lattice}}
          \\&\implies \text{\prope{weak modus ponens}}
        \end{align*}

      \item Proof that on a \structe{Boolean lattice}, $\limpr=\limpc$:
        \begin{align*}
          x\limpr y
            &\eqd ( \negat{x} \meet y ) \join (x \meet y ) \join ( \negat{x} \meet \negat{y})
            &&    \text{by definition of $\limpr$}
          \\&=    \brs{\negat{x} \meet (y \join \negat{y})} \join (x \meet y ) 
           %&&    \text{by \prope{distributive} property of \structe{Boolean lattice}s \xref{def:boolean}}
            &&    \text{by \prope{distributive} property \ifxref{boolean}{def:boolean}}
          \\&=    \brs{\negat{x} \meet \lid} \join (x \meet y ) 
            &&    \text{by \prope{excluded middle} property of $\negat$ \xref{thm:latn_ortho}}
          \\&=    \negat{x} \join (x \meet y ) 
            &&    \text{by definition of $\lid$ and $\meet$ \xref{def:meet}}
          \\&=    \negat{x} \join y
            &&    \text{by property of \structe{Boolean lattice}s \ifxref{boolean}{lem:boa_xcxy}}
          \\&\eqd x\limpc y
            &&    \text{by definition of $\limpc$}
        \end{align*}
    \end{enumerate}
\end{enumerate}
\end{proof}


%=======================================
\section{Logics}
%=======================================
%%---------------------------------------
%\begin{definition}
%\footnote{
%    \citerpgc{xu2003}{27}{354040175X}{Definition 2.2.1}\\
%    \citeP{xu1992}\\
%    \citeP{xu1993}\\
%    \citePppc{jun1998}{53}{54}{Definition 2.1}\\
%    \citePppc{liu1999}{24}{25}{Definition 1}\\
%    \citePppc{xu2007}{813}{814}{Definition 1}\\
%    \citePpc{liu2010}{386}{Definition 1}
%  }
%\label{def:lia}
%%---------------------------------------
%Let $\latL\eqd\latnX$ be a \structe{lattice with negation} \xref{def:latn}.
%\defbox{
%  \begin{array}{>{\indentx}FrclCDD}
%    \mc{7}{M}{$\impalgX$ is a \structd{quasi-lattice implication algebra} if}\\
%      (1). & x \limp (y \limp z) &=& y \limp (x \limp z)           & \forall x,y,z\in\setX &                    & and \\
%      (2). & x \limp x          &=& \lid                        & \forall x    \in\setX &                    & and \\
%      (3). & x \limp y          &=& \negat{y} \limp \negat{x}      & \forall x,y  \in\setX & (\prope{antitone}) & and \\
%      (4). & (x \limp y) \limp y &=& (y \limp x) \limp x           & \forall x,y  \in\setX &                    & and \\
%      (5). & \mc{3}{l}{(x \limp y = y \limp x = \lid) \implies x=y}  & \forall x,y  \in\setX &                &     \\
%    \mc{7}{M}{A \structe{quasi-lattice implication algebra} is a \structd{lattice implication algebra} if}      \\
%      (6). & (x \join y) \limp z &=& (x\limp z)\meet(y\limp z)    & \forall x,y,z\in\setX &                    & and \\
%      (7). & (x \meet y) \limp z &=& (x\limp z)\join(y\limp z)    & \forall x,y,z\in\setX &                    & 
%  \end{array}
%  }
%\end{definition}

\begin{figure}[th]
  \begin{center}
    \psset{yunit=12mm}\gsize%============================================================================
% Daniel J. Greenhoe
% LaTeX file
% lattice of logics
%============================================================================
  %\psset{xunit=1.2mm,yunit=1.5mm}
  \begin{pspicture}(-6,-0.3)(6,4.5)%
     \footnotesize
     \psset{%
       cornersize=relative,
       framearc=0.25,
       subgriddiv=1,
       gridlabels=4pt,
       gridwidth=0.2pt,
       }%
     \begin{tabstr}{0.75}
     \rput(0,4){\rnode{minimal}  {\psframebox{\begin{tabular}{c}\struct{logic}\end{tabular}}}}%
     \rput(3,3){\rnode{fuzzy}  {\psframebox{\begin{tabular}{c}\struct{fuzzy logic}\end{tabular}}}}%
     \rput(-3,2.5){\rnode{intuitionalistic}  {\psframebox{\begin{tabular}{c}\struct{intuitionalistic logic}\end{tabular}}}}%
     \rput(3,2){\rnode{demorgan}  {\psframebox{\begin{tabular}{c}\struct{de Morgan logic}\end{tabular}}}}%
     \rput(0,1){\rnode{ortho}   {\psframebox{\begin{tabular}{c}\struct{ortho logic}\end{tabular}}}}%
     \rput(0,0){\rnode{boolean} {\psframebox{\begin{tabular}{c}\struct{Boolean logic} / \struct{classic logic}\end{tabular}}}}%
     \end{tabstr}
     %
     \psset{doubleline=true}%
     \ncline{<-}{minimal}{intuitionalistic}\ncline{<-}{minimal}{fuzzy}%
     \ncline{<-}{fuzzy}{demorgan}%
     \ncline{<-}{intuitionalistic}{ortho}\ncline{<-}{demorgan}{ortho}%
     \ncline{<-}{ortho}{boolean}%
     %
     %\psgrid[unit=10mm](-8,-1)(8,9)%
  \end{pspicture}

  \end{center}
  \caption{lattice of logics\label{fig:latlogics}}
\end{figure}

%%---------------------------------------
%\begin{definition}
%\footnote{
%  \citePpc{strasburger}{136}{Definition 2.1}
%  }
%\label{def:logic}
%%---------------------------------------
%\defboxt{
%  A \structd{logic} is a \structe{lattice with negation} $\logicX$ \xref{latn}.
%  }
%\end{definition}

%---------------------------------------
\begin{definition}
\footnote{
  \citePpc{strasburger2005}{136}{Definition 2.1}, % 3764373040
  \citePpc{devries2007}{11}{Definition 16}
  }
\label{def:logic}
\label{def:logfuz}
%---------------------------------------
Let $\limp$ be an \rele{implication} \xref{def:limp} defined on a \structe{lattice with negation}
    $\latL\eqd\latnX$ \xref{def:latn}.
%Let \prope{Boolean} be defined as in \prefpp{def:boolean}.
\defbox{\begin{array}{MM}%
  ${\scy\logicX}$ is a  \structd{logic}                 &if $\negat$ is a     \fncte{minimal negation}.         \\% &\xref{def:latn}.\\
  ${\scy\logicX}$ is a  \structd{fuzzy logic}           &if $\negat$ is a     \fncte{fuzzy negation}.           \\% &\xref{def:latn}.\\
  ${\scy\logicX}$ is an \structd{intuitionalistic logic}&if $\negat$ is an    \fncte{intuitionalistic negation}.\\% &\xref{def:latn}.\\
  ${\scy\logicX}$ is a  \structd{de Morgan logic}       &if $\negat$ is a     \fncte{de Morgan negation}.       \\% &\xref{def:latn}.\\
  ${\scy\logicX}$ is a  \structd{Kleene logic}          &if $\negat$ is a     \fncte{Kleene negation}.          \\% &\xref{def:latn}.\\
  ${\scy\logicX}$ is an \structd{ortho logic}           &if $\negat$ is an    \fncte{ortho negation}.           \\% &\xref{def:latn}.\\
  ${\scy\logicX}$ is a  \structd{Boolean logic}         &if $\negat$ is an    \fncte{ortho negation} and $\latL$ is \prope{Boolean}. \\% &\xref{def:latn}.\\
  %                                                     &and $\latL$ is  \prope{Boolean}.    % &\xref{def:boolean}
\end{array}}
\end{definition}

%---------------------------------------
\begin{definition}
\footnote{
  \citerpg{novak1999}{18}{0792385950}
  }
\label{def:lequ}
%---------------------------------------
Let $\latL\eqd\logicX$ be a \structe{logic} \xref{def:logic}.
\defboxt{
  The function $\hxs{\lequ}$ in $\clFxx$ is an \fnctd{equivalence} on $\latL$ if
  \\\indentx$\begin{array}{rclC}
    x\lequ y &\eqd& (x\limp y)\meet(y\limp x) & \forall x,y\in\setX
  \end{array}$
  }
\end{definition}



%---------------------------------------
\begin{example}[\exmd{Aristotelian logic}/\exmd{classical logic}]
\footnote{
  \citerppgc{novak1999}{17}{18}{0792385950}{\scshape Example 2.1}
  }
\label{ex:log_L2}
%---------------------------------------
\exboxp{
  The \structe{classical bi-variate logic} is defined below. %
  It is a 2 element \structe{Boolean logic} \xref{def:logic}.
  with $\latL\eqd\latn{\setn{\lT,\lF}}{\join}{\meet}{\negat}{\lF}{\lT}{\orel}$ and a \fncte{classical implication} $\limp$
  with \prope{strong entailment} and \prope{strong modus ponens}.
  The value $\lT$ represents ``\vale{true}" and $\lF$ represents ``\vale{false}".
  \\\indentx
  \tbox{\includegraphics{graphics/logic_L2_tf.pdf}}%
  %$\begin{array}{M}%
  %  \psset{yunit=8mm}%
  %  \gsize%============================================================================
% Daniel J. Greenhoe
% LaTeX file
% lattice (2^{x,y,z}, subseteq)
% nominal unit = 10mm
%============================================================================
{\psset{yunit=0.75\psunit}%
\begin{pspicture}(-0.15,-0.2)(1.2,1.2)%
  %---------------------------------
  % settings
  %---------------------------------
  %\psset{%
  %  %
  %  }%
  %---------------------------------
  % nodes
  %---------------------------------
  \Cnode(0,1){t}%
  \Cnode(0,0){f}%
  %---------------------------------
  % node connections
  %---------------------------------
  \ncline{t}{f}%
  %---------------------------------
  % node labels
  %---------------------------------
  %\uput[0](t) {$\lid=\lzero^\negat$}%
  %\uput[0](m) {$\frac{1}{2}=\frac{1}{2}^\negat$}%
  %\uput[0](f) {$\lzero=\lid^\negat$}%
  %\uput[0](t) {$\lT=\negat\lF$}%
  %\uput[0](n) {$\lN=\negat\lN$}%
  %\uput[0](f) {$\lF=\negat\lT$}%
  %
  \uput[0](t) {$\lT=\negat\lF$}%
  \uput[0](f) {$\lF=\negat\lT$}%
  %\uput[0](t) {$\negat\lF$}%
  %\uput[0](f) {$\negat\lT$}%
  %
 % \uput[180](t) {(``true")}%
 %%\uput[180](m) {$\brp{\begin{array}{M}``neutral"/\\``undefined"\end{array}}$}%
 % \uput[180](m) {(``neutral")}%
 % \uput[180](f) {(``false")}%
\end{pspicture}%
}%%
  %\end{array}$
  \indentx
  $x\limp y 
    \eqd\brb{\begin{array}{lD}
             \lT              & $\forall x\orel y$\\
             y                & otherwise
           \end{array}}
    = \brb{\begin{array}{c|cc}
              \limp & \lT & \lF\\
              \hline      %------------
              \lT         & \lT & \lF\\
              \lF         & \lT & \lT
            \end{array}\qquad\scy\forall x,y\in\setX}
    = \negat{x} \join y$
  %\\Equivalence \xref{def:lequ} in this logic is\indentx
  %$x\lequ y 
  %  = \brb{\begin{array}{c|cc}
  %            \lequ & \lT & \lF\\
  %            \hline      %------------
  %            \lT         & \lT & \lF\\
  %            \lF         & \lF & \lT
  %          \end{array}\quad\scy\forall x,y\in\setX}$
  }
\end{example}
\begin{proofns}
\begin{enumerate}
  \item Proof that $\negat$ is an \fncte{ortho negation}: by \prefpp{def:negor}
  \item Proof that $\limp$ is an \fncte{implication} with \prope{strong entailment} and \prope{strong modus ponens}:
    \begin{enumerate}
        \item $\latL$ is \prope{Boolean} and therefore is \prope{orthocomplemented}.
        \item $\limp$ is equivalent to the \fncte{classical implication} $\limpc$ \xref{ex:limpx}.
        \item By \prefpp{ex:limpx}, $\limp$ has \prope{strong entailment} and \prope{strong modus ponens}.
    \end{enumerate}
\end{enumerate}
\end{proofns}

The \structe{classical logic} (previous example) can be generalized in several ways.
Arguably one of the simplest of these is the 3-valued logic due to Kleene (next example).
%---------------------------------------
\begin{example}[\exmd{Kleene 3-valued logic}]
\footnote{
  \citePp{kleene1938}{153},
  \citerppc{kleene1952}{332}{339}{\textsection 64. The 3-valued logic},
  \citePp{avron1991}{277}
  }
\label{ex:logic_kleene3}
%---------------------------------------
\exboxp{
  The \structe{Kleene 3-valued logic} $\logicX$ is defined below.
  The function $\negat$ is a \prope{Kleene negation} \xxref{def:negkl}{ex:negat_L3_kl}
  defined on a 3 element \prope{linearly ordered lattice} \xref{def:toset}.
  The function $\limp$ is the \fncte{classical implication} $x\limp y\eqd \negat{x}\join y$.
  The values $\lT$ represents ``\vale{true}", $\lF$ represents ``\vale{false}", and $\lN$ represents ``\vale{neutral}" or ``\vale{undecided}".
  \\\indentx
  \tbox{\includegraphics{graphics/logic_L3_tnf.pdf}}%
  %$\begin{array}{M}%
  %  \psset{yunit=8mm}%
  %  \gsize%============================================================================
% Daniel J. Greenhoe
% LaTeX file
% lattice (2^{x,y,z}, subseteq)
% nominal unit = 10mm
%============================================================================
{\psset{yunit=0.75\psunit}%
\begin{pspicture}(-0.15,-0.2)(1.2,2.2)%
  %---------------------------------
  % settings
  %---------------------------------
  %\psset{%
  %  %
  %  }%
  %---------------------------------
  % nodes
  %---------------------------------
  \Cnode(0,2){t}%
  \Cnode(0,1){n}%
  \Cnode(0,0){f}%
  %---------------------------------
  % node connections
  %---------------------------------
  \ncline{t}{n}%
  \ncline{n}{f}%
  %---------------------------------
  % node labels
  %---------------------------------
  %\uput[0](t) {$\lid=\lzero^\negat$}%
  %\uput[0](m) {$\frac{1}{2}=\frac{1}{2}^\negat$}%
  %\uput[0](f) {$\lzero=\lid^\negat$}%
  %\uput[0](t) {$\lT=\negat\lF$}%
  %\uput[0](n) {$\lN=\negat\lN$}%
  %\uput[0](f) {$\lF=\negat\lT$}%
  %
  \uput[0](t) {$\lT=\negat\lF$}%
  \uput[0](n) {$\lN=\negat\lN$}%
  \uput[0](f) {$\lF=\negat\lT$}%
  %\uput[0](t) {$\negat\lF$}%
  %\uput[0](n) {$\negat\lN$}%
  %\uput[0](f) {$\negat\lT$}%
  %
 % \uput[180](t) {(``true")}%
 %%\uput[180](m) {$\brp{\begin{array}{M}``neutral"/\\``undefined"\end{array}}$}%
 % \uput[180](m) {(``neutral")}%
 % \uput[180](f) {(``false")}%
\end{pspicture}%
}%%
  %\end{array}$
  \indentx\indentx
  $x\limp y 
    \eqd \brb{\begin{array}{lD}
                 \negat{x}\join y & $\forall x\in\setX$\\
               \end{array}}
    =    \brb{\begin{array}{c|ccc}
              \limp   & \lT & \lN & \lF\\
              \hline        %------------------
              \lT           & \lT & \lN & \lF\\
              \lN           & \lT & \lN & \lN\\
              \lF           & \lT & \lT & \lT
            \end{array}\quad\scy\forall x,y\in\setX}$
  }
\end{example}
\begin{proofns}
\begin{enumerate}
  \item Proof that $\negat$ is a \fncte{Kleene negation}: see \prefpp{ex:negat_L3_kl}
  \item Proof that $\limp$ is an \fncte{implication}: 
        This follows directly from the definition of $\limp$ and the definition of an \fncte{implication} \xref{def:limp}.
  \item Proof that $\limp$ does not have \prope{strong entailment}: $n\limp n=n=n\join n\neq\lid$.
  \item Proof that $\limp$ does not have \prope{strong modus ponens}: $n\limp\lzero=n=\negat{n}\join\lzero\nleq\lzero$.
\end{enumerate}
\end{proofns}

A lattice and negation alone do not uniquely define a logic. 
/-Lukasiewicz also introduced a 3-valued logic with identical lattice structure to Kleene,
but with a different implication relation (next example).
Historically, /-Lukasiewicz's logic was introduced before Kleene's. 
%---------------------------------------
\begin{example}[\exmd{/-Lukasiewicz 3-valued logic}]
\footnote{
  \citePpgc{lukasiewicz1920}{17}{0198243049}{II. The principles of consequence},
  \citePpc{avron1991}{277}{/-Lukasiewicz.}
  }
\label{ex:logic_lukas3}
%---------------------------------------
\\\begin{minipage}{\tw-25mm}
%\exboxp{
  The \structe{/-Lukasiewicz 3-valued logic} $\logicX$ is defined to the right and below.
  The function $\negat$ is a \prope{Kleene negation} \xref{def:negkl}
  defined on a 3 element \prope{linearly ordered lattice} \xref{def:toset}.
  The implication has \prope{strong entailment} but \prope{weak modus ponens}.
  In the implication table below, values that differ from the classical $x\limp y\eqd \negat{x}\join y$ are \colorbox{shadedcell}{shaded}.
  %The values $\lT$ represents ``\vale{true}", $\lF$ represents ``\vale{false}", and $\lN$ represents ``\vale{neutral}" or ``\vale{undecided}".
\end{minipage}%
\hfill%
  %\\%\indentx
\tbox{\includegraphics{graphics/logic_L3_tnf.pdf}}%
%\begin{minipage}{20mm}%
%  %$\begin{array}{M}%
%    \psset{yunit=8mm}%
%    \gsize%============================================================================
% Daniel J. Greenhoe
% LaTeX file
% lattice (2^{x,y,z}, subseteq)
% nominal unit = 10mm
%============================================================================
{\psset{yunit=0.75\psunit}%
\begin{pspicture}(-0.15,-0.2)(1.2,2.2)%
  %---------------------------------
  % settings
  %---------------------------------
  %\psset{%
  %  %
  %  }%
  %---------------------------------
  % nodes
  %---------------------------------
  \Cnode(0,2){t}%
  \Cnode(0,1){n}%
  \Cnode(0,0){f}%
  %---------------------------------
  % node connections
  %---------------------------------
  \ncline{t}{n}%
  \ncline{n}{f}%
  %---------------------------------
  % node labels
  %---------------------------------
  %\uput[0](t) {$\lid=\lzero^\negat$}%
  %\uput[0](m) {$\frac{1}{2}=\frac{1}{2}^\negat$}%
  %\uput[0](f) {$\lzero=\lid^\negat$}%
  %\uput[0](t) {$\lT=\negat\lF$}%
  %\uput[0](n) {$\lN=\negat\lN$}%
  %\uput[0](f) {$\lF=\negat\lT$}%
  %
  \uput[0](t) {$\lT=\negat\lF$}%
  \uput[0](n) {$\lN=\negat\lN$}%
  \uput[0](f) {$\lF=\negat\lT$}%
  %\uput[0](t) {$\negat\lF$}%
  %\uput[0](n) {$\negat\lN$}%
  %\uput[0](f) {$\negat\lT$}%
  %
 % \uput[180](t) {(``true")}%
 %%\uput[180](m) {$\brp{\begin{array}{M}``neutral"/\\``undefined"\end{array}}$}%
 % \uput[180](m) {(``neutral")}%
 % \uput[180](f) {(``false")}%
\end{pspicture}%
}%%
%  %\end{array}$
%\end{minipage}
\\\indentx
  $x\limp y 
    \eqd \brb{\begin{array}{lD}
                 \lT              & $\forall x\orel y$\\
                 \negat{x}\join y & otherwise
               \end{array}}
    = \brb{\begin{array}{c|ccc}
              \limp & \lT & \lN & \lF\\
              \hline      %------------------
              \lT         & \lT & \lN & \lF\\
              \lN         & \lT & \scell\lT & \lN\\
              \lF         & \lT & \lT & \lT
            \end{array}\quad\scy\forall x,y\in\setX}
    = \brb{\begin{array}{lD}
                 \lT              & for $x=y=\lN$\\
                 \negat{x}\join y & otherwise
               \end{array}}$
%  }
\end{example}
\begin{proofns}
\begin{enumerate}
  \item Proof that $\negat$ is a \fncte{Kleene negation}: see \prefpp{ex:negat_L3_kl}
  \item Proof that $\limp$ is an \fncte{implication}: 
        This follows directly from the definition of $\limp$ and the definition of an \fncte{implication} \xref{def:limp}.
 %\item Proof that $\limp$ does not have \prope{strong entailment}: $n\limp n=n=n\join n\neq\lid$.
  \item Proof that $\limp$ does not have \prope{strong modus ponens}: $n\limp\lzero=n=\negat{n}\join\lzero\nleq\lzero$.
\end{enumerate}
\end{proofns}

%---------------------------------------
\begin{example}[\exmd{RM$_3$ logic}]
\footnote{
  \citePpp{avron1991}{277}{278}\\
  \citeP{sobocinski1952}
  }
\label{ex:logic_rm3}
%---------------------------------------
\exboxp{
  The \structe{RM$_3$ logic} $\logicX$ is defined below.
  The function $\negat$ is a \prope{Kleene negation} \xref{def:negkl}
  defined on a 3 element \prope{linearly ordered lattice} \xref{def:toset}.
  The implication function has \prope{weak entailment} by \prope{strong modus ponens}.
  In the implication table below, values that differ from the classical $x\limp y\eqd \negat{x}\join y$ are \colorbox{shadedcell}{shaded}.
  %The order structure and negation are the same as in \prefpp{ex:logic_lukas3}.
  \\\indentx\tbox{\includegraphics{graphics/logic_L3_tnf.pdf}}%
  %$\begin{array}{M}%
  %  \psset{yunit=8mm}%
  %  \gsize%============================================================================
% Daniel J. Greenhoe
% LaTeX file
% lattice (2^{x,y,z}, subseteq)
% nominal unit = 10mm
%============================================================================
{\psset{yunit=0.75\psunit}%
\begin{pspicture}(-0.15,-0.2)(1.2,2.2)%
  %---------------------------------
  % settings
  %---------------------------------
  %\psset{%
  %  %
  %  }%
  %---------------------------------
  % nodes
  %---------------------------------
  \Cnode(0,2){t}%
  \Cnode(0,1){n}%
  \Cnode(0,0){f}%
  %---------------------------------
  % node connections
  %---------------------------------
  \ncline{t}{n}%
  \ncline{n}{f}%
  %---------------------------------
  % node labels
  %---------------------------------
  %\uput[0](t) {$\lid=\lzero^\negat$}%
  %\uput[0](m) {$\frac{1}{2}=\frac{1}{2}^\negat$}%
  %\uput[0](f) {$\lzero=\lid^\negat$}%
  %\uput[0](t) {$\lT=\negat\lF$}%
  %\uput[0](n) {$\lN=\negat\lN$}%
  %\uput[0](f) {$\lF=\negat\lT$}%
  %
  \uput[0](t) {$\lT=\negat\lF$}%
  \uput[0](n) {$\lN=\negat\lN$}%
  \uput[0](f) {$\lF=\negat\lT$}%
  %\uput[0](t) {$\negat\lF$}%
  %\uput[0](n) {$\negat\lN$}%
  %\uput[0](f) {$\negat\lT$}%
  %
 % \uput[180](t) {(``true")}%
 %%\uput[180](m) {$\brp{\begin{array}{M}``neutral"/\\``undefined"\end{array}}$}%
 % \uput[180](m) {(``neutral")}%
 % \uput[180](f) {(``false")}%
\end{pspicture}%
}%%
  %\end{array}$
  \indentx\indentx
  $x\limp y 
    \eqd \brb{\begin{array}{lD}
                 \lT   & $\forall x<y$\\
                 \lN   & $\forall x=y$\\
                 \lF   & $\forall x>y$
               \end{array}}
    =    \brb{\begin{array}{c|ccc}
              \limp   & \lT & \lN & \lF\\
              \hline        %------------------
              \lT           & \lT & \scell\lF & \lF\\
              \lN           & \lT & \lN & \scell\lF\\
              \lF           & \lT & \lT & \lT
            \end{array}\qquad\scy\forall x,y\in\setX}$
  }
\end{example}
\begin{proofns}
\begin{enumerate}
  \item Proof that $\negat$ is a \fncte{Kleene negation}: see \prefpp{ex:negat_L3_kl}
  \item Proof that $\limp$ is an \fncte{implication}: 
        This follows directly from the definition of $\limp$ and the definition of an \fncte{implication} \xref{def:limp}.
  \item Proof that $\limp$ does not have \prope{strong entailment}: $n\limp n=n=n\join n\neq\lid$.
  %\item Proof that $\limp$ does not have \prope{strong modus ponens}: $n\limp\lzero=n=\negat{n}\join\lzero\nleq\lzero$.
\end{enumerate}
\end{proofns}

In a 3-valued logic, the negation does not necessarily have to be as in the previous three examples.
The next example offers a different negation.
%---------------------------------------
\begin{example}[\exmd{Heyting 3-valued logic}/\exmd{Ja/'skowski's first matrix}]
\footnote{
  \citerpg{karpenko2006}{45}{0955117038},
  \citerpgc{johnstone1982}{9}{0521337798}{\textsection 1.12},
  \citeP{heyting1930a},
  \citeP{heyting1930b},
  \citeP{heyting1930c},
  \citeP{heyting1930d},
  \citeP{jaskowski1936},
  \citer{mancosu1998}
  }
\label{ex:logic_heyting3}
%---------------------------------------
\exboxp{
  The \structe{Heyting 3-valued logic} $\logicX$ is defined below.
  The negation $\negat$ is both \prope{intuitionistic} and \prope{fuzzy} \xref{def:negint},
  and is defined on a 3 element \prope{linearly ordered lattice} \xref{def:toset}.
  The implication function has both \prope{strong entailment} and \prope{strong modus ponens}.
  In the implication table below, values that differ from the classical $x\limp y\eqd \negat{x}\join y$ are \colorbox{shadedcell}{shaded}.
  \\\indentx\tbox{\includegraphics{graphics/logic_L3_tnf_fft.pdf}}%
  %$\begin{array}{M}%
  %  \psset{yunit=8mm}%
  %  \gsize%============================================================================
% Daniel J. Greenhoe
% LaTeX file
% lattice (2^{x,y,z}, subseteq)
% nominal unit = 10mm
%============================================================================
{\psset{yunit=0.75\psunit}%
\begin{pspicture}(-0.15,-0.2)(1.2,2.2)%
  %---------------------------------
  % settings
  %---------------------------------
  %\psset{%
  %  %
  %  }%
  %---------------------------------
  % nodes
  %---------------------------------
  \Cnode(0,2){t}%
  \Cnode(0,1){n}%
  \Cnode(0,0){f}%
  %---------------------------------
  % node connections
  %---------------------------------
  \ncline{t}{n}%
  \ncline{n}{f}%
  %---------------------------------
  % node labels
  %---------------------------------
  %\uput[0](t) {$\lid=\lzero^\negat$}%
  %\uput[0](m) {$\frac{1}{2}=\frac{1}{2}^\negat$}%
  %\uput[0](f) {$\lzero=\lid^\negat$}%
  %\uput[0](t) {$\lT=\negat\lF$}%
  %\uput[0](n) {$\lN=\negat\lN$}%
  %\uput[0](f) {$\lF=\negat\lT$}%
  %
  \uput[0](t) {$\lT=\negat\lF$}%
  \uput[0](n) {$\lN$}%
  \uput[0](f) {$\lF=\negat\lN=\negat\lT$}%
  %\uput[0](t) {$\negat\lF$}%
  %\uput[0](n) {$\negat\lN$}%
  %\uput[0](f) {$\negat\lT$}%
  %
 % \uput[180](t) {(``true")}%
 %%\uput[180](m) {$\brp{\begin{array}{M}``neutral"/\\``undefined"\end{array}}$}%
 % \uput[180](m) {(``neutral")}%
 % \uput[180](f) {(``false")}%
\end{pspicture}%
}%%
  %\end{array}$
  \indentx\indentx
  $x\limp y 
    \eqd \brb{\begin{array}{lD}
                 \lT & $\forall x\orel y$\\
                 y   & otherwise
               \end{array}}
    =    \brb{\begin{array}{c|ccc}
              \limp   & \lT & \lN & \lF\\
              \hline        %------------------
              \lT           & \lT & \lN & \lF\\
              \lN           & \lT & \scell\lT &\lF\\
              \lF           & \lT & \lT & \lT
            \end{array}\quad\scy\forall x,y\in\setX}$
  }
\end{example}
\begin{proofns}
\begin{enumerate}
  \item Proof that $\negat$ is a \fncte{Kleene negation}: see \prefpp{ex:negat_L3_intfuz}
  \item Proof that $\limp$ is an \fncte{implication}: by definition of \fncte{implication} \xref{def:limp}
\end{enumerate}
\end{proofns}

Of course it is possible to generalize to more than 3 values (next example).
%---------------------------------------
\begin{example}[\exmd{/-Lukasiewicz 5-valued logic}]
\footnote{
    \citerpgc{xu2003}{29}{354040175X}{Example 2.1.3}\\
    \citePpc{jun1998}{54}{Example 2.2}
  }
%---------------------------------------
\exboxp{
  The \structe{/-Lukasiewicz 5-valued logic} $\logicX$ is defined below.
  The implication function has \prope{strong entailment} but \prope{weak modus ponens}.
  In the implication table below, values that differ from the classical $x\limp y\eqd \negat{x}\join y$ are \colorbox{shadedcell}{shaded}.
  %a 5 element \structe{lattice with negation} \xref{def:latn}
  %$\latn{\set{\lT,\lP,\lN,\lM,\lF}}{\join}{\meet}{\negat}{\lF}{\lT}{\orel}$ as illustrated below
  %with values $\lT$ representing ``\vale{true}", $\lF$ representing ``\vale{false}", and $\lP$, $\lN$ and $\lM$ representing intermediate values,
  %and with an implication connective $\limp$ defined as
  \\\indentx\tbox{\includegraphics{graphics/logic_L5_tpnmf.pdf}}%
  %$\begin{array}{M}%
  %  \psset{yunit=5mm}%
  %  \gsize%============================================================================
% Daniel J. Greenhoe
% LaTeX file
% lattice (2^{x,y,z}, subseteq)
% nominal unit = 10mm
%============================================================================
\begin{pspicture}(-0.5,-\latbot)(3.5,4.5)%
  %---------------------------------
  % settings
  %---------------------------------
  %\psset{%
  %  %
  %  }%
  %---------------------------------
  % nodes
  %---------------------------------
  \Cnode(0,4){t}%
  \Cnode(0,3){p}%
  \Cnode(0,2){n}%
  \Cnode(0,1){m}%
  \Cnode(0,0){f}%
  %---------------------------------
  % node connections
  %---------------------------------
  \ncline{t}{p}%
  \ncline{p}{n}%
  \ncline{n}{m}%
  \ncline{m}{f}%
  %---------------------------------
  % node labels
  %---------------------------------
  %\uput[0](t) {$\lid=\lzero^\negat$}%
  %\uput[0](m) {$\frac{1}{2}=\frac{1}{2}^\negat$}%
  %\uput[0](f) {$\lzero=\lid^\negat$}%
  %\uput[0](t) {$\lT=\negat\lF$}%
  %\uput[0](n) {$\lN=\negat\lN$}%
  %\uput[0](f) {$\lF=\negat\lT$}%
  %
  \uput[0](t) {$\lT=\negat\lF$}%
  \uput[0](p) {$\lP=\negat\lM$}%
  \uput[0](n) {$\lN=\negat\lN$}%
  \uput[0](m) {$\lM=\negat\lP$}%
  \uput[0](f) {$\lF=\negat\lT$}%
  %\uput[0](t) {$\negat\lF$}%
  %\uput[0](p) {$\negat\lM$}%
  %\uput[0](n) {$\negat\lN$}%
  %\uput[0](m) {$\negat\lP$}%
  %\uput[0](f) {$\negat\lT$}%
  %
 % \uput[180](t) {(``true")}%
 %%\uput[180](m) {$\brp{\begin{array}{M}``neutral"/\\``undefined"\end{array}}$}%
 % \uput[180](m) {(``neutral")}%
 % \uput[180](f) {(``false")}%
\end{pspicture}%%
  %\end{array}$
  \indentx\indentx
  $x\limp y 
    \eqd \brbl{\begin{array}{c|ccccc}
              \limp & \lT & \lP & \lN & \lM & \lF\\
              \hline%------------------------------
              \lT   & \lT & \lP & \lN & \lM & \lF\\
              \lP   & \lT & \scell\lT & \lN & \lM & \lM\\
              \lN   & \lT & \scell\lT & \scell\lT & \scell\lM & \lN\\
              \lM   & \lT & \scell\lT & \scell\lT & \scell\lT & \lP\\
              \lF   & \lT & \lT & \lT & \lT & \lT
            \end{array}}\qquad\scy\forall x,y\in\setX$
  }
\end{example}
\begin{proof}

\end{proof}

All the previous examples in this section are \prope{linearly ordered}.
The following examples employ logics that are not.
%---------------------------------------
\begin{example}[\exmd{Boolean 4-valued logic}]
\citetblp{
  \citePp{belnap1977}{13},
  \citerpgc{restall2000}{177}{041521534X}{Example 8.44},
  \citePpgc{pavicic2008}{28}{0080931669}{Definition 2, \rele{classical implication}},
  %\citePpgc{pavicic2008}{28}{0080931669}{Definition 2, $a\limp_0 b\eqd \negat{a} \setu b$ (\rele{classical implication})}
  \citeP{mittelstaedt1970},
  \citePpc{finch1970}{102}{(1.1)}, %{ $\brp{x\limp y}=\brp{y\meet x}\join Nx$}\\
  \citePpg{smets2006}{270}{9042021306}
  }
\label{ex:logic_m2_boolean}
%---------------------------------------
\exboxp{
  The \structe{Boolean 4-valued logic} is defined below.
  The negation function $\negat$ is an \fncte{ortho negation} \xref{ex:negat_m2_ortho}
  defined on an \structe{$M_2$ lattice}.
  The value $\lT$ represents ``\vale{true}", $\lF$ represents ``\vale{false}", and $\lM$ and $\lN$ represent some intermediate values.
  \\\indentx\tbox{\includegraphics{graphics/logic_M2_tmnf.pdf}}%
  %$\begin{array}{M}%
  %  \psset{unit=8mm}%
  %  \gsize%============================================================================
% Daniel J. Greenhoe
% LaTeX file
% lattice (2^{x,y,z}, subseteq)
% nominal unit = 7.5mm
%============================================================================
\begin{pspicture}(-1.75,-0.2)(1.75,2.2)%
  %---------------------------------
  % settings
  %---------------------------------
  %\psset{%
  %  %
  %  }%
  %---------------------------------
  % nodes
  %---------------------------------
  \Cnode(0,2){t}%
  \Cnode(-1,1){m}\Cnode(1,1){n}%
  \Cnode(0,0){f}%
  %---------------------------------
  % node connections
  %---------------------------------
  \ncline{t}{m}\ncline{t}{n}%
  \ncline{f}{m}\ncline{f}{n}%
  %---------------------------------
  % node labels
  %---------------------------------
  %\uput[0](t) {$\lid=\lzero^\negat$}%
  %\uput[0](m) {$\frac{1}{2}=\frac{1}{2}^\negat$}%
  %\uput[0](f) {$\lzero=\lid^\negat$}%
  %\uput[0](t) {$\lT=\negat\lF$}%
  %\uput[0](n) {$\lN=\negat\lN$}%
  %\uput[0](f) {$\lF=\negat\lT$}%
  %
  \uput[0](t) {$\lT=\negat\lF$}%
  \uput[90](m) {$\lB=\negat\lN$}%
  \uput[-90](n) {$\lN=\negat\lB$}%
  \uput[180](f) {$\lF=\negat\lT$}%
  %\uput[0](t) {$\negat\lF$}%
  %\uput[0](n) {$\negat\lN$}%
  %\uput[0](f) {$\negat\lT$}%
\end{pspicture}%%
  %\end{array}$
  \indentxx
  $x\limp y \eqd \negat{x} \join y
    = \brbl{\begin{array}{c|cccc}
              \limp   & \lT & \lB & \lN & \lF\\
              \hline        %-----------& -----------
              \lT           & \lT & \lB & \lN & \lF\\
              \lB           & \lT & \lT & \lN & \lN\\
              \lN           & \lT & \lB & \lT & \lB\\
              \lF           & \lT & \lT & \lT & \lT
            \end{array}}\qquad\scy\forall x,y\in\setX$
  }
\end{example}

%---------------------------------------
\begin{example}[\exmd{Sasaki hook} / \exmd{quantum implication}]
\footnote{
  \citePpgc{pavicic2008}{28}{0080931669}{Definition 2}, %, {\rele{quantum implication}, \rele{Sasaki hook}}\\
  \citeP{mittelstaedt1970},
  \citePpc{finch1970}{102}{(1.1)}, %{ $\brp{x\limp y}=\brp{y\meet x}\join Nx$}\\
  \citePpg{smets2006}{270}{9042021306}
  }
\label{ex:logic_sasaki}
%---------------------------------------
\exboxp{
  The \structe{Sasaki hook logic} $\logicX$ is defined below.
  The order structure and negation are the same as in \prefpp{ex:logic_m2_boolean}.
  \\\indentx$\begin{array}{M}%
    \psset{unit=8mm}%
    \gsize%============================================================================
% Daniel J. Greenhoe
% LaTeX file
% lattice (2^{x,y,z}, subseteq)
% nominal unit = 7.5mm
%============================================================================
\begin{pspicture}(-1.75,-0.2)(1.75,2.2)%
  %---------------------------------
  % settings
  %---------------------------------
  %\psset{%
  %  %
  %  }%
  %---------------------------------
  % nodes
  %---------------------------------
  \Cnode(0,2){t}%
  \Cnode(-1,1){m}\Cnode(1,1){n}%
  \Cnode(0,0){f}%
  %---------------------------------
  % node connections
  %---------------------------------
  \ncline{t}{m}\ncline{t}{n}%
  \ncline{f}{m}\ncline{f}{n}%
  %---------------------------------
  % node labels
  %---------------------------------
  %\uput[0](t) {$\lid=\lzero^\negat$}%
  %\uput[0](m) {$\frac{1}{2}=\frac{1}{2}^\negat$}%
  %\uput[0](f) {$\lzero=\lid^\negat$}%
  %\uput[0](t) {$\lT=\negat\lF$}%
  %\uput[0](n) {$\lN=\negat\lN$}%
  %\uput[0](f) {$\lF=\negat\lT$}%
  %
  \uput[0](t) {$\lT=\negat\lF$}%
  \uput[90](m) {$\lB=\negat\lN$}%
  \uput[-90](n) {$\lN=\negat\lB$}%
  \uput[180](f) {$\lF=\negat\lT$}%
  %\uput[0](t) {$\negat\lF$}%
  %\uput[0](n) {$\negat\lN$}%
  %\uput[0](f) {$\negat\lT$}%
\end{pspicture}%%
  \end{array}$
    \indentxx$x\limp y 
    \eqd \negat{x} \join \brp{x \meet y}
    = \brbl{\begin{array}{c|cccc}
              \limp   & \lT & \lB & \lN & \lF\\
              \hline        %------------------------
              \lT           & \lT & \lB & \lN & \lF\\
              \lB           & \lT & \lT & \lN & \lN\\
              \lN           & \lT & \lB & \lT & \lB\\
              \lF           & \lT & \lT & \lT & \lT
            \end{array}}\qquad\scy\forall x,y\in\setX$
  }
\end{example}


All the previous examples in this section are \prope{distributive}; the previous example was \prope{Boolean}.
The next example is \prope{non-distributive}, and \prope{de Morgan} (but \prope{non-Boolean}).
Note for a given order structure, the method of negation may not be unique;
in the previous and following examples both have identical lattices, but are negated differently.
%---------------------------------------
\begin{example}[\exmd{BN$_4$ logic}]
\footnote{
  \citerpgc{restall2000}{171}{041521534X}{Example 8.39}
  }
\label{ex:logic_bn4}
%---------------------------------------
\exboxp{
  The \structe{BN$_4$ logic} is defined below.
  The function $\negat$ is a \prope{de Morgan negation} \xref{ex:negat_m2_bn4}
  defined on a 4 element \prope{M$_2$ lattice}.
  The value $\lT$ represents ``\vale{true}", $\lF$ represents ``\vale{false}", 
            $\lB$ represents ``\vale{both}" (both true and false), and
            $\lN$ represents ``\vale{neither}".
    In the implication table below, the values that differ from those of the \fncte{classical implication} $\limpc$ 
    are \colorbox{shadedcell}{shaded}.
  \\\indentx\tbox{\includegraphics{graphics/logic_BN4_tbnf.pdf}}%
  %$\begin{array}{M}%
  %  \psset{unit=8mm}%
  %  \gsize%============================================================================
% Daniel J. Greenhoe
% LaTeX file
% nominal unit = 10mm
%============================================================================
{\psset{unit=0.75\psunit}%
\begin{pspicture}(-1.8,-0.2)(1.8,2.2)%
  %---------------------------------
  % settings
  %---------------------------------
  %\psset{%
  %  %
  %  }%
  %---------------------------------
  % nodes
  %---------------------------------
  \Cnode(0,2){t}%
  \Cnode(-1,1){m}\Cnode(1,1){n}%
  \Cnode(0,0){f}%
  %---------------------------------
  % node connections
  %---------------------------------
  \ncline{t}{m}\ncline{t}{n}%
  \ncline{f}{m}\ncline{f}{n}%
  %---------------------------------
  % node labels
  %---------------------------------
  \uput[0](t) {$\lT=\negat\lF$}%
  \uput[90](m) {$\lB=\negat\lB$}%
  \uput[-90](n) {$\lN=\negat\lN$}%
  \uput[180](f) {$\lF=\negat\lT$}%
\end{pspicture}%
}%%
  %\end{array}$
  \indentx\indentx
  $x\limp y 
    \eqd \brbl{\begin{array}{c|cccc}
              \limp   & \lT & \lN & \lB & \lF\\
              \hline        %------------------------
              \lT           & \lT & \lN & \scell\lF & \lF\\
              \lN           & \lT & \scell\lT & \scell\lN & \lN\\
              \lB           & \lT & \scell\lN & \lB & \scell\lF\\
              \lF           & \lT & \lT & \lT & \lT
            \end{array}}\qquad\scy\forall x,y\in\setX$
  }
\end{example}


%---------------------------------------
\begin{example}
\label{ex:limpx_o6}
%---------------------------------------
\exbox{
  \begin{array}{m{\tw-49mm}}
    The tables that follow are the 6 implications defined in \prefpp{ex:limpx} on the 
    \structe{O$_6$ lattice with ortho negation} \xref{def:negor}, 
    or the \structe{O$_6$ orthocomplemented lattice} \xref{def:o6},
    illustrated to the right.
    In the tables, the values that differ from those of the \fncte{classical implication} $\limpc$ 
    are \colorbox{shadedcell}{shaded}.
  \end{array}
  \tbox{\includegraphics{graphics/lat6_o6_negor.pdf}}%
  %\begin{array}{M}
  % \psset{unit=5mm}
  % \gsize%============================================================================
% Daniel J. Greenhoe
% LaTeX file
% nominal unit = 10mm
%============================================================================
{\psset{unit=0.75\psunit}%
\begin{pspicture}(-1.5,-\latbot)(1.5,3.2)
  %---------------------------------
  % nodes
  %---------------------------------
  \Cnode(0,3){t}
  \Cnode(-1,2){c}\Cnode(1,2){d}%
  \Cnode(-1,1){x}\Cnode(1,1){y}%
  \Cnode(0,0){b}
  %---------------------------------
  % node connections
  %---------------------------------
  \ncline{t}{c}\ncline{t}{d}%
  \ncline{c}{x}\ncline{d}{y}%
  \ncline{b}{x}\ncline{b}{y}%
  %---------------------------------
  % node labels
  %---------------------------------
  %\uput[0](t) {$\lid=\negat\lzero$}%     
  %\uput[0](d) {$d=\negat{a}$}%     
  %\uput[180](c) {$\negat{b}=c$}%     
  %\uput[0](y) {$b=\negat{c}$}%     
  %\uput[180](x) {$\negat{d}=a$}%     
  %\uput[0](b) {$\lzero=\negat\lid$}%     
  %---------------------------------
  \uput[0](t){$\lid$}\uput[180](t){$\negat\lzero$}%     
  \uput[180](d){$\negat{a}$}\uput[0](d){$d$}%     
  \uput[180](c){$c$}\uput[0](c){$\negat{b}$}%     
  \uput[180](y){$\negat{c}$}\uput[0](y){$b$}%     
  \uput[180](x) {$a$}\uput[0](x){$\negat{d}$}%     
  \uput[180](b){$\lzero$}\uput[0](b){$\negat\lid$}%     
\end{pspicture}
}%
  %\end{array}
  }
  \\\indentx
 $\begin{array}[t]{|c|cccccc|}
    \hline
    \limpc  & \lT & d   & c   & b   & a   & \lF \\
    \hline        %------------------------
    \lT           & \lT & d   & c   & b   & a   & \lF \\ 
    d             & \lT & \lT & c   & \lT & a   & a   \\
    c             & \lT & d   & \lT & b   & \lT & b   \\
    b             & \lT & \lT & c   & \lT & c   & c   \\
    a             & \lT & d   & \lT & d   & \lT & d   \\
    \lF           & \lT & \lT & \lT & \lT & \lT & \lT \\
    \hline
  \end{array}$\hfill
 $\begin{array}[t]{|c|cccccc|}
    \hline
    \limps  & \lT & d   & c   & b   & a   & \lF \\
    \hline        %------------------------
    \lT           & \lT & d   & c    & b   & a   & \lF \\ 
    d             & \lT & \lT & \scell{a}  & \lT & a   & a   \\
    c             & \lT & \scell{b}  & \lT & b   & \lT & b   \\
    b             & \lT & \lT & c    & \lT & c   & c   \\
    a             & \lT & \scell{\lT}& \lT & d   & \lT & d   \\
    \lF           & \lT & \lT & \lT  & \lT & \lT & \lT \\
    \hline
  \end{array}$\hfill
 $\begin{array}[t]{|c|cccccc|}
    \hline
    \limpd  & \lT & d   & c   & b   & a   & \lF \\
    \hline        %------------------------
    \lT           & \lT & d   & c   & b         & a         & \lF \\ 
    d             & \lT & \lT & c   & \lT       & a         & a   \\
    c             & \lT & d   & \lT & b         & \lT       & b   \\
    b             & \lT & \lT & c   & \lT       & \scell{a} & c   \\
    a             & \lT & d   & \lT & \scell{b} & \lT       & d   \\
    \lF           & \lT & \lT & \lT & \lT       & \lT       & \lT \\
    \hline
  \end{array}$
\hfill\mbox{}\\\indentx
 $\begin{array}[t]{|c|cccccc|}
    \hline
    \limpk  & \lT & d   & c   & b   & a   & \lF \\
    \hline        %------------------------
    \lT           & \lT & d         & c         & b         & a           & \lF \\ 
    d             & \lT & \lT       & \scell{a} & \lT       & a           & a   \\
    c             & \lT & \scell{b} & \lT       & b         & \lT         & b   \\
    b             & \lT & \lT       & c         & \lT       & \scell{a}   & c   \\
    a             & \lT & d         & \lT       & \scell{b} & \lT         & d   \\
    \lF           & \lT & \lT       & \lT       & \lT       & \lT         & \lT \\
    \hline
  \end{array}$\hfill
 $\begin{array}[t]{|c|cccccc|}
    \hline
    \limpn  & \lT & d   & c   & b   & a   & \lF \\
    \hline        %------------------------
    \lT           & \lT & d         & c         & b         & a           & \lF \\ 
    d             & \lT & \lT       & \scell{a} & \lT       & a           & a   \\
    c             & \lT & \scell{b} & \lT       & b         & \lT         & b   \\
    b             & \lT & \lT       & c         & \lT       & \scell{a}   & c   \\
    a             & \lT & d         & \lT       & \scell{b} & \lT         & d   \\
    \lF           & \lT & \lT       & \lT       & \lT       & \lT         & \lT \\
    \hline
  \end{array}$\hfill
 $\begin{array}[t]{|c|cccccc|}
    \hline
    \limpr  & \lT & d   & c   & b   & a   & \lF \\
    \hline        %------------------------
    \lT           & \lT & d         & c         & b         & a           & \lF \\ 
    d             & \lT & \lT       & \scell{a} & \lT       & a           & a   \\
    c             & \lT & \scell{b} & \lT       & b         & \lT         & b   \\
    b             & \lT & \lT       & c         & \lT       & \scell{a}   & c   \\
    a             & \lT & d         & \lT       & \scell{b} & \lT         & d   \\
    \lF           & \lT & \lT       & \lT       & \lT       & \lT         & \lT \\
    \hline
  \end{array}$\hfill\mbox{}
\end{example}

%---------------------------------------
\begin{example}
\footnote{
    \citerppgc{xu2003}{29}{30}{354040175X}{Example 2.1.4}
  }
%---------------------------------------
\exboxp{
  A 6 element logic is defined below.
  The function $\negat$ is a \fncte{Kleene negation} \xref{ex:negat_o6slash}.
  The implication has \prope{strong entailment} but \prope{weak modus ponens}.
  In the implication table below, the values that differ from those of the \fncte{classical implication} $\limpc$ 
  are \colorbox{shadedcell}{shaded}.
  \\\indentx
  \tbox{\includegraphics{graphics/logic_O6slash_tpqmnf.pdf}}%
  %$\begin{array}{M}%
  %  \psset{unit=8mm}%
  %  \gsize%============================================================================
% Daniel J. Greenhoe
% LaTeX file
% lattice (2^{x,y,z}, subseteq)
% nominal unit = 10mm
%============================================================================
{\psset{unit=0.75\psunit}%
\begin{pspicture}(-1.8,-0.2)(1.8,3.2)%
  %---------------------------------
  % settings
  %---------------------------------
  %\psset{%
  %  %
  %  }%
  %---------------------------------
  % nodes
  %---------------------------------
  \Cnode(0,3){t}%
  \Cnode(-1,2){p}\Cnode(1,2){q}%
  \Cnode(-1,1){m}\Cnode(1,1){n}%
  \Cnode(0,0){f}%
  %---------------------------------
  % node connections
  %---------------------------------
  \ncline{t}{p}\ncline{t}{q}%
  \ncline{m}{p}\ncline{n}{q}%
  \ncline{f}{m}\ncline{f}{n}%
  \ncline{p}{n}% slash
  %---------------------------------
  % node labels
  %---------------------------------
  %\uput[0](t) {$\lT=\negat\lF$}%
  %\uput[90](p) {$\negat{n}=p$}%
  %\uput[-90](q) {$q=\negat{m}$}%
  %\uput[-90](n) {$n=\negat{p}$}%
  %\uput[90](m) {$\negat{q}=m$}%
  %\uput[180](f) {$\lF=\negat\lT$}%
  %---------------------------------
  \uput[180](t) {$\lT$}\uput[0](t) {$\negat\lF$}%
  \uput[180](p){$p$}\uput[0](p){$\negat{n}$}%
  \uput[180](q){$\negat{m}$}\uput[0](q){$q$}%
  \uput[195](n){$\negat{p}$}\uput[0](n){$n$}%
  \uput[180](m){$m$}\uput[0](m){$\negat{q}$}%
  \uput[180](f){$\lF$}\uput[0](f){$\negat\lT$}%
\end{pspicture}%
}%%
  %\end{array}$
  \indentx\indentx
  $x\limp y 
    \eqd \brbl{\begin{array}{c|cccccc}
              \limp   & \lT & p   & q   & m   & n   & \lF \\
              \hline        %------------------------
              \lT           & \lT & p   & q   & m   & n   & \lF \\ 
              p             & \lT & \scell\lT & q   & p   & \scell{q}   & n   \\
              q             & \lT & p   & \lT & m   & p   & m   \\
              m             & \lT & \lT & q   & \lT & q   & q   \\
              n             & \lT & \scell\lT & \lT & p   & \scell\lT & p   \\
              \lF           & \lT & \lT & \lT & \lT & \lT & \lT 
            \end{array}}\qquad\scy\forall x,y\in\setX$
  }
\end{example}
\begin{proofns}
\begin{enumerate}
  \item Proof that $\negat$ is a \fncte{Kleene negation}: see \prefpp{ex:negat_o6slash}
  \item Proof that $\limp$ is an \fncte{implication}: 
        This follows directly from the definition of $\limp$ and the definition of an \fncte{implication} \xref{def:limp}.
  %\item Proof that $\limp$ does not have \prope{strong entailment}: $n\limp n=n=n\join n\neq\lid$.
  \item Proof that $\limp$ does not have \prope{strong modus ponens}: 
    \\$\begin{array}{lclclclclcl}
      \negat{p}\meet(p\limp m) &=& n\meet p &=& n &\orel& p &=& \negat{p}\join m &\nleq& m\\
      \negat{n}\meet(n\limp m) &=& n\meet p &=& n &\orel& p &=& \negat{p}\join m &\nleq& m\\
      \negat{p}\meet(p\limp\lzero) &=& n\meet n &=& n &\orel& n &=& \negat{p}\join \lzero &\nleq& \lzero\\
      \negat{n}\meet(n\limp\lzero) &=& p\meet n &=& n &\orel& p &=& \negat{n}\join \lzero &\nleq& \lzero
    \end{array}$
\end{enumerate}
\end{proofns}


For an example of an 8-valued logic, see \citeP{kamide2013}.
For examples of 16-valued logics, see \citeP{shramko2005}.




%=======================================
\section{Classical two-valued logic}
%=======================================
%---------------------------------------
\begin{definition}[{Aristotelian logic}/{classical logic}]
\footnote{
  \citerppgc{novak1999}{17}{18}{0792385950}{\scshape Example 2.1}
  }
\label{def:2vlogic}
\label{def:lnot}
%---------------------------------------
\defboxp{
  The \structd{classical 2-value logic} is a 2 element \structe{lattice with ortho negation} \xref{def:negor}
  $\logicTF$ as illustrated below
  with values $\lT$ representing ``\vale{true}", $\lF$ representing ``\vale{false}", 
  and with an implication connective $\implies$ as specified below:
  \\\indentx
  \tbox{\includegraphics{graphics/logic_L2_tf.pdf}}%
  %$\begin{array}{M}%
  %  \psset{yunit=8mm}%
  %  \gsize%============================================================================
% Daniel J. Greenhoe
% LaTeX file
% lattice (2^{x,y,z}, subseteq)
% nominal unit = 10mm
%============================================================================
{\psset{yunit=0.75\psunit}%
\begin{pspicture}(-0.15,-0.2)(1.2,1.2)%
  %---------------------------------
  % settings
  %---------------------------------
  %\psset{%
  %  %
  %  }%
  %---------------------------------
  % nodes
  %---------------------------------
  \Cnode(0,1){t}%
  \Cnode(0,0){f}%
  %---------------------------------
  % node connections
  %---------------------------------
  \ncline{t}{f}%
  %---------------------------------
  % node labels
  %---------------------------------
  %\uput[0](t) {$\lid=\lzero^\negat$}%
  %\uput[0](m) {$\frac{1}{2}=\frac{1}{2}^\negat$}%
  %\uput[0](f) {$\lzero=\lid^\negat$}%
  %\uput[0](t) {$\lT=\negat\lF$}%
  %\uput[0](n) {$\lN=\negat\lN$}%
  %\uput[0](f) {$\lF=\negat\lT$}%
  %
  \uput[0](t) {$\lT=\negat\lF$}%
  \uput[0](f) {$\lF=\negat\lT$}%
  %\uput[0](t) {$\negat\lF$}%
  %\uput[0](f) {$\negat\lT$}%
  %
 % \uput[180](t) {(``true")}%
 %%\uput[180](m) {$\brp{\begin{array}{M}``neutral"/\\``undefined"\end{array}}$}%
 % \uput[180](m) {(``neutral")}%
 % \uput[180](f) {(``false")}%
\end{pspicture}%
}%%
  %\end{array}$
  \indentx
  $x\implies y 
    \eqd\brb{\begin{array}{lD}
             \lT              & $\forall x\orel y$\\
             y                & otherwise
           \end{array}}
    = \brb{\begin{array}{c|cc}
              \implies & \lT & \lF\\
              \hline      %------------
              \lT         & \lT & \lF\\
              \lF         & \lT & \lT
            \end{array}\qquad\scy\forall x,y\in\setX}
    = \negat{x} \join y$
  }
\end{definition}

%---------------------------------------
\begin{theorem}
\label{def:iff}
\label{def:log_op}
\label{def:land}
\label{def:lor}
%---------------------------------------
\thmboxt{
  If $\logicTF$ is the \structe{classical 2-value logic} \xref{def:2vlogic},\\ 
  then the \fnctd{logical OR} $\lor$, \fnctd{logical AND} $\land$, and \fnctd{logical equivalence} $\iff$ operations\\ 
  are defined as follows: 
  \\\indentx
    $\begin{array}{c|cc}
      \lor  & \lT & \lF\\
      \hline%------------
      \lT   & \lT & \lT\\
      \lF   & \lT & \lF
    \end{array}
    \qquad\qquad
    \begin{array}{c|cc}
      \land & \lT & \lF\\
      \hline%------------
      \lT   & \lT & \lF\\
      \lF   & \lF & \lF
    \end{array}
    \qquad\qquad
    \begin{array}{c|cc}
      \iff  & \lT & \lF\\
      \hline%------------
      \lT   & \lT & \lF\\
      \lF   & \lF & \lT
    \end{array}$
  }

\end{theorem}
\begin{proof}
\begin{enumerate}
  \item Proof for \ope{logical OR} operation $\join$: This follows from the \structe{lattice} \xref{def:lattice} properties of $\latL_2$.
  \item Proof for \ope{logical AND} operation $\meet$: This follows from the \structe{lattice} \xref{def:lattice} properties of $\latL_2$.
  \item Proof for \ope{logical if and only if} operation $\iff$: This follows from the definition of $\implies$ \xref{def:2vlogic}
        and \prefpp{def:lequ}.
\end{enumerate}
\end{proof}

One of the most useful facts concerning propositional logic systems is that they
form a \hie{Boolean algebra} (next theorem).
Because they are a Boolean algebra, a number of useful properties
automatically follow (next theorem) from the properties of Boolean algebras
(\prefp{thm:boo_prop}).
%---------------------------------------
\begin{theorem}[Boolean algebra properties]
\footnote{
  %\citerpg{svozil}{72}{981020809X},
  \citerpg{maclane1999}{488}{0821816462},
  \citerpg{givant2009}{10}{0387402934},
  \citerpp{muller1909}{20}{21},
  \citer{schroder1890},
  \citerppu{whitehead1898}{35}{37}{http://books.google.com/books?id=P_A8AAAAIAAJ\&pg=PA35},
  \citerp{peano1889e}{88}
  }
\label{thm:log_boolean}
\label{thm:logic}
%---------------------------------------
Let $\setft$ be the set of logical properties \prope{false} and \prope{true} \xref{ax:plogic}.
Let $\lor$ be the \ope{logical or} and $\land$ the \ope{logical and} operations \xref{def:land}.
Let $\implies$ be the \rele{logical implies} relation \xref{def:implies}.
\thmboxt{
$\lattice{\setft}{\implies}{\lor}{\land}$ is a \hie{Boolean algebra}. In particular
for all $x,y,z\in\setft$,
\footnotesize
\\$\begin{array}{lcl | lcl || D}
     x \lor  x         &=&  x
  &  x\land  x         &=&  x
  & (\prope{idempotent})
  \\
     x \lor  y         &=&  y \lor  x
  &  x\land  y         &=&  y \land  x
  & (\prope{commutative})
  \\
     x\lor ( y\lor  z)   &=& ( x\lor   y) \lor   z
  &  x\land  ( y\land   z) &=& ( x\land   y) \land   z
  & (\prope{associative})
  \\
     x \lor ( x\land  y) &=&  x
  &  x\land ( x \lor  y) &=&  x
  & (\prope{absorptive})% \prope{contractive}
  \\
    x\lor( y\land z) &=& ( x\lor y) \land ( x\lor z)
  & x\land( y\lor z) &=& ( x\land y) \lor  ( x\land z)
  & (\prope{distributive})
  \\
    x \lor  \lfalse     &=& x
  & x \land \ltrue      &=& x
  & (\prope{identity})
  \\
    x \lor  \ltrue     &=& \ltrue
  & x \land \lfalse    &=& \lfalse
  & (\prope{bounded})
  \\
    x \lor x'        &=& \ltrue
  & x \land x'       &=& \lfalse
  & (\prope{complemented})\footnotemark
  \\
    \brp{x'}'         &=& x
  &                   &&
  %& (\prope{uniquely complemented})
  & (\prope{uniquely comp.})
  \\
    (x\lor y)'       &=& x' \land y'
  & (x\land y)'       &=& x' \lor y'
  & (\prope{de Morgan's laws})
  \\ \hline
    \mc{3}{H|}{\text{property with emphasizing $\lor$}}
  & \mc{3}{H||}{\text{dual property emphasizing $\land$}}
  & \mc{1}{H}{\text{property name}}
\end{array}$}
\footnotetext{
  \begin{tabular}[t]{l MNM l}
    The property & x \lor  x' &=& \ltrue  & is also called the \prope{law of the excluded middle}.\\
    The property & x \land x' &=& \lfalse & is also called     \prope{non-contradiction} or \prope{explosion}.
  \end{tabular}
  \\References: \begin{tabular}[t]{l}
    \citerp{renedo2003}{71}\\
    \citerppg{restall2004}{73}{75}{0199265178}\\
    \citerpp{restall2001}{1}{3}
  \end{tabular}
  }
\end{theorem}
\begin{proof}
This follows directly from the fact that the \structe{classical 2-valued logic} \xref{def:2vlogic}
is a \structe{Boolean algebra} \xref{def:boolean} and from \prefpp{thm:boo_prop}.
\end{proof}

%---------------------------------------
\begin{definition}[additional logic operations]
\label{def:log_rel}
\label{def:lxor}
\footnote{
  \citerpgc{givant2009}{32}{0387402934}{disjunction, conjunction, negation},
  \citerpgc{shiva1998}{83}{0824700821}{inhibit, transfer},
  \citerppgc{whitesitt1995}{68}{69}{0486684830}{\hie{Sheffer stroke functions} $\bnor=\uparrow$, $\bnand=\downarrow$},
  \citerppgc{quine1979}{45}{48}{0674554515}{joint denial $\downarrow$, alternate denial $|$},
  \citePpc{bernstein1934}{876}{implication $\supset$}
  %\url{http://www.swif.uniba.it/lei/foldop/foldoc.cgi?Sheffer+stroke},
  %\url{http://www.swif.uniba.it/lei/foldop/foldoc.cgi?dagger+function}
  }
%---------------------------------------
Let $\logsysd$ be a propositional logic system.
Let $x'\eqd \lnot x$ and $y'\eqd \lnot y$.%, and juxtaposition represent $\land$.
The following table defines additional operations on $\setft$ in terms of
$\lor$, $\land$, and $\lnot$.
\\\begin{center}
\begin{tabular}{|l|N|MNM>{\scriptstyle}M|}
  \hline
  \mc{1}{|R|}{name}          & \mc{1}{R|}{symbol} & \mc{4}{R|}{definition}
  \\\hline                   
  \opd{joint denial}         & \hxs{\lnor}     & x \lnor    y &\eqd& x'\land y'                           & \forall x,y\in\setft \\ % 0001
  \opd{inhibit x}            & \hxs{\linx}     & x \linx    y &\eqd& x'\land y                            & \forall x,y\in\setft \\ % 0010
  \opd{inhibit y}            & \hxs{\liny}     & x \liny    y &\eqd& x \land y'                           & \forall x,y\in\setft \\ % 0100
  \opd{complete disjunction} & \hxs{\lxor}     & x \lxor    y &\eqd& \brp{x'\land y} \lor \brp{x \land y'}& \forall x,y\in\setft \\ % 0110
  \opd{alternative denial}   & \hxs{\lnand}    & x \lnand   y &\eqd& x'\lor y'                            & \forall x,y\in\setft \\ % 0111
 %\opd{equivalence}          & \hxs{\lequiv}   & x \lequiv  y &\eqd& \brp{x\land y} \lor \brp{x'\land y'} & \forall x,y\in\setft \\ % 1001
 %\opd{transfer y}           & \hxs{\ltrany}   & x \ltrany  y &\eqd& y                                    & \forall x,y\in\setft \\ % 1010
 %\opd{implication}          & \hxs{\limpl}    & x \limpl   y &\eqd& x' \lor y                            & \forall x,y\in\setft \\ % 1011
 %\opd{transfer x}           & \hxs{\ltranx}   & x \ltranx  y &\eqd& x                                    & \forall x,y\in\setft \\ % 1100
 %\opd{implied by}           & \hxs{\limplby}  & x \limplby y &\eqd& x \lor y'                            & \forall x,y\in\setft \\ % 1101
  \hline
\end{tabular}
\end{center}
\end{definition}

There are a total of $2^4=16$ possible binary operations on the set of relations
$\setft\cprod\setft\limp\setft$.
The following table summarizes these 16 operations.%
\footnote{
  \citerpg{shiva1998}{83}{0824700821}
  }
\label{tbl:log_op}
\begin{longtable}{|l|N |NNNN| MNM >{\scriptstyle}M|}
   \hline
   \mc{10}{|B|}{logic operations}
   \\\hline
   \mc{2}{B|}{ } & \mc{4}{B|}{$\opair{x}{y}=$} & \mc{4}{B|}{ }
   \\
   \mc{2}{|B|}{name and symbol}         &
  %\mc{1}{B|}{symbol}                   &
   \mc{1}{@{\,}B@{\,}|}{$\ltrue\ltrue$} &
   \mc{1}{@{\,}B@{\,}|}{$\ltrue\lfals$} &
   \mc{1}{@{\,}B@{\,}|}{$\lfals\ltrue$} &
   \mc{1}{@{\,}B@{\,}|}{$\lfals\lfals$} &
   \mc{4}{B|}{operation in terms of $\lor$, $\land$, and $\lnot$}
   \\\hline
   \opd{zero}                 & \hxsd{\lzero}  & \lfals & \lfals & \lfals & \lfals &   \lzero     &=& x  \land x'                          & \forall x  \in\setft \\ % 0000,  0 
   \opd{joint denial}         & \hxsd{\lnor}   & \lfals & \lfals & \lfals & \ltrue & x \lnor    y &=& x' \land y'                          & \forall x,y\in\setft \\ % 0001,  1
   \opd{inhibit $x$}          & \hxsd{\linx}   & \lfals & \lfals & \ltrue & \lfals & x \linx    y &=& x' \land y                           & \forall x,y\in\setft \\ % 0010,  2
   \opd{complement $x$}       & \hxsd{\lnotx}  & \lfals & \lfals & \ltrue & \ltrue & x \lnotx   y &=& x'                                   & \forall x,y\in\setft \\ % 0011,  3
   \opd{inhibit $y$}          & \hxsd{\liny}   & \lfals & \ltrue & \lfals & \lfals & x \liny    y &=& x  \land y'                          & \forall x,y\in\setft \\ % 0100,  4
   \opd{complement $y$}       & \hxsd{\lnoty}  & \lfals & \ltrue & \lfals & \ltrue & x \lnoty   y &=& y'                                   & \forall x,y\in\setft \\ % 0101,  5
   \opd{complete disjunction} & \hxsd{\lxor}   & \lfals & \ltrue & \ltrue & \lfals & x \lxor    y &=& \brp{x'\land y} \lor \brp{x\land y'} & \forall x,y\in\setft \\ % 0110,  6
   \opd{alternative denial}   & \hxsd{\lnand}  & \lfals & \ltrue & \ltrue & \ltrue & x \lnand   y &=& x' \lor y'                           & \forall x,y\in\setft \\ % 0111,  7
   \opd{conjunction}          & \hxsd{\land}   & \ltrue & \lfals & \lfals & \lfals & x \land    y &=& x  \land y                           & \forall x,y\in\setft \\ % 1000,  8
   \opd{equivalence}          & \hxsd{\lequiv} & \ltrue & \lfals & \lfals & \ltrue & x \lequiv  y &=& \brp{x\land y} \lor \brp{x'\land y'} & \forall x,y\in\setft \\ % 1001,  9
   \opd{transfer y}           & \hxsd{\ltrany} & \ltrue & \lfals & \ltrue & \lfals & x \ltrany  y &=& y                                    & \forall x,y\in\setft \\ % 1010, 10
   \opd{implication}          & \hxsd{\limpl}  & \ltrue & \lfals & \ltrue & \ltrue & x \limpl   y &=& x' \lor y                            & \forall x,y\in\setft \\ % 1011, 11
   \opd{transfer x}           & \hxsd{\ltranx} & \ltrue & \ltrue & \lfals & \lfals & x \ltranx  y &=& x                                    & \forall x,y\in\setft \\ % 1100, 12
   \opd{implied by}           & \hxsd{\limplby}& \ltrue & \ltrue & \lfals & \ltrue & x \limplby y &=& x  \lor y'                           & \forall x,y\in\setft \\ % 1101, 13
   \opd{disjunction}          & \hxsd{\lor}    & \ltrue & \ltrue & \ltrue & \lfals & x \lor     y &=& x  \lor y                            & \forall x,y\in\setft \\ % 1110, 14
   \opd{identity}             & \hxsd{\lid}    & \ltrue & \ltrue & \ltrue & \ltrue &   \lid       &=& x  \lor x'                           & \forall x  \in\setft \\ % 1111, 15
   \hline
\end{longtable}



%=======================================
%\section{Generalization and application to other fields}
%=======================================
The 16 logic operations of propositional logic can all
be represented using the logic operations of \hie{disjunction} $\lor$,
\hie{conjunction} $\land$, and \hie{negation} $\lnot$.
Using these representations, all 16 operations can be generalized to
\hie{Boolean algebras} using the equivalent Boolean algebra/lattice operations of
\hie{join}, \hie{meet}, and \hie{complement}.\citepg{givant2009}{32}{0387402934}

In addition to Boolean algebras, the 16
operations can also have equivalent operations on \hie{algebra of sets} where the
logic operations essentially define the set operations as in
\begin{align*}
  \setA \setu \setB &= \set{x\in\setX}{\brp{x\in\setA} \lor \brp{x\in\setB}}  \\
  \setA \seti \setB &= \set{x\in\setX}{\brp{x\in\setA} \land \brp{x\in\setB}} \\
  \setA \setd \setB &= \set{x\in\setX}{\brp{x\in\setA} \linx \brp{x\in\setB}} \\
  \setA \sets \setB &= \set{x\in\setX}{\brp{x\in\setA} \lxor \brp{x\in\setB}} \\
  \cmpA             &= \set{x\in\setX}{\lnot\brp{x\in\setA}}
\end{align*}
Computer science also makes use of some of the 16 logic operations,
where \hie{disjunction} becomes \hie{OR}, and \hie{conjunction} becomes \hie{AND}.
So, there are four fields (Boolean algebra, logic, set theory,
computer science) that all use essentially the same operations, but
sometimes call them by different names.
The following table attempts to identify to these terms across the four fields:%
\footnote{\url{http://groups.google.com/group/sci.math/browse_thread/thread/c1e9a7beb9a82311}}
\\
\begin{footnotesize}\label{ba_log_set_cs}%
\begin{longtable}{|M| Nl| Nl| Nl| Nl|}
  \hline
  \mc{9}{|B|}{terminology}
  \\\hline
  \mc{1}{|B}{} & \mc{2}{|B}{Boolean algebra} & \mc{2}{|B}{logic} & \mc{2}{|B}{algebra of sets} & \mc{2}{|B|}{computer science}
  \\\hline
  0000 & \hxsd{\bzero  } & \opd{bottom}                   & \hxs{\lzero  } &  \ope{false}                & \hxs{\szero  } & \opd{empty set}            & \symxd{\cszero  } & \opd{zero}          \\
  0001 & \hxsd{\bnor   } & \opd{rejection}                & \hxs{\lnor   } &  \ope{joint denial}         & \hxs{\snor   } & \opd{rejection}            & \symxd{\csnor   } & \opd{nor}           \\
  0010 & \hxsd{\binx   } & \opd{inhibit $x$}              & \hxs{\linx   } &  \ope{inhibit $x$}          & \hxs{\sinx   } & \opd{inhibit $x$}          & \symxd{\csinx   } & \opd{inhibit $x$}   \\
  0011 & \hxsd{\bnotx  } & \opd{complement $x$}           & \hxs{\lnotx  } &  \ope{negation $x$}         & \hxs{\snotx  } & \opd{complement $x$}       & \symxd{\csnotx  } & \opd{not $x$}       \\
  0100 & \hxsd{\biny   } & \opd{exception}                & \hxs{\liny   } &  \ope{inhibit $y$}          & \hxs{\siny   } & \opd{difference}           & \symxd{\csiny   } & \opd{difference}    \\
  0101 & \hxsd{\bnoty  } & \opd{complement $y$}           & \hxs{\lnoty  } &  \ope{negation $y$}         & \hxs{\snoty  } & \opd{complement $y$}       & \symxd{\csnoty  } & \opd{not $y$}       \\
  0110 & \hxsd{\bxor   } & \opd{Boolean addition}         & \hxs{\lxor   } &  \ope{complete disjunction} & \hxs{\sxor   } & \opd{symmetric difference} & \symxd{\csxor   } & \opd{exclusive-or}  \\
  0111 & \hxsd{\bnand  } & \opd{Sheffer stroke}           & \hxs{\lnand  } &  \ope{alternate denial}     & \hxs{\snand  } & \opd{Sheffer stroke}       & \symxd{\csnand  } & \opd{nand}          \\
  1000 & \hxsd{\band   } & \opd{meet}                     & \hxs{\land   } &  \ope{conjuction}           & \hxs{\sand   } & \opd{intersection}         & \symxd{\csand   } & \opd{and}           \\
  1001 & \hxsd{\bequiv } & \opd{biconditional}            & \hxs{\lequiv } &  \ope{equivalence}          & \hxs{\sequiv } & \opd{equivalence}          & \symxd{\csequiv } & \opd{equivalence}   \\
  1010 & \hxsd{\btrany } & \opd{projection $y$}           & \hxs{\ltrany } &  \ope{transfer $y$}         & \hxs{\strany } & \opd{projection $y$}       & \symxd{\cstrany } & \opd{projection $y$}\\
  1011 & \hxsd{\bimpl  } & \opd{implication}              & \hxs{\limpl  } &  \ope{implication}          & \hxs{\simpl  } & \opd{implication}          & \symxd{\csimpl  } & \opd{implication}   \\
  1100 & \hxsd{\btranx } & \opd{projection $x$}           & \hxs{\ltranx } &  \ope{transfer $x$}         & \hxs{\stranx } & \opd{projection $x$}       & \symxd{\cstranx } & \opd{projection $x$}\\
  1101 & \hxsd{\bimplby} & \opd{adjunction}               & \hxs{\limplby} &  \ope{implied by}           & \hxs{\sadj}    & \opd{adjunction}           & \symxd{\csimplby} & \opd{adjunction}    \\
  1110 & \hxsd{\bor    } & \opd{join}                     & \hxs{\lor    } &  \ope{disjunction}          & \hxs{\sor    } & \opd{union}                & \symxd{\csor    } & \opd{or}            \\
  1111 & \hxsd{\bid    } & \opd{top}                      & \hxs{\lid    } &  \ope{true}                 & \hxs{\sid    } & \opd{universal set}        & \symxd{\csid    } & \opd{one}           \\
  \hline
\end{longtable}
\end{footnotesize}


\qboxnpq
  { \href{http://en.wikipedia.org/wiki/Bertrand_Russell}{Bertrand Russell}
    \href{http://www-history.mcs.st-andrews.ac.uk/Timelines/TimelineF.html}{(1872--1970)},
    \href{http://www-history.mcs.st-andrews.ac.uk/BirthplaceMaps/Places/UK.html}{British mathematician},
    \index{Russull, Bertrand}
    \index{quotes!Russull, Bertrand}
    \footnotemark
  }
  {../common/people/russell1907_wkp_pdomain.jpg} % http://en.wikipedia.org/wiki/File:Russell1907-2.jpg
  {I spent September in extending his [Peano's] methods to the 
  logic of relations.\ldots
  %It seems to me in retrospect that, through that 
  %month, every day was warm and sunny. 
  %The Whiteheads stayed with us at Fernhurst, 
  %and I explained my new ideas to Mm. 
  %Every evening the discussion ended with some difficulty, 
  %and every morning I found that the difficulty of the previous 
  %evening had solved itself while I slept. 
  The time was one of intellectual intoxication. 
  My sensations resembled those one has after climbing a mountain in a mist, 
  when, on reaching the summit, 
  the mist suddenly clears, and the country becomes visible 
  for forty miles in every direction.\ldots
  %For years I had been endeavouring to analyse the fundamental notions of mathematics, 
  %such as order and cardinal numbers. 
  Suddenly, in the space of a few weeks, 
  I discovered what appeared to be definitive answers to the problems 
  which had baffled me for years. 
  And in the course of discovering these answers, I was introducing a 
  new mathematical technique, by which regions formerly abandoned 
  to the vaguenesses of philosophers were conquered for 
  the precision of exact formulae. Intellectually, the month of 
  September 1900 was the highest point of my life. I went about 
  saying to myself that now at last I had done something worth 
  doing, and I had the feeling that I must be careful not to be run 
  over in the street before I had written it down.}
\citetblt{
  quote: & \citerpp{russell1951}{217}{218}\\
  image: & \url{http://en.wikipedia.org/wiki/File:Russell1907-2.jpg}, public domain
  }

