%============================================================================
% Wavelet Structure and Design
% Archive file
% Daniel J. Greenhoe
%============================================================================



%=======================================
\section{First section}
%=======================================


Therefore, for the projection of any function $\ff(x)$ onto the subspace
$\spV_n$ (the approximation of $\ff(x)$ at scale $n$) is
\begin{align*}
  \mcomr{\opV_n \ff(x)}{projection of $\ff(x)$ onto scaling subspace $\spV_n$}
    &= \mcom{\sum_m \inprod{\ff(x)}{\fphi(2^nx-m)}\;\fphi(2^nx-m)}{in terms of scaling basis functions}.
  \\
  \mcomr{\opW_n \ff(x)}{projection of $\ff(x)$ onto wavelet subspace $\spW_n$}
    &= \mcom{\sum_m \inprod{\ff(x)}{\fpsi(2^nx-m)}\;\fpsi(2^nx-m)}{in terms of wavelet basis functions}.
\end{align*}

Furthermore, by expanding $\opV_n$ into a coarser scaling operator $\spV_{n-j}$
and $j$ wavelet operators, we can project a function $\ff(x)$ onto the scaling
subspace $\spV_n$ as
\begin{align*}
  \opV_n \ff(x)
    &= \mcom{\sum_m \inprod{\ff(x)}{\fphi(2^{n-j}x-m)}\;\fphi(2^{n-j}x-m)}
            {scaling expansion in space $\spV_{n-j}$}
     + \mcom{\sum_{k=1}^j\sum_m \inprod{\ff(x)}{\fpsi(2^{n-k}x-m)}\;\fpsi(2^{n-k}x-m)}
            {wavelet expansion over spaces $\spW_{n-j},\ldots,\spW_{n-2},\spW_{n-1}$}.
\end{align*}


\begin{figure}[th]
\begin{tabular}{ccc}
  $ h_n $, $ g_n $ & $\fphi(t)$ & $\fpsi(t)$  \\
  \hline
\\
  \parbox[][5\tw/16][t]{5\tw/16}{
    rectangular:\\
    \begin{tabular}[b]{l|r|r}
      $n$ & $\sqrt{2} h_n $ & $\sqrt{2} g_n $ \\
      \hline
      0   & $1$  & $ 1$  \\
      1   & $1$  & $-1$
    \end{tabular}
    } &
  \parbox[][5\tw/16][t]{5\tw/16}{\includegraphics*[width=5\tw/16, clip=true]{d1sc_x3125.eps}}
  \parbox[][5\tw/16][t]{5\tw/16}{\includegraphics*[width=5\tw/16, clip=true]{d1wc_x3125.eps}}
\\
  \parbox[][5\tw/16][t]{5\tw/16}{
    tent: \\
    \begin{tabular}[b]{l|r|r}
      $n$ & $\sqrt{2} h_n $ & $\sqrt{2} g_n $ \\
      \hline
      0   & $\frac{1}{2}$  & $ \frac{1}{2}$  \\
      1   & $1$            & $-1$            \\
      2   & $\frac{1}{2}$  & $ \frac{1}{2}$
    \end{tabular}
    } &
  \parbox[][4cm][t]{5cm}{\epsfig{file=tents.eps, width=5\tw/16, clip=}} &
  \parbox[][4cm][t]{5cm}{\epsfig{file=tentw.eps, width=5\tw/16, clip=}}
\\
  \parbox[][4cm][t]{4cm}{
    B-spline: \\
    \begin{tabular}[b]{l|r|r}
      $n$ & $\sqrt{2} h_n $ & $\sqrt{2} g_n $ \\
      \hline
      0   & $\frac{1}{8}$  & $ \frac{1}{8}$  \\
      1   & $\frac{4}{8}$  & $-\frac{4}{8}$  \\
      2   & $\frac{6}{8}$  & $ \frac{6}{8}$  \\
      3   & $\frac{4}{8}$  & $-\frac{4}{8}$  \\
      4   & $\frac{1}{8}$  & $-\frac{1}{8}$
    \end{tabular}
    } &
  \parbox[][4cm][t]{5cm}{\epsfig{file=Bspls.eps, width=5\tw/16, clip=}} &
  \parbox[][4cm][t]{5cm}{\epsfig{file=Bsplw.eps, width=5\tw/16, clip=}}
\\
  \parbox[][4cm][t]{4cm}{
    Daubechies-4: \\
    \begin{tabular}[b]{l|r|r}
      $n$ & $\sqrt{2} h_n $ & $\sqrt{2} g_n $ \\
      \hline
      0   & $\frac{1+\sqrt{3}}{4}$  & $ \frac{1-\sqrt{3}}{4}$  \\
      1   & $\frac{3+\sqrt{3}}{4}$  & $-\frac{3-\sqrt{3}}{4}$  \\
      2   & $\frac{3-\sqrt{3}}{4}$  & $ \frac{3+\sqrt{3}}{4}$  \\
      3   & $\frac{1-\sqrt{3}}{4}$  & $-\frac{1+\sqrt{3}}{4}$  \\
    \end{tabular}
    } &
  \parbox[][4cm][t]{5cm}{\epsfig{file=D2sc_x3125.eps, width=5\tw/16, clip=}} &
  \parbox[][4cm][t]{5cm}{\epsfig{file=D2wc_x3125.eps, width=5\tw/16, clip=}}
\end{tabular}
\caption{
  Examples of scaling functions.
  \label{fig:scaling_ex}
  }
\end{figure}



Definition~\ref{def:hn} includes a mysterious ``$\sqrt{2}$" factor.
This factor is not really necessary.
Some authors do not include it (use ``$1$" instead);
other authors replace it with a factor of $2$.
However, one reason for using the $\sqrt{2}$ factor is that
with it and if $\seq{\fphi(t-n)}{n\in\Z}$ is orthonormal,
then the square of the norm of the scaling function is simply the
sum of the squared absolute values of the scaling coefficients.
This result is presented in the next proposition.
%--------------------------------------
\begin{proposition}
%--------------------------------------
Let $\fphi(t)$ be a scaling function with
scaling coefficients $\seq{ h_n }{n\in\Z}$. Then
\formbox{
  \mcom{\inprod{\fphi(t-n)}{\fphi(t-m)}=\kdelta_{n-m}}
       {$\seq{\fphi(t-n)}{n\in\Z}$ is orthnormal}
  \hs{4ex}\implies\hs{4ex}
  \norm{\fphi(t)}^2 = \sum_n | h_n |^2
  }
\end{proposition}
\begin{proof}
\begin{align*}
  \norm{\fphi(t)}^2
    &= \norm{\sqrt{2}\sum_n  h_n \fphi(2t-n)}^2
    && \text{by dilation equation---Theorem~\ref{thm:h->phi}}
  \\&= |\sqrt{2}|^2 \norm{\sum_n  h_n \fphi(2t-n)}^2
    && \text{by definition of $\norm{\cdot}$ (page~\pageref{def:norm})}
  \\&= 2 \sum_n \norm{ h_n \fphi(2t-n)}^2
    && \text{by Pythagorean Theorem (page~\pageref{thm:pythag})}
  \\&= 2 \sum_n | h_n |^2 \; \norm{\fphi(2t-n)}^2
    && \text{by definition of $\norm{\cdot}$ (page~\pageref{def:norm})}
  \\&= 2 \sum_n | h_n |^2 \; \int_t \fphi(2t-n) \fphi^\ast(2t-n) \dt
    && \text{by definition of $\norm{\cdot}$ in $\spLL$}
  \\&= 2 \sum_n | h_n |^2 \; \int_u \fphi(u-n) \fphi^\ast(u-n) \frac{1}{2}\du
    && \text{let } u=2t \implies t=\frac{1}{2}u
  \\&= \frac{2}{2} \sum_n | h_n |^2 \; \mcom{\inprod{\fphi(u-n)}{\fphi(u-n)}}{1}
    && \text{by definition of $\norm{\cdot}$ in $\spLL$}
  \\&= \sum_n | h_n |^2
    && \text{by left hypothesis}
\end{align*}
\end{proof}



%--------------------------------------
\begin{definition}
\label{def:ws}
\index{wavelet system}
%--------------------------------------
Let $\fh,\fg\in\spII$.
A {\bf wavelet system} is the
ordered pair $(\fh,\fg)$ that satisfies
\defbox{\begin{array}{ll@{\qquad}D}
  1. & \fphi(t) = \sqrt{2} \sum_n  h_n  \fphi(2t-n)
     & \text{($\fh$ generates $\fphi(t)$)}  \\
  2. & \fpsi(t) = \sqrt{2} \sum_n  g_n  \fphi(2t-n)
     & \text{($\fg$ generates $\fpsi(t)$)} \\
  3. & \seq{2^{j}\fphi(2^jt-n)}{n\in\Z} \spans \spV_j
     & \text{($\{\fphi(t-n)\}$ generates $\seq{\spV_j}{j\in\Z}$)} \\
  4. & \seq{2^{j}\fpsi(2^jt-n)}{n\in\Z} \spans \spW_j
     & \text{($\{\fpsi(t-n)\}$ generates $\seq{\spW_j}{j\in\Z}$)} \\
  5. & \inprod{\fphi(t-n)}{\fphi(t-m)}=\kdelta_{nm}
     & \text{($\seq{\fphi(t-n)}{n\in\Z}$ is orthonormal)} \\
  6. & \inprod{\fpsi(t-n)}{\fpsi(t-m)}=\kdelta_{nm}
     & \text{($\seq{\fpsi(t-n)}{n\in\Z}$ is orthonormal)} \\
  7. & \spV_0 \perp \spW_0
     & \text{($\spV_0$ is orthogonal to $\spW_0$)}
\end{array}}
and generates the 4-tuple
$\left(\seq{\spV_j}{j\in\Z}, \seq{\spW_j}{j\in\Z}, \fphi, \fpsi \right)$.
\addtocounter{footnote}{-1}
\footnotetext{
   {\em multiresolutional analysis}:
   Definition~\ref{def:mra} page~\pageref{def:mra}
   }
\stepcounter{footnote}
\footnotetext{
   {\em complimentary subspaces}:
   Definition~\ref{def:direct_sum} page~\pageref{def:direct_sum}
   }
\end{definition}

\begin{figure}[t] \color{figcolor}
\begin{center}
\begin{fsL}
\setlength{\unitlength}{0.10mm}
\begin{picture}(300,300)(-130,-130)
  %\graphpaper[10](0,0)(200,200)
  \thinlines
  \input{circ512.inc}
  \put(-130 ,   0){\line(1,0){260} }
  \put(   0 ,-130){\line(0,1){260} }
  \put( 140 ,   0){\makebox(0,0)[l]{$\Reb{z}$}}
  \put(   0 , 140){\makebox(0,0)[b]{$\Imb{z}$}}

  \put( 120 , 120){\makebox(0,0)[lb]{$z=e^{i\omega}$}}
  %\put(-100,    0){\circle{15}}
  \put( 115 , 115){\vector(-1,-1){43}}

  %\put(-150 , -50){\makebox(0,0)[tr]{$\Zh(-1)=\Fh(\pi)=0$}}
  %\put(-150 , -50){\vector( 1, 1){43}}

  \put( 150 , -50){\makebox(0,0)[tl]{$\Zh(1)=\Fh(0)=\sqrt{2}$}}
  \put( 150 , -50){\vector(-1, 1){50}}
\end{picture}
\end{fsL}
\end{center}
\caption{
   Scaling functions requirements at $\Zh(1)$
   \label{fig:h(z)=sqrt2}
   }
\end{figure}


%=======================================
\section{Approximation using wavelets}
%=======================================
%=======================================
\subsection {Wavelet function families}
%=======================================
The scaling function $\fphi(x)$ generates a family
of related scaling functions $\{\fphi_{k,n}(x)\}$.
Likewise the wavelet function $\fpsi(x)$ generates a family
of wavelet functions $\{\fpsi_{k,n}(x)\}$.
Each element in these sets has
has a {\em scale} parameter $k$ and a {\em translation} parameter $n$.
%Figure~\ref{fig:haar-jn} shows the Haar scaling function at multiple
%scales and translations.

%--------------------------------------
\begin{definition}
\label{def:phi_family}
\index{scaling family}
\label{def:psi_family}
\index{wavelet family}
%--------------------------------------
Let {\bf the set of scaling functions}
$\set{\fphi_{j,n}:\R\to\C}{j,n\in\Z}$ generated by $\fphi(x)$ and
{\bf the set of wavelet functions}
$\set{\fpsi_{j,n}:\R\to\C}{j,n\in\Z}$ generated by $\fpsi(x)$
be defined as
\defbox{\begin{array}{rcl}
   \fphi_{j,n}(x) &\eqd& \sqrt{2^j} \fphi(2^jx-n)  \\
   \fpsi_{j,n}(x) &\eqd& \sqrt{2^j} \fpsi(2^jx-n)
\end{array}}
where $j$ represents the {\bf scale}
and $n$ represents {\bf translation} or {\em shift} of each element.
\end{definition}



%=======================================
\subsection{Choices of function representation}
%=======================================
\setlength{\unitlength}{8mm}
\begin{figure}[h] \color{figcolor}
\begin{center}
\begin{picture}(16,6)
\put(0,5){\framebox(16,1){$\spV_j$}}

\put(0,4){\framebox(8,1){$\spV_{j-1}$}}
\put(8,4){\framebox(8,1){$\spW_{j-1}$}}

\put(0,3){\framebox(4,1){$\spV_{j-2}$}}
\put(4,3){\framebox(4,1){$\spW_{j-2}$}}
\put(8,3){\framebox(8,1){$\spW_{j-1}$}}

\put(0,2){\framebox(2,1){$\spV_{j-3}$}}
\put(2,2){\framebox(2,1){$\spW_{j-3}$}}
\put(4,2){\framebox(4,1){$\spW_{j-2}$}}
\put(8,2){\framebox(8,1){$\spW_{j-1}$}}

\put(0,1){\makebox(16,1){$\vdots$}}

\put(0,0){\framebox(1,1){$\spV_{0}$}}
\put(1,0){\framebox(1,1){$\spW_{0}$}}
\put(2,0){\framebox(2,1){$\spW_{1}$}}
\put(4,0){\makebox (4,1){$\cdots$}}
\put(8,0){\framebox(8,1){$\spW_{j-1}$}}
\end{picture}
\caption{
   Decomposition of $\spV_j$ into $1$ scaling space and $j$ wavelet subspaces
   \label{fig:Vj_V0_Wj}
   }
\end{center}
\end{figure}




\setlength{\unitlength}{8mm}
\begin{figure}[h] \color{figcolor}
\begin{center}
\begin{picture}(16,6)
\put(0,5){\framebox(16,1){$\spV_j$}}

\put(0,4){\framebox(8,1){$\spV_{j-1}$}}
\put(8,4){\framebox(8,1){$\spW_{j-1}$}}

\put(0,3){\framebox(4,1){$\spV_{j-2}$}}
\put(4,3){\framebox(4,1){$\spW_{j-2}$}}
\put(8,3){\framebox(8,1){$\spW_{j-1}$}}

\put(0,2){\framebox(2,1){$\spV_{j-3}$}}
\put(2,2){\framebox(2,1){$\spW_{j-3}$}}
\put(4,2){\framebox(4,1){$\spW_{j-2}$}}
\put(8,2){\framebox(8,1){$\spW_{j-1}$}}

\put(0,1){\makebox(16,1){$\vdots$}}

\put(0,0){\framebox(1,1){}}
\put(0.0625,0){\framebox(0.0625,1){}}
\put(0.125,0){\framebox(0.125,1){}}
\put(0.25,0){\framebox(0.25,1){}}
\put(0.5,0){\framebox(0.5,1){}}
\put(1,0){\framebox(1,1){$\cdots$}}
\put(2,0){\framebox(2,1){$\spW_{j-3}$}}
\put(4,0){\framebox(4,1){$\spW_{j-2}$}}
\put(8,0){\framebox(8,1){$\spW_{j-1}$}}
\end{picture}
\caption{
   Decomposition of $\spV_j$ into wavelet subspaces only
   \label{fig:Vj_Wj}
   }
\end{center}
\end{figure}




Wavelet analysis provides three forms for representing a function
$\ff(t)$ in the scaling space $\spV_j$: \\
\begin{tabular}{clll}
   1. & using one high resolution scaling function\footnotemark     & (Theorem~\ref{thm_fx_v}) \\
   2. & using one low  resolution scaling function and $j$ wavelets & (Theorem~\ref{thm_fx_vw}) \\
   3. & using an infinite number of wavelet functions               & (Theorem~\ref{thm_fx_w})
\end{tabular}
\footnotetext{
   The ultimate example of using a single high resolution function to represent
   a function is when $\fphi(x)=\delta(x)$ (Dirac delta function).
   In this case the scale is effectively $m=\infty$ and
      \[
         \sum_n \inprod{\ff(x)}{\fphi_{m,n}(x)} \fphi_{m,n}(x)
              = \int_x \inprod{\ff(y)}{\delta(y-x)} \delta(x) \dx
              = \int_x \ff(x) \delta(x)  \dx
              = \ff(x).
      \]
   Reference: \cite[page 37]{burrus}
}

When a function is decomposed using a scaling function and multiple wavelet functions,
the portion of $\ff(t)$ extracted by the scaling function can be interpretted as
the ``corse approximation" of $\ff(t)$, while
the portion extracted by the wavelet functions can be interpretted
as the ``detail" of $\ff(t)$.


The space $\spV_j$ can be made arbitrariliy large ($k$ can be made arbitrarily large)
to accomodate $\ff(t)$.
However, if the size of $\spV_j$ is limited such that $\ff(t)$ does not ``fit" in $\spV_j$,
then the wavelet analysis will not give an exact representatioin of $\ff(t)$
but rather will give an {\em approximation} of $\ff(t)$.



%--------------------------------------
\begin{proposition}
\label{prop:Vk}
%--------------------------------------
The scaling subspace $\spV_j$ is composed of the smaller space $\spV_0$
and $j$ wavelet subspaces $\spW_0, \spW_1, \ldots, \spW_{j-1}$ such that
(see Figure~\ref{fig:Vj_V0_Wj})
\formbox{
   \spV_j = \spV_0 \oplus \spW_0 \oplus \spW_1 \oplus \cdots \oplus \spW_{j-1}
%       = \spV_0 \oplus \bigoplus\limits_{m=0}^{k-1}{\spW_{m}}
  }
\end{proposition}
\begin{proof}
\begin{eqnarray*}
   \spV_j &=& \spV_{j-1} \oplus \spW_{j-1} \\
       &=& \spV_{j-2} \oplus \spW_{j-2} \oplus \spW_{j-1} \\
       &=& \spV_{j-3} \oplus \spW_{j-3} \oplus \spW_{j-2} \oplus \spW_{j-1} \\
       &\vdots&  \\
       &=& \spV_{0} \oplus \spW_{0} \oplus \spW_1 \oplus \cdots \oplus \spW_{j-1} \\
       &=& \spV_{0} \oplus \bigoplus\limits_{m=0}^{j-1}{\spW_{m}}
\end{eqnarray*}
\end{proof}

%--------------------------------------
\begin{proposition}
\label{prop:VkW}
%--------------------------------------
The scaling subspace $V_j$ can be decomposed into an infinite number of
wavelet subspaces $\spW_{j-1}, \spW_{j-2}, \spW_{j-3}, \ldots, \spW_{-\infty}$ such that
(see Figure~\ref{fig:Vj_Wj}) \citepp{burrus}{12}{16}
\formbox{
  \spV_j = \spW_{j-1} \oplus \spW_{j-2} \oplus \spW_{j-3} \oplus \cdots \oplus \spW_{-\infty}
  }
\end{proposition}
\begin{proof}
\begin{eqnarray*}
   \spV_j &=& \spV_{j-1} \oplus \spW_{j-1} \\
       &=& \spV_{j-2} \oplus \spW_{j-2} \oplus \spW_{j-1} \\
       &=& \spV_{j-3} \oplus \spW_{j-3} \oplus \spW_{j-2} \oplus \spW_{j-1} \\
       &=& \cdots \\
       &=& \bigoplus\limits_{m=-\infty}^{j-1}{\spW_m}
\end{eqnarray*}
\end{proof}

The next theorem demonstrates representing $\ff(t)$ with scaling functions only.
%--------------------------------------
\begin{theorem}
\label{thm_fx_v}
%--------------------------------------
Let $\left(\{\spV_j\},\{\spW_j\}, \fphi,\fpsi,\fh,\fg\right)$
be a wavelet system and
$\opP_j:\spLL\to\spV_j$ the $j$-scale projection operator.
\formbox{\begin{array}{>{\ds}l}
   \exists \seq{\alpha_n}{n\in\Z} \st
   \\ \\
   \opP_j\ff(t) = \mcoml{\sum_n \alpha_n \fphi_{j,n}(t)}{only scaling functions at scale $j$}
  \end{array}}
\end{theorem}
\begin{proof}
By Definition~\ref{def:mra} (MRA, page~\pageref{def:mra}),
$\fphi(t)$ is a Riesz basis for $\spV_0$.
Throught the MRA scaling property,
this can be extended to any other space $\spV_j$.
\end{proof}

The next theorem demonstrates
representing $\ff(t)$ with one scaling function and multiple wavelets.
%--------------------------------------
\begin{theorem}
\label{thm_fx_vw}
%--------------------------------------
Let $\left(\{\spV_j\},\{\spW_j\}, \fphi,\fpsi,\fh,\fg\right)$
be a wavelet system and
$\opP_j:\spLL\to\spV_j$ the $j$-scale projection operator.
\formbox{\begin{array}{>{\ds}l}
   \exists \seq{\alpha_n}{n\in\Z}\text{ and } \seq{\beta_{mn}}{m,n\in\Z} \st
   \\ \\
   \opP_j \ff(t) =
     \mcom {\sum_n \alpha_n \fphi_{0,n}(x)}{scaling at scale $0$} +
     \mcoml{\sum_{m=0}^{j-1} \sum_n \beta_{mn} \fpsi_{m,n}(t)}{wavelets at scales $0,1,\ldots,(j-1)$}
   \end{array}}
\end{theorem}
\begin{proof}
See Proposition~\ref{prop:Vk} (page~\pageref{prop:Vk});
\end{proof}

The next theorem demonstrates
representing \ff(t) with wavelets only (and no scaling function).
%--------------------------------------
\begin{theorem}
\label{thm_fx_w}
%--------------------------------------
Let $\left(\{\spV_j\},\{\spW_j\}, \fphi,\fpsi,\fh,\fg\right)$
be a wavelet system and
$\opP_j:\spLL\to\spV_j$ the $j$-scale projection operator.
\formbox{\begin{array}{>{\ds}l}
   \exists \seq{\alpha_{mn}}{m,n\in\Z} \st
   \\ \\
   \opP_j \ff(t)
     = \mcoml{\sum_{m=-\infty}^{j-1} \sum_n \alpha_{mn} \fpsi_{m,n}(t)}
             {no scaling, countably infinite number of wavelets}
  \end{array}}
\end{theorem}
\begin{proof}
See Proposition~\ref{prop:VkW} (page~\pageref{prop:VkW});
\end{proof}



%---------------------------------------
\begin{theorem}[Neumann Expansion Theorem]
\index{Neumann Expansion Theorem}
\index{theorems!Neumann Expansion Theorem}
\label{thm:op_net2}
\citep{michel}{415}
%---------------------------------------
Let $\opA:\spX\to\spX$ be an operator on a vector space $\spX$.
Let $\opA^0\eqd \opI$.
\formbox{\begin{array}{ll}
  \left.\begin{array}{lrclD}
    1. & \opA          &\in& \oppB(\spX,\spX) & ($\opA$ is bounded) \\
    2. & \normop{\opA} &<&   1         
  \end{array}\right\}
  \implies
  \left\{\begin{array}{lrc>{\ds}l}
    1. & (\opI-\opA)^{-1} &&\text{ exists} \\
    2. & \normop{(\opI-\opA)^{-1}} &\le& \frac{1}{1-\normop{\opA}} \\
    3. & (\opI-\opA)^{-1} &=& \sum_{n=0}^\infty \opA^n  \\
       & \mc{3}{c}{\text{ with uniform convergence}}
  \end{array}\right.
\end{array}}
\end{theorem}







%--------------------------------------
\begin{theorem}
\label{thm:wav_net}
%--------------------------------------
Let $\wavsys$ be a \hi{wavelet system}.
\formbox{
  \sum_n \abs{h_n} \ge 1
  }
\end{theorem}
\begin{proof}
\begin{align*}
  &&
  \vphi &= \sum_n h_n \opT^n \opD \vphi
  \\\implies&&
  \left(\opI - \sum_n h_n \opT^n \opD \right)\vphi &= \vzero
  \\\implies&&
  \left(\opI - \sum_n h_n \opT^n \opD \right)^{-1} & \text{must not exist}
  \\\implies&&
  \normop{\sum_n h_n \opT^n \opD} & \ge 1 
    && \text{by Neumann Expansion Theorem \prefpo{thm:op_net2}}
  \\\implies&&
  1
      &\le \normop{\sum_n h_n \opT^n \opD}
     &&    \text{by Neumann Expansion Theorem \prefpo{thm:op_net2}}
  \\&&&\le \sum_n  \normop{h_n \opT^n \opD}
     &&    \text{by generalized triangle inequality \prefpo{thm:norm_tri}}
  \\&&&=   \sum_n  \abs{h_n}\; \cancelto{1}{\normop{ \opT^n \opD}}
     &&    \text{by homogeneous property of norm \prefpo{def:norm}}
  \\&&&=   \sum_n  \abs{h_n}
     &&    \text{by \prefp{thm:unitary_prop}}
\end{align*}
\end{proof}




%--------------------------------------
\begin{example}
\label{ex:sw_gh_d3}
%--------------------------------------
The figure below illustrates scaling and wavelet coefficients and functions
for the \hie{Daubechies-$p3$} wavelet system.\\
  \parbox[b][35\tw/160][c]{5\tw/16}{\footnotesize\begin{tabular}[b]{l|r|r}
    $n$ & $h_n $ & $ g_n $ \\
    \hline
      0   & $ 0.332671$  & $ 0.035226$  \\
      1   & $ 0.806892$  & $ 0.085441$  \\
      2   & $ 0.459878$  & $-0.135011$  \\
      3   & $-0.135011$  & $-0.459878$  \\
      4   & $-0.085441$  & $ 0.806892$  \\
      5   & $ 0.035226$  & $-0.332671$  \\
  \end{tabular}}
  \includegraphics*[width=5\tw/16, clip=true]{../common/wavelets/d3sc_x3125.eps}
  \includegraphics*[width=5\tw/16, clip=true]{../common/wavelets/d3wc_x3125.eps}
\end{example}

%--------------------------------------
\begin{example}
\label{ex:sw_gh_d4}
%--------------------------------------
The figure below illustrates scaling and wavelet coefficients and functions
for the \hie{Daubechies-$p4$} wavelet system.\\
  \parbox[b][35\tw/160][c]{5\tw/16}{\footnotesize\begin{tabular}[b]{l|r|r}
    $n$ & $h_n $ & $ g_n $ \\
    \hline
      0   & $ 0.230378$  & $-0.010597$  \\
      1   & $ 0.714847$  & $-0.032883$  \\
      2   & $ 0.630881$  & $ 0.030841$  \\
      3   & $-0.027984$  & $ 0.187035$  \\
      4   & $-0.187035$  & $-0.027984$  \\
      5   & $ 0.030841$  & $-0.630881$  \\
      6   & $ 0.032883$  & $ 0.714847$  \\
      7   & $-0.010597$  & $-0.230378$  \\
  \end{tabular}}
  \includegraphics*[width=5\tw/16, clip=true]{../common/wavelets/d4sc_x3125.eps}
  \includegraphics*[width=5\tw/16, clip=true]{../common/wavelets/d4wc_x3125.eps}
\end{example}

%--------------------------------------
\begin{example}
\citep{strang89}{616}
\label{ex:sw_gh_tent}
%--------------------------------------
The figure below illustrates scaling and wavelet coefficients and functions
for the \hie{tent function}.\\
  \parbox[b][35\tw/160][c]{5\tw/16}{\begin{tabular}[b]{l|r|r}
    $n$ & $\sqrt{2} h_n $ & $\sqrt{2} g_n $ \\
    \hline
      0   & $\frac{1}{2}$  & $ \frac{1}{2}$  \\
      1   & $1$            & $-1$            \\
      2   & $\frac{1}{2}$  & $ \frac{1}{2}$
  \end{tabular}}
  \includegraphics*[width=5\tw/16, clip=true]{../common/wavelets/tents.eps}
  \includegraphics*[width=5\tw/16, clip=true]{../common/wavelets/tentw.eps}
\end{example}

%--------------------------------------
\begin{example}
\citep{strang89}{616}
\label{ex:sw_gh_bspline}
%--------------------------------------
The figure below illustrates scaling and wavelet coefficients and functions
for a \hie{B-sline}.\\
  \parbox[b][35\tw/160][c]{5\tw/16}{\begin{tabular}[b]{l|r|r}
    $n$ & $\sqrt{2} h_n $ & $\sqrt{2} g_n $ \\
    \hline
      0   & $\frac{1}{8}$  & $ \frac{1}{8}$  \\
      1   & $\frac{4}{8}$  & $-\frac{4}{8}$  \\
      2   & $\frac{6}{8}$  & $ \frac{6}{8}$  \\
      3   & $\frac{4}{8}$  & $-\frac{4}{8}$  \\
      4   & $\frac{1}{8}$  & $-\frac{1}{8}$
  \end{tabular}}
  \includegraphics*[width=5\tw/16, clip=true]{../common/wavelets/Bspls.eps}
  \includegraphics*[width=5\tw/16, clip=true]{../common/wavelets/Bsplw.eps}
\end{example}



\parbox[c][][c]{\textwidth/2}{
%\begin{figure}[ht]
\color{figcolor}
\begin{center}
\begin{fsL}
\setlength{\unitlength}{0.10mm}
\begin{picture}(500,600)(-400,-600)
  %\graphpaper[10](0,0)(600,200)
  \thicklines
  \put(   0,   0){\line(-1,-1){100} }  \put(   0,   0){\line( 1,-1){100} }
  \put(-100,-100){\line(-1,-1){100} }  \put(-100,-100){\line( 1,-1){100} }
  \put(-200,-200){\line(-1,-1){100} }  \put(-200,-200){\line( 1,-1){100} }
  \put(-400,-400){\line(-1,-1){100} }  \put(-400,-400){\line( 1,-1){100} }

  \put( 100,-100){\line(-1,-5){100} }
  \put(   0,-200){\line( 0,-1){400} }
  \put(-100,-300){\line( 1,-3){100} }
  \put(-200,-400){\line( 1,-1){200} }
  \put(-300,-500){\line( 3,-1){300} }
  \put(-500,-500){\line( 5,-1){500} }

  \put(   0,   0){\circle*{15}}
  \put(-100,-100){\circle*{15}}  \put( 100,-100){\circle*{15}}
  \put(-200,-200){\circle*{15}}  \put(   0,-200){\circle*{15}}
  \put(-300,-300){\circle*{15}}  \put(-100,-300){\circle*{15}}
  \put(-400,-400){\circle*{15}}  \put(-200,-400){\circle*{15}}
  \put(-500,-500){\circle*{15}}  \put(-300,-500){\circle*{15}}
  \put(   0,-600){\circle*{15}}

  \put(   0,  10){\makebox(0,0)[b] { $\opV_n$}}
  \put(-110,-100){\makebox(0,0)[r] { $\opV_{n-1}$}}
  \put( 110,-100){\makebox(0,0)[l] { $\opW_{n-1}$}}
  \put(-210,-200){\makebox(0,0)[r] { $\opV_{n-2}$}}
  \put(  10,-200){\makebox(0,0)[l] { $\opW_{n-2}$}}
  \put(-310,-300){\makebox(0,0)[r] { $\opV_{n-3}$}}
  \put( -90,-300){\makebox(0,0)[l] { $\opW_{n-3}$}}
  \put(-410,-400){\makebox(0,0)[r] { $\opV_{2}$}}
  \put(-190,-400){\makebox(0,0)[l] { $\opW_{2}$}}
  \put(-510,-500){\makebox(0,0)[r] { $\opV_{1}$}}
  \put(-290,-500){\makebox(0,0)[l] { $\opW_{1}$}}
  \put(   0,-610){\makebox(0,0)[t] { $\setn{0}$}}
  \put(-350,-350){\makebox(0,0)[c] { $\vdots$}}
  \put(-250,-350){\makebox(0,0)[c] { $\vdots$}}
\end{picture}
\end{fsL}
\end{center}
}
\parbox[c][][c]{\textwidth/2}{
%---------------------------------------
\begin{example}[Multiresolutional Analysis]
\label{ex:lat_mra}
%---------------------------------------
A {\em \hi{multiresolutional analysis}} (\hi{MRA}) is a technique for
partitioning a Hilbert space into {\em scaling spaces} $\seqn{\spV_n}{n\in\Z}$
and {\em\hi{wavelet subspaces}} $\seqn{\spW_n}{n\in\Z}$ such that
$\spV_{n+1} = \spV_n \oplus \spW_n$, where $\oplus$ is the
{\em\hi{direct product}}.
\exbox{\begin{tabular}{l@{\quad}lll}
  \mc{4}{l}{The MRA  subspaces form a lattice}\\
  \mc{4}{l}{$(\set{\spV_k,\spW_k}{k=1,2,\ldots,n},\subseteq,\oplus,\seti)$ that is}\\
   & 1. & non-distributive  & (lattice contains $N5$)\\
   & 2. & non-modular       & (lattice contains $N5$)  \\
   & 3. & non-complemented
\end{tabular}}
The MRA lattice is illustrated in the figure to the left.
Alternatively, this could be the lattice of \hi{projection operator}s that
project onto the subspaces.
\end{example}
}


%=======================================
\section{Approximation using wavelets}
%=======================================
Suppose a vector (e.g. a function) is a member of some large space $\spX$.
In practice, it is normally not possible to represent that vector exactly;
rather, some approximation of the vector is necessary.
Of course many approximations are possible depending on the amount of ``resolution"
or scale selected.
\prefpp{thm:wav_approx} shows that in a wavelet system, 
the approximation of a vector $\vx$ at some scale $n$ can be performed in three ways:
\begin{dingautolist}{"C0}
  \item Projecting $\vx$ onto one high resolution scaling subspace $\spV_n$
        (and {\em no} wavelet subspaces).
  \item Projecting $\vx$ onto one low resolution scaling subspace $\spV_1$ 
        and $n-1$ wavelet subspaces $\spW_1,\,\spW_2,\,\dots,\,\spW_{n-1}$.
  \item Projecting $\vx$ onto one an infinite number of wavelet subspaces
        $\spW_{n-1},\spW_{n-2},\spW_{n-3},\ldots$
        (and {\em no} scaling subspaces).
\end{dingautolist}

%---------------------------------------
\begin{theorem}
\label{thm:wav_approx}
%---------------------------------------
Let $\spX$ be a vector space with 
wavelet subspace lattice
$\wavlatsubs$
and wavelet operator lattice
$(\setn{\seq{\opP_n}{n\in\Z},\seq{\opQ_n}{n\in\Z}},\,\sqsubseteq,\,+,\,\cdot)$.
\formbox{\begin{array}{rcl@{\qquad}>{\scriptsize}p{7\tw/16}}
  \spV_n
    &=& \opP_n \spX
    &   (one high resolution scaling subspace) 
  \\&=& (\opP_1 + \opQ_1 + \opQ_2 + \cdots + \opQ_{n-1})\spX
    &   (one low resolution scaling subsspace and $n-1$ wavelet subspaces)
  \\&=& (\opQ_{-\infty} + \cdots + \opQ_{n-2} + \opQ_{n-1})\spX
    &   (infinite number of wavelet subspaces)
  \end{array}}
\end{theorem}
\begin{proof}
  \begin{align*}
    \spV_n
      &= \spV_{n-1} + \spW_{n-1}
      && \text{by \prefp{def:seqWn}}
    \\&= \spV_{n-2} + \spW_{n-2} + \spW_{n-1}
      && \text{by \prefp{def:seqWn}}
    \\&= \spV_{n-3} + \spW_{n-3} + \spW_{n-2} + \spW_{n-1}
      && \text{by \prefp{def:seqWn}}
    \\&\quad\vdots
    \\&= \spV_{1} + \spW_{1} + \spW_{2} +\cdots+ \spW_{n-1}
      && \text{by \prefp{def:seqWn}}
    \\&\quad\vdots
    \\&= \spW_{\infty}+\cdots + \spW_{1} + \spW_{2} +\cdots+ \spW_{n-1}
      && \text{by \prefp{def:seqWn}}
  \end{align*}
\end{proof}


 
%============================================================================
% Daniel J. Greenhoe
% LaTeX File
%============================================================================

%======================================
\chapter{The Structure of Wavelets}
%======================================

%=======================================
%\section {Vector space architecture}
%=======================================
%=======================================
\section{Wavelet transversal operators}
\label{sec:wav_lat_op}
%=======================================
All Hilbert spaces can be constructed as a lattice of subspaces.
Transforms in Hilbert spaces are a collection of projection operators that 
project a given vector onto those subspaces.
Thus a transform is essentially defined by the subspace lattice structure.

In such a subspace lattice structure, we define two \hie{transversal operators}
that move vectors within the lattice.
These transversal operators are\\
  \begin{tabular}{ll<{:}ll}
    \circOne & \hie{translation operator} & $\opT$ &---moves a vector within a subspace. \\
    \circTwo & \hie{dilation operator}    & $\opD$ &---moves a vector ``up" one subspace.
  \end{tabular}

\begin{minipage}[t]{8\tw/16}
%\begin{figure}[t]%
%  \color{figcolor}%
  \begin{center}%
  \begin{fsL}%
  \setlength{\unitlength}{\textwidth/800}%
  \begin{picture}(500,200)(-250,-50)%
    %{\color{graphpaper}\graphpaper[10](-250,-50)(500,200)}%
    {\color{axis}%
      \put(-220, 0){\line(1, 0){440} }%
      \put(0, 0){\line(0, 1){120} }%
      \multiput(-200,-10)(50,0){9}{\line(0, 1){20}}%
      \put( 230,  0){\makebox(0,0)[l]{$t$}}%
      \put( 200,-20){\makebox(0,0)[t]{$2$}}%
      \put( 100,-20){\makebox(0,0)[t]{$1$}}%
      \put(   0,-20){\makebox(0,0)[t]{$0$}}%
      \put(-100,-20){\makebox(0,0)[t]{$-1$}}%
      \put(-200,-20){\makebox(0,0)[t]{$-2$}}%
    }%
    {\color{figcolor}%
      \put(   0,110){\makebox(0,0)[b]{$\fx(t)$}}%
      \put(-100, 0){\line( 1, 1){100} }%
      \put( 100, 0){\line(-1, 1){100} }%
    }%
    {\color{red}%
      \put( 150,110){\makebox(0,0)[b]{$\opT \fx(t)$}}%
      \put( 150,100){\vector( 0,-1){50} }%
      \put(   0, 0){\line( 1, 1){100} }%
      \put( 200, 0){\line(-1, 1){100} }%
    }%
    {\color{green}%
      \put(-150,110){\makebox(0,0)[b]{$\opTi\fx(t)$}}%
      \put(-150,100){\vector( 0,-1){50} }%
      \put(-200, 0){\line( 1, 1){100} }%
      \put(   0, 0){\line(-1, 1){100} }%
    }%
  \end{picture}%
  \end{fsL}%
  \end{center}%
  \begin{center}%
  \begin{fsL}%
  \setlength{\unitlength}{\textwidth/800}%
  \begin{picture}(500,200)(-250,-50)%
    %{\color{graphpaper}\graphpaper[10](-250,-50)(500,200)}%
    {\color{axis}%
      \put(-220, 0){\line(1, 0){440} }%
      \put(0, 0){\line(0, 1){150} }%
      \multiput(-200,-10)(50,0){9}{\line(0, 1){20}}%
      \put( 230,  0){\makebox(0,0)[l]{$t$}}%
      \put( 200,-20){\makebox(0,0)[t]{$2$}}%
      \put( 100,-20){\makebox(0,0)[t]{$1$}}%
      \put(   0,-20){\makebox(0,0)[t]{$0$}}%
      \put(-100,-20){\makebox(0,0)[t]{$-1$}}%
      \put(-200,-20){\makebox(0,0)[t]{$-2$}}%
    }%
    {\color{figcolor}%
      \put( 130,80){\makebox(0,0)[l]{$\fx(t)$}}%
      \put( 130,70){\vector( -1,-1){50} }%
      \put(-100, 0){\line( 1, 1){100} }%
      \put( 100, 0){\line(-1, 1){100} }%
    }%
    {\color{red}%
      \put( 110,120){\makebox(0,0)[l]{$\opD \fx(t)$}}%
      \put( 100,120){\vector(-1,0){90} }%
      \put( -50, 0){\line( 1, 3){50} }%
      \put(  50, 0){\line(-1, 3){50} }%
    }%
    {\color{green}%
      \put(-100,110){\makebox(0,0)[b]{$\opDi\fx(t)$}}%
      \put(-100,100){\vector( 0,-1){65} }%
      \put(-200, 0){\line( 3, 1){200} }%
      \put( 200, 0){\line(-3, 1){200} }%
    }%
  \end{picture}%
  \end{fsL}%
  \end{center}%
%  \caption{
%    Translation $\opT$ and dilation operator $\opD$
%    \label{fig:wav_TD}
%    }
%\end{figure}
\end{minipage}
\begin{minipage}[c]{8\tw/16}
These two operators are illustrated in the figures to the left
and defined in \prefpp{def:wav_opD}.
Additionally, translation and dilation in the case of the \hie{Haar} wavelet transform
is illustrated in \prefpp{fig:haar-jn}.
\end{minipage}
\footnotetext{\hie{operator:} \prefpp{def:operator}}

\begin{figure}[t]
\setlength{\unitlength}{8mm}
\begin{center}
\begin{tabular}{cc}
   \includegraphics*[width=6\tw/16, height=6\tw/16, clip=true]{../common/wavelets/haar_sj.eps} &
   \includegraphics*[width=6\tw/16, height=6\tw/16, clip=true]{../common/wavelets/haar_sk.eps} \\
   \footnotesize varying dilation, constant translation   & 
   \footnotesize varying translation, constant dilation
\end{tabular}
\caption{
   \label{fig:haar-jn}
   Haar scaling function at varying dilations and translations.
   }
\end{center}
\end{figure}



%---------------------------------------
\begin{definition}
\label{def:wav_opD}
\label{def:wav_opT}
\index[xsym]{$\opD$|textbf}
\index[xsym]{$\opT$|textbf}
%---------------------------------------
\defbox{\begin{array}{lcll}
  \opT  \fx(t) &\eqd&         \fx(t-1)  & \text{where $\opT$  is called the \hid{translation operator}} \\
  \opD  \fx(t) &\eqd& \sqrt{2}\fx(2t)  & \text{where $\opD$  is called the \hid{dilation operator}} \\
\end{array}}
\end{definition}



\begin{minipage}[t]{8\tw/16}
%\begin{figure}[t]
\begin{fsL}
\[\begin{array}{*{5}{>{\ds}c}}
     && \fcolorbox{blue}{bg_blue}{\parbox[c]{3\tw/16}{\centering {operator}}} \index{operator}
  \\ & \Nearrow & \Uparrow & \Nwarrow
  \\ \fcolorbox{blue}{bg_blue}{\parbox[c]{3\tw/16}{\centering {normal} $\opNa\opN=\opN\opNa$}} \index{operator!normal}
     && \fcolorbox{blue}{bg_blue}{\parbox[c]{3\tw/16}{\centering {projection} $\opP^2=\opP$}}  \index{operator!projection}
     && \fcolorbox{blue}{bg_blue}{\parbox[c]{3\tw/16}{\centering {isometric} $\opAa\opA=\opI$}} \index{operator!isometric}
  \\ \Uparrow & \Nwarrow  & & \Nearrow
  \\ \fcolorbox{blue}{bg_blue}{\parbox[c]{3\tw/16}{\centering {self-adjoint} $\opA=\opAa$}}  \index{operator!self-adjoint}
     && \fcolorbox{blue}{bg_blue}{\parbox[c]{3\tw/16}{\centering {invertible} $\opAa=\opAi$}} \index{operator!invertible}
  \\ && \Uparrow
  \\ && \fcolorbox{red}{bg_red}{\parbox[c]{3\tw/16}{\centering {unitary} $\opU\opUa=\opUa\opU=\opI$}} \index{operator!unitary}
\end{array}\]
\end{fsL}
%\caption{
%   Relations between operator properties
%   \label{fig:op_prop}
%   }
%\end{figure}
\end{minipage}
\begin{minipage}[t]{8\tw/16}
  \paragraph{Properties of operators.}
  The translation operator $\opT$ and dilation operator $\opD$ are both
  \hie{unitary} (\prefp{thm:TD_unitary})
  but are \hie{not commutative} with respect to each other (\prefp{thm:TD_DT}).
  The unitary property is especially powerful, because this implies that the operators are also
  \hie{invertible}, \hie{normal}, and \hie{isometric}.\footnotemark
  These relationships are illustrated in the figure to the left.
  These properties imply many other properties;
  noteably the isometric property implies that the operators preserve the ``length" 
  of any vector (\prefp{cor:wav_TD_prop}).
\end{minipage}
\footnotetext{\hie{operator types}: \prefpp{def:op_types}}




%---------------------------------------
\begin{theorem}
\label{thm:wav_opTi}
\label{thm:wav_opDi}
\index[xsym]{$\opDi$|textbf}
\index[xsym]{$\opTi$|textbf}
%---------------------------------------
Let $\spX$ be a vector space, $\opT$ the translation operator on $\spX$,
$\opD$ the dilation operator on $\spX$.
Then the \hie{inverse} \footnote{\hie{inverse operator:} \prefp{def:op_inv}} $\opTi$ of $\opT$ 
and  the \hie{inverse} $\opDi$ of $\opD$
are
\formbox{
  \opTi \fx(t) \eqd \fx(t+1)  
  \qquad\text{and}\qquad
  \opDi \fx(t) \eqd \cwt \:\fx\left(\frac{1}{2}t\right)  
  }
\end{theorem}
\begin{proof}
\begin{align*}
  \intertext{1a. Proof that $\opTi\opT = \opI$:}
  \opTi\opT\fx(t)
    &= \opTi\fx(t-1)
    && \text{by defintion of $\opT$ \prefpo{def:wav_opT}}
  \\&= \fx([t+1]-1)
    && \text{by hypothesis}
  \\&= \opI\fx(t)
    && \text{by definition of $\opI$ \prefpo{def:opI}}
  \\
  \intertext{1b. Proof that $\opT\opTi = \opI$:}
  \opT\opTi\fx(t)
    &= \opT\fx(t+1)
    && \text{by hypothesis}
  \\&= \fx([t-1]+1)
    && \text{by defintion of $\opT$ \prefpo{def:wav_opT}}
  \\&= \opI\fx(t)
    && \text{by definition of $\opI$ \prefpo{def:opI}}
  \\
  \intertext{2a. Proof that $\opDi\opD = \opI$:}
  \opDi\opD\fx(t)
    &= \opDi \sqrt{2}\fx(2t)
    && \text{by defintion of $\opD$ \prefpo{def:wav_opT}}
  \\&= \left(\cwt \right)\sqrt{2}\fx\left(2\left[\frac{1}{2}t\right]\right)
    && \text{by hypothesis}
  \\&= \opI\fx(t)
    && \text{by definition of $\opI$ \prefpo{def:opI}}
  \\
  \intertext{2b. Proof that $\opD\opDi = \opI$:}
  \opD\opDi\fx\left(t\right)
    &= \opD \left[\cwt \fx\left(\frac{1}{2}t\right) \right]
    && \text{by hypothesis}
  \\&= \sqrt{2} \left[\cwt \fx\left(\frac{1}{2}[2t]\right) \right]
    && \text{by defintion of $\opD$ \prefpo{def:wav_opT}}
  \\&= \opI\fx(t)
    && \text{by definition of $\opI$ \prefpo{def:opI}}
\end{align*}
\end{proof}


%---------------------------------------
\begin{theorem}
\label{thm:TD_DT}
%---------------------------------------
Let $\spV$ be a vector space,
$\opD:\spV\to\spV$ be a dilation operator and $\opT:\spV\to\spV$ a translation operator.
\formbox{
  \opT^{2n}\opD = \opD \opT^n 
  \qquad\scriptstyle
  \forall n\in\Z
  }
\end{theorem}
\begin{proof}
\begin{align*}
  \intertext{1. Proof for $n=0$:}
    \left.\opT^{2n}\opD \fx(t)\right|_{n=0}
      &= \opT^{2\cdot0}\opD
    \\&= \opI \opD
    \\&= \opD \opI
    \\&= \opD \opT^0
    \\&= \left. \opD \opT^n \right|_{n=0}
  \\
  \intertext{2. Proof for $n>0$:}
    \opT^{2n}\opD \fx(t)
      &= \opT^{2n} \sqrt{2}\,\fx(2t)
      && \text{by \prefp{def:wav_opD}}
    \\&= \sqrt{2}\,\fx\brp{2t-2n}
      && \text{by \prefp{def:wav_opT}}
    \\&= \sqrt{2}\,\fx\brp{2\brs{t-n}}
    \\&= \opD \fx\brp{t-n}
      && \text{by \prefp{def:wav_opD}}
    \\&= \opD \opT^n \fx\brp{t}
      && \text{by \prefp{def:wav_opT}}
  \\
  \intertext{3. Proof for $n<0$:}
    \opT^{2n}\opD \fx(t)
      &= \opT^{2n} \sqrt{2}\,\fx(2t)
      && \text{by \prefp{def:wav_opD}}
    \\&= \sqrt{2}\,\fx\brp{2t-2n}
      && \text{by \prefp{thm:wav_opTi}}
    \\&= \sqrt{2}\,\fx\brp{2\brs{t-n}}
    \\&= \opD \fx\brp{t-n}
      && \text{by \prefp{def:wav_opD}}
    \\&= \opD \opT \fx\brp{t}
      && \text{by \prefp{thm:wav_opTi}}
\end{align*}
\end{proof}


%---------------------------------------
\begin{theorem}
\label{thm:TD_unitary}
%---------------------------------------
Let $\opT$ the translation operator with inverse $\opTi$ and adjoint $\opTa$.
Let $\opD$ the dilation    operator with inverse $\opDi$ and adjoint $\opDa$.
\footnote{\hie{operator adjoint}: \prefp{def:op_adjoint}}
\formbox{\begin{array}{l rcl @{\qquad}D}
  1. & \opTa &=& \opTi  & (translation operator $\opT$ is \hie{unitary} and thus also normal and isometric) \\
  2. & \opDa &=& \opDi  & (dilation   operator $\opD$ is \hie{unitary} and thus also normal and isometric)
\end{array}}
\end{theorem}
\begin{proof}
\begin{align*}
  \intertext{1. Proof that $\opTa=\opTi$:}
  \inprod{\opT\ff(t)}{\fg(t)}
    &= \inprod{\ff(t-1)}{\fg(t)}
    && \text{by definition of $\opT$ \prefpo{def:wav_opT}}
  \\&= \int_x\ff(t-1) \fg^\ast(t) \dt
  \\&= \int_x\ff(t) \fg^\ast(t+1) \dt
  \\&= \inprod{\ff(t)}{\fg(t+1)}
  \\&= \inprod{\ff(t)}{\mcom{\opTi}{$\opTa$}\fg(t)}
    && \text{by \prefp{thm:wav_opTi}}
  \\
  \intertext{2. Proof that $\opDa=\opDi$:}
  \inprod{\opD\ff(t)}{\fg(t)}
    &= \inprod{\sqrt{2}\ff(2t)}{\fg(t)}
    && \text{by definition of $\opD$ \prefpo{def:wav_opD}}
  \\&= \int_t  \sqrt{2} \ff(2t) \fg^\ast(t) \dt
  \\&= \int_u  \sqrt{2} \ff(u) \fg^\ast\left(\frac{1}{2}u\right) \frac{1}{2}\du
    && \text{let $u\eqd 2t \quad\implies\quad \dt=\frac{1}{2}\du$}
  \\&= \int_u  \ff(u) \left[\frac{1}{\sqrt{2}}\fg\left(\frac{1}{2}u\right)\right]^\ast \du
  \\&= \inprod{\ff(t)}{\frac{1}{\sqrt{2}}\fg\left(\frac{1}{2}t\right)}
  \\&= \inprod{\ff(t)}{\mcom{\opDi}{$\opDa$}\fg(t)}
    && \text{by \prefp{thm:wav_opDi}}
\end{align*}
\end{proof}


%---------------------------------------
\begin{corollary}
\label{cor:wav_TD_prop}
%---------------------------------------
Let $\spV$ be a vector space,
$\opD:\spV\to\spV$ be a dilation operator and $\opT:\spV\to\spV$ a translation operator.
\formbox{\begin{array}{ll@{\qquad}l@{\qquad}D}
  1. & \opTa\opT = \opT\opTa = \opI
     & \opDa\opD = \opD\opDa = \opI
     & (unitary)
  \\
  2. & \normop{\opT} = 1
     & \normop{\opD} = 1
     & (unit length)
  \\
  3. & \inprod{\opT\vx}{\opT\vy}  = \inprod{\vx}{\vy}  
     & \inprod{\opD\vx}{\opD\vy}  = \inprod{\vx}{\vy}  
     & (surjective) 
  \\
  4. & \norm{\opT\vx-\opT\vy} = \norm{\vx-\vy}             
     & \norm{\opD\vx-\opD\vy} = \norm{\vx-\vy}             
     & (isometric in distance) 
  \\
  5. & \norm{\opT\vx} = \norm{\vx}             
     & \norm{\opD\vx} = \norm{\vx}             
     & (isometric in length) 
\end{array}}
\end{corollary}
\begin{proof}
\begin{liste}
  \item By \prefpp{thm:TD_unitary}, $\opT$ and $\opD$ are \hie{unitary}.
  \item The properties listed follow from the unitary property and \prefpp{thm:unitary_equiv}
        and \prefpp{thm:unitary_prop}.
\end{liste}
\end{proof}


%---------------------------------------
\begin{theorem}
\label{thm:wav_FD}
%---------------------------------------
Let $\spV$ be a vector space,
$\opD:\spV\to\spV$ be a dilation operator and $\opT:\spV\to\spV$ a translation operator.
Let $\opFT$ be the Fourier Transform operator.
\formbox{\begin{array}{rcl}
  \opFT\opD &=& \opDi\opFT  \\
  \opD\opFT &=& \opFT\opDi
\end{array}}
\end{theorem}
\begin{proof}
\begin{align*}
  \opFT\opD\ff(t)
    &= \opF \brs{\sqrt{2}\,\ff(2t)}
    && \text{by definition of $\opD$ \prefpo{def:wav_opD}}
  \\&= \frac{1}{\sqrt{2\pi}}\,\int_t \brs{\sqrt{2}\,\ff(2t)} e^{-i\omega t} \dt
    && \text{by definition of $\opFT$ \prefpo{def:opFT}}
  \\&= \frac{1}{\sqrt{2\pi}}\,\int_u \brs{\sqrt{2}\,\ff(u)} e^{-i\omega \frac{u}{2}} \frac{1}{2}\du
    && \text{let $u\eqd 2t \implies t=\frac{1}{2}u$}
  \\&= \cwt \, \frac{1}{\sqrt{2\pi}}\,
       \int_u \ff(u) e^{-i\frac{\omega}{2} u} \du
  \\&= \cwt \, \frac{1}{\sqrt{2\pi}}\,
       \int_u \ff(u) e^{-i\frac{\omega}{2} u} \du
  \\&= \opDi\,\brs{ \frac{1}{\sqrt{2\pi}}\,
       \int_u \ff(u) e^{-i\omega u} \du}
    && \text{by \prefp{thm:wav_opDi}}
  \\&= \opDi\, \opFT \, \ff(t)
  \\
  \opD\opFT\,\ff(t)
    &= \opD \brs{\frac{1}{\sqrt{2\pi}} \int_t \ff(t) e^{-i\omega t} \dt  }
    && \text{by definition of $\opFT$ \prefpo{def:opFT}}
  \\&= \frac{\sqrt{2}}{\sqrt{2\pi}} \int_t \ff(t) e^{-i2\omega t} \dt
    && \text{by definition of $\opD$ \prefpo{def:wav_opD}}
  \\&= \frac{\sqrt{2}}{\sqrt{2\pi}} \int_u \ff\brp{\frac{u}{2}} e^{-i\omega u} \frac{1}{2}\du
    && \text{let $u\eqd 2t \implies t=\frac{1}{2}u$}
  \\&= \frac{1}{\sqrt{2\pi}} \int_u \brs{\cwt  \ff\brp{\frac{u}{2}} }e^{-i\omega u} \du
  \\&= \frac{1}{\sqrt{2\pi}} \int_u \brs{\opDi \ff}(u) \,e^{-i\omega u} \du
    && \text{by \prefp{thm:wav_opDi}}
  \\&= \opFT\opDi \ff(t)
    && \text{by definition of $\opFT$ \prefpo{def:opFT}}
\end{align*}
\end{proof}


%---------------------------------------
\begin{corollary}
\label{cor:wav_DFD}
%---------------------------------------
Let $\spV$ be a vector space,
$\opD:\spV\to\spV$ be a dilation operator and $\opT:\spV\to\spV$ a translation operator.
Let $\opFT$ be the Fourier Transform operator.
\formbox{\begin{array}{rcl cl}
  \opD  &=& \opFT\opDi\opFTi &=& \opFTi\opDi\opFT \\
  \opFT &=& \opD\opFT\opD &=& \opDi\opFT\opDi  
\end{array}}
\end{corollary}
\begin{proof}
These results follow from \prefpp{thm:wav_FD}.
\end{proof}

%---------------------------------------
\begin{theorem}
\label{thm:wav_FT}
%---------------------------------------
Let $\spV$ be a vector space,
$\opD:\spV\to\spV$ be a dilation operator and $\opT:\spV\to\spV$ a translation operator.
Let $\opFT$ be the Fourier Transform operator.
\formbox{\begin{array}{rcl}
  \opFT\opT^n  &=& e^{-i\omega n} \opFT
\end{array}}
\end{theorem}
\begin{proof}
\begin{align*}
  \opFT\opT^n \ff(t)
    &= \opFT \ff(t-n)
    && \text{by definition of $\opT$ \prefpo{def:wav_opT}}
  \\&= \frac{1}{\sqrt{2\pi}} \int_t \ff(t-n) e^{-i\omega t} \dt
    && \text{by definition of $\opFT$ \prefpo{def:opFT}}
  \\&= \frac{1}{\sqrt{2\pi}} \int_u \ff(u) e^{-i\omega (u+n)} \du
    && \text{where $u\eqd t-n$}
  \\&= e^{-i\omega (n)}\;\brs{\frac{1}{\sqrt{2\pi}} \int_u \ff(u) e^{-i\omega u} \du}
  \\&= e^{-i\omega (n)}\;\opFT \ff(t)
\end{align*}
\end{proof}


%---------------------------------------
\begin{theorem}
%---------------------------------------
Let $\opT$ the \hie{translation operator} such that
$\opT\seqn{x_n} = \seqn{x_{n-1}}$.
That is,
\[
  \opT
  \mcom{\begin{array}{|l||*{16}{c|}}
    \hline
    n   & \cdots &  -2 & -1  & 0   & 1    & 2   & 3   & \cdots \\
    \hline
    x_n & \cdots & x_{-2} & x_{-1} & x_0 &  x_1 & x_2 & x_3 & \cdots \\
    \hline
  \end{array}}{original sequence $\vx=\seq{x_n}{n\in\Z}$}
  =
  \mcom{\begin{array}{|l||*{16}{c|}}
    \hline
    n   & \cdots &  -2 & -1  & 0   & 1    & 2   & 3    & \cdots \\
    \hline
    x_n & \cdots & x_{-3} & x_{-2} & x_{-1} &  x_0 & x_1 & x_2 & \cdots \\
    \hline
  \end{array}}{shifted sequence $\opT\vx=\seq{x_{n-1}}{n\in\Z}$}.
\]
Then
\formbox{
  \vphi 
  =
  \mcom{\begin{array}{|l||*{16}{c|}}
    \hline
    n   & \cdots &  -2     & -1     & 0   & 1   & 2   & 3   & 4   & \cdots \\
    \hline
    \phi_n & \cdots & \lambda^{2}\phi_0 & \lambda \phi_0 & \phi_0 & \lambda^{-1} \phi_0 & \lambda^{-2} \phi_0 & \lambda^{-3} \phi_0 & \lambda^{-4} \phi_0 & \cdots \\
    \hline
  \end{array}}{\hie{geometric progression} with \hie{common ratio} $\lambda$}
  \iff
  \mcom{\opT\vphi = \lambda\vphi}{$\vphi$ is an eigenvector}
  }
\end{theorem}
\begin{proof}
\begin{align*}
  \intertext{$\imark$ Proof that 
    $\vphi=
       \begin{array}{|>{\scriptscriptstyle}l||*{16}{>{\scriptscriptstyle}c|}}
         \hline
         n   & \cdots &  -2     & -1     & 0   & 1   & 2   & 3   & 4   & \cdots \\
         \hline
         \phi_n & \cdots & \lambda^{2}\phi_0 & \lambda^{1}\phi_0 & \phi_0 & \lambda^{-1} \phi_0 & \lambda^{-2} \phi_0 & \lambda^{-3} \phi_0 & \lambda^{-4} \phi_0 & \cdots \\
         \hline
       \end{array}
    $
    $\implies$
    $\opT\vphi=\lambda\vphi$:}
  %   
  \opT\vphi 
    &= \opT\;
       \begin{array}{|l||*{16}{c|}}
         \hline
         n   & \cdots &  -2     & -1     & 0   & 1   & 2   & 3   & 4   & \cdots \\
         \hline
         \phi_n & \cdots & \lambda^{2}\phi_0 & \lambda^{1}\phi_0 & \phi_0 & \lambda^{-1} \phi_0 & \lambda^{-2} \phi_0 & \lambda^{-3} \phi_0 & \lambda^{-4} \phi_0 & \cdots \\
         \hline
       \end{array}
    && \text{by left hypothesis}
  \\&= \quad\begin{array}{|l||*{16}{c|}}
         \hline
         n   & \cdots &  -2     & -1     & 0   & 1   & 2   & 3   & 4   & \cdots \\
         \hline
         \phi_n & \cdots & \lambda^{3}\phi_0 & \lambda^{2}\phi_0 & \lambda \phi_0 & \phi_0 & \lambda^{-1} \phi_0 & \lambda^{-2} \phi_0 & \lambda^{-3} \phi_0  & \cdots \\
         \hline
       \end{array}
    && \text{by definition of $\opT$}
  \\&= \lambda\vphi
    && \text{by left hypothesis}
\end{align*}



\begin{align*}
  \intertext{$\imark$ Proof that 
    $\vphi=
       \begin{array}{|>{\scriptscriptstyle}l||*{16}{>{\scriptscriptstyle}c|}}
         \hline
         n   & \cdots &  -2     & -1     & 0   & 1   & 2   & 3   & 4   & \cdots \\
         \hline
         \phi_n & \cdots & \lambda^{2}\phi_0 & \lambda^{1}\phi_0 & \phi_0 & \lambda^{-1} \phi_0 & \lambda^{-2} \phi_0 & \lambda^{-3} \phi_0 & \lambda^{-4} \phi_0 & \cdots \\
         \hline
       \end{array}
    $
    $\impliedby$
    $\opT\vphi=\lambda\vphi$:}
  %   
  \opT\vphi &= \opT \seqn{\phi_n} = \seqn{\phi_{n-1}}
  \\
  \lambda\vphi &= \lambda\seqn{\phi_n} = \seqn{\lambda\phi_n}
  \\ \opT\vphi = \lambda\vphi  \implies \phi_{n-1} &= \lambda\phi_n \qquad \forall n\in\Z
  \\ \implies \phi_{n} &= \lambda^{-1}\phi_{n-1} \qquad \forall n\in\Z
  \\ \implies \phi_{1} &= \lambda^{-1}\phi_{0}
  \\ \implies \phi_{2} &= \lambda^{-1}\phi_{1} = \lambda^{-2} \phi_0
  \\ \implies \phi_{3} &= \lambda^{-1}\phi_{2} = \lambda^{-3} \phi_0
  \\ \phi_n = \lambda^{-n}\phi_0 \implies \phi_{n+1} = \lambda^{-n-1}\phi_0
  \\ \implies \phi_n &= \lambda^{-n}\phi_0 \qquad \forall n\in\Z
  \\ \implies \vphi &= 
     \begin{array}{|l||*{16}{c|}}
       \hline
       n   & \cdots &  -2     & -1     & 0   & 1   & 2   & 3   & 4   & \cdots \\
       \hline
       \phi_n & \cdots & \lambda^{2}\phi_0 & \lambda \phi_0 & \phi_0 & \lambda^{-1} \phi_0 & \lambda^{-2} \phi_0 & \lambda^{-3} \phi_0 & \lambda^{-4} \phi_0 & \cdots \\
       \hline
     \end{array}
\end{align*}

\end{proof}




%---------------------------------------
\begin{theorem}
%---------------------------------------
Let $\opD$ the \hie{dilation operator} such that
$\opD\seqn{x_n} = \seqn{x_{2n}}$.
That is,
\[
  \opD
  \mcom{\begin{array}{|l||*{16}{c|}}
    \hline
    n   & \cdots &  -2 & -1  & 0   & 1    & 2   & 3   & \cdots \\
    \hline
    x_n & \cdots & x_{-2} & x_{-1} & x_0 &  x_1 & x_2 & x_3 & \cdots \\
    \hline
  \end{array}}{original sequence $\vx=\seq{x_n}{n\in\Z}$}
  =
  \mcom{\begin{array}{|l||*{16}{c|}}
    \hline
    n   & \cdots &  -2 & -1  & 0   & 1    & 2   & 3    & \cdots \\
    \hline
    x_n & \cdots & x_{-4} & x_{-2} & x_{0} &  x_2 & x_4 & x_6 & \cdots \\
    \hline
  \end{array}}{dilated sequence $\opD\vx=\seq{x_{2n}}{n\in\Z}$}.
\]
Then
\formbox{\begin{array}{>{\ds}l}
  \vphi 
  = \seq{\phi_n \st \phi_{2n}=\lambda^{\ffr(n)} \phi_{\fo(n)}}{n\in\Z}
  \iff
  \mcom{\opD\vphi = \lambda\vphi}{$\vphi$ is an eigenvector}
  \\
  \text{where}\footnotemark
  \quad\left\{\begin{array}{rc>{\ds}ll}
    \ffr(n) &\eqd& \text{number of factors of $2$ in $2n$} & \text{(the \hie{ruler function})}\\
    \fo(n)  &\eqd& \frac{2n}{2^{\ffr(n)}}                  & \text{(the \hie{odd part} of $n$)}
  \end{array}\right.
\end{array}}
\footnotetext{\begin{tabular}[t]{ll}
  $\ffr(n)$: & \citer{sloane}, \href{http://www.research.att.com/~njas/sequences/A001511}{A001511} \\
             & \citerpp{thomae1875}{14}{15} \\
  $\fo(n)$:  & \citer{sloane}, \href{http://www.research.att.com/~njas/sequences/A000265}{A000265}
\end{tabular}}
\end{theorem}
\begin{proof}
\begin{align*}
  \intertext{$\imark$ Proof that 
    $\vphi= \seq{\phi_n \st \lambda^{\ffr(n)} \phi_{\fo(n)}}{n\in\Z}
     \impliedby
     \opD\vphi = \lambda\vphi
    $:}
  \opD\seqn{\phi_n} &= \seqn{\phi_{2n}} \\
  \lambda \seqn{\phi_n} &= \seqn{\lambda\phi_n}
  \\ \opD\vphi = \lambda\vphi \implies \phi_{2n} = \lambda\phi_n \qquad \forall n\in\Z
  \\ \implies \phi_{0 } &= \lambda \phi_0    &&= \lambda   \phi_0
  \\          \phi_{2 } &= \lambda \phi_1    &&= \lambda   \phi_1
  \\          \phi_{4 } &= \lambda \phi_2    &&= \lambda^2 \phi_1
  \\          \phi_{6 } &= \lambda \phi_3    &&= \lambda   \phi_3
  \\          \phi_{8 } &= \lambda \phi_4    &&= \lambda^3 \phi_1 
  \\          \phi_{10} &= \lambda \phi_5    &&= \lambda   \phi_5
  \\          \phi_{12} &= \lambda \phi_6    &&= \lambda^2 \phi_3 
  \\          \phi_{14} &= \lambda \phi_7    &&= \lambda   \phi_7
  \\          \phi_{16} &= \lambda \phi_8    &&= \lambda^4 \phi_1 
  \\          \phi_{18} &= \lambda \phi_9    &&= \lambda   \phi_9
  \\          \phi_{20} &= \lambda \phi_{10} &&= \lambda^2 \phi_5
  \\                  & \vdots
  \\ \implies \phi_n  &= \lambda^{\ffr(n)} \phi_{\fo(n)} 
  \\
  \intertext{$\imark$ Proof that 
    $\vphi= \seq{\phi_n \st \lambda^{\ffr(n)} \phi_{\fo(n)}}{n\in\Z}
     \implies
     \opD\vphi = \lambda\vphi
    $:}
  \opD\vphi
    &= \opD\seq{\phi_n \st \phi_{2n}=\lambda^{\ffr(n)} \phi_{\fo(n)}}{n\in\Z}
    && \text{by left hypothesis}
  \\&= \seqn{\phi_{2n} \st \phi_{2n}=\lambda^{\ffr(n)} \phi_{\fo(n)}}
    && \text{by definition of $\opD$}
  \\&= \seqn{\phi_{2n} \st \phi_{2n}=\lambda \phi_n}
    && \text{by previous development}
  \\&= \seqn{\lambda \phi_n}
  \\&= \lambda \vphi
\end{align*}
\end{proof}





%---------------------------------------
\begin{theorem}
\label{thm:wav_opT_spectrum}
%---------------------------------------
Let $\opT\in\oppB(\spX,\spX)$ be the \hie{translation operator}.
\formbox{\begin{array}{l}
  \begin{array}{l>{\ds}rc>{\ds}l}
    1. & \opT &=& \sum_n \lambda_n \opP_n  \\
    2. & \sum_n \opP_n &=& \opI \\
    3. & \opP_n\opP_m &=& \kdelta_{n-m} \opP_n \\
    4. & \oppDim(\spH_n) &<& \infty \\
    5. & \mc{3}{l}{\seto{\set{\lambda_n}{\lambda_n\ne 0}}  \text{is countably infinite}} \\
  \end{array}
  \\
  where \quad\left\{
    \begin{array}{>{\ds}rcl@{\qquad}D}
      \seq{\lambda_n}{n\in\Z} &\eqd& \oppSpecp(\opT) & (eigenvalues of $\opT$) \\
      \spH_n &\eqd& \oppN(\opT-\lambda_n\opI)        & ($\lambda_n$ is the eigenspace of $\opT$ at $\lambda_n$ in $\spY$) \\
      \spH_n &=& \opP_n \spY                         & ($\opP_n$ is the projection operator that generates $\spH_n$)
    \end{array}
  \right.
  \end{array}}
\end{theorem}
\begin{proof}
\begin{dingautolist}{"AC}
  \item By \prefpp{thm:TD_unitary}, $\opT$ is \hie{unitary}.
  \item Because $\opT$ is unitary, by \prefpp{def:op_types} $\opT$ is also \hie{normal}.
  \item Because $\opT$ is normal and compact, by the \hie{Spectral Theorem} (\prefp{thm:spectral_theorem}), 
        it has the expansion shown in this theorem (\pref{thm:wav_opT_spectrum}).
\end{dingautolist}
\end{proof}



%---------------------------------------
\begin{theorem}
\label{thm:wav_opD_spectrum}
%---------------------------------------
Let $\opD\in\oppB(\spX,\spX)$ be the \hie{dilation operator}.
\formbox{\begin{array}{l}
  \begin{array}{l>{\ds}rc>{\ds}l}
    1. & \opD &=& \sum_n \lambda_n \opP_n  \\
    2. & \sum_n \opP_n &=& \opI \\
    3. & \opP_n\opP_m &=& \kdelta_{n-m} \opP_n \\
    4. & \oppDim(\spH_n) &<& \infty \\
    5. & \mc{3}{l}{\seto{\set{\lambda_n}{\lambda_n\ne 0}}  \text{is countably infinite}} \\
  \end{array}
  \\
  where \quad\left\{
    \begin{array}{>{\ds}rcl@{\qquad}D}
      \seq{\lambda_n}{n\in\Z} &\eqd& \oppSpecp(\opD) & (eigenvalues of $\opD$) \\
      \spH_n &\eqd& \oppN(\opD-\lambda_n\opI)        & ($\lambda_n$ is the eigenspace of $\opD$ at $\lambda_n$ in $\spY$) \\
      \spH_n &=& \opP_n \spY                         & ($\opP_n$ is the projection operator that generates $\spH_n$)
    \end{array}
  \right.
  \end{array}}
\end{theorem}
\begin{proof}
\begin{dingautolist}{"AC}
  \item By \prefpp{thm:TD_unitary}, $\opD$ is \hie{unitary}.
  \item Because $\opD$ is unitary, by \prefpp{def:op_types} $\opD$ is also \hie{normal}.
  \item Because $\opD$ is normal and compact, by the \hie{Spectral Theorem} (\prefp{thm:spectral_theorem}), 
        it has the expansion shown in this theorem (\pref{thm:wav_opD_spectrum}).
\end{dingautolist}
\end{proof}






\begin{enumerate}

%\if 0
\item Proof 2:
\begin{enumerate}
\item Proof that $\fP_m^{(n)}(0)=\frac{(p+n-1)!}{(p-1)!}$:\\
  \begin{align*}
  \left.\deriv{^n}{y^n} \fP_m(y) \right|_{y=0}
    &= \left.
       \deriv{^n}{y^n}
       \sum_{k=0}^{p-1} {p-1+k\choose k} y^k
       \right|_{y=0}
  \\&= \left.
       \sum_{k=n}^{p-1} {p-1+k\choose k} \left[\frac{k!}{(k-n)!}y^{k-n}\right]
       \right|_{y=0}
  \\&= \sum_{k=n}^{p-1} {p-1+k\choose k} \left[\frac{k!}{(k-n)!} \kdelta_{kn}\right]
  \\&= {p-1+n\choose n} [n!]
  \\&= \frac{(p-1+n)!)}{(p-1)!n!} [n!]
  \\&= \frac{(p+n-1)!}{(p-1)!}
  \end{align*}


\item Proof that
\begin{align*}
  \fP_m(y)
    &= \sum_{n=0}^\infty \frac{1}{n!} P^{(n)}(0) y^p
    && \text{by Maclaurin series \prefpo{thm:taylor}}
  \\&= \sum_{n=0}^{p-1} \frac{1}{n!} P^{(n)}(0) y^p
    && \text{by B\'ezout's Theorem \prefpo{thm:bezout}}
  \\&= \sum_{n=0}^{p-1} \frac{1}{n!} \frac{(p+n-1)!}{(p-1)!}  y^p
    && \text{by (a).}
  \\&= \sum_{n=0}^{p-1} {p+n-1 \choose n}  y^p
    && \text{by definition of ${n\choose k}$}
\end{align*}
\end{enumerate}
%\fi


%\if 0
\item Proof that
  \[
    P^{(n)}(0) =  \kdelta_n +
    (-1)^{n-1} p!
    \sum_{m=0}^{n-1} {n\choose n-m} (-1)^m \frac{1}{(p-n+m)!}
    P^{(m)}(0):
  \]

\begin{eqnarray*}
  \kdelta_n
    &=& \left. \kdelta_n \right|_{y=0}
  \\&=& \left.
        \deriv{^p}{y^p} 1
        \right|_{y=0}
  \\&=& \left.
        \deriv{^p}{y^p} \left[ (1-y)^p P(y) + y^p P(1-y) \right]
        \right|_{y=0}
  \\&=& \left.\left[
        \sum_{k=0}^p {n\choose k} \left[\deriv{^k}{y^k}(1-y)^p\right] P^{(n-k)}(y)
      + \sum_{k=0}^p {n\choose k} \left[\deriv{^k}{y^k} y^p\right] \left[(-1)^{n-k} P^{(n-k)}(1-y)\right]
        \right]\right|_{y=0}
  \\&=& \left.\left[
        \sum_{k=0}^p {n\choose k} (-1)^k \frac{p!}{(p-k)!} (1-y)^{p-k} P^{(n-k)}(y  )
      + \sum_{k=0}^p {n\choose k}        \frac{p!}{(p-k)!}    y^{p-k} (-1)^{n-k} P^{(n-k)}(1-y)
        \right]\right|_{y=0}
  \\&=& \sum_{k=0}^p {n\choose k} (-1)^k \frac{p!}{(p-k)!} P^{(n-k)}(0)
  \\&=& \sum_{m=0}^p {n\choose n-m} (-1)^{n-m} \frac{p!}{(p-n+m)!} P^{(m)}(0)
        \hs{6ex}m=n-k \iff k=n-m
  \\&=& P^{(n)}(0)
      + \sum_{m=0}^{n-1} {n\choose n-m} (-1)^{n-m} \frac{p!}{(p-n+m)!} P^{(m)}(0)
  \\&=& P^{(n)}(0)
      + (-1)^p p! \sum_{m=0}^{n-1} {n\choose n-m} (-1)^m \frac{1}{(p-n+m)!} P^{(m)}(0)
\end{eqnarray*}
%\fi



%\if 0
\item $p$ even case:

\[\begin{array}{|r||l|l|l|l|}
  \hline
  n  & r_n   & q_n   & s_n=s_{n-2}-q_ns_{n-1} & t_n=t_{n-2}-q_nt_{n-1}  \\
  \hline
  -1  & \ds (1-y)^p
      & \ds -
      & \ds 1
      & \ds 0
      \\
   0  & \ds y^p
      & \ds -
      & \ds 0
      & \ds 1
      \\
  \hline
   1  & \ds  \sum_{k=0}^{p-1} {p\choose k}(-1)^k y^k
      & \ds  1
      & \ds  1
      & \ds -1
      \\
   2  & \ds \frac{1}{p} y \sum_{k=0}^{p-2} {p\choose k}(-1)^k y^k
      & \ds -\frac{1}{p} y
      & \ds \frac{1}{p} y
      & \ds 1 - \frac{1}{p} y
      \\
   3  & \ds \sum_{k=0}^{p-2} {p\choose k}(-1)^k
            \frac{(p-1)(p-k+1)-2k}{(p-1)(p-k+1)}\;y^k
      & \ds -\frac{2p}{p-1}
      & \ds 1 +\frac{2}{p-1}y
      & \ds \frac{p+1}{p-1} - \frac{2}{p-1}y
      \\
   4  & \ds
      & \ds \frac{3(p-1)}{p(p+1)} y
      & \ds \frac{-2p+4}{p(p+1)} y  -\frac{6}{p(p+1)} y^2
      & \ds 1 -\frac{4}{p} y + \frac{6}{p(p+1)} y^2
      \\
 \vdots & \ds \vdots
      & \ds \vdots
      & \ds \vdots
      & \ds \vdots
      \\
 2p-1 & \ds 1
      & \ds
      & \ds P(y)
      & \ds P(1-y)
      \\
  \hline
 2p   & \ds 0
      & \ds
      & \ds -
      & \ds -
      \\
  \hline
\end{array}\]
\begin{eqnarray*}
  \sum_{k=0}^{p-2} {p\choose k}(-1)^k y^k
  + \frac{2}{p-1} y \sum_{k=0}^{p-3} {p\choose k}(-1)^k y^k
    &=& \sum_{k=0}^{p-2} {p\choose k}(-1)^k y^k
      + \frac{2}{p-1} \sum_{k=0}^{p-3} {p\choose k}(-1)^k y^{k+1}
  \\&=& \sum_{k=0}^{p-2} {p\choose k}(-1)^k y^k
      + \frac{2}{p-1} \sum_{k=1}^{p-2} {p\choose k-1}(-1)^{k-1} y^{k}
  \\&=& \sum_{k=0}^{p-2} {p\choose k}(-1)^k y^k
      - \frac{2}{p-1} \sum_{k=1}^{p-2} \frac{k}{p-k+1}{p\choose k}(-1)^k y^{k}
  \\&=& \sum_{k=0}^{p-2} {p\choose k}(-1)^k y^k
      - \sum_{k=1}^{p-2} \frac{2k}{(p-1)(p-k+1)}{p\choose k}(-1)^k y^{k}
  \\&=& \sum_{k=0}^{p-2} {p\choose k}(-1)^k y^k
      - \sum_{k=0}^{p-2} \frac{2k}{(p-1)(p-k+1)}{p\choose k}(-1)^k y^{k}
  \\&=& \sum_{k=0}^{p-2} {p\choose k}(-1)^k
        \left[1-\frac{2k}{(p-1)(p-k+1)}\right]y^k
  \\&=& \sum_{k=0}^{p-2} {p\choose k}(-1)^k
        \left[\frac{(p-1)(p-k+1)-2k}{(p-1)(p-k+1)}\right]y^k
\end{eqnarray*}
%\fi
\end{enumerate}


