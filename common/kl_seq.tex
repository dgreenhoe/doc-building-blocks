%============================================================================
% LaTeX File
% Daniel J. Greenhoe
%============================================================================

%======================================
\chapter{KL Expansion---Discrete Case}
%======================================
%=======================================
\section{Definitions}
\label{sec:KL}
%=======================================
%---------------------------------------
\begin{definition}
\label{def:white_seq}
%---------------------------------------
Let $\rvx(n)$ be a \fncte{random sequence} \xref{def:randseq}
and $\Rxx(m)$ the \fncte{autocorrelation} \xref{def:Rxxm} of $\rvx(n)$.
Let $\kdelta(n)$ be the \fncte{Kronecker delta function} \xref{def:kdelta}.
\defboxt{
  $\rvx(n)$ is \propd{white} if
  \\\indentx$\begin{array}{cMMD}
    (1) & $\ds\rvx(n)$ is \prope{wide sense stationary} (WSS) & \xref{def:wss} & and\\
    (2) & $\ds\Rxx(m)=K\kdelta(m)$\quad for some $K>0$
  \end{array}$
  \\The random sequence $\rvx(n)$ is \propd{colored} if $\rvx(n)$ is \prope{WSS} but yet \emph{not} \prope{white}.
  }
\end{definition}

%---------------------------------------
\begin{definition}
\label{def:opRnm}
%---------------------------------------
Let $\rvx(n)$ and $\rvy(n)$ be \fncte{random sequence}s \xref{def:randseq}.
Let $\Rxx(n,m)$ be the \fncte{autocorrelation} \xref{def:Rxxnm} of $\rvx(n)$.
\defboxt{
  The \opd{autocorrelation operator} $\opRx$ of $\rvy(n)$ is defined as
  \\\indentx$\ds\brs{\opRx\rvy}(n) \eqd \sum_{m\in\Z} \Rxx(m,n) \rvy(m)$
  \\where $\Rxx(m,n)$ is the \fncte{kernel} \xref{rem:kernel} of $\brs{\opRx\rvy}(n)$.
  }
\end{definition}

Now for some key points:
\begin{liste}
  \item If a random sequence $\rvx(n)$ is \prope{white} \xref{def:white_seq}
        and the set $\Psi=\setn{\fpsi_1(n),\fpsi_2(n),\ldots,\fpsi_\xN(n)}$
        is \textbf{any} set of orthonormal basis functions, then the inner products
        $\inprod{\rvx(n)}{\fpsi_p(n)}$ and $\inprod{\rvx(n)}{\fpsi_q(n)}$ are \prope{uncorrelated}
        for $p\ne  q$ \xref{thm:fdotp_white}.
  \item However, if $\rvx(n)$ is \propb{colored}, then the inner products are \emph{not}
        in general uncorrelated.
  \item \emph{But} if the elements of $\Psi$ are chosen to be the \fncte{eigenfunction}s \xref{def:eigen}
        of $\opRx$ such that $\ds\opRx \fpsi_n = \lambda_n \fpsi_n$,
        then the set $\setn{\fpsi_n(n)}$ are \prope{orthogonal} \xref{cor:opRn_eigen} and
        the inner products \textbf{are} \prope{uncorrelated}, even if $\rvx(n)$ is not white.
%This criterion is called the  Karhunen-Lo\`{e}ve criterion for $\rvx(n)$ \xref{thm:kle}.
\end{liste}

%=======================================
\section{Properties}
%=======================================
%---------------------------------------
\begin{theorem}
\label{thm:fdotp_zeromean}
%---------------------------------------
Let $\rvx(n)$ be a \fncte{random sequence} \xref{def:randseq}.
Let $\fpsi_p(n)$ be a function.
Let $\fdotx_p$ be the \ope{projection} of $\rvx(n)$ onto $\fpsi_p(n)$ such that
\\\indentx$\ds \fdotx_p \eqd \inprod{\rvx(n)}{\fpsi_p(n)} \eqd \sum_{n\in\Z} \rvx(n) \fpsi_p^\ast(n)$
\thmbox{
  \mcom{ \pE\brs{\rvx(n)} = 0 }{\prope{zero-mean}}
  \quad\implies\quad
  \mcom{ \pE\brs{\fdotx_p} = 0 }{\prope{zero-mean}}
  }
\end{theorem}
\begin{proof}
  \begin{align*}
    \pE\fdotx_p
      &\eqd \pE\brs{ \inprod{\rvx(n)}{\fpsi_p(n)} }
      && \text{by definition of $\fdotx_p$}
    \\&\eqd \pE\brs{ \sum_{n\in\Z} \rvx(n) \fpsi_p^\ast(n) }
      && \text{by definition of $\inprodn$}
    \\&= \sum_{n\in\Z} \pE\brs{\rvx(n)} \fpsi_p^\ast(n)
      && \text{by \prope{linearity}}
      && \text{\ifxref{expectation}{thm:pE_linop}}
    \\&= \sum_{n\in\Z} 0 \cdot \fpsi_p^\ast(n)
      && \text{by \prope{zero-mean} property}
    \\&= 0
  \end{align*}
\end{proof}

%---------------------------------------
\begin{theorem}
\label{thm:fdotp_white}
%---------------------------------------
Let $\rvx(n)$ be a \fncte{random sequence} \xref{def:randseq}.
Let $\fpsi_p(n)$ be a function.
Let $\fdotx_p$ be the \ope{projection} of $\rvx(n)$ onto $\fpsi_p(n)$ such that
\\\indentx$\ds \fdotx_p \eqd \inprod{\rvx(n)}{\fpsi_p(n)} \eqd \sum_{n\in\Z} \rvx(n) \fpsi_p^\ast(n)$
\thmbox{
  \mcom{ \pE\brs{\rvx(n)\rvx(m)}   = \sigma^2 \kdelta(n-m)  }{\prope{uncorrelated}}
  \quad\implies\quad
  \mcom{ \pE\brs{\fdotx_p\fdotx_q} = \sigma^2 \kdelta_{p-q} }{\prope{uncorrelated}}
  }
\end{theorem}
\begin{proof}
  \begin{align*}
   \pE\brs{ \fdotx_q \fdotx_p^\ast }
      &= \pE\brs{ \inprod{\rvx(n)}{\fpsi_q(n)} \inprod{\rvx(n)}{\fpsi_p(n)}^\ast }
      && \text{by definition of $\fdotx_p$}
    \\&\eqd \pE\brs{ \brp{\sum_{n\in\Z} \rvx(n) \fpsi_q(n)}
                     \brp{\sum_{m\in\Z} \rvx(m) \fpsi_p(m)}^\ast }
      && \text{by definition of $\inprodn$}
    \\&= \pE\brs{ \brp{\sum_{n\in\Z} \rvx(n) \fpsi_q(n)}
                  \brp{\sum_{m\in\Z} \rvx(m) \fpsi_p(m)}^\ast }
      && \text{by def. of $\pCov$}
    \\&= \pE\brs{ \sum_{n\in\Z}\sum_{m\in\Z} \rvx(n)\rvx^\ast(m) \fpsi_q(n) \fpsi_p^\ast(m) }
    \\&= \sum_{n\in\Z}\sum_{m\in\Z} \pE\brs{\rvx(n)\rvx^\ast(m)} \fpsi_q(n) \fpsi_p^\ast(m)
    \\&= \sum_{n\in\Z}\sum_{m\in\Z} \sigma^2\kdelta(n-m) \fpsi_q(n) \fpsi_p^\ast(m)
      && \text{by \prope{uncorrelated} hypothesis}
      && \text{(B)}
    \\&= \sigma^2 \sum_{n\in\Z}  \fpsi_q(n) \fpsi_p^\ast(n)
    \\&\eqd \sigma^2 \inprod{\fpsi_q(n)}{\fpsi_p(n)}
      && \text{by definition of $\inprodn$}
    \\&= \brbl{\begin{array}{lM}
             \sigma^2 & for $p = q$ \\
             0        & for $p\ne q$.
          \end{array}}
      && \text{by \prope{orthonormal} property}
    \\&\eqd \sigma^2 \kdelta_{p-q}
      &&\text{by definition of \fncte{Kronecker delta} $\kdelta$}
      && \text{\xref{def:kdelta}}
  \end{align*}
\end{proof}

%---------------------------------------
\begin{theorem}
\label{thm:opRn}
%---------------------------------------
Let $\rvx(n)$ and $\rvy(n)$ be \fncte{random sequence}s \xref{def:randseq}
with \fncte{inner product} \xref{def:inprod} defined as
\\\indentx$\ds\inprod{\rvx}{\rvy} \eqd \sum_{n\in\Z} \rvx(n) \rvy^\ast(n)$\\
Let $\opRx$ be the \ope{autocorrelation} operator \xref{def:opRnm} with respect to $\rvx(n)$.
\thmbox{\begin{array}{F>{\ds}lDD}
    (1). & \inprod{\opRx \rvx}{\rvx} \ge 0                     & \text{(\prope{non-negative})} & and\\
    (2). & \inprod{\opRx \rvx}{\fy} = \inprod{\rvx}{\opRx \fy} & \text{(\prope{self-adjoint})} &
\end{array}}
\end{theorem}
\begin{proof}

\begin{align*}
\intertext{1. Proof that $\opRx$ is \prope{non-negative}:}
   \inprod{\opRx \fy}{\fy}
     &= \inprod{\sum_{m\in\Z}\Rxx(n,m) \fy(m) }{\fy(n)}
     && \text{by definition of $\opRx$}
     && \text{\xref{def:opRnm}}
   \\&= \inprod{\sum_{m\in\Z}\pE\brs{\rvx(n)\rvx^\ast(m)} \fy(m) }{\fy(n)}
     && \text{by definition of $\Rxx(n,m)$}
     && \text{\xref{def:Rxx}}
   \\&= \pE\brs{\inprod{\sum_{m\in\Z}\rvx(n)\rvx^\ast(m) \fy(m) }{\fy(n)}}
     && \text{by \prope{linearity} of $\inprodn$ and $\sum$}
     && \text{\ifsxref{vsinprod}{def:inprod}}
   \\&= \pE\brs{\sum_{m\in\Z}\rvx^\ast(m) \fy(m)  \inprod{\rvx(n)}{\fy(n)}}
     && \text{by \prope{additivity} property of $\inprodn$}
     && \text{\ifsxref{vsinprod}{def:inprod}}
   \\&= \pE\brs{\inprod{\fy(m)}{\rvx(m)} \inprod{\rvx(n) }{\fy(n)}}
     && \text{by local definition of $\inprodn$}
   \\&= \pE\brs{\inprod{\rvx(m)}{\fy(m)}^\ast \inprod{\rvx(n) }{\fy(n)}}
     && \text{by \prope{conjugate symmetry} prop.}
     && \text{\ifsxref{vsinprod}{def:inprod}}
   \\&= \pE{\abs{\inprod{\rvx(n) }{\fy(n)}}^2}
     && \text{by definition of $\absn$}
     && \text{\xref{def:abs}}
   \\&\ge 0
     && \text{by \prope{strictly positive} property of norms}
     && \text{\ifxref{vsnorm}{def:norm}}
\intertext{2. Proof that $\opRx$ is \prope{self-adjoint}:}
   \inprod{\brs{\opRx \rvx}(n)}{\fy}
     &= \inprod{\sum_{m\in\Z}\Rxx(n,m) \rvx(m) }{\fy(n)}
     && \text{by definition of $\opRx$}
     && \text{\xref{def:opRnm}}
   \\&= \sum_{m\in\Z}\rvx(m) \inprod{\Rxx(n,m)  }{\fy(n)}
     && \text{by \prope{additive} property of $\inprodn$}
     && \text{\ifsxref{vsinprod}{def:inprod}}
   \\&= \sum_{m\in\Z}\rvx(m) \inprod{\fy(n)}{\Rxx(n,m)}^\ast
     && \text{by \prope{conjugate symmetry} prop.}
     && \text{\ifsxref{vsinprod}{def:inprod}}
   \\&= \inprod{ \rvx(m) }{\inprod{\fy(n)}{\Rxx(n,m)} }
     && \text{by local definition of $\inprodn$}
   \\&= \inprod{ \rvx(m) }{\sum_{n\in\Z}\fy(n) \Rxx^\ast(n,m)  }
     && \text{by definition of $\inprodn$}
     && \text{\ifsxref{vsinprod}{def:inprod}}
   \\&= \inprod{ \rvx(m) }{\sum_{n\in\Z}\fy(n) \Rxx(m,n)  }
     && \text{by property of $\Rxx$}
     && \text{\xref{thm:Rxx_prop}}
   \\&= \inprod{ \rvx(m) }{\mcom{\opRx}{$\opRxa$} \fy }
     && \text{by definition of $\opRx$}
     && \text{\xref{def:opRnm}}
   \\\implies& \opRx=\opRxa
     && \text{by definition of \ope{adjoint} $\opRxa$}
     && \text{\ifxref{operator}{def:adjoint}}
   \\\implies& \text{$\opRx$ is \prope{self-adjoint}}
     && \text{by definition of \prope{self-adjoint}}
     && \text{\ifxref{operator}{def:op_selfadj}}
\end{align*}
\end{proof}

%---------------------------------------
\begin{corollary}
%\footnote{
%  \citerpp{keener}{114}{119}
%  }
\label{cor:opRn_eigen}
%---------------------------------------
Let $\seq{\lambda_n}{n\in\Z}$ be the \fncte{eigenvalue}s and
    $\seq{\fpsi_n}{n\in\Z}$ be the \fncte{eigenfunction}s \xref{def:eigen} of
    operator $\opRx$ such that
    $\opRx \fpsi_n = \lambda_n \fpsi_n$ for all $n\in\Z$.
\corbox{\begin{array}{FrclM}
  (1). & \lambda_n \in \R &&
       & (\prope{real-valued})
       \\
  (2). & \lambda_n\ne \lambda_m &\implies& \inprod{\fpsi_n}{\fpsi_m}=0
       & (\prope{orthogonal})
       \\
  (3). & \norm{\fpsi_n}^2>0 &\implies& \lambda_n\ge0
       & (\prope{non-negative})
       \\
  (4). & \norm{\fpsi_n(t)}^2>0, \inprod{\opRx\ff}{\ff} > 0 &\implies& \lambda_n>0
       & ($\opRx$ \prope{positive definite} $\implies$ $\lambda_n$ \prope{positive})
\end{array}}
\end{corollary}
\begin{proof}
\begin{enumerate}
\item Proof that eigenvalues are \prope{real-valued}:
  \begin{align*}
    \text{$\opRx$ is \prope{self-adjoint}}
      &
      && \text{by \prefp{thm:opRn}}
    \\&\implies \text{eigenvalues of $\opRx$ are \prope{real}}
      && \text{\ifsxref{operator}{thm:self_adjoint}}
  \end{align*}

\item Proof that eigenfunctions associated with distinct eigenvalues are orthogonal:
Because $\opRx$ is \prope{self-adjoint}, this property follows\ifsxref{operator}{thm:self_adjoint}.

\item Proof that eigenvalues are \prope{non-negative}:
      \begin{align*}
         0 &\le \inprod{\opRx \fpsi_n}{\fpsi_n}
           &&   \text{by definition of \prope{non-negative definite}}
         \\&=   \inprod{\lambda_n \fpsi_n}{\fpsi_n}
           &&   \text{by definition of \fncte{eigenvalue} ($\opRx\fpsi_n=\lambda_n\fpsi_n$)}
         \\&=   \lambda_n \inprod{\fpsi_n}{\fpsi_n}
           &&   \text{by \prope{homogeneous} property of \fncte{inner product}s}
           &&   \text{\ifxref{vsinprod}{def:inprod}}
         \\&=   \lambda_n \norm{\fpsi_n}^2
           &&   \text{by definition of norm induced by inner-product}
           &&   \text{\ifxref{vsnorm}{def:norm}}
      \end{align*}

\item Proof that eigenvalues are \prope{positive} if $\opRx$ is \prope{positive definite}:
      \begin{align*}
         0 &< \inprod{\opRx \fpsi_n}{\fpsi_n}
           && \text{by definition of \prope{positive definite}}
         \\&= \inprod{\lambda_n \fpsi_n}{\fpsi_n}
           && \text{by hypothesis}
         \\&= \lambda_n \inprod{\fpsi_n}{\fpsi_n}
           && \text{by \prope{homogeneous} property of $\inprodn$}
           && \text{\ifxref{vsinprod}{def:inprod}}
         \\&= \lambda_n \norm{\fpsi_n}^2
           && \text{by \thme{induced norm} theorem}
           && \text{\ifxref{vsinprod}{thm:norm=inprod}}
      \end{align*}
\end{enumerate}
\end{proof}

%---------------------------------------
\begin{theorem}[\thmd{Karhunen-Lo/`eve Expansion}]
\footnote{
  \citerpp{keener}{114}{119}
  }
\label{thm:kle}
%---------------------------------------
Let $\opRx$ be the \ope{autocorrelation operator} \xref{def:opRnm}
of a \fncte{random sequence} $\rvx(n)$.
Let $\seq{\lambda_n}{n\in\Z}$ be the eigenvalues of $\opRx$
and $\seq{\fpsi_n}{n\in\Z}$ are the eigenfunctions of $\opRx$ such that
    $\opRx \fpsi_n = \lambda_n \fpsi_n$.
\thmbox{
  \mcom{\norm{\fpsi_n}=1}
       {$\setn{\fpsi_n(p)}$ are \prope{normalized}}
  \implies
  \mcom{\pE\brp{\abs{\rvx(m)-\sum_{n\in\Z}\inprod{\rvx(m)}{\fpsi_n(m)} \fpsi_n(m) }^2} = 0}
       {\prope{convergence in probability}}
  \qquad
  \brp{\begin{array}{M}
    $\setn{\fpsi_n(m)}$ is a\\
    \structe{basis} for $\rvx(m)$
  \end{array}}
  }
\end{theorem}
\begin{proof}
\begin{enumerate}
  \item Define $\ds\dotx_n \eqd \inprod{\rvx(m)}{\fpsi_n(m)}
                           \eqd \sum_{m\in\Z} \rvx(m) \fpsi_n(m)$
        \label{idef:kle_xdot}
 %\item Define $\ds\opRx\rvx(p) \eqd \int_{u\in\R} \Rxx(p,u)\rvx(u)\du$

  \item lemma: \label{ilem:kle_mercer}
        $\ds\pE\brs{\rvx(m)\rvx(m)} = \sum_{n\in\Z}\lambda_n \abs{\fpsi_n(m)}^2$. Proof:
              $\begin{array}[t]{MMM}
                   by  & \prope{non-negative} property & \xref{thm:opRn}
                 \\and & \thme{Mercer's Theorem}       & \xref{thm:mercer}
               \end{array}$

  \item lemma: \label{ilem:kle_1}
    \begin{align*}
       &\pE\brs{\rvx(p) \brp{\sum_{n\in\Z}\dotx_n \fpsi_n(p)}^\ast}
       \\&\eqd \pE\brs{\rvx(p) \brp{\sum_{n\in\Z}\int_{u\in\R}\rvx(u)\fpsi_n^\ast(u)\du \fpsi_n(p)}^\ast}
         && \text{by definition of $\dotx$}
         && \text{\xref{idef:kle_xdot}}
       \\&= \sum_{n\in\Z}\brp{\int_{u\in\R}\pE\brs{\rvx(p)\rvx^\ast(u)}\fpsi_n(u)\du} \fpsi_n^\ast(p)
         && \text{by \prope{linearity}}
         && \text{\ifxref{expectation}{thm:pE_linop}}
       \\&\eqd \sum_{n\in\Z}\brp{\int_{u\in\R}\Rxx(p,u)\fpsi_n(u)\du} \fpsi_n^\ast(p)
         && \text{by definition of $\Rxx(p,u)$}
         && \text{\xref{def:Rxx}}
       \\&\eqd \sum_{n\in\Z}\brp{\opRx\fpsi_n(p) \fpsi_n^\ast(p)}
         && \text{by definition of $\opRx$}
         && \text{\xref{def:opRnm}}
       \\&= \sum_{n\in\Z}\lambda_n\fpsi_n(p) \fpsi_n^\ast(p)
         && \text{by property of \structe{eigen-system}}
       \\&= \sum_{n\in\Z}\lambda_n \abs{\fpsi_n(p)}^2
    \end{align*}

  \item lemma: \label{ilem:kle_2}
    \begin{align*}
       &\pE\brs{\sum_{n\in\Z}\dotx_n \fpsi_n(p)\brp{\sum_{m\in\Z}\dotx_m \fpsi_m(p)}^\ast}
       \\&\eqd \pE\brs{\sum_{n\in\Z}\int_{u\in\R}\rvx(u)\fpsi_n^\ast(u)\du   \fpsi_n(p)\brp{\sum_{m\in\Z}\int_v \rvx(v)\fpsi_m^\ast(v)\dv \fpsi_m(p)}^\ast}
         && \text{by definition of $\dotx$
                  \xref{idef:kle_xdot}}
       \\&= \sum_{n\in\Z}\sum_{m\in\Z}\int_u\brp{\int_v \pE\brs{\rvx(u)\rvx^\ast(v)}\fpsi_m(v)\dv} \fpsi_n^\ast(u)\du   \fpsi_n(p)   \fpsi_m^\ast(p)
         && \text{by \prope{linearity}
                  \ifxref{expectation}{thm:pE_linop}}
       \\&= \sum_{n\in\Z}\sum_{m\in\Z}\int_u\brp{\int_v \Rxx(u,v)\fpsi_m(v)\dv} \fpsi_n^\ast(u)\du   \fpsi_n(p)   \fpsi_m^\ast(p)
         && \text{by definition of $\Rxx(p,u)$
                  \xref{def:Rxx}}
       \\&= \sum_{n\in\Z}\sum_{m\in\Z}\int_u\brp{\opRx\fpsi_m(u)} \fpsi_n^\ast(u)\du   \fpsi_n(p)   \fpsi_m^\ast(p)
         && \text{by definition of $\opRx$
                  \xref{def:opRnm}}
       \\&= \sum_{n\in\Z}\sum_{m\in\Z}\int_u\brp{\lambda_m\fpsi_m(u)} \fpsi_n^\ast(u)\du   \fpsi_n(p)   \fpsi_m^\ast(p)
         && \text{by property of \structe{eigen-system}}
       \\&= \sum_{n\in\Z}\sum_{m\in\Z}\lambda_m \brp{\int_{u\in\R}\fpsi_m(u) \fpsi_n^\ast(u)\du }   \fpsi_n(p)   \fpsi_m^\ast(p)
         && \text{by \prope{linearity}}
       \\&= \sum_{n\in\Z}\sum_{m\in\Z}\lambda_m \norm{\fpsi(p)}^2 \kdelta_{mn}   \fpsi_n(p)   \fpsi_m^\ast(p)
         && \text{by \prope{orthogonal} property
                  \xref{thm:opR_eigen}}
       \\&= \sum_{n\in\Z}\sum_{m\in\Z}\lambda_m \kdelta_{mn}   \fpsi_n(p)   \fpsi_m^\ast(p)
         && \text{by \prope{normalized} hypothesis}
       \\&= \sum_{n\in\Z}\lambda_n   \fpsi_n(p)   \fpsi_n^\ast(p)
         \qquad\text{by definition of \fncte{Kronecker delta} $\kdelta$}
         && \text{\xref{def:kdelta}}
       \\&= \sum_{n\in\Z}\lambda_n \abs{\fpsi_n(p)}^2
    \end{align*}

\item Proof that $\setn{\fpsi_n(p)}$ is a \structe{basis} for $\rvx(p)$:
  \begin{align*}
    &\pE\brp{\abs{\rvx(p)-\sum_{n\in\Z}\dotx_n \fpsi_n(p) }^2}
    \\&= \pE\brp{\brs{\rvx(p)-\sum_{n\in\Z}\dotx_n \fpsi_n(p) }\brs{\rvx(p)-\sum_{m\in\Z}\dotx_m \fpsi_m(p) }^\ast}
    \\&= \pE\brp{
         \rvx(p)\rvx^\ast(p)
       - \rvx(p)\brs{\sum_{n\in\Z}\dotx_n \fpsi_n(p)}^\ast
       - \rvx^\ast(p)\sum_{n\in\Z}\dotx_n \fpsi_n(p)
       + \brs{\sum_{n\in\Z}\dotx_n \fpsi_n(p)} \brs{\sum_{m\in\Z}\dotx_m \fpsi_m(p) }^\ast
       }
    \\&= \pE\brp{\rvx(p)\rvx^\ast(p)}
       - \pE\brs{\rvx(p)\brs{\sum_{n\in\Z}\dotx_n \fpsi_n(p)}^\ast}
       - \pE\brs{\rvx^\ast(p)\sum_{n\in\Z}\dotx_n \fpsi_n(p)}
       + \pE\brs{\sum_{n\in\Z}\dotx_n \fpsi_n(p) \brs{\sum_{m\in\Z}\dotx_m \fpsi_m(p) }^\ast }
    \\&\qquad\text{by \prope{linearity} of $\pE$ \ifxref{expectation}{thm:pE_linop}}
    \\&= \mcom{\sum_{n\in\Z}\lambda_n \abs{\fpsi_n(p)}^2}{by \pref{ilem:kle_mercer}}
       - \mcom{\sum_{n\in\Z}\lambda_n \abs{\fpsi_n(p)}^2}{by \pref{ilem:kle_1}}
       - \mcom{\brs{\sum_{n\in\Z}\lambda_n \abs{\fpsi_n(p)}^2}^\ast}{by \pref{ilem:kle_1}}
       + \mcom{\sum_{n\in\Z}\lambda_n \abs{\fpsi_n(p)}^2}{by \pref{ilem:kle_2}}
    \\&= 0
  \end{align*}
\end{enumerate}
\end{proof}

%=======================================
\section{Quasi-basis}
%=======================================
The \ope{autocorrelation operator} $\opRx$ \xref{def:opRnm} in the discrete case can be approximated using
a \ope{correlation matrix}. In the \prope{zero-mean} case, this becomes
\\\indentx$\ds\opRx\eqd
\brs{\begin{array}{cccc}
  \pE[\rvy_1 \rvy_1] & \pE[\rvy_1 \rvy_2] & \cdots & \pE[\rvy_1 \rvy_n]\\
  \pE[\rvy_2 \rvy_1] & \pE[\rvy_2 \rvy_2] &        & \pE[\rvy_2 \rvy_n]\\
  \vdots             & \vdots             & \ddots & \vdots            \\
  \pE[\rvy_n \rvy_1] & \pE[\rvy_n \rvy_2] & \cdots & \pE[\rvy_n \rvy_n]
\end{array}}
$

The eigen-vectors (and hence a quasi-basis) for $\opRx$ can be found using a
\ope{Cholesky Decomposition}.

%---------------------------------------
\begin{proposition}
\footnote{
  See
  \citerpgc{clarkson1993}{131}{0849386098}{\textsection ``Appendix 3A --- Positive Semi-Definite Form of the Autocorrelation Matrix"}
  }
%---------------------------------------
\propboxt{
  The \ope{autocorrelation matrix} $\opRx$ is \propb{Toeplitz}.
  }
\end{proposition}

%---------------------------------------
\begin{remark}
%---------------------------------------
For more information about the properties of \opb{Toeplitz matrices}, see
\begin{enume}
  \item \citer{grenander1958},
  \item \citer{widom1965},
  \item \citeP{gray1971},
  \item \citerpgc{smylie1973}{408}{9780323148368}{\textsection \scshape ``B. Properties of the Toeplitz Matrix"},
  \item \citer{grenander1984},
  \item \citer{haykin1979},
  \item \citer{haykin1983},
  \item \citerppgc{marple1987}{80}{92}{9780132141499}{\textsection {\scshape``3.8 The Toeplitz Matrix"}},
  \item \citerg{bottcher1999}{9780387985701},
  \item \citeP{gray2006},
  \item \citerppgc{marple2019}{80}{93}{9780486780528}{\textsection {\scshape``3.8 The Toeplitz Matrix"}}.
\end{enume}
\end{remark}

%=======================================
\section{Examples}
%=======================================
Sunspot data --- see \prefpp{ex:sunspot_period_Eigen} and \prefpp{ex:sunspot_white_Eigen}.