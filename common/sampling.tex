%============================================================================
% XeLaTeX File
% Daniel J. Greenhoe
%============================================================================
%---------------------------------------
\chapter{Sampling}
\label{chp:sampling}
%---------------------------------------

%--------------------------------------
\section{A basis for sampling}
%--------------------------------------
\begin{figure}[h]
  \centering
  \includegraphics{graphics/samplat.pdf}
\caption{A basis for sampling \label{fig:samplat}}
\end{figure}
  To perform \opb{sampling}, we \ope{project} 
  continuous functions onto a very special basis
  to get a \fnctb{sequence}, as illustrated in \prefpp{fig:samplat}.
\begin{align*}
  \dot{\fx}(n)
    &\eqd \inprod{\fx(t)}{\delta(t-n)}
  \\&\eqd \int_{t=-\infty}^{t=\infty} \fx(t)\delta(t-n) \dt
  \\&=    \fx(n)
\end{align*}

%--------------------------------------

%--------------------------------------
\includegraphics{graphics/sin_pi_t.pdf}
\includegraphics{graphics/sin_pi_t_sampled.pdf}

%--------------------------------------

%--------------------------------------
  \begin{tabular}{|c|c|}
    \hline
    %\mc{1}{|c|}{subspace}&\mc{1}{c|}{transform}&\mc{1}{c|}{approximation}
    Sampling (analysis) & Approximation (synthesis)
    \\\hline\hline
    %$\spV_0$
      \includegraphics{graphics/haar0_sin_t.pdf}
    & \includegraphics{graphics/haar0_sin_a.pdf}
    \\\hline
    %$\spV_1$
      \includegraphics{graphics/haar1_sin_t.pdf}
    & \includegraphics{graphics/haar1_sin_a.pdf}
    \\\hline
    %$\spV_2$
      \includegraphics{graphics/haar2_sin_t.pdf}
    & \includegraphics{graphics/haar2_sin_a.pdf}
    \\\hline
  \end{tabular}

Approximation getting closer with higher sample rate!
But can we ever get back the original?
If so, how fast do we need to sample?

%--------------------------------------

%--------------------------------------
\tbox{The \thmb{Sample Theorem} \xref{thm:sampling}\\answers this question:}
\quad
\tbox{\includegraphics{graphics/sinclat.pdf}}
\\\indentx\begin{tabular}{cl}
  \imark & \emph{If} your signal is \propb{band-limited}, and
       \\& \emph{if} you sample at a rate of \emph{at least} $2\times$ the highest frequency (the \hie{Nyquist frequency}), and 
       \\& \emph{if} you happen to have an \emph{ideal} low-pass filter,
       \\& \emph{then} you can get the original signal back (perfect synthesis!).
  \\\imark & \emph{But} if you don't sample fast enough, you get \opb{aliasing}.
\end{tabular}

When aliasing occurs, a high frequency component can ``masquerade" as (``pretend" to be, 
``impersonate", ``assume the identity" of, or ``take on the alias" of) 
an entirely different low frequency component.
That is, is forces a high frequency component to take up residence as an 
\hie{alien} (\hie{alias} and \hie{alien} have the same Latin root \hie{al} meaning 
``beyond"\footnote{%
  \url{https://www.etymonline.com/word/alias}, 
  \url{https://www.etymonline.com/word/alien}
  }) 
in a low frequency location.

%--------------------------------------
\begin{example}[\exmd{Aliasing using \emph{Audacity}}]
%--------------------------------------
Here is an experiment with aliasing you can try using the free program 
\hie{Audacity} and the \hie{Nyquist programming language plugin}.\footnote{%
  \hie{Audacity}\rtrademark: 
  ``Free, open source, cross-platform audio software".
  \url{https://www.audacityteam.org/};
  \hie{Nyquist plugin}:
  \url{https://www.audacityteam.org/about/nyquist/}
  }
\\\exboxt{
  \tbox{\includegraphics[height=60mm]{../common/math/graphics/audacity_hzosc.jpg}}\qquad
  \begin{tabular}{rl}
    1. & Set Project Rate to 10000 (Hz)\\
    2. & Tracks $\rightarrow$ New Track $\rightarrow$ Mono Track\\
    3. & Select 1 second to 11 seconds\\
    4. & Effect $\rightarrow$ Nyquist prompt $\rightarrow$ (hzosc 9900) 
  \end{tabular}
  }\\
In this case, the 9900 Hz sinusoid will be aliased to show up as a 100 Hz sinusoid
(more impressive if you happen to have a good subwoofer handy).
\end{example}


%============================================================================
% Daniel J. Greenhoe
% LaTeX/XeLaTeX file
%============================================================================
%=======================================
%\chapter{Cardinal Sequence and Series}
%=======================================

%=======================================
\section{Cardinal Series and Sampling}
%=======================================
%======================================
\subsection{Cardinal series basis}
\label{sec:cardinal}
%======================================
The \prope{Paley-Wiener} class of functions (next definition) are those with a bandlimited Fourier transform.
The cardinal series forms an orthogonal basis for such a space \xrefP{thm:cardinalSeries}.
In a \structe{frame} $\seqnZ{\vx_n}$ with \ope{frame operator} $\opS$ on a \structe{Hilbert Space} $\spH$
with \fncte{inner product} $\inprodn$, 
a function $\ff(x)$ in the space spanned by the frame can be represented by
\\\indentx$\ds\ff(x) = \sum_{n\in\Z} \mcom{\inprod{\ff}{\opSi\vx_n}}{``\fncte{Fourier coefficient}"}\vx_n$.\\
If the frame is \prope{orthonormal} (giving an \structe{orthonormal basis}), then $\opS=\opSi=\opI$ and 
\\\indentx$\ds\ff(x) = \sum_{n\in\Z} \inprod{\ff}{\vx_n}\vx_n$.\\
In the case of the cardinal series, 
the \fncte{Fourier coefficients}\ifsxref{frames}{def:fcoef} are particularly 
simple---these coefficients are samples of $\ff$ taken at regular intervals \xrefP{thm:sampling}.
In fact, one could represent the coefficients using inner product notation with the 
\structe{Dirac delta distribution} $\delta$ \ifxref{relation}{def:dirac} as
follows:
\\\indentx$\ds\inprod{\ff(x)}{\delta(x-n\tau)} \eqd \int_{\R} \ff(x)\delta(x-n\tau) \dt \eqd \ff(n\tau)$

%--------------------------------------
\begin{definition}
\footnote{
  %\citerp{higgins1985}{56}\\
  \citerpgc{higgins1996}{52}{0198596995}{Definition 6.15}
  %\citerp{hardy1941}{332}
  }
\label{def:PW}
%--------------------------------------
\defbox{\begin{array}{M}
  A function $\ff\in\clFcc$ is in the \hid{Paley-Wiener} class of functions $\ds\spPW^p_\sigma$ if\\
  there exists $\fF\in\spL^p\intoo{-\sigma}{\sigma}$ such that 
  \\\qquad$\ds\ff(x) = \int_{-\sigma}^{\sigma} \fF(\omega)e^{ix\omega}\dw$
  \qquad({\scs $\ff$ has a \prope{bandlimited} Fourier transform $\fF$  with bandwidth $\sigma$})
  \\for $p\in\intco{1}{\infty}$ and $\sigma\in\intoo{0}{\infty}$.
\end{array}}
\end{definition}

%--------------------------------------
\begin{theorem}[\thmd{Paley-Wiener Theorem for Functions}]
\footnote{
  \citerpgc{boas1954}{103}{0123745829}{6.8.1 Theorem of Paley and Wiener},
  \citerpgc{katznelson2004}{212}{0521543592}{7.4 Theorem},
  \citerppgc{zygmund1968v2}{272}{273}{0521890535}{(7$\cdot$2) \scshape Theorem of Paley-Wiener},
  %\citerpg{yosida1971}{161}{3540055061},
  \citerpg{yosida1980}{161}{3540586547},
  \citerpgc{rudinr}{375}{0070542341}{19.3 Theorem},
  \citerpgc{young2001}{85}{0127729550}{Theorem 18}
  }
%--------------------------------------
Let $\ff$ be an \structe{entire function} (the domain of $\ff$ is the entire complex plane $\C$).
Let $\sigma\in\Rp$.
\thmbox{
  \brb{\ff\in\spPW^2_\sigma}
  \iff
  \brb{\begin{array}{F>{\ds}lDD}
    1. & \exists C\in\Rp\st\abs{\ff(z)} \le Ce^{\sigma\abs{z}} & (\prope{exponential type}) & and\\
    2. & \ff\in\spLLR
    %2. & \int_{\R} \abs{\ff(x)}^2 \ds < \infty 
  \end{array}}
  }
\end{theorem}

\includegraphics{graphics/sinc.pdf}\\
%--------------------------------------
\begin{theorem}[\thmd{Cardinal sequence}]
%\begin{theorem}
\footnote{
  \citerpgc{higgins1996}{52}{0198596995}{Definition 6.15},
  \citerc{hardy1941}{\prope{orthonormality}},
  \citerpc{higgins1985}{56}{H1.; historical notes}
  %
  %\citerp{higgins1985}{47}\\
  }
\label{thm:cardinalSeries}
%--------------------------------------
%Let $\ff\in\spLLR$ and $\Ff(\omega)$ have bandwidth $W$ such that
%$\Ff(\omega)=0$ for $\abs{\frac{\omega}{2\pi}}>W$. \\
%If the sample rate $$ then
\thmbox{
  %\mcom{\frac{1}{\tau}\ge 2\sigma}{sample rate $\frac{1}{\tau}$ $\ge$ $2\times$ bandwidth $\sigma$}
  \brb{\frac{1}{\tau}\ge 2\sigma}
  \implies
  \text{The sequence}\quad
  %\setbigleft{\frac{\sin\left[\frac{\pi}{\tau}(x-n\tau)\right]}{\frac{\pi}{\tau}(x-n\tau)}}{n\in\Z}
  \seqnZ{\frac{\sin\left[\frac{\pi}{\tau}(x-n\tau)\right]}{\frac{\pi}{\tau}(x-n\tau)}}
  \text{is an \prope{orthonormal} \structe{basis} for $\spPW_\sigma^2$.}
  }
\end{theorem}


%--------------------------------------
\begin{theorem}[\thmd{Sampling Theorem}]
%\begin{theorem}
\footnote{
  %\citerpc{higgins1985}{56}{H1.; historical notes}\\
  %\citerp{higgins1985}{47}\\
  \citor{etwhittaker1915},
  \citor{kotelnikov1933e},
  \citor{jmwhittaker1935},
  \citePc{shannon1948}{Theorem 13},
  \citePp{shannon1949}{11}
  \citerpg{marks1991}{1}{0387973915},
  \citer{nashed1991},
  \citerpg{higgins1996}{5}{0198596995},
  \citerppgc{young2001}{90}{91}{0127729550}{\scshape The Paley-Wiener Space}.
  The {\em sampling theorem} was ``discovered" and published by multiple people: 
  Nyquist in 1928 (DSP?), 
  Whittaker in 1935 (interpolation theory),
  and Shannon in 1949 (communication theory).
  references: \citerp{mallat}{43},
  \citerp{os}{143}.
  }
\label{thm:t_sampling}
\label{thm:sampling}
%--------------------------------------
%Let $\ff\in\spLLR$ and $\Ff(\omega)$ have bandwidth $W$ such that
%$\Ff(\omega)=0$ for $\abs{\frac{\omega}{2\pi}}>W$. \\
%If the sample rate  then
\thmbox{
  \brb{\begin{array}{F>{\ds}lD}
    1. & \ff\in\spPW_\sigma^2 & and\\
    2. & \frac{1}{\tau}\ge 2\sigma
  \end{array}}
  \qquad\implies\qquad
  \ff(x) = \mcom{\ds\sum_{n=1}^\infty \ff(n\tau) \frac{\sin\brs{\frac{\pi}{\tau}(x-n\tau)}}{\frac{\pi}{\tau}(x-n\tau)}}
                {\hie{Cardinal series}}.
  }
\end{theorem}
\begin{proof}
\[ \mbox{Let }\hspace{3ex}
   \fs(x) \eqd \frac{\sin\left[\frac{\pi}{\tau}x\right]}{\frac{\pi}{\tau}x}
   \iff
   \Fs(\omega) = \left\{\begin{array}{ll}
      \tau & : |f|\le \frac{1}{2\tau} \\
      0 & : \mbox{otherwise}
      \end{array}\right.
\]

\begin{enumerate}
  \item Proof that the set is \prope{orthonormal}: see \citer{hardy1941}

  \item Proof that the set is a \prope{basis}:
    \begin{align*}
      \ff(x)
        &= \int_\omega \Ff(\omega) e^{i\omega t} \dw
        && \text{by \thme{inverse Fourier transform}}
        && \text{\xref{thm:opFTi}}
      \\&= \int_\omega \opT \Ffd(\omega)\Fs(\omega) e^{i\omega t} \dw
        && \text{if $W\le\frac{1}{2T}$}
      \\&= \opT \ffd(x) \conv \fs(x) 
        && \text{by \thme{Convolution theorem}}
        && \text{\xref{thm:conv}}
      \\&= \opT \int_u [\ffd(u)] \fs(x-u) \du
        && \text{by \ope{convolution} definition}
        && \text{\xref{def:conv}} 
      \\&= \opT \int_u \left[ \sum_{n\in\Z}\ff(u)\delta(u-n\tau) \right] \fs(x-u) \du
        && \text{by \ope{sampling} definition}
        && \text{\xref{thm:f_sampling}}
      \\&= \opT \sum_{n\in\Z} \int_u \ff(u)\fs(x-u)\delta(u-n\tau) \du
      \\&= \opT \sum_{n\in\Z} \ff(n\tau)\fs(x-n\tau)
        && \text{by prop. of \fncte{Dirac delta}}
      \\&= \opT \sum_{n\in\Z} \ff(n\tau) 
             \frac{\sin\left[\frac{\pi}{\tau}(x-n\tau)\right]}
                  {\frac{\pi}{\tau}(x-n\tau)       }
        && \text{by definition of $\fs(x)$}
    \end{align*}
\end{enumerate}
\end{proof}


%======================================
\subsection{Sampling}
\label{sec:sampling}
\index{sampling}
%======================================
If $\ffd(x)$ is the function $\ff(x)$ sampled at rate $1/T$, 
then $\Ffd(\omega)$ is simply $\Ff(\omega)$
{\em replicated} every $1/T$ Hertz and {\em scaled} by $1/T$.
This is proven in Theorem~\ref{thm:f_sampling} (next) and 
illustrated in \prefpp{fig:f_sampling}.


\begin{figure}[ht]
\index{nyquist sampling rate}
\setlength{\unitlength}{0.1mm}
\begin{center}
\begin{fsL}
\begin{tabular}{c}
%
\begin{picture}(1550,200)(-750,-50)
  %\graphpaper[10](0,0)(600,200)
  \thicklines
  \put(  40,  70){\makebox(0,0)[lb]{$\Ff(\omega)$} }
  \put(-750,   0){\line( 1,0){1500} }
  \put(   0, -50){\line( 0,1){ 200} }
  \put(-100,   0){\line( 1,1){ 100} }
  \put( 100,   0){\line(-1,1){ 100} }
  \put(- 10, 100){\line( 1,0){  20} }
  \put(- 20, 100){\makebox(0,0)[r]{$A$} }
  \put(-100, -10){\line( 0,1){  20} }
  \put( 100, -10){\line( 0,1){  20} }
  \put(-100, -20){\makebox(0,0)[t]{$-W$} }
  \put( 100, -20){\makebox(0,0)[t]{$+W$} }
  \put( 770,   0){\makebox(0,0)[l]{$f$} }
\end{picture}
\\
\begin{picture}(1550,200)(-750,-50)
  %\graphpaper[10](0,0)(600,200)
  \thicklines
  \put(  20, 140){\makebox(0,0)[lt]{$\Ffd(\omega)$ at sample rate $\frac{1}{\tau}=3W$ (``oversampling")} }
  \multiput(-600,0)(+300,0){5}{
     \put(-100,   0){\line( 1,1){ 100} }
     \put( 100,   0){\line(-1,1){ 100} }
     }
  \put(-750,   0){\line( 1,0){1500} }
  \put(   0, -50){\line( 0,1){ 200} }
  \put(- 10, 100){\line( 1,0){  20} }
  \put(- 20, 100){\makebox(0,0)[r]{$\frac{A}{\tau}$} }
  \put(-100, -10){\line( 0,1){  20} }
  \put( 100, -10){\line( 0,1){  20} }
  \put(-100, -20){\makebox(0,0)[t]{$-W$} }
  \put( 100, -20){\makebox(0,0)[t]{$+W$} }
  \put( 770,   0){\makebox(0,0)[l]{$f$} }

  \put(-600, -10){\line( 0,1){  20} }
  \put(-600, -20){\makebox(0,0)[t]{$\frac{-2}{\tau}$} }
  \put(-300, -10){\line( 0,1){  20} }
  \put(-300, -20){\makebox(0,0)[t]{$\frac{-1}{\tau}$} }
  \put( 300, -10){\line( 0,1){  20} }
  \put( 290, -20){\makebox(0,0)[lt]{$\frac{1}{\tau}=3W$} }
  \put( 600, -10){\line( 0,1){  20} }
  \put( 600, -20){\makebox(0,0)[t]{$\frac{2}{\tau}$} }
\end{picture}
\\
\begin{picture}(1550,200)(-750,-50)
  %\graphpaper[10](0,0)(600,200)
  \thicklines
  \put(  20, 80){\makebox(0,0)[lb]{$\Ffd(\omega)$ at sample rate $\frac{1}{\tau}=2W$ (at Nyquist rate)} }
  \multiput(-600,0)(+200,0){7}{
     \put(-100,   0){\line( 3,2){ 100} }
     \put( 100,   0){\line(-3,2){ 100} }
     }
  \put(-750,   0){\line( 1,0){1500} }
  \put(   0, -50){\line( 0,1){ 200} }
  \put(- 10,  67){\line( 1,0){  20} }
  \put(- 20,  67){\makebox(0,0)[r]{$\frac{A}{\tau}$} }
  \put(-100, -10){\line( 0,1){  20} }
  \put( 100, -10){\line( 0,1){  20} }
  \put(-100, -20){\makebox(0,0)[t]{$-W$} }
  \put( 100, -20){\makebox(0,0)[t]{$+W$} }
  \put( 770,   0){\makebox(0,0)[l]{$f$} }

  \put(-600, -10){\line( 0,1){  20} }
  \put(-600, -20){\makebox(0,0)[t]{$\frac{-3}{\tau}$} }
  \put(-400, -10){\line( 0,1){  20} }
  \put(-400, -20){\makebox(0,0)[t]{$\frac{-2}{\tau}$} }
  \put(-200, -10){\line( 0,1){  20} }
  \put(-200, -20){\makebox(0,0)[t]{$\frac{-1}{\tau}$} }
  \put( 200, -10){\line( 0,1){  20} }
  \put( 190, -20){\makebox(0,0)[lt]{$\frac{1}{\tau}=2W$} }
  \put( 400, -10){\line( 0,1){  20} }
  \put( 400, -20){\makebox(0,0)[t]{$\frac{2}{\tau}$} }
  \put( 600, -10){\line( 0,1){  20} }
  \put( 600, -20){\makebox(0,0)[t]{$\frac{3}{\tau}$} }
\end{picture}
\\
\begin{picture}(1550,200)(-750,-50)
  %\graphpaper[10](0,0)(600,200)
  \thicklines
  \put(  20, 43){\makebox(0,0)[lb]{$\Ffd(\omega)$ at sample rate $\frac{1}{\tau}=W$ (``undersampling")} }
  \multiput(-600,0)(+100,0){13}{
     \put(-100,   0){\line( 3,1){ 100} }
     \put( 100,   0){\line(-3,1){ 100} }
     }
  \put(-750,   0){\line( 1,0){1500} }
  \put(   0, -50){\line( 0,1){ 200} }
  \put(- 10,  33){\line( 1,0){  20} }
  \put(- 20,  33){\makebox(0,0)[r]{$\frac{A}{\tau}$} }
  \put(-100, -10){\line( 0,1){  20} }
  \put( 100, -10){\line( 0,1){  20} }
  %\put(-100, -20){\makebox(0,0)[t]{$-W$} }
  %\put( 100, -20){\makebox(0,0)[t]{$+W$} }
  \put( 770,   0){\makebox(0,0)[l]{$f$} }

  \put(-600, -10){\line( 0,1){  20} }
  \put(-600, -20){\makebox(0,0)[t]{$\frac{-6}{\tau}$} }
  \put(-500, -10){\line( 0,1){  20} }
  \put(-500, -20){\makebox(0,0)[t]{$\frac{-5}{\tau}$} }
  \put(-400, -10){\line( 0,1){  20} }
  \put(-400, -20){\makebox(0,0)[t]{$\frac{-4}{\tau}$} }
  \put(-300, -10){\line( 0,1){  20} }
  \put(-300, -20){\makebox(0,0)[t]{$\frac{-3}{\tau}$} }
  \put(-200, -10){\line( 0,1){  20} }
  \put(-200, -20){\makebox(0,0)[t]{$\frac{-2}{\tau}$} }
  \put(-100, -10){\line( 0,1){  20} }
  \put(-100, -20){\makebox(0,0)[t]{$\frac{-1}{\tau}$} }
  \put( 100, -10){\line( 0,1){  20} }
  \put( 100, -20){\makebox(0,0)[ct]{$\frac{1}{\tau}=W$} }
  \put( 200, -10){\line( 0,1){  20} }
  \put( 200, -20){\makebox(0,0)[t]{$\frac{2}{\tau}$} }
  \put( 300, -10){\line( 0,1){  20} }
  \put( 300, -20){\makebox(0,0)[t]{$\frac{3}{\tau}$} }
  \put( 400, -10){\line( 0,1){  20} }
  \put( 400, -20){\makebox(0,0)[t]{$\frac{4}{\tau}$} }
  \put( 500, -10){\line( 0,1){  20} }
  \put( 500, -20){\makebox(0,0)[t]{$\frac{5}{\tau}$} }
  \put( 600, -10){\line( 0,1){  20} }
  \put( 600, -20){\makebox(0,0)[t]{$\frac{6}{\tau}$} }
\end{picture}
\end{tabular}
\end{fsL}
\end{center}
\caption{
   Sampling in frequency domain
   \label{fig:f_sampling}
   }
\end{figure}

%--------------------------------------
\begin{theorem}
\label{thm:f_sampling}
%--------------------------------------
Let $\ff,\ffd\in\spLLR$ and $\Ff,\Ffd\in\spLLR$ 
be their respective fourier transforms.
Let $\ffd(x)$ be the {\bf sampled} $\ff(x)$ such that
  \[ \ffd(x) \eqd \sum_{n\in\Z} \ff(x) \delta(x-n\tau). \]
\thmbox{
  \mcom{\brb{\ffd(x) \eqd \sum_{n\in\Z} \ff(x) \delta(x-n\tau)}}{\fnctd{Shah function}}
  \quad\implies\quad
  \brb{\Ffd(\omega) = \frac{2\pi}{\tau}\sum_{n\in\Z} \Ff\brp{\omega -\frac{2\pi}{\tau}n}}
   }
\end{theorem}
\begin{proof}
\begin{align*}
   \Ffd(\omega) 
     &\eqd \int_t \ffd(x) e^{-i\omega t} \dt
   \\&=    \int_t \left[\sum_{n\in\Z} \ff(x) \delta(x-n\tau)\right] e^{-i\omega t} \dt
   \\&=    \sum_{n\in\Z} \int_t \ff(x) \delta(x-n\tau) e^{-i\omega t} \dt
   \\&=    \sum_{n\in\Z} \ff(n\tau) e^{-i\omega n\tau}
     &&    \text{by definition of $\delta$}
     &&    \text{\ifxref{relation}{def:dirac}}
   \\&=    \frac{2\pi}{\tau} \sum_{n\in\Z} \Ff\left(\omega + \frac{2\pi}{\tau}n \right)
     &&    \text{by \thme{IPSF}}
     &&    \text{\xref{thm:ipsf}}
   \\&=    \frac{2\pi}{\tau} \sum_{n\in\Z} \Ff\left(\omega-\frac{2\pi}{\tau}n\right)
\end{align*}
\end{proof}

Suppose a waveform $\ff(x)$ is sampled at every time $T$
generating a sequence of sampled values $\ff(n\tau)$.
Then in general, we can {\em approximate} $\ff(x)$ by 
using interpolation between the points $\ff(n\tau)$.
Interpolation can be performed using several interpolation techniques. %:\\
%\begin{tabular}{lll}
%  $\imark$ & 0th order interpolation: & convolution with rectangular functions \\
%  $\imark$ & 1st order interpolation: & convolution with triangular functions  \\
%  $\imark$ & 3rd order interpolation: & cubic splines \footnotemark            \\
%  $\imark$ & lowpass smoothing        & convolution with the ``sinc" function 
%\end{tabular}
%\footnotetext{
%  A {\em cubic spline} is a third order polynomial $ax^3+bx^2+cx+d$.
%  Such a polynomial has 4 {\em degrees of freedom}. 
%  Two of these degrees of freedom are used to make the polymomial match
%  at the interpolation interval endpoints.
%  The other two degrees of freedom are used to make the slopes of the 
%  polynomial match at the interval endpoints.
%  }

In general all techniques lead only to an approximation of $\ff(x)$.
However, if $\ff(x)$ is \prope{bandlimited} with bandwidth 
$W\le\frac{1}{2T}$,
then $\ff(x)$ is {\em perfectly reconstructed} (not just approximated)
from the sampled values $\ff(n\tau)$ \xref{thm:sampling}.
 %using ``\hie{lowpass smoothing}" (the last technique in the list).
%This perfect reconstruction is demonstrated in \prefp{thm:sampling}.











