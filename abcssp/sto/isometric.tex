%============================================================================
% LaTeX File
% Daniel J. Greenhoe
%============================================================================

%=======================================
\subsection{Isometric spaces}
%=======================================
%--------------------------------------
\begin{definition}
\footnote{
  \citerpc{thron1966}{153}{definition 19.4},
  \citerpgc{giles1987}{124}{0521359287}{Definition 6.22},
  \citerpgc{khamsi2001}{15}{0471418250}{Definition 2.4},
  \citerpg{kubrusly2001}{110}{0817641742}
  }
\label{def:isometry}
\label{def:isometric}
%--------------------------------------
Let $\metspaceX$ and $\opair{\setY}{\metrican}$ be \structe{distance space}s \xref{def:distance}.
\\\defboxt{
  The function $\ff\in\clFxy$ is an \hid{isometry} on $\clF{\metspaceX}{\opair{\setY}{\metrican}}$ if
    \quad$\ds\metric{x}{y}= \metrica{\ff(x)}{\ff(y)} \quad\scy \forall x,y\in\setX$.
  \\
  The spaces $\metspaceX$ and $\opair{\setY}{\metrican}$ are \propd{isometric} if there exists an isometry on $\clF{\metspaceX}{\opair{\setY}{\metrican}}$.
}
\end{definition}

%--------------------------------------
\begin{theorem}
\footnote{
  \citerpc{thron1966}{153}{theorem 19.5}
  }
\label{thm:isometry_inv}
%--------------------------------------
Let $\metspaceX$ and $\opair{\setY}{\metrican}$ be \structe{distance space}s.\\
Let $\ff$ be a function in $\clFxy$ and $\ffi$ its inverse in $\clFyx$.
\thmbox{
  \brb{\text{$\ff$ is an \propd{isometry} on $\clF{\metspaceX}{\opair{\setY}{\metrican}}$}}
  \qquad\iff\qquad
  \brb{\text{$\ffi$ is an \propd{isometry} on $\clF{\opair{\setY}{\metrican}}{\metspaceX}$}}
  }
\end{theorem}


If a function $\metrican$ is a \structe{metric} and a function $\fg$ is \prope{injective}, then 
the function $\metric{x}{y}\eqd\metrica{\fg(x)}{\fg(y)}$ is also a \structe{metric} (next theorem).
For an example of this with $\metrica{x}{y}\eqd\abs{x-y}$ and $\fg\eqd\arctan(x)$, see \prefpp{ex:ms_atan}.

%---------------------------------------
\begin{theorem}[\thm{Pullback metric}/\thm{$\fg$-transform metric}]
\footnote{
  \citerpg{deza2009}{81}{3642002331}
  %\url{http://groups.google.com/group/sci.math/msg/6d091dfa5061cc08}
  }
\label{thm:met_sumpf}
\label{thm:pullback}
%---------------------------------------
Let $\setX$ and $\setY$ be sets.
%Let $\fg$ be a function in $\clFxy$.
\thmbox{
  \brb{\begin{array}{FMDD}
      (1). & $\metrican$ is a \fncte{metric} on $\setY$ & \xref{def:metric}   & and 
    \\(2). & $\fg$ is a \structe{function} in $\clFxy$  & \xref{def:function} & and
    \\(3). & $\fg$ is \prope{injective}                 & \xref{def:injective}&
  \end{array}}
  \quad\implies\quad
  \brb{\begin{array}{MC}
    $\metric{x}{y} =  \metrica{\fg(x)}{\fg(y)}$ &\forall x,y\in\setX \\
    is a \fncte{metric} on $\setX$
  \end{array}}
  }
\end{theorem}
\begin{proof}
\begin{align*}
  \intertext{1. Proof that ${x}={y}\implies\metric{x}{y}=0$:}
  \metric{x}{y}
    &\eqd \metrica{\fphi(x)}{\fphi(y)}
    &&    \text{by definition of $\metricn$}
  \\&=    \metrica{\fphi(x)}{\fphi(x)}
    &&    \text{by ${x}={y}$ hypothesis}
  \\&=    0
    &&    \text{by \prope{nondegenerate} property of metric $\metrican$ \xref{def:metric}}
  \\&=    0
  \\
  \intertext{2. Proof that ${x}={y}\impliedby\metric{x}{y}=0$:}
  0
    &=    \metric{x}{y}
    &&    \text{by right hypothesis}
  \\&\eqd  \metrica{\fphi(x)}{\fphi(y)}
    &&    \text{by definition of $\metricn$}
  \\\implies \metrica{\fphi(x)}{\fphi(y)}&=0 \text{ for $n=1,2,\ldots,\xN$}
    &&    \text{because $\metrican$ is non-negative}
  \\\implies {x}&={y}
    &&    \text{by \prope{injective} hypothesis (3)}
  \\
  \intertext{3. Proof that $\metric{x}{y}\le\metric{z}{x}+\metric{z}{y}$:}
  \metric{x}{y}
    &\eqd  \metrica{\fphi(x)}{\fphi(y)}
    &&    \text{by definition of $\metricn$}
  \\&\le   \big( \metrica{\fphi(x)}{\fphi(z)} + \metric{\fphi(z)}{\fphi(y)} \big)
    &&    \text{by \prope{subadditive} property of $\metrican$ \xref{def:metric}}
  \\&=     \metrica{\fphi(z)}{\fphi(x)} +  \metrica{\fphi(z)}{\fphi(y)} 
    &&    \text{by \prope{symmetry} property of metric $\metrican$ \xref{def:metric}}
  \\&\eqd \metric{z}{x} + \metric{z}{y}
    &&    \text{by definition of $\metricn$}
\end{align*}
\end{proof}

%---------------------------------------
\begin{example}[\exmd{Inverse tangent metric}]
\label{ex:ms_atan}
\footnote{
  \citerp{copson1968}{25},
  \citerpg{khamsi2001}{14}{0471418250}
  }
\index{metrics!inverse tangent}
\index{inverse tangent metric}
%---------------------------------------
%\\\begin{minipage}{3\tw/4-3mm}%
%Let $\setX$ be a set and
%$\vx\eqd\tuplexn{x_k\in\setX}$ and 
%$\vy\eqd\tuplexn{y_k\in\setX}$
%be sequences on $\setX$.
Let $\seqn{x_1,x_2,\ldots,x_\xN}$ and $\seqn{y_1,y_2,\ldots,y_\xN}$ 
be points in $\R^\xN$.
\exbox{
  \brb{
    \metric{\seqn{x_1,x_2,\ldots,x_\xN}}{\seqn{y_1,y_2,\ldots,y_\xN}}
    \eqd \sum_{n=1}^\xN \abs{\arctan x_n - \arctan y_n}
  }
  %\quad\implies\quad
  \qquad
  \text{is a \structe{metric}}.
  %\\
  %2. & $\metricn$ is \emph{not generated by a norm}.
  %\\
  %3. & $\ball{0}{1}$ in $\opair{\R^n}{\metricn}$ is \emph{not convex}.
  }
%\end{minipage}%
%\hfill%
%\begin{minipage}{\tw/4}%
%  \begin{center}
%  \begin{fsL}
%  \setlength{\unitlength}{\tw/300}
%  \begin{picture}(300,300)(-130,-130)%
%    %{\color{graphpaper}\graphpaper[10](-150,-150)(300,300)}%
%    \thicklines%
%    \color{axis}%
%      \put(-130,   0){\line(1,0){260} }%
%      \put(   0,-130){\line(0,1){260} }%
%      \put( 140,   0){\makebox(0,0)[l]{$x$}}%
%      \put(   0, 140){\makebox(0,0)[b]{$y$}}%
%      \put(-100, -10){\line(0,1){20} }%
%      \put( 100, -10){\line(0,1){20} }%
%      \put( -10,-100){\line(1,0){20} }%
%      \put( -10, 100){\line(1,0){20} }%
%      \put( -15, 100){\makebox(0,0)[r]{$+1$} }%
%      \put( -15,-100){\makebox(0,0)[r]{$-1$} }%
%      \put(-100, -15){\makebox(0,0)[t]{$-1$} }%
%      \put( 100, -15){\makebox(0,0)[t]{$+1$} }%
%    \color{blue}%
%      \qbezier( 100,0)(0,0)(0, 100)%
%      \qbezier( 100,0)(0,0)(0,-100)%
%      \qbezier(-100,0)(0,0)(0,-100)%
%      \qbezier(-100,0)(0,0)(0, 100)%
%    \color{red}%
%      \qbezier[20]( 75,0)(37.5,37.5)(0, 75)%
%      \put( 10,90){\makebox(0,0)[bl]{$\scriptstyle \lambda\vv_\circ+(1-\lambda)\vw_\circ$} }%
%      \put( 90,90){\vector(-1,-1){50} }%
%    %\color{black}%
%      %\put(-100,-100){\vector(1,1){125} }%
%      %\put(-120, -110){\makebox(0,0)[tl]{$\scriptstyle x^2+y^2-2xy-2x-2y+1=0$} }%
%      %\put(-120, -140){\makebox(0,0)[tl]{\scriptsize(parabolic equation)} }%
%  \end{picture}
%  \end{fsL}
%  \end{center}
%\end{minipage}%
\end{example}
\begin{proof}
\begin{enume}
  \item The function $\metric{x}{y}\eqd\abs{x-y}$ is a \structe{metric} (the \hie{usual metric}, \prefp{def:d_usual}).
  \item The function  $\fg(x)\eqd\arctan(x)$ is \prope{injective} in $\clFrr$.
  \item Therefore, $\metricn$ is a \structe{Pullback metric} (or \structe{$\fg$-transform metric}), 
        and by \prefpp{thm:met_sumpf}, $\metricn$ is a \structe{metric}.
\end{enume}
\end{proof}

