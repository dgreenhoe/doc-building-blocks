%============================================================================
% Daniel J. Greenhoe
% XeLaTeX File
%============================================================================

%======================================
\chapter{Vanishing Moments Constraint}
\label{chp:vanish}
%======================================
%Besides the \prope{orthonormality} constraint \xref{def:ows} and
%the resulting \prope{partition of unity} constraint \xref{def:pun},
One of the most common wavelet design constraints is
the number of \prope{vanishing moments} $p$ on the wavelet function such that
  \\\indentx$\ds\inprod{\fpsi(x)}{x^n}=0 \qquad\text{for}\quad n=0,1,2,\ldots,p-1.$\\
This chapter investigates wavelet design under the vanishing moment constraint.

%=======================================
\section{Moments}
\label{sec:vanish}
%=======================================
%--------------------------------------
\begin{definition}
\footnote{
  \citerpp{jawerth}{16}{17},
  \citerp{sweldens93}{2},
  \citerp{vidakovic}{83}
  %\citerp{goswami}{102}    \\
  %\citerpp{mallat}{241}{243}
  }
\label{def:Mn}
%--------------------------------------
\defboxt{
  The quantity $M_n$ is the \hid{$n$th moment} of a function $\ff(x)\in\spLLR$ if
  \\\indentx$\ds M_n \eqd \int_{\R} x^n \ff(x) \dx \qquad \text{for }n\in\Znn$.
  }
\end{definition}

%--------------------------------------
\begin{definition}
\footnote{
  \citerp{mallat}{241}
  }
\label{def:vanish}
%--------------------------------------
Let $M_n$ be the \fncte{$n$th moment} \xref{def:Mn} of a function $\ff(x)\in\spLLR$ \xref{def:spLLR}.
\defboxt{
  The function $\ff(x)$ has an \propd{$n$th vanishing moment} if $M_n=0$.\\
  The function $\ff(x)$ has \propd{$p$ vanishing moments} if $M_0=M_1=\cdots=M_{p-1}=0$ and $M_p\neq0$.
  }
\end{definition}

Let $\wavsys$ be a \structe{wavelet system} \xref{def:wavsys}.
Let $\opP$ be the \fncte{projection operator} that generates the subspace $\spV_0$
and $\opQ$ the \fncte{projection operator} that generates the subspace $\spW_0$.
The number of {\em vanishing moments} of a wavelet is important
when using wavelets for the analysis and synthesis of a function $\ff(x)$ as in
\[ \ff(x) =
   \mcom{
     \sum_{n\in\Z} \mcom{\inprod{\ff(x)}{\opTrn^n\fphi(x)}}{Fourier coefficient} \opTrn^n\fphi(x)
     }{\structe{Fourier expansion} \xref{def:fex} of $\opP\ff(x)$}
   +
   \mcom{
     \sum_{k=0}^\infty \sum_{n\in\Z}\mcom{\inprod{\ff(x)}{\opDil^k\opTrn^n\fpsi(x)}}{Fourier coefficient} \opDil^k\opTrn^n\fpsi(x)
     }{\structe{Fourier expansion} \xref{def:fex} of $\opQ\ff(x)$}
\]
All differentiable functions have a \fncte{Taylor polynomial expansion}\ifsxref{polynom}{thm:taylor}.
The \fncte{ Fourier coefficients} \xref{def:fcoef} of this polynomial decrease as the order of each term increases.
The inner-product of the function and a wavelet with many vanishing moments
results in a smaller value, giving a better approximation for a fixed
number of Fourier coefficients.

%--------------------------------------
\begin{theorem}
\label{thm:vanish_poly}
%--------------------------------------
\thmbox{
  \brbr{\begin{array}{FMD}
    1. & a function $\ff(x)$ has $p$ vanishing moments   & and \\
    2. & a polynomial $\fq(x)$ is of order $p-1$ or less
  \end{array}}
  \qquad\implies\qquad
  \mcoml{\inprod{\ff}{\fq}=0}
       {$\ff$ is \prope{orthogonal} to $\fq$}
  }
\end{theorem}
\begin{proof}
\begin{align*}
  \inprodr{\ff(x)}{\sum_{n=0}^{p-1}a_n x^n}
    &= \sum_{n=0}^{p-1}a_n^\ast \inprod{\ff(x)}{x^n}
     = \sum_{n=0}^{p-1}a_n^\ast \cdot 0
     = 0
\end{align*}
\end{proof}

%=======================================
\section {Vanishing moments and the wavelet function}
%=======================================
The number of vanishing moments a wavelet has is closely related
to how the derivatives of the Fourier transforms of the wavelet
and wavelet coefficients behave (next theorem).

%--------------------------------------
\begin{theorem}
\footnote{
  \citerpp{vidakovic}{82}{83},
  \citerpp{mallat}{241}{242}
  }
\label{thm:vanish}
\index{vanishing moments}
%--------------------------------------
Let $\wavsys$ be a \hi{wavelet system}.
Let $\Fpsi(\omega)$ be the \fncte{Fourier transform} \xref{def:opFT} of $\fpsi(x)$.
Let $\Dh(\omega)$ and $\Dg(\omega)$ be the \fncte{DTFT}s \xref{def:dtft} of $\seqn{h_n}$ and $\seqn{g_n}$, respectively.
\thmbox{
  \brb{\begin{array}{F@{\qquad}>{\ds}r c >{\ds}rcl@{\qquad}F}
    (A) & \mcom{\inprod{\fpsi(x)}{x^n}=0}{$\fpsi(x)$ has a vanishing $n$th moment}
      &\iff&
    \left.\opddwn  \Fpsi(\omega)\right|_{\omega=0} &=& 0         &(B)
      \\&&\iff&
    \left.\opddwn  \Dg(\omega)\right|_{\omega=0}   &=& 0         & (C)
      \\&&\iff&
    \sum_{k\in\Z} k^n  g_n &=& 0 & (D)
  \end{array}}
  \qquad\scy\forall n\in\Znn
  }
\end{theorem}
\begin{proof}
\begin{enumerate}
  \item Proof that $(A)\iff(B)$: by \prefp{lem:vanish_deriv}

  \item Proof that $(B)\implies(C)$:
    \begin{align*}
      0
        &= \left.\opddwn  \Fpsi(2\omega)\right|_{\omega=0}
        && \text{by left hypothesis}
      \\&= \left.\opddwn  \Dg(\omega)\Fphi(\omega)\right|_{\omega=0}
        && \text{by \prefp{prop:vsmra_real_Fpsi}}
      \\&= \left.\sum_{k=0}^n {n \choose k} \Dg^{(k)}(\omega)\Fphi^{(n-k)}(\omega)\right|_{\omega=0}
        && \text{by \thme{Leibnitz GPR}}
        && \text{\xref{lem:LGPR}}
      \\&= \brs{
           \Dg^{(n)}(\omega)\Fphi(\omega)
           +\sum_{k=0}^{n-1} {n \choose k} \Dg^{(k)}(\omega)\Fphi^{(n-k)}(\omega)
           }_{\omega=0}
      \\&= \Dg^{(n)}(0) \Fphi(0)
           +\sum_{k=0}^{n-1} {n \choose k} \Dg^{(k)}(0)\Fphi^{(n-k)}(0)
    \end{align*}


  \[\begin{array}{rrcll}
    \implies&  \Dg^{(0)}(0) &=& 0   \\
    \implies&  \Dg^{(1)}(0) &=& 0   \\
    \implies&  \Dg^{(2)}(0) &=& 0   \\
    \implies&  \Dg^{(3)}(0) &=& 0   \\
    \implies&  \Dg^{(4)}(0) &=& 0   \\
    \vdots  &  \mc{1}{c}{\vdots}    \\
    \implies&  \Dg^{(n)}(0) &=& 0   \\
    \implies&  \Dg^{(n)}(0) &=& 0  & \text{for $n=0,1,2,\ldots$}
  \end{array}\]

  \item Proof that $(B)\impliedby(C)$:
    \begin{align*}
      \left.\opddwn  \Fpsi(2\omega)\right|_{\omega=0}
        &= \left.\opddwn  \Dg(\omega)\Fphi(\omega)\right|_{\omega=0}
        && \text{by \prefp{prop:vsmra_real_Fpsi}}
      \\&= \left.\sum_{k=0}^n {n \choose k} \Dg^{(k)}(\omega)\Fphi^{(n-k)}(\omega)\right|_{\omega=0}
        && \text{by \thme{Leibnitz GPR}}
        && \text{\xref{lem:LGPR}}
      \\&= \left.\sum_{k=0}^n {n \choose k} \;0\; \Fphi^{(n-k)}(\omega)\right|_{\omega=0}
        && \text{by right hypothesis}
      \\&= 0
    \end{align*}

  \item Proof that $(C)\iff(D)$: by \prefp{thm:dtft_ddw}
\end{enumerate}
\end{proof}



%=======================================
\section {Vanishing moments and the scaling function}
%=======================================
%=======================================
\subsection{Results}
%=======================================
%--------------------------------------
\begin{lemma} %[\thmd{Strang-Fix condition}]
\label{lem:vanish_hphi}
%\footnote{
%  \citerpp{jawerth}{16}{17}    \\
%  \citerp{sweldens93}{2} \\
%  \citerp{vidakovic}{83}  \\
%  \citerp{goswami}{102}    \\
%  \citerpp{mallat}{241}{243}
%  }
%--------------------------------------
Let $\wavsys$ be a wavelet system and $M_n$ the \hie{$n$th moment} \xref{def:Mn} of $\fphi(x)$.
\lembox{
  \left.\opddwn  \Dh(\omega)\right|_{\omega=\pi}=0
  \qquad\implies\qquad
  \left.\opddwn  \Fphi(\omega)\right|_{\omega=2\pi k}= \cft(-i)M_n\delta_k
  }
\end{lemma}
\begin{proof}
\begin{enumerate}
  \item By \prefp{lem:har_moment}, $\left. \opddwn  \Fphi(\omega)\right|_{\omega=0}= \cft (-i)^n M_n$.

  \item Then,
    \begin{align*}
        \left. \opddwn  \Fphi(\omega)                           \right|_{\omega=2\pi k}
          &= \left. \opddwn
                   \Dh\left(\frac{\omega}{2}\right)
                   \Fphi\left(\frac{\omega}{2}\right)
             \right|_{\omega=2\pi k}
          && \text{by \prefp{prop:vsmra_real_Fpsi}}
        \\&= \left. \sum_{m=0}^n {n\choose m}
                    \Dh  ^{(m  )}\left(\frac{\omega}{2}\right)
                    \Fphi^{(n-m)}\left(\frac{\omega}{2}\right)
             \right|_{\omega=2\pi k}
        && \text{by \thme{Leibnitz GPR}}
        && \text{\xref{lem:LGPR}}
        \\&= \sum_{m=0}^n {n\choose m} \mcom{\Dh^{(m)}(\pi k)}{0 for $k$ odd} \Fphi^{(n-m)}(\pi k)
        \\&= \brbl{\begin{array}{>{\ds}lM}
               \sum_{m=0}^n {n\choose m} \Dh^{(m)}(\pi k) \Fphi^{(n-m)}(\pi k)   & for $k$ even \\
               0                                                                 & for $k$ odd
             \end{array}}
          && \text{by left hypothesis}
    \end{align*}

  \item And so
    \begin{align*}
        \Fphi^{(n)}(2\pi0) &= (-i)^n M_n
        \\
        \Fphi^{(n)}(2\pi1) &= 0
        \\
        \Fphi^{(n)}(2\pi2)
          &= \ds \sum_{m=0}^n {n\choose m} \Dh^{(m)}(2\pi1) \mcom{\Fphi^{(n-m)}(2\pi1)}{$0$}
           = 0
        \\
        \Fphi^{(n)}(2\pi3) &= 0
        \\
        \Fphi^{(n)}(2\pi4)
          &= \ds \sum_{m=0}^n {n\choose m} \Dh^{(m)}(2\pi2) \mcom{\Fphi^{(n-m)}(2\pi2)}{$0$}
           = 0
        \\
        \Fphi^{(n)}(2\pi5) &= 0
        \\
        \Fphi^{(n)}(2\pi6)
          &= \ds \sum_{m=0}^n {n\choose m} \Dh^{(m)}(2\pi3) \mcom{\Fphi^{(n-m)}(2\pi3)}{$0$}
           = 0
        \\
        \vdots & \vdots
        \\
        \Fphi^{(n)}(2\pi k) &= \kdelta_k(-i)^n M_n
    \end{align*}
\end{enumerate}
\end{proof}

%\prefpp{thm:vanish} gives alternative forms for expressing
%the property of vanishing moments of a wavelet.
%The \structe{CQF} \xref{def:cqf} constraint on the \structe{wavelet coefficients} \xref{def:gn}
%establishes a relationship between wavelets and their associated scaling functions. % ($\spV_0\perp\spW_0$).
%%The definition of a \structe{wavelet system} \xref{def:wavsys} establishes a relationship between wavelets and their associated scaling functions ($\spV_0\perp\spW_0$).
%With this relationship, the vanishing moments of a wavelet are also related
%to forms of expression of the scaling function:
%--------------------------------------
\begin{theorem}
\footnote{
  \citerpp{vidakovic}{82}{83},
  \citerpp{mallat}{241}{242}
  }
\index{vanishing moments}
\label{thm:gh_cqf}
%--------------------------------------
Let $\wavsys$ be a \structe{wavelet system} \xref{def:wavsys}.
%Let $\Fpsi(\omega)$ be the \fncte{Fourier transform} \xref{def:opFT} of $\fpsi(x)$.
Let $\Dh(\omega)$ and $\Dg(\omega)$ be the \ope{DTFT}s \xref{def:dtft} of $\seqn{h_n}$ and $\seqn{g_n}$, respectively.
\thmboxt{
  Let $g_n = \pm(-1)^n h^\ast_{N-n}$ (\prope{CQF condition}, {\scs\prefp{def:cqf}}). Then\\
  %\brb{\begin{array}{FMD}
  %  A. & $\setn{\opTrn^n\fphi(x)}$ is \prope{orthonormal} & and \\
  %  B. & $\spV_0\orthog\spW_0$                            & and \\
  %  C. & $g_n = (-1)^n h^\ast_{N-n}$                      &
  %\end{array}}
  %\implies
  $\brb{\begin{array}{>{\ds}rc >{\ds}rcl@{\qquad}D}
       \mcom{\inprod{\fpsi(x)}{x^n}=0}{(1) \prope{$n$th vanishing moment}}
         &\iff&         \left.\opddwn  \Dg(\omega)\right|_{\omega=0}     &=& 0 & (2)
       \\&\iff&         \left.\opddwn  \Dh(\omega)\right|_{\omega=\pi}   &=& 0 & (3)
       \\&\iff&         \sum_{k\in\Z} (-1)^k k^n  h_k                    &=& 0 & (4)
       \\&\iff&         \sum_{k\in\Z} k^n  g_k                           &=& 0 & (5)
  \end{array}}
  \qquad\scy\forall n\in\Znn$
  }
\end{theorem}
\begin{proof}
\begin{enumerate}
  \item Proof that $(1)\iff (2)$: by \prefpp{thm:vanish}

  \item Proof that $(2) \iff (3)$: by \prefp{thm:cqf_ddw}

  %\item Proof that $(2) \implies (3)$:
  %\item lemma: Proof that $\Dg(\pi)=\sqrt{2}$: \label{item:gh_cqf_Fgpi} % proof moved to wavstruct
  %\item Proof that $(2) \implies (3)$: % note: this requires orthogonality
  %  \begin{align*}
  %      0
  %        &= \left. \opddwn  \; 0 \right|_{\omega=\pi}
  %      \\&= \left. \opddwn  \; 2\Shs(\omega) \right|_{\omega=\pi}
  %        && \text{by (B) and \prefp{lem:Swfgo}}
  %      \\&= \left. \opddwn  \brs{
  %           \Dh(\omega)\Dg^\ast(\omega)\Sphi(\omega)+\Dh(\omega+\pi)\Dg^\ast(\omega+\pi)\Sphi(\omega)
  %           }\right|_{\omega=\pi}
  %        && \text{by \prefp{thm:gen_quadcon}}
  %      \\&= \left. \opddwn  \brs{
  %           \Dh(\omega)\Dg^\ast(\omega)+\Dh(\omega+\pi)\Dg^\ast(\omega+\pi)
  %           }\right|_{\omega=\pi}
  %        && \text{by (A) and \prefp{lem:Swfgo}}
  %      \\&= \left.\brs{
  %           \sum_{k=0}^n {n\choose k} \Dh^{(k)}(\omega)\Dg^{(n-k)\ast}(\omega)
  %          +\sum_{k=0}^n {n\choose k} \Dh^{(k)}(\omega+\pi)\Dg^{(n-k)\ast}(\omega+\pi)
  %           }\right|_{\omega=\pi}
  %        && \text{by Leibnitz GPR \prefpo{lem:LGPR}}
  %      \\&= \sum_{k=0}^n {n\choose k} \Dh^{(k)}(\pi)\Dg^{(n-k)\ast}(\pi)
  %          +\sum_{k=0}^n {n\choose k} \Dh^{(k)}(0)\Dg^{(n-k)\ast}(0)
  %        && \text{by \prefp{prop:dtft_2pi}}
  %      \\&= \sum_{k=0}^n {n\choose k} \Dh^{(k)}(\pi)\Dg^{(n-k)\ast}(\pi)
  %          +\sum_{k=0}^n {n\choose k} \Dh^{(k)}(0)\; \cancelto{0}{\Dg^{(n-k)\ast}(0)}
  %        && \text{by (2)}
  %      %\\&= \sum_{k=0}^n {n\choose k} \Dh^{(k)}(\pi)\Dg^{(n-k)\ast}(\pi)
  %      \\&= \Dh^{(n)}(\pi)\Dg^{\ast}(\pi)
  %          +\sum_{k=0}^{n-1} {n\choose k} \Dh^{(k)}(\pi)\Dg^{(n-k)\ast}(\pi)
  %      \\&= \sqrt{2}\;\Dh^{(n)}(\pi)
  %          +\sum_{k=0}^{n-1} {n\choose k} \Dh^{(k)}(\pi)\Dg^{(n-k)\ast}(\pi)
  %        && \text{by \prefp{item:gh_cqf_Fgpi}}
  %  \end{align*}
  %
  %\[\begin{array}{r rcl l}
  %  \implies & \Dh^{(0)}(\pi) &=& 0 \\
  %  \implies & \Dh^{(1)}(\pi) &=& 0 \\
  %  \implies & \Dh^{(2)}(\pi) &=& 0 \\
  %  \implies & \Dh^{(3)}(\pi) &=& 0 \\
  %  \implies & \Dh^{(4)}(\pi) &=& 0 \\
  %  \vdots   & \mc{1}{c}{\vdots}    \\
  %  \implies & \Dh^{(n)}(\pi) &=& 0 \\
  %  \implies & \Dh^{(n)}(\pi) &=& 0 & \text{for $n=0,1,2,\ldots$}
  %\end{array}\]

  %\item Proof that (2)$\impliedby$(3):
  %      \begin{align*}
  %          0
  %            &= \brlr{\opddwn  \; 0}_{\omega=0}
  %          \\&= \brlr{\opddwn  \; 2\Shs(\omega)}_{\omega=0}
  %            && \text{by (B) and \prefp{lem:Swfgo}}
  %          \\&= \brlr{\opddwn  \brs{
  %                 \Dh(\omega)\Dg^\ast(\omega)\Sphi(\omega)+\Dh(\omega+\pi)\Dg^\ast(\omega+\pi)\Sphi(\omega)
  %                 }}_{\omega=0}
  %            && \text{by \prefp{thm:gen_quadcon}}
  %          \\&= \brlr{\opddwn  \brs{
  %                 \Dh(\omega)\Dg^\ast(\omega)+\Dh(\omega+\pi)\Dg^\ast(\omega+\pi)
  %                 }}_{\omega=0}
  %            && \text{by (A) and \prefp{lem:Swfgo}} %{; see also \prefp{thm:quadcon}}
  %          \\&= \left.\brs{
  %               \sum_{k=0}^n {n\choose k} \Dh^{(k)}(\omega)\Dg^{(n-k)\ast}(\omega)
  %              +\sum_{k=0}^n {n\choose k} \Dh^{(k)}(\omega+\pi)\Dg^{(n-k)\ast}(\omega+\pi)
  %               }\right|_{\omega=0}
  %            && \text{by Leibnitz GPR \prefpo{lem:LGPR}}
  %          \\&= \left.\brs{
  %               \sum_{k=0}^n {n\choose k} \Dh^{(k)}(\omega)\Dg^{(n-k)\ast}(\omega)
  %              +\sum_{k=0}^n {n\choose k} \;0\;\Dg^{(n-k)\ast}(\omega+\pi)
  %               }\right|_{\omega=0}
  %            && \text{by (3)}
  %          \\&= \left.
  %               \sum_{k=0}^n {n\choose k} \Dh^{(k)}(\omega)\Dg^{(n-k)\ast}(\omega)
  %               \right|_{\omega=0}
  %          \\&= \Dh^{(0)}(0)\Dg^{(n)\ast}(0)
  %              +\sum_{k=1}^n {n\choose k} \Dh^{(k)}(0)\Dg^{(n-k)\ast}(0)
  %          \\&= \sqrt{2}\;\Dg^{(n)\ast}(0)
  %              +\sum_{k=1}^n {n\choose k} \Dh^{(k)}(0)\Dg^{(n-k)\ast}(0)
  %            && \text{by \prefp{thm:admiss}}
  %            %&& \text{by admissibility condition (\pref{thm:admiss})}
  %      \end{align*}
  %
  %    \[\begin{array}{r rcl l}
  %      \implies & \Dg^{(0)}(0) &=& 0   \\
  %      \implies & \Dg^{(1)}(0) &=& 0   \\
  %      \implies & \Dg^{(2)}(0) &=& 0   \\
  %      \implies & \Dg^{(3)}(0) &=& 0   \\
  %      \implies & \Dg^{(4)}(0) &=& 0   \\
  %      \vdots   & \mc{1}{c}{\vdots}    \\
  %      \implies & \Dg^{(n)}(0) &=& 0
  %    \end{array}\]

\item Proof that (3)$\iff$(4): by \prefp{thm:dtft_ddw}
\item Proof that (2)$\iff$(5): by \prefp{thm:dtft_ddw}; see also \prefp{thm:vanish}.
\item Proof that CQF$\notimpliedby$(1): by \prefp{thm:dtft_ddw}; see also \prefp{thm:vanish}.

\end{enumerate}
\end{proof}

\begin{figure}
  \centering%
  \begin{tabular*}{\textwidth-40mm}{@{\extracolsep{\fill}}cc}%
    \includegraphics{../common/math/graphics/pdfs/bspline_pz.pdf}&\includegraphics{../common/math/graphics/pdfs/Dp_pz.pdf}\\%
    order $p$ Cardinal B-spline&Daubechies-p%
  \end{tabular*}%
  \caption{
     Zero locations for B-cardinal spline $N_p(x)$ and Daubechies-p scaling coefficients
     \label{fig:vanish_spline_zero}
     }
\end{figure}
These relationships imply that a compactly supported scaling function
with $p$ vanishing moments has $p$ zeros at $z=-1$
(due to the spline component) as illustrated in \pref{fig:vanish_spline_zero}.

%=======================================
\subsection{Scaling function and polynomials}
%=======================================
Linear combinations of $\setxZ{\opTrn^n\fphi(x)}$ of a wavelet system with $p$ vanishing moments
can perfectly represent any polynomial of order $p-1$ or less \xref{thm:sfix_poly}.
%The key to proving this is the \prope{Strang-Fix condition} \xref{lem:sfixceq}.
\prefp{ex:sfixc5} illustrates this using the Strang-Fix condition \xref{lem:sfixc}
to represent any polynomial of dgree $4$ or less.
%--------------------------------------
\begin{lemma} %[\thmd{Strang-Fix condition}]
\label{lem:sfixc}
\footnote{
  \citerpp{jawerth}{16}{17},
  \citerp{sweldens93}{2},
  \citerp{vidakovic}{83},
  \citerp{goswami}{102},
  \citerpp{mallat}{241}{243}
  }
%--------------------------------------
Let $\wavsys$ be a wavelet system and $M_n$ the \hie{$n$th moment} \xref{def:Mn} of $\fphi(x)$.
\lembox{
  \brb{\begin{array}{M}
    $\fpsi$ has $p$\\
    vanishing\\
    moments
  \end{array}}
  \implies
  \brb{\begin{array}{FlclD}
    1. & \ds \sum_{k\in\Z} (x-k)^n \fphi(x-k) &=& M_n & and
    \\
    2. & \ds \left.\opddwn  \Fphi(\omega) \right|_{\omega=2\pi k}&=& \cft (-i)^n \kdelta_k M_n
  \end{array}}
  \quad\scy\text{for $n=1,2,\ldots,p-1$}
  }
\end{lemma}
\begin{proof}
\begin{enumerate}
  \item Proof that (1)$\iff$(2): by \prefp{lem:sfixceq}


  \item Proof for ($\implies$) assertion:
    \begin{align*}
      \inprod{\fpsi(x)}{x^n} =0
        &\iff     \left.\opddwn  \Dh(\omega)\right|_{\omega=\pi}=0
        &&        \text{by \prefp{thm:gh_cqf}}
      \\&\implies \left.\opddwn  \Fphi(\omega)\right|_{\omega=2\pi k}= \cft(-i)M_n\delta_k
        &&        \text{by \prefp{lem:vanish_hphi}}
      \\&\iff     \sum_{k\in\Z} (x-k)^n \fphi(x-k)= M_n
        &&        \text{by \prope{Strang-Fix condition}}
        &&        \text{\xref{lem:sfixceq}}
    \end{align*}

  %\item Proof that
  %  $\left.\opddwn  \Dh(\omega)\right|_{\omega=\pi}=0
  %   \implies
  %   \left.\opddwn  \Fphi(\omega)\right|_{\omega=2\pi k}=0
  %  $:
  %  \begin{align*}
  %      \left. \opddwn  \Fphi(\omega)                           \right|_{\omega=0}
  %        &= \left. \opddwn  \cft \int_\R \fphi(x)e^{-i\omega x} \dx \right|_{\omega=0}
  %        && \text{by definition of $\Fphi$ \xref{def:opFT}}
  %      \\&= \left. \cft \int_\R \fphi(x)\opddwn  e^{-i\omega x} \dx \right|_{\omega=0}
  %      \\&= \left. \cft \int_\R \fphi(x) (-it)^n e^{-i\omega x} \dx             \right|_{\omega=0}
  %      \\&= \left. \cft (-i)^n \int_\R x^n \fphi(x) e^{-i\omega x} \dx          \right|_{\omega=0}
  %      \\&= \cft (-i)^n \int_\R x^n \fphi(x) \dx
  %      \\&= \cft (-i)^n M_n
  %        && \text{by definition of $M_n$}
  %      \\
  %      \\
  %      \left. \opddwn  \Fphi(\omega)                           \right|_{\omega=2\pi k}
  %        &= \left. \opddwn
  %                 \Dh\left(\frac{\omega}{2}\right)
  %                 \Fphi\left(\frac{\omega}{2}\right)
  %           \right|_{\omega=2\pi k}
  %        && \text{by \prefp{prop:vsmra_real_Fpsi}}
  %      \\&= \left. \sum_{m=0}^n {n\choose m}
  %                  \Dh  ^{(m  )}\left(\frac{\omega}{2}\right)
  %                  \Fphi^{(n-m)}\left(\frac{\omega}{2}\right)
  %           \right|_{\omega=2\pi k}
  %        && \text{by Leibnitz GPR \prefpo{lem:LGPR}}
  %      \\&= \sum_{m=0}^n {n\choose m} \mcom{\Dh^{(m)}(\pi k)}{0 for $k$ odd} \Fphi^{(n-m)}(\pi k)
  %      \\&= \left\{\begin{array}{ll}
  %             \sum_{m=0}^n {n\choose m} \Dh^{(m)}(\pi k) \Fphi^{(n-m)}(\pi k)
  %                & \text{for $k$ even} \\
  %             0  & \text{for $k$ odd}
  %           \end{array}\right.
  %        && \text{by left hypothesis}
  %  \\
  %  \\
  %      \Fphi^{(n)}(2\pi0) &= (-i)^n M_n
  %      \\
  %      \Fphi^{(n)}(2\pi1) &= 0
  %      \\
  %      \Fphi^{(n)}(2\pi2)
  %        &= \ds \sum_{m=0}^n {n\choose m} \Dh^{(m)}(2\pi1) \mcom{\Fphi^{(n-m)}(2\pi1)}{$0$}
  %         = 0
  %      \\
  %      \Fphi^{(n)}(2\pi3) &= 0
  %      \\
  %      \Fphi^{(n)}(2\pi4)
  %        &= \ds \sum_{m=0}^n {n\choose m} \Dh^{(m)}(2\pi2) \mcom{\Fphi^{(n-m)}(2\pi2)}{$0$}
  %         = 0
  %      \\
  %      \Fphi^{(n)}(2\pi5) &= 0
  %      \\
  %      \Fphi^{(n)}(2\pi6)
  %        &= \ds \sum_{m=0}^n {n\choose m} \Dh^{(m)}(2\pi3) \mcom{\Fphi^{(n-m)}(2\pi3)}{$0$}
  %         = 0
  %      \\
  %      \vdots & \vdots
  %      \\
  %      \Fphi^{(n)}(2\pi k) &= \kdelta_k(-i)^n M_n
  %  \end{align*}

\end{enumerate}
\end{proof}


%---------------------------------------
\begin{theorem}
\label{thm:sfix_poly}
\index{Strang-Fix condition}
%---------------------------------------
Let $\wavsys$ be a wavelet system with $p$ vanishing moments.
Let $\fq(x)$ be a \fncte{polynomial}.
Let
  \\\indentx$\setQ_p\eqd\set{\fq(x)}{\text{$\fq(x)$ is a \fncte{polynomial} of order $p-1$ or less}}$.\\
\thmbox{
  \brb{\begin{array}{FMD}
    1. & $\fpsi$ has $p$ vanishing\\
       & moments                     & and \\
    2. & $\fq(x)$ is of order $p-1$
  \end{array}}
  \quad\implies\quad
  \brb{\begin{array}{FMD}
    1. & There exists $\tuple{b_n}{0}{p-1}$ such that \\
       & $\ds\fq(x) = \sum_{n=0}^{p-1} b_n \sum_{k\in\Z} k^n\fphi(x-k)$ & and\\
    2. & $\setxZ{\opTrn^n\fphi(x)}$ is a \structe{basis} for $\setQ_p$.
  \end{array}}
  }
\end{theorem}
\begin{proof}
This follows from \prefp{lem:sfixc}
(Strang-Fix condition).
See \prefp{ex:sfixc5}.
\end{proof}

%---------------------------------------
\begin{example}
\label{ex:sfixc5}
%---------------------------------------
Let $\wavsys$ be a \structe{wavelet system} \xref{def:wavsys} in which
$\fpsi(x)$ has $p=5$ vanishing moments.
\exbox{\begin{array}{lclclclclclclclcl}
  \ds \sum_{k\in\Z}     \fphi(x-k) &=& M_0
  \\
  \ds \sum_{k\in\Z} k   \fphi(x-k) &=& M_0x   &-&  M_1
  \\
  \ds \sum_{k\in\Z} k^2 \fphi(x-k) &=& M_0x^2 &-& 2M_1x   &+& M_2
  \\
  \ds \sum_{k\in\Z} k^3 \fphi(x-k) &=& M_0x^3 &-& 3M_1x^2 &+& 3M_2x   &-&  M_3
  \\
  \ds \sum_{k\in\Z} k^4 \fphi(x-k) &=& M_0x^4 &-& 4M_1x^3 &+& 6M_2x^2 &-& 4M_3x &+& M_4
  \\
  \ds \sum_{k\in\Z} k^5 \fphi(x-k) &=& M_0x^5 &-& 5M_1x^4 &+& 10M_2x^3 &-& 10M_3x^2 &+& 5M_4x &-& M_5
\end{array}}

These equations can be represented in matrix algebra form as
\exbox{
  \brs{\begin{array}{>{\ds}l}
    \sum_{k\in\Z} k^4 \fphi(x-k) \\
    \sum_{k\in\Z} k^3 \fphi(x-k) \\
    \sum_{k\in\Z} k^2 \fphi(x-k) \\
    \sum_{k\in\Z} k   \fphi(x-k) \\
    \sum_{k\in\Z}     \fphi(x-k)
  \end{array}}
  =
  \brs{\begin{array}{rrrrr}
    M_0 & -4M_1 &  6M_2 & -4M_3 &  M_4 \\
      0 &   M_0 & -3M_1 &  3M_2 & -M_3 \\
      0 &     0 &   M_0 & -2M_1 &  M_2 \\
      0 &     0 &     0 &   M_0 & -M_1 \\
      0 &     0 &     0 &     0 &  M_0
  \end{array}}
  \brs{\begin{array}{l}
    x^4 \\
    x^3 \\
    x^2 \\
    x   \\
    1
  \end{array}}
  }

The operator matrix is triangular and therefore it's inverse
always exists.
The monomials $1, x, x^2, x^3$ and $x^4$ can be computed by
computing the inverse of the operator matrix.
\exbox{\begin{array}{*{7}{>{\ds}l>{\ds}c}}
  1   &=& \frac{1}{M_0} \sum_{k\in\Z} \fphi(x-k)
  \\
  x   &=& \frac{1}{M_0}\sum_{k\in\Z} k \fphi(x-k)
      &+& \frac{M_1}{M_0}
  \\
  x^2 &=& \frac{1}{M_0}\sum_{k\in\Z} k^2 \fphi(x-k)
      &+& \frac{2M_1}{M_0^2}\sum_{k\in\Z} k \fphi(x-k)
      &+& \frac{2M_1^2-M_2M_0}{M_0^2}
  \\
  x^3 &=& \frac{1}{M_0}\sum_{k\in\Z} k^3 \fphi(x-k)
      &+& \frac{3M_1 }{M_0^2}\sum_{k\in\Z} k^2 \fphi(x-k)
      &+& \frac{6M_1^2-3M_2M_0}{M_0^3}\sum_{k\in\Z} k \fphi(x-k)
      \\&&\qquad&+& \frac{6M_1^3-6M_2M_1M_0+M_3M_0^2}{M_0^3}
\end{array}}
\\
Note that each monomial $x^n$ is simply a linear combination
of $\seq{\fphi(x-k)}{k\in\Z}$.
\end{example}
\begin{proof}
\begin{align*}
  M_0
    &= \sum_{k\in\Z} (x-k)^0 \fphi(x-k)
  \\&= \sum_{k\in\Z} \fphi(x-k)
  \\
  \\
  M_1
    &= \sum_{k\in\Z} (x-k)^1 \fphi(x-k)
  \\&= t \sum_{k\in\Z} \fphi(x-k) - \sum_{k\in\Z} k \fphi(x-k)
  \\&= M_0t  - \sum_{k\in\Z} k \fphi(x-k)
  \\
  \\
  M_2
    &= \sum_{k\in\Z} (x-k)^2 \fphi(x-k)
  \\&= x^2 \sum_{k\in\Z}\fphi(x-k) - 2t \sum_{k\in\Z} k\fphi(x-k) + \sum_{k\in\Z} k^2 \fphi(x-k)
  \\&= x^2 M_0 - 2t(M_0t - M_1) + \sum_{k\in\Z} k^2 \fphi(x-k)
  \\&= -M_0x^2 + 2M_1x + \sum_{k\in\Z} k^2 \fphi(x-k)
  \\
  \\
  M_3
    &= \sum_{k\in\Z} (x-k)^3 \fphi(x-k)
  \\&=  x^3 \sum_{k\in\Z}     \fphi(x-k)
      - 3x^2 \sum_{k\in\Z} k   \fphi(x-k)
      + 3x   \sum_{k\in\Z} k^2 \fphi(x-k)
      -      \sum_{k\in\Z} k^3 \fphi(x-k)
  \\&=  x^3 M_0
      - 3x^2 \brs{ M_0t - M_1 }
      + 3x   \brs{ M_0x^2 - 2M_1x + M_2 }
      -      \sum_{k\in\Z} k^3 \fphi(x-k)
  \\&= (M_0-3M_0+3M_0)x^3 + (3M_1-6M_1)x^2 + 3M_2 x -\sum_{k\in\Z} k^3 \fphi(x-k)
  \\&= M_0x^3 -3M_1 x^2 + 3M_2t - \sum_{k\in\Z} k^3 \fphi(x-k)
  \\
  \\
  M_4
    &= \sum_{k\in\Z} (x-k)^4  \fphi(x-k)
  \\&=  x^4 \sum_{k\in\Z}     \fphi(x-k)
      - 4x^3 \sum_{k\in\Z} k   \fphi(x-k)
      + 6x^2 \sum_{k\in\Z} k^2 \fphi(x-k)
      - 4x   \sum_{k\in\Z} k^3 \fphi(x-k)
      +      \sum_{k\in\Z} k^4 \fphi(x-k)
  \\&=  x^4 M_0
      - 4x^3 \brs{ M_0t   -  M_1  }
      + 6x^2 \brs{ M_0x^2 - 2M_1x   + M_2  }
      - 4x   \brs{ M_0x^3 - 3M_1x^2 + 3M_2t - M_3  }
      \\&\qquad+      \sum_{k\in\Z} k^4 \fphi(x-k)
  \\&=  x^4 M_0
      + \brs{ - 4M_0x^4   + 4M_1x^3  }
      + \brs{ 6M_0x^4 - 12M_1x^3  + 6M_2x^2  }
      \\&\qquad+ \brs{ - 4M_0x^4 + 12M_1x^3 - 12M_2x^2 + 4M_3x  }
      +  \sum_{k\in\Z} k^4 \fphi(x-k)
  \\&= (M_0- 4M_0+6M_0- 4M_0)x^4
      + (4M_1- 12M_1+ 12M_1)x^3
      + (6M_2-12M_2)x^2
      + 4M_3x
      \\&\qquad+ \sum_{k\in\Z} k^4 \fphi(x-k)
  \\&= -M_0x^4
      + 4M_1x^3
      - 6M_2x^2
      + 4M_3x
      + \sum_{k\in\Z} k^4 \fphi(x-k)
  \\
  \\
  M_5
    &= \sum_{k\in\Z} (x-k)^5  \fphi(x-k)
  \\&=  x^5 \sum_{k\in\Z}     \fphi(x-k)
      - 5x^4 \sum_{k\in\Z} k   \fphi(x-k)
      +10x^3 \sum_{k\in\Z} k^2 \fphi(x-k)
      -10x^2 \sum_{k\in\Z} k^3 \fphi(x-k)
      \\&\qquad+ 5t   \sum_{k\in\Z} k^4 \fphi(x-k)
      -      \sum_{k\in\Z} k^5 \fphi(x-k)
  \\&=  x^5 M_0
      - 5x^4 \brs{ M_0x-M_1                 }
      +10x^3 \brs{ M_0x^2-2M_1x+M_2         }
      -10x^2 \brs{ M_0x^3-3M_1x^2+3M_2x-M_3 }
      \\&\qquad+ 5t   \brs{ M_0x^4-4M_1x^3+6M_2x^2-4M_3x+M_4 }
      -      \sum_{k\in\Z} k^5 \fphi(x-k)
  \\&=  x^5 M_0
      \brs{ -5M_0x^5+5M_1x^4                 }
      + \brs{ 10M_0x^5 -20M_1x^4 + 10M_2x^3  }
      \\&\qquad+ \brs{ -10M_0x^5 +30M_1x^4 -30M_2x^3 +10M_3x^2 }
      \\&\qquad+ \brs{ 5M_0x^5 - 20M_1x^4 +30M_2x^3 - 20M_3x^2 +5M_4x }
      - \sum_{k\in\Z} k^5 \fphi(x-k)
  \\&= (M_0-5M_0+10M_0+5M_0-10M_0)x^5
       +(5M_1-20M_1+30M_1-20M_1)x^4
       \\&\qquad+(10M_2-30M_2+30M_2)x^3
       +(10M_3-20M_3)x^2
       + 5M_4x
      - \sum_{k\in\Z} k^5 \fphi(x-k)
  \\&= M_0x^5 - 5M_1x^4 + 10M_2x^3 - 10M_3x^2 + 5M_4x
      - \sum_{k\in\Z} k^5 \fphi(x-k)
\end{align*}

\begin{align*}
  &\brs{\begin{array}{*{8}{r}}
       M_0 & -3M_1 &  3M_2 & -M_3  & 1 & 0 & 0 & 0
     \\  0 &   M_0 & -2M_1 &  M_2  & 0 & 1 & 0 & 0
     \\  0 &     0 &   M_0 & -M_1  & 0 & 0 & 1 & 0
     \\  0 &     0 &     0 &  M_0  & 0 & 0 & 0 & 1
  \end{array}}\;\rightarrow
  \;
  \brs{\begin{array}{*{8}{r}}
       M_0 & -3M_1 &  3M_2 &    0  & 1 & 0 & 0 &  \frac{M_3}{M_0}
     \\  0 &   M_0 & -2M_1 &    0  & 0 & 1 & 0 & -\frac{M_2}{M_0}
     \\  0 &     0 &   M_0 &    0  & 0 & 0 & 1 &  \frac{M_1}{M_0}
     \\  0 &     0 &     0 &  M_0  & 0 & 0 & 0 &  1
  \end{array}}\;\rightarrow
  \\
  &\brs{\begin{array}{*{8}{r}}
       M_0 & -3M_1 &     0 &    0  & 1 & 0 & -\frac{3M_2}{M_0} & -\frac{3M_2M_1}{M_0^2}+\frac{M_3}{M_0}
     \\  0 &   M_0 &     0 &    0  & 0 & 1 &  \frac{2M_1}{M_0} & \frac{2M_1^2}{M_0^2}-\frac{M_2}{M_0}
     \\  0 &     0 &   M_0 &    0  & 0 & 0 &  1                &  \frac{M_1}{M_0}
     \\  0 &     0 &     0 &  M_0  & 0 & 0 &  0                &  1
  \end{array}}\;\rightarrow
  \\
  &\brs{\begin{array}{*{8}{r}}
       M_0 &     0 &     0 &    0  & 1 & \frac{3M_1}{M_0} & \frac{6M_1^2}{M_0^2}-\frac{3M_2}{M_0} & \frac{6M_1^3}{M_0^3}-\frac{3M_1M_2}{M_0^2} - \frac{3M_2M_1}{M_0^2}+\frac{M_3}{M_0}
     \\  0 &   M_0 &     0 &    0  & 0 & 1                &  \frac{2M_1}{M_0}                     & \frac{2M_1^2}{M_0^2}-\frac{M_2}{M_0}
     \\  0 &     0 &   M_0 &    0  & 0 & 0                &  1                                    & \frac{M_1}{M_0}
     \\  0 &     0 &     0 &  M_0  & 0 & 0                &  0                                    & 1
  \end{array}}\;\rightarrow
  \\
  &\brs{\begin{array}{*{8}{r}}
       1 &     0 &     0 &    0  & \frac{1}{M_0} & \frac{3M_1}{M_0^2} & \frac{6M_1^2}{M_0^3}-\frac{3M_2}{M_0^2} & \frac{6M_1^3}{M_0^4}-\frac{3M_1M_2}{M_0^3} - \frac{3M_2M_1}{M_0^3}+\frac{M_3}{M_0^2}
     \\0 &     1 &     0 &    0  & 0 & \frac{1}{M_0}    &  \frac{2M_1}{M_0^2}                     & \frac{2M_1^2}{M_0^3}-\frac{M_2}{M_0^2}
     \\0 &     0 &     1 &    0  & 0 & 0                &  \frac{1}{M_0}                          & \frac{M_1}{M_0^2}
     \\0 &     0 &     0 &    1  & 0 & 0                &  0                                      & \frac{1}{M_0}
  \end{array}}\;\rightarrow
  \\
  &\brs{\begin{array}{*{8}{r}}
       1 &     0 &     0 &    0  & \frac{1}{M_0} & \frac{3M_1}{M_0^2} & \frac{6M_1^2-3M_2M_0}{M_0^3} & \frac{6M_1^3-6M_2M_1M_0+M_3M_0^2}{M_0^4}
     \\0 &     1 &     0 &    0  & 0 & \frac{1}{M_0}    &  \frac{2M_1}{M_0^2}                     & \frac{2M_1^2-M_2M_0}{M_0^3}
     \\0 &     0 &     1 &    0  & 0 & 0                &  \frac{1}{M_0}                          & \frac{M_1}{M_0^2}
     \\0 &     0 &     0 &    1  & 0 & 0                &  0                                      & \frac{1}{M_0}
  \end{array}}
\end{align*}

\begin{align*}
  \brs{\begin{array}{>{\ds}l}
    x^3  \\
    x^2  \\
    x    \\
    1
  \end{array}}
  &=
  \brs{\begin{array}{*{4}{>{\ds}l}}
      \frac{1}{M_0} & \frac{3M_1}{M_0^2} & \frac{6M_1^2-3M_2M_0}{M_0^3} & \frac{6M_1^3-6M_2M_1M_0+M_3M_0^2}{M_0^4}
    \\0             & \frac{1}{M_0}      &  \frac{2M_1}{M_0^2}          & \frac{2M_1^2-M_2M_0}{M_0^3}
    \\0             & 0                  &  \frac{1}{M_0}               & \frac{M_1}{M_0^2}
    \\0             & 0                  &  0                           & \frac{1}{M_0}
  \end{array}}
  \brs{\begin{array}{>{\ds}l}
      \sum_{k\in\Z} k^3 \fphi(x-k)
    \\\sum_{k\in\Z} k^2 \fphi(x-k)
    \\\sum_{k\in\Z} k   \fphi(x-k)
    \\\sum_{k\in\Z}     \fphi(x-k)
  \end{array}}
\end{align*}


\begin{align*}
  x^3 &=  \frac{1}{M_0}                            \sum_{k\in\Z} k^3 \fphi(x-k)
       +  \frac{3M_1}{M_0^2}                       \sum_{k\in\Z} k^2 \fphi(x-k)
       +  \frac{6M_1^2-3M_2M_0}{M_0^3}             \sum_{k\in\Z} k   \fphi(x-k)
    \\& + \frac{6M_1^3-6M_2M_1M_0+M_3M_0^2}{M_0^4}
          \mcom{\sum_{k\in\Z} \fphi(x-k)}{$M_0$}
    \\&=  \frac{1}{M_0}                            \sum_{k\in\Z} k^3 \fphi(x-k)
       +  \frac{3M_1}{M_0^2}                       \sum_{k\in\Z} k^2 \fphi(x-k)
       +  \frac{6M_1^2-3M_2M_0}{M_0^3}             \sum_{k\in\Z} k   \fphi(x-k)
    \\&\qquad+ \frac{6M_1^3-6M_2M_1M_0+M_3M_0^2}{M_0^3}
\\
  x^2 &=
      \frac{1}{M_0}                \sum_{k\in\Z} k^2 \fphi(x-k)
     +\frac{2M_1}{M_0^2}           \sum_{k\in\Z}     \fphi(x-k)
     +\frac{2M_1^2-M_2M_0}{M_0^2}
\\
  x &=  \frac{1}{M_0}\sum_{k\in\Z} k \fphi(x-k) + \frac{M_1}{M_0}
\end{align*}

\end{proof}


%---------------------------------------
\begin{conjecture}
\footnote{
  %\citeoeis{A110555}
  \citeOc{oeisA110555}{A110555}{$T(n,k) \eqd \sum_{k=0,1,\ldots,n}{n\choose k}(-1)^k$}
  }
\label{cnj:sfixc}
%---------------------------------------
Let $\wavsys$ be a wavelet system with $p$ vanishing moments
and let $M_n$ be the \hi{$n$th moment} \xref{def:Mn} of $\fphi(x)$.
\cnjbox{
  \sum_{k\in\Z} k^n \fphi(x-k)
      = \sum_{m=0}^n {n\choose m}(-1)^m M_m x^{n-m}
  }
\end{conjecture}
\begin{proof}
This conjecture is supported by the results of \prefpp{ex:sfixc5}.
However, I have no proof at this time.
\begin{align*}
  (-1)^n M_n
    &= (-1)^n \sum_{k\in\Z} (x-k)^n  \fphi(x-k)
  \\&= (-1)^n \sum_{k\in\Z} \sum_{m=0}^n {n\choose m}x^{n-m} (-k)^m \fphi(x-k)
  \\&= (-1)^n \sum_{m=0}^n {n\choose m}(-1)^m x^{n-m} \sum_{k\in\Z} k^m \fphi(x-k)
  \\&= \brs{(-1)^{2n} \sum_{k\in\Z} k^n \fphi(x-k)}
      \;+\; (-1)^n \sum_{m=0}^{n-1} {n\choose m}(-1)^m x^{n-m} \sum_{k\in\Z} k^m \fphi(x-k)
  \\&= \brs{\sum_{k\in\Z} k^n \fphi(x-k)}
      \;+\; (-1)^n \sum_{m=0}^{n-1} {n\choose m}(-1)^m x^{n-m} \sum_{k\in\Z} k^m \fphi(x-k)
  \\
  \\
  \sum_{k\in\Z} k^n \fphi(x-k)
    &= -(-1)^n \sum_{m=0}^{n-1} {n\choose m}(-1)^m x^{n-m} \sum_{k\in\Z} k^m \fphi(x-k)
        + (-1)^n M_n
  \\&=  \sum_{m=0}^{n-1} {n\choose m}(-1)^m
         \mcom{\brs{(-1)^{n-1}\sum_{k\in\Z} k^m \fphi(x-k)}}{$M_m?$}
         x^{n-m}
        + (-1)^n M_n
\\
\\
  M_m &\eqq (-1)^{n-1}\sum_{k\in\Z}  k^m \fphi(x-k)
\\
  M_m
    &= \sum_{k\in\Z} (x-k)^m \fphi(x-k)
  \\&= \sum_{k\in\Z} \sum_{p=0}^m {m\choose p}x^{m-p}(-k)^p \fphi(x-k)
  \\&= (-1)^m \sum_{k\in\Z} k^m \fphi(x-k)
      + \mcom{\sum_{k\in\Z} \sum_{p=0}^{m-1}{m\choose p}x^{m-p}(-k)^p \fphi(x-k)}{$0?$}
  \\
  \\
  \sum_{k\in\Z} k^n \fphi(x-k)
    &= -(-1)^n \sum_{m=0}^{n-1} {n\choose m}(-1)^m x^{n-m} \sum_{k\in\Z} k^m \fphi(x-k)
        + (-1)^n M_n
  \\&=  \sum_{m=0}^{n-1} {n\choose m}(-1)^m
         (-1)^{n-1}\brs{\sum_{k\in\Z} k^m \fphi(x-k)}
         x^{n-m}
        + (-1)^n M_n
  \\&=  \sum_{m=0}^{n-1} {n\choose m}(-1)^m
         (-1)^{n-1}
         \brs{
           \sum_{p=0}^{m-1} {m\choose p}(-1)^p
           (-1)^{m-1}\brs{\sum_{k\in\Z} k^p \fphi(x-k)}
           x^{m-p}
           + (-1)^m M_m
         }
         x^{n-m}
        \\&\qquad+ (-1)^n M_n
\end{align*}
\end{proof}

%=======================================
\subsection{Scaling function and continuity}
%=======================================
%--------------------------------------
\begin{theorem}
\label{thm:vanish_continuous}
%--------------------------------------
Let $\wavsys$ be a \structe{wavelet system} \xref{def:wavsys}.
Let $\Fphi(\omega)$ be the \fncte{Fourier transform} \xref{def:opFT} of $\fphi(x)$.
\thmbox{
  \brb{\begin{array}{FM}
    (1). & $\fpsi$ has $p\ge2$ \prope{vanishing moments}\qquad\scs and\\
    (2). & $g_n = (-1)^n h^\ast_{N-n}$\qquad\scs[\prope{CQF condition} \xref{def:cqf}]
  \end{array}}
  \implies
  \brb{\begin{array}{M}
    $\Fphi(\omega)$ is \prope{continuous}\\
    at $\omega=0$
  \end{array}}
  }
\end{theorem}
\begin{proof}
  \begin{align*}
    \text{(1) and (2)}
      &\implies \brlr{\opddwn  \Fphi(\omega)}_{\omega=2\pi k}=\cft (-i)^n \kdelta_k M_n
      && \text{by \prefp{lem:sfixc}}
    \\&\implies \brlr{\opddw \Fphi(\omega)}_{\omega=0} = \cft (-i)^1 M_1
    \\&\implies \brlr{\opddw \Fphi(\omega)}_{\omega=0} = 0
      && \text{by (1) and def. of \prope{vanishing moments} \xref{def:vanish}}
    \\&\implies \brlr{\opddw \Fphi(\omega)}_{\omega=0} < \infty
    \\&\implies \text{$\Fphi(\omega)$ is \prope{continuous} at $\omega=0$}
  \end{align*}
\end{proof}

%=======================================
\section{Sufficient conditions}
%=======================================
How can we design a \structe{wavelet system} \xref{def:wavsys} to have \prope{$p$ vanishing moments}?
One way is to design the \structe{scaling coefficient sequence} \xref{def:hn} $\seqn{h_n}$ in the \ope{z-domain} \xref{def:opZ}
such that $\Zh(z)$ includes the factor $\brp{\frac{1+z^{-1}}{2}}^p$ (next theorem).
This factor is included in the $\Zh(z)$ of the \structe{Daubechies-p wavelet system}\ifsxref{compactp}{def:Dp}.
A similar factor \xref{lem:FTNn} also appears in the \ope{Fourier transform} \xref{def:opFT}
of \structe{B-spline}s \xref{def:Bspline}.

%---------------------------------------
\begin{lemma}
\label{lem:van_factor}
\footnote{
  \citerp{dau}{155},
  \citerpp{vidakovic}{80}{82}
  }
%---------------------------------------
Let $\wavsys$ be a \structe{wavelet system} \xref{def:wavsys}.
Let $\seqnZ{q_n}$ be a sequence with \ope{z-transform} \xref{def:opZ} $\Zq(z)$.
%Let the \prope{CQF condition} be defined as in \prefp{def:cqf}.
\lembox{
  \brb{\begin{array}{Frc>{\ds}l}
    (1). & \Zh(z) &=& \brp{\frac{1+z^{-1}}{2}}^p\,\Zq(z) \qquad\text{\scs and} \\
    (2). & g_n    &=& (-1)^n h^\ast_{N-n}                \quad\text{\scs[\prope{CQF condition} \xref{def:cqf}]}
  \end{array}}
  \implies
  \brb{\begin{array}{M}
    $\fpsi$ has $p$\\
    vanishing moments
  \end{array}}
  }
\end{lemma}
\begin{proof}
\begin{align*}
            && \Zh(z)
                   &= \brp{\frac{1+z^{-1}}{2}}^p\,\Zq(z)
                  &&  \text{by (1)}
  \\\implies&& \Dh(\omega)
                   &= \brp{\frac{1+e^{i\omega}}{2}}^p\,\Fq(\omega)
                   && \text{by \prefp{def:dtft}}
                   %\text{\scs by \prefp{def:opZ} and \prefp{def:dtft}}
  \\\implies&& \brlr{\opddwn\Dh(\omega)}_{\omega=\pi}
                   &= \brlr{\opddwn\brp{\frac{1+e^{i\omega}}{2}}^p\,\Fq(\omega)}_{\omega=\pi}
               \\&&&= \brlr{\sum_{k=0}^n\bcoef{n}{k} \brs{\opddwn[k]\brp{\frac{1+e^{i\omega}}{2}}^p} \, \brs{\opddwn[n-k]\Fq(\omega)}}_{\omega=\pi}
                 %&& \text{by \thme{Leibnitz GPR} \xref{lem:LGPR}}
                  && \text{by \prefp{lem:LGPR}}
               \\&&&=\brlr{\sum_{k=0}^n\bcoef{n}{k} \brs{\frac{p!}{2^p(p-k)!}\prod_{k=1}^p\brs{ie^{i\omega}\brp{1+e^{i\omega}}^k}} \, \brs{\opddw^{n-k}\Fq(\omega)}}_{\omega=\pi}
               \\&&&= 0
  \\\iff         &&& \text{$\fpsi$ has $p$ vanishing moments}
                  && \text{by \prefp{thm:gh_cqf}}
\end{align*}
\end{proof}



%%---------------------------------------
%\begin{theorem}
%\label{thm:compact_form}
%\footnote{
%  \citerp{dau}{155},
%  \citerpp{vidakovic}{80}{82}
%  }
%%---------------------------------------
%Let $\wavsys$ be an \hib{orthonormal} wavelet system.
%\thmboxt{
%  $\ds\left.\begin{array}{FlD}
%    1. & |\fpsi(x)| \le \frac{C_1}{(1+|x|)^\alpha}, \; \alpha>p
%       & \text{and} \\
%    2. & \text{$\fpsi(\omega)$ is $N$ times differentiable}
%       & \text{and} \\
%    3. & |\fphi(x)| \le \frac{C_2}{(1+|x|)^\alpha}, \; \alpha>p
%  \end{array}\right\}
%  \implies
%  \left\{\begin{array}{FlD}
%    1. & \ds \text{$\fpsi$ has $p$ vanishing moments.} & and\\
%    2. & \ds \Dh(\omega) = \left(\frac{1+e^{-i\omega}}{2}\right)^p \fQ(\omega)
%  \end{array}\right.$
%  \\
%  where $\fQ(\omega)$ is $2\pi$ periodic and
%  $\fQ(\omega)$ is $p-1$ times differentiable.
%  }
%\end{theorem}
%%\begin{proof}
%%No proof at this time.
%%\end{proof}




%=======================================
\section{Examples}
%=======================================
%--------------------------------------
\begin{example}[\exmd{order 1 B-spline wavelet system}]
\label{ex:pun_n=3}
%--------------------------------------
Let $\wavsys$ be a \structe{wavelet system}.
\exbox{
  \brb{\begin{array}{FMMD}
      1. & $\support\fphi(x)=\intcc{0}{2}$ & \xref{thm:mra_support}  & and
    \\2. & \prope{admissibility condition} & \xref{thm:admiss}       & and
    \\3. & \prope{partition of unity}      & \xref{thm:pun_zero}     & and
    \\4. & \prope{1 vanishing moment}      & \xref{thm:gh_cqf}       &
  \end{array}}
  \implies
  \brb{\begin{array}{r|>{\ds}l}
      n            & \fh_{n}
    \\\hline
      0            & \frac{\sqrt{2}}{4}\cdot1
    \\1            & \frac{\sqrt{2}}{4}\cdot2
    \\2            & \frac{\sqrt{2}}{4}\cdot1
    \\\text{other} & 0 
  \end{array}}
  }
\\\indentx\tbox{\includegraphics{../common/math/graphics/pdfs/n1_pz.pdf}}
  \indentx\tbox{\includegraphics{../common/math/graphics/pdfs/n1_h.pdf}}
  \indentx\tbox{\includegraphics{../common/math/graphics/pdfs/b1_g.pdf}}
\end{example}
\begin{proof}
\begin{enumerate}
  \item Proof that (1) $\implies$ that only $h_0$, $h_1$, and $h_2$ are non-zero: by \prefp{thm:mra_support}.

  \item System of Equations:
        \\\indentx$\ds\begin{array}{rcl MM}
           \fh_0 +\fh_1 +  \fh_2 &=& \sqrt{2} & (\prope{admissibility}        & \xref{thm:admiss}   
        \\ \fh_0 -\fh_1 +  \fh_2 &=& 0        & (\prope{partition of unity}   & \xref{thm:pun_zero} 
        \\0\fh_0 -\fh_1 + 2\fh_2 &=& 0        & (\prope{1st vanishing moment} & \xref{thm:gh_cqf} 
        \end{array}$

  \item System of Equations in Matrix Algebra form:
    \begin{align*}
      \brs{\begin{array}{rrc}
        \\  1 &  1 &  1
        \\  1 & -1 &  1
        \\  0 & -1 &  2
      \end{array}}
      \brs{\begin{array}{l}
           h_0
        \\ h_1
        \\ h_2
      \end{array}}
      &= \brs{\begin{array}{l}
              \sqrt{2}
           \\ 0
           \\ 0
         \end{array}}
    \end{align*}
  
  \item Solution for $\brs{\begin{array}{ccc} h_0 & h_1 & h_2\end{array}}^T$:
    \begin{align*}
      \brs{\begin{array}{rrc|l}
            1 &  1 &  1 & \sqrt{2}
        \\  1 & -1 &  1 & 0
        \\  0 & -1 &  2 & 0
      \end{array}}
      &\rightarrow
      \brs{\begin{array}{rrc|l}
            0 &  2 &  0 & \sqrt{2}
        \\  1 & -1 &  1 & 0
        \\  0 & -1 &  2 & 0
      \end{array}}
      &&\rightarrow
      \brs{\begin{array}{rrc|l}
            0 &  0 &  4 & \sqrt{2}
        \\  1 & -1 &  1 & 0
        \\  0 &  1 & -2 & 0
      \end{array}}
      &&\rightarrow
      \brs{\begin{array}{rrc|l}
            0 &  0 &  1 & \sqrt{2}/4
        \\  1 &  0 & -1 & 0
        \\  0 &  1 & -2 & 0
      \end{array}}
      \\&\rightarrow
      \brs{\begin{array}{rrc|l}
            0 &  0 &  1 & 1\cdot\sqrt{2}/4
        \\  1 &  0 &  0 & 1\cdot\sqrt{2}/4
        \\  0 &  1 &  0 & 2\cdot\sqrt{2}/4
      \end{array}}
      &&\rightarrow
      \brs{\begin{array}{rrc|l}
            1 &  0 &  0 & 1\cdot\sqrt{2}/4
        \\  0 &  1 &  0 & 2\cdot\sqrt{2}/4
        \\  0 &  0 &  1 & 1\cdot\sqrt{2}/4
      \end{array}}
      &&\implies
      \brs{\begin{array}{l}
           h_0
        \\ h_1
        \\ h_2
      \end{array}}
      = 
      \frac{\sqrt{2}}{4}\brs{\begin{array}{l}
              1
           \\ 2
           \\ 1
      \end{array}}
    \end{align*}

  \item Note: $h_0$, $h_1$, $h_2$ can also be produced using other systems of equations including the following:
    \begin{dingautolist}{"AC}
      %\item \prope{admissibility condition} and \prope{orthonormality} \xref{ex:ortho_n=3}
      \item \fncte{Daubechies-$p2$} wavelets computed using spectral techniques\ifsxref{compactp}{ex:dau-p2}
    \end{dingautolist}
\end{enumerate}
\end{proof}

%Examples of wavelets with $p$ vanishing moments include the B-spline wavelets and the Daubechies-$p$ wavelets.
%
%\begin{tabular}{|>{\scs}rlcll|}
%  \hline
%     &  name & $p$ & &
%  \\\hline
%  %1. &  Haar                            & 1 & \pref{ex:pun_n=2}       & \prefpo{ex:pun_n=2}\\
%  %2. &  order 1 B-spline wavelet system & 2 & \pref{ex:sw_gh_tent}    & \prefpo{ex:sw_gh_tent}\\
%  %3. &  order 3 B-spline wavelet system & 4 & \pref{ex:sw_gh_bspline} & \prefpo{ex:sw_gh_bspline}\\
%  1. &  order 0 B-spline & 1 & \pref{ex:N0_hg}                           & \prefpo{ex:N0_hg} \\
%  2. &  order 1 B-spline & 2 & \pref{ex:N1_hg}                           & \prefpo{ex:N1_hg} \\
%  3. &  order 2 B-spline & 3 & \pref{ex:N2_hg}                           & \prefpo{ex:N2_hg} \\
%  4. &  Daubechies-p2    & 2 & \ifdochas{compactp}{\pref{ex:dau-p2}}     & \ifdochas{compactp}{\prefpo{ex:dau-p2}}\\
%  5. &  Daubechies-p3    & 3 & \ifdochas{compactp}{\pref{ex:dau-p3}}     & \ifdochas{compactp}{\prefpo{ex:dau-p3}} \\
%  6. &  Symlet-p4        & 4 & \ifdochas{compactp}{\pref{ex:symlet_p4}}  & \ifdochas{compactp}{\prefpo{ex:symlet_p4}}\\
%  \hline
%\end{tabular}
%

%\[\begin{array}{llcllll}
%  1. & \ds\sum_{n\in\Z}  h_n   &=& \sqrt{2}
%     & \text{(admissiblility condition)}
%     & \text{\pref{thm:admiss}}
%     & \text{page~\pageref{thm:admiss}}
%\\
%\\
%  2. & \ds\sum_{m\in\Z} h_m \fh^\ast_{m-2n} &=& \kdelta_n
%     & \text{(quadrature condition)}
%     & \text{\pref{thm:quadcon}}
%     & \text{page~\pageref{thm:quadcon}}
%\\
%\\
%  3. & \ds\sum_{m\in\Z} (-1)^m m^p  h_m   &=& 0
%     & \text{(vanishing $p$th moment)}
%     & \text{\pref{thm:vanish}}
%     & \text{page~\pageref{thm:vanish}}
%\end{array}\]
%
%Note that the ``zero at $z=-1$" constraint
%(\pref{thm:zero_unity_evenodd} page~\pageref{thm:zero_unity_evenodd})
%is just the vanishing 0th moment (and hence included in 3. for $p=0$).
%Also, the ``partition of unity" and ``even-odd" constraints
%are, by the same theorem, equivalent  to the
%``zero at $z=-1$" constraint.





%%---------------------------------------
%\begin{theorem}
%\label{thm:vsmra_real}
%%---------------------------------------
%Let $\spLLR$ be the set of real Lebesgue measurable functions.
%\thmbox{
%  \mcom{\left.\begin{array}{llclC}
%    1. & \opT  \fx(x) &\eqd&         \fx(x-1) & \forall \fx\in\spLLR \\
%    2. & \opD  \fx(x) &\eqd& \sqrt{2}\fx(2t)  & \forall \fx\in\spLLR
%  \end{array}\right\}}{definition of $\opT$ and $\opD$ (\prefp{def:vsmra_real_T})}
%  \quad\implies\quad
%  \text{$\vsmrasysLL$ is a \vsmratext}.
%}
%\end{theorem}
%\begin{proof}
%  \begin{enumerate}
%    \item Proof that $\opT^n\opD = \opD\opT^{2n}$:  By \prefpp{prop:vsmra_real_commute}
%    \item Proof that $\brp{\sum_{n\in\Z} \opT^n}$ is normal:
%      \begin{enumerate}
%        \item By \prefpp{prop:vsmra_real_sa}, $\brp{\sum_{n\in\Z} \opT^n}$ is self-adjoint.
%        \item Because self-adjoint $\implies$ normal,
%              $\brp{\sum_{n\in\Z} \opT^n}$ is also normal.
%      \end{enumerate}
%    \item Proof that $\normop{\opT}=\normop{\opD}=1$:
%      \begin{enumerate}
%        \item By \prefpp{prop:vsmra_real_Ta} and \prefpp{prop:vsmra_real_Ti},
%          $\opTi=\opTa$ and $\opDi=\opDa$.
%        \item Therefore, $\opT$ and $\opD$ are both unitary operators because
%          $\opTa\opT=\opT\opTa=\opI$ and $\opDa\opD=\opD\opDa=\opI$.
%        \item Because $\opT$ and $\opD$ are unitary operators,
%          by \prefpp{thm:unitary_prop}
%          $\normop{\opT}=\normop{\opD}=1$.
%    \end{enumerate}
%  \end{enumerate}
%\end{proof}
%

%%--------------------------------------
%\begin{theorem}
%\label{thm:Phi=HPhi}
%\index{dilation equation}
%\index{refinement equation}
%\index{two-scale difference equation}
%\raggedright
%%--------------------------------------
%Let $\spX$ be a vector space with unit vector $\vunit$,
%wavelet basis lattice $\wavlatbasis$,
%and ``\hie{Fourier transforms}"
%\[ \begin{array}{rcl cl}
%     \Fphi(\omega) &\eqd& \inprod{\fphi(x)}{\fkern{x}{\omega}}
%                   &\eqd& \inprodint{\fphi(x)}{\fkernea{x}{\omega}}
%     \\
%     \Fpsi(\omega) &\eqd& \inprod{\fpsi(x)}{\fkern{x}{\omega}}
%                   &\eqd& \inprodint{\fpsi(x)}{ \fkernea{x}{\omega}}
%\end{array} \]
%and ``\hie{discrete Fourier transforms}"
%\[ \begin{array}{rcl cl @{\qquad}l}
%     \Dh(\omega) &\eqd& \inproda{\fphi(x)}{\fkern{n}{\omega}}
%                 &\eqd& \sum_{n\in\Z} h_n \fkernea{n}{\omega}
%                 &      n \in\Z
%     \\
%     \Dg(\omega) &\eqd& \inproda{\fpsi(x)}{\fkern{n}{\omega}}
%                 &\eqd& \sum_{n\in\Z} g_n \fkernea{n}{\omega}
%                 &      n \in\Z
%\end{array} \]
%\thmbox{
%\renewcommand{\arraystretch}{1}
%\begin{array}{rc>{\ds}l c>{\ds}l}
%   \Fphi(\omega) &=& \frac{1}{\sqrt{2}} \Dh\left(\frac{\omega}{2}\right)\Fphi\left(\frac{\omega}{2}\right)
%                 &=& \mcomr{\inprod{\vphi}{\vunit}}{(normally set to $1$)} \;
%                     \prod_{n=1}^\infty \frac{1}{\sqrt{2}} \Dh\left(\frac{\omega}{2^n}\right)
%   \\
%   \Fpsi(\omega) &=& \frac{1}{\sqrt{2}} \Dg\left(\frac{\omega}{2}\right)\Fphi\left(\frac{\omega}{2}\right)
%   \\
%\end{array}
%}
%\end{theorem}
%\begin{proof}
%\begin{align*}
%   \intertext{\ding{"C0} Proof that
%     $\Fphi(\omega)=\cwt  \Dh\left(\frac{\omega}{2}\right)\Fphi\left(\frac{\omega}{2}\right)$:}
%   \opF\vphi
%     &= \inprod{\vphi}{\fkern{x}{\omega}}
%   \\&= \inprod{\sum_{n\in\Z}  h_n  \opT^n\opD\vphi}{\fkern{x}{\omega}}
%     && \text{by dilation equation \prefpo{thm:dilation_eq}}
%   \\&= \sum_{n\in\Z}  h_n \inprod{\opT^n\opD\vphi}{\fkern{x}{\omega}}
%     && \text{by definition of $\inprod{\cdot}{\cdot}$ \prefpo{def:inprod}}
%   \\&= \sum_{n\in\Z}  h_n \inprod{\vphi}{\left(\opT^n\opD\right)^\ast \fkern{x}{\omega}}
%     && \text{by definition of operator adjoint \prefpo{def:op_adj}}
%   \\&= \sum_{n\in\Z}  h_n \inprod{\vphi}{\cwt \fkern{n}{\frac{1}{2}\omega}\fkern{x}{\frac{1}{2}\omega}}
%     && \text{by \prefp{lem:TDk}}
%   \\&= \sum_{n\in\Z}  h_n  \;
%        \cwt  \;
%        \fkernea{n}{\frac{1}{2}\omega} \;
%        \inprod{\vphi}{\fkern{x}{\frac{1}{2}\omega}}
%     && \text{by property of inner-products \prefpo{def:inprod}}
%   \\&= \cwt  \;
%        \inprod{h_n}{\fkern{n}{\frac{1}{2}\omega}} \;
%        \inprod{\vphi}{\fkern{x}{\frac{1}{2}\omega}}
%     && \text{by assumption}
%  \\&= \cwt \, \Dh\left(\frac{\omega}{2}\right)\; \Fphi\left(\frac{\omega}{2}\right)
%    && \text{by definitions of $\Dg$ and $\Fphi$}
%\\
%   \intertext{\ding{"C1} Proof that
%     $\Fpsi(\omega)=\cwt  \Dg\left(\frac{\omega}{2}\right)\Fphi\left(\frac{\omega}{2}\right)$:
%     essentially same as above but substitute $g_n$ for $h_n$.
%     }
%\\
%   \intertext{\ding{"C2} Proof that
%     $\Fphi(\omega)=\int_\R \fphi(x) \dx\: \prod_{n=1}^\infty \frac{1}{\sqrt{2}} \Dh\left(\frac{\omega}{2^n}\right)$:
%     }
%   \Fphi(\omega)
%     &= \frac{1}{\sqrt{2}} \Dh\left(\frac{\omega}{2}\right)\Fphi\left(\frac{\omega}{2}\right)
%     && \text{by part \ding{"C0}}
%   \\&= \brs{
%          \frac{1}{\sqrt{2}} \Dh\left(\frac{\omega}{2}\right)
%          \frac{1}{\sqrt{2}} \Dh\left(\frac{\omega}{4}\right)
%        }\; \Fphi\left(\frac{\omega}{4}\right)
%   \\&= \brs{
%          \frac{1}{\sqrt{2}} \Dh\left(\frac{\omega}{2}\right)
%          \frac{1}{\sqrt{2}} \Dh\left(\frac{\omega}{4}\right)
%          \frac{1}{\sqrt{2}} \Dh\left(\frac{\omega}{8}\right)
%        }\; \Fphi\left(\frac{\omega}{8}\right)
%   \\&\vdots
%   \\&= \brs{\prod_{n=1}^\infty \frac{1}{\sqrt{2}} \Dh\left(\frac{\omega}{2^n}\right)}
%        \brs{ \lim_{n\to\infty} \Fphi\left(\frac{\omega}{n}\right) }
%   \\&= \Fphi\left(0\right) \;
%        \brs{\prod_{n=1}^\infty \frac{1}{\sqrt{2}} \Dh\left(\frac{\omega}{2^n}\right)}
%   \\&= \left. \inprod{\vphi}{\fkern{x}{\omega}} \right|_{\omega=0} \;
%        \brs{ \prod_{n=1}^\infty \frac{1}{\sqrt{2}} \Dh\left(\frac{\omega}{2^n}\right) }
%     && \text{by definition of $\Dh(\omega)$}
%   \\&= \inprod{\vphi}{\vunit} \;
%        \prod_{n=1}^\infty \frac{1}{\sqrt{2}} \Dh\left(\frac{\omega}{2^n}\right)
%\end{align*}
%\end{proof}

\begin{figure}
  \centering%
  $\begin{array}{|*{3}{>{\ds}c|}}
     \hline
      \includegraphics{../common/math/graphics/pdfs/n0_h.pdf}
     &\includegraphics{../common/math/graphics/pdfs/n1_h.pdf}
     &\includegraphics{../common/math/graphics/pdfs/n2_h.pdf}
    %&\includegraphics{../common/math/graphics/pdfs/bspline_phi_h.pdf}
     \\
      \fN_0(x) %= \sqrt{2}\sum_{k=0}^1 \brp{\frac{1}{\sqrt{2}}\bcoef{1}{k}} \fN_0(2x-k)
     &\fN_1(x) %= \sqrt{2}\sum_{k=0}^2 \brp{\frac{1}{\sqrt{2}}\bcoef{2}{k}} \fN_1(2x-k)
     &\fN_2(x) %= \sqrt{2}\sum_{k=0}^2 \brp{\frac{1}{\sqrt{2}}\bcoef{2}{k}} \fN_1(2x-k)
     \\\hline
       \mc{2}{|c|}{\includegraphics{../common/math/graphics/pdfs/n3_h.pdf}}
      &\includegraphics{../common/math/graphics/pdfs/n4_h.pdf}
     \\
       \mc{2}{|c|}{\fN_3(x)} %= \sqrt{2}\sum_{k=0}^4 \brp{\frac{1}{\sqrt{2}}\bcoef{4}{k}} \fN_3(2x-k)
      &\fN_4(x)
     \\\hline
  \end{array}$
  \caption{Some B-splines \label{fig:partuni_Nde}}
\end{figure}
%As it turns out, all \structe{B-splines} satisfy the \prope{partition of unity} property \xref{thm:bspline_punity}
%\emph{and} the \prope{multi-resolution} property \xref{thm:bspline_2x}.
%See \pref{ex:pun_n=2} (next), \prefpp{app:bspline}, and \prefpp{fig:partuni_Nde} for examples.
%All of the scaling functions in the examples in this section form a \hie{partition of unity},
%but yet are not orthogonal.
%In particular, note that for all of them $\sum_{n\in\Z} (-1)^n\fh_n = 0$ (see \pref{thm:pun_zero}).


%%--------------------------------------
%\begin{example}[\exm{order 1 B-spline wavelet system}]
%\footnote{
%  \citerp{strang89}{616},
%  \citerppgc{dau}{146}{148}{0898712742}{\textsection 5.4}
%  }
%\label{ex:sw_gh_tent}
%\exmx{tent function}
%%--------------------------------------
%The following figures illustrate scaling and wavelet coefficients and functions
%for the \hie{B-Spline $B_2$}, or \fncte{tent function}. % \xref{ex:pounity_tent}.
%The partition of unity formed by the scaling function $\fphi(x)$ is illustrated in \prefpp{ex:n1_pounity}. %\prefp{ex:pounity_tent}.
%\\\exbox{\begin{array}{m{40mm}m{51mm}m{51mm}}
%  $\begin{array}{>{\scy}c|r|r}
%    n & \mc{1}{c}{h_n} & \mc{1}{c}{g_n} \\
%    \hline
%      0   &  \brp{\frac{\sqrt{2}}{4}}  &   \brp{\frac{\sqrt{2}}{4}}  \\
%      1   & 2\brp{\frac{\sqrt{2}}{4}}  & -2\brp{\frac{\sqrt{2}}{4}}            \\
%      2   &  \brp{\frac{\sqrt{2}}{4}}  &   \brp{\frac{\sqrt{2}}{4}}
%  \end{array}$
%  &\includegraphics{../common/math/graphics/pdfs/tent_phi_h.pdf}&\includegraphics{../common/math/graphics/pdfs/tent_psi_g.pdf}
%\end{array}}
%\end{example}
%\begin{proof}
% \item These results follow from \prefp{thm:Bsplineh}.
%\\\indentx$\brp{\begin{array}{*{9}{c}}
%    &   &   &   & 1 &   &   &   &  \\
%    &   &   & 1 &   & 1 &   &   &  \\
%    &   & 1 &   & 2 &   & 1 &   &
%\end{array}}$
%\end{enumerate}
%\end{proof}
%
%--------------------------------------
\begin{example}[\exm{order 3 B-spline wavelet system}]
\footnote{
  \citerp{strang89}{616}
  }
\label{ex:sw_gh_bspline}
\exmx{B-spline}
%--------------------------------------
The following figures illustrate scaling and wavelet coefficients and functions
for a \hie{B-spline}.
\\\exbox{\begin{array}{m{40mm}m{51mm}m{51mm}}
  $\begin{array}{>{\scy}c|c|r}
    n & \mc{1}{c}{h_n} & \mc{1}{c}{g_n} \\
    \hline
      0   & { }\brp{\frac{\sqrt{2}}{16}}  &  { }\brp{\frac{\sqrt{2}}{16}}  \\
      1   & {4}\brp{\frac{\sqrt{2}}{16}}  & -{4}\brp{\frac{\sqrt{2}}{16}}  \\
      2   & {6}\brp{\frac{\sqrt{2}}{16}}  &  {6}\brp{\frac{\sqrt{2}}{16}}  \\
      3   & {4}\brp{\frac{\sqrt{2}}{16}}  & -{4}\brp{\frac{\sqrt{2}}{16}}  \\
      4   & { }\brp{\frac{\sqrt{2}}{16}}  &  { }\brp{\frac{\sqrt{2}}{16}}
  \end{array}$
%  &\includegraphics{../common/math/graphics/pdfs/bspline_phi_h.pdf}&\includegraphics{../common/math/graphics/pdfs/bspline_psi_g.pdf}
\end{array}}
\\\indentx\tbox{\includegraphics{../common/math/graphics/pdfs/n3_h.pdf}}
  \indentx\tbox{\includegraphics{../common/math/graphics/pdfs/n3_pz.pdf}}
\end{example}
\begin{proof}
\begin{enumerate}
  \item These results follow from \prefp{thm:Bsplineh}.
        \\\indentx$\brp{\begin{array}{c|*{9}{c}}
            n=0: &   &   &   &   & 1 &   &   &   &
          \\n=1: &   &   &   & 1 &   & 1 &   &   &
          \\n=2: &   &   & 1 &   & 2 &   & 1 &   &
          \\n=3: &   & 1 &   & 3 &   & 3 &   & 1 &
          \\n=4: & 1 &   & 4 &   & 6 &   & 4 &   & 1
        \end{array}}$
  \item Alternative \ldots
  \begin{align*}
    \Zh(z)
      &= \sqrt{2}\brs{\frac{(z+1)^4} {(1+1)^4 z^4}}
    \\&= \frac{\sqrt{2}}{2^4}\brs{\frac{\sum_{n=0}^4 \bcoef{4}{n}z^n} {z^4}}
      && \text{by \thme{Binomial Theorem} \ifxref{polynom}{thm:binomial}}
    \\&= \frac{\sqrt{2}}{16} \brs{\frac{z^4 + 4z^3 + 6z^2 + 4z^ + 1}{z^4} }
      && \text{by definition of $\bcoef{n}{k}$ \ifxref{binomial}{def:bcoef}}
    \\&= \frac{\sqrt{2}}{16} \brs{ 1 + 4z^{-1} + 6z^{-2} + 4z^{-3} + 1z^{-4}}
  \end{align*}
\end{enumerate}
\end{proof}


