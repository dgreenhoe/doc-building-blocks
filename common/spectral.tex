%============================================================================
% Daniel J. Greenhoe
% LaTeX file
%============================================================================


%=======================================
\chapter{Spectral Theory}
%=======================================

%=======================================
\section{Operator Spectrum}
%=======================================
%---------------------------------------
\begin{definition}
\label{def:eigen}
\footnote{
  %\citerp{michel1993}{439}
  \citerp{bollobas1999}{168},
  \citer{descartes1637},
  \citer{descartes1637e},
  \citer{cayley1858},
  \citerp{hilbert1904r1}{67},
  \citer{hilbert1912},
  \cithr{steen1973}
  }
%---------------------------------------
Let $\opA\in\clB{\spX}{\spY}$ be an operator over the linear spaces
$\spX=(\setX,F,\oplus,\otimes)$ and $\spY\eqd(\setY,F,\oplus,\otimes)$.
Let $\oppN(\opA)$ be the \hie{null space} of $\opA$.
\defbox{\begin{array}{l}
  \text{An \hid{eigenvalue} of $\opA$ is any value $\lambda$ such that there exists $\vx$ such that $\opA\vx=\lambda\vx$.} \\
  \text{The \hid{eigenspace} $\spH_\lambda$ of $\opA$ at eigenvalue $\lambda$ is $\oppN(\opA-\lambda\opI)$.} \\
  \text{An \hid{eigenvector} of $\opA$ associated with eigenvalue $\lambda$ is any element of $\oppN(\opA-\lambda\opI)$.}
\end{array}}
\end{definition}


%--------------------------------------
\begin{example}
\label{ex:spec_D}
\footnote{\citerp{pedersen2000}{79}}
%--------------------------------------
Let $\opD$ be the differntial operator.
\exbox{\begin{array}{rcl@{\qquad}D}
    \mc{4}{l}{\text{The set $\set{e^{\lambda x}}{\lambda\in\C}$ are the eigenvectors of $\opD$. }} \\
    \oppRes(\opD)   &=& \emptyset       & ($\opD$ has no non-spectral points whatsoever) \\
    \oppSpecp(\opD) &=& \oppSpec(\opD)  & (the spectrum of $\opD$ is all eigenvalues) \\
    \oppSpecc(\opD) &=& \emptyset       & ($\opD$ has no continuous spectrum) \\
    \oppSpecr(\opD) &=& \emptyset       & ($\opD$ has no resolvent spectrum)
\end{array}}
\end{example}
\begin{proof}
\begin{align*}
  (\opD-\lambda \opI) e^{\lambda x} 
    &= \opD e^{\lambda x}  - \lambda \opI e^{\lambda x}
  \\&= \lambda e^{\lambda x}  - \lambda e^{\lambda x}
  \\&= 0 
  && \forall \lambda\in\C
\end{align*}
This theorem and proof needs more work and investigation 
to prove/disprove its claims.\problem
\end{proof}



\begin{table}
\[\begin{array}{|lD || c|c|c||}
  \hline
  \hline
  \mc{2}{|c||}{\text{quantity}}   & 
  \oppN(\opA-\lambda\opI)=\setn{\vzero} &
  \overline{\oppR(\opA -\lambda\opI)}=\spX &
  (\opA -\lambda\opI)^{-1} \in \clB{\spX}{\spY} 
  \\&&
  \text{\scriptsize($\vx=\vzero$ is the only solution)} &
  \text{\scriptsize(dense)}  &
  \text{\scriptsize(continuous/bounded)} 
  \\
  \hline
  \hline
  \oppRes(\opA) & (resolvent set)         & \ltrue     & \ltrue   & \ltrue         \\
 %\oppSpec (\opA) & (spectrum)            & \lfals     &          &                \\
  \oppSpecp(\opA) & (point spectrum)      & \lfals     &          &                \\
  \oppSpecr(\opA) & (residual spectrum)   & \ltrue     & \lfals   &                \\
  \oppSpecc(\opA) & (continuous spectrum) & \ltrue     & \ltrue   & \lfals         \\
  \hline
  \hline
\end{array}\]
\caption{Spectrum of an operator $\opA$ \label{tbl:op_spectrum}}
\end{table}

%---------------------------------------
\begin{definition}
\label{def:op_res_set}
\label{def:op_spectrum}
\footnote{
  \citerp{michel1993}{439}
  }
%---------------------------------------
Let $\opA\in\clB{\spX}{\spY}$ be an operator over the linear spaces
$\spX=(\setX,F,\oplus,\otimes)$ and $\spY\eqd(\setY,F,\oplus,\otimes)$.
\defboxp{\indxs{\oppRes}\indxs{\oppSpec}
  The \hid{resolvent set} $\oppRes(\opA)$ of operator $\opA$ is defined as
  \[
    \oppRes(\opA) \eqd 
    \set{\lambda\in F}
        {\begin{array}{ll @{\qquad}D @{\qquad}>{$}l<{$}}
          1. & \oppN(\opA -\lambda\opI)=\setn{\vzero}   & (no non-zero eigenvectors) & and \\
          2. & \overline{\oppR(\opA -\lambda\opI)}=\spX  & (the range is dense in $\spX$).      &   and  \\  
          3. & (\opA -\lambda\opI)^{-1} \in \clB{\spX}{\spY}  & (inverse is continuous/bounded).   & 
         \end{array}}
  \]
  The \hid{spectrum} $\oppSpec(\opA)$ of operator $\opA$ is defined as
  \[ \oppSpec(\opA) \eqd F \setd \oppRes(\opA). \]
%  \set{\lambda\in F}{\lambda\notin \oppRes(\opA)}
  }
\end{definition}

%---------------------------------------
\begin{definition}
\label{def:op_spectrum_pcr}
\footnote{
  \citerp{bollobas1999}{168},
  %\citerp{michel1993}{440},
  \citorpp{hilbert1906r4}{169}{172}
  }
%---------------------------------------
Let $\opA\in\clB{\spX}{\spY}$ be an operator over the linear spaces
$\spX=(\setX,F,\oplus,\otimes)$ and $\spY\eqd(\setY,F,\oplus,\otimes)$.
\defbox{\indxs{\oppSpecp}\indxs{\oppSpecr}\indxs{\oppSpecc}
  \begin{array}{rcl}
  \mc{3}{l}{\text{The \hid{point spectrum} $\oppSpecp(\opA)$ of operator $\opA$ is defined as }} \\
  \qquad\oppSpecp(\opA) 
  &\eqd& 
  \set{\lambda\in F}
      {\begin{array}{ll @{\qquad}D @{\qquad}>{$}l<{$}}
        1. & \oppN(\opA -\lambda\opI)\supseteq\setn{\vzero}   & (has non-zero eigenvector) &  \\
      \end{array}}
  \\
  \mc{3}{l}{\text{The \hid{residual spectrum} $\oppSpecr(\opA)$ of operator $\opA$ is defined as }} \\
  \qquad\oppSpecr(\opA) 
  &\eqd& 
  \set{\lambda\in F}
      {\begin{array}{ll @{\qquad}D @{\qquad}>{$}l<{$}}
        1. & \oppN(\opA -\lambda\opI)=\setn{\vzero} & (no non-zero eigenvectors) & and \\
        2. & \overline{\oppR(\opA -\lambda\opI)}\ne \spX  & (not dense in $\spX$---has gaps).      &     
      \end{array}}
  \\
  \mc{3}{l}{\text{The \hid{continuous spectrum} $\oppSpecc(\opA)$ of operator $\opA$ is defined as }} \\
  \qquad\oppSpecc(\opA) 
  &\eqd& 
  \set{\lambda\in F}
      {\begin{array}{ll @{\qquad}D @{\qquad}>{$}l<{$}}
        1. & \oppN(\opA -\lambda\opI)=\setn{\vzero}   & (no non-zero eigenvectors) & and \\
        2. & \overline{\oppR(\opA -\lambda\opI)}=\spX           & (dense in $\spX$).      &  and \\
        3. & (\opA -\lambda\opI)^{-1} \notin \clB{\spX}{\spY}  & (not continuous / not bounded)   & 
      \end{array}}
\end{array}}
\end{definition}


\begin{minipage}[c]{10\tw/16-1ex}
  \begin{center}%
  \begin{fsL}%
  \setlength{\unitlength}{\tw/(1000)}%
  \begin{picture}(1000,700)(-700,-600)%
  %{\color{graphpaper}\graphpaper[100](-700,-600)(1000,700)}%
    \thicklines%
      {\color{blue}%
        \put(-700,-600){\framebox(200,100)[c]{}}%
        }%
      {\color{black}%
        \put(-700,-600){\makebox(200,100)[c]{$\lambda\in\oppRes(\opA)$}}%
        }%
    \begin{picture}(0,0)(0,0)%
      {\color{blue}%
        \put(-100,   0){\line( 1, 1){100}}%
        \put( 100,   0){\line(-1, 1){100}}%
        \put(-100,   0){\line( 1,-1){100}}%
        \put( 100,   0){\line(-1,-1){100}}%
        \put( 100,-200){\framebox(200,100)[c]{}}%
        }%
      {\color{red}%
        \put(-100,   0){\line(-1,-1){100}}%
        \put( 100,   0){\line( 1,-1){100}}%
        \put(-160,-40){\makebox(0,0)[br]{true}}%
        \put( 160,-40){\makebox(0,0)[bl]{false}}%
        }%
      {\color{black}%
        \put(   0, 0){\makebox(0,0)[c]{$\vx=\vzero$ only?}}%
        \put( 100,-200){\makebox(200,100)[c]{$\lambda\in\oppSpecp(\opA)$}}%
        }%
    \end{picture}%
    \begin{picture}(0,0)(200,200)%
      {\color{blue}%
        \put(-100,   0){\line( 1, 1){100}}%
        \put( 100,   0){\line(-1, 1){100}}%
        \put(-100,   0){\line( 1,-1){100}}%
        \put( 100,   0){\line(-1,-1){100}}%
        \put( 100,-200){\framebox(200,100)[c]{}}%
        }%
      {\color{red}%
        \put(-100,   0){\line(-1,-1){100}}%
        \put( 100,   0){\line( 1,-1){100}}%
        \put(-160,-40){\makebox(0,0)[br]{true}}%
        \put( 160,-40){\makebox(0,0)[bl]{false}}%
        }%
      {\color{black}%
        \put(   0, 0){\makebox(0,0)[c]{dense?}}%
        \put( 100,-200){\makebox(200,100)[c]{$\lambda\in\oppSpecr(\opA)$}}%
        }%
    \end{picture}%
    \begin{picture}(0,0)(400,400)%
      {\color{blue}%
        \put(-100,   0){\line( 1, 1){100}}%
        \put( 100,   0){\line(-1, 1){100}}%
        \put(-100,   0){\line( 1,-1){100}}%
        \put( 100,   0){\line(-1,-1){100}}%
        \put( 100,-200){\framebox(200,100)[c]{}}%
        }%
      {\color{red}%
        \put(-100,   0){\line(-1,-1){100}}%
        \put( 100,   0){\line( 1,-1){100}}%
        \put(-160,-40){\makebox(0,0)[br]{true}}%
        \put( 160,-40){\makebox(0,0)[bl]{false}}%
        }%
      {\color{black}%
        \put(   0, 0){\makebox(0,0)[c]{not continuous?}}%
        \put( 100,-200){\makebox(200,100)[c]{$\lambda\in\oppSpecc(\opA)$}}%
        }%
    \end{picture}%
  \end{picture}%
  \end{fsL}%
  \end{center}%
\end{minipage}%
\hfill
\begin{minipage}[c]{6\tw/16-1ex}
The spectral components' definitions are illustrated in the figure to the left and 
summarized in \prefpp{tbl:op_spectrum}.
Let a family of operators $\opB(\lambda)$ be defined with respect to an operator $\opA$
such that $\opB(\lambda)\eqd(\opA-\lambda\opI)$.
Normally, we might expect a ``normal" or ``regular" or even ``mundane" operator 
$\opB(\lambda)$ to have the properties
  \begin{dingautolist}{"AC}
    \item $\opB(\lambda)\vx=0$ if and only if $\vx=\vzero$
    \item $\opB(\lambda)\vx$ spans virtually all of $\spX$ as we vary $\vx$
    \item $\opBi(\lambda)$ is continuous.
  \end{dingautolist}
\end{minipage}
After all, these are the properties that we would have if $\opB(\lambda)$ 
were simply an \prop{affine} operator in the field of real numbers---
such as $\brs{\opB(\lambda)}(x)\eqd \brs{\lambda}(x)=\lambda x$
which is $0$ if and only if $x=0$, has range $\oppR(\lambda)=\R$,
and its inverse $\lambda^{-1}x$ is continuous.

If for some $\lambda$ the operator $\opB(\lambda)$ 
does have all these ``regular" properties, 
then that $\lambda$ part of the \hie{resolvent set} of $\opA$ and $\lambda$ is called
\hie{regular}.
However if for some $\lambda$ the operator $\opB(\lambda)$ 
fails any of these conditions, then that $\lambda$ part of the \hie{spectrum} of $\opA$.
And which conditions it fails determines which component of the spectrum it is in.

%---------------------------------------
\begin{theorem}
\footnote{
  \citerp{michel1993}{440}
  }
%---------------------------------------
Let $\opA\in\clB{\spX}{\spY}$ be an operator.
\thmbox{
  \oppSpec(\opA) = \oppSpecp(\opA) \setu \oppSpecc(\opA) \setu \oppSpecr(\opA)
  }
\end{theorem}

%---------------------------------------
\begin{theorem}[\thmd{Spectral Theorem}]
\footnote{
  \citerp{michel1993}{457},
  \citerp{bollobas1999}{200}
  }
\label{thm:spectral_theorem}
\index{spectral theorem}
\index{theorems!spectral theorem}
\footnote{
  \citer{hilbert1906r4},
  \citer{hilbert1912},
  \citer{vonNeumann1929},
  \citer{deWitt}
  }
%---------------------------------------
Let $\opN\in\clFxy$ be an operator.
\thmbox{\begin{array}{l}
  \left.\begin{array}{Fl}
    (1). & \mcom{\opNa\opN=\opN\opNa}{$\opN$ is \hie{normal}}\\
    (2). & \text{$\opN$ is \hie{compact}}
  \end{array}\right\}
  \qquad\implies\qquad
  \left\{\begin{array}{F>{\ds}rc>{\ds}l}
    (1). & \opN &=& \sum_n \lambda_n \opP_n  \\
    (2). & \sum_n \opP_n &=& \opI \\
    (3). & \opP_n\opP_m &=& \kdelta_{n-m} \opP_n \\
    (4). & \oppDim(\spH_n) &<& \infty \\
    (5). & \mc{3}{l}{\seto{\set{\lambda_n}{\lambda_n\ne 0}}  \text{is \prope{countably infinite}}} \\
  \end{array}\right.
  \\
  where
  \\\quad
  \begin{array}{>{\ds}rcl@{\qquad}D}
    \seq{\lambda_n}{n\in\Z} &\eqd& \oppSpecp(\opN) & (eigenvalues of $\opN$) \\
    \spH_n &\eqd& \oppN(\opN-\lambda_n\opI)        & ($\lambda_n$ is the eigenspace of $\opN$ at $\lambda_n$ in $\spY$) \\
    \spH_n &=& \opP_n \spY                         & ($\opP_n$ is the projection operator that generates $\spH_n$)
  \end{array}
  \end{array}}
\end{theorem}

%=======================================
\section{Fredholm kernels}
%=======================================
%---------------------------------------
\begin{definition}
\label{def:op_int}
\footnote{
  \citerp{michel1993}{425}
  }
\index{Fredholm integral equation of the first kind|textbf}
%---------------------------------------
\defbox{\begin{array}{l@{\qquad}l}
  \mc{2}{l}{\text{A \hid{Fredholm operator} $\opK$ is defined as}}
  \\&
  \brs{\opK\ff}(t) 
  \eqd 
  \mcom{\int_a^b \mcom{\kappa(t,s)}{kernel} \ff(s) \dd{s} }
       {Fredholm integral equation of the first kind \footnotemark}
  \qquad \forall \ff\in \spL_2([a,b])
  \end{array}}
\footnotetext{
   The equation $\int_u \kappa(t,s) \ff(s) \dd{s}$ is a 
   \fnctd{Fredholm integral equation of the first kind} and
   $\kappa(t,u)$ is the {\bf kernel} of the equation.
   References: 
     \citer{fredholm1900},
     \citerp{fredholm1903}{365},
     \citerp{michel1993}{97},
     \citerp{keener}{101}
   }
\end{definition}

%---------------------------------------
\begin{example}
%---------------------------------------
Examples of \prop{Fredholm operators} include
\[
\begin{array}{llrclrcl}
  1. & \text{Fourier Transform}
     & [\opFT \fx](f) &=& \int_t \fx(t) e^{-i2\pi ft}\dt 
     & \kappa(t,f)    &=& e^{-i2\pi ft}
\\
  2. & \text{Inverse Fourier Transform}
     & [\opFTi \Fx](t) &=& \int_f \Fx(f) e^{i2\pi ft}\df 
     & \kappa(f,t)    &=& e^{i2\pi ft}
\\
  3. & \text{Laplace operator}
     & [\opL \fx](s) &=& \int_t \fx(t) e^{-st}\dt 
     & \kappa(t,s)   &=& e^{-st}
\\
  4. & \text{autocorrelation operator}
     & [\opR \fx](t) &=& \int_s R(t,s) \fx(s) \dds 
     & \kappa(t,s)   &=& R(t,s) %\text{ (autocorrelation function)}
\end{array}
\]
\end{example}


%---------------------------------------
\begin{theorem}
\label{thm:intop_KkKk}
%---------------------------------------
Let $\opK$ be a \prop{Fredholm operator} with kernel $\kappa(t,s)$ and adjoint $\opKa$.
\thmbox{
  \brs{\opK\ff}(t) = \int_\setA \kappa(t,s) \ff(s) \dds
  \qquad\iff\qquad
  \brs{\opKa\ff}(t) = \int_\setA \kappa^\ast(s,t) \ff(s) \dds
  }
\end{theorem}
\begin{proof}
\begin{align*}
   \brs{\opK\ff}(t) 
     &= \int_\setA \kappa(t,s) \ff(s) \dds
   \\\iff
   \inprod{\brs{\opK \ff}(t)}{\fg(t)}
     &= \inprod{\int_s \kappa(t,s) \ff(s) \dds}{\fg(t)}
     && \text{by left hypothesis}
   \\&= \int_s \ff(s) \inprod{\kappa(t,s)  }{\fg(t)} \dds
     && \text{by additivity property of $\inprodn$\ifsxref{vsinprod}{def:inprod}}
   \\&= \int_s \ff(s) \inprod{\fg(t)}{\kappa(t,s)}^\ast \dds
     && \text{by conjugate symmetry property of $\inprodn$\ifsxref{vsinprod}{def:inprod}}
   \\&= \inprod{ \ff(s) }{\inprod{\fg(t)}{\kappa(t,s)} }
     && \text{by local definition of $\inprodn$}
   \\&= \inprod{ \ff(s) }{\mcom{\int_t \kappa^\ast(t,s) \fg(t) \dt}{$\brs{\opKa\fg}(s)$} }
     && \text{by local definition of $\inprodn$}
   \\\iff
   \brs{\opKa\fg}(s) 
     &= \int_\setA \kappa^\ast(t,s) \fg(t) \dt
     && \text{by right hypothesis}
   \\\iff
   \brs{\opKa\fg}(\sigma) 
     &= \int_\setA \kappa^\ast(\tau,\sigma) \fg(\tau) \dtau
     && \text{by change of variable: $\tau=t,\, \sigma=s$}
   \\\iff
   \brs{\opKa\ff}(t) 
     &= \int_\setA \kappa^\ast(s,t) \ff(s) \dds
     && \text{by change of variable: $t=\sigma,\, s=\tau,\, \ff=\fg$}
\end{align*}
\end{proof}

%---------------------------------------
\begin{theorem}
\label{thm:intop_KKkk}
\footnote{
  \citerp{michel1993}{430}
  }
%---------------------------------------
Let $\opK$ be an \prop{Fredholm operator} with kernel $\kappa(t,s)$ and adjoint $\opKa$.
\thmbox{
  \mcom{\opK=\opKa}{$\opK$ is self-adjoint}
  \qquad\iff\qquad
  \mcom{\kappa(t,s) = \kappa^\ast(s,t)}{kernel is \prop{conjugate symmetric}}
  }
\end{theorem}
\begin{proof}
\begin{align*}
  \opK = \opKa
    &\iff \int_\setA \kappa(t,s) \ff(s) \dds = \int_\setA \kappa^\ast(s,t) \ff(s) \dds
    &&    \text{by \prefp{thm:intop_KkKk}}
  \\&\iff \kappa(t,s) =  \kappa^\ast(s,t) 
\end{align*}
\end{proof}



%---------------------------------------
\begin{theorem}[Mercer's Theorem]
\label{thm:mercer}
\footnote{
  \citerp{gohberg}{198},
  \citerpp{courant}{138}{140},
  \citerp{mercer1909}{439}
  }
\index{Mercer's Theorem}
\index{theorems!Mercer's Theorem}
%---------------------------------------
Let $\opK$ be an \prop{Fredholm operator} with kernel $\kappa(t,s)$ and
eigensystem $\seq{(\lambda_n,\phi_n(t))}{n\in\Z}$.
\thmbox{
  \left.\begin{array}{ll}
    1.& \mcom{\int_a^b \int_a^b \kappa(t,s) \ff(t) \ff^\ast(s) \ds\dt \ge 0}{positive} \quad \text{and }\\
    2.& \kappa(t,s) \text{ is continuous on } [a,b]\times[a,b]
  \end{array}\right\}
  \implies
  \left\{\begin{array}{ll}
    1.& \kappa(t,s) = \sum_n \lambda_n \fphi_n(t) \fphi_n^\ast(s) \quad\text{and}\\
    2.& \kappa(t,s) \text{ converges absolutely and} \\
      & \text{uniformly on $[a,b]\times[a,b]$}
  \end{array}\right.
  }
\end{theorem}
\begin{proof}

\end{proof}

