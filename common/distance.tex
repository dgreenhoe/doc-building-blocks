%============================================================================
% LaTeX File
% Daniel J. Greenhoe
%============================================================================

%======================================
\chapter{Distance Spaces}
\label{chp:dspace}
%======================================

%  \begin{figure}[t]
%  \[\begin{array}{*{5}{>{\ds}c}}
%       && \fcolorbox{blue}{bg_blue}{\parbox[c]{3\tw/16}{\centering topological space}}
%    \\ && \\setUparrow & \Nwarrow
%    \\ \fcolorbox{blue}{bg_blue}{\parbox[c]{3\tw/16}{\centering vector space}}
%       && \fcolorbox{red}{bg_red}{\parbox[c]{3\tw/16}{\centering metric space}}
%       && \fcolorbox{blue}{bg_blue}{\parbox[c]{3\tw/16}{\centering measure space}}
%    \\ & \Nwarrow & \\setUparrow && \\setUparrow
%    \\ && \fcolorbox{blue}{bg_blue}{\parbox[c]{3\tw/16}{\centering normed linear space}}
%       && \fcolorbox{blue}{bg_blue}{\parbox[c]{3\tw/16}{\centering probability space}}
%    \\ &  \Nearrow  & & \Nwarrow
%    \\ \fcolorbox{blue}{bg_blue}{\parbox[c]{3\tw/16}{\centering inner-product space}}
%       &&&&
%       \fcolorbox{blue}{bg_blue}{\parbox[c]{3\tw/16}{\centering Banach space}}
%    \\ & \Nwarrow && \Nearrow
%    \\ && \fcolorbox{blue}{bg_blue}{\parbox[c]{3\tw/16}{\centering Hilbert space}}
%    \\ &\Nearrow & & \Nwarrow
%    \\ \fcolorbox{blue}{bg_blue}{\parbox[c]{3\tw/16}{\centering$\spII$}}
%       &&&&
%       \fcolorbox{blue}{bg_blue}{\parbox[c]{3\tw/16}{\centering$\spLL$}}
%    \\ & \Nwarrow && \Nearrow
%    \\ && \fcolorbox{blue}{bg_blue}{\parbox[c]{3\tw/16}{\centering$0$}}
%  \end{array}\]
%  \end{figure}


%\qboxnps
%  {
%    \href{http://en.wikipedia.org/wiki/Eric_Temple_Bell}{Eric Temple Bell}
%    (\href{http://www-history.mcs.st-andrews.ac.uk/Timelines/TimelineG.html}{1883--1960}),
%    mathematician and author
%    \index{Bell, Eric Temple}
%    \footnotemark
%  }
%  {../common/people/bell.jpg}
%  {The cowboys have a way of trussing up a steer or a pugnacious bronco
%    which fixes the brute so that it can neither move nor think.
%    This is the hog-tie, and it is what Euclid did to geometry.}
%  \footnotetext{\begin{tabular}[t]{ll}
%    quote: & \citerp{crayshaw}{191} \\
%           & \url{http://www-groups.dcs.st-and.ac.uk/~history/Quotations/Bell.html} \\
%    image: & \url{http://www-history.mcs.st-andrews.ac.uk/PictDisplay/Bell.html}
%  \end{tabular}}

A \structe{distance space} \xref{def:dspace} can be defined as a \structe{metric space} \xref{def:metric}
without the \prope{triangle inequality} constraint.
Much of the material in this section about \structe{distance space}s is
standard in \structe{metric space}s.
However, this paper works through this material again to demonstrate ``how far we can go", and can't go, 
without the \prope{triangle inequality}.

%======================================
\section{Fundamental structure of distance spaces}
%======================================
%======================================
\subsection{Definitions}
%======================================
%---------------------------------------
\begin{definition}
\footnote{
  \citePpc{menger1928}{76}{``Abstand $a$ $b$ definiert ist\ldots" (distance from $a$ to $b$ is defined as\ldots")},
  \citePpc{wilson1931a}{361}{\textsection1., ``distance", ``semi-metric space"},
  \citePp{blumenthal1938}{38},
  \citerpgc{blumenthal1953}{7}{0828402426}{``{\scshape Definition 5.1.} A distance space is called semimetric provided\ldots"},
  \citePpc{galvin1984}{67}{``distance function"},
  \citerpgc{laos1998}{118}{9810231806}{``distance space"},
  \citerpgc{khamsi2001}{13}{0471418250}{``semimetric space"},
  \citePpc{bessenyei2014}{2}{``semimetric space"}, %{{\scshape Conventions And Basic Notions}},
  \citerpgc{deza2014}{3}{3662443422}{``\textbf{distance} (or \textbf{dissimilarity})"}
  }
\label{def:distance}
\label{def:dspace}
%---------------------------------------
%Let $\setX$ be a set and $\Rnn$ the set of non-negative real numbers.
\defboxt{
  A function $\hxs{\distancen}$ in the set $\clF{\setX\times\setX}{\R}$\ifsxref{relation}{def:clFxy} is a \fnctd{distance} if
  \\\indentx$\begin{array}{F rcl CDD}
        1. & \distance{x}{y} &\ge& 0               & \forall x,y   \in\setX & (\prope{non-negative})   & and 
      \\2. & \distance{x}{y} &=  & 0  \iff x=y     & \forall x,y   \in\setX & (\prope{nondegenerate})  & and 
      \\3. & \distance{x}{y} &=  & \distance{y}{x} & \forall x,y   \in\setX & (\prope{symmetric})      & 
    \end{array}$
  \\The pair $\dspaceX$ is a \structd{distance space} if $\distancen$ is a \fncte{distance} on a set $\setX$.
  \\A \fncte{distance} is also called a \fnctd{dissimilarity}.
  }
\end{definition}

%In a \structe{metric space} \xref{def:metric}\index{space!metric},
%it is sometimes useful to know the maximum distance between any two points in the set.
%This maximum distance is called the \fncte{diameter} of the set 
%(\pref{def:diam}, next definition).
%Some \structe{distance space}s \xref{def:distance} and all \structe{metric space}s \xref{def:metric} induce 
%\structe{topological space}s \xref{def:topology}.
%However the \fncte{set function} \xref{def:setf}
%\fncte{set diameter} (next definition) and the related property of \prope{boundedness} \xref{def:bounded} 
%are fundamentally \structe{distance space} concepts, 
%not topological ones.\footnote{in \structe{metric space}: \citerpg{munkres2000}{121}{0131816292}}
%---------------------------------------
\begin{definition}
\label{def:diam}
\footnote{
  in \structe{metric space}:
  %\citerpg{davis2005}{28}{0071243399} \\
  \citerp{hausdorff1937e}{166},
  \citerp{copson1968}{23},
  \citerp{michel1993}{267},
  \citerpg{molchanov2005}{389}{185233892X}
  }
%---------------------------------------
Let $\dspaceX$ be a \structe{distance space} and $\psetX$ be the \structe{power set} of $\setX$ \xref{def:pset}.
\defboxt{
  The \vald{diameter} in $\dspaceX$ of a set $\setA\in\psetX$ is
  \quad$\ds
  \diam\setA \eqd
    \brp{\begin{array}{ll}
      0                            & \text{for } \setA=\emptyset \\
      \sup\set{\distance{x}{y}}{x,y\in\setA} & \text{otherwise}
    \end{array}}$
  }
\end{definition}

%The \fncte{diameter} of a set is a special case of a  
%\prefpp{sec:topmetric} presents some topological properties of metric spaces.
%---------------------------------------
\begin{definition}
\footnote{
  in \structe{metric space}:
  \citerpc{thron1966}{154}{definition 19.5},
  \citerpg{bruckner1997}{356}{013458886X}
  }
\label{def:bounded}
%---------------------------------------
Let $\dspaceX$ be a \structe{distance space}.
Let $\psetX$ be the \structe{power set} \xref{def:pset} of $\setX$.
\defboxt{
  A set $\setA$ is \propd{bounded} in $\dspaceX$ if
  \\\indentx$\setA\in\psetX$ and $\diam\setA<\infty$.
  }
\end{definition}

%%---------------------------------------
%\begin{remark}
%%---------------------------------------
%Although the function $\diam$ is a \fncte{set function}, it is not a \fncte{measure}.
%\end{remark}
%\begin{proof}
%Let $\setA$ and $\setB$ be sets that are \prope{disjoint} and \prope{disconnected}
%with respect to each other.
%\begin{align*}
%  \diam{\setA \setu \setB}
%    &= \diam\setA + \diam\setB + \inf\set{\distance{x}{y}}{x,y\in\setA}
%  \\&\ge \diam\setA + \diam\setB
%  \\\implies & \text{$\diam$ is not a measure}
%    && \text{by definition of measure \ifdochas{measure}{\prefpo{def:measure}}}
%\end{align*}
%\end{proof}


%%---------------------------------------
%\begin{definition}
%\label{def:ods}
%%---------------------------------------
%\defboxp{
%  The triple $\omsD$ is an \structd{ordered distance space} if
%  $\dspaceX$ is a \structe{distance space} \xref{def:dspace}
%  and $\osetX$ is an \structe{ordered set} \xref{def:poset}.
%  }
%\end{definition}

%=======================================
\subsection{Properties}
%=======================================
%%---------------------------------------
%\begin{remark}
%\footnote{
%  \citePpc{heath1961}{810}{{\scshape Theorem}},
%  \citePpc{galvin1984}{71}{{\scshape 2.3 Lemma}}
%  }
%%---------------------------------------
%Although every \structe{metric space} \xref{def:metric} induces a \structe{topology} \xref{def:topology},
%a \structe{distance space} does \textbf{not} in general induce a topology.
%\end{remark}
%
%---------------------------------------
\begin{remark}
%---------------------------------------
%Let $\dspaceX$ be a \structb{distance space} \xref{def:dspace}.
Let $\seqxZ{x_n}$ be a \fncte{sequence} in a \structe{distance space} $\dspaceX$.
The \structe{distance space} $\dspaceX$ does not necessarily have all the nice properties that a 
\structe{metric space} \xref{def:metric} has.
In particular, note the following:
\\\indentx\remboxp{$\begin{array}{FMcMM}
  1. & $\distancen$   is a \fncte{distance} in $\dspaceX$ &\notimplies& $\distancen$ is \prope{continuous} in $\dspaceX$                  & \xref{ex:dspace_21}.\\
  2. & $\balln$ is an \structe{open ball} in $\dspaceX$ &\notimplies& $\balln$ is \prope{open} in $\dspaceX$                              & \xref{ex:dspace_1n}.\\
  3. & $\baseB$ is the set of all                                    &\notimplies& $\baseB$ is a \structe{base} for a                     & \xref{ex:dspace_1n}.\footnotemark\\
     &                            \structe{open ball}s in $\dspaceX$ &           &                                    topology on $\setX$ & \\
  4. & $\seqn{x_n}$ is \prope{convergent} in $\dspaceX$ &\notimplies& limit is \prope{unique}                                             & \xref{ex:dspace_01}.\\
  5. & $\seqn{x_n}$ is \prope{convergent} in $\dspaceX$ &\notimplies& $\seqn{x_n}$ is \prope{Cauchy} in $\dspaceX$                        & \xref{ex:dspace_1n}.\\
\end{array}$}
%In a \structe{metric space} \xref{def:metric}, if a \fncte{sequence} has a limit, that limit is unique.
%This is not true in general for \structe{distance space}s \xref{def:distance}.
%See \prefpp{ex:dspace_01} for a distance space which converges to two distinct limits.
\footnotetext{
  \citePpc{heath1961}{810}{{\scshape Theorem}},
  \citePpc{galvin1984}{71}{{\scshape 2.3 Lemma}}
  }
\end{remark}


%=======================================
\section{Open sets in distance spaces}
%=======================================
%=======================================
\subsection{Definitions}
%=======================================
%\pref{def:ball} (next) defines the \structe{open ball}.
%In a \structe{metric space} \xref{def:metric}\index{space!metric}, 
%sets are often specified in terms of an \prope{open ball};
%and an open ball is specified in terms of a metric.
%---------------------------------------
\begin{definition}
\footnote{
  in \structe{metric space}:
  \citerp{ab}{35}
  }
\label{def:ball}
\label{def:ballc}
%---------------------------------------
Let $\dspaceX$ be a \structe{distance space} \xref{def:distance}.
Let $\Rp$ be the \structe{set of positive real numbers} \xref{def:Rx}.
\\\defboxt{$\begin{array}{Mrcl}
  An \structd{open ball}   centered at $x$ with radius $r$ is the set & \ball{x}{r}  &\eqd& \set{y\in\setX}{\distance{x}{y}<r}.\\
  A  \structd{closed ball} centered at $x$ with radius $r$ is the set & \ballc{x}{r} &\eqd& \set{y\in\setX}{\distance{x}{y}\le r}.
\end{array}$}
\end{definition}

%Open balls will often ``appear" different in different metric spaces. 
%Some examples include the following \xref{ex:taxicab}:\\
%\setlength{\unitlength}{\tw/6000}
%\begin{tabular}{cll>{\footnotesize(}l<{\footnotesize)}}
%%\begin{tabular}{cl}%<{:}l>{\footnotesize(}l<{\footnotesize)}}
%  \begin{picture}(300,350)(-130,-130)
%    \thicklines
%    \color{axis}%
%      \put(-130,   0){\line(1,0){260} }%
%      \put(   0,-130){\line(0,1){260} }%
%    \color{blue}%
%      \qbezier( 100,0)( 50, 50)(0, 100)%
%      \qbezier(-100,0)(-50, 50)(0, 100)%
%      \qbezier(-100,0)(-50,-50)(0,-100)%
%      \qbezier( 100,0)( 50,-50)(0,-100)%
%  \end{picture}
%  & \prope{taxi-cab metric}%  & \pref{ex:ms_taxi} & \prefpo{ex:ms_taxi} 
%  \\
%  \begin{picture}(300,300)(-130,-130)
%    \thicklines
%    \color{axis}%
%      \put(-130,   0){\line(1,0){260} }%
%      \put(   0,-130){\line(0,1){260} }%
%    \color{blue}%============================================================================
% NCTU - Hsinchu, Taiwan
% LaTeX File
% Daniel Greenhoe
%
% Unit circle with radius 100
%============================================================================

\qbezier( 100,   0)( 100, 41.421356)(+70.710678,+70.710678) % 0   -->1pi/4
\qbezier(   0, 100)( 41.421356, 100)(+70.710678,+70.710678) % pi/4-->2pi/4
\qbezier(   0, 100)(-41.421356, 100)(-70.710678,+70.710678) %2pi/4-->3pi/4
\qbezier(-100,   0)(-100, 41.421356)(-70.710678,+70.710678) %3pi/4--> pi 
\qbezier(-100,   0)(-100,-41.421356)(-70.710678,-70.710678) % pi  -->5pi/4
\qbezier(   0,-100)(-41.421356,-100)(-70.710678,-70.710678) %5pi/4-->6pi/4
\qbezier(   0,-100)( 41.421356,-100)( 70.710678,-70.710678) %6pi/4-->7pi/4
\qbezier( 100,   0)( 100,-41.421356)( 70.710678,-70.710678) %7pi/4-->2pi


%
%  \end{picture}
%  & \prope{Euclidean metric}% & \pref{ex:ms_euclidean} & \prefpo{ex:ms_euclidean} 
%  \\
%  \begin{picture}(300,300)(-130,-130)
%    \thicklines
%    \color{axis}%
%      \put(-130,   0){\line(1,0){260} }%
%      \put(   0,-130){\line(0,1){260} }%
%    \color{blue}%
%      \put(-100,-100){\line( 1, 0){200} }%
%      \put(-100,-100){\line( 0, 1){200} }%
%      \put( 100, 100){\line(-1, 0){200} }%
%      \put( 100, 100){\line( 0,-1){200} }%
%  \end{picture}
%  & \prope{sup metric}% & \pref{ex:ms_sup} & \prefpo{ex:ms_sup} 
%%  \\
%%  \begin{picture}(300,300)(-130,-130)%
%%    %{\color{graphpaper}\graphpaper[10](-150,-150)(300,300)}%
%%    \thicklines%
%%    \color{axis}%
%%      \put(-130,   0){\line(1,0){260} }%
%%      \put(   0,-130){\line(0,1){260} }%
%%    \color{blue}%
%%      \qbezier( 100,0)(0,0)(0, 100)%
%%      \qbezier( 100,0)(0,0)(0,-100)%
%%      \qbezier(-100,0)(0,0)(0,-100)%
%%      \qbezier(-100,0)(0,0)(0, 100)%
%%  \end{picture}
%%  & \prope{parabolic metric} & \pref{ex:ms_parabolic} & \prefpo{ex:ms_parabolic}
%%  \\ 
%%  \begin{picture}(400,400)(-200,-200)%
%%    \thicklines%
%%    \color{axis}%
%%      \put(-200,   0){\line(1,0){400} }%
%%      \put(   0,-200){\line(0,1){400} }%
%%    \color{blue}%
%%      \qbezier( 100,0)(100,100)(0, 100)%
%%      \qbezier(-171,0)(-50,50)(0, 100)%
%%      \qbezier(-171,0)(-50,-50)(0,-171)%
%%      \qbezier( 100,0)(50,-50)(0,-171)%
%%  \end{picture}
%%  & \prope{exponential metric} & \pref{ex:ms_32x} & \prefpo{ex:ms_32x}
%%  \\ 
%%  \begin{picture}(300,300)(-150,-150)%
%%    \thicklines%
%%    \color{axis}%
%%      \put(-130,   0){\line(1,0){260} }%
%%      \put(   0,-130){\line(0,1){260} }%
%%    \color{blue}%
%%      \qbezier( 100,0)(70,70)(0, 100)%
%%      \qbezier(-100,0)(-70,70)(0, 100)%
%%      \qbezier(-100,0)(-70,-70)(0,-100)%
%%      \qbezier( 100,0)(70,-70)(0,-100)%
%%  \end{picture}
%%  & \prope{tangential metric} & \pref{ex:ms_tan} & \prefpo{ex:ms_tan}
%\end{tabular}

%%---------------------------------------
%\begin{theorem}
%\footnote{
%  \citerpp{isham1999}{10}{11} \\
%  \citor{birkhoff1933}
%  }
%%---------------------------------------
%\thmbox{\begin{array}{rc>{\ds}l}
%  \distancesub{1}{x}{y} \join \distancesub{2}{x}{y} 
%    &\eqd& \max\brb{\distancesub{1}{x}{y},\, \distancesub{2}{x}{y}} 
%    \\
%  \distancesub{1}{x}{y} \meet \distancesub{2}{x}{y} 
%    &\eqd& \inf_{\seqn{x=x_1,x_2,\ldots,x_n=y}} \sum_{i=1}^{n-1} \brb{\distancesub{1}{x_i}{x_{i+1}},\, \distancesub{2}{x_i}{x_{i+1}}}
%\end{array}}
%\end{theorem}

%\begin{figure}[h]
%  \centering%
%  \psset{unit=6mm}%
%  %============================================================================
% Daniel J. Greenhoe
% LaTeX file
%============================================================================
{\psset{unit=0.30mm}
\begin{pspicture}(-152,-80)(152,90)
  %-------------------------------------
  % options
  %-------------------------------------
  \psset{
    dotsize=5pt,
    linestyle=dashed,
    fillstyle=none,
    %labelsep=5pt,
    }
  %-------------------------------------
  % design support
  %-------------------------------------
  %\psgrid[unit=100\psunit](-1,-1)(1,1)
  %-------------------------------------
  % axes
  %-------------------------------------
  %\psline[linecolor=axis]{<->}(-120,0)(120,0)% x-axis
  %\psline[linecolor=axis]{<->}(0,-120)(0,120)% y-axis
  %-------------------------------------
  % nodes
  %-------------------------------------
  \pnode(  0,  7){o}% origin (and center of outer ball)
  \pnode( 41, 25){p1}% a point p (and center of an inner ball)
  \pnode(  0, 55){p2}% a point p (and center of an inner ball)
  \pnode( 83, 19){p3}% a point p (and center of an inner ball)
  \pnode( 67,-37){p4}% a point p (and center of an inner ball)
  \pnode(-23,-29){p5}% a point p (and center of an inner ball)
  \pnode(-59, 17){p6}% a point p (and center of an inner ball)
  \pnode(-13, 11){p7}% a point p (and center of an inner ball)
  \pnode( 20,-53){p8}% a point p (and center of an inner ball)
  %-------------------------------------
  % objects
  %-------------------------------------
  \psccurve[linecolor=blue](110,0)(80,80)(0,70)(-110,0)(-50,-30)(0,-70)% open set
  \psdot(p1)\pscircle[linecolor=red](p1){33}\uput{4pt}[0](p1){$p_1$}%
  \psdot(p2)\pscircle[linecolor=red](p2){15}\uput{4pt}[90](p2){$p_2$}%
  \psdot(p3)\pscircle[linecolor=red](p3){20}\uput{4pt}[0](p3){$p_3$}%
  \psdot(p4)\pscircle[linecolor=red](p4){11}\uput{4pt}[45](p4){$p_4$}%
  \psdot(p5)\pscircle[linecolor=red](p5){19}\uput{4pt}[0](p5){$p_5$}%
  \psdot(p6)\pscircle[linecolor=red](p6){30}\uput{4pt}[90](p6){$p_6$}%
  \psdot(p7)\pscircle[linecolor=red](p7){31}\uput{4pt}[0](p7){$p_7$}%
  \psdot(p8)\pscircle[linecolor=red](p8){15}\uput{4pt}[90](p8){$p_8$}%
\end{pspicture}
}%
%}%


%  \caption{\structe{open set} \xref{def:dspace_open} \label{fig:ms_open}}
%\end{figure}
%---------------------------------------
\begin{definition}
\label{def:dspace_open}
%\label{def:dspace_closed}
%---------------------------------------
Let $\dspaceX$ be a \structe{distance space}. % \xref{def:dspace}.
Let $\setX\setd\setA$ be the \ope{set difference} of $\setX$ and a set $\setA$.
\defboxp{
  A set $\setU$ is \propd{open} in $\dspaceX$ if $\setU\in\psetX$ and 
  %\\\indentx$x\in\setU \qquad\implies\qquad$there exists $r\in\Rp$ such that $\ball{x}{r}\subseteq \setU$.
  for every $x$ in $\setU$ there exists $r\in\Rp$ such that $\ball{x}{r}\subseteq \setU$.
  A set $\setU$ is an \structd{open set} in $\dspaceX$ if $\setU$ is \prope{open} in $\dspaceX$. 
  A set $\setD$ is \propd{closed} in $\dspaceX$ if $\brp{\setX\setd\setD}$ is \prope{open}.
  A set $\setD$ is a \structd{closed set} in $\dspaceX$ if $\setD$ is \prope{closed} in $\dspaceX$. 
  }
\end{definition}

%=======================================
\subsection{Properties}
%=======================================
%\pref{thm:dspace_open} (next) identifies four fundamental properties of open sets in
%distance spaces.
%These properties are the same as those defining a topology\ifsxref{topology}{def:topology}.
%---------------------------------------
\begin{theorem}
\footnote{
  in \structe{metric space}:
  \citerpp{dieudonne1969}{33}{34},
  \citerpg{rosenlicht}{39}{0486650383}
  %\citerp{giles1987}{215} \\
  %\citerpg{davis2005}{19}{0071243399}
  %\citerpg{ab}{35}{0120502577}
  }
\label{thm:dspace_open}
\index{space!metric}
%---------------------------------------
Let $\dspaceX$ be a \structe{distance space}. % \xref{def:dspace}.
Let $\xN$ be any (finite) positive integer.
Let $\Gamma$ be a \structe{set} possibly with an uncountable number of elements.
\thmbox{\begin{array}{F Mll>{\ds}l l}
    1. &                                                                &                         &          & \setX                                   & \text{is \prope{open}.}\\
    2. &                                                                &                         &          & \emptyset                               & \text{is \prope{open}.}\\
    3. & each element in $\setxn{\setU_n}$                              & \text{is \prope{open}} & \implies & \setopi_{n=1}^\xN \setU_n               & \text{is \prope{open}.}\\
    4. & each element in $\set{\setU_\gamma\in\psetX}{\gamma\in\Gamma}$ & \text{is \prope{open}} & \implies & \setopu_{\gamma\in\Gamma} \setU_\gamma  & \text{is \prope{open}.} 
  \end{array}}
\end{theorem}
\begin{proof}
\begin{enumerate}
  \item Proof that $\setX$ is \prope{open} in $\dspaceX$:
    \begin{enumerate}
      \item By definition of \structe{open set} \xref{def:dspace_open}, 
            $\setX$ is \prope{open} $\iff \forall x\in\setX \quad\exists r \st \ball{x}{r}\subseteq \setX$.
      \item By definition of \structe{open ball} \xref{def:ball}, it is always true that $\ball{x}{r}\subseteq\setX$ in $\dspaceX$.
      \item Therefore, $\setX$ is \prope{open} in $\dspaceX$.
    \end{enumerate}

  \item Proof that $\emptyset$ is \prope{open} in $\dspaceX$:
    \begin{enumerate}
      \item By definition of \structe{open set} \xref{def:dspace_open}, 
            $\emptyset$ is \prope{open} $\iff \forall x\in\setX \quad\exists r \st \ball{x}{r}\subseteq \emptyset$.
      \item By definition of \structe{empty set} $\emptyset$ \xref{def:emptyset}, this is always true because no $x$ is in $\emptyset$.
      \item Therefore, $\emptyset$ is \prope{open} in $\dspaceX$.
    \end{enumerate}

  \item Proof that $\setopu\setU_\gamma$ is \prope{open} in $\dspaceX$:
    \begin{enumerate}
      \item By definition of \structe{open set} \xref{def:dspace_open}, 
            $\setopu\setU_\gamma$ is \prope{open} $\iff \forall x\in\setopu\setU_\gamma \quad\exists r \st \ball{x}{r}\subseteq\setopu\setU_\gamma$.
      \item If $x\in\setopu\setU_\gamma$, then there is at least one $\setU\in\setopu\setU_\gamma$ that contains $x$.
      \item By the left hypothesis in (4), that set $\setU$ is open and so for that $x$,
            $\exists r \st \ball{x}{r}\subseteq\setU\subseteq\setopu\setU_\gamma$.
      \item Therefore, $\setopu\setU_\gamma$ is \prope{open} in $\dspaceX$.
    \end{enumerate}

  \item Proof that $\setU_1$ and $\setU_2$ are \prope{open} $\implies$ $\setU_1\seti\setU_2$ is \prope{open}: \label{ilem:ms_open_seti}
    \begin{enumerate}
      \item By definition of \structe{open set} \xref{def:dspace_open}, 
            $\setU_1\seti\setU_2$ is \prope{open} $\iff \forall x\in\setU_1\seti\setU_2\quad\exists r \st \ball{x}{r}\subseteq\setU_1\seti\setU_2$.
      \item By the left hypothesis above, $\setU_1$ and $\setU_2$ are \prope{open};
            and by the definition of \structe{open set}s \xref{def:dspace_open}, 
            there exists $r_1$ and $r_2$ such that
            $\ball{x}{r_1}\subseteq\setU_1$ and $\ball{x}{r_2}\subseteq\setU_2$.
      \item Let $r\eqd\min\setn{r_1,r_2}$. Then $\ball{x}{r}\subseteq\setU_1$ and $\ball{x}{r}\subseteq\setU_2$.
      \item By definition of \ope{set intersection} $\seti$ then, $\ball{x}{r}\subseteq\setU_1\seti\setU_2$.
      \item By definition of \ope{open set} \xref{def:dspace_open}, $\setU_1\seti\setU_2$ is \prope{open}.
    \end{enumerate}

  \item Proof that $\setopi_{n=1}^\xN\setU_n$ is \prope{open} (by induction):
    \begin{enumerate}
      \item Proof for $\xN=1$ case: $\setopi_{n=1}^\xN\setU_n=\setopi_{n=1}^1\setU_n=\setU_1$ is \prope{open} by hypothesis.
      \item Proof that $\xN$ case $\implies$ $\xN+1$ case: 
        \begin{align*} 
          \setopi_{n=1}^{\xN+1}\setU_n
            &= \brp{\setopi_{n=1}^{\xN}\setU_n}\seti\setU_{\xN+1}
            && \text{by property of $\setopi$}
          \\&\implies \text{\prope{open}} 
            && \text{by ``$\xN$ case" hypothesis and \prefp{ilem:ms_open_seti}}
        \end{align*}
    \end{enumerate}
\end{enumerate}
\end{proof}

%---------------------------------------
\begin{corollary}
\label{cor:dspace_open}
%---------------------------------------
Let $\dspaceX$ be a \structe{distance space}.
%Let an $\prope{open}$ set in $\psetX$ be defined as in \prefpp{def:dspace_open}.
\corboxt{
  The set 
  \quad$\topT\eqd\set{\setU\in\psetX}{\text{$\setU$ is \prope{open} in $\dspaceX$}}$\quad
  is a \structe{topology} on $\setX$, 
  \\and $\topspaceX$ is a \structe{topologogical space}.
  }
\end{corollary}
\begin{proof}
This follows directly from the definition of an \prope{open} set \xref{def:dspace_open},
\prefpp{thm:dspace_open}, and the definition of \structe{topology} \xref{def:topology}.
\end{proof}

%%---------------------------------------
%\begin{corollary}
%\label{cor:dspace_open}
%%---------------------------------------
%Let $\dspaceX$ be a \structe{distance space} \xref{def:dspace}.
%Let $\Gamma$ be a \structe{set} possibly with an uncountable number of elements.
%Let $\baseU\eqd\set{\setU_\gamma\in\psetX}{\gamma\in\Gamma}$.
%Let $\baseB$ be a family of sets such that $\baseB\subseteq\psetX$.
%\corbox{
%  \brb{\begin{array}{FMD}
%    1. & each set in $\baseU$ is \prope{open} in $\dspaceX$ & and
%  \\2. & $\baseU\subseteq\baseB$ &and 
%  \\3. & $\setX\in\baseB$ & and
%  \\4. & $\setU_1,\setU_2\in\baseU\implies\setU_1\seti\setU_2\in\baseB$ & and
%  \end{array}}
%  \quad\implies\quad
%  \brb{\begin{array}{M}
%    $\baseB$ is a \structe{base} for a\\\structe{topology} \xref{def:topology} on $\setX$
%  \end{array}}
%  }
%\end{corollary}
%\begin{proof}
%This follows directly from \prefpp{thm:dspace_open}, the definition of \structe{topology} \xref{def:topology},
%and the definition of \structe{base} \xref{def:base}.
%\end{proof}

Of course it is possible to define a very large number of topologies even on a finite set with just a handful of elements;\footnote{
  For a finite set $\setX$ with $n$ elements, there are 
  29 topologies on $\setX$ if $n=3$, 
  6942 topologies on $\setX$ if $n=5$, and
  and 8,977,053,873,043 (almost 9 trillion) topologies on $\setX$ if $n=10$.
  References:  \citeoeis{A000798},
  \citerp{brown1996}{31},
  \citerpg{comtet1974}{229}{9027704414},
  \citer{comtet1966},
  \citerp{chatterji1967}{7},
  \citer{evans1967},
  \citerp{krishnamurthy1966}{157}
  }
and it is possible to define an infinite number of topologies even on a \prope{linearly ordered} infinite set
like the \structe{real line} $\osetR$.\footnote{%
  For examples of topologies on the real line, see the following:
      \citerpgc{adams2008}{31}{0131848690}{"six topologies on the real line"},
      \citerppgc{salzmann2007}{64}{70}{0521865166}{Weird topologies on the real line},
      \citerpgc{murdeshwar1990}{53}{8122402461}{``often used topologies on the real line"},
      \citerppgc{joshi1983}{85}{91}{0852264445}{\textsection4.2 Examples of Topological Spaces}
  }
Be that as it may, \pref{def:dspacetop} (next definition) defines a single but convenient 
\structe{topological space} in terms of a \structe{distance space}.
Note that every \structe{metric space} conveniently and naturally induces a \structe{topological space}
because the \structe{open ball}s of the metric space form a \structe{base} for the \structe{topology}.
This is not the case for all {distance space}s.
But if the open balls of a \structe{distance space} are all \prope{open}, 
then those open balls induce a topology (next theorem).\footnote{
  \structe{metric space}: \prefp{def:metric}; 
  \structe{open ball}: \prefp{def:ball};
  \structe{base}: \prefp{def:base};  
  \structe{topology}: \prefp{def:topology};
  not all open balls are open in a distance space: \prefpp{ex:dspace_01} and \prefpp{ex:dspace_1n};
  }
%---------------------------------------
\begin{definition}
\label{def:dspacetop}
%---------------------------------------
Let $\dspaceX$ be a \structe{distance space}. % \xref{def:dspace}.
%Let $\psetX$ be the \structe{power set} \xref{def:psetx} of $\setX$.
\defboxt{
  The set 
  \quad$\topT\eqd\set{\setU\in\psetX}{\text{$\setU$ is \prope{open} in $\dspaceX$}}$\quad
  is the \structd{topology induced by $\dspaceX$ on $\setX$}. 
  \\The pair $\topspaceX$ is called the \structd{topological space induced by $\dspaceX$}. 
  }
\end{definition}

For any \structe{distance space} $\dspaceX$, no matter how strange, 
there is guaranteed to be at least one \structe{topological space induced by $\dspaceX$}---and that 
is the \structe{indiscrete topological space} \xref{ex:idts} because for any distance space $\dspaceX$, 
$\emptyset$ and $\setX$ are \prope{open set}s in $\dspaceX$ \xref{thm:dspace_open}.



%In a \structe{metric space}, the set of all \structe{open ball}s 
%is a \structe{base} for a \structe{topology}.
%This is largely due to the fact that in a \structe{metric space}, all \structe{open ball}s are \prope{open}.
%However, in a \structe{distance space}, \structe{open ball}s are not always \prope{open} \xref{ex:dspace_1n}.
%But if they all are, then the open balls induce a topology (next theorem).\footnote{
%  \structe{metric space}: \prefp{def:metric}; 
%  \structe{open ball}: \prefp{def:ball};
%  \structe{base}: \prefp{def:base};  
%  \structe{topology}: \prefp{def:topology};
%  not all \structe{open ball}s are \prope{open} in a \structe{distance space}: \prefp{ex:dspace_1n};
%  }
%---------------------------------------
\begin{theorem}
\label{thm:baseoball}
%---------------------------------------
Let $\baseB$ be the set of all \structe{open ball}s in a \structe{distance space}
$\dspaceX$. % \xref{def:dspace}.
\thmbox{
  \brb{\text{every \structe{open ball} in $\baseB$ is \prope{open}}}
  \quad\iff\quad
  \brb{\text{$\baseB$ is a \structe{base} for a \structe{topology}}}
  }
\end{theorem}
\begin{proof}
    \begin{align*}
      \mathrlap{\boxed{\text{every \structe{open ball} in $\baseB$ is \prope{open}}}}
      \\&\implies\text{for every $x$ in $\setB_y\in\baseB$ there exists $r\in\Rp$ such that $\ball{x}{r}\subseteq \setB_y$}
        && \text{by definition of \prope{open} \xref{def:dspace_open}}
      \\&\implies \brb{\begin{array}{M}
                    for every $x\in\setX$ and for every $\setB_y\in\baseB$ containing $x$,\\
                    there exists $\setB_x\in\baseB$ such that\qquad $x\in\setB_x\subseteq\setB_y$.
                  \end{array}}
        && \text{because $\forall\opair{x}{r}\in\setX\times\Rp$, $\ball{x}{r}\subseteq\setX$}
      \\&\implies \boxed{\text{$\baseB$ is a \structe{base} for $\topT$}}
        && \text{by \prefp{thm:basex}}
      %
      \\&\implies \brb{\begin{array}{M}
                    for every $x\in\setX$ and for every $\setU\subseteq\topT$ containing $x$,\\
                    there exists $\setB_x\in\baseB$ such that\qquad $x\in\setB_x\subseteq\setU$.
                  \end{array}}
        && \text{by \prefp{thm:basex}}
      \\&\implies \brb{\begin{array}{M}
                    for every $x\in\setX$ and for every $\setB_y\in\baseB\subseteq\topT$ containing $x$,\\
                    there exists $\setB_x\in\baseB$ such that\qquad $x\in\setB_x\subseteq\setB_y$.
                  \end{array}}
        && \text{by definition of \structe{base} \xref{def:base}}
      \\&\implies \brb{\begin{array}{M}
                    for every $x\in\setB_y\in\baseB\subseteq\topT$,\\
                    there exists $\setB_x\in\baseB$ such that\qquad $x\in\setB_x\subseteq\setB_y$.
                  \end{array}}
      \\&\implies \boxed{\text{every \structe{open ball} in $\baseB$ is \prope{open}}}
        && \text{by definition of \prope{open} \xref{def:dspace_open}}
    \end{align*}
\end{proof}

%%---------------------------------------
%\begin{remark}
%\label{rem:dspace_tspace}
%%---------------------------------------
%In a \structe{metric space} \xref{def:metric}, the set of all \structe{open ball}s \xref{def:ball} 
%is a \structe{base} \xref{def:base} for a \structe{topology} \xref{def:topology}.
%That is, the open balls induce a topology on the base set.
%In a \structe{distance space} \xref{def:dspace}, however, this is \emph{not} the case \xxref{thm:baseoball}{ex:dspace_1n}.
%But this by no means implies that all the beautiful structure of topological spaces is unavailable to us in a distance space.
%Rather, any set of \structe{open set}s \xref{def:dspace_open} in a \structe{distance space} $\dspaceX$ that
%satifies the four relations in \prefpp{thm:dspace_open} \emph{is}, by \prefpp{def:topology}, a \structe{topology} on $\setX$.
%For example, regardless of how strange the \fncte{distance function} $\distancen$ may be, 
%of course the \structe{discrete topology} $\psetX$
%and \structe{indiscrete topology} $\setn{\emptyset,\setX}$ \xref{ex:discretetop}
%are still legitimate topologies on $\setX$.
%And in any \structe{topological space} we may construct on a distance space, 
%all the structure of topological space, such as \structe{derived sets} \xxref{def:clsA}{def:bndA},
%and fundamental properties, such as the \thme{closed set theorem} \xref{thm:cst}, 
%is still available to us.
%\end{remark}

%=======================================
\section{Sequences in distance spaces}
%=======================================
%=======================================
\subsection{Definitions}
%=======================================
%---------------------------------------
\begin{definition}
\footnote{
  in \structe{metric space}:
  \citerpg{rosenlicht}{45}{0486650383},
  \citerpgc{giles1987}{37}{0521359287}{3.2 Definition},
  \citerpgc{khamsi2001}{13}{0471418250}{Definition 2.1}
  %\citerpgc{thomson2008}{30}{143484367X}{Definition 2.1}
  %\citor{cauchy1821}
  ``$\to$" symbol: \citorpc{leathem1905}{13}{section III.11}  % referenced by bromwich1955 page 3
  }
\label{def:dspace_converge}
\label{def:dspace_limit}
%---------------------------------------
%Let $\topspaceX$ be the \structe{topological space} induced by a \structe{distance space} $\dspaceX$ \xref{def:dspace}.
Let $\seqxZ{x_n\in\setX}$ be a  \fncte{sequence} in a \structe{distance space} $\dspaceX$. % \xref{def:dspace}.
\defboxp{
  The sequence $\seqn{x_n}$ \propd{converge}s to a \propd{limit} $x$ if
    for any $\varepsilon\in\Rp$, there exists $\xN\in\Z$
    such that for all $n>\xN$,
    $\distance{x_n}{x}<\varepsilon$.
  \\This condition can be expressed in any of the following forms:
  \\$\indentx\begin{array}{>{\scy}rM@{\qquad}>{\scy}rM}
      1. & The \opd{limit} of the sequence $\seqn{x_n}$ is $x$.             & 3. & $\ds\lim_{n\to\infty} \seqn{x_n} = x$.
    \\2. & The sequence $\seqn{x_n}$ is \propd{convergent} with limit $x$.    & 4. & $\ds\seqn{x_n} \to x$.                
  \end{array}$
  \\A \fncte{sequence} that converges is \propd{convergent}.
    %a \fncte{sequence} that does not converge is said to \propd{diverge}, or is \propd{divergent}.
    %An element $x\in\setA$ is a \vald{limit point} of $\setA$ if it is the limit of some $\setA$-valued sequence $\seqn{x_n\in\setA}$.
  }
\end{definition}

%---------------------------------------
\begin{definition}
\footnote{
  in \structe{metric space}:
  \citerpgc{apostol1975}{73}{0201002884}{4.7},
  \citerpg{rosenlicht}{51}{0486650383}
  }
\label{def:cauchy}
\index{Cauchy sequences}
\index{sequences!Cauchy}
%---------------------------------------
Let $\seqxZ{x_n\in\setX}$ be a \fncte{sequence} in a \structe{distance space} $\dspaceX$. % \xref{def:distance}.
\\\defboxp{
  The sequence $\seqn{x_n}$ is a \structd{Cauchy sequence} in $\dspaceX$ if
  \\\indentx for every $\varepsilon\in\Rp$, there exists $\xN\in\Z$ such that $\forall n,m>\xN,\; \distance{x_n}{x_m}<\varepsilon$\qquad{\scs(\prope{Cauchy condition})}.
  }
\end{definition}


%%---------------------------------------
%\begin{definition}
%\footnote{
%  \citerpgc{blumenthal1953}{9}{0828402426}{{\scshape Definition 6.3}}
%  %\citerpgc{berberian1961}{27}{0821819127}{Theorem~II.4.1}
%  %\citerp{pedersen2000}{4}
%  }
%\label{def:dspace_continuous}
%%---------------------------------------
%%Let $\omsR$ be an \prope{ordered distance space} \xref{def:ods}
%%on the \structe{set of real numbers} $\R$ with the \fncte{usual metric} $\distancea{x}{y}\eqd\abs{x-y}$.
%Let $\dspaceX$ be a \structe{distance space} \xref{def:dspace}.
%\\\defboxp{
%  The \fncte{distance function} $\distancen$ is \propd{continuous} at $\opair{x}{y}$ if
%  \\\indentx$
%  \mcom{\seqn{x_n} \to x \text{ and } \seqn{y_n} \to y}{convergence in $\dspaceX$}
%  \quad \implies \quad
%  \mcom{\seqn{\distance{x_n}{y_n}} \to \distance{x}{y}}{convergence in $\opair{\R}{\distancean}$}
%  $\\
%  The \fncte{distance function} $\distancen$ is \propd{continuous} if $\distancen$ is \prope{continuous} at each $\opair{x}{y}$ in $\setX^2$.
%  }
%\end{definition}


%A \fncte{sequence} is said to be \prope{complete} in a \structe{distance space} $\dspaceX$
%if every \prope{Cauchy sequence} in $\dspaceX$ \prope{converges} to a \struct{limit} in $\dspaceX$ (next definition).
%---------------------------------------
\begin{definition}
\footnote{
  in \structe{metric space}:
  \citerpg{rosenlicht}{52}{0486650383}
  }
\label{def:complete}
%---------------------------------------
Let $\seqxZ{x_n\in\setX}$ be a \fncte{sequence} in a \structe{distance space} $\dspaceX$.
\\\defboxt{
  The sequence $\seqxZ{x_n\in\setX}$ is \propd{complete} in $\dspaceX$ if
  \\\indentx$\text{$\seqn{x_n}$ is \prope{Cauchy} in $\dspaceX$}
    \quad \implies \quad
    \text{$\seqn{x_n}$ is \prope{convergent} in $\dspaceX$.}$
  }
\end{definition}


%=======================================
\subsection{Properties}
%=======================================
%---------------------------------------
\begin{proposition}
\footnote{
  in \structe{metric space}:
  \citerpgc{giles1987}{49}{0521359287}{Theorem 3.30}
  }
\label{prop:cauchy==>bounded}
%---------------------------------------
Let $\seqxZ{x_n\in\setX}$ be a \fncte{sequence} in a \structe{distance space} $\dspaceX$. % \xref{def:dspace}.
\propbox{
  \brb{\begin{array}{M}
    $\seqn{x_n}$ is \prope{Cauchy} 
    in $\dspaceX$
  \end{array}}
  \qquad\implies\qquad
  \brb{\begin{array}{M}
    $\seqn{x_n}$ is \prope{bounded} 
    in $\dspaceX$
  \end{array}}
}
\end{proposition}
\begin{proof}
\begin{align*}
  \text{$\seqn{x_n}$ is \prope{Cauchy}}
    &\implies \text{for every $\varepsilon\in\Rp$},\; \exists\xN\in\Z \st \forall n,m>\xN,\; \distance{x_n}{x_m}<\varepsilon
    \quad \text{(by \pref{def:cauchy})}
  \\&\implies \exists\xN\in\Z \st \forall n,m>\xN,\; \distance{x_n}{x_m}<1
    \qquad \text{(arbitrarily choose $\varepsilon\eqd1$)}
  \\&\implies \exists\xN\in\Z \st \forall n,m\in\Z,\; \distance{x_n}{x_{m+1}}<\max\setn{\setn{1}\setu\set{\distance{x_p}{x_q}}{p,q\ngtr N}}
  \\&\implies \text{$\seqn{x_n}$ is \prope{bounded}}
    \qquad \text{(by \prefp{def:bounded})}
\end{align*}
\end{proof}


%---------------------------------------
\begin{proposition}
\footnote{
  in \structe{metric space}:
  \citerpg{rosenlicht}{52}{0486650383}
  }
\label{prop:cauchy_subseq}
%---------------------------------------
Let $\seqxZ{x_n\in\setX}$ be a \fncte{sequence} in a \structe{distance space} $\dspaceX$. % \xref{def:dspace}.
Let $\ff\in\clFzz$ \xref{def:clFxy} be a \prope{strictly monotone} function such that $\ff(n)<\ff(n+1)$.
\propbox{
  \mcom{\text{$\seq{x_n}{n\in\Z}$ is \prope{Cauchy}}}{sequence is \prope{Cauchy}}
  \qquad\implies\qquad
  \mcom{\text{$\seq{x_{\ff(n)}}{n\in\Z}$ is \prope{Cauchy}}}{subsequence is also \prope{Cauchy}}
  }
\end{proposition}
\begin{proof}
\begin{align*}
  &\text{$\seq{x_n}{n\in\Z}$ is \prope{Cauchy}}
  \\&\implies \text{for any given } \varepsilon>0,\; \exists\xN \st \forall n,m>\xN,\; \distance{x_n}{x_m}<\varepsilon
    &&        \text{by \prefp{def:cauchy}}
  \\&\implies \text{for any given } \varepsilon>0,\; \exists\xN' \st \forall \ff(n),\ff(m)>\xN',\; \distance{x_{\ff(n)}}{x_{\ff(m)}}<\varepsilon
  \\&\implies \seq{x_{\ff(n)}}{n\in\Z} \text{ is \prope{Cauchy}}
    &&        \text{by \prefp{def:cauchy}}
\end{align*}
\end{proof}



%%---------------------------------------
%\begin{proposition}
%\footnote{
%  \citerppg{deza2014}{6}{7}{3662443422}
%  }
%%---------------------------------------
%\propboxt{
%  If $\fq$ is a \fncte{quasi-metric} \xref{def:qmetric}, then 
%  \\\indentx$\begin{array}{rcl MD}
%    \distancen_1(x,y) &\eqd& \min\setn{\fq(x,y),\fq(y,x)}                    & is a \fncte{metric} & (\fncte{bi-distance}).\\
%    \distancen_2(x,y) &\eqd& \brp{\fq^r(x,y),\fq^r(y,x)}^\frac{1}{r},\,r\ge1 & is a \fncte{metric}.
%  \end{array}$
%  }
%\end{proposition}

%---------------------------------------
\begin{theorem}
\footnote{
  in \structe{metric space}:
  \citerpgc{kubrusly2001}{128}{0817641742}{Theorem 3.40},
  \citerpgc{haaser1991}{75}{0486665097}{6$\cdot$10, 6$\cdot$11 Propositions},
  \citerpgc{bryant1985}{40}{0521318971}{Theorem 3.6, 3.7},
  \citerppg{sutherland1975}{123}{124}{0198531613} %{Proposition 9.23}
  %\citerpgc{rosenlicht}{52}{0486650383}{Proposition}\\
  }
\label{thm:comcls}
%\label{prop:comcls_com}
%---------------------------------------
%Let $\dspaceX$ be a \structe{metric space}.
%Let $\dspaceA$ be a \structe{subspace} of a \structe{distance space} $\dspaceX$ \xref{def:dspace}.
Let $\dspaceX$ be a \structe{distance space}. % \xref{def:dspace}.
%Let $\setA$ be a subset of $\setX$.
Let $\clsA$ be the \structe{closure} \xref{def:clsA} of a $\setA$ in a 
\structe{topological space induced by $\dspaceX$}. % \xref{def:dspacetop}.
\thmbox{
  \brb{\begin{array}{FMMD}
    1. & \vale{limit}s are \prope{unique} in $\dspaceX$ & \xref{def:dspace_converge} & and\\
    2. & $\dspaceA$ is \prope{complete} in $\dspaceX$ & \xref{def:complete}        &
  \end{array}}
  \quad\implies\quad
  \mcom{\text{$\setA$ is \prope{closed} in $\dspaceX$}}{$\setA=\clsA$}
  %\\
  %\brb{\begin{array}{FMD}
  %  1. & $\dspaceX$ is \prope{complete} in $\dspaceX$ &  and \\
  %  2. &  $\setA$ is \prope{closed} in $\dspaceX$       &($\setA=\clsA$)
  %\end{array}}
  %&\implies&
  %\brb{\text{$\dspaceA$ is \prope{complete} in $\dspaceX$}}
  }
\end{theorem}
\begin{proof}
\begin{enumerate}
  \item Proof that $\setA\subseteq\clsA$: by \prefp{lem:intAAclsA}
  \item Proof that $\clsA\subseteq\setA$ (proof that $x\in\clsA$ $\implies$ $x\in\setA$):
    \begin{enumerate}
      \item Let $x$ be a point in $\clsA$ ($x\in\clsA$).
      \item Define a \fncte{sequence} of open balls $\seqn{\ball{x}{\frac{1}{1}},\,\ball{x}{\frac{1}{2}},\,\ball{x}{\frac{1}{3}},\,\ldots}$.
      \item Define a \fncte{sequence} of points $\seqn{x_1,\,x_2,\,x_3,\,\ldots}$ such that $x_n\in\ball{x_n}{\frac{1}{n}}\seti\setA$.
      \item Then $\seqn{x_n}$ is \prope{convergent} in $\setX$ with limit $x$ by \prefp{def:dspace_converge}
      \item and  $\seqn{x_n}$ is \prope{Cauchy} in $\setA$ by \prefp{def:cauchy}.
      %\item By the left hypothesis ($\dspaceA$ is \prope{complete}), $\seqn{x_n}$ is therefore also \prope{convergent} in $\setA$.\\
      \item By the hypothesis 2, $\seqn{x_n}$ is therefore also \prope{convergent} in $\setA$.\\
            Let this limit be $y$. Note that $y\in\setA$.\label{item:com_cls_yY}
      %\item By \prefp{prop:xn_to_xy}, limits are \prope{unique}, so $y=x$. \label{item:com_cls_yx}
      \item By hypothesis 1, limits are \prope{unique}, so $y=x$. \label{item:com_cls_yx}
      \item Because $y\in\setA$ (\pref{item:com_cls_yY}) and $y=x$ (\pref{item:com_cls_yx}), so $x\in\setA$.
      \item Therefore, $x\in\clsA\implies x\in\setA$ and $\clsA\subseteq\setA$.
    \end{enumerate}
\end{enumerate}
\end{proof}

%---------------------------------------
\begin{proposition}
\footnote{
  in \structe{metric space}:
  \citerpg{rosenlicht}{46}{0486650383}
  }
%---------------------------------------
Let $\seq{x_n}{n\in\Z}$ be a sequence in a \structe{distance space} $\dspaceX$.
Let $\ff:\Z\to\Z$ be a strictly increasing function such that $\ff(n)<\ff(n+1)$.
\propbox{
  \mcomr{\seq{x_n}{n\in\Z} \to x}{sequence converges to limit $x$}
  \qquad\implies\qquad
  \mcoml{\seq{x_{\ff(n)}}{n\in\Z} \to x}{subsequence converges to the same limit $x$}
  }
\end{proposition}
\begin{proof}
\begin{align*}
  \seq{x_n}{n\in\Z} \to x
    &\implies \forall \varepsilon>0,\; \exists\xN \st \forall n>\xN,\; \distance{x_n}{x}<\varepsilon
    &&        \text{by \prefp{thm:ms_converge}}
  \\&\implies \forall \varepsilon>0,\; \exists \ff(N) \st \forall \ff(n)>\ff(N),\; \distance{x_{\ff(n)}}{x}<\varepsilon
  \\&\implies \seq{x_{\ff(n)}}{n\in\Z} \to x
    &&        \text{by \prefp{thm:ms_converge}}
\end{align*}
\end{proof}



%%---------------------------------------
%\begin{theorem}
%\footnote{
%  \citerpgc{kubrusly2001}{128}{0817641742}{Corollary 3.41}
%  }
%\label{thm:comcomcls}
%%---------------------------------------
%%Let $\dspaceA$ be a \structe{subspace} of a metric space $\dspaceX$ ($\setA\subseteq\setX$).
%Let $\dspaceX$ be a \structe{metric space} \xref{def:metric}.
%Let $\setA$ be a subset of $\setX$.
%Let $\clsA$ be the \structe{closure} \xref{def:clsA} of $\setA$ in $\dspaceX$.
%\thmboxp{
%If $\dspaceX$ is \prope{complete} \xref{def:complete}, then
%  \\\indentx$
%  \brb{\text{$\dspaceA$ is \prope{complete}}}
%  \qquad\iff\qquad
%  \mcom{\text{$\setA$ is \prope{closed} in $\dspaceX$}}{$\setA=\clsA$}
%  $
%  }
%\end{theorem}
%\begin{proof}
%Note that in this corollary, the metric space $\dspaceX$ is assumed to be \prope{complete}.
%\begin{enumerate}
%  \item Proof that \prope{complete} $\implies$ \prope{closed}: by \pref{thm:comcls} (1).
%  \item Proof that \prope{complete} $\impliedby$ \prope{closed}: by \prope{complete} hypothesis and \pref{thm:comcls} (2).
%    \begin{enumerate}
%      \item By left hypothesis 2, $\setA$ is closed in $\dspaceX$.
%      \item By \prefpp{thm:insubset_closed} and because $\setA$ is closed in $\dspaceX$,
%            sequences converge in $\setA$.
%      \item Therefore by \prefpp{def:complete}, $\dspaceA$ is complete.
%    \end{enumerate}
%\end{enumerate}
%\end{proof}

%%---------------------------------------
%\begin{example}
%%---------------------------------------
%Let $\Q$ be the set of \sete{rational numbers}.
%\exbox{%\begin{array}{ll}
%  \text{The metric space $(\Q,\distance{x}{y}=\abs{x-y})$ is {\em not} \prope{complete}.}
%  %2. & \text{The metric space $(\R,\distance{x}{y}=\abs{x-y})$ {\em is} complete.    }\\
%}%\end{array}}
%\end{example}
%\begin{proofns}
%Let $\seq{x_n}{n\in\Znn}$ be the sequence of values approximating $\pi$ truncated to $n$
%decimal points:
%  \[ \seq{x_n}{n\in\Znn} \eqd \seqn{3,\, 3.1,\, 3.14,\, 3.141,\, 3.1415,\, 3.14159,\, 3.141592,\,\ldots} \]
%This is a Cauchy sequence.
%However, this sequence (and all sequences converging to an irrational number)
%does not converge to a rational number ($\Q$) and thus is not in
%the metric space $(\Q,\distancen)$ and thus $(\Q,\distancen)$ is {\em not complete}.
%%But of course this sequence (and all other like sequences) do converge to real numbers and thus
%%$(\R,\distancen)$ is a complete metric space.
%\end{proofns}

%%---------------------------------------
%\begin{example}[\exmd{Cauchy's convergence criterion}/\exmd{Cauchy's criterion}]
%\footnote{
%  \citerpgc{sohrab2003}{54}{0817642110}{Theorem 2.2.5}
%  }
%%---------------------------------------
%Let $\seqxZ{r_n\in\R}$ be a \textbf{real} sequence.
%\exboxp{
%  The metric space $\opair{\seq{r_n}}{\abs{r_n-r_m}}$ is \prope{complete}.
%  }
%\end{example}

%---------------------------------------
\begin{theorem}[\thmd{Cantor intersection theorem}]
\footnote{
  in \structe{metric space}:
  \citerp{davis2005}{28},
  \citerp{hausdorff1937e}{150}
  }
\label{thm:cantor_int}
\label{thm:cit}
\index{Cantor intersection theorem}
\index{theorems!Cantor intersection}
%---------------------------------------
Let $\dspaceX$ be a \prope{distance space} \xref{def:dspace},
$\seqnZ{\setA_n}$ a \fncte{sequence} with each $\setA_n\in\psetX$, and $\seto{\setA}$ the number of elements in $\setA$.
\thmbox{
  \brb{\begin{array}{FMCMD}
    1. & $\dspaceX$ is \prope{complete}                &                 & \xref{def:complete}       & and \\
    2. & $\setA_n$ is \prope{closed}                     & \forall n\in\Zp & \xref{def:closedset}    & and \\
    3. & $\diam \setA_{n} \ge \diam \setA_{n+1}$         & \forall n\in\Zp & \xref{def:diam}         & and \\
    4. & $\diam \seqxZ{\setA_n} \to 0$                   &                 & \xref{def:dspace_limit}
  \end{array}}
  \qquad\implies\qquad
  \brb{\seto{\ds\setopi_{n\in\Zp} \setA_n } = 1}
  }
\end{theorem}
\begin{proof}
\begin{enumerate}
\item Proof that $\seto{\setopi_{n\in\Z} \setA_n}<2$:
  \begin{enumerate}
    \item Let $\setA\eqd\seti \setA_n$.
    \item $x\ne y$ and $\{x,y\}\in \setA \implies \distance{x}{y}>0$ and $\{x,y\}\subseteq \setA_n \forall n$
    \item $\exists n \st \diam \setA_n < \distance{x}{y}$ by left hypothesis 4
    \item $\implies \exists n \st \sup\set{\distance{x}{y}}{x,y\in \setA_n}<\distance{x}{y}$
    \item This is a contradiction, so $\{x,y\}\notin \setA$ and $\seto{\setopi \setA_n}<2$.
  \end{enumerate}
                                                 
\item Proof that $\seto{\seti \setA_n}\ge1$:
  \begin{enumerate}
    \item Let $x_n\in \setA_n$ and $x_m\in \setA_m$
    \item $\forall \varepsilon,\; \exists\xN\in\Zp \st \setA_N < \varepsilon$
    \item $\forall m,n>\xN,\; x_n\in \setA_n\subseteq \setA_N$ and $x_m\in \setA_m\subseteq \setA_N$
    \item $\distance{x_n}{x_m}\le\diam \setA_N < \varepsilon \implies \{x_n\}$ is a Cauchy sequence
    \item Because $\{x_n\}$ is complete, $x_n\to x$.
    \item $\implies x\in \cls{\brp{\setA_n}} = \setA_n$
    \item $\implies \seto{\setA_n}\ge1$
  \end{enumerate}
\end{enumerate}
\end{proof}

%---------------------------------------
\begin{definition}
\footnote{
  \citerpgc{blumenthal1953}{9}{0828402426}{{\scshape Definition 6.3}}
  }
\label{def:dspace_cont}
%---------------------------------------
Let $\dspaceX$ be a \structe{distance space}.
Let $\setC$ be the set of all \prope{convergent} sequences in $\dspaceX$.
\thmboxp{
  The \fncte{distance function} $\distancen$ is \propd{continuous} in $\dspaceX$ if
  \\\indentx$\ds
    \seqn{x_n},\seqn{y_n}\in\setC
    \quad\implies\quad
    \lim_{n\to\infty}\seqn{\distance{x_n}{y_n}}=\distance{\lim_{n\to\infty}\seqn{x_n}}{\lim_{n\to\infty}\seqn{y_n}}
    $.\\
  A \fncte{distance function} is \propd{discontinuous} if it is not \prope{continuous}.
  }
\end{definition}

%---------------------------------------
\begin{remark}
%---------------------------------------
Rather than defining \prope{continuity} of a \structe{distance function} in terms of 
the \thme{sequential characterization of continuity} \xref{def:dspace_cont},
we could define continuity using an \thme{inverse image characterization of continuity}" \xref{def:dspacetop}.
Assuming an equivalent \structe{topological space} is used for both characterizations, the 
two characterizations are equivalent \xref{thm:limcont}.
In fact, one could construct an equivalence such as the following:
%%---------------------------------------
%\begin{corollary}
%\label{cor:limcont}
%%---------------------------------------
%Let $\dspaceX$ be a \structe{distance space}.
%Let $\topspaceX$ be the \structe{topological space induced by $\dspaceX$} \xref{def:dspacetop}.
%Let $\topspace{\R}{\topS}$ be the \structe{usual topological space over $\R$}.
%Let $\setC$ be the set of all \prope{convergent} sequences in $\dspaceX$.
\rembox{
  \brb{\begin{tabstr}{0.75}\begin{array}{M}
    \\
    $\distancen$ is \prope{continuous}  in $\clF{\setX^2}{\R}$\\
    \xref{def:continuous}\\
    {\scs(\thme{inverse image characterization of continuity})}
  \end{array}\end{tabstr}}
  \quad\iff\quad
  \brb{\begin{tabstr}{0.75}\begin{array}{M}
    $\ds\seqn{x_n},\seqn{y_n}\in\setC\quad\implies$\\
    $\ds\lim_{n\to\infty}\seqn{\distance{x_n}{y_n}}=\distance{\lim_{n\to\infty}\seqn{x_n}}{\lim_{n\to\infty}\seqn{y_n}}$\\
    \xref{def:converge}\\
    {\scs(\thme{sequential characterization of continuity})}
  \end{array}\end{tabstr}}
  }\\
%\end{corollary}
%\begin{proof}
Note that just as $\seqn{x_n}$ is a sequence in $\setX$, so the ordered pair $\opair{\seqn{x_n}}{\seqn{y_n}}$
is a sequence in $\setX^2$.
The remainder %of this proof 
follows from \prefpp{thm:limcont}.
%\end{proof}
However, use of the \thme{inverse image characterization} is somewhat troublesome
because we would need a topology on $\setX^2$, and we don't immediately have one defined and ready to use.
In fact, we don't even immediately have a distance space on $\setX^2$ defined or even open balls in such a distance space.
The result is, for the scope of this chapter, it is arguably not worthwhile constructing the extra structure, 
but rather this chapter instead
uses the \thme{sequential characterization} as a definition \xref{def:dspace_cont}.
\end{remark}



%=======================================
\section{Examples}
%=======================================
Similar distance functions and several of the observations for the examples 
in this section can be found in \citerppg{blumenthal1953}{8}{13}{0828402426}.

In a \structe{metric space}, all \structe{open ball}s are \prope{open},
the \structe{open ball}s form a \structe{base} for a \structe{topology}, 
the limits of \prope{convergent} sequences are \prope{unique}, 
and the \fncte{metric function} is \prope{continuous}.
In the \structe{distance space} of the next example, none of these properties hold.
%---------------------------------------
\begin{example}
\footnote{
  A similar distance function $\distancen$ and \prefp{item:dspace_01} 
  can in essence be found in \citerpg{blumenthal1953}{8}{0828402426}.
  %Definitions for \pref{ex:dspace_01}:
  %$\opair{x}{y}$: \prefpp{def:opair};
  %$\intoo{a}{b}$ and $\intoc{a}{b}$: \prefpp{def:intxx};
  %$\abs{x}$: \prefpp{def:abs};
  %$\clF{\R\times\R}{\R}$: \prefpp{def:clFxy};
  %\fncte{distance}: \prefpp{def:distance};
  %\structe{open ball}: \prefpp{def:ball};
  %\prope{open}: \prefpp{def:dspace_open};
  %\structe{base}: \prefpp{def:base};
  %\structe{topology}: \prefpp{def:topology};
  %\structe{open set}: \prefpp{def:dspace_open};
  %\structe{topological space induced by $\dspaceRd$}: \prefpp{def:dspacetop};
  %\prope{discontinuous}: \prefpp{def:dspace_cont};
  }
\label{ex:dspace_01}
%---------------------------------------
%Let $\setX\eqd\R$.
Let $\opair{x}{y}$ be an \structe{ordered pair} in $\R^2$.
Let $\intoo{a}{b}$ be an \structe{open interval} and $\intoc{a}{b}$ a \structe{half-open interval} in $\R$.
Let $\abs{x}$ be the \fncte{absolute value} of $x\in\R$.
The function $\distance{x}{y}\in\clF{\R\times\R}{\R}$ such that
\\\indentx$\distance{x}{y} \eqd \brb{\begin{array}{lMD}
      y         & $\forall \opair{x}{y}\in\setn{4}\times\intoc{0}{2}$ & (\structe{vertical half-open interval})\\
      x         & $\forall \opair{x}{y}\in\intoc{0}{2}\times\setn{4}$ & (\structe{horizontal half-open interval})\\
      \abs{x-y} & otherwise                                           & (\prope{Euclidean})
    \end{array}}$\quad is a \fncte{distance} on $\R$.
\\
Note some characteristics of the \structe{distance space} $\dspaceRd$:
\begin{enumerate}
  \item $\dspaceRd$ is not a \structe{metric space} because $\distancen$ does not satisfy the \prope{triangle inequality}:
    \\\indentx$\distance{0}{4}
        \eqd \abs{0-4} = 4
        \nleq 2
        = \abs{0-1}     + 1
        \eqd \distance{0}{1} + \distance{1}{4}
       $

  \item Not every \structe{open ball} in $\dspaceRd$ is \prope{open}.\label{item:dspace_01_oballo}\\
        For example, the \structe{open ball} $\ball{3}{2}$ is \prope{not open} 
        because $4\in\ball{3}{2}$ \emph{but} for all $0<\varepsilon<1$
        \\\indentx$\ball{4}{\varepsilon}=\intoo{4-\varepsilon}{4+\varepsilon}\setu\intoo{0}{\varepsilon}
           \nsubseteq\intoo{1}{5}
           = \ball{3}{2}$

  \item The \structe{open balls} of $\dspaceRd$ do not form a \structe{base} for a \structe{topology} on $\R$.\\
        This follows directly from \pref{item:dspace_01_oballo} and \prefpp{thm:baseoball}.

  \item In the \structe{distance space} $\dspaceRd$, limits are \prope{not unique};\\
        For example, the sequence $\seqn{\sfrac{1}{n}}_1^\infty$ converges both to the limit $0$ and the limit $4$ in $\dspaceRd$:
        \label{item:dspace_01}
    \\\indentx$\begin{array}{*{5}{>{\ds}l}}
      \lim_{n\to\infty}\distance{x_n}{0} 
        &\eqd \lim_{n\to\infty}\distance{\sfrac{1}{n}}{0}
        &\eqd \lim_{n\to\infty}\abs{\sfrac{1}{n}-0}
        &= 0
       %&\implies \lim_{n\to\infty}\seqn{\sfrac{1}{n}}=0
        &\quad\implies\quad \seqn{\sfrac{1}{n}}\to0
        \\
      \lim_{n\to\infty}\distance{x_n}{4} 
        &\eqd \lim_{n\to\infty}\distance{\sfrac{1}{n}}{4}
        &\eqd \lim_{n\to\infty}\seqn{\sfrac{1}{n}}
        &= 0
       %&\implies \lim_{n\to\infty}\seqn{\sfrac{1}{n}}=4
        &\quad\implies\quad \seqn{\sfrac{1}{n}}\to4
    \end{array}$

  \item The \structe{topological space $\topspaceX$ induced by $\dspaceRd$} also 
        yields limits of $0$ and $4$ for the sequence $\seqn{\sfrac{1}{n}}_1^\infty$, just 
        as it does in \pref{item:dspace_01}.
        This is largely due to the fact that, for small $\varepsilon$, 
        the open balls $\ball{0}{\varepsilon}$ and $\ball{4}{\varepsilon}$ are \prope{open}.
    \begin{align*}
        \text{$\ball{0}{\varepsilon}$ is \prope{open}}
          &\implies \text{for each $\setU\in\topT$ that contains $0$, $\exists\xN\in\Zp \st \sfrac{1}{n}\in\setU\quad\forall n>\xN$}
        \\&\iff \seqn{\sfrac{1}{n}}\to0
          \qquad\text{by definition of \prope{convergence} \xref{def:converge}}
        \\
        \text{$\ball{4}{\varepsilon}$ is \prope{open}}
          &\implies \text{for each $\setU\in\topT$ that contains $4$, $\exists\xN\in\Zp \st \sfrac{1}{n}\in\setU\quad\forall n>\xN$}
        \\&\iff \seqn{\sfrac{1}{n}}\to4
          \qquad\text{by definition of \prope{convergence} \xref{def:converge}}
    \end{align*}  

  %\item The distance function $\distancen\in\clF{\R\times\R}{\R}$ is \prope{discontinuous} \label{item:dspace_01_discon}
  %      with respect to the \structe{Euclidean topologies of $\R$ and $\R^2$}.
  %      Consider the following: 
  %      Let $\opair{x}{y}\in\R^2$ be an \structe{ordered pair}.
  %      Let $\ball{\opair{x}{y}}{\varepsilon}$ be an \structe{open ball} in the \structe{Euclidean metric space}
  %      $\metspace{\R^2}{\metrican}$ where
  %      \\$\metrica{\opair{x_1}{x_2}}{\opair{y_1}{y_2}}\eqd\sqrt{(x_1-x_2)^2+(y_1-y_2)^2}$\qquad%
  %        ${\scy\forall \opair{x_1}{x_2},\opair{y_1}{y_2}\in\R^2}$.\footnote{
  %      Note that $\C\eqd\R^2$, where $\C$ is the set of \structe{complex numbers}, and thus $\dspaceRd$ 
  %      can be viewed as a \structe{distance space} over $\C$ and the  
  %      \fncte{distance function} $\metrican$ somewhat like a \fncte{projection} 
  %      from $\C^2=\R^4$ onto a ``real axis" $\R$ in $\C$.
  %      It is like a \fncte{projection} in the sense that it maps from $\C^2=\R^4$ to $\R$ and also because 
  %      it has a property similar to the \prope{idempotent} property in the sense that 
  %      \\\indentx$\metrica{\opair{\metrica{\opair{x_1}{x_2}}{\opair{y_1}{y_2}}}{0}}{\opair{0}{0}}=
  %      \metrica{\opair{x_1}{x_2}}{\opair{y_1}{y_2}}$ .
  %      } 
  %      \qquad Then
  %      \begin{align*}
  %        \distancen^{-1}\brs{\intoo{0}{2}}
  %          &\supseteq \setn{4}\times\intoo{0}{2}
  %          && \text{by definition of \fncte{distance} $\distancen$}
  %        \\&\nsupseteq\ball{\opair{4}{1}}{\varepsilon}\quad\forall \varepsilon>0
  %          && \text{by definition of \structe{open ball} $\balln$}
  %        \\&\implies \text{$\distancen^{-1}\brs{\intoo{0}{2}}$ is \prope{not open}}
  %          &&\text{by \prefpp{def:dspace_open}}
  %        \\&\implies \text{$\distancen$ is \prope{discontinuous}}
  %          &&\text{by \prefpp{def:continuous}}
  %      \end{align*}

  \item The distance function $\distancen$ is \prope{discontinuous} \xref{def:dspace_cont}:
  %\item As an alternative to the \pref{item:dspace_01_discon} method of using \structe{open set}s to prove that $\distancen$ is \prope{discontinuous}, we could use convergent sequences to prove the same:
    \begin{align*}
      \lim_{n\to\infty}\seqn{\distance{1-\sfrac{1}{n}}{4-\sfrac{1}{n}}}
        &=\lim_{n\to\infty}\seqn{\abs{\brp{1-\sfrac{1}{n}}-\brp{4-\sfrac{1}{n}}}}
         = \abs{1-4} = 3 \neq 4 = \distance{0}{4}
      \\&= \distance{\lim_{n\to\infty}\seqn{1-\sfrac{1}{n}}}{\lim_{n\to\infty}\seqn{4-\sfrac{1}{n}}}
    \end{align*}
   %which by \prefpp{cor:limcont} implies that $\distancen$ is \prope{discontinuous}.
   %   \item Therefore, by \prefpp{cor:limcont}, $\distancen$ is \prope{discontinuous}.
   % \end{enumerate}
\end{enumerate}
\end{example}

In a \structe{metric space}, all \prope{convergent} sequences are also \prope{Cauchy}.
However, this is not the case for all \structe{distance space}s, as demonstrated next:
%---------------------------------------
\begin{example}
\footnote{
  The distance function $\distancen$ and \prefp{item:dspace_1n_cauchy} 
  can in essence be found in \citerpg{blumenthal1953}{9}{0828402426}
  }
\label{ex:dspace_1n}
%---------------------------------------
%Let $\setX\eqd\set{\frac{1}{n}}{n=1,2,3,\ldots}\setu\setn{0}$ be a \structe{set}.
%The function $\distance{x}{y}\in\clF{\setX^2}{\R}$ such that
The function $\distance{x}{y}\in\clF{\R\times\R}{\R}$ such that
\\\indentx$\distance{x}{y} \eqd \brb{\begin{array}{lMD}
      \abs{x-y} & for $x=0$ or $y=0$ or $x=y$ & (\prope{Euclidean})\\
      1         & otherwise                   & (\prope{discrete})
    \end{array}}$\quad is a \fncte{distance} on $\R$. 
\\
Note some characteristics of the \structe{distance space} $\dspaceRd$:
\begin{enumerate}
  \item $\dspaceX$ is not a \structe{metric space} because the \prope{triangle inequality} does not hold:
      \\\indentx
        $\distance{\frac{1}{4}}{\frac{1}{2}}
        = 1
        \nleq \frac{3}{4}
        = \abs{\frac{1}{4}-0}     + \abs{0-\frac{1}{2}}
        = \distance{\frac{1}{4}}{0} + \distance{0}{\frac{1}{2}}
        $

  \item The \structe{open ball} $\ball{\frac{1}{4}}{\frac{1}{2}}$ is \prope{not open}
        because for any $\varepsilon\in\Rp$, no matter how small,  \label{item:dspace_1n_oballo}
        \\\indentx
         $\ball{0}{\varepsilon} = \intoo{-\varepsilon}{+\varepsilon}
            \nsubseteq \setn{0,\,\frac{1}{4}}
            =\set{x\in\setX}{\distance{\frac{1}{4}}{x}<\frac{1}{2}}
            \eqd\ball{\frac{1}{4}}{\frac{1}{2}}$
  
  \item Even though not all the \structe{open ball}s are \prope{open}, 
        it is still possible to have an \structe{open set} in $\dspaceX$. 
        For example, the set $\setU\eqd\setn{1,\,2}$ is \prope{open}:
        \\\indentx$\begin{array}{rclclclcl}
            \ball{1}{1}
            &\eqd& \set{x\in\setX}{\distance{1}{x}<1}
            &=& \setn{1}
            &\subseteq& \setn{1,\,2}
            &\eqd& \setU
            \\
            \ball{2}{1}
            &\eqd& \set{x\in\setX}{\distance{2}{x}<1}
            &=& \setn{2}
            &\subseteq& \setn{1,\,2}
            &\eqd& \setU
          \end{array}$

  \item By \pref{item:dspace_1n_oballo} and \prefpp{thm:baseoball}, 
        the \structe{open ball}s of $\dspaceRd$ do not form a \structe{base} for a \structe{topology} on $\R$.

  \item Even though the open balls in $\dspaceRd$ do not induce a topology on $\setX$, it is still possible to 
        find a set of \structe{open set}s in $\dspaceX$ that \emph{is} a topology. 
        For example, the set
          $\setn{\emptyset,\,\setn{1,2},\,\R}$
        is a topology on $\R$.

  \item In $\dspaceRd$, limits of \prope{convergent} sequences are \prope{unique}:
    \\\indentx$\ds
      \seqn{x_n}\to x \quad\implies\quad
      \lim_{n\to\infty}\distance{x_n}{x}=\brb{\begin{array}{rclMD}
        \lim\abs{x_n-0} &=& 0 & for $x=0$ & OR\\
            \abs{x-x}   &=& 0 & for constant $\seqn{x_n}$ for $n>\xN$ & OR\\
                    1   &\neq& 0 & otherwise
      \end{array}}$\\
     which says that there are only two ways for a sequence to converge: either $x=0$ or the sequence eventually becomes constant
     (or both). Any other sequence will \prope{diverge}. Therefore we can say the following:
    \begin{enumerate}
      \item If $x=0$ and the sequence is not constant, then the limit is \prope{unique} and $0$.
      \item If $x=0$ and the sequence is constant, then the limit is \prope{unique} and $0$.
      \item If $x\neq0$ and the sequence is constant, then the limit is \prope{unique} and $x$.
      \item If $x\neq0$ and the sequence is not constant, then the sequence diverges and there is no limit.
    \end{enumerate}

  \item In $\dspaceRd$, a \prope{convergent} sequence is not necessarily \prope{Cauchy}.\label{item:dspace_1n_cauchy}
    For example,
    \begin{enumerate}
      \item the sequence $\seqnZp{\sfrac{1}{n}}$ is \prope{convergent} with limit $0$:
       $\ds\lim_{n\to\infty}\distance{\sfrac{1}{n}}{0}
        = \lim_{n\to\infty}\sfrac{1}{n}
        = 0$
      \item However, even though $\seqn{\sfrac{1}{n}}$ is \prope{convergent}, it is \prope{not Cauchy}:
       $\ds\lim_{n,m\to\infty}\distance{\sfrac{1}{n}}{\sfrac{1}{m}}
       = 1
       \neq 0
      $
    \end{enumerate}

  \item The \fncte{distance function} $\distancen$ is \prope{discontinuous} in $\dspaceX$:
    \begin{align*}
        \lim_{n\to\infty}\seqn{\distance{\sfrac{1}{n}}{2-\sfrac{1}{n}}} 
        &= 1 
      \\&\neq 2 = \distance{0}{2} = \distance{\lim_{n\to\infty}\seqn{\sfrac{1}{n}}}{\lim_{n\to\infty}\seqn{2-\sfrac{1}{n}}}
    \end{align*}
    %which by \prefpp{cor:limcont} implies that $\distancen$ is \prope{discontinuous}.
\end{enumerate}
\end{example}

%In any \structe{distance space} in which the \prope{triangle inequality} holds 
%(a \structe{metric space}),
%the \fncte{distance} function is always \prope{continuous}. % \xref{prop:metric_continuous}.
%This is even true in the case of the \fncte{discrete metric} % \xref{ex:dmetric}
%which is induced from any other metric using the \prope{non-continuous} 
%\fncte{discrete metric preserving function}. % \xref{ex:mpf_discrete}.
%However, distance functions are not all \prope{continuous}, as demonstrated next.
%---------------------------------------
\begin{example}
\footnote{
  The distance function $\distancen$ and \prefp{item:dspace_21_cont}
  can in essence be found in \citerpg{blumenthal1953}{9}{0828402426}
  }
\label{ex:dspace_21}
%---------------------------------------
%Let $\setX\eqd\set{\frac{1}{n}}{n=1,2,3,\ldots}\setu\setn{0}$ be a \structe{set}.
The function $\distance{x}{y}\in\clF{\R\times\R}{\R}$ such that
\\\indentx$\distance{x}{y} \eqd \brb{\begin{array}{rMD}
      2\abs{x-y} & $\forall \opair{x}{y}\in\setn{\opair{0}{1},\,\opair{1}{0}}$ & (\prope{dilated Euclidean})\\
       \abs{x-y} & otherwise                                                   & (\prope{Euclidean})
    \end{array}}$\quad is a \fncte{distance} on $\R$.
\\
Note some characteristics of the \structe{distance space} $\dspaceRd$:
\begin{enumerate}
  \item $\dspaceRd$ is \emph{not} a \structe{metric space} because $\distancen$ does \emph{not} 
        satisfy the \prope{triangle inequality}:
      \\\indentx$\ds\distance{0}{1}
        \eqd 2\abs{0-1}
        = 2
        \nleq 1
        = \abs{0-\sfrac{1}{2}}  + \abs{\sfrac{1}{2}-1}
        \eqd \distance{0}{\sfrac{1}{2}} + \distance{\sfrac{1}{2}}{1}
        $

  \item The function $\distancen$ is \prope{discontinuous}: \label{item:dspace_21_cont}
        %In particular, it is \prope{not continous} at the point $\opair{\lim\seqn{1-\frac{1}{n}}}{\lim\seqn{\frac{1}{n}}}$.
    \begin{align*}
      &\lim_{n\to\infty}\seqn{\distance{1-\sfrac{1}{n}}{\sfrac{1}{n}}}
        \eqd\lim_{n\to\infty}\seqn{\abs{1-\sfrac{1}{n}-\sfrac{1}{n}}}
        =1
      \\&\qquad\neq 2 
        =2\abs{0-1}
        \eqd\distance{0}{1}
        =\distance{\lim_{n\to\infty}\seqn{1-\sfrac{1}{n}}}{\lim_{n\to\infty}\seqn{\sfrac{1}{n}}}
    \end{align*}
    %which by \prefpp{cor:limcont} implies that $\distancen$ is \prope{discontinuous}.

  \item In $\dspaceX$, \structe{open ball}s are \prope{open}: \label{item:dspace_21_oballo}
    \begin{enumerate}
      \item $\metrica{x}{y}\eqd\abs{x-y}$ is a \fncte{metric} and thus all open balls in that do not contain both $0$ and $1$ are \prope{open}.
      \item By \prefpp{ex:mpf_ascaled}, $\metricb{x}{y}\eqd2\abs{x-y}$ is also a \fncte{metric} and thus all open balls containing $0$ and $1$ only are \prope{open}.
      \item The only question remaining is with regards to open balls that contain $0$, $1$ and some other element(s) in $\R$.
            But even in this case, open balls are still open. For example:
        \\\indentx$\ball{-1}{2} = \intoo{-1}{2} = \intoo{-1}{1}\setu\intoo{1}{2}$\\
        Note that both $\intoo{-1}{1}$ and $\intoo{1}{2}$ are \prope{open}, and thus by \prefpp{thm:dspace_open},
        $\ball{-1}{2}$ is \prope{open} as well.
    \end{enumerate}

  \item By \pref{item:dspace_21_oballo} and \prefpp{thm:baseoball}, 
        the \structe{open ball}s of $\dspaceRd$ \emph{do} form a \structe{base} for a \structe{topology} on $\R$.

  \item In $\dspaceX$, the limits of \prope{convergent} sequences are \prope{unique}.
        This is demonstrated in \prefpp{ex:pdspace_21} using additional structure developed in \pref{chp:pdspace}.

  \item In $\dspaceX$, \prope{convergent} sequences are \prope{Cauchy}.\\
        This is also demonstrated in \prefpp{ex:pdspace_21}.

\end{enumerate}
\end{example}

The \fncte{distance function}s in \prefpp{ex:dspace_01}--\prefpp{ex:dspace_21} were all \prope{discontinuous}.
In the absence of the \prope{triangle inequality} and in light of these examples, 
one might try replacing the \prope{triangle inequality} with the weaker requirement of \prope{continuity}.
However, as demonstrated by the next example, this also leads to an arguably disastrous result.
%---------------------------------------
\begin{example}
\footnote{
  \citerppg{blumenthal1953}{12}{13}{0828402426},
  \citerppg{laos1998}{118}{119}{9810231806}
  }
\label{ex:dspace_xy2}
%---------------------------------------
The function $\distancen\in\clF{\R\times\R}{\R}$ such that $\distance{x}{y}\eqd(x-y)^2$ is a \fncte{distance} on $\R$.
\\Note some characteristics of the \structe{distance space} $\dspaceRd$:
\begin{enumerate}
  \item $\dspaceRd$ is \emph{not} a \structe{metric space} because the \prope{triangle inequality} does not hold:
    \\\indentx$\ds\distance{0}{2} \eqd \brp{0-2}^2 = 4 \nleq 2 = \brp{0-1}^2 + \brp{1-2}^2 \eqd \distance{0}{1} + \distance{1}{2} $

  \item The \fncte{distance function} $\distancen$ is \prope{continuous} in $\dspaceX$.
        This is demonstrated in the more general setting of \pref{chp:pdspace} in \prefpp{ex:pdspace_xy2}.

  \item Calculating the length of curves in $\dspaceX$ leads to a paradox:\footnote{
        This is the method of ``inscribed polygons" for calculating the length of a curve and goes back to Archimedes:
        \citerpg{brunschwig2003}{26}{0674021568},
        \citerpc{walmsley1920}{200}{\textsection158},
        }
    \begin{enumerate}
      \item Partition $\intcc{0}{1}$ into $2^\xN$ consecutive line segments connected at the points 
            \\\indentx$\seqn{0,\,\frac{1}{2^\xN},\,\frac{2}{2^\xN},\,\frac{3}{2^\xN},\,\ldots,\,\frac{2^{\xN-1}1}{2^\xN},\,1}$
      \item Then the distance, as measured by $\distancen$, between any two consecutive points is
            \\\indentx$\distance{p_n}{p_{n+1}}\eqd\brp{p_n-p_{n+1}}^2=\brp{\frac{1}{2^\xN}}^2=\frac{1}{2^{2\xN}}$
      \item But this leads to the paradox that the total length of $\intcc{0}{1}$ is 0:
            \\\indentx$\ds %1=\brp{0-1}^2\eqd\distance{0}{1}
              \lim_{\xN\to\infty}\sum_{n=0}^{2^{\xN}-1}\frac{1}{2^{2\xN}}
              =\lim_{\xN\to\infty}\frac{2^\xN}{2^{2\xN}}
              =\lim_{\xN\to\infty}\frac{1}{2^{\xN}}
              =0
             $
    \end{enumerate}
\end{enumerate}
\end{example}
