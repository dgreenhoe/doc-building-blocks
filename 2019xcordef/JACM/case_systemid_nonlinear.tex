%============================================================================
% Daniel J. Greenhoe
% LaTeX file
%============================================================================
%============================================================================
\subsection{Case study: Non-linear system identification}
\label{sec:case_systemid_nonlinear}
%============================================================================

\begin{figure}[h]
  \centering
  \begin{tabular}{|c|}
    \hline
    \tbox{\includegraphics{graphics/opT_estH_mnoise.pdf}}
    \\\hline
  \end{tabular}
  \caption{Least Square estimation \xref{prop:estHls}\label{fig:estHls}}
\end{figure}

%\begin{minipage}{\tw-50mm}
%Let $\opS$ be the \structe{system} illustrated to the right.
%\textbf{If} there is no measurement noise on the input and output and \textbf{if}
%$\opH$ is \prope{linear time invariant}, then
%$\FH = \Swxy/\Swxx$ \xref{cor:Rxyh}.
%But what if there is output measurement noise?
%And what if $\opH$ is not \prope{LTI}?
%What is the best least-squares estimate of $\FH$?
%The answer depends on how you define ``the best".
%\end{minipage}
%\hfill\tbox{\includegraphics{graphics/sysHw_xy.pdf}}

%---------------------------------------
\begin{remark}
\label{rem:fCost}
%---------------------------------------
The defintion of ``best" or ``optimal" is given by a \fncte{cost function} $\fCost(\estH)$.
There are several possible cost functions.
One possibility is to define an error $\rve(n)\eqd\rvq(n)-\rvr(n)$.
We note that if $\estH$ is closely tuned to match $\opT$, then 
not only should $\rve(n)$ be close to 0 for all $n\in\Z$, 
but the \fncte{auto-correlation} $\estRee(m)$ of $\rve(n)$ should also be close to 0 for all $m\in\Z$.
Moreover by extension, the \fncte{auto-spectral density} $\Swee(\omega)\eqd\opDTFT\estRee(m)$ 
should also be close to 0. 
As such, we can define an arguably relevant \fncte{cost function} for the system $\opS$ of
\prefpp{fig:estHls} in terms of $\Swxx$, $\Swyy$ and $\Swxy$. % as in \prefpp{def:fCost}.
In the case of Papoulis's $\Rxy(m)$, the development of such a cost function $\fCost(\estH)$ 
might look something like this:
\begin{align*}
  &(1).\quad\fCostrq\brp{\estH} %|
  \\&\eqd \opDTFT\pE\brp{\brs{\rvr(n)-\rvq(n)}\brs{\rvr(0)-\rvq(0)}^\ast}
    && \text{by definition of $\fCostrq$}
    && \text{\xref{def:fCost}}
  \\&=\opFT\brs{
            \pE\brs{\rvr(n)\rvr^\ast(0)}
           -\pE\brs{\rvr(n)\rvq^\ast(0)}
           -\pE\brs{\rvq(n)\rvr^\ast(0)}
           +\pE\brs{\rvq(n)\rvq^\ast(0)}
           }
    && \text{by \prope{linearity} of $\pE$}
    && \text{\xref{prop:pE_linop}}
  \\&\eqd \opFT\brs{\Rrr(m) - \Rrq(m) - \Rqr(m) + \Rqq(m)}
    && \text{by definition of $\Rxy$}
    && \text{\xref{def:Rxym}}
  \\&\eqd \boxed{\Swrr(\omega) - \Swrq(\omega) - \Swqr(\omega) + \Swqq(\omega)}
    && \text{by definition of $\Swxy$}
    && \text{\xref{def:Swxy}}
  %\\&= \Swrr(\omega) - \Swrq(\omega) - \Swrq^\ast(\omega) + \Swqq(\omega)
  %  && \text{by \prefp{cor:Swxy}}
  %\\&= \Swpp(\omega)\abs{\estH(\omega)}^2 - \Swrq(\omega) - \Swrq^\ast(\omega) + \Swqq(\omega)
  %  && \text{by \pref{cor:RxySwxy}}
  %\\&= \Swpp(\omega)\abs{\estH(\omega)}^2
  %   - \Swpy(\omega)\estH(\omega)
  %   - \Swpy^\ast(\omega)\estH^\ast(\omega)
  %   + \Swqq(\omega)
  %  && \text{by \pref{prop:dual_mnoise}}
\end{align*}
\end{remark}

Taking cue from the result of \pref{rem:fCost}, we arrive at \emph{a} definition of cost:

%---------------------------------------
\begin{definition}
\label{def:fCost}
%---------------------------------------
Let $\opS$ be a system defined as in \prefpp{fig:estHls}.
Define the following \fncte{cost function}s for spectral \prope{least-squares} estimates:
\defbox{\begin{array}{>{\ds}r*{2}{c>{\ds}l}}
      \fCostrq(\estH) &\eqd& 
        \Swrr(\omega) - \Swrq(\omega) - \Swqr(\omega) + \Swqq(\omega)
  \end{array}}
\end{definition}

%---------------------------------------
\begin{remark}
%---------------------------------------
Note that by \prefpp{cor:Swxx_real}, $\Swqr=\Swrq^\ast$ for (1)--(4)\ldots and thus 
the cost function $\fCost$ for (1)--(4) is \fncte{real-valued}. 
This in general is \emph{not} true for (5)--(8).
This in itself provides an argument, however weak that argument may be, 
for \emph{not} selecting any of (5)--(8) as a standard for the definition of $\Rxy(m)$.
\end{remark}

Now for each of the eight $\Rxy(m)$ definitions, we can transform the expression of $\fCostrq(\estH)$
as given by \pref{def:fCost} into expressions involving $\estH$ (next lemma). 
In doing so, one might hope to be in a good position to take partial derivatives of the real and imaginary parts of $\estH$
to find an optimal \prope{least-squares-like} solution for $\estH$.

%---------------------------------------
\begin{lemma}
\label{lem:H1LS_cost}
%---------------------------------------
Let $\fCostrq(\estH)$ be defined as in \pref{def:fCost}.
Let (1)--(8) below correspond to the eight definitions of $\Rxy(m)$ in \pref{def:Rxym}.
%\lembox{\begin{array}{>{\ds}rc>{\ds}l}
%  \fCostrq\brp{\estH}
%    &=& \Swpp     (\omega)\abs{\estH     (\omega)}^2
%      - \Swpy     (\omega)     \estH     (\omega)
%      - \Swpy^\ast(\omega)     \estH^\ast(\omega)
%      + \Swqq     (\omega)
%\end{array}}
%\end{lemma}
%\begin{proof}
\begin{small}
%\begin{align*}
%    %---------------------------
%  (1).\,\fCostrq\brp{\estH} %|
%    %---------------------------
%    &\eqd \Swrr(\omega) - \Swrq(\omega) - \Swqr(\omega) + \Swqq(\omega)
%    && \text{by definition of $\fCostrq$}
%    && \text{\xref{def:fCost}}
%  \\&= \Swrr(\omega) - \Swrq(\omega) - \Swrq^\ast(\omega) + \Swqq(\omega)
%    && \text{by \prefp{cor:Swxy}}
%
%  \\&= \Swpp(\omega)\abs{\estH(\omega)}^2 - \Swrq(\omega) - \Swrq^\ast(\omega) + \Swqq(\omega)
%    && \text{by \prefp{cor:RxySwxy}}
%
%  \\&= \Swpp(\omega)\abs{\estH(\omega)}^2
%     - \Swpy(\omega)\estH(\omega)
%     - \Swpy^\ast(\omega)\estH^\ast(\omega)
%     + \Swqq(\omega)
%    && \text{by \prefp{prop:dual_mnoise}}
%\end{align*}
\begin{align*}
  (1).\quad\fCostrq\brp{\estH}
     &= \ocom{\Swrr(\omega)- \Swrq(\omega) - \Swrq^\ast(\omega) + \Swqq(\omega)}
             {by \pref{def:fCost} and \prefp{cor:Swxy}}
    &&= \ocom{
        \Swpp(\omega)\abs{\estH(\omega)}^2
      - \Swpy(\omega)\estH(\omega)
      - \Swpy^\ast(\omega)\estH^\ast(\omega)
      + \Swqq(\omega)
      }{by \prefp{cor:RxySwxy} and \prefp{prop:dual_mnoise}}
    \\
  (2).\quad\fCostrq\brp{\estH}
     &= \Swrr(\omega)- \Swrq(\omega) - \Swrq^\ast(\omega) + \Swqq(\omega)
    &&= \Swpp(\omega)\abs{\estH(\omega)}^2
      - \Swpy(\omega)\estH^\ast(-\omega)
      - \Swpy^\ast(\omega)\estH(-\omega)
      + \Swqq(\omega)
    \\
  (3).\quad\fCostrq\brp{\estH}
     &= \Swrr(\omega)- \Swrq(\omega) - \Swrq^\ast(\omega) + \Swqq(\omega)
    &&= \Swpp(\omega)\abs{\estH(-\omega)}^2
      - \Swpy(\omega)\estH(\omega)
      - \Swpy^\ast(\omega)\estH^\ast(\omega)
      + \Swqq(\omega)
    \\
  (4).\quad\fCostrq\brp{\estH}
     &= \Swrr(\omega)- \Swrq(\omega) - \Swrq^\ast(\omega) + \Swqq(\omega)
    &&= \Swpp(\omega)\abs{\estH(-\omega)}^2
      - \Swpy(\omega)\estH^\ast(-\omega)
      - \Swpy^\ast(\omega)\estH(-\omega)
      + \Swqq(\omega)
    \\
  (5).\quad\fCostrq\brp{\estH}
     &= \Swrr(\omega)- \Swrq(\omega) - \Swrq(-\omega) + \Swqq(\omega)
    &&= \Swpp(\omega)\estH(\omega)\estH(-\omega)
      - \Swpy(\omega)\estH(\omega)
      - \Swpy(-\omega)\estH(-\omega)
      + \Swqq(\omega)
    \\
  (6).\quad\fCostrq\brp{\estH}
     &= \Swrr(\omega)- \Swrq(\omega) - \Swrq(-\omega) + \Swqq(\omega)
    &&= \Swpp(\omega)\estH(\omega)\estH(-\omega)
      - \Swpy(\omega)\estH(\omega)
      - \Swpy(-\omega)\estH(-\omega)
      + \Swqq(\omega)
    \\
  (7).\quad\fCostrq\brp{\estH}
     &= \Swrr(\omega)- \Swrq(\omega) - \Swrq(-\omega) + \Swqq(\omega)
    &&= \Swpp(\omega)\estH^\ast(\omega)\estH^\ast(-\omega)
      - \Swpy(\omega)\estH(\omega)
      - \Swpy(-\omega)\estH^\ast(\omega)
      + \Swqq(\omega)
    \\
  (8).\quad\fCostrq\brp{\estH}
     &= \Swrr(\omega)- \Swrq(\omega) - \Swrq(-\omega) + \Swqq(\omega)
    &&= \Swpp(\omega)\estH(\omega)\estH(-\omega)
      - \Swpy(\omega)\estH^\ast(-\omega)
      - \Swpy(-\omega)\estH(-\omega)
      + \Swqq(\omega)
\end{align*}
%\end{proof}
\end{small}
\end{lemma}

%%---------------------------------------
%\begin{remark}
%\label{rem:case_systemid_nonlinear}
%%---------------------------------------
For the Papoulis $\Rxy(m)$ definition (1), the $\fCostrq$ expression demonstrated in \pref{lem:H1LS_cost} is very 
useful. 
In particular, we can 
%However, it does illustrate what properties might be useful in identifying an optimal least squares solution.
%In particular, least squares methods involve 
set the partial derivatives 
$\pderiv{}{\estH_R}\estH(\omega)$ and $\pderiv{}{\estH_I}\estH(\omega)$
of the real and imaginary parts 
of $\estH(\omega)$ to zero and solve the resulting two equations to find an optimal $\estH$ (as in \prefp{prop:estHls}).

However, this becomes troublesome in the case when encountering $\estH(-\omega)$ and the 
\fncte{impulse response} of $\estH$ is \prope{complex-valued}---in which case in general $\estH(-\omega)\neq\estH^\ast(\omega)$.

Note that except for (1), \emph{all} of the expressions demonstrated in \pref{lem:H1LS_cost} contain an $\estH(-\omega)$ 
and/or $\estH^\ast(-\omega)$. 

This trouble provides an argument, however a weak one it might be, for choosing (1) as the standard definition of $\Rxy(m)$.
%\end{remark}

%---------------------------------------
\begin{proposition}
\label{prop:estHls}
%---------------------------------------
Let $\opS$ be the \structe{system} illustrated in \prefp{fig:estHls}.
\propbox{
  \brb{\begin{array}{FMMD}
      (A).& $\rvx$, $\rvu$, and $\rvv$ are & \prope{WSS}          & and
    \\(B).& $\rvx$, $\rvu$, and $\rvv$ are & \prope{uncorrelated} & and
    \\(C).& $\rvu$ and $\rvv$ are          & \prope{zero-mean}    & and
    \\(D).& $\estH$ is                     & \prope{LTI}
  \end{array}}
  \implies
  \brb{\begin{array}{>{\ds}r>{\ds}c>{\ds}l}
    \argmin_{\estH}\fCostrq(\estH) &=& \frac{\Swxy^\ast(\omega)}{\Swxx(\omega)-\Swuu(\omega)}
    \\\mc{3}{M}{for (1)}
  \end{array}}
  }
\end{proposition}
\begin{proof}
\begin{align*}
  (1).\quad 
    0 &= \pderiv{}{\estH_R}\fCostrq\brs{\estH(\omega)}
      &= 2\estH_R(\omega)\Swpp(\omega) -  \Swpy(\omega) -  \Swpy^\ast(\omega) + \cancelto{0}{\pderiv{}{\estH_R}\Swqq(\omega)}
     &&\implies \estH_R(\omega) = \frac{\Real\Swpy^\ast(\omega)}{\Swpp(\omega)}
     \\
    0 &= \pderiv{}{\estH_I}\fCostrq\brs{\estH(\omega)}
      &= 2\estH_I(\omega)\Swpp(\omega) -  i\Swpy(\omega) + i\Swpy^\ast(\omega) + \cancelto{0}{\pderiv{}{\estH_R}\Swqq(\omega)}
     &&\implies \estH_I(\omega) = \frac{\Imag\Swpy^\ast(\omega)}{\Swpp(\omega)}
\end{align*}
\begin{align*}
    \implies \estH(\omega)
      &\eqd \estH_R(\omega) + i\estH_I(\omega)
        \frac{\Real\Swpy^\ast(\omega)}{\Swpp(\omega)}
        + \frac{i\Imag\Swpy^\ast(\omega)}{\Swpp(\omega)}
    \\&= \frac{\Swpy^\ast(\omega)}{\Swpp(\omega)}
    \\&= \frac{\Swxy^\ast(\omega)}{\Swxx(\omega)-\Swuu(\omega)}
      && \text{by \prefp{prop:xvy}}
\end{align*}
\end{proof}

It follows immediately from \pref{prop:estHls} that, for (1) and in the special case
of no input noise ($\rvu(n)=0$), the standard estimate\footnote{
  \citerppgc{bendat1980}{98}{100}{0471058874}{5.1.1 Optimal Character of Calculations; note: proof minimizing $\Swvv$ but yields same result},
  \citerppgc{bendat1993}{106}{109}{0471570559}{5.1.1 Optimality of Calculations},
  \citerppgc{bendat2010}{187}{190}{1118210824}{6.1.4 Optimum Frequency Response Functions}
  }
$\estHa$ is the optimal
least-squares estimate of $\FH$ (next).

%---------------------------------------
\begin{corollary}
\label{cor:H1LSb}
%---------------------------------------
Let $\opS$ be the \structe{system} illustrated in \prefp{fig:estHls}.
\corbox{
  \brb{\begin{array}{FMD}
      (1). & hypotheses of \pref{prop:estHls} & and
    \\(2). & $\rvu(n)=0$
  \end{array}}
  \implies
  \brb{\estH(\omega) = \estHa(\omega) \eqd \frac{\Swxy^\ast(\omega)}{\Swxx(\omega)}
  }
  \qquad
  \begin{array}{M}
    for (1)
  \end{array}
  }
\end{corollary}
