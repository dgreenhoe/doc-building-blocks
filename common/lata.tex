%============================================================================
% Daniel J. Greenhoe
% LaTeX file
%============================================================================

%=======================================
\section{Atomic lattices}
%=======================================
%---------------------------------------
\begin{definition}
\citetbl{
  \citePp{larson1975}{178},
  \citePpc{birkhoff1938}{800}{see footnote \ddag}
  }
\label{def:lata}
\label{def:atomic}
%---------------------------------------
Let $\latL\eqd\latbX$ be a \structe{bounded lattice} \xref{def:latb}.
\\\defboxt{\begin{tabular}{l}
    $x$ is an  \hid{atom}      of $\latL$ if $x$ \prope{covers} \xref{def:cover} \lzero.\\
    $x$ is an  \hid{anti-atom} of $\latL$ if $x$ is \prope{covered by} \lid.\\
    $\latL$ is \hid{atomic} if every $x\in\setX\setd\lzero$ can be represented as joins of atoms of $\latL$. \\
    $\latL$ is \hid{anti-atomic} if every $x\in\setX\setd\lid$ can be represented as meets of anti-atoms of $\latL$.
  \end{tabular}}
\end{definition}

\begin{figure}
\psset{unit=\latunit}%
\gsize%
\centering%
\begin{tabular}{|cc|cc|cc|}%
\hline%
\mc{2}{|G}{atomic lattices}&%
\mc{2}{|G}{anti-atomic}&%
\mc{2}{|G|}{atomic and anti-atomic}%
\\\hline%
\includegraphics{graphics/lat7_m2inm2.pdf}&%
\includegraphics{graphics/lat7_m2m2l.pdf}&%
\includegraphics{graphics/lat7_m2inm2_dual.pdf}&%
\includegraphics{graphics/lat7_m2m2l_dual.pdf}&%
\includegraphics{graphics/lat7_m5.pdf}&%
\includegraphics{graphics/lat8_l2e3.pdf}%
\\
\mc{6}{|G|}{neither atomic nor anti-atomic}%
\\
\mc{1}{|c}{\includegraphics{graphics/lat7_m2onm2.pdf}}&%
\mc{1}{c}{ \includegraphics{graphics/lat7_l4m3.pdf}}&%
\mc{1}{c}{ \includegraphics{graphics/lat7_l5inm2.pdf}}&%
\mc{1}{c}{ \includegraphics{graphics/lat7_l3ino6.pdf}}&%
\mc{1}{c}{ \includegraphics{graphics/lat7_l3ino6l.pdf}}&%
\mc{1}{c|}{\includegraphics{graphics/lat4_l4.pdf}}%
\\\hline%
\end{tabular}
\caption{%
  Selected \prope{atomic}, \prope{anti-atomic}, and neither atomic nor anti-atomic lattices (see \prefp{ex:lata})
  \label{fig:lata}
  }%
\end{figure}
%---------------------------------------
\begin{example}
\label{ex:lata}
%---------------------------------------
\prefpp{fig:lata} illustrates some examples of lattices that are \prope{atomic}, \prope{anti-atomic},
both, and neither.
\end{example}

