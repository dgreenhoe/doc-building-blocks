%============================================================================
% Daniel J. Greenhoe
% LaTeX File
% Daubechies-3 Pole Zero plot
% nominal font size = \gsize
% nominal unit = 10mm
%============================================================================
\begin{pspicture}(-1.5,-2)(4.1,2)%
  %-------------------------------------
  % settings
  %-------------------------------------
  \psset{radius=1mm,linewidth=0.75pt}%
  %-------------------------------------
  % nodes
  %-------------------------------------
  \pnode(0,0){origin}%          z-plane origin
  \pnode(-1,0){pzeroes}%        p zeroes location
  \pnode(0,0){poles}%           2p-1 poles location
  %-------------------------------------
  % asymptotic regions
  %   |z-1| = sqrt(2) (arc with radius sqrt(2) centered at z=1)
  %   |z+1| = sqrt(2) (arc with radius sqrt(2) centered at z=-1)
  % sqrt(2) ~= 1.4142135623730950488016887242097
  %-------------------------------------
  \pscustom[fillstyle=solid,fillcolor=pzasymshade,linestyle=solid,linecolor=pzasym]{%
    \psarc(-1,0){1.414214}{-45}{45}%
    \psarcn( 1,0){1.414214}{135}{-135}% 
    }%
  %\pnode(0.2,0.73){zm1P}%
  %\pnode(2,1){zp1P}%
  %\rput[bl](-1.5,0.433){\rnode[r]{zm1L}{$\scy\abs{z-1}=\sqrt{2}$}}%
  %\rput[br](2.9,1.6){\rnode[b]{zp1L}{$\scy\abs{z+1}=\sqrt{2}$}}%
  %\ncline[linecolor=pzasym]{->}{zm1L}{zm1P}%
  %\ncline[linecolor=pzasym,nodesepA=2pt]{->}{zp1L}{zp1P}%
  %-------------------------------------
  % axes
  %-------------------------------------
  \psaxes[linecolor=axis,labels=none]{<->}(0,0)(-1.5,-1.5)(3.5,1.5)%
  \uput{1pt}[0](3.5,0){\color{axis}$\scy\Reb{z}$}%
  \uput{2pt}[210](0,1.5){\color{axis}$\scy\Imb{z}$}%
  %-------------------------------------
  % unit circle
  %-------------------------------------
  \pscircle[linecolor=unitcircle](origin){1}%            unit circle
  \rput[tr](-1.5mm,-1.5mm){\rnode[b]{circleL}{$\scy\abs{z}=1$}}% circle label
  \pnode(-0.707,-0.707){circleP}% circle point
  \ncline[linecolor=unitcircle,linestyle=dotted,nodesepA=0pt]{->}{circleL}{circleP}% circle pointer
  %-------------------------------------
  % B-spline zeroes (at z=-1)
  %-------------------------------------
  \Cnode[linewidth=1pt](-1,0){zp}%
  \rput[bl](-1.45,-1.9){\rnode[t]{zpL}{$p$}$=3$ zeroes}%
  \ncline[linestyle=dotted,linecolor=zero,nodesepA=0pt]{->}{zpL}{zp}%
  \uput{1mm}[30](-1,0){\color{zero}$\scy3$}%
  %-------------------------------------
  % orthogonality zeroes
  %-------------------------------------
  \Cnode [linecolor=zero,linewidth=1pt]( 0.2872513780,  0.1528923339 ){z1}% 
  \Cnode [linecolor=zero,linewidth=1pt]( 0.2872513780, -0.1528923339 ){z2}% 
  \Cnode*[linecolor=zero,linewidth=1pt]( 2.7127486220,  1.4438867826 ){z3}% 
  \Cnode*[linecolor=zero,linewidth=1pt]( 2.7127486220, -1.4438867826 ){z4}% 
  %-------------------------------------
  % radial lines
  %-------------------------------------
  \ncline[linestyle=dashed]{origin}{z3}%
  \ncline[linestyle=dashed]{origin}{z4}%
  %-------------------------------------
  % poles
  %-------------------------------------
  %\rput(poles){\color{pole}$\mathbf\times$}%
  \rput(poles){%
    \psline[linecolor=pole,linewidth=1pt](-0.1,+0.1)(+0.1,-0.1)%
    \psline[linecolor=pole,linewidth=1pt](-0.1,-0.1)(+0.1,+0.1)%
    }%
  \rput[tl](-1.45,1.9){$2$\rnode[br]{polesL}{$p$}$-1=5$ poles}%
  \ncline[linestyle=dotted,linecolor=pole,nodesepA=0pt,nodesepB=5pt]{->}{polesL}{poles}%
  \uput{1mm}[30](0,0){\color{pole}$\scy5$}%
  %-------------------------------------
  % discarded zeroes region
  %-------------------------------------
 %\psframe[linestyle=dotted](0.8,-1.9)(3,1.9)%
  \psframe[linestyle=dotted](3,-1.9)(2.4,1.9)%
  \rput[b]{-90}(3,0){\color{blue}\footnotesize$p-1=2$ discarded zeroes}%
\end{pspicture}%
%
%\begin{picture}(450,400)(-150,-200)
%  %\graphpaper[10](0,0)(200,200)
%  \footnotesize%
%  \thicklines%
%  \color{axis}%
%    \put(-130,   0){\line(1,0){420} }%
%    \put(   0,-130){\line(0,1){260} }%
%    \put( 300,   0){\makebox(0,0)[l]{$\Reb{z}$}}%
%    \put(   0, 140){\makebox(0,0)[b]{$\Imb{z}$}}%
%  \color{circle}%
%    %\input{circ512.inc}
%    \color{circle}%
%    \qbezier( 100,   0)( 100, 41.421356)(+70.710678,+70.710678)% 0   -->1pi/4
%    \qbezier(   0, 100)( 41.421356, 100)(+70.710678,+70.710678)% pi/4-->2pi/4
%    \qbezier(   0, 100)(-41.421356, 100)(-70.710678,+70.710678)%2pi/4-->3pi/4
%    \qbezier(-100,   0)(-100, 41.421356)(-70.710678,+70.710678)%3pi/4--> pi
%    \qbezier(-100,   0)(-100,-41.421356)(-70.710678,-70.710678)% pi  -->5pi/4
%    \qbezier(   0,-100)(-41.421356,-100)(-70.710678,-70.710678)%5pi/4-->6pi/4
%    \qbezier(   0,-100)( 41.421356,-100)( 70.710678,-70.710678)%6pi/4-->7pi/4
%    \qbezier( 100,   0)( 100,-41.421356)( 70.710678,-70.710678)%7pi/4-->2pi
%    %\put( 110, 110){\makebox(0,0)[lb]{$z=e^{i\omega}$}}
%    %\put( 105, 105){\vector(-1,-1){33}}
%  \color{pzasym}%
%    \qbezier(41.4214,0)(41.4214, 58.5786)(0, 100)%      % inner upper arc
%    \qbezier(41.4214,0)(41.4214,-58.5786)(0,-100)%      % inner lower arc
%    \qbezier(0, 100)(41.421, 141.421)(100, 141.421)%outer upper arc
%    \qbezier(100, 141.421)(158.579,141.421)(200,100)%
%    \qbezier(200,100)(241.421,58.5786)(241.421,0)%
%    \qbezier(0,-100)(41.421,-141.421)(100,-141.421)%outer lower arc
%    \qbezier(100,-141.421)(158.579,-141.421)(200,-100)%
%    \qbezier(200,-100)(241.421,-58.5786)(241.421,0)%
%  \color{pole}%
%    \put(-110,-110){\makebox(0,0)[tl]{5 poles}}%
%    \put(-100,-100){\vector(1, 1){91}}%
%    \put(   0,    0){\makebox(0,0)[c]{$\times$}}%
%    \put(   0,    0){\makebox(0,0)[c]{$\times$}}%
%    \put(   0,    0){\makebox(0,0)[c]{\hspace{1em}$^5$}}%
%  \color{zero}%
%    \put(-10,110){\makebox(0,0)[br]{p=3 zeros}}%
%    \put(-100,100){\vector( 0,-1){60}}%
%    \put(-100,    0){\circle{15}}%
%    \put(-100,    0){\circle{15}$^3$}%
%    \qbezier[30](0,0)(135.5,62)(271,144)%
%    \qbezier[30](0,0)(135.5,-62)(271,-144)%
%    \put(260,-144){\makebox(0,0)[rc]{discarded zero $\rightarrow$}}%
%    \put(260, 144){\makebox(0,0)[rc]{discarded zero $\rightarrow$}}%
%    \put( 028.72513780,  015.28923339 ){\circle {15}}% 
%    \put( 028.72513780, -015.28923339 ){\circle {15}}% 
%    \put( 271.27486220,  144.38867826 ){\circle*{15}}% 
%    \put( 271.27486220, -144.38867826 ){\circle*{15}}% 
%\end{picture}                                   
