%============================================================================
% LaTeX File
% Daniel J. Greenhoe
%============================================================================
%=======================================
\chapter{Density Estimation}
\label{chp:pdfest}
%=======================================
%=======================================
\section{Introduction}
%=======================================
Moment estimation is extremely useful for model-building
when a \emph{type} of parameterized
\fncte{probability density function} (\fncte{pdf}) $\ppx(x)$ \xref{def:pdf}
is known or more likely \emph{assumed},
but the \emph{parameters} themselves are \emph{not known}.
Examples of parameterized pdfs include the following:
\\\begin{tabular}{clll}
    \imark & The \fncte{Uniform Distribution}    & with parameter mean $\pmean$                        & \xref{def:uniform}.
  \\\imark & The \fncte{Gamma Distribution}      & with parameter $b$                                  & \xref{def:gamma}.
  \\\imark & The \fncte{Chi-square Distribution} & with parameter variance $\pvar$                     & \xref{def:chisq}.
  \\\imark & The \fncte{Gaussian Distribution}   & with parameters mean $\pmean$ and variance $\pvar$  & \xref{def:normal}.
  \\\imark & The \fncte{Rayleigh Distribution}   & with parameters radius $r$ and variance $\pvar$     & \xref{def:rayleigh}.
\end{tabular}

In the case of distributions with mean and/or variance parameters,
these parameters can be estimated using the techniques in this chapter,
and the technique of estimating the pdf in this manner is called
\opd{parametric density estimation}.\footnote{
  \citerpgc{gramacki2017}{2}{9783319716886}{\textsection ``1.1 Background"}
  }
If a distribution type is not known and not assumed, then
the distribution itself must be estimated
using \opd{nonparametric density estimation} \xref{chp:pdfest}

%=======================================
\section{Parametric density estimation}
%=======================================
%---------------------------------------
\begin{corollary}
\label{cor:mean_white_gaussian}
%---------------------------------------
\corbox{
  \brb{\begin{array}{FMD}
      (A). & $\seqn{\rvx_n}$ is \prope{uncorrelated} & and
    \\(B). & $\rvx_n$ is \prope{Gaussian}
  \end{array}}
  \implies
  \brb{\begin{array}{FMD}
      (1). & $\ds\estML[\mu] = \frac{1}{\xN}\sum_{n=1}^{\xN} \rvx_n$ & and
    \\(2). & $\estML[\mu]$ is \prope{consistent}              & and
    \\(3). & $\estML[\mu]$ is \prope{efficient}               &
  \end{array}}
  }
\end{corollary}
\begin{proof}
This result follows directly from \prefpp{thm:estML-CR} with
\\\indentx$\begin{array}{rc>{\ds}lM}
   \rvy(t)          &\eqd& \rvx(t;\theta) + \fv(t) & where $\fv(t)$ is a \prope{zero-mean} white Gaussian noise process
   \\\rvx(t;\theta) &\eqd& \fg(\theta)
   \\               &\eqd& \theta
   \\               &\eqd& \mu
   \\\rvx_n         &\eqd& \fdoty_n
   \\               &\eqd& \inprod{\fy(t)}{\fpsi_n(t)}
   \\               &\eqd& \inprod{\fy(t)}{\delta(t-n\tau)}
   \\               &\eqd& \int_{t\in\R} \fy(t) \delta(t-n\tau) \dt
   \\               &=&    \fy(n\tau)
\end{array}$

Alternatively, the results follow from \prefpp{thm:estML_amplitude}.
\end{proof}

%%---------------------------------------
%\begin{minipage}{\tw-115mm}
%\begin{example}
%%---------------------------------------
%The R code to the right
%calculates the \fncte{average} \xref{def:average} number of sunspots
%from 1749 into 2013.
%\end{example}
%\end{minipage}
%\hfill
%\begin{minipage}{110mm}
%\begin{lstlisting}[language=R]
%require(stats);
%N        = 50;
%mean     = 3;
%stddev   = 10;
%x        = stats::rnorm(N, mean, stddev);
%estimate = sum(x) / N;
%print(estimate);
%plot (x, col="blue", type="l");
%lines(c(start(x)[1],end(x)[1]), c(mean,mean), col="green");
%lines(c(start(x)[1],end(x)[1]), c(estimate,estimate), col="red");
%\end{lstlisting}
%\end{minipage}

%=======================================
\section{Nonparametric density estimation}
%=======================================
Some techniques of \ope{nonparametric density estimation} include\footnote{
  \citerg{silverman1986}{9780412246203},
  \citerppgc{tsybakov2008}{1}{27}{9780387790527}{\textsection1.1--\textsection1.3},
  \citerg{gramacki2017}{9783319716886},
  \citerppgc{vidakovic}{217}{245}{9780471293651}{Chapter 7 ``Density Estimation"}
  }
\begin{listi}
  \item Histogram %/ Smoothed Histograms
  \item Average Shifted Histogram
  \item \ope{Kernel Density Estimation} (\ope{KDE})
  \item Fourier-based
  \item Wavelet-based
  \item Fast Gauss Transform
\end{listi}

KDE's are conceptually quite simple, but do have some difficulties:
\begin{listi}
  \item KDEs are good for 6 dimensions or less---the ``curse of dimensionality"\footnote{
        \citerp{bellman1954}{206},
        \citerpc{bellman1961}{94}{\textsection ``5.16 The Curse of Dimensionality"},
        \citerp{bellman1971}{44},
        \citerpgc{gramacki2017}{59}{9783319716886}{\textsection ``The Curse of Dimensionality"}
        }
  \item KDEs have a \vale{bandwidth} parameter and the computation of this parameter can be quite demanding.
\end{listi}

