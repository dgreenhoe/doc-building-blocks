%============================================================================
% XeLaTeX File
% Daniel J. Greenhoe
%============================================================================
%=======================================
\chapter{DSP Calculus}
%=======================================

%=======================================
\section{Why it's important}
%=======================================
For modeling real-world processes above the quantum level, measurements are \prope{continuous} in time---that is,
the first derivative of a function over time representing the measurement \prope{exists}.

But even for ``simple" physical systems, it is not just the first derivative that matters.
For example, the classical ``vibrating string" vertical displacement $\fu(x,t)$ wave equation can be described as
        \\\indentx$\ds\ppderiv{u}{x^2} - \frac{1}{c^2}\ppderiv{u}{t^2} = 0$

\begin{figure}
  \centering
  \begin{tabular}{c}
    \includegraphics[width=\tw-10mm]{graphics/sunspots.pdf}\\
    \includegraphics[width=\tw-10mm,height=50mm]{graphics/earthquake_kapi_20180928.pdf}
  \end{tabular}
  \caption{Sunspot and earthquake measurements\label{fig:sunspot}}
\end{figure}
Not only do physical systems demonstrate heavy dependence on the derivatives of their measurement functions,
but also commonly exhibit \prope{oscillation}, as demonstrated by sunspot activity over the last 300 years or
earthquake activity \xref{fig:sunspot}.

In fact, derivatives and oscillations are fundamentally linked
as demonstrated by the fact that 
all solutions of homogeneous second order differential equations
        are linear combinations of sine and cosine functions\ifsxref{harTrig}{thm:D2f_cos_sin}:
        \\\indentx$\ds  \brb{\opDiff\ff + \ff=0}
  %     {\fncte{2nd order homogeneous differential equation}}
  \quad\iff\quad
  \brb{\ff(x) = \ff(0)\,\cos(x) + \ff^\prime(0)\,\sin(x)}
  \qquad\msizes\forall\ff\in\spC,\,\forall x\in\R$

Derivatives are calculated \prope{locally} about a point.
Oscillations are observed \prope{globally} over a range,
and analyzed (decomposed) by projecting the function onto a sequence of basis functions---sinusoids 
in the case of Fourier Transform family.
Projection is accomplished using inner products, and often these are calculated using \ope{integration}.
Note that derivatives and integrals are also fundamentally linked as demonstrated by the
\thme{Fundamental Theorem of Calculus}\ldots which shows that integration 
can be calculated using anti-differentiation:
\\\indentx$\ds\int_a^b \ff(x)\dx = \fF(b) - \fF(a)$\qquad where $\fF(x)$ is the \fncte{antiderivative} of $\ff(x)$.

%In nature, take displacement, velocity, and acceleration for example:
%\\\indentx$\begin{array}{rc>{\ds}l}
%  \fx(t) &=& \int_{u=0}^t \fv(u) \du + \mcom{\fx(0)}{initial condition}
%  \\
%  \fv(t) &=& \int_{u=0}^t \fa(u) \du + \mcom{\fv(0)}{initial condition}
%\end{array}$

Brook Taylor showed that for \prope{analytic} functions,\footnote{
  \prope{analytic} functions: Functions for which all their derivatives exist.}
knowledge of the derivatives of a function at a location $x=a$
%\\\indentx$\seqn{\ff(a),\, \frac{1}{1!}\ff'(a),\,  \frac{1}{2!}\ff''(a),\,  \frac{1}{3!}\ff'''(a),\cdots}$\\
%of the Taylor polynomial at the point $x=a$ 
allows you to determine (predict) arbitrarily closely all the points $\ff(x)$ in the vicinity of $x=a$:\footnote{
\citePp{robinson1982}{886}
}
\\\indentx$\ds\ff(x) = \ff(a) + \frac{1}{1!}\ff'  (a)\brs{x-a}
                              + \frac{1}{2!}\ff'' (x)\brs{x-a}^2
                              + \frac{1}{3!}\ff'''(x)\brs{x-a}^3
                              + \cdots
                  $

On the other hand, the \ope{Fourier Transform} is a kind of counter-part of the Taylor expansion:\footnote{
        \citePp{robinson1982}{886}
        }
        %\begin{tabular}{|@{2}{p{\tw/2-5mm}|}}
        \\\begin{tabular}{|c|l|l|}
            \hline
              & \mc{1}{|c|}{Taylor coefficients} & \mc{1}{c|}{Fourier coefficients}
            \\\hline
              \imark&Depend on derivatives $\ds\dndxn\ff(x)$        &Depend on integrals   $\ds\int_{x\in\R} \ff(x)e^{-i\omega x} \dx$
            \\\imark&Behavior in the vicinity of a point.           &Behavior over the entire function.
            \\\imark&Demonstrate trends locally.                    &Demonstrate trends globally, such as oscillations.
            \\\imark&Admits \prope{analytic} functions only.        &Admits \prope{non-analytic} functions as well.
            \\\imark&Function must be \prope{continuous}.           &Function can be \prope{discontinuous}.
            \\\hline
        \end{tabular}

%=======================================
\section{Fourier Transform calculus}
%=======================================
%---------------------------------------
\begin{theorem}
\label{thm:opLT_diff}
%---------------------------------------
Let $\opLT$ be the \ope{Laplace Transform} operator.
\mbox{}\\
\thmbox{
    \brb{\lim_{t\to-\infty}\fx(t)=0}
    \qquad\implies\qquad
    \brb{\opLT \brs{\ddt \fx(t)} = s \brs{\opLT\fx}(s)}
  }
\end{theorem}
\begin{proof}
\begin{align*}
  \boxed{\opLT \brs{\ddt \fx(t)}} 
    &\eqd \int_{t\in\R} \mcom{\brs{\ddt \fx(t)}}{$\dv$} \mcom{e^{-st}}{$u$} \dt
    && \text{by definition of $\opLT$}
  \\&= \brlr{\mcom{e^{-st}}{$u$} \mcom{\fx(t)}{$v$}}_{t=-\infty}^{t=+\infty}
      -\int_{t\in\R} \mcom{\fx(t)}{$v$} \mcom{(-s)e^{-s t}}{$\du$} \dt
    && \text{by \thme{Integration by Parts}}
  \\&= \cancelto{0}{e^{-s \infty}}\fx(\infty) - e^{s \infty}\cancelto{0}{\fx(-\infty)} 
      -(-s)\mcom{\int_{t\in\R} \fx(t) e^{-st} \dt}{\ope{Laplace Transform} of $\fx(t)$}
    && \text{by left hypothesis}
  \\&= \boxed{s \brs{\opLT\fx}(s)}
\end{align*}
\end{proof}

%---------------------------------------
\begin{corollary}
%---------------------------------------
Let $\opFT$ be the \ope{Fourier Transform} operator.
\corbox{
    \brb{\lim_{t\to-\infty}\fx(t)=0}
    \qquad\implies\qquad
    \brb{\opFT \brs{\ddt \fx(t)} = i\omega \brs{\opFT\fx}(\omega)}
  }
\end{corollary}
\begin{proof}
\begin{align*}
  \opFT \brs{\ddt \fx(t)}
    &\eqd \brlr{\opLT\brs{\ddt \fx(t)}(s)}_{s=i\omega}
    && \text{by definitions of $\opLT$ and $\opFT$}
  \\&= \brlr{s \brs{\opLT\fx(t)}(s)}_{s=i\omega}
    && \text{by \prefp{thm:opLT_diff}}
  \\&= i\omega \brs{\opFT\fx}(\omega)
\end{align*}
\end{proof}

%---------------------------------------
\begin{theorem}
\label{thm:opLT_int}
%---------------------------------------
Let $\opLT$ be the \ope{Laplace Transform} operator.
\thmbox{
    %\brb{\lim_{t\to-\infty}\fx(t)=0}
    %\qquad\implies\qquad
    \opLT \int_{u=-\infty}^{u=t} \fx(u) \du = \frac{1}{s} \brs{\opLT\fx}(s)
  }
\end{theorem}
\begin{proof}
\begin{enumerate}
  \item Define the \fncte{Heaviside function} $\fh(t)$ as\qquad \label{idef:heaviside}
    $\ds \fh(t)\eqd\brbl{\begin{array}{lM}
                           1 & for $t\ge0$
                         \\0 & otherwise
                         \end{array}}$

  \item Remainder of proof\ldots
    \begin{align*}
      \boxed{\opLT \int_{u=-\infty}^{u=t} \fx(u) \du}
        &\eqd \int_{t=-\infty}^{t=+\infty} \brs{\int_{u=-\infty}^{u=t} \fx(u) \du} e^{-s t} \dt
        && \text{by definition of $\opLT$}
      \\&= \int_{t=-\infty}^{t=+\infty} \brs{\int_{u=-\infty}^{u=+\infty} \fx(u) \fh(t-u) \du} e^{-s t} \dt
        && \brp{\begin{array}{M}
             by definition of \fncte{Heaviside function}\\
             \pref{idef:heaviside}
           \end{array}}
      \\&= \int_{v=-\infty}^{v=+\infty} \int_{u=-\infty}^{u=+\infty} \fx(u) \fh(v)  e^{-s (u+v)} \du \dv
        && \brp{\begin{array}{Mr@{\hspace{2pt}}c@{\hspace{2pt}}l}
             where      & v&\eqd&t-u\\ 
             $\implies$ & t&=&u+v
           \end{array}}
      \\&= \brs{\int_{v=-\infty}^{v=+\infty} \fh(v) e^{-s v} \dv} 
            \mcom{\brs{\int_{u=-\infty}^{u=+\infty} \fx(u)   e^{-s u} \du }}
                 {\ope{Laplace Transform} of $\fx(t)$}
      \\&= \brs{\int_{v=0}^{v=\infty} e^{-s v} \dv} \brs{\opLT\fx}(s)
        && \brp{\begin{array}{M}
             by definition of \fncte{Heaviside function}\\
             \pref{idef:heaviside}
           \end{array}}
      \\&= \brlr{\frac{1}{-s}e^{-s v}}_{v=0}^{v=\infty} \brs{\opLT\fx}(s)
        && \text{by \thme{Fundamental Theorem of Calculus}}
      \\&= \boxed{\frac{1}{s} \brs{\opLT\fx}(s)}
    \end{align*}
\end{enumerate}
\end{proof}

%---------------------------------------
\begin{corollary}
%---------------------------------------
Let $\opFT$ be the \ope{Fourier Transform} operator.
\corbox{
    \opFT \int_{u=-\infty}^{u=t} \fx(u) \du = \frac{1}{i\omega} \brs{\opFT\fx}(\omega)
  }
\end{corollary}
\begin{proof}
\begin{align*}
  \opFT \int_{u=-\infty}^{u=t} \fx(u) \du
    &\eqd \brlr{\opLT \int_{u=-\infty}^{u=t} \fx(u) \du}_{s=i\omega}
  \\&=    \brlr{\frac{1}{s} \brs{\opLT\fx(t)}(s)}_{s=i\omega}
    && \text{by \prefp{thm:opLT_int}}
  \\&=    \frac{1}{i\omega} \brs{\opFT\fx(t)}(\omega)
\end{align*}
\end{proof}

%=======================================
\section{Digital differentiation methods}
%=======================================
%=======================================
%\subsection{Method 1: Difference}
%=======================================


Digital Differentiation Method \#1: \ope{Difference}\footnote{\citerpgc{williams1986}{69}{9780132018562}{Difference}}
 
\begin{align*}
  \fy[n]
    &\eqd \fx[n] - \fx[n-1]
  \\\\
  \opZ\brb{\fy[n]} &= \opZ\brb{\fx[n] - \fx[n-1]}
  \\
  \ZY(z) &= \ZX(z) + z^{-1}\ZX(z)
  \\\\
  \frac{\ZY(z)}{\ZX(z)} &= 1-z^{-1} \quad=\quad \boxed{\frac{z-1}{z}}
  \qquad\brb{\begin{array}{MM}
    How many zeros? & Where?\\
    How many poles? & Where?
  \end{array}}
\end{align*}

%---------------------------------------
%\newpage\mbox{} 
%---------------------------------------
 
\begin{figure}[h]
  \centering
  \begin{tabular}{|c|c|}
      \hline
      \includegraphics{graphics/diff.pdf} & \includegraphics{graphics/cendiff.pdf}
    \\Difference method & Central Difference method
    \\\hline
  \end{tabular}
  \caption{Digital differentiation methods\label{fig:differentiation}}
\end{figure}

Is digital differentiation equivalent to continuous differentiation?\footnote{\citerpgc{williams1986}{70}{9780132018562}{Figure 2.14(a)}}
 
\begin{align*}
  \abs{\frac{z-1}{z}}_{z=e^{i\omega}}
    &= \abs{\frac{e^{i\omega}-1}
                 {e^{i\omega}}
           }
  \\&= \abs{\frac{e^{i\omega/2}\brp{e^{i\omega/2}-e^{-i\omega/2}}}
                 {e^{i\omega}}
           }
   &&= \abs{\mcom{e^{-i\omega/2}}{phase}\,\mcom{{2\sin\brp{\frac{\omega}{2}}}}{magnitude}}
  \\&= \boxed{{2\sin\brp{\frac{\omega}{2}}}} 
    && \text{for $0\le\omega\le\pi$}
\end{align*}

%=======================================
\subsection{Digital Differentiation Method \#2: \ope{Central Difference}}
\footnote{\citerpgc{williams1986}{69}{9780132018562}{Difference}}
%=======================================
 
\begin{align*}
  \fy[n]
    &\eqd \frac{\fx[n] - \fx[n-2]}{2}
  \\\\
  Y(z) &= \frac{X(z) + z^{-1}X(z)}{2}
  \\\\
  \frac{Y(z)}{X(z)} &= \frac{1-z^{-1}}{2} \quad=\quad {\frac{z^2-1}{2z^2}}
  \\\\
                    &= \boxed{\frac{(z+1)(z-1)}{2z^2}} 
  \qquad\brb{\begin{array}{MM}
    How many zeros? & Where?\\
    How many poles? & Where?
  \end{array}}
\end{align*}


Central Difference = Continuous Differentiation?\footnote{\citerpgc{williams1986}{70}{9780132018562}{Figure 2.14(b)}}
 
\begin{align*}
  \abs{\frac{z^2-1}{2z^2}}_{z=e^{i\omega}}
    &= \abs{\frac{e^{2i\omega}-1}
                 {2e^{2i\omega}}
           }
     = \abs{\brp{\frac{e^{i\omega}}{e^{2i\omega}}}
            \frac{\brp{e^{i\omega}-e^{-i\omega}}}{2}
           }
  \\&= \abs{\brp{e^{-i\omega}}
            \frac{\brs{cos(\omega)+i\sin(\omega)}-\brs{\cos(\omega)+i\sin(-\omega)}}{2}
           }
  \\&= \abs{\brp{e^{-i\omega}}
            \frac{\brs{cos(\omega)+i\sin(\omega)}-\brs{\cos(\omega)-i\sin(\omega)}}{2}
           }
  \\&= \abs{\brp{e^{-i\omega+\pi/2}}
            \frac{2\sin(\omega)}{2}
           }
     = \boxed{\abs{\sin(\omega)}}
\end{align*}


%=======================================
\section{Digital integration}
%=======================================
%=======================================
\subsection{Digital Integration Method \#1: \opd{Summation}}
%=======================================

{\begin{align*}
  \fy[n]
    &\eqd \fx[n] + \mcom{\fx[n-1] + \fx[n-2] + \fx[n-3] + \fx[n-4] + \fx[n-5] + \cdots}{{$\fy[n-1]$}}
  \\
  \fy[n] &=    \fx[n] + \fy[n-1]
  \\\\
  \opZ\brb{\fy[n]} &= \opZ\brb{\fx[n] + \fy[n-1]}
  \\
  Y(z) &= X(z) + z^{-1}Y(z)
  \\
  Y(z)\brs{1-z^{-1}} &= X(z)
  \\\\
  \frac{Y(z)}{X(z)} &= \frac{1}{1-z^{-1}} \quad=\quad \boxed{\frac{z}{z-1}}
  \qquad\brb{\begin{array}{MM}
    How many zeros? & Where?\\
    How many poles? & Where?
  \end{array}}
\end{align*}}


%---------------------------------------
%\newpage\mbox{} 
%---------------------------------------
%=======================================
\subsection{Digital Integration Method \#2: \opd{Trapezoid}}
%=======================================
 
{\begin{align*}
  \fy[n]
    &\eqd \frac{\fx[n]+\fx[n-1]}{2} + \frac{\fx[n-1]+\fx[n-2]}{2} + \frac{\fx[n-2]+\fx[n-3]}{2} + \cdots
  \\&=    \sfrac{1}{2}\fx[n] + \mcom{\fx[n-1] + \fx[n-2] + \fx[n-3] + \fx[n-4] + \fx[n-5] + \cdots}{{$\fy[n-1]+\sfrac{1}{2}\fx[n-1]$}}
  \\&=    \sfrac{1}{2}\fx[n] + \fy[n-1]+\sfrac{1}{2}\fx[n-1]
  \\\\
  \fy[n]-\fy[n-1]&=    \sfrac{1}{2}\brs{\fx[n] + +\fx[n-1]} 
  \\\\
  Y(z)\brs{1-z^{-1}} &= \sfrac{1}{2} X(z)\brs{1+z^{-1}}
  \\\\
  \frac{Y(z)}{X(z)} 
    &= \brp{\frac{1}{2}} \frac{1+z^{-1}}{1-z^{-1}}
     = \boxed{\brp{\frac{1}{2}} \frac{z+1}{z-1}} 
  \qquad\brb{\begin{array}{MM}
    How many zeros? & Where?\\
    How many poles? & Where?
  \end{array}}
\end{align*}}

%=======================================
\subsection{Digital Integration Method \#3:  \opd{Simpson's Rule}}
%=======================================

% 
%{\begin{align*}
%  \fy[n]
%    &\eqd \frac{\fx[n  ]+4\fx[n -1]+\fx[n -2]}{3}   
%     +    \frac{\fx[n-1]+4\fx[n -2]+\fx[n -3]}{3}   
%     +    \frac{\fx[n-2]+4\fx[n -3]+\fx[n -4]}{3} 
%     +    \frac{\fx[n-3]+4\fx[n -4]+\fx[n -5]}{3} 
%     +    \frac{\fx[n-4]+4\fx[n -5]+\fx[n -6]}{3} 
%     +    \frac{\fx[n-5]+4\fx[n -6]+\fx[n -7]}{3} 
%     +    \frac{\fx[n-6]+4\fx[n -7]+\fx[n -8]}{3} 
%     +    \frac{\fx[n-7]+4\fx[n -8]+\fx[n -9]}{3} 
%     +    \frac{\fx[n-8]+4\fx[n -9]+\fx[n-10]}{3} 
%     +    \frac{\fx[n-9]+4\fx[n-10]+\fx[n-11]}{3} 
%     +    \cdots  
%  \\&=    \sfrac{1}{3}\fx[n]+\sfrac{5}{3}\fx[n-1] 
%        + \mcom{2\fx[n-2] + 2\fx[n-3] + 2\fx[n-4] + \cdots}
%               {$\fy[n-2]+\sfrac{5}{3}\fx[n-2]+\sfrac{1}{3}\fx[n-3]$}
%
%
%  \\&=    \sfrac{1}{3}\fx[n]+\sfrac{5}{3}\fx[n-1] 
%        + \mcom{2\fx[n-2] + 2\fx[n-3] + 2\fx[n-4] + \cdots}
%               {$\fy[n-2]+\sfrac{5}{3}\fx[n-2]+\sfrac{1}{3}\fx[n-3]$}
%
%
%  \\&=    \sfrac{1}{3}\fx[n]+\mcom{\sfrac{5}{3}\fx[n-1] + 2\fx[n-2] + 2\fx[n-3] + 2\fx[n-4] + \cdots}
%                                  {$\fy[n-1]+\sfrac{2}{3}\fx[n-1]+\sfrac{1}{3}\fx[n-2]$}
%  \\&=    \fy[n-1] - \sfrac{1}{3}\brp{\fx[n]+ 2\fx[n-1]+\fx[n-2]}
%
%
%
%
%     +    \frac{\fx[n-1]+4\fx[n-2]+\fx[n-3]}{3}   
%     +    \frac{\fx[n-2]+4\fx[n-3]+\fx[n-4]}{3} 
%     +    \cdots  
%  \\&=    \sfrac{1}{2}\fx[n] + \mcom{\fx[n-1] + \fx[n-2] + \fx[n-3] + \fx[n-4] + \fx[n-5] + \cdots}{{$\fy[n-1]+\sfrac{1}{2}\fx[n-1]$}}
%  \\&=    \sfrac{1}{2}\fx[n] + \fy[n-1]+\sfrac{1}{2}\fx[n-1]
%  \\\\
%  \fy[n]-\fy[n-1]&=    \sfrac{1}{2}\brs{\fx[n] + +\fx[n-1]} 
%  \\\\
%  Y(z)\brs{1-z^{-1}} &= \sfrac{1}{2} X(z)\brs{1+z^{-1}}
%  \\\\
%  \frac{Y(z)}{X(z)} 
%    &= \brp{\frac{1}{2}} \frac{1+z^{-1}}{1-z^{-1}}
%     = \boxed{\brp{\frac{1}{2}} \frac{z+1}{z-1}} 
%  \qquad\brb{\begin{array}{MM}
%    How many zeros? & Where?\\
%    How many poles? & Where?
%  \end{array}}
%\end{align*}}
%

%---------------------------------------
%\newpage\mbox{} 
%---------------------------------------
 
\begin{figure}[h]
  \centering
  \begin{tabular}{|c|c|}
    \hline
    \includegraphics{graphics/IntSum.pdf}&\includegraphics{graphics/IntTrap.pdf}
  \\summation integration & trapezoid integration
  \\\hline
  \end{tabular}
  \caption{Comparison of digital integration methods to analytic integration\label{fig:dspint}}
\end{figure}
Is digital summation integration equivalent to continuous integration?
Not really \xref{fig:dspint}.

\begin{align*}
  \abs{\frac{z}{z-1}}_{z=e^{i\omega}}
    &= \abs{\frac{e^{i\omega}}{e^{i\omega}-1}}
  \\&= \abs{\frac{e^{i\omega}}{e^{i\omega/2}\brp{e^{i\omega/2}-e^{-i\omega/2}}}}
   &&= \abs{\mcom{e^{i\omega/2}}{phase}\,\mcom{\frac{1}{2\sin\brp{\frac{\omega}{2}}}}{magnitude}}
  \\&= \boxed{\frac{1}{2\sin\brp{\frac{\omega}{2}}}} 
    && \text{for $0\le\omega\le\pi$}
\end{align*}


Is digital trapezoid integration equivalent to continuous integration? 
Not really \xref{fig:dspint}.
 
\begin{align*}
  \abs{\frac{1}{2}\brp{\frac{z+1}{z-1}}}_{z=e^{i\omega}}
    &= \frac{1}{2}
       \abs{\frac{e^{i\omega}+1}{e^{i\omega}-1}}
  \\&= \frac{1}{2}
       \abs{\frac{e^{i\omega/2}\brp{e^{i\omega/2}+e^{-i\omega/2}}}
                 {e^{i\omega/2}\brp{e^{i\omega/2}-e^{-i\omega/2}}}
           }
   &&= \frac{1}{2}
       \abs{\frac{2\cos\brp{\frac{\omega}{2}}}
                 {2\sin\brp{\frac{\omega}{2}}}
           }
  \\&= \boxed{\frac{1}{2} \abs{\cot\brp{\frac{\omega}{2}}}}
    && \text{for $0\le\omega\le\pi$}
\end{align*}





