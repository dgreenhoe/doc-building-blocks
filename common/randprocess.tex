%============================================================================
% LaTeX File
% Daniel J. Greenhoe
%============================================================================

%======================================
\chapter{Continuous Random Processes}
\label{app:random_processes}
%======================================
\qboxnps
  {Aristotle (384 BC -- 322 BC)
    \index{Aristotle}
    \index{quotes!Aristotle}
    \footnotemark
  }
  {../common/people/aristot.jpg}
  {A likely impossibility is always preferable to an
  unconvincing possibility.}
  \footnotetext{\begin{tabular}[t]{ll}
    quote: & \url{http://en.wikiquote.org/wiki/Aristotle} \\
    image: & \url{http://en.wikipedia.org/wiki/Aristotle}
  \end{tabular}}



%=======================================
%\section{Continuous-time random processes}
%=======================================
%=======================================
\section{Definitions}
%=======================================
%---------------------------------------
\begin{definition}
\index{random variable}
\index{random process}
\citetbl{
  \citerp{papoulis}{63},
  \citerp{papoulis}{285}
  }
%---------------------------------------
Let $\ps$ be a \structe{probability space}.\\
\defboxt{
  The function $\rvx:\pso\to\R$ is a \fnctd{random variable}.\\
  The function $\rvy:\R\times\pso\to\R$ is a \fnctd{random process}.
  }
\end{definition}

The random process $\rvx(t,\omega)$, where $t$ commonly represents time
and $\omega\in\pso$ is an outcome of an experiment,
can take on more specialized forms depending on whether
$t$ and $\omega$ are fixed or allowed to vary.
These forms are illustrated in \prefp{fig:X(t,w)}\footnote{\citerpp{papoulis}{285}{286}}
and \prefp{fig:X(t,w)graph}.

\begin{figure}[ht]\color{figcolor}
\begin{center}
   \begin{tabular}{|c||c|c|}
      \hline
         $\rvx(t,\omega)$ &  fixed $t$      & variable $t$   \\
      \hline
      \hline
         fixed    $\omega$ & number          & time function  \\
      \hline
         variable $\omega$ & random variable & random process \\
      \hline
   \end{tabular}
\caption{
   Specialized forms of a random process $\rvx(t,\omega)$
   \label{fig:X(t,w)}
   }
\end{center}
\end{figure}

\begin{figure}[ht]\color{figcolor}
\begin{center}
\includegraphics[height=8cm,width=12cm]{../common/x_tw.eps}
\end{center}
\caption{
  Example of a random process $\rvx(t,\omega)$
  \label{fig:X(t,w)graph}
}
\end{figure}

%---------------------------------------
\begin{definition}
\footnote{
  \citerpgc{papoulis1984}{216}{0070484686}{$R_{xy}(t_1,t_2)=E\brb{\rvx(t_1)\rvy^\ast(t_2)}$ (9-35)},
  }
\label{def:Rxx}
\label{def:Rxy}
%---------------------------------------
Let $\rvx(t)$ and $\rvy(t)$ be random processes.\\
\defbox{
  \begin{array}{MlMlc>{\ds}l}
       The \fnctd{mean}                      & \pmeanx(t) & of $\rvx(t)$              is & \pmeanx(t)&\eqd& \pE\brs{\rvx(t)}                    %& \text{(\prope{mean functional})} \\
     \\The \fnctd{cross-correlation}         & \Rxy(t)    & of $\rvx(t)$ and $\rvy(t)$is & \Rxy(t,u) &\eqd& \pE\brs{\rvx(t)\rvy^\ast(u)}        %& \text{(\prope{cross-correlation bilinear functional})} \\
     \\The \fnctd{auto-correlation function} & \Rxx(t)    & of $\rvx(t)$              is & \Rxx(t,u) &\eqd& \pE\brs{\rvx(t)\rvx^\ast(u)}        %& \text{(\prope{auto-correlation bilinear functional})} \\
  \end{array}
  }
\end{definition}

%---------------------------------------
\begin{remark}
\footnote{
  \citer{fredholm1900},
  \citerp{fredholm1903}{365},
  \citerp{michel1993}{97},
  \citerp{keener}{101}
  }
%---------------------------------------
The equation $\int_{u\in\R}\Rxx(t,u)\ff(u) \du$ is a
   \ope{Fredholm integral equation of the first kind} and
   $\Rxx(t,u)$ is the \ope{kernel} of the equation.
\end{remark}

%---------------------------------------
%\section{Properties}
%---------------------------------------

%---------------------------------------
\begin{theorem}
\label{thm:Rxx_prop}
\index{cross-correlation}
\index{symmetric!conjugate}
\index{conjuage symmetric}
%---------------------------------------
Let $\rvx(t)$ and $\fy(t)$ be random processes with
cross-correlation $\Rxy(t,u)$ and
let $\Rxx(t,u)$ be the auto-correlation of $\rvx(t)$.
\thmbox{
\begin{array}{rcll}
   \Rxx(t,u) &=& \Rxx^\ast(u,t) & \text{(\prope{conjugate symmetric})}\\
   \Rxy(t,u) &=& \Ryx^\ast(u,t) &
\end{array}
}
\end{theorem}
\begin{proof}
\begin{align*}
   \Rxx(t,u)
      &\eqd \pE\brs{\rvx(t) \rvx^\ast(u)}
      &=    \pE\brs{\rvx^\ast(u) \rvx(t)}
      &=    \brp{ \pE\brs{\rvx(u) \rvx^\ast(t)}}^\ast
      &\eqd \Rxx^\ast(u,t)
\\
   \Rxy(t,u)
      &\eqd \pE\brs{\rvx(t) \rvy^\ast(u)}
      &=    \pE\brs{\rvy^\ast(u) \rvx(t) }
      &=    \brp{\pE\brs{\rvy(u) \rvx^\ast(t)}}^\ast
      &\eqd \Ryx^\ast(u,t)
\end{align*}
\end{proof}

%=======================================
\section{Eigen-analysis of random processes}
\label{sec:KL}
%=======================================
%---------------------------------------
\begin{definition}
\label{def:opR}
%---------------------------------------
Let $\rvx(t)$ be random processes with \fncte{auto-correlation} function \xref{def:Rxx} $\Rxx(t,u)$.
\defboxt{
  The \opd{auto-correlation operator} $\opR$ of $\rvx(t)$ is defined as
  \\\indentx$\opR\ff  \eqd \int_{u\in\R}\Rxx(t,u)\ff(u) \du$
  }
\end{definition}

%---------------------------------------
\begin{definition}
\label{def:white}
%---------------------------------------
Let $\rvx(t)$ be a \fncte{random process} with \fncte{auto-correlation} $\Rxx(\tau)$ \xref{def:Rxx}.
\defboxt{
  A \fncte{random process} $\rvx(t)$ is \propd{white} if $\Rxx(\tau)=\delta(\tau)$
  }
\end{definition}

If a random process $\rvx(t)$ is \prope{white} \xref{def:white}
and the set $\Psi=\setn{\fpsi_1(t),\fpsi_2(t),\ldots,\fpsi_N(t)\}}$
is \textbf{any} set of orthonormal basis functions, then the innerproducts
$\inprod{n(t)}{\fpsi_n(t)}$ and $\inprod{n(t)}{\fpsi_m(t)}$ are \prope{uncorrelated}
for $m\ne  n$.
However, if $\rvx(t)$ is \propb{colored} (not white), then the innerproducts are not
in general uncorrelated.
But if the elements of $\Psi$ are chosen to be the eigenfunctions of $\opR$ such
that $\ds\opR \fpsi_n = \lambda_n \fpsi_n$,
then by \prefpp{thm:Rxx_prop}, the set $\setn{\fpsi_n(t)}$ are \prope{orthogonal} and
the innerproducts \textbf{are} \prope{uncorrelated} even though $\rvx(t)$ is
not white.
This criterion is called the  Karhunen-Lo\`{e}ve criterion for $\rvx(t)$.

%---------------------------------------
\begin{theorem}
\label{thm:opR}
%---------------------------------------
Let $\opR$ be an \ope{auto-correlation} operator.
\thmbox{\begin{array}{lll}
  \inprod{\opR \rvx}{\rvx} \ge 0
    & \forall \rvx\in\spX
    & \text{(\prope{non-negative})}  \\
  \inprod{\opR \rvx}{\fy} = \inprod{\rvx}{\opR \fy}
    & \forall \rvx,\fy\in\spX
    & \text{(\prope{self-adjoint})}
\end{array}}
\end{theorem}
\begin{proof}
\begin{align*}
\intertext{1. Proof that $\opR$ is \prope{non-negative}:}
   \inprod{\opR \fy}{\fy}
     &= \inprod{\int_{u\in\R}\Rxx(t,u) \fy(u) \du}{\fy(t)}
     && \text{by definition of $\opR$}
     && \text{\xref{def:opR}}
   \\&= \inprod{\int_{u\in\R}\pE\brs{\rvx(t)\rvx^\ast(u)} \fy(u) \du}{\fy(t)}
     && \text{by definition of $\Rxx(t,u)$}
     && \text{\xref{def:Rxx}}
   \\&= \pE\brs{\inprod{\int_{u\in\R}\rvx(t)\rvx^\ast(u) \fy(u) \du}{\fy(t)}}
     && \text{by \prope{linearity} of $\inprodn$ and $\int$}
     && \text{\ifsxref{vsinprod}{def:inprod}}
   \\&= \pE\brs{\int_{u\in\R}\rvx^\ast(u) \fy(u) \du \inprod{\rvx(t)}{\fy(t)}}
     && \text{by \prope{additivity} property of $\inprodn$}
     && \text{\ifsxref{vsinprod}{def:inprod}}
   \\&= \pE\brs{\inprod{\fy(u)}{\rvx(u)} \inprod{\rvx(t) }{\fy(t)}}
     && \text{by local definition of $\inprodn$}
     && \text{\ifsxref{vsinprod}{def:inprod}}
   \\&= \pE\brs{\inprod{\rvx(u)}{\fy(u)}^\ast \inprod{\rvx(t) }{\fy(t)}}
     && \text{by \prope{conjugate symmetry} prop.}
     && \text{\ifsxref{vsinprod}{def:inprod}}
   \\&= \pE{\abs{\inprod{\rvx(t) }{\fy(t)}}^2}
     && \text{by definition of $\absn$}
     && \text{\xref{def:abs}}
   \\&\ge 0
\intertext{2. Proof that $\opR$ is self-adjoint:}
   \inprod{\brs{\opR \rvx}(t)}{\fy}
     &= \inprod{\int_{u\in\R}\Rxx(t,u) \rvx(u) \du}{\fy(t)}
     && \text{by definition of $\opR$}
     && \text{\xref{def:opR}}
   \\&= \int_{u\in\R}\rvx(u) \inprod{\Rxx(t,u)  }{\fy(t)} \du
     && \text{by \prope{additive} property of $\inprodn$}
     && \text{\ifsxref{vsinprod}{def:inprod}}
   \\&= \int_{u\in\R}\rvx(u) \inprod{\fy(t)}{\Rxx(t,u)}^\ast \du
     && \text{by \prope{conjugate symmetry} prop.}
     && \text{\ifsxref{vsinprod}{def:inprod}}
   \\&= \inprod{ \rvx(u) }{\inprod{\fy(t)}{\Rxx(t,u)} }
     && \text{by local definition of $\inprodn$}
     && \text{\ifsxref{vsinprod}{def:inprod}}
   \\&= \inprod{ \rvx(u) }{\int_{t\in\R}\fy(t) \Rxx^\ast(t,u)\dt }
     && \text{by local definition of $\inprodn$}
     && \text{\ifsxref{vsinprod}{def:inprod}}
   \\&= \inprod{ \rvx(u) }{\int_{t\in\R}\fy(t) \Rxx(u,t)\dt }
     && \text{by property of $\Rxx$}
     && \text{\xref{thm:Rxx_prop}}
   \\&= \inprod{ \rvx(u) }{\mcom{\opR}{$\opRa$} \fy }
     && \text{by definition of $\opR$}
     && \text{\xref{def:opR}}
   \\\implies&\qquad \opR=\opRa \qquad\implies \text{$\opR$ is \prope{self adjoint}}
\end{align*}
\end{proof}

%---------------------------------------
\begin{theorem}
\footnote{
  \citerpp{keener}{114}{119}
  }
\label{thm:opR_eigen}
%---------------------------------------
Let $\seq{\lambda_n}{n\in\Z}$ be the eigenvalues and
    $\seq{\fpsi_n}{n\in\Z}$ be the eigenfunctions of
    operator $\opR$ such that
    $\opR \fpsi_n = \lambda_n \fpsi_n$.
\thmbox{\begin{array}{rlp{7cm}}
  1. & \lambda_n \in \R
     & (eigenvalues of $\opR$ are \prope{real})
     \\
  2. & \lambda_n\ne \lambda_m \implies \inprod{\fpsi_n}{\fpsi_m}=0
     & (eigenfunctions associated with distinct eigenvalues are \prope{orthogonal})
     \\
  3. & \norm{\fpsi_n(t)}^2>0 \implies \lambda_n\ge0
     & (eigenvalues are \prope{non-negative})
     \\
  4. & \norm{\fpsi_n(t)}^2>0, \inprod{\opR\ff}{\ff} > 0 \implies \lambda_n>0
     & (if $\opR$ is \prope{positive definite}, then eigenvalues are \prope{positive})
\end{array}}
\end{theorem}
\begin{proof}
\begin{enumerate}
\item Proof that eigenvalues are \prope{real-valued}:
Because $\opR$ is self-adjoint, its eigenvalues are real\ifsxref{operator}{thm:self_adjoint}.

\item eigenfunctions associated with distinct eigenvalues are orthogonal:
Because $\opR$ is self-adjoint, this property follows\ifsxref{operator}{thm:self_adjoint}.

\item Proof that eigenvalues are \prope{non-negative}:
\begin{align*}
   0 &\ge \inprod{\opR \fpsi_n}{\fpsi_n}
     &&   \text{by definition of non-negative definite}
   \\&=   \inprod{\lambda_n \fpsi_n}{\fpsi_n}
     &&   \text{by hypothesis}
   \\&=   \lambda_n \inprod{\fpsi_n}{\fpsi_n}
     &&   \text{by definition of inner-products}
   \\&=   \lambda_n \norm{\fpsi_n}^2
     &&   \text{by definition of norm induced by inner-product}
\end{align*}

\item Eigenvalues are \prope{positive} if $\opR$ is \prope{positive definite}:
\begin{align*}
   0 &> \inprod{\opR \fpsi_n}{\fpsi_n}
     && \text{by definition of \prope{positive definite}}
   \\&= \inprod{\lambda_n \fpsi_n}{\fpsi_n}
     && \text{by hypothesis}
   \\&= \lambda_n \inprod{\fpsi_n}{\fpsi_n}
     && \text{by definition of inner-products}
   \\&= \lambda_n \norm{\fpsi_n}^2
     && \text{by definition of norm induced by inner-product}
\end{align*}

\end{enumerate}
\end{proof}

%---------------------------------------
\begin{theorem}[\thmd{Karhunen-Lo/`eve Expansion}]
\footnote{
  \citerpp{keener}{114}{119}
  }
\label{thm:kle}
%---------------------------------------
Let $\opR$ be the \ope{auto-correlation operator} \xref{def:opR}
of a \fncte{random process} $\rvx(t)$.
Let $\seq{\lambda_n}{n\in\Z}$ be the eigenvalues of $\opR$
and $\seq{\fpsi_n}{n\in\Z}$ are the eigenfunctions of $\opR$ such that
    $\opR \fpsi_n = \lambda_n \fpsi_n$.
\thmbox{
  \mcom{\norm{\fpsi_n(t)}=1}{$\setn{\fpsi_n(t)}$ are \prope{normalized}}
  \implies
  \mcom{\pE\brp{\abs{\rvx(t)-\sum_{n\in\Z}\inprod{\rvx(t)}{\fpsi_n(t)} \fpsi_n(t) }^2} = 0}
       {\prope{convergence in probability}}
    \qquad($\setn{\fpsi_n(t)}$ is a \structe{basis} for $\rvx(t)$)
  }
\end{theorem}
\begin{proof}
\begin{enumerate}
  \item Define $\ds\dotx_n\eqd \inprod{\rvx(t)}{\fpsi_n(t)}$ \label{idef:kle_xdot}
  \item Define $\ds\opR\rvx(t) \eqd \int_{u\in\R} \Rxx(t,u)\rvx(u)\du$ \label{idef:kle_R}

  \item lemma: \label{ilem:kle_mercer}
        $\ds\pE\brs{\rvx(t)\rvx(t)} = \sum_{n\in\Z}\lambda_n \abs{\fpsi_n(t)}^2$. Proof:
        \begin{align*}
          \pE\brs{\rvx(t)\rvx(t)}
            &= \sum_{n\in\Z}\lambda_n \abs{\fpsi_n(t)}^2
            && \begin{array}{MMM}
                   by  & \prope{non-negative} property & \xref{thm:opR}
                 \\and & \thme{Mercer's Theorem}       & \xref{thm:mercer}
               \end{array}
        \end{align*}

  \item lemma: \label{ilem:kle_1}
    \begin{align*}
       \pE\brs{\rvx(t) \brp{\sum_{n\in\Z}\dotx_n \fpsi_n(t)}^\ast}
         &\eqd \pE\brs{\rvx(t) \brp{\sum_{n\in\Z}\int_{u\in\R}\rvx(u)\fpsi_n^\ast(u)\du \fpsi_n(t)}^\ast}
         && \text{by definition of $\dotx$}
         && \text{\xref{idef:kle_xdot}}
       \\&= \sum_{n\in\Z}\brp{\int_{u\in\R}\pE\brs{\rvx(t)\rvx^\ast(u)}\fpsi_n(u)\du} \fpsi_n^\ast(t)
         && \text{by \prope{linearity}}
         && \text{\xref{thm:pE_linop}}
       \\&\eqd \sum_{n\in\Z}\brp{\int_{u\in\R}\Rxx(t,u)\fpsi_n(u)\du} \fpsi_n^\ast(t)
         && \text{by definition of $\Rxx(t,u)$}
         && \text{\xref{def:Rxx}}
       \\&\eqd \sum_{n\in\Z}\brp{\opR\fpsi_n(t) \fpsi_n^\ast(t)
         && \text{by definition of $\opR$}
         && \text{\xref{idef:kle_R}}
       \\&= \sum_{n\in\Z}\lambda_n\fpsi_n(t) \fpsi_n^\ast(t)
         && \text{by property of \structe{eigen-system}}
       \\&= \sum_{n\in\Z}\lambda_n \abs{\fpsi_n(t)}^2
    \end{align*}

  \item lemma: \label{ilem:kle_2}
    \begin{align*}
       \pE\brs{\sum_{n\in\Z}\dotx_n \fpsi_n(t)\brp{\sum_{m\in\Z}\dotx_m \fpsi_m(t)}^\ast}
         &\eqd \pE\brs{\sum_{n\in\Z}\int_{u\in\R}\rvx(u)\fpsi_n^\ast(u)\du   \fpsi_n(t)\brp{\sum_{m\in\Z}\int_v \rvx(v)\fpsi_m^\ast(v)\dv \fpsi_m(t)}^\ast}
         && \text{by definition of $\dotx$}
         && \text{\xref{idef:kle_xdot}}
       \\&= \sum_{n\in\Z}\sum_{m\in\Z}\int_u\brp{\int_v \pE\brs{\rvx(u)\rvx^\ast(v)}\fpsi_m(v)\dv} \fpsi_n^\ast(u)\du   \fpsi_n(t)   \fpsi_m^\ast(t)
         && \text{by \prope{linearity}}
         && \text{\xref{thm:pE_linop}}
       \\&= \sum_{n\in\Z}\sum_{m\in\Z}\int_u\brp{\int_v \Rxx(u,v)\fpsi_m(v)\dv} \fpsi_n^\ast(u)\du   \fpsi_n(t)   \fpsi_m^\ast(t)
         && \text{by definition of $\Rxx(t,u)$}
         && \text{\xref{def:Rxx}}
       \\&= \sum_{n\in\Z}\sum_{m\in\Z}\int_u\brp{\opR\fpsi_m(u)} \fpsi_n^\ast(u)\du   \fpsi_n(t)   \fpsi_m^\ast(t)
         && \text{by definition of $\opR$}
         && \text{\xref{idef:kle_R}}
       \\&= \sum_{n\in\Z}\sum_{m\in\Z}\int_u\brp{\lambda_m\fpsi_m(u)} \fpsi_n^\ast(u)\du   \fpsi_n(t)   \fpsi_m^\ast(t)
         && \text{by property of \structe{eigen-system}}
       \\&= \sum_{n\in\Z}\sum_{m\in\Z}\lambda_m \brp{\int_{u\in\R}\fpsi_m(u) \fpsi_n^\ast(u)\du }   \fpsi_n(t)   \fpsi_m^\ast(t)
         && \text{by \prope{linearity}}
       \\&= \sum_{n\in\Z}\sum_{m\in\Z}\lambda_m \norm{\fpsi(t)}^2 \kdelta_{mn}   \fpsi_n(t)   \fpsi_m^\ast(t)
         && \text{by \prope{orthogonal} property} 
         && \text{\xref{thm:opR_eigen}}
       \\&= \sum_{n\in\Z}\sum_{m\in\Z}\lambda_m \kdelta_{mn}   \fpsi_n(t)   \fpsi_m^\ast(t)
         && \text{by \prope{normalized} hypothesis}
       \\&= \sum_{n\in\Z}\lambda_n   \fpsi_n(t)   \fpsi_n^\ast(t)
         && \text{by definition of \fncte{Kronecker delta} $\kdelta$}
         && \text{\xref{def:kdelta}}
       \\&= \sum_{n\in\Z}\lambda_n \abs{\fpsi_n(t)}^2
    \end{align*}

\item Proof that $\setn{\fpsi_n(t)}$ is a \structe{basis} for $\rvx(t)$:
  \begin{align*}
    &\pE\brp{\abs{\rvx(t)-\sum_{n\in\Z}\dotx_n \fpsi_n(t) }^2}
    \\&= \pE\brp{\brs{\rvx(t)-\sum_{n\in\Z}\dotx_n \fpsi_n(t) }\brs{\rvx(t)-\sum_{m\in\Z}\dotx_m \fpsi_m(t) }^\ast}
    \\&= \pE\brp{
         \rvx(t)\rvx^\ast(t)
       - \rvx(t)\brs{\sum_{n\in\Z}\dotx_n \fpsi_n(t)}^\ast
       - \rvx^\ast(t)\sum_{n\in\Z}\dotx_n \fpsi_n(t)
       + \brs{\sum_{n\in\Z}\dotx_n \fpsi_n(t)} \brs{\sum_{m\in\Z}\dotx_m \fpsi_m(t) }^\ast
       }
    \\&= \pE\brp{\rvx(t)\rvx^\ast(t)}
       - \pE\brs{\rvx(t)\brs{\sum_{n\in\Z}\dotx_n \fpsi_n(t)}^\ast}
       - \pE\brs{\rvx^\ast(t)\sum_{n\in\Z}\dotx_n \fpsi_n(t)}
       + \pE\brs{\sum_{n\in\Z}\dotx_n \fpsi_n(t) \brs{\sum_{m\in\Z}\dotx_m \fpsi_m(t) }^\ast }
    \\&\qquad\text{by \prope{linearity} \xref{thm:pE_linop}}
    \\&= \mcom{\sum_{n\in\Z}\lambda_n \abs{\fpsi_n(t)}^2}{by \pref{ilem:kle_mercer}}
       - \mcom{\sum_{n\in\Z}\lambda_n \abs{\fpsi_n(t)}^2}{by \pref{ilem:kle_1}}
       - \mcom{\brs{\sum_{n\in\Z}\lambda_n \abs{\fpsi_n(t)}^2}^\ast}{by \pref{ilem:kle_1}}
       + \mcom{\sum_{n\in\Z}\lambda_n \abs{\fpsi_n(t)}^2}{by \pref{ilem:kle_2}}
    \\&= 0
       - \sum_{n\in\Z}\lambda_n \abs{\fpsi_n(t)}^2
       + \sum_{n\in\Z}\lambda_n \abs{\fpsi_n(t)}^2
    \\&= 0
  \end{align*}
\end{enumerate}
\end{proof}

