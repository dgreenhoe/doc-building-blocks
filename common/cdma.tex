%============================================================================
% NCTU - Hsinchu, Taiwan
% LaTeX File
% Daniel Greenhoe
%============================================================================


%======================================
\chapter{Spread Spectrum}
\label{chp:cdma}
%======================================
%--------------------------------------
\section{Introduction}
\index{Time Division Multiple Access}     
\index{Frequency Division Multiple Access}
\index{Code Division Multiple Access}
\index{TDMA} 
\index{FDMA} 
\index{CDMA} 
\index{DS}
\index{FH}
\index{Direct Sequence}
\index{Frequency Hopping}    
%--------------------------------------
\paragraph{Communication channel multiple access.}
A communication system provides the ability for 
a set of information to be 
sent from a transmitter to a receiver through a physical channel.
If multiple sets of information need to be sent through the channel,
then this channel must be shared.
Multiple access of a channel can be achieved by separating the information sets in
time, frequency, or code.  
These three multiple access techniques are referred to as

\begin{tabular}{clll}
   $\bullet$ & TDMA & Time Division Multiple Access:      & separation in time \\
   $\bullet$ & FDMA & Frequency Division Multiple Access: & separation in frequency \\
   $\bullet$ & CDMA & Code Division Multiple Access:      & separation by code 
\end{tabular}

\paragraph{CDMA Modulation}
Communication through a channel is typically performed by transmitted information 
{\em modulating} (affecting some parameter of) a {\em carrier} waveform.
There are two basic types of CDMA modulation:

\begin{tabular}{clll}
   $\bullet$ & DS & Direct Sequence   &  \\
   $\bullet$ & FH & Frequency Hopping &  
\end{tabular}

In FH-CDMA modulation, an information sequence modulates the frequency 
of a sinusoidal carrier waveform.  
FH-CDMA will not be further discussed in this chapter.

\index{pn-sequence}
\index{pseudo-noise sequence}
In DS-CDMA modulation, an information sequence modulates a 
{\em pseudo-noise sequence} (pn-sequence).
This pn-sequence and the information which modulates it
are typically both binary sequences.
The modulation operation itself is a simple {\em modulo 2 addition} operation in mathematics,
which is equivalent to an {\em exclusive OR} operation in logic,
which may be implemented with an {\em exclusive OR gate} in hardware.

\paragraph{Types of PN-Sequences}
Generating good PN-sequences is one of the keys to effective DS-CDMA 
communication system design.  
A sequence is simply a function $f$ whose domain is the set of integers
and range is some set $R$.
This report is limited to {\em binary} pn-sequences,
which are functions with range $\{0,1\}$ of the form
   \[ f:\Z\to\{0,1\}.  \]

\index{m-sequence} \index{Gold sequence}
The most basic binary pn-sequence is the {\em m-sequence} 
(maximal length sequence).
From this basic sequence, other sequences can be constructed 
such as {\em Gold} sequences.

%--------------------------------------
\section{Generating m-sequences mathematically}
\label{sec:math}
%--------------------------------------
%--------------------------------------
\subsection{Definitions}
\label{sec:def}
%--------------------------------------
An m-sequence can be represented as the coefficients of a {\em polynomial} 
over a {\em finite field}.
Any {\em field} is defined by the triplet $(S,+,\cdot)$,
where 

\begin{tabular}{clrlll}
   $S$:     & a set \\
   $+$:     & addition   operation in the form       &$+:$     &$S\times S\to S$ \\
   $\cdot$: & multiplication   operation in the form &$\cdot:$ &$S\times S\to S $
\end{tabular}


%--------------------------------------
\begin{definition} {\bf Galois Field 2, GF(2)} \\
\label{def:gf2}
\index{Galois field}
\index{GF(2)}
%--------------------------------------
GF(2) is the field $(S,+,\cdot)$ with members of the triplet defined as

\begin{tabular}{ccccc}
\begin{math}
\begin{array}{ccl}
   S     &=& \{0,1\}                        \\
   +     &:& \{0,1\}\times\{0,1\}\to\{0,1\} \\
   \cdot &:& \{0,1\}\times\{0,1\}\to\{0,1\} 
\end{array}
\end{math}
&
 such that 
&
\begin{math}
\begin{array}{cc|c}
   a & b & a+b \\
   \hline 
   0 & 0 & 0 \\
   0 & 1 & 1 \\
   1 & 0 & 1 \\
   1 & 1 & 0 \\
\end{array}
\end{math}
&and&
\begin{math}
\begin{array}{cc|c}
   a & b & a\cdot b \\
   \hline 
   0 & 0 & 0 \\
   0 & 1 & 0 \\
   1 & 0 & 0 \\
   1 & 1 & 1 \\
\end{array}
\end{math}
\end{tabular}

\end{definition}

\index{GF(2)!polynomials over}
M-sequences can be generated and represented as 
{\em polynomials over GF(2)}.
A polynomial over GF(2) is a polynomial with coefficients selected from GF(2).
An example of a polynomial over GF(2) is 
   \[ 1 + x^2 + x^5 + x^6 + x^7 + x^9. \]

The generation of an m-sequence is equivalent to polynomial division,
which is very similar to integer division.

%--------------------------------------
\begin{definition} {\bf Polynomial division} \\
%--------------------------------------
The quantities of polynomial division are identified as follows:

\begin{center}
\begin{tabular}{ccc}
$\ds \frac{d(x)}{p(x)} = q(x) + \frac{r(x)}{p(x)}$
&where&
\begin{tabular}{llll}
   d(x) & is the dividend \\
   p(x) & is the divisor  \\
   q(x) & is the quotient \\
   r(x) & is the remainder.
\end{tabular}
\end{tabular}
\end{center}
\end{definition}

The ring of integers $\Z$ contains some special elements called {\em primes}
which can only be divided\footnote{
   The expression ``$a$ divides $b$" means that $b/a$ has remainder 0.
}
 by themselves or 1.
Rings of polynomials have a similar elements called {\em primitive polynomials}.

%--------------------------------------
\begin{definition} {\bf Primitive polynomial} \\
\index{primitive polynomial}
%--------------------------------------
A primitive polynomial $p(x)$ of order $n$ has the properties
\begin{enumerate}
\setlength{\itemsep}{0ex}
   \item $p(x)$ cannot be factored
   \item the smallest order polynomial that $p(x)$ can divide is $x^{2^n-1}+1=0$.
\end{enumerate}
\end{definition}

Some examples\citepp{wicker}{465}{475} of primitive polynomials over $GF(2)$ are

\begin{center}
\begin{tabular}{cl}
   order & primitive polynomial \\
   \hline
    2 & $p(x) = x^2 + x + 1 $ \\
    3 & $p(x) = x^3 + x + 1 $ \\
    4 & $p(x) = x^4 + x + 1 $ \\
    5 & $p(x) = x^5 + x^2 + 1 $ \\
    5 & $p(x) = x^5 + x^4 + x^2 + x + 1 $ \\
   16 & $p(x) = x^{16} + x^{15} + x^{13} + x^4 + 1 $\\
   31 & $p(x) = x^{31} + x^{28} + 1 $
\end{tabular}
\end{center}

An m-sequence is the remainder when dividing any non-zero polynomial by a primitive 
polynomial.  
We can define an {\em equivalence relation}\footnote{
   An equivalence relation $\equiv$ must satisfy three properties:

\begin{tabular}{cll}
   1. & reflexivity:   & $a\equiv a$ \\
   2. & symmetry:      & if $a\equiv b$ then $b\equiv a$. \\
   3. & transitivity:  & if $a\equiv b$ and  $b\equiv c$ then $a\equiv c$. 
\end{tabular}

reference: \cite[p.7]{ab}
}
on polynomials which defines two polynomials as
{\em equivalent with respect to $p(x)$}
when their remainders are equal.

%--------------------------------------
\begin{definition} {\bf Equivalence relation $\equiv$} \\
\label{def:equiv}
\index{equivalence relation}
%--------------------------------------

\begin{center}
Let  \hspace{5mm}
   $\ds \frac{a_1(x)}{p(x)} = q_1(x) + \frac{r_1(x)}{p(x)}$
\hspace{5mm} and \hspace{5mm}
   $\ds \frac{a_2(x)}{p(x)} = q_2(x) + \frac{r_2(x)}{p(x)}$.
\end{center}

Then $a_1(x)\equiv a_2(x)$ with respect to $p(x)$ if $r_1(x)=r_2(x)$. 
\end{definition}

Using the equivalence relation of Definition \ref{def:equiv},
we can develop two very useful equivalent representations of
polynomials over GF(2).  
We will call these two representations the {\em exponential} representation
and the {\em polynomial} representation.

%--------------------------------------
\begin{example}
\label{ex:representations}
%--------------------------------------
By Definition \ref{def:equiv} and 
under $p(x)=x^3+x+1$, we have the following equivalent representations:

\begin{center}
\begin{math}
\begin{array}{lcrclclclllll}
   \frac{x^0}{x^3+x+1} &=& 0     &+& \frac{1  }{x^3+x+1}     &\implies& x^0 &\equiv& 1   \\
   \frac{x^1}{x^3+x+1} &=& 0     &+& \frac{x  }{x^3+x+1}     &\implies& x^1 &\equiv& x   \\
   \frac{x^2}{x^3+x+1} &=& 0     &+& \frac{x^2}{x^3+x+1}     &\implies& x^2 &\equiv& x^2 \\
   \frac{x^3}{x^3+x+1} &=& 1     &+& \frac{x+1}{x^3+x+1}     &\implies& x^3 &\equiv& x+1 \\
   \frac{x^4}{x^3+x+1} &=& x     &+& \frac{x^2+x}{x^3+x+1}   &\implies& x^4 &\equiv& x^2+x \\
   \frac{x^5}{x^3+x+1} &=& x^2+1 &+& \frac{x^2+x+1}{x^3+x+1} &\implies& x^5 &\equiv& x^2+x+1 \\
   \frac{x^6}{x^3+x+1} &=& x^3+x+1 &+& \frac{x^2+1}{x^3+x+1} &\implies& x^6 &\equiv& x^2+1 \\
   \frac{x^7}{x^3+x+1} &=& x^4+x^2+x+1 &+& \frac{1}{x^3+x+1} &\implies& x^7 &\equiv& 1 
\end{array}
\end{math}
\end{center}

\index{cycle}
Notice that $x^7\equiv x^0$, and so a cycle is formed with 
$2^3-1=7$ elements in the cycle.
The monomials to the left of the $\equiv$ are the {\em exponential} 
representation and the polynomials to the right are the {\em polynomial}
representation.
Additionally, the polynomial representation may be put in a vector form giving a 
{\em vector} representation.
The vectors may be interpreted as a binary number and represented as a decimal numeral.

\begin{center}
\begin{math}
\begin{array}{cccc}
   \mbox{exponential} & \mbox{polynomial} & vector & decimal \\
   x^0 & 1       &[001] & 1  \\
   x^1 & x       &[010] & 2  \\
   x^2 & x^2     &[100] & 4  \\
   x^3 & x+1     &[011] & 3  \\
   x^4 & x^2+x   &[110] & 6  \\
   x^5 & x^2+x+1 &[111] & 7  \\
   x^6 & x^2+1   &[101] & 5  
\end{array}
\end{math}
\end{center}

\end{example}


%--------------------------------------
\subsection{Generating m-sequences using polynomial division}
%--------------------------------------
An m-sequence is generated by dividing any non-zero polynomial of order less than $m$
by a primitive polynomial of order $m$.  
The m-sequence is the coefficients of the resulting polynomial.
M-sequences will repeat every $2^m-1$ values.
This is the maximum sequence length possible when the 
sequence is generated by division in polynomials over GF(2).

%--------------------------------------
\begin{example}
\label{ex:1/p(x)}
%--------------------------------------
We can generate an m-sequence of length 
$2^3-1=7$ by dividing 1 by the primitive polynomial $x^3+x+1$.

\begin{fsL}
\[
\begin{array}{rrll}
           &   & x^{-3} + x^{-5} + x^{-6} +& x^{-7} + x^{-10} + x^{-12} + x^{-13} + x^{-14} + x^{-17} + \cdots \\
   \cline{3-4}
   x^3+x+1 & \vline & 1 \\
           &        & 1 + x^{-2} + x^{-3} \\
   \cline{3-3}
           &        & x^{-2} + x^{-3} \\
           &        & x^{-2} + x^{-4} + x^{-5} \\
   \cline{3-3}
           &        & x^{-3} + x^{-4} + x^{-5} \\
           &        & x^{-3} + x^{-5} + x^{-6} \\
   \cline{3-3}
           &        & x^{-4} + x^{-6}          \\
           &        & x^{-4} + x^{-6} + x^{-7} \\
   \cline{3-3}
           &        & x^{-7}                   \\
           &        & x^{-7} + x^{-9} + x^{-10} \\
   \cline{3-3}
           &        & x^{-9} + x^{-10}         \\
           &        & x^{-9} + x^{-11} + x^{-12} \\
   \cline{3-3}
           &        & x^{-10} + x^{-11} + x^{-12} \\
           &        & x^{-10} + x^{-12} + x^{-13} \\
   \cline{3-3}
           &        & x^{-11} + x^{-13}          \\
           &        & x^{-11} + x^{-13} + x^{-14} \\
   \cline{3-3}
           &        & x^{-14}           \\
           &        & \vdots           
\end{array}
\]
\end{fsL}
The coefficients, starting with the $x^{-1}$ term, 
of the resulting polynomial form the m-sequence
\[ 0010111 \; 0010111 \; \cdots \]
which repeats every $2^3-1=7$ elements.
\end{example}


Note that the division operation in Example \ref{ex:1/p(x)}
can be performed using vector notation rather than polynomial notation.

%--------------------------------------
\begin{example}
%--------------------------------------
Generate an m-sequence of length $2^3-1=7$ by dividing 1 by the primitive polynomial
$x^3+x+1$ using vector notation.

\begin{fsL}
\[
\begin{array}{rrllllllllllllllllllllllllllllllllllll}
           &   &   &.&0&0&1&0&1&1&1 &0&0&1&0&1&1&1 &0&\cdots \\
   \cline{3-20}
   1011    &\vline   & 1 &.&0&0&0&0&0&0&0 &0&0&0&0&0&0&0 &0 \\
           &   & 1 & &0&1&1 \\
   \cline{3-7}
           &   & 0 & &0&1&1&0 \\
           &   &   & &0&0&0&0 \\
   \cline{5-8}
           &   &   & &0&1&1&0&0 \\
           &   &   & & &1&0&1&1 \\
   \cline{6-9}
           &   &   & & &0&1&1&1&0 \\
           &   &   & & & &1&0&1&1 \\
   \cline{7-10}
           &   &   & & & &0&1&0&1&0 \\
           &   &   & & & & &1&0&1&1 \\
   \cline{8-11}
           &   &   & & & & &0&0&0&1&0 \\
           &   &   & & & & & &0&0&0&0 \\
   \cline{9-12}
           &   &   & & & & & &0&0&1&0&0 \\
           &   &   & & & & & & &0&0&0&0 \\
   \cline{10-13}
           &   &   & & & & & & &0&1&0&0&0 \\
           &   &   & & & & & & & &1&0&1&1 \\
   \cline{11-14}
           &   &   & & & & & & & &0&0&1&1&0 \\
           &   &   & & & & & & & & &0&0&0&0 \\
   \cline{12-15}
           &   &   & & & & & & & & &0&1&1&0&0 \\
           &   &   & & & & & & & & & &1&0&1&1 \\
   \cline{13-16}
           &   &   & & & & & & & & & &0&1&1&1&0 \\
           &   &   & & & & & & & & & & &0&0&0&0 \\
           &   &   & & & & & & & & & & & &\vdots& &  \\
\end{array}
\]
\end{fsL}

The coefficients, starting to the right of the binary point, 
is again the sequence
\[ 0010111 \; 0010111 \; \cdots. \]
\end{example}


%--------------------------------------
\subsection{Multiplication modulo a primitive polynomial}
\label{sec:math-xmod}
%--------------------------------------
If $p(x)$ is a primitive polynomial,
by Definition \ref{def:equiv}
the product of two polynomials is equivalent (with respect to $p(x)$)
of the product {\em modulo} $p(x)$.
The ability to multiplying two polynomials modulo a primitive polynomial
is very useful for manipulating m-sequences.

In general, the product of two polynomials can be evaluated as follows.
Let 
\begin{eqnarray*}
   a(x) &\eqd& a_mx^m + a_{m-1}x^{m-1} + \cdots + a_2x^2 + a_1x + a_0 \\
   b(x) &\eqd& b_mx^m + b_{m-1}x^{m-1} + \cdots + b_2x^2 + b_1x + b_0
\end{eqnarray*}

Then 
\begin{scriptsize}
\begin{eqnarray*}
   a(x)b(x)
        &=& \left( a_mx^m + a_{m-1}x^{m-1} + \cdots + a_2x^2 + a_1x + a_0 \right)
            \left( b_mx^m + b_{m-1}x^{m-1} + \cdots + b_2x^2 + b_1x + b_0 \right)\\
        &=& a_0b_0 + (a_0b_1+a_1b_0)x + (a_0b_2+a_1b_1+a_2b_0)x^2 + \cdots + a_mb_mx^{2m}\\
        &=& \left( \sum\limits_{i=0}^{m-1}x^i \sum\limits_{j=0}^i a_jb_{i-j} \right) +
            \left( \sum\limits_{i=m}^{2m}x^i \sum\limits_{j=0}^{2m-i} a_{i-m+j}b_{m-j} \right)
\end{eqnarray*}
\end{scriptsize}

The product modulo $p(x)$ is obtained when the terms involving 
$x^m$, $x^{m+1}$, \ldots, $x^{2m}$ are replaced by their 
equivalent polynomial representations
(see Section \ref{sec:def}).

%--------------------------------------
\begin{example}
\label{ex:math-xmod}
%--------------------------------------
Suppose we want to find $(a_2x^2 + a_1x + a_0)(b_2x^2+b_1x+b_0)$ modulo $x^3+x+1$.
\begin{scriptsize}
\begin{eqnarray*}
   a(x)b(x) &=& (a_2x^2+a_1x+a_0)(b_2x^2+b_1x+b_0)
      \\&=& a_0b_0 + (a_0b_1 + a_1b_0)x + (a_0b_2 + a_1b_1 + a_2b_0)x^2 + 
            (a_1b_2 + a_2b_1)x^3 + a_2b_2x^4
      \\&=& a_0b_0 + (a_0b_1 + a_1b_0)x + (a_0b_2 + a_1b_1 + a_2b_0)x^2 + 
            (a_1b_2 + a_2b_1)(x+1) + a_2b_2(x^2+x)
      \\&=& (a_0b_0 + a_1b_2 + a_2b_1) + 
            (a_0b_1 + a_1b_0 + a_1b_2 + a_2b_1 + a_2b_2)x + 
            (a_0b_2 + a_1b_1 + a_2b_0 + a_2b_2)x^2 
\end{eqnarray*}
\end{scriptsize}
Notice that if the $a_i$ and $b_i$ coefficients are known, 
the resulting product has only three terms.
\end{example}


%--------------------------------------
\section{Generating m-sequences in hardware}
\label{sec:hw}
%--------------------------------------
Section \ref{sec:math} has already demonstrated how to generate m-sequences
mathematically.
If we further know how to implement each of those mathematical operations 
efficiently in hardware, we are done.
That is what this section is about.

%--------------------------------------
\subsection{Field operations}
%--------------------------------------
The mapping tables for GF(2) addition and multiplication
given in Definition \ref{def:gf2} (page \pageref{def:gf2})
are exactly the same as those for the hardware {\em exclusive OR (XOR)} gate
and the {\em AND} gate, respectively.

%--------------------------------------
\subsection{Polynomial multiplication and division using DF1}
\index{direct form 1}
%--------------------------------------
Suppose we want to construct a circuit to compute the 
rational expression $f(x)\frac{b(x)}{a(x)}$.
This is a common problem in {\em Digital Signal Processing (DSP)};
we can borrow results from there.
DSP is generally concerned with polynomials over the field of real or complex numbers.
However, a field is a field, and all fields (whether, real, complex, or GF(2))
support both addition and multiplication;\footnote{
   {\bf Fields:} Roughly speaking, a {\em group} is a set together with an operation 
   on that set.  
   An {\em additive group} is a set $S$ 
   with an addition operation $+:S\times S\to S$.
   A {\em multiplicative group} is a set $S$ 
   with a multiplication operation $\cdot:S\times S\to S$.
   A {\em field} is constructed using two groups: 
   An addition group and a multiplication group.
   See Appendix \ref{app:inprod} page \pageref{app:inprod}.\\
   Reference: \cite[p.123]{durbin}.
   }
the rules change somewhat, but the basic structure is the same regardless.
Alternatively, just as a typical digital filter operates over the real 
or complex field, 
{\bf the m-sequence generator} described in this section
{\bf is a digital filter which operates over the field GF(2)}.

A sequential hardware multiplier-divider for polynomials is simple.
\begin{liste}
   \item Each $x$ in $f(x)$, $b(x)$, and $a(x)$ represents a delay of one clock cycle.
In DSP terminology, a delay of one clock cycle is represented by $z^{-1}$.
Thus, $x=z^{-1}$.
   \item Let $f(x)=f_0 + f_1x + f_2x^2 + \cdots$.\\
         Then let $\dot{f}(n)$ be the sequence $\dot{f}(i)=f_i$, with $i\in\Z$.
   \item Let $b(x)= b_0 + b_1x + b_2x^2 + \cdots + b_mx^m$.\\
         Let $\dot{b}(n)$ be the sequence $\dot{b}(i)=b_i$, with $i\in\Z$.
   \item Let $a(x)=1 + a_1x + a_2x^2 + \cdots + a_mx^m$.\\
         Let $\dot{a}(n)$ be the sequence $\dot{a}(i)=a_i$, with $i\in\Z$.
\end{liste}
Then the multiplier-divider (for any mathematical field) can be implemented as shown
in Figure \ref{fig:df1}.  
This structure is called the {\em Direct Form I} implementation\citep{os}{344};
it implements the rational expression
\[
   f(x) \frac{b_mx^m + b_{m-1}x^{m-1} + \cdots + b_2x^2 + b_1x + b_0}
             {a_mx^m + a_{m-1}x^{m-1} + \cdots + a_2x^2 + a_1x + 1  }
\]

\begin{figure}[ht]
\center{\epsfig{file=df1.eps, height=6cm, clip=}}
%\center{\includegraphics[0,0][4in,4in]{df1.eps}}
\caption{
   Direct Form I Implementation for $\ds f(x)\frac{b(x)}{a(x)}$
   \label{fig:df1}
   }
\end{figure}

In GF(2), the blocks in the figure can be implemented very simply:
\begin{liste}
   \item Each $x=z^{-1}$ element can be implemented as a simple D flip-flop.
   \item An $a_i=1$ or $b_i=1$ coefficient is implemented as a wire (closed circuit).
   \item An $a_i=0$ or $b_i=0$ coefficient is implemented as a no-connect (open circuit).
\end{liste}

%--------------------------------------
\begin{example}
\label{ex:df1}
%--------------------------------------
Suppose we want to build a hardware circuit to 
generate an m-sequence specified by the rational expression
\[ \frac{x^2+x}{x^3+x+1}. \]
To do this we can set $f(x)=1$, $b(x)=x^2+x$ and $a(x)=x^3+x+1$.
The resulting structure is shown in Figure \ref{fig:ex-df1}.
\begin{figure}[ht]
\center{\epsfig{file=ex-df1.eps, height=6cm, clip=}}
\caption{
   Direct Form I Implementation for Example \ref{ex:df1}
   \label{fig:ex-df1}
   }
\end{figure}
Notice that the two flip-flops on the left are for initialization 
only and are not used in the steady state operation of the m-sequence
generator.
In fact, they can be eliminated altogether by proper initialization of 
the flip-flops on the right.
\end{example}

%--------------------------------------
\subsection{Polynomial multiplication and division using DF2}
\label{sec:df2}
\index{direct form 2}
%--------------------------------------
The Direct Form I structure shown in Figure \ref{fig:df1} can be transformed
to a new structure by transformation rules based on 
{\em Mason's Gain Formula}.\footnote{
   The transformation rules are as follows:
   \begin{enumerate}
   \setlength{\itemsep}{0ex}
   \item Reverse the direction of all signal paths.
   \item Replace all nodes with addition operators.
   \item Replace all addition operators with nodes.
   \end{enumerate}
   \cite[p.363]{os}
   }
The resulting structure is known as Direct Form II\citep{os}{347} 
and is illustrated in 
Figure \ref{fig:df2}.
\begin{figure}[ht]
\center{\epsfig{file=df2.eps, height=6cm, clip=}}
\caption{
   Direct Form II Implementation
   \label{fig:df2}
   }
\end{figure}
Again, when using the DF2 structure for m-sequence generation,
the $\dot{f}(n)$ sequence can be eliminated by proper initialization of
the delay elements (flip flops).

%--------------------------------------
\subsection{Hardware polynomial modulo multiplier}
\label{sec:hw-xmod}
%--------------------------------------
The mathematics of polynomial multiplication modulo a primitive polynomial 
was already presented in Section \ref{sec:math-xmod}
and demonstrated in Example \ref{ex:math-xmod} (page \pageref{ex:math-xmod}).
It is straight forward to implement these equations in hardware:
\begin{liste}
   \item every $a_ib_j$ bitwise multiply operation is implemented with an AND gate
   \item every $+$ between consecutive $a_ib_j$ terms is implemented with an XOR gate
\end{liste}
Note that {\bf the hardware modulo multiplier can be implemented 
using only combinatorial logic}(!);
No sequential circuitry (such as flip-flops) are needed.



