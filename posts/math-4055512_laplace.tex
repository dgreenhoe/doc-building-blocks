Daniel J. Greenhoe
https://math.stackexchange.com/questions/4055512/


$\newcommand{\eqd}{\triangleq}$
$\newcommand{\eqa}{\approx}$
$\newcommand{\abs}[1]{{\left\lvert #1 \right\rvert}}$
$\newcommand{\brp}[1]{{\left(#1\right)}}$
$\newcommand{\brs}[1]{{\left[#1\right]}}$
$\newcommand{\brbl}[1]{{\left\{#1\right.}}$
$\newcommand{\brlr}[1]{\left.#1\right|}$
$\newcommand{\deriv} [2]   {{\frac{\mathrm{d}#1}{\mathrm{d}#2} }}$
$\newcommand{\ddx}  [0]  {\frac{\mathrm{d}}{\dx}}$
$\newcommand{\dndxn}  [0]  {\frac{\mathrm{d^n}}{{\dx}^n}}$
$\newcommand{\ddt}{\frac{\mathrm{d}}{dt}}$
$\newcommand{\dddt}{\frac{\mathrm{d^2}}{dt^2}}$
$\newcommand{\R}{\Bbb{R}}$
$\newcommand{\intcc} [2]  {{\left[#1:#2\right]}}$
$\newcommand{\intoo} [2]  {{\left(#1:#2\right)}}$
$\newcommand{\intoc} [2]  {{\left(#1:#2\right]}}$
$\newcommand{\intco} [2]  {{\left[#1:#2\right)}}$
$\newcommand{\opLT}{\mathrm{L}}$
$\newcommand{\opFT}{\mathrm{F}}$
$\newcommand{\ff}{\mathrm{f}}$
$\newcommand{\fF}{\mathrm{F}}$
$\newcommand{\fG}{\mathrm{G}}$
$\newcommand{\fg}{\mathrm{g}}$
$\newcommand{\fphi}{\mathrm{\phi}}$
$\newcommand{\dx}{\mathrm{dx}}$
$\newcommand{\dS}{\mathrm{ds}}$
$\newcommand{\dtau}{\mathrm{d\tau}}$
$\newcommand{\du}{\mathrm{du}}$
$\newcommand{\dv}{\mathrm{dv}}$
$\newcommand{\step}{\mu}$
$\newcommand{\ds}{\displaystyle}$
$\newcommand{\thme}[1]{#1}$
$\newcommand{\opFT}{\mathrm{F}}$
$\newcommand{\opFT}{\mathrm{L}}$
$\newcommand{\opIFT}{\mathrm{IFT}}$
$\newcommand{\thme}[1]{#1}$
$\newcommand{\Real}{\mathrm{Re}}$
$\newcommand{\mcom}[2]{{\displaystyle\underbrace{\displaystyle#1}_{\text{#2}}}}$


Transforms are used in analysis (analysis: ?v???s??, [meaning](https://www.amazon.com/dp/0943575338) �dissolution�, or separation into components) to separate functions into components. 
https://www.etymonline.com/word/analysis
Before Fourier, this separation was performed using the differential operator $\ddx$ and the result was the Taylor Expansion or Taylor Polynomial (or we could even call it the Taylor Transform). This worked well for analytic functions---e.g., as long as the function was continuous. But then [along came Fourier](https://course.ece.cmu.edu/~ece792/handouts/Robinson82.pdf) and showed that there was another way: instead of analyzing (decomposing) a function $\ff(x)$ using rates of change calculated by the derivative operator $\ddx$, analysis could instead be performed by projection using the integration operator $\int\ddx$.

The Fourier Transform projects onto the sinusoidal basis $e^{-i\omega x}=cos(\omega x) - i\sin(\omega x)$. This is generalized by Laplace by projecting onto $e^{-sx}$. Just as transforms break functions into components, transforms can themselves be broken into components. One way to do this with the Laplace Transform $\opL$ is to simply let $s\eqd\alpha + i\omega$, then
\begin{align*}
  \opLT[\ff(x)] 
    &\eqd \int_\R \ff(x) e^{-sx}\dx 
  \\&\eqd \int_\R \ff(x) e^{-(\alpha+i\omega)x}\dx 
  \\&= \mcom{\int_\R \ff(x) e^{-\alpha x}\dx}{decay transform} 
      +\mcom{\int_\R \ff(x) e^{-i\omega x}\dx}{Fourier Transform}
\end{align*}
...and the Laplace Transform $\opLT$ is decomposed into a Fourier Transform and a kind of "decay transform".

The Fourier Transform demonstrates the amount of harmonic "energy" at a given location ("frequency") $\omega$. Here the "decay transform" demonstrates rate of decay of a function. 
For example, let 
$$\begin{align*}
  \ff(x) 
    &\eqd \brbl{\begin{array}{ll}
            1 & \text{for $0\leq x<1$}\\
            0 & \text{otherwise}
          \end{array}}
    &&\text{and}&
  \fg(x) 
    &\eqd \brbl{\begin{array}{ll}
            1 & \text{for $0\leq x<2$}\\
            0 & \text{otherwise}
          \end{array}}
\end{align*}$$
Note that $\ff(x)$ has faster decay than does $\fg(x)$. 
Now take the Laplace Transform of both:
\begin{align*}
  \opLT\brs{\ff(x)}(s) 
    &\eqd \int_\R\ff(x) e^{-sx}\dx
  \\&\eqd \int_\R\brs{\step(x)-\step(x-1)} e^{-sx}\dx
  \\&= \frac{1}{s} - e^{-s}\frac{1}{s}
  \\&= \frac{1-e^{-s}}{s}
  \\
  \opLT\brs{\fg(x)}(s) 
    &\eqd \int_\R\brs{\step(x)-\step(x-2)} e^{-sx}\dx
  \\&= \frac{1-e^{-2s}}{s}
\end{align*}


I think this question and the struggle to answer it is similar to the question of what imaginary numbers are (see Euler quote below). In the case of imaginary numbers�they are useful because the roots of polynomials, even those with real coefficients, don't always have solutions in the set of real numbers...but *do* always have solutions in the set of complex numbers. In fact, those roots are always either *on* the real axis, or occur in complex conjugate pairs mirrored across the real axis.

But first, why use Laplace at all? A big motivation is that real-world physical systems involving things like electric fields (e.g. capacitors), magnetic fields (e.g. inductors), 4-dimensional state information (e.g. position, velocity, acceleration) are described in terms of first and second order differential equations (e.g. $i(t)=C\deriv{v}{t}$). In fact, in the physical world, it's just very hard to get away from the differential operators $\ddt$ and $\dddt$ because the *only* solutions (the only possible $\ff(t)$) for the second order differential equation $\dddt\ff+\ff=0$ **[must](https://books.google.com/books?vid=ISBN0486650383&pg=PA157)** be of the form $\ff(t)=\alpha\cos(t) + \beta\sin(t)$; at the same time $\sin(t)$ and $\cos(t)$ are hard to get away from because they [are eigenvectors](https://books.google.com/books?id=Nfk59L8L5hcC&pg=PA34) for any linear time-invariant (LTI) system. And it turns out that Laplace allows handling differential equations with the quasi-greatest of ease---turning differential equations into algebraic ones (e.g. differentiation becomes multiplication by $s$ and integration division by $s$).

So there you are, working away solving complex system differential equations transformed to algebraic ones with help from Laplace, just as complex geometric shapes involving arcs and curves and what-not can be [transformed to polynomials in the Cartesian plane](https://wild.maths.org/ren%C3%A9-descartes-and-fly-ceiling). All your Laplace equations are easy to write down---they even all have real coefficients. But here it comes again---the problem of roots---and once again they are not in general all real-valued. But because $s$ is complex-valued, you are still OK with that.

But why care about roots at all? We care because special things happen when we get anywhere near a root. If the root is in the numerator (a "zero"), the system responds with a decrease in magnitude (decrease in voltage, current, velocity, what-have-you); if the root is in the denominator (a "pole"), the system responds with an increase.

These pole/zero locations have direct impact on the both time and frequency response of a system. The frequency response is calculated as you move up and down on the imaginary axis (because $s=i\omega$ to get a Fourier Transform). Even if there is no pole or zero directly *on* this imaginary axis, when the value of $\omega$ is such that $i\omega$ is close to a pole, the frequency response magnitude (at that value of $\omega$) will tend to increase; and when $i\omega$ is close to a zero, it tends to decrease. The closer the pole or zero is to the imaginary axis, the more dramatic the effect.

Likewise, a zero close to the real axis in the complex $s$-plane will have a *damping* effect, and a pole an excitation effect. Note that the Laplace Transform of $e^{-\alpha t}\mu(t)$, where $\mu(t)$ is the step function and a real-valued $\alpha$ is a damping parameter, [is](https://archive.org/details/in.ernet.dli.2015.141269/page/n31/) $\frac{1}{s+\alpha}$. For real-valued $\alpha$, this means that as $\alpha$ gets larger, the location of the pole moves farther and farther away on the negative real axis...making it's influence less and less...which corresponds to less and less damping from the $e^{-\alpha t}$ term.

