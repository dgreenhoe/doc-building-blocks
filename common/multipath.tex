%============================================================================
% LaTeX File
% Daniel Greenhoe
%============================================================================

%======================================
\chapter{Multipath fading Channel}
\label{chp:mpath}
\index{multipath fading channel}
\index{scintillation}
\index{multipath}
\index{fading}
%======================================
%======================================
\section{Channel model}
\index{Nyquist signaling rate}
%======================================
\begin{figure}[ht]
\begin{center}
\begin{fsK}
\setlength{\unitlength}{0.2mm}                  
\begin{picture}(700,200)(0,-100)
  \thinlines                                      
  %\graphpaper[10](0,0)(500,150)

  \put(   0 ,   0 ){\vector(1,0){100} }
  \put(   0 ,   0 ){\makebox(100,  50)[b]{$a_n$} }
  \put( 100 ,  80 ){\makebox(0,0)[lb]{signal rate $\ds \frac{1}{T}\le 2W$ (Nyquist signaling rate)} }
  \put( 150 ,  80 ){\vector(0,-1){30} }
  \put( 100 , -50 ){\framebox(100,100){transmitter}}
  \put( 100 , -40 ){\makebox( 100, 80)[b]{$\opT$} }
  \put( 200 ,   0 ){\vector(1,0){100} }
  \put( 200 ,   0 ){\makebox(100,  50)[b]{$s(t)$} }
  \put( 300 , -50 ){\framebox(100,100){}}
  \put( 300 , -40 ){\makebox( 100, 80)[b]{$\opC$} }
  \put( 400 ,   0 ){\vector(1,0){100} }
  \put( 400 ,   0 ){\makebox(100,  50)[b]{$r(t)$} }
  \put( 500 , -50 ){\framebox(100,100){receiver}}
  \put( 500 , -40 ){\makebox( 100, 80)[b]{$\opR$} }
  \put( 600 ,   0 ){\vector(1,0){100} }
  \put( 600 ,   0 ){\makebox(100,  50)[b]{$r(nT_2)$} }

  %\put( 200 ,-60 ){\makebox(0, 0)[lt]{$\ds s(t)=\sum_n a_n \lambda(t-nT)$} }
  \put( 400 ,-60 ){\makebox(0, 0)[lt]{$\ds r_r(t)=\sum_n a_n(t) s[t-\tau_n(t)]$} }

  \put( 295 ,-30 ){
  \setlength{\unitlength}{0.03mm} 
  \begin{picture}(700,400)(-100,-200) 
  \thinlines                                      
  
  \put(-100,   0){\line    (  1,  0)   {100}                  }

  \put(   0,-150){\line    (  0,  1)   {300}                  }
  \put(   0,  50){\vector  (  1,  0)   {100}                  }
  \put(   0, 150){\vector  (  1,  0)   {100}                  }
  \put(   0,-150){\vector  (  1,  0)   {100}                  }

  \put( 100, 150){\makebox(0,0)[lc] {$\Box$}                  }
  \put( 100,  50){\makebox(0,0)[lc] {$\Box$}                  }
  \put( 100,- 25){\makebox(0,0)[lc] {$\;\cdot$}              }
  \put( 100,- 50){\makebox(0,0)[lc] {$\;\cdot$}              }
  \put( 100,- 75){\makebox(0,0)[lc] {$\;\cdot$}              }
  \put( 100,-150){\makebox(0,0)[lc] {$\Box$}                  }

  %\put( 150, 150){\makebox(0,0)[lc] {$\rightarrow$}                  }
  %\put( 150,  50){\makebox(0,0)[lc] {$\rightarrow$}                  }
  %\put( 150,-150){\makebox(0,0)[lc] {$\rightarrow$}                  }

  \put( 200, 150){\makebox(0,0)[lc] {$\triangleright$}                  }
  \put( 200,  50){\makebox(0,0)[lc] {$\triangleright$}                  }
  \put( 200,-150){\makebox(0,0)[lc] {$\triangleright$}                  }

  \put( 250, 150){\makebox(0,0)[lt] {$\searrow$}                  }
  \put( 250,  50){\makebox(0,0)[lc] {$\rightarrow$}                  }
  \put( 250,-150){\makebox(0,0)[lb] {$\nearrow$}                  }

  \put( 345,   0){\makebox (  0,  0)[lc]   {$\oplus$}                  }
  \put( 407,  -3){\makebox (  0,  0)[lc]   {$\rightarrow$}                  }
  \end{picture}                                   
  }
\end{picture}                                   
\end{fsK}
\end{center}
\caption{
   Multipath system model
   \label{fig:mpath_model}
   }
\end{figure}


\begin{figure}[ht]
\begin{center}
\begin{fsK}
\setlength{\unitlength}{0.2mm}                  
\begin{picture}(700,400)(-100,-200) 
  \thinlines                                      
  %\graphpaper[10](0,0)(700,100)                  

  \put(-100,   0){\line    (  1,  0)   {100}                  }
  \put(-100,   0){\vector  (  1,  0)   { 40}                  }
  \put(-100,  10){\makebox ( 80, 50)[b]{$s(t)$}               }

  \put(   0,-150){\line    (  0,  1)   {300}                  }
  \put(   0,  50){\vector  (  1,  0)   {100}                  }
  \put(   0, 150){\vector  (  1,  0)   {100}                  }
  \put(   0,-150){\vector  (  1,  0)   {100}                  }
  \put(   0, 160){\makebox (100, 50)[b]{path 1}               }
  \put(   0,  60){\makebox (100, 50)[b]{path 2}               }
  \put(   0,-140){\makebox (100, 50)[b]{path $N$}             }

  \put( 100, 125){\framebox(100, 50)   {delay $\tau_1(t)$}       }
  \put( 100,  25){\framebox(100, 50)   {delay $\tau_2(t)$}       }
  \put( 100,- 75){\makebox (100, 50)   {$\vdots$}             }
  \put( 100,-175){\framebox(100, 50)   {delay $\tau_N(t)$}       }
  \put( 200,  50){\line    (  1,  0)   { 50}                  }
  \put( 200, 150){\line    (  1,  0)   { 50}                  }
  \put( 200,-150){\line    (  1,  0)   { 50}                  }

  \put( 250, 125){\makebox ( 40, 50)   {$\alpha_1(t)$}           }
  \put( 250, 125){\line    (  0,  1)   { 50}                  }
  \put( 250, 125){\line    (  2,  1)   { 50}                  }
  \put( 250, 175){\line    (  2, -1)   { 50}                  }

  \put( 250,  25){\makebox ( 40, 50)   {$\alpha_2(t)$}           }
  \put( 250,  25){\line    (  0,  1)   { 50}                  }
  \put( 250,  25){\line    (  2,  1)   { 50}                  }
  \put( 250,  75){\line    (  2, -1)   { 50}                  }

  \put( 250,-175){\makebox ( 40, 50)   {$\alpha_N(t)$}           }
  \put( 250,-175){\line    (  0,  1)   { 50}                  }
  \put( 250,-175){\line    (  2,  1)   { 50}                  }
  \put( 250,-125){\line    (  2, -1)   { 50}                  }

  \put( 300,  50){\line    (  1,  0)   { 50}                  }
  \put( 300, 150){\line    (  1,  0)   { 50}                  }
  \put( 300,-150){\line    (  1,  0)   { 50}                  }

  \put( 350, 150){\vector  (  1, -3)   { 46.8}                  }
  \put( 350,  50){\vector  (  1, -1)   { 43}                  }
  \put( 350,-150){\vector  (  1,  3)   { 47}                  }

  \put( 400,   0){\circle              { 20}                  }
  \put( 400,   0){\makebox (  0,  0)   {$+$}                  }
  \put( 410,   0){\vector  (  1,  0)   { 90}                  }
  \put( 440,  10){\makebox ( 90,  0)[bl]{$\ds \fr_r(t)=\sum_n \alpha_n(t)s[t-\tau_n(t)]$}}
 %\put( 440, -10){\makebox ( 90,  0)[tl]{$\ds \fr(t)=\int_y \alpha(t;y)s[t-\tau(t;y)]\dy$}}

  \put(- 20,-200){\dashbox{5}(450,400)   {}                     }
\end{picture}                                   
\end{fsK}
\end{center}
\caption{
   Multipath system model
   \label{fig:mp_model}
   }
\end{figure}

\paragraph{Sources of interference.}
In the multipath-fading channel, there are two sources of interference:
\hie{multipath} and \hie{fading}.
These are briefly described next and illustrated in 
\prefpp{fig:mp_model}.
\begin{liste}
  \item \hib{multipath}: 
  Multipath is a process caused by multiple signal paths in a channel.
  Each path $n$ is characterized by a scaling coefficient $\alpha_n$
  and a delay $\tau_n$.  \\
  These weighted delays create a filter with some frequency response 
  at time $t$. \\
  The stochastic bandwidth of this filter is the 
  \hie{coherence bandwidth} $(\delf)_c$. \\
  We would like the bandwidth $W$ of the transmitted signal $\fs(t)$ 
  to fit comfortably within the coherence bandwidth such that 
  $W\ll(\delf)_c$. 
  In this case we say that the channel is \hie{frequency non-selective.}

\item \hib{fading}:
  Fading is a process caused by the values of the scaling coefficients
  and delays changing with time $t$.
  When the path $n$ scaling coefficient $\alpha_n$ tends to zero,
  the signal traversing that path is attenuated and we say that it 
  ``fades".
  A measure of how fast paths change is the \hie{coherence time}
  $(\delt)_c$.
  We would like the paths to remain stable for at least as long as 
  a symbol period $T$ such that $T\ll(\delt)_c$.
  In this case we say that the channel is \hie{slowly fading}.
\end{liste}
  
\paragraph{Channel operator space.}
Many communication systems can be modeled as illustrated in 
\prefpp{fig:mp_model}.
The system may be \hie{discrete} (finite $N$) 
or \hie{continuous} (infinite $N$);
The system response may characterized by its
\hie{real-time response} or by its \hie{instantaneous response}.
These four possibilities are given in the following table:

\begin{tabular}{| l || l | l |}
  \hline
  $\fr(t)$      & discrete & continuous \\
  \hline
  \hline
  real-time     & $\ds \fr_{r}(t)=\sum_n \alpha_n(t)s[t-\tau_n(t)]$
                & $ \fr_{rc}(t)=\int_y \alpha(t;y)s[t-\tau(t;y)]\dy$ \\
  \hline
  instantaneous & $\ds \fr(\tau;t)=\sum_n \alpha_n(t)s[\tau-\tau_n(t)]$     
                & $ \fr_c(\tau;t)=\int_y \alpha(t;y)s[\tau-\tau(t;y)]\dy$ \\
  \hline
\end{tabular}

In the instantaneous response, the values of the system parameters
$\alpha_n(t)$ and $\tau_n(t)$ are ``frozen" at time instant $t$,
the system response is then given as a function of $\tau$.
In this chapter, analysis will be performed using 
the discrete instantaneous response.

%\begin{defx}
%this is my definition.
%\end{defx}
%--------------------------------------
\begin{definition}
\label{def:mpath}
\index{operator!channel}
\index{real-time             }
\index{response-time         }
\index{reflection coefficient}
\index{path delay            }
%--------------------------------------
Let channel operator $\opC:\{s:\R\to\R\}\to\{r:\R\to\R\}$ 
be such that 
  \[ [\opC s](\tau;t) = \sum_n \alpha_n(t) s[\tau-\tau_n(t)] \]
and under the constraints
\begin{enume}
   \item $\alpha_n(t)$ is zero mean
   \item $\alpha_n(t)$ and $\alpha_m(t)$ are uncorrelated for $n\not=m$.
   \item $\tau_n(t)$ and $\tau_m(t)$ are uncorrelated for $n\not=m$.
   \item $\alpha(t)$ and $\tau(t)$ are uncorrelated.
   \item the impulse response of $\opC$  is WSS with respect to real-time $t$.
   \item $\tau(t)$ are continuous with respect to real-time $t$.
\end{enume}
Let $\fh:\R^2\to\R$ be the impulse response of $\opC$ such that 
  \[ \fh(\tau;t) = [\opC\delta](\tau;t)
                 = \sum_n \alpha_n(t) \delta[\tau-\tau_n(t)]. \]
The following terms apply to the listed quantities:

\begin{tabular}[b]{ll}
   $t$:        & real-time               \\
   $\tau$:     & response-time           \\
   $\alpha_n$: & reflection coefficient  \\
   $\tau_n$:   & path delay 
\end{tabular}
\end{definition}

Justification in real-world environments 
for the constraints of \prefpp{def:mpath} is as follows:
\begin{enume}
  \item This is just for mathematical convenience.
        We make the DC value equal to ``0".
  \item The amount of energy reflected from two different surfaces
        ($\alpha_n$ and $\alpha_m$) are uncorrelated.
  \item The length of two signal paths ($\tau_n$ and $\tau_m$) 
        are uncorrelated.
  \item The amount of energy reflected from a surface ($\alpha(t)$)
        and the length of the signal path ($\tau(t)$) are 
        uncorrelated.
  \item The statistical properties of the channel do not change with time.
  \item The continuity constraint is especially important in the real-time
case when $s(t)$ is a very short pulse, or even an impulse $\delta(t)$.
For example, in the impulse case, $\delta[t-\tau(t)]$ is only non-zero
when $t=\tau(t)$. But if $\tau(t)$ is not continuous, it may never equal $t$
and the impulse is completely lost even when $\alpha(t)\ne0$. 
Having the continuity constraint helps fix the problem.
  \end{enume}

%======================================
\section{Receiver statistics}
%======================================
%---------------------------------------
\begin{proposition}
%---------------------------------------
\[\Eb{\fr(\tau;t)} = 0 \]
\end{proposition}
\begin{proof}
\begin{eqnarray*}
   \Eb{\fr(\tau;t)} 
     &=& \Eb{\sum_n \alpha_n(t)s[\tau-\tau_n(t)]}
      =  \sum_n \Eb{\alpha_n(t)}s[\tau-\tau_n(t)]
      =   \sum_n 0\cdot\Eb{s[\tau-\tau_n(t)]}
      =  0.
\end{eqnarray*}
\end{proof}

%---------------------------------------
\begin{proposition}
\label{prop:mp_tau}
%---------------------------------------
Operation $\opC$ is uncorrelated with respect to $\tau$
($\opC$ is white with respect to $\tau$).
\end{proposition}
\begin{proof}
By \prefpp{def:mpath}, 
$\tau_n(t)$ and $\tau_m(t)$ are uncorrelated for $m\not=n$.
Different values of $\tau$ correspond to different 
path delays $\tau_n(t),\tau_m(t)$.
Thus $\opC$ is uncorrelated with respect to $\tau$.
\end{proof}

Suppose 
$\Rhh'(\tau_1,\tau_2;t_1,t_2)\eqd\Eb{\fh(\tau_1;t_1)\fh(\tau_2;t_2)}$
is the autocorrelation function of the impulse response $\fh(\tau;t)$.
We already have two key characteristics of $\fh(\tau;t)$:
\begin{enume}
  \item $\fh(\tau;t)$ is uncorrelated with respect to $\tau$
        (by \prefp{prop:mp_tau}).\\
        So we only care about the case $\tau=\tau_1=\tau_2$.
  \item $\fh(\tau;t)$ is WSS with respect to $t$
        (by \prefpp{def:mpath}). \\
        So we only care about the case $\delt=t_1-t_2$.
\end{enume}
Because of these two characteristics, the autocorrelation function
can be simplified to 
\[ \Rhh(\tau;\delt)=\Rhh(\tau;t_1-t_2)=\Rhh'(\tau_1,\tau_2;t_1,t_2).\]

%---------------------------------------
\begin{definition}
\label{def:mp_Rhh}
%---------------------------------------
Let $\Rhh:\R^2\to\R$ be the \hib{autocorrelation} function
of impulse response $\fh:\R^2\to\R$ such that
  \[ \Rhh(\tau;\delt)\eqd\Eb{\fh(\tau;t+\delt)\fh^\ast(\tau;t)}. \]
\end{definition}

%======================================
\section{Multipath measurement functions}
%======================================
The Fourier transform can operate over $\Rhh(\tau;\delt)$
with respect to $\tau$, $\delt$, or both to generate three
new functions $\mpathRft(f)$, $\mpathRtf(f)$, and $\mpathRff(f)$.
This provides a total of four equivalent functions for 
measuring multipath.
These four functions are formally defined in 
\prefpp{def:mp_cf} and illustrated in \prefpp{fig:mp_cf}.

%--------------------------------------
\begin{definition}
\label{def:mp_cf}
\index{measurement functions}
\index{time correlation                      }
\index{spaced-frequency spaced-time function }
\index{scattering function                   }
\index{Doppler function                      }
%--------------------------------------
Let $\Rhh:\R^2\to\R$, $\mpathRft:\R^2\to\R$, $\mpathRtf:\R^2\to\R$, and 
$\mpathRff:\R^2\to\R$ be defined as\\
\defboxp{\begin{tabular}[b]{lp{7cm}lcl}
   1. & autocorrelation function
      & $\Rtt(\tau;\delt)$ &$\eqd$& $\Eb{\fh(\tau;t+\delt)\fh^\ast(\tau;t)}$
\\
   2. & spaced-frequency spaced-time function
      & $\mpathRft(\delf;\delt)$ &$\eqd$& $\opFT_\tau \Rtt(\tau;\delt)$
\\
   3. & scattering function
      & $\mpathRtf(\tau;\lambda)$ &$\eqd$& $\opFT_{\delt} \Rtt(\tau;\delt)$
\\
   4. & Doppler function
      & $\mpathRff(\delf;\lambda)$ &$\eqd$& $\opFT_\tau\opFT_{\delt} \Rtt(\tau;\delt)$
\end{tabular}

The arguments of these functions are designated as\\
\begin{tabular}[b]{ll}
   $\tau$    & delay                  \\
   $\delf$   & frequency difference   \\
   $\delt$   & time difference        \\
   $\lambda$ & Doppler frequency.
\end{tabular}}
\end{definition}


\begin{figure}[ht]
\begin{center}
\begin{fsK}
\setlength{\unitlength}{0.2mm}                  
\begin{picture}(400,400)(-200,-200) 
  \thinlines                                      
  %\graphpaper[10](0,0)(700,100)                  

  \put(- 50, 100){\framebox(100,100)   {}                     }
  \put(- 50, 110){\makebox (100, 80)[t]{$\Rtt(\tau;\delt)$}   }
  \put(- 50, 110){\makebox (100, 80)[c]{autocorrelation}     }
  \put(- 50, 110){\makebox (100, 80)[b]{function}     }

  \put(- 50,-200){\framebox(100,100)   {}                     }
  \put(- 50,-190){\makebox (100, 80)[t]{$\mpathRff(\delf;\lambda)$}   }
  \put(- 50,-190){\makebox (100, 80)[c]{doppler}     }
  \put(- 50,-190){\makebox (100, 80)[b]{function}     }

  \put(-200,- 50){\framebox(100,100)   {}                     }
  \put(-200,- 40){\makebox (100, 80)[t]{$\mpathRft(\delf;\delt)$}   }
  \put(-200,- 40){\makebox (100, 80)[c]{spaced-frequency}}
  \put(-200,- 40){\makebox (100, 80)[b]{spaced-time}}

  \put( 100,- 50){\framebox(100,100)   {}                     }
  \put( 100,- 40){\makebox (100, 80)[t]{$\mpathRtf(\tau;\lambda)$}   }
  \put( 100,- 40){\makebox (100, 80)[c]{scattering}}
  \put( 100,- 40){\makebox (100, 80)[b]{function}}

  \put(- 60, 150){\vector  ( -1, -1)   {90}                  }
  \put(  60, 150){\vector  (  1, -1)   {90}                  }
  \put( 150,- 60){\vector  ( -1, -1)   {90}                  }
  \put(-150,- 60){\vector  (  1, -1)   {90}                  }

  \put(-110, 110){\makebox (  0,  0)[br]{$\opFT$}}
  \put( 110, 110){\makebox (  0,  0)[bl]{$\opFT$}}
  \put(-110,-110){\makebox (  0,  0)[tr]{$\opFT$}}
  \put( 110,-110){\makebox (  0,  0)[tl]{$\opFT$}}

\end{picture}                                   
\end{fsK}
\end{center}
\caption{
   Multipath measurement functions
   \label{fig:mp_cf}
   }
\end{figure}

The Fourier transform of a random process (in time) is also 
a random process (in ``frequency").
The Fourier transform of the random process $\fh(\tau;t)$
with respect to $\tau$ is therefore a random process and has 
an autocorrelation function.
This autocorrelation function is equivalent to 
the spaced-frequency-spaced-time function $\mpathRft(\delf;\delt)$
as shown next.
%---------------------------------------
\begin{proposition}
%---------------------------------------
Let $\Fh:\R^2\to\C$ be the Fourier transform of $\fh:\R^2\to\R$ 
such that
  \[ \Fh(f;t) \eqd [\opFT \fh(\tau;t)](f;t) 
              \eqd \int_\tau \fh(\tau;t) e^{-i2\pi f\tau}\dtau.
  \]
Then
  \[ \Eb{\Fh(f_1;t+\delt)\Fh^\ast(f_2;t)} = \mpathRft(\delf;\delt). \]
\end{proposition}
\begin{proof}
\if 0
\begin{eqnarray*}
   \mpathRft(\delf;\delt)
     &=& \opFT_\tau \Rtt(\tau;\delt)
   \\&=& \int_{\tau}\Rtt(\tau;\delt) e^{-i2\pi \delf \tau} \dtau
   \\&=& \int_{\tau}\Eb{\fh(\tau;t+\delt) \fh^\ast(\tau;t) }
              e^{-i2\pi (f_1-f_2)\tau}\dtau
   \\&=& \int_{\tau_1}\int_{\tau_2}\Eb{
             \fh(\tau_1;t+\delt) \fh^\ast(\tau_2;t) }
              e^{-i2\pi f_1\tau_1}e^{i2\pi f_2\tau_2} 
            \dtau_2 \dtau_1
   \\&=& \Eb{\int_{\tau_1}\int_{\tau_2}
             \fh     (\tau_1;t+\delt) e^{-i2\pi f_1\tau_1}
             \fh^\ast(\tau_2;t      ) e^{i2\pi f_2\tau_2} 
            \dtau_2 \dtau_1}
   \\&=& \Eb{      \int_{\tau_1}\fh(\tau_1;t+\delt) e^{-i2\pi f_1\tau_1} \dtau_1
             \left(\int_{\tau_2}\fh(\tau_2;t      ) e^{-i2\pi f_2\tau_2} \;d\tau_2\right)^\ast
            }
   \\&=& \Eb{\Fh(f_1;t+\delt)\Fh^\ast(f_2;t)}
\end{eqnarray*}
\fi
\begin{eqnarray*}
   \Eb{\Fh(f_1;t+\delt)\Fh^\ast(f_2;t)}
     &=& \Eb{      \int_{\tau_1}\fh(\tau_1;t+\delt) e^{-i2\pi f_1\tau_1} \dtau_1
             \left(\int_{\tau_2}\fh(\tau_2;t      ) e^{-i2\pi f_2\tau_2} \;d\tau_2\right)^\ast
            }
   \\&=& \Eb{\int_{\tau_1}\int_{\tau_2}
             \fh     (\tau_1;t+\delt) e^{-i2\pi f_1\tau_1}
             \fh^\ast(\tau_2;t      ) e^{i2\pi f_2\tau_2} 
            \dtau_2 \dtau_1}
   \\&=& \int_{\tau_1}\int_{\tau_2}\Eb{
             \fh(\tau_1;t+\delt) \fh^\ast(\tau_2;t) }
              e^{-i2\pi f_1\tau_1}e^{i2\pi f_2\tau_2} 
            \dtau_2 \dtau_1
   \\&=& \int_{\tau}\Eb{
             \fh(\tau;t+\delt) \fh^\ast(\tau;t) }
              e^{-i2\pi (f_1-f_2)\tau}
            \dtau
   \\&=& \int_{\tau}\Rtt(\tau;\delt)
              e^{-i2\pi \delf \tau}
            \dtau
   \\&=& \opFT_\tau \Rtt(\tau;\delt)
   \\&=& \mpathRft(\delf;\delt)
\end{eqnarray*}
%\\ \\
The following proof fails (diverges).
However I still include it here anyway.
Maybe someone can show me what I did wrong:
\begin{align*}
  \Eb{ \Fh(\tau;\lambda_1) \Fh^\ast(\tau;\lambda_2) }
    &= \Eb{ \Fh(\tau;\lambda_1) \Fh^\ast(\tau;\lambda_2) }
  \\&= \Eb{   \int_t \fh(\tau;t) e^{-i2\pi\lambda_1 t} \dt
       \left( \int_u \fh(\tau;u) e^{-i2\pi\lambda_2 u} \du \right)^\ast
       }
  \\&= \Eb{ \int_t \fh     (\tau;t) e^{-i2\pi\lambda_1 t} \dt
            \int_u \fh^\ast(\tau;u) e^{ i2\pi\lambda_2 u} \du 
       }
  \\&= \int_t \int_u  \Eb{ \fh(\tau;t) \fh^\ast(\tau;u) }
            e^{-i2\pi\lambda_1t} e^{ i2\pi\lambda_2u} \du \dt 
  \\&= \int_t \int_u  \Eb{ \fh(\tau;u+\delt) \fh^\ast(\tau;u) }
            e^{-i2\pi\lambda_1(u+\delt)} e^{ i2\pi\lambda_2u} \du \dt 
       \hspace{1cm}\delt=t-u \iff t=u+\delt
  \\&= \int_u \int_{\delt}  \Rhh(\tau;\delt)
            e^{-i2\pi\lambda_1(u+\delt)} e^{ i2\pi\lambda_2u} \ddelt \du 
  \\&= \int_u e^{-i2\pi(\lambda_1-\lambda_2)u} \du 
       \int_{\delt}  \Rhh(\tau;\delt) e^{-i2\pi\lambda_1\delt}  \ddelt 
  \\&= \delta(\lambda_1-\lambda_2)\mpathRtf(\tau;\lambda_1)
\end{align*}
\end{proof}
%======================================
\section{Profile functions}
%======================================
Setting one of the two inputs in each 
measurement function of \prefpp{def:mp_cf} to zero
generates four new ``profile" functions.
The width of these four profile functions are four critical parameters.
The four profile functions and four critical parameters are
defined in \prefpp{def:mp_cp} and 
illustrated in \prefpp{fig:mp_pf}.


\begin{figure}[ht]
\begin{center}
\begin{fsL}
\setlength{\unitlength}{0.2mm}
\begin{tabular}{c|c}
\hib{Multipath} measures 
&
\hib{Fading} measures
\\
\hline
\\
multipath intensity profile  & spaced-time correlation function
\\
\begin{picture}(230,135)(-100,-35)
  %\graphpaper[10](0,0)(200,200)                  
  \thicklines                                      
  \put(-100 ,   0 ){\line    (   1,   0){200} }
  \put(   0 ,- 10 ){\line    (   0,   1){110} }
  \put( 110 ,   0 ){\makebox (   0,   0)[l]{$\tau$} }
  \put(  40 ,  40 ){\makebox (   0,   0)[bl]{$\Rtt(\tau;0)$} }
  \thinlines
  \qbezier(  0, 80)( 40, 20)( 80,  0)
 %\qbezier(  0, 80)(-40, 20)(-80,  0)

  %\put(- 80 ,-  5 ){\line    (  0,- 1)   { 30} }
  %\put(  80 ,-  5 ){\line    (  0,- 1)   { 30} }
  \put(  80 ,- 15 ){\makebox (  0,  0)[t]{$T_m$} }

 %\put(- 20 ,- 20 ){\vector  ( -1,  0)   { 60} }
 %\put(  20 ,- 20 ){\vector  (  1,  0)   { 60} }
\end{picture}
&
\begin{picture}(230,135)(-100,-35)
  %\graphpaper[10](0,0)(200,200)                  
  \thicklines                                      
  \put(-100 ,   0 ){\line    (   1,   0){200} }
  \put(   0 ,- 10 ){\line    (   0,   1){110} }
  \put( 110 ,   0 ){\makebox (   0,   0)[l]{$\delt$} }
  \put(  40 ,  40 ){\makebox (   0,   0)[bl]{$\Rtt(0;\delt)$} }
  \thinlines
  \qbezier(  0, 80)( 40, 20)( 80,  0)
  \qbezier(  0, 80)(-40, 20)(-80,  0)

  \put(- 80 ,-  5 ){\line    (  0,- 1)   { 30} }
  \put(  80 ,-  5 ){\line    (  0,- 1)   { 30} }
  \put(-100 ,- 15 ){\makebox (200,  0)[t]{$(\delt)_c$} }

  \put(- 20 ,- 20 ){\vector  ( -1,  0)   { 60} }
  \put(  20 ,- 20 ){\vector  (  1,  0)   { 60} }
\end{picture}
\\
multipath spread &  coherence time 
\\
$\Bigg\downarrow\opFT$      & $\Bigg\downarrow\opFT$  
\\
spaced-frequency correlation function   & Doppler power spectrum
\\
\begin{picture}(230,135)(-100,-35)
  %\graphpaper[10](0,0)(200,200)                  
  \thicklines                                      
  \put(-100 ,   0 ){\line    (   1,   0){200} }
  \put(   0 ,- 10 ){\line    (   0,   1){110} }
  \put( 110 ,   0 ){\makebox (   0,   0)[l]{$\delf$} }
  \put(  40 ,  40 ){\makebox (   0,   0)[bl]{$\mpathRft(\delf;0)$} }
  \thinlines
  \qbezier( -32,  48)(   0, 144)(  32,  48)
  \qbezier(- 80,   0)( -48,   0)( -32,  48)
  \qbezier(  32,  48)(  48,   0)(  80,   0)

  \put(- 80 ,-  5 ){\line    (  0,- 1)   { 30} }
  \put(  80 ,-  5 ){\line    (  0,- 1)   { 30} }
  \put(-100 ,- 15 ){\makebox (200,  0)[t]{$(\delf)_c$} }

  \put(- 30 ,- 20 ){\vector  ( -1,  0)   { 50} }
  \put(  30 ,- 20 ){\vector  (  1,  0)   { 50} }
\end{picture}
&
\begin{picture}(230,135)(-100,-35)
  %\graphpaper[10](0,0)(200,200)                  
  \thicklines                                      
  \put(-100 ,   0 ){\line    (   1,   0){200} }
  \put(   0 ,- 10 ){\line    (   0,   1){110} }
  \put( 110 ,   0 ){\makebox (   0,   0)[l]{$\lambda$} }
  \put(  40 ,  40 ){\makebox (   0,   0)[bl]{$\mpathRtf(0;\lambda)$} }
  \thinlines
  \qbezier( -32,  48)(   0, 144)(  32,  48)
  \qbezier(- 80,   0)( -48,   0)( -32,  48)
  \qbezier(  32,  48)(  48,   0)(  80,   0)

  \put(- 80 ,-  5 ){\line    (  0,- 1)   { 30} }
  \put(  80 ,-  5 ){\line    (  0,- 1)   { 30} }
  \put(-100 ,- 15 ){\makebox (200,  0)[t]{$B_d$} }

  \put(- 20 ,- 20 ){\vector  ( -1,  0)   { 60} }
  \put(  20 ,- 20 ){\vector  (  1,  0)   { 60} }
\end{picture}
\\
coherence bandwidth                  & Doppler spread  \\
$\ds (\delf)_c \eqa \frac{1}{T_m}$   & $\ds (\delt)_c \eqa \frac{1}{B_d}$
                 
\end{tabular}                             
\end{fsL}
\end{center}
\caption{
   Profile functions with critical parameters
   \label{fig:mp_pf}
   }
\end{figure}

%--------------------------------------
\begin{definition}
\label{def:mp_pf}
\index{multipath intensity profile          }
\index{spaced-time correlation function     }
\index{Doppler power spectrum               }
\index{spaced-frequency correlation function}
\label{def:mp_cp}
   \index{multipath spread   }
   \index{coherence time     }
   \index{Doppler spread     }
   \index{coherence bandwidth}
%--------------------------------------
The following four \hib{profile functions} are defined as \\
\defboxp{\begin{tabular}[b]{lp{7cm}lcl}
   1. & multipath intensity profile 
      & $\Rtt(\tau;0)$ 
\\
   2. & spaced-time correlation function
      & $\mpathRft(0;\delt)$ 
\\
   3. & Doppler power spectrum
      & $\mpathRtf(0;\lambda)$ 
\\
   4. & spaced-frequency correlation function
      & $\mpathRft(\delf;0)$
\end{tabular}}

The following four \hib{critical parameters} are defined as\\
\defboxp{\begin{tabular}[b]{llll}
   1. & multipath spread
      & $T_m$
      & is the width of $\Rtt(\tau;0)$
\\
   2. & coherence time
      & $(\delt)_c$
      & is the width of $\mpathRft(0;\delt)$ 
\\
   3. & Doppler spread
      & $B_d$
      & is the width of $\mpathRtf(0;\lambda)$ 
\\
   4. & coherence bandwidth
      & $(\delf)_c$ 
      & is the width of $\mpathRft(\delf;0)$
\end{tabular}}
\end{definition}





%======================================
\subsection*{Multipath intensity profile $\Rtt(\tau;0)$}
%======================================
\paragraph{Power.}
The \hie{multipath intensity profile} $\Rtt(\tau;0)$ 
is a measure of the power (the ``intensity") 
of a signal as a function of the 
path delay $\tau$ (each path has a delay $\tau$). 
This is demonstrated by
\begin{eqnarray*}
   \Rtt(\tau;0) 
     &\eqd& \Eb{\fh(\tau;t+0)\fh^\ast(\tau;t)}
   \\&=&    \E{|\fh(\tau;t)|^2}
   \\&=&    \E{|\fh(\tau;0)|^2}
            \hspace{1cm}\mbox{(because $\fh(\tau;t)$ is WSS with respect to $t$).}
\end{eqnarray*}

\paragraph{Path correlation.}
As a signal traverses two paths where one is longer and longer paths relative to the other, 
the resulting two signals are less and less correlated.
If they are delayed by more than the \hie{multipath spread} $T_m$,
then they are uncorrelated.

%======================================
\subsection*{Spaced-time correlation profile $\mpathRft(0;\delt)$}
%======================================
The \hie{spaced-time correlation profile} $\mpathRft(0;\delt)$
measures the time auto-correlation of a signal traveling through a 
single path.
A signal is uncorrelated with a delayed version of itself if the delay 
is greater than the \hie{coherence time} $(\delt)_c$.

%======================================
\subsection*{Doppler power spectrum $\mpathRtf(0;\lambda)$}
%======================================
The \hie{Doppler power spectrum} $\mpathRtf(\tau;0)$ 
is a measure of signal power density as a function of $\lambda$.


%======================================
\subsection*{Spaced-frequency correlation function $\mpathRft(\delf;0)$}
%======================================
The \hie{spaced-frequency correlation function} $\mpathRft(\delf;0)$
measures the correlation of two sinusoids. 
If two sinusoids are separted in fequency by more than the 
\hie{coherence bandwidth} $(\delf)_c$, then they are uncorrelated.


%======================================
\section{Channel classification}
%======================================

%--------------------------------------
\begin{definition}
\index{frequency non-selective channel }
\index{frequency selective channel     }
\index{slowly   fading channel         }
\index{fast fading channel             }
\index{underspread channel             }
\index{overspread channel              }
\index{Shannon signalling rate}
%--------------------------------------
For a signal $s(t)$ in a multipath channel let
\begin{liste}
   \item $T$ be the signalling period
   \item $W$ be the bandwidth.
\end{liste}

Then $s(t)$ is 

\defboxp{\begin{tabular}[b]{lllllc}
   frequency non-selective channel & if & $W\ll (\delf)_c$ & or & $W\gg T_m$    &$\star$  \\
   frequency selective channel     & if & $W\gg (\delf)_c$ & or & $W\ll T_m$     \\
\\
   slowly   fading channel         & if & $T\ll (\delt)_c$ & or & $T\gg B_d$  &$\star$   \\
   fast fading channel             & if & $T\gg (\delt)_c$ & or & $T\ll B_d$.\\
\\
   underspread channel             & if & $T_mB_d<1$ \\
   overspread channel              & if & $T_mB_d>1$
\end{tabular}}
\end{definition}

The ``underspead/overspread" definitions are related to the 
\hie{Nyquist signaling rate}.\footnote{
  \hie{Nyquist signaling theorem}: 
  Theorem~\ref{thm:nst} page~\pageref{thm:nst}.
  }
The Nyquist signaling theorem states
the signaling rate $1/T$ is related to 
the transmitted signal bandwidth $W$ by $1/T \le 2W$.
So at the maximum rate, $TW=1/2\eqa 1$.
\[\begin{array}{rcll}
  TW     &\eqa& 1           & \mbox{(by Nyquist signaling theorem)                      } \\
  B_d    &\ll&  T           & \mbox{(for slowly fading channel)                         } \\
  T_m    &\ll&  W           & \mbox{(for frequency non-selective channel)               } \\
  T_mB_d &<&    TW \eqa 1   & \mbox{(for slowly fading, fequency non-selective channel).}
\end{array}\]


%======================================
\section{Multipath-fading countermeasures}
%======================================
There are two general classes of multipath-fading countermeasures:
\begin{dingautolist}{"AC}
  \item diversity techniques 
  \item Rake receiver. 
\end{dingautolist}

Diversity techniques for compensating for multipath are\citepp{proakis}{821}{822}
\begin{enume}
  \item frequency diversity
  \item time diversity
  \item antenna diversity
  \item path diversity
  \item angle of arrival diversity
  \item polarization diversity
\end{enume}

The rake receiver is a transversal filter with coefficients 
optimized for channel operation.





