%============================================================================
% Daniel J. Greenhoe
% XeLaTeX file
% introduction
%============================================================================
%=======================================
\chapter*{Acknowledgements}
\addcontentsline{toc}{section}{Acknowledgements}
%=======================================
It is not far from the truth to say that much of the research presented in this text started with this email from 
%my academic advisor, 
Professor Po-Ning Chen of National Chiao-Tung University, Taiwan:\footnote{%
  \begin{tabular}[t]{lll}%
      Po-Ning Chen:                          & \zht{陳伯寧}       &(pinyin: Ch/'en B/'o Ni/'ng)
    \\National Chiao-Tung University:        & \zht{國立交通大學} &(G/'uo L/`i Ji/-ao T/-ong D/`a X/'ue).
    \\Taiwan:                                & \zht{台灣}         &(pinyin: T/'ai W/-an).
    %\\Communications Engineering Department: & \zht{電信工程學系} &(Di/`an X/`in G/-ong Ch/'eng X/'ue X/`i)
    %\\address: Hsinchu, Taiwan:              & \zht{新竹,台灣}   &(X/-in Zh/'u, T/'ai W/-an)
  \end{tabular}}
\begin{center}\fbox{\begin{minipage}{\tw-20mm}\ttfamily\raggedright %\fntmono\raggedright
\hfill Sat, 25 Jan 2014 03:45:51 -0800 (PST)

Dear Dan:
So far people are mostly (and are used to) dealing with ``frequency-transform" of a numerical sequences. 
However, sometimes, we need to find out the "occurrence frequency" of a symbolic sequence. 
I am wondering ``Can we design a wavelet transform for a symbolic sequence like DNA?" 
What do you think? See the attached paper as an example.

Po-Ning
\end{minipage}}\end{center}

%I responded the same day saying, ``Thank you for the paper. So far I have read the introduction. I have
%never heard of ``Genomic Signal Processing" before. That is very
%interesting, and maybe something that could help a lot of people some
%day.\ldots I will investigate Genomic Signal Processing more. Thank you for
%introducing me to the topic."\footnote{The ``attached paper" referred to by Professor Chen's email is \citeP{galleani2010}.}
%
%This text is not about genomic signal processing in particular. 
%But it is about symbolic sequence processing, of which genomic signal processing is a special 
%and motivational case.

So I would like to say one more time to %my academic advisor, 
Professor Po-Ning Chen, 
``Thank you for introducing me to the topic"(!)
which has to a large extent resulted in the research described in this text.\footnote{%
  The ``attached paper" referred to by Professor Chen's email is \citeP{galleani2010}.}

%Moreover, it would be difficult to effectly communicate the ideas in this text without the use of 
%the typesetting software {\XeLaTeX}, and {\LaTeX} upon which it is based.
%Even though I am from the United States and {\LaTeX} has its origins in the United States,
%it seems I never really knew what it was before coming to Taiwan and being introduced to it by Professor Chen.
%And likewise {\XeLaTeX} has it's origins in the United States as well, but again I was introduced to it in Taiwan,
%this time by Professor Peng-Hua Wang.\footnote{%
%  Peng-Hua Wang (Chinese: \zht{王鵬華}, pinyin: W/'ang P/'eng Hu/'a).
%  }
%So I would like to thank Professor Chen for introducing me to \LaTeX,
%and I would like to thank Professor Wang for introducing me to \XeLaTeX.
%
%I would also like to thank 
%Professor Po-Ning Chen,\footnote{Po-Ning Chen (Chinese: \zht{陳伯寧}, pinyin: Ch/'en B/'o N/'ing)}
%Professor Yuan-Pei Lin,\footnote{Yuan-Pei Lin (Chinese: \zht{林源倍}, pinyin: L/'in Y/'uan B/`ei)}
%Professor Hsiao-Feng Lu,\footnote{Hsiao-Feng Lu (Chinese: \zht{陸曉峰}, pinyin: L/`u Xi/>ao F/-eng)}
%and
%Professor Zhong-Jin Lu\footnote{Zhong-Jin Lu (Chinese: \zht{呂忠津}, pinyin: L/>:u Zh/-ong J/-in)}
%who served on the committee for my oral defense, which was largely based on the material in this text.
%
%Finally, I would like to thank National Chiao-Tung University (NCTU), the Communications Engineering Department of NCTU, 
%and the people of Taiwan.\footnote{\begin{tabular}[t]{lll}%
%    Taiwan:                                & \zht{台灣}         &(pinyin: T/'ai W/-an).
%  \\National Chiao-Tung University:        & \zht{國立交通大學} &(G/'uo L/`i Ji/-ao T/-ong D/`a X/'ue).
%  \\Communications Engineering Department: & \zht{電信工程學系} &(Di/`an X/`in G/-ong Ch/'eng X/'ue X/`i)
%  \\address: Hsinchu, Taiwan:              & \zht{新竹,台灣}   &(X/-in Zh/'u, T/'ai W/-an)
%\end{tabular}}
%Although born and raised in the United States,
%Taiwan has graciously allowed me to live, study, and work here for several years now.
%To live in a country not one's own is always a privilege, an honor, and a great opportunity to learn about life.
%I am very happy to have had and to continue to have this privilege.
%\\[2ex]
%{\color{blue}---\enghw{Daniel J. Greenhoe} (\zhthw{柯晨光})}





