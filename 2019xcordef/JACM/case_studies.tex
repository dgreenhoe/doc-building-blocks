%============================================================================
% Daniel J. Greenhoe
% LaTeX file
%============================================================================
%============================================================================
\section{Case studies}
%============================================================================
It has been suggested by the giants that the usefulness of a mathematical idea can be measured by 
\begin{liste}
  \item how useful it is in applications\footnote{
    ``I regard as quite useless the reading of large treatises of pure analysis:
    too large a number of methods pass at once before the eyes.
    It is in the works of applications that one must study them;
    one judges their ability there and one apprises the manner of making use of them."
    ---Joseph Louis Lagrange (1736--1813).
    \citerp{stopple2003}{xi}
    }
  and
  \item how well it connects and is connected to the larger web of mathematical ideas.\footnote{
    \index{quotes!Hardy, G.H.}
    ``The ``seriousness" of a mathematical theorem lies,
    not in its practical consequences,
    which are usually negligible,
    but in the {\em significance} of the mathematical ideas which it connects.
    We may say, roughly, that a mathematical idea is ``significant" if it can be
    connected, in a natural illuminating way,
    with a large complex of other mathematical ideas."
    ---G.H. Hardy (1877--1947).
    \citerc{hardy1940}{section 11}
    }
\end{liste}

As such, this section which presents applications, 
may prove useful in gauging the usefulness of the preceding sections.