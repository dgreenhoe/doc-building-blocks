%============================================================================
% Daniel J. Greenhoe
% XeLaTeX file
%============================================================================

%=======================================
\chapter{Normed Algebras}
%=======================================
%=======================================
\section{Algebras}
%=======================================

All \structe{linear space}s\ifsxref{vector}{def:vspace} are equipped with an operation by which vectors in the spaces can be added together.
Linear spaces also have an operation that allows a scalar and a vector to be ``multiplied" together.
But linear spaces in general have no operation that allows two vectors to be multiplied together.
A linear space together with such an operator is an \structd{algebra}.\footnote{\citerpg{fuchs1995}{2}{052148412X}}

There are many many possible algebras---many more than one can shake a stick at,
as indicated by Michiel Hazewinkel in his book, \hie{Handbook of Algebras}:
``Algebra, as we know it today (2005), 
consists of many different ideas, concepts and results. 
A reasonable estimate of the number of these different items 
would be somewhere between 50,000 and 200,000. 
Many of these have been named and many more could (and perhaps should) 
have a ``name" or other convenient designation."\footnote{\citerpg{hazewinkel2000}{v}{044450396X}}

%---------------------------------------
\begin{definition}
\footnote{
  \citerpg{folland1995}{1}{0849384907}
  }
\label{def:unital_algebra}
\label{def:ualg}
%---------------------------------------
Let $\algA$ be an \structe{algebra}.
\defbox{
  \text{An algebra $\algA$ is \hid{unital} if
  $\quad\exists u\in\algA \st ux = xu = x \qquad\forall x\in\algA$}
  }
\end{definition}

%---------------------------------------
\begin{definition}
\footnote{
  \citerppg{folland1995}{3}{4}{0849384907}
  }
\indxs{\oppSpec}\indxs{\oppRes}\indxs{\oppRad}
%---------------------------------------
Let $\algA$ be an \structe{unital algebra} \xref{def:ualg} with unit $e$.
\defbox{\begin{array}{M>{\ds}rc>{\ds}lC}
  The \structd{spectrum} of $x\in\algA$ is        & \oppSpec(x)        &\eqd& \set{\lambda\in\C}{\lambda e - x \text{ is not invertible}}.\\
  The \structd{resolvent} of $x\in\algA$ is       & \oppRes_x(\lambda) &\eqd& (\lambda e - x)^{-1}   & \forall\lambda\notin\oppSpec(x).\\
  The \structd{spectral radius} of $x\in\algA$ is & \oppRad(x)         &\eqd& \sup\set{\abs{\lambda}}{\lambda\in\oppSpec(x)}.
\end{array}}
\end{definition}


%=======================================
\section{Star-Algebras}
%=======================================
%---------------------------------------
\begin{definition}
\footnote{
  %\citerpg{folland1995}{1}{0849384907},
  \citerp{rickart1960}{178},
  \citeIpg{gelfand1964}{241}{0821820222}
  }
\label{def:star_algebra}
\label{def:staralg}
\index{star-algebra}
\index{algebras!$\invo$-algebra} 
\indxs{\invo}
%---------------------------------------
Let $\algA$ be an \structe{algebra}.
\defbox{\begin{array}{F rcl CDD}
  \mc{7}{M}{The pair $\opair{\algA}{\invo}$ is a \structd{$\invo$-algebra}, or \structd{star-algebra}, if}
    \\1.& \brp{x+y}^\invo       &=& x^\invo + y^\invo     & \forall x,y\in\algA                 & (\prope{distributive})     & and 
    \\2.& (\alpha x)^\invo      &=& \bar{\alpha}x^\invo   & \forall x\in\algA,\, \alpha\in\C    & (\prope{conjugate linear}) & and 
    \\3.& (xy)^\invo            &=& y^\invo x^\invo       & \forall x,y \in \algA               & (\prope{antiautomorphic})  & and 
    \\4.& x^{\invo\invo}        &=& x                     & \forall x \in \algA                 & (\prope{involutory}) 
  \\\mc{7}{M}{The operator $\invo$ is called an \opd{involution} on the algebra $\algA$.}
\end{array}}
\end{definition}


%---------------------------------------
\begin{proposition}
\footnote{
  \citerpg{folland1995}{5}{0849384907}
  }
\label{prop:nalg_x*-1}
%---------------------------------------
Let $\opair{\algA}{\invo}$ be an \structe{unital $\invo$-algebra}.
\propbox{
  \text{$x$ is invertible}
  \qquad\implies\qquad
  \brbl{\begin{array}{FLCD}
    1. & \text{$x^\invo$ is \prope{invertible}}     & \forall x\in\algA  & and \\
    2. & \brp{x^\invo}^{-1} = \brp{x^{-1}}^\invo    & \forall x\in\algA
  \end{array}}
  }
\end{proposition}
\begin{proof}
Let $e$ be the unit element of $\opair{\algA}{\invo}$.
\begin{enumerate}
  \item Proof that $e^\invo = e$:\label{item:nalg_x*-1_e}
    \begin{align*}
      x\,e^\invo 
        &= \brp{x\,e^\invo}^{\invo\invo}
        && \text{by \prope{involutory} property of $\invo$}
        && \text{\xref{def:staralg}}
      \\&= \brp{x^\invo\,e^{\invo\invo}}^{\invo}
        && \text{by \prope{antiautomorphic} property of $\invo$}
        && \text{\xref{def:staralg}}
      \\&= \brp{x^\invo\,e}^{\invo}
        && \text{by \prope{involutory} property of $\invo$}
        && \text{\xref{def:staralg}}
      \\&= \brp{x^\invo}^{\invo}
        && \text{by definition of $e$}
      \\&= x
        && \text{by \prope{involutory} property of $\invo$}
        && \text{\xref{def:staralg}}
      \\
      e^\invo\,x 
        &= \brp{e^\invo\,x}^{\invo\invo}
        && \text{by \prope{involutory} property of $\invo$}
        && \text{\xref{def:staralg}}
      \\&= \brp{e^{\invo\invo}\,x^\invo}^{\invo}
        && \text{by \prope{antiautomorphic} property of $\invo$}
        && \text{\xref{def:staralg}}
      \\&= \brp{e\,x^\invo}^{\invo}
        && \text{by \prope{involutory} property of $\invo$}
        && \text{\xref{def:staralg}}
      \\&= \brp{x^\invo}^{\invo}
        && \text{by definition of $e$}
      \\&= x
        && \text{by \prope{involutory} property of $\invo$}
        && \text{\xref{def:staralg}}
    \end{align*}

  \item Proof that $\brp{x^\invo}^{-1} = \brp{x^{-1}}^\invo$:
    \begin{align*}
      \brp{x^{-1}}^\invo\,\brp{x^\invo}
        &= \brs{x\,\brp{x^{-1}}}^\invo
        && \text{by \prope{antiautomorphic} and \prope{involution} properties of $\invo$}
        && \text{\xref{def:staralg}}
      \\&= e^\invo
      \\&= e
        && \text{by \prefp{item:nalg_x*-1_e}}
      \\
      \brp{x^\invo}\,\brp{x^{-1}}^\invo
        &= \brs{x^{-1}\,x}^\invo
        && \text{by \prope{antiautomorphic} and \prope{involution} properties of $\invo$}
        && \text{\xref{def:staralg}}
      \\&= e^\invo
      \\&= e
        && \text{by \prefp{item:nalg_x*-1_e}}
    \end{align*}
\end{enumerate}
\end{proof}


%---------------------------------------
\begin{definition}
\label{def:op_adjoint}
\label{def:*_selfadj}
\footnote{
  \citerp{rickart1960}{178},
  \citeIpg{gelfand1964}{242}{0821820222}
  }
%---------------------------------------
Let $\opair{\algA}{\normn}$ be a \structe{$\invo$-algebra} \xref{def:staralg}.
\defboxp{
  \begin{liste}
    \item An element $x\in\algA$ is \hid{hermitian} or \hid{self-adjoint} if $x^\invo=x$.
    \item An element $x\in\algA$ is \hid{normal} if $xx^\invo=x^\invo x$.
    \item An element $x\in\algA$ is a \hid{projection} if 
          $xx=x$ (\prope{involutory}) and $x^\invo=x$ (\prope{hermitian}).
  \end{liste}
  }
\end{definition}


%---------------------------------------
\begin{theorem}
\footnote{
  \citerpg{michel1993}{429}{048667598X}
  }
\label{thm:nalg_hermitian}
\index{operator!adjoint}
\index{conjugate linear} \index{antilinear} \index{semilinear}
%---------------------------------------
Let $\opair{\algA}{\normn}$ be a \structe{$\invo$-algebra} \xref{def:staralg}.
\thmbox{
  \mcom{ x= x^\invo \text{ and }  y= y^\invo}{$ x$ and $ y$ are \prope{hermitian}}
  \qquad\implies\qquad
  \left\{
    \begin{array}{lcl @{\qquad}D}
       x+ y   &=& ( x+ y  )^\invo & \text{($x+ y$ is self adjoint)} \\
      %\alpha  x &=& \bar{\alpha}  x^\invo & \text{($\alpha  x$ is self adjoint)} \\
       x^\invo   &=& ( x^\invo  )^\invo & \text{($x^\invo$ is self adjoint)} \\
      \mc{4}{l}{\ds \mcom{ x  y   = ( x  y  )^\invo}{$( x y)$ is \prope{hermitian}} \iff \mcom{ x y= y x}{commutative}}
    \end{array}
  \right.
  }
\end{theorem}
\begin{proof}
\begin{align*}
  \brp{ x+ y}^\invo
    &=  x^\invo +  y^\invo
    && \text{by \prope{distributive} property of $\invo$}
    && \text{\xref{def:staralg}}
  \\&=  x +  y
    && \text{by left hypothesis}
  \\\\
  %\inprod{(\alpha x)\vx}{\vy}
  %  &= \inprod{\vx}{(\alpha x)^\invo\vy}
  %  && \text{by definition of adjoint \xref{def:op_adjoint}}
  %\\&= \inprod{\vx}{\bar{\alpha} x\vy}
  %  && \text{by \prefp{def:star_algebra}}
  %\\
  \brp{ x^\invo}^\invo
    &=  x
    && \text{by \prope{involutory} property of $\invo$}
    && \text{\xref{def:staralg}}
  \\\\
  \intertext{Proof that $ x y=( x y)^\invo \implies  x y= y x$}
   x y
    &= \brp{ x y}^\invo
    && \text{by left hypothesis}
  \\&=  y^\invo x^\invo
    && \text{by \prope{antiautomorphic} property of $\invo$}
    && \text{\xref{def:staralg}}
  \\&=  y x
    && \text{by left hypothesis}
  \\
  \intertext{Proof that $ x y=( x y)^\invo \impliedby  x y= y x$}
  \brp{ x y}^\invo
    &= \brp{ y x}^\invo
    && \text{by left hypothesis}
  \\&=  x^\invo y^\invo
    && \text{by \prope{antiautomorphic} property of $\invo$}
    && \text{\xref{def:staralg}}
  \\&=  x y
    && \text{by left hypothesis}
\end{align*}
\end{proof}


%---------------------------------------
\begin{definition}[Hermitian components]
\footnote{
  \citerpg{michel1993}{430}{048667598X},
  \citerp{rickart1960}{179},
  \citeIpg{gelfand1964}{242}{0821820222}
  }
\label{def:nalg_Re}
\label{def:nalg_Im}
\label{def:Re}
\label{def:Im}
%---------------------------------------
Let $\opair{\algA}{\normn}$ be a \structe{$\invo$-algebra} \xref{def:staralg}.
\defbox{
  \begin{array}{MLcL}
    The \opd{real part}      of $x$ is defined as & \hxs{\Real} x &\eqd& \frac{1}{2  }\Big( x+ x^\invo \Big)  \\
    The \opd{imaginary part} of $x$ is defined as & \hxs{\Imag} x &\eqd& \frac{1}{2i }\Big( x- x^\invo \Big)
  \end{array}
  }
\end{definition}

%---------------------------------------
\begin{theorem}
\footnote{
  \citerpg{michel1993}{430}{048667598X},
  \citerpg{halmos}{42}{0821813781}  
  }
\label{thm:nalg_re_sa}
%---------------------------------------
Let $\opair{\algA}{\invo}$ be a \structe{$\invo$-algebra} \xref{def:staralg}.
\thmbox{\begin{array}{rcl @{\qquad}C @{\qquad}D}
  \Real x &=& \brp{\Real x}^\invo & \forall  x\in\algA & ($\Real x$ is \prope{hermitian})\\
  \Imag x &=& \brp{\Imag x}^\invo & \forall  x\in\algA & ($\Imag x$ is \prope{hermitian})
\end{array}}
\end{theorem}
\begin{proof}
  \begin{align*}
    \brp{\Real x}^\invo
      &= \brp{\frac{1}{2  }\brp{ x+ x^\invo}}^\invo
      && \text{by definition of $\Re$}
      && \text{\xref{def:nalg_Re}}
    \\&= \frac{1}{2  }\brp{ x^\invo+ x^{\invo\invo}}
      && \text{by \prope{distributive} property of $\invo$}
      && \text{\xref{def:staralg}}
    \\&= \frac{1}{2  }\brp{ x^\invo+ x}
      && \text{by \prope{involutory} property of $\invo$}
      && \text{\xref{def:staralg}}
    \\&= \Real x
      && \text{by definition of $\Re$}
      && \text{\xref{def:nalg_Re}}
    \\
    \brp{\Imag x}^\invo
      &= \brp{\frac{1}{2i}\brp{ x- x^\invo}}^\invo
      && \text{by definition of $\Im$}
      && \text{\xref{def:nalg_Im}}
    \\&= \frac{1}{2i }\brp{ x^\invo- x^{\invo\invo}}
      && \text{by \prope{distributive} property of $\invo$}
      && \text{\xref{def:staralg}}
    \\&= \frac{1}{2i }\brp{ x^\invo- x}
      && \text{by \prope{involutory} property of $\invo$}
      && \text{\xref{def:staralg}}
    \\&= \Imag x
      && \text{by definition of $\Im$}
      && \text{\xref{def:nalg_Im}}
  \end{align*}
\end{proof}

%---------------------------------------
\begin{theorem}[\thmd{Hermitian representation}]
\label{thm:nalg_re_im}
\footnote{
  \citerpg{michel1993}{430}{048667598X},
  \citerp{rickart1960}{179},
  %\citeIpg{gelfand1964}{242}{0821820222} \\
  \citeIp{gelfand1943r}{7}
  }
\index{hermitian components}
%---------------------------------------
Let $\opair{\algA}{\invo}$ be a \structe{$\invo$-algebra} \xref{def:staralg}.
\thmbox{
   a =  x + i y
  \qquad\iff\qquad
   x=\Real a \quad\text{and}\quad  y=\Imag a
  }
\end{theorem}
\begin{proof}
  \begin{liste}
    \item Proof that $ a =  x + i y \implies  x=\Real a \quad\text{and}\quad  y=\Imag a$:
      \begin{align*}
                   &&  a  &=  x + i y                 && \text{by left hypothesis}
        \\\implies &&  a^\invo &= \brp{ x+i y}^\invo  
                   && \text{by definition of \fncte{adjoint}}
                   && \text{\xref{def:op_adjoint}}
        \\         &&       &=  x^\invo - i y^\invo   
                   && \text{by \prope{distributive} property of $\invo$}
                   && \text{\xref{def:star_algebra}}
        \\         &&       &=  x - i y           && \text{by \prefp{thm:nalg_re_sa}}
        \\\implies &&  x  &=  a  - i y          && \text{by solving for $ x$ in $ a = x+i y$ equation}
        \\         &&  x  &=  a^\invo + i y          && \text{by solving for $ x$ in $ a^\invo= x-i y$ equation}
        \\\implies &&  x+ x &=  a+ a^\invo         && \text{by adding previous 2 equations}
        \\\implies && 2 x &=  a+ a^\invo             && \text{by solving for $ x$ in previous equation}
        \\\implies &&  x  &= \frac{1}{2}\brp{ a+ a^\invo}
        \\         &&       &= \Real a                
                   && \text{by definition of $\Re$} 
                   && \text{\xref{def:nalg_Re}}
        \\         && 
        \\         && i y &=  a  -  x           && \text{by solving for $i y$ in $ a = x+i y$ equation}
        \\         && i y &= - a^\invo +  x          && \text{by solving for $i y$ in $ a = x+i y$ equation}
        \\\implies && i y+i y &=  a- a^\invo       && \text{by adding previous 2 equations}
        \\\implies &&  y  &= \frac{1}{2i}\brp{ a- a^\invo} && \text{by solving for $i y$ in previous equations}
        \\         &&       &= \Imag a                
                   && \text{by definition of $\Im$}
                   && \text{\xref{def:nalg_Im}}
      \end{align*}

    \item Proof that $ a =  x + i y \impliedby  x=\Real a \quad\text{and}\quad  y=\Imag a$:
      \begin{align*}
         x + i y
          &= \Real a + i\,\Imag a
          && \text{by right hypothesis}
        \\&= \mcom{\frac{1}{2}\brp{ a+ a^\invo}}{$\Real a$} + i\mcom{\frac{1}{2i}\brp{ a- a^\invo}}{$\Imag a$}
          && \text{by definition of $\Re$ and $\Im$}
          && \text{\xref{def:nalg_Re}}
        \\&= \brp{\frac{1}{2} a+\frac{1}{2} a} + 
             \cancelto{0}{\brp{\frac{1}{2} a^\invo-\frac{1}{2} a^\invo}}
        \\&=  a
      \end{align*}
  \end{liste}
\end{proof}

%=======================================
\section{Normed Algebras}
%=======================================
%---------------------------------------
\begin{definition}
\footnote{
  \citerp{rickart1960}{2},
  \citerpgc{berberian1961}{103}{0821819127}{Theorem IV.9.2}
  %\citerpg{folland1995}{1}{0849384907}
  }
\label{def:normed_algebra}
\label{def:nalg}
%---------------------------------------
Let $\algA$ be an algebra.
\defbox{\begin{array}{M}
  The pair $\opair{\algA}{\normn}$ is a \hid{normed algebra} if
  \\\qquad$\ds\norm{xy} \le \norm{x}\norm{y} 
    \qquad \forall x,y\in\algA 
    \qquad \text{\scriptsize(\hid{multiplicative condition})}
  $\\
  A normed algebra $\opair{\algA}{\normn}$ is a \hid{Banach algebra} if
  $\opair{\algA}{\normn}$ is also a Banach space.
\end{array}}
\end{definition}

%---------------------------------------
\begin{proposition}
%---------------------------------------
\propbox{\text{
  $\opair{\algA}{\normn}$ is a normed algebra
  $\qquad\implies\qquad$
  multiplication is \hib{continuous} in $\opair{\algA}{\normn}$
  }}
\end{proposition}
\begin{proof}
  \begin{enumerate}
    \item Define $\ff(x)\eqd zx$. That is, the function $\ff$ represents multiplication of $x$ times some 
          arbitrary value $z$. \label{item:ffnorm}

    \item Let $\delta\eqd \norm{x-y}$ and $\epsilon\eqd\norm{\ff(x)-\ff(y)}$. \label{item:deltanorm}

    \item To prove that multiplication ($\ff$) is \hie{continuous} with respect to the metric generated by $\normn$,
          we have to show that we can always make $\epsilon$ arbitrarily small for some $\delta>0$.

    \item And here is the proof that multiplication is indeed continuous in $\opair{\algA}{\normn}$: 
      \begin{align*}
        \norm{\ff(x)-\ff(y)}
          &\eqd\norm{zx-zy}
          &&   \text{by definition of $\ff$}
          &&   \text{\xref{item:ffnorm}}
        \\&=   \norm{z(x-y)}
        \\&\le \norm{z}\,\norm{x-y}
          &&   \text{by definition of \structe{normed algebra}}
          &&   \text{\xref{def:normed_algebra}}
        \\&\eqd\norm{z}\,\delta
          &&   \text{by definition of $\delta$}
          &&   \text{\xref{item:deltanorm}}
        \\&\le \epsilon
          &&   \text{for some value of $\delta>0$}
      \end{align*}
  \end{enumerate}
\end{proof}

%---------------------------------------
\begin{theorem}[\thmd{Gelfand-Mazur Theorem}]
\footnote{
  \citerpg{folland1995}{4}{0849384907},
  \citePc{mazur1938}{(statement)},
  \citePc{gelfand1941}{(proof)}
  }
%---------------------------------------
Let $\C$ be the field of complex numbers.
\thmbox{
  \brbr{\begin{array}{l}
    \text{$\opair{\algA}{\normn}$ is a Banach algebra} \\
    \text{every nonzero $x\in\algA$ is invertible}
  \end{array}}
  \qquad\implies\qquad
  \algA \isomorphic \C
  \quad\text{($\algA$ is isomorphic to $\C$)}
}
\end{theorem}


%=======================================
\section{C* Algebras}
%=======================================

%---------------------------------------
\begin{definition}
\footnote{
  \citerpg{folland1995}{1}{0849384907},
  \citeIpg{gelfand1964}{241}{0821820222},
  \citeP{gelfand1943},
  \citeI{gelfand1943r} 
  }
%\label{def:cstar_algebra}
\label{def:cstar}
\index{algebras!$C^\invo$-algebra} 
%---------------------------------------
\defboxt{
  The triple $\hxs{\otriple{\algA}{\normn}{\invo}}$ is a \structd{$C^\invo$ algebra} if
  \\\indentx$\begin{array}{FLlD}
      1. & \opair{\algA}{\normn}       & \text{is a Banach algebra}  & and  
    \\2. & \opair{\algA}{\invo}        & \text{is a $\invo$-algebra} & and  
    \\3. & \norm{x^\invo x}=\norm{x}^2 & \forall x\in\algA           &.
  \end{array}$\\
  A \structd{$C^\invo$ algebra} $\otriple{\algA}{\normn}{\invo}$ is also called a \structd{C star algebra}.
  }
\end{definition}


%---------------------------------------
\begin{theorem}
\footnote{
  \citerpg{folland1995}{1}{0849384907},
  \citeIp{gelfand1943r}{4},
  \citeP{gelfand1943}
  }
%---------------------------------------
Let $\algA$ be an algebra.
\thmbox{
  \text{$\otriple{\algA}{\normn}{\invo}$ is a \hid{$C^\invo$ algebra}}
  \qquad\implies\qquad
  \norm{x^\invo}=\norm{x}
  }
\end{theorem}
\begin{proof}
\begin{align*}
  \norm{x}
    &= \frac{1}{\norm{x}}\;\norm{x}^2
  \\&= \frac{1}{\norm{x}}\;\norm{x^\invo x}
    && \text{by definition of \structe{$C^\invo$-algebra}} 
    && \text{\xref{def:cstar}}
  \\&\le \frac{1}{\norm{x}}\;\norm{x^\invo}\norm{x}
    && \text{by definition of \structe{normed algebra}}
    && \text{\xref{def:normed_algebra}}
  \\&= \norm{x^\invo}
  \\
  \norm{x^\invo}
    &\le \norm{x^{\invo\invo}}
    && \text{by previous result}
  \\&= \norm{x}
    && \text{by \prope{involution} property of $\invo$}
    && \text{\xref{def:star_algebra}}
\end{align*}
\end{proof}

