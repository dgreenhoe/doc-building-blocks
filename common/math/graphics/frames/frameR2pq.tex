%============================================================================
% Daniel J. Greenhoe
% LaTeX file
% frame for R^2 with N frame vectors
% nominal unit = 10mm
%============================================================================
\begin{pspicture}(-1.25,-1.25)(1.5,1.5)%
  %-------------------------------------
  % parameters
  %-------------------------------------
  \psset{linewidth=2pt,labelsep=1pt}%
  \pnode(0,0){o}% origin
  %-------------------------------------
  % frame vector definitions
  %-------------------------------------
  \pnode(0.8660254, 0.5){p}% p = r(cos30, sin30)
  \pnode(-0.5,0.8660254){q}% q = r(cos120,sin120)
  \pnode(p){x1}% x1 = p
  \pnode(0.3660254, 1.3660254){x2}% x2 = p + q
  \pnode(1.3660254,-0.3660254){x3}% x3 = p - q
  %-------------------------------------
  % axes
  %-------------------------------------
  \psaxes[linewidth=0.75pt,linecolor=axis,ticks=none,labels=none]{<->}(0,0)(-1.25,-1.25)(1.5,1.5)%
  \uput{2pt}[120](1.5,0){$x$}%
  \uput{3pt}[210](0,1.5){$y$}%
  \pscircle[linecolor=red,linestyle=dashed,linewidth=0.75pt](0,0){1}% unit circle
  %-------------------------------------
  % annotation
  %-------------------------------------
 %\uput[ 90](p){$\vp$}%
  \uput[180](q){$\vq$}%
  \uput[90](x1){$\vx_1\!\eqd\!\vp$}%
  \uput[0](x2){$\vx_2$}%
  \uput[-120](x3){$\vx_3$}%
  %\uput[-120](x3){$\vx_3\!\eqd\!\vp\!-\!\vq$}%
  %\uput[90](x1){$\vx_1\!\eqd\!\vp$}%
  %-------------------------------------
  % frame vector plots
  %-------------------------------------
  \ncline[linecolor=red]{->}{o}{p}%
  %\naput[nodesep=0pt,nrot=:U]{$1$}%  
  \ncline[linecolor=red]{->}{o}{q}%
  \ncline{->}{o}{x1}%
  \ncline{->}{o}{x2}%
  \ncline{->}{o}{x3}%
  \ncline{->}{o}{x4}%
\end{pspicture}%