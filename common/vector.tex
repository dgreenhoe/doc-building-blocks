%============================================================================
% LaTeX File
% Daniel J. Greenhoe
%============================================================================





%======================================
\chapter{Linear spaces}
%======================================
\qboxnpq
  {\href{http://en.wikipedia.org/wiki/Giuseppe_Peano}{Giuseppe Peano}
   \href{http://www-history.mcs.st-andrews.ac.uk/Timelines/TimelineF.html}{(1858--1932)},
   \href{http://www-history.mcs.st-andrews.ac.uk/BirthplaceMaps/Places/Italy.html}{Italian} mathematician,
   credited with being one of the first to introduce the concept of the \structe{linear space} (\structe{vector space}).\footnotemark}
  {../common/people/PeanoGiuseppe_wkp_pdomain.jpg}
  {The geometric \struct{calculus}, in general, 
   consists in a system of operations on geometric entities, and their consequences, 
   analogous to those that \struct{algebra} has on the numbers.
   It permits the expression in formulas of the results of geometric constructions,
   the representation with equations of propositions of \struct{geometry},
   and the substitution of a transformation of equations for a verbal argument.}
  \citetblt{
    quote: & \citerp{peano1888e}{ix} \\
    image  & \scs\url{http://en.wikipedia.org/wiki/File:Giuseppe_Peano.jpg}, public domain
    }
%\qboxnpqt
%  {
%    Jules Henri Poincar\'e (1854-1912), physicist and mathematician
%    \index{Poincar\'e, Jules Henri}
%    \index{quotes!Poincar\'e, Jules Henri}
%    \footnotemark
%  }
%  {../common/people/poincare.jpg}
%  {
%    \ldots et que nous nommons le temps et l'espace. \ldots
%    ce n'est pas la nature qui nous les impose,
%    c'est nous qui les imposons \`a la nature parce que nous les trouvons
%    commodes, \ldots
%  }
%  {
%    \ldots and which are called time and space. \ldots
%    it is not nature which imposes them upon us,
%    it is we who impose them upon nature because
%    we find them convenient.
%  }
%\citetblt{
%  quote:       & \citorc{poincare_vos}{Introduction, paragraph 10}  \\
%  translation: & \citorp{poincare_vos_e}{13}  \\
%  image:       & \url{http://en.wikipedia.org/wiki/Image:Poincare_jh.jpg}
%  }


%\qboxnpq
%  {
%    Karl Friedrich Gauss, mathematician (1777--1855)
%    in an November 8, 1824 letter to Franz Adolph Taurinus
%    \index{Gauss, Karl Friedrich}
%    \footnotemark
%  }
%  {../common/people/gauss.jpg}
%  {But it seems to me that in spite of the word-mastery of the metaphysicians,
%   we know really too little, or even nothing at all,
%   about the true nature of space to be able to confuse something that seems
%   unnatural with absolutely impossible.
%   If non-Euclidean geometry is the real one and the constant is incomparable
%   to the magnitudes that we encounter on earth or in the heavens then it can be
%   determined aposteriori.
%   I have therefore occasionally for fun expressed the wish that Euclidean geometry
%   not be the real one, for then we would have a priori an absolute measure.}
%  \footnotetext{\begin{tabular}[t]{ll}
%    quote: & \url{http://www.ltn.lv/~podnieks/} \\
%           & Burris(2003), \url{http://www.math.uwaterloo.ca/~snburris/htdocs/noneucl.pdf}, page 12 \\
%           & \url{http://resolver.sub.uni-goettingen.de/purl?PPN236010751} \\
%    image: & \url{http://en.wikipedia.org/wiki/Karl_Friedrich_Gauss}
%  \end{tabular}}


\begin{figure}[th]
  \centering
  \psset{yunit=1.5\latunit}
  %============================================================================
% Daniel J. Greenhoe
% LaTeX file
%============================================================================
\begin{pspicture}(-7,0.5)(5,9)%
  %-------------------------------------
  % settings
  %-------------------------------------
   \psset{%
    arrowsize=4mm,
    arrowlength=0.6,
    arrowinset=0.1,
    %linecolor=blue,
    %linewidth=1pt,
     cornersize=relative,
     framearc=0.25,
    % gridcolor=graph,
    % subgriddiv=1,
    % gridlabels=4pt,
    % gridwidth=0.2pt,
    xunit=1\latunit,
    yunit=1.25\latunit,
     }%
  %-------------------------------------
  % nodes
  %-------------------------------------
   \begin{tabstr}{0.75}%
     \rput(-1, 8){\rnode{spaces}    {\psframebox{\begin{tabular}{c}abstract spaces\end{tabular}}}}%
     \rput(-4, 7){\rnode{lin}       {\psframebox{\begin{tabular}{c}linear spaces\ifnxref{vector}{def:vspace}\end{tabular}}}}%
     \rput( 2, 7){\rnode{top}       {\psframebox{\begin{tabular}{c}topological spaces\ifnxref{vstopo}{def:toplinspace}\end{tabular}}}}%
     \rput( 2, 6){\rnode{metric}    {\psframebox{\begin{tabular}{c}metric spaces\ifnxref{metric}{def:metric}\end{tabular}}}}%
     \rput( 2, 4.5){\rnode{commetric} {\psframebox{\begin{tabular}{c}complete metric spaces\ifnxref{seq}{def:ms_complete}\end{tabular}}}}%
     \rput(-4, 5){\rnode{metriclin} {\psframebox{\begin{tabular}{c}metric linear spaces\end{tabular}}}}%
     \rput(-4, 4){\rnode{normlin}   {\psframebox{\begin{tabular}{c}normed linear spaces\ifnxref{vsnorm}{def:norm}\end{tabular}}}}%
     \rput( 2, 3){\rnode{banach}    {\psframebox{\begin{tabular}{c}Banach spaces\ifnxref{seq}{def:banach}\end{tabular}}}}%
     \rput(-4, 3){\rnode{inprod}    {\psframebox{\begin{tabular}{c}inner-product spaces\ifnxref{vsinprod}{def:inprod}\end{tabular}}}}%
     \rput( 2, 2){\rnode{hilbert}   {\psframebox{\begin{tabular}{c}Hilbert spaces\ifnxref{seq}{def:hilbert}\end{tabular}}}}%
     \rput( 2, 1){\rnode{zero}      {\psframebox{\begin{tabular}{c}$\spZero$\ifnxref{subspace}{prop:subspace_0X}\end{tabular}}}}%
   \end{tabstr}%
  %-------------------------------------
  % connecting lines/arrows
  %-------------------------------------
   %\ncline[doubleline=true]{->}{lin}{spaces} double arrow seems to cause trouble for xdvipdfmx
   \ncline{lin}{spaces}
   \ncline{top}{spaces}%
   \ncline{com}{spaces}%
   \ncline{metriclin}{lin}%
   \ncline{normlin}{metriclin}%
   \ncline{inprod}{normlin}
   \ncline{banach}{normlin}%
   \ncline{hilbert}{inprod}%
   \ncline{hilbert}{banach}%
   \ncline{zero}{hilbert}%
   \ncline{metric}{top}%
   \ncline{metriclin}{metric}
   \ncline{commetric}{metric}%
   \ncline{banach}{commetric}%
  %-------------------------------------
  % labeling
  %-------------------------------------
   %\psccurve[linestyle=dashed,linecolor=red,fillstyle=none]%
   %  (0,-5)(20,4)(70,60)(20,60)(15,55)(-5,38)(-20,35)(-10,20)(-25,5)%
   %\psline[linecolor=red]{->}(60,75)(60,68)%
   %\uput[135](60,75){complete spaces}%
   %\psline[linecolor=red]{->}(26,80)(15,74)%
   %\psline[linecolor=red]{->}(30,80)(30,62.5)%
   %\uput[90](28,80){analytic spaces}%
   %
   %\psgrid[unit=10mm](-8,-1)(8,9)%
\end{pspicture}%

  \caption{Lattice of mathematical spaces\label{fig:vector_spaces}}
\end{figure}%




%======================================
\section{Definition and basic results}
%======================================
A \structe{metric space}\ifsxref{metric}{def:metric} is a \structe{set} together with nothing else save a \structe{metric} 
that gives the space a \structe{topology} \xref{def:topology}.
A \structe{linear space} (next definition) in general has no topology but does have some additional 
\structe{algebraic} structure\ifsxref{algebra}{app:algebra} that 
is useful in generalizing a number of mathematical concepts.
If one wishes to have both algebraic structure and a topology, then this can be accomplished by appending 
a \structe{topology} to a \structe{linear space} giving a \structe{topological linear space} \xref{def:toplinspace}, 
a \structe{metric} giving a \structe{metric linear space}\ifsxref{metric}{def:metric},
an \structe{inner product} giving an \structe{inner product space} \xref{def:inprod}, % or a \hie{Hilbert space} \xref{def:hilbert},
or a \structe{norm} giving a \structe{normed linear space} \xref{def:norm}. % or a \structe{Banach space} \xref{def:banach}.
%---------------------------------------
\begin{definition}
\footnote{
  \citerppgc{kubrusly2001}{40}{41}{0817641742}{Definition 2.1 and following remarks},
  \citerp{haaser1991}{41},
  \citerpp{halmos1948}{1}{2},
  \citerc{peano1888}{Chapter IX},
  \citerpp{peano1888e}{119}{120},
  \citePpp{banach1922}{134}{135}
  }
\label{def:vspace}
\index{space!vector}
\index{space!linear}
%---------------------------------------
%Let $\F\eqd\otriple{\setS}{+}{\cdot}$ be a field.
Let $\fieldF$ be a \structe{field}\ifsxref{algebra}{def:field}.
Let $\setX$ be a set, 
let $+$ be an \structe{operator} \xref{def:operator} in $\clF{\setX^2}{\setX}$, 
and let $\otimes$ be an operator in $\clF{\F\times\setX}{\setX}$.
\defboxp{
  The structure $\spO\eqd\linearspaceX$ is a \structd{linear space} over $\fieldF$ if
  %\\\indentx$\ds\begin{array}{l rcl @{\quad}C @{\quad}D@{}r@{}}
  \\\indentx$\ds\begin{array}{>{\scriptstyle}r rcl @{\quad}C @{\quad}D@{}r@{}}
    \cline{7-7}
    1.& \exists \vzero\in\setX \st \vx + \vzero &=& \vx
      & \forall \vx\in\setX
      & ($+$ \structe{identity})
      & \ast\vline
      \\
    2.& \exists \vy\in\setX \st \vx+\vy &=& \vzero
      & \forall \vx \in\setX
      & ($+$ \structe{inverse})
      & \vline
      \\
    3.& (\vx+\vy)+\vz &=& \vx+(\vy+\vz)
      & \forall \vx,\vy,\vz\in\setX
      & ($+$ is \prope{associative})
      & \text{ }\vline
      \\
    4.& \vx+\vy &=& \vy+\vx
      & \forall \vx,\vy\in\setX
      & ($+$ is \prope{commutative})
      & \vline
      \\\cline{7-7}
    5.& 1\cdot \vx &=& \vx
      & \forall \vx\in\setX
      & ($\cdot$ \prope{identity})
      \\
    6.& \alpha\cdot(\beta\cdot\vx) &=& (\alpha\cdot\beta)\cdot\vx
      & \forall \alpha,\beta\in\setS \text{ and } \vx\in\setX
      & ($\cdot$ \prope{associates} with $\cdot$)
      \\
    7.& \alpha\cdot(\vx+\vy) &=& (\alpha \cdot\vx)+(\alpha\cdot\vy)
      & \forall \alpha\in\setS \text{ and } \vx,\vy\in\setX
      & ($\cdot$ \prope{distributes} over $+$)
      \\
    8.& (\alpha+\beta)\cdot\vx &=& (\alpha\cdot \vx)+(\beta\cdot \vx)
      & \forall \alpha,\beta\in\setS \text{ and } \vx\in\setX
      & ($\cdot$ \prope{pseudo-distributes} over $+$)
  \end{array}$
  \\
  The set $\setX$ is called the \structd{underlying set}.
  The elements of $\setX$ are called \structd{vectors}.
  The elements of $\F$ are called \structd{scalars}.
  A \structe{linear space} is also called a \structd{vector space}.
  If $\F\eqd\R$, then $\spO$ is a \structd{real linear space}.
  If $\F\eqd\C$, then $\spO$ is a \structd{complex linear space}.
  }
\end{definition}

%---------------------------------------
\begin{definition}
\label{def:lsub}
%---------------------------------------
Let $\spL_1\eqd\linearspace{\setX_1}{+}{\cdot}{\F_1}{\addf}{\dottimes}$
and $\spL_2\eqd\linearspace{\setX_2}{+}{\cdot}{\F_2}{\addf}{\dottimes}$.
\defboxt{$\spO_2$ is a \structd{linear subspace} of $\spO_1$ if
  \\\indentx$\begin{array}{FrcllD}
      1. & \mc{3}{M}{$\spL_1$ is a \structe{linear space}} & \xref{def:vspace} & and 
    \\2. & \mc{3}{M}{$\spL_2$ is a \structe{linear space}} & \xref{def:vspace} & and 
    \\3. & \F_2    &\subseteq& \F_1                        &                   & and
    \\4. & \setX_2 &\subseteq& \setX_1                     &                   & and
  \end{array}$}
\end{definition}

%---------------------------------------
\begin{remark}
\footnote{
  \citerp{akhiezer1e}{1},
  \citerp{haaser1991}{41}
  }
%---------------------------------------
By the first four conditions ($\ast]$) listed in \pref{def:vspace},
$\opair{\setX}{+}$ is a \hib{commutative group} (or \hib{abelian group}).
\end{remark}

Often when discussing a linear space, %the operator $+$ is expressed simply as $+$ and
the operator $\cdot$ is simply expressed with juxtaposition
(e.g. $\alpha\vx$ is equivalent to $\alpha\cdot\vx$). %is equivalent to $\alpha\otimes\vx$).
In doing this, there is no risk of ambiguity between %vector-vector addition
scalar-vector multiplication and scalar-scalar multiplication
because the operands uniquely identify the precise operator.\footnote{
  {\em Operator overload} is a technique in which
  two fundamentally different operators or functions
  share the same symbol or label. It is
  inherent in the programming language {\em C++} and is therein called
  {\em operator overload}.
  In C++, you can define two (or more) operators or functions that
  share the same symbol or name, but yet are completely different.
  Two such operators (or functions) are distinguished from each other
  by the type of their operands.
  So for example, in C++, you can define an
  $m\times n$ matrix {\em type} and use operator overload to
  define a $+$ operator that operates on this new matrix type.
  So if variables $x$ and $y$ are of floating point type and
  $A$ and $B$ are of the matrix type,
  you can then add either type using the same syntax style:\\
    \begin{tabular}{l@{\hs{6ex}}l}
      x+y  & (add two floating point numbers) \\
      A+B  & (add two matrices)
    \end{tabular}\\
  Even though both of these operations ``look" the same, they
  are of course fundamentally different.
  }


%---------------------------------------
\begin{example}[\exmd{tuples in $\F^\xN$}]
\footnote{
  \citerpgc{kubrusly2001}{41}{0817641742}{Example 2D}
  }
\label{ex:lsp_FN}
%---------------------------------------
Let $\tuplexn{x_n}$ be an \structe{$\xN$-tuple} \xref{def:tuple} over a \structe{field}\ifsxref{algebra}{def:field} $\fieldF$.
\exbox{\begin{array}{N>{\ds}rc>{\ds}l@{\qquad}C@{\qquad}D}
  Let    & \setX                     &\eqd& \set{\tuplexn{x_n}}{x_n\in\F}         &                              & and
  \\     & \tuplexn{x_n}+\tuplexn{y_n} &\eqd& \tuplexn{x_n\dotplus y_n}             & \forall \tuplen{x_n}\in\setX & and
  \\     & \alpha\cdot\tuplexn{x_n}   &\eqd& \tuplexn{\alpha\dottimes x_n}         & \forall \tuplen{x_n}\in\setX, \alpha\in\F. & 
  \\\mc{6}{M}{Then the structure $\linearspaceX$ is a \prope{linear space}.}
\end{array}}
\end{example}

%%---------------------------------------
%\begin{remark}
%\footnote{
%  \citerpgc{kubrusly2001}{41}{0817641742}{Example 2D}
%  }
%%---------------------------------------
%Note that the tuples $\tuplexn{x_n}$ in example \pref{ex:lsp_FN} can not be extended a sequence $\seqxn{x_n}$
%without first equipping the linear space with a \hie{topology}.
%This is because 
%\end{remark}

%---------------------------------------
\begin{example}[real numbers]
\footnote{
  \citerpgc{kubrusly2001}{41}{0817641742}{Example 2D},
  \citor{hamel1905}
  }
%---------------------------------------
Let  $\field{\R}{+}{\cdot}$ be the field of real numbers.
\exbox{\begin{array}{M}%\begin{array}{N>{\ds}rc>{\ds}l@{\qquad}C@{\qquad}D}
  %Let    & x+y      &\eqd& x\dotplus y     & \forall x,y\in\R  & and
  %\\     & x\cdot y &\eqd& x\dottimes y    & \forall x,y\in\R. &
  %\\\mc{6}{M}{Then the structure $\linearspaceXR[\R]$ is a \prope{linear space}}.
  %\\\mc{6}{M}{That is, the field $\fieldR$ with $+$ and $\cdot$ is a linear space over itself.}
    {The structure $\linearspace{\R}{+}{\cdot}{\R}{+}{\cdot}$ is a \prope{linear space}}.
  \\{That is, the field of real numbers forms a linear space over itself.}
\end{array}}
\end{example}

%---------------------------------------
\begin{example}[functions]
\footnote{
  \citerpgc{kubrusly2001}{42}{0817641742}{Example 2F}
  }
%---------------------------------------
Let $\fieldF$ be a field.
Let $\clFxy$ be the set of all functions with domain $\setX$ and range $\setY$.
\exbox{\begin{array}{N>{\ds}rc>{\ds}l@{\qquad}C@{\qquad}D@{\qquad}D}
  Let    & \brs{\ff+\fg}(x)          &\eqd& \ff(x) + \fg(x)                  & \forall \ff,\fg\in\clFxy & (\prope{pointwise addition}) & and
  \\     & \brs{\alpha\cdot\ff}(x)   &\eqd& \alpha\cdot\brs{\ff(x)}          & \forall \ff\in\clFxy, \alpha\in\F. & 
  \\\mc{7}{M}{Then the structure $\linearspace{\clFxy}{+}{\cdot}{\F}{\dotplus}{\dottimes}$ is a \prope{linear space}.}
\end{array}}
\end{example}

%---------------------------------------
\begin{example}[functions onto $\F$]
\footnote{
  \citerpgc{kubrusly2001}{41}{0817641742}{Example 2E}
  }
%---------------------------------------
Let $\field{\F}{+}{\cdot}$ be a field.
Let $\clFxf$ be the set of all functions with domain $\setX$ and range $\F$.
\exbox{\begin{array}{N>{\ds}rc>{\ds}l@{\qquad}C@{\qquad}D@{\qquad}D}
  Let    & \brs{\ff+\fg}(x)          &\eqd& \ff(x) + \fg(x)                  & \forall \ff,\fg\in\clFxf & (\prope{pointwise addition}) & and
  \\     & \brs{\alpha\cdot\ff}(x)   &\eqd& \alpha\cdot\brs{\ff(x)}          & \forall \ff\in\clFxf,\, \alpha\in\F. & 
  \\\mc{7}{M}{Then the structure $\linearspace{\clFxf}{+}{\cdot}{\F}{+}{\cdot}$ is a \prope{linear space}.}
\end{array}}
\end{example}

%---------------------------------------
\begin{theorem}[Additive identity properties]
\footnote{
  \citerpgc{berberian1961}{6}{0821819127}{Theorem 1},
  \citerpg{michel1993}{77}{048667598X}
  }
\label{thm:vs_addid}
\index{additive identity}
%--------------------------------------
Let $\linearspaceX$ be a linear space,
$0$ the \structe{additive identity element} \xref{def:group} with respect to $\dotplus$,
and $\vzero$ the \structe{additive identity element} with respect to $+$.
\thmbox{\begin{array}{F lcl cMC}
    1. & 0\vx             &=& \vzero                 &        &                             & \forall \vx\in\setX
  \\2. & \alpha\vzero     &=& \vzero                 &        &                             & \forall \alpha\in\F
  \\3. & \alpha\vx        &=& \vzero                 &\implies& $\alpha=0$ or $\vx=\vzero$  & 
  \\4. & \vx+\vx          &=& \vx                    &\implies& $\vx=\vzero$                & 
  \\5. & \mc{3}{M}{$\alpha\neq0$ and $\vx\neq\vzero$}&\implies& $\alpha\vx\neq\vzero$       & 
\end{array}}
\end{theorem}
\begin{proof}
  \begin{align*}
    \intertext{1. Proof that $0\vx=\vzero$:}
      0\vx
        &= 0\vx + 0\vzero
        && \text{by definition of $+$ additive identity element}
      \\&= 0\vx + 0\vx + (-0\vx)
        && \text{by definition of $+$ additive inverse}
      \\&= (0+0)\vx + (-0\cdot\vx)
        && \text{by definition of $+$ additive identity element}
      \\&= 0\vx + (-0\vx)
        && \text{by \pref{def:vspace} property 4}
      \\&= \vzero
        && \text{by definition of $+$ additive identity element}
      %
    \intertext{2. Proof that $\alpha\vzero=\vzero$:}
      \alpha\vzero
        &= \alpha\brp{0\vx}
        && \text{by item 1}
      \\&= \brp{\alpha0}\vx
        && \text{by \pref{def:vspace} property 6}
      \\&= 0\vx
      \\&= \vzero
        && \text{by item 1}
      %
    \intertext{3. Proof that $\alpha\neq0$ and $\vx\neq\vzero$ $\implies$ $\alpha\vx\neq\vzero$: Suppose $\alpha\vx=\vzero$. Then}
      \vx 
        &= \brp{\frac{1}{\alpha}\alpha}\vx
      \\&= \frac{1}{\alpha}\brp{\alpha\vx}
      \\&= \frac{1}{\alpha}\vzero
      \\&= \vzero
        && \text{by item 2}
      \\&\implies \vx=\vzero 
      \\&\text{This is a \hie{contradiction} and so $\alpha\vx\neq\vzero$.}
      %  
    \intertext{4. Proof that $\alpha\vx=\vzero$ $\implies$ $\alpha=0$ or $\vx=\vzero$: contrapositive argument of item 3}
      %  
    \intertext{5. Proof that $\vx+\vx=\vx\implies\vx=\vzero$:}
      \vx
        &= \vx + \vzero
        && \text{by \prope{additive identity} property \xref{def:vspace}}
      \\&= \vx + \brs{\vx+(-\vx)}
        && \text{by \prope{additive inverse} property \xref{def:vspace}}
      \\&= \brs{\vx + \vx}+(-\vx)
        && \text{by \prope{associative} property \xref{def:vspace}}
      \\&= \vx+(-\vx)
        && \text{by left hypothesis}
      \\&= \vzero
        && \text{by \prope{additive inverse} property \xref{def:vspace}}
  \end{align*}
  
\end{proof}

%---------------------------------------
\begin{definition}
\footnote{
  \citerpgc{berberian1961}{7}{0821819127}{Definition~1}
  }
\label{def:vs_addinv}
\index{additive inverse}
%--------------------------------------
Let $\spO\eqd\linearspaceX$ be a linear space with vectors $\vx,\vy\in\setX$.
Let $-\vy$ be the additive inverse of $\vy$ such that $\vy+(-\vy)=\vzero$.
\defbox{\begin{array}{M}
  The \hid{difference} of $\vx$ and $\vy$ is $\vx+(-\vy)$ and is denoted
  \\\qquad$\vx-\vy$.
\end{array}}
\end{definition}

%---------------------------------------
\begin{theorem}[Additive inverse properties]
\footnote{
  \citerpgc{berberian1961}{7}{0821819127}{Corollary 1},
  \citerpg{michel1993}{77}{048667598X},
  %\citerp{prasad}{13} \\
  \citorc{peano1888}{Chapter IX},
  \citorpp{peano1888e}{119}{120},
  \citorpp{banach1922}{134}{135}
  }
\index{additive inverse}
\label{thm:vs_addinv}
%--------------------------------------
Let $\spO\eqd\linearspaceX$ be a linear space,
   $\vzero$ the \structe{additive identity element} with respect to $+$,
and $-\vx$ the \structe{additive inverse} \xref{def:group} of $\vx$ with respect to $+$.
\thmbox{\begin{array}{Flcl@{\qquad}C@{\qquad}D}
 %\\1.& (-1)\vx           &=& -\vx                          & \forall \vx\in\setX
    1.& \vx+\vy           &=& \vzero \qquad \implies \vx = -\vy & \forall \vx,\vy\in\setX                 & (additive inverse is \prope{unique})
  \\2.& (-\alpha)\vx      &=& -(\alpha\vx) = \alpha(-\vx)       & \forall \vx\in\setX,\,\alpha\in\F       & 
  \\3.& \alpha(\vx-\vy)   &=& \alpha\vx - \alpha\vy             & \forall \vx,\vy\in\setX,\,\alpha\in\F   & (\prope{distributive})
  \\4.& (\alpha-\beta)\vx &=& \alpha\vx - \beta\vx              & \forall \vx\in\setX,\,\alpha,\beta\in\F & (\prope{distributive})
\end{array}}
\end{theorem}
\begin{proof}
\begin{enumerate}
  %\item Proof that $(-1)\vx=-\vx$:
  %  \begin{align*}
  %    \vx + [(-1)\vx]
  %      &= [1\vx] + [(-1)\vx]
  %      && \text{by definition of $\otimes$ multiplicative identity element}
  %    \\&= (1-1)\vx
  %      && \text{by \pref{def:vspace} property 4}
  %    \\&= 0\vx
  %      && \text{by property of $+$ additive inverse element}
  %    \\&= \vzero
  %      && \text{by previous result 1.}
  %  \end{align*} 
  \item Proof that $\vx+\vy=0 \implies \vx=-\vy$: \label{item:vs_addinv_unique}
    \begin{align*}
      \vx
        &= \vx - \vzero
      \\&= \vx - (\vx+\vy)
        && \text{by left hypothesis}
      \\&= (\vx-\vx)-\vy
      \\&= \vzero - \vy
      \\&= -\vy
    \end{align*}
    
  \item Proof that $(-\alpha)\vx=-(\alpha\vx)$: \label{item:vs_addinv_minus1}
    \begin{align*}
      \vzero
        &= 0\vx
        && \text{by \prefp{thm:vs_addid}}
      \\&= (\alpha-\alpha)\vx
        && \text{by field property of $\F$}
      \\&= \brs{\alpha+(-\alpha)}\vx
        && \text{by field property of $\F$}
      \\&= \alpha\vx + (-\alpha)\vx
        && \text{by \prefp{def:vspace}}
      \\\implies -(\alpha\vx) &= (-\alpha)\vx
        && \text{by \prefp{item:vs_addinv_unique}}
    \end{align*}

  \item Proof that $\alpha(-\vx)=-(\alpha\vx)$: \label{item:vs_addinv_minus2}
    \begin{align*}
      \vzero
        &= \alpha\vzero
        && \text{by \prefp{thm:vs_addid}}
      \\&= \alpha\brs{\vx+(-\vx)}
        && \text{by definition of additive identity element $-\vx$}
      \\&= \alpha\vx + \alpha(-\vx)
        && \text{by \prefp{def:vspace}}
      \\&= \alpha\vx + \alpha(-\vx)
      \\\implies -(\alpha\vx) &= \alpha(-\vx)
        && \text{by \prefp{item:vs_addinv_unique}}
    \end{align*}

  \item Proof that $\alpha(\vx-\vy) = \alpha\vx - \alpha\vy$:
    \begin{align*}
      \alpha(\vx-\vy)
        &= \alpha\brs{\vx+(-\vy)}
        && \text{by \prefp{def:vs_addinv}}
      \\&= \alpha\vx+ \alpha(-\vy)
        && \text{by \prefp{def:vspace}}
      \\&= \alpha\vx+ (-\alpha\vy)
        && \text{by \prefp{item:vs_addinv_minus2}}
      \\&= \alpha\vx-\alpha\vy
        && \text{by \prefp{def:vs_addinv}}
    \end{align*}

  \item Proof that $(\alpha-\beta)\vx = \alpha\vx - \beta\vx$: 
    \begin{align*}
      (\alpha-\beta)\vx
        &= \brs{\alpha+(-\beta)}\vx
        && \text{by field properties of $\F$}
      \\&= \alpha\vx + (-\beta)\vx
        && \text{by \pref{def:vspace}}
      \\&= \alpha\vx + \brs{-(\beta\vx)}
        && \text{by \prefp{item:vs_addinv_minus1}}
      \\&= \alpha\vx -(\beta\vx)
        && \text{by \prefp{def:vs_addinv}}
    \end{align*}

\end{enumerate}
\end{proof}


%--------------------------------------
\begin{theorem}
\footnote{
  \citerpg{michel1993}{77}{048667598X},
  %\citerp{prasad}{13},
  \citorc{peano1888}{Chapter IX},
  \citorpp{peano1888e}{119}{120},
  \citorpp{banach1922}{134}{135}
  }
\index{additive identity}
\index{additive inverse}
%--------------------------------------
Let $\linearspaceX$ be a linear space,
$\vzero$ the additive identity element with respect to $+$,
and $-\vx$ additive inverse of $\vx$ with respect to $+$.
\thmbox{
  \begin{array}{Fllrcl@{\qquad}C}
      1. & \alpha\vx = \alpha\vy \text{ and } \alpha\ne 0     &\implies&    \vx    &=& \vy   & \forall \vx,\vy\in\setX
    \\2. & \alpha\vx = \beta\vx \text{ and } \vx\ne \vzero    &\implies&    \alpha &=& \beta & \forall \vx,\vy\in\setX,\; \alpha,\beta\in\F
    \\3. & \vz+\vx=\vz+\vy                                    &\implies&    \vx    &=& \vy   & \forall \vx,\vy,\vz\in\setX
   %\\4. & \vx_1+\vy=\vz \text{\quad and\quad} \vx_2+\vy=\vz  &\implies&    \vx_1  &=&\vx_2  & \forall \vx_1,\vx_2,\vy,\vz\in\setX
  \end{array}
  }
\end{theorem}
\begin{proof}
\begin{align*}
  \intertext{1. Proof that $\alpha\vx = \alpha\vy$ and $\alpha\ne 0$ $\implies \vx = \vy$:}
    \vzero &= \frac{1}{\alpha}(\vzero)
      && \text{by left hypothesis ($\alpha\ne 0$)}
    \\&= \frac{1}{\alpha}\Big( \alpha\vx - \alpha\vy \Big)
      && \text{by left hypothesis ($\alpha\vx = \alpha\vy$)}
    \\&= \frac{1}{\alpha}\alpha\Big( \vx - \vy \Big)
      && \text{by \prefp{def:vspace}}
    \\&= \vx - \vy
  %
  \intertext{2. Proof that $\alpha\vx = \beta\vx$ and $\vx\ne \vzero$ $\implies \alpha = \beta$:}
    \vzero
      &= \alpha\vx + (-\alpha\vx)
      && \text{by definition of additive inverse}
    \\&= \beta\vx + (-\alpha\vx)
      && \text{by left hypothesis}
    \\&= \beta\vx + (-\alpha)\vx
      && \text{by \prefp{thm:vs_addinv}}
    \\&= \brs{\beta+(-\alpha)}\vx
      && \text{by \prefp{def:vspace}}
    \\\implies \beta-\alpha&=0
      && \text{by \prefp{thm:vs_addid}}
    \\\implies \alpha&=\beta
      && \text{by field properties of $\F$}
  %
  \intertext{3. Proof that $\vz+\vx=\vz+\vy \implies \vx = \vy$: }
    \vzero
      &= (\vz+\vx)-(\vz+\vy)
    \\&= (\vx+\vz)-(\vz+\vy)
      && \text{by \pref{def:vspace} property 1}
    \\&= (\vx+\vz)+[(-1)\vz+(-1)\vy]
      && \text{by \pref{def:vspace} property 3}
    \\&= (\vx+\vz)+(-\vz-\vy)
      && \text{by previous result 2.}
    \\&= \vx+(\vz-\vz)-\vy
    \\&= \vx-\vy
  %
  %\intertext{4. Proof that $\vx_1+\vy=\vz$ and $\vx_2+\vy=\vz$ $\implies$ $\vx_1=\vx_2$:}
  %    \vx_1+\vy &= \vx_2+\vy                 && \text{by item 3}
  %  \\(\vx_1+\vy)+(-\vy) &= (\vx_2+\vy)+(-\vy) && 
  %  \\\vx_1+(\vy-\vy) &= \vx_2+(\vy-\vy) && 
  %  \\\vx_1+\vzero   &= \vx_2+\vzero   && 
  %  \\\vx_1          &= \vx_2          && 
\end{align*}
\end{proof}




%======================================
\section{Order on Linear Spaces}
%======================================
%--------------------------------------
\begin{definition}
\label{def:lsp_order}
\footnote{
  \citerppg{ab2006}{1}{2}{1402050070}
 }
%--------------------------------------
Let $\spO\eqd\RlinearspaceX$ be a real linear space.
\defbox{\begin{array}{>{\qquad\scriptstyle}rlcl@{\qquad}CF}
  \mc{5}{M}{The pair $\opair{\spO}{\orel}$ is an \hid{ordered linear space} if}
  \\
    1. & \vx\orel\vy &\implies& \vx+\vz\orel\vy+\vz     & \forall \vz\in\setX  & and\\
    2. & \vx\orel\vy &\implies& \alpha\vx\orel\alpha\vy & \forall \alpha\in\F
  \\
  \mc{5}{M}{A vector $\vx$ is \hid{positive} if $\vzero\orel\vx$.}
  \\
  \mc{5}{M}{The \hid{positive cone} $\spX^+$ of $\opair{\spX}{\orel}$ is the set $\spXp\eqd\set{\vx\in\spX}{\vzero\orel\vx}$.}
  %\\
  %The ordered linear space $\opair{\spX}{\orel}$ may also be represented as 
  %$\lattice{\spX}{\orel}{\join}{\meet}$,
  %where $\join$ is the greatest lower bound induced by $\orel$, and
  %$\meet$ is the least upper bound induced by $\orel$.
  \end{array}}
\end{definition}

%--------------------------------------
\begin{definition}
\footnote{
  \citerpg{ab2006}{2}{1402050070}
 }
\label{def:lsp_lattice}
\label{def:rieszspace}
%--------------------------------------
Let $\opair{\spX}{\orel}$ be an ordered linear space.
\defbox{\begin{array}{M}
  The tuple $\latL\eqd\lattice{\spX}{\orel}{\join}{\meet}$ is a \hid{Riesz space} if 
  $\latL$ is a \hib{lattice}.\\
  A \structe{Riesz space} is also called a \hid{vector lattice}.
  \end{array}}
\end{definition}

%--------------------------------------
\begin{theorem}
\footnote{
  \citerpgc{ab2006}{3}{1402050070}{Theorem 1.2}
 }
\label{thm:lsp_lat_prop}
%--------------------------------------
Let $\lattice{\spX}{\orel}{\join}{\meet}$ be a Riesz space \xref{def:rieszspace}.
\thmbox{
  \begin{array}{lcl|lcl|C}
       \vx \join \vy &=& -\brs{\brp{-\vx}\meet\brp{-\vy}}              
    &  \vx \meet \vy &=& -\brs{\brp{-\vx}\join\brp{-\vy}}              
    &  \forall \vx,\vy    \in\spX 
    \\ 
       \vx + \brp{\vy\join\vz} &=& \brp{\vx+\vy} \join \brp{\vx+\vz}   
    &  \vx + \brp{\vy\meet\vz} &=& \brp{\vx+\vy} \meet \brp{\vx+\vz}   
    &  \forall \vx,\vy,\vz\in\spX 
    \\ 
       \alpha\brp{\vx\join\vy} &=& \brp{\alpha\vx}\join\brp{\alpha\vy}
    &  \alpha\brp{\vx\meet\vy} &=& \brp{\alpha\vx}\meet\brp{\alpha\vy} 
    & \forall \vx,\vy    \in\spX, \alpha\oreld0 
    \\ 
      \vx + \vy     &=& \brp{\vx \meet \vy} + \brp{\vx\join\vy}
    &&&
    & \forall \vx,\vy    \in\spX, \alpha\in\F 
  \end{array}
  }
\end{theorem}
\begin{proof}
\begin{enumerate}
  \item Proof that $\vx \join \vy = -\brs{\brp{-\vx}\meet\brp{-\vy}}$: \label{item:lsp_lat_prop_xVy}
    %\begin{longtable}{|ONM|ONM|}
    \\$\ds\begin{array}{|rcl|rcl|}
      \hline
      (-\vx) \meet (-\vy) &\orel& -\vx        &     (-\vx) \meet (-\vy) &\orel& -\vy \\
      \vx &\orel& -\brs{(-\vx) \meet (-\vy)}  &   \vy &\orel& -\brs{(-\vx) \meet (-\vy)} \\
      \vx\join \vy &\orel& -\brs{(-\vx) \meet (-\vy)}  &&& \\
      \hline
      \vx                  &\orel& \vx \join \vy      &     \vy                 &\orel& \vx \join \vy\\
      -\brp{\vx \join \vy} &\orel& -\vx               &    -\brp{\vx \join \vy} &\orel& -\vy         \\
      -\brp{\vx \join \vy} &\orel& (-\vx) \meet (-\vy)&                         &     &              \\
      -\brs{(-\vx) \meet (-\vy)} &\orel& \vx \join \vy&                         &     &              \\
      \hline
    %\end{longtable}
    \end{array}$

  \item Proof that $\vx \meet \vy = -\brs{\brp{-\vx}\join\brp{-\vy}}$: \label{item:lsp_lat_prop_xmy}
    \begin{longtable}{|ONM@{\qquad}E|}
      \hline
      \vx\join \vy         &=& -\brs{(-\vx) \meet (-\vy)}        & by \pref{item:lsp_lat_prop_xVy} \\
      (-\vx)\join (-\vy)   &=& -\brs{(-(-\vx)) \meet (-(-\vy))}  & replace $\vx$ with $-\vx$ and $\vy$ with $\-\vy$\\
      (-\vx)\join (-\vy)   &=& -\brs{\vx \meet \vy}              & $-(-\vx)=\vx$\\ 
      -\brs{\vx \meet \vy} &=& (-\vx) \join (-\vy)               & by symmetry of $=$ relation\\
      \vx \meet \vy        &=& -\brs{(-\vx) \join (-\vy)}        & multipy both sides by $-1$\\
      \hline
    \end{longtable}

  \item Proof that $\vx + \brp{\vy\join\vz} = \brp{\vx+\vy} \join \brp{\vx+\vz}$: \label{item:lsp_lat_prop_xyVz}
    \begin{longtable}{|ONM|ONM|}
         \hline
         \vx+\vy &\orel& \vx+\brp{\vy\join\vz}  & \vx+\vz &\orel& \vx+\brp{\vy\join\vz}
      \\ \brp{\vx+\vy} \join \brp{\vx+\vz} &\orel& \vx+\brp{\vy\join\vz} &&&
      \\ \hline
         \vy &=&     -\vx+\brp{\vx+\vy}                         &  \vz &=& -\vx+\brp{\vx+\vz}
      \\     &\orel& -\vx+\brs{\brp{\vx+\vy}\join\brp{\vx+\vz}} &      &\orel& -\vx+\brs{\brp{\vx+\vy}\join\brp{\vx+\vz}} 
      \\ \vy\join\vz  &\orel& -\vx+\brs{\brp{\vx+\vy}\join\brp{\vx+\vz}} &&&
      \\ \vx + \brp{\vy\join\vz}  &\orel& \brp{\vx+\vy}\join\brp{\vx+\vz} &&&
      \\ \hline
    \end{longtable}

  \item Proof that $\vx + \brp{\vy\meet\vz} = \brp{\vx+\vy} \meet \brp{\vx+\vz}$: %\label{item:lsp_lat_prop_xyVz}
    \begin{longtable}{|ONM|ONM|}
         \hline
         \vx+\vy &\oreld& \vx+\brp{\vy\meet\vz}  & \vx+\vz &\oreld& \vx+\brp{\vy\meet\vz}
      \\ \brp{\vx+\vy} \meet \brp{\vx+\vz} &\oreld& \vx+\brp{\vy\meet\vz} &&&
      \\ \hline
         \vy &=&     -\vx+\brp{\vx+\vy}                         &  \vz &=& -\vx+\brp{\vx+\vz}
      \\     &\oreld& -\vx+\brs{\brp{\vx+\vy}\meet\brp{\vx+\vz}} &      &\oreld& -\vx+\brs{\brp{\vx+\vy}\meet\brp{\vx+\vz}} 
      \\ \vy\meet\vz  &\oreld& -\vx+\brs{\brp{\vx+\vy}\meet\brp{\vx+\vz}} &&&
      \\ \vx + \brp{\vy\meet\vz}  &\oreld& \brp{\vx+\vy}\meet\brp{\vx+\vz} &&&
      \\ \hline
    \end{longtable}

  \item Proof that $\alpha\brp{\vx\join\vy} = \brp{\alpha\vx}\join\brp{\alpha\vy}$ for $\alpha\ge0$:
    \begin{longtable}{|ONM|ONM|E|}
        \hline
        \vx       &\orel& \vx\join\vy              & \vy       &\orel& \vx\join\vy             &
      \\\alpha\vx &\orel& \alpha\brp{\vx\join\vy}  & \alpha\vy &\orel& \alpha\brp{\vx\join\vy} & by \prefp{def:lsp_order}
      \\\brp{\alpha\vx} \join \brp{\alpha\vy} &\orel& \alpha\brp{\vx\join\vy}  &&&&
      \\\hline
        \alpha\vx &\orel& \brp{\alpha\vx}\join\brp{\alpha\vy} & \alpha\vy &\orel& \brp{\alpha\vx}\join\brp{\alpha\vy} &
      \\\vx &\orel& \alpha^{-1}\brp{\alpha\vx}\join\brp{\alpha\vy} & \vy &\orel& \alpha^{-1}\brp{\alpha\vx}\join\brp{\alpha\vy}  &
      \\\vx \join \vy &\orel& \alpha^{-1}\brp{\alpha\vx} \join \brp{\alpha\vy} &&&&
      \\\alpha\brp{\vx \join \vy} &\orel& \brp{\alpha\vx} \join \brp{\alpha\vy} &&&&
      \\\hline
    \end{longtable}

  \item Proof that $\alpha\brp{\vx\meet\vy} = \brp{\alpha\vx}\meet\brp{\alpha\vy}$ for $\alpha\ge0$:
    \begin{longtable}{|ONM|ONM|E|}
        \hline
        \vx       &\oreld& \vx\meet\vy             & \vy       &\oreld& \vx\meet\vy               &
      \\\alpha\vx &\oreld& \alpha\brp{\vx\meet\vy} & \alpha\vy &\oreld& \alpha\brp{\vx\meet\vy}   & by \prefp{def:lsp_order}
      \\\brp{\alpha\vx} \meet \brp{\alpha\vy} &\oreld& \alpha\brp{\vx\meet\vy} &&&&
      \\\hline
        \alpha\vx &\oreld& \brp{\alpha\vx}\meet\brp{\alpha\vy} & \alpha\vy &\oreld& \brp{\alpha\vx}\meet\brp{\alpha\vy} &
      \\\vx &\oreld& \alpha^{-1}\brp{\alpha\vx}\meet\brp{\alpha\vy} & \vy &\oreld& \alpha^{-1}\brp{\alpha\vx}\meet\brp{\alpha\vy} &
      \\\vx \meet \vy &\oreld& \alpha^{-1}\brp{\alpha\vx} \meet \brp{\alpha\vy} &&&&
      \\\alpha\brp{\vx \meet \vy} &\oreld& \brp{\alpha\vx} \meet \brp{\alpha\vy} &&&&
      \\\hline
    \end{longtable}

  \item Proof that $\vx+\vy=\brp{\vx \meet \vy} + \brp{\vx\join\vy}$:
    \begin{longtable}{|ONM|ONM|}
      \hline
         \vx &\orel& \vx\join\vy         & \vy &\orel& \vx\join\vy                             
      \\ \vx+\vy &\orel& \brp{\vx\join\vy}+\vy         & \vx+vy &\orel& \vx+\brp{\vx\join\vy} 
      \\ \vx+\vy-\brp{\vx\join\vy} &\orel& \vy         & \vx+vy-\brp{\vx\join\vy} &\orel& \vx
      \\ \vx+\vy-\brp{\vx\join\vy} &\orel& \vx\meet\vy &&&
      \\ \vx+\vy &\orel& \brp{\vx\join\vy} + \brp{\vx\meet\vy} &&&
      \\ \hline
         \vx\meet\vy &\orel& \vx         & \vx\meet\vy &\orel& \vy 
      \\ 0 &\orel& \vx-\brp{\vx\meet\vy} & 0 &\orel& \vy-\brp{\vx\meet\vy}
      \\ \vy &\orel& \vy+\vx-\brp{\vx\meet\vy} & \vx &\orel& \vx+\vy-\brp{\vx\meet\vy}
      \\ \vy &\orel& \vx+\vy-\brp{\vx\meet\vy} & \vx &\orel& \vx+\vy-\brp{\vx\meet\vy}
      \\ \vx\join\vy &\orel& \vx+\vy-\brp{\vx\meet\vy}            &&&
      \\ \brp{\vx\meet\vy} + \brp{\vx\join\vy} &\orel& \vx+\vy    &&&
      \\ \hline
    \end{longtable}


\end{enumerate}
\end{proof}

%--------------------------------------
\begin{definition}
\footnote{
  \citerpg{ab2006}{4}{1402050070},
  \citerpg{istratescu1987}{129}{9027721823}
  %\citerpgc{istratescu1987}{129}{9027721823}{differs from Deza}\\
  }
\label{def:vs_vxp}
\label{def:vs_vxn}
\label{def:vs_vxa}
\index[xsym]{$\vxp$}
\index[xsym]{$\vxn$}
\index[xsym]{$\vxa$}
%--------------------------------------
Let $\lattice{\spX}{\orel}{\join}{\meet}$ be a \structe{Riesz space} \xref{def:rieszspace}.
\defbox{\begin{array}{l@{\quad}M@{\quad}M}
    \vxp & is defined as $\vxp\eqd\vx\join\vzero$         & and is called the \hid{positive part}  of $\vx$.\\
    \vxn & is defined as $\vxn\eqd\brp{-\vx}\join\vzero$  & and is called the \hid{negative part}  of $\vx$. \\
    \vxa & is defined as $\vxa\eqd\vx\join\brp{-\vx}$     & and is called the \hid{absolute value} of $\vx$.
  \end{array}}
\end{definition}

%--------------------------------------
\begin{theorem}
\footnote{
  \citerpgc{ab2006}{4}{1402050070}{Theorem 1.3}
  }
\label{thm:vs_xpxn}
%--------------------------------------
Let $\lattice{\spX}{\orel}{\join}{\meet}$ be a \structe{Riesz space} \xref{def:rieszspace}.
\thmbox{
  \brbr{\begin{array}{rclD}
    \vy-\vz &=& \vx & and \\
    \vy \meet \vz &=& \vzero
  \end{array}}
  \qquad\iff\qquad
  \brbl{\begin{array}{rclD}
    \vy &=& \vxp & and \\
    \vz &=& \vxn
  \end{array}}
  }
\end{theorem}
\begin{proof}
  \begin{enumerate}
    \item Proof that left hypothesis $\implies$ right hypothesis:
      \begin{align*}
        \vxp
          &= \vx \join \vzero
          && \text{by definition of $\vxp$ \prefp{def:vs_vxp}}
        \\&= \brp{\vy-\vz} \join \vzero
          && \text{by left hypothesis}
        \\&= \brp{\vy-\vz} \join \brp{\vz-\vz}
          && 
        \\&= \brp{\vy\join\vz} - \vz
          && \text{by \prefp{thm:lsp_lat_prop}}
        \\&= \brs{\vy+\vz-\brp{\vy\meet\vz}} - \vz
          && \text{by \prefp{thm:lsp_lat_prop}}
        \\&= \vy-\brp{\vy\meet\vz}
          && 
        \\&= \vy-\vzero
          && \text{by left hypothesis}
        \\&= \vy
        \\
        \vxn
          &= \brp{-\vx} \join \vzero
          && \text{by definition of $\vxn$ \prefp{def:vs_vxn}}
        \\&= \brp{\vz-\vy} \join \vzero
          && \text{by left hypothesis}
        \\&= \brp{\vz-\vy} \join \brp{\vy-\vy}
          && 
        \\&= \brp{\vz\join\vy} - \vy
          && \text{by \prefp{thm:lsp_lat_prop}}
        \\&= \brs{\vz+\vy-\brp{\vz\meet\vy}} - \vz
          && \text{by \prefp{thm:lsp_lat_prop}}
        \\&= \vz-\brp{\vz\meet\vy}
          && 
        \\&= \vz-\vzero
          && \text{by left hypothesis}
        \\&= \vz
      \end{align*}

    \item Proof that left hypothesis $\impliedby$ right hypothesis:
      \begin{align*}
        \vy - \vz
          &= \vxp - \vxn
          && \text{by right hypothesis}
        \\&= \brs{\vx\join\vzero} - \brs{\brp{-\vx}\join\vzero}
          && \text{by \prefp{def:vs_vxp}}
        \\&= \brp{\vx\join\vzero} + \brp{\vx\meet{\vzero}}
          && \text{by \prefp{thm:lsp_lat_prop}}
        \\&= \vx
          && \text{by \prefp{thm:lsp_lat_prop}}
        \\
        \vy \meet \vz
          &= \vxp \meet \vxn
          && \text{by right hypothesis}
        \\&= \brs{\vxn + \brp{\vxp - \vxn}} \meet \brs{\vxn + \vzero}
        \\& \vxn + \brs{\brp{\vxp-\vxn} \meet \vzero}
          && \text{by \prefp{thm:lsp_lat_prop}}
        \\& \vxn + \brs{\brp{\vy-\vz} \meet \vzero}
          && \text{by right hypothesis}
        \\& \vxn + \brp{\vx \meet \vzero}
          && \text{by previous result}
        \\& \vxn - \brs{-\vx \join \vzero}
          && \text{by \prefp{thm:lsp_lat_prop}}
        \\& \vxn - \vxn
          && \text{by definition of $\vxn$ \xref{def:vs_vxn}}
        \\& \vzero
      \end{align*}
  \end{enumerate}
\end{proof}

%--------------------------------------
\begin{theorem}
\footnote{
  \citerpg{ab2006}{4}{1402050070}
  }
%--------------------------------------
Let $\lattice{\spX}{\orel}{\join}{\meet}$ be a \structe{Riesz space} \xref{def:rieszspace}.
Let $\vxp$ the \structe{positive part} of $\vx\in\spX$,
$\vxn$ the \structe{negative part} of $\vx\in\spX$, and $\vxa$ the \fncte{absolute value} \xref{def:vs_vxa} of $\vx\in\spX$.
\thmbox{
  \begin{array}{rclCD}
    \vxa   &=& \vxp + \vxn       & \forall \vx\in\spX \\
    \vx    &=& \brp{\vx-\vy}^+ + \brp{\vx\meet\vy}  & \forall \vx\in\spX
  \end{array}
  }
\end{theorem}
\begin{proof}
\begin{align*}
  \vxa
    &= \vx \join \brp{-\vx}
    && \text{by definition of $\vxa$ \xref{def:vs_vxa}}
  \\&= \brp{2\vx-\vx} \join \brp{\vzero-\vx}
  \\&= \brp{-\vx+2\vx} \join \brp{-\vx+\vzero}
    && \text{by \prope{commutative} property \xref{def:vspace}}
  \\&= \brp{-\vx} + \brp{2\vx \join \vzero}
    && \text{by \prefp{thm:lsp_lat_prop}}
  \\&= \brp{2\vx \join 2\vzero} - \vx
    && \text{by the \prope{commutative} property \xref{def:vspace}}
  \\&= 2\brp{\vx \join \vzero} - \vx
    && \text{by \prefp{thm:lsp_lat_prop}}
  \\&= 2\vxp - \vx
    && \text{by definition of $\vxp$ \xref{def:vs_vxp}}
  \\&= 2\vxp - \brp{\vxp - \vxn}
    && \text{by 1}
  \\&= \vxp + \vxn
  \\
  \vx 
    &= \vx + \vzero
    && \vx + \vy - \vy
  \\&= \brp{\vx\join\vy}+\brp{\vx\meet\vy} - \vy
    && \text{by \prefp{thm:lsp_lat_prop}}
  \\&= \brs{\brp{\vx-\vy}\join\brp{\vy-\vy}}+\brp{\vx\meet\vy}
    && \text{by \prefp{thm:lsp_lat_prop}}
  \\&= \brs{\brp{\vx-\vy}\join\vzero}+\brp{\vx\meet\vy}
  \\&= \brp{\vx-\vy}^+ +\brp{\vx\meet\vy}
    && \text{by definition of $\vxp$ \xref{def:vs_vxp}}
\end{align*}
\end{proof}

%--------------------------------------
\begin{theorem}
\footnote{
  \citerpgc{ab2006}{5}{1402050070}{Theorem 1.4}
  }
%--------------------------------------
Let $\lattice{\spX}{\orel}{\join}{\meet}$ be a \structe{Riesz space} \xref{def:rieszspace}.
Let $\vxp$ the \structe{positive part} of $\vx\in\spX$,
$\vxn$ the \structe{negative part} of $\vx\in\spX$, and $\vxa$ the \structe{absolute value} \xref{def:vs_vxa} of $\vx\in\spX$.
\thmbox{
  \begin{array}{>{\scriptstyle}rrcl@{\qquad}C}
    1. & \vx\join\vy &=& \frac{1}{2}\brp{\vx+\vy+\abs{\vx-\vy}}       & \forall \vx,\vy\in\spX \\
    2. & \vx\meet\vy &=& \frac{1}{2}\brp{\vx+\vy-\abs{\vx-\vy}}       & \forall \vx,\vy\in\spX \\
    3. & \abs{\vx-\vy} &=& \brp{\vx\join\vy} - \brp{\vx\meet\vy}      & \forall \vx,\vy\in\spX \\
    4. & \abs{\vx}\join\abs{\vy} &=& \frac{1}{2}\brp{\abs{\vx+\vy}+\abs{\vx-\vy}} & \forall \vx,\vy\in\spX \\
    5. & \vxa\join\vya &=& \frac{1}{2}\abs{\abs{\vx+\vy}-\abs{\vx-\vy}} & \forall \vx,\vy\in\spX 
  \end{array}
  }
\end{theorem}
\begin{proof}
\begin{align*}
  \brp{\vx+\vy+\abs{\vx-\vy}}
    &= \brp{\vx+\vy} + \brs{\brp{\vx-\vy}\join\brp{\vy-\vx}}
   %&& \text{by definition of $\vxa$ \xref{def:vs_vxa}}
    && \text{by \prefp{def:vs_vxa}}
  \\&= \brs{\brp{\vx+\vy} + \brp{\vx-\vy}} \join \brs{\brp{\vx+\vy}+\brp{\vy-\vx}}
    && \text{by \prefp{thm:lsp_lat_prop}}
  \\&= \brp{2\vx} \join \brp{2\vy}
  \\&= 2\brp{\vx\join\vy}
    && \text{by \prefp{thm:lsp_lat_prop}}
  \\
  \brp{\vx+\vy-\abs{\vx-\vy}}
    &= \brp{\vx+\vy} - \brs{\brp{\vx-\vy}\join\brp{\vy-\vx}}
   %&& \text{by definition of $\vxa$ \xref{def:vs_vxa}}
    && \text{by \prefp{def:vs_vxa}}
  \\&= \brp{\vx+\vy} - \brs{\brp{-\brp{\vy-\vx}}\join\brp{-\brp{\vx-\vy}}}
  \\&= \brp{\vx+\vy} + \brs{\brp{\vy-\vx}\meet\brp{\vx-\vy}}
    && \text{by \prefp{thm:lsp_lat_prop}}
  \\&= \brs{\brp{\vx+\vy}+\brp{\vy-\vx}}  \meet \brs{\brp{\vx+\vy}+\brp{\vx-\vy}}
    && \text{by \prefp{thm:lsp_lat_prop}}
  \\&= \brp{2\vy}  \meet \brp{2\vx}
  \\&= 2\brp{\vy\meet\vx}
    && \text{by \prefp{thm:lsp_lat_prop}}
  \\&= 2\brp{\vx\meet\vy}
  \\
  \abs{\vx-\vy}
    &= \frac{1}{2}\brp{\vx+\vy+\abs{\vx-\vy}} - \frac{1}{2}\brp{\vx+\vy+\abs{\vx-\vy}}
  \\&= \brp{\vx\join\vy} - \brp{\vx\meet\vy}
    && \text{by 1 and 2}
  \\
  \abs{\vx+\vy}+\abs{\vx-\vy}
    &= \frac{1}{2}\brp{\vzero + \abs{2\vx+2\vy}} + \abs{\vx-\vy}
  \\&= \frac{1}{2}\brs{\brp{\vx+\vy}+\brp{-\vx-\vy} + \abs{\brp{\vx+\vy}-\brp{-\vx-\vy}}} + \abs{\vx-\vy}
  \\&= \brs{\brp{\vx+\vy}\join\brp{-\vx-\vy}} + \abs{\vx-\vy}
    && \text{by 1}
  \\&= \brs{\brp{\vx+\vy}+\abs{\vx-\vy}}\join\brs{\brp{-\vx-\vy}+\abs{\vx-\vy}}
    && \text{by \prefp{thm:lsp_lat_prop}}
  \\&= 2\brp{\vx\join\vy} \join 2\brs{\brp{-\vy}+\brp{-\vx}+\abs{\brp{-\vy}-\brp{-\vx}}}
    && \text{by 1}
  \\&= 2\brp{\vx\join\vy} \join 2\brs{\brp{-\vy}\join\brp{-\vx}}
    && \text{by 1}
  \\&= 2\brp{\brs{\vx\join\brp{-\vx}}\join \brp{\vy\join\brp{-\vy}}}
  \\&= 2\brp{\abs{\vx}\join\abs{\vy}}
   %&& \text{by definition of $\vxa$ \xref{def:vs_vxa}}
    && \text{by \prefp{def:vs_vxa}}
  \\
  \abs{\abs{\vx+\vy}-\abs{\vx-\vy}}
    &= 2\brp{\abs{\vx+\vy}\join\abs{\vx-\vy}}-\brp{\abs{\vx+\vy}+\abs{\vx-\vy}}
    && \text{by 1}
  \\&= \brp{\abs{\vx+\vy+\vx-\vy}+\abs{\vx+\vy-\vx+\vy}}-2\brp{\vxa\join\vya}
    && \text{by 3}
  \\&= 2\brp{\abs{\vx}+\abs{\vy}}-2\brp{\vxa\join\vya}
  \\&= 2\brp{\vxa\join\vya}
    && \text{by \prefp{thm:lsp_lat_prop}}
\end{align*}
\end{proof}

%--------------------------------------
\begin{definition}
\label{def:vs_disjoint}
\footnote{
  \citerpg{ab2006}{5}{1402050070}
  }
\index[xsym]{$\perp$}
%--------------------------------------
Let $\lattice{\spX}{\orel}{\join}{\meet}$ be a \structe{Riesz space} \xref{def:rieszspace}.
Let $\vxp$ the \structe{positive part} of $\vx\in\spX$,
$\vxn$ the \structe{negative part} of $\vx\in\spX$, and $\vxa$ the \structe{absolute value} \xref{def:vs_vxa} of $\vx\in\spX$.
\defbox{\begin{array}{M}
  {$\vx$ and $\vy$ are \hid{disjoint}, denoted by $\vx\perp\vy$, if }
  \\\qquad$\abs{\vx}\meet\abs{\vy}=\vzero$.
  \\
  {Two subsets $\spU$ and $\spV$ of $\spX$ are \hid{disjoint}, denoted by $\spU\perp\spV$ if}
  \\\qquad$\vx\perp\vy \qquad \forall \vx\in\spU\text{ and }\vy\in\spV$
  \end{array}}
\end{definition}

%--------------------------------------
\begin{definition}
\label{def:vs_disjointc}
\footnote{
  \citerpg{ab2006}{5}{1402050070}
  }
\index[xsym]{$d$}
%--------------------------------------
Let $\lattice{\spX}{\orel}{\join}{\meet}$ be a \structe{Riesz space} \xref{def:rieszspace}.
Let $\vxp$ the \structe{positive part} of $\vx\in\spX$,
$\vxn$ the \structe{negative part} of $\vx\in\spX$, and $\vxa$ the \structe{absolute value} \xref{def:vs_vxa} of $\vx\in\spX$.
Let $\spY$ be a subset of $\spX$.
\defbox{\begin{array}{M}
  $\spY^d$ is the \hid{disjoint complement} of $\spY$ if \qquad$\spY^d \eqd \set{\vx\in\spX}{\vx\perp\vy \quad\forall \vy\in\spY}$.
  \\The quantity $\spY^{dd}$ is defined as $\brp{\spY^d}^d$.
  \end{array}}
\end{definition}

%--------------------------------------
\begin{definition}
\footnote{
  \citerpg{ab2006}{7}{1402050070}
  }
%--------------------------------------
Let $\lattice{\spX}{\orel}{\join}{\meet}$ be a \structe{Riesz space} \xref{def:rieszspace}.
Let $\vxp$ the \structe{positive part} of $\vx\in\spX$,
$\vxn$ the \structe{negative part} of $\vx\in\spX$, and $\vxa$ the \structe{absolute value} \xref{def:vs_vxa} of $\vx\in\spX$.
\defbox{
  \begin{array}{rcl}
    \abs{\setA}     &\eqd& \set{\abs{\va}}{\va\in\setA}  \\
    \setA^+         &\eqd& \set{\va^+}{\va\in\setA}  \\
    \setA^-         &\eqd& \set{\va^-}{\va\in\setA}  \\
    \setA\join\setB &\eqd& \set{\va\join\vb}{\va\in\setA\text{ and }\vb\in\setB}  \\
    \setA\meet\setB &\eqd& \set{\va\meet\vb}{\va\in\setA\text{ and }\vb\in\setB}  \\
    \vx\join\setA   &\eqd& \set{\vx\join\va}{\va\in\setA}  \\
    \vx\meet\setA   &\eqd& \set{\vx\meet\va}{\va\in\setA}  
  \end{array}
  }
\end{definition}


