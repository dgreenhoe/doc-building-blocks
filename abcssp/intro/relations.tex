%============================================================================
% Daniel J. Greenhoe
% LaTeX file
%============================================================================
%=======================================
\subsection{Relations}
%=======================================



%In the set $\setn{a,b}$, the order in which the elements are listed
%does \emph{not} matter. That is, $\setn{a,b}$ is equivalent to $\setn{b,a}$.
%However, in structures as fundamental as \structe{relation}s and \structe{function}s,
%order does (very much) matter.
%To help with this, we have the concept of the \structe{ordered pair}, % (next definition),
%in which the order the elements are listed \emph{is} significant.
%%That is, $\opair{a}{b}$ is equivalent to $(b,a)$ if and only if $a=b$.
%%The ordered pair can be defined in different ways.
One of the most fundamental structures in mathematics is the \structe{ordered pair},
and one of the most common definitions of \structe{ordered pair} is due to
%mathematician Kazimierz Kuratowski \index{Kuratowski, Kazimierz}
\hie{Kuratowski} (1921) and is presented next:%
\footnote{%\emph{Alternative ordered pair definition}:
  As an alternative to the Kuratowski definition, 
  the \structe{ordered pair} can also be taken as an \emph{axiom}. References: 
      \citerp{bourbaki_tos}{72},
      \citerp{munkres2000}{13}
  }
%---------------------------------------
\begin{definition}
\footnote{
  \citerpg{suppes1972}{32}{0486616304},
  \citerpg{halmos1960}{23}{0387900926},
  %\citerp{menini2004}{39} \\
  %\citerp{wolf}{164}  \\
  \citerp{kuratowski_stt}{39},
  \citePc{kuratowski1921}{Def. V, page 171},
  \citeP{wiener1914}
  %\citer{veblen1904} -- I did not find any reference to ordered pairs in this paper (42 pages long) \\
  }
\label{def:(a,b)}
\label{def:opair}
%---------------------------------------
\defboxt{
  \text{The \structd{ordered pair} $\opair{a}{b}$ is defined as }
  \qquad
  $\opair{a}{b} \eqd \setn{ \setn{a},\setn{a,b} }$. }
\end{definition}

\pref{prop:(a,b)_extract} (next) and \pref{cor:(a,b)=(c,d)} demonstrate that the 
the definition of \structe{ordered pair} given by \pref{def:(a,b)}
allows $a$ and $b$ to be unambiguously extracted from $\opair{a}{b}$ 
and that $\opair{a}{b}$ is well defined.
%\prefpp{def:(a,b)} defined the \structe{ordered pair} using Kuratowski's
%somewhat cryptic expression.
% %$\opair{a}{b} \eqd \setn{\setn{a},\, \setn{a,b}}$.
%\pref{prop:(a,b)_extract} (next) helps show that this expression is reasonable
%in that $a$ and $b$ can be extracted from $\opair{a}{b}$
%and \pref{cor:(a,b)=(c,d)} further demonstrates that $\opair{a}{b}$ is well defined.
%%from the set $\setn{\; \setn{a},\; \setn{a,b}\; }$
%%by simple set operations on its elements.
%%Also, a key test to the usefulness of the definition of ordered pairs is whether
%%or not $\opair{a}{b}=\opair{c}{d}$ if and only if $a=c \text{ and } b=d$.
%%In fact, this statement is true and demonstrated by \prefpp{cor:(a,b)=(c,d)}.
%---------------------------------------
\begin{proposition}
\label{prop:(a,b)_extract}
%---------------------------------------
\thmbox{\begin{array}{rcl clcl}
   \setn{a} &=&  \setopi\opair{a}{b} &=& \setopi\setn{ \setn{a},\setn{a,b} } &=& \setn{a}\seti\setn{a,b} \\
   \setn{b} &=&  \setops\opair{a}{b} &=& \setops\setn{ \setn{a},\setn{a,b} } &=& \setn{a}\sets\setn{a,b}
\end{array}}
\end{proposition}

%---------------------------------------
\begin{corollary}
\label{cor:(a,b)=(c,d)}
\footnote{
  \citerpg{apostol1975}{33}{0201002884},
  \citerpg{hausdorff1937e}{15}{0828401195}
  }
%---------------------------------------
\corbox{
  \opair{a}{b}=(c,d)
  \qquad\iff\qquad
  \brb{a=c \text{ and } b=d}
  }
\end{corollary}
\begin{proof}
$\begin{array}[t]{lclclclM}
  \setn{a}     &=& \Seti\opair{a}{b} &=& \Seti(c,d) &=& \setn{c} & {by \pref{prop:(a,b)_extract} and left hypothesis}\\
  \setn{b}     &=& \Sets\opair{a}{b} &=& \Sets(c,d) &=& \setn{d} & {by \pref{prop:(a,b)_extract} and left hypothesis}\\
  \opair{a}{b} &=& \opair{c}{d}      & &            & &          & {by right hypothesis}
\end{array}$
%\begin{align*}
%  %\intertext{1. Proof that $\opair{a}{b}=(c,d) \implies a=c \text{ and } b=d$:}
%    \setn{a}
%      &= \Seti\opair{a}{b}
%      && \text{by \prefp{thm:(a,b)_extract}}
%    \\&= \Seti(c,d)
%      && \text{by left hypothesis}
%    \\&= \setn{c}
%      && \text{by \prefp{thm:(a,b)_extract}}
%    \\
%    \setn{b}
%      &= \Sets\opair{a}{b}
%      && \text{by \prefp{thm:(a,b)_extract}}
%    \\&= \Sets(c,d)
%      && \text{by left hypothesis}
%    \\&= \setn{d}
%      && \text{by \prefp{thm:(a,b)_extract}}
%    \\
%  %\intertext{2. Proof that $\opair{a}{b}=(c,d) \impliedby a=c \text{ and } b=d$:}
%    \opair{a}{b}
%      &= (c,d)
%      && \text{by right hypothesis}
%\end{align*}
\end{proof}



%%---------------------------------------
%\begin{minipage}{\tw-40mm}
%\begin{definition}
%\label{def:rel_and}
%%---------------------------------------
%Let $\setX$ and $\setY$ be sets, $\ltrue$ denote the logical property of ``true", 
%and $\lfals$ the logical property of ``false". % \xref{ax:plogic}.
%\defboxp{
%  The \fnctd{relational and} $\owedge$ is a \structe{set} defined as
%  \\\indentx
%  $\owedge \eqd 
%    \setn{
%      \opair{\opair{\lfals}{\lfals}}{\lfals},\;
%      \opair{\opair{\lfals}{\ltrue}}{\lfals},\;
%      \opair{\opair{\ltrue}{\lfals}}{\lfals},\;
%      \opair{\opair{\ltrue}{\ltrue}}{\ltrue}
%      }$
%    }
%\end{definition}
%The set $\owedge$ is illustrated in the table to the right.
%\end{minipage}\hfill
%$\begin{tabstr}{0.75}\begin{array}{|cc|c|}
%      \hline
%      x & y & x\owedge y  \\
%      \hline
%      \lfals & \lfals & \lfals \\
%      \lfals & \ltrue & \lfals \\
%      \ltrue & \lfals & \lfals \\
%      \ltrue & \ltrue & \ltrue \\
%      \hline
%  \end{array}\end{tabstr}$

%---------------------------------------
\begin{definition}
\footnote{
  \citerpg{halmos1960}{24}{0387900926},
  G. Frege, 2007 August 25, \url{http://groups.google.com/group/sci.logic/msg/3b3294f5ac3a76f0}
  }
\label{def:AxB}
\label{def:cprod}
%---------------------------------------
Let $\setX$ and $\setY$ be sets. 
%, and let $\owedge$ be the ``relational and" relation of \prefpp{def:rel_and}.
\defboxt{
  The \structd{Cartesian product} $\cprodXY$ is defined as
 %\\\indentx$\ds \cprodXY \eqd \set{\opair{x}{y}}{(x\in\setX)\; \owedge \; (y\in\setY)}$
  \\\indentx$\ds \cprodXY \eqd \set{\opair{x}{y}}{(x\in\setX)\; \text{and} \; (y\in\setY)}$
  }
\end{definition}


%\mbox{}\hfill%
%  \begin{tabstr}{0.5}
%  \begin{tabular}{|cc|cc|}
%    \hline
%      \includegraphics{../common/math/graphics/pdfs/2xyfnct1.pdf}
%    & \includegraphics{../common/math/graphics/pdfs/2xyfnct2.pdf}
%    & \includegraphics{../common/math/graphics/pdfs/2xynotfnct1.pdf}
%    & \includegraphics{../common/math/graphics/pdfs/2xynotfnct2.pdf}
%    \\\mc{2}{|c|}{\scs two \structe{relation}s in $\clRxy$ that \emph{are} \structe{function}s}
%     &\mc{2}{c|}{\scs two \structe{relation}s in $\clRxy$ that are \emph{not} \structe{function}s}
%    %\\  a relation     & a relation     & a relation         & a relation
%    %\\  that \emph{is} & that \emph{is} & that is \emph{not} & that is \emph{not}
%    %\\  a function     & a function     & a function         & a function
%    \\\hline
%  \end{tabular}\end{tabstr}%
%\hfill\mbox{}\\
%---------------------------------------
\begin{definition}
\footnote{
  \citerpg{maddux2006}{4}{0444520139},
  \citerppg{halmos1960}{26}{30}{0387900926},
  \citerpg{suppes1972}{86}{0486616304},
  \citerpg{kelley1955}{10}{0387901256},
  %\citerp{kbr}{161}\\
  \citer{bourbaki1939},
  \citerpg{bottazzini}{7}{0387963022},
  \citerpgc{comtet1974}{4}{9027704414}{$\seto{\clFxy}$};
  The notation $\clRxy$ is motivated by the fact that for finite $\setX$ and $\setY$,
  $\seto{\clRxy}=2^{\seto{\setX}\cdot\seto{\setY}}$.
  }
\label{def:clRxy}
\label{def:relation}
%---------------------------------------
Let $\setX$ and $\setY$ be \structe{sets}.
\defboxp{
%The \structd{Cartesian product} $\setX\times\setY$ of $\setX$ and $\setY$ is the set
%  $\setX\times\setY \eqd \set{\opair{x}{y}}{\text{$x\in\setX$ and $y\in\setY$}}$.
%An \structd{ordered pair} $\opair{x}{y}$ on $\setX$ and $\setY$ is any element in $\setX\times\setY$.
A \structd{relation} $\rel$ on $\setX$ and $\setY$ is any subset of $\setX\times\setY$ such that $\rel\subseteq\setX\times\setY$.
The set $\clRxy$ is the \structd{set of all relations} in $\setX\times\setY$.
}
\end{definition}


%\mbox{}\hfill$\begin{tabstr}{0.75}\begin{array}[t]{l@{\qquad}l@{\qquad}l}
%   \begin{array}[t]{|cc|c|}
%      %\hline
%      %\mc{3}{|Z|}{``implies" or ``only if"}\\
%      %\mc{3}{|c|}{\implies:\setft^2\to\setft}\\
%      \hline
%      x & y & x\implies y  \\
%      \hline
%      \lfals & \lfals & \ltrue \\
%      \lfals & \ltrue & \ltrue \\
%      \ltrue & \lfals & \lfals \\
%      \ltrue & \ltrue & \ltrue \\
%      \hline
%   \end{array}
%&
%   \begin{array}[t]{|cc|c|}
%      %\hline
%      %\mc{3}{|Z|}{``implied by" or ``if"}\\
%      %\mc{3}{|c|}{\impliedby:\setft^2\to\setft}\\
%      \hline
%      x & y & x\impliedby y  \\
%      \hline
%      \lfals & \lfals & \ltrue \\
%      \lfals & \ltrue & \lfals \\
%      \ltrue & \lfals & \ltrue \\
%      \ltrue & \ltrue & \ltrue \\
%      \hline
%   \end{array}
%&
%   \begin{array}[t]{|cc|c|}
%      %\hline
%      %\mc{3}{|Z|}{``if and only if"}\\
%      %\mc{3}{|c|}{\iff:\setft^2\to\setft}\\
%      \hline
%      x & y & x\iff y  \\
%      \hline
%      \lfals & \lfals & \ltrue \\
%      \lfals & \ltrue & \lfals \\
%      \ltrue & \lfals & \lfals \\
%      \ltrue & \ltrue & \ltrue \\
%      \hline
%   \end{array}
%\end{array}\end{tabstr}$\hfill\mbox{}\\
%%---------------------------------------
%\begin{definition}[logic order relation]
%\label{def:==>}
%\label{def:implies}
%\label{def:onlyif}
%%---------------------------------------
%Let $\ltrue$ and $\lfalse$ represent the logical properties of ``\prope{true}" and
%``\prope{false}", respectively. % \xref{ax:plogic}.
%\defboxp{\indxs{\implies}\index{logical operations!implies}
%  The \fnctd{implies}, or \fnctd{only if}, relation $\implies$ is defined as
%  \\\indentx$\implies\quad\eqd\quad\setn{\opair{\opair{\lfals}{\lfals}}{\ltrue},\; \opair{\opair{\lfals}{\ltrue}}{\ltrue},\; \opair{\opair{\ltrue}{\lfals}}{\lfals},\; \opair{\opair{\ltrue}{\ltrue}}{\ltrue}}$
%  }
%\end{definition}

%---------------------------------------
\begin{definition}
\footnote{
  \citerpg{maddux2006}{4}{0444520139},
  \citerppg{halmos1960}{26}{30}{0387900926},
  \citerpg{suppes1972}{86}{0486616304},
  \citerpg{kelley1955}{10}{0387901256},
  %\citerp{kbr}{161}\\
  \citer{bourbaki1939},
  \citerpg{bottazzini}{7}{0387963022},
  \citerpgc{comtet1974}{4}{9027704414}{$\seto{\clFxy}$};
  The notation $\clFxy$ is motivated by the fact that for finite $\setX$ and $\setY$,
  $\seto{\clFxy}=\seto{\setY}^\seto{\setX}$.
  }
\label{def:clFxy}
\label{def:function}
%---------------------------------------
Let $\setX$ and $\setY$ be \structe{sets}.
\defboxt{
A \structe{relation} $\ff\in\clRxy$ is a \structd{function} if
\\\indentx 
$\brb{\opair{x}{y_1}\in\ff\quad\text{and}\quad\opair{x}{y_2}\in\ff}
   \quad\implies\quad
   \brb{y_1=y_2}$.
\\
The set $\clFxy$ is the \structd{set of all functions} in $\clRxy$. %\footnotemark
}
\end{definition}





%\mbox{}\hfill$\begin{tabstr}{0.75}\begin{array}{l*{3}{@{\qquad}l}}
%   \begin{array}[t]{|c|c|}
%      \hline
%      %\mc{2}{|c|}{\cellcolor{blue}\text{\bfseries\color{white}\ope{logical not}}}\\\hline
%      x & \lnot x  \\
%      \hline
%       \lfals & \ltrue \\
%       \ltrue & \lfals \\
%      \hline
%   \end{array}
%&
%   \begin{array}[t]{|cc|c|}
%      \hline
%      %\mc{3}{|c|}{\cellcolor{blue}\text{\bfseries\color{white}\ope{logical or}}}\\\hline
%      x & y & x\lor y  \\
%      \hline
%      \lfals & \lfals & \lfals \\
%      \lfals & \ltrue & \ltrue \\
%      \ltrue & \lfals & \ltrue \\
%      \ltrue & \ltrue & \ltrue \\
%      \hline
%   \end{array}
%&
%   \begin{array}[t]{|cc|c|}
%      \hline
%      %\mc{3}{|c|}{\cellcolor{blue}\text{\bfseries\color{white}\ope{logical and}}}\\\hline
%      x & y & x\land y  \\
%      \hline
%      \lfals & \lfals & \lfals \\
%      \lfals & \ltrue & \lfals \\
%      \ltrue & \lfals & \lfals \\
%      \ltrue & \ltrue & \ltrue \\
%      \hline
%   \end{array}
%&
%   \begin{array}[t]{|cc|c|}
%      \hline
%      %\mc{3}{|c|}{\cellcolor{blue}\text{\bfseries\color{white}\ope{logical exclusive-or}}}\\\hline
%      x & y & x\lxor y  \\
%      \hline
%      \lfals & \lfals & \lfals \\
%      \lfals & \ltrue & \ltrue \\
%      \ltrue & \lfals & \ltrue \\
%      \ltrue & \ltrue & \lfals \\
%      \hline
%   \end{array}
%\end{array}\end{tabstr}$\hfill\mbox{}\\
%%---------------------------------------
%\begin{definition}[logic functions]
%\label{def:flogic}
%%\label{def:lnot}
%%\label{def:land}
%%\label{def:lor}
%%\label{def:lxor}
%\index{propositional logic}
%\index{logic!propositional}
%\index{logical operation}
%\index{logical operation!AND}
%\index{logical operation!OR}
%\index{logical operation!NOT}
%\index{logical operation!XOR}
%%---------------------------------------
%Let $\ltrue$ and $\lfalse$ represent the logical properties of ``\prope{true}" and 
%``\prope{false}", respectively. % \xref{ax:plogic}.
%The following logic functions are defined according to the expressions below:
%\defbox{\begin{array}{rclD}
%      \lnot &\eqd& \setn{\opair{\lfals}{\ltrue},\; \opair{\ltrue}{\lfals}}                                                                                                                     & (\prope{logical not})
%    \\\lor  &\eqd& \setn{\opair{\opair{\lfals}{\lfals}}{\lfals},\; \opair{\opair{\lfals}{\ltrue}}{\ltrue},\; \opair{\opair{\ltrue}{\lfals}}{\ltrue},\; \opair{\opair{\ltrue}{\ltrue}}{\ltrue}} & (\prope{logical or})
%    \\\land &\eqd& \setn{\opair{\opair{\lfals}{\lfals}}{\lfals},\; \opair{\opair{\lfals}{\ltrue}}{\lfals},\; \opair{\opair{\ltrue}{\lfals}}{\lfals},\; \opair{\opair{\ltrue}{\ltrue}}{\ltrue}} & (\prope{logical and})
%    \\\lxor &\eqd& \setn{\opair{\opair{\lfals}{\lfals}}{\lfals},\; \opair{\opair{\lfals}{\ltrue}}{\ltrue},\; \opair{\opair{\ltrue}{\lfals}}{\ltrue},\; \opair{\opair{\ltrue}{\ltrue}}{\lfals}} & (\prope{logical exclusive or})
%  \end{array}}
%\end{definition}
%
%
%%---------------------------------------
%\begin{definition}
%\label{def:<==}
%\label{def:<==>}
%\label{def:impliedby}
%\label{def:if}
%%\label{def:iff}
%%---------------------------------------
%Let $\ltrue$ and $\lfalse$ represent the logical properties of ``\prope{true}" and
%``\prope{false}", respectively.
%\defbox{
%  \indxs{\impliedby}\indxs{\iff}\index{logical operations!implied by}\index{logical operations!if and only if}
%  \begin{array}{lclcl@{\qquad}C}
%    (x \impliedby y) & \eqd & (y \implies x) &       &                  & \forall x,y\in\setft \\
%    (x \iff y)       & \eqd & (x \implies y) & \land & (y \implies x)   & \forall x,y\in\setft
%  \end{array}}
%\end{definition}


%Note that the ``logical and" \emph{function} $\land$ as defined in \pref{def:flogic} (previous)
%and the ``\prope{relational and} "\emph{relation} $\owedge$ as defined in \prefpp{def:AxB} 
%are equivalent relations.

%%---------------------------------------
%\begin{definition}
%\label{def:subset}
%%---------------------------------------
%\defboxp{
%  A set $\setA$ is a \propd{subset} of a set $\setX$, expressed $\setA\subseteq\setX$ if
%  \quad$x\in\setA\;\implies\; x\in\setX$.
%  }
%\end{definition}

A {function} does not always have an inverse that is also a function.
But unlike functions, \emph{every} relation has an inverse that is also a relation.
Note that since all functions are relations,
every function \emph{does} have an inverse that is at least a relation,
and in some cases this inverse is also a function.
%---------------------------------------
\begin{definition}
\footnote{
  \citerpgc{suppes1972}{61}{0486616304}{Defintion 6, inverse=``converse"}\\
  \citerpg{kelley1955}{7}{0387901256}\\
  \citorpc{peirce1883}{188}{inverse=``converse"}
  }
\index{relation!inverse}
\label{def:rel_inverse}
\label{def:inverse}
%---------------------------------------
Let $\rel$ be a relation in $\clRxy$.
\defboxp{
  $\symxd{\reli}$ is the \relxd{inverse} of relation $\rel$ if
  \\\indentx
  $\reli\eqd\set{\opair{y}{x}\in\cprodYX}{\opair{x}{y}\in\rel}$
  \\
  The inverse relation $\reli$ is also called the \relxd{converse} of $\rel$.
}
\end{definition}


%---------------------------------------
\begin{definition}
\label{def:pset}
%---------------------------------------
Let $\setX$ be a set.
\defboxt{
  The quantity $\psetx$ is the \structe{power set of $\setX$} such that
  \\\indentx$\begin{array}{rclD}
      \psetx &\eqd& \setn{\setA\subseteq\setX} & (the set of all subsets of $\setX$).
  \end{array}$
  }
\end{definition}


%$\begin{array}{|*{5}{c|}}
%  \hline
%   \includegraphics[width=\tw/4-5mm]{../common/graphics/math/venn_union.pdf}
%  &\includegraphics[width=\tw/4-5mm]{../common/graphics/math/venn_intersect.pdf}
%  &\includegraphics[width=\tw/4-5mm]{../common/graphics/math/venn_setdiff.pdf}
%  &\includegraphics[width=\tw/4-5mm]{../common/graphics/math/venn_symdiff.pdf}
% %&\includegraphics[width=\tw/4-5mm]{../common/graphics/math/venn_cmpA.pdf}
%  \\
%   \setA\setu\setB
%  &\setA\seti\setB
%  &\setA\setd\setB
%  &\setA\sets\setB
% %&\cmpA
%  \\\hline
%\end{array}$
%%---------------------------------------
%\begin{definition}
%\label{def:set_ops}
%\footnote{
%  \citerppg{verescagin2002}{1}{2}{0821827316},
%  \citerpp{ab}{2}{4},
%  Origin of $\setu$ and $\seti$: \citer{peano1888}, \citer{peano1888e}
%  }
%\index{sets!operations}
%%---------------------------------------
%Let $\psetX$ be the \structe{power set} \xref{def:pset} on a set $\setX$.
%\defboxp{\indxs{\setu}\indxs{\seti}\indxs{\setd}\indxs{\sets}
%   \\\indentx$\begin{array}{l>{\eqd}cl@{\qquad}C@{\qquad}D}
%   \setA\setu\setB
%     && \set{x\in\setX}{ (x\in\setA) \lor  (x\in\setB) }
%     & \forall \setA,\setB\in\psetX
%     &(\opd{union})
%     \\
%   \setA\seti\setB
%     && \set{x\in\setX}{ (x\in\setA) \land (x\in\setB) }
%     & \forall \setA,\setB\in\psetX
%     &(\opd{intersection})
%   \\
%   \setA\setd\setB
%     && \set{x\in\setX}{ (x\in\setA) \land \lnot(x\in\setB) }
%     & \forall \setA,\setB\in\psetX
%     &(\opd{difference})
%     \\
%   \setA\sets\setB
%     && \set{x\in\setX}{(x\in\setA) \lxor (x\in\setB)}
%     & \forall \setA,\setB\in\psetX
%     &(\opd{symmetric difference})
%     \\
%   \cmpA
%     && \set{x\in\setX}{\lnot(x\in\setA)  }
%     & \forall \setA\in\psetX
%     &(\hid{complement})
%   %\cmpA
%   %  && \setX\setd\setA
%   %  & \forall \setA\in\psetX
%   %  &(\opd{complement})
%\end{array}$\\
%Moreover, 
%  $\setopu\setn{\setA,\setB}\eqd\setA\setu\setB$, 
%  $\setopi\setn{\setA,\setB}\eqd\setA\seti\setB$, and 
%  $\setops\setn{\setA,\setB}\eqd\setA\sets\setB$.
%}
%\end{definition}


%\prefpp{prop:(a,b)_extract} and \pref{cor:(a,b)=(c,d)} demonstrate that the 
%the definition of \structe{ordered pair} given by \prefpp{def:(a,b)}
%allows $a$ and $b$ to be unambiguously extracted from $\opair{a}{b}$ 
%and that $\opair{a}{b}$ is well defined.
%%\prefpp{def:(a,b)} defined the \structe{ordered pair} using Kuratowski's
%%somewhat cryptic expression.
%% %$\opair{a}{b} \eqd \setn{\setn{a},\, \setn{a,b}}$.
%%\pref{prop:(a,b)_extract} (next) helps show that this expression is reasonable
%%in that $a$ and $b$ can be extracted from $\opair{a}{b}$
%%and \pref{cor:(a,b)=(c,d)} further demonstrates that $\opair{a}{b}$ is well defined.
%%%from the set $\setn{\; \setn{a},\; \setn{a,b}\; }$
%%%by simple set operations on its elements.
%%%Also, a key test to the usefulness of the definition of ordered pairs is whether
%%%or not $\opair{a}{b}=\opair{c}{d}$ if and only if $a=c \text{ and } b=d$.
%%%In fact, this statement is true and demonstrated by \prefpp{cor:(a,b)=(c,d)}.
%%---------------------------------------
%\begin{proposition}
%\label{prop:(a,b)_extract}
%%---------------------------------------
%\thmbox{\begin{array}{rcl clcl}
%   \setn{a} &=&  \setopi\opair{a}{b} &=& \setopi\setn{ \setn{a},\setn{a,b} } &=& \setn{a}\seti\setn{a,b} \\
%   \setn{b} &=&  \setops\opair{a}{b} &=& \setops\setn{ \setn{a},\setn{a,b} } &=& \setn{a}\sets\setn{a,b}
%\end{array}}
%\end{proposition}
%
%%---------------------------------------
%\begin{corollary}
%\label{cor:(a,b)=(c,d)}
%\footnote{
%  \citerpg{apostol1975}{33}{0201002884},
%  \citerpg{hausdorff1937e}{15}{0828401195}
%  }
%%---------------------------------------
%\corboxp{$
%  \opair{a}{b}=(c,d)
%  \qquad\iff\qquad
%  \brb{a=c \text{ and } b=d}
%  $}
%\end{corollary}
%\begin{proof}
%$\begin{array}[t]{lclclclM}
%  \setn{a}     &=& \Seti\opair{a}{b} &=& \Seti(c,d) &=& \setn{c} & {by \pref{prop:(a,b)_extract} and left hypothesis}\\
%  \setn{b}     &=& \Sets\opair{a}{b} &=& \Sets(c,d) &=& \setn{d} & {by \pref{prop:(a,b)_extract} and left hypothesis}\\
%  \opair{a}{b} &=& \opair{c}{d}      & &            & &          & {by right hypothesis}
%\end{array}$
%%\begin{align*}
%%  %\intertext{1. Proof that $\opair{a}{b}=(c,d) \implies a=c \text{ and } b=d$:}
%%    \setn{a}
%%      &= \Seti\opair{a}{b}
%%      && \text{by \prefp{thm:(a,b)_extract}}
%%    \\&= \Seti(c,d)
%%      && \text{by left hypothesis}
%%    \\&= \setn{c}
%%      && \text{by \prefp{thm:(a,b)_extract}}
%%    \\
%%    \setn{b}
%%      &= \Sets\opair{a}{b}
%%      && \text{by \prefp{thm:(a,b)_extract}}
%%    \\&= \Sets(c,d)
%%      && \text{by left hypothesis}
%%    \\&= \setn{d}
%%      && \text{by \prefp{thm:(a,b)_extract}}
%%    \\
%%  %\intertext{2. Proof that $\opair{a}{b}=(c,d) \impliedby a=c \text{ and } b=d$:}
%%    \opair{a}{b}
%%      &= (c,d)
%%      && \text{by right hypothesis}
%%\end{align*}
%\end{proof}

%%---------------------------------------
%\begin{definition}
%\label{def:C}
%%---------------------------------------
%Let $\R$ be the \structe{set of real numbers}.
%\defboxp{The \structd{set of complex numbers} $\Cnum$ is defined as $\Cnum=\R\times\R$.}
%\end{definition}

%---------------------------------------
\begin{definition}
\label{def:Xn}
%---------------------------------------
Let $\setY$ be a \structe{set}.
\defboxt{The structure $\setY^n$ for $n\in\Zp$ is a \structe{set} defined as
  \\\indentx$\begin{array}{lclMD}
    \setY^1 &\eqd& \setY                  &  & and\\
    \setY^n &\eqd& \setY\times\setY^{n-1} & for $n=2,3,4,\ldots$
  \end{array}$
  }
\end{definition}

%---------------------------------------
\begin{definition}
\label{def:C}
%---------------------------------------
\defboxt{
  The \structd{set of complex numbers} $\Cnum$ is defined as $\Cnum\eqd\R^2$.
  }
\end{definition}

%---------------------------------------
\begin{definition}
\label{def:ntuple}
\label{def:tuple}
\label{def:triple}
\label{def:otriple}
%---------------------------------------
Let $\setY_1$, $\setY_2$, \ldots $\setY_\xN$ be \structe{sets}.
\defboxp{
  The structure $\oquad{x_1}{x_2}{\cdots}{x_n}$ is an \structd{n-tuple} on 
  $\setY_1\times\setY_2\times\cdots\times\setY_\xN$ if 
  $\oquad{x_1}{x_2}{\cdots}{x_n}$ is an element in the set
  $\setY_1\times\setY_2\times\cdots\times\setY_\xN$.
  A \structe{3-tuple} is also called an \structd{ordered triple} 
  or simply a \structd{triple}.
  %An n-tuple is denoted in this paper in the form $\ds\tuplexn{x_n}$ or simply as $\ds\tuplen{x_n}$.
  }
\end{definition}

%---------------------------------------
\begin{definition}
\footnote{
  \citerp{munkres2000}{16},
  \citerpg{kelley1955}{7}{0387901256}
  }
\label{def:range}
\label{def:domain}
\label{def:image}
\label{def:null}
%---------------------------------------
Let $\rel\in\clRxy$ be a \structe{relation} \xref{def:relation}.
\defbox{%
  \begin{array}{Mrcl}
    {The \relxd{domain} of $\rel$ is }
      & \symxd{\oppD}(\rel) &\eqd& \set{x\in\setX}{\exists y \st \opair{x}{y}\in\rel}.
      \\
    {The \relxd{image set} of $\rel$ is}
      & \symxd{\oppI}(\rel) &\eqd& \set{y\in\setY}{\exists x \st \opair{x}{y}\in\rel}.
      \\
    {The \relxd{null space} of $\rel$ is}
      & \symxd{\oppN}(\rel) &\eqd& \set{x\in\setX}{\opair{x}{0}\in\rel}.
      \\
    {The \relxd{range} of $\rel$ is}
      & \mc{3}{l}{\text{any set $\symxd{\oppR}(\rel)$ such that $\oppI(\rel)\subseteq\oppR(\rel)$.}}
  \end{array}
  }
\end{definition}

%---------------------------------------
\begin{definition}
\label{def:seto}
%---------------------------------------
%Let $\psetx$ be the \structe{power set} \xref{def:pset} of a set $\setX$.
\defboxt{
  The \fncte{set function}\footnotemark
  $\seto{\setA}\in\clF{\psetx}{\Zext}$ is the \fncte{cardinality of $\setA$} such that
  \\\indentx$\seto{\setA}\eqd\brb{\begin{array}{MM}
      the number of elements in $\setA$ & for \prope{finite} $\setA$\\
      $\infty$                          & otherwise
  \end{array}}\qquad\forall\setA\in\psetx$
  }
\footnotetext{\structe{set function}:
  \citerpgc{pap1995}{8}{0792336585}{Definition 2.3: extended real-valued set function},
  \citerpgc{halmos1950}{30}{0387900888}{\textsection7. {\scshape measure on rings}}
  }
\end{definition}

\begin{figure}[h]
  \gsize%
  \centering%
  \psset{unit=5mm}%
  %%============================================================================
% Daniel J. Greenhoe
% XeLaTeX / LaTeX file
% types of functions
%============================================================================
{\psset{
  labelsep=2.8mm,
  }
\begin{tabular*}{\tw}{c@{\extracolsep\fill}ccc}
\begin{pspicture}(-1,0.5)(3,5)
  \psellipse[linecolor=red,linewidth=1pt](0,2.5)(0.5,2)
  \psellipse[linecolor=red,linewidth=1pt](2,2.5)(0.5,2)
  %
  \Cnode*(0,4){x4}\Cnode*(2,4){y4}%
  \Cnode*(0,3){x3}\Cnode*(2,3){y3}%
  \Cnode*(0,2){x2}\Cnode*(2,2){y2}%
  \Cnode*(0,1){x1}\Cnode*(2,1){y1}%
  %
  \uput{3.5mm}[90]{0}(x4){$\setX$}
  \uput{3.5mm}[90]{0}(y4){$\setY$}
  %
  \uput[180]{0}(x4){$x_4$}\uput[0]{0}(y4){$y_4$}%
  \uput[180]{0}(x3){$x_3$}\uput[0]{0}(y3){$y_3$}%
  \uput[180]{0}(x2){$x_2$}\uput[0]{0}(y2){$y_2$}%
  \uput[180]{0}(x1){$x_1$}\uput[0]{0}(y1){$y_1$}%
  %
  \ncline[linecolor=blue]{->}{x4}{y3}%
  \ncline[linecolor=blue]{->}{x3}{y4}%
  \ncline[linecolor=blue]{->}{x2}{y3}%
  \ncline[linecolor=blue]{->}{x1}{y2}%
\end{pspicture}
&
\begin{pspicture}(-1,0.5)(3,5)
  \psellipse[linecolor=red,linewidth=1pt](0,2.5)(0.5,2)
  \psellipse[linecolor=red,linewidth=1pt](2,2.5)(0.5,2)
  %
  \Cnode*(0,4){x4}\pnode(2,4){y4}%
  \Cnode*(0,3){x3}\Cnode*(2,3.5){y3}%
  \Cnode*(0,2){x2}\Cnode*(2,2.5){y2}%
  \Cnode*(0,1){x1}\Cnode*(2,1.5){y1}%
  %
  \uput{3.5mm}[90]{0}(x4){$\setX$}
  \uput{3.5mm}[90]{0}(y4){$\setY$}
  %
  \uput[180]{0}(x4){$x_4$}%
  \uput[180]{0}(x3){$x_3$}\uput[0]{0}(y3){$y_3$}%
  \uput[180]{0}(x2){$x_2$}\uput[0]{0}(y2){$y_2$}%
  \uput[180]{0}(x1){$x_1$}\uput[0]{0}(y1){$y_1$}%
  %
  \ncline[linecolor=blue]{->}{x4}{y2}
  \ncline[linecolor=blue]{->}{x3}{y3}
  \ncline[linecolor=blue]{->}{x2}{y1}
  \ncline[linecolor=blue]{->}{x1}{y1}
\end{pspicture}
&
\begin{pspicture}(-1,0.5)(3,5)
  \psellipse[linecolor=red,linewidth=1pt](0,2.5)(0.5,2)
  \psellipse[linecolor=red,linewidth=1pt](2,2.5)(0.5,2)
  %
  \pnode (0,4)  {x4}\Cnode*(2,4){y4}%
  \Cnode*(0,3.5){x3}\Cnode*(2,3){y3}%
  \Cnode*(0,2.5){x2}\Cnode*(2,2){y2}%
  \Cnode*(0,1.5){x1}\Cnode*(2,1){y1}%
  %
  \uput{3.5mm}[90]{0}(x4){$\setX$}
  \uput{3.5mm}[90]{0}(y4){$\setY$}
  %
                          \uput[0]{0}(y4){$y_4$}%
  \uput[180]{0}(x3){$x_3$}\uput[0]{0}(y3){$y_3$}%
  \uput[180]{0}(x2){$x_2$}\uput[0]{0}(y2){$y_2$}%
  \uput[180]{0}(x1){$x_1$}\uput[0]{0}(y1){$y_1$}%
  %
  \ncline[linecolor=blue]{->}{x3}{y3}
  \ncline[linecolor=blue]{->}{x2}{y4}
  \ncline[linecolor=blue]{->}{x1}{y2}
\end{pspicture}
&
\begin{pspicture}(-1,0.5)(3,5)
  \psellipse[linecolor=red,linewidth=1pt](0,2.5)(0.5,2)
  \psellipse[linecolor=red,linewidth=1pt](2,2.5)(0.5,2)
  %
  \Cnode*(0,4){x4}\Cnode*(2,4){y4}%
  \Cnode*(0,3){x3}\Cnode*(2,3){y3}%
  \Cnode*(0,2){x2}\Cnode*(2,2){y2}%
  \Cnode*(0,1){x1}\Cnode*(2,1){y1}%
  %
  \uput{3.5mm}[90]{0}(x4){$\setX$}
  \uput{3.5mm}[90]{0}(y4){$\setY$}
  %
  \uput[180]{0}(x4){$x_4$}\uput[0]{0}(y4){$y_4$}%
  \uput[180]{0}(x3){$x_3$}\uput[0]{0}(y3){$y_3$}%
  \uput[180]{0}(x2){$x_2$}\uput[0]{0}(y2){$y_2$}%
  \uput[180]{0}(x1){$x_1$}\uput[0]{0}(y1){$y_1$}%
  %
  \ncline[linecolor=blue]{->}{x4}{y3}
  \ncline[linecolor=blue]{->}{x3}{y4}
  \ncline[linecolor=blue]{->}{x2}{y2}
  \ncline[linecolor=blue]{->}{x1}{y1}
\end{pspicture}
\\  ``\prope{into}"   & ``\prope{onto}"    & ``\prope{one-to-one}"    & ``\prope{one-to-one and onto}"
\\  (arbitrary function in $\clFxy$) & \prope{surjective} in $\clFxy$ & \prope{injective} in $\clFxy$        & \prope{bijective} in $\clFxy$
\end{tabular*}}%
  \begin{tabular*}{\tw}{c@{\extracolsep\fill}ccc}
     \includegraphics{../common/math/graphics/pdfs/function_x4y4.pdf}
    &\includegraphics{../common/math/graphics/pdfs/surjective_x4y4.pdf}
    &\includegraphics{../common/math/graphics/pdfs/injective_x4y4.pdf}
    &\includegraphics{../common/math/graphics/pdfs/bijective_x4y4.pdf}
    \\  ``\prope{into}"   & ``\prope{onto}"    & ``\prope{one-to-one}"    & ``\prope{one-to-one and onto}"
    \\  (arbitrary function in $\clFxy$) & \prope{surjective} in $\clFxy$ & \prope{injective} in $\clFxy$        & \prope{bijective} in $\clFxy$
  \end{tabular*}%
  \caption{types of functions \xref{def:ftypes}}
\end{figure}
%---------------------------------------
\begin{definition}
\footnote{
  \citerppg{michel1993}{14}{15}{048667598X},
  \citerpg{fuhrmann2012}{2}{1461403375},
  \citerpg{comtet1974}{5}{9027704414},
  \citerppg{durbin2000}{16}{17}{0471321478}
  }
\label{def:ftypes}
\label{def:injective}
\label{def:bijective}
%---------------------------------------
Let $\ff$ be a \structe{function} in $\clFxy$ \xref{def:clFxy}.
\defbox{\begin{array}{MMM}
   $\ff$ is \hid{surjective} & (also called \hid{onto})                   & {if \quad$\ff(\setX)=\setY$}.  \\
   $\ff$ is \hid{injective}  & (also called \hid{one-to-one})             & {if \quad$\ff(x) = \ff(y) \implies x=y$}. \\
   $\ff$ is \hid{bijective}  & (also called \textbf{one-to-one and onto}) & {if \quad$\ff$ is both \prope{surjective} and \prope{injective}}.
\end{array}}
\end{definition}






