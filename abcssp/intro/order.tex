%============================================================================
% Daniel J. Greenhoe
% XeLaTeX file
% graph theory
%============================================================================
%=======================================
\subsection{Order}
%=======================================
%---------------------------------------
\begin{definition}
\footnote{
  %\citerpp{menini2004}{44}{45} \\
  \citerpg{maclane1999}{470}{0821816462},
  \citerpg{beran1985}{1}{902771715X},
  %\citerpg{halmos1960}{54}{0387900926}\\
  %\citorp{kuratowski1921}{164}\\
  \citePpc{korselt1894}{156}{I, II, (1)},
  \citePpc{dedekind1900}{373}{I--III}.
  %\citorc{harriot1631}{\S 1}
  An \structe{order relation} is also called a \reld{partial order relation}.
  An \structe{ordered set} is also called a \structd{partially ordered set} or \structd{poset}.
  }
\label{def:orel}
\label{def:oset}
\label{def:poset}
\label{def:order}
\label{def:comparable}
\label{def:incomparable}
\label{def:unordered}
%---------------------------------------
Let $\setX$ be a set. 
\defboxp{\indxs{\orel}
  A relation $\orel$ is an \reld{order relation} in $\clRxx$ \xref{def:clRxy} if
  \\\indentx$\begin{array}{Fl@{\qquad}C@{\qquad}DD@{}r@{}D}
    \cline{6-6}
    1. & x \orel x
       & \forall x\in\setX
       & (\prope{reflexive})
       & and \hspace{2ex}
       & \text{ }\vline
       & \text{\hspace{2ex}\rele{preorder}}
       \\
    2. & x \orel y \text{ and } y \orel z \implies x \orel z
       & \forall x,y,z\in\setX
       & (\prope{transitive})
       & and
       & \vline
    \\\cline{6-6}
    3. & x \orel y \text{ and } y \orel x \implies x=y
       & \forall x,y\in\setX
       & (\prope{anti-symmetric}).
       & 
       & 
  \end{array}$
  \\
  An \structd{ordered set} is the pair $\opair{\setX}{\orel}$.
  If $\orel=\emptyset$ \xref{def:emptyset}, then $\opair{\setX}{\orel}$ is an \structd{unordered set}.
  %If $\orel$ is the \structe{empty set}, then $\opair{\setX}{\orel}$ is an \structd{unordered set}.
  %The set $\setX$ is called the \hid{base set} of $\opair{\setX}{\orel}$.
  If $x\orel y$ or $y\orel x$, then elements $x$ and $y$ are said to be \propd{comparable}; %, denoted $x\sim y$.
  otherwise they are \propd{incomparable}. %, denoted $x||y$.
  %The relation $<$ is the relation $\orel\setd=$ (``less than but not equal to"), 
  %where $\setd$ is the \ope{set difference} operator, and $=$ is the equality relation.
}
  %\footnotetext{%
  %An \structe{order relation} is also called a \reld{partial order relation}.
  %An \structe{ordered set} is also called a \structd{partially ordered set} or \structd{poset}.
  %}
\end{definition}

%---------------------------------------
\begin{definition}
\footnote{
  \citerpg{maclane1999}{470}{0821816462},
  \citePp{ore1935}{410}
  }
\label{def:toset}
\label{def:chain}
\index{ordered set!totally}
\index{ordered set!linearly}
%---------------------------------------
\defboxt{
  A relation $\orel$ is a \hid{linear order relation} on $\setX$ if
  \\\indentx$\begin{array}{FlCDD}
    1. & \mc{2}{M}{$\orel$ is an \structe{order relation}}      & (\prefp{def:orel})    & and \\
    2. & x\orel y \text{ or } y\orel x  & \forall x,y\in\setX   & (\prope{comparable}).
  \end{array}$
  \\
  A \structd{linearly ordered set} is the pair $\opair{\setX}{\orel}$.\\
  A linearly ordered set is also called a \structd{totally ordered set}, 
  a \structd{fully ordered set}, and a \structd{chain}.
  }
\end{definition}

The familiar relations $\oreld$, $<$, and $>$ (next) 
can be defined in terms of the 
order relation $\orel$ (\pref{def:orel}---previous).
%---------------------------------------
\begin{definition}
\footnote{
  \citePp{peirce1880ajm}{2}
  }
\label{def:<}
\label{def:>}
\label{def:ge}
\label{def:oreld}
%---------------------------------------
Let $\opair{\setX}{\orel}$ be an ordered set.
\defboxt{
  The relations $\symxd{\oreld},\,\symxd{<},\,\symxd{>}\in\clRxx$ are defined as follows:
  \\\indentx$\begin{array}{lclMlC}
    x\oreld y & \iffdef & y\orel x &     &         & \forall x,y\in\setX\\
    x\lneq y       & \iffdef & x\orel  y & and & x\ne y  & \forall x,y\in\setX\\
    x\gneq y       & \iffdef & x\oreld y & and & x\ne y  & \forall x,y\in\setX
  \end{array}$
  \\
  The relation $\oreld$ is called the \relxd{dual} of $\orel$.
  }
%\ifdochas{relation}{\footnotetext{
%  \relxe{inverse} relation: \prefp{def:rel_inverse}
%  }}
\end{definition}

%---------------------------------------
\begin{example}[\exmd{Coordinatewise order relation}]
\footnote{
  \citerpg{shen2002}{43}{0821827316}
  }
\label{ex:order_coordinatewise}
\index{order relations!coordinatewise}
%---------------------------------------
Let $\opair{\setX}{\orel}$ be an ordered set.\\
Let $\vx\eqd\oquad{x_1}{x_2}{\ldots}{x_n}$ and $\vy\eqd\oquad{y_1}{y_2}{\ldots}{y_n}$.
\exbox{\begin{array}{M}
  The \hid{coordinatewise order relation} $\orela$ on the Cartesian product $\setX^n$
  \\is defined for all $\vx,\vy\in\setX^n$ as
  \\\indentx
  $\vx\orela\vy
  \qquad\iffdef\qquad
  \brb{ x_1\orel y_1 \text{ and } x_2\orel y_2 \text{ and } \ldots \text{ and } x_n\orel y_n }$
\end{array}}
\end{example}

%---------------------------------------
\begin{example}[\exmd{Lexicographical order relation}]
\footnote{
  \citerpg{shen2002}{44}{0821827316},
  \citerpg{halmos1960}{58}{0387900926},
  \citerpg{hausdorff1937e}{54}{0828401195}
  }
\label{ex:order_lex}
\index{order relations!lexicographical}
\index{order relations!dictionary}
\index{order relations!alphabetic}
%---------------------------------------
Let $\opair{\setX}{\orel}$ be an ordered set.\\
Let $\vx\eqd\oquad{x_1}{x_2}{\ldots}{x_n}$ and $\vy\eqd\oquad{y_1}{y_2}{\ldots}{y_n}$.
\exbox{\begin{array}{M}
  The \hid{lexicographical order relation} $\orela$ on the Cartesian product $\setX^n$
  \\is defined forall $\vx,\vy\in\setX^n$ as
  \\
  $\vx\orela\vy
  \iffdef
  \left\{\begin{array}{>{\big(}l l >{$}l<{$} l l<{\big)} >{$}l<{$} }
      & x_1 < y_1 &     &                                     & & or
    \\& x_2 < y_2 & and & x_1 = y_1                           & & or
    \\& x_3 < y_3 & and & \opair{x_1}{x_2} = \opair{y_1}{y_2} & & or
    \\& \ldots        & \ldots & \ldots                           & & or 
    \\& x_{n-1} < y_{n-1} & and & \oquad{x_1}{x_2}{\ldots}{x_{n-2}} = \oquad{y_1}{y_2}{\ldots}{y_{n-2}} & & or
    \\& x_{n} \leq y_{n} & and & \oquad{x_1}{x_2}{\ldots}{x_{n-1}} = \oquad{y_1}{y_2}{\ldots}{y_{n-1}} & &
  \end{array}\right\}$
  \\The lexicographical order relation is also called the \hid{dictionary order relation} 
  \\or \hid{alphabetic order relation}.
\end{array}}
\end{example}



%---------------------------------------
\begin{definition}
\footnote{
  \citerpg{apostol1975}{4}{0201002884},
  \citePp{ore1935}{409},
  \citePp{duthie1942}{2}, %{$a$=\hie{lower extremity}, $b$=\hie{upper extremity}}\\
  \citePpc{ore1935}{425}{\hie{quotient structures}}
  }
\label{def:intxx}
\label{def:intcc}
\label{def:intoc}
\label{def:intco}
\label{def:intoo}
\label{def:intcc_order}
%---------------------------------------
\defboxt{\indxs{\intccn}\indxs{\intoon}\indxs{\intocn}\indxs{\intcon}
  In an \structe{ordered set} $\osetX$,
  \\\indentx$\begin{array}{MlclMD}
    the \structe{set} & \intcc{x}{y} &\eqd& \set{z\in\setX}{x\le z\le y} & is a  \structd{closed interval}    on $\osetX$ & and \\
    the \structe{set} & \intoc{x}{y} &\eqd& \set{z\in\setX}{x<   z\le y} & is a  \structd{half-open interval} on $\osetX$ & and \\
    the \structe{set} & \intco{x}{y} &\eqd& \set{z\in\setX}{x\le z<   y} & is a  \structd{half-open interval} on $\osetX$ & and \\
    the \structe{set} & \intoo{x}{y} &\eqd& \set{z\in\setX}{x<   z<   y} & is an \structd{open interval}.     on $\osetX$ & 
  \end{array}$
  }
\end{definition}

%%--------------------------------------
%\begin{example}
%%--------------------------------------
%Examples of \structe{open interval}s of $\Z$ include
%$\emptyset$, $\Z$, $\Znn$, $\Zp$, $\setn{0,1,2,3,4,5}$, and $\setn{-2,-1,0,1,2,3}$.
%\end{example}

%--------------------------------------
\begin{definition}
\label{def:abs}
%--------------------------------------
Let $\osetR$ be the \structe{ordered set of real numbers} \xref{def:oset}.
\\\defboxt{
  The \fnctd{absolute value} $\absn\in\clFrr$ is defined as\footnotemark
  \qquad$\abs{x}\eqd
    \brb{\begin{array}{rM}
     -x & for $x\orel0$\\
      x & otherwise
    \end{array}}$.
  }
\end{definition}
\footnotetext{
  A more general definition for \fncte{absolute value} is available for any \structe{commutative ring}:
  Let $R$ be a \structe{commutative ring}. %\xref{def:com_ring}.
  A function $\absn$ in $\clF{R}{R}$ is an \fnctd{absolute value}, or \fnctd{modulus}, on $R$ if
    \\\indentx
    $\begin{array}{F rclCDD}
      1. & \abs{ x}  &\ge& 0                     & x  \in\R & (\prope{non-negative})                              & and 
    \\2. & \abs{ x}  &=  & 0 \iff x=0            & x  \in\R & (\prope{nondegenerate})                             & and 
    \\3. & \abs{xy}  &=  & \abs{x}\cdot\abs{y}   & x,y\in\R & (\prope{homogeneous} / \prope{submultiplicative})   & and 
    \\4. & \abs{x+y} &\le& \abs{x}+\abs{y}       & x,y\in\R & (\prope{subadditive} / \prope{triangle inequality}) & 
    \end{array}$\\
  Reference: \citerpg{cohn}{312}{1852335874}
  }

%---------------------------------------
\begin{definition}
\footnote{
  \citerpg{barvinok2002}{5}{0821872311}
%  \citerpg{beran1985}{4}{902771715X}\\
%  \citerpc{alfsen1963}{432}{(5.1)}\\
%  \citerpg{vel1993}{6}{0444815058}\\
%  \citerpg{stern1999}{284}{0521461057}\\
%  \citerpg{denkowski2003}{116}{0306474565}
  }
\label{def:convex_set}
\label{def:convex}
%---------------------------------------
Let $\osetX$ be an \structe{ordered set}.
\defboxt{
  A subset $\setD\subseteq\setX$ is \propd{convex} in $\setX$ if
  \\\indentx$x,y\in\setD \qquad\implies\qquad \intoo{x}{y}\subseteq\setD$.
  %A set that is {\bf not} convex is \propd{concave}.
  }
\end{definition}

%--------------------------------------
\begin{example}
%--------------------------------------
\prope{Convex} subsets of $\Z$ under the usual integer ordering relation include
\\\indentx
  $\emptyset$, $\Z$, $\Znn$, $\Zp$, $\setn{0,1,2,3,4,5}$, and $\setn{-2,-1,0,1,2,3}$.
\end{example}

%---------------------------------------
\begin{definition}
\label{def:lub}\label{def:sup}\label{def:join}
\index{bound!least upper bound}
\index{bound!supremum}
%---------------------------------------
Let $\opair{\setX}{\orel}$ be an ordered set. % and $\psetx$ the power set of $\setX$.
\defboxp{\indxs{\join}\indxs{\supA}\indxs{\joinop\setA}
  For any set $\setA\in\psetx$, $c$ is an \vald{upper bound} of $\setA$ in $\opair{\setX}{\orel}$ if
  $\begin{array}{ll}
    x \orel c & \forall x\in\setA .
  \end{array}$
  An element $b$ is the \vald{least upper bound}, or \vald{\lub}, of $\setA$ in $\opair{\setX}{\orel}$ if
  \\\indentx$\begin{array}{Mcl}
    $b$ and $c$ are \vale{upper bound}s of $\setA$ &\implies& b\orel c .
  \end{array}$\\
  The least upper bound of the set $\setA$ is denoted $\joinop\setA$.\\
  %It is also called the \hid{supremum} of $\setA$, which is denoted $\supA$.
  The \opd{join} $x\join y$ of $x$ and $y$ is defined as $x\join y\eqd\joinop\setn{x,y}$.
  }
\end{definition}

%---------------------------------------
\begin{definition}
\label{def:glb}\label{def:inf}\label{def:meet}
\index{bound!greatest lower bound}
\index{bound!infimum}
%---------------------------------------
Let $\opair{\setX}{\orel}$ be an ordered set. % and $\psetx$ the power set of $\setX$.
\defboxp{\indxs{\join}\indxs{\supA}\indxs{\joinop\setA}
  For any set $\setA\in\psetx$, $p$ is a \vald{lower bound} of $\setA$ in $\opair{\setX}{\orel}$ if
  $\begin{array}{ll}
    p \orel x & \forall x\in\setA .
  \end{array}$
  An element $a$ is the \vald{greatest lower bound}, or \vald{\glb}, of $\setA$ in $\opair{\setX}{\orel}$ if
  \\\indentx$\begin{array}{Mcl}
    $a$ and $p$ are \vale{lower bound}s of $\setA$ &\implies& p\orel a .
  \end{array}$\\
  The greatest lower bound of the set $\setA$ is denoted $\meetop\setA$.\\
  %It is also called the \vald{infimum} of $\setA$, which is denoted $\infA$.
  The \opd{meet} $x\meet y$ of $x$ and $y$ is defined as $x\meet y\eqd\meetop\setn{x,y}$.
  }
\end{definition}

%---------------------------------------
\begin{lemma}
\label{lem:lubX}
%---------------------------------------
Let $\osetX$ be an \structe{ordered set} \xref{def:oset}.
Let $\joinop\setA$ be the \structe{least upper bound} \xref{def:lub} of a set $\setA\in\psetx$ \xref{def:pset}.
Let $\meetop\setA$ be the \structe{greatest lower bound} \xref{def:glb} of a set $\setA\in\psetx$.
\lembox{
  \brb{\setA=\setX}
  \quad\implies\quad
  \brb{\begin{array}{FrclD}
    1. & \joinop\setA &=& \set{a\in\setX}{x\orel a,\,\forall x,a\in\setX} &and\\
    2. & \meetop\setA &=& \set{a\in\setX}{a\orel x,\,\forall x,a\in\setX} &.
  \end{array}}
  }
\end{lemma}

%---------------------------------------
\begin{definition}
\label{def:ceil}
\label{def:floor}
%---------------------------------------
Let $\opair{\R}{\orel}$ be the \structe{standard ordered set of real numbers}.
The \fnctd{floor   function} $\floor{x}\in\clFrz$ and
the \fnctd{ceiling function} $\ceil{x}\in\clFrz$ are defined as
\\\indentx
  $\mcom{\ds\floor{x} \eqd \joinop\set{n\in\Z}{n\orel  x}}{\fncte{floor function}}$
  \quad and \quad
  $\mcom{\ds\ceil{x}  \eqd \meetop\set{n\in\Z}{n\oreld x}}{\fncte{ceiling function}}$.
\end{definition}

