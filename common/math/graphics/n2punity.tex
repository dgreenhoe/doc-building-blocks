%============================================================================
% Daniel J. Greenhoe
% LaTeX file
% 2nd order B-spline partition of unity
% demonstration that  SUM N_2(x-k) = 1
%                    k\in Z
% nominal unit = 10mm
%============================================================================
\begin{pspicture}(-5,-0.6)(5,1.5)%
  %-------------------------------------
  % axes
  %-------------------------------------
  \psaxes[linewidth=0.75pt,linecolor=axis,yAxis=false,Dx=1]{<->}(0,0)(-5,0)(5,1.5)% x-axis
  \psaxes[linewidth=0.75pt,linecolor=axis,xAxis=false,Dx=1]{->}(0,0)(-5,0)(5,1.5)%  y-axis
  %-------------------------------------
  % annotation
  %-------------------------------------
  \multirput(-4.5,0)(0.5,0){19}{% vertical guide lines
    \psline[linecolor=red,linewidth=0.75pt,linestyle=dashed](0,0)(0,1)%
    }%
  \psline[linecolor=red,linewidth=0.75pt,linestyle=dashed](-5,1)(5,1)% horizontal guide line
  %-------------------------------------
  % B-spline plots
  %-------------------------------------
  \multirput(-7,0)(1,0){13}{%
    \psplot[plotpoints=64]{0}{1}{ 1   x 2 exp mul 2 div}%
    \psplot[plotpoints=64]{1}{2}{-2   x 2 exp mul +6 x mul add -3 add 2 div}% 
    \psplot[plotpoints=64]{2}{3}{ 1   x 2 exp mul -6 x mul add +9 add 2 div}% 
    }%
\end{pspicture}
