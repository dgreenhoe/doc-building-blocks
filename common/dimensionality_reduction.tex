%============================================================================
% LaTeX File
% Daniel J. Greenhoe
%============================================================================
%=======================================
\chapter{Dimensionality Reduction}
\label{chp:dimreduct}
%=======================================
%=======================================
%\section{Introduction}
%=======================================
A number of statistical estimation algorithms yield subspace structures with 
high dimensionality---one key example is \ope{Kernel Density Estimation} (\ope{KDE})\xref{chp:pdfest}.
Richard Bellman described this problem as the 
``curse of dimensionality".\footnote{
  \citerp{bellman1954}{206},
  \citerpc{bellman1961}{94}{\textsection ``5.16 The Curse of Dimensionality"},
  \citerp{bellman1971}{44},
  \citerpgc{gramacki2017}{3, 59}{9783319716886}{\textsection3.9 ``The Curse of Dimensionality"}
  }

Here are some references about \ope{dimensionality reduction}:
\\\indentx\begin{tabular}{cll}
    \citeP{cunningham2015} & (a 2015 paper): & ``\emph{Linear Dimensionality Reduction: Survey, Insights, and Generalizations}"
  \\\citeP{sorzano2014}    & (a 2014 paper): & ``\emph{A survey of dimensionality reduction techniques}"
  \\\citer{lee2007}        & (a 2007 book):  & ``\emph{Nonlinear Dimensionality Reduction}" 
\end{tabular}
