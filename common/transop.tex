%============================================================================
% Daniel J. Greenhoe
% LaTeX File
%============================================================================

%======================================
\chapter{Translation and Dilation}
%======================================
\qboxnpqt
  {Ren\'e Descartes, philosopher and mathematician (1596--1650)
   \index{Descartes, Ren\'e}
   \index{quotes!Descartes, Ren\'e}
   \footnotemark}
  {../common/people/descartes_fransHals_bw_wkp_pdomain.jpg}
  {Je me plaisois surtout aux math\'ematiques,
    \`a cause de la certitude et de l'\'evidence de leurs raisons:
    mais je ne remarquois point encore leur vrai usage;
    et, pensant qu'elles ne servoient qu'aux arts m\'ecaniques,
    je m'\'etonnois de ce que leurs fondements \'etant si fermes et si solides,
    on n'avoit rien b\^ati dessus de plus relev\'e:}
  {I was especially delighted with the mathematics,
    on account of the certitude and evidence of their reasonings;
    but I had not as yet a precise knowledge of their true use;
    and thinking that they but contributed to the advancement of the mechanical arts,
    I was astonished that foundations, so strong and solid,
    should have had no loftier superstructure reared on them.}
  \citetblt{
    quote: & \citer{descartes_method} \\
    translation: & \citerc{descartes_method_eng}{part I, paragraph 10} \\
    image: & \scs\url{http://en.wikipedia.org/wiki/File:Frans_Hals_-_Portret_van_Ren\%C3\%A9_Descartes.jpg}, public domain
    }

%======================================
\section{Definitions}
%======================================
The space $\spLLR$ is a subspace of the space $\hxs{\spRR}$, the set of all functions with \structe{domain}
$\hxs{\R}$ (the set of real numbers) and \structe{range} $\R$.
The space $\spRR$ is a subspace of the space $\hxs{\spCC}$, the set of all functions with \structe{domain}
$\hxs{\C}$ (the set of complex numbers) and \structe{range} $\C$.
That is,
$\ds\spLLR \subseteq \spRR \subseteq \spCC$.
In general, the notation $\clFxy$ represents the set of all functions with domain $\setX$ and range $\setY$ \xref{def:spXY}.
Although this notation may seem curious, note that for finite $\setX$ and finite $\setY$, the number of functions (elements)
in $\clFxy$ is $\seto{\clFxy}=\seto{\setY}^\seto{\setX}$.
%---------------------------------------
\begin{definition}
\label{def:spXY}
%---------------------------------------
Let $\hxs{\setX}$ and $\hxs{\setY}$ be sets.
\defbox{\begin{array}{M}
  The space $\clFxy$ represents the set of all functions with \structe{domain} $\setX$ and \structe{range} $\setY$ such that
  \\\indentx$\ds \hxs{\clFxy} \eqd \set{\ff(x)}{\ff(x):\setX\to\setY}$
\end{array}}
\end{definition}

\ifdochasnot{relation}{%
%---------------------------------------
\begin{definition}
\footnote{
  \citerp{ab}{126},
  %\citerp{halmos1950}{15},
  \citerp{hausdorff1937e}{22},
  \citorp{poussin1915}{440}
  }
\label{def:setind}
\index{function!characteristic}
\index{function!indicator}
%---------------------------------------
Let $\setX$ be a set.
\defbox{\begin{array}{M}
  The \fnctd{indicator function} $\hxs{\setind}\in\clF{\psetx}{\setn{0,1}}$ is defined as
  \\\indentx$\ds
  \setind_{\setA}(x) =
    \brbl{\begin{array}{ll @{\qquad}C}
      1  & $ for $ x\in \setA       & \forall x\in\setX,\; \setA\in\psetx \\
      0  & $ for $ x\notin \setA   & \forall x\in\setX,\; \setA\in\psetx
    \end{array}}
    $
  \\
  The indicator function $\setind$ is also called the \fnctd{characteristic function}.
\end{array}}
\end{definition}
}

%---------------------------------------
\begin{definition}
\footnote{
  \citerppgc{walnut2002}{79}{80}{0817639624}{Definition 3.39},
  \citerppg{christensen2003}{41}{42}{0817642951},
  \citerpgc{wojtaszczyk1997}{18}{0521578949}{Definitions 2.3,2.4},
  \citerpg{kammler2008}{A-21}{0521883407},
  \citerpg{bachman2002}{473}{9780387988993},
  \citerpg{packer2004}{260}{0821834029}, %{section 3.1}\\
  \citerpg{zayed2004}{}{0817643044},
  \citerpgc{heil2011}{250}{0817646868}{Notation 9.4},
  \citerpg{casazza1998}{74}{0817639594},
  \citerp{goodman1993}{639},
  \citePpc{heil1989}{633}{Definition 1.3.1},
  \citerp{dai1996}{81},
  \citerpg{dai1998}{2}{0821808001}
  %\citerpg{dai1998}{21}{0821808001}
  }
\label{def:opT}
\label{def:opD}
\label{def:opTD}
\label{def:opTrn}
\label{def:opDil}
%---------------------------------------
\defbox{\begin{array}{Mrc>{\ds}lCM}
    $\hxs{\opTrn_\tau}$   is a \opd{translation operator} on $\spCC$ if & \hxs{\opTrn_\tau}  \ff(x) &\eqd& \ff(x-\tau)  & \forall \ff\in\spCC & .
  \\$\hxs{\opDil_\alpha}$ is a \opd{dilation operator}    on $\spCC$ if & \hxs{\opDil_\alpha}\ff(x) &\eqd& \ff(\alpha x)& \forall \ff\in\spCC & .
  \\\mc{5}{M}{Moreover, $\hxs{\opTrn}\eqd\opTrn_1$ \quad{and}\quad $\opDil\eqd\sqrt{2}\opDil_2$} & .
\end{array}}
\end{definition}

%---------------------------------------
\begin{example}
%---------------------------------------
Let $\opT$ and $\opD$ be defined as in \prefpp{def:opTD}.
\exbox{
\begin{array}{cc}
  \includegraphics{../common/math/graphics/pdfs/opTrn.pdf}&\includegraphics{../common/math/graphics/pdfs/opDil.pdf}%
\end{array}}
\end{example}

%=======================================
\section{Algebraic properties}
%=======================================
%---------------------------------------
\begin{proposition}
\label{prop:opT_periodic}
%---------------------------------------
Let $\opTrn_\tau$ be a \fncte{translation operator} \xref{def:opT}.
\propbox{
  \sum_{n\in\Z}\opTrn_\tau^n\ff(x) = \sum_{n\in\Z}\opTrn_\tau^n\ff(x+\tau)
  \qquad \scy \forall \ff\in\spRR
  \qquad \brp{\text{\scs$\ds\sum_{n\in\Z}\opTrn_\tau^n\ff(x)$ is \prope{periodic} with period $\tau$}}
  }
\end{proposition}
\begin{proof}
\begin{align*}
  \sum_{n\in\Z}\opTrn_\tau^n\ff(x+\tau)
    &= \sum_{n\in\Z}\ff(x-n\tau+\tau)
    && \text{by definition of $\opTrn_\tau$} && \text{\xref{def:opT}}
  \\&= \sum_{m\in\Z}\ff(x-m\tau)
    && \text{where $m\eqd n-1$}         && \implies\;n=m+1
  \\&= \sum_{m\in\Z}\opTrn_\tau^m\ff(x)
    && \text{by definition of $\opTrn_\tau$} && \text{\xref{def:opT}}
\end{align*}
\end{proof}

In a linear space, every operator has an \structe{inverse}\ifsxref{relation}{def:rel_inverse}.
Although the inverse always exists as a \rele{relation}\ifsxref{relation}{def:relation},
it may not exist as a \rele{function}\ifsxrefs{relation}{def:function}or as an \rele{operator}\ifsxref{relation}{def:operator}.
But in some cases the inverse of an operator is itself an operator.
%\ifdochas{operator}{\footnote{\hie{operator inverse}: \prefp{def:op_inv}}}
The inverses of the operators $\opTrn$ and $\opDil$ both exist as operators,
as demonstrated next.

%=======================================
\section{Operator properties}
%=======================================
%---------------------------------------
\begin{proposition}[\thmd{transversal operator inverses}]
\label{prop:opTi}
\label{prop:opDi}
%---------------------------------------
Let $\opTrn$ and $\opDil$ be as defined in \prefp{def:opT}.
\propbox{
  \begin{array}{lrcl C@{\qquad}D}
    \mc{6}{M}{$\opTrn$ has an \ope{inverse} $\opTrni$ in $\spCC$ expressed by the relation}
      \\& \opTrni\ff(x) &=& \ff(x+1)               & \forall \ff\in\spCC  & (\hid{translation operator inverse}).
      \\
    \mc{6}{M}{$\opDil$ has an \ope{inverse} $\opDili$ in $\spCC$ expressed by the relation}
      \\& \opDili\ff(x) &=& \cwt\:\ff\brp{\half x} & \forall \ff\in\spCC  & (\hid{dilation operator inverse}).
  \end{array}
  }
\end{proposition}
\begin{proof}
\begin{align*}
  \intertext{1. Proof that $\opTrni$ is the inverse of $\opTrn$:}
  \opTrni\opTrn\ff(x)
    &= \opTrni\ff(x-1)
    && \text{by defintion of $\opTrn$}                       && \text{\xref{def:opT}}
  \\&= \ff([x+1]-1)
  \\&= \ff(x)
  \\&= \ff([x-1]+1)
  \\&= \opTrn\ff(x+1)
    && \text{by defintion of $\opTrn$}                       && \text{\xref{def:opT}}
  \\&= \opTrn\opTrni\ff(x)
  \\\implies & \opTrni\opTrn=\opI=\opTrn\opTrni
  %
  \intertext{2. Proof that $\opDili$ is the inverse of $\opDil$:}
  \opDili\opDil\ff(x)
    &= \opDili \sqrt{2}\ff\brp{2x}
    && \text{by defintion of $\opDil$}                       && \text{\xref{def:opT}}
  \\&= \brp{\cwt }\sqrt{2}\ff\brp{2\brs{\half x}}
  \\&= \ff(x)
  \\&= \sqrt{2} \brs{\cwt \ff\left(\half [2x]\right)}
  \\&= \opDil \brs{\cwt \ff\brp{\half x}}
    && \text{by defintion of $\opDil$}                       && \text{\xref{def:opT}}
  \\&= \opDil\opDili\ff(x)
  \\\implies & \opDili\opDil=\opI=\opDil\opDili
\end{align*}
\end{proof}

%---------------------------------------
\begin{proposition}
\label{prop:DjTn}
%---------------------------------------
Let $\opTrn$ and $\opDil$ be as defined in \prefp{def:opT}.\\
Let $\opDil^0=\opTrn^0\eqd\opI$ be the \ope{identity operator}.
\propbox{
  \opDil^j\opTrn^n\ff(x) = 2^{j/2}\ff\brp{2^jx-n} \qquad\scy\forall j,n\in\Z,\,\ff\in\spCC
  }
\end{proposition}
%\begin{proof}
%\begin{enumerate}
%  \item Proof for $j=0$:
%    \begin{align*}
%      \opDil^0\opTrn^n\ff(x)
%        &= \opI\opTrn^n\ff(x)
%        && \text{by definition of $\opDil^0$}
%      \\&= \opTrn^n\ff(x)
%
%        = 2^{j/2}\ff\brp{2^jx-n} \qquad\scy\forall j,n\in\Z,\,\ff\in\spCC
%
%    \end{align*}
%\end{enumerate}
%\end{proof}

%=======================================
\section{Linear operator properties}
%=======================================

%20171225%\ifdochasnot{sums}{
%20171225%%--------------------------------------
%20171225%\begin{definition}
%20171225%\label{def:sum}
%20171225%\footnote{
%20171225%  \citerpgc{berberian1961}{8}{0821819127}{Definition~I.3.1},
%20171225%  \citorpc{fourier1820}{280}{``$\sum$" notation}  %{http://gallica.bnf.fr/ark:/12148/bpt6k33707/f285.image}
%20171225%  }
%20171225%%--------------------------------------
%20171225%Let $+$ be an addition operator on a tuple $\tuple{x_n}{m}{\xN}$.
%20171225%\defbox{\begin{array}{M}
%20171225%  The \hid{summation} of $\tuplen{x_n}$ from index $m$ to index $\xN$ with respect to $+$ is
%20171225%  \\\qquad$\ds
%20171225%  \sum_{n=m}^\xN x_n \eqd
%20171225%    \brbl{\begin{array}{>{\ds}l@{\qquad}M}
%20171225%      0                            & for $\xN<m$\\
%20171225%     %x_\xN                          & for $N=m$ \\
%20171225%    \brp{\sum_{n=m}^{\xN-1} x_n}+x_\xN & for $\xN\ge m$
%20171225%    \end{array}}
%20171225%  $
%20171225%\end{array}}
%20171225%\end{definition}
%20171225%}
%20171225%
%20171225%\ifdochasnot{series}{
%20171225%An infinite summation $\sum_{n=1}^\infty \fphi_n$ is meaningless outside some topological space (e.g. metric space, normed space, etc.).
%20171225%The sum $\sum_{n=1}^\infty \fphi_n$ is an abbreviation for $\lim_{\xN\to\infty}\sum_{n=1}^\xN \fphi_n$
%20171225%(the limit of partial sums). And the concept of limit is also itself meaningless outside of a topological space.
%20171225%
%20171225%%---------------------------------------
%20171225%\begin{definition}
%20171225%\footnote{
%20171225%  \citerpg{klauder2010}{4}{0817647902},
%20171225%  \citerpg{kubrusly2001}{43}{0817641742},
%20171225%  \citerppg{bachman1966}{3}{4}{0486402517}
%20171225%  %\citerppg{bachman2000fa}{3}{4}{0486402517}\\
%20171225%  }
%20171225%\label{def:suminf}
%20171225%%---------------------------------------
%20171225%Let $\topspaceX$ be a topological space and $\lim$ be the limit induced by the topology $\topT$.
%20171225%\defbox{\begin{array}{>{\ds}rc>{\ds}lc>{\ds}l}
%20171225%  \sum_{n=1}^\infty       \vx_n &\eqd& \sum_{n\in\Zp} \vx_n &\eqd& \lim_{\xN\to\infty} \sum_{n=1}^\xN \vx_n\\
%20171225%  \sum_{n=-\infty}^\infty \vx_n &\eqd& \sum_{n\in\Z}  \vx_n &\eqd& \lim_{\xN\to\infty} \brp{\sum_{n=0}^\xN \vx_n} + \brp{\lim_{\xN\to-\infty} \sum_{n=-1}^\xN \vx_n}
%20171225%\end{array}}
%20171225%\end{definition}
%20171225%}

%%---------------------------------------
%\begin{proposition}
%\label{prop:opTD_sum}
%\label{prop:opTDi_sum}
%%---------------------------------------
%%Let $\opTrn$ be the \fncte{translation operator} and $\opDil$ the \fncte{dilation operator}. % \xref{def:opD}.
%Let $\opTrn$ and $\opDil$ be as in \prefp{def:opT}.
%\propbox{\begin{array}{>{\ds}rc>{\ds}l@{\qquad}C}
%  \opDil^j\opTrn^n\sum_{m\in\Z} \ff(x) &=& \sum_{m\in\Z}\opDil^j\opTrn^n \ff(x) & \forall \ff\in\spCC,\,n,j\in\Z
%  %\opDil\sum_{n\in\Z} \ff(x) &=& \sum_{n\in\Z}\opDil \ff(x) & \forall \ff\in\spCC
%\end{array}}
%\end{proposition}
%\begin{proof}
%\begin{align*}
%  \opDil^j\opTrn^n \sum_{n\in\Z} \ff(x)
%    &= \sum_{m\in\Z} 2^{j/2}\ff\brp{2^j-n}
%  \\&= \sum_{m\in\Z}\opDil^j\opTrn^n\ff(x)
%    && \text{by \prefp{thm:L_prop}}
%\end{align*}
%\end{proof}


%--------------------------------------
\begin{theorem}
\label{thm:opTD_linear}
%--------------------------------------
\thmbox{\begin{array}{>{\ds}l        c>{\ds}l                                                    C                               D}
    \mc{3}{H}{Mapping}                                                                         & \mc{1}{|H}{Domain}            & \mc{1}{|H}{Property}
  \\\hline%                          & &                                                       &                               &
    \opTrn_\tau  \brs{\ff(x)+\fg(x)} &=& \brs{\opTrn_\tau\ff(x)} + \brs{\opTrn_\tau\fg(x)}     & \forall x       \in\C         & (\prope{additive})
  \\\opTrn_\tau  \brs{\alpha\fg(x)}  &=& c\opTrn_\tau\fg(x)                                    & \forall x,\alpha\in\C\;c\in\R & (\prope{homogeneous})
  \\\opDil_\alpha\brs{\ff(x)+\fg(x)} &=& \brs{\opDil_\alpha\ff(x)} + \brs{\opDil_\alpha\fg(x)} & \forall x       \in\C         & (\prope{additive})
  \\\opDil_\alpha\brs{\alpha\fg(x)}  &=& c\opDil_\alpha\fg(x)                                  & \forall x,\alpha\in\C\;c\in\R & (\prope{homogeneous})
\end{array}}
\end{theorem}
\begin{proof}
\begin{align*}
    \opTrn_\tau\brs{\ff(x)+\fg(x)}
      &\eqd \ff(x-\tau)+\fg(x-\tau)
      && \text{by definition of $\opTrn$}
      && \text{\xref{def:opTrn}}
    \\&\eqd \brs{\opTrn_\tau\ff(x)} + \brs{\opTrn_\tau\fg(x)}
      && \text{by definition of $\opTrn$}
      && \text{\xref{def:opTrn}}
    \\
    \opTrn_\tau\brs{c\fg(x)}
      &\eqd c\fg(x-\tau)
      && \text{by definition of $\opTrn$}
      && \text{\xref{def:opTrn}}
    \\&\eqd c\opTrn_\tau\fg(x)
      && \text{by definition of $\opTrn$}
      && \text{\xref{def:opTrn}}
    \\
    \opDil_\alpha\brs{\ff(x)+\fg(x)}
      &\eqd \ff(\alpha x)+\fg(\alpha x)
      && \text{by definition of $\opDil$}
      && \text{\xref{def:opDil}}
    \\&\eqd \brs{\opDil_\alpha\ff(x)} + \brs{\opDil_\alpha\fg(x)}
      && \text{by definition of $\opDil$}
      && \text{\xref{def:opTrn}}
    \\
    \opDil\brs{\alpha\fg(x)}
      &\eqd \alpha\fg(\alpha x)
      && \text{by definition of $\opDil$}
      && \text{\xref{def:opDil}}
    \\&\eqd \alpha\opDil\fg(x)
      && \text{by definition of $\opDil$}
      && \text{\xref{def:opDil}}
\end{align*}
\end{proof}

%---------------------------------------
\begin{corollary}
\label{cor:opTrn_linear}
%---------------------------------------
\corbox{
  \begin{array}{F>{\ds}lc>{\ds}lCD}
      1. & \opTrn 0        &=& 0                                                                   &                              & and
    \\2. & \opTrn(-\ff)    &=& -(\opTrn\ff)                                                        & \forall \vx\in\spX           & and
    \\3. & \opTrn(\ff-\fg) &=& \opTrn\ff - \opL\fg                                                 & \forall \vx,\vy\in\spX       & and
    \\4. & \opTrn\brp{\sum_{n=1}^\xN \alpha_n\ff_n}  &=& \sum_{n=1}^\xN \alpha_n\brp{\opTrn\ff_n}  & \vx_n\in\spX,\,\alpha_n\in\F &
  \end{array}
}
\end{corollary}
\begin{proof}
\begin{enumerate}
  \item By \prefpp{thm:opTD_linear}, $\opTrn$ is \prope{additive} and \prope{homogeneous}.
  \item By (1) and by \prefpp{def:linop}, $\opTrn$ is a \prope{linear} operator.
  \item Then the properties follow by \prefpp{thm:L_prop}.
\end{enumerate}
\end{proof}

%---------------------------------------
\begin{corollary}
\label{cor:opDil_linear}
%---------------------------------------
\corbox{
  \begin{array}{F>{\ds}lc>{\ds}lCD}
      1. & \opDil 0        &=& 0                                                                   &                              & and
    \\2. & \opDil(-\ff)    &=& -(\opDil\ff)                                                        & \forall \vx\in\spX           & and
    \\3. & \opDil(\ff-\fg) &=& \opDil\ff - \opL\fg                                                 & \forall \vx,\vy\in\spX       & and
    \\4. & \opDil\brp{\sum_{n=1}^\xN \alpha_n\ff_n}  &=& \sum_{n=1}^\xN \alpha_n\brp{\opDil\ff_n}  & \vx_n\in\spX,\,\alpha_n\in\F &
  \end{array}
}
\end{corollary}
\begin{proof}
\begin{enumerate}
  \item By \prefpp{thm:opTD_linear}, $\opDil$ is \prope{additive} and \prope{homogeneous}.
  \item By (1) and by \prefpp{def:linop}, $\opDil$ is a \prope{linear} operator.
  \item Then the properties follow by \prefpp{thm:L_prop}.
\end{enumerate}
\end{proof}

%--------------------------------------
\begin{proposition}
\label{prop:DjTnfg}
%--------------------------------------
Let $\opTrn$ and $\opDil$ be as in \prefp{def:opT}.
\propbox{
  \opDil^j\opTrn^n\brs{\ff\fg} = 2^{-j/2}\;\brs{\opDil^j\opTrn^n\ff}\;\brs{\opDil^j\opTrn^n\fg}
  \qquad\scy\forall j,n\in\Z,\,\ff\in\spCC
  }
\end{proposition}
\begin{proof}
  \begin{align*}
    \opDil^j\opTrn^n\brs{\ff(x)\fg(x)}
      &= 2^{j/2}
         \ff\brp{2^jx-n}
         \fg\brp{2^jx-n}
      && \text{by \prefp{prop:DjTn}}
    \\&= 2^{-j/2}
         \brs{2^{j/2}\ff\brp{2^jx-n}}
         \brs{2^{j/2}\fg\brp{2^jx-n}}
    \\&= 2^{-j/2}
         \brs{\opDil^j\opTrn^n\ff(x)} \brs{\opDil^j\opTrn^n\fg(x)}
      && \text{by \prefp{prop:DjTn}}
  \end{align*}
\end{proof}

In general the operators $\opTrn$ and $\opDil$ are \prope{noncommutative} ($\opTrn\opDil\neq\opDil\opTrn$),
as demonstrated by \pref{cnt:DTTD} (next) and \prefpp{prop:DTTD}.
%---------------------------------------
\begin{counterex}
\label{cnt:DTTD}
%---------------------------------------
\cntboxt{%
  \tbox{
    As illustrated to the right,\\ % for the function $\ff\in\clFrr$,\\
    it is \textbf{not} always true that\\
    $\opTrn\opDil=\opDil\opTrn$:
    }
  \qquad
  \begin{tabular}{ccc}
     \includegraphics{../common/math/graphics/pdfs/tentfx.pdf}%
    &\includegraphics{../common/math/graphics/pdfs/TD_tentfx.pdf}%
    &\includegraphics{../common/math/graphics/pdfs/DT_tentfx.pdf}%
  \end{tabular}%
  }
\end{counterex}

%---------------------------------------
\begin{proposition}[\thmd{commutator relation}]
\footnote{
  \citerpgc{christensen2003}{42}{0817642951}{equation (2.9)},
  \citerpg{dai1998}{21}{0821808001},
  \citerp{goodman1993}{641},
  \citerp{goodman1993feb}{110}
  }
\label{prop:DTTD}
%---------------------------------------
Let $\opTrn$ and $\opDil$ be as in \prefp{def:opT}.
\propbox{
  \begin{array}{>{\ds}rc>{\ds}lC}
    \opDil^j\opTrn^n &=& \opTrn^{2^{-j/2}n}\opDil^j & \forall j,n\in\Z\\
    \opTrn^n\opDil^j &=& \opDil^j\opTrn^{2^jn}      & \forall n,j\in\Z
  \end{array}
  }
\end{proposition}
\begin{proof}
  \begin{align*}
    \opDil^j \opTrn^{2^jn}\ff(x)
      &= 2^{j/2}\,\ff(2^jx-2^jn)
      && \text{by \prefp{prop:DjTnfg}}
    \\&= 2^{j/2}\,\ff\brp{2^j\brs{x-n}}
      && \text{by \prope{distributivity} of the field $\fieldR$} && \text{\xref{def:algebra}}
    \\&= \opTrn^n 2^{j/2}\,\ff\brp{2^jx}
      && \text{by definition of $\opTrn$}  && \text{\xref{def:opT}}
    \\&= \opTrn^n \opDil^j\ff\brp{x}
      && \text{by definition of $\opDil$}  && \text{\xref{def:opD}}
    \\
    \\
    \opDil^j \opTrn^n\ff(x)
      &= 2^{j/2}\,\ff(2^jx-n)
      && \text{by \prefp{prop:DjTnfg}}
    \\&= 2^{j/2}\,\ff\brp{2^j\brs{x-2^{-j/2}n}}
      && \text{by \prope{distributivity} of the field $\fieldR$}  && \text{\xref{def:algebra}}
    \\&= \opTrn^{2^{-j/2}n} 2^{j/2}\,\ff\brp{2^jx}
      && \text{by definition of $\opTrn$}  && \text{\xref{def:opT}}
    \\&= \opTrn^{2^{-j/2}n} \opDil^j\ff\brp{x}
      && \text{by definition of $\opDil$}  && \text{\xref{def:opD}}
  \end{align*}
\end{proof}



%=======================================
\section{Inner product space properties}
%=======================================
%\prefp{def:opT} defines the dilation operator $\opDil$ and translation operator $\opTrn$.
In an inner product space\ifsxref{vsinprod}{def:inprod},
every operator has an \ope{adjoint} \ifxref{operator}{prop:op_adjoint}
and this adjoint is always itself an operator.
In the case where the adjoint and inverse of an operator $\opU$ coincide,
then $\opU$ is said to be \prope{unitary}\ifsxref{operator}{def:op_unitary}.
And in this case, $\opU$ has several nice properties (see \pref{prop:wavstrct_TD_unitary_1} and \prefp{thm:TD_unitary}).
\pref{prop:wavstrct_Ta} (next) gives the adjoints of $\opDil$ and $\opTrn$,
and \prefpp{prop:TD_unitary} demonstrates that both
$\opDil$ and $\opTrn$ are \prop{unitary}.
Other examples of unitary operators include the \hie{Fourier Transform operator} $\opFT$
\ifsxref{harFour}{cor:ft_unitary} and the \hie{rotation matrix operator}\ifsxref{operator}{ex:operator_rotation_unitary}.
%---------------------------------------
\begin{proposition} %[Adjoints and inverses of transversal operators]
\label{prop:wavstrct_Ta}
\label{prop:wavstrct_Da}
\label{prop:opTa}
%---------------------------------------
Let $\opTrn$ be the \ope{translation operator} \xref{def:opT} with \ope{adjoint} $\hxs{\opTrna}$
and $\opDil$    the \ope{dilation operator} with \ope{adjoint} $\hxs{\opDila}$\ifsxref{operator}{def:adjoint}.
%in the inner product space $\inprodspaceX[\clFrc]$.
\propbox{
  \begin{array}{rcl C@{\qquad}D@{\qquad}D}
    \hxs{\opTrna}\ff(x) &=& \ff(x+1)
                &   \forall \ff\in\spLLR
       & (\ope{translation operator adjoint})
    \\
    \hxs{\opDila}\ff(x) &=& \cwt \:\ff\brp{\half x}
                &   \forall \ff\in\spLLR
       & (\ope{dilation operator adjoint})
  \end{array}
  }
\end{proposition}
\begin{proof}
\begin{align*}
  \intertext{1. Proof that $\opTrna\ff(x)=\ff(x+1)$:}
  \inprod{\fg(x)}{\opTrna\ff(x)}
    &= \inprod{\fg(u)}{\opTrna\ff(u)}
    && \text{by change of variable $x\rightarrow u$}
  \\&= \inprod{\opTrn\fg(u)}{\ff(u)}
    && \text{by definition of adjoint $\opTrna$}             && \text{\ifxref{operator}{def:adjoint}}
  \\&= \inprod{\fg(u-1)}{\ff(u)}
    && \text{by definition of $\opTrn$}                      && \text{\xref{def:opT}}
  \\&= \inprod{\fg(x)}{\ff(x+1)}
    && \text{where $x\eqd u-1$ $\implies$ $u=x+1$}
  \\\implies& \opTrna\ff(x) = \ff(x+1)
  \\
  \intertext{2. Proof that $\opDila\ff(x)=\cwt \:\ff\brp{\half x}$:}
  \inprod{\fg(x)}{\opDila\ff(x)}
    &= \inprod{\fg(u)}{\opDila\ff(u)}
    && \text{by change of variable $x\rightarrow u$}
  \\&= \inprod{\opDil\fg(u)}{\ff(u)}
    && \text{by definition of $\opDila$}                     && \text{\ifxref{operator}{def:adjoint}}
  \\&= \inprod{\sqrt{2}\fg(2u)}{\ff(u)}
    && \text{by definition of $\opDil$}                      && \text{\xref{def:opD}}
  \\&= \int_{u\in\R} \sqrt{2}\fg(2u) \ff^\ast(u) \du
    && \text{by definition of $\inprodn$}                    && \text{\ifxref{vsinprod}{def:inprod}}
  \\&= \int_{x\in\R} \fg(x) \brs{\sqrt{2}\ff\brp{\frac{x}{2}} \half }^\ast \dx
    && \text{where $x=2u$}
  \\&= \inprod{\fg(x)}{\cwt  \ff\brp{\frac{x}{2}}}
    && \text{by definition of $\inprodn$}                    && \text{\ifxref{vsinprod}{def:inprod}}
  \\&\implies \opDila\ff(x)=\cwt \:\ff\brp{\frac{x}{2}}
\end{align*}
\end{proof}

%---------------------------------------
\begin{proposition}
\footnote{
  \citerpgc{christensen2003}{41}{0817642951}{Lemma 2.5.1},
  \citerpgc{wojtaszczyk1997}{18}{0521578949}{Lemma 2.5}
  }
\label{prop:TD_unitary}
%---------------------------------------
%Let $\opTrn$ be the translation operator with inverse $\hxs{\opTrni}$ and adjoint $\opTrna$
%and $\opDil$ be the dilation    operator with inverse $\hxs{\opDili}$ and adjoint $\opDila$.
Let $\opTrn$ and $\opDil$ be as in \prefpp{def:opT}.\\
Let $\opTrni$ and $\opDili$ be as in \prefpp{prop:opTi}.
\propbox{
  \begin{array}{MM}
    $\opTrn$ is \prope{unitary} in $\spLLR$ & ($\opTrni=\opTrna$ in $\spLLR$).\\
    $\opDil$ is \prope{unitary} in $\spLLR$ & ($\opDili=\opDila$ in $\spLLR$).
  \end{array}
  }
\end{proposition}
\begin{proof}
\begin{align*}
    \opTrni &= \opTrna && \text{by \prefp{prop:opTi} and \prefp{prop:wavstrct_Ta}}
  \\      &\implies \quad \text{$\opTrn$ is \prope{unitary}}
          && \text{by the definition of \prope{unitary} operators\ifsxref{operator}{def:op_unitary}}
  \\
  \\\opDili &= \opDila && \text{by \prefp{prop:opDi} and \prefp{prop:wavstrct_Da}}
  \\      &\implies \quad \text{$\opDil$ is \prope{unitary}}
          && \text{by the definition of \prope{unitary} operators\ifsxref{operator}{def:op_unitary}}
\end{align*}
\end{proof}



%=======================================
\section{Normed linear space properties}
%=======================================
%---------------------------------------
\begin{proposition}
\label{prop:opD_constant}
%---------------------------------------
Let $\opDil$ be the \fncte{dilation operator} \xref{def:opD}.
\propbox{
  \brb{\begin{array}{FMD}
    (1). & $\ds\opDil\ff(x) = \sqrt{2}\ff(x)$ & and\\
    (2). & $\ff(x)$ is \prope{continuous}
  \end{array}}
  \qquad\iff\qquad
  \brb{\text{$\ff(x)$ is a \prope{constant}}}
  \qquad\scy\forall \ff\in\spLLR
  }
\end{proposition}
\begin{proof}
\begin{enumerate}
  \item Proof that (1)$\impliedby$ \prope{constant} property:
    \begin{align*}
      \opDil\ff(x)
        &\eqd \sqrt{2}\ff\brp{2x}
        &&    \text{by definition of $\opDil$}  && \text{\xref{def:opD}}
      \\&=    \sqrt{2}\ff\brp{x}
        &&    \text{by \prope{constant} hypothesis}
    \end{align*}

  \item Proof that (2)$\impliedby$ \prope{constant} property:
    \begin{align*}
      \norm{\ff(x)-\ff(x+h)}
        &= \norm{\ff(x)-\ff(x)}
           \qquad\text{by \prope{constant} hypothesis}
      \\&= \norm{0}
      \\&= 0
           \qquad\text{by \prope{nondegenerate} property of $\normn$} && \text{\ifxref{vsnorm}{def:norm}}
      \\&\le \varepsilon
      \\&\implies \forall h>0,\,\exists\varepsilon\st\norm{\ff(x)-\ff(x+h)}<\varepsilon
      \\&\iffdef  \text{$\ff(x)$ is \prope{continuous}}
    \end{align*}

  \item Proof that (1,2)$\implies$ \prope{constant} property:
    \begin{enumerate}
      \item Suppose there exists $x,y\in\R$ such that $\ff(x)\neq\ff(y)$. \label{item:opD_constant_assumption}
      \item Let $\seqnZp{x_n}$ be a sequence with limit $x$ and $\seqnZp{y_n}$ a sequence with limit $y$ \label{item:opD_constant_xn}
      \item Then
        \begin{align*}
          0
            &< \norm{\ff(x)-\ff(y)}
            && \text{by assumption in \prefp{item:opD_constant_assumption}}
          \\&= \lim_{n\to\infty}\norm{\ff\brp{x_n}-\ff\brp{y_n}}
            && \text{by (2) and definition of $\seqn{x_n}$ and $\seqn{y_n}$ in \prefp{item:opD_constant_xn}}
          \\&= \lim_{n\to\infty}\norm{\ff\brp{2^m x_n}-\ff\brp{2^\ell y_n}}\quad\forall m,\ell\in\Z
            && \text{by (1)}
          \\&= 0
        \end{align*}
      \item But this is a \emph{contradiction}, so $\ff(x)=\ff(y)$ for all $x,y\in\R$, and $\ff(x)$ is \prope{constant}.
    \end{enumerate}
\end{enumerate}
\end{proof}

%---------------------------------------
\begin{remark}
%---------------------------------------
\remboxt{
  In \prefp{prop:opD_constant}, it is not possible to remove the \prope{continuous}\\
  constraint outright, as demonstrated by the next two counterexamples.
  }
\end{remark}

%---------------------------------------
\begin{counterex}
%---------------------------------------
Let $\ff(x)$ be a function in $\spRR$.
\cntbox{
   \begin{array}{M}
   Let $\ds\ff(x)\eqd\brbl{\begin{array}{lM}
     0 & for $x=0$\\
     1 & otherwise.
   \end{array}}$
   \\
   Then $\opDil\ff(x)\eqd\sqrt{2}\ff\brp{2x}=\sqrt{2}\ff(x)$, but $\ff(x)$ is \prope{not constant}.
   \end{array}
   \hspace{10mm}
   \begin{array}{M}
   {\footnotesize
   \psset{unit=8mm}%
   \begin{pspicture}(-2.5,-0.6)(2.5,2)%
     \psaxes[linecolor=axis,yAxis=false]{<->}(0,0)(-2.5,0)(2.5,2)%
     \psaxes[linecolor=axis,xAxis=false]{ ->}(0,0)(-2.5,0)(2.5,2)%
     \uput[0](2.5,0){$x$}%
     \psline[linecolor=blue,linestyle=dotted]   (-2.5,1)(-2,  1)%
     \psline[linecolor=blue,linestyle=solid,dotsize=5pt]{-o}(-2,  1)( 0,  1)%
     \psline[linecolor=blue,linestyle=solid,dotsize=5pt]{o-}( 0,  1)( 2,  1)%
     \psline[linecolor=blue,linestyle=dotted]   ( 2,  1)( 2.5,1)%
     \psdot[dotsize=4pt](0,0)%
     \rput[bl](0.5,0.6){$\ff(x)$}%
   \end{pspicture}}%
   \end{array}
  }
\end{counterex}

%---------------------------------------
\begin{counterex}
%---------------------------------------
Let $\ff(x)$ be a function in $\spRR$.\\
Let $\hxs{\Q}$ be the set of \structe{rational numbers} and
$\R\setd\Q$ the set of \structe{irrational numbers}.
\cntbox{
   \begin{array}{M}
   Let $\ds\ff(x)\eqd\brbl{\begin{array}{rM}
     1 & for $x\in\Q$\\
    -1 & for $x\in\R\setd\Q$.
   \end{array}}$
   \\
   Then $\opDil\ff(x)\eqd\sqrt{2}\ff\brp{2x}=\sqrt{2}\ff(x)$, but $\ff(x)$ is \prope{not constant}.
   \end{array}
   \hspace{10mm}
   \begin{array}{M}
   {\footnotesize
   \psset{xunit=8mm,yunit=5mm}%
   \begin{pspicture}(-2.5,-2)(2.5,2)%
     \psaxes[linecolor=axis]{<->}(0,0)(-2.5,-1.9)(2.5,1.9)%
     %\psaxes[linecolor=axis,xAxis=false]{ ->}(0,0)(-2.5,-1.9)(2.5,1.9)%
     \uput[0](2.5,0){$x$}%
     \psline[linecolor=blue,linestyle=dotted]   (-2.5,1)(2.5,  1)%
     \psline[linecolor=red,linestyle=dotted]   (-2.5,-1)(2.5, -1)%
     %\rput[bl](0.5,0.6){$\ff(x)$}%
   \end{pspicture}}%
   \end{array}
  }
\end{counterex}



%--------------------------------------
\begin{proposition}[Operator norm]
\label{prop:wavstrct_TD_unitary_1}
%--------------------------------------
%Let $\opTrn$ be the translation operator with inverse $\opTrni$ and adjoint $\opTrna$
%and $\opDil$ be the dilation    operator with inverse $\opDili$ and adjoint $\opDila$.
%Let $\inprodn$ be the inner product induced by the operator $\int$ in $\spLLR$,
%$\normn$ the norm induced by $\inprodn$, and
%$\normopn$ be the \hi{operator norm}\ifsxref{operator}{def:op_norm} induced by $\normn$.
Let $\opTrn$ and $\opDil$ be as in \prefp{def:opT}.
Let $\opTrni$ and $\opDili$ be as in \prefp{prop:opTi}.
Let $\opTrna$ and $\opDila$ be as in \prefp{prop:opTa}.
Let $\normn$ and $\inprodn$ be as in \prefp{def:spLLR}.
Let $\normopn$ be the \hi{operator norm}\ifsxref{operator}{def:op_norm} induced by $\normn$.
\propbox{
  \normop{\opTrn} = \normop{\opDil} = \normop{\opTrna} = \normop{\opDila} = \normop{\opTrni} = \normop{\opDili} = 1
  }
%\propbox{\begin{array}{lclclD}
%    \normop{\opTrn}                   &=& \normop{\opDil}               &=& 1                 & (\prope{unit length})
%  \\\normop{\opTrna}                  &=& \normop{\opDila}              &=& 1                 & (\prope{unit length})
%  \\\normop{\opTrni}                  &=& \normop{\opDili}              &=& 1                 & (\prope{unit length})
%\end{array}}
\end{proposition}
\begin{proof}
  These results follow directly from the fact that $\opTrn$ and $\opDil$ are \prope{unitary}
  \ifsxref{operator}{prop:TD_unitary} and from
  \ifdochaselse{operator}{\prefp{thm:unitary_equiv} and \prefp{thm:unitary_prop}.}
                         {properties of unitary operators.}
\end{proof}

%--------------------------------------
\begin{theorem}
\label{thm:TD_unitary}
%--------------------------------------
%Let $\opTrn$ be the translation operator with inverse $\opTrni$ and $\opDil$ the dilation operator with inverse $\opDili$.
Let $\opTrn$ and $\opDil$ be as in \prefp{def:opT}.\\
Let $\opTrni$ and $\opDili$ be as in \prefp{prop:opTi}.
Let $\normn$ and $\inprodn$ be as in \prefp{def:spLLR}.
\thmbox{\begin{array}{Flclcl@{\qquad}C@{\qquad}D}
    1. & \norm{\opTrn\ff}                    &=& \norm{\opDil\ff}                &=& \norm{\ff}        & \forall\ff    \in\spLLR & (\prope{isometric in length})
  \\2. & \norm{\opTrn\ff-\opTrn\fg}          &=& \norm{\opDil\ff-\opDil\fg}      &=& \norm{\ff-\fg}    & \forall\ff,\fg\in\spLLR & (\prope{isometric in distance})
  \\3. & \norm{\opTrni\ff-\opTrni\fg}        &=& \norm{\opDili\ff-\opDili\fg}    &=& \norm{\ff-\fg}    & \forall\ff,\fg\in\spLLR & (\prope{isometric in distance})
  \\4. & \inprod{\opTrn\ff}{\opTrn\fg}       &=& \inprod{\opDil\ff}{\opDil\fg}   &=& \inprod{\ff}{\fg} & \forall\ff,\fg\in\spLLR & (\prope{surjective})
  \\5. & \inprod{\opTrni\ff}{\opTrni\fg}     &=& \inprod{\opDili\ff}{\opDili\fg} &=& \inprod{\ff}{\fg} & \forall\ff,\fg\in\spLLR & (\prope{surjective})
\end{array}}
\end{theorem}
\begin{proof}
  These results follow directly from the fact that $\opTrn$ and $\opDil$ are \prope{unitary}
  \xrefP{prop:TD_unitary} and from
  \ifdochaselse{operator}{\prefp{thm:unitary_equiv} and \prefp{thm:unitary_prop}.}
                         {properties of unitary operators.}
\end{proof}



%---------------------------------------
\begin{proposition}
\label{prop:vsmra_real_sa}
%---------------------------------------
%Let $\opTrn$ be the translation operator.
Let $\opTrn$ be as in \prefp{def:opT}.
%in a Hilbert space $\spH\eqd\HspaceX[\clFrc]$\ifsxref{seq}{def:hilbert}.
Let $\opAa$ be the \ope{adjoint}\ifsxrefs{operator}{def:norm_adjoint} of an operator $\opA$.
\ifdochas{operator}{Let the property ``\prope{self adjoint}" be defined as in \prefpp{def:op_selfadj}.}
\propbox{
  \begin{array}{lcl@{\qquad\qquad}C}
  \ds\brp{\sum_{n\in\Z} \opTrn^n} &=& \ds\brp{\sum_{n\in\Z} \opTrn^n}^\ast
    & \ds\brp{\text{The operator } \brs{\sum_{n\in\Z} \opTrn^n} \text{ is \prope{self-adjoint}}}
  \end{array}
  }
\end{proposition}
\begin{proof}
   %Let $\opA\eqd\brp{\sum_{n\in\Z} \opTrn^n}$.
    \begin{align*}
      %\inprod{(\opA\ff)(x)}{\fg(x)}
        \inprod{\brp{\sum_{n\in\Z} \opTrn^n}\ff(x)}{\fg(x)}
        &= \inprod{\sum_{n\in\Z} \ff(x-n)}{\fg(x)}
        && \text{by definition of $\opTrn$}  && \text{\xref{def:opT}}
      \\&= \inprod{\sum_{n\in\Z} \ff(x+n)}{\fg(x)}
        && \text{by \prope{commutative} property} && \text{\xref{def:field}}
      \\&= \sum_{n\in\Z} \inprod{\ff(x+n)}{\fg(x)}
        && \text{by \prope{additive} property of $\inprodn$} && \text{\ifxref{vsinprod}{def:inprod}}
      \\&= \sum_{n\in\Z} \inprod{\ff(u)}{\fg(u-n)}
        && \text{where $u\eqd x+n$}
      \\&= \inprodr{\ff(u)}{\sum_{n\in\Z} \fg(u-n)}
        && \text{by \prope{additive} property of $\inprodn$} && \text{\ifxref{vsinprod}{def:inprod}}
      %\\&= \inprod{\ff(u)}{\sum_{n\in\Z} \fg(u+n)}
      %  && \text{by \prop{commutative} property of addition}
      \\&= \inprodr{\ff(x)}{\sum_{n\in\Z} \fg(x-n)}
        &&  \text{by change of variable: $u\rightarrow x$}
      \\&= \inprodr{\ff(x)}{\sum_{n\in\Z} \opTrn^{n}\fg(x)}
        && \text{by definition of $\opTrn$} && \text{\xref{def:opT}}
      %\\&= \inprod{\ff(x)}{(\opA\fg)(x)}
      %  && \text{by definition of $\opA$}
      \\&\iff \brp{\sum_{n\in\Z}\opTrn^n} = \brp{\sum_{n\in\Z}\opTrn^n}^\ast
      %\\&\iff \opA=\opAa
        && \text{by definition of \ope{adjoint}} && \text{\ifsxref{operator}{prop:op_adjoint}}
      \\&\iff \text{$\brp{\sum_{n\in\Z}\opTrn^n}$ is \prope{self-adjoint}}
        && \text{by definition of \prope{self-adjoint}} && \text{\ifsxref{operator}{def:op_selfadj}}
    \end{align*}
\end{proof}




%=======================================
\section{Fourier transform properties}
%=======================================
%---------------------------------------
\begin{proposition}
\label{prop:wavstrct_BTD}
%---------------------------------------
%Let $\opTrn$ be the translation operator and $\opDil$ be the dilation operator in the space $\clFrc$.
Let $\opTrn$ and $\opDil$ be as in \prefp{def:opT}.\\
Let $\opBT$ be the \hie{two-sided Laplace transform} defined as
  $\brs{\opBT \ff}(s) \eqd \cft  \int_{\R} \ff(x) e^{-sx} \dx$.
\propbox{
  \begin{array}{F>{\ds}rc>{\ds}l c>{\ds}lCD}
      1.& \opBT\opTrn^n      &=& e^{-s n} \opBT      & &        & \forall n\in\Z &
    \\2.& \opBT\opDil^j      &=& \opDil^{-j}\opBT    & &        & \forall j\in\Z &
    \\3.& \opDil\opBT        &=& \opBT\opDili        & &        & \forall n\in\Z &
    \\4.& \opBT\opDili\opBTi &=& \opBTi\opDili\opBT  &=& \opDil & \forall n\in\Z & ($\opDili$ is \prope{similar} to $\opDil$)
    \\5.& \opDil\opBT\opDil  &=& \opDili\opBT\opDili &=& \opBT  & \forall n\in\Z &
  \end{array}
  }
\end{proposition}
\begin{proof}
\begin{align*}
  \opBT\opTrn^n \ff(x)
    &= \opBT \ff(x-n)
    && \text{by definition of $\opTrn$} && \text{\xref{def:opT}}
  \\&= \cft  \int_\R \ff(x-n) e^{-sx} \dx
    && \text{by definition of $\opBT$}
  \\&= \cft  \int_\R \ff(u) e^{-s (u+n)} \du
    && \text{where $u\eqd x-n$}
  \\&= e^{-sn}\;\brs{\cft  \int_\R \ff(u) e^{-s u} \du}
  \\&= e^{-sn}\;\opBT \ff(x)
    && \text{by definition of $\opBT$}
  \\
  \\
  \opBT\opDil^j\ff(x)
    &= \opBT \brs{2^{j/2}\,\ff\brp{2^jx}}
    && \text{by definition of $\opDil$} && \text{\xref{def:opD}}
  \\&= \cft \,\int_\R \brs{2^{j/2}\,\ff\brp{2^jx}} e^{-sx} \dx
    && \text{by definition of $\opBT$}
  \\&= \cft \,\int_\R \brs{2^{j/2}\,\ff(u)} e^{-s 2^{-j}} 2^{-j}\du
    && \text{let $u\eqd 2^jx \implies x=2^{-j}u$}
  \\&= \cwt \, \cft \,
       \int_\R \ff(u) e^{-s 2^{-j} u} \du
  \\&= \opDili\,\brs{ \cft \,
       \int_\R \ff(u) e^{-s u} \du}
    && \text{by \prefp{prop:wavstrct_Da} and} && \text{\prefp{prop:TD_unitary}}
  \\&= \opDil^{-j}\, \opBT \, \ff(x)
    && \text{by definition of $\opBT$}
  \\
  \opDil\opBT\,\ff(x)
    &= \opDil \brs{\cft  \int_\R \ff(x) e^{-sx} \dx  }
    && \text{by definition of $\opBT$}
  \\&= \frac{\sqrt{2}}{\sqrt{2\pi}} \int_\R \ff(x) e^{-2sx} \dx
    && \text{by definition of $\opDil$} && \text{\xref{def:opD}}
  \\&= \frac{\sqrt{2}}{\sqrt{2\pi}} \int_\R \ff\brp{\frac{u}{2}} e^{-s u} \half \du
    && \text{let $u\eqd 2x \implies x=\half u$}
  \\&= \cft  \int_\R \brs{\cwt  \ff\brp{\frac{u}{2}} }e^{-s u} \du
  \\&= \cft  \int_\R \brs{\opDili \ff}(u) \,e^{-s u} \du
    && \text{by \prefp{prop:wavstrct_Da} and} &&\text{\prefp{prop:TD_unitary}}
  \\&= \opBT\opDili \ff(x)
    && \text{by definition of $\opBT$}
  \\
  \opBTi\opDili\opBT
    &= \opBTi\opBT\opDil
    && \text{by previous result}
  \\&= \opDil
    && \text{by definition of operator inverse} && \text{\ifxref{operator}{def:op_inv}}
  \\
  \opBT\opDili\opBTi
    &= \opDil\opBT\opBTi
    && \text{by previous result}
  \\&= \opDil
    && \text{by definition of operator inverse} && \text{\ifxref{operator}{def:op_inv}}
  \\
  \opDil\opBT\opDil
    &= \opDil\opDili\opBT
    && \text{by previous result}
  \\&= \opBT
    && \text{by definition of operator inverse} && \text{\ifxref{operator}{def:op_inv}}
  \\
  \opDili\opBT\opDili
    &= \opDili\opDil\opBT
    && \text{by previous result}
  \\&= \opBT
    && \text{by definition of operator inverse} && \text{\ifxref{operator}{def:op_inv}}
\end{align*}
%These results follow from \prefp{prop:vsmra_real_FD}.
\end{proof}


%---------------------------------------
\begin{corollary}
%\label{prop:vsmra_real_FT}
%\label{prop:vsmra_real_FD}
\label{cor:wavstrct_FTD}
\label{cor:FTD}
%---------------------------------------
%Let $\opTrn$ be the translation operator and $\opDil$ be the dilation operator.
%in the space $\clFrc$, and
%Let $\opFT$ be the Fourier transform\ifsxref{harFour}{def:ft}. % as defined below:
Let $\opTrn$ and $\opDil$ be as in \prefp{def:opT}.
Let $\Ff(\omega)\eqd\opFT\ff(x)$ be the \ope{Fourier Transform} \xref{def:opFT} of some function $\ff\in\spLLR$ \xref{def:spLLR}.
%  \[\brs{\opFT \ff}(\omega) \eqd \cft  \int_{x\in\R} \ff(x) e^{-i\omega t} \dx\]
\corbox{
  \begin{array}{F>{\ds}rc>{\ds}l c>{\ds}l}
      1.& \opFT\opTrn^n  &=& e^{-i\omega n} \opFT
    \\2.& \opFT\opDil^j  &=& \opDil^{-j}\opFT
    \\3.& \opDil\opFT    &=& \opFT\opDili
    \\4.& \opDil         &=& \opFT\opDili\opFTi   &=& \opFTi\opDili\opFT
    \\5.& \opFT          &=& \opDil\opFT\opDil    &=& \opDili\opFT\opDili
  \end{array}
  }
\end{corollary}
\begin{proof}
These results follow directly from \prefp{prop:wavstrct_BTD} with $\opFT = \left.\opBT\right|_{s=i\omega}$.
\end{proof}

%---------------------------------------
\begin{proposition}
\label{prop:FTDf}
%---------------------------------------
%Let $\opTrn$ be the translation operator and $\opDil$ be the dilation operator.
%Let $\opTrn$ and $\opDil$ be as in \prefp{def:opT}.
%Let $\opFT$ be the Fourier transform\ifsxref{harFour}{def:ft}. % as defined below:
%Let $\Ff(\omega)\eqd\opFT\ff(x)$ for some function $\ff\in\spLLR$.
%  \[\brs{\opFT \ff}(\omega) \eqd \cft  \int_{x\in\R} \ff(x) e^{-i\omega t} \dx\]
Let $\opTrn$ and $\opDil$ be as in \prefp{def:opT}.
Let $\Ff(\omega)\eqd\opFT\ff(x)$ be the \ope{Fourier Transform} \xref{def:opFT} of some function $\ff\in\spLLR$ \xref{def:spLLR}.
\propbox{
  \opFT\opDil^j\opTrn^n \ff(x)  = \frac{1}{2^{j/2}} e^{-i\frac{\omega}{2^j} n} \Ff\brp{\frac{\omega}{2^j}}
  }
\end{proposition}
\begin{proof}
     \begin{align*}
       \opFT\opDil^j\opTrn^n\ff(x)
         &= \opDil^{-j}\opFT\opTrn^n\ff(x)
         && \text{by \prefp{cor:FTD} (3)}
       \\&= \opDil^{-j} e^{-i\omega n}\opFT\ff(x)
         && \text{by \prefp{cor:FTD} (3)}
       \\&= \opDil^{-j} e^{-i\omega n}\Ff\brp{\omega}
       \\&= 2^{-j/2} e^{-i2^{-j}\omega n}\Ff\brp{2^{-j}\omega}
         && \text{by \prefp{prop:opDi}}
     \end{align*}
\end{proof}

%---------------------------------------
\begin{proposition}
\label{prop:Fsum_af}
%---------------------------------------
Let $\opTrn$ be the translation operator \xref{def:opT}.
Let $\Ff(\omega)\eqd\opFT\ff(x)$ be the \ope{Fourier Transform} \xref{def:opFT} of a function $\ff\in\spLLR$.
Let $\Da(\omega)$ be the \ope{DTFT} \xref{def:dtft} of a sequence $\seqxZ{a_n}\in\spllR$ \xref{def:spllR}.
\propbox{
  \opFT\sum_{n\in\Z}a_n \opTrn^n\fphi(x) = \Da(\omega)\Fphi(\omega)
  \qquad\scy\forall\seqn{a_n}\in\spllR,\,\fphi(x)\in\spLLR
  }
\end{proposition}
\begin{proof}
     \begin{align*}
       \opFT\sum_{n\in\Z}a_n \opTrn^n\fphi(x)
         &= \sum_{n\in\Z}a_n \opFT\opTrn^n\fphi(x)
       \\&= \sum_{n\in\Z}a_n e^{-i\omega n}\opFT\fphi(x)
         && \text{by \prefp{cor:FTD}}
       \\&= \brs{\sum_{n\in\Z}a_n e^{-i\omega n}}\Fphi(\omega)
         && \text{by definition of $\Fphi(\omega)$}
       \\&= \Da(\omega)\Fphi(\omega)
         && \text{by definition of \ope{DTFT} \xref{def:dtft}}
     \end{align*}
\end{proof}


%======================================
%\section{Poisson Summation Formulas}
\label{sec:psf}
%======================================
%\qboxnps
%  {
%  Sim\'eon Denis Poisson (1781--1840) French mathematician and physicist
%  \index{Poisson, Sim\'eon Denis}
%  \index{quotes!Poisson, Sim\'eon Denis}
%  \footnotemark
%  }
%  {../common/people/poisson.jpg}
%  {Life is good for only two things, discovering mathematics and teaching mathematics.}
%  \footnotetext{\begin{tabular}[t]{ll}
%    %quote: & \url{http://www-history.mcs.sx-andrews.ac.uk/Biographies/Poisson.html} \\
%    quote: & \citerpg{eves1990}{486}{0030295580}\\
%    image: & \url{http://en.wikipedia.org/wiki/Poisson}
%  \end{tabular}}

%The \hie{Poisson Summation Formula} (PSF, \prefpo{thm:psf}) and
%\hie{Inverse Poisson Summation Forumla} (IPSF, \prefpo{thm:ipsf})
%are fundamental and extremely powerful theorems in Harmonic analysis.


%--------------------------------------
\begin{definition}
\label{def:opS}
%--------------------------------------
Let $\spLLRBu$ be the \structe{space of Lebesgue square-integrable functions} \xref{def:spLLRBu}.
Let $\spllR$ be the \structe{space of all absolutely square summable sequences over $\R$} \xref{def:spLLR}.
\defboxt{
  $\opS$  is the \opd{sampling operator} in $\clO{\spLLR}{\spllR}$ if
  \quad$\ds\brs{\opS\ff(x)}(n) \eqd \ff\brp{\frac{2\pi}{\tau} n}\qquad \scy\forall \ff\in\spLLRBu,\,\tau\in\Rp$
  }
\end{definition}


%--------------------------------------
\begin{theorem}[\thmd{Poisson Summation Formula}---\thmd{PSF}]
\footnote{
  \citerp{andrews}{624},
  \citerpg{knappb2005}{389}{0817632506},
  \citerp{lasser}{254},
  %\citerp{goswami}{44}  \\
  \citerpp{rudinr}{194}{195},
  \citerp{folland}{337}
  }
\label{thm:psf}
%--------------------------------------
Let $\Ff(\omega)$ be the \fncte{Fourier transform} \xref{def:opFT} of a function $\ff(x)\in\spLLR$.
Let $\opS$ be the \ope{sampling operator} \xref{def:opS}.
%Let $\opFT$ be the \hie{Fourier Transform} operator,
%    $\opFS$ the \hie{Fourier Series} operator,
%    $\opFSi$ the \hie{inverse Fourier Series} operator,
%and $\Ff(\omega)\eqd\opFT\ff(x)$ (the Fourier transform of a function $\ff(x)$).
\thmboxt{
   $\ds
   \mcom{\ds\sum_{n\in\Z} \opTrn_{\tau}^n\ff\brp{x} =
            \sum_{n\in\Z} \ff\brp{x+n\tau}}
        {summation in ``time"}
   =
   \mcom{\ds\sqrt{\frac{2\pi}{\tau}}\: \opFSi \opS\opFT\brs{\ff(x)}}
        {operator notation}
        %{\parbox{6\tw/16}{proportional to the inverse Fourier series
        % of the Fourier transform of $\ff(x)$
        % sampled at $\frac{2\pi}{\tau}$ intervals}
        %}
   =
   \mcom{\ds\frac{\sqrt{2\pi}}{\tau} \sum_{n\in\Z} \Ff\brp{\frac{2\pi}{\tau}n} e^{i\frac{2\pi}{\tau}nx}}
        {summation in ``frequency"}
  $
  %\\\qquad
  %  $\begin{array}{Ml}
  %    where & \text{$\opS\in\clO{\spLLR}{\spllR}$ is the \ope{sampling operator} defined as}\\
  %          & \brs{\opS\ff(x)}(n) \eqd \ff\brp{\frac{2\pi}{\tau} n}\qquad \forall \ff\in\spLLRBu,\,\tau\in\Rp
  %  \end{array}$
  }
\end{theorem}

\begin{proof}
%\begin{enumerate}
%\item Proof using Fourier series (\prefpo{def:opFS}):
\begin{enumerate}
  \item lemma: If $\ds \fh(x)    \eqd \sum_{n\in\Z} \ff(x+n\tau)$ then $\fh\equiv\opFSi\opFS\fh$.
        Proof: \label{ilem:psf_h}\\
        Note that $\fh(x)$ is \prope{periodic} with period $\tau$\ifsxref{sums}{lem:series_sumT}.
        %\[
        %   \fh(x+\tau) \eqd \sum_{n\in\Z} \ff(x+T+n\tau) = \sum_{n\in\Z} \ff(x+(n+1)\tau) = \sum_{n\in\Z} \ff(x+n\tau) \eqd \fh(x).
        %\]
        Because $\fh$ is periodic, it is in the domain of $\opFS$ and thus $\fh\equiv\opFSi\opFS\fh$.
  \item Proof of PSF (this theorem---\pref{thm:psf}):
    \begin{align*}
       \sum_{n\in\Z} \ff(x+n\tau)
           &= \opFSi \opFS \sum_{n\in\Z} \ff(x+n\tau)
           && \text{by \prefp{ilem:psf_h}}
         \\&= \opFSi \brn{
                \mcom{\brs{
                  \fsscale\int_0^\tau \brp{\sum_{n\in\Z} \ff(x+n\tau)} e^{-i\frac{2\pi}{\tau}kx} \dx
                  }}{$\opFS \brs{\sum_{n\in\Z} \ff(x+n\tau)}$}
                }
            && \text{by definition of $\opFS$} && \text{\xref{def:opFS}}
         \\&= \opFSi \brs{
                \fsscale \sum_{n\in\Z}
                \int_{0}^\tau \ff(x+n\tau) e^{-i\frac{2\pi}{\tau}kx} \dx
                }
         \\&= \opFSi \brs{
                \fsscale
                \sum_{n\in\Z}
                \int_{u=n\tau}^{u=(n+1)\tau} \ff(u) e^{-i\frac{2\pi}{\tau}k(u-n\tau)} \du
                }
           && \text{where $u\eqd x+n\tau$ $\implies$} && \text{$x=u-n\tau$}
         \\&= \opFSi \brs{
                \fsscale \sum_{n\in\Z}
                \cancelto{\scy1}{e^{i2\pi kn}}\;
                \int_{u=n\tau}^{u=(n+1)\tau} \ff(u) e^{-i\frac{2\pi}{\tau}ku} \du
                }
         \\&= \sqrt{\frac{2\pi}{\tau}} \opFSi \brn{
                \mcom{\brs{\fscale \int_{u\in\R} \ff(u) e^{-i\left(\frac{2\pi}{\tau}k\right)u} \du}}
                     {$\brs{\opFT\ff}\brp{\frac{2\pi}{\tau}k}$}
                }
           && \text{by evaluation of $\opFSi$} && \text{\xref{thm:opFSi}}
         \\&= \sqrt{\frac{2\pi}{\tau}}\: \opFSi \brs{\brs{\opFT\ff(x)}\brp{\frac{2\pi}{\tau}k}}
           && \text{by definition of $\opFT$} && \text{\xref{def:opFT}}
         \\&= \sqrt{\frac{2\pi}{\tau}}\: \opFSi\opS\opFT \ff
           && \text{by definition of $\opS$}
           && \text{\xref{def:opS}}
         \\&= \frac{\sqrt{2\pi}}{\tau}\:
              \sum_{n\in\Z} \Ff\brp{\frac{2\pi}{\tau}n} e^{i\frac{2\pi}{\tau}nx}
           && \text{by evaluation of $\opFSi$} && \text{\xref{thm:opFSi}}
    \end{align*}
\end{enumerate}
%\item Proof using \hie{spectral theorem}\ifdochas{operator}{ (\prefpo{thm:spectral_theorem})}:
%\begin{enumerate}
%  \item Define operator $\opA$ as
%        \[ (\opA \ff)(x) \eqd \sum_{n\in\Z} \ff(x+n) \]
%
%  \item $\opA$ can be expressed as an integral operator with
%        kernel $\fh(t,u) = \sum_{n\in\Z} \delta(u-x-n)$:
%        \begin{align*}
%          (\opA \ff)(x)
%            &= \sum_{n\in\Z} \ff(x+n)
%          \\&= \sum_{n\in\Z} \int_{u\in\R} \delta(u-x-n) \ff(u) \du
%            && \text{by \prefp{def:dirac}}
%          \\&= \int_{u\in\R} \mcom{\sum_{n\in\Z} \delta(u-x-n)}{kernel $\fh(t,u)$} \ff(u) \du
%        \end{align*}
%
%  \item $\opA$ has eigenpairs $\brp{2\pi\sum_{n\in\Z} \delta(\omega-2\pi n),\, e^{i\omega x}}$:
%        \begin{align*}
%          \opA e^{i\omega x}
%            &= \sum_{n\in\Z} e^{i\omega(x+n)}
%          \\&= \mcomr{\brp{\sum_{n\in\Z} e^{i\omega n}}}{Dirichlet kernel}\: e^{i\omega x}
%          \\&= \mcom{\brp{2\pi\sum_{n\in\Z} \delta(\omega-2\pi n)}}{eigenvalue}\:
%               \mcoml{e^{-i\omega x}}{eigenvector}
%            && \text{by \prefp{prop:dkernel}}
%        \end{align*}
%
%  \item $\opA$ is \hie{self adjoint}:%
%    \ifdochas{operator}{\footnote{\hie{self ajoint}: \prefp{def:op_selfadj}}}
%        \begin{align*}
%          \inprod{(\opA\ff)(x)}{\fg(x)}
%            &= \inprod{\sum_{n\in\Z} \ff(x+n)}{\fg(x)}
%          \\&= \sum_{n\in\Z} \inprod{\ff(x+n)}{\fg(x)}
%            && \text{by additivity property of $\inprodn$ \ifdochas{vsinprod}{\prefpo{def:inprod}}}
%          \\&= \sum_{n\in\Z} \inprod{\ff(u)}{\fg(u-n)}
%            && \text{where $u=x+n$}
%          \\&= \inprod{\ff(u)}{\sum_{n\in\Z} \fg(u-n)}
%          \\&= \inprod{\ff(u)}{\sum_{n\in\Z} \fg(u+n)}
%          \\&= \inprod{\ff(x)}{\sum_{n\in\Z} \fg(x+n)}
%          \\&= \inprod{\ff(x)}{(\opA\fg)(x)}
%          \\&\implies \opA=\opAa
%        \end{align*}
%
%  \item Because the operator $\opA$ is self adjoint, it is also \hie{normal}.%
%        \ifdochas{operator}{\footnote{\hie{normal}: \prefp{def:op_normal}}}
%
%  \item Because the operator $\opA$ is normal, it can be evaluated using the \hie{spectral theorem}
%        \footnote{\hie{spectral theorem}: \ifdochas{operator}{\prefp{thm:spectral_theorem}}}.
%        \begin{align*}
%          \sum_{n\in\Z} \ff(x+n\tau)
%            &= (\opA \ff)(x)
%            && \text{by definition of $\opA$}
%          \\&= \sum_{n\in\Z} \lambda_n \opP_n \ff
%            && \text{by spectral theorem \ifdochas{operator}{\prefpo{thm:spectral_theorem}}}
%          \\&= \sum_{n\in\Z} \brs{2\pi\sum_{m\in\Z} \delta(\omega_n-2\pi m)}\:
%                      \inprod{\ff(x)}{e^{i\omega_n t}}
%          \\&= \sum_{n\in\Z} \brs{2\pi\sum_{m\in\Z} \delta(\omega_n-2\pi m)}\:
%                      \sqrt{2\pi}(\opFT\ff)(\omega_n)
%          \\&= \brp{\sqrt{2\pi}}^\frac{3}{2}
%               \sum_{n\in\Z} \brs{\sum_{m\in\Z} \delta(\omega_n-2\pi m)}\:
%                      (\opFT\ff)(2\pi m)
%        \end{align*}
%\end{enumerate}
%\end{enumerate}
\end{proof}


%--------------------------------------
\begin{theorem}[\thmd{Inverse Poisson Summation Formula}---\thmd{IPSF}]
\footnote{
  \citerp{gauss1900werke8}{88}
  }
\label{thm:ipsf}
\index{Inverse Poisson Summation Formula}
\index{theorems!Inverse Poisson Summation Formula}
\index{IPSF}
%--------------------------------------
\\Let $\Ff(\omega)$ be the \fncte{Fourier transform} \xref{def:opFT} of a function $\ff(x)\in\spLLR$.
%Let $\opFT$ be the \hie{Fourier Transform} operator with inverse $\opFTi$,
%    $\opFS$ be the \hie{Fourier Series} operator with inverse $\opFSi$,
%and $\Ff(\omega)$ be the Fourier transform of a function $\ff(x)$
%such that $\Ff \eqd \opFT\ff$.
\thmbox{
   \mcom{\sum_{n\in\Z} \opTrn_{2\pi/\tau}^n\Ff\brp{\omega} \eqd
         \sum_{n\in\Z} \Ff\brp{\omega - \frac{2\pi}{\tau}n}}
        {summation in ``frequency"}
   \qquad=\qquad
   %\mcom{\sqrt{\tau}\: \opFSi \opS \opFTi\Ff }
   %     {\parbox{5\tw/16}{
   %     proportional to the inverse Fourier series of the
   %     inverse Fourier transform of $\Ff(\omega)$
   %     sampled at $kT$ intervals and reversed in time about $x=0$
   %     }}
   %=
   \mcom{\frac{\tau}{\sqrt{2\pi}} \sum_{n\in\Z} \ff(n\tau) e^{-i\omega n\tau}}
        {summation in ``time"}
%  \\\qquad
%    \begin{array}{>{$}l<{$}>{\ds}l}
%      where & \text{$\opS\in\clO{\spLLR}{\spllR}$ is the \ope{sampling operator} defined as}\\
%            & \brs{\opS\ff(x)}(n) \eqd \ff\brp{-n\tau}\qquad\scriptstyle \forall \ff\in\spLLRBu
%    \end{array}
  }
\end{theorem}
\begin{proof}
\begin{enumerate}
  \item lemma: \label{ilem:ipsf_h}
        If $\fh(\omega) \eqd \sum_{n\in\Z} \Ff\left( \omega + \frac{2\pi}{\tau}n  \right)$,
               then  $\fh\equiv\opFSi\opFS\fh$. Proof: \\
        Note that $\fh(\omega)$ is periodic with period $2\pi/T$:
    \[
       \fh\left(\omega+\frac{2\pi}{\tau}\right)
       \eqd \sum_{n\in\Z} \Ff\left( \omega+\frac{2\pi}{\tau}+ \frac{2\pi}{\tau}n \right)
       =      \sum_{n\in\Z} \Ff\left( \omega+ (n+1)\frac{2\pi}{\tau} \right)
       =      \sum_{n\in\Z} \Ff\left( \omega + \frac{2\pi}{\tau}n  \right)
       \eqd \fh(\omega)
    \]
    Because $\fh$ is periodic, it is in the domain of $\opFS$ and is equivalent to $\opFSi\opFS\fh$.
  \item Proof of IPSF (this theorem---\pref{thm:ipsf}):
    \begin{align*}
      &\sum_{n\in\Z} \Ff\left( \omega + \frac{2\pi}{\tau}n \right)
      \\&= \opFSi\opFS\brn{ \sum_{n\in\Z} \Ff\left( \omega + \frac{2\pi}{\tau}n \right) }
        && \text{by \prefp{ilem:ipsf_h}}
      \\&= \opFSi\brn{
             \mcom{\brs{ \sqrt{\frac{\tau}{2\pi}}\int_0^\frac{2\pi}{\tau} \sum_{n\in\Z} \Ff\brp{\omega + \frac{2\pi}{\tau}n} e^{-i\omega\frac{2\pi}{2\pi/\tau}k} \dw }}
                  {$\opFS\brs{\sum_{n\in\Z} \Ff\brp{\omega + \frac{2\pi}{\tau}n}}$}
             }
        && \text{by definition of $\opFS$} && \text{\xref{def:opFS}}
      \\&= \opFSi\brn{
             \brs{ \sqrt{\frac{\tau}{2\pi}}\sum_{n\in\Z} \int_0^\frac{2\pi}{\tau} \Ff\brp{\omega + \frac{2\pi}{\tau}n} e^{-i\omega Tk} \dw }
             }
      \\&= \opFSi\brn{
             \brs{ \sqrt{\frac{\tau}{2\pi}}\sum_{n\in\Z} \int_{u=\frac{2\pi}{\tau}n}^{u=\frac{2\pi}{\tau}(n+1)}
                   \Ff\brp{u} e^{-i\brp{u-\frac{2\pi}{\tau}n} Tk} \du }
             }
        && \text{where $u\eqd\omega + \frac{2\pi}{\tau}n$ $\implies$} && \text{$\omega=u-\frac{2\pi}{\tau}n$}
      \\&= \opFSi\brn{
             \brs{\sqrt{\frac{\tau}{2\pi}} \sum_{n\in\Z}
                  \cancelto{\scy1}{e^{i2\pi nk}}\;
                  \int_{\frac{2\pi}{\tau}n}^{\frac{2\pi}{\tau}(n+1)} \Ff(u) e^{-iu\tau k} \du
                 }
             }
      \\&= \opFSi\brn{
             \brs{\sqrt{\frac{\tau}{2\pi}}
                  \int_{\R} \Ff(u) e^{-iu \tau k} \du
                 }
             }
      \\&= \sqrt{\tau}\: \opFSi\brn{
             \mcom{\brs{
                  \fscalei
                  \int_{\R} \Ff(u) e^{iu(-\tau k)} \du
                 }}{$\brs{\opFTi\Ff}(-k\tau)$}
             }
      \\&= \sqrt{\tau}\: \opFSi \brs{ \brs{\opFTi\Ff}(-k\tau) }
        && \text{by value of $\opFTi$} && \text{\xref{thm:opFTi}}
      \\&= \sqrt{\tau}\: \opFSi \opS \opFTi \:\Ff
        && \text{by definition of $\opS$}
        && \text{\xref{def:opS}}
      \\&= \sqrt{\tau}\: \opFSi \opS \ff(x)
        && \text{by definition of $\opFT$} && \text{\xref{def:opFT}}
      \\&= \sqrt{\tau}\: \opFSi \ff(-k\tau)
        && \text{by definition of $\opS$}
        && \text{\xref{def:opS}}
      \\&= \sqrt{\tau}\: \frac{1}{\sqrt{\frac{2\pi}{\tau}}}\:
           \sum_{k\in\Z} \ff(-k\tau)e^{i2\pi\frac{1}{\frac{2\pi}{\tau}} k\omega}
        && \text{by definition of $\opFSi$} && \text{\xref{thm:opFSi}}
      \\&= \frac{\tau}{\sqrt{\frac{2\pi}{\tau}}}\:
           \sum_{k\in\Z} \ff(-k\tau)e^{i k\tau\omega}
        && \text{by definition of $\opFSi$} && \text{\xref{thm:opFSi}}
      \\&= \frac{\tau}{\sqrt{2\pi}}\sum_{m\in\Z} \ff(m\tau) e^{-i\omega m\tau}
        && \text{let $m\eqd-k$}
    \end{align*}

\end{enumerate}
\end{proof}


%---------------------------------------
\begin{remark}
%---------------------------------------
  The left hand side of the \fncte{Poisson Summation Formula} \xref{thm:psf}
  is very similar to the \fncte{Zak Transform} $\opZ$:
  \footnote{
    \citerp{janssen1988}{24},
    \citerp{zayed}{482}
    %\url{http://mathworld.wolfram.com/ZakTransform.html}
    }
  \\\indentx$\ds(\opZ\ff)(t,\omega) \eqd \sum_{n\in\Z} \ff(x+n\tau)e^{i2\pi n\omega}$
\end{remark}

%---------------------------------------
\begin{remark}
%---------------------------------------
A generalization of the \fncte{Poisson Summation Formula} \xref{thm:psf} is the
\hid{Selberg Trace Formula}.\footnote{
  \citerp{lax}{349},
  \citor{selberg1956},
  \citer{terras1999}
  }
\end{remark}

%=======================================
\section{Basis theory properties}
%=======================================
%--------------------------------------
\begin{example}[\exmd{linear functions}]
\footnote{
  \citerpgc{higgins1996}{2}{0198596995}{1.1 General introduction}
  }
\label{ex:TD_flinear}
%--------------------------------------
Let $\opTrn$ be the \structe{translation operator} \xref{def:opT}.
%Let $\ff$ be a function in $\spLLR$.
Let $\hxs{\clLcc}$ be the set of all \prope{linear} functions in $\spLLR$.
\exbox{\begin{array}{FMD}
  1. & $\setn{x,\,\opTrn x}$ is a \structe{basis} for $\clLcc$ & and \\
  2. & $\ff(x) = \ff(1)x - \ff(0)\opTrn x$                     & $\forall\ff\in\clLcc$
\end{array}}
\end{example}
\begin{proof}
By left hypothesis, $\ff$ is \prope{linear}; so let $\ff(x)\eqd ax + b$
\begin{align*}
  \ff(1)x - \ff(0)\opTrn x
    &= \ff(1)x - \ff(0)(x-1)
    && \text{by \prefp{def:opT}}
  \\&= \left.(ax+b)\right|_{x=1}x - \left.(ax+b)\right|_{x=0}(x-1)
    && \text{by left hypothesis and definition of $\ff$}
  \\&= (a+b)x - b(x-1)
  \\&= ax+bx-bx+b
  \\&= ax+b
  \\&= \ff(x)
    && \text{by left hypothesis and definition of $\ff$}
\end{align*}
\end{proof}

%--------------------------------------
\begin{example}[\exmd{Cardinal Series}]
\label{ex:TD_cardinalseries}
%--------------------------------------
Let $\opTrn$ be the \structe{translation operator} \xrefP{def:opT}.
The \prope{Paley-Wiener} class of functions $\hxs{\spPW_\sigma^2}$\ifsxref{interpo}{def:PW}
are those functions which are ``\hie{bandlimited}"
with respect to their Fourier transform\ifsxref{harFour}{def:ft}.
The cardinal series forms an orthogonal basis for such a space\ifsxref{interpo}{thm:cardinalSeries}.
The \fncte{Fourier coefficients}\ifsxref{frames}{def:fcoef} for a projection of a function $\ff$ onto the Cardinal series basis elements is particularly
simple---these coefficients are samples of $\ff(x)$ taken at regular intervals\ifsxref{interpo}{thm:t_sampling}.
In fact, one could represent the coefficients using inner product notation with the
\structe{Dirac delta distribution} $\delta$ \ifxref{relation}{def:dirac} as
follows: %\footnote{see \prefp{sec:cardinal} for more details}
  \\\indentx$\ds\inprod{\ff(x)}{\opTrn^n\delta(x)} \eqd \int_{\R} \ff(x)\delta(x-n) \dx \eqd \ff(n)$
\exbox{\begin{array}{FMD}
  1. & $\ds\setxZp{\opTrn^n\frac{\sin\brp{\pi x}}{\pi x}}$ is a \structe{basis} for $\spPW_\sigma^2$ & and\\
  2. & $\ds\ff(x) = \mcom{\sum_{n=1}^\infty \ff(n) \opTrn^n\frac{\sin\brp{\pi x}}{\pi x}}{\structe{Cardinal series}}$ & $\forall \ff\in\spPW_\sigma^2,\,\sigma\le \half $
\end{array}}
\end{example}
\ifdochas{interpo}{\begin{proof}
See \prefp{thm:cardinalSeries}.
\end{proof}}

%--------------------------------------
\begin{example}[\exmd{Fourier Series}]
\label{ex:TD_fs}
%--------------------------------------
%Define an alternative dilation operator $\opDil_n$ as
%\\\indentx$\ds\brs{\opDil_n\ff}(x)\eqd\ff(nx)$, $n\in\Z$.
\exbox{\begin{array}{Frc>{\ds}lCD}
  (1). & \mc{4}{M}{$\ds\setxZ{\opDil_n e^{ix}}$ is a \structe{basis} for $\spL\intoo{0}{2\pi}$}         & and\\
  (2). & \ff(x)   &=&    \cft \sum_{n\in\Z} \alpha_n \opDil_n e^{ix}            & \forall x\in\intoo{0}{2\pi},\,\ff\in\spL\intoo{0}{2\pi}   & where\\
       & \alpha_n &\eqd& \cft \int_0^{2\pi} \ff(x) \opDil_n e^{-ix} \dx         & \forall\ff\in\spL\intoo{0}{2\pi}
\end{array}}
\end{example}
\begin{proof}
See \prefp{thm:opFSi}.
\end{proof}

%--------------------------------------
\begin{example}[\exmd{Fourier Transform}]
\label{ex:TD_ft}
\footnote{cross reference: \prefp{def:opFT}}
%--------------------------------------
%Define an alternative dilation operator $\opDil_\omega$ as
%\\\indentx$\ds\brs{\opDil_\omega\ff}(x)\eqd\ff(\omega x)$, $\omega\in\R$.
\exbox{\begin{array}{Frc>{\ds}lCD}
  (1). & \mc{4}{M}{$\ds\set{\opDil_\omega e^{ix}}{\scy\omega\in\R}$ is a \structe{basis} for $\spLLR$}         & and\\
  (2). & \ff(x)      &=&    \cft \int_\R \Ff(\omega) \opDil_x e^{i\omega} \dw  & \forall\ff\in\spLLR   & where\\
       & \Ff(\omega) &\eqd& \cft \int_\R \ff(x) \opDil_\omega e^{-ix} \dx  & \forall\ff\in\spLLR
\end{array}}
\end{example}

%--------------------------------------
\begin{example}[\exmd{Gabor Transform}]
\footnote{
  \citeP{gabor1946},
  \citergc{qian1996}{0132543842}{Chapter 3},
  \citerpgc{forster2009}{32}{0817648909}{Definition 1.69}
  }
\label{ex:TD_gt}
%--------------------------------------
%Define an alternative dilation operator $\opDil_\omega$ and an alternative translation operator $\opTrn_\tau$ as
%\\\indentx$\brs{\opDil_\omega\ff}(x)\eqd\ff(\omega x)$, \qquad and\qquad $\brs{\opTrn_\tau\ff}(x)\eqd\ff(x-\tau)$,\qquad$\scy\omega,\tau\in\R$.
\exbox{\begin{array}{Frc>{\ds}lCD}
  (1). & \mc{4}{M}{$\ds\setbigleft{\brp{\opTrn_\tau e^{-\pi x^2}}\brp{\opDil_\omega e^{ix}}}{\scy\tau,\omega\in\R}$ is a \structe{basis} for $\spLLR$}         & and\\
  (2). & \ff(x)                  &=&    \int_\R \fG\opair{\tau}{\omega} \opDil_x e^{i\omega} \dw & \forall x\in\R,\,\ff\in\spLLR   & where\\
       & \fG\opair{\tau}{\omega} &\eqd& \int_\R \ff(x) \brp{\opTrn_\tau e^{-\pi x^2}}\brp{\opDil_\omega e^{-ix}} \dx & \forall x\in\R,\,\ff\in\spLLR
\end{array}}
\end{example}

%--------------------------------------
\begin{example}[\exmd{wavelets}]
\label{ex:TD_wavelets}
%--------------------------------------
Let $\fpsi(x)$ be a \fncte{wavelet}.
\exbox{\begin{array}{Frc>{\ds}lCD}
  (1). & \mc{4}{M}{$\ds\setbigleft{\opDil^k\opTrn^n \fpsi(x)}{\scy k,n\in\Z}$ is a \structe{basis} for $\spLLR$}         & and\\
  (2). & \ff(x)    &=&    \sum_{k\in\Z}\sum_{n\in\Z} \alpha_{k,n} \opDil^k\opTrn^n\fpsi(x) & \forall \ff\in\spLLR        & where\\
       & \alpha_n  &\eqd& \int_\R \ff(x) \opDil^k\opTrn^n\fpsi^\ast(x) \dx & \forall\ff\in\spLLR
\end{array}}
\end{example}

%%\pref{def:opDalpha} gives some generalizations of \prefp{def:opD}, as well as some additional operators used
%%in the remaining examples in this section.
%%---------------------------------------
%\begin{definition}
%\footnote{
%  \citerppgc{walnut2002}{79}{80}{0817639624}{Definition 3.39}\\
%  \citerppg{christensen2003}{41}{42}{0817642951}\\
%  \citerpg{kammler2008}{A-21}{0521883407}\\
%  \citerpg{bachman2002}{473}{9780387988993}\\
%  \citerpg{packer2004}{260}{0821834029}\\ %{section 3.1}\\
%  \citerpg{zayed2004}{}{0817643044}\\
%  \citerpgc{heil2011}{250}{0817646868}{Notation 9.4}\\
%  \citerpg{casazza1998}{74}{0817639594}\\
%  \citerp{goodman1993}{639}\\
%  \citerp{dai1996}{81}\\
%  \citerpg{dai1998}{2}{0821808001}
%  %\citerpg{dai1998}{21}{0821808001}
%  }
%\label{def:opDalpha}
%\label{def:opE}
%\label{def:opM}
%%---------------------------------------
%%Let $\C$ be the set of complex numbers,
%%and $\hxs{\spLLR}$ the set of all functions with range $\C$ and domain $\C$.
%%We define the following operators in $\spLLR$.
%%Define the operators $\opTrn:\spLLR\to\spLLR$ and $\opDil:\spLLR\to\spLLR$ as follows:
%\defbox{\begin{array}{M}
%  %The operator $\opTrn$ is a \hid{translation operator} and $\opDil$ is a \hid{dilation operator} if
%  %\\\indentx
%  $\begin{array}{F >{\ds}lc>{\ds}l @{\qquad}C@{\qquad}D@{\qquad}D@{\qquad}rcl}
%    %1. & \hxs{\opTrn_\alpha}  \ff(x) &\eqd&         \ff(x-\alpha)
%    %   & \forall \ff\in\spLLR
%    %   & (\hid{generalized translation operator})
%    %   & and
%    %   & \hxs{\opTrn}&\eqd& \opTrn_1
%    %   \\
%    1. & \hxs{\opDil_\alpha}  \ff(x) &\eqd& \sqrt{\alpha}\ff(\alpha x)
%       & \forall \ff\in\spRR
%       & (\hid{generalized dilation operator})
%      %& and
%      %& \hxs{\opDil}&\eqd& \opDil_2
%       \\
%    2. & \hxs{\opE_\alpha}  \ff(x) &\eqd& e^{i2\pi\alpha x}\ff(x)
%       & \forall \ff\in\spRR
%       & (\hid{modulation operator})
%      %& and
%      %& \hxs{\opE}&\eqd& \opE_1
%       \\
%    %3. & \hxs{\opM_\ff}  \ff(x) &\eqd& \brs{\ff\circ\ff}(x)
%    %   & \forall \ff,\ff\in\spLLR
%    %   & (\hid{multiplication operator})
%    %  %& and
%    %  %& \hxs{\opM}&\eqd& \opM_x
%  \end{array}$
%  %\\
%  %Moreover, $$, $\opDil\eqd\opDil_2$, $\opE\eqd\opE_1$, and $\opM\eqd\opM_x$.
%\end{array}}
%\end{definition}
%
%
%
%%--------------------------------------
%\begin{example}[\exm{Poisson Summation Formula}]
%\label{ex:psf}
%\index{Poisson Summation Formula}
%\index{theorems!Poisson Summation Formula}
%%--------------------------------------
%Let $\opE$ be the \structe{modulation operator} and $\opTrn$ the \structe{translation operator} \xrefP{def:opT}.
%Let $\Ff(\omega)$ be the Fourier transform \ifxref{harFour}{def:opFT} of a function $\ff(x)$.
%\exbox{
%  %\mcom{\ds \sum_{n\in\Z} \Ff\brp{{2\pi}n} e^{i{2\pi}nt}}
%  \mcom{\ds \sum_{n\in\Z} \opE_n\Ff\brp{{2\pi}n}}
%       {modulated summation in ``frequency"}
%  =
%  \mcom{\ds\cft \sum_{n\in\Z} \opTrn^n\ff(x)}
%       {summation in ``time"}
%  \qquad\scy\forall \ff\in\spLLR
%  }
%\end{example}
%\begin{proof}
%See \prefp{thm:psf}.
%\end{proof}
%
%%%--------------------------------------
%%\begin{example}[\exm{Legendre Polynomials}]
%%\footnote{
%%  \citerpgc{jackson1941}{47}{0486438082}{(5)}
%%  }
%%\label{ex:legendre_polynomials}
%%%---------------------------------------
%%Let $\opM$ be the \structe{multiplication operator} \xrefP{def:opM}.
%%Let $\fP_n(x)$ be the \structe{$n$th order Legendre polynomial}.
%%\exbox{
%%  \fP_{n+1}(x)   = 2x\fP_n(x) - \frac{n}{n+1}\fP_{n-1}(x)
%%  }
%%\end{example}
%
%%--------------------------------------
%\begin{example}[\exm{B-splines}]
%\label{ex:bspline_recursion}
%%---------------------------------------
%Let $\opM$ be the \structe{multiplication operator} and $\opTrn$ the \structe{translation operator} \xrefP{def:opT}.
%Let $\fN_n(x)$ be the \structe{$n$th order cardinal B-spline}\ifsxref{spline}{def:Nn}.
%\exbox{
%  \fN_n(x)   = \frac{1}{n}x\fN_{n-1}(x) - \frac{1}{n}x\opTrn\fN_{n-1}(x) + \frac{n+1}{n}\opTrn\fN_{n-1}(x)  \qquad\scy\forall n\in\Znn\setd\setn{1},\,  \forall x\in\R
%  }
%\end{example}
%\begin{proof}
%See \prefp{thm:bspline_recursion}.
%\end{proof}
%
%%--------------------------------------
%\begin{example}[\exm{Fourier Series analysis}]
%\label{ex:fs}
%%---------------------------------------
%Let $\opDil_\alpha$ be the \structe{dilation operator} and $\opE$ the \structe{modulation operator} \xrefP{def:opE}.
%Let $\opFS$ be the Fourier Series operator\ifsxref{harPoly}{def:opFS}.
%\exbox{\begin{array}{M}
%  The \structe{inverse Fourier Series} operator $\opFSi$ is given by
%  \\\indentx
%  $\ds \ff(n) = \sum_{n\in\Z} \Ff(n) \frac{1}{\sqrt{n}}\opDil_n e^{-i2\pi t} = \sum_{n\in\Z} \opE_n \Ff(n)
%  \qquad\scy\forall\ff\in\spLLR$
%  \\where
%  \\\indentx
%  $\ds\Ff(\omega) \eqd \int_0^1 \opE_{-n} \ff(x) \dx$
%\end{array}}
%\end{example}
%\begin{proof}
%See \prefp{thm:opFSi}.
%\end{proof}


