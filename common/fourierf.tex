%============================================================================
% LaTeX File
% Daniel Greenhoe
%============================================================================

%======================================
\chapter{Fourier Analysis}
\index{harmonic analysis}
\index{fourier analysis}
\label{app:fourier}
%======================================
%======================================
\section{Operations}
%======================================
Harmonic analysis provides four operations depending on 
whether the time/frequency is {\em continuous} or {\em discrete}
(see Figure~\ref{fig:fourier}).

\begin{figure}[h]
\begin{center}
   \begin{tabular}{|c||c|c|}
      \hline
      $[\opFT x(t)](f)$     &  continuous $t$      & discrete $t$   \\
      \hline
      \hline
         continuous $f$ & Fourier Transform (FT) & Discrete-Time Fourier Transform (DTFT) \\
                        & $\opFT:\spLL\to\spLL$    &  $\opDTFT:\spII\to\spLL$ \\
      \hline
         discrete   $f$ & Fourier Series (FS)  & Discrete Fourier Transform (DFT) \\
                        & $\opFS:\spLL\to\spII$  & $\opDFT:\spII\to\spII$ \\
      \hline
   \end{tabular}
\caption{
   harmonic analysis operations
   \label{fig:fourier}
   }
\end{center}
\end{figure}

%--------------------------------------
\begin{definition}
\label{def:ha_ops}
\label{def:fs}
\index{Fourier Transform}               \index{Fourier Series} 
\index{Discrete Time Fourier Transform} \index{Discrete Fourier Transform}
\index{FT} \index{FS} \index{DTFT} \index{DFT}
%--------------------------------------
Let $x,\ft{x}:\R\to\C$ and $y,\Fx:\Z\to\C$. Then 
$\opFT  :\spLL\to\spLL$
$\opFS  :\spLL\to\spII$, 
$\opDTFT:\spLL\to\spLL$, and 
$\opDFT :\spII\to\spII$
are linear operators such that

\begin{align*}
   \ft{x}(f) &\eqd& \opFT   x &\eqd& \ds \int_t   x(t) e^{-i2\pi ft         } \dt & \mbox{(Fourier Transform---FT)                } \\
   \Fx(k) &\eqd& \opFS   x &\eqd& \ds \int_0^T x(t) e^{-i\frac{2\pi}{T}kt} \dt & \mbox{(Fourier Series---FS)                   } \\
             &    & \opDTFT y &\eqd& \ds \sum_n   y(n) e^{-i2\pi fnT        }     & \mbox{(Discrete Time Fourier Transform---DTFT)} \\
             &    & \opDFT  y &\eqd& \ds \sum_n   y(n) e^{-i2\pi kn         }     & \mbox{(Discrete Fourier Transform---DFT)      } 
\end{align*}
\end{definition}

Under Definition~\ref{def:ha_ops}, Harmonic Analysis operations 
can be expressed in either of two powerful forms:

\begin{tabular}{cll}
   $\imark$ & linear operators & (Proposition~\ref{prop:ha_linops})\\
   $\imark$ & innerproducts \footnotemark
             & (Proposition~\ref{prop:ha_inprod})
\end{tabular}
\footnotetext{{\em Innerproducts}: Appendix~\ref{app:inprod} page~\pageref{app:inprod}}

Linear operator expressions allow for the immediate application 
of powerful tools from linear operator theory such as 
eigenvalues/eigenvector pairs and spectral analysis
involving self adjoint operators.

Innerproduct expressions allow for the immediate application 
of powerful tools from innerproduct space such as 
frames, Riesz basis, and projections.

%--------------------------------------
\begin{proposition}[Harmonic operators as linear operators]
\label{prop:ha_linops}
\index{linear operator}
%--------------------------------------
Let
\begin{liste}
   \item $x:\R\to\C$ and $y:\Z\to\C$
   \item $\kernel$ be a {\bf kernel}.
\end{liste}

Then
\formbox{
\begin{array}{rcllrcl}
   \left[\opFT    x\right](f) &=& \ds \int_t   \kernelb{f,t} x(t)  \dt & \mbox{ where } & \kernelb{f,t} &\eqd& e^{-i2\pi ft         } \\
   \left[ \opFS   x\right](k) &=& \ds \int_0^T \kernelb{k,t} x(t)  \dt & \mbox{ where } & \kernelb{k,t} &\eqd& e^{-i\frac{2\pi}{T}kt} \\
   \left[ \opDTFT y\right](f) &=& \ds \sum_n   \kernelb{f,n} y(n)      & \mbox{ where } & \kernelb{f,n} &\eqd& e^{-i2\pi fnT        } \\
   \left[ \opDFT  y\right](k) &=& \ds \sum_n   \kernelb{k,n} y(n)      & \mbox{ where } & \kernelb{k,n} &\eqd& e^{-i2\pi kn         } 
\end{array}
}
\end{proposition}

%--------------------------------------
\begin{proposition}[Harmonic operators as innerproducts]
\label{prop:ha_inprod}
\index{innerproduct}
%--------------------------------------
Let
\begin{liste}
   \item $x,y:\R\to\C$ 
   \item $a,b:\Z\to\C$.
\end{liste}

Then
\formbox{
\begin{array}{rcllrcl}
   \opFT   x &=& \ds \inprod{x(t)}{e^{i2\pi ft} }  &\mbox{ where } & \ds \inprod{x}{y} &\eqd& \ds \int_t x(t)y^\ast(t) \dt \\ 
   \opFS   x &=& \ds \inprod{x(t)}{e^{i\frac{2\pi}{T}k t} }  &\mbox{ where } & \ds \inprod{x}{y} &\eqd& \ds \int_0^T x(t)y^\ast(t) \dt \\
   \opDTFT a &=& \ds \inprod{a(n)}{e^{i2\pi fnT} } &\mbox{ where } & \ds \inprod{x}{y} &\eqd& \ds \sum_n a(n) b^\ast(n) \\
   \opDFT  a &=& \ds \inprod{a(n)}{e^{i2\pi kn} }  &\mbox{ where } & \ds \inprod{x}{y} &\eqd& \ds \sum_n a(n) b^\ast(n) \\
\end{array}
}
\end{proposition}


%======================================
\section{Spectral representation}
\index{spectral decomposition}
%======================================
The operations of Definition~\ref{def:ha_ops} can be used to 
give a spectral representation of a function $\fx(t)$.
\begin{liste}
   \item If $\fx(t)$ is periodic, then the Fourier Series operator
         $\opFS$ provides a spectral representation of $\fx(t)$
         (Theorem~\ref{thm:ha_sr_fs}).
   \item If $\fx(t)$ is in the space $\spLL$, then the Fourier Transform operator
         $\opFT$ provides a spectral representation of $\fx(t)$
         (Theorem~\ref{thm:ha_sr_ft}).
\end{liste}
%--------------------------------------
\begin{theorem}
\label{thm:ha_sr_fs}
%--------------------------------------
If a funtion $\fx:\R\to\C$ is periodic with period $T$, then
\formbox{
   \fx(t) = \sum_k \Fx(k) e^{i\frac{2\pi}{T}kt}.
   }
\end{theorem}

\begin{proof}
\begin{eqnarray*}
     \ds \sum_k \Fx(k) e^{i\frac{2\pi}{T}kt}
       &\eqd& \ds \sum_k \inprod{\fx}{ e^{i\frac{2\pi}{T}kt} } e^{i\frac{2\pi}{T}kt}  
     \\&\eqd& \ds \frac{1}{T} \sum_k  e^{i\frac{2\pi}{T}kt} \int_0^T \fx(u) e^{-i\frac{2\pi}{T}ku} \du 
     \\&=&    \ds \int_0^T \fx(u) \frac{1}{T} \sum_k e^{i\frac{2\pi}{T}k(t-u)}  \du 
     \\&=&    \ds \int_0^T \fx(u) \sum_k \delta(t-u-kT)  \du 
     \\&=&    \ds \sum_k \int_0^T \fx(u) \delta(t-u-kT)  \du 
     \\&=&    \ds \sum_k \int_{kT}^{(k+1)T} \fx(u-kT) \delta(t-u)  \du 
     \\&=&    \ds \sum_k \int_{kT}^{(k+1)T} \fx(u) \delta(t-u)  \du 
     \\&=&    \ds \int_{-\infty}^{+\infty} \fx(u) \delta(t-u)  \du 
     \\&=&    \ds \fx(t) 
\end{eqnarray*}
\end{proof}



%--------------------------------------
\begin{theorem}
\label{thm:ha_sr_ft}
%--------------------------------------
If $(\fx:\R\to\C)\in\spLL$, then
\formbox{
   \fx(t) = \int_f \Fx(f) e^{i2\pi ft} \df.
   }
\end{theorem}

\begin{proof}
\begin{eqnarray*}
   \int_f \Fx(f) e^{i2\pi ft} \df
     &=&    \int_f \int_u \fx(u) e^{-i2\pi fu}\du e^{i2\pi ft} \df
   \\&=&    \int_u \fx(u) \int_f e^{i2\pi f(t-u)} \df\du 
   \\&\eqq& \int_u \fx(u) \delta(t-u) \du 
   \\&=&    \fx(t)
\end{eqnarray*}
\end{proof}




%======================================
\section{Poisson Summation Formulas}
\label{sec:psf}
\index{Poisson Summation Formula}
\index{Sampling Theorem                                         }
\index{Dirichlet Kernel                                         }
\index{Harmonic shifted orthonormality requirement              }
\index{Zero-ISI necessary and sufficient criterion              }
\index{Duobinary ISI necessary and sufficient criterion         }
\index{Modified duobinary ISI necessary and sufficient criterion}
%======================================
The {\em Poisson Summation Formula} (PSF) and 
{\em Inverse Poisson Summation Forumla} (IPSF)
are fundamental and extremely powerful theorems in Harmonic analysis.
Using PSF and IPSF, several other critical results easily follow.
Examples include\\
\begin{tabular}{llll}
   $\imark$ & Sampling Theorem 
             & Theorem~\ref{thm:f_sampling} 
             & page~\pageref{thm:f_sampling}
\\
   $\imark$ & Dirichlet Kernel 
             & Theorem~\ref{thm:comb}
             & page~\pageref{thm:comb}
\\
   $\imark$ & Harmonic shifted orthonormality requirement
             & Theorem~\ref{thm:ortho_tf}
             & page~\pageref{thm:ortho_tf}
\\
   $\imark$ & Zero-ISI necessary and sufficient criterion
%             & Theorem~\ref{thm:unity}
%             & page~\pageref{thm:unity}
\\
   $\imark$ & Duobinary ISI necessary and sufficient criterion
%             & Theorem~\ref{thm:db_zeroISI}
%             & page~\pageref{thm:db_zeroISI}
\\
   $\imark$ & Modified duobinary ISI necessary and sufficient criterion
%             & Theorem~\ref{thm:mdb_zeroISI}
%             & page~\pageref{thm:mdb_zeroISI}
\end{tabular}
%



%======================================
\section{Parseval Relations}
\label{sec:parseval}
\index{Parseval relations}
%======================================
%--------------------------------------
\begin{theorem}[Parseval's equation]
\citep{mallat}{26}
\citep{goswami}{40}
\label{thm:GPT}
\index{Parseval's equation}
\index{theorems!Parseval's equation}
%--------------------------------------
Let $\ds \inprod{\fg(t)}{\fh(t)} \eqd \int_t \fg(t) \fh^\ast(t) \dt.$
Then 
\footnote{
  {\em Generalization of Parseval's formula}: 
  Theorem~\ref{thm:inprod_parseval} page~\pageref{thm:inprod_parseval}.
  }
\formbox{
  \inprod{\fg(t)}{\fh(t)} = \inprod{\Fg(f)}{\Fh(f)}.
  }
\end{theorem}

\begin{proof}
\begin{eqnarray*}
   \inprod{\fg(t)}{\fh(t)}
     &=&    \inprod{\inprod{\Fg(f)}{e^{-i2\pi ft}}}{\fh(t)}
   \\&=&    \inprod{\int_f\Fg(f)e^{-i2\pi ft}\df}{\fh(t)}
   \\&=&    \int_f\Fg(f) \inprod{e^{-i2\pi ft}}{\fh(t)} \df
   \\&=&    \int_f\Fg(f) \inprod{\fh(t)}{e^{-i2\pi ft}}^\ast \df
   \\&=&    \int_f\Fg(f) \Fh^\ast(f) \df
   \\&=&    \inprod{\Fg(f)}{\Fh(f)}
\end{eqnarray*}
\end{proof}

The energy in the time domain representation of a signal 
is equal to the energy in 
the frequency representation as shown in the next corollary.
%--------------------------------------
\begin{corollary}[\thmd{Plancherel's formula}]
\footnote{
  \citerp{mallat}{26},
  \citerp{goswami}{40}
  }
\label{cor:plancheral}
\index{Plancheral's formula}
%--------------------------------------
Let
\begin{eqnarray*}
   \inprod{\fg(t)}{\fh(t)} &\eqd& \int_t \fg(t) \fh^\ast(t) \; \dt \\
   \norm{\fg(t)}^2 &\eqd& \inprod{\fg(t)}{\fg(t)}.
\end{eqnarray*}

Then
\footnote{
  {\em Generalization of Plancheral's formula}: 
  \ppref{thm:inprod_plancheral}.
  }
\corbox{
   \norm{ \fg(t)}^2 = \norm{ \Fg(f)}^2.
   }
\end{corollary}

\begin{proof}
\begin{align*}
   \norm{ \fg(t)}^2
     &=& \ds \inprod{\fg(t)}{\fg(t)}
   \\&=& \ds \inprod{\Fg(f)}{\Fg(f)}  & \mbox{by Theorem \ref{thm:GPT}}
   \\&=& \ds \norm{\Fg(f)}^2
\end{align*}
\end{proof}









%======================================
\section{Shifted orthonormality}
%======================================
%--------------------------------------
\begin{theorem}[Harmonic shifted orthonormality requirement]
\label{thm:ortho_tf}
\index{Harmonic shifted orthonormality requirement}
%--------------------------------------
Let $n\in\Z$ and $\Fphi(f) \eqd [\opFT \phi](f)$. Then \\
\framebox{
\begin{tabular}{p{5cm}cp{5cm}}
  \mc{1}{c}{$\ds \int_t \phi(t) \phi^\ast(t-n) \dt =\kdelta_n $}
  &$\implies$&
  \mc{1}{c}{$\ds \sum_n | \Fphi(f-n)|^2 = 1$}
\\
  \cline{1-1}\cline{3-3}
  orthonormality requirement in time domain
  &&
  orthonormality requirement in frequency domain
\end{tabular}
}
\end{theorem}
\begin{proof}
First note that 
\[ [\opFT\phi^\ast(-t)](f) 
   \eqd \int_t \phi^\ast(-t) e^{-i2\pi ft}\dt
   =    \int_u \phi^\ast(u) e^{i2\pi fu}\du 
   =    \left[ \int_u \phi(u) e^{-i2\pi fu}\du \right]^\ast
   \eqd \Fphi^\ast(f).
\]

{\bf ``only if"} case: 
$\left[\int_t \phi(t) \phi^\ast(t-n) \dt =\kdelta_n\right] \implies
  \left[\sum_n | \Fphi(f-n)|^2 = 1 \right]$
\begin{eqnarray*}
  1 &=& \sum_n \kdelta_n e^{-i2\pi fn}
  \\&=& \sum_n \int_t \phi(t)\phi^\ast(t-n) \dt \;e^{-i2\pi fn}
        \hspace{1cm}\mbox{by left hypothesis}
  \\&=& \sum_n \left[ \int_t \phi(t)\phi^\ast(-(u-t))\dt \right]_{u=n} e^{-i2\pi fn}
  \\&=& \sum_n \left[ \phi(u) \conv \phi^\ast(-u)\right](n) e^{-i2\pi fn}
        \hspace{1cm}\mbox{by convolution definition (Definition~\ref{def:conv})}
  \\&=& \sum_n \left[\opFT \phi(u) \conv \phi^\ast(-u)\right](f+n) 
        \hspace{1cm}\mbox{by IPSF (Theorem~\ref{thm:ipsf})}
  \\&=& \sum_n \Fphi(f+n) \Fphi^\ast(f+n)
        \hspace{1cm}\mbox{by Convolution Theorem (Theorem~\ref{thm:conv})}
  \\&=& \sum_n |\Fphi(f+n)|^2
\end{eqnarray*}

{\bf ``if"} case: 
$\left[\int_t \phi(t) \phi^\ast(t-n) \dt =\kdelta_n\right] \Longleftarrow
  \left[\sum_n | \Fphi(f-n)|^2 = 1 \right]$

Mallat\citep{mallat}{225} claims that the ``if" case ($\Longleftarrow$)
is also true.\attention
However, I have not been able to prove it. 
Below is two attempts to prove it.
However, {\bf this portion of the proof is incomplete} (not a real proof):
\begin{eqnarray*}
  \int_t \fphi(t)\fphi^\ast(t-n) \dt
    &=& \left[ \int_t \fphi(t)\fphi^\ast(-(u-t)) \dt \right]_{u=n}
  \\&=& \left[ \fphi(u) \conv \fphi^\ast(-u) \right](n)
  \\&=& \left[ \opFTi \Fphi(f) \Fphi^\ast(f) \right](n)
  \\&=& \int_f \Fphi(f) \Fphi^\ast(f) e^{i2\pi fn} \df
  \\&=& \int_f | \Fphi(f) |^2 \df
\\
\\
\\
  \sum_n \kdelta_n
    &=& 1
  \\&=& \sum_n |\Fphi(n)|^2
        \hspace{1cm}\mbox{by right hypothesis}
  \\&=& \sum_n \Fphi(n) \Fphi^\ast(n)
  \\&=& \sum_n \left[ \opFT \fphi(u) \conv \fphi^\ast(-u) \right](n)
  \\&=& \sum_n \left[ \fphi(u) \conv \fphi^\ast(-u) \right](n)
        \hspace{1cm}\mbox{by IPSF (Theorem~\ref{thm:ipsf} page~\pageref{thm:ipsf})}
  \\&=& \sum_n \left[ \int_t \fphi(t)\fphi^\ast(-(u-t)) \dt \right]_{u=n}
  \\&=& \sum_n \int_t \fphi(t) \fphi^\ast(t-n) \dt 
\end{eqnarray*}
\end{proof}








