%============================================================================
% Daniel J. Greenhoe
% XeLaTeX file
%============================================================================
%=======================================
\chapter{Verbs}
%=======================================
%=======================================
\section{Sentence structure}
%=======================================
\ftbox{\zhtsdH{主}{詞}{zh/>u}{c/'i}{master}{word}{subject}}
\hfill\begin{minipage}{\tw-80mm}
As in English, verbs generally come \emph{after} the subject. 
However, just as in English (e.g. when waxing poetic),
the verb may come  \emph{before}.
\pref{ex:verbsubject} (next) illustrates such structure.
\end{minipage}\hfill
\ftbox{\zhtsdH{動}{詞}{d/`ong}{c/'i}{to move}{word}{verb}}

%---------------------------------------
\begin{example}[\exmd{verb-subject structure}]
\mbox{}\\
\label{ex:verbsubject}
%---------------------------------------
%有一天,飛來了一隻大蝴蝶。
\exboxt{
  \zhtss{有}{y/>ou}{have}
  \zhtsd{一}{天}{y/-i}{ti/-an}{one}{day}{one day}
  \zhtsC
  \zhtss{飛}{f/-ei}{fly}
  \zhtsd{來}{了}{l/'ai}{le}{come}{}{came}
  \zhtsd{一}{隻}{y/-i}{zh/-i}{a}{(MW)}{a}
  \zhtss{大}{d/`a}{big}
  \zhtsd{蝴}{蝶}{h/'u}{di/'e}{butterfly}{butterfly}{butterfly}
  \zhtsP\footnotemark
  \engbox{One day,\\flying along\\came a butterfly.}
  }
  \citetblt{\citerp{butterfly_kite}{1}}
\end{example}

%In fact, not only is the explicit use of a conjunction not always necessary, but the 
%explicit use of a verb as well,
%as illustrated in \pref{ex:noand} (next).
In English, a subject is normally followed by a verb, and 
consecutive adjectives are often delineated with an ``and" or a ``but" \ldots
as in, ``He \textbf{was} very happy \textbf{and} very pleased with himself."
In Chinese however, neither of these is always necessary, as illustrated in \pref{ex:noverb} (next).
Further examples of no explicit conjunction are provided in \prefpp{ex:and}.

%---------------------------------------
\begin{example}[\exmd{no explicit verb or conjunction}]
\mbox{}\\
\label{ex:noverb}
%---------------------------------------
\exboxt{
  %小白兔好高興,好得意
  \zhtss{小}{xi/>ao}{little}
  \zhtss{白}{b/>ai}{white}
  \zhtss{兔}{t/`u}{rabbit}
  \zhtss{好}{h/>ao}{very}
  \zhtsd{高}{興}{g/-ao}{x/`ing}{high}{happy}{happy}
  \zhtsC
  \zhtss{好}{h/>ao}{very}
  \zhtsd{得}{意}{d/'e}{y/`i}{get}{meaning}{pleased with oneself}
  \zhtsP\footnotemark
  \engbox{Little White Rabbit\\was very happy,\\and very pleased\\with himeself.}
  \citetblt{\citerp{butterfly_kite}{7}}
  }
\end{example}



%=======================================
\section{Number and tense}
%=======================================
\quad\begin{minipage}{\tw-65mm}
In English, \hie{verb} forms change substantially depending on
  \\\indentx\begin{tabular}{lll}
    $\imark$ & \hie{number} & (e.g. he walks, they walk)     \\
    $\imark$ & \hie{tense}  & (e.g. they walked, they walk) \\
  \end{tabular}
\end{minipage}\\
In particular, English verbs can be either \prope{singular} or \prope{plural},
and moreover can be either \prope{past tense} or \prope{present tense}.
Note that in English, while of course it is possible to express grammatical \prope{future tense},
\textbf{English verbs have no future tense}.\citetbl{\citerp{greenbaum1996}{81}}
Rather, grammatical future tense is expressed by appending a modifier word such as ``will" before
a present tense verb as in, ``I \emph{will} brush my teeth before going to bed."
\footnotetext{
  English verbs have no future tense. %(\zhtzy{英文的動詞沒有未來式}).
  %For future tense using English, add the word \hib{will} before the verb
  %(\zhtzy{英文的未來式, 在動詞的前面加``will"}).
  %For future tense in Chinese, add the character \zhtzy{將} before the verb (\zhtzy{中文的未來式, 在動詞的前面加``將"}).
  Verbs in \zcomi{Italian}{義大利語} have future tense. \citerpg{fasold2006}{87}{0521847680}.
  Example:
  \begin{tabular}[t]{|l||l|l|l|}
    \hline
             & past     & present & future
    \\\hline
    English: & left     & leave   & will leave \\
    Italian: & lasciato & partire & partir\.a \\
    Chinese: & $\times$ & \zhtzy{離開} & \zhtzy{將離開}
    \\\hline
  \end{tabular}
  }

In Chinese grammar, life is even simpler.
Verb forms do not change with either number or tense.
That is, \textbf{Chinese verbs have no tense}\footnotemark
        \footnotetext{
          Other languages in which verbs have no tense include the Australian
          language \hie{Dyirbal} (\citerpg{comrie1985}{40}{0521281385}) and
          and \zcomi{Burmese}{緬甸語} (\citerpg{comrie1985}{50}{0521281385}).
          At the other extreme,
          the African laguange \hie{Bamileke-Dschang} has 11 tenses.
          (\citerpg{trask1999}{207}{0415157420})
          }
and \textbf{Chinese verbs have no number}.
Just as in English, Chinese has 3 tenses, but Chinese verbs themselves do not.\citetbl{\citerp{li1989}{13}}
%(\zhtzy{中文的句法 有三個時態, 但是中文的動詞 沒有時態。})
%(\zhtzy{英文的句法有三個時態, 但是英文的動詞只有兩個時態:現在式和過去式。})

%Some common verbs are listed in \prefpp{voc:verbs} and some examples listed in \prefpp{sen:verbs_ex}.
%        \zcomi{Example}{例子}:
%        \\\begin{tabular}{ll>{\zhtzye}l}
%          present tense: & Today, I \textbf{run} home. & 今天我\underline{跑}回家。\\
%          past    tense: & Yesterday, I \textbf{ran} home. & 昨天我\underline{跑}回家。
%        \end{tabular}

\begin{minipage}{\tw-47mm}
English verbs together with modifier words such as ``will"
have 3 tenses and 4 aspects, giving 3$\times$4=12 states altogether.
However, some of these states are \emph{not} available in Chinese without the
addition of some word or words describing \emph{situation}.
For example in English one can say, ``I walked."
But in Chinese, there is no way to express this accurately without
specifying situation as in ``Yesterday I walked" or ``Last week I walked."
\end{minipage}
\ftbox{\zhtsdH{情}{況}{q/'ing}{ku/`ang}{sentiment}{condition}{situation}}

%---------------------------------------
\begin{example}[\exmd{verb tenses}]
\mbox{}\\
%---------------------------------------
\begin{tabular}{|l|l|l|}
  \hline
  \hline
  \zcomi{past tense}{過去式}  & \zcomi{present tense}{現在式} & \zcomi{future tense}{未來式}
  \\\hline
  I \textbf{walked}. & I \textbf{walk}. & I \textbf{will walk}.
  \\
  %\zhtzye (沒有好中文的翻譯) & \zhtzye 我走路。 & \zhtzye 我將走路。
  (no Chinese equivalent) & \zhtzye 我走路。 & \zhtzye 我將走路。
  \\\hline
  I \textbf{had} a dog. & I \textbf{have} a dog. & I \textbf{will have} a dog.
  \\
  %\zhtzye (沒有好中文的翻譯) & \zhtzye 我有一隻狗。 & \zhtzye 你將有一隻狗。
  (no Chinese equivalent) & \zhtzye 我有一隻狗。 & \zhtzye 你將有一隻狗。
  \\\hline
  You \textbf{gave} me a ball. & You \textbf{give} me a ball. & You \textbf{will give} me a ball.
  \\
  %\zhtzye (沒有好中文的翻譯) & \zhtzye 你給我一個球。 & \zhtzye 你將給我一個球。
  (no Chinese equivalent) & \zhtzye 你給我一個球。 & \zhtzye 你將給我一個球。
  \\\hline
\end{tabular}
\end{example}


\begin{tblvv}{Common verbs}{voc:verbs}
  \tblx to do          & \zhtssp{作}{z/`uo}{to do}
  \tblc to operate     & \zhtsds{操}{作}{c/-ao}{z/`uo}{to handle}{to do}{to operate}
  \tblh to see         & \zhtssp{看}{k/`an}{to see}
  \tblc to look at     & \zhtsds{看}{著}{k/`an}{zhe}{see}{(adverbial particle)}{look at}
  \tblh to look about  & \zhtsds{探}{望}{t/`an}{w/`ang}{to investigate}{to look afar}{to look about}
  \tblc to look around & \zhtsds{張}{望}{zh/-ang}{w/`ang}{to spread}{to look afar}{to look around}
  \tblh to stare at    & \zhtss{盯}{d/-ing}{fix eyes on} \zhtss{著}{zhe}{(adv. part.)} \zhtss{看}{k/`an}{see}
  \tblc to glare at    & \zhtsds{瞪}{著}{d\`eng}{zhe}{glare}{(adv. part.)}{glare at}
  \tblh to listen      & \zhtss{聽}{t/-ing}{listen}
  \tblc to smell       & \zhtssp{聞}{w/'en}{smell}
  \tblh to say         & \zhtss {說}{sh/-uo}{say  }
  \tblc to talk        & \zhtsds{講}{話}{ji/>ang}{h/`ua}{to speak}{talks}{talk}
  \tblh to want        & \zhtssp{要}{y/`ao}{want}
  \tblc to need        & \zhtsds{需}{要}{x/-u}{y/`ao}{need}{want}{need}
  \tblh to like        & \zhtsds{喜}{歡}{x/>i}{h/-uan}{happy}{cheerful}{like}
  \tblc to love        & \zhtss {愛}{a/`i}{love}
  \tblh to eat         & \zhtssp{吃}{ch/-i}{eat}
  \tblc to drink       & \zhtssp{喝}{h/-e}{drink}
  \tblh to wish        & \zhtsds{願}{意}{y/`uan}{y/`i}{hope}{idea}{to wish} \citex{ugly_duckling}{30}
\end{tblvv}


\begin{tbls}{Examples of common verbs}{sen:verbs_ex}
  \tblx I see you. &
    \zhtss{我}{w\v{o}}{I}%
    \zhtsd{看}{到}{k/`an}{d/`ao}{see}{reach}{see}%
    \zhtss{你}{n/>i}{you}\zhtsP
  \tblh I look at you. &
    \zhtss{我}{w\v{o}}{I}%
    \zhtsd{看}{著}{k/`an}{zhe}{see}{(particle)}{look at}%
    \zhtss{你}{n/>i}{you}\zhtsP
  \tblh From inside their homes they stuck out their heads and looked about. &
    \zhtsd{他}{們}{t/-a}{m/'en}{he}{(plural)}{they}%
    \zhtss{從}{c/'ong}{from}\zhtss{家}{ji/-a}{home}\zhtss{裡}{l/>i}{inside}%
    \zhtsd{伸}{出}{sh/-en}{ch/-u}{to stretch}{to exit}{to stick out}%
    \zhtss{頭}{t/'ou}{head}\zhtss{來}{l/'ai}{to come}%
    \zhtsd{探}{望}{t/`an}{w/`ang}{to investigate}{to look afar}{to look about}\zhtsP
    \citetbl{\citerp{baobao_lion}{4}}
  \tblh The crow hurriedly exited the hole and peeped around. &
    \zhtsd{烏}{鴉}{w/-u}{y/-a}{black}{crow}{crow}%
    \zhtsd{急}{忙}{j/'i}{m/'ang}{urgent}{busy}{hurriedly}%
    \zhtss{出}{ch/-u}{to exit}%
    \zhtss{洞}{d/`ong}{hole}%
    \zhtsd{張}{望}{zh/-ang}{w/`ang}{to spread}{to look afar}{to look around}\zhtsP%
    \citetbl{\citerp{10crows}{57}}
\end{tbls}

%=======================================
\section{Aspect}
%\grammarLessonHeader{Aspect}{動詞的體}
%=======================================
%\begin{vocabulary}
%     \ventry{simple}     {簡單式}  {}{"sImp.l}
%  \\ \ventry{progressive}{進行式}  {}{pr@"grEsIv}
%  \\ \ventry{perfect}    {完成式}  {}{"p\v{3}fIkt}
%\end{vocabulary}
%
Verbs in English have two aspects---progressive and perfect.\cittrp{greenbaum1996}{81}
A verb can be used with no aspect, one aspect, or both aspects (a total of 4 possibilities):
    \begin{enume}
      \item simple (no aspect)
      \item progressive
      \item perfect
      \item perfect progressive
    \end{enume}
Aspect is used together with tense to give verbs 12 different possible verb forms.
    \\\begin{tabular}{|p{\tw/4-2em}||*{3}{p{\tw/4-2em}|}}
                            & \mc{3}{|B|}{tense}\\
      \mc{1}{|B|}{aspect}   & past                     & present                     & future
      \\\hline\hline
      simple                & past simple              & present simple              & future simple
      \\\hline
      progressive           & past progressive         & present progressive         & future progressive
      \\\hline
      perfect               & past perfect             & present perfect             & future perfect
      \\\hline
      perfect progressive   & past perfect progressive & present perfect progressive & future perfect progressive
      \\\hline
    \end{tabular}

%In English, to change from from simple aspect to progressive aspect, do two things:
%    \begin{enume}
%      \item Add a \textbf{form of be} before the verb (\zhtzy{在動詞的前面加一個} ``form of be"): is, are, was, were, be
%      \item Add \textbf{-ing} to the end of the verb  (\zhtzy{在動詞的後面加} ``ing"):
%    \end{enume}
%    In Chinese, add \zhtzy{正在} in front of the verb.

\begin{minipage}{\tw-42mm}
In English, to change from from simple aspect to perfect aspect, add a form of ``\hib{have}" in front of the verb:
        \hib{had} (past tense), \hib{have} (present tense).
        Also, the verb changes form, often ending with an ``-n" (but not always).
    In Chinese, add \zht{了} somewhere after the verb.\footnotemark
\end{minipage}\hfill%
\ftbox{\zhtss{了}{le}{(a change of state)}}
\citetblt{\citerp{li1989}{185}}

%---------------------------------------
\begin{example}[\exmd{Perfect aspect}]
\mbox{}\\
%---------------------------------------
\begin{longtable}{|>{\scriptsize}r|llll|l|>{have }l|}
  \mc{1}{|B|}{ }&\mc{4}{B|}{present}&\mc{1}{B|}{past}&\mc{1}{B|}{perfect}
  \\%
  \mc{1}{|B|}{ }&\mc{1}{B|}{English}
                &\mc{1}{B|}{Traditional}
                &\mc{1}{B|}{Simplified}
                &\mc{1}{B|}{PinYin}&\mc{1}{B|}{    }&\mc{1}{B|}{       }
  \\
  \cnto &  {bite}  & \fntzht 咬   & \fntzhs 咬   &\color{red}y/>ao        & {bit}    &  {biten}
  \cntn &  {blow}  & \fntzht 吹   & \fntzhs 吹   &\color{red}ch/-ui       & {blew}   &  {blown}
  \cntn &  {buy}   & \fntzht 買   & \fntzhs 買   &\color{red}m/>ai        & {bought} &  {bought}
  \cntn &  {come}  & \fntzht 來   & \fntzhs 來   &\color{red}l/'ai        & {came}   &  {come}
  \cntn &  {do}    & \fntzht 做   & \fntzhs 做   &\color{red}z/`uo        & {did}    &  {done}
  \cntn &  {draw}  & \fntzht 畫   & \fntzhs 畫   &\color{red}h/`ua        & {drew}   &  {drawn}
  \cntn &  {drive} & \fntzht 開   & \fntzhs 開   &\color{red}k/-ai        & {drove}  &  {driven}
  \cntn &  {eat}   & \fntzht 吃   & \fntzhs 吃   &\color{red}ch/-i        & {ate}    &  {eaten}
  \cntn &  {feel}  & \fntzht 感覺 & \fntzhs 感覺 &\color{red}g/>an j/'ue  & {felt}   &  {felt}
  \cntn &  {fight} & \fntzht 打架 & \fntzhs 打架 &\color{red}d/>a ji/`a   & {fought} &  {fought}
  \cntn &  {fly}   & \fntzht 飛   & \fntzhs 飛   &\color{red}f/-ei        & {flew}   &  {flown}
  \cntn &  {forget}& \fntzht 忘記 & \fntzhs 忘記 &\color{red}w/`ang j/`i  & {forgot} &  {forgotten}
  \cntn &  {give}  & \fntzht 給   & \fntzhs 給   &\color{red}g/>ei        & {gave}   &  {given}
  \cntn &  {go}    & \fntzht 去   & \fntzhs 去   &\color{red}q/`u         & {went}   &  {gone}
  \cntn &  {hear}  & \fntzht 聽   & \fntzhs 聽   &\color{red}t/-ing       & {heard}  &  {heard}
  \cntn &  {ride}  & \fntzht 騎   & \fntzhs 騎   &\color{red}q/'i         & {rode}   &  {ridden}
  \cntn &  {run}   & \fntzht 跑   & \fntzhs 跑   &\color{red}p/>ao        & {ran}    &  {run}
  \cntn &  {see}   & \fntzht 看   & \fntzhs 看   &\color{red}k/`an        & {saw}    &  {seen}
  \cntn &  {seek}  & \fntzht 尋找 & \fntzhs 尋找 &\color{red}x/'un zh/>ao & {sought} &  {sought}
  \cntn &  {sell}  & \fntzht 賣   & \fntzhs 賣   &\color{red}m/`ai        & {sold}   &  {sold}
  \cntn &  {sing}  & \fntzht 唱   & \fntzhs 唱   &\color{red}ch/`ang      & {sang}   &  {sung}
  \cntn &  {sleep} & \fntzht 睡   & \fntzhs 睡   &\color{red}sh/`ui       & {slept}  &  {slept}
  \cntn &  {swim}  & \fntzht 游泳 & \fntzhs 游泳 &\color{red}y/'ou y/>ong & {swam}   &  {swum}
  \cntn &  {take}  & \fntzht 拿   & \fntzhs 拿   &\color{red}n/'a         & {took}   &  {taken}
  \cntn &  {tear}  & \fntzht 撕   & \fntzhs 撕   &\color{red}s/-i         & {tore}   &  {torn}
  \cntn &  {teach} & \fntzht 教書 & \fntzhs 教書 &\color{red}ji/-ao sh/-u & {taught} &  {taught}
  \cntn &  {think} & \fntzht 想   & \fntzhs 想   &\color{red}xi/>ang      & {thought}&  {thought}
  \cntn &  {write} & \fntzht 寫   & \fntzhs 寫   &\color{red}xi/>e        & {wrote}  &  {written}
  \\\hline
\end{longtable}
\end{example}

%\showSolution


%=======================================
%\bannerGrammartz{Present progessive tense}{現在進行時態}
%=======================================
\begin{minipage}{\tw-50mm}
In English, to change from simple aspect to progressive aspect, 
do two things:\quad\begin{tabular}[t]{cll}
              \imark&Add a \textbf{form of be} before the verb & and %\footnote{\zhtzy{在動詞的前面加一個be動詞}: is, are, was, were, be} and
            \\\imark&Add   \textbf{-ing}       after  the verb & %\footnote{\zhtzy{在動詞的後面加} ``ing"}
          \end{tabular}\\
In Chinese, add \zht{正在} in front of the verb.
%  \begin{tabular}{|>{\scriptsize}r|l|l|}
%    \hline
%    & \zcom{present simple tense}{現在簡單式} & \zcom{present progressive tense}{現在進行式}
%    \\\hline
%    \cnto & play   & is playing
%    \cntn & ride   & is riding
%    \cntn & study  & is studying
%    \cntn & walk   & is walking
%   %\cntn & rains   & is raining
%    \\\hline
%  \end{tabular}
%%\newpage
%  \item When a \zcom{situation}{情況} is \zcom{currently}{目前}
%\zcom{in the process of happening}{在過程中}.
%\zcom{use}{使用} \zcom{present progressive tense}{現在進行時態}.
%\\
%({\zhtzye
%當情況正在進行中時, 使用現在進行式。
%})\cittrpg{murphy1989}{2}{0521348439}
\end{minipage}\hfill%
\ftbox{\zhtsd{正}{在}{zh/`eng}{z/`ai}{exactly}{present}{in the process of}}

%\begin{tabular}{|l|>{\scriptsize}r|p{\tw/3-3mm}|>{\cellcolor[gray]{0.65}}p{\tw/3-3mm}|}
%  \hline
%        &                     & \zcomi{correct}{正確的}    & \zcomi{incorrect}{不正確的}
%  \\\zcomi{question}{問題} &  & \zcomi{answer}{回答}    & \zcomi{answer}{回答}
%  \\\hline
%                          & \cnto & He is playing basketball.     & He plays basketball.    \\
%                          & \cntx & He is  playing soccer.        & He plays  soccer.       \\
%  What is Ernie doing?    & \cntx & He is riding a bicycle.       & He rides a bicycle.     \\
%  \zhtzye(Ernie在做什麼?)& \cntx & He is playing the piano.      & He plays the piano.     \\
%                          & \cntx & He is studying math.          & He studies math.        \\
%                          & \cntx & He is walking to the library. & He walks to the library.\\
% %\cnth & How's the weather?   & It is raining.             & It rains.
%  \hline
%\end{tabular}


%---------------------------------------
\begin{example}
\mbox{}\\
%---------------------------------------
\begin{tabular}{|m{3\tw/16-3mm}||*{3}{m{4\tw/16-2mm}|}}
    \hline
    \mc{4}{|G|}{Example (\zhtzy{例子})}
    \\\hline
    & past & present & future
    \\\hline\hline
    \hie{simple}     & I \textbf{ran} home.       (no equivalent Chinese)
                     & I \textbf{run} home.       \zhtzy{我跑 回家。}
                     & I \textbf{will run} home.  \zhtzy{我將跑 回家。}
    \\\hline
    \hie{progressive}& I \textbf{\underline{was} runn\underline{ing}} home.      (no equivalent Chinese)
                     & I \textbf{\underline{am}  runn\underline{ing}} home.      \zhtzy{我\underline{正在} 跑回家。}
                     & I \textbf{\underline{will be} runn\underline{ing}} home.  \zhtzy{我\underline{將正在} 跑回家。}
    \\\hline
    \hie{perfect}    & I \textbf{had run} home.        no equivalent Chinese
                     & I \textbf{have run} home.       \zhtzy{我跑 回家了。}
                     & I \textbf{will have run} home.  \zhtzy{我將跑 回家了。}
    \\\hline
    \hie{perfect progressive}& I \textbf{\underline{had been} runn\underline{ing}} home.
                       (no equivalent Chinese)
                     & I \textbf{\underline{have been}  runn\underline{ing}} home.
                       (no equivalent Chinese)
                     & I \textbf{\underline{will have been} runn\underline{ing}} home.
                       (no equivalent Chinese)
    \\\hline
\end{tabular}
\end{example}

%---------------------------------------
\begin{example}
\mbox{}\\
%---------------------------------------
\begin{tabular}{|m{3\tw/16-3mm}||*{3}{m{4\tw/16-2mm}|}}
    \hline
    \mc{4}{|G|}{Example (\zhtzy{例子})}
    \\\hline
    & past & present & future
    \\\hline\hline
    \hie{simple}     & She {ate} an apple.
                       (no Chinese equiv.)
                     & She \textbf{eats} an apple.
                       \zhtzy{她吃 一個蘋果。}
                     & She
                       {will}
                       {eat} an apple.
                       \zhtzy{她將吃 一個蘋果。}
    \\\hline
    \hie{pro-gressive}& She
                        {was}
                        {eating} an apple.
                        (no Chinese equiv.)
                      & She
                        {is}
                        {eating} an apple.
                        \zhtzy{她正在吃 一個蘋果。}
                      & She
                        {will}
                        {be}
                        {eating} an apple.
                        \zhtzy{她將正在吃 一個蘋果。}
    \\\hline
    \hie{perfect}    & She
                        {had}
                        {eaten} an apple.
                       (no Chinese equiv.)
                     & She
                       {has}
                       {eaten} an apple.
                       \zhtzy{她吃了 一個蘋果。}
                     & She
                        {will}
                        {have}
                        {eaten} an apple.
                       \zhtzy{她將吃了 一個蘋果。}
    \\\hline
    \hie{perfect progressive}
                     & She
                        {had}
                        {been}
                        {eating} an apple.
                       (no Chinese equiv.)
                     & She
                        {has}
                        {been}
                       {eating} an apple.
                       (no Chinese equiv.)
                     & She  {will}
                        {have}
                        {been}
                        {eating} an apple.
                       (no Chinese equiv.)
    \\\hline
\end{tabular}
\end{example}


%---------------------------------------
\begin{example}
\mbox{}\\
%---------------------------------------
\begin{tabular}{|m{3\tw/16-3mm}||*{3}{m{4\tw/16-2mm}|}}
    \hline
    \mc{4}{|G|}{Example (\zhtzy{例子})}
    \\\hline
    & past & present & future
    \\\hline\hline
    \hie{simple}     & She  {saw} an apple.
                       (no Chinese equiv.)
                     & She \textbf{sees} an apple.
                       \zhtzy{她看見 一個蘋果。}
                     & She
                        {will}
                        {see} an apple.
                       \zhtzy{她將看見 一個蘋果。}
    \\\hline
    \hie{pro-gressive}& She
                         {was}
                         {seeing} an apple.
                        (no Chinese equiv.)
                      & She
                         {is}
                         {seeing} an apple.
                        \zhtzy{她正在看 一個蘋果。}
                      & She
                         {will}
                         {be}
                         {seeing} an apple.
                        \zhtzy{她將正在看 一個蘋果。}
    \\\hline
    \hie{perfect}    & She
                        {had}
                        {seen} an apple.
                       (no Chinese equiv.)
                     & She
                        {has}
                        {seen} an apple.
                       \zhtzy{她看了 一個蘋果。}
                     & She
                        {will}
                        {have}
                        {seen} an apple.
                       \zhtzy{她將看了 一個蘋果。}
    \\\hline
    \hie{perfect progressive}
                     & She
                        {had}
                        {been}
                        {seeing} an apple.
                       (no Chinese equiv.)
                     & She
                        {has}
                        {been}
                        {seeing} an apple.
                       (no Chinese equiv.)
                     & She
                        {will}
                        {have}
                        {been}
                        {seeing} an apple.
                       (no Chinese equiv.)
    \\\hline
\end{tabular}
\end{example}

%%---------------------------------------
%\begin{example}
%\mbox{}\\
%%---------------------------------------
%\begin{tabular}{|m{3\tw/16-4em}||*{3}{m{4\tw/16+1em}|}}
%    \hline
%    \mc{4}{|G|}{Example (\zhtzy{例子})}
%    \\\hline
%    & past & present & future
%    \\\hline\hline
%    \hie{simple}     & He  {bought} two.
%                     & He \textbf{buys} two.
%                       \zhtzy{他買兩個。}
%                     & He
%                        {will}
%                        {buy} two.
%    \\\hline
%    \hie{pro-gressive}& He
%                         {was}
%                         {buying} two.
%                      & He
%                         {is}
%                         {buying} two.
%                      & He
%                         {will}
%                         {be}
%                         {buying} two.
%    \\\hline
%    \hie{perfect}    & He
%                        {had}
%                        {bought} two.
%                     & He
%                        {has}
%                        {bought} two.
%                     & He
%                        {will}
%                        {have}
%                        {bought} two.
%    \\\hline
%    \hie{perfect progressive}
%                     & He
%                        {had}
%                        {been}
%                        {buying} two.
%                     & He
%                        {has}
%                        {been}
%                        {buying} two.
%                     & He
%                        {will}
%                        {have}
%                        {been}
%                        {buying} two.
%    \\\hline
%\end{tabular}
%\end{example}


%=======================================
\section{(Verb) around}
%=======================================
\ftbox{\zhtss{來}{l/'ai}{come}}\hfill
\begin{minipage}{\tw-60mm}
In English, one may say ``walk around" or ``run around", etc.
A common Chinese equivalent is ``\blank \zht{來} \blank \zht{去}",
as illustrated in \pref{ex:verb_around} (next).
\end{minipage}\hfill
\ftbox{\zhtss{去}{q/`u}{go}}

%---------------------------------------
\begin{example}[\exmd{(verb) around}]
\mbox{}\\
\label{ex:verb_around}
%---------------------------------------
 %蝴蝶在花朵上飛來飛去
  \zhtsd{蝴}{蝶}{h/'u}{di/'e}{butterfly}{butterfly}{butterfly}
  \zhtss{在}{z/`ai}{at}
  \zhtsd{花}{朵}{h/-ua}{d/>uo}{flower}{flower MW}{a flower}
  \zhtss{上}{sh/`ang}{above}
    \tcom{\zhtss{飛}{f/-ei}{fly}
          \zhtss{來}{l/'ai}{come}
          \zhtss{飛}{f/-ei}{fly}
          \zhtss{去}{q/`u}{go}}{flying around}
  \citetbl{\citerp{butterfly_kite}{3}}
  \engbox{The butterfly was flying\\around above a flower}
\end{example}

%=======================================
\section{(Verb) it down}
%=======================================
\begin{minipage}{\tw-37mm}
In English, one may ``write something down".
Similarly in Chinese, a common sentence pattern is ``(verb) \zht{下來}",
as illustrated in \pref{ex:verb_itdown} (next).
\end{minipage}\hfill
\ftbox{\zhtsd{下}{來}{xi/`a}{l/'ai}{down}{come}{to come down}}

%---------------------------------------
\begin{example}[\exmd{(verb) it down}]
\mbox{}\\
\label{ex:verb_itdown}
%---------------------------------------
 %我要把她畫下來
  \zhtss{我}{w/>o}{I}
  \zhtss{要}{y/`ao}{to want}
  \zhtss{把}{b/>a}{of}
  \zhtss{她}{t/-a}{her}
  \zhtss{畫}{h/`ua}{to draw}
  \zhtsds{下}{來}{xi/`a}{l/'ai}{down}{come}{to come down}%
  \zhtsP
  \citetbl{\citerp{butterfly_kite}{3}}
  \engbox{I want to draw\\a picture of her.}
\end{example}


