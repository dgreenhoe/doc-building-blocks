%============================================================================
% XeLaTeX File
% Daniel J. Greenhoe
%============================================================================

%=======================================
\chapter{Filtering Tricks}
%=======================================

%---------------------------------------
\section*{The Magic Wand: The z-domain IIR filter}
%---------------------------------------

What we want:

\indentx\begin{tabular}{|cll|}
    \hline
    \imark & linear         & (double the input means double the output)
  \\\imark & time-invariant & (doesn't change with time)
  \\\imark & tractable      & (we can build it)
  \\\hline
\end{tabular}


\includegraphics{graphics/linear.pdf}

Which one is \prope{linear}?



\includegraphics{graphics/nonti.pdf}

Is the filter \prope{time-invariant}?



\includegraphics{graphics/cosLcos.pdf}

\prope{linear} + \prope{time-invariant} = \prope{LTI}


%---------------------------------------
\section*{Linear operators}
%---------------------------------------
Note:

\fbox{\begin{tabular}{llll}
    A  & \opb{filter}    & in   \hie{engineering} & is an
  \\   & \opb{operator}  & in   \hie{mathematics} & is a
  \\   & \opb{black box} & in a \hie{system} diagram.
\end{tabular}}



\qboxnqt
  {
    Jules Henri Poincar\'e (1854-1912), physicist and mathematician
    \index{Poincar\'e, Jules Henri}
    \index{quotes!Poincar\'e, Jules Henri}
    \footnotemark
  }
  %{../common/people/small/poincare.jpg}
  {Je ne sais si je n'ai d\'ej\`a` dit quelque part que la Math\'ematique est 
  l'art de donner le m\^eme nom \`a des choses diff\'erentes. 
  Il convient que ces choses, diff\'erentes par la mati\`ere, 
  soient semblables par la forme, 
  \ldots
  qu'elles puissent, 
  pour ainsi dire, se couler dans le m\^eme moule. 
  Quand le langage a \'et\'e bien choisi, 
  on est tout \'etonn\'e 
  de voir que toutes les d\'emonstrations, faites pour un objet connu, 
  s'appliquent imm\'ediatement \`a beaucoup d'objets nouveaux ; 
  \ldots
  on n'a rien \`a y changer, pas m\^eme les mots, puisque les noms sont devenus les m\^emes.
  }
  %
  {I think I have already said somewhere that mathematics is the art
   of giving the same name to different things. 
   It is enough that these things, though differing in matter, 
   should be similar in form, 
   \ldots
   to permit of their being, so to speak,
   run in the same mould.
   When language has been well chosen, 
   one is astonished to find that all
   demonstrations made for a known object apply immediately to many new objects:
   \ldots
   nothing requires to be changed, 
   not even the terms,
   since the names have become the same.
   }
  \citetblt{
    quote:   & \citerc{poincare_sam}{book 1, chapter 2, paragraph 20} \\
             & \url{http://fr.wikisource.org/wiki/Science_et_m\%C3\%A9thode_-_Livre_premier\%2C_\%C2\%A7_II} \\
    trans.:  & \citerp{poincare_sam_eng}{34} \\
   %image:        & \url{http://en.wikipedia.org/wiki/Image:Poincare_jh.jpg}
    }



\fbox{\begin{tabular}{llll}
    A  & \opb{relation} & is a \ope{map}      & from one set to another.
  \\A  & \opb{function} & is a \ope{relation} & where 1 point cannot map to 2 points.
  \\An & \opb{operator} & is a \ope{function} & that maps vectors to vectors.
\end{tabular}}

$\begin{array}{|c|c|}
   \hline
   \includegraphics{graphics/123le12.pdf}
  &\includegraphics{graphics/fnctsq.pdf}
  \\\ope{relation}&\ope{function}
  \\\hline
\end{array}$




%---------------------------------------
%\begin{definition}
\label{def:linop}
\label{def:clL}
%---------------------------------------
%Let $\spX\eqd\linearspaceX$ and $\spY\eqd\linearspaceY$ be linear spaces.\\
  \defbox{%
    \index{operator!linear}
    \begin{array}{>{\scy\qquad}r rcl @{\qquad}C @{\qquad}D@{\qquad}D}
      \mc{7}{M}{An operator $\opL$ is \hid{linear} if \footnotemark}
        \\1. & \opL(\vx + \vy)  &=&  \opL \vx + \opL \vy & %\forall \vx,\vy\in\spX                      
                                                         & (\prope{additive}) & and
        \\2. & \opL(\alpha \vx) &=&  \alpha\opL \vx      & %\forall \vx\in\spX,\quad \forall\alpha\in\F 
                                                         & (\prope{homogeneous}).
    %\\\mc{7}{M}{The set of all linear operators from $\spX$ to $\spY$ is denoted $\clLxy$ such that}
    %\\\mc{7}{l}{\qquad\clLxy \eqd \set{\opL\in\clOxy}{\text{$\opL$ is linear}}}.
    \end{array}
  }
\footnotetext{
  \citerpg{kubrusly2001}{55}{0817641742},
  \citerpg{ab}{224}{0120502577},
  %\citerpg{michel1993}{95,407}{048667598X} \\
  \citorp{hilbert1927}{6},
  \citorp{stone1932}{33}
  }
%\end{definition}


%---------------------------------------
%\begin{theorem}
\label{thm:L_prop}
%---------------------------------------
%Let $\opL$ be an operator from a linear space $\spX$ to a linear space $\spY$, both over a field $\F$.
\thmbox{
  \text{$\opL$ is \prope{linear}}
  \qquad\implies\qquad
  \brbl{\begin{array}{>{\scy}r>{\ds}lc>{\ds}lC}
      1. & \opL\vzero                            &=& \vzero                                & %
    \\2. & \opL(-\vx)                            &=& -(\opL\vx)                            & %\forall \vx\in\spX
    \\3. & \opL(\vx-\vy)                         &=& \opL\vx - \opL\vy                     & %\forall \vx,\vy\in\spX
    \\4. & \opL\brp{\sum_{n=1}^N \alpha_n\vx_n}  &=& \sum_{n=1}^N \alpha_n\brp{\opL\vx_n}  & %\vx_n\in\spX,\,\alpha_n\in\F
  \end{array}}\footnotemark
}
\footnotetext{
  \citerpgc{berberian1961}{79}{0821819127}{Theorem~IV.1.1}
  }


%---------------------------------------
%\begin{theorem}
%---------------------------------------
Let $\spW$, $\spX$, $\spY$, and $\spZ$ be linear spaces over a field $\F$.
\thmbox{\begin{array}{>{\scy}rlcl@{\qquad}D}
    1. & \opL\brp{\opM\opN}      &=& \brp{\opL\opM}\opN                          & (\prope{associative})
  \\2. & \opL\brp{\opM\addo\opN} &=& \brp{\opL\opM}\addo\brp{\opL\opN}           & (\prope{left distributive})
  \\3. & \brp{\opL\addo\opM}\opN &=& \brp{\opL\opN}\addo\brp{\opM\opN}           & (\prope{right distributive})
  \\4. & \alpha\brp{\opL\opM}    &=& \brp{\alpha\opL}\opM = \opL\brp{\alpha\opM} & (\prope{homogeneous})
\end{array}\footnotemark}
\footnotetext{
  \citerpgc{berberian1961}{88}{0821819127}{Theorem~IV.5.1}
  }
%\end{theorem}

\tbox{$\ds X(z)\eqd\sum_{n\in\Z} \fx[n] z^{-n}$}
\tbox{\includegraphics{graphics/ziir2.pdf}}
\tbox{$\ds Y(z)\eqd\sum_{n\in\Z} \fy[n] z^{-n}$}

\indentx\begin{tabular}{cl}
    \imark&Is it \prope{linear}?
  \\\imark&Is it \prope{time-invariant}?
  \\\imark&Is it \prope{tractable}?
\end{tabular}

\tbox{$\ds X(z)$}
\tbox{\includegraphics{graphics/ziir2.pdf}}
\tbox{$\ds Y(z)$}
\begin{align*}
  H(z)\eqd \frac{Y(z)}{X(z)}
    &=\frac{b_0 z^2 + b_1 z + b_2}{a_0 z^2 + a_1 z + a_2}
    \qquad=\frac{z^{-2}}{z^{-2}}\times\frac{b_0 z^2 + b_1 z + b_2}{a_0 z^2 + a_1 z + a_2}
  \\&=\frac{b_0 + b_1 z^{-1} + b_2z^{-2}}{a_0 + a_1 z^{-1} + a_2z^{-2}}
  \\
  \\Y(z)\brs{a_0 + a_1 z^{-1} + a_2z^{-2}}
    &= X(z) \brs{b_0 + b_1 z^{-1} + b_2z^{-2}}
\end{align*}

If $a_0=1$ then

\begin{align*}
  Y(z) &=  b_0X(z) + b_1z^{-1}X(z)  + b_2z^{-2}X(z) - a_1 z^{-1}Y(z) + a_2z^{-2}Y(z)
\end{align*}

\begin{align*}
  a X(z)
    &\eqd a \opZ \seqn{\fx[n]}                && \text{by definition of $X(z)$}
  \\&\eqd a \sum_{n\in\Z} \fx[n] z^{-n}       && \text{by definition of $\opZ$ operator}
  \\&\eqd \sum_{n\in\Z} \brp{a \fx[n]} z^{-n} && \text{by \prope{distributive} property}
  \\&\eqd \opZ\seqn{a\fx[n]}                  && \text{by definition of $\opZ$ operator}
\end{align*}

\begin{align*}
  z^{-2}X(z) 
    &= z^{-2} \opZ\seqn{\fx[n]}
   &&\eqd z^{-2}\sum_{n=-\infty}^{n=+\infty} \fx[n] z^{-n}
  \\&=          \sum_{n=-\infty}^{n=+\infty} \fx[n] z^{-n-2}
  \\&=          \sum_{m-2=-\infty}^{m-2=+\infty} \fx[m-2] z^{-m}
    && \text{where $m\eqd n+2$ $\implies$ $n=m-2$}
  \\&=          \sum_{m=-\infty}^{m=+\infty} \fx[m-2] z^{-m}
  \\&=          \sum_{n=-\infty}^{n=+\infty} \fx[n-2] z^{-n}
    && \text{by change of free variable $m\rightarrow n$}
  \\&\eqd \opZ\seqn{\fx[n-2]}
    && \text{by definition of $\opZ$ operator}
\end{align*}

%---------------------------------------
\section*{Trick \# 1: From z-domain back to time-domain}
%---------------------------------------

\begin{align*}
  Y(z) &=  b_0X(z) + b_1z^{-1}X(z)  + b_2z^{-2}X(z) - a_1 z^{-1}Y(z) + a_2z^{-2}Y(z)
  \\\\
  \fy[n] &= b_0\fx[n] + b_1\fx[n-1] + b_2\fx[n-2] - a_1\fy[n-1] - a_2\fy[n-2]
\end{align*}

\includegraphics{graphics/iir2n.pdf}



Example

$\ds{\frac{3z^2 + 5z + 7}{2z^2 + 10z + 12}}$
=
$\ds{\frac{3z^2 + 5z + 7}{2\brp{z^2 + 5z + 6}}}$
=
$\ds{\frac{\brp{\sfrac{3}{2}z^2 + \sfrac{5}{2}z + \sfrac{7}{2}}}
               {z^2 + 5z + 6}}$
=
$\ds{\frac{\brp{\sfrac{3}{2} + \sfrac{5}{2}z^{-1} + \sfrac{7}{2}z^{-2}}}
               {1 + 5z^{-1} + 6z^{-2}}}$

\includegraphics{graphics/dfI_order2_156.pdf}

%---------------------------------------
\section*{Trick \# 2: The mirror trick}
%---------------------------------------
  \includegraphics{graphics/pz_minphase.pdf}%

If you want real coefficients, choose poles and zeros in conjugate pairs.

%\begin{align*}
%  \brp{z-p_1}\brp{z-p_1^*}
%    &= \brp{z-re^{i\theta}}\brp{z-re^{-i\theta}}
%  \\&= z^2 -r\brs{e^{-i\theta}+e^{i\theta}}z + 1
%  \\&= z^2 -r\brp{\brs{\cos\theta - i\sin\theta}+\brs{\cos\theta + i\sin\theta}}z + 1
%  \\&= z^2 -2r\cos(\theta) z + 1
%\end{align*}
\begin{align*}
  \brp{z-p_1}\brp{z-p_1^*}
    &= \brs{z-\brp{a+ib}} \brs{z-\brp{a-ib}}
  \\&= z^2 +\brs{-a+ib-ib-a}z - \brs{ib}^2
  \\&= z^2 -2a z + b^2
\end{align*}

Example: 

\includegraphics{graphics/pz_realcoefs_11.pdf}

\begin{align*}
  H(z)   &= G\frac{\brs{z-z_1}\brs{z-z_2}}
                  {\brs{z-p_1}\brs{z-p_2}}
          = G\frac{\brs{z-\brp{1+i}}\brs{z-\brp{1-i}}}
                  {\brs{z-\brp{-\sfrac{2}{3}+i\sfrac{1}{2}}}\brs{z-\brp{-\sfrac{2}{3}-i\sfrac{1}{2}}}}
       \\&= G\frac{z^2 - z\brs{\brp{1-i}+\brp{1+i}} + \brp{1-i}\brp{1+i}}
                  {z^2 - z\brs{\brp{-\sfrac{2}{3}+i\sfrac{1}{2}}+\brp{-\sfrac{2}{3}+i\sfrac{1}{2}}} + \brp{-\sfrac{2}{3}+i\sfrac{1}{2}}\brp{-\sfrac{2}{3}+i\sfrac{1}{2}}}
       \\&= G\frac{z^2 - 2z + 2}
                  {z^2 - \sfrac{4}{3}z + \brp{\sfrac{4}{3}+\sfrac{1}{4}}}
          = G\frac{z^2 - 2z + 2}
                  {z^2 - \sfrac{4}{3}z + \sfrac{19}{12}}
\end{align*}

%---------------------------------------
\section*{Trick \# 3: The 0Hz Gain}
%---------------------------------------
\tbox{$\ds X(z)$}
\tbox{\includegraphics{graphics/ziir2.pdf}}
\tbox{$\ds Y(z)$}

$\ds\begin{array}{rclM}
  \ds\brb{\opDTFT \fx[n]}(\omega) &\ds\eqd& \ds\sum_{n\in\Z} \fx[n] e^{-i\omega n}  & DTFT\\
  \ds\brb{\opZ    \fx[n]}(z)      &\ds\eqd& \ds\sum_{n\in\Z} \fx[n] z^{-n}          & z-Transform
\end{array}$

\[\implies\qquad z=e^{i\omega} \qquad z=1 \text{ at $\omega=0$}\]

\begin{align*}
  \brlr{H(z)}_{z=e^{i0}=1}
    &= \brlr{\frac{b_0 z^2 + b_1 z + b_2}{a_0 z^2 + a_1 z + a_2}}_{z=e^{i0}=1}%
     = \frac{b_0     + b_1   + b_2}{a_0     + a_1   + a_2}
\end{align*}


%---------------------------------------
\section*{Trick \# 4: The Fs/2 Gain}
%---------------------------------------
\tbox{$\ds X(z)$}
\tbox{\includegraphics{graphics/ziir2.pdf}}
\tbox{$\ds Y(z)$}


\[z=e^{i\omega} \qquad z=-1 \text{ at $\omega=\pi$}\]

\begin{align*}
  \brlr{H(z)}_{z=e^{i\pi}}
    &= \brlr{\frac{b_0 z^2 + b_1 z + b_2}{a_0 z^2 + a_1 z + a_2}}_{z=e^{i\pi}=-1}%
    &= \brlr{\frac{b_0 (-1)^2 + b_1 (-1) + b_2}{a_0 (-1)^2 + a_1 (-1) + a_2}}_{z=e^{i\pi}=-1}%
  \\&= \frac{b_0     - b_1   + b_2}{a_0     - a_1   + a_2}
\end{align*}

%---------------------------------------
\section*{Trick \# 5: For a stable filter, put all the poles inside the unit circle}
%---------------------------------------

\begin{tabular}{cc}
    \includegraphics{graphics/pz_unstable.pdf}
   &\includegraphics{graphics/pz_stable.pdf}%
  \\\prope{unstable}                          
   &\prope{stable}
\end{tabular}


%---------------------------------------
\section*{Trick \# 6: For minimum phase, put all the zeros inside the unit circle}
%---------------------------------------
\begin{tabular}{cc}
  \includegraphics{graphics/pz_realcoefs.pdf}%
  &\includegraphics{graphics/pz_minphase.pdf}%
  \\\emph{not} minimum phase & \prope{minimum phase}
\end{tabular}


The impulse response of a minimum phase filter has most of its energy concentrated
near the beginning of its support

%--------------------------------------
\thmd{Robinson's Energy Delay Theorem}
\footnote{
  \citerpg{dumitrescu2007}{36}{1402051247},
  \citor{robinson1962},  % referenced by claerbout1976
  \citorc{robinson1966}{???},  % referenced by online thesis
  \citerpp{claerbout1976}{52}{53}
  %\citerp{os}{291}\\
  %\citerp{mallat}{253}
  }
\label{thm:ztr_redp}
\index{minimum phase!energy}
\index{energy}
%--------------------------------------
\\
%Let $\fp(z)\eqd\sum_{n=0}^\xN a_n z^{-n}$ 
%and $\fq(z)\eqd\sum_{n=0}^\xN b_n z^{-n}$ 
%be polynomials.
\thmbox{
  \brb{\begin{array}{lMD}
    \fp & is \prope{minimum phase} & and\\
    \fq & is \emph{not} minimum phase & 
  \end{array}}
  \implies
  \mcom{\sum_{n=0}^{m-1} \abs{a_n}^2}{\parbox{20mm}{``energy" of the first $m$ coefficients of $\fp(z)$}} \ge 
  \mcom{\sum_{n=0}^{m-1} \abs{b_n}^2}{\parbox{20mm}{``energy" of the first $m$ coefficients of $\fq(z)$}} 
  %\qquad \forall 0\le m\le\xN
  }




Minimum phase filters are easy to invert: each zero becomes a pole and each pole becomes a zero

$\begin{array}{ccccc}
     \tbox{\includegraphics{graphics/pz_minphase.pdf}}
    &\tbox{$\times$}&
     \tbox{\includegraphics{graphics/pz_minphase_inv.pdf}}
    &\tbox{$=$}&
     \tbox{$1$}
  \\
  \ds\frac{\brp{z-z_1}\brp{z-z_2}\brp{z-z_3}\brp{z-z_4}}
          {\brp{z-p_1}\brp{z-p_2}\brp{z-p_3}\brp{z-p_4}}
  &\times&
  \ds\frac{\brp{z-p_1}\brp{z-p_2}\brp{z-p_3}\brp{z-p_4}}
          {\brp{z-z_1}\brp{z-z_2}\brp{z-z_3}\brp{z-z_4}}
    &=& 1
\end{array}$



%---------------------------------------
\section*{Trick \# 7: For more symmetry, put some zeros inside and some outside the unit circle}
%---------------------------------------

  \exboxt{\begin{tabular}{c|c}
    Daubechies-8 & Symlet-8
    \\\hline
    \includegraphics{graphics/D8_pz.pdf}&\includegraphics{graphics/S8_pz.pdf}\\
    \includegraphics{graphics/d8_phi_h.pdf}&\includegraphics{graphics/s8_phi_h.pdf}
    %\includegraphics{graphics/d8_psi_g.pdf}&\includegraphics{graphics/s8_psi_g.pdf}
  \end{tabular}}





%---------------------------------------
\section*{Trick \# 8: Really must invert a non-minimum phase filter? All-pass filter to save the day!}
%---------------------------------------

\begin{tabular}{ccccc}
     \tbox{\includegraphics{graphics/pz_unstable2.pdf}}
    &\tbox{$\times$}&
     \tbox{\includegraphics{graphics/pz_allpass.pdf}}
    &\tbox{$=$}&
     \tbox{\includegraphics{graphics/pz_unall.pdf}}
  \\unstable&&all-pass&&stable!
\end{tabular}



\begin{align*}
  \abs{A\brp{z}}_{z=e^{i\omega}}
    &= \frac{1}{r}\abs{\frac{z-r          e^{i\phi}}
                            {z-\frac{1}{r}e^{i\phi}}}_{z=e^{i\omega}}
   &&= \abs{\frac{ z- re^{i\phi}}
                 {rz-  e^{i\phi}}}_{z=e^{i\omega}}
  \\&= \abs{e^{i\phi}\brp{
            \frac{e^{-i\phi}z-r}
                 {rz- e^{i\phi}}}}_{z=e^{i\omega}}
   &&= \abs{z\brp{
            \frac{e^{-i\phi}-rz^{-1}}
                 {rz- e^{i\phi}}}}_{z=e^{i\omega}}
  \\&= \abs{-z\brp{
            \frac{rz^{-1}- e^{-i\phi}}
                 {rz     - e^{ i\phi}}}}_{z=e^{i\omega}}
   &&= \abs{\mcom{e^{i\pi}}{$-1$}e^{i\omega}\brp{
            \frac{re^{-i\omega} - e^{-i\phi}}
                 {re^{ i\omega} - e^{ i\phi}}}}
  \\&= \abs{\frac{1}{e^{-iv}}\brp{
            \frac{re^{-i\omega} - e^{-i\phi}}
                 {\brp{re^{ i\omega} - e^{ i\phi}}^\ast}}}
   &&= \abs{\frac{re^{-i\omega} - e^{-i\phi}}
                 {re^{-i\omega} - e^{-i\phi}}}
  \\&= 1
\end{align*}





%---------------------------------------
\section*{Trick \# 9: The Reappearing Roots: Now you don't see them, now you do.}
%---------------------------------------

True or False? This filter has no poles:

  $\ds H(z)= b_0 + b_1 z^{-1} + b_2 z^{-2}$

\includegraphics{graphics/dfI_order2_fir.pdf}


\begin{align*}
  H(z)
    &= b_0 + b_1 z^{-1} + b_2 z^{-2}
     = \frac{z^2}{z^2} \times \frac{b_0 + b_1 z^{-1} + b_2 z^{-2}}{1}
     = \frac{b_0 z^2 + b_1 z^{1} + b_2 }{z^2}
\end{align*}

\includegraphics{graphics/pz_pole00.pdf}

%---------------------------------------
\section*{Trick \# 10: Have a low-pass but a high-pass need? Just put a minus in front of z!}
%---------------------------------------
Suppose $\Zh(z)$ is a \propb{low-pass} filter. \qquad Then what is  $\Zh(-z)$?
\begin{align*}
  \abs{\Fg(\omega)}^2
    &\eqd \abs{\Zh(-z)}_{z=e^{i\omega}}
    && \text{by definition of $\Fg(\omega)$}
  \\&= \abs{\Zh(e^{-i\pi}z)}_{z=e^{i\omega}}
  \\&= \abs{\Zh(z)}_{z=e^{i\omega}e^{-i\pi}}
  \\&= \abs{\Zh(z)}_{z=e^{i(\omega-\pi)}}
  \\&\eqd \abs{\Fh(\omega-\pi)}^2
    && \text{by definition of $\Fh(\omega)$}
\end{align*}


