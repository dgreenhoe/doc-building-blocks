%============================================================================
% LaTeX File
% Daniel J. Greenhoe
%============================================================================

%======================================
\chapter{Polynomials}
%======================================
%\qboxnps{
%  \href{http://en.wikipedia.org/wiki/Jacques_Charles_Fran\%C3\%A7ois_Sturm}{Jacques Charles Francois Sturm}
%  \href{http://www-history.mcs.st-andrews.ac.uk/Timelines/TimelineE.html}{(1803--1855)},
%  \href{http://www-history.mcs.st-andrews.ac.uk/BirthplaceMaps/Places/France.html}{French-German} mathematician, \\
%  {\footnotesize (which he often said when presenting {\em Sturm's Method}
%  of locating polynomial zeros.)}
%  \footnotemark
%  \index{Sturm}\index{quotes!Sturm}
%  }
%  {../common/people/sturm.jpg}
%  {Here is the theorem whose name I bear.}
%  \citetblt{
%    quote: & \citerp{akl}{297} \\
%    image: & \url{http://en.wikipedia.org/wiki/Image:Sturm.jpeg}
%    }

%=======================================
\section{Definitions}
%=======================================
%--------------------------------------
\begin{definition}
\label{def:polynomial}
\footnote{
  \citerpg{barbeau1989}{1}{0387406271},
  \citerpg{fuhrmann2012}{11}{1461403375},
  \citerpg{borwein1995}{2}{0387945091}
  }
%--------------------------------------
Let $\fieldF$ be a \structe{field}.
\defboxt{
  A function $\fp$ in $\spFF$ is a \hid{polynomial} over $\fieldF$ if it is of the form
  \\\indentx$\ds\fp(x)\eqd\sum_{n=0}^\xN \alpha_n x^n \qquad \alpha_n\in\F,\,\alpha_\xN\neq0.$
  \\The \hid{degree} of $\fp$ is $\xN$. A \hid{coefficient} of $\fp$ is any element of $\tuplexN{\alpha_n}$.
  \\The \hid{leading coefficient} of $\fp$ is $\alpha_\xN$.
  }
\end{definition}

%--------------------------------------
\begin{definition}
\label{def:poly_eq}
\footnote{
  \citerpg{fuhrmann2012}{11}{1461403375}
  }
%--------------------------------------
%Let $\fp$ be a \structe{polynomial} \xref{def:polynomial} of degree $\xN$ and $\fq$ a polynomial of degree $\xM$
%over a \structe{field} $\fieldF$ \xref{def:polynomial}.
Let $\fieldF$ be a \structe{field}.
\defboxt{
  A polynomial $\fp$ of degree $\xN$ over the field $\F$ and\\
  a polynomial $\fq$ of degree $\xM$ over the field $\F$ are \hid{equal} if
  \\\indentx$\begin{array}{FMD}
    1. & $\xN=\xM$ & and\\
    2. & $\alpha_n=\beta_n$ \qquad for\qquad $n=0,1,\ldots,\xN$
  \end{array}$.
  \\The expression $\fp(x)=\fq(x)$ (or $\fp=\fq$) denotes that $\fp$ and $\fq$ are \prope{equal}.
  }
\end{definition}

%%--------------------------------------
%\begin{definition}
%\label{def:poly}
%\footnote{\citerp{michel1993}{69}
%\index{polynomial|textbf}
%%--------------------------------------
%Let $\vp$ and $\vq$ be sequences and $F=(\setX,+,\cdot)$ a \structe{field} over a set $\setX$.
%\begin{align*}
%  \vp
%    &\eqd \seqn{\alpha_{n-1},\ldots,\alpha_2,\alpha_1,\alpha_0},
%  \\
%  \vq
%    &\eqd \seqn{\beta_{m-1},\ldots,\beta_2,\beta_1,\beta_0}
%\end{align*}
%
%\defbox{\parbox{\textwidth-10pt}{
%The set
%$P= \set{(\alpha_{n-1},\; \alpha_{n-2},\ldots,\alpha_2,\alpha_1,\alpha_0)}
%        {\alpha_k\in\F,\; n\in\Znn}
%$
%is a set of {\bf polynomials over the \structe{field} $\fieldF$} if
%for all $\vp,\vq\in P$
%\[
%\begin{array}{lrcl}
%  1.&   \vp\oplus\vq
%    &=& \vq\oplus\vp
%  \\
%  2.&   \epsilon\vp
%    &=& (\epsilon\alpha_{n-1},\ldots,\epsilon\alpha_2,\epsilon\alpha_1,\epsilon\alpha_0)
%  \\
%  3.&   \vp\oplus\vq
%    &=& \ds\left\{\begin{array}{ll}
%          (\alpha_{n-1},\ldots,\alpha_m,\quad
%           \alpha_{m-1}+\beta_{m-1}, \ldots, \alpha_1+\beta_1,
%           \alpha_0+\beta_0
%          )
%          & \text{ for $n\ge m$}
%          \\
%          (\beta_{m-1},\ldots,\beta_n,\quad
%           \alpha_{n-1}+\beta_{n-1}, \ldots, \alpha_1+\beta_1,
%           \alpha_0+\beta_0
%          )
%          & \text{ for $m\ge n$}
%          \\
%        \end{array}\right.
%  \\
%  4.&   \epsilon(\vp\oplus\vq)
%    &=& (\epsilon\vp)\oplus(\epsilon\vq)
%  \\
%  5.&   (\gamma+\epsilon)\vp
%    &=& (\gamma\vp)\oplus(\epsilon\vp)
%  \\
%  6.&   (\gamma\epsilon)\vp
%    &=& \gamma(\epsilon\vp)
%  \\
%  7.&   \vp\otimes\vq
%    &=& \ds\sum_{k=0}^{n+m-2}\alpha_k\beta_{n+m-2-k}
%\end{array}
%\]
%}}
%\end{definition}
%
%%--------------------------------------
%\begin{theorem}
%\footnote{\citerp{michel1993}{69}
%\label{thm:poly}
%\index{polynomial}
%%--------------------------------------
%Let $P$ be a set of polynomials.
%\thmbox{\begin{tabular}{ll}
%  1. & $P$ is a vector space.
%  \\
%  2. & $P$ is an algebra.
%  \\
%  3. & $P$ is a ring
%  \\
%  4. & $P$ is a \structe{field} $\iff$ the order of $P$ is $0$.
%\end{tabular}}
%\end{theorem}
%




%=======================================
\section{Ring properties}
%=======================================
%=======================================
\subsection{Polynomial Arithmetic}
%=======================================
%--------------------------------------
\begin{theorem}[polynomial addition]
\label{thm:poly_add}
\footnote{
  \citerpg{fuhrmann2012}{11}{1461403375}
  }
%--------------------------------------
Let $\fieldF$ be a \structe{field}.
\thmbox{\begin{array}{>{\ds}lM>{\ds}l}
  \mcom{\left(\sum_{n=0}^\xN \alpha_n x^n \right)}{$\fp(x)$} \;+\;
  \mcom{\left(\sum_{n=0}^\xM \beta_n  x^n \right)}{$\fq(x)$}
  =
  \mcom{\sum_{n=0}^{\max(\xN,\xM)} \gamma_n x^n}{$\fp(x)+\fq(x)$}
  &where&
  \gamma_n \eqd \brbl{
    \begin{array}{lM}
      \alpha_n + \beta_n & for $n\le \min(\xN,\xM)$\\
      \alpha_n      & for $n>\xM$\\
      \beta_n       & for $n>\xN$
    \end{array}}
  \\
  &for all&
  x,\,\alpha_n,\,\beta_n\in\F
\end{array}}
\end{theorem}

%=======================================
%\subsection{Multiplication}
%=======================================
Polynomial multiplication is equivalent to
convolution \xref{def:convd} of the coefficients \xref{def:polynomial}.
\footnote{
  \fncte{Convolution}: In fact, using \hie{GNU Octave}\texttrademark\hspace{1ex} or \hie{MatLab}\texttrademark,
  polynomial multiplication can be performed using convolution.
  For example, the operation
  $(x^3 + 5x^2 + 7x + 9)(4x^2 + 11)$ can be calculated in
  \hie{GNU Octave}\texttrademark\hspace{1ex} or \hie{MatLab}\texttrademark\hspace{1ex} with
  conv([1 5 7 9],[4 0 11])
  %\verb.conv([1 5 7 9],[4 0 11]).
  }
%--------------------------------------
\begin{theorem}[polynomial multiplication]
\label{thm:polymult}
\footnote{\citerp{apostol1975}{237}}
\index{Cauchy product}
%--------------------------------------
Let $\seqn{\alpha_n\in\C}$, $\seqn{b_n\in\C}$, and $x\in\C$.
\thmbox{
  \brp{\sum_{n=0}^\xN \alpha_n x^n }
  \brp{\sum_{n=0}^\xM \beta_n  x^n }
  =
  \sum_{n=0}^{\xN+\xM}
  \mcom{\brp{\sum_{k=\max(0,n-\xM)}^{\min(n,\xN)} \alpha_n \beta_{k-n} }}
       {Cauchy product}
  x^n
  }
\end{theorem}
\begin{proof}
%\begin{enumerate}
%\item
\begin{align*}
  \left(\sum_{n=0}^\xN \alpha_n x^n \right)\left(\sum_{m=0}^\xM \beta_m x^m \right)
    &= \sum_{n=0}^\xN \sum_{m=0}^\xM \alpha_n \beta_m x^{n+m}
  \\&= \sum_{n=0}^\xN \sum_{k=n}^{\xM+n} \alpha_n \beta_{k-n} x^k
    && k\eqd n+m \iff m=k-n
  %\\&\vdots
  \\&= \sum_{n=0}^{\xN+\xM} \brp{\sum_{k=\max(0,n-\xM)}^{\min(n,\xN)} \alpha_n \beta_{k-n} } x^n
  %\\&= \sum_{n=0}^{\xN+\xM}
  %     \left(\sum_{k=0}^n \alpha_k \beta_{n-k} \right) x^n
\end{align*}

 Perhaps the easiest way to see the relationship is by illustration with
      a matrix of product terms:
\[\begin{array}{>{\color{blue}}l | *{6}{l}}
      & \color{blue}\beta_0       & \color{blue}\beta_1       & \color{blue}\beta_2       &\color{blue} \beta_3       & \color{blue}\cdots & \color{blue}\beta_\xM           \\
  \hline
  \alpha_0 & \alpha_0 \beta_0     & \alpha_0 \beta_1x   & \alpha_0 \beta_2x^2 & \alpha_0 \beta_3x^3 & \cdots & \alpha_0 \beta_\xM x^\xM     \\
  \alpha_1 & \alpha_1 \beta_0 x   & \alpha_1 \beta_1x^2 & \alpha_1 \beta_2x^3 & \alpha_1 \beta_3x^4 & \cdots & \alpha_1 \beta_\xM x^{1+\xM} \\
  \alpha_2 & \alpha_2 \beta_0 x^2 & \alpha_2 \beta_1x^3 & \alpha_2 \beta_2x^4 & \alpha_2 \beta_3x^5 & \cdots & \alpha_2 \beta_\xM x^{2+\xM} \\
  \alpha_3 & \alpha_3 \beta_0 x^3 & \alpha_3 \beta_1x^4 & \alpha_3 \beta_2x^5 & \alpha_3 \beta_3x^6 & \cdots & \alpha_3 \beta_\xM x^{3+\xM} \\
  \vdots & \vdots & \vdots    & \vdots    & \vdots    & \ddots & \vdots        \\
  \alpha_\xN & \alpha_\xN \beta_0x^\xN & \alpha_\xN \beta_1 x^{\xN+1} & \alpha_\xN \beta_2 x^{\xN+2} & \alpha_\xN \beta_3x^{\xN+3} & \cdots & \alpha_\xN \beta_\xM x^{\xN+\xM}
\end{array}\]
\begin{enumerate}
\item The expression $\sum_{n=0}^\xN \sum_{m=0}^\xM \alpha_n \beta_m x^{n+m}$
      is equivalent to adding {\em horizontally}
      from left to right, from the first row to the last.

\item If we switched the order of summation to
      $\sum_{m=0}^\xM \sum_{n=0}^\xN \alpha_n \beta_m x^{n+m}$,
      then it would be equivalent to adding {\em vertically}
      from top to bottom,
      from the first column to the last.

\item For $\xN=\xM=\infty$, the expression
      $\sum_{n=0}^{\xN+\xM} \left(\sum_{k=0}^n \alpha_k \beta_{n-k} \right) x^n$
      is equivalent to adding {\em diagonally}
      starting from the upper left corner and proceeding
      towards the lower right.

\item For finite $\xN$ and $\xM$\ldots
  \begin{enumerate}
    \item The upper limit on the inner summation puts two constraints on $k$:
      \\\indentx$\brb{\begin{array}{lllD}
        k &\le& n  & and \\
        k &\le& \xN
      \end{array}}
      \implies
      k\le\min(n,\xN)$

    \item The lower limit on the inner summation also puts two constraints on $k$:
      \\\indentx$\brb{\begin{array}{lllD}
        k &\ge& 0     & and \\
        k &\ge& n-\xM
      \end{array}}
      \implies
      k\ge\max(0,n-\xM)$
  \end{enumerate}
\end{enumerate}
\end{proof}

%=======================================
%\subsection {Division}
%=======================================
Polynomial division can be performed in a manner very similar to integer division
(both integers and polynomials are \hie{rings}).



%--------------------------------------
\begin{definition}[Polynomial division]
%--------------------------------------
The quantities of polynomial division are defined as follows:
\defbox{
  \frac{\fd(x)}{\fp(x)} = \fq(x) + \frac{\fr(x)}{\fp(x)}
  \qquad\text{where}\qquad
  \brb{\begin{array}{lMD}
    \fd(x) & is the \hid{dividend}  & and \\
    \fp(x) & is the \hid{divisor}   & and \\
    \fq(x) & is the \hid{quotient}  & and \\
    \fr(x) & is the \hid{remainder}.
  \end{array}}
  }
\end{definition}

\ifexclude{wsd}{
The ring of integers $\Z$ contains some special elements called {\em primes}
which can only be divided\footnote{
   The expression ``$a$ divides $b$" means that $b/a$ has remainder 0.
}
 by themselves or 1.

Rings of polynomials have a similar elements called {\em primitive polynomials}.

%--------------------------------------
\begin{definition}
%--------------------------------------
\defbox{\begin{tabular}{ll}
  \mc{2}{l}{\text{A \hib{primitive polynomial} is any polynomial $p(x)$ that satisfies}}
  \\\qquad 1. & $p(x)$ cannot be factored
  \\\qquad 2. & the smallest order polynomial that $p(x)$ can divide is $x^{2^n-1}+1=0$.
\end{tabular}}
\end{definition}

%--------------------------------------
\begin{example}
\footnote{
  \citePpu{hansen1992}{s47}{https://www.ams.org/journals/mcom/1992-59-200/S0025-5718-1992-1134730-7/S0025-5718-1992-1134730-7.pdf},
  \citerppg{wicker}{465}{475}{0132008092}
  }
%--------------------------------------
Some examples of primitive polynomials over $GF(2)$ are
\exbox{\begin{array}{cl|cl}
   \text{order} & \text{primitive polynomial} & \text{order} & \text{primitive polynomial}
   \\\hline
      2 & \fp(x) = x^2    + x      + 1  &  5 & \fp(x) = x^5    + x^4    + x^2 + x + 1     
   \\ 3 & \fp(x) = x^3    + x      + 1  & 16 & \fp(x) = x^{16} + x^{15} + x^{13} + x^4 + 1
   \\ 4 & \fp(x) = x^4    + x      + 1  & 31 & \fp(x) = x^{31} + x^{28} + 1               
   \\ 5 & \fp(x) = x^5    + x^2    + 1  &
\end{array}}
\end{example}

An m-sequence is the remainder when dividing any non-zero polynomial by a primitive
polynomial.
We can define an {\em equivalence relation}%
\ifdochas{found}{\footnote{
   \hie{equivalence relation}: \prefpp{def:eq_rel}}}
on polynomials which defines two polynomials as
{\em equivalent with respect to $p(x)$}
when their remainders are equal.

%--------------------------------------
\begin{definition}[Equivalence relation]
\label{def:equiv}
\index{equivalence relation}
%--------------------------------------
\begin{center}
Let  \hspace{5mm}
   $\ds \frac{\alpha_1(x)}{p(x)} = \fq_1(x) + \frac{r_1(x)}{p(x)}$
\hspace{5mm} and \hspace{5mm}
   $\ds \frac{\alpha_2(x)}{p(x)} = \fq_2(x) + \frac{r_2(x)}{p(x)}$.
\end{center}

Then $\alpha_1(x)\equiv \alpha_2(x)$ with respect to $p(x)$ if $r_1(x)=r_2(x)$.
\end{definition}

Using the equivalence relation of Definition \ref{def:equiv},
we can develop two very useful equivalent representations of
polynomials over GF(2).
We will call these two representations the {\em exponential} representation
and the {\em polynomial} representation.

%--------------------------------------
\begin{example}
\label{ex:representations}
\index{cycle}
%--------------------------------------
By Definition \ref{def:equiv} and
under $p(x)=x^3+x+1$, we have the following equivalent representations:
\exbox{\begin{array}{lcrclclclllll}
   \frac{x^0}{x^3+x+1} &=& 0     &+& \frac{1  }{x^3+x+1}     &\implies& x^0 &\equiv& 1   \\
   \frac{x^1}{x^3+x+1} &=& 0     &+& \frac{x  }{x^3+x+1}     &\implies& x^1 &\equiv& x   \\
   \frac{x^2}{x^3+x+1} &=& 0     &+& \frac{x^2}{x^3+x+1}     &\implies& x^2 &\equiv& x^2 \\
   \frac{x^3}{x^3+x+1} &=& 1     &+& \frac{x+1}{x^3+x+1}     &\implies& x^3 &\equiv& x+1 \\
   \frac{x^4}{x^3+x+1} &=& x     &+& \frac{x^2+x}{x^3+x+1}   &\implies& x^4 &\equiv& x^2+x \\
   \frac{x^5}{x^3+x+1} &=& x^2+1 &+& \frac{x^2+x+1}{x^3+x+1} &\implies& x^5 &\equiv& x^2+x+1 \\
   \frac{x^6}{x^3+x+1} &=& x^3+x+1 &+& \frac{x^2+1}{x^3+x+1} &\implies& x^6 &\equiv& x^2+1 \\
   \frac{x^7}{x^3+x+1} &=& x^4+x^2+x+1 &+& \frac{1}{x^3+x+1} &\implies& x^7 &\equiv& 1
\end{array}}

Notice that $x^7\equiv x^0$, and so a cycle is formed with
$2^3-1=7$ elements in the cycle.
The monomials to the left of the $\equiv$ are the {\em exponential}
representation and the polynomials to the right are the {\em polynomial}
representation.
Additionally, the polynomial representation may be put in a vector form giving a
{\em vector} representation.
The vectors may be interpreted as a binary number and represented as a decimal numeral.
\exbox{\begin{array}{c *{5}{c@{\,}} cc}
   \text{exponential} & \mc{5}{c}{\text{polynomial}} & \text{vector} & \text{decimal} \\
   x^0 &     & &   & & 1 & [001] & 1  \\
   x^1 &     & & x & &   & [010] & 2  \\
   x^2 & x^2 & &   & &   & [100] & 4  \\
   x^3 &     & & x &+& 1 & [011] & 3  \\
   x^4 & x^2 &+& x & &   & [110] & 6  \\
   x^5 & x^2 &+& x &+& 1 & [111] & 7  \\
   x^6 & x^2 &+&   & & 1 & [101] & 5
\end{array}}
\end{example}

%--------------------------------------
\begin{example}
\label{ex:1/p(x)}
%--------------------------------------
We can generate an m-sequence of length
$2^3-1=7$ by dividing 1 by the primitive polynomial $x^3+x+1$.




\begin{fsL}
\[
\begin{array}{rrll}
           &   & x^{-3} + x^{-5} + x^{-6} +& x^{-7} + x^{-10} + x^{-12} + x^{-13} + x^{-14} + x^{-17} + \cdots \\
   \cline{3-4}
   x^3+x+1 & \vline & 1 \\
           &        & 1 + x^{-2} + x^{-3} \\
   \cline{3-3}
           &        & x^{-2} + x^{-3} \\
           &        & x^{-2} + x^{-4} + x^{-5} \\
   \cline{3-3}
           &        & x^{-3} + x^{-4} + x^{-5} \\
           &        & x^{-3} + x^{-5} + x^{-6} \\
   \cline{3-3}
           &        & x^{-4} + x^{-6}          \\
           &        & x^{-4} + x^{-6} + x^{-7} \\
   \cline{3-3}
           &        & x^{-7}                   \\
           &        & x^{-7} + x^{-9} + x^{-10} \\
   \cline{3-3}
           &        & x^{-9} + x^{-10}         \\
           &        & x^{-9} + x^{-11} + x^{-12} \\
   \cline{3-3}
           &        & x^{-10} + x^{-11} + x^{-12} \\
           &        & x^{-10} + x^{-12} + x^{-13} \\
   \cline{3-3}
           &        & x^{-11} + x^{-13}          \\
           &        & x^{-11} + x^{-13} + x^{-14} \\
   \cline{3-3}
           &        & x^{-14}           \\
           &        & \vdots
\end{array}
\]
\end{fsL}
The coefficients, starting with the $x^{-1}$ term,
of the resulting polynomial form the m-sequence
\[ 0010111 \; 0010111 \; \cdots \]
which repeats every $2^3-1=7$ elements.
\end{example}


Note that the division operation in Example \ref{ex:1/p(x)}
can be performed using vector notation rather than polynomial notation.

%--------------------------------------
\begin{example}
%--------------------------------------
Generate an m-sequence of length $2^3-1=7$ by dividing 1 by the primitive polynomial
$x^3+x+1$ using vector notation.

\begin{fsL}
\[
\begin{array}{rrllllllllllllllllllllllllllllllllllll}
           &   &   &.&0&0&1&0&1&1&1 &0&0&1&0&1&1&1 &0&\cdots \\
   \cline{3-20}
   1011    &\vline   & 1 &.&0&0&0&0&0&0&0 &0&0&0&0&0&0&0 &0 \\
           &   & 1 & &0&1&1 \\
   \cline{3-7}
           &   & 0 & &0&1&1&0 \\
           &   &   & &0&0&0&0 \\
   \cline{5-8}
           &   &   & &0&1&1&0&0 \\
           &   &   & & &1&0&1&1 \\
   \cline{6-9}
           &   &   & & &0&1&1&1&0 \\
           &   &   & & & &1&0&1&1 \\
   \cline{7-10}
           &   &   & & & &0&1&0&1&0 \\
           &   &   & & & & &1&0&1&1 \\
   \cline{8-11}
           &   &   & & & & &0&0&0&1&0 \\
           &   &   & & & & & &0&0&0&0 \\
   \cline{9-12}
           &   &   & & & & & &0&0&1&0&0 \\
           &   &   & & & & & & &0&0&0&0 \\
   \cline{10-13}
           &   &   & & & & & & &0&1&0&0&0 \\
           &   &   & & & & & & & &1&0&1&1 \\
   \cline{11-14}
           &   &   & & & & & & & &0&0&1&1&0 \\
           &   &   & & & & & & & & &0&0&0&0 \\
   \cline{12-15}
           &   &   & & & & & & & & &0&1&1&0&0 \\
           &   &   & & & & & & & & & &1&0&1&1 \\
   \cline{13-16}
           &   &   & & & & & & & & & &0&1&1&1&0 \\
           &   &   & & & & & & & & & & &0&0&0&0 \\
           &   &   & & & & & & & & & & & &\vdots& &  \\
\end{array}
\]
\end{fsL}

The coefficients, starting to the right of the binary point,
is again the sequence
\[ 0010111 \; 0010111 \; \cdots. \]
\end{example}

} % wsd exclude

%=======================================
\subsection {Greatest common divisor}
%=======================================
%--------------------------------------
\begin{theorem}[Extended Euclidean Algorithm]
\label{thm:xEuclidAlgo}
\index{Extended Euclidean Algorithm}
\index{theorems!Extended Euclidean Algorithm}
\footnote{
  \citerp{wicker}{53},
  \citerpg{fuhrmann2012}{11}{1461403375}
  }

%--------------------------------------
Let $\fr_1(x)$ and $\fr_2(x)$ be polynomials.
The following algorithm computes their greatest common divisor
$\gcd(\fr_1(x),\fr_2(x))$, and factors $\fa(x)$ and $\fb(x)$ such that
\[ \fr_1(x)\fa(x) + \fr_2(x)\fb(x) = \gcd(\fr_1,\fr_2) \]

\thmbox{\renewcommand{\arraystretch}{1.1}\begin{array}{|r||l|l|l|l|}
  \hline
     & \text{remainder} & \text{quotient} & \text{factor} & \text{factor} \\
  n  & \fr_n=\fr_{n-2}-\fq_n \fr_{n-1}  & \fq_n   & \alpha_n=a_{n-2}-\fq_n \alpha_{n-1} & \beta_n=b_{n-2}-\fq_n \beta_{n-1}  \\
  \hline
  \hline
   1  & \fr_1(x)
      & -
      & 1
      & 0
      \\
   2  & \fr_2(x)
      & -
      & 0
      & 1
      \\
  \hline
   3  & \fr_1-q_3 \fr_2
      & \fq_3
      & 1
      & -q_3
      \\
   4  & \fr_2-q_4r_3
      & \fq_4
      & - \fq_4
      & 1 + \fq_4 \fq_1
      \\
   5  & \fr_1 - \fq_5 \fr_2
      & \fq_5
      & 1 + \fq_5 \fq_4
      & -q_3 - \fq_5(1 + \fq_4 \fq_3)
      \\
   \vdots& \vdots
      & \vdots
      & \vdots
      & \vdots
      \\
   n  & \gcd(\fr_1(x),\fr_2(x))
      & \fq_n
      & \fa(x)=a_{n-2}-q_n \alpha_{n-1}
      & \fb(x)=b_{n-2}-q_n \beta_{n-1}
      \\
   n+1& 0
      & \fq_{n+1}
      &
      &
      \\
  \hline
\end{array}}
\end{theorem}
\begin{proof}
\begin{align*}
  \fr_1
    &= \fq_3 \fr_2 + \fr_3
  \\&= \fq_3 \fr_2 + \fr_3
\end{align*}
\end{proof}

%--------------------------------------
\begin{example}
\label{ex:eea_n=2}
\index{Extended Euclidean Algorithm}
%--------------------------------------
Let
  \[ \fu(x)\eqd (1-x)^2  \hspace{4ex} \fv(x)\eqd x^2. \]

The greatest common divisor and factors of $\fu$ and $\fv$ are such that
\[
   \mcom{(1-x)^2}{$\fu(x)$} \mcom{(1+2x)}{$\fa(x)$}   +
   \mcom{(x^2)  }{$\fv(x)$} \mcom{(3-2x)}{$\fb(x)$} =
   \mcom{1}{$\gcd$}
\]
Because $\gcd(\fu,\fv)=1$, $\fu(x)$ and $\fv(x)$ are said to be
{\em relatively prime}.
\end{example}
\begin{proof}
\[\begin{array}{|r||l|l|l|l|}
  \hline
  n  & \fr_n=r_{n-2}-r_{n-1}q_n  & \fq_n   & \alpha_n=a_{n-2}-q_n \alpha_{n-1} & \beta_n=b_{n-2}-q_n \beta_{n-1}  \\
  \hline
  \hline
  -1  & (1-x)^2=1-2x+x^2 = \fu(x)
      & -
      & 1
      & 0
      \\
   0  & x^2 = \fv(x)
      & -
      & 0
      & 1
      \\
  \hline
   1  & 1-2x
      &  1
      &  1
      & -1
      \\
   2  &  \frac{1}{2}x
      & -\frac{1}{2}x
      &  \frac{1}{2}x
      & 1-\frac{1}{2}x
      \\
   3  & 1    = \gcd(\;(1-x)^2,\;x^2\;)
      & -4
      & 1+2x = \fa(x)
      & 3-2x = \fb(x)
      \\
   4  & 0
      & \frac{1}{2}x
      & -
      & -
      \\
  \hline
\end{array}\]
\end{proof}







%--------------------------------------
\begin{example}
\label{ex:eea_n=3}
\index{Extended Euclidean Algorithm}
%--------------------------------------
Let
  \[ \fu(x)\eqd (1-x)^3  \hspace{4ex} \fv(x)\eqd x^3. \]

The greatest common divisor and factors of $\fu$ and $\fv$ are such that
\[
   \mcom{(1-x)^3}{$\fu(x)$} \mcom{(1+3x+6x^2)}{$\fa(x)$}   +
   \mcom{(x^3)  }{$\fv(x)$} \mcom{(10-15x+6x^2)}{$\fb(x)$} =
   \mcom{1}{$\gcd$}
\]
Because $\gcd(\fu,\fv)=1$, $\fu(x)$ and $\fv(x)$ are said to be
{\em relatively prime}.
\end{example}
\begin{proof}
\[\begin{array}{|r||l|l|l|l|}
  \hline
  n  & \fr_n=r_{n-2}-r_{n-1}q_n  & \fq_n   & \alpha_n=a_{n-2}-q_n \alpha_{n-1} & \beta_n=b_{n-2}-q_n \beta_{n-1}  \\
  \hline
  \hline
  -1  & (1-x)^3=1-3x+3x^2-x^3
      & -
      & 1
      & 0
      \\
   0  & x^3
      & -
      & 0
      & 1
      \\
  \hline
   1  & 1-3x+3x^2
      & -1
      &  1
      &  1
      \\
   2  & -\frac{1}{3}x + x^2
      & \frac{1}{3}x
      & -\frac{1}{3}x
      & 1-\frac{1}{3}x
      \\
   3  & 1 -2x
      & 3
      & 1+x
      & -2+x
      \\
   4  & \frac{1}{6}x
      & -\frac{1}{2}x
      & \frac{1}{6}x + \frac{1}{2}x^2
      & 1-\frac{4}{3}y+\frac{1}{2}x^2
      \\
   5  & 1 = \gcd((1-x)^3,x^3)
      & -12
      & 1+3x+6x^2 = \fa(x)
      & 10-15x+6x^2 = \fb(x)
      \\
      \hline
   6  & 0
      & \frac{1}{6}x
      &
      &
      \\
  \hline
\end{array}\]
\end{proof}



%--------------------------------------
\begin{example}
\label{ex:eea_n=4}
\index{Extended Euclidean Algorithm}
%--------------------------------------
Let
  \[ \fu(x)\eqd (1-x)^4  \hspace{4ex} \fv(x)\eqd x^4. \]

The greatest common divisor and factors of $\fu$ and $\fv$ are such that
\[ \mcom{(1-x)^4}{$\fu(x)$} \mcom{(1 + 4x + 10x^2 + 20x^3)}{$\fa(x)$}   +
   \mcom{(x^4)  }{$\fv(x)$} \mcom{(35 -84x + 70x^2 -20x^3)}{$\fb(x)$} =
   \mcom{1}{$\gcd$}
\]
Because $\gcd(\fu,\fv)=1$, $\fu(x)$ and $\fv(x)$ are said to be
{\em relatively prime}.
\end{example}
\begin{proof}
\[\begin{array}{|r||l|l|l|l|}
  \hline
  n  & \fr_n=r_{n-2}-r_{n-1}q_n  & \fq_n   & \alpha_n=a_{n-2}-q_n \alpha_{n-1} & \beta_n=b_{n-2}-q_n \beta_{n-1}  \\
  \hline
  \hline
  -1  & (1-x)^4=1-4x+6^2-4x^3+x^4
      & -
      & 1
      & 0
      \\
   0  & x^4
      & -
      & 0
      & 1
      \\
  \hline
   1  & 1-4x+6x^2-4x^3
      &  1
      &  1
      & -1
      \\
   2  & \frac{1}{4}x - x^2 +\frac{3}{2}x^3
      & -\frac{1}{4}x
      & \frac{1}{4}x
      & 1-\frac{1}{4}x
      \\
   3  & 1 - \frac{10}{3}x + \frac{10}{3}x^2
      & -         \frac{8}{3}
      & 1 + \frac{2}{3}x
      & \frac{5}{3} - \frac{2}{3}x
      \\
   4  & -\frac{1}{5}x + \frac{1}{2}x^2
      & \frac{3}{2}\cdot\frac{3}{10}x
      & -\frac{1}{5}x - \frac{3}{10}x^2
      & 1 - x + \frac{3}{10}x^2
      \\
   5  & 1 - 2x
      & \frac{20}{3}
      & 1 + 2x + 2x^2
      &  -5 + 6x - 2x^2
      \\
   6  & \frac{1}{20}x
      & -\frac{1}{4}x
      & \frac{1}{20}x + \frac{1}{5}x^2  + \frac{1}{2}x^3
      & 1 -\frac{9}{4}x + \frac{18}{10}x^2 - \frac{1}{2}x^3
      \\
   7  & 1 = \gcd((1-x)^4,x^4)
      & -40
      & 1 + 4x + 10x^2 + 20x^3
      & 35 -84x + 70x^2 -20x^3
      \\
  \hline
   8  & 0
      & \frac{1}{20}x
      & -
      & -
      \\
  \hline
\end{array}\]
\end{proof}

\qboxnpq
  {{/'E}tienne B{/'e}zout, 1779 \footnotemark}
  {../common/people/bezout_wkp_pdomain.jpg}
  {Infinitesimal analysis was considered so attractive and important because of its numerous and useful applications;
  as such, it attracted upon itself all research attention and efforts.
  Concurrently, algebraic analysis appeared to be a \structe{field} where nothing remained to be done,
  or where whatever remained to be done would have only been worthless speculation.
  \ldots
  Nevertheless, the major contributors to infinitesimal analysis are well aware of the need to improve algebraic analysis:
  Their own progress depends upon it.}
\citetblt{
  quote: & \citer{bezout1779}\\
  translation: & \citerp{bezout1779e}{xv}\\
  image:& \scs\url{http://en.wikipedia.org/wiki/File:Etienne_Bezout2.jpg}, public domain
  }

%--------------------------------------
\begin{theorem}[\thm{B{/'e}zout's Identity}]
\label{lem:bezout}
\label{thm:bezout}
\footnote{
  \citerpgc{bourbaki2003algebraII}{2}{3540007067}{Theorem 1 Chapter VII},
  \citerppgc{fuhrmann2012}{15}{17}{1461403375}{Corollary 1.31, Corollary 1.38},
  \citerpg{adhikari2003}{182}{8173714290},
  \citerpg{warner1990}{381}{0486663418},
  \citerp{dau}{169},
  \citerp{mallat}{250}
  }
\footnote{
  Historical information:
  \citorc{bezout1779}{???},
  \citorc{bezout1779e}{???},
  \citorc{bachet1621}{???},
  \citerppgc{childs2009}{37}{46}{0387745270}{some history on page 46},
  \url{http://serge.mehl.free.fr/chrono/Bachet.html},
  \url{http://serge.mehl.free.fr/chrono/Bezout.html}
%  \citorpc{bezout1779e}{???}
  %\footnote{\url{http://mathworld.wolfram.com/BezoutsTheorem.html}}
  }
%--------------------------------------
Let $\fp_1(x)$ be a polynomial of degree $n_1$
and $\fp_2(x)$ be a polynomial of degree $n_2$.
\thmbox{\renewcommand{\arraystretch}{1}
  \mcom{\gcd(\fp_1(x),\; \fp_2(x))=1}
       {\parbox{20ex}{$\fp_1(x)$ and $\fp_2(x)$ are relatively prime}}
  \qquad\implies\qquad
  \left\{\begin{array}{ll}
    1. & \exists \fq_1(x),\fq_2(x) \st
    \\ \\ &
       {\renewcommand{\arraystretch}{0.7}
       \begin{array}{l@{}l l l@{}lll}
         \mc{3}{l}{\mbox{degree }n_2-1} & \mc{3}{l}{\mbox{degree }n_1-1} \\
         & \downarrow &          & & \downarrow                        \\
         \fp_1(x)     & \fq_1(x) &+& \fp_2(x)   & \fq_2(x) &=& 1       \\
         \uparrow     &          & & \uparrow                          \\
         \mc{3}{l}{\mbox{degree }n_1} & \mc{4}{l}{\mbox{degree }n_2}
       \end{array}
       } \\ \\
    2. & \text{order of } \fq_1(x)=n_2-1  \\
    3. & \text{order of } \fq_2(x)=n_1-1
  \end{array}\right.
}
\end{theorem}
\begin{proof}
No proof at this time.
\end{proof}





%=======================================
\section{Roots}
%=======================================
\qboxnpq
  {\href{http://en.wikipedia.org/wiki/Descartes}{Ren\'e Descartes}
   \href{http://www-history.mcs.st-andrews.ac.uk/Timelines/TimelineC.html}{(1596--1650)},
   \href{http://www-history.mcs.st-andrews.ac.uk/BirthplaceMaps/Places/France.html}{French}
   philosopher and mathematician\footnotemark
  \index{Descartes, Ren\'e}
  \index{quotes!Descartes}
  }
  %{../common/people/descart.jpg}
  {../common/people/descartes_fransHals_wkp_bw.jpg}
  {Neither the true nor the false roots are always real;
   sometimes they are imaginary;
   that is, while we can always conceive of as many roots for each equation
   as I have already assigned,
   yet there is not always a definite quantity corresponding to each root so conceived of.
   Thus, while we may conceive of the equation
   $x^3-6x^2+13x-10=0$ as having three roots,
   yet there is only one real root, 2, while the other two,
   however we may increase, diminish, or multiply them in accordance with the rules just laid down,
   remain always imaginary.}
  \citetblt{%
    quote:   & \citer{descartes1637}  \\
    English: & \citerp{descartes1637e}{175} \\
    image: & \scs\url{http://en.wikipedia.org/wiki/File:Frans_Hals_-_Portret_van_Ren\%C3\%A9_Descartes.jpg}, public domain
    }


%--------------------------------------
\begin{theorem}[\thmd{Fundamental Theorem of Algebra}]
\footnote{
  \citerppgc{prasolov2004}{1}{2}{3540407146}{Section 1.1.1},
  \citerpgc{borwein1995}{11}{0387945091}{Theorem 1.2.1}
  }
%--------------------------------------
Let $\fp(x)$ be a polynomial over a field $\fieldF$.
\thmbox{
  %\mcom{\fp(x)=\sum_{n=0}^\xN \gamma_n x^n}{$\xN$th order polnomial}
  \brb{\text{degree of $\fp(x)$ is $\xN$}}
  \implies
  \brb{\begin{array}{M}
    $\mcom{\text{$\exists \tuplexN{x_n}$ such that $\fp(x_n)=0$ for $n=1,2,\ldots,\xN$}}
         {$\fp(x)$ has $\xN$ zeros}$
    \\
    where $x_n$ and $x_m$ are not necessarily distinct for $n\ne m$.
  \end{array}}
  }
\end{theorem}

%--------------------------------------
\begin{corollary}
%--------------------------------------
Let $\fp(x)=\sum_{n=0}^\xN \alpha_n x^n$ be a polynomial over a field $\fieldF$.
%Let $\fp(x)$ be a polynomial of degree $\xN$ over a field $\fieldF$.
\corbox{
  \brb{\mcom{\begin{array}{M}
    There exists $\tuplexN{x_n}$\\
    such that $\fp(x_n)=0$ for $n=0,1,\ldots,\xN$\\
    and where $x_n$ and $x_m$ are\\
    not necessarily distinct for $n\ne m$.
  \end{array}}{$\xN$ zeros of $\fp(x)$}}
  \implies
  \brb{\begin{array}{>{\ds}l}
    \fp(x) = \brp{\frac{\alpha_0}{\ds\prod_{n=1}^\xN (-x_n)}}
    \mcom{\prod_{n=1}^\xN (x-x_n)}{$\xN$ factors}
  \end{array}}
  }
\end{corollary}

\begin{figure}%
  \centering
  \begin{tabular}{|c|c|c|c|c|}
    \hline
     \includegraphics{../common/math/graphics/pdfs/unitcircle_roots1.pdf}
    &\includegraphics{../common/math/graphics/pdfs/unitcircle_roots2.pdf}
    &\includegraphics{../common/math/graphics/pdfs/unitcircle_roots3.pdf}
    &\includegraphics{../common/math/graphics/pdfs/unitcircle_roots4.pdf}
    &\includegraphics{../common/math/graphics/pdfs/unitcircle_roots5.pdf}
    \\
    $x+1=0$ & $x^2+1=0$ & $x^3+1=0$ & $x^4+1=0$ & $x^5+1=0$
    \\\hline
  \end{tabular}
  \caption{
     Roots of $x^{n}+1=0$
     %\label{fig:h(x)=sqrt2}
     }
\end{figure}

%--------------------------------------
\begin{lemma}
%--------------------------------------
%Let $\setZ$ be the set of all zeros of a polynomial $\fp(x)$ over a field $\fieldF$.
\lembox{
  \brb{x^\xN + 1 = 0}
  \qquad\implies\qquad
  \brb{\text{roots of x} =
    \set{\exp\brs{i\frac{\pi}{\xN}(2n+1)}}
        {n=0,1,\ldots,\xN-1}
    }
  }
\end{lemma}
\begin{proof}
\begin{align*}
  e^{i\xN\theta_n - i2\pi n} &= -1                     && n\in\Z   \\
  \xN\theta_n - 2\pi n       &= \pi                    && n=0,1,\ldots,\xN-1   \\
  \xN\theta_n                &=  2\pi n + \pi                                   \\
  \theta_n                 &=  \frac{\pi}{\xN}(2n + 1)
\end{align*}
\end{proof}

%---------------------------------------
\begin{theorem}
\label{thm:hsym<==>1/x*}
%---------------------------------------
Let $N\in\Zp$, $I=\set{n\in\Z}{-N\le n\le N}$ and
  $\ds\fp(x) \eqd \sum_{n=-N}^N \alpha_n x^n  \qquad \forall x\in\C$.
\thmbox{
  \mcom{\alpha_n = \alpha_{-n}^\ast \quad \forall n\in I}
       {$\seq{\alpha_n}{}$ is Hermitian symmetric}
  \qquad\iff\qquad
  \fp(x) = \fp^\ast\left(\frac{1}{x^\ast}\right)  \quad \forall x\in\C
  }
\end{theorem}
\begin{proof}
\begin{align*}
\intertext{1. Proof that
  $\alpha_n=\alpha_{-n}^\ast \implies \fp(x) = \fp^\ast\left(\frac{1}{x^\ast}\right)$:
  }
  \fp(x)
    &\eqd\sum_{n=-N}^N \alpha_n x^n
    &&   \text{by definition of $\fp(x)$}
  \\&=   \alpha_0 + \sum_{n=1}^N \alpha_n x^n + \sum_{n=1}^N \alpha_{-n} x^{-n}
  \\&=   \alpha_0 + \sum_{n=1}^N \alpha_n x^n + \sum_{n=1}^N \alpha_n^\ast x^{-n}
    &&   \text{by left hypothesis}
  \\&=   \alpha_0 + \sum_{n=1}^N \alpha_n^\ast x^{-n} + \sum_{n=1}^N \alpha_n x^n
  \\&=   \alpha_0
       + \sum_{n=1}^N \alpha_n^\ast \left(\frac{1}{x}\right)^n
       + \sum_{n=1}^N \alpha_n \left(\frac{1}{x}\right)^{-n}
  \\&=   \left[\alpha_0
       + \sum_{n=1}^N \alpha_n \left(\frac{1}{x^\ast}\right)^n
       + \sum_{n=1}^N \alpha_n^\ast \left(\frac{1}{x^\ast}\right)^{-n}
         \right]^\ast
    &&   \ifdochas{numsys}{\text{by \prefp{thm:conj}}}
  \\&=   \left[\alpha_0
       + \sum_{n=1}^N \alpha_n    \left(\frac{1}{x^\ast}\right)^n
       + \sum_{n=1}^N \alpha_{-n} \left(\frac{1}{x^\ast}\right)^{-n}
         \right]^\ast
    &&   \text{by left hypothesis}
  \\&=   \left[
           \sum_{n=-N}^N \alpha_n    \left(\frac{1}{x^\ast}\right)^n
         \right]^\ast
  \\&= \fp^\ast\left(\frac{1}{x^\ast}\right)
    &&   \text{by definition of $\fp(x)$}
  \\
\intertext{2. Proof that
  $\alpha_n=\alpha_{-n}^\ast \impliedby \fp(x) = \fp^\ast\left(\frac{1}{x^\ast}\right)$:
  }
  \sum_{n=-N}^N \alpha_n x^n
    &\eqd\fp(x)
    &&   \text{by definition of $\fp(x)$}
  \\&=   \fp^\ast\left(\frac{1}{x^\ast}\right)
    &&   \text{by right hypothesis}
  \\&\eqd\left[
           \sum_{n=-N}^N \alpha_n    \left(\frac{1}{x^\ast}\right)^n
         \right]^\ast
    &&   \text{by definition of $\fp(x)$}
  \\&=   \sum_{n=-N}^N \alpha_n^\ast \left(\frac{1}{x}\right)^n
    &&   \ifdochas{numsys}{\text{by \prefp{thm:conj}}}
  \\&=   \sum_{n=-N}^N \alpha_{-n}^\ast x^{n}
    &&   \text{by symmetry of summation indices}
  \\ \\
  \implies
    &\quad \alpha_n=\alpha_{-n}^\ast
    &&   \text{by matching of polynomial coefficients}
\end{align*}
\end{proof}

\begin{figure}
  \centering
  \includegraphics{../common/math/graphics/pdfs/unitcircle_RecipricolConjugate.pdf}
  \caption{Recipricol conjugate pairs}
\end{figure}

%---------------------------------------
\begin{theorem}
\label{thm:hsym==>rconj}
%---------------------------------------
Let $N\in\Zp$, $I=\set{n\in\Z}{-N\le n\le N}$ and
  \[ \fp(x) \eqd \sum_{n=-N}^N \alpha_n x^n  \qquad \forall x\in\C \]
\thmbox{
  \mcom{\alpha_n = \alpha_{-n}^\ast \quad \forall n\in I}
       {$\seq{\alpha_n}{}$ is Hermitian symmetric}
  \quad\implies\quad
  \mcom{
    \Big[
    \sigma \text{ is a root of }\fp(x)
    \iff
    \frac{1}{\sigma^\ast} \text{ is a root of }\fp(x)
    \Big]
    }
    {roots occur in conjugate recipricol pairs}
  }
\end{theorem}
\begin{proof}
\begin{align*}
  \alpha_n = \alpha_{-n}^\ast \quad \forall n\in I
    &&& \text{by left hypothesis}
  \\&\implies \fp(x) = \fp^\ast\left(\frac{1}{x^\ast}\right)  \quad \forall x\in\C
    && \text{by \prefp{thm:hsym<==>1/x*}}
  \\&\implies
    \Big[
    \sigma \text{ is a root of }\fp(x)
    \iff
    \frac{1}{\sigma^\ast} \text{ is a root of }\fp(x)
    \Big]
\end{align*}

If $\sigma$ is a zero of $\fp(x)$,
then so is $\frac{1}{\sigma^\ast}$ because
\[ \fp\left(\frac{1}{\sigma^\ast}\right) = \fp^\ast(\sigma) = 0^\ast = 0.\]
\end{proof}




\begin{figure}[ht]\color{figcolor}
\begin{center}
\begin{fsL}
\setlength{\unitlength}{0.2mm}
\begin{picture}(200,230)(-100,-100)
  %\graphpaper[10](0,0)(200,200)
  \thicklines
  \put(-100 ,   0 ){\line(1,0){200} }
  \put(   0 ,-100 ){\line(0,1){200} }
  \thicklines
  \qbezier[16](  0, 40)(  0, 40)( 70, 40)
  \qbezier[16](  0,-40)(  0,-40)( 70,-40)
  \qbezier[16]( 70, 40)( 70,  0)( 70,-40)

  \put(  -5,   40 ){\makebox(  0,0)[r]{$+b$} }
  \put(  -5,  -40 ){\makebox(  0,0)[r]{$-b$} }
  \put(  75,    5 ){\makebox(  0,0)[l]{$a$} }
  \put( 105 ,   0 ){\makebox(  0,0)[l]{$\Re$}  }
  \put(   0 , 105 ){\makebox(  0,0)[b]{$\Im$}  }

  \put(  70 ,  40 ){\circle{10}}
  \put(  70 , -40 ){\circle{10}}
  \put( -50 ,   0 ){\circle{10}}

  \put(  80 ,  40 ){\makebox(0,0)[l]{zero at $x_1=a + i b$}}
  \put(  80 , -40 ){\makebox(0,0)[l]{zero at $x_2=a - i b=x_1^\ast$}}
  \put( -50 ,  10 ){\makebox(0,0)[rb]{zero at $x_2=c + i 0$}}
\end{picture}
\end{fsL}
\end{center}
\caption{
   Conjugate pairs of roots
   \label{fig:zeros_cpairs}
   }
\end{figure}
\pref{thm:zeros_cpairs} (next) states that the roots of real polynomials
occur in complex conjugate pairs;
this is illustrated in \prefpp{fig:zeros_cpairs}.
%--------------------------------------
\begin{theorem}
\label{thm:zeros_cpairs}
\footnote{\citerp{korn}{17}}
%--------------------------------------
Let $\fp(x)=\sum_{n=0}^\xN \alpha_n x^n$ be a $\xN$th order polynomial.
\thmbox{
  \left[\mcom{\seq{\alpha_n\in\R}{n=0,1,\ldots,\xN}}{coefficients are real}\right]
  \implies
  \left[ \mcom{\fp(x_0)=0 \iff \fp(x_0^\ast)=0}{zeros occur in conjugate pairs} \right]
  }
\end{theorem}

%--------------------------------------
\begin{theorem}[Routh-Hurwitz Criterion]
\label{thm:zeros_rhc}
\footnote{\citerp{korn}{17}}
\index{Routh-Hurwitz Criterion}
\index{theorems!Routh-Hurwitz Criterion}
%--------------------------------------
Let $\fp(x)=\sum_{n=0}^\xN \alpha_n x^n$ be a $\xN$th order polynomial with
$\alpha_n\in\R$ and
\[\begin{array}{>{\ds}l}
  d_0 \eqd \alpha_0
  \qquad
  d_1 \eqd \alpha_1
  \qquad
  d_2 \eqd \begin{array}{|ll|}
             \alpha_1 & \alpha_0 \\
             \alpha_3 & \alpha_2
           \end{array}
  \qquad
  d_3 \eqd \begin{array}{|lll|}
             \alpha_1 & \alpha_0 & 0   \\
             \alpha_3 & \alpha_2 & \alpha_1 \\
             \alpha_5 & \alpha_4 & \alpha_3
           \end{array}
  \qquad
  d_4 \eqd \begin{array}{|llll|}
             \alpha_1 & \alpha_0 & 0   & 0   \\
             \alpha_3 & \alpha_2 & \alpha_1 & \alpha_0 \\
             \alpha_5 & \alpha_4 & \alpha_3 & \alpha_2 \\
             \alpha_7 & \alpha_6 & \alpha_5 & \alpha_4
           \end{array}
  \\
  d_n \eqd \begin{array}{|llll|}
             \alpha_1      & \alpha_0      & \cdots & 0       \\
             \alpha_3      & \alpha_2      & \cdots & 0       \\
             \ddots   & \ddots   & \ddots & \vdots  \\
             \alpha_{2n-3} & \alpha_{2n-4} & \cdots & \alpha_{n-2} \\
             \alpha_{2n-1} & \alpha_{2n-2} & \cdots & \alpha_n
           \end{array}
  \end{array}\]
Let $\opS\seqn{x_n}$ be the number of sign changes of some sequence
$\seqn{x_n}$ after eliminating all zero elements ($x_n=0$).
\thmbox{\begin{array}{*{3}{>{\ds}l}}
  \mcom{|\set{x_n}{\fp(x_n)=0,\; \Re[x_n]>0}|}
       {number of roots in right half plane}
    &=& \mcom{\opS\seqn{d_0,\;d_1,\;d_1d_2,\;d_2d_3,\;\ldots,\;d_{p-2}d_{p-1},\;\alpha_p}}
             {number of sign changes}
  \\&=& \mcom{\opS\seqn{d_0,\;d_1,\;\frac{d_2}{d_1},\; \frac{d_3}{d_2},\; \ldots,\; \frac{d_p}{d_{p-1}} }}
             {number of sign changes}
  \end{array}}
\end{theorem}

%--------------------------------------
\begin{theorem}[Descartes rule of signs]
\label{thm:zeros_drs}
\footnote{\citerp{korn}{17}}
\index{Descartes rule of signs}
\index{theorems!Descartes rule of signs}
%--------------------------------------
Let $\fp(x)=\sum_{n=0}^\xN \alpha_n x^n$ be a $\xN$th order polynomial with
$\alpha_n\in\R$.
\thmbox{
  \mcom{|\set{x_n}{\fp(x_n)=0,\; \Re[x_n]>0},\; \Im[x_n]=0|}
       {number of roots on right real axis}
    =
        \mcom{\opS\seqn{\alpha_n} - 2m}
             {number of sign changes $-$ even integer}
        \text{ where $m\in\Znn$}
  }
\end{theorem}

%--------------------------------------
\begin{theorem}
\footnote{\citerp{korn}{18}}
%--------------------------------------
Let $\fp(x)=\sum_{n=0}^\xN \alpha_n x^n$ be a $\xN$th order polynomial with
$\alpha_n\in\R$.
\thmbox{
  \mcom{\alpha_0,\; \alpha_1,\;\ldots,\; \alpha_{k-1}\;\ge 0}
       {first $k$ coefficients are nonnegative}
  \implies
  \left\{\begin{array}{>{\ds}l}
    \mcom{|\set{x_n}{\fp(x_n)=0,\; \Im[x_n]=0}|}{number of real roots} <
    \mcom{1 + \left(\frac{q}{\alpha_0}\right)^\frac{1}{k}}{upper bound}
    \\
    \text{where }
    q \eqd \mcom{\max\set{\;\abs{\alpha_n}\;}{\alpha_n<0}}{largest negative coefficient}
  \end{array}\right.
  }
\end{theorem}

%--------------------------------------
\begin{theorem}[Rolle's Theorem]
\label{thm:zeros_rolle}
\footnote{
  \citerp{korn}{18}
  }
\index{Rolle's Theorem}
\index{theorems!Rolle's Theorem}
%--------------------------------------
Let $\fp(x)=\sum_{n=0}^\xN \alpha_n x^n$ be a $\xN$th order polynomial with
$\alpha_n\in\R$.
%Let $\set{r_n}{r_n\in\R,\; \fp(r_n)=0,\; x_1\le x_2\le \cdots \le x_m,\; n=1,2,\ldots,m}$
%be the {\bf real consecutive roots} of $\fp(x)$.
%Let $\set{q_n}{q_n\in\R,\; \fp'(q_n)=0}$
%be the {\bf real roots} of $\fp'(x)$.
The number of real zeros of $\fp'(x)$ between any two real
consecutive real zeros of $\fp(x)$ is {\bf odd}.
\end{theorem}

%%--------------------------------------
%\begin{theorem}[Budan's Theorem]
%\label{thm:zeros_budan}
%\footnote{\citerp{korn}{18}
%%--------------------------------------
%
%\end{theorem}

%%--------------------------------------
%\begin{theorem}[Sturm's Method]
%\label{thm:zeros_sturm}
%\footnote{\citerp{korn}{18}
%%--------------------------------------
%
%\end{theorem}



%--------------------------------------
\begin{definition}
\footnote{
  \citerpg{fuhrmann2012}{22}{1461403375}
  }
%--------------------------------------
Let $\fieldF$ be a \structe{field}.
\defboxt{
  $\ds\frac{\fp(x)}{\fq(x)}$ is a \hid{rational function} \\
  if $\fp(x)$ and $\fq(x)$ are \structe{polynomials} over $\fieldF$.
  }
\end{definition}

%--------------------------------------
\begin{example}
%--------------------------------------
\exboxt{
  An example of a rational function using polynomials in $x^{-1}$ is
  \\\indentx$\begin{array}{rcl}
    A(x) &=& \ds\frac{b_0 + \beta_1x^{-1} + \beta_2x^{-2} + \beta_3x^{-3} }
                     {1   + \alpha_1x^{-1} + \alpha_2x^{-2} + \alpha_3x^{-3}}
  \end{array}$
  \\
  This can be expressed as a rational function using polynomials in $x$\\
  by multiplying numerator and denominator by $x^3$:
  \\\indentx$\begin{array}{rclcl}
    A(x) &=& \ds\frac{x^3}{x^3}\;A(x)
         &=& \ds\frac{b_0x^3 + \beta_1x^{2} + \beta_2x + \beta_3 }
                     {x^3   + \alpha_1x^{2} + \alpha_2x + \alpha_3}
  \end{array}$
  }
\end{example}

%--------------------------------------
\begin{definition}
\label{def:zero}
\label{def:pole}
%--------------------------------------
\defboxt{
  The \hid{zeros} of a rational function $H(x)=\frac{B(x)}{A(x)}$
  are the roots of $B(x)$.
  \\
  The \hid{poles} of a rational function $H(x)=\frac{B(x)}{A(x)}$
  are the roots of $A(x)$.
  }
\end{definition}


%=======================================
\section{Polynomial expansions}
%=======================================
\qboxnps
  {\href{http://en.wikipedia.org/wiki/Euclid}{Euclid}
   \href{http://www-history.mcs.st-andrews.ac.uk/Timelines/TimelineA.html}{($\sim$300BC)},
   \href{http://www-history.mcs.st-andrews.ac.uk/BirthplaceMaps/Places/Greece.html}{Greek mathematician},
   demonstrating the \thme{Binomial theorem} for exponent $n=2$ as in $(x+y)^2=x^2 + 2xy + y^2$.
    \index{Euclid}
    \index{quotes!Euclid}
    \footnotemark
  }
  {../common/people/euclid_wkp_pdomain_bw.jpg}
  {Thus, if a straight-line is cut at random,
   then the square on the whole (straight-line) is equal to
   the (sum of the) squares on the pieces (of the straight-line),
   and twice the rectangle contained by the pieces.}
  %{If a straight line be cut at random,
  % the square on the whole is equal to the squares on the
  % segments and twice the rectangle of the segments.
  %}
  \citetblt{
    quote: & \citorc{euclid}{Book II, Proposition 4},
             \citerp{coolidge1949}{147} \\
    image: & \url{http://commons.wikimedia.org/wiki/File:Euklid-von-Alexandria_1.jpg}, public domain
    }

\ifdochasnot{taylor}{
%--------------------------------------
\begin{theorem}[\thmd{Taylor Series}]
\footnote{
  \citerpgc{flanigan1983}{221}{0486613887}{Theorem 15},
  \citerpg{strichartz1995}{281}{0867204710},
  \citerpgc{sohrab2003}{317}{0817642110}{Theorem 8.4.9},
  \citer{taylor},
  \citer{taylor1717},
  \citer{maclaurin}
  %\citerc{as}{pages 14, 880}
  %\citerp{mallat}{164}
  }
\label{thm:taylor}
%--------------------------------------
Let $\spC$ be the space of all \prope{analytic} functions
and $\hxs{\opDif}$ in $\clF{\spC}{\spC}$ the \ope{differentiation operator}.
\thmboxt{
  A \opd{Taylor series} about the point $a$ of a function $\ff\in\clF{\spC}{\spC}$ is
  $\ds\begin{array}{rc>{\ds}l@{\qquad}C@{\qquad}D}
    \ff(x) &=& \sum_{n=0}^\infty \frac{\brs{\opDif^n\ff}(a)}{n!}\:(x-a)^n
           &   \forall a\in\R,\,\ff\in\spC
           &   (\ope{Taylor series} about the point $a$) \index{series!Taylor}
    %\\
    %\ff(x) &=& \sum_{n=0}^\infty \frac{\brs{\opDif^n\ff}(0)}{n!}\: x^n
    %       &   \forall \ff\in\spC
    %       &   (\hie{Maclaurin series}) \index{series!Maclaurin}
  \end{array}$
  \\A \opd{Maclaurin series} is a \fncte{Taylor series} about the point $a=0$.
  }
\end{theorem}
}

%---------------------------------------
\begin{theorem}[\thmd{Binomial Theorem}]
\footnote{
  \citerpgc{graham1994}{162}{0201558025}{(5.12)},
  \citerpgc{rotman2010}{84}{0821847414}{Proposition 2.5},
  \citerpgc{bourbaki2003algebraI}{99}{3540642439}{Corollary 1},
  \citerppgc{warner1990}{189}{190}{0486663418}{Theorem 21.1},
  \citerpc{metzler1908}{169}{any real exponent},
  %\citerpc{hall1894}
  \citor{coolidge1949}
  }
\label{thm:binomial}
\index{Binomial Theorem}
\index{theorems!Binomial Theorem}
%---------------------------------------
\thmbox{
  (x+y)^n = \sum_{k=0}^n {n\choose k} x^{n-k} y^k
  \hspace{3ex}\mbox{where}\hspace{3ex}
  {n\choose k} \eqd \frac{n!}{(n-k)! k!}
  }
\end{theorem}
\begin{proof}
This theorem is proven using two different techniques.
Either is sufficient.
The first requires the Maclaurin series resulting in a more compact proof,
but requires the additional (here unproven) Maclaurin series.
The second proof uses induction resulting in a longer proof,
but does not require any external theorem.

\begin{enumerate}
\item Proof using Maclaurin series:
\begin{align*}
  (x+y)^n
    &= \sum_{k=0}^\infty \frac{1}{k!} \deriv{^k}{y^k}\Big[(x+y)^n\Big]_{y=0} y^k
    \qquad\text{by Maclaurin series (\prefp{thm:taylor})}
  \\&= \sum_{k=0}^\infty \frac{1}{k!} \Big[n(n-1)(n-2)\cdots(n-k+1)(x+y)^{n-k}\Big]_{y=0} y^k
  \\&= \sum_{k=0}^\infty \frac{1}{k!} \frac{n!}{(n-k)!} x^{n-k} y^k
  \\&= \sum_{k=0}^\infty {n\choose k} x^{n-k} y^k
    \qquad\text{by definition of ${n\choose k}$}
  \\&= \sum_{k=0}^n {n\choose k} x^{n-k} y^k
      +\cancelto{0}{\sum_{k=n+1}^\infty {n\choose k} x^{n-k} y^k}
  \\&= \sum_{k=0}^n {n\choose k} x^{n-k} y^k
    \qquad\text{because $(x+y)^n$ has order $n$}
\end{align*}

\item Proof using induction:
\begin{enumerate}
\item Proof that
  $(x+y)^{n} = \sum_{k=0}^{n} {n \choose k} x^k y^{n-k}$
  is true for $n=0$:
\begin{align*}
  \left.\sum_{k=0}^{n} {n \choose k} x^k y^{n-k} \right|_{n=0}
    &= {0 \choose 0} x^0 y^{0-0}
  \\&= 1
  \\&= (x+y)^n|_{n=0}
\end{align*}

\item Proof that
  $(x+y)^{n} = \sum_{k=0}^{n} {n \choose k} x^k y^{n-k}$
  is true for $n=1$:
\begin{align*}
  \left.\sum_{k=0}^{n} {n \choose k} x^k y^{n-k} \right|_{n=1}
    &= {1 \choose 0} x^0 y^{1-0} + {1 \choose 1} x^1 y^{1-1}
  \\&= y + x
  \\&= (x+y)^n|_{n=1}
\end{align*}

\item Proof that
  $(x+y)^{n} = \sum_{k=0}^{n} {n \choose k} x^k y^{n-k}
   \implies
   (x+y)^{n+1} = \sum_{k=0}^{n+1} {n \choose k} x^k y^{n+1-k}:
  $
\begin{align*}
  &\sum_{k=0}^{n+1} {n+1 \choose k} x^k y^{n+1-k}
  \\&= x^{n+1} + y^{n+1} + \sum_{k=1}^{n} {n+1 \choose k} x^k y^{n+1-k}
  \\&= x^{n+1} + y^{n+1}
     + \sum_{k=1}^{n} \left[ {n \choose k-1}+ {n \choose k} \right] x^k y^{n+1-k}
    \qquad\text{by \thme{Pascal's Rule} \ifxref{binomial}{thm:Pascals_Rule}}
  \\&= x^{n+1} + y^{n+1}
     + \sum_{k=1}^{n} {n \choose k-1} x^k y^{n+1-k}
     + \sum_{k=1}^{n} {n \choose k  } x^k y^{n+1-k}
  \\&= x^{n+1} + y^{n+1}
     + \left[ \sum_{k=0}^{n} {n \choose k  } x^{k+1} y^{n+1-(k+1)} -x^{n+1}\right]
     + \left[ \sum_{k=0}^{n} {n \choose k  } x^k y^{n+1-k} - y^{n+1} \right]
  \\&= x \sum_{k=0}^{n} {n \choose k } x^k y^{n-k}
     + y \sum_{k=0}^{n} {n \choose k } x^k y^{n-k}
  \\&= x (x+y)^n + y(x+y)^n
    \qquad\text{by left hypothesis}
  \\&= (x+y)(x+y)^n
  \\&= (x+y)^{n+1}
\end{align*}
\end{enumerate}
\end{enumerate}
\end{proof}

