%============================================================================
% Daniel J. Greenhoe
% LaTeX file
%============================================================================
%=======================================
\subsection{Fourier Analysis}
%=======================================
%=======================================
%\subsubsection{Fourier analysis of die sequences}
%=======================================
%---------------------------------------
\begin{example}[\exmd{length 1200 non-stationary die sequence with 10Hz oscillating mean}]
\label{ex:nonstat34}
\addcontentsline{toc}{subsubsection}{* length 1200 non-stationary die sequence with 10Hz oscillating mean}
\mbox{}\\
%---------------------------------------
\begin{enumerate}
  \item \label{item:nonstat34_psp}
     Suppose we have a length $\xN\eqd1200$ die sequence $\seqn{x_n}$ with the following distribution:
     \\\indentx$\begin{array}{lDlD}
       \psp(\dieA)=\psp(\dieB)=\psp(\dieD)=\psp(\dieE)=\psp(\dieF)=0.15 &and& \psp(\dieC)=0.25            &     \\
       \mc{3}{M}{\qquad for $n\in\set{p+ (2m)\frac{\xM}{2}}{p=0,1,\ldots,\frac{\xM}{2}-1,\,m=0,1,2,\ldots,9}$} & and  \\
       \psp(\dieA)=\psp(\dieB)=\psp(\dieC)=\psp(\dieE)=\psp(\dieF)=0.15 &and& \psp(\dieD)=0.25 &                \\
       \mc{3}{M}{\qquad for $n\in\set{p+(2m+1)\frac{\xM}{2}}{p=0,1,\ldots,\frac{\xM}{2}-1,\,m=0,1,2,\ldots,9}$} & 
     \end{array}$\\
     where $\xM\eqd120$.
     That is, the distribution of the sequence oscillates every $\sfrac{\xM}{2}=60$ samples between one that favors $\dieC$ 
     and one that favors $\dieD$.
     Moreover, if we were to evaluate the sequence using a \ope{Discrete Fourier Transform} operator $\dft$ \xref{def:dft}, 
     we might expect to see a strong component at $\frac{\xN}{\xM}=10$ 
     (or 10 Hz---the distribution goes through 10 cycles during the course of the sequence).
  
  \item \label{item:nonstat34_R1pam}
        Suppose we first use the \fncte{PAM die random variable} \xref{def:rv_diepam} to map
        the sequence of \pref{item:nonstat34_psp} into $\R^1$.
        The magnitude of the $\dft:\R^1\to\C^1$ of the mapped sequence is as follows:
     \\\begin{tabular}{|>{\scs}c|}
          \hline
          \includegraphics{../common/math/sspplots/diedftR1_1525_1200m120.pdf}%
        \\\hline
     \end{tabular}\\
     Looking at the above result, it would be next to impossible to discern that the distribution had a significantly strong
     oscillation of 10 cycles.
     In fact, the magnitude of the DFT at 10Hz is only $0.895699$, or $10\log_{10}(0.895699)=-0.478377$ dB.
     There are exactly 456 out of a total $\sfrac{\xN}{2}=600$ values that are greater than 
     the DFT magnitude at 10Hz.\footnote{\seessp{diedft_1525_1200m120.xlg}}
     % 
     That is, to either a human observer or a machine algorithm, the 10Hz component is effectively lost in the noise.
     
  \item \label{item:nonstat34_C1qpsk}
    Suppose we next use the \fncte{QPSK die random variable} \xref{def:rv_dieqpsk} to map
    the sequence into the complex plane.
    The magnitude of the $\dft:\C^1\to\C^1$ operation on the mapped sequence is as follows:
    \\\begin{tabular}{|>{\scs}c|}
         \hline
         \includegraphics{../common/math/sspplots/diedftC1_1525_1200m120.pdf}%
       \\\hline
    \end{tabular}\\
    %Again, the 10Hz component is effectively lost in the noise.
    The magnitude of the DFT at 10Hz is $0.589990$, or $10\log_{10}(0.589990)=-2.291552$ dB.
    There are exactly 831 out of a total $\xN=1200$ values that are greater than 
    the DFT magnitude at 10Hz.\footnote{\seessp{diedft_1525_1200m120.xlg}}
    Again, the 10Hz component is effectively lost in the noise.
     
  \item \label{item:nonstat34_R6}
    Suppose we next use the \fncte{$\R^6$ die random variable} \xref{def:rv_dieR6} to map
    the sequence into $\R^6$.
    The magnitude of $\dft:\R^6\to\C^6$ of the mapped sequence is as follows:
    \\\begin{tabular}{|>{\scs}c|}
         \hline
         \includegraphics{../common/math/sspplots/diedftR6_1525_1200m120.pdf}%
       \\\hline
    \end{tabular}\\
    The magnitude at 10Hz is $1.556295$, or $10\log_{10}(1.556295)=1.920920$ dB.
    Besides the DC component (0Hz component), this is the uniquely greatest value of the 600 samples.
    And in fact, there are only 5 out of a total $\sfrac{\xN}{2}=600$ samples
    that are $0.90\times1.556295$ or greater.\footnote{\seessp{diedft_1525_1200m120.xlg} 
      The 5 largest values are the points
      $\opair{0}{14.251433}$, $\opair{10}{1.556295}$, $\opair{344}{1.513501}$, $\opair{456}{1.405843}$ and $\opair{557}{1.468970}$.
      } 
    %And, besides the DC component, the 10Hz component is the uniquely greatest value of the 600 samples.
    % dieC1dft_1200m120_20160312_044601.log
    Thus, using the $\R^6$ mapping technique of this example, 
    it is much simpler to detect the 10Hz oscillating distribution.
\end{enumerate}
\end{example}

%---------------------------------------
\begin{example}[\exmd{length 12000 non-stationary die sequence with 10Hz oscillating mean}]
\label{ex:nonstat34_12000m1200}
\addcontentsline{toc}{subsubsection}{* length 12000 non-stationary die sequence with 10Hz oscillating mean}
\mbox{}\\
%---------------------------------------
\begin{enumerate}
  \item \label{item:nonstat34_12000m1200_psp}
    Suppose we have a length $\xN\eqd12000$ die sequence $\seqn{x_n}$ with the following distribution:
    \\\indentx$\begin{array}{lDlD}
      \psp(\dieA)=\psp(\dieB)=\psp(\dieD)=\psp(\dieE)=\psp(\dieF)=0.16 &and& \psp(\dieC)=0.20            &     \\
      \mc{3}{M}{\qquad for $n\in\set{p+ (2m)\frac{\xM}{2}}{p=0,1,\ldots,\frac{\xM}{2}-1,\,m=0,1,2,\ldots,9}$} & and  \\
      \psp(\dieA)=\psp(\dieB)=\psp(\dieC)=\psp(\dieE)=\psp(\dieF)=0.16 &and& \psp(\dieD)=0.20 &                \\
      \mc{3}{M}{\qquad for $n\in\set{p+(2m+1)\frac{\xM}{2}}{p=0,1,\ldots,\frac{\xM}{2}-1,\,m=0,1,2,\ldots,9}$} & 
    \end{array}$\\
    where $\xM\eqd1200$.
    That is, the distribution of the sequence oscillates every $\sfrac{\xM}{2}=600$ samples between one that favors $\dieC$ 
    and one that favors $\dieD$.
    If we were to evaluate the sequence using the \ope{Discrete Fourier Transform} operator, 
    we again might expect to see a strong component at $\frac{\xN}{\xM}=10$ 
    (or 10 Hz---the distribution goes through 10 cycles during the course of the sequence).
  
  \item \label{item:nonstat34_12000m1200_R1pam}
    Suppose we first use the \fncte{PAM die random variable} \xref{def:rv_diepam} to map
    the sequence of \pref{item:nonstat34_12000m1200_psp} into $\R^1$.
    In the magnitude of $\dft:\R^1\to\C^1$ there are 1130 values out of a possible $\sfrac{\xN}{2}=6000$ 
    values greater than the value at 10Hz (that value being $2.174512$).\footnote{\seessp{diedft_1620_12000m1200.xlg}}
    As in \prefpp{ex:nonstat34}, the subtle 10Hz component is effectively lost in the noise.
     
  \item \label{item:nonstat34_12000m1200_C1qpsk}
    Suppose we next use the \fncte{QPSK die random variable} \xref{def:rv_dieqpsk} to map
    the sequence into the complex plane.
    There are exactly 1932 out of a total $\xN=12000$ values that are greater than the DFT value at 10Hz
    (that value being $1.348693$).\footnote{\seessp{diedft_1620_12000m1200.xlg}}
    As in \prefpp{ex:nonstat34}, the subtle 10Hz component is effectively lost in the noise.
     
  \item \label{item:nonstat34_12000m1200_R6}
    Suppose we next use the \fncte{$\R^6$ die random variable} \xref{def:rv_dieR6} to map
    the sequence into $\R^6$.
    The magnitude of $\dft:\R^6\to\C^6$ of the mapped sequence is as follows:
    \\\begin{tabular}{|>{\scs}c|}
         \hline
         \includegraphics{../common/math/sspplots/diedftR6_1620_12000m1200.pdf}%
       \\\hline
    \end{tabular}\\
    Besides the DC component, the value at 100Hz (that value being ($1.965018$) 
    is the uniquely greatest value of the $\sfrac{\xN}{2}=6000$ samples;
    and it is $10\log_{10}(1.965018/1.660189)=0.699\cdots$dB larger than the next largest value.\footnote{
    \seessp{diedft_1620_12000m1200.xlg}
    The 10 largest values are
    $\opair{    0}{44.763194}$, $\opair{   10}{ 1.965018}$, $\opair{   90}{ 1.602474}$, $\opair{ 1223}{ 1.660189}$,
    $\opair{ 1313}{ 1.555349}$, $\opair{ 2385}{ 1.551028}$, $\opair{ 3039}{ 1.550918}$, $\opair{ 4154}{ 1.563756}$,
    $\opair{ 4187}{ 1.586362}$, and $\opair{ 5147}{ 1.623052}$.}
    Thus, even though the oscillating distribution is very subtle
    (even more subtle than that of \prefpp{ex:nonstat34}), the $\R^6$ mapping technique and subsequent analysis
    are able to detect it.
\end{enumerate}
\end{example}

%---------------------------------------
\begin{example}[\exmd{length 12000 non-stationary die sequence with 100Hz oscillating mean}]
\label{ex:nonstat34_12000m120}
\addcontentsline{toc}{subsubsection}{* length 12000 non-stationary die sequence with 100Hz oscillating mean}
\mbox{}\\
%---------------------------------------
\begin{enumerate}
  \item \label{item:nonstat34_12000m120_psp}
     Suppose we have a length $\xN\eqd12000$ die sequence $\seqn{x_n}$ with the following distribution:
     \\\indentx$\begin{array}{lDlD}
       \psp(\dieA)=\psp(\dieB)=\psp(\dieD)=\psp(\dieE)=\psp(\dieF)=0.16 &and& \psp(\dieC)=0.20            &     \\
       \mc{3}{M}{\qquad for $n\in\set{p+ (2m)\frac{\xM}{2}}{p=0,1,\ldots,\frac{\xM}{2}-1,\,m=0,1,2,\ldots,9}$} & and  \\
       \psp(\dieA)=\psp(\dieB)=\psp(\dieC)=\psp(\dieE)=\psp(\dieF)=0.16 &and& \psp(\dieD)=0.20 &                \\
       \mc{3}{M}{\qquad for $n\in\set{p+(2m+1)\frac{\xM}{2}}{p=0,1,\ldots,\frac{\xM}{2}-1,\,m=0,1,2,\ldots,9}$} & 
     \end{array}$\\
     where $\xM\eqd120$.
     That is, the distribution of the sequence oscillates every $\sfrac{\xM}{2}=60$ samples between one that favors $\dieC$ 
     and one that favors $\dieD$.
     If we were to evaluate the sequence using the \ope{Discrete Fourier Transform} operator, 
     we might expect to see a strong component at $\frac{\xN}{\xM}=100$
     (or 100 Hz---the distribution goes through 100 cycles during the course of the sequence).
  

  \item \label{item:nonstat34_12000m120_R1pam}
    Suppose we first use the \fncte{PAM die random variable} \xref{def:rv_diepam} to map
    the sequence of \pref{item:nonstat34_12000m120_psp} into $\R^1$.
    In the magnitude $\dft:\R^1\to\C^1$ there are 1320 values out of a possible $\sfrac{\xN}{2}=6000$ 
    values greater than the value at 100Hz
    that value being $2.081469$).\footnote{\seessp{diedft_1620_12000m120.xlg}}
    The subtle 100Hz component is effectively lost in the noise.
     
  \item \label{item:nonstat34_12000m120_C1qpsk}
    Suppose we next use the \fncte{QPSK die random variable} \xref{def:rv_dieqpsk} to map
    the sequence into the complex plane.
    There are exactly 1555 out of a total $\xN12000$ values that are greater than the DFT value at 100Hz
    that value being $1.425427$).\footnote{\seessp{diedft_1620_12000m120.xlg}}
    The subtle 100Hz component is effectively lost in the noise.
     
  \item \label{item:nonstat34_12000m120_R6}
    Suppose we next use the \fncte{$\R^6$ die random variable} \xref{def:rv_dieR6} to map
    the sequence into $\R^6$.
    The magnitude of $\dft:\R^6\to\C^6$ of the mapped sequence is as follows:
    \\\begin{tabular}{|>{\scs}c|}
         \hline
         \includegraphics{../common/math/sspplots/diedftR6_1620_12000m120.pdf}%
       \\\hline
    \end{tabular}\\
     Besides the DC component, the value at 100Hz (that value being $2.256927$) 
     is the uniquely greatest value of the $\sfrac{\xN}{2}=6000$.
     and it is $10\log_{10}(2.256927/1.599335)=1.495\cdots$dB larger than the next largest value.\footnote{
     \seessp{diedft_1620_12000m120.xlg}
     The 10 largest values are
       $\opair{   0}{44.763060}$, 
       $\opair{  90}{ 1.577597}$, 
       $\opair{ 100}{ 2.256927}$, 
       $\opair{ 486}{ 1.599335}$, 
       $\opair{1223}{ 1.585154}$, 
       $\opair{1313}{ 1.547956}$,
       $\opair{3039}{ 1.553522}$, 
       $\opair{3162}{ 1.561863}$,
       $\opair{5147}{ 1.558487}$, and
       $\opair{5567}{ 1.533659}$.
     }
     Thus, even though the oscillating distribution is very subtle, the $\R^6$ mapping technique
     and subsequent analysis are able to detect it.
\end{enumerate}
\end{example}

%=======================================
%\subsubsection{Fourier analysis of DNA sequences}
%=======================================
\begin{figure}%
  \centering%
  \gsize%
  $\begin{array}{|r|rMl|c|c|c|c|}
   \hline
   \mc{1}{|M|}{cycle}   & \mc{3}{|M|}{domain} & \psp(\dnaA) & \psp(\dnaC)  & \psp(\dnaG) & \psp(\dnaT)  \\\hline\hline
     0 &     0 &--&   599 & 0.24  & \cellcolor{bgp}  0.28  & 0.24  & 0.24                  \\
       &   600 &--&  1199 & 0.24  &                  0.24  & 0.24  & \cellcolor{bgp}  0.28 \\\hline
     1 &  1200 &--&  1799 & 0.24  & \cellcolor{bgp}  0.28  & 0.24  & 0.24                  \\
       &  1800 &--&  2399 & 0.24  &                  0.24  & 0.24  & \cellcolor{bgp}  0.28 \\\hline
     2 &  2400 &--&  2999 & 0.24  & \cellcolor{bgp}  0.28  & 0.24  & 0.24                  \\
       &  3000 &--&  3599 & 0.24  &                  0.24  & 0.24  & \cellcolor{bgp}  0.28 \\\hline
     3 &  3600 &--&  4199 & 0.24  & \cellcolor{bgp}  0.28  & 0.24  & 0.24                  \\
       &  4200 &--&  4799 & 0.24  &                  0.24  & 0.24  & \cellcolor{bgp}  0.28 \\\hline
     4 &  4800 &--&  5399 & 0.24  & \cellcolor{bgp}  0.28  & 0.24  & 0.24                  \\
       &  5400 &--&  5999 & 0.24  &                  0.24  & 0.24  & \cellcolor{bgp}  0.28 \\\hline
  \end{array}$%
  \hfill%
  $\begin{array}{|r|rMl|c|c|c|c|}%
   \hline%
   \mc{1}{|M|}{cycle}   & \mc{3}{|M|}{domain} & \psp(\dnaA) & \psp(\dnaC)  & \psp(\dnaG) & \psp(\dnaT)  \\\hline\hline
     5 &  6000 &--&  6599 & 0.24  & \cellcolor{bgp}  0.28  & 0.24  & 0.24                  \\
       &  6600 &--&  7199 & 0.24  &                  0.24  & 0.24  & \cellcolor{bgp}  0.28 \\\hline
     6 &  7200 &--&  7799 & 0.24  & \cellcolor{bgp}  0.28  & 0.24  & 0.24                  \\
       &  7800 &--&  8399 & 0.24  &                  0.24  & 0.24  & \cellcolor{bgp}  0.28 \\\hline
     7 &  8400 &--&  8999 & 0.24  & \cellcolor{bgp}  0.28  & 0.24  & 0.24                  \\
       &  9000 &--&  9599 & 0.24  &                  0.24  & 0.24  & \cellcolor{bgp}  0.28 \\\hline
     8 &  9600 &--& 10199 & 0.24  & \cellcolor{bgp}  0.28  & 0.24  & 0.24                  \\
       & 10200 &--& 10799 & 0.24  &                  0.24  & 0.24  & \cellcolor{bgp}  0.28 \\\hline
     9 & 10800 &--& 11399 & 0.24  & \cellcolor{bgp}  0.28  & 0.24  & 0.24                  \\
       & 11400 &--& 11999 & 0.24  &                  0.24  & 0.24  & \cellcolor{bgp}  0.28 \\\hline
  \end{array}$
  \caption{Distribution used in \prefpp{ex:nonstatdna}\label{fig:nonstatdna}}
\end{figure}
%---------------------------------------
\begin{example}[\exmd{length 12000 non-stationary artificial DNA sequence with 10Hz oscillating mean}]
\label{ex:nonstatdna}
\addcontentsline{toc}{subsubsection}{* length 12000 non-stationary artificial dna sequence with 10Hz oscillating mean}
\mbox{}\\
%---------------------------------------
\begin{enumerate}
  \item \label{item:nonstatdna_psp}
     Suppose we have a length $\xN\eqd12000$ die sequence $\seqn{x_n}$ with the following distribution (see also \prefp{fig:nonstatdna}):
     \\\indentx$\begin{array}{lDlD}
       \psp(\dnaA)=\psp(\dnaT)=\psp(\dnaG)=0.24 &and& \psp(\dnaC)=0.28            &     \\
       \mc{3}{M}{\qquad for $n\in\set{p+ 2m   \frac{\xM}{2}}{p=0,1,\ldots,\frac{\xM}{2}-1,\,m=0,1,2,\ldots,9}$} & and  \\
       \psp(\dnaA)=\psp(\dnaC)=\psp(\dnaG)=0.24 &and& \psp(\dnaT)=0.28            &     \\
       \mc{3}{M}{\qquad for $n\in\set{p+(2m+1)\frac{\xM}{2}}{p=0,1,\ldots,\frac{\xM}{2}-1,\,m=0,1,2,\ldots,9}$} & 
     \end{array}$\\
     where $\xM\eqd1200$.
     That is, the distribution of the sequence oscillates every $\sfrac{\xM}{2}=600$ samples between one that favors $\dnaC$ 
     and one that favors $\dnaT$.
     Moreover, if we were to evaluate the sequence using a \ope{Discrete Fourier Transform} (DFT) operator, 
     we might expect to see a strong component at $\frac{\xN}{\xM}=10$ 
     (or 10 Hz---the distribution goes through 10 cycles during the course of the sequence).
  
  \item \label{item:nonstatdna_R1pam}
    Suppose we first use the \fncte{PAM DNA random variable} \xref{def:rv_dnapam} to map
    the DNA sequence into $\R^1$.
    The magnitude of $\dft:\R^1\to\C^1$ of the sequence after applying this mapping is as follows:
    \\\begin{tabular}{|>{\scs}c|}
         \hline
         \includegraphics{../common/math/sspplots/dnadftR1.pdf}%
       \\\hline
    \end{tabular}\\
    %Looking at the above result, it would be next to impossible to discern that the distribution had a significantly strong
    %oscillation of 10 cycles.
    The magnitude of the DFT at 10Hz is only $1.163575$ ($10\log_{10}(1.163575)=0.657944$ dB).
    There are exactly 2023 out of a total $\sfrac{\xN}{2}=6000$ values that are greater than the DFT value at 10Hz
    (that value being $1.163575$).\footnote{\seessp{dnadft_12000m1200.xlg}}
    Here again, the 10Hz component is effectively lost in the noise.
    
  \item \label{item:nonstatdna_C1qpsk}
    Suppose we next use the \fncte{QPSK DNA random variable} \xref{def:rv_dieqpsk} to map
    the DNA sequence into the complex plane.
    The magnitude of $\dft:\C^1\to\C^1$ of the sequence after applying this mapping is as follows:
    \\\begin{tabular}{|>{\scs}c|}
         \hline
         \includegraphics{../common/math/sspplots/dnadftC1.pdf}%
       \\\hline
    \end{tabular}\\
    The DFT at 10Hz is $1.888671$, or $10\log_{10}(1.888671)=2.761563$ dB.
    There are exactly 343 out of a total $\xN=6000$ values that are greater than the DFT value at 10Hz.
    (that value being $1.888671$).\footnote{\seessp{dnadft_12000m1200.xlg}}
    Using this mapping it would be difficult to detect the subtle but significant 10Hz component.
     
  \item \label{item:nonstatdna_R4}
    Suppose we next use the \fncte{$\R^4$ DNA random variable} \xref{def:rv_dnaR4} to map
    the sequence into $\R^4$.
    The magnitude of $\dft:\R^4\to\C^4$ of the mapped sequence is as follows:
     \\\begin{tabular}{|>{\scs}c|}
          \hline
          \includegraphics{../common/math/sspplots/dnadftR4.pdf}%
        \\\hline
     \end{tabular}\\
     The magnitude of the DFT at 10Hz is $1.932042$ ($10\log_{10}(1.932042)=2.860166$ dB).
     Besides itself and the DC component, there are only two out of a total $\sfrac{\xN}{2}=6000$ samples
     that are greater or equal to this value.\footnote{
      \seessp{dnadft_12000m1200.xlg}\\
       The 4 largest values are at
         $\opair{   0}{54.791926}$, 
         $\opair{  10}{ 1.932042}$, 
         $\opair{4187}{ 1.962836}$, and 
         $\opair{5147}{ 2.057553}$.
       } 
     Thus, using the $\R^4$ mapping technique and subsequent analysis of this example, 
     it is much simpler to detect the 10Hz oscillation.
\end{enumerate}
\end{example}

%---------------------------------------
\begin{example}[\exmd{Fourier analysis of Ebola DNA sequence}]
\label{ex:dftebola}
\addcontentsline{toc}{subsubsection}{* Fourier analysis of Ebola DNA sequence}
\mbox{}\\
%---------------------------------------
\begin{enumerate}
  \item \label{item:dftebola_psp}
     Consider the Ebola DNA sequence described in \prefpp{ex:dna_ebola}.\footnote{\citeD{ncbiEbola}}
     DNA sequences commonly exhibit a strong DFT harmonic component at $\sfrac{2\pi}{3}$ 
     radians.\footnote{\citePp{galleani2010}{771}}
  
  \item \label{item:dftebola_R1pam}
    Suppose we first use the \fncte{PAM DNA random variable} \xref{def:rv_dnapam} to map
    the DNA sequence into $\R^1$.
    The magnitude of $\dft:\R^1\to\C^1$ of the sequence after applying this mapping is as follows:
    \\\begin{tabular}{|>{\scs}c|}
         \hline
         \includegraphics{../common/math/sspplots/dna_AF086833_ebola_dftR1.pdf}%
       \\\hline
    \end{tabular}\\
    The component at $\sfrac{2\pi}{3}$ is easy to pick out with a signal to noise ratio (SNR) of\\ 
    $10\log_{10}(4.290296/1.123163)\approx5.8$ dB.\footnote{\seessp{dna_AF086833_ebola_dft.xlg}}
    Here, the noise value 1.123163 is the \fncte{RMS} (\fncte{root mean square}) of the DFT magnitude 
    sequence from $n=1$ to $n=\xN/2-1$ computed as follows:
    \\\indentx$\ds\sqrt{\frac{1}{\xN/2-1}\sum_{n=1}^{n=\xN/2-1} x_n^2}$.
     
  \item \label{item:dftebola_C1qpsk}
    Suppose we next use the \fncte{QPSK DNA random variable} \xref{def:rv_dieqpsk} to map
    the dna sequence into the complex plane.
    The magnitude of $\dft:\C^1\to\C^1$ of the sequence after applying this mapping is as follows:
     \\\begin{tabular}{|>{\scs}c|}
          \hline
          \includegraphics{../common/math/sspplots/dna_AF086833_ebola_dftC1.pdf}%
        \\\hline
     \end{tabular}\\
    The component at $\sfrac{2\pi}{3}$ is again easy to pick out with a signal to noise ratio (SNR) of\\
    $10\log_{10}(6.412578/0.998659)\approx8.1$ dB.\footnote{\seessp{dna_AF086833_ebola_dft.xlg}}
    Here, the noise value 0.998659 is the \fncte{RMS} of the DFT magnitude 
    sequence from $n=1$ to $n=\xN-1$.
    %C^1 rms = 0.998685, max=6.412578   8.0760413380372487836054657324457 dB

  \item \label{item:dftebola_R4}
    Suppose we next use the \fncte{$\R^4$ DNA random variable} \xref{def:rv_dnaR4} to map
    the sequence into $\R^4$.
    The magnitude of $\dft:\R^4\to\C^4$ of the mapped sequence is as follows:
    \\\begin{tabular}{|>{\scs}c|}
         \hline
         \includegraphics{../common/math/sspplots/dna_AF086833_ebola_dftR4.pdf}%
       \\\hline
    \end{tabular}\\
    The component at $\sfrac{2\pi}{3}$ is again easy to pick out with a signal to noise ratio (SNR) of\\ 
    $10\log_{10}(3.944811/0.860665)\approx 6.6$ dB.\footnote{\seessp{dna_AF086833_ebola_dft.xlg}}
    Here, the noise value 0.860665 is the \fncte{RMS} of the DFT magnitude 
    sequence from $n=1$ to $n=\xN/2-1$.
    %R^4 rms = 0.860710, max=3.944811    6.6116935177820583786215400221832 dB

  \item In conclusion, for this application, there is only a small advantage to using 
        the $\R^4$ mapping \xrefr{item:dftebola_R4}
        versus the $\R^1$ mapping \xrefr{item:dftebola_R1pam}, 
        and even a demonstrable disadvantage when compared to the $\C^1$ mapping \xrefr{item:dftebola_C1qpsk}.
\end{enumerate}
\end{example}


%---------------------------------------
\begin{example}[\exmd{Fourier analysis of SARS-CoV-2 DNA sequence}]
\label{ex:dftSARS-CoV-2}
\addcontentsline{toc}{subsubsection}{* Fourier analysis of SARS-CoV-2 DNA sequence}
\mbox{}\\
%---------------------------------------
\begin{enumerate}
  \item \label{item:dftSARS-CoV-2_psp}
     Consider the SARS-CoV-2 DNA sequence described in \prefpp{ex:dna_SARSCoV2}.\footnote{\citeD{ncbiSARSCoV2}}
     %DNA sequences commonly exhibit a strong DFT harmonic component at $\sfrac{2\pi}{3}$ 
     %radians.\footnote{\citePp{galleani2010}{771}}
  
  \item \label{item:dftSARS-CoV-2_R1pam}
    %Suppose we first use the \fncte{PAM DNA random variable} \xref{def:rv_dnapam} to map
    %the DNA sequence into $\R^1$.
    The magnitude of $\dft:\R^1\to\C^1$ of the sequence after applying this mapping is as follows:
    \\\begin{tabular}{|>{\scs}c|}
         \hline
         \includegraphics{../common/math/graphics/pdfs/dna_SARSCoV2_dftR1.pdf}%
       \\\hline
    \end{tabular}
    %The component at $\sfrac{2\pi}{3}$ is easy to pick out with a signal to noise ratio (SNR) of\\ 
    %$10\log_{10}(4.290296/1.123163)\approx5.8$ dB.\footnote{\seessp{dna_AF086833_SARS-CoV-2_dft.xlg}}
    %Here, the noise value 1.123163 is the \fncte{RMS} (\fncte{root mean square}) of the DFT magnitude 
    %sequence from $n=1$ to $n=\xN/2-1$ computed as follows:
    %\\\indentx$\ds\sqrt{\frac{1}{\xN/2-1}\sum_{n=1}^{n=\xN/2-1} x_n^2}$.
     
  \item \label{item:dftSARS-CoV-2_C1qpsk}
    %Suppose we next use the \fncte{QPSK DNA random variable} \xref{def:rv_dieqpsk} to map
    %the dna sequence into the complex plane.
    The magnitude of $\dft:\C^1\to\C^1$ of the sequence after applying this mapping is as follows:
     \\\begin{tabular}{|>{\scs}c|}
          \hline
          \includegraphics{../common/math/graphics/pdfs/dna_SARSCoV2_dftC1.pdf}%
        \\\hline
     \end{tabular}
    %The component at $\sfrac{2\pi}{3}$ is again easy to pick out with a signal to noise ratio (SNR) of\\
    %$10\log_{10}(6.412578/0.998659)\approx8.1$ dB.\footnote{\seessp{dna_AF086833_SARS-CoV-2_dft.xlg}}
    %Here, the noise value 0.998659 is the \fncte{RMS} of the DFT magnitude 
    %sequence from $n=1$ to $n=\xN-1$.
    %C^1 rms = 0.998685, max=6.412578   8.0760413380372487836054657324457 dB

  \item \label{item:dftSARS-CoV-2_R4}
    %Suppose we next use the \fncte{$\R^4$ DNA random variable} \xref{def:rv_dnaR4} to map
    %the sequence into $\R^4$.
    The magnitude of $\dft:\R^4\to\C^4$ of the mapped sequence is as follows:
    \\\begin{tabular}{|>{\scs}c|}
         \hline
         \includegraphics{../common/math/graphics/pdfs/dna_SARSCoV2_dftR4.pdf}%
       \\\hline
    \end{tabular}
    %The component at $\sfrac{2\pi}{3}$ is again easy to pick out with a signal to noise ratio (SNR) of\\ 
    %$10\log_{10}(3.944811/0.860665)\approx 6.6$ dB.\footnote{\seessp{dna_AF086833_SARS-CoV-2_dft.xlg}}
    %Here, the noise value 0.860665 is the \fncte{RMS} of the DFT magnitude 
    %sequence from $n=1$ to $n=\xN/2-1$.
    %R^4 rms = 0.860710, max=3.944811    6.6116935177820583786215400221832 dB

  %\item In conclusion, for this application, there is only a small advantage to using 
  %      the $\R^4$ mapping \xrefr{item:dftSARS-CoV-2_R4}
  %      versus the $\R^1$ mapping \xrefr{item:dftSARS-CoV-2_R1pam}, 
  %      and even a demonstrable disadvantage when compared to the $\C^1$ mapping \xrefr{item:dftSARS-CoV-2_C1qpsk}.
\end{enumerate}
\end{example}


