
% LaTeX File
% Daniel J. Greenhoe
%============================================================================





%======================================
\chapter{Infinite Sums}
%======================================

%=======================================
%\section{Combinitorial relations}
%=======================================

\qboxnpq{Niels Henrik Abel, in a January 16, 1826 letter to Holmbo/'e \footnotemark}
  {../common/people/abelnh_wkp_pdomain.jpg}
  {The divergent series are the invention of the devil, and it is a shame to base on them any demonstration whatsoever.
   By using them, one may draw any conclusion he pleases and that is why these series have produced so many 
   fallacies and so many paradoxes\ldots}
\citetblt{
  quote: & \citerpgc{kline1972}{973}{0195061373}{Chapter 47}\\ %{Chapter 47 Divergent Series}\\
  image: & \url{http://en.wikipedia.org/wiki/File:Niels_Henrik_Abel.jpg}, public domain
  }

\qboxnps{Oliver Heaviside (1850--1925) \footnotemark}
{../common/people/heaviside_wkp_pdomain.jpg}
{The series is divergent; therefore we may be able to do something with it.}
\citetblt{
  quote: & \citerpgc{kline1972}{1096}{0195061373}{Chapter 47}\\ % Divergent Series\\
  image: & \url{http://en.wikipedia.org/wiki/File:Oliver_Heaviside2.jpg}, public domain
  }

\qboxns{Ivor Grattan-Guinness (1990)\footnotemark}
{Some modern appraisals of the cavalier style of 18th-century mathematicians in handling 
infinite series conveys the impression that these poor men set their brains aside when 
confronted by them.}
\citetblt{\citerpg{grattan1990}{163}{3764322373}}

%=======================================
\section{Convergence}
\label{sec:series_convergence}
%=======================================
An infinite summation $\sum_{n=1}^\infty x_n$ is meaningless outside some 
topological space (e.g. metric space, normed space, etc.).
The sum $\sum_{n=1}^\infty \vx_n$ is an abbreviation for $\lim_{\xN\to\infty}\sum_{n=1}^\xN \vx_n$
(next definition); and the concept of \structe{limit}\ifsxref{topology}{def:limit} is also itself meaningless outside of a 
\structe{topological space}\ifsxref{topology}{def:topology}.
%---------------------------------------
\begin{definition}
\footnote{
  \citerpg{klauder2010}{4}{0817647902},
  \citerpg{kubrusly2001}{43}{0817641742},
  \citerppg{bachman1966}{3}{4}{0486402517}
  %\citerppg{bachman2000fa}{3}{4}{0486402517}\\
  }
\label{def:suminf}
%---------------------------------------
Let $\topspaceX$ be a topological space and $\lim$ be the limit generated by the topology $\topT$.
\defbox{\begin{array}{>{\ds}rc>{\ds}lc>{\ds}l}
  \sum_{n=1}^\infty       \vx_n &\eqd& \sum_{n\in\Zp} \vx_n &\eqd& \lim_{\xN\to\infty} \sum_{n=1}^\xN \vx_n\\
  \sum_{n=-\infty}^\infty \vx_n &\eqd& \sum_{n\in\Z}  \vx_n &\eqd& \lim_{\xN\to\infty} \brp{\sum_{n=0}^\xN \vx_n} + \brp{\lim_{\xN\to-\infty} \sum_{n=-1}^\xN \vx_n}
\end{array}}
\end{definition}

In general, the order of summation of an infinite series \emph{does} matter.
%In fact, in the case of the \hie{Logarithmic Series
% (next example). 

%---------------------------------------
\begin{definition}
\footnote{
  \citerpgc{kadets1997}{5}{3764354011}{\scshape Theorem 1.1.1 (Riemann's theorem)},
  \citerpc{bromwich1908}{64}{\scshape IV. Absolute Convergence.}, %{\scshape Chapter IV. Absolute Convergence.}
  %\citerpg{klauder2010}{4}{0817647902}
  \citerp{szasz1952}{2}
  }
\label{def:abscon}
%---------------------------------------
Let $\setP$ be the set of all \structe{permutations} in $\clFnn$.
\defbox{\begin{array}{M}
  A series $\ds\sum_{n=1}^\infty \vx_n$ is \hid{absolutely convergent} if \qquad$\ds\sum_{n=1}^\infty \vx_n=\sum_{n=1}^\infty \vx_{\fp(n)} \quad \forall \fp\in\setP$
  \\
  A series is \hid{conditionally convergent} if it is \prope{convergent}
  \\\indentx but not \prope{absolutely convergent}.
\end{array}}
\end{definition}

%---------------------------------------
\begin{theorem}[\thm{Riemann Series Theorem}]
\footnote{
  \citerpgc{kadets1997}{5}{3764354011}{\scshape Theorem 1.1.1 (Riemann's theorem)},
  \citerpc{bromwich1908}{68}{Article 28. Riemann's Theorem}
  }
%---------------------------------------
Let $\fp(n)$ be a permutation on $\Zp$.
Let $\seqxZ{a_n}$ be a sequence of real numbers.
\thmbox{
  \brb{\begin{array}{M}
    $\ds\sum_{n=1}^\infty a_n$ is\\
    \prope{conditionally convergent}
  \end{array}}
  \quad\implies\quad
  \brb{\begin{array}{M}
  For every $x\in\R$ there exists $\fp$ \\
  such that $\ds\sum_{n=1}^\infty a_{\fp(n)}=x$\\
  or such that $\ds\sum_{n=1}^\infty a_{\fp(n)}$ is \prope{divergent}
  \end{array}}
  }
\end{theorem}

%---------------------------------------
\begin{theorem}
\footnote{
  \citerpgc{kadets1997}{5}{3764354011}{\scshape Theorem 1.1.1 (Riemann's theorem)}
  %\citerpg{klauder2010}{4}{0817647902}
  }
%---------------------------------------
\thmbox{
  \sum_{n=1}^\infty \abs{\vx_n}<\infty 
  \qquad\implies\qquad
  \brb{
  \sum_{n=1}^\infty \vx_n
  \quad
  \text{is \prope{absolutely convergent}}.
  }
  }
\end{theorem}


%---------------------------------------
\begin{example}[\exmd{Logarithmic Series}]
\footnote{
  %\citerpg{klauder2010}{4}{0817647902}\\
  \citerppc{bromwich1908}{51}{52}{Article 21 Example 1},
  \citerpc{hall1894}{191}{article 223},
  \citerppgc{jolley1961}{14}{15}{0486441601}{item (71)},
  \citeruc{oeis}{http://oeis.org/A002939}{$2n(2n-1)$},
  %\citerc{tijdeman2002}{\small\scshape ``4. On the transcendence of infinite sums."}\\
  %\citeruc{oeis}{http://oeis.org/A118239}{\fncte{Engel expansion} of $\cosh(1)$}\\
  \citerpgc{graham1994}{99}{0201558025}{n.w. diagonal of \fncte{spiral function}}, %{north-west diagonal of \fncte{spiral function}}\\
  \parbox{\tw-10mm}{
  Many many thanks to Po-Ning Chen (Chinese: \zht{???}, pinyin: Ch/'en B/'o Ni/'ng) for his consultation regarding this series.
  %``Since both sums of positive numbers and negative numbers are infinite, 
  %playing with the order can give you any convergent value you want."---Po-Ning Chen (\zht{???}), 2012 September 9
  }
  }
\label{ex:logseries}
%---------------------------------------
%Let $\fp(n)$ be a permutation that partitions the natural numbers into odd and even values.
Consider the sum 
  $\sum_{n=1}^\infty (-1)^{(n-1)} \frac{1}{n}$. 
To which value this sum converges, or whether it even converges at all, depends on the order in which the terms are summed.
This is demonstrated by the following series:
\exbox{\begin{array}{>{\ds}rc>{\ds}l}
  \mc{3}{M}{\imark If the series is added in the given order, the result is $\ln2$:}
  \\
  \sum_{n=1}^\infty (-1)^{(n-1)} \frac{1}{n} 
    &\eqd& \lim_{\xN\to\infty} \sum_{n=1}^\xN (-1)^{(n-1)} \frac{1}{n} 
    %&=&\mcom{\brp{1 - \frac{1}{2}}}{$\frac{1}{2}$} 
    %   + \mcom{\brp{\frac{1}{3} - \frac{1}{4}}}{$\frac{1}{12}$} 
    %   + \mcom{\brp{\frac{1}{5} - \frac{1}{6}}}{$\frac{1}{30}$} 
    %   + \mcom{\brp{\frac{1}{7} - \frac{1}{8}}}{$\frac{1}{56}$} 
    %   +\ldots
    = \ln 2
  \\ \mc{3}{M}{\imark But if the order is changed, the sum can be any real value:}
  \\ \mc{3}{M}{\quad Let $x$ be any real value (even an irrational one such as $\pi$ or $\sqrt{2}$).}
  \\
  \text{Let}\qquad
  \fp(N) &\eqd& \brbl{\begin{array}{MMlcc}
    %$1$                    & if & N=1 \\
    next unused odd  value & if & \sum_{n=1}^{\xN-1} (-1)^{(\fp(n)-1)} \frac{1}{\fp(n)} &\le& x\\
    next unused even value & if & \sum_{n=1}^{\xN-1} (-1)^{(\fp(n)-1)} \frac{1}{\fp(n)} &>&   x
    \end{array}}
  \\
  \sum_{n=1}^\infty (-1)^{(\fp(n)-1)} \frac{1}{\fp(n)} 
    &\eqd& \lim_{\xN\to\infty} \sum_{n=1}^\xN (-1)^{(\fp(n)-1)} \frac{1}{\fp(n)}
    =    x
  \\\mc{3}{M}{\imark The series can even be summed in such a way that it does not converge at all:}
  \\\mc{3}{M}{\quad Let $\fq(n)$ be a permutation that partitions the natural numbers into}
  \\\mc{3}{M}{\quad odd and even values such that $\seqn{x_{\fq(n)}}=\seqn{1,3,5,\ldots,2,4,6,\ldots}$.}
  \\
  \sum_{n=1}^\infty (-1)^{(\fq(n)-1)} \frac{1}{\fq(n)} 
    &=& \mcom{\ds\sum_{n=1}^\infty \frac{1}{2n-1}}{$\infty$} - \mcom{\ds\sum_{n=1}^\infty \frac{1}{2n}}{$\infty$}
    \qquad\implies\qquad \text{diverges}
\end{array}}
\end{example}
\begin{proof}
\begin{enumerate}
  \item Proof that $\ds\sum_{n=1}^\infty (-1)^{(n-1)} \frac{1}{n}=\ln 2$ using polynomial expansion:\cittrppc{bromwich1908}{51}{52}{Article 21 Example 1}
    \begin{enumerate}
      \item Lemma: Proof that $\frac{1}{1+x}=\sum_{k=0}^{2n-1} (-1)^k x^k + \frac{x^{2n}}{1+x}$:\label{item:logseries_xk}
        \begin{align*}
          (1+x)\brp{\sum_{k=0}^{2n-1} (-1)^{k} x^k + \frac{x^{2n}}{1+x}}
            &= \sum_{k=0}^{2n-1} (-1)^{k} x^k + \sum_{k=0}^{2n-1} (-1)^{k} x^{k+1} + x^{2n}
          \\&= 1 + \sum_{k=1}^{2n-1} (-1)^{k} x^k - \sum_{k=1}^{2n-1} (-1)^{k} x^{k} + x^{2n}
          \\&= 1 + x^{2n}
        \end{align*}

      \item Lemma: Proof that $\lim_{n\to\infty}\int_0^1 \frac{x^{2n}}{1+x} \dx=0$:\label{item:logseries_2n1}
        \begin{align*}
          0
            &< \lim_{n\to\infty}\int_0^1 \frac{x^{2n}}{1+x} \dx 
          \\&< \lim_{n\to\infty} \int_0^1 x^{2n} \dx 
          \\&= \lim_{n\to\infty} \left. \frac{x^{2n+1}}{2n+1} \right|_0^1
          \\&= \lim_{n\to\infty} \frac{1}{2n+1}
          \\&= 0
        \end{align*}
  
      \item Proof that sum = $\ln2$:
        \begin{align*}
          \ln2
            &= \ln2 - \ln1
          \\&= \int_1^2 \frac{1}{x} \dx
          \\&= \int_0^1 \frac{1}{x+1} \dx
          \\&= \lim_{n\to\infty}\int_0^1 \brb{\sum_{k=0}^{2n-1} (-1)^k x^k + \frac{x^{2n}}{1+x}} \dx 
            && \text{by \pref{item:logseries_xk}}
          \\&= \sum_{k=0}^{2n-1} (-1)^k \int_0^1  x^k \dx + \lim_{n\to\infty}\int_0^1 \frac{x^{2n}}{1+x} \dx 
          \\&= \lim_{n\to\infty}\sum_{k=0}^{2n-1} (-1)^k \frac{1}{k} + 0
            && \text{by \pref{item:logseries_2n1}}
          \\&= \lim_{n\to\infty}\sum_{k=1}^{2n} (-1)^{k-1} \frac{1}{k}
          \\&= \sum_{n=1}^{\infty} (-1)^{n-1} \frac{1}{n}
        \end{align*}
    \end{enumerate}

  \item Proof that $\ds\sum_{n=1}^\infty (-1)^{(n-1)} \frac{1}{n}=\ln 2$ using Taylor expansion:\cittrpc{hall1894}{191}{article 223}
    \begin{enumerate}
      \item Lemma: Proof that $\ds\ln(x+1) = \sum_{n=1}^\infty (-1)^{(n+1)}\frac{1}{n}x^n$:
        \begin{align*}
          \ln(x+1)
            &= \sum_{n=0}^\infty \frac{\brs{\opD^n\ln(x+1)}(0)}{n!}\:x^n
          \qquad \text{by \hie{Taylor series expansion} \ifdochas{polynom}{(\prefp{thm:taylor})}}
          \\&= \frac{\ln(1+0)}{0!}x^0 + \frac{\frac{1}{1+0}}{1!}x^1 - \frac{\frac{1}{(1+0)^2}}{2!}x^2 + \frac{\frac{2}{(1+0)^3}}{3!}x^3 - \frac{\frac{6}{(1+0)^4}}{4!}x^4 + \frac{\frac{24}{(1+0)^5}}{5!}x^5 - \frac{\frac{120}{(1+0)^6}}{6!}x^6 + \cdots
          \\&= 0 + x - \frac{1}{2}x^2 + \frac{1}{3}x^3 - \frac{1}{4}x^4 + \frac{1}{5}x^5 - \frac{1}{6}x^6 + \cdots
          \\&= \sum_{n=1}^\infty (-1)^{(n+1)}\frac{1}{n}x^n
        \end{align*}
      
      \item Proof that $\ds\sum_{n=1}^\infty (-1)^{(n-1)} \frac{1}{n}=\ln 2$:
        \begin{align*}
            &= \mcom{\brp{1 - \frac{1}{2}}}{$\frac{1}{2}$} 
               + \mcom{\brp{\frac{1}{3} - \frac{1}{4}}}{$\frac{1}{12}$} 
               + \mcom{\brp{\frac{1}{5} - \frac{1}{6}}}{$\frac{1}{30}$} 
               + \mcom{\brp{\frac{1}{7} - \frac{1}{8}}}{$\frac{1}{56}$} 
               +\ldots
          \\&= \left.\sum_{n=1}^\infty (-1)^{(n+1)}\frac{1}{n}x^n\right|_{x=1}
          \\&= \ln 2
            && \text{by Lemma in item 1}
          \\&\approx 0.693147
        \end{align*}
    \end{enumerate}
  
  \item Proof that $\ds\sum_{n=1}^\infty (-1)^{(\fp(n)-1)} \frac{1}{\fp(n)}= x$: 
        If the partial sum is less than $x$, positive values are added.
        If the partial sum is greater than $x$, negative values are added.
        The limit is $x$.  %\cittrpg{klauder2010}{4}{0817647902}

  \item Proof that $\ds\sum_{n=1}^\infty (-1)^{(\fq(n)-1)} \frac{1}{\fq(n)}=\infty-\infty$:
    \begin{align*} 
      \sum_{n=1}^\infty (-1)^{(\fq(n)-1)} \frac{1}{\fq(n)} 
        &= \mcom{\ds\sum_{n=1}^\infty (-1)^{(2n-1-1)} \frac{1}{2n-1}}{odd indices} 
         + \mcom{\ds\sum_{n=1}^\infty (-1)^{(2n-1)} \frac{1}{2n}}{even indices}
      \\&= \mcom{\ds\sum_{n=1}^\infty \frac{1}{2n-1}}{$\infty$} - \mcom{\ds\sum_{n=1}^\infty \frac{1}{2n}}{$\infty$}
      %\\&\eqq \infty - \infty
      \\&\rightarrow \text{(diverges)}
    \end{align*}
\end{enumerate}
\end{proof}

Divergent series could even result in decisions that may be considered extremely irrational, as demonstrated
by \exme{St. Petersberg Paradox} (next).
%---------------------------------------
\begin{example}[\exm{St. Petersburg Paradox}]
\footnote{
  \citerppg{szekely1986}{27}{28}{9027718997}, %{7. \scshape St. Petersburg Paradox}\\
  \citorppc{bernoulli1783e}{31}{32}{\textsection 17},
  \citorpgc{montmort1713}{402}{9780821837818}{1713 letter from Nicolas Bernoulli}
  }
%---------------------------------------
There is a lottery with a prize pot of \$1.
A coin is tossed.
If the coin is a tail, the money in the lottery is doubled (\$2, \$4, \$8, \$16, \ldots).
If the coin is a head, you win the money and the game is finished.

How much money would you be willing to play this game?
The answer to this question for some people may depend on the expected value of how much money would be won.
But the expected value of the amount of money you would win is
\\\indentx$ 
  \frac{1}{2}  \times \$1 + 
  \frac{1}{4}  \times \$2 + 
  \frac{1}{8}  \times \$4 + 
  \frac{1}{16} \times \$8 + \ldots 
  = 
  \frac{1}{2} + 
  \frac{1}{2} + 
  \frac{1}{2} + 
  \frac{1}{2} + \ldots
  = \infty
  $

Since the expected value of the win is infinity, you should be willing to pay any finite amount of money to play this game
(even trillions of dollars).
But yet common sense would tell most people that this would be an unwise investment.
\end{example}


%=======================================
\section{Multiplication}
%=======================================
%--------------------------------------
\begin{theorem}
\label{thm:seq_mult}
\footnote{
  \citerpgc{apostol1975}{204}{0201002884}{Definition 8.45} %note that convolution is a single element of the Cauchy product
  %\citerpg{apostol1975}{237}{0201002884}
  }
\index{Cauchy product}
%--------------------------------------
Let $\tuple{x_n}{1}{\xN}$ and $\tuple{y_n}{1}{\xN}$ be sequences over a ring $\ring$.
\thmbox{
  \left(\sum_{n=0}^p x_n  \right)
  \left(\sum_{m=0}^q y_m  \right)
  =
  \sum_{n=0}^{p+q} 
  \mcom{\ds\left(\sum_{k=\max(0,n-q)}^{\min(n,p)} x_k y_{n-k} \right)}
       {Cauchy product} 
  }
\end{theorem}
\begin{proof}
\begin{enumerate}
\item 
\begin{align*}
  \left(\sum_{n=0}^p x_n \right)\left(\sum_{m=0}^q y_m   \right)
    &= \sum_{n=0}^p \sum_{m=0}^q x_n y_m z^{n+m}
  \\&= \sum_{n=0}^p \sum_{k=n}^{q+n} x_n y_{k-n}  
    && k=n+m \qquad m=k-n
  \\&\vdots
  \\&= \sum_{n=0}^{p+q} 
       \left(\sum_{k=0}^n x_k y_{n-k} \right)  
\end{align*}

\item Perhaps the easiest way to see the relationship is by illustration with
      a matrix of product terms:
\[\begin{array}{>{\color{blue}}l | *{6}{l}}
      & \color{blue}y_0       & \color{blue}y_1       & \color{blue}y_2       &\color{blue} y_3       & \color{blue}\cdots & \color{blue}y_q           \\
  \hline
  x_0 & x_0y_0 & x_0y_1 & x_0y_2 & x_0y_3 & \cdots & x_0y_q  \\
  x_1 & x_1y_0 & x_1y_1 & x_1y_2 & x_1y_3 & \cdots & x_1y_q  \\
  x_2 & x_2y_0 & x_2y_1 & x_2y_2 & x_2y_3 & \cdots & x_2y_q  \\
  x_3 & x_3y_0 & x_3y_1 & x_3y_2 & x_3y_3 & \cdots & x_3y_q  \\
  \vdots & \vdots & \vdots    & \vdots    & \vdots    & \ddots & \vdots        \\
  x_p & x_py_0 & x_py_1 & x_py_2 & x_py_3 & \cdots & x_py_q
\end{array}\]
\begin{enumerate}
\item The expression $\sum_{n=0}^p \sum_{m=0}^q x_n y_m z^{n+m}$
      is equivalent to adding {\em horizontally} 
      from left to right, from the first row to the last.

\item If we switched the order of summation to 
      $\sum_{m=0}^q \sum_{n=0}^p x_n y_m z^{n+m}$,
      then it would be equivalent to adding {\em vertically} 
      from top to bottom, 
      from the first column to the last.

\item However the final result expression
      $\sum_{n=0}^{p+q} \left(\sum_{k=0}^n x_k y_{n-k} \right)  $
      is equivalent to adding {\em diagonally} 
      starting from the upper left corner and proceding 
      to the lower right.

\item Upper limit on inner summation:
      Looking at the $x_k$ terms, we see that there are two constraints
      on $k$:
  \[\left.\begin{array}{lll}
    k &\le& n  \\
    k &\le& p
  \end{array}\right\}
  \implies
  k\le\min(n,p)\]

\item Lower limit on inner summation:
      Looking at the $x_k$ terms, we see that there are two constraints
      on $k$:
  \[\left.\begin{array}{lll}
    k &\ge& 0  \\
    k &\ge& n-q
  \end{array}\right\}
  \implies
  k\ge\max(0,n-q)\]
\end{enumerate}
\end{enumerate}
\end{proof}



%--------------------------------------
\begin{corollary}
\label{cor:seq_mult}
\index{Cauchy product}
%--------------------------------------
Let $\tuplen{x_n\in\C}$ and $\tuplen{y_n\in\C}$.
\corbox{
  \left(\sum_{n=0}^\infty x_n  \right)
  \left(\sum_{m=0}^\infty y_m  \right)
  =
  \sum_{n=0}^\infty 
  \mcom{\ds\brp{\sum_{k=0}^n x_k y_{n-k}}}
       {Cauchy product} 
  }
\end{corollary}
\begin{proof}
\begin{align*}
  \left(\sum_{n=0}^{p=\infty} x_n  \right)
  \left(\sum_{m=0}^{q=\infty} y_m  \right)
    &= \sum_{n=0}^{\infty} 
       \left(\sum_{k=\max(0,n-\infty)}^{\min(n,\infty)} x_k y_{n-k} \right)
    && \text{by \prefp{thm:seq_mult}}
  \\&= \sum_{n=0}^{\infty} \brp{(\sum_{k=0}^n x_k y_{n-k}}
\end{align*}
\end{proof}


%---------------------------------------
\begin{theorem}
\footnote{
  \citerppc{hardy1949}{227}{228}{\scshape Theorem 160},
  \citerpc{bromwich1908}{66}{Article 27.},
  \citorppc{cauchy1821}{147}{148}{$6.^e$ Th/'eor\`eme}
  %\citerpgc{kadets1997}{5}{3764354011}{\scshape Theorem 1.1.1 (Riemann's theorem)}\\
  }
%---------------------------------------
Let $\ds X\eqd\sum_{n=0}^\infty x_n$,
    $\ds Y\eqd\sum_{n=0}^\infty y_n$, and
    $\ds Z\eqd\brp{\sum_{n=0}^\infty\sum_{k=0}^n x_k y_{n-k}}$.
\thmbox{
  \brb{\begin{array}{MD}
    $X$ is \prope{absolutely convergent} & and \\
    $Y$ is \prope{absolutely convergent} & 
  \end{array}}
  \implies
  \brb{\begin{array}{MD}
    $Z$ is \prope{absolutely convergent} & and \\
    $\ds Z=XY$.
  \end{array}}
  }
\end{theorem}

%---------------------------------------
\begin{theorem}
\footnote{
  \citerpc{hardy1949}{228}{\scshape Theorem 161},
  \citerppc{bromwich1908}{85}{86}{Article 35.},
  \citor{mertens1875}
  %http://content.ebscohost.com/pdf23_24/pdf/2010/HXX/01Feb10/47696609.pdf?T=P&P=AN&K=47696609&S=R&D=aph&EbscoContent=dGJyMNXb4kSep7c40dvuOLCmr0qeqLFSsaq4S6%2BWxWXS&ContentCustomer=dGJyMPGssk2xqLJNuePfgeyx44Hy
  }
%---------------------------------------
Let $\ds X\eqd\sum_{n=0}^\infty x_n$,
    $\ds Y\eqd\sum_{n=0}^\infty y_n$, and
    $\ds Z\eqd\brp{\sum_{n=0}^\infty\sum_{k=0}^n x_k y_{n-k}}$.
\thmbox{
  \brb{\begin{array}{FMD}
    1. & $X$ is \prope{absolutely convergent} & and\\
    2. & $Y$ is \prope{convergent} & 
  \end{array}}
  \implies
  \brb{\begin{array}{FMD}
    1. & $Z$ is \prope{convergent} & and \\
    2. & $Z=XY$
  \end{array}}
  }
\end{theorem}

%---------------------------------------
\begin{theorem}
\footnote{
  \citerpc{hardy1949}{228}{\scshape Theorem 162},
  \citor{abel1826}
  }
%---------------------------------------
Let $\ds X\eqd\sum_{n=0}^\infty x_n$,
    $\ds Y\eqd\sum_{n=0}^\infty y_n$, and
    $\ds Z\eqd\brp{\sum_{n=0}^\infty\sum_{k=0}^n x_k y_{n-k}}$.
\thmbox{
  \brb{\begin{array}{FMD}
    1. & $X$ is \prope{convergent} & and\\
    2. & $Y$ is \prope{convergent} & and\\
    3. & $Z$ is \prope{convergent} & 
  \end{array}}
  \implies
  \brb{Z=XY}
  }
\end{theorem}

%=======================================
\section{Summability}
%=======================================
Cauchy and Abel, the 19th century champions of rigour in analysis, firmly rejected any and all divergent sums.
However in more recent times, certain classes of divergent sums have been found to be extremely useful.
Often such sums are ones that are said to be \hie{summable}. %, and such a sum is said to have the property of \hie{summability}.
%The property of \prope{summability} is a kind of generalization of \prope{convergence}
%in the sense that all convergent sums are \prope{summable}, but not all summable sums are \prope{convergent}.

%---------------------------------------
\begin{definition}
\footnote{
  \citerppg{zygmund1968v1}{75}{76}{0521890535},
  \citerpc{hardy1949}{96}{5.4 Ces/`aro means},
  \citerppc{whittaker1920}{155}{156}{8.43, 8.431},
  %\citerpgc{thomson2008}{129}{143484367X}{Definition 3.54}\\
  \citor{cesaro1890}
  }
\label{def:cesaro}
%---------------------------------------
\defbox{\begin{array}{M}
  The series $\ds \sum_{n=0}^\infty x_n$  is \hid{summable by the $k$-th arithmetic mean of Ces/`aro to limit $x$},\\
  or \hid{summable $\opair{C}{k}$ to the limit $x$}, if
  \\\qquad
  $\lim_{n\to\infty} \frac{\ds S^k_n}{\ds A^k_n} = x$ \quad for $n\in\Znn$ and where
  \\
  $\begin{array}{cMc}
    S^k_n \eqd \brbl{\begin{array}{>{\ds}lM}
                         \sum_{m=0}^n x_m       & for $k=0$\\
                         \sum_{m=0}^n S^{k-1}_m & for $k=1,2,3,\ldots$
                       \end{array}}
  &\quad and \quad&
    A^k_n \eqd \brbl{\begin{array}{>{\ds}lM}
                      1                      & for $k=0$\\
                      \sum_{m=0}^n A^{k-1}_m & for $k=1,2,3,\ldots$
                    \end{array}}
  \end{array}$
\end{array}}
\end{definition}

%---------------------------------------
\begin{proposition}
\footnote{
  \citerppg{zygmund1968v1}{75}{76}{0521890535},
  \citerpgc{thomson2008}{129}{143484367X}{Definition 3.54},
  \citerp{szasz1952}{13}%{``method of arithmetic means"}
  }
\label{prop:cesaro}
%---------------------------------------
\propbox{\begin{array}{>{\ds}lM>{\ds}lD}
  \sum_{n=0}^\infty x_n&  is summable $\opair{C}{0}$ to the limit $x$ if 
    & \lim_{\xN\to\infty}\sum_{n=0}^\xN x_n = x
    & (normal convergence)
  \\
  \sum_{n=0}^\infty x_n&  is summable $\opair{C}{1}$ to the limit $x$ if 
    & \lim_{\xN\to\infty}\frac{1}{\xN+1}\sum_{n=0}^\xN s_n = x
    & (arthimetic mean)
  \\\mc{4}{M}{\qquad where $\ds s_n \eqd \sum_{m=0}^n x_m$}
\end{array}}
\end{proposition}

%---------------------------------------
\begin{definition}
\footnote{
  \citerpc{whittaker1920}{155}{8.42}
  }
\label{def:summable_Euler}
%---------------------------------------
\defbox{\begin{array}{M}
  The series $\ds \sum_{n=0}^\infty a_n$  is \hid{summable by Euler's method to limit $a$} if
  \\\qquad
  $\ds\lim_{x\to1-0} \sum_{n=0}^\infty a_n x^n  = a$ 
\end{array}}
\end{definition}

%---------------------------------------
\begin{example}
\footnote{
  \citerpgc{thomson2008}{130}{143484367X}{Example 3.56},
  \citerpc{whittaker1920}{155}{8.42}
  }
\label{ex:cesaro_pm1}
%---------------------------------------
\exbox{\begin{array}{M}
  The series $\quad\ds\sum_{n=0}^\infty (-1)^n = 1-1+1-1+1-1+\cdots\quad$ is \prope{divergent} \\
  (it is \emph{not} summable $\opair{C}{0}$),\\
  but yet it \emph{is} summable $\opair{C}{1}$  to the limit $\frac{1}{2}$.\\
  It is also summable by Euler's method to the limit $\frac{1}{2}$.
\end{array}}
\end{example}
\begin{proof}
\begin{enumerate}
  \item Proof for Ces/`aro summability:
    \begin{enumerate}
      \item Note that the sequence of partial sums $s_n$ is $s_0=1$, $s_1=0$, $s_2=1$, $s_3=0$, $s_4=1$, \ldots. That is
        \\\qquad $\ds s_n = \brbl{\begin{array}{lM}
                              1 & for $n$ even\\
                              0 & for $n$ odd
                            \end{array}}$
      \item Then
         \begin{align*}
          \lim_{n\to\infty} \frac{1}{n+1}\sum_{k=0}^n s_k
            &= \lim_{n\to\infty} \frac{1}{2n+1}\sum_{k=0}^{2n} s_{k}
          \\&= \lim_{n\to\infty} \frac{1}{2n+1}\brp{\mcom{\ds\sum_{k=0}^{n} s_{2k}}{even terms} + \mcom{\ds\sum_{k=0}^{n-1} s_{2k+1}}{odd terms}}
          \\&= \lim_{n\to\infty} \frac{1}{2n+1}\brp{\sum_{k=0}^{n} 1 + \sum_{k=0}^{n-1} 0 }
          \\&= \lim_{n\to\infty} \frac{n+1}{2n+1}
          \\&= \frac{1}{2}
         \end{align*}
    \end{enumerate}

  \item Proof for Euler summability:
    \begin{align*}
      \lim_{x\to1-0}\sum_{n=0}^\infty (-1)^n
        &= \lim_{x\to1-0} \lim_{n\to\infty}\brp{\frac{1}{1+x}=\sum_{k=0}^{2n-1} (-1)^k x^k + \frac{x^{2n}}{1+x}}
      \\&= \lim_{x\to1-0} \frac{1}{1+x}
        && \text{by \prefpp{item:logseries_xk} of \pref{ex:logseries}}
      \\&= \frac{1}{2}
    \end{align*}
\end{enumerate}
\end{proof}

%=======================================
\section{Convergence in Banach spaces}
%=======================================
The properties of \prope{strong convergence} and \prope{weak convergence} are defined on \fncte{sequence}s\ifsxref{seq}{def:strong_converge}.
An infinite sum $\sum_{n=1}^\infty \vx_n$ in a Banach space is the limit of a sequence of partial sums $\seqn{\sum_{n=1}^\xN\vx_n}$,
so the properties of strong and weak convergence apply to infinite sums as well.
\pref{def:eqs} (next) assigns special equality symbols for these sums.
%---------------------------------------
\begin{definition}
\label{def:eqs}
\label{def:eqw}
%---------------------------------------
Let $\spB\eqd\normspaceX$ be a Banach space.
\defbox{\begin{array}{M}
  The expression $\ds\vx\hxs{\eqs}\sum_{n=1}^\infty\vx_n$ denotes that the sum \hid{converges strongly} to $\vx$.\\
  The expression $\ds\vx\hxs{\eqw}\sum_{n=1}^\infty\vx_n$ denotes that the sum \hid{converges weakly} to $\vx$.
\end{array}}
\end{definition}

%=======================================
\section{Convergence tests for real sequences}
%=======================================
%---------------------------------------
\begin{theorem}[\thme{comparison test}]
\footnote{
  \citerpgc{bonar2006}{26}{0883857456}{Theorem 1.53 (Limit Comparison Test Strengthened)},
  \citerpgc{heinbockel2010}{152}{1426949545}{Comparison Tests}
  }
\label{thm:series_comparison}
%---------------------------------------
\thmbox{
  \brb{\begin{array}{FMCD}
    1. & $\sum_{n=1}^\infty\seqn{y_n}$ \prope{converges} & & and\\
    2. & $x_n\le y_n$ & \forall n\in\Zp
  \end{array}}
  \implies
  \text{$\sum_{n=1}^\infty\seqn{x_n}$ \prope{converges}}
  }
\end{theorem}

