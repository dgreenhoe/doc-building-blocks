%============================================================================
% NCTU - Hsinchu, Taiwan
% LaTe\rvX File
% Daniel Greenhoe
%============================================================================

%======================================
\chapter{Information Theory}
\label{chp:capacity}
\index{information theory}
%======================================
\begin{figure}[ht]
\color{figcolor}
\begin{center}
\begin{fsK}
\setlength{\unitlength}{0.17mm}                  
\begin{picture}(900,150)(-100,-50)
  \thicklines                                      
  %\graphpaper[10](0,0)(700,100)                  
  \put(-100, 125 ){\vector(0,-1){40}}
  \put(-100, 130 ){\makebox(0,0)[bl]{$R_d$ (data rate)}}
  \put(-100,  60 ){\makebox(30,0)[br]{$\{u_n\}$} }
  \put(-100,  50 ){\vector(1,0){50} }

  \put(-70, -50 ){\dashbox{4}( 280,160){} }
  \put(-70, -40 ){\makebox( 280,160)[b]{transmitter} }
  \put(   50, 130 ){\makebox(0,0)[bl]{$R_c$ (signal rate)}}
  \put(   75, 125 ){\vector(0,-1){40}}
  \put(   50,  60 ){\makebox(50,50)[b]{$\{y_n\}$} }
  \put(   50,  50 ){\vector(1,0){50} }

  \put(- 50,  00 ){\framebox( 100,100){} }
  \put(- 50,  10 ){\makebox( 100,80)[t]{channel} }
  \put(- 50,  10 ){\makebox( 100,80)[ ]{coder} }
  \put(- 50,  10 ){\makebox( 100,80)[b]{$(L,N)$} }
  \put( 100,  00 ){\framebox( 100,100){modulator} }
  \put( 200,  50 ){\vector(1,0){100} }

  \put( 350, 125 ){\vector(0,-1){25}}
  \put( 300, 130 ){\makebox(0,0)[bl]{$\frac{L}{N}=\frac{R_d}{R_c}\eqd R<C$ (channel capacity)}}
  \put( 300,  00 ){\framebox( 100,100){} }
  \put( 300,  30 ){\makebox( 100, 40)[t]{channel} }
  \put( 300,  10 ){\makebox( 100, 40)[b]{$\pp(\rvY|\rvX)$} }
  \put( 210,  60 ){\makebox( 90, 50)[b]{$s(t)$} }
  \put( 400,  60 ){\makebox( 80, 50)[b]{$r(t)$} }

  \put( 400,  50 ){\vector(1,0){100} }
  \put( 500,  00 ){\framebox(100,100){demodulator} }
  \put( 600,  60 ){\makebox(50,50)[b]{$\{\hat{y}_n\}$} }
  \put( 600,  50 ){\vector(1,0){50}}
  \put( 650,  00 ){\framebox(100,100){} }
  \put( 650,  30 ){\makebox(100,40)[t]{channel} }
  \put( 650,  30 ){\makebox(100,40)[b]{decoder} }
  \put( 480, -50 ){\dashbox{4}( 280,160){} }
  \put( 480, -40 ){\makebox( 280,160)[b]{receiver} }

  \put( 760,  60 ){\makebox(40,50)[b]{$\{\ue_n\}$} }
  \put( 750,  50 ){\vector(1,0){50}}
\end{picture}                                   
\end{fsK}
\end{center}
\caption{
   Memoryless modulation system model
   \label{fig:i_mod_model}
   }
\end{figure}




%=======================================
\section{Information Theory}
%=======================================
%=======================================
\subsection{Definitions}
%=======================================
The \fncte{Kullback Leibler distance} $\iD{\pp_1}{\pp_2}$ \xref{def:kld} is a 
measure between two probability density functions $\pp_1$ and $\pp_2$.
It is not a true distance measure\footnote{
  {\em Distance measure}: \prefpp{def:metric}
  }
but it behaves in a similar manner.
If $\pp_1=\pp_2$, then the \fncte{KL distance} is 0.
If $\pp_1$ is very different from $\pp_2$, then $|\iD{\pp_1}{\pp_2}|$ 
will be much larger.

%--------------------------------------
\begin{definition}
\citetbl{
  \citerp{cover}{18}
  }
\label{def:kld}
\index{Kullback Leibler distance}
\index{relative entropy}
%--------------------------------------
Let $\pp_1$ and $\pp_2$ be probability density functions.
Then the {\bf Kullback Leibler distance}
(the \fncte{KL distance}, also called the {\bf relative entropy})
of $\pp_1$ and $\pp_2$ is
\defbox{
  \iD{\pp_1}{\pp_2} \eqd \pE \log_2 \frac{ \pp_1(\rvX) }{\pp_2(\rvX)} 
  \hspace{3ex}\mbox{bits}
  }
If the base of logarithm is $e$ (the ``natural logarithm") rather than $2$,
then the units are {\em nats} rather than {\em bits}.
\end{definition}

The \fncte{mutual information} $\iI(\rvX;\rvY)$ of random variable $\rvX$ and $\rvY$ is
the \fncte{KL distance} between their \fncte{joint distribution} $\pp(\rvX,\rvY)$ and the 
product of their \fncte{marginal distribution}s $\pp(\rvX)$ and $\pp(\rvY)$.
If $\rvX$ and $\rvY$ are independent, then the \fncte{KL distance} between 
joint and marginal product is $\log1=0$ and they have no 
\fncte{mutual information} ($\iI(\rvX;\rvY)=0$).
If $\rvX$ and $\rvY$ are highly correlated, then the \fncte{joint distribution} is
much different than the product of the marginals making the \fncte{KL distance}
greater and along with it the \fncte{mutual information} greater as well.
%--------------------------------------
\begin{definition}[Mutual information]
\footnote{
  \citerpp{cover}{18}{19}
  }
\label{def:I(X;Y)}
\index{information}
\index{information!mutual information}
%--------------------------------------
\defbox{
  \iI(\rvX;\rvY) \eqd \iD{\pp(\rvX,\rvY)}{\pp(\rvX)\pp(\rvY)} 
           \eqd \pExy \log_2 \frac{ \pp(\rvX,\rvY) }{\pp(\rvX)\pp(\rvY)} 
                \hspace{3ex}\mbox{bits}
  }
\end{definition}

The {\em self information} $\iI(\rvX;\rvX)$ of random variable $\rvX$ is the 
\fncte{mutual information} between $\rvX$ and itself.
That is, it is a measure of the information contained in $\rvX$.
Self information $\iI(\rvX;\rvX)$ can also be viewed as the \fncte{KL distance} between
the constant $1$ (no information because $1$ is completely known)
and $\pp(\rvX)$.
%--------------------------------------
\begin{definition}[Self information]
\footnote{
  \citerpp{cover}{18}{19}
  }
\label{def:I(X;X)}
\index{information!self information}
%--------------------------------------
\defbox{
  \iI(\rvX;\rvX) \eqd \iD{1}{\pp(\rvX)} 
           \eqd \pEx \log_2 \frac{1}{\pp(\rvX)} 
                \hspace{3ex}\mbox{bits}
  }
\end{definition}

The \hie{entropy} $\iH(\rvX)$ of a random variable $\rvX$ is equivalent to
the self information $\iI(\rvX;\rvX)$ of $\rvX$.
That is, the entropy of $\rvX$ is a measure of the information contained
in $\rvX$.

Likewise, the {\em conditional entropy} $\iH(\rvX|\rvY)$ 
of $\rvX$ given $\rvY$ is the information
contained in $\rvX$ given $\rvY$ has occurred. 
If $\rvX$ and $\rvY$ are independent, then $\rvX$ does not care about the occurrence of
$\rvY$. Thus in this case, 
the occurrence of $\rvY=y$ does not change the amount of information
provided by $\rvX$ and $\iH(\rvX|\rvY)=\iH(\rvX)$.
If $\rvX$ and $\rvY$ are highly correlated, 
the occurrence of $\rvY=y$ tells us a lot about what the value of $\rvX$ might
turn out to be.
Thus in this case, the information provided by $\rvX$ given $\rvY$ is greatly reduced
and $\iH(\rvX|\rvY)<<\iH(\rvX)$.

The {\em joint entropy} $\iH(\rvX,\rvY)$ of $\rvX$ and $\rvY$ is the amount of information 
contained in the ordered pair $(\rvX,\rvY)$.

%--------------------------------------
\begin{definition}[Entropy]
\footnote{
  \citerpp{cover}{15}{17}
  }
\label{def:H(X)}
\label{def:H(XY)}
\index{entropy}
\index{entropy!joint entropy}
\index{entropy!conditional entropy}
%--------------------------------------
\defbox{
\begin{array}{l@{\hspace{1cm}}rcl@{\hspace{1cm}}l}
  \mbox{entropy of $\rvX$}                       : & \iH(\rvX)        &\eqd& \pEx  \log_2 \frac{1}{\pp(\rvX)  }  & \mbox{bits} \\
  \mbox{joint entropy of $\rvX,\rvY$}               : & \iH(\rvX,\rvY)      &\eqd& \pExy \log_2 \frac{1}{\pp(\rvX,\rvY)}  & \mbox{bits} \\
  \mbox{conditional entropy of $\rvX$ given $\rvY$} : & \iH(\rvX|\rvY)      &\eqd& \pExy \log_2 \frac{1}{\pp(\rvX|\rvY)}  & \mbox{bits}   
\end{array}
}
\end{definition}


%=======================================
\subsection{Relations}
%=======================================
\begin{figure}[ht]
\begin{center}\begin{footnotesize}
\setlength{\unitlength}{0.4mm}
\begin{picture}(150,180)(0,0)
  {\color[rgb]{0,0,1}
  \put(  50,  50){\oval(100,100)}
  \put(  50, 105){\makebox(0,0)[b]{$\iH(\rvX)$}}
  \put(  25,  50){\makebox(0,0){$\iH(\rvX|\rvY)$}}
  }
  {\color[rgb]{1,0,0}
  \put( 100,  50){\oval(100,100)}
  \put( 100, 105){\makebox(0,0)[b]{$\iH(\rvY)$}}
  \put( 125,  50){\makebox(0,0){$\iH(\rvY|\rvX)$}}
  }
  {\color[rgb]{0.5,0,0.5}
  \put(  75, 120){\makebox(0,0)[b]{$\iH(\rvX,\rvY)$}}
  \put(  75,  50){\makebox(0,0)[b]{$\iI(\rvX;\rvY)$}}
  }
\end{picture}
\end{footnotesize}\end{center}
\caption{
  Relationship between information and entropy
  \label{fig:HI}
  }
\end{figure}

%--------------------------------------
\begin{theorem}
%--------------------------------------
\thmbox{ \iH(\rvX,\rvY) = \iH(\rvY,\rvX)  }
\end{theorem}
\begin{proof}
\begin{eqnarray*}
  \iH(\rvX,\rvY)
    &\eqd& \pExy \log \frac{1}{\pp_{xy}(\rvX,\rvY)}
  \\&=&    \pEyx \log \frac{1}{\pp_{yx}(\rvY,\rvX)}
  \\&\eqd& \iH(\rvY,\rvX)
\end{eqnarray*}
\end{proof}

%--------------------------------------
\begin{theorem}[Entropy chain rule]
\label{thm:chain}
\index{chain rule!entropy}
\index{Entropy chain rule}
\index{theorems!Entropy chain rule}
%--------------------------------------
\thmbox{
  \begin{array}{rcl}
    \iH(\rvX,\rvY) &=& \ds \iH(\rvX|\rvY) + \iH(\rvY)  \\
             &=& \ds \iH(\rvY|\rvX) + \iH(\rvX). \\
    \iH(\rvX_1,\rvX_2,\ldots,\rvX_N) &=& \ds \sum_{n=1}^{N-1}\iH(\rvX_n|\rvX_{n+1},\ldots,\rvX_N) + \iH(\rvX_N)
  \end{array}
}
\end{theorem}
\begin{proof}
\begin{eqnarray*}
  \iH(\rvX,\rvY)
    &\eqd& \pExy \log \frac{1}{\pp(\rvX,\rvY)}
  \\&=&    \pExy \log \frac{1}{\pp(\rvX|\rvY)\pp(\rvY)}
  \\&=&    \pExy \log \frac{1}{\pp(\rvX|\rvY)} + \pExy \log \frac{1}{\pp(\rvY)}
  \\&=&    \pExy \log \frac{1}{\pp(\rvX|\rvY)} + \pEy \log \frac{1}{\pp(\rvY)}
  \\&=&    \iH(\rvX|\rvY) + \iH(\rvY)
\\
\\
  \iH(\rvX,\rvY)
    &\eqd& \pExy \log \frac{1}{\pp(\rvX,\rvY)}
  \\&=&    \pExy \log \frac{1}{\pp(\rvY|\rvX)\pp(\rvX)}
  \\&=&    \pExy \log \frac{1}{\pp(\rvY|\rvX)} + \pExy \log \frac{1}{\pp(\rvX)}
  \\&=&    \pExy \log \frac{1}{\pp(\rvY|\rvX)} + \pEy \log \frac{1}{\pp(\rvX)}
  \\&=&    \iH(\rvY|\rvX) + \iH(\rvX)
\\
\\
  \iH(\rvX_1,\rvX_2,\ldots,\rvX_N) 
    &=& \iH(\rvX_1|\rvX_2,\ldots,\rvX_N) + \iH(\rvX_2,\ldots,\rvX_N)
  \\&=& \iH(\rvX_1|\rvX_2,\ldots,\rvX_N) + \iH(\rvX_2|\rvX_3,\ldots,\rvX_N) + \iH(\rvX_3,\ldots,\rvX_N)
  \\&=& \iH(\rvX_1|\rvX_2,\ldots,\rvX_N) + \iH(\rvX_2|\rvX_3,\ldots,\rvX_N) + \iH(\rvX_3|\rvX_4,\ldots,\rvX_N) + \iH(\rvX_4,\ldots,\rvX_N)
  \\&=& \sum_{n=1}^{N-1}\iH(\rvX_n|\rvX_{n+1},\ldots,\rvX_n) + \iH(\rvX_N)
\end{eqnarray*}
\end{proof}


%--------------------------------------
\begin{theorem}
%--------------------------------------
\thmbox{
  \begin{array}{rcl}
    \iI(\rvX;\rvY) &=& \iH(\rvX) - \iH(\rvX|\rvY)           \\
    \iI(\rvX;\rvY) &=& \iH(\rvY) - \iH(\rvY|\rvX)           \\
    \iI(\rvX;\rvY) &=& \iH(\rvX) + \iH(\rvY) - \iH(\rvX,\rvY)  \\
    \iI(\rvX;\rvY) &=& \iI(\rvY;\rvX)                    \\
    \iI(\rvX;\rvX) &=& \iH(\rvX)
  \end{array}
}
\end{theorem}
\begin{proof}
\begin{eqnarray*}
  \iI(\rvX;\rvY)
    &\eqd& \pExy \log_2 \frac{ \pp(\rvX,\rvY) }{\pp(\rvX)\pp(\rvY)}
  \\&=&    \pExy \log_2 \frac{ \pp(\rvX|\rvY) }{\pp(\rvX)}
  \\&=&    \pExy \log_2 \frac{ 1 }{\pp(\rvX)} + \pExy \log_2 \pp(\rvX|\rvY) 
  \\&=&    \pExy \log_2 \frac{ 1 }{\pp(\rvX)} - \pExy \log_2 \frac{1}{\pp(\rvX|\rvY)}
  \\&\eqd& \iH(\rvX) - \iH(\rvX|\rvY)
\\
\\
  \iI(\rvX;\rvY)
    &\eqd& \pExy \log_2 \frac{ \pp(\rvX,\rvY) }{\pp(\rvX)\pp(\rvY)}
  \\&=&    \pExy \log_2 \frac{ \pp(\rvY|\rvX) }{\pp(\rvY)}
  \\&=&    \pExy \log_2 \frac{ 1 }{\pp(\rvY)} + \pExy \log_2 \pp(\rvY|\rvX) 
  \\&=&    \pExy \log_2 \frac{ 1 }{\pp(\rvY)} - \pExy \log_2 \frac{1}{\pp(\rvY|\rvX)}
  \\&\eqd& \iH(\rvY) - \iH(\rvY|\rvX)
\\
\\
  \iI(\rvX;\rvY)
    &=&    \iH(\rvY) - \iH(\rvY|\rvX)
  \\&=&    \iI(\rvY;\rvX)
\\
\\
  \iI(\rvX;\rvX)
    &\eqd& \pExy \log_2 \frac{ \pp(\rvX,\rvX) }{\pp(\rvX)\pp(\rvX)}
  \\&=&    \pExy \log_2 \frac{ \pp(\rvX)   }{\pp(\rvX)\pp(\rvX)}
  \\&=&    \pExy \log_2 \frac{ 1        }{\pp(\rvX)      }
  \\&\eqd& \iH(\rvX)
\\
\\
  \iI(\rvX;\rvY)
    &\eqd& \iH(\rvX) - \iH(\rvX|\rvY)
  \\&=&    \iH(\rvX) - [ \iH(\rvX,\rvY) - \iH(\rvY) ]
  \\&=&    \iH(\rvX) + \iH(\rvY) - \iH(\rvX,\rvY) 
\end{eqnarray*}
\end{proof}

%--------------------------------------
\begin{theorem}[Information chain rule]
\index{chain rule!information}
\index{information chain rule}
\index{theorems!information chain rule}
%--------------------------------------
\thmbox{
  \iI(\rvX_1,\rvX_2,\ldots,\rvX_N;\rvY)
    = \sum_{n=1}^{N-1}\iI(\rvX_n|\rvX_{n+1},\ldots,\rvX_N) 
        + \iI(\rvX_N) 
  }
\end{theorem}
\begin{proof}
\begin{eqnarray*}
  \iI(\rvX_1,\rvX_2,\ldots,\rvX_N;\rvY)
    &=& \iH(\rvX_1,\rvX_2,\ldots,\rvX_N) - \iH(\rvX_1,\rvX_2,\ldots,\rvX_N|\rvY)
  \\&=& \sum_{n=1}^{N-1}\iH(\rvX_n|\rvX_{n+1},\ldots,\rvX_N) + \iH(\rvX_N) 
        - \sum_{n=1}^{N-1}\iH(\rvX_n|\rvX_{n+1},\ldots,\rvX_N,\rvY) - \iH(\rvX_N|\rvY) 
  \\&=& \sum_{n=1}^{N-1}\left[ 
        \iH(\rvX_n|\rvX_{n+1},\ldots,\rvX_N) - \iH(\rvX_n|\rvX_{n+1},\ldots,\rvX_N,\rvY) 
        \right]
        + \left[ \iH(\rvX_N) - \iH(\rvX_N|\rvY) \right]
  \\&=& \sum_{n=1}^{N-1}\iI(\rvX_n|\rvX_{n+1},\ldots,\rvX_N) 
        + \iI(\rvX_N) 
\end{eqnarray*}
\end{proof}

%=======================================
\subsection{Properties}
%=======================================
%---------------------------------------
\begin{theorem}
\citetbl{
  \citerp{cover}{26}
  }
%---------------------------------------
\thmbox{
\begin{array}{rcl}
  \iD{\pp_1}{\pp_2} &\ge& 0  \\
  \iI(\rvX;\rvY)          &\ge& 0
\end{array}
}
\end{theorem}
\begin{proof}
\begin{eqnarray*}
  \iD{\pp_1}{\pp_2}
    &\eqd& \pEx\log\frac{\pp_1(\rvX)}{\pp_2(\rvX)}
  \\&=&    \pEx\left[-\log\frac{\pp_2(\rvX)}{\pp_1(\rvX)} \right]
  \\&\ge&  -\log\pEx\left[\frac{\pp_2(\rvX)}{\pp_1(\rvX)} \right]
    \hspace{1cm}\mbox{by \thme{Jensen's Inequality} \xref{thm:jensen}}
  \\&=&    -\log\int_x \pp_1(x)\frac{\pp_2(x)}{\pp_1(x)} \dx
  \\&=&    -\log\int_x \pp_2(x) \dx
  \\&=&    -\log(1)
  \\&=&    0
\end{eqnarray*}
\end{proof}
%=======================================
\section{Channel Capacity}
%=======================================
%--------------------------------------
\begin{definition}
\label{def:iC}
%--------------------------------------
Let $(L,N)$ be a block coder with $N$ output bits for each $L$ input bits.
\[
\begin{array}{rcll}
  R   &\eqd& \frac{L}{N}    & \mbox{coding rate}      \\
  \iC &\eqd& \max \iI(\rvX;\rvY)  & \mbox{channel capacity} \\
  \iE(R) &\eqd& \max_\rho \max_Q [\iE_0(\rho,Q)-\rho R ]            & \mbox{random coding exponent}
\end{array}
\]
\end{definition}

%--------------------------------------
\begin{theorem}[noisy channel coding theorem]
\citetbl{
  \citerp{gallager}{143}
  }
\label{thm:ncct}
\index{Noisy channel coding theorem}
\index{theorems!Noisy channel coding theorem}
%--------------------------------------
If
\formbox{ R < \iC}
then it is possible to construct an encoder and decoder such that 
the probability of error $P_e$ is arbitrarily small. Specifically
\formbox{ P_e \le e^{-N\iE(R)}  }

For $0\le R\ge\iC$, the function $\iE(R)$ is
\begin{enume}
  \item positive
  \item decreasing
  \item convex
\end{enume}
\end{theorem}





\begin{figure}[ht]
\color{figcolor}
\setlength{\unitlength}{0.2mm}
\begin{center}
\begin{picture}(200,100)(-50,0)
  \put(  0,  0){\line(1, 0){120}}
  \put(  0,  0){\line(0, 1){120}}
  %\put(125,  0){\makebox(0,0)[l]{$e$}}
  \put(100, -5){\makebox(0,0)[t]{$\iC$}}
  %\put( -5,100){\makebox(0,0)[r]{$\iC$}}
  \qbezier(0,100)(20,20)(100,0)
  \qbezier[32](0,100)(50,100)(100,100)
  \qbezier[32](100,0)(100,50)(100,100)
  \put( 50,90){\makebox(0,0)[t]{$\iE(R)$}}
\end{picture}
\end{center}
\caption{
  Typical $\iE(R)$
  \label{fig:E(R)}
  }
\end{figure}


\begin{figure}[ht]
\color{figcolor}
\setlength{\unitlength}{0.2mm}
\begin{center}
\begin{tabular}{cccc}
\begin{picture}(200,100)(-50,0)
  \put(  0,100){\vector(1, 0){100}}
  \put(  0,  0){\vector(1, 0){100}}
  \put(  0,100){\vector(1,-1){100}}
  \put(  0,  0){\vector(1, 1){100}}
  \put(-30, 50){\makebox(0,0)[r]{$\rvX$}}
  \put(130, 50){\makebox(0,0)[l]{$\rvY$}}
  \put(-05,100){\makebox(0,0)[r]{$0$}}
  \put(-05,  0){\makebox(0,0)[r]{$1$}}
  \put(105,100){\makebox(0,0)[l]{$0$}}
  \put(105,  0){\makebox(0,0)[l]{$1$}}
  \put( 50,105){\makebox(0,0)[b]{$1-\epsilon$}}
  \put( 50, -5){\makebox(0,0)[t]{$1-\epsilon$}}
  \put( 27, 77){\makebox(0,0)[bl]{$\epsilon$}}
  \put( 23, 27){\makebox(0,0)[br]{$\epsilon$}}
\end{picture}
&
\begin{picture}(200,100)(-50,0)
  \put(  0,  0){\line(1, 0){120}}
  \put(  0,  0){\line(0, 1){120}}
  \put(125,  0){\makebox(0,0)[l]{$p$}}
  \put( -5,100){\makebox(0,0)[r]{$1$}}
  \put(100, -5){\makebox(0,0)[t]{$1$}}
  \put( 50, -5){\makebox(0,0)[t]{$\frac{1}{2}$}}
  \qbezier[32](50,0)(50,50)(50,100)
  \qbezier(0,0)(50,200)(100,0)
  \qbezier[16](0,100)(25,100)(50,100)
  \put( 80,80){\makebox(0,0)[l]{$\iH(\rvX)$}}
\end{picture}
&
\begin{picture}(200,100)(-50,0)
  \put(  0,  0){\line(1, 0){120}}
  \put(  0,  0){\line(0, 1){120}}
  \put(125,  0){\makebox(0,0)[l]{$e$}}
  \put(100, -5){\makebox(0,0)[t]{$1$}}
  \put( 50, -5){\makebox(0,0)[t]{$\frac{1}{2}$}}
  \put( -5,100){\makebox(0,0)[r]{$1$}}
  \qbezier[32](50,0)(50,50)(50,100)
  \qbezier(0,0)(50,200)(100,0)
  \qbezier[16](0,100)(25,100)(50,100)
  \put( 80,80){\makebox(0,0)[l]{$\iH(\rvY|\rvX)$}}
\end{picture}
&
\begin{picture}(200,100)(-50,0)
  \put(  0,  0){\line(1, 0){120}}
  \put(  0,  0){\line(0, 1){120}}
  \put(125,  0){\makebox(0,0)[l]{$\epsilon$}}
  \put(100, -5){\makebox(0,0)[t]{$1$}}
  \put( 50, -5){\makebox(0,0)[t]{$\frac{1}{2}$}}
  \put( -5,100){\makebox(0,0)[r]{$1$}}
  \qbezier(0,100)(50,-100)(100,100)
  \qbezier[32](0,100)(50,100)(100,100)
  \qbezier[32](100,0)(100,50)(100,100)
  \put( 50,90){\makebox(0,0)[t]{$\iI(\rvX;\rvY)$}}
\end{picture}
\end{tabular}
\end{center}
\caption{
  Binary symmetric channel (BSC)
  \label{fig:bsc}
  }
\end{figure}

%=======================================
\section{Specific channels}
%=======================================
%=======================================
\subsection{Binary Symmetric Channel (BSC)}
%=======================================
The properties of the {\em binary symmetric channel (BSC)} 
are illustrated in \prefpp{fig:bsc} and stated in 
\pref{thm:bsc} (next).
%--------------------------------------
\begin{theorem}[Binary symmetric channel]
\label{thm:bsc}
\index{Binary symmetric channel}
\index{theorems!Binary symmetric channel}
%--------------------------------------
Let $\opC:\rvX\to \rvY$ be a channel operation with $\rvX,\rvY\in\{0,1\}$ and
\begin{eqnarray*}
  p &\eqd& \pP{\rvX=1} \\
  \pP{\rvY=1|\rvX=0} &=& \pP{\rvY=0|\rvX=1} \eqd \epsilon
\end{eqnarray*}
Then
\thmbox{\begin{array}{rcl}
  \pP{\rvY=1} &=& \epsilon+p-2\epsilon p
\\
  \pP{\rvY=0} &=& 1-p-\epsilon +2\epsilon p
\\
  \iH(\rvX)    &=&    p     \log_2 \frac{1}{p} +
           (1-p) \log_2 \frac{1}{(1-p)}
\\
  \iH(\rvY)    &=&    (1-p-\epsilon +2\epsilon p) \log_2 \frac{1}{1-p-\epsilon +2\epsilon p} + 
           (\epsilon +p-2\epsilon p)   \log_2 \frac{1}{\epsilon +p-2\epsilon p}
\\
  \iH(\rvY|\rvX)  &=&    (1-\epsilon ) \log_2 \frac{1}{1-\epsilon } +
           \epsilon      \log_2 \frac{1}{\epsilon   }
\\
  \iI(\rvX;\rvY)  &=&    (1-p-\epsilon +2\epsilon p) \log_2 \frac{1}{1-p-\epsilon +2\epsilon p} + 
           (\epsilon +p-2\epsilon p)   \log_2 \frac{1}{\epsilon +p-2\epsilon p} \\&&
           - (1-\epsilon ) \log_2 \frac{1}{1-\epsilon } +
           - \epsilon      \log_2 \frac{1}{\epsilon   }
\\
  \iC       &=&    1  + \epsilon \log_2 \epsilon  + (1-\epsilon ) \log_2 (1-\epsilon )
\end{array}}
\end{theorem}
\begin{proof}
\begin{eqnarray*}
  \pP{\rvX=1} &\eqd& p    \\
  \pP{\rvX=0} &=& 1-p     \\
  \pP{\rvY=1}
    &=& \pP{\rvY=1|\rvX=0}\pP{\rvX=0} + \pP{\rvY=1|\rvX=1}\pP{\rvX=1}
  \\&=& \epsilon(1-p) + (1-\epsilon)p
  \\&=& \epsilon-\epsilon p + p-\epsilon p
  \\&=& \epsilon+p-2\epsilon p
\\
  \pP{\rvY=0} 
    &=& \pP{\rvY=0|\rvX=0}\pP{\rvX=0} + \pP{\rvY=0|\rvX=1}\pP{\rvX=1}
  \\&=& (1-\epsilon )(1-p) + \epsilon p
  \\&=& 1-p-\epsilon +\epsilon p+\epsilon p
  \\&=& 1-p-\epsilon +2\epsilon p
\\
\\
  \iH(\rvX)
    &\eqd& \pEx \log_2 \frac{1}{\pp(\rvX)}
  \\&=&    \sum_{n=0}^1 \pP{\rvX=n} \log_2 \frac{1}{\pP{\rvX=n}}
  \\&=&    \pP{\rvX=0} \log_2 \frac{1}{\pP{\rvX=0}} +
           \pP{\rvX=1} \log_2 \frac{1}{\pP{\rvX=1}}
  \\&=&    p     \log_2 \frac{1}{p} +
           (1-p) \log_2 \frac{1}{(1-p)}
\\
\\
  \iH(\rvY)
    &\eqd& \pEy \log_2 \frac{1}{\pp(\rvY)}
  \\&=&    \sum_{n=0}^1 \pP{\rvY=n} \log_2 \frac{1}{\pP{\rvY=n}}
  \\&=&    \pP{\rvY=0} \log_2 \frac{1}{\pP{\rvY=0}} + 
           \pP{\rvY=1} \log_2 \frac{1}{\pP{\rvY=1}}
  \\&=&    (1-p-\epsilon +2\epsilon p) \log_2 \frac{1}{1-p-\epsilon +2\epsilon p} + 
           (\epsilon +p-2\epsilon p)   \log_2 \frac{1}{\epsilon +p-2\epsilon p}
\\
\\
  \iH(\rvY|\rvX)
    &\eqd& \pExy \log_2 \frac{1}{\pp(\rvY|\rvX)}
  \\&=&    \sum_{m=0}^1\sum_{n=0}^1 \pP{\rvX=m,\rvY=n} \log_2 \frac{1}{\pP{\rvY=n|\rvX=m}}
  \\&=&    \sum_{m=0}^1\sum_{n=0}^1 \pP{\rvY=n|\rvX=m}\pP{\rvX=m} 
           \log_2 \frac{1}{\pP{\rvY=n|\rvX=m}}
  \\&=&    \pP{\rvY=0|\rvX=0}\pP{\rvX=0} \log_2 \frac{1}{\pP{\rvY=0|\rvX=0}} + \\&&
           \pP{\rvY=0|\rvX=1}\pP{\rvX=1} \log_2 \frac{1}{\pP{\rvY=0|\rvX=1}} + \\&&
           \pP{\rvY=1|\rvX=0}\pP{\rvX=0} \log_2 \frac{1}{\pP{\rvY=1|\rvX=0}} + \\&&
           \pP{\rvY=1|\rvX=1}\pP{\rvX=1} \log_2 \frac{1}{\pP{\rvY=1|\rvX=1}} 
  \\&=&    (1-\epsilon ) (1-p) \log_2 \frac{1}{1-\epsilon } +
           \epsilon      p     \log_2 \frac{1}{\epsilon   } +
           \epsilon      (1-p) \log_2 \frac{1}{\epsilon   } +
           (1-\epsilon ) p     \log_2 \frac{1}{1-\epsilon } 
  \\&=&    (1-p-\epsilon +\epsilon p+p-\epsilon p) \log_2 \frac{1}{1-\epsilon } +
           (\epsilon p+\epsilon -\epsilon p)   \log_2 \frac{1}{\epsilon   }
  \\&=&    (1-\epsilon ) \log_2 \frac{1}{1-\epsilon } +
           \epsilon      \log_2 \frac{1}{\epsilon   }
\\
\\
  \iI(\rvX;\rvY)
    &=& \iH(\rvY) - \iH(\rvY|\rvX)
  \\&=&    (1-p-\epsilon +2\epsilon p) \log_2 \frac{1}{1-p-\epsilon +2\epsilon p} + 
           (\epsilon +p-2\epsilon p)   \log_2 \frac{1}{\epsilon +p-2\epsilon p}
           - (1-\epsilon ) \log_2 \frac{1}{1-\epsilon } +
           - \epsilon      \log_2 \frac{1}{\epsilon   }
\\
\\
  \iC
    &\eqd& \max_p \iI(\rvX;\rvY)
  \\&=&    \left. \iI(\rvX;\rvY) \right|_{p=\frac{1}{2}}
  \\&=&    \frac{1}{2} \log_2 \frac{1}{\frac{1}{2}} + 
           \frac{1}{2} \log_2 \frac{1}{\frac{1}{2}}
           - (1-\epsilon ) \log_2 \frac{1}{1-\epsilon } +
           - \epsilon      \log_2 \frac{1}{\epsilon   }
  \\&=&    1  + \epsilon \log_2 \epsilon  + (1-\epsilon ) \log_2 (1-\epsilon )
\end{eqnarray*}
\end{proof}

%---------------------------------------
\begin{remark}
%---------------------------------------
\mbox{}\\\rembox{%
  \begin{array}{MM}
      When $\epsilon =0$  &(noiseless channel), the channel capacity is $1$ bit (maximum capacity).
    \\When $\epsilon=1$   &(inverting channel), the channel capacity is still $1$ bit.
    \\When $\epsilon=1/2$ &(totally random channel), the channel capacity is $0$.
    \\When $p=1$ ($1$     &is always transmitted), the entropy of $\rvX$ is $0$.
    \\When $p=0$ ($0$     &is always transmitted), the entropy of $\rvX$ is $0$.
    \\When $p=1/2$        &(totally random transmission), the entropy of $\rvX$ is 1 bit (maximum entropy).
  \end{array}}
\end{remark}

%=======================================
\subsection{Gaussian Noise Channel}
%=======================================

\begin{figure}[ht] \color{figcolor}
\begin{center}
\begin{fsL}
\setlength{\unitlength}{0.20mm}                  
\begin{picture}(700,150)(-100,-50) 
  \thicklines                                      
  %\graphpaper[10](0,0)(500,100)                  
  \put(-100 ,  60 ){\makebox( 100,0)[b]{$\su$} }
  \put(-100 ,  50 ){\vector(1,0){100} }

  \put(  00 ,  10 ){\makebox( 100, 80)[t]{transmit} }
  \put(  00 ,  10 ){\makebox( 100, 80)[c]{operation} }
  \put(  00 ,  10 ){\makebox( 100, 80)[b]{$\opT$} }
  \put(  00 ,  00 ){\framebox( 100,100){} }

  \put( 100 ,  60 ){\makebox( 100,0)[b]{$\rvX$} }
  \put( 100 ,  50 ){\vector(1,0){140} }


  \put( 200 ,  00 ){\makebox(100, 95)[t]{$Z$} }
  \put( 260,   50 ){\line  (1,0){ 45} }
  \put( 250 ,  80 ){\vector(0,-1){20} }
  \put( 250,   50) {\circle{20}                   }
  \put( 200 ,  00 ){\dashbox(100,100){$+$} }
  \put( 200 ,  10 ){\makebox(100, 90)[b]{channel $\opC$} }

  %\put( 200 ,  10 ){\makebox( 100, 80)[t]{channel} }
  %\put( 200 ,  10 ){\makebox( 100, 80)[c]{operation} }
  %\put( 200 ,  10 ){\makebox( 100, 80)[b]{\opC} }
  %\put( 200 ,  00 ){\framebox(100,100){} }

  \put( 300 ,  60 ){\makebox( 100,0)[b]{$\rvY$} }
  \put( 300 ,  50 ){\vector(1,0){100} }

  \put( 400 ,  00 ){\framebox(100,100){} }
  \put( 400 ,  10 ){\makebox( 100, 80)[t]{receive} }
  \put( 400 ,  10 ){\makebox( 100, 80)[c]{operation} }
  \put( 400 ,  10 ){\makebox( 100, 80)[b]{$\opR$} }

  \put( 500 ,  60 ){\makebox( 100,0)[b]{$\sue$} }
  \put( 500 ,  50 ){\vector(1,0){100} }

  %\put(- 90 , -10 ){\makebox( 0, 0)[tl]{$\vu\eqd\su$} }
  %\put( 110 , -10 ){\makebox( 0, 0)[tl]{$s(t;\vu)=\opT\vu$} }
  %\put( 310 , -10 ){\makebox( 0, 0)[tl]{$r(t;\vu)=\opC\opT\vu$} }
  %\put( 510 , -10 ){\makebox( 0, 0)[tl]{$\sue=\opR\opC\opT\vu$} }

\end{picture}                                   
\end{fsL}
\end{center}
\caption{
   Additive noise system model
   \label{fig:i_addNoise_model}
   }
\end{figure}


%--------------------------------------
\begin{theorem}
%--------------------------------------
Let $Z\sim\pN{0}{\sigma^2}$. Then
\thmbox{  \iH(Z) = \frac{1}{2}\log_2 2\pi e \sigma^2  }
\end{theorem}
\begin{proof}
\begin{align*}
  \iH(Z)
    &= \pEz \log \frac{1}{\pp(Z)}
  \\&= -\pEz \log \pp(z) 
  \\&= -\pEz
        \log \left[\frac{1}{\sqrt{2\pi\sigma^2}}e^{\frac{-z^2}{2\sigma^2}} \right] 
  \\&= -\pEz \left[
       -\frac{1}{2}\log(2\pi\sigma^2) 
       + \frac{-z^2}{2\sigma^2} \log e 
       \right] 
  \\&= \frac{1}{2} \pEz \left[
       \log(2\pi\sigma^2) 
       + \frac{\log e}{\sigma^2}z^2  
       \right] 
  \\&= \frac{1}{2} \left[
       \log(2\pi\sigma^2) + \frac{\log e}{\sigma^2}\pEz z^2  
       \right] 
  \\&= \frac{1}{2} \left[
       \log(2\pi\sigma^2) + \frac{\log e}{\sigma^2}(\sigma^2+0)
       \right] 
  \\&= \frac{1}{2} \left[
       \log(2\pi\sigma^2) + \log e
       \right] 
  \\&= \frac{1}{2} \log(2\pi e\sigma^2) 
\end{align*}
\end{proof}

%--------------------------------------
\begin{theorem}
\footnote{
  \citerp{cover}{241}
  }
%--------------------------------------
Let $\rvY=\rvX+Z$ be a Gaussian channel with $\pE \rvX^2=P$ and
$Z\sim\pN{0}{\sigma^2}$. Then
\thmbox{ 
  \iI(\rvX;\rvY) \le \frac{1}{2}\log\left( 1 + \frac{P}{\sigma^2}\right) = \iC 
  \hspace{1cm}\mbox{bits per usage}
  }
\end{theorem}

%--------------------------------------
\begin{theorem}
\footnote{
  \citerp{cover}{250}
  }
%--------------------------------------
Let $\rvY=\rvX+Z$ be a bandlimited Gaussian channel with $\pE \rvX^2=P$ and
$Z\sim\pN{0}{\sigma^2}$ and bandwidth $W$. Then
\thmbox{ 
  \iC = W \log\left( 1 + \frac{P}{\sigma^2 W}\right) 
  \qquad\mbox{bits per second}
  }
\end{theorem}
%\begin{proof}
%By \prefpp{thm:nst}, $R_c \le 2W$.
%No complete proof at this time. \attention
%
%\end{proof}




