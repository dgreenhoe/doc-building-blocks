%============================================================================
% Daniel J. Greenhoe
% LaTeX file
% 
%
% approximation of a sin(pi t) in the Haar k=1 subspace 
%
%  sqrt(2)
% --------- * sqrt(2)^k = 0.63661977236758134307553505349006
%    pi
% nominal xunit = 4mm
% nominal yunit = 8mm
%============================================================================
\begin{pspicture}(-7,-1.5)(8,1.5)%
  \psaxes[linecolor=axis,labels=none,ticks=all]{<->}(0,0)(-7,-1.5)(7,1.5)%
  \multirput(-4,0)(2,0){5}{% dotted connecting segments
    \psline[linestyle=dotted](0,-0.6366)(0,0.6366)%
    }%
  \multirput(-4,0)(4,0){3}{%
    \psline{*-o}(0,0.6366)(1,0.6366)
    \psline{*-o}(1,0.6366)(2,0.6366)
    }%
  \multirput(-6,0)(4,0){3}{%
    \psline{*-o}(0,-0.6366)(1,-0.6366)
    \psline{*-o}(1,-0.6366)(2,-0.6366)
    }%
  \uput[180](0,0.6366){$\frac{\sqrt{2}}{\pi}$}% y=1
  \uput[0](0,-0.6366){$\frac{-\sqrt{2}}{\pi}$}% y=1
  \uput[0](7,0){$x$}%
  \psplot[plotpoints=100,linestyle=dashed,linecolor=red,linewidth=1pt]{-6}{6}{x 90 mul sin}%
  %\rput[b](17.5,10){$\inprod{\ff(t-n)}{\sin(\pi t)}$}%
\end{pspicture}
