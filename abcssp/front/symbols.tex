%============================================================================
% LaTeX File
% Daniel J. Greenhoe
%============================================================================
%--------------------------------------
\section*{Symbols}%
\markboth{{Symbols}}{{Symbols}}
\addcontentsline{toc}{section}{Symbols}%
%--------------------------------------
\qboxnpqt
  {
    \href{http://en.wikipedia.org/wiki/Descartes}{Ren\'e Descartes} 
    \href{http://www-history.mcs.st-andrews.ac.uk/Timelines/TimelineC.html}{(1596--1650)}, 
    \href{http://www-history.mcs.st-andrews.ac.uk/BirthplaceMaps/Places/France.html}{French} philosopher and mathematician
    \index{Descartes, Ren\'e}
    \index{quotes!Descartes, Ren\'e}
    \footnotemark
  }
  {../common/people/descartes_fransHals_bw_wkp_pdomain.jpg}
  {{\bf rugula XVI.}
      Quae vero praesentem mentis attentionem non requirunt,
      etiamsi ad conclusionem necessaria sint,
      illa melius est per brevissimas notas designare quam per integras figuras:
      ita enim memoria non poterit falli, nec tamen interim cogitatio distrahetur
      ad haec retinenda, dum aliis deducendis incumbit.}
  {{\bf Rule XVI.}
      As for things which do not require the immediate attention of the mind,
      however necessary they may be for the conclusion,
      it is better to represent them by very concise symbols rather than by
      complete figures.
      It will thus be impossible for our memory to go wrong,
      and our mind will not be distracted by having to retain these
      while it is taken up with deducing other matters.}
  \footnotetext{
    quote:       \citerc{descartes_rules}{rugula XVI},
    translation: \citerc{descartes_rules_eng}{rule XVI},
    image:       Frans Hals (circa 1650), \url{http://en.wikipedia.org/wiki/Descartes}, public domain
    }

%\qboxnps
%  {\href{http://en.wikipedia.org/wiki/Gottfried_Leibniz}{Gottfried Leibniz}
%   \href{http://www-history.mcs.st-andrews.ac.uk/Timelines/TimelineC.html}{(1646--1716)},
%   \href{http://www-history.mcs.st-andrews.ac.uk/BirthplaceMaps/Places/Germany.html}{German mathematician},
%   \index{Leibniz, Gottfried}
%   \index{quotes!Leibniz, Gottfried}
%   \footnotemark
%  }
%  {../common/people/leibniz.jpg}
%  {As regards signs, I see it clearly that it is to the interest of the Republic of Letters
%   and especially of students, that learned men should reach agreement on signs.}
%  \citetblt{
%    quote: & \citerc{cajori2}{paragraph 540} \\
%    image: & \url{http://en.wikipedia.org/wiki/Gottfried_Leibniz}
%    }

\qboxnps
  {\href{http://en.wikipedia.org/wiki/Gottfried_Leibniz}{Gottfried Leibniz}
   \href{http://www-history.mcs.st-andrews.ac.uk/Timelines/TimelineC.html}{(1646--1716)},
   \href{http://www-history.mcs.st-andrews.ac.uk/BirthplaceMaps/Places/Germany.html}{German mathematician},
   \index{Leibniz, Gottfried}
   \index{quotes!Leibniz, Gottfried}
   \footnotemark
  }
  {../common/people/leibniz_wkp_pdomain_bw.jpg}
  {In signs one observes an advantage in discovery which is greatest when they express
   the exact nature of a thing briefly and, as it were, picture it;
   then indeed the labor of thought is wonderfully diminished.}
  \footnotetext{
    quote: \citerc{cajori2}{paragraph 540},
    image: \scs\url{http://en.wikipedia.org/wiki/File:Gottfried_Wilhelm_von_Leibniz.jpg}, public domain
    }
%\clearpage

%test\[ \oppD \oppR \oppI \oppS\] end test

%\begin{longtable}{c@{\qquad}>{$}l<{$}ll}
\begin{longtable}{c@{\qquad}Mll}
    & $symbol$ & description & reference  \\
    \hline
  \endfirsthead
    & $symbol$ & description  & reference \\
    \hline
  \endhead
    \hline
    \mc{3}{r}{{\sffamily\itshape\scriptsize\ldots continued on next page\ldots}}
  \endfoot
  %  \hline
  %  \hline
  \endlastfoot
  \mc{3}{l}{sets:}
  \\& \emptyset              & \structe{empty set}              & \prefpp{def:emptyset}
  \\& \Z                     & \structe{integers}               & \prefpp{def:Z}  %$\ldots,-3,-2,-1,0,1,2,3,\ldots$
  \\& \Znn                   & \structe{whole numbers}          & \prefpp{def:Z} %$0,1,2,3,\ldots$
  \\& \Zp                    & \structe{natural numbers}        & \prefpp{def:Z} %$1,2,3,\ldots$
  \\& \Zext                  & \structe{extended set of integers} & \prefpp{def:Z}
 %\\& \Znp                   & \structe{non-positive integers}  & $\ldots,-3,-2,-1,0$
 %\\& \Zn                    & \structe{negative integers}      & $\ldots,-3,-2,-1$
 %\\& \Zo                    & \structe{odd integers}           & $\ldots,-3,-1,1,3,\ldots$
 %\\& \Ze                    & \structe{even integers}          & $\ldots,-4,-2,0,2,4,\ldots$
 %\\& \Q                     & \structe{rational numbers}       & $\frac{m}{n}$ with $m\in\Z$ and $n\in\Z\setd0$
  \\& \R                     & \structe{real numbers}           & \prefpp{def:R}
  \\& \Rnn                   & \structe{non-negative real numbers}      & \prefpp{def:Rnn} %$[0,\infty)$
  %\\& \Rnp                   & \structe{non-positive real numbers}      & $(-\infty,0]$
  \\& \Rp                    & \structe{positive real numbers}          & \prefpp{def:Rp} %$(0,\infty)$
  %\\& \Rn                    & \structe{negative real numbers}          & $(-\infty,0)$
  \\& \Rx                    & \structe{extended real numbers}  & \prefpp{def:Rx} %$\Rx\eqd\R\setu\setn{-\infty,\,\infty}$
  \\& \C                     & \structe{complex numbers}        & \prefpp{def:C}
 %\\& \F                     & \structe{arbitrary field}        & (often either $\R$ or $\C$)
  \\
  \mc{3}{l}{values:}
  \\& \pi                    & \structe{pi}                     & $3.14159265\ldots$                                   %&
  \\& \infty                 & \structe{positive infinity}      &
  \\& -\infty                & \structe{negative infinity}      &
  \\
  \mc{3}{l}{relations:}
  \\& \relation              & \structe{relation}                                         & \prefpp{def:relation}       
  \\& \opairn                & \structe{ordered pair}                                     & \prefpp{def:opair}
  \\& \setX\cprod\setY       & \structe{Cartesian product of $\setX$ and $\setY$}         & \prefpp{def:cprod}
  \\& \opair{a}{b}           & \structe{ordered pair}                                     & \prefpp{def:opair}
  \\& \reli                  & \relxe{inverse} of relation $\rel$                         & \prefpp{def:inverse}
  \\& \setX^\xN              &                                                            & \prefpp{def:Xn}
  \\& \otriple{a}{b}{c}      & \fncte{ordered triple}                                     & \prefpp{def:otriple}
 %\\& \owedge                & \prop{relational and}
  \\& \abs{z}                & \prop{absolute value} of a complex number $z$      & \prefpp{def:abs}
  \\& =                      & \prop{equality relation}
  \\& \eqd                   & \prop{equality by definition}
  \\& \to                    & \prop{maps to}
  \\& \in                    & is an element of
  \\& \notin                 & is not an element of
  \\& \oppD(\relation)       & \structe{domain}     of a relation $\relation$    & \prefpp{def:domain}
  \\& \oppI(\relation)       & \structe{image}      of a relation $\relation$    & \prefpp{def:image}
  \\& \oppR(\relation)       & \structe{range}      of a relation $\relation$    & \prefpp{def:range}
  \\& \oppN(\relation)       & \structe{null space} of a relation $\relation$    & \prefpp{def:null}
  \\
  \mc{3}{l}{relations on sets:}
  \\& \subseteq              & \prop{subset}
 %\\& \subsetneq             & \prop{proper subset}
  \\& \subset                & \prop{proper subset}
  \\& \supseteq              & \prop{super set}
 %\\& \supsetneq             & \prop{proper superset}
  \\& \supset                & \prop{proper superset}
  \\& \nsubseteq             & is not a subset of
  \\& \nsubset               & is not a proper subset of
  \\
  \mc{3}{l}{sets of relations:}
  \\& \clRxy                 & \structe{set of all relations in $\setX\cprod\setY$}       & \prefpp{def:clRxy}
  \\& \clFxy                 & \structe{set of all functions in $\setX\cprod\setY$}       & \prefpp{def:clFxy}
  %\\& \clSj{\setX}{\setY}& \mc{2}{l}{set of \hie{surjective} functions from $\setX$ to $\setY$}
  %\\& \clIj{\setX}{\setY}& \mc{2}{l}{set of \hie{injective}  functions from $\setX$ to $\setY$}
  %\\& \clBj{\setX}{\setY}& \mc{2}{l}{set of \hie{bijective}  functions from $\setX$ to $\setY$}
  %\\& \clB {\spX}{\spY}  & \mc{2}{l}{set of \hie{bounded} functions/operators from $\spX$ to $\spY$}
  %\\& \clL {\spX}{\spY}  & \mc{2}{l}{set of \hie{linear bounded} functions/operators from $\spX$ to $\spY$}
  %\\& \clC {\spX}{\spY}  & \mc{2}{l}{set of \hie{continuous} functions/operators from $\spX$ to $\spY$}
  \\
  \mc{3}{l}{operations on sets:}
  \\& \setA \setu \setB      & set \prop{union}
  \\& \setA \seti \setB      & set \prop{intersection}
  \\& \setA \sets \setB      & set \prop{symmetric difference}
  \\& \setA \setd \setB      & set \prop{difference}
  \\& \cmpA                  & set \prop{complement}
  \\& \seto{\cdot}           & set \prop{order}                   & \prefpp{def:seto}
 %\\& \setind_A(x)           & \mc{2}{l}{set \prop{indicator function} or \prop{characteristic function}}
  \\
  \mc{3}{l}{logic:}
  %\\& \ltrue                 & ``\prop{true}" condition
  %\\& \lfals                 & ``\prop{false}" condition
  \\& \lnot                  & logical \prop{NOT} operation
  \\& \land                  & logical \prop{AND} operation
  \\& \lor                   & logical \prop{inclusive OR} operation
  \\& \lxor                  & logical \prop{exclusive OR} operation
  \\& \implies               & ``\prop{implies}";             & ``\prop{only if}"
  \\& \impliedby             & ``\prop{implied by}";          & ``\prop{if}"
  \\& \iff                   & ``\prop{if and only if}";      & ``\prop{implies and is implied by}"
  \\& \forall                & \prop{universal quantifier}:   & ``\prop{for each}"
  \\& \exists                & \prop{existential quantifier}: & ``\prop{there exists}"
  \\
  \mc{3}{l}{order:}
  \\& \join                  & \prop{join} or \prop{least upper bound}    & \prefpp{def:join}
  \\& \meet                  & \prop{meet} or \prop{greatest lower bound} & \prefpp{def:meet}
  \\& \le                    & \prop{reflexive ordering relation}         & \prefpp{def:orel} %``less than or equal to"
  \\& \ge                    & \prop{reflexive ordering relation}         & \prefpp{def:ge}  %``greater than or equal to"
  \\& <                      & \prop{irreflexive ordering relation}       & \prefpp{def:<}  %``less than"
  \\& >                      & \prop{irreflexive ordering relation}       & \prefpp{def:>}  %``greater than"
  \\& \osetX                 & \structe{ordered set}                      & \prefpp{def:oset}
  \\& \intcc{a}{b}           & \structe{closed interval} from $a$ to $b$  & \prefpp{def:intcc}
  \\& \intoc{a}{b}           & \structe{half-open interval} from $a$ to $b$  & \prefpp{def:intoc}
  \\& \intco{a}{b}           & \structe{half-open interval} from $a$ to $b$  & \prefpp{def:intco}
  \\& \intoo{a}{b}           & \structe{open interval} from $a$ to $b$  & \prefpp{def:intoo}
  \\& \abs{x}                & \fncte{absolute value} of $x$            & \prefpp{def:abs}
  \\& \ceil{x}               & \structe{ceiling} of $x$                 & \prefpp{def:ceil}
  \\& \floor{x}              & \structe{floor} of $x$                   & \prefpp{def:floor}
  \\
  %\mc{3}{l}{measures on sets:}
  %\\& \seto{\setX}           & \mc{2}{l}{\prop{order} or \prop{counting measure} of a set $\setX$}
  %\\
  \mc{3}{l}{ring operations:}
  \\& \sum_{n\in\Dom}x_n       & \fncte{sum} over $\Dom$ of $\seqn{x_n}$ & \prefpp{def:sum}
  \\& \fM_\fphi\brp{\seqn{x_n}} & weighted \ope{$\fphi$-mean}  & \prefpp{def:seq_Mphi}
  \\& \fM_{\fphi(x;p)}\brp{\tuplen{x_n}} & \fncte{power mean} with parameter $p$ & \prefpp{def:pmean}
  \\
  \mc{3}{l}{set structures:}
  \\& \topT            & a  \prop{topology of sets}                        & \prefpp{def:topology}
  \\& \ssR             & a  \prop{ring of sets}     
  \\& \ssA             & an \prop{algebra of sets}  
  \\& \ssP{\setX}      & \prop{power set}         on a set $\setX$         & \prefpp{def:pset}
  \\
  \mc{3}{l}{topology:}
  \\& \topspaceX       & \structe{topological space} on a set $\setX$ & \prefpp{def:topspace}
  \\& \clsA            & \ope{closure} of a set $\setA$      & \prefpp{def:clsA}
  \\& \intA            & \structe{interior} of a set $\setA$ & \prefpp{def:intA}
  \\& \lim_{n\to\infty} \seqn{x_n} & \fncte{limit} of $\seqn{x_n}$ & \prefpp{def:limit}
  \\& \seqn{x_n} \to x & \fncte{limit} of $\seqn{x_n}$ & \prefpp{def:limit}
  \\
  \mc{3}{l}{distance spaces:}
  \\& \metric{p}{q}          & \fncte{distance} function           & \prefpp{def:distance}
  \\& \dspaceX               & \structe{distance space}            & \prefpp{def:dspace}
  \\& \ball{x}{r}            & \structe{open ball} centered at $x$ with radius $r$  & \prefpp{def:ball}
  \\& \ballc{x}{r}           & \structe{closed ball} centered at $x$ with radius $r$  & \prefpp{def:ballc}
  \\& \diam \setA            & \fncte{diameter} of a set $\setA$   & \prefpp{def:diam}
  \\& \ptf                   & \fncte{power triangle function}     & \prefpp{def:ptf}
  \\& \trirelD               & \rele{power triangle inequality}    & \prefpp{def:trirel}
  \\
  \mc{3}{l}{probability:}
  \\& \ps                   & \structe{probability space}               & \prefpp{def:ps}
  \\& \pso                  & \structe{set of outcomes},                & \prefpp{def:ps}
  \\& \pse                  & \structe{\txsigma-algebra} on $\pso$      & \prefpp{def:ps}
  \\& \psp                  & \structe{probability function} on $\pse$  & \prefpp{def:ps}
  \\& \rvX                  & \structe{random variable}                 & \prefpp{def:rvt}
  \\& \pE(\rvX)             & \fncte{traditional expected value} of $\rvX$ & \prefpp{def:pE}
  \\& \pVar(\rvX)           & \fncte{traditional variance} of $\rvX$    & \prefpp{def:pVar}
  \\
  %\mc{3}{l}{sets of set structures:}
  %\\& \sssT{\setX}      & \mc{2}{l}{\prop{set of topologies}       on a set $\setX$}
  %\\& \sssR{\setX}      & \mc{2}{l}{\prop{set of rings of sets}    on a set $\setX$} 
  %\\& \sssA{\setX}      & \mc{2}{l}{\prop{set of algebras of sets} on a set $\setX$} 
  %\\
  \mc{3}{l}{graphs:}
    \\& \opair{\setX}{\setE} & \structe{graph}          & \prefpp{def:graph}
    \\& \oquad{\setX}{\setE}{\metricn}{\fw}  & \structe{weighted graph}  & \prefpp{def:wgraph}
    \\& \gcen(\grG)        & \structe{center} of a graph $\grG$  & \prefpp{def:gcen}
    \\
  \mc{3}{l}{outcome subspaces:}
    \\& \epsX  & \structe{extended probability space}                                   & \prefpp{def:ocs}
    \\& \Rxy(n)            & \fncte{cross-correlation} of $\seqn{x_n}$ and $\seqn{y_n}$ & \prefpp{def:ocsRxx}
    \\& \Rxx(n)            & \fncte{auto-correlation}  of $\seqn{x_n}$                  & \prefpp{def:ocsRxx}
    \\& \ocsmom_n(x,y)     & \structe{$n$th-moment}  from $x$ to $y$                    & \prefpp{def:ocsmom}
    \\& \ocsmom(x,y)       & \structe{moment}        from $x$ to $y$                    & \prefpp{def:ocsmom}
    \\& \ocscen(\ocsG)     & \structe{outcome center} of $\ocsG$                        & \prefpp{def:ocscen}
    \\& \ocscena(\ocsG)    & \structe{arithmetic center} of $\ocsG$                     & \prefpp{def:ocscenx}
    \\& \ocsceng(\ocsG)    & \structe{geometric center}  of $\ocsG$                     & \prefpp{def:ocscenx}
    \\& \ocscenh(\ocsG)    & \structe{harmonic center}   of $\ocsG$                     & \prefpp{def:ocscenx}
    \\& \ocscenm(\ocsG)    & \structe{minimal center}    of $\ocsG$                     & \prefpp{def:ocscenx}
    \\& \ocscenM(\ocsG)    & \structe{maxmin center}     of $\ocsG$                     & \prefpp{def:ocscenx}
    \\& \ocsE(\rvX)        & \fncte{outcome expected value}                             & \prefpp{def:ocsE} 
    \\& \ocsVar(\rvX;\pE_x)& \fncte{outcome variance}                                   & \prefpp{def:ocsVar}
    \\& \ocsVar(\rvX)      & \fncte{outcome variance}                                   & \prefpp{def:ocsVar}
    \\& \omsD              & \structe{ordered distance space}                           & \prefpp{def:oms}
    \\& \omsZ              & \structe{integer line}                                     & \prefpp{def:Zline}
    \\& \omsR              & \structe{real line}                                        & \prefpp{def:Rline}
    %\\& \ocs{\setn{\diceA,\diceB,\diceC,\diceD,\diceE,\diceF}}{\ocsd}{\ocsr}{\ocsp} & \structe{die outcome subspace} & \prefpp{def:fdie}
    %\\& \ocs{\setn{\circOne,\circTwo,\circThree,\circFour,\circFive,\circSix}}{\ocsd}{\emptyset}{\ocsp} & \structe{spinner outcome subspace} & \prefpp{def:spinner}
    %\\& \ocs{\setn{\symA,\symT,\symC,\symG}}{\ocsd}{\emptyset}{\ocsp} & \structe{DNA outcome subspace} & \prefpp{def:dna}
    %\\& \ocs{\setn{\symA,\symT,\symC,\symG,\symN}}{\ocsd}{\emptyset}{\ocsp} & \structe{scaffold DNA outcome subspace} & \prefpp{def:dnan}
    \\& \mc{2}{l}{$\ocs{\setn{\diceA,\diceB,\diceC,\diceD,\diceE,\diceF}}{\ocsd}{\ocsr}{\ocsp}                    $} & \prefpp{def:fdie}
    \\& \mc{2}{l}{$\ocs{\setn{\circOne,\circTwo,\circThree,\circFour,\circFive,\circSix}}{\ocsd}{\emptyset}{\ocsp}$} & \prefpp{def:spinner}
    \\& \mc{2}{l}{$\ocs{\setn{\symA,\symT,\symC,\symG}}{\ocsd}{\emptyset}{\ocsp}                                  $} & \prefpp{def:dna}
    \\& \mc{2}{l}{$\ocs{\setn{\symA,\symT,\symC,\symG,\symN}}{\ocsd}{\emptyset}{\ocsp}                            $} & \prefpp{def:dnan}
    \\
  \mc{3}{l}{sequences:}
    \\& \seqnD{x_n} & \structe{sequence} over domain $\Dom$ & \prefpp{def:seq}
    \\& \seqn{x_n}  & \structe{sequence}                    & \prefpp{def:seq}
    \\& \seq{z_n}{\Dom}\eqd\seq{x_n}{\Dom_1}\conv\seq{y_n}{\Dom_2} & \ope{convolution} of $\seqn{x_n}$ and $\seqn{y_n}$ & \prefpp{def:conv}
    \\& \dft\seqn{x_n}  & \ope{Discrete Fourier Transform} on $\seqn{x_n}$ & \prefpp{def:dft}
    \\& \alpha+\seqn{x_n}   & \fncte{constant-sequence addition} & \prefpp{def:axn}
    \\& \seqn{x_n}+\alpha   & \fncte{constant-sequence addition} & \prefpp{def:axn}
    \\& \alpha\seqn{x_n}    & \fncte{constant-sequence multiplication} & \prefpp{def:axn}
    \\& \seqn{x_n}\alpha    & \fncte{constant-sequence multiplication} & \prefpp{def:axn}
    \\
  \mc{3}{l}{linear spaces:}
 %\\& \fieldF                & \structe{field}                            & \prefpp{def:field}
  \\& \linearspaceX          & \structe{linear space} over $\setX$        & \prefpp{def:linspace}
  \\& \metlinspaceX          & \structe{metric linear space}              & \prefpp{def:vs_metric}
  \\& \oml{\R^\xN}{\omlr}{\omld} & \structe{$\R^\xN$ ordered distance linear space} & 
  \\& \normn                 & \prop{vector norm}                         & \prefpp{def:norm}
 %\\& \normopn               & \prop{operator norm}
  %\\& \inprodn               & \prop{inner-product}                       & \prefpp{def:inprod}
  %\\& \oppS(\spV)            & \prop{span} of a linear space $\spV$       & \prefpp{def:span}
  \\
  %\mc{3}{l}{algebras:}
  %\\& \Re                    & \mc{2}{l}{\prop{real part} of an element in a $\invo$-algebra}
  %\\& \Im                    & \mc{2}{l}{\prop{imaginary part} of an element in a $\invo$-algebra}
  %\\
  \mc{3}{l}{ring operations:}
    \\& \oplus                & \ope{addition operator}  & \prefpp{def:add}
    \\& \otimes               & \ope{multiplication operator} & \prefpp{def:mpy}
  %\\
 % \mc{3}{l}{operators on linear spaces:}
  %\\& \opFT              & \mc{2}{l}{\ope{Fourier Transform} operator \ifxref{harFour}{def:opFT}}                %& $\spLLC\to\spLLC$
  %\\& \opFS              & \mc{2}{l}{\ope{Fourier Series} operator \ifxref{harTrig}{def:opFS}}                   %& $\spLLab{0}{2\pi}\to\spllC$
  %\\& \opDTFT            & \mc{2}{l}{\ope{Discrete Time Fourier Series} operator \ifxref{dsp}{def:dtft}}     %& $\spllC\to\spLLab{0}{2\pi}$
  %\\& \opZ               & \mc{2}{l}{\ope{Z-Transform} operator \ifxref{dsp}{def:opZ}}                       %& 
  %\\& \Ff(\omega)        & \mc{2}{l}{\ope{Fourier Transform} of a function $\ff(x)\in\spLLR$}
  %\\& \Dx(\omega)        & \mc{2}{l}{\ope{Discrete Time Fourier Transform} of a sequence $\seqxZ{x_n\in\C}$} 
  %\\& \Zx(z)             & \mc{2}{l}{\ope{Z-Transform} of a sequence $\seqxZ{x_n\in\C}$} 
\end{longtable}

%--------------------------------------
%\footnotetext{
%  Notation $\R, \Znn, \Zp$, etc.:
%  \hie{Bourbaki notation}.
%  Reference: \citerp{davis2005}{9}
%  }
%--------------------------------------



%\qboxnps
%  {\href{http://en.wikipedia.org/wiki/Gottfried_Leibniz}{Gottfried Leibniz}
%   \href{http://www-history.mcs.st-andrews.ac.uk/Timelines/TimelineC.html}{(1646--1716)},
%   \href{http://www-history.mcs.st-andrews.ac.uk/BirthplaceMaps/Places/Germany.html}{German mathematician},
%   \index{Leibniz, Gottfried}
%   \index{quotes!Leibniz, Gottfried}
%   \footnotemark
%  }
%  {../common/people/leibniz.jpg}
%  {All my discoveries were simply improvements in notation.}
%  \citetblt{
%    quote: & \citerp{hein2003}{73} \\
%    image: & \url{http://en.wikipedia.org/wiki/Gottfried_Leibniz}
%    }

%\qboxnps
%  {\href{http://en.wikipedia.org/wiki/Gottfried_Leibniz}{Gottfried Leibniz}
%   \href{http://www-history.mcs.st-andrews.ac.uk/Timelines/TimelineC.html}{(1646--1716)},
%   \href{http://www-history.mcs.st-andrews.ac.uk/BirthplaceMaps/Places/Germany.html}{German mathematician},
%   \index{Leibniz, Gottfried}
%   \index{quotes!Leibniz, Gottfried}
%   \footnotemark
%  }
%  {../common/people/leibniz.jpg}
%  {I perform the calculus by certain new signs of wonderful convenience\ldots}
%  \citetblt{
%    quote: & \citerc{cajori2}{paragraph 540} \\
%    image: & \url{http://en.wikipedia.org/wiki/Gottfried_Leibniz}
%    }
%
%\qboxnps
%  {\href{http://en.wikipedia.org/wiki/Gottfried_Leibniz}{Gottfried Leibniz}
%   \href{http://www-history.mcs.st-andrews.ac.uk/Timelines/TimelineC.html}{(1646--1716)},
%   \href{http://www-history.mcs.st-andrews.ac.uk/BirthplaceMaps/Places/Germany.html}{German mathematician},
%   \index{Leibniz, Gottfried}
%   \footnotemark
%  }
%  {../common/people/leibniz.jpg}
%  {One of the secrets of analysis consists in the characteristic, that is,
%   in the art of skillful employment of the available signs.}
%  \citetblt{
%    quote: & \citerc{cajori2}{paragraph 540} \\
%    image: & \url{http://en.wikipedia.org/wiki/Gottfried_Leibniz}
%    }

%\qboxnps
%  {\href{http://www-history.mcs.st-andrews.ac.uk/Biographies/Newton.html}{Isaac Newton}
%   \href{http://www-history.mcs.st-andrews.ac.uk/Timelines/TimelineC.html}{(1643 - 1727)},
%   \href{http://www-history.mcs.st-andrews.ac.uk/BirthplaceMaps/Places/UK.html}{British mathematician and physicist},
%   \index{Newton, Isaac}
%   \index{quotes!Newton, Isaac}
%   \footnotemark
%  }
%  {../common/people/newton.jpg}
%  {Mr. Newton doth not place his Method in Forms of Symbols,
%   nor confine himself to any particular Sort of Symbols for Fluents and Fluxions.}
%  \citetblt{
%    quote: & \citerp{newton1714}{204} \\
%           & \citerp{newton1714_hall}{294} \\
%           & \citerp{newton1714_wilkins}{19} \\
%           & \citerc{cajori2}{paragraph 538} \\
%    image: & \url{http://www-history.mcs.st-andrews.ac.uk/Thumbnails/Newton.jpg}
%    }




















