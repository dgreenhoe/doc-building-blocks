%============================================================================
% LaTeX File
% Daniel J. Greenhoe
%============================================================================
%=======================================
\chapter{Spectral Estimation}
%=======================================
\qboxnq{Helen Steiner Rice (1900--1981); from \emph{A Bend in the Road}\footnotemark}{\centering
    Sometimes we come to life's crossroads
  \\and view what we think is the end.
  \\But God has a much wider vision
  \\and He knows that it's only a bend.\ldots
  }
\footnotetext{
  \citerpgc{rice2012}{116}{9781620294345}{Note: not sure about page number}.
  Many many thanks to Mary L. Greenhoe for bringing this poem to the author's attention.
  }
%=======================================
\section{Nonparametric spectral estimation}
%=======================================
Quality of spectral estimators\footnote{
  \citerppgc{proakis2002}{452}{457}{9780130622198}{\textsection ``8.2.4 Performance Characteristics of Nonparametric Power Spectrum Estimators"},
  \citerppgc{proakis1996}{916}{919}{9780133737622}{\textsection ``12.2.4 Performance Characteristics of Nonparametric Power Spectrum Estimators"},
  \citerpgc{rao2018}{731}{9789811080814}{``Table 12.1 Comparison of performance of classical methods"},
  \citerpgc{salivahanan2001}{606}{9780074639962}{\textsection ``12.5 Power Spectrum Estimation: Non-Parametric Methods"},
 %\citerppgc{ifeachor1993}{577}{610}{020154413X}{Chapter 10 ``Spectrum estimation and analysis"},
  \citerppgc{ifeachor2002}{706}{707}{0201596199}{\textsection ``11.3.7 Comparison of the power spectral density estimation methods"},
  \citerpg{chitode2009dsp}{P-100}{9788184316469},
  \citerp{abdaheer2009}{204}
  }

\thmbox{\begin{array}{Mrcl}
    \ope{Periodogram}:               & Q &=& 1
  \\\ope{Welch Method 0\% overlap}:  & Q &=& 0.78\xN\Delta f
  \\\ope{Welch Method 50\% overlap}: & Q &=& 1.39\xN\Delta f
  \\\ope{Bartlett Method}:           & Q &=& 1.11\xN\Delta f
  \\\ope{Blackman-Tukey Method}:     & Q &=& 2.34\xN\Delta f
\end{array}}


\begin{figure}
  \centering
  \begin{tabular}{l}
   %\includegraphics{../common/math/graphics/pdfs/sunspots_1749-2004.pdf}\\
    \includegraphics{../common/math/graphics/pdfs/sunspots.pdf}\\
    \includegraphics{../common/math/graphics/pdfs/sunspots_welchPSD_4segs}
  \end{tabular}
  \caption{Estimation of sunspot periodicity using Welch PSD estimation\label{fig:sunspot_welch}}
\end{figure}
%---------------------------------------
\begin{example}
\footnote{
  \citeDc{wdcsilo_sunspot}{\url{http://www.sidc.be/silso/DATA/SN_m_tot_V2.0.txt}},
  \citeD{rdocumentation_sunspotmonth}
  }
%---------------------------------------
Using the PSD, we can estimate the periodicity of sunspots \xref{fig:sunspot_welch} as
\exbox{\begin{array}{M|M}
    number of segments = 2 & estimated periodicty = 11.0277777777777786 years
  \\number of segments = 3 & estimated periodicty = 11.0208333333333321 years
  \\number of segments = 4 & estimated periodicty = 11.0277777777777786 years
  \\number of segments = 8 & estimated periodicty = 11.0000000000000000 years
\end{array}}
\\
That is, period = $\frac{1}{f}=\frac{1}{0.007557\times12}\eqa11.0273$ years.
Here is some R code that can be used to experiment with:
\begin{lstlisting}[language=R]
require(stats);
require(graphics);
require(datasets);
require(ramify);
require(bspec);
data(sunspots, package="datasets");
mypsd = function(x, numSegments=4) 
  {
  xts       = as.ts(as.vector(x)); # year indices seems to confuse welchPSD
  N         = length(xts);         # length of time series
  Fs        = 12;                  # sample rate = 12 samples per year
  estMean   = mean(xts);           # estimated mean
  segLength = N / numSegments;     # segment length
  xpsd      = bspec::welchPSD(xts - estMean, seglength = segLength);
  psdMax    = max(xpsd$power);     # maximum PSD value
  binMax    = ramify::argmax(as.matrix(xpsd$power), rows = FALSE);
  freqMax   = xpsd$frequency[binMax] * Fs; # dominate non-DC frequency
  periodT   = 1 / freqMax;         # estimated period
  plot(xpsd$power, type="b", xlim=c(1,50), ylim=c(0,psdMax), col="blue");
  periodT;                         # return estimated period
  }
 estT = mypsd(x, numSegments=4);
 print(sprintf("estimated period = %.16f years", estT));
\end{lstlisting}
\end{example}

%=======================================
\section{Bandwidth-Time Product}
%=======================================
BT-product references:
\begin{enume}
  \item \citerppgc{haykin2014}{25}{28}{9781118476772}{\textsection ``2.4 The Inverse Relationship between Time-Domain and Frequency-Domain Representations"},
  \item \citerppgc{marple1987}{144}{146}{9780132141499}{{\textsection ``\scshape5.4 Resolution and the Stability-Time-Bandwidth Product"}}
\end{enume}

