%============================================================================
% Daniel J. Greenhoe
% XeLaTeX file
% introduction
%============================================================================
%=======================================
\section{Introduction}
%=======================================
\begin{minipage}{\tw-10mm}
In \ope{traditional stochastic processing}, a \fncte{real-valued random variable} $\rvX$
(or we might even say ``a \fncte{traditional random variable} $\rvX$)
first maps the underlying \structe{probability space} $\ps$ to 
$\omlsRDB$ where $\borel$ is the \structe{usual Borel \txsigma-algebra}
on the ``real line" $\omlsRD$ (see \prefp{fig:die_realline}),
and then operations (such as the expected value operation $\pE(X)$) is performed on $\rvX$. 
Here, \structe{real line} refers to the structure $\omsR$, where $\R$ is the \structe{set of real numbers},
$\orel$ is the standard linear order relation on $\R$, and $\metric{x}{y}\eqd\abs{x-y}$ is the 
\fncte{usual metric} on $\R$.
\end{minipage}\hfill%
\tbox{\includegraphics{sto/graphics/realline.pdf}}
\begin{figure}[h]%
  \centering%
  \includegraphics{sto/graphics/die_realline.pdf}%
  \caption{\fncte{traditional random variable} $\rvX$ mapping a \structe{stochastic process} to the \structe{real line} \label{fig:die_realline}}
\end{figure}

This is all well and good when the physical process being analyzed (in the case of statistical estimation 
or system analysis) or being processed (as in the case of signal processing, including digital signal processing)
is also \prope{linearly ordered} and has a \structe{metric geometry} similar to the one induced 
by the \fncte{usual metric} on $\R$.

Be that as it may, in several real world applications, this is simply not the case. 
Take these processes for example:

\begin{minipage}{\tw-50mm}\imarks
  The values of a \structe{fair die} $\setn{\diceA,\diceB,\diceC,\diceD,\diceE,\diceF}$
  have absolutely no order structure, 
  and have no metric except the \fncte{discrete metric}. 
  On a fair die, $\diceB$ is not greater or less than $\diceA$; rather $\diceB$ and $\diceA$ are simply symbols without order.
  Moreover, $\diceA$ is not ``closer" to $\diceB$ than it is to $\diceC$;
  rather, $\diceA$, $\diceB$, and $\diceC$ are simply symbols without any inherit order or metric geometry.
\end{minipage}\hfill\tbox{\includegraphics{sto/graphics/ocs_fdie.pdf}}

\begin{minipage}{\tw-50mm}\imarks
  \structe{Genomic Signal Processing} (\structe{GSP}) analyzes biological sequences called \structe{genome}s.
        These sequences are constructed over a set of 4 symbols that are commonly referred to as 
        $\symA$, $\symT$, $\symC$, and $\symG$,
        each of which corresponds to a nucleobase (adenine,  thymine, cytosine, and guanine, respectively).\footnotemark
        A typical genome sequence contains a large number of symbols 
        (about 3 billion for humans, 29751 for the SARS virus).\footnotemark
\end{minipage}\hfill\tbox{\includegraphics{sto/graphics/dna.pdf}}
\addtocounter{footnote}{-2}
\footnotetext{
  \citePc{mendel1853e}{Mendel (1853): gene coding uses discrete symbols},
  \citePpc{watson1953}{737}{Watson and Crick (1953): gene coding symbols are adenine,  thymine, cytosine, and guanine},
  \citePp{watson1953may}{965},
  \citerpg{pommerville2013}{52}{1449647960}
  }
\stepcounter{footnote}
\footnotetext{%
  \citeWuc{genbank}{http://www.ncbi.nlm.nih.gov/genome/guide/human/}{Homo sapiens, NC\_000001--NC\_000022 (22 chromosome pairs), NC\_000023 (X chromosome), NC\_000024 (Y chromosome), NC\_012920 (mitochondria)},
  \citeWuc{genbank}{http://www.ncbi.nlm.nih.gov/nuccore/30271926}{SARS coronavirus, NC\_004718.3}
  \citePc{gregory2006}{homo sapien chromosome 1},
  \citePc{he2004}{SARS coronavirus}
  }

\begin{minipage}{\tw-50mm}\imarks
  A \structe{linear congruential pseudo-random number generator}
  induced by the equation $y_{n+1}=(y_n+2)\mod5$ with $y_0=1$.
  The sequential nature of the structure induces both a natural order and distance.
  %See \prefpp{ex:lcg7x1m9_xyz}--\prefpp{ex:lcg7x1m9_dgraph} for further demonstration.
\end{minipage}\hfill\tbox{\includegraphics{sto/graphics/oms_wring5shortd.pdf}}

In all three processes, the symbols in general have an order structure and a \structe{metric geometry} %\xref{rem:mgeo} 
that is fundamentally dissimilar from that of the \structe{real line}. %mapped to by the random variables $\rvX$ and $\rvY$.
Therefore, statistical inferences based on the \structe{real line} will likely lead to results 
that arguably have little relationship with intuition or reality.

So we can observe the following:
\begin{enume}
  \item A traditional random variable $\rvX$ maps to the real line.
  \item The structure of the underlying stochastic process (the domain of $\rvX$) may be very dissimilar to that of the real line.
  \item To fix the problem, we need random variables that map to alternative structures that are more similar to the underlying stochastic processes.
  \item Such a structure should have both an \rele{order relation} and a \rele{distance function} defined on it
        that are similar to the stochastic process. In particular, such a structure should be an 
        \structe{ordered distance space}. And ideally, the stochastic process and the ordered distance space should be 
        both \propb{isomorphic} and \propb{isometric} with respect to each other.
\end{enume}

This text proposes two \structe{ordered distance space}s for stochastic processing:

\begin{minipage}{\tw-50mm}
    \imarks \fbox{\pref{chp:sto}} proposes mapping to \structe{directed weighted graph}s.\\
    In such a structure, order is represented by direction of it's edges,
    and distance by the lengths of it's edges.
    Furthermore, probability can be represented by weights assigned to it's vertices.
\end{minipage}\quad\tbox{\includegraphics{sto/graphics/ocs_wdie.pdf}}
                  %\includegraphics{sto/graphics/oms_wring5shortd.pdf}%

\begin{minipage}{\tw-50mm}
  \imarks \fbox{\pref{chp:ssp}} proposes mapping to an \structe{ordered distance linear space} 
  $\omlsRnD$, where $\fieldRa$ is a \structe{field}, 
  $+$ is the vector addition operator on $\R^n\times\R^n$, 
  and $\cdot$ is the scalar-vector multiplication operator on $\R\times\R^n$.
  Probability has no natural representation here, and must be assigned through a separate 
  \fncte{probability function} $\psp$.
\end{minipage}\quad\tbox{\includegraphics{ssp/graphics/rdie_r3met.pdf}}%


Mapping to a weighted graph as in \pref{chp:sto} 
is useful for analysis of a single random variable $\rvX$.
But the traditional expectation value $\pE(\rvX)$ of $\rvX$ is often a poor choice of a statistic.
%when the stochastic process that $\rvX$ maps from is a structure other than the real line or 
%some substructure of the real line.
For example, the traditional expected value of a fair die is $\pE(X)=\frac{1}{6}(1+2+\cdots+6)=3.5$.
But this result has no relationship with reality or with intuition because the result implies that we expect 
the value of $\diceC$ or $\diceD$ more than we expect the outcome of say $\diceA$ or $\diceB$.
The fact is, that for a fair die, we would expect any pair of values equally.
The reason for this is that the values of the face of a fair die are merely symbols with no order, 
and with no metric geometry other than the \structe{discrete metric geometry}.
Weighted graphs on the other hand offer structures more similar to the underlying stochastic process.
And the \fncte{expectation} $\pE\rvX$ 
of $\rvX$ can be defined simply as the \fncte{center} of the \structe{weighted graph} % ~166 words
(as illustrated above for a \structe{weighted die} with $\pE\rvX$ shaded in blue). % ~166 words

However, the mapping has limitations with regards to a \structe{sequence} of random variables in 
performing sequence analysis (using for example \ope{Fourier analysis} or \ope{wavelet analysis}),
in performing sequence processing (using for example \ope{FIR filtering} or \ope{IIR filtering}),
in making   diagnostic measurements (using a post-transform metric space),
or in making ``optimal" decisions 
(based on ``distance" measurements in a \structe{metric space} or more generally a \structe{distance space}).  %~224
\\\indentx\includegraphics{ssp/plots/fdie_5eed_51_seq.pdf}

Mapping to an \structe{ordered distance linear space} $\spY$ as in \pref{chp:ssp}
provides the \structe{linear space} component of $\spY$,
which in turn provides a much more convenient framework for \fncte{sequence} analysis and processing
(as compared to the weighted graph).
The \structe{ordered set} and \structe{distance space} components of $\spY$ allow one to preserve the 
order structure and distance geometry inherent in the underlying %~280
stochastic process, which can provide a less distorted (as compared to the \structe{real line})
framework for analysis, diagnostics, and optimal decision making. % ~299

For any given stochastic process, there are an infinite number of possible random variable mappings,
with some being ``better"---with respect to some criteria (probably involving \prope{isomorphic}
and \prope{isometric} properties)---than others. 
Multiple mappings using random variables $\rvW$, $\rvX$, $\rvY$, and $\rvZ$
for the ``\structe{real die}" stochastic process $\ocsG$ are illustrated in \prefpp{fig:intro_realdieXRYZ}.
\begin{figure}[h]
  \centering%
  \gsize%
  %{%============================================================================
% Daniel J. Greenhoe
% LaTeX file
% discrete metric real die mapping to 4 outcome spaces
%============================================================================
{%\psset{unit=0.5\psunit}%
\begin{pspicture}(-6,-3.6)(6,3.8)%
  %---------------------------------
  % options
  %---------------------------------
  \psset{%
    radius=1.25ex,
    labelsep=2.5mm,
    linecolor=blue,%
    fillstyle=none,
    }%
  %---------------------------------
  % die graph
  %---------------------------------
  \rput(0,0){%\psset{unit=2\psunit}%
    \uput{1}[210](0,0){\Cnode[fillstyle=solid,fillcolor=snode](0,0){D4}}%
    \uput{1}[150](0,0){\Cnode[fillstyle=solid,fillcolor=snode](0,0){D5}}%
    \uput{1}[ 90](0,0){\Cnode[fillstyle=solid,fillcolor=snode](0,0){D6}}%
    \uput{1}[ 30](0,0){\Cnode[fillstyle=solid,fillcolor=snode](0,0){D3}}%
    \uput{1}[-30](0,0){\Cnode[fillstyle=solid,fillcolor=snode](0,0){D2}}%
    \uput{1}[-90](0,0){\Cnode[fillstyle=solid,fillcolor=snode](0,0){D1}}%
    \rput(0,0){$\ocsG$}%
    }
  \rput(D6){$\diceF$}%
  \rput(D5){$\diceE$}%
  \rput(D4){$\diceD$}%
  \rput(D3){$\diceC$}%
  \rput(D2){$\diceB$}%
  \rput(D1){$\diceA$}%
  %
  \ncline{D5}{D6}%
  \ncline{D4}{D5}\ncline{D4}{D6}%
  \ncline{D3}{D5}\ncline{D3}{D6}%
  \ncline{D2}{D3}\ncline{D2}{D4}\ncline{D2}{D6}%
  \ncline{D1}{D2}\ncline{D1}{D3}\ncline{D1}{D4}\ncline{D1}{D5}%
  %
  \uput[ 210](D4){$\frac{1}{6}$}
  \uput[ 150](D5){$\frac{1}{6}$}
  \uput[  90](D6){$\frac{1}{6}$}
  \uput[  30](D3){$\frac{1}{6}$}
  \uput[ -30](D2){$\frac{1}{6}$}
  \uput[ -90](D1){$\frac{1}{6}$}
  %---------------------------------
  % isomorphic mapping Y
  %---------------------------------
  \rput(-4.5,0){%\psset{unit=2\psunit}%
    \uput{1}[210](0,0){\Cnode[fillstyle=solid,fillcolor=snode](0,0){Y4}}%
    \uput{1}[150](0,0){\Cnode[fillstyle=solid,fillcolor=snode](0,0){Y5}}%
    \uput{1}[ 90](0,0){\Cnode[fillstyle=solid,fillcolor=snode](0,0){Y6}}%
    \uput{1}[ 30](0,0){\Cnode[fillstyle=solid,fillcolor=snode](0,0){Y3}}%
    \uput{1}[-30](0,0){\Cnode[fillstyle=solid,fillcolor=snode](0,0){Y2}}%
    \uput{1}[-90](0,0){\Cnode[fillstyle=solid,fillcolor=snode](0,0){Y1}}%
    \rput(0,0){$\omsH$}%
    }
  \rput(Y6){$6$}%
  \rput(Y5){$5$}%
  \rput(Y4){$4$}%
  \rput(Y3){$3$}%
  \rput(Y2){$2$}%
  \rput(Y1){$1$}%
  %
  \ncline{Y5}{Y6}%
  \ncline{Y4}{Y5}\ncline{Y4}{Y6}%
  \ncline{Y3}{Y5}\ncline{Y3}{Y6}%
  \ncline{Y2}{Y3}\ncline{Y2}{Y4}\ncline{Y2}{Y6}%
  \ncline{Y1}{Y2}\ncline{Y1}{Y3}\ncline{Y1}{Y4}\ncline{Y1}{Y5}%
  %
  \uput[210](Y4){$\frac{1}{6}$}%
  \uput[150](Y5){$\frac{1}{6}$}%
  \uput[ 90](Y6){$\frac{1}{6}$}%
  \uput[ 30](Y3){$\frac{1}{6}$}%
  \uput[-30](Y2){$\frac{1}{6}$}%
  \uput[-90](Y1){$\frac{1}{6}$}%
  %
  \ncarc[arcangle=-22,linewidth=0.75pt,linecolor=blue]{->}{D6}{Y6}%
  \ncarc[arcangle=-67,linewidth=0.75pt,linecolor=blue]{->}{D5}{Y5}%
  \ncarc[arcangle= 67,linewidth=0.75pt,linecolor=blue]{->}{D4}{Y4}%
  \ncarc[arcangle=-67,linewidth=0.75pt,linecolor=blue]{->}{D3}{Y3}%
  \ncarc[arcangle= 67,linewidth=0.75pt,linecolor=blue]{->}{D2}{Y2}%
  \ncarc[arcangle= 22,linewidth=0.75pt,linecolor=blue]{->}{D1}{Y1}%
  %---------------------------------
  % random variable mapping Z to extended outcome space mapping Z
  %---------------------------------
  \rput(4.5,0){%\psset{unit=2\psunit}%
    \uput{1}[210](0,0){\Cnode(0,0){Z4}}%
    \uput{1}[150](0,0){\Cnode(0,0){Z5}}%
    \uput{1}[ 90](0,0){\Cnode(0,0){Z6}}%
    \uput{1}[ 30](0,0){\Cnode(0,0){Z3}}%
    \uput{1}[-30](0,0){\Cnode(0,0){Z2}}%
    \uput{1}[-90](0,0){\Cnode(0,0){Z1}}%
    \Cnode[fillstyle=solid,fillcolor=snode](0,0){Z0}%
    \uput{1.25}[60](0,0){$\omsK$}%
    }%
  \rput(Z6){$6$}%
  \rput(Z5){$5$}%
  \rput(Z4){$4$}%
  \rput(Z3){$3$}%
  \rput(Z2){$2$}%
  \rput(Z1){$1$}%
  \rput(Z0){$0$}%
  %
  \ncline{Z5}{Z6}%
  \ncline{Z4}{Z5}\ncline{Z4}{Z6}%
  \ncline{Z3}{Z5}\ncline{Z3}{Z6}%
  \ncline{Z2}{Z3}\ncline{Z2}{Z4}\ncline{Z2}{Z6}%
  \ncline{Z1}{Z2}\ncline{Z1}{Z3}\ncline{Z1}{Z4}\ncline{Z1}{Z5}%
  \ncline{Z0}{Z1}\ncline{Z0}{Z2}\ncline{Z0}{Z3}\ncline{Z0}{Z4}\ncline{Z0}{Z5}\ncline{Z0}{Z6}%
  %
  \uput[210](Z4){$\frac{1}{6}$}
  \uput[150](Z5){$\frac{1}{6}$}
  \uput[ 90](Z6){$\frac{1}{6}$}
  \uput[ 30](Z3){$\frac{1}{6}$}
  \uput[-30](Z2){$\frac{1}{6}$}
  \uput[-90](Z1){$\frac{1}{6}$}
  \uput[  0](Z0){$\sfrac{0}{6}$}
  %
  \ncarc[arcangle= 22,linewidth=0.75pt,linecolor=green]{->}{D6}{Z6}%
  \ncarc[arcangle= 67,linewidth=0.75pt,linecolor=green]{->}{D5}{Z5}%
  \ncarc[arcangle=-67,linewidth=0.75pt,linecolor=green]{->}{D4}{Z4}%
  \ncarc[arcangle= 67,linewidth=0.75pt,linecolor=green]{->}{D3}{Z3}%
  \ncarc[arcangle=-67,linewidth=0.75pt,linecolor=green]{->}{D2}{Z2}%
  \ncarc[arcangle=-22,linewidth=0.75pt,linecolor=green]{->}{D1}{Z1}%
  %---------------------------------
  % random variable mapping W from G to real line 
  %---------------------------------
  \rput(0,-3){%\psset{unit=0.75\psunit}%
    \multirput(-2.5,0)(1,0){6}{\psline(0,-0.1)(0,0.1)}%
    \pnode(3.5,0){WB}%
    \pnode(2.5,0){W6}%
    \pnode(1.5,0){W5}%
    \pnode(0.5,0){W4}%
    \pnode(0,0){W34}%
    \pnode(-0.5,0){W3}%
    \pnode(-1.5,0){W2}%
    \pnode(-2.5,0){W1}%
    \pnode(-3.5,0){WA}%
    }
  \uput[-90](W6){$6$}%
  \uput[-90](W5){$5$}%
  \uput[-90](W4){$4$}%
  \uput[-90](W3){$3$}%
  \uput[-90](W2){$2$}%
  \uput[-90](W1){$1$}%
  %
  \ncline{<->}{WA}{WB}%
  \pscircle[fillstyle=solid,linecolor=snode,fillcolor=snode](W34){1ex}%
  \pscircle[fillstyle=none,linecolor=red,fillcolor=red](W34){1ex}%
  %\rput(W233){\pscircle[fillstyle=none,linecolor=red,fillcolor=red](0,0){1ex}}%
  %
  \uput[ 0](WB){$\omsR$}
  \uput[90](W6){$\frac{1}{6}$}
  \uput[90](W5){$\frac{1}{6}$}
  \uput[90](W4){$\frac{1}{6}$}
  \uput[90](W3){$\frac{1}{6}$}
  \uput[90](W2){$\frac{1}{6}$}
  \uput[90](W1){$\frac{1}{6}$}
  %
  \ncarc[arcangle= 67,linewidth=0.75pt,linecolor=purple]{->}{D6}{W6}%
  \ncarc[arcangle=-67,linewidth=0.75pt,linecolor=purple]{->}{D5}{W5}%
  \ncarc[arcangle= 22,linewidth=0.75pt,linecolor=purple]{->}{D4}{W4}%
  \ncarc[arcangle= 45,linewidth=0.75pt,linecolor=purple]{->}{D3}{W3}%
  \ncarc[arcangle= 22,linewidth=0.75pt,linecolor=purple]{->}{D2}{W2}%
  \ncarc[arcangle=-10,linewidth=0.75pt,linecolor=purple]{->}{D1}{W1}%
  %
  %---------------------------------
  % random variable mapping X from G to integer line 
  %---------------------------------
  \rput(0,3){%\psset{unit=0.75\psunit}%
    \pnode(3.5,0){XB}%
    \Cnode(2.5,0){X6}%
    \Cnode(1.5,0){X5}%
    \Cnode[fillstyle=solid,fillcolor=snode](0.5,0){X4}%
    \Cnode[fillstyle=solid,fillcolor=snode](-0.5,0){X3}%
    \Cnode(-1.5,0){X2}%
    \Cnode(-2.5,0){X1}%
    \pnode(-3.5,0){XA}%
    }
  \rput(X6){$6$}%
  \rput(X5){$5$}%
  \rput(X4){$4$}%
  \rput(X3){$3$}%
  \rput(X2){$2$}%
  \rput(X1){$1$}%
  %
  \ncline[linestyle=dotted]{X6}{XB}%
  \ncline{X5}{X6}%
  \ncline{X4}{X5}%
  \ncline{X3}{X4}%
  \ncline{X2}{X3}%
  \ncline{X1}{X2}%
  \ncline[linestyle=dotted]{X1}{XA}%
  %
  \uput[ 0](XB){$\omsZ$}
  \uput[90](X6){$\frac{1}{6}$}
  \uput[90](X5){$\frac{1}{6}$}
  \uput[90](X4){$\frac{1}{6}$}
  \uput[90](X3){$\frac{1}{6}$}
  \uput[90](X2){$\frac{1}{6}$}
  \uput[90](X1){$\frac{1}{6}$}
  %
  \ncarc[arcangle= 10,linewidth=0.75pt,linecolor=red]{->}{D6}{X6}%
  \ncarc[arcangle= 22,linewidth=0.75pt,linecolor=red]{->}{D5}{X5}%
  \ncarc[arcangle= 67,linewidth=0.75pt,linecolor=red]{->}{D4}{X4}%
  \ncarc[arcangle=-45,linewidth=0.75pt,linecolor=red]{->}{D3}{X3}%
  \ncarc[arcangle=-67,linewidth=0.75pt,linecolor=red]{->}{D2}{X2}%
  \ncarc[arcangle= 67,linewidth=0.75pt,linecolor=red]{->}{D1}{X1}%
  %
  %---------------------------------
  % labels
  %---------------------------------
  \rput(2.3,0){$\rvZ(\cdot)$}%
  \rput(-2.3,0){$\rvY(\cdot)$}%
  \rput(0,2){$\rvX(\cdot)$}%
  \rput(0,-2.2){$\rvW(\cdot)$}%
\end{pspicture}
}%}%
  {\includegraphics{sto/graphics/rdie_wxyz.pdf}}%
  \caption{several random variable mappings for the \structe{real die} \label{fig:intro_realdieXRYZ}}
\end{figure}


%=======================================
\section{Further mathematical support}
%=======================================
%Here is an overview of this text with more detail:

%\imarks \fbox{\pref{chp:sto}} proposes mapping to \structe{weighted graph}s.\\
%  This is useful for computing single statistics such as the expected value $\pE(\rvX)$ of a random variable $\rvX$. 
%The traditional expectation value $\pE(\rvX)$ of $\rvX$ is then often a poor choice of a statistic
%when the stochastic process that $\rvX$ maps from is a structure other than the real line or 
%some substructure of the real line.
%For example, the traditional expected value of a fair die is $\pE(X)=\frac{1}{6}(1+2+\cdots+6)=3.5$.
%But this result has no relationship with reality or with intuition because the result implies that we expect 
%the value of $\diceC$ or $\diceD$ more than we expect the outcome of say $\diceA$ or $\diceB$.
%The fact is, that for a fair die, we would expect any pair of values equally.
%The reason for this is that the values of the face of a fair die are merely symbols with no order, 
%and with no metric geometry other than the \structe{discrete metric geometry}.
%%\pref{chp:sto} proposes an alternative statistical system, based somewhat on graph theory, 
%%that takes into account the order structure and metric geometry of the underlying stochastic process.
%Weighted graphs on the other hand offer structures more similar to the underlying stochastic process.
%And the \fncte{expectation} $\pE\rvX$ 
%of $\rvX$ can be defined simply as the \fncte{center} of the \structe{weighted graph}. % ~166 words

%\fbox{\imarks \pref{chp:ssp}} proposes mapping to $\xN$-dimensional real spaces $\R^\xN$.\\
%Mapping to a weighted graph is useful for analysis of a single random variable.
%However, the mapping has limitations with regards to a \structe{sequence} of random variables in 
%performing sequence analysis (using for example \ope{Fourier analysis} or \ope{wavelet analysis}),
%in performing sequence processing (using for example \ope{FIR filtering} or \ope{IIR filtering}),
%in making   diagnostic measurements (using a post-transform metric space),
%or in making ``optimal" decisions 
%(based on ``distance" measurements in a \structe{metric space} or more generally a \structe{distance space}).  %~224
%
%\imarks \fbox{\pref{chp:ssp}}, rather than mapping to a \structe{weighted graph}, 
%proposes instead mapping to an \structe{ordered distance linear space} 
%$\spY\eqd\omlsRnD$, where $\fieldRa$ is a \structe{field}, 
%$+$ is the vector addition operator on $\R^n\times\R^n$, 
%and $\cdot$ is the scalar-vector multiplication operator on $\R\times\R^n$.
%The \structe{linear space} component of $\spY$ provides a much more convenient % ~254
%(as compared to the \structe{weighted graph}) 
%framework for sequence analysis and processing.
%The \structe{ordered set} and \structe{distance space} components of $\spY$ allow one to preserve the 
%order structure and distance geometry inherent in the underlying %~280
%stochastic process, which in turn likely provides a less distorted (as compared to the \structe{real line})
%framework for analysis, diagnostics, and optimal decision making. % ~299

Further mathematical support for these two methods is provided in appendices.
Here is partial summary of what may be found there:

\begin{minipage}{\tw-50mm}\imarks
  \fbox{\pref{app:larc}} introduces what is herein called the \fncte{Lagrange arc distance} function.
  It is important in this text because it is used in \pref{chp:ssp} in the processing of \fncte{real die sequence}s 
  in $\R^3$ and \fncte{spinner sequence}s in $\R^2$.
  The function is an extension to \emph{all} of $\R^\xN$ of the \fncte{spherical metric}, which has as domain
  only a ``spherical" \emph{surface} in $\R^\xN$.
  The \fncte{Lagrange arc distance} $\distance{p}{q}$ draws arcs between certain pairs of points $\opair{p}{q}$ using 
  \ope{Lagrange interpolation}.
\end{minipage}\hfill\tbox{\includegraphics{larc/graphics/larc_metex.pdf}}
%\includegraphics{larc/graphics/larc_R2_balls.pdf}
    

\begin{minipage}{\tw-50mm}\imarks
  \fbox{\pref{app:distance}} presents \structe{distance space}s.
  A \structe{distance space} is a \structe{metric space} in which the \prope{triangle inequality} does not
  necessarily hold.
  \structe{Distance space}s are important in this text because the \fncte{Lagrange arc distance} is a \fncte{distance} function,
  and \emph{not} a \fncte{metric} in that %it is \emph{not} always true that
  $\distance{p}{r}\orel\distance{p}{q}+\distance{q}{r}$
  does \emph{not} necessarily hold for all \structe{triple}s $\otriple{p}{q}{r}$ in $\R^\xN$.
\end{minipage}\hfill\tbox{\includegraphics{larc/graphics/larc_trieq.pdf}}

\imarks \fbox{\pref{app:pds}} introduces what is herein called the \structe{power distance space}. 
  This space is a generalization of the \structe{metric space},
  and is a \structe{distance space} that satisfies what is herein called the \fncte{power triangle inequality}:
  \\\indentx$\begin{array}{rclCC}
    \distance{x}{y} &\orel& 2\sigma\brs{\frac{1}{2}\distancep{p}{x}{z} + \frac{1}{2}\distancep{p}{z}{y}}^\frac{1}{p} & \forall \opair{p}{\sigma}\in\Rx\times\Rnn, & x,y,z\in\setX
  \end{array}$\\
  It turns out that the inequality $2\sigma\le2^\frac{1}{p}$ has special significance with regards to these spaces
  and appears repeatedly in \pref{app:pds}.
  It is plotted in \prefpp{fig:intro_sigmap}.
  \structe{Power distance space}s are not explicitly used in this text. 
  However, they may prove useful to future research in symbolic sequence processing space
  in which the \prope{triangle inequality} fails to hold, but in which some of the key properties of a metric space 
  are still required.
\begin{figure}[h]
  \gsize%
  \centering%
  \includegraphics{pds/graphics/trirel_sigmap.pdf}
  \caption{$\sigma = \frac{1}{2}(2^{\frac{1}{p}}) = 2^{\frac{1}{p}-1}$ or $p=\frac{\ln2}{\ln(2\sigma)}$ 
  \label{fig:intro_sigmap}
  }
\end{figure}





%\fbox{\imarks \pref{app:larc}} introduces the \emph{Lagrange arc distance} function.
%%It is important in this text because it is used in \pref{chp:ssp} in the processing of \fncte{real die sequence}s  and \fncte{spinner sequence}s.
%The \fncte{Lagrange arc distance} is an extension of the \fncte{spherical metric}.
%The \fncte{spherical metric} $\metrican_r$ operates on the surface of a \structe{sphere} with radius $r$
%centered at the origin in a \structe{linear space} $\R^\xN$.
%Thus, for any pair of points $\opair{p}{q}$ \emph{on the surface} of this sphere,
%$\opair{p}{q}$ is in the \structe{domain} of $\metrican_r$ and 
%$\metrican_r\opair{p}{q}$ is the ``distance" between those points.
%However, if $x$ and $y$ \emph{are} both in $\R^\xN$ but are \emph{not} on the surface of a    % ~57 words
%common sphere centered at the origin, then $\opair{p}{q}$ is \emph{not} in the domain
%of $\metrican_r$ and $\metrican_r(p,q)$ is simply \emph{undefined}.                             % ~73
%In symbolic sequence processing, however, it would be useful to have an \prope{extension} $\metricn$ of $\metrican$
%to the entire space $\R^\xN$ (rather than just on a surface in $\R^\xN$).  % ~101
%The \fncte{Lagrange arc distance} is such an extension.
%%This paper introduces an extension to the spherical metric using a polar form of \ope{linear interpolation}.
%%\fncte{Lagrange polynomial} interpolation.
%%The extension is herein called the \fncte{Lagrange arc distance}.
%It has as its domain the entire space $\R^\xN$, % (rather than just a surface \emph{in} $\R^\xN$),
%is \prope{homogeneous}, and is \prope{continuous} everywhere in $\R^\xN$ except at the origin.
%However the extension does come at a cost: 
%The \fncte{Lagrange arc distance} $\distance{p}{q}$, as its name suggests, 
%is a \fncte{distance function} rather than a \fncte{metric}.
%In particular, the \prope{triangle inequality} does not in general hold.
%Moreover, it is \prope{not translation invariant}, 
%%is \prope{discontinuous} at the origin of $\R^\xN$ (but \prope{continuous everywhere else}),
%does \prope{not induce a norm},
%and balls in the \structe{distance space} $\opair{\R^\xN}{\distancen}$ are \prope{not convex}.
%On the other hand, empirical evidence suggests that 
%the \fncte{Lagrange arc distance} results in structure similar to that of 
%the \fncte{Euclidean metric} in that balls in $\R^2$ and $\R^3$ generated by the two functions are in 
%some regions of $\R^\xN$ very similar in form.
%
%\fbox{\imarks \pref{app:distance}} presents \structe{distance space}s.
%A \structe{distance space} is a \structe{metric space} in which the \prope{triangle inequality} does not
%necessarily hold.
%\structe{Distance space}s are important in this text because the \fncte{Lagrange arc distance} is a \fncte{distance} function.
%A questions that natually follows is,
%``What happens if we remove the \prope{triangle inequality} all together?"
%\pref{sec:dspace} introduces \structe{distance space}s and demonstrates that some properties 
%commonly associated with \structe{metric space}s also hold in any \structe{distance space}:
%\\\indentx$\begin{array}{FMD}
%      D1. & $\emptyset$ and $\setX$ are \prope{open}                                          & \xref{thm:dspace_open}      
%    \\D2. & the intersection of a finite number of open sets is \prope{open}                  & \xref{thm:dspace_open}      
%    \\D3. & the union of an arbitrary    number of open sets is \prope{open}                  & \xref{thm:dspace_open}      
%    \\D4. & every \prop{Cauchy} sequence is \prope{bounded}                                   & \xref{prop:cauchy==>bounded}
%    \\D5. & any subsequence of a \prope{Cauchy} sequence is also \prope{Cauchy}               & \xref{prop:cauchy_subseq}   
%    \\D6. & the \thme{Cantor Intersection Theorem} holds                                      & \xref{thm:cit}              
%\end{array}$\\
%The following five properties (M1--M5) \emph{do} hold in any \structe{metric space}.
%However, the examples %from \pref{sec:dspace} 
%listed below demonstrate that the five properties do \emph{not} 
%hold in all \structe{distance spaces}: %\footnote{The examples referred to for M1--M5 can all in essense be found in \citerppgc{blumenthal1953}{8}{12}{0828402426}{6. Topology of semimetric spaces}}
%\\\indentx$\begin{array}{FMDM}
%      M1. & the \fncte{metric function} is \prope{continuous}                  &fails to hold in&\pref{ex:dspace_01}--\pref{ex:dspace_21}
%    \\M2. & \structe{open ball}s are \prope{open}                              &fails to hold in&\pref{ex:dspace_01} and \pref{ex:dspace_1n}
%    \\M3. & the \structe{open ball}s form a \structe{base} for a topology      &fails to hold in&\pref{ex:dspace_01} and \pref{ex:dspace_1n}
%    \\M4. & the limits of \structe{convergent sequence}s are \prope{unique}    &fails to hold in&\pref{ex:dspace_01}
%    \\M5. & \prope{convergent} sequences are \prope{Cauchy}                    &fails to hold in&\pref{ex:dspace_1n}
%\end{array}$
%
%\fbox{\imarks \pref{app:pds}} introduces what is herein called \structe{power distance space}s. 
%A \structe{power distance space} is a \structe{distance space} that also satisfies what is herein called 
%the \prope{power triangle inequality} property:
%  \\\indentx$\begin{array}{rclCC}
%    \distance{x}{y} &\orel& 2\sigma\brs{\frac{1}{2}\distancep{p}{x}{z} + \frac{1}{2}\distancep{p}{z}{y}}^\frac{1}{p} & \forall \opair{p}{\sigma}\in\Rx\times\Rnn, & x,y,z\in\setX
%  \end{array}$,
%\\
%which is a generalization of the \fncte{triangle inequality} property.
%Even if the \prope{triangle inequality} property does not hold in some symbolic sequence processing space,
%in some cases the properties M1--M5 \emph{may} still hold:
%In particular, \pref{app:pds} shows for what values of $\opair{p}{\sigma}$ the properties M1--M5 hold %. % in these spaces.
%%Here is a summary of the results
%%for what values of $\opair{p}{\sigma}$ in $\Rx\times\Rp$
%%the five basic properties listed earlier for metric spaces also hold 
%in a \structe{power distance space} $\pdspaceX$: %, for all $x,y,z\in\setX$:
%\\\indentx$\begin{array}{FMMM}
%    (M1) & holds for any $\opair{p}{\sigma}\in(\Rx\setd\setn{0})\times\Rp$ such that $2\sigma =   2^{\frac{1}{p}}$ & \xref{thm:pdspace_continuous}
%  \\(M2) & holds for any $\opair{p}{\sigma}\in(\Rx\setd\setn{0})\times\Rp$ such that $2\sigma \le 2^{\frac{1}{p}}$ & \xref{cor:oball_open}
%  \\(M3) & holds for any $\opair{p}{\sigma}\in(\Rx\setd\setn{0})\times\Rp$ such that $2\sigma \le 2^{\frac{1}{p}}$ & \xref{cor:tspace_base}
%  \\(M4) & holds for any $\opair{p}{\sigma}\in\Rx\times\Rp$                                                        & \xref{thm:xn_to_xy}
%  \\(M5) & holds for any $\opair{p}{\sigma}\in\Rx\times\Rp$                                                        & \xref{thm:convergent==>cauchy}
%\end{array}$
%\\
%\structe{Power distance space}s are not explicitly used in this text. 
%However, they may prove useful to future research in symbolic sequence processing space
%in which the \prope{triangle inequality} fails to hold, but in which some or all of the properties M1--M5 are required.





