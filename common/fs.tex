%============================================================================
% LaTeX File
% Daniel J. Greenhoe
%============================================================================
%=======================================
\chapter{Fourier Series}
\label{chp:fs}
\label{app:fs}
%=======================================
\qboxnpqt
  {
    A competition awards committee consisting of the
    mathematical giants
    \href{http://en.wikipedia.org/wiki/Joseph_Louis_Lagrange}{Lagrange},
    \href{http://en.wikipedia.org/wiki/Laplace}{Laplace},
    \href{http://en.wikipedia.org/wiki/Legendre}{Legendre}, and others,
    commenting on
    \href{http://en.wikipedia.org/wiki/Joseph_Fourier}{Fourier's}
    \href{http://en.wikipedia.org/wiki/1807}{1807} landmark paper
    \href{http://gallica.bnf.fr/ark:/12148/bpt6k33707/f220n7}
          {\hie{Dissertation on the propagation of heat in solid bodies}}
    that introduced the \hie{Fourier Series}.
    \footnotemark
  }
  {../common/people/fourier1807z.jpg}
  %{ \ldots et la nouveaut\'e du sujet, jointe \`a son importance,
  { \ldots et la nouveaut\'e de l'objet, jointe \`a son importance,
    a d\'etermin\'e la classe \`a couronner cet ouvrage,
    en observant cependant que la mani\`ere dont l`auteur parvient
    \`a ses \'equations n`est pas exempte de difficult\'es, et que son analyse,
    pour les int\'egrer, laisse encore quelque chose \`a d\'esirer,
    soit relativement \`a la g\'en\'eralit\'e, soit m\^eme du cot\'e de la rigueur.}
  { \ldots and the innovation of the subject,
    together with its importance,
    convinced the committee to crown this work.
    %By observing however that the way in which the author [Fourier] arrives at his equations
    By observing however that the way in which the author arrives at his equations
    is not free from difficulties, and the analysis of which,
    to integrate them, still leaves something to be desired,
    either relative to generality, or even on the side of rigour.}
  \footnotetext{\begin{tabular}[t]{lp{\tw-31mm}}
%   quote:       & \url{http://www.todayinsci.com/F/Fourier_JBJ/fourglos.htm#commenx} \\
%    quote:       & \url{http://www.todayinsci.com/F/Fourier_JBJ/fourglos.htm} \\
    quote:       & \citerp{mdf1812jan6}{374},
                   \citerp{edf1812jan6}{112},
                   \citerpg{kahane2008}{199}{0821844245}
                   \tabularnewline
    translation: & assisted by \href{http://translate.google.com/}{Google Translate},
                   \citerc{castanedo2005}{chapter 2 footnote 5}
%                & \url{http://www.theses.ulaval.ca/2005/23016/ch03.html#ftn.N11235} \\
                   \tabularnewline
%    image: & \url{http://en.wikipedia.org/wiki/Joseph_Fourier}
  paper: & \citer{fourier1807}%
  \end{tabular}%
  }

%=======================================
\section{Definition}
%=======================================
The \ope{Fourier Series} expansion of a periodic function
is simply a complex trigonometric polynomial.
In the special case that the periodic function is even,
then the Fourier Series expansion is a cosine polynomial.

%--------------------------------------
\begin{definition}
\footnote{
  \citerpg{katznelson2004}{3}{0521543592}
  }
\label{def:fs}
\label{def:opFS}
%--------------------------------------
\defbox{\begin{array}{M}\indxs{\opFS}
  The \opd{Fourier Series operator} $\opFS:\spLLR\to\spllR$ is defined as
  \\\indentx$\ds
    \brs{\opFS \ff}(n)
    \eqd
    \fsscale \int_0^\tau \ff(x) \fkernea{x}{\frac{2\pi}{\tau}n} \dx
    \qquad\forall\ff\in\set{\ff\in\spLLR}{\text{$\ff$ is periodic with period $\tau$}}
  $
\end{array}}
\end{definition}

%=======================================
\section{Inverse Fourier Series operator}
%=======================================
%--------------------------------------
\begin{theorem}
\label{thm:opFSi}
%--------------------------------------
Let $\opFS$ be the Fourier Series operator.
\thmbox{\begin{array}{M}\indxs{\opFSi}
  The \opd{inverse Fourier Series} operator $\opFSi$ is given by
  \\\indentx$\ds
  \brs{\opFSi\seqnZ{\Fx_n}}(x) \eqd \fsscalei \sum_{n\in\Z} \Fx_n \fkerne{x}{\frac{2\pi}{\tau}n}
  \qquad\scy\forall\seqn{\Fx_n}\in\spllR
  $
\end{array}}
\end{theorem}
\begin{proof}
The proof of the pointwise convergence of the Fourier Series is notoriously difficult.
It was conjectured in 1913 by Nokolai Luzin that the Fourier Series for all square summable periodic functions
are pointwise convergent:
\citePbullet\quad\fullcite{luzin1913}

Fifty-three years later (1966) at a conference in Moscow, Lennart Axel Edvard Carleson presented one of the most spectacular
results ever in mathematics; he demonstrated that the Luzin conjecture is indeed correct.
Carleson formally published his result that same year:
\citePbullet\quad\fullcite{carleson1966}

Carleson's proof is expounded upon in Reyna's (2002) 175 page book:
\citebullet\quad\fullcite{reyna2002}

Interestingly enough, Carleson started out trying to disprove Luzin's conjecture.
Carleson said this in an interview published in 2001:\footnote{
  \citer{carleson2001}
  }
\quotetext{%
  Well, the problem of course presents itself already when you are a student
  and I was thinking of the problem on and off,
  but the situation was more interesting than that.
  The great authority in those days was Zygmund and he was completely convinced
  that what one should produce was not a proof but a counter-example.
  When I was a young student in the United States,
  I met Zygmund and I had an idea how to  produce some very complicated functions for a counter-example
  and Zygmund encouraged me very much to do so.
  I was thinking about it for about 15 years on and off,
  on how to make these counter-examples work and the interesting thing that happened
  was that I suddenly realized why there should be a counter-example and how you should produce it.
  I thought I really understood what was the background and then to my amazement I could prove that this
  ``correct" counter-example couldn't exist and therefore I suddenly realized that what you should try to do was the opposite,
  you should try to prove what was not fashionable, namely to prove convergence.
  The most important aspect in solving a mathematical problem is the conviction of what is the true result!
  Then it took like 2 or 3 years using the technique that had been developed during the past 20 years or so.
  It is actually a problem related to analytic functions basically even though it doesn't look that way.%
  }

For now, if you just want some intuitive justification for the Fourier Series,
and you can somehow imagine that the Dirichlet kernel generates a \fncte{comb function} of \fncte{Dirac delta} functions,
then perhaps what follows may help (or not). It is certainly not mathematically rigorous and is by
no means a real proof (but at least it is less than 175 pages).

\begin{align*}
     \brs{\opFSi \opFS \fx}(x)
       &=    \opFSi \mcom{\brs{\fsscale \int_0^\tau \fx(x) \fkernea{x}{\frac{2\pi}{\tau}n} \dx}}
                         {$\opFS \fx$}
       &&    \text{by definition of $\opFS$}
       &&    \text{\xref{def:fs}}
     \\&=    \fsscalei \sum_{n\in\Z} \brs{\fsscale\int_0^\tau \fx(u) \fkernea{u}{\frac{2\pi}{\tau}n} \du }
             \fkerne{x}{\frac{2\pi}{\tau}n}
       &&    \text{by definition of $\opFSi$}
     \\&=    \fsscalei \sum_{n\in\Z} \fsscale\int_0^\tau \fx(u) \fkernea{u}{\frac{2\pi}{\tau}n} \fkerne{x}{\frac{2\pi}{\tau}n}\du
     \\&=    \fsscalei \sum_{n\in\Z} \fsscale\int_0^\tau \fx(u) \fkerne{(x-u)}{\frac{2\pi}{\tau}n} \du
     \\&=    \int_0^\tau \fx(u)
             \mcom{\frac{1}{\tau}\sum_{n\in\Z} \fkerne{(x-u)}{\frac{2\pi}{\tau}n}}
                  {$\lim_{N\to\infty}\Dn(x)$}
             \du
     \\&=    \int_0^\tau \fx(u) \brs{\sum_{n\in\Z} \delta(x-u-n\tau)} \du
       %&&    \text{by \prefp{prop:dkernel}}
     \\&=    \sum_{n\in\Z} \int_{u=0}^{u=\tau} \fx(u) \delta(x-u-n\tau)\du
     \\&=    \sum_{n\in\Z} \int_{v-n\tau=0}^{v-n\tau=\tau} \fx(v-n\tau) \delta(x-v)\dv
       &&    \text{where $v\eqd u+n\tau$}
     \\&=    \sum_{n\in\Z} \int_{v=n\tau}^{v=(n+1)\tau} \fx(v-n\tau) \delta(x-v)\dv
       &&    \text{where $v\eqd u+n\tau$}
     \\&=    \sum_{n\in\Z} \int_{v=n\tau}^{v=(n+1)\tau} \fx(v) \delta(x-v)\dv
       &&    \text{because $\fx$ is periodic with period $\tau$}
     \\&=    \int_{\R} \fx(v) \delta(x-v)\dv
     \\&=    \fx(x)
     \\&=    \opI \Fx(n)
       &&    \text{by definition of $\opI$}
       &&    \text{\ifxref{operator}{def:opI}}
     \\
     \\
     \brs{\opFS \opFSi \Fx}(n)
       &=    \opFS \brs{ \fsscalei\sum_{k\in\Z} \Fx(k) \fkerne{x}{\frac{2\pi}{\tau}k}  }
       &&    \text{by definition of $\opFSi$}
     \\&=    \fsscale\int_0^\tau \brs{ \fsscalei\sum_{k\in\Z} \Fx(k) \fkerne{x}{\frac{2\pi}{\tau}k}  } \fkernea{x}{\frac{2\pi}{\tau}n} \dx
       &&    \text{by definition of $\opFS$}
       &&    \text{\xref{def:fs}}
     \\&=    \frac{1}{\tau} \int_0^\tau \brs{ \sum_{k\in\Z} \Fx(k) \fkerne{x}{\frac{2\pi}{\tau}(k-n)}  }  \dx
     \\&=    \sum_{k\in\Z} \Fx(k) \brs{ \frac{1}{\tau}  \int_0^\tau \fkerne{x}{\frac{2\pi}{\tau}(k-n)} \dx }
     \\&=    \sum_{k\in\Z} \Fx(k) \frac{1}{\tau} \brs{  \frac{1}{i\frac{2\pi}{\tau}(k-n)}  \fkerne{x}{\frac{2\pi}{\tau}(k-n)}}_0^\tau
     \\&=    \sum_{k\in\Z} \Fx(k) \frac{1}{i2\pi(k-n)}\brs{ \fkerne{}{2\pi(k-n)}-1  }
     \\&=    \sum_{k\in\Z} \Fx(k) \, \kdelta(k-n)\, \lim_{x\to0}\brs{\frac{\fkerne{}{2\pi x}-1 }{i2\pi x}}
     \\&=    \Fx(n) \, \left.\frac{\opddx\left(\fkerne{}{2\pi x}-1\right) }{\opddx(i2\pi x)}\right|_{x=0}
       &&    \text{by \thme{l'H{/<o}pital's rule}}
     \\&=    \Fx(n) \left.\frac{i2\pi\fkerne{}{2\pi x}}{i2\pi }\right|_{x=0}
     \\&=    \Fx(n)
     \\&=    \opI \Fx(n)
       &&    \text{by definition of $\opI$}
       &&    \text{\ifxref{operator}{def:opI}}
\end{align*}
\end{proof}

%--------------------------------------
\begin{theorem}
\label{thm:opFSa}
%--------------------------------------
\thmbox{\begin{array}{M}\indxs{\opFSa}
  The \opd{Fourier Series adjoint} operator $\opFSa$ is given by
  \\\indentx$\ds \opFSa = \opFSi $
\end{array}}
\end{theorem}
\begin{proof}
\begin{align*}
  \inprod{ \opFS \fx(x) }{ \Fy(n) }_{\Z}
    &= \inprod{ \fsscale \int_0^\tau \fx(x) \fkernea{x}{\frac{2\pi}{\tau}n} \dx }{ \Fy(n) }_{\Z}
    && \text{by definition of $\opFS$}
    && \text{\xref{def:fs}}
  \\&= \fsscale \int_0^\tau \fx(x) \inprod{ \fkernea{x}{\frac{2\pi}{\tau}n} }{ \Fy(n) }_{\Z}  \dx
    && \text{by additivity property of $\inprodn$}
    && \text{\ifxref{vsinprod}{def:inprod}}
  \\&= \int_0^\tau \fx(x) \fsscale \inprod{ \Fy(n) }{ \fkernea{x}{\frac{2\pi}{\tau}n} }_{\Z}^\ast  \dx
    && \text{by property of $\inprodn$}
    && \text{\ifxref{vsinprod}{def:inprod}}
  \\&= \int_0^\tau \fx(x) \brs{\opFSi\Fy(n)}^\ast  \dx
    && \text{by definition of $\opFSi$}
    && \text{\xref{thm:opFSi}}
  \\&= \inprod{\fx(x)}{\mcom{\opFSi}{$\opFSa$}\Fy(n)}_{\R}
\end{align*}
\end{proof}

The Fourier Series operator has several nice properties:
\begin{liste}
  \item $\opFS$ is \prope{unitary}
    \ifdochas{operator}{\footnote{{\em unitary operators}: \prefp{def:op_unitary}}}
    (\prefp{cor:fs_unitary}).
  \item Because $\opFS$ is unitary, it automatically has several other nice
        properties such as being \prope{isometric}, and satisfying
        \fncte{Parseval's equation}, satisfying \fncte{Plancheral's formula}, and more
        \xref{cor:fs_prop}.
\end{liste}

%---------------------------------------
\begin{corollary}
\label{cor:fs_unitary}
\index{operator!unitary}  \index{unitary operator}
%---------------------------------------
Let $\opI$ be the identity operator
and let $\opFS$ be the Fourier Series operator with
adjoint $\opFSa$.
\corbox{
 \brb{\opFS\opFSa = \opFSa\opFS = \opI}
 \qquad \brp{\begin{array}{M}
   $\opFS$ is \propb{unitary} \ldots
   and thus also \prope{normal} and \prope{isometric}
 \end{array}}
 }
\end{corollary}
\begin{proof}
This follows directly from the fact that $\opFSa=\opFSi$ \xref{thm:opFSa}.
\end{proof}

%---------------------------------------
\begin{corollary}
\label{cor:fs_prop}
\index{operator!unitary}  \index{unitary operator}
%---------------------------------------
Let $\opFS$ be the Fourier series operator with adjoint $\opFSa$ and inverse $\opFSi$.
\corbox{\begin{array}{rcl cl D}
  \oppR(\opFS)                  &=&  \oppR(\opFSi)                   &=&    \spLLR            & \\
  %\oppN(\opFS)                  &=&  \oppN(\opFSi)                   &\eqq& \setn{0}          & \\
  \normop{\opFS   }             &=&  \normop{\opFSi}                 &=&    1                 & (\prope{unitary})              \\
  \inprod{\opFS \vx}{\opFS \vy} &=&  \inprod{\opFSi \vx}{\opFSi \vy} &=&    \inprod{\vx}{\vy} & (\fncte{Parseval's equation})  \\
  \norm{\opFS  \vx}             &=&  \norm{\opFSi \vx}               &=&    \norm{\vx}        & (\fncte{Plancherel's formula}) \\
  \norm{\opFS \vx-\opFS \vy}    &=&  \norm{\opFSi \vx-\opFSi \vy}    &=&    \norm{\vx-\vy}    & (\prope{isometric})
\end{array}}
\end{corollary}
\begin{proof}
These results follow directly from the fact that $\opFS$
is unitary \xref{cor:fs_unitary} and from
the properties of unitary operators\ifsxref{operator}{thm:unitary_prop}.
\end{proof}

%=======================================
\section{Fourier series for compactly supported functions}
%=======================================
%--------------------------------------
\begin{theorem}
\label{thm:opFSc}
%--------------------------------------
\thmboxt{
  The set
  \\\indentx$\ds
    \setxZ{\fsscalei \fkerne{x}{\frac{2\pi}{\tau}n}}$
  \\
  is an \structe{orthonormal basis} for all functions $\ff(x)$ with support in $\intcc{0}{\tau}$.
  }
\end{theorem}
