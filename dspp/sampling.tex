%============================================================================
% XeLaTeX File
% Daniel J. Greenhoe
%============================================================================
%---------------------------------------
\chapter{Sampling}
%---------------------------------------
%%--------------------------------------
%\section*{Functions}
%%--------------------------------------
%\vfill
%\begin{minipage}{\tw-10mm}%
%{\psset{unit=20mm}
%\psset{fillstyle=solid,linewidth=3pt}%
%  \begin{pspicture}(-1,-2)(9,2.8)
%    \psellipse[linecolor=set](0,0)(1,2)%
%    \psellipse[linecolor=set](4,0)(1,2)%
%    \psellipse[linecolor=set](8,0)(1,2)%
%    \psdot(0, 1)% 1
%    \psdot(0, 0)% 2
%    \psdot(0,-1)% 3
%    \psdot(4, 1)% A
%    \psdot(4,-1)% B
%    \psdot(8, 1)% 1
%    \psdot(8, 0)% 2
%    \psdot(8,-1)% 3
%    \psline[linecolor=blue] {->}(0, 1)(4, 1)%                (3,A)
%    \psline[linecolor=red]  {->}(0, 0)(4, 1)%                (2,A)
%    %\psline[linecolor=red]  {->}(0, 0)(4,-1)%                (2,B)
%    \psline[linecolor=green]{->}(0,-1)(4,-1)%                (1,B)
%    %\psline[linecolor=blue] {->}(4, 1)(8, 1)%                (A,1)
%    \psline[linecolor=red]  {->}(4, 1)(8, 0)%                (A,2)
%    %\psline[linecolor=red]  {->}(4,-1)(8, 0)%                (B,2)
%    \psline[linecolor=green]{->}(4,-1)(8,-1)%                (B,1)
%    \uput [90]{0}(0, 2){$\setX$}
%    \uput [90]{0}(4, 2){$\setY$}
%    \uput [90]{0}(8, 2){$\setX$}
%    \uput [90]{0}(2, 2){$\rel$}
%    \uput [90]{0}(6, 2){$\reli$}
%    \uput[180]{0}(-0, 1){$1$}
%    \uput[180]{0}(-0, 0){$2$}
%    \uput[180]{0}(-0,-1){$3$}
%    \uput[270]{0}(4, 1){$A$}
%    \uput [90]{0}(4,-1){$B$}
%    \uput  [0]{0}(8, 1){$1$}
%    \uput  [0]{0}(8, 0){$2$}
%    \uput  [0]{0}(8,-1){$3$}
%    {\scriptsize
%    \uput[90] {0}(2, 1)    {$\opair{1}{A}$}
%    \uput[-22.5]{0}(2, 0.5)  {$\opair{2}{A}$}
%    \uput[22.5] {0}(2,-0.5)  {$\opair{2}{B}$}
%    \uput[270]{0}(2,-1)    {$\opair{3}{B}$}
%    \uput[90] {0}(6,   1)  {$\opair{A}{1}$}
%    \uput[202.5]{0}(6, 0.5){$\opair{A}{2}$}
%    \uput[115.5]{0}(6,-0.5){$\opair{B}{2}$}
%    \uput[270]{0}(6,  -1)  {$\opair{B}{3}$}
%    }
%    %\psgrid[gridcolor=green,subgridcolor=green](-1,-2)(9,3)
%  \end{pspicture}%
%}%
%\end{minipage}%
%\vfill\mbox{}



%--------------------------------------
%\section*{Mathematical spaces}
%--------------------------------------
%  %============================================================================
% Daniel J. Greenhoe
% LaTeX file
%============================================================================
\begin{pspicture}(-7,0.5)(5,9)%
  %-------------------------------------
  % settings
  %-------------------------------------
   \psset{%
    arrowsize=4mm,
    arrowlength=0.6,
    arrowinset=0.1,
    %linecolor=blue,
    %linewidth=1pt,
     cornersize=relative,
     framearc=0.25,
    % gridcolor=graph,
    % subgriddiv=1,
    % gridlabels=4pt,
    % gridwidth=0.2pt,
    xunit=1\latunit,
    yunit=1.25\latunit,
     }%
  %-------------------------------------
  % nodes
  %-------------------------------------
   \begin{tabstr}{0.75}%
     \rput(-1, 8){\rnode{spaces}    {\psframebox{\begin{tabular}{c}abstract spaces\end{tabular}}}}%
     \rput(-4, 7){\rnode{lin}       {\psframebox{\begin{tabular}{c}linear spaces\ifnxref{vector}{def:vspace}\end{tabular}}}}%
     \rput( 2, 7){\rnode{top}       {\psframebox{\begin{tabular}{c}topological spaces\ifnxref{vstopo}{def:toplinspace}\end{tabular}}}}%
     \rput( 2, 6){\rnode{metric}    {\psframebox{\begin{tabular}{c}metric spaces\ifnxref{metric}{def:metric}\end{tabular}}}}%
     \rput( 2, 4.5){\rnode{commetric} {\psframebox{\begin{tabular}{c}complete metric spaces\ifnxref{seq}{def:ms_complete}\end{tabular}}}}%
     \rput(-4, 5){\rnode{metriclin} {\psframebox{\begin{tabular}{c}metric linear spaces\end{tabular}}}}%
     \rput(-4, 4){\rnode{normlin}   {\psframebox{\begin{tabular}{c}normed linear spaces\ifnxref{vsnorm}{def:norm}\end{tabular}}}}%
     \rput( 2, 3){\rnode{banach}    {\psframebox{\begin{tabular}{c}Banach spaces\ifnxref{seq}{def:banach}\end{tabular}}}}%
     \rput(-4, 3){\rnode{inprod}    {\psframebox{\begin{tabular}{c}inner-product spaces\ifnxref{vsinprod}{def:inprod}\end{tabular}}}}%
     \rput( 2, 2){\rnode{hilbert}   {\psframebox{\begin{tabular}{c}Hilbert spaces\ifnxref{seq}{def:hilbert}\end{tabular}}}}%
     \rput( 2, 1){\rnode{zero}      {\psframebox{\begin{tabular}{c}$\spZero$\ifnxref{subspace}{prop:subspace_0X}\end{tabular}}}}%
   \end{tabstr}%
  %-------------------------------------
  % connecting lines/arrows
  %-------------------------------------
   %\ncline[doubleline=true]{->}{lin}{spaces} double arrow seems to cause trouble for xdvipdfmx
   \ncline{lin}{spaces}
   \ncline{top}{spaces}%
   \ncline{com}{spaces}%
   \ncline{metriclin}{lin}%
   \ncline{normlin}{metriclin}%
   \ncline{inprod}{normlin}
   \ncline{banach}{normlin}%
   \ncline{hilbert}{inprod}%
   \ncline{hilbert}{banach}%
   \ncline{zero}{hilbert}%
   \ncline{metric}{top}%
   \ncline{metriclin}{metric}
   \ncline{commetric}{metric}%
   \ncline{banach}{commetric}%
  %-------------------------------------
  % labeling
  %-------------------------------------
   %\psccurve[linestyle=dashed,linecolor=red,fillstyle=none]%
   %  (0,-5)(20,4)(70,60)(20,60)(15,55)(-5,38)(-20,35)(-10,20)(-25,5)%
   %\psline[linecolor=red]{->}(60,75)(60,68)%
   %\uput[135](60,75){complete spaces}%
   %\psline[linecolor=red]{->}(26,80)(15,74)%
   %\psline[linecolor=red]{->}(30,80)(30,62.5)%
   %\uput[90](28,80){analytic spaces}%
   %
   %\psgrid[unit=10mm](-8,-1)(8,9)%
\end{pspicture}%


%--------------------------------------
\section*{Bases}
%--------------------------------------
\includegraphics[width=\tw-50mm]{graphics/baslat_cose.pdf}  \newpage
\includegraphics[width=\tw-50mm]{graphics/baslat_cheby.pdf} \newpage
\includegraphics[width=\tw-50mm]{graphics/baslat_cosh.pdf}  \newpage
\includegraphics[width=\tw-60mm]{graphics/baslat_h3.pdf}    \newpage
\includegraphics[width=\tw-130mm]{graphics/baslat_d1.pdf}    \newpage
\includegraphics[width=\tw-130mm]{graphics/baslat_d2.pdf}    \newpage


%--------------------------------------
\section*{Translation and Dilation}
%--------------------------------------
\includegraphics[width=\tw/2-10mm]{graphics/opTrn.pdf}\hspace{20mm}
\includegraphics[width=\tw/2-10mm]{graphics/opDil.pdf}

%--------------------------------------
\section*{Sines and Cosines}
%--------------------------------------
      \[\begin{array}{rcl @{\qquad}c@{\qquad} lll}
        \cos(x) &\eqd& \ff(x)
          & \text{where} 
          & \mcom{\brn{\opDiff\ff}+ \ff=0}{differential equation}  
          & \mcom{\ff(0)=1           }{1st initial condition}
          & \mcom{\brs{\opDif\ff}(0)=0 }{2nd initial condition}
        \\
        \sin(x) &\eqd& \fg(x)
          & \text{where} 
          & \mcom{\brn{\opDiff\fg}+ \fg=0}{differential equation}  
          & \mcom{\fg(0)=0           }{1st initial condition}
          & \mcom{\brs{\opDif\fg}(0)=1 }{2nd initial condition}
      \end{array}\]

%--------------------------------------
\section*{Sampling}
%--------------------------------------
\includegraphics[width=\tw-50mm]{graphics/sin_pi_t.pdf}\newpage
\includegraphics[width=\tw-50mm]{graphics/haar0_sin_t.pdf}


  \begin{tabular}{|l|l|l|}
    \hline
    \mc{1}{|c|}{subspace}&\mc{1}{c|}{transform}&\mc{1}{c|}{approximation}
    \\\hline\hline
    $\spV_0$
    & \includegraphics[height=36mm]{graphics/haar0_sin_t.pdf}
    & \includegraphics[height=36mm]{graphics/haar0_sin_a.pdf}
    \\\hline
    $\spV_1$
    & \includegraphics[height=36mm]{graphics/haar1_sin_t.pdf}
    & \includegraphics[height=36mm]{graphics/haar1_sin_a.pdf}
    \\\hline
    $\spV_2$
    & \includegraphics[height=36mm]{graphics/haar2_sin_t.pdf}
    & \includegraphics[height=36mm]{graphics/haar2_sin_a.pdf}
    \\\hline
  \end{tabular}



%--------------------------------------
\section*{Poles and zeros}
%--------------------------------------

  \includegraphics[height=135mm]{graphics/vecres.pdf}



  \begin{tabular*}{\textwidth-40mm}{@{\extracolsep{\fill}}cc}%
    \includegraphics[width=\tw/2-10mm]{graphics/bspline_pz.pdf}&\includegraphics[width=\tw/2-10mm]{graphics/Dp_pz.pdf}\\%
    Cardinal B-spline&Daubechies-p%
  \end{tabular*}%
\newpage
  \exbox{\begin{tabular}{cc}
     $\begin{array}{l|r|r}
      n & h_n  & g_n  \\
      \hline
        0   &  0.2303778133 & -0.0105974018 \\
        1   &  0.7148465706 & -0.0328830117 \\
        2   &  0.6308807679 &  0.0308413818 \\
        3   & -0.0279837694 &  0.1870348117 \\
        4   & -0.1870348117 & -0.0279837694 \\
        5   &  0.0308413818 & -0.6308807679 \\
        6   &  0.0328830117 &  0.7148465706 \\
        7   & -0.0105974018 & -0.2303778133
    \end{array}$&\tbox{\includegraphics{graphics/D4_pz.pdf}}\\
    \includegraphics[height=50mm]{graphics/d4_phi_h.pdf}&\includegraphics[height=50mm]{graphics/d4_psi_g.pdf}
  \end{tabular}}


\newpage
\qboxnpqt
  {
    Leonhard Euler (1707--1783), mathematician
    \index{Euler, Leonhard}
    \index{quotes!Euler, Leonhard}
    \footnotemark
  }
  %{../common/people/euler.jpg}
  {../common/people/euler1753bw_handmann_wkp_pdomain.jpg}
  {Weil nun alle m\"oglichen Zahlen, die man sich nur immer vorstellen mag,
    entweder gr\"o{\ss}er oder kleiner als 0, oder etwa 0 selbst sind, so ist klar, da{\ss}
    die Quadratwurzeln von Negativzahlen nicht einmal zu den m\"oglichen Zahlen
    gerechnet werden k\"onnen.
    Folglich m\"ussen wir sagen, da{\ss} dies unm\"ogliche Zahlen sind.
    Und dieser Umstand leitet uns auf den Begri{\ss} von solchen Zahlen,
    welche ihrer Natur nach unm\"oglich sind, und gew\"ohnlich imagin\"are
    oder eingebildete Zahlen genannt werden, weil sie blo\ss in der Einbildung
    vorhanden sind.}
  {And, since all numbers which it is possible to conceive,
    are either greater or less than 0, or are 0 itself,
    it is evident that we cannot rank the square root of a negative
    number amongst possible numbers, and we must therefore say that
    it is an \prop{impossible quantity}.
    In this manner we are led to the idea of numbers,
    which from their nature are impossible;
    and therefore they are usually called \hie{imaginary quantities},
    because they exist merely in the imagination.}
  \footnotetext{\begin{tabular}[t]{ll}
    quote: & \citorp{euler1770}{60} \\
    %www.math.uni-konstanz.de/~hoffmann/Funktionentheorie/kap1.pdf
    %www.springer.com/cda/content/document/cda_downloaddocument/9783540435549-c1.pdf?SGWID=0-0-45-130631-p2253211
    translation: & \citorp{euler1770e}{43} \\
    %image: & \url{http://en.wikipedia.org/wiki/Image:Leonhard_Euler.jpg}
    image: & \url{http://en.wikipedia.org/wiki/File:Leonhard_Euler.jpg}, public domain
  \end{tabular}}



