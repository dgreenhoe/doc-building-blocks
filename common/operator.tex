%============================================================================
% LaTeX File
% Daniel J. Greenhoe
%============================================================================

%======================================
\chapter{Operators on Linear Spaces}
\label{chp:operator}
%======================================
\qboxnps
  {\href{http://en.wikipedia.org/wiki/Gottfried_Leibniz}{Gottfried Leibniz}
   \href{http://www-history.mcs.st-andrews.ac.uk/Timelines/TimelineC.html}{(1646--1716)},
   \href{http://www-history.mcs.st-andrews.ac.uk/BirthplaceMaps/Places/Germany.html}{German} mathematician,
   in a September 8, 1679 letter to
   \href{http://en.wikipedia.org/wiki/Christian_Huygens}{Christian Huygens}.
   \index{Leibniz, Gottfried}
   \index{quotes!Leibniz, Gottfried}
   \footnotemark
  }
  {../common/people/leibniz_wkp_pdomain_bw.jpg}
  {And I am not afraid to say that
   there is a way to advance algebra as far beyond what Vieta and Descartes have left us
   as Vieta and Descartes carried it beyond the ancients.\ldots
   we need still another analysis which is distinctly geometrical or linear,
   and which will express {\em situation} directly as algebra expresses {\em magnitude} directly.}
  \citetblt{
    quote: & \citorppg{leibniz_huygens1679}{248}{249}{902770693X} \\
   %quote: & \citerpp{leibniz_letters}{248}{249} \\
          %&\citerpp{leibniz_letters}{248}{249} \\
          %   \cithrp{fearnley-sander}{1}
    image: & \scs\url{http://en.wikipedia.org/wiki/File:Gottfried_Wilhelm_von_Leibniz.jpg}, public domain
    }




%=======================================
\section{Operators on linear spaces}
%=======================================
\begin{figure}[t]
  \begin{center}
  %============================================================================
% Daniel J. Greenhoe
% LaTeX file
% lattice ({factors of 30}, |)
%============================================================================


  \psset{xunit=1.2mm,yunit=1.5mm}
  \begin{pspicture}(-60,-24)(60,32)%
     \footnotesize
     \psset{%
       linecolor=blue,
       cornersize=relative,
       framearc=0.25,
       gridcolor=graph,
       subgriddiv=1,
       gridlabels=4pt,
       gridwidth=0.2pt,
       }%
     \begin{tabstr}{0.75}
     \rput(  0, 30){\rnode{relations} {\psframebox{\begin{tabular}{c}relations\end{tabular}}}}%
     \rput(  0, 20){\rnode{functions} {\psframebox{\begin{tabular}{c}functions\end{tabular}}}}%
     \rput(  0, 10){\rnode{op}        {\psframebox{\begin{tabular}{c}operators\\\ifxref{operator}{def:operator}\end{tabular}}}}%
     \rput( 40,  0){\rnode{nonlinop}  {\psframebox{\begin{tabular}{c}non-linear   operators\end{tabular}}}}%
     \rput(  0,  0){\rnode{linop}     {\psframebox{\begin{tabular}{c}linear       operators\\\ifxref{operator}{def:linop}\end{tabular}}}}%
     \rput(-40,-10){\rnode{opN}       {\psframebox{\begin{tabular}{c}normal       operators\\$\opNa\opN=\opN\opNa$\\\ifxref{operator}{def:op_normal}\end{tabular}}}}%
     \rput(  0,-10){\rnode{opP}       {\psframebox{\begin{tabular}{c}projection   operators\\$\opP^2=\opP$\\\ifxref{operator}{def:opP}\end{tabular}}}}%
     \rput( 40,-10){\rnode{opIso}     {\psframebox{\begin{tabular}{c}isometric    operators\\$\opAa\opA=\opI$\\\ifxref{operator}{def:op_isometric}\end{tabular}}}}%
     \rput(-40,-20){\rnode{opSA}      {\psframebox{\begin{tabular}{c}self adjoint operators\\$\opA=\opAa$\\\ifxref{operator}{def:op_selfadj}\end{tabular}}}}%
     \rput(  0,-20){\rnode{opU}       {\psframebox{\begin{tabular}{c}unitary      operators\\$\opU\opUa=\opUa\opU=\opI$\\\ifxref{operator}{def:op_unitary}\end{tabular}}}}%
     \end{tabstr}
     %
     \ncline{relations}{functions}%
     \ncline{functions}{op}%
     \ncline{op}{linop}%
     \ncline{op}{nonlinop}%
     \ncline{linop}{opN}%
     \ncline{linop}{opP}%
     \ncline{linop}{opIso}%
     \ncline{opN}  {opSA}%
     \ncline{opU}  {opN}%
     \ncline{opU}  {opIso}%
     %
     %\psgrid[unit=10mm](-8,-1)(8,9)%
  \end{pspicture}

  \caption{Some operator types\label{fig:operator_oplat}}
  \end{center}
\end{figure}%

%=======================================
\subsection{Operator Algebra}
%=======================================
An operator is simply a function that maps from a linear space to another linear space (or to the same linear space).

\ifdochasnot{vector}{
%---------------------------------------
\begin{definition}
\footnote{
  \citerppgc{kubrusly2001}{40}{41}{0817641742}{Definition 2.1 and following remarks},
  \citerp{haaser1991}{41},
  \citerpp{halmos1948}{1}{2},
  \citerc{peano1888}{Chapter IX},
  \citerpp{peano1888e}{119}{120},
  \citePpp{banach1922}{134}{135}
  }
\label{def:vspace}
\index{space!vector}
\index{space!linear}
%---------------------------------------
%Let $\F\eqd\otriple{\setS}{+}{\cdot}$ be a field.
Let $\fieldF$ be a \structe{field}\ifsxref{algebra}{def:field}.
Let $\setX$ be a set, 
let $+$ be an \structe{operator} \xref{def:operator} in $\clF{\setX^2}{\setX}$, 
and let $\otimes$ be an operator in $\clF{\F\times\setX}{\setX}$.
\defboxp{
  The structure $\spO\eqd\linearspaceX$ is a \hid{linear space} over $\fieldF$ if
  %\\\indentx$\ds\begin{array}{l rcl @{\quad}C @{\quad}D@{}r@{}}
  \\\indentx$\ds\begin{array}{>{\scriptstyle}r rcl @{\quad}C @{\quad}D@{}r@{}}
    \cline{7-7}
    1.& \exists \vzero\in\setX \st \vx + \vzero &=& \vx
      & \forall \vx\in\setX
      & ($+$ \structe{identity})
      & \ast\vline
      \\
    2.& \exists \vy\in\setX \st \vx+\vy &=& \vzero
      & \forall \vx \in\setX
      & ($+$ \structe{inverse})
      & \vline
      \\
    3.& (\vx+\vy)+\vz &=& \vx+(\vy+\vz)
      & \forall \vx,\vy,\vz\in\setX
      & ($+$ is \prope{associative})
      & \text{ }\vline
      \\
    4.& \vx+\vy &=& \vy+\vx
      & \forall \vx,\vy\in\setX
      & ($+$ is \prope{commutative})
      & \vline
      \\\cline{7-7}
    5.& 1\cdot \vx &=& \vx
      & \forall \vx\in\setX
      & ($\cdot$ \prope{identity})
      \\
    6.& \alpha\cdot(\beta\cdot\vx) &=& (\alpha\cdot\beta)\cdot\vx
      & \forall \alpha,\beta\in\setS \text{ and } \vx\in\setX
      & ($\cdot$ \prope{associates} with $\cdot$)
      \\
    7.& \alpha\cdot(\vx+\vy) &=& (\alpha \cdot\vx)+(\alpha\cdot\vy)
      & \forall \alpha\in\setS \text{ and } \vx,\vy\in\setX
      & ($\cdot$ \prope{distributes} over $+$)
      \\
    8.& (\alpha+\beta)\cdot\vx &=& (\alpha\cdot \vx)+(\beta\cdot \vx)
      & \forall \alpha,\beta\in\setS \text{ and } \vx\in\setX
      & ($\cdot$ \prope{pseudo-distributes} over $+$)
  \end{array}$
  \\
  The set $\setX$ is called the \structd{underlying set}.
  The elements of $\setX$ are called \structd{vectors}.
  The elements of $\F$ are called \structd{scalars}.
  A linear space is also called a \structd{vector space}.
  If $\F\eqd\R$, then $\spO$ is a \structd{real linear space}.
  If $\F\eqd\C$, then $\spO$ is a \structd{complex linear space}.
  }
\end{definition}
}

%---------------------------------------
\begin{definition}
\footnote{
  \citerpg{heil2011}{42}{0817646868}
  }
\label{def:operator}
\index{operator!definition}
%---------------------------------------
%Let $\spX\eqd\linearspaceX$ and $\spY\eqd\linearspaceY$ be linear spaces.~%
%Let $\setX$ and $\setY$ be sets.
\defboxt{
  A function $\opA$ in $\clOxy$ is an \hid{operator} in $\clOxy$ if\\
  $\spX$ and $\spY$ are both \structe{linear spaces} \xref{def:vspace}.\\ %\footnotemark\\
    %1. & $\opA$ is a \structe{function} in $\clFxy$                               & and\\
    %2. & There exists a linear space $\spX\eqd\linearspaceX$ & and\\
    %3. & There exists a linear space $\spY\eqd\linearspaceY$. &\\
  %The set of all operators from $\spX$ to $\spY$ is denoted $\hxs{\clOxy}$.
  %\mc{3}{M}{If $\spY=\F$, then $\opA$ is a \hid{functional}.}
  }
%\footnotetext{\begin{tabular}[t]{ll}
%  \structe{function}:     & \prefp{def:rel_f}\\
%  \structe{linear space}: & \prefp{def:vspace}
%\end{tabular}}
\end{definition}

Two operators $\opA$ and $\opB$ in $\clOxy$ are \hid{equal} if 
$\ds \opA\vx=\opB\vx$ for all $\vx\in\spX$\ifsxref{relation}{def:f=g}.
The inverse relation\ifsxref{relation}{def:rel_inverse} of an operator $\opA$ in $\clOxy$
always exists as a \hie{relation} in $\clR{\spX}{\spY}$\ifsxref{relation}{def:relation},
but may not always be a \hie{function} (may not always be an operator)
in $\clOxy$\ifsxref{relation}{def:rel_f}.


The operator $\opI\in\clOxx$ is the \hie{identity} operator if $\opI\vx=\opI$ for all $\vx\in\spX$.
%(see next definition and also\ifsxref{relation}{ex:opI}).
%---------------------------------------
\begin{definition}
\footnote{
  \citerpg{michel1993}{411}{048667598X}
  }
\label{def:opI}
\label{def:op_inv}
\index{operator!identity}
\index{identity operator}
%---------------------------------------
Let $\clOxx$ be the set of all operators with from a \structe{linear space} $\spX$ to $\spX$.
Let $\opI$ be an operator in $\clOxx$.
Let $\relid(\spX)$ be the \structe{identity element}\ifsxref{relation}{def:rel_id} in $\clOxx$.
\defbox{\text{
  $\hxs{\opI}$ is the \hid{identity operator} in $\clOxx$ if \qquad $\opI=\relid(\spX)$.
  }}
\end{definition}




%=======================================
%\subsection{Functions of operators}
%\index{operator!functions of}
%\index{function!of operator}
%=======================================
%%---------------------------------------
%\begin{definition}
%\label{def:op_taylor}
%\citepp{hassani}{53}{54}
%%---------------------------------------
%Let $\opA$ be an operator.
%\defbox{\begin{array}{rc>{\ds}l@{\qquad}D}
%  \ff(\opA) &=& \sum_{n=0}^\infty \frac{f^{(n)}(a)}{n!}(\opA-\opI a)^n
%            &   (\prop{Taylor series} about the point $a$)
%  \\
%  \ff(\opA) &=& \sum_{n=0}^\infty \frac{f^{(n)}(0)}{n!}\opA^n
%            &   (\prop{Maclaurin series})
%  \end{array}}
%\end{definition}
%
%%---------------------------------------
%\begin{example}
%\citep{hassani}{54}
%%---------------------------------------
%\exbox{
%  e^{\opA} = \sum_{n=0}^\infty \frac{\opA^n}{n!}
%  }
%\end{example}
%\begin{proof}
%\begin{align*}
%  e^{\opA}
%    &= \sum_{n=0}^\infty \frac{e^{(n)}(0)}{n!}\opA^n
%    && \text{by Maclaurin series (\prefp{def:op_taylor})}
%  \\&= \sum_{n=0}^\infty \frac{1}{n!}\opA^n
%\end{align*}
%\end{proof}



%=======================================
\subsection{Linear operators}
\index{linear operators}
%=======================================

%---------------------------------------
\begin{definition}
\footnote{
  \citerpg{kubrusly2001}{55}{0817641742},
  \citerpg{ab}{224}{0120502577},
  %\citerpg{michel1993}{95,407}{048667598X} \\
  \citorp{hilbert1927}{6},
  \citorp{stone1932}{33}
  }
\label{def:linop}
\label{def:clL}
%---------------------------------------
Let $\spX\eqd\linearspaceX$ and $\spY\eqd\linearspaceY$ be linear spaces.\\
  \defbox{%
    \indxs{\clOxy}
    \index{operator!linear}
    \begin{array}{>{\scy\qquad}r rcl @{\qquad}C @{\qquad}D@{\qquad}D}
      \mc{7}{M}{An operator $\opL\in\clOxy$ is \hid{linear} if}
        \\1. & \opL(\vx + \vy)    &=&  \opL \vx + \opL \vy & \forall \vx,\vy\in\spX  & (\prope{additive}) & and
        \\2. & \opL(\alpha \vx) &=&  \alpha\opL \vx    & \forall \vx\in\spX,\quad \forall\alpha\in\F & (\prope{homogeneous}).
    \\\mc{7}{M}{The set of all linear operators from $\spX$ to $\spY$ is denoted $\clLxy$ such that}
    \\\mc{7}{l}{\qquad\clLxy \eqd \set{\opL\in\clOxy}{\text{$\opL$ is linear}}}.
    \end{array}
  }
\end{definition}

%---------------------------------------
\begin{theorem}
\footnote{
  \citerpgc{berberian1961}{79}{0821819127}{Theorem~IV.1.1}
  }
\label{thm:L_prop}
%---------------------------------------
Let $\opL$ be an operator from a linear space $\spX$ to a linear space $\spY$, both over a field $\F$.
\thmbox{
  \text{$\opL$ is \prope{linear}}
  \qquad\implies\qquad
  \brbl{\begin{array}{>{\scy}r>{\ds}lc>{\ds}lC}
      1. & \opL\vzero    &=& \vzero                           & 
    \\2. & \opL(-\vx)    &=& -(\opL\vx)                       & \forall \vx\in\spX
    \\3. & \opL(\vx-\vy) &=& \opL\vx - \opL\vy                & \forall \vx,\vy\in\spX
    \\4. & \opL\brp{\sum_{n=1}^N \alpha_n\vx_n}  &=& \sum_{n=1}^N \alpha_n\brp{\opL\vx_n}  & \vx_n\in\spX,\,\alpha_n\in\F
  \end{array}}
}
\end{theorem}
\begin{proof}
\begin{enumerate}
  \item Proof that $\opL\vzero=\vzero$:
    \begin{align*}
      \opL\vzero
        &= \opL\brp{0\cdot\vzero}
        && \text{by additive identity property\ifsxref{vector}{thm:vs_addid}}
      \\&= 0\cdot\brp{\opL\vzero}
        && \text{by \prope{homogeneous} property of $\opL$ \xref{def:linop}}
      \\&= \vzero
        && \text{by additive identity property\ifsxref{vector}{thm:vs_addid}}
    \end{align*}

  \item Proof that $\opL(-\vx)    = -(\opL\vx)$:
    \begin{align*}
      \opL\brp{-\vx}
        &= \opL\brp{-1\cdot\vx}
        && \text{by additive inverse property\ifsxref{vector}{thm:vs_addinv}}
      \\&= -1\cdot\brp{\opL\vx}
        && \text{by \prope{homogeneous} property of $\opL$ \xref{def:linop}}
      \\&= -\brp{\opL\vx}
        && \text{by additive inverse property\ifsxref{vector}{thm:vs_addinv}}
    \end{align*}

  \item Proof that $\opL(\vx-\vy) = \opL\vx - \opL\vy$:
    \begin{align*}
      \opL(\vx-\vy)
        &= \opL(\vx+(-\vy))
        && \text{by additive inverse property\ifsxref{vector}{thm:vs_addinv}}
      \\&= \opL(\vx) + \opL(-\vy)
        && \text{by \prope{linearity} property of $\opL$ \xref{def:linop}}
      \\&= \opL\vx - \opL\vy
        && \text{by 2.}
    \end{align*}

  \item Proof that $\opL\brp{\sum_{n=1}^N \alpha_n\vx_n}  = \sum_{n=1}^N \alpha_n\brp{\opL\vx_n}$:
    \begin{enumerate}
      \item Proof for $N=1$:
        \begin{align*}
          \opL\brp{\sum_{n=1}^N \alpha_n\vx_n}
            &= \opL\brp{\alpha_1\vx_1}
            && \text{by $N=1$ hypothesis}
          \\&= \alpha_1\brp{\opL\vx_1}
            && \text{by \prope{homogeneous} property of \prefp{def:linop}}
        \end{align*}

      \item Proof that $N$ case $\implies$ $N+1$ case:
        \begin{align*}
          \opL\brp{\sum_{n=1}^{N+1} \alpha_n\vx_n}
            &= \opL\brp{\alpha_{N+1}\vx_{N+1} + \sum_{n=1}^{N} \alpha_n\vx_n}
          \\&= \opL\brp{\alpha_{N+1}\vx_{N+1}} + \opL\brp{\sum_{n=1}^{N} \alpha_n\vx_n}
            && \text{by \prope{linearity} property of \prefp{def:linop}}
          \\&= \alpha_{N+1}\opL\brp{\vx_{N+1}} + \sum_{n=1}^{N} \opL\brp{\alpha_n\vx_n}
            && \text{by left $N+1$ hypothesis}
          \\&= \sum_{n=1}^{N+1} \opL\brp{\alpha_n\vx_n}
        \end{align*}
    \end{enumerate}
\end{enumerate}
\end{proof}

%---------------------------------------
\begin{theorem}
\footnote{
  \citerppg{michel1993}{98}{104}{048667598X},
  \citerppgc{berberian1961}{80}{85}{0821819127}{Theorem~IV.1.4 and Theorem~IV.3.1}
  %\citep{prasad}{16}
  }
%---------------------------------------
%Let $\clOxy$ be the set of linear operators from $\spX$ to $\spY$.
Let $\clLxy$ be the set of all linear operators from a linear space $\spX$ to a linear space $\spY$.
Let $\oppN(\opL)$ be the \hie{null space} of an operator $\opL$ in $\clOxy$
and $\oppI(\opL)$ the \structe{image set}\ifsxref{relation}{def:rel_null} of $\opL$ in $\clOxy$.
\thmbox{
  \begin{array}{ll@{\qquad}C@{\qquad}D}
    \clLxy & \text{ is a linear space}
                     &
                     & (\prop{space of linear transforms})
                     \\
    \oppN(\opL)      & \text{is a linear subspace of $\spX$}
                     & \forall \opL\in\clOxy
                     \\
    \oppI(\opL)      & \text{is a linear subspace of $\spY$}
                     & \forall \opL\in\clOxy
  \end{array}
  }
\end{theorem}
\begin{proof}
\begin{enumerate}
  \item Proof that $\oppN(\opL)$ is a linear subspace of $\spX$:
    \begin{enumerate}
      \item $\vzero\in\oppN(\opL) \implies \oppN(\opL)\ne \emptyset$
      \item $\oppN(\opL)\eqd\set{\vx\in\spX}{\opL\vx=\vzero}\subseteq\spX$
      \item $\vx+\vy\in\oppN(\opL) \implies \vzero=\opL(\vx+\vy)=\opL(\vy+\vx)\implies \vy+\vx\in\oppN(\opL)$
      \item $\alpha\in\F,\; \vx\in\spX
            \implies \vzero=\opL\vx
            \implies \vzero=\alpha\opL\vx
            \implies \vzero=\opL(\alpha\vx)
            \implies \alpha\vx \in \oppN(\opL)$
    \end{enumerate}

  \item Proof that $\oppI(\opL)$ is a linear subspace of $\spY$:
    \begin{enumerate}
      \item $\vzero\in\oppI(\opL) \implies \oppI(\opL)\ne \emptyset$
      \item $\oppI(\opL)\eqd\set{\vy\in\spY}{\exists\vx\in\spX\st\vy=\opL\vx}\subseteq\spY$
      \item $\vx+\vy\in\oppI(\opL) \implies \exists\vv\in\spX\st\opL\vv=\vx+\vy=\vy+\vx \implies \vy+\vx\in\oppI(\opL)$
      \item $\alpha\in\F,\; \vx\in\oppI(\opL)
            \implies \exists\vx\in\spX \st \vy=\opL\vx
            \implies \alpha\vy=\alpha\opL\vx=\opL(\alpha\vx)
            \implies \alpha\vx \in \oppI(\opL)$
    \end{enumerate}
\end{enumerate}
\end{proof}


%---------------------------------------
\begin{example}
\footnote{
  \citerpg{eidelman2004}{3}{0821836463}
  }
%---------------------------------------
Let $\clC{\intcc{a}{b}}{\R}$ be the set of all \prope{continuous} functions from the closed real interval $\intcc{a}{b}$ to $\R$.
\exbox{
  \text{$\clC{\intcc{a}{b}}{\R}$ is a linear space.}
  }
\end{example}


%---------------------------------------
\begin{theorem}
\footnote{
  \citerpgc{berberian1961}{88}{0821819127}{Theorem~IV.1.4}
  }
%---------------------------------------
Let $\clLxy$ be the set of linear operators from a linear space $\spX$ to a linear space $\spY$.
Let $\oppN(\opL)$ be the \structe{null space} of a linear operator $\opL\in\clLxy$.
\thmbox{\begin{array}{l@{\qquad}c@{\qquad}l}
  \opL\vx = \opL\vy                  &\iff& \vx-\vy\in\oppN(\opL)\\
  \text{$\opL$ is \prope{injective}} &\iff& \oppN(\opL)=\setn{\vzero}
\end{array}}
\end{theorem}
\begin{proof}
\begin{enumerate}
  \item Proof that $\opL\vx = \opL\vy \implies   \vx-\vy\in\oppN(\opL)$:
    \begin{align*}
      \opL(\vx-\vy)
        &= \opL\vx - \opL\vy
        && \text{by \prefp{thm:L_prop}}
      \\&= \vzero
        && \text{by left hypothesis}
      \\&\implies\quad \vx-\vy\in\oppN(\opL)
        && \text{by definition of \structe{null space}\ifsxref{relation}{def:rel_null}}
    \end{align*}

  \item Proof that $\opL\vx = \opL\vy \impliedby \vx-\vy\in\oppN(\opL)$:
    \begin{align*}
      \opL\vy
        &= \opL\vy + \vzero
        && \text{by definition of linear space (\prefp{def:vspace})}
      \\&= \opL\vy + \opL(\vx-\vy)
        && \text{by right hypothesis}
      \\&= \opL\vy + (\opL\vx-\opL\vy)
        && \text{by \prefp{thm:L_prop}}
      \\&= (\opL\vy -\opL\vy) + \opL\vx
        && \text{by \prope{associative} and \prope{commutative} properties (\prefp{def:vspace})}
      \\&= \opL\vx
    \end{align*}

  \item Proof that $\opL$ is \prope{injective} $\iff$   $\oppN(\opL)=\setn{\vzero}$:
    \begin{align*}
      \text{$\opL$ is \prope{injective}}
        &\iff \brb{\brp{\opL\vx=\opL\vy\quad\iff\quad\vx=\vy}\quad\forall\vx,\vy\in\setX}
      \\&\iff \brb{\brs{\opL\vx-\opL\vy=\vzero\quad\iff\quad(\vx-\vy)=\vzero}\quad\forall\vx,\vy\in\setX}
      \\&\iff \brb{\brs{\opL(\vx-\vy)=\vzero\quad\iff\quad(\vx-\vy)=\vzero}\quad\forall\vx,\vy\in\setX}
      \\&\iff \oppN(\opL)=\setn{\vzero}
    \end{align*}

\end{enumerate}
\end{proof}

%---------------------------------------
\begin{theorem}
\footnote{
  \citerpgc{berberian1961}{88}{0821819127}{Theorem~IV.5.1}
  }
%---------------------------------------
Let $\spW$, $\spX$, $\spY$, and $\spZ$ be linear spaces over a field $\F$.
\thmbox{\begin{array}{>{\scy}rlcl@{\qquad}C@{\qquad}D}
    1. & \opL\brp{\opM\opN}      &=& \brp{\opL\opM}\opN                          & \forall \opL\in\clLzw,\, \opM\in\clLyz,\, \opN\in\clLxy         & (\prope{associative})
  \\2. & \opL\brp{\opM\addo\opN} &=& \brp{\opL\opM}\addo\brp{\opL\opN}           & \forall \opL\in\clLyz,\, \opM\in\clLxy,\, \opN\in\clLxy  & (\prope{left distributive})
  \\3. & \brp{\opL\addo\opM}\opN &=& \brp{\opL\opN}\addo\brp{\opM\opN}           & \forall \opL\in\clLyz,\, \opM\in\clLyz,\, \opN\in\clLxy  & (\prope{right distributive})
  \\4. & \alpha\brp{\opL\opM}    &=& \brp{\alpha\opL}\opM = \opL\brp{\alpha\opM} & \forall \opL\in\clLyz,\, \opM\in\clLxy,\, \alpha\in\F    & (\prope{homogeneous})
\end{array}}
\end{theorem}
\begin{proof}
\begin{enumerate}
  \item Proof that $\opL\brp{\opM\opN}      = \brp{\opL\opM}\opN$: Follows directly from property of \prope{associative} operators\ifsxref{relation}{thm:op_associative}.

  \item Proof that $\opL\brp{\opM\addo\opN} = \brp{\opL\opM}\addo\brp{\opL\opN}$:  
    \begin{align*}
      \brs{\opL\brp{\opM\addo\opN}}\vx
        &= \opL\brs{\brp{\opM\addo\opN}\vx} 
        && \ifdochas{relation}{\text{by \prefp{def:op+x}}}
      \\&= \opL\brs{\brp{\opM\vx}\addo\brp{\opN\vx}} 
        && \ifdochas{relation}{\text{by \prefp{def:op+x}}}
      \\&= \brs{\opL\brp{\opM\vx}} \addo \brs{\opL\brp{\opN\vx}}
        && \text{by \prope{additive} property \prefp{def:linop}}
      \\&= \brs{\brp{\opL\opM}\vx}\addo\brs{\brp{\opL\opN}\vx}
        && \ifdochas{relation}{\text{by \prefp{def:op+x}}}
    \end{align*}

  \item Proof that $\brp{\opL\addo\opM}\opN = \brp{\opL\opN}\addo\brp{\opM\opN}$:  Follows directly from property of \prope{associative} operators\ifsxref{relation}{thm:op_associative}.

  \item Proof that $\alpha\brp{\opL\opM} = \brp{\alpha\opL}\opM$: Follows directly from \prope{associative} property of linear operators\ifsxref{relation}{thm:op_associative}.
  \item Proof that $\alpha\brp{\opL\opM} = \opL\brp{\alpha\opM}$:
    \begin{align*}
      \brs{\alpha\brp{\opL\opM}}\vx
        &= \alpha\brs{\brp{\opL\opM}\vx}
        && \ifdochas{relation}{\text{by \prefp{def:op+x}}}
      \\&= \opL\brs{\alpha\brp{\opM\vx}}
        && \text{by \prope{homogeneous} property \prefp{def:linop}}
      \\&= \opL\brs{\brp{\alpha\opM}\vx}
        && \ifdochas{relation}{\text{by \prefp{def:op+x}}}
      \\&= \brs{\opL\brp{\alpha\opM}}\vx
        && \ifdochas{relation}{\text{by \prefp{def:op+x}}}
    \end{align*}
\end{enumerate}
\end{proof}
  
%---------------------------------------
\begin{theorem}[Fundamental theorem of linear equations]
\index{Fundamental theorem of linear equations}
\index{theorems!Fundamental theorem of linear equations}
\citerpg{michel1993}{99}{048667598X}
%---------------------------------------
%Let $\opL\in\clOxy$ be a linear operator.
Let $\clOxy$ be the set of all operators from a linear space $\spX$ to a linear space $\spY$.
Let $\oppN(\opL)$ be the \hie{null space} of an operator $\opL$ in $\clOxy$
and $\oppI(\opL)$ the \hie{image set} of $\opL$ in $\clOxy$
(\prefp{def:rel_null}).
\thmbox{
  \dim \oppI(\opL) + \dim\oppN(\opL) = \dim \spX
  \qquad\scriptstyle
  \forall \opL\in\clOxy
  }
\end{theorem}
\begin{proof}
Let $\set{\psi_k}{k=1,2,\ldots,p}$ be a basis for $\spX$
constructed such that $\{\psi_{p-n+1},\psi_{p-n+2},\ldots,\psi_p\}$
is a basis for $\oppN(\opL)$.
\begin{align*}
  \text{Let } p &\eqd \dim\spX.         \\
  \text{Let } n &\eqd \dim\oppN(\opL).  \\
  \\
  \dim\oppI(\opL)
    &= \dim \set{\vy\in\spY}{\exists\vx\in\spX \st \vy=\opL\vx}
  \\&= \dim \set{\vy\in\spY}{\exists(\alpha_1,\alpha_2,\ldots,\alpha_p) \st \vy=\opL\sum_{k=1}^p\alpha_k\psi_k}
  \\&= \dim \set{\vy\in\spY}{\exists(\alpha_1,\alpha_2,\ldots,\alpha_p) \st
            \vy=\sum_{k=1}^p\alpha_k\opL\psi_k}
  \\&= \dim \set{\vy\in\spY}{\exists(\alpha_1,\alpha_2,\ldots,\alpha_p) \st
            \vy=\sum_{k=1}^{p-n}\alpha_k\opL\psi_k+\sum_{k=1}^n\alpha_k\opL\psi_k}
  \\&= \dim \set{\vy\in\spY}{\exists(\alpha_1,\alpha_2,\ldots,\alpha_p) \st
            \vy=\sum_{k=1}^{p-n}\alpha_k\opL\psi_k+\vzero}
  \\&= p-n
  \\&= \dim\spX - \dim\oppN(\opL)
\end{align*}
Note: This ``proof" may be missing some necessary detail. \problem
\end{proof}

%=======================================
\section{Operators on Normed linear spaces}
%=======================================

%=======================================
\subsection{Operator norm}
%=======================================
\ifdocnot{vsnorm}{
%--------------------------------------
\begin{definition}
\footnote{
  \citerppg{ab}{217}{218}{0120502577},
  \citorp{banach1932}{53},
  \citorp{banach1932e}{33},
  \citorp{banach1922}{135}
 %\citerpg{michel1993}{344}{048667598X} \\
 %\citerp{horn}{259}  \\
  }
\label{def:norm}
\index{space!normed vector}
\index{inequality!triangle}
\index[xsym]{$\normn$}
%--------------------------------------
Let $\spV=\oquad{\setX}{\F}{\adds}{\cdot}$ be a linear space and
$\F$ be a field with absolute value function $\abs{\cdot}\in\clFfr$\ifsxref{algebra}{def:abs}.
\defbox{\begin{array}{>{\qquad}F rcl @{\qquad}C @{\qquad}D @{\qquad}D}
  \mc{7}{M}{A \fnctd{norm} is any functional $\normn$ in $\clFxr$ that satisfies}
  \\1. & \norm{ \vx}      &\ge& 0                     & \forall \vx \in\setX            & (\prope{strictly positive})  \nocite{michel1993} & and  %page 115
  \\2. & \norm{ \vx}      &=  & 0 \iff \vx=\vzero     & \forall \vx \in\setX            & (\prope{nondegenerate})                                    & and
  \\3. & \norm{a\vx}      &=  & |a|\norm{\vx}         & \forall \vx \in\setX, \; a\in\C & (\prope{homogeneous})                                      & and
  \\4. & \norm{\vx+\vy}   &\le& \norm{\vx}+\norm{\vy} & \forall \vx,\vy \in\setX        & (\prope{subadditive}/\prop{triangle inquality}).
  \\
  \mc{7}{M}{A \structd{normed linear space} is the pair $\opair{\spV}{\normn}$.}
\end{array}}
\end{definition}
}

%---------------------------------------
\begin{definition}
\footnote{
  \citerpg{rudinf}{92}{0070542252},
  \citerpg{ab}{225}{0120502577}
  }
\label{def:op_norm}
\label{def:normop}
\label{def:opnorm}
\index{operator!norm}   \index{norm}
%---------------------------------------
Let $\clLxy$ be the space of linear operators over normed linear spaces $\spX$ and $\spY$.
\footnote{
  The operator norm notation $\normop{\cdot}$ is introduced (as a Matrix norm) in \\
  \citerp{horn}{290}
  }
\defbox{\indxs{\normopn}
  \begin{array}{M}
  The \hid{operator norm} $\normopn$ is defined as
    \\\qquad$\ds
      \normop{\opA} \eqd \sup_{\vx\in\spX}\set{\norm{\opA\vx}}{\norm{\vx} \le 1}
      \qquad\forall\opA\in\clLxy
      $
    \\
    The pair $\opair{\clLxy}{\normopn}$
    is the \hid{normed space of linear operators} on $\opair{\spX}{\spY}$.
  \end{array}}
\end{definition}

\pref{prop:op_norm} (next) shows that the functional defined in
\pref{def:op_norm} (previous) is a \hie{norm}.%
\ifdochas{vsnorm}{\footnote{
  \emph{norm} $\normn$: \prefpp{def:norm}
  }}

%---------------------------------------
\begin{proposition}
\label{prop:op_norm}
\footnote{
  \citerpg{rudinf}{93}{0070542252}
  }
\index{triangle inequality}
\index{inequality!triangle}
%---------------------------------------
Let $\opair{\clLxy}{\normopn}$ be the normed space of linear operators
over the normed linear spaces
$\spX\eqd\normspaceX$ and $\spY\eqd\normspaceY$.
\propbox{\begin{array}{>{\qquad}F>{\ds}rc>{\ds}l @{\qquad}C @{\qquad}D @{\qquad}D}
  \mc{7}{M}{The functional $\normopn$ is a \hib{norm} on $\clLxy$. In particular,}
  \\
    1.&\normop{\opA}        &\ge&   0
    & \forall \opA\in\clLxy
    & (\prope{non-negative})
    & and
    \\
    2.&\normop{\opA}        &=&   0 \quad \iff \opA\eqo\vzero
    & \forall \opA\in\clLxy
    & (\prope{nondegenerate})
    & and
    \\
    3.&\normop{\alpha\opA}  &=&   |\alpha|\normop{\opA}
    & \forall \opA\in\clLxy,\, \alpha\in\F
    & (\prope{homogeneous})
    & and
    \\
    4.&\normop{\opA\addo  \opB} &\le& \normop{\opA} + \normop{\opB}
    & \forall \opA\in\clLxy
    & (\prope{subadditive}).
  \\
  \mc{7}{M}{Moreover, $\opair{\clLxy}{\normopn}$ is a \hid{normed linear space}.}
\end{array}}
\end{proposition}
\begin{proof}
\begin{align*}
  \intertext{1. Proof that $\normop{\opA}>0$ for $\opA\ne \vzero$:}
  \normop{\opA}
    &\eqd \sup_{\vx\in\spX} \set{\norm{\opA\vx}}{\norm{\vx}\le1}
    &&    \text{by definition of $\normopn$ \xref{def:opnorm}}
  \\&>    0
  \intertext{2. Proof that $\normop{\opA}=0$ for $\opA\eqo\vzero$:}
  \normop{\opA}
    &\eqd \sup_{\vx\in\spX} \set{\norm{\opA\vx}}{\norm{\vx}\le1}
    &&    \text{by definition of $\normopn$ \xref{def:opnorm}}
  \\&=    \sup_{\vx\in\spX} \set{\norm{\vzero\vx}}{\norm{\vx}\le1}
  \\&=    0
  \\
  \intertext{3. Proof that $\normop{\alpha\opA}=|\alpha|\normop{\opA}$:}
  \normop{\alpha\opA}
    &\eqd \sup_{\vx\in\spX} \set{\norm{\alpha\opA\vx}}{\norm{\vx}\le1}
    &&    \text{by definition of $\normopn$ \xref{def:opnorm}}
  \\&=    \sup_{\vx\in\spX} \set{|\alpha|\norm{\opA\vx}}{\norm{\vx}\le1}
    &&    \text{by definition of $\normopn$ \xref{def:opnorm}}
  \\&=    |\alpha|\sup_{\vx\in\spX} \set{\norm{\opA\vx}}{\norm{\vx}\le1}
    &&    \text{by definition of $\sup$}
  \\&=    |\alpha| \; \normop{\opA}
    &&    \text{by definition of $\normopn$ \xref{def:opnorm}}
  \\
  \intertext{4. Proof that $\normop{\opA\addo  \opB} \le \normop{\opA} + \normop{\opB}$:}
  \normop{\opA\addo  \opB}
    &\eqd \sup_{\vx\in\spX} \set{\norm{(\opA\addo \opB)\vx}}{\norm{\vx}\le1}
    &&    \text{by definition of $\normopn$ \xref{def:opnorm}}
  \\&=    \sup_{\vx\in\spX} \set{\norm{\opA\vx+\opB\vx}}{\norm{\vx}\le1}
    && \ifdochas{relation}{\text{by \prefp{def:op+x}}}
  \\&\le  \sup_{\vx\in\spX} \set{\norm{\opA\vx}+\norm{\opB\vx}}{\norm{\vx}\le1}
    &&    \text{by definition of $\normopn$ \xref{def:opnorm}}
  \\&\le  \sup_{\vx\in\spX} \set{\norm{\opA\vx}}{\norm{\vx}\le1}
       +  \sup_{\vx\in\spX} \set{\norm{\opB\vx}}{\norm{\vx}\le1}
  \\&\eqd \normop{\opA} + \normop{\opB}
    &&    \text{by definition of $\normopn$ \xref{def:opnorm}}
\end{align*}
\end{proof}

%---------------------------------------
\begin{lemma}
\label{lem:opnorm}
%---------------------------------------
Let $\opair{\clLxy}{\normopn}$ be the normed space of linear operators over normed linear spaces
$\spX\eqd\normspaceX$ and $\spY\eqd\normspaceY$.
\lembox{
  \normop{\opL}
    %\eqd \sup_{\vx} \set{\norm{\opL\vx}}{\norm{\vx}\le1}
    =    \sup_{\vx} \set{\norm{\opL\vx}}{\norm{\vx}=1}
    \qquad\forall \vx\in\clLxy
  }
\end{lemma}
\begin{proof}
  \footnote{\begin{minipage}[t]{2\tw/16+1ex}%
      \fontsize{0.5mm}{0.5mm}\usefont{T1}{put}{b}{n}%
      \begin{Verbatim}[frame=single, gobble=8, label={\footnotesize\rmfamily email}]
        Hi, Dan,
        OK, let's discuss it in Friday,
        Btw, there is something I missed in my last mail, the following is more
        complete than previous one.
        (a) sup_x { || Lx ||  : ||x||<=1 } >= sup_x { || Lx ||  : ||x|| =1 } , since
        sup of a set is always greater than sup of its subset.
        (b) Assume\vy_{1},..\vy_{n}.. be a sequence in the set {\vx : ||x||<=1 } that
        can achieve sup_x { || Lx ||  : ||x||<=1 }, we can define z_{n}=\vy_{n} / ||
       \vy_{n} ||, thus z_{n} is in {\vx : ||x||=1 } .
        Also, we have sup_n { || L z_{n} || } >= sup_n { || L Y_{n} || },
        since ||\vy_{n} || <= 1, and for each n,
        || L z_{n} ||= || L Y_{n} || / ||\vy_{n} ||  >=  || L Y_{n} ||.
        Thus, sup_x { || Lx ||  : ||x|| =1 } >= sup_n { || L z_{n} || }
            >= sup_n { || L Y_{n} || } = sup_x { || Lx || : ||x||<=1 }
        (c) From (a) and (b), we have
        sup_x { || Lx ||  : ||x||<=1 } = sup_x { || Lx ||  : ||x|| =1 }.

        Chien Yao
      \end{Verbatim}
    \end{minipage}
    \hfill
    \begin{minipage}[t]{13\tw/16-3ex}
      Many many thanks to former NCTU Ph.D. student
      \href{http://web.ntpu.edu.tw/~yshan/it07.pdf}{Chien Yao} (Chinese: \zht{\zhtYao\zhtJian}; PinYin: Y{/'a}o Ji/`an) 
      for his brilliant help with this proof.
      (If you are viewing this text as a pdf file,
       zoom in on the figure to the left to see text from Chien Yao's 2007 April 16 email.)
    \end{minipage}
  }


\begin{enumerate}
  \item Proof that $\sup_{\vx} \set{\norm{\opL\vx}}{\norm{\vx}\le1}
       \ge \sup_{\vx} \set{\norm{\opL\vx}}{\norm{\vx}=1}$:
    \begin{align*}
      \sup_{\vx} \set{\norm{\opL\vx}}{\norm{\vx}\le1}
        &\ge \sup_{\vx} \set{\norm{\opL\vx}}{\norm{\vx}=1}
        &&   \text{because $\setA\subseteq\setB \implies \sup\setA \le \sup\setB$}
    \end{align*}

  \item Let the subset $\setY\subsetneq\setX$ be defined as
    \[ \setY \eqd \set{\vy\in\setX}
                      {\begin{array}{l>{\ds}l}
                         1. & \norm{\opL\vy} = \sup_{\vx\in\setX} \set{\norm{\opL\vx}}{\norm{\vx}\le1}
                              \text{ and } \\
                         2. & 0<\norm{\vy}\le 1
                      \end{array}}
    \]


  \item Proof that $\sup_{\vx} \set{\norm{\opL\vx}}{\norm{\vx}\le1}
       \le \sup_{\vx} \set{\norm{\opL\vx}}{\norm{\vx}=1}$:
    \begin{align*}
      \sup_{\vx\in\setX} \set{\norm{\opL\vx}}{\norm{\vx}\le1}
        &=    \norm{\opL \vy}
        &&    \text{by definition of set $\setY$}
      \\&=    \frac{\norm{\vy}}{\norm{\vy}} \norm{\opL \vy}
      \\&=    \norm{\vy} \norm{\frac{1}{\norm{\vy}}\opL \vy}
        &&    \text{by homogeneous property (\prefpo{def:norm})}
      \\&=    \norm{\vy} \norm{\opL \frac{\vy}{\norm{\vy}}}
        &&    \text{by homogeneous property (\prefpo{def:linop})}
      \\&\le  \norm{\vy} \sup_{\vy\in\setY} \setn{\norm{\opL \frac{\vy}{\norm{\vy}}}}
        &&    \text{by definition of supremum}
      \\&=    \norm{\vy} \sup_{\vy\in\setY} \set{\norm{\opL \frac{\vy}{\norm{\vy}}}}{\norm{\frac{\vy}{\norm{\vy}}}=1}
        &&    \text{because $\norm{\frac{\vy}{\norm{\vy}}}=1$ for all $\vy\in\setY$}
      \\&\le  \sup_{\vy\in\setY} \set{\norm{\opL \frac{\vy}{\norm{\vy}}}}{\norm{\frac{\vy}{\norm{\vy}}}=1}
        &&    \text{because $0<\norm{\vy}\le1$}
      \\&\le  \sup_{\vx\in\setX} \set{\norm{\opL \vx}}{\norm{\vx}=1}
        &&    \text{because $\frac{\vy}{\norm{\vy}}\in\setX\qquad\forall \vy\in\setY$}
    \end{align*}

  \item By (1) and (3),
    \[ \sup_{\vx\in\setX}\set{ \norm{\opL \vx}}{\norm{\vx}\le1}
       =
       \sup_{\vx\in\setX}\set{ \norm{\opL \vx}}{\norm{\vx}=1}
    \]

\end{enumerate}
\end{proof}


%---------------------------------------
\begin{proposition}
\footnote{
  \citerpg{michel1993}{410}{048667598X}
  }
%---------------------------------------
Let $\opI$ be the identity operator in the normed space of linear operators
$\opair{\clLxx}{\normopn}$.
\propbox{
  \normop{\opI} = 1
  }
\end{proposition}
\begin{proof}
\begin{align*}
  \normop{\opI}
    &\eqd \sup\set{\norm{\opI\vx}}{\norm{\vx}\le1}
    &&    \text{by definition of $\normopn$ (\prefp{def:op_norm})}
  \\&=    \sup\set{\norm{\vx}}{\norm{\vx}\le1}
    &&    \text{by definition of $\opI$ (\prefp{def:opI})}
  \\&=    1
\end{align*}
\end{proof}


%---------------------------------------
\begin{theorem}
\footnote{
  \citerpg{rudinf}{103}{0070542252},
  \citerpg{ab}{225}{0120502577}
  }
\label{thm:LxLx}
\label{thm:KLKL}
%---------------------------------------
Let $\opair{\clLxy}{\normopn}$ be the normed space of linear operators over normed linear spaces
$\spX$ and $\spY$.
\thmbox{\begin{array}{rcl@{\qquad\qquad}l}
  \norm  {\opL \vx}  &\le& \normop{\opL}\:\norm{\vx}    & \forall \opL\in\clLxy,\,\vx\in\spX \\
  \normop{\opK \opL} &\le& \normop{\opK}\:\normop{\opL} & \forall \opK,\opL\in\clLxy
  \end{array}}
\end{theorem}
\begin{proof}
\begin{align*}
  \intertext{1. Proof that $\norm  {\opL \vx} \le \normop{\opL}\:\norm{\vx}$:}
  \norm{\opL \vx}
    &=    \frac{\norm{\vx}}{\norm{\vx}} \norm{\opL \vx}
  \\&=    \norm{\vx} \norm{\frac{1}{\norm{\vx}}\opL \vx}
    &&    \text{by property of norms}
  \\&=    \norm{\vx} \norm{\opL \frac{\vx}{\norm{\vx}}}
    &&    \text{by property of linear operators}
  \\&\eqd \norm{\vx} \norm{\opL \vy}
    &&    \text{where $y\eqd\frac{\vx}{\norm{\vx}}$}
  \\&\le  \norm{\vx} \sup_y \norm{\opL \vy}
    &&    \text{by definition of supremum}
  \\&=    \norm{\vx} \sup_y\set{ \norm{\opL \vy}}{\norm{\vy}=1}
    &&    \text{because $\norm{\vy}=\norm{\frac{\vx}{\norm{\vx}}}=\frac{\norm{\vx}}{\norm{\vx}}=1$}
  \\&\eqd \norm{\vx} \norm{\opL}
    &&    \text{by definition of operator norm}
  \\
  \intertext{2. Proof that $\normop{\opK \opL} \le \normop{\opK}\:\normop{\opL}$:}
  \normop{\opK\opL}
    &\eqd \sup_{\vx\in\spX} \set{\norm{(\opK\opL)\vx}}{\norm{\vx}\le1}
    &&    \text{by \prefp{def:op_norm} ($\normop{\cdot}$)}
  \\&=    \sup_{\vx\in\spX} \set{\norm{\opK(\opL\vx)}}{\norm{\vx}\le1}
    && \ifdochas{relation}{\text{by \prefp{def:op+x}}}
  \\&\le  \sup_{\vx\in\spX} \set{\normop{\opK}\;\norm{\opL\vx}}{\norm{\vx}\le1}
    &&    \text{by 1.}
  \\&\le  \sup_{\vx\in\spX} \set{\normop{\opK}\;\normop{\opL}\;\norm{\vx}}{\norm{\vx}\le1}
    &&    \text{by 1.}
  \\&=    \sup_{\vx\in\spX} \set{\normop{\opK}\;\normop{\opL}\;1}{\norm{\vx}\le1}
    &&    \text{by definition of $\sup$}
  \\&=    \normop{\opK}\;\normop{\opL}
    &&    \text{by definition of $\sup$}
\end{align*}
\end{proof}



%=======================================
\subsection{Bounded linear operators}
%=======================================
%---------------------------------------
\begin{definition}
\footnote{
  \citerppg{rudinf}{92}{93}{0070542252}
  }
\label{def:clB}
\label{def:op_bounded}
\index{operator!bounded}   \index{bounded operator}
\index{operator!unbounded}
\indxs{\clBxy}
%---------------------------------------
Let $\opair{\clLxy}{\normopn}$ be a normed space of linear operators.
\defbox{\begin{array}{M}
  An operator $\opB$ is \hid{bounded} if $\normop{\opB} < \infty$.
  \\
  The quantity $\clBxy$ is the set of all \hid{bounded linear operators} on $\opair{\spX}{\spY}$ such that
    \\\qquad$\ds\clBxy  \eqd \set{\opL\in\clLxy}{\normop{\opL}<\infty}$.
\end{array}}
\end{definition}

%An operator $\opL$ is \hib{continuous} if for all $\epsilon>0$ there exists $\delta>0$ such that (see \prefp{def:ms_continous})
%\\\qquad$\ds\norm{\vx-\vy}<\delta \qquad\implies\qquad \norm{\opL\vx - \opL\vy} < \epsilon.$

%%---------------------------------------
%\begin{definition}
%\footnote{
%  \citerpg{michel1993}{408}{048667598X}
%  }
%\label{def:clC}
%\label{def:op_cont}
%\index{operator!bounded}   \index{bounded operator}
%\index{operator!unbounded}
%%---------------------------------------
%Let $\opair{\clLxy}{\normopn}$ be the normed space of linear operators over normed linear spaces
%$\spX$ and $\spY$.
%\defbox{\begin{array}{M}
%  An operator $\opL$ is \hid{continuous} if\quad
%    for all $\epsilon>0$ there exists $\delta>0$ such that
%    \\\qquad$\ds\norm{\vx-\vy}<\delta \qquad\implies\qquad \norm{\opL\vx - \opL\vy} < \epsilon.$\\
%  %The set $\clCxy$ is defined to be the set of all \hid{continuous operators} on $\opair{\spX}{\spY}$ such that
%  %  \\\qquad$\ds\hxs{\clCxy} \eqd \set{\opC\in\clOxy}{\text{$\opC$ is continuous}}.$
%\end{array}}
%\end{definition}


%---------------------------------------
\begin{theorem}
\label{thm:op_cont_eq}
\footnote{
  \citerpg{ab}{227}{0120502577}
  %\citerpg{michel1993}{408}
  %\citep{pedersen2000}{27}
  }
%---------------------------------------
Let $\opair{\clLxy}{\normopn}$ be the set of linear operators over normed linear spaces\\
$\spX\eqd\normspaceX$ and $\spY\eqd\normspaceY$.%
%\crossreftbl{
%  \crossrefp{continuous}{thm:ms_continous}\\  % {def:continuous}\\
%  \crossrefp{bounded}{def:bounded}
%  }
\thmbox{\begin{array}{>{\qquad}FlCc}
  \mc{3}{M}{The following conditions are all {\em equivalent}:}\\
     1. & \text{$\opL$ is continuous at {\em a single point} $\vx_0\in\setX$}
        & \forall \opL\in\clLxy
        & \iff
   \\2. & \text{$\opL$ is \emph{continuous} (at every point $\vx\in\setX$)}
        & \forall \opL\in\clLxy
        & \iff
   \\3. & \normop{\opL}<\infty \text{ ($\opL$ is \emph{bounded})}
        & \forall \opL\in\clLxy
        & \iff
   \\4. & \exists M\in\R \st \norm{\opL\vx} \le M\norm{\vx}
        & \forall \opL\in\clLxy,\,\vx\in\setX
        &
\end{array}}
\end{theorem}
\begin{proof}
\begin{enumerate}
  \item Proof that $1\implies2$:
    \begin{align*}
      \epsilon &> \norm{\opL\vx - \opL\vx_0}
        && \text{by hypothesis 1}
      \\&= \norm{\opL(\vx - \vx_0)}
        && \text{by linearity (\prefp{def:linop})}
      \\&= \norm{\opL(\vx +\vy - \vx_0-\vy)}
      \\&= \norm{\opL(\vx +\vy) - \opL(\vx_0+\vy)}
        && \text{by linearity (\prefp{def:linop})}
      \\\implies&\text{$\opL$ is continuous at point $\vx+\vy$}
      \\\implies&\text{$\opL$ is continuous at every point in $\setX$}
        && \text{(hypothesis 2)}
    \end{align*}
  %
  \item Proof that $2\implies1$: obvious:
  %
  %\item Proof that $4\implies1$:
  %  \begin{align*}
  %    \lim_{n\to\infty} \norm{\opL\vx - \opL\vx_n}
  %      &=   \lim_{n\to\infty} \norm{\opL(\vx - \vx_n)}
  %      &&   \text{by linearity of $\opL$ (\prefp{def:linop})}
  %    \\&\le \lim_{n\to\infty} \, M \, \norm{\vx - \vx_n}
  %      &&   \text{by hypothesis 4}
  %    \\&\le M \, \lim_{n\to\infty} \, \norm{\vx - \vx_n}
  %    \\&\le M \cdot 0
  %    \\&\le 0
  %      &&   \text{because $M<\infty$}
  %    \\\implies & \text{$\opL$ is continuous at point $x$}
  %      &&   \text{(hypothesis 1)}
  %  \end{align*}
  %
  \item Proof that $4\implies2$: \citep{bollobas1999}{29}
    \begin{align*}
      \normop{\opL\vx} \le M\norm{\vx}
        &\implies \normop{\opL(\vx-\vy)} \le M\norm{\vx-\vy}
        && \text{by hypothesis 4}
      \\&\implies \normop{\opL\vx-\opL\vy} \le M\norm{\vx-\vy}
        &&        \text{by linearity of $\opL$ \xref{def:linop}}
      \\&\implies \normop{\opL\vx-\opL\vy}\le\epsilon \text{ whenever } M\norm{\vx-\vy}<\epsilon
      \\&\implies \normop{\opL\vx-\opL\vy}\le\epsilon \text{ whenever } \norm{\vx-\vy}<\frac{\epsilon}{M}
        && \text{(hypothesis 2)}
    \end{align*}
  %
  \item Proof that $3\implies4$:
    \begin{align*}
      \norm  {\opL \vx}
        &\le \mcom{\normop{\opL}}{$M$}\:\norm{\vx}
        &&   \text{by \prefp{thm:LxLx}}
      \\&=   M\:\norm{\vx}
        &&   \text{where $M\eqd\normop{\opL}<\infty$ (by hypothesis 1)}
    \end{align*}
  %
  \item Proof that $1\implies3$: \citep{ab}{227}
    \begin{align*}
      \normop{\opL} = \infty
        &\implies \set{\norm{\opL\vx}}{\norm{\vx}\le1} = \infty
      \\&\implies \exists\seqn{\vx_n} \st \norm{\vx_n}=1 \text{ and } \normop{\opL} = \set{\norm{\opL\vx_n}}{\norm{\vx_n}\le1} = \infty
      \\&\implies \norm{\vx_n}=1 \text{ and } \infty=\normop{\opL} = \norm{\opL\vx_n}
      \\&\implies \norm{\vx_n}=1 \text{ and } \norm{\opL\vx_n} \ge n
      \\&\implies \frac{1}{n}\norm{\vx_n}=\frac{1}{n} \text{ and } \frac{1}{n}\norm{\opL\vx_n} \ge 1
      \\&\implies \norm{\frac{\vx_n}{n}}=\frac{1}{n} \text{ and } \norm{\opL\frac{\vx_n}{n}} \ge 1
      \\&\implies \lim_{n\to\infty}\norm{\frac{\vx_n}{n}}=0 \text{ and } \lim_{n\to\infty}\norm{\opL\frac{\vx_n}{n}} \ge 1
      \\&\implies \text{$\opL$ is not continuous at $0$}
    \end{align*}
    But by hypothesis, $\opL$ {\em is} continuous.
    So the statement $\normop{\opL}=\infty$ must be \emph{false} and thus
    $\normop{\opL}<\infty$ ($\opL$ is \emph{bounded}).
\end{enumerate}
\end{proof}



%=======================================
\subsection{Adjoints on normed linear spaces}
%=======================================
%---------------------------------------
\begin{definition}
\label{def:norm_adjoint}
\label{def:adjoint}
%---------------------------------------
Let $\clBxy $ be the space of bounded linear operators on normed linear spaces $\spX$ and $\spY$.
Let $\spXd$ be the \structe{topological dual space}\ifsxref{vstopo}{def:spXd} of $\spX$.
\defbox{\begin{array}{M}
    $\opBa$ is the \opd{adjoint} of an operator $\opB\in\clBxy$ if\\
    \qquad$\ds\ff(\opB\vx) = \brs{\opBa\ff}(\vx) \qquad\forall\ff\in\spXd,\,\vx\in\spX$
  \end{array}
  }
\end{definition}


%---------------------------------------
\begin{theorem}
\footnote{
  \citerp{bollobas1999}{156}
  }
\label{thm:op_adjoint}
%---------------------------------------
Let $\clBxy $ be the space of bounded linear operators on normed linear spaces $\spX$ and $\spY$.
\thmbox{\begin{array}{rcl@{\qquad}C}
  (\opA\addo  \opB)^\ast &=& \opAa \addo \opBa & \forall \opA,\opB\in\clBxy \\
  (\lambda\opA)^\ast &=& \lambda\opAa  & \forall \opA,\opB\in\clBxy \\
  (\opA\opB)^\ast    &=& \opBa\opAa    & \forall \opA,\opB\in\clBxy
\end{array}}
\end{theorem}
\begin{proof}
{\begin{align*}
  \brs{\opA\addo \opB}^\ast\ff(\vx)
    &= \ff\brp{\brs{\opA\addo \opB}\vx}
    && \text{by definition of adjoint}&&\text{\xref{def:norm_adjoint}}
  \\&= \ff\brp{\opA\vx+\opB\vx}
    && \text{by definition of linear operators}&&\text{\xref{def:linop}}
  \\&= \ff\brp{\opA\vx} + \ff\brp{\opB\vx}
    && \text{by definition of \fncte{linear functional}}&&\text{\ifxref{functionals}{def:functional}}
  \\&= \opAa\ff\brp{\vx} + \opBa\ff\brp{\vx}
    && \text{by definition of \ope{adjoint}}&&\text{\xref{def:norm_adjoint}}
  \\&= \brs{\opAa+\opBa}\ff\brp{\vx}
    && \text{by definition of \fncte{linear functional}}&&\text{\ifxref{functionals}{def:functional}}
  \\
  \\
  \brs{\lambda\opA}^\ast\ff(\vx)
    &= \ff\brp{\brs{\lambda\opA}\vx}
    && \text{by definition of \ope{adjoint}}&&\text{\xref{def:norm_adjoint}}
  \\&= \lambda\ff\brp{\opA\vx}
    && \text{by definition of \fncte{linear functional}}&&\text{\ifxref{functionals}{def:functional}}
  \\&= \brs{\lambda\opAa}\ff\brp{\vx}
    && \text{by definition of \ope{adjoint}}&&\text{\xref{def:norm_adjoint}}
  \\
  \\
  \brs{\opA\opB}^\ast\ff(\vx)
    &= \ff\brp{\brs{\opA\opB}\vx}
    && \text{by definition of \ope{adjoint}}&&\text{\xref{def:norm_adjoint}}
  \\&= \ff\brp{\opA\brs{\opB\vx}}
    && \text{by definition of \ope{linear operators}}&&\text{\xref{def:linop}}
  \\&= \brs{\opAa\ff}\brp{\opB\vx}
    && \text{by definition of \ope{adjoint}}&&\text{\xref{def:norm_adjoint}}
  \\&= \opBa\brs{\opAa\ff}\brp{\vx}
    && \text{by definition of \ope{adjoint}}&&\text{\xref{def:norm_adjoint}}
  \\&= \brs{\opBa\opAa}\ff\brp{\vx}
    && \text{by definition of \ope{adjoint}}&&\text{\xref{def:norm_adjoint}}
\end{align*}}
\end{proof}

%---------------------------------------
\begin{theorem}
\footnote{
  \citerpg{rudinf}{98}{0070542252}
  }
%---------------------------------------
Let $\clBxy $ be the space of bounded linear operators on normed linear spaces $\spX$ and $\spY$.
Let $\opBa$ be the adjoint of an operator $\opB$.
\thmbox{
  \normop{\opB} = \normop{\opBa}
  \qquad\forall\opB\in\clBxy
  }
\end{theorem}
\begin{proof}
\problem
\begin{align*}
  \normop{\opB}
    &\eqd \sup\set{\norm{\opB\vx}}{\norm{\vx}\le1}
    &&    \text{by \prefp{def:op_norm}}
  \\&\eqq \sup\set{|\fg(\opB\vx;\vyd)|}{\norm{\vx}\le1,\; \norm{\vyd}\le1}
  \\&=    \sup\set{|\ff(\vx;\opBa\vyd)|}{\norm{\vx}\le1,\; \norm{\vyd}\le1}
  \\&\eqd \sup\set{\norm{\opBa\vyd}}{\norm{\vx}\le1,\; \norm{\vyd}\le1}
  \\&=    \sup\set{\norm{\opBa\vyd}}{\norm{\vyd}\le1}
  \\&\eqd \normop{\opBa}
    &&    \text{by \prefp{def:op_norm}}
\end{align*}
\end{proof}


%=======================================
\subsection{More properties}
%=======================================
\qboxnpq
  {\href{http://en.wikipedia.org/wiki/Stanis\%C5\%82aw_Ulam}{Stanislaus M. Ulam}
   \href{http://www-history.mcs.st-andrews.ac.uk/Timelines/TimelineG.html}{(1909--1984)},
   \href{http://www-history.mcs.st-andrews.ac.uk/BirthplaceMaps/Places/Germany.html}{Polish mathematician}
    \index{Ulam, Stanislaus M.}
    \index{quotes!Ulam, Stanislaus M.}
    \footnotemark
  }
  {../common/people/ulam.jpg}
  {Beginning with the third year of studies,
   most of my mathematical work was really started in conversations with Mazur and Banach.
   And according to Banach some of my own contributions were characterized by a certain ``strangeness"
   in the formulation of problems and in the outline of possible proofs.
   As he told me once some years later,
   he was surprised how often these ``strange" approaches really worked.}
  \footnotetext{\begin{tabular}[t]{ll}
    quote: & \citerp{ulam1991}{33} \\
    image: & \url{http://www-history.mcs.st-andrews.ac.uk/Biographies/Ulam.html}
  \end{tabular}}


%---------------------------------------
\begin{theorem}[\thm{Mazur-Ulam theorem}]
\label{thm:mazur-ulam}
\footnote{
  \citerp{oikhberg2007}{598},
  \citerp{vaisala2003}{634},
  \citerp{giles2000}{11},
  \citerp{dunford1957}{91},
  \citor{mazur1932}
  }
%---------------------------------------
Let $\fphi\in\clLxy$ be a function on normed linear spaces
$\opair{\spX}{\normn_\spX}$ and $\opair{\spY}{\normn_\spY}$.
Let $\opI\in\clLxx$ be the identity operator on $\opair{\spX}{\normn_\spX}$.
\thmbox{\begin{array}{ll}
  \left.\begin{array}{llD}
    1. & \mcom{\fphi^{-1}\fphi = \fphi\fphi^{-1} = \opI}
              {\prop{bijective}}
       & and \\
    2. & \mcom{\norm{\fphi\vx-\fphi\vy}_\spY = \norm{\vx-\vy}_\spX  \quad\sst\forall \vx,\vy\in\spX}
              {\prop{isometric}}
  \end{array}\right\}
  \implies
  \mcom{\fphi\brp{[1-\lambda]\vx+\lambda\vy} = [1-\lambda]\fphi\vx + \lambda\fphi\vy
        \forall \lambda\in\R}
       {\prop{affine}}
\end{array}}
\end{theorem}
\begin{proof}
Proof not yet complete.\attention
\begin{enumerate}
  \item Let $\fpsi$ be the \hie{reflection} of $\vz$ in $\spX$ such that
        $\fpsi \vx = 2\vz - \vx$
        \label{item:mazur-ulam_psi}
    \begin{enumerate}
      \item $\norm{\fpsi\vx-\vz} = \norm{\vx-\vz}$
    \end{enumerate}

  \item Let $\ds\lambda \eqd \sup_\fg\setn{\norm{\fg\vz-\vz}}$ \label{item:mazur-ulam_lambda}

  \item Proof that $\fg\in\setW \implies \fg^{-1}\in\setW$: \label{item:mazur-ulam_gi}
    \begin{align*}
      \intertext{Let $\hat{\vx}\eqd \fg^{-1}\vx$ and $\hat{\vy}\eqd \fg^{-1}\vy$.}
      \norm{\fg^{-1}\vx - \fg^{-1}\vy}
        &= \norm{\hat{\vx} - \hat{\vy}}
        && \text{by definition of $\hat{\vx}$ and $\hat{\vy}$}
      \\&= \norm{\fg\hat{\vx} - \fg\hat{\vy}}
        && \text{by left hypothesis}
      \\&= \norm{\fg\fg^{-1}\vx - \fg\fg^{-1}\vy}
        && \text{by definition of $\hat{\vx}$ and $\hat{\vy}$}
      \\&= \norm{\vx - \vy}
        && \text{by definition of $\fg^{-1}$}
    \end{align*}

  \item Proof that $\fg\vz = \vz$:
    \begin{align*}
      2\lambda
        &= 2\sup\setn{\norm{\fg\vz-\vz}}
        && \text{by definition of $\lambda$ \pref{item:mazur-ulam_lambda}}
      \\&\le 2\norm{\fg\vz-\vz}
        && \text{by definition of $\sup$}
      \\&= \norm{2\vz-2\fg\vz}
      \\&= \norm{\fpsi\fg\vz-\fg\vz}
        && \text{by definition of $\fpsi$ \pref{item:mazur-ulam_psi}}
      \\&= \norm{\fg^{-1}\fpsi\fg\vz-\fg^{-1}\fg\vz}
        && \text{by \pref{item:mazur-ulam_gi}}
      \\&= \norm{\fg^{-1}\fpsi\fg\vz-\vz}
        && \text{by definition of $\fg^{-1}$}
      \\&= \norm{\fpsi\fg^{-1}\fpsi\fg\vz-\vz}
      \\&= \norm{\fg^\ast\vz-\vz}
      \\&\le \lambda
        && \text{by definition of $\lambda$ \pref{item:mazur-ulam_lambda}}
      \\&\implies 2\lambda \le \lambda
      \\&\implies \lambda=0
      \\&\implies \fg\vz = \vz
    \end{align*}

  \item Proof that $\fphi\brp{\frac{1}{2}\vx+\frac{1}{2}\vy} = \frac{1}{2}\fphi\vx + \frac{1}{2}\fphi\vy$:
    \begin{align*}
      \fphi\brp{\frac{1}{2}\vx+\frac{1}{2}\vy}
        &=
      \\&= \frac{1}{2}\fphi\vx + \frac{1}{2}\fphi\vy
    \end{align*}

  \item Proof that $\fphi\brp{[1-\lambda]\vx+\lambda\vy} = [1-\lambda]\fphi\vx + \lambda\fphi\vy$:
    \begin{align*}
      \fphi\brp{[1-\lambda]\vx+\lambda\vy}
        &=
      \\&= [1-\lambda]\fphi\vx + \lambda\fphi\vy
    \end{align*}
\end{enumerate}
\end{proof}


%---------------------------------------
\begin{theorem}[Neumann Expansion Theorem]
\index{Neumann Expansion Theorem}
\index{theorems!Neumann Expansion Theorem}
\label{thm:op_net}
\footnote{
  \citerpg{michel1993}{415}{048667598X}
  }
%---------------------------------------
Let $\opA\in\clOxx$ be an operator on a linear space $\spX$.
Let $\opA^0\eqd \opI$.
\thmbox{\begin{array}{ll}
  \left.\begin{array}{llD}
    1. & \opA          \in \clBxx  & ($\opA$ is bounded) \\
    2. & \normop{\opA} <   1
  \end{array}\right\}
  \implies
  \left\{\begin{array}{lrc>{\ds}l}
    1. & (\opI-\opA)^{-1} &&\text{ exists} \\
    2. & \normop{(\opI-\opA)^{-1}} &\le& \frac{1}{1-\normop{\opA}} \\
    3. & (\opI-\opA)^{-1} &=& \sum_{n=0}^\infty \opA^n  \\
       & \mc{3}{c}{\text{ with uniform convergence}}
  \end{array}\right.
\end{array}}
\end{theorem}


%%---------------------------------------
%\begin{definition}
%\footnote{
%  \citerp{rudinf}{312},
%  \citerpg{michel1993}{431}{048667598X}
%  }
%\label{def:op_types}
%\label{def:op_normal}
%\label{def:op_isometric}
%\label{def:op_unitary}
%\index{operator!normal}      \index{normal operator}
%\index{operator!isometric}   \index{isometric operator}
%\index{operator!unitary}     \index{unitary operator}
%%---------------------------------------
%Let $\spX\eqd\normspaceX$ be a normed linear space.
%Let $\opN$,  $\opM$,  and $\opU$   be operators in $\clBxx$ with adjoints
%    $\opNa$, $\opMa$, and $\opUa$, respectively.
%%Let $\spI$ be the identity operator.
%\defbox{\begin{array}{M lclC}
%  An operator $\opM$ is \hid{isometric} if & \norm{\opM\vx-\opM\vy} &=& \norm{\vx-\vy} & \forall \vx,\vy\in\setX.\\
%  An operator $\opU$ is \hid{unitary}   if & \norm{\opU\vx}         &=& \norm{\vx}     & \forall \vx\in\setX.\\
%  An operator $\opN$ is \hid{normal}    if & \norm{\opNa\vx}        &=& \norm{\opN\vx} & \forall \vx\in\setX.
%\end{array}}
%\end{definition}
%\addtocounter{footnote}{-1}
%\citetblt{
%  \citor{frobenius1878} \\
%  \citorp{frobenius1878b}{391}
%  \cithrp{dieudonne1969}{167}
%  }
%\stepcounter{footnote}
%\citetblt{
%  \citorp{autonne1901}{209} \\
%  \citor{autonne1902} \\
%  \citor{schur1909}
%  \cithr{steen1973}
%  }


%--------------------------------------
\begin{theorem}
%--------------------------------------
Let $\normspaceX$ be a normed linear space.
\thmbox{
  \text{$\opM$ is \prope{isometric}}
  \qquad\implies\qquad
  \no
  }
\end{theorem}



%=======================================
\section{Operators on Inner product spaces}
%=======================================
%=======================================
\subsection{General Results}
%=======================================
\ifdochasnot{vsinprod}{
%--------------------------------------
\begin{definition}
\label{def:inprod}
\footnote{
  \citerp{haaser1991}{277},
  \citerpg{ab}{276}{0120502577},
  \citorp{peano1888e}{72}
  }
\index[xsym]{$\inprodn$}
\index{$\inprodn$}
\index{space!inner product}
%--------------------------------------
Let $\spO\eqd\linearspaceX$ be a linear space.
\defboxt{
  A function $\inprodn\in\clFxxf$ is an \fnctd{inner product} on $\spO$ if
  \\\indentx$\ds\begin{array}{F rcl @{\qquad}C @{\qquad}D @{\qquad}D}
   1. & \inprod{\vx    }{\vx} &\ge& 0
      & \forall \vx\in\setX
      & (\prop{non-negative})
      & and
      \\
   2. & \inprod{\vx    }{\vx} &=& 0 \iff \vx=\vzero
      & \forall \vx\in\setX
      & (\prop{nondegenerate})
      & and
      \\
   3. & \inprod{\alpha\vx}{\vy}    &=& \alpha\inprod{\vx}{{\vy}}
      & \forall \vx,\vy\in\setX,\;\forall\alpha\in\C
      & (\prop{homogeneous})
      & and
      \\
   4. & \inprod{\vx+\vy}{\vu} &=& \inprod{\vx}{{\vu}} + \inprod{\vy}{{\vu}}
      & \forall \vx,\vy,\vu\in\setX
      & (\prop{additive})
      & and
      \\
   5. & \inprod{\vx    }{\vy} &=& \inprod{\vy}{\vx}^\ast
      & \forall \vx,\vy\in\setX
      & (\prop{conjugate symmetric}).
  \end{array}$\\
  An inner product is also called a \fnctd{scalar product}.\\
  An \structd{inner product space} is the pair $\opair{\spO}{\inprodn}$.
  }
\end{definition}
}

%---------------------------------------
\begin{theorem}
\label{thm:op_inprod}
\footnote{
  \citerpgc{rudinf}{310}{0070542252}{Theorem 12.7, Corollary}
  }
%---------------------------------------
Let $\opA,\opB\in\clBxx$ be bounded linear operators on an inner product space $\spX\eqd\inprodspaceX$.
\thmbox{\begin{array}{rclC @{\qquad}c@{\qquad} rclC}
  \inprod{\opB\vx}{\vx}&=&0 &\forall \vx\in\setX
  &\iff&
  \opB\vx&=&\vzero &\forall\vx\in\setX
  \\
  \inprod{\opA\vx}{\vx}&=&\inprod{\opB\vx}{\vx} &\forall \vx\in\setX
  &\iff&
  \opA&=&\opB
\end{array}}
\end{theorem}
\begin{proof}
\begin{enumerate}
\item Proof that $\inprod{\opB\vx}{\vx}=0\implies\opB\vx=\vzero$:
  \begin{align*}
    0
      &= \inprod{\opB(\vx+\opB\vx)}{(\vx+\opB\vx)} + i\inprod{\opB(\vx+i\opB\vx)}{(\vx+i\opB\vx)}
      && \text{by left hypothesis}
    \\&=  \brb{\inprod{\opB\vx+\opB^2\vx)}{\vx+\opB\vx}}
       + i\brb{\inprod{\opB\vx+i\opB^2\vx)}{\vx+i\opB\vx}}
      %&& \text{by \prope{additive} property of $\opB$} 
      && \text{by \prefp{def:linop}}
    \\&=  \brb{\inprod{\opB\vx}{\vx}+ \inprod{\opB\vx}{\opB\vx}+ \inprod{\opB^2\vx}{\vx}+   \inprod{\opB^2\vx}{\opB\vx}}
      %&& \text{by \prope{additive} property of $\inprodn$} 
      %&& \text{\xref{def:inprod}}
       && \text{by \prefp{def:inprod}}
      \\&\qquad + i\brb{\inprod{\opB\vx}{\vx}-i\inprod{\opB\vx}{\opB\vx}+i\inprod{\opB^2\vx}{\vx}-i^2\inprod{\opB^2\vx}{\opB\vx}}
    \\&=\mathrlap{%
          \brb{0                    + \inprod{\opB\vx}{\opB\vx}+ \inprod{\opB^2\vx}{\vx}+   0                          }
       + i\brb{0                    -i\inprod{\opB\vx}{\opB\vx}+i\inprod{\opB^2\vx}{\vx}-i^20                          }
       \quad\text{by left hypothesis}
       }
    \\&=  \brb{                       \inprod{\opB\vx}{\opB\vx}+ \inprod{\opB^2\vx}{\vx}                               }
       +  \brb{                       \inprod{\opB\vx}{\opB\vx}- \inprod{\opB^2\vx}{\vx}                               }
    \\&=  2\inprod{\opB\vx}{\opB\vx}
    \\&=  2\norm{\opB\vx}^2
    \\&\implies\opB\vx=\vzero
      %&& \text{by \prope{nondegenerate} property of $\normn$} 
      %&& \text{\xref{def:norm}}
      && \text{by \prefp{def:norm}}
  \end{align*}

\item Proof that $\inprod{\opB\vx}{\vx}=0\impliedby\opB\vx=\vzero$: by property of inner products\ifsxref{vsinprod}{thm:inprod_prop}.

\item Proof that $\inprod{\opA\vx}{\vx}=\inprod{\opB\vx}{\vx}\implies\opA\eqo\opB$:
  \begin{align*}
    0
      &= \inprod{\opA\vx}{\vx}-\inprod{\opB\vx}{\vx}
      && \text{by left hypothesis}
    \\&= \inprod{\opA\vx-\opB\vx}{\vx}
      && \text{by \prope{additivity} property of $\inprodn$ (\prefp{def:inprod})}
    \\&= \inprod{(\opA-\opB)\vx}{\vx}
      && \text{by definition of operator addition\ifsxref{relation}{def:op+x}}
    \\\implies (\opA-\opB)\vx&=\vzero
      && \text{by item 1}
    \\\implies \opA &= \opB
      && \text{by definition of operator subtraction\ifsxref{relation}{def:op+x}}
  \end{align*}

\item Proof that $\inprod{\opA\vx}{\vx}=\inprod{\opB\vx}{\vx}\impliedby\opA\eqo\opB$:
  \begin{align*}
    \inprod{\opA\vx}{\vx}
      &= \inprod{\opB\vx}{\vx}
      && \text{by $\opA\eqo\opB$ hypothesis}
  \end{align*}
\end{enumerate}
\end{proof}

%=======================================
\subsection{Operator adjoint}
%=======================================
A fundamental concept of operators on inner product spaces is the
\hie{operator adjoint} (\prefp{prop:op_adjoint}).
The adjoint of an operator is a kind of generalization of the conjugate
of a complex number in that
\begin{liste}
  \item Both are \hie{star-algebras} (\prefp{thm:op_star}).
  \item Both support decomposition into ``real" and ``imaginary" parts\ifdochas{normalg}{ (\prefp{thm:nalg_re_im})}.
\end{liste}
Structurally, the operator adjoint provides a convenient symmetric relationship
between the \hie{range space} and \hie{null space} of an operator
(\prefp{thm:oppN_oppR}).

%---------------------------------------
\begin{proposition}
\label{prop:op_adjoint}
\footnote{
  \citerpg{michel1993}{220}{048667598X},
  \citerpg{rudinf}{311}{0070542252},
  \citerp{giles2000}{182},
  %\nocite{pedersen2000} % page 55
  %\nocite{ab} % page 297
  \citorp{vonNeumann1929}{49},
  \citorp{stone1932}{41}
  }
%---------------------------------------
Let $\clBhh$ be the space of bounded linear operators on a Hilbert space $\spH$.\footnote{%
  \structe{bounded operator}: \prefpp{def:clB}; 
  \structe{adjoint}:          \prefpp{def:norm_adjoint}
  }
\propbox{\begin{array}{l@{\qquad}l}
  \mc{2}{l}{\text{An operator $\opBa$ is the \structe{adjoint} of $\opB\in\clBhh$ if}}\\
  \qquad\inprod{\opB\vx}{\vy} = \inprod{\vx}{\opBa\vy} \qquad \forall \vx,\vy\in\spH.
\end{array}}
\end{proposition}
\begin{proof}
\begin{enumerate}
  \item For fixed $\fy$, $\ff(\vx)\eqd\inprod{\vx}{\vy}$ is a \structe{functional} 
        in $\clFxf$\ifsxref{functionals}{def:functional}.
  \item $\opBa$ is the \structe{adjoint} of $\opB$ because
    \begin{align*}
      \inprod{\opB\vx}{\vy}
        &\eqd \ff(\opB\vx)
        && \ifdochas{functionals}{\text{by \prefpp{def:functional}}}
      \\&\eqd \opBa\ff(\vx)
        && \text{by \prefpp{def:norm_adjoint}}
      \\&=    \inprod{\vx}{\opBa\vy}
        && \ifdochas{functionals}{\text{by \prefpp{def:functional}}}
    \end{align*}
\end{enumerate}
\end{proof}

%---------------------------------------
\begin{example}[Matrix algebra: $\opAa=\opA^H$]
\index{adjoint}
%---------------------------------------
In matrix algebra,
\exbox{\begin{tabular}{>{$\imark\quad$}l}
  The inner product operation $\inprod{\vx}{\vy}$ is represented by
         $\vy^H \vx$.  \\
  The linear operator is represented as a matrix $\vA$. \\
  The operation of $\vA$ on vector $\vx$ is represented as $\vA\vx$. \\
  The adjoint of matrix $\vA$ is the Hermitian matrix $\vA^H$.
  \end{tabular}}
\end{example}
\begin{proof}
\begin{align*}
  \inprod{\vA\vx}{\vy}
    &\eqd  \vy^H \vA\vx
     =     [(\vA\vx)^H \vy ]^H
     =     [\vx^H\vA^H \vy ]^H
     =      (\vA^H \vy)^H \vx
     \eqd  \inprod{\vx}{\vA^H \vy}
\end{align*}
\end{proof}



%%---------------------------------------
%\begin{definition}
%\label{def:op_ops}
%\citep{bollobas1999}{205}
%\index{square}         \index{operator!square}
%\index{square root}    \index{operator!square root}
%\index{absolute value} \index{operator!absolute value}
%%---------------------------------------
%Let $\opA\in\clOxy$ be an operator with adjoint $\opAa$.
%\defbox{\begin{array}{>{$}l<{$} rcl}
%  The {\bf square        } of an operator $\opA$ is          & \opA^2      &\eqd& \opA\opA           \\
%  The {\bf square root   } of an operator $\opA$ is     & \sqrt{\opA} &\eqd& \opB \st \opB^2 = \opA  \\
%  The {\bf absolute value} of an operator $\opA$ is  & \abs{\opA}  &\eqd& \sqrt{\opAa\opA}
%\end{array}}
%\end{definition}




Structures that satisfy the four conditions of the next theorem are known as
\structe{*-algebras} (``\hie{star-algebras}"\ifdochas{normalg}{, \prefp{def:star_algebra}}).
Other structures which are *-algebras include
the \structe{field of complex numbers} $\C$ and
any \structe{ring of complex square $n\times n$ matrices}.\footnote{\citerpg{sakai}{1}{3540636331}}
%---------------------------------------
\begin{theorem}[\thm{operator star-algebra}]
\footnote{
  \citerpp{halmos}{39}{40},
  \citerpg{rudinf}{311}{0070542252}
  }
\label{thm:op_star}
%---------------------------------------
Let $\spH$ be a Hilbert space with operators $\opA,\opB\in\clBhh$ and
with adjoints $\opAa,\opBa\in\clBhh$.
Let $\bar{\alpha}$ be the complex conjugate of some $\alpha\in\C$.
\thmbox{\begin{array}{>{\qquad\scy}rrcl@{\qquad}C@{\qquad}D@{\qquad}D}
  \mc{7}{M}{The pair $\opair{\spH}{\ast}$ is a \structe{$\ast$-algebra} (\structe{star-algebra}).
  In particular,}
  \\
    1. & (\opA\addo \opB)^\ast   &=& \opAa + \opBa          & \forall \opA,\opB\in\spH & (\prope{distributive})     & and \\
    2. & (\alpha \opA)^\ast      &=& \bar{\alpha} \opA^\ast & \forall \opA,\opB\in\spH & (\prope{conjugate linear}) & and \\
    3. & (\opA \opB)^\ast        &=& \opBa \opAa            & \forall \opA,\opB\in\spH & (\prope{antiautomorphic})  & and \\
    4. & \opA^{\ast\ast}         &=& \opA                   & \forall \opA,\opB\in\spH & (\prope{involutary})
\end{array}}
\end{theorem}
\begin{proof}
\begin{align*}
  \inprod{\vx}{(\opA\addo \opB)^\ast\vy}
    &= \inprod{(\opA\addo \opB)\vx}{\vy}
    && \text{by definition of adjoint} &&\text{\xref{prop:op_adjoint}}
  \\&= \inprod{\opA\vx}{\vy} + \inprod{\opB\vx}{\vy}
    && \text{by definition of inner product} && \text{\xref{def:inprod}}
  \\&= \inprod{\vx}{\opAa\vy} + \inprod{\vx}{\opBa\vy}
    && \text{by definition of operator addition} &&\text{\ifxref{relation}{def:op+x}}
  \\&= \inprod{\vx}{\opAa\vy +\opBa\vy}
    && \text{by definition of inner product} &&\text{\xref{def:inprod}}
  \\&= \inprod{\vx}{(\opAa +\opBa)\vy}
    && \text{by definition of operator addition} &&\text{\ifxref{relation}{def:op+x}}
  \\
  \\
  \inprod{\vx}{(\alpha\opA)^\ast\vy}
    &= \inprod{(\alpha\opA)\vx}{\vy}
    && \text{by definition of adjoint} &&\text{\xref{prop:op_adjoint}}
  \\&= \inprod{\alpha (\opA\vx)}{\vy}
    && \text{by definition of scalar multiplication} &&\text{\ifxref{relation}{def:op+x}}
  \\&= \alpha \inprod{\opA\vx}{\vy}
    && \text{by definition of inner product} &&\text{\xref{def:inprod}}
  \\&= \alpha \inprod{\vx}{\opAa\vy}
    && \text{by definition of adjoint} &&\text{\xref{prop:op_adjoint}}
  \\&= \inprod{\vx}{\alpha^\ast \opAa\vy}
    && \text{by definition of inner product} &&\text{\xref{def:inprod}}
  \\
  \\
  \inprod{\vx}{(\opA\opB)^\ast\vy}
    &= \inprod{(\opA\opB)\vx}{\vy}
    && \text{by definition of adjoint} &&\text{\xref{prop:op_adjoint}}
  \\&= \inprod{\opA(\opB\vx)}{\vy}
    && \text{by definition of operator multiplication} &&\text{\ifsxref{relation}{def:op+x}}
  \\&= \inprod{(\opB\vx)}{\opAa\vy}
    && \text{by definition of adjoint} &&\text{\xref{prop:op_adjoint}}
  \\&= \inprod{\vx}{\opBa \opAa\vy}
    && \text{by definition of adjoint} &&\text{\xref{prop:op_adjoint}}
  \\
  \\
  \inprod{\vx}{\opA^{\ast\ast}\vy}
    &= \inprod{\opAa\vx}{\vy}
    && \text{by definition of adjoint} &&\text{\xref{prop:op_adjoint}}
  \\&= \inprod{\vy}{\opAa\vx}^\ast
    && \text{by definition of inner product} &&\text{\xref{def:inprod}}
  \\&= \inprod{\opA\vy}{\vx}^\ast
    && \text{by definition of adjoint} &&\text{\xref{prop:op_adjoint}}
  \\&= \inprod{\vx}{\opA\vy}
    && \text{by definition of inner product} && \text{\xref{def:inprod}}
\end{align*}
\end{proof}

%---------------------------------------
\begin{theorem}
\label{thm:oppN_oppR}
\footnote{
  \citerpg{rudinf}{312}{0070542252}
  }
%\footnote{more properties in \citerpg{michel1993}{426}{048667598X}}
\index{operator!null space}
\index{operator!range}
%---------------------------------------
Let $\clOxy$ be the set of all operators from a linear space $\spX$ to a linear space $\spY$.
Let $\oppN(\opL)$ be the \hie{null space} of an operator $\opL$ in $\clOxy$
and $\oppI(\opL)$ the \hie{image set} of $\opL$ in $\clOxy$\ifsxref{relation}{def:rel_null}.
\thmbox{\begin{aligned}
  \oppN(\opA ) &= \oppI(\opAa)^\perp  \\
  \oppN(\opAa) &= \oppI(\opA )^\perp
  \end{aligned}}
\end{theorem}
\begin{proof}
\begin{align*}
  \oppI(\opAa)^\perp
    &= \set{y\in\spH}{\inprod{\vy}{\vu}=0 \quad \forall u\in\oppI(\opAa)}
    && %\text{by \prefp{def:space_op} \prefpo{def:space_op}}
  \\&= \set{y\in\spH}{\inprod{\vy}{\opAa\vx}=0 \quad \forall\vx\in\spH}
    && \ifdochas{relation}{\text{by \prefp{def:rel_range}}}
  \\&= \set{y\in\spH}{\inprod{\opA\vy}{\vx}=0 \quad \forall\vx\in\spH}
    && \text{by definition of $\opAa$}&&\text{\xref{prop:op_adjoint}}
  \\&= \set{y\in\spH}{\opA\vy=0 }
  \\&= \oppN(\opA)
    && \text{by definition of $\oppN(\opA)$\ifsxref{relation}{def:rel_null}}
  \\
  \\
  \oppI(\opA)^\perp
    &= \set{y\in\spH}{\inprod{\vy}{\vu}=0 \quad \forall u\in\oppI(\opA)}
    && %\text{by definition of $\opA^\perp$ \prefp{def:space_op}}
  \\&= \set{y\in\spH}{\inprod{\vy}{\opA\vx}=0 \quad \forall\vx\in\spH}
    && \text{by definition of $\oppI$ \ifdochas{relation}{\prefp{def:rel_range}}}
  \\&= \set{y\in\spH}{\inprod{\opAa\vy}{\vx}=0 \quad \forall\vx\in\spH}
    && \text{by definition of $\opAa$}&&\text{\xref{prop:op_adjoint}}
  \\&= \set{y\in\spH}{\opAa\vy=0 }
  \\&= \oppN(\opAa)
    && \text{by definition of $\oppN(\opA)$\ifsxref{relation}{def:rel_null}}
\end{align*}
\end{proof}


%=======================================
\section{Special Classes of Operators}
%=======================================
%=======================================
\subsection{Projection operators}
%=======================================
%---------------------------------------
\begin{definition}
\footnote{
  \citerpgc{rudinf}{133}{0070542252}{5.15 Projections},
  \citerpg{kubrusly2001}{70}{0817641742},
  \citerpg{bachman1966}{6}{0486402517},
  \citerpgc{halmos1958}{73}{0387900934}{\textsection 41. Projections}
  }
\label{def:opP}
\index{operator!projection}
\index{projection operator}
%---------------------------------------
Let $\clBxy $ be the space of bounded linear operators on normed linear spaces $\spX$ and $\spY$.
Let $\opP$ be a bounded linear operator in $\clBxy$.
\defbox{\text{
  $\opP$ is a \opd{projection} operator if \quad $\opP^2=\opP$.
  }}
\end{definition}

%---------------------------------------
\begin{theorem}
\label{thm:op_RPNP}
\footnote{
  \citerppg{michel1993}{120}{121}{048667598X}
  }
%---------------------------------------
Let $\clBxy $ be the space of bounded linear operators on normed linear spaces $\spX$ and $\spY$.
Let $\opP$ be a bounded linear operator in $\clBxy$
with \hie{null space} $\oppN(\opP)$ and \hie{image set} $\oppI(\opP)$\ifsxref{relation}{def:rel_null}.
%\footnote{%
%  \begin{tabular}{ll}
%    \structe{null space}: & \prefp{def:rel_null} \\
%    \structe{image set}:  & \prefp{def:rel_image}
%  \end{tabular}}
\thmbox{
  \left.\begin{array}{>{\scy}rrclDD}
    1. & \opP^2   &=& \opP                 & ($\opP$ is a projection operator)     & and \\
    2. & \spO     &=& \spX \adds \spY      & ($\spY$ compliments $\spX$ in $\spO$) & and \\
    3. & \opP\spO &=& \spX                 & ($\opP$ projects onto $\spX$)         &
  \end{array}\right\}
  \implies
  \left\{\begin{array}{>{\scy}rrclD}
    1. & \oppI(\opP) &=& \spX                           & and \\
    2. & \oppN(\opP) &=& \spY                           & and \\
    3. & \spO        &=& \oppI(\opP) \adds \oppN(\opP)
  \end{array}\right.
  }
\end{theorem}
\begin{proof}
\begin{align*}
  \oppI(\opP)
    &= \opP\spO
  \\&= \opP(\spO_1 + \spO_2)
  \\&= \opP\spO_1 + \opP\spO_2
  \\&= \spO_1 + \setn{\vzero}
  \\&= \spO_1
  \\
  \\
  \oppN(\opP)
    &= \set{\vx\in\spO}{\opP\vx=\vzero}
  \\&= \set{\vx\in(\spO_1+\spO_2)}{\opP\vx=\vzero}
  \\&= \set{\vx\in\spO_1}{\opP\vx=\vzero} + \set{\vx\in\spO_2}{\opP\vx=\vzero}
  \\&= \setn{\vzero} + \spO_2
  \\&= \spO_2
\end{align*}
\end{proof}

%---------------------------------------
\begin{theorem}
\label{thm:op_PIP}
\footnote{
  \citerpg{michel1993}{121}{048667598X}
  }
%---------------------------------------
Let $\clBxy $ be the space of bounded linear operators on normed linear spaces $\spX$ and $\spY$.
Let $\opP$ be a bounded linear operator in $\clBxy$.
\thmbox{
  \mcom{\opP^2=\opP}{$\opP$ is a projection operator}
  \qquad\iff\qquad
  \mcom{(\opI-\opP)^2=(\opI-\opP)}{$(\opI-\opP)$ is a projection operator}
  }
\end{theorem}
\begin{proof}
\begin{align*}
  \intertext{$\imark$ Proof that
    $\opP^2=\opP \implies (\opI-\opP)^2=(\opI-\opP)$:}
  (\opI-\opP)^2
    &= (\opI-\opP)(\opI-\opP)
    && \ifdochas{relation}{\text{by \prefp{def:op+x}}}
  \\&= \opI(\opI-\opP) + (-\opP)(\opI-\opP)
    && \ifdochas{relation}{\text{by \prefp{def:op+x}}}
  \\&= \opI-\opP -\opP\opI + \opP^2
    && \ifdochas{relation}{\text{by \prefp{def:op+x}}}
  \\&= \opI-\opP -\opP + \opP
    && \text{by left hypothesis}
  \\&= \opI-\opP
    && \ifdochas{relation}{\text{by \prefp{def:op+x}}}
  \\
  \intertext{$\imark$ Proof that
    $\opP^2=\opP \impliedby (\opI-\opP)^2=(\opI-\opP)$:}
  \opP^2
    &= \mcom{\opI -\opP -\opP + \opP^2}{$(\opI-\opP)^2$} -(\opI -\opP-\opP)
  \\&= (\opI-\opP)^2 -(\opI -\opP-\opP)
  \\&= (\opI-\opP) -(\opI -\opP-\opP)
    && \text{by right hypothesis}
  \\&= \opP
    && \ifdochas{relation}{\text{by \prefp{def:op+x}}}
\end{align*}
\end{proof}

%%---------------------------------------
%\begin{theorem}
%\footnote{
%  \citerpg{rudinf}{314}{0070542252}
%  }
%%---------------------------------------
%Let $\spH$ be a Hilbert space and $\opP$ an operator in $\clOhh$
%with adjoint $\opPa$, \hie{null space} $\oppN(\opP)$, and \hie{image set} $\oppI(\opP)$.%
%\footnote{%
%  \begin{tabular}{ll}
%    \structe{null space}: & \prefp{def:rel_null} \\
%    \structe{image set}:  & \prefp{def:rel_image}
%  \end{tabular}}
%\thmbox{\begin{array}{>{\quad}F rcl CDc}
%  \mc{7}{M}{If $\opP$ is a \structe{projection operator}, then the following are equivalent:}\\
%  \cnto & \opPa     &=& \opP
%        &
%        &($\opP$ is \prope{self-adjoint})
%        & \iff
%  \cntn & \opPa\opP &=& \opP\opPa
%        &
%        &($\opP$ is \prope{normal})
%        & \iff
%  \cntn & \oppI(\opP) &=& \oppN(\opP)^\perp
%        &
%        &
%        & \iff
%  \cntn & \inprod{\opP\vx}{\vx} &=& \norm{\opP\vx}^2
%        & \forall \vx\in\setX
%        &
%        &
%  \end{array}}
%\end{theorem}
%\begin{proof}
%This proof is incomplete at this time. \problem
%\begin{align*}
%\intertext{Proof that (1)$\implies$(2):}
%  \opPa\opP
%    &= \opP^{\ast\ast}\opPa
%    && \text{by (1)}
%  \\&= \opP\opPa
%    && \text{by \prefp{thm:op_star}}
%  \\
%\intertext{Proof that (1)$\implies$(3):}
%  \oppI(\opP)
%    &= \oppN(\opPa)^\perp
%    && \text{by \prefp{thm:oppN_oppR}}
%  \\&= \oppN(\opP)^\perp
%    && \text{by (1)}
%  \\
%\intertext{Proof that (3)$\implies$(4):}
%\intertext{Proof that (4)$\implies$(1):}
%\end{align*}
%\end{proof}
%





%=======================================
\subsection{Self Adjoint Operators}
%=======================================
%---------------------------------------
\begin{definition}
  \footnote{
    Historical works regarding self-adjoint operators:
    \quad  \citorp{vonNeumann1929}{49},
    \qquad ``linearer Operator R selbstadjungiert oder Hermitesch",
    \quad  \citorpc{stone1932}{50}{``self-adjoint transformations"}
    }
\label{def:op_selfadj}
%---------------------------------------
Let $\opB\in\clBhh$ be a bounded operator with adjoint $\opBa$ on a Hilbert space $\spH$.
\defbox{
  \text{The operator $\opB$ is said to be \propd{self-adjoint} or \propd{hermitian} if $\opB\eqo\opBa$}.
  }
\end{definition}

%---------------------------------------
\begin{example}[Autocorrelation operator]
\index{autocorrelation}
\index{operator!autocorrelation}
%---------------------------------------
Let $\fx(t)$ be a random process with autocorrelation
\\\indentx$\Rxx(t,u) \eqd \mcom{\pE[ \fx(t)\fx^\ast(u) ]}{expectation}$.
\\Let an autocorrelation operator $\opR$ be defined as
  $\ds
    [\opR \ff](t) \eqd \int_\R \mcom{\Rxx(t,u)}{kernel} \ff(u) \du.
  $
\exbox{
  \opR = \opRa
  \qquad
  \text{(The autocorrelation operator $\opR$ is \prope{self-adjoint})}
  }
\end{example}
\begin{proof}
\begin{align*}
  \intertext{1. First note that the \structe{autocorrelation kernel} $\Rxx(t,u)$ is \prope{hermitian symmetric}:}
      \Rxx(t,u)
        &\eqd \pE  \fx(t) \fx^\ast(u)
         =     [\pE \fx^\ast(t) \fx(u)]^\ast
         =     [\pE \fx(u) \fx^\ast(t)]^\ast
      \\&=     \Rxx^\ast(u,t)
  \intertext{2. Proof that the \structe{autocorrelation operator} $\opR$ is \prope{self-adjoint}:}
      \inprod{\opR \ff}{\fg}
        &= \inprod{\int_{u\in\R}\Rxx(t,u) \ff(u) \du}{\fg(t)}
        && \text{by definition of $\opR$}
      \\&= \int_{u\in\R}\ff(u) \inprod{\Rxx(t,u) }{\fg(t)} \du
      \\&= \int_{u\in\R}\ff(u) \int_t \Rxx(t,u) \fg^\ast(t) \dt \du
      \\&= \int_{u\in\R}\ff(u) \int_t \Rxx^\ast(u,t) \fg^\ast(t) \dt \du
        && \text{by 1.}
      \\&= \int_{u\in\R}\ff(u) \left[\int_t \Rxx(u,t) \fg(t) \dt\right]^\ast \du
      \\&= \int_{u\in\R}\ff(u) \left[\opR \fg \right]^\ast \du
        && \text{by definition of $\opR$}
      \\&= \inprod{ \ff}{\opR \fg}
  \end{align*}
\end{proof}




%---------------------------------------
\begin{theorem}
\label{thm:self_adjoint}
\index{operator!self-adjoint}
\footnote{
  \citerpp{lax}{315}{316},
  \citerpp{keener}{114}{119}
  }
%---------------------------------------
Let $\opS:\spH\to\spH$ be an operator over a Hilbert space $\spH$
with eigenvalues $\{\lambda_n\}$ and eigenfunctions $\{\fpsi_n\}$
such that $\opS \fpsi_n = \lambda_n\fpsi_n$ and let
$\norm{\vx}\eqd\sqrt{\inprod{\vx}{\vx}}$.
\thmbox{
  \mcom{\opS = \opSa}{$\opS$ is \prope{self adjoint}}
  \implies
  \left\{\begin{array}{rlp{52mm}}
    1. & \inprod{\opS\vx}{\vx}\in\R
       & {\scs
         (the hermitian quadratic form of $\opS$ is real)}
       \\
    2. & \lambda_n \in \R
       & {\scs
         (eigenvalues of $\opS$ are real)}
       \\
    3. & \lambda_n\ne \lambda_m\implies\inprod{\psi_n}{\psi_m}=0
       & {\scs
         (eigenfunctions associated with distinct eigenvalues are orthogonal)}
  \end{array}\right.
}
\end{theorem}

\begin{proof}
  \begin{align*}
    \intertext{1. Proof that $\opS=\opSa  \implies  \inprod{\opS\vx}{\vx}\in\R$:}
    \inprod{\vx}{\opS\vx}
      &= \inprod{\opS\vx}{\vx}
      && \text{by left hypothesis}
    \\&= \inprod{\vx}{\opS\vx}^\ast
      && \text{by definition of $\inprodn$ \prefp{def:inprod}}
    \\
    \intertext{2. Proof that $\opS=\opSa \implies \lambda_n \in \R $:}
    \lambda_n\norm{\psi_n}^2
      &= \lambda_n \inprod{\psi_n}{\psi_n}
      && \text{by definition}
    \\&= \inprod{\lambda_n \psi_n}{\psi_n}
      && \text{by definition of $\inprodn$ \prefp{def:inprod}}
    \\&= \inprod{\opS \psi_n}{\psi_n}
      && \text{by definition of eigenpairs}
    \\&= \inprod{\psi_n}{\opS \psi_n}
      && \text{by left hypothesis}
    \\&= \inprod{\psi_n}{\lambda_n \psi_n}
      && \text{by definition of eigenpairs}
    \\&= \lambda_n^\ast \inprod{\psi_n}{\psi_n}
      && \text{by definition of $\inprodn$ \prefp{def:inprod}}
    \\&= \lambda_n^\ast \norm{\psi_n}^2
      && \text{by definition}
    \\
    \intertext{3. Proof that $\opS=\opSa \implies
     [\lambda_n\ne \lambda_m\implies\inprod{\psi_n}{\psi_m}=0]$:}
    \lambda_n\inprod{\psi_n}{\psi_m}
      &= \inprod{\lambda_n\psi_n}{\psi_m}
      && \text{by definition of $\inprodn$ \prefp{def:inprod}}
    \\&= \inprod{\opS\psi_n}{\psi_m}
      && \text{by definition of eigenpairs}
    \\&= \inprod{\psi_n}{\opS\psi_m}
      && \text{by left hypothesis}
    \\&= \inprod{\psi_n}{\lambda_m\psi_m}
      && \text{by definition of eigenpairs}
    \\&= \lambda_m^\ast\inprod{\psi_n}{\psi_m}
      && \text{by definition of $\inprodn$ \prefp{def:inprod}}
    \\&= \lambda_m\inprod{\psi_n}{\psi_m}
      && \text{because $\lambda_m$ is real}
    \\
    \intertext{This implies for $\lambda_n\ne \lambda_m\ne 0$,
               $\inprod{\psi_n}{\psi_m}=0$.}
\end{align*}
\end{proof}



%=======================================
\subsection{Normal Operators}
%=======================================
%---------------------------------------
\begin{definition}
\footnote{
  \citerpg{rudinf}{312}{0070542252},
  \citerpg{michel1993}{431}{048667598X},
  \citerp{dieudonne1969}{167},
  \citor{frobenius1878},
  \citorp{frobenius1878b}{391}
  }
\label{def:op_normal}
\index{operator!normal}      \index{normal operator}
\index{operator!isometric}   \index{isometric operator}
\index{operator!unitary}     \index{unitary operator}
%---------------------------------------
Let $\clBxy $ be the space of bounded linear operators on normed linear spaces $\spX$ and $\spY$.
Let $\opNa$ be the adjoint of an operator $\opN\in\clBxy$.
\defbox{\begin{array}{M rcl}
  $\opN$ is \hid{normal} if & \opNa\opN &=& \opN\opNa .
\end{array}}
\end{definition}


%---------------------------------------
\begin{theorem}
\footnote{
  \citerppg{rudinf}{312}{313}{0070542252}
  }
\label{thm:op_norm_prop}
\index{operator!normal}
\index{normal}
%---------------------------------------
Let $\clBhh $ be the space of bounded linear operators on a Hilbert space $\spH$.
Let $\oppN(\opN)$ be the \hie{null space} of an operator $\opN$ in $\clBhh$
and $\oppI(\opN)$ the \hie{image set} of $\opN$ in $\clBhh$\ifsxref{relation}{def:rel_null}.
%\footnote{\structe{image set}: \prefp{def:rel_null}}
\thmbox{
  \mcom{\opNa\opN=\opN \opNa}{$\opN$ is normal}
  \qquad\iff\qquad
  \norm{\opNa\vx} = \norm{\opN\vx} \qquad \forall\vx\in\spH
  }
\end{theorem}
\begin{proof}
\begin{align*}
  \intertext{1. Proof that
    $\opNa\opN=\opN \opNa  \implies \norm{\opNa\vx} = \norm{\opN\vx} $:}
  \norm{\opN\vx}^2
    &= \inprod{\opN\vx}{\opN\vx}
    && \text{by definition}
  \\&= \inprod{\vx}{\opNa \opN\vx}
    && \text{by \prefp{prop:op_adjoint} (definition of $\opNa$)}
  \\&= \inprod{\vx}{\opN \opNa\vx}
    && \text{by left hypothesis ($\opN$ is normal)}
  \\&= \inprod{\opN\vx}{\opNa\vx}
    && \text{by \prefp{prop:op_adjoint} (definition of $\opNa$)}
  \\&= \norm{\opNa\vx}^2
    && \text{by definition}
  \\
  \intertext{2. Proof that
    $\opNa\opN=\opN \opNa  \impliedby \norm{\opNa\vx} = \norm{\opN\vx} $:}
  \inprod{\opNa \opN\vx}{\vx}
    &= \inprod{\opN\vx}{\opN^{\ast\ast}\vx}
    && \text{by \prefp{prop:op_adjoint} (definition of $\opNa$)}
  \\&= \inprod{\opN\vx}{\opN\vx}
    &&    \text{by \prefp{thm:op_star} (property of adjoint)}
  \\&= \norm{\opN\vx}^2
    && \text{by definition}
  \\&= \norm{\opNa\vx}^2
    && \text{by right hypothesis $(\norm{\opNa\vx} = \norm{\opN\vx})$ }
  \\&= \inprod{\opNa\vx}{\opNa\vx}
    && \text{by definition}
  \\&= \inprod{\opN \opNa\vx}{\vx}
    && \text{by \prefp{prop:op_adjoint} (definition of $\opNa$)}
  \end{align*}
\end{proof}

%---------------------------------------
\begin{theorem}
\footnote{
  \citerppg{rudinf}{312}{313}{0070542252}
  }
\label{thm:op_norm_prop2}
\index{operator!normal}
\index{normal}
%---------------------------------------
Let $\clBhh $ be the space of bounded linear operators on a Hilbert space $\spH$.
Let $\oppN(\opN)$ be the \hie{null space} of an operator $\opN$ in $\clBhh$
and $\oppI(\opN)$ the \hie{image set} of $\opN$ in $\clBhh$\ifsxref{relation}{def:rel_null}.
%\footnote{\structe{image set}: \prefp{def:rel_null}}
\thmbox{
  \mcom{\opNa\opN=\opN\opNa}{$\opN$ is normal}
  \qquad\implies\qquad
  \mcom{\oppN(\opNa) = \oppN(\opN)}{$\opN$ and $\opNa$ have the same null space}
  }
\end{theorem}
\begin{proof}
\begin{align*}
  \oppN(\opNa)
    &= \set{\vx}{\opNa\vx=0 \quad \forall\vx\in\spX}
    && \text{\ifdochas{relation}{by \ifdochas{relation}{\prefp{def:rel_null}}} (definition of $\oppN$)}
  \\&= \set{\vx}{\norm{\opNa\vx}=0 \quad \forall\vx\in\spX}
    && \text{by definition of $\normn$ (\prefp{def:norm})}
  \\&= \set{\vx}{\norm{\opN\vx}=0 \quad \forall\vx\in\spX}
%    && \text{by left hypothesis and \prefp{thm:norm_iff_prop} ($\norm{\opNa\vx}=\norm{\opN\vx}$)}
  \\&= \set{\vx}{\opN\vx=0 \quad \forall\vx\in\spX}
    && \text{by definition of $\normn$ (\prefp{def:norm})}
  \\&= \oppN(\opN)
    && \text{\ifdochas{relation}{by \ifdochas{relation}{\prefp{def:rel_null}}} (definition of $\oppN$)}
  \end{align*}
\end{proof}


%%---------------------------------------
%\begin{theorem}
%\footnote{
%  \citerppg{rudinf}{312}{313}{0070542252}
%  }
%\label{thm:op_norm_prop}
%\index{operator!normal}
%\index{normal}
%%---------------------------------------
%Let $\clBhh $ be the space of bounded linear operators on a Hilbert space $\spH$.
%Let $\oppN(\opN)$ be the \hie{null space} of an operator $\opN$ in $\clBhh$
%and $\oppI(\opN)$ the \hie{image set} of $\opN$ in $\clBhh$.
%\footnote{\structe{image set}: \prefp{def:rel_null}}
%\thmbox{
%  \mcom{\opNa\opN=\opN\opNa}{$\opN$ is normal}
%  \qquad\implies\qquad
%  \brbl{\mcom{\lambda_n\ne \lambda_m\implies\inprod{\psi_n}{\psi_m}=0}
%             {eigenfunctions associated with distinct eigenvalues are orthogonal}
%       }
%  }
%\end{theorem}
%\begin{proof}
%The proof in (1) is flawed. This implies that (2) is also flawed. [Rudin] claims both to be true.\citep{rudinf}{313} \problem
%\begin{align*}
%  \intertext{1. Proof that
%  $\opNa\opN=\opN\opNa \implies \opNa\fpsi = \lambda^\ast \fpsi$:}
%  \opN \fpsi
%    &= \lambda \fpsi
%  \\\impliedby
%  \\0
%    &= \oppN(\opN - \lambda \opI)
%  \\&= \oppN([\opN - \lambda \opI]^\ast)
%    && \text{by $\oppN(\opN)=\oppN(\opNa)$}
%  \\&= \oppN(\opNa - [\lambda \opI]^\ast)
%    && \text{by \prefp{thm:op_star}}
%  \\&= \oppN(\opNa - \lambda^\ast \opI^\ast)
%    && \text{by \prefp{thm:op_star}}
%  \\&= \oppN(\opNa - \lambda^\ast \opI)
%  \\\implies
%  \\(\opNa-\lambda^\ast \opI)\fpsi
%    &= 0
%  \\\iff \opNa \fpsi
%    &= \lambda^\ast \fpsi
%  \\
%  \intertext{2. Proof that
%  $\opNa\opN=\opN\opNa \implies
%   [\lambda_n\ne \lambda_m\implies\inprod{\psi_n}{\psi_m}=0]
%  $:}
%  \lambda_n\inprod{\psi_n}{\psi_m}
%    &= \inprod{\lambda_n\psi_n}{\psi_m}
%    && \text{by definition of $\inprodn$ \prefp{def:inprod}}
%  \\&= \inprod{\opN\psi_n}{\psi_m}
%    && \text{by definition of eigenpairs}
%  \\&= \inprod{\psi_n}{\opNa\psi_m}
%    && \text{by \prefp{prop:op_adjoint} (definition of adjoint)}
%  \\&= \inprod{\psi_n}{\lambda_m^\ast \psi_m}
%    && \text{by (4.)}
%  \\&= \lambda_m \inprod{\psi_n}{\psi_m}
%    && \text{by definition of $\inprodn$ \prefp{def:inprod}}
%  \intertext{This implies for $\lambda_n\ne \lambda_m\ne 0$,
%             $\inprod{\psi_n}{\psi_m}=0$.}
%  \end{align*}
%\end{proof}

%=======================================
\subsection{Isometric operators}
%=======================================
An operator on a pair of normed linear spaces is \prope{isometric} (next definition) 
if it is an \structe{isometry}. % \xrefP{def:isometry}.
%--------------------------------------
\begin{definition}
\label{def:op_isometric}
%--------------------------------------
Let $\normspaceX$ and $\normspaceY$ be \structe{normed linear spaces} \xrefP{def:norm}.
\defbox{\begin{array}{M}
  An operator $\opM\in\clLxy$ is \hid{isometric} if
  \\\qquad$\ds\norm{\opM\vx}= \norm{\vx} \qquad\scy \forall \vx\in\setX$.
\end{array}}
\end{definition}

%--------------------------------------
\begin{theorem}
\footnote{
  \citerpgc{kubrusly2001}{239}{0817641742}{Proposition 4.37},
  \citerpgc{berberian1961}{27}{0821819127}{Theorem IV.7.5}
  }
\label{thm:op_normiso}
%--------------------------------------
Let $\normspaceX$ and $\normspaceY$ be \structe{normed linear spaces}.
Let $\opM$ be a linear operator in $\clLxy$.
\thmbox{
  \mcom{\norm{\opM\vx}=\norm{\vx}\quad\scy\forall\vx\in\setX}{isometric in length}
  \qquad\iff\qquad
  \mcom{\norm{\opM\vx-\opM\vy}=\norm{\vx-\vy}\quad\scy\forall\vx,\vy\in\setX}{isometric in distance}
  }
\end{theorem}
\begin{proof}
\begin{enumerate}
  \item Proof that $\norm{\opM\vx}=\norm{\vx}$ $\implies$ $\norm{\opM\vx-\opM\vy}=\norm{\vx-\vy}$:
    \begin{align*}
      \norm{\opM\vx-\opM\vy}
        &= \norm{\opM(\vx-\vy)}
        && \text{by definition of linear operators (\prefp{def:linop})}
      \\&= \norm{\opM\vu}
        && \text{let $\vu\eqd\vx-\vy$}
      \\&=\norm{\vx-\vy}
        && \text{by left hypothesis}
    \end{align*}

  \item Proof that $\norm{\opM\vx}=\norm{\vx}$ $\impliedby$ $\norm{\opM\vx-\opM\vy}=\norm{\vx-\vy}$:
    \begin{align*}
      \norm{\opM\vx}
        &= \norm{\opM(\vx-\vzero)}
      \\&= \norm{\opM\vx-\opM\vzero)}
        && \text{by definition of linear operators (\prefp{def:linop})}
      \\&= \norm{\vx-\vzero}
        && \text{by right hypothesis}
      \\&=\norm{\vx}
    \end{align*}

\end{enumerate}
\end{proof}

Isometric operators have already been defined \xrefP{def:op_isometric} in the more general normed linear spaces,
while \prefpp{thm:op_normiso} demonstrated that in a normed linear space $\spX$,
$\norm{\opM\vx}=\norm{\vx}\iff\norm{\opM\vx-\opM\vy}=\norm{\vx-\vy}$ for all $\vx,\vy\in\spX$.
Here in the more specialized inner product spaces, \pref{thm:op_inprodiso} (next) demonstrates two additional
equivalent properties.

%---------------------------------------
\begin{theorem}
\footnote{
  \citerpgc{michel1993}{432}{048667598X}{Theorem 7.5.8},
  \citerpgc{kubrusly2001}{391}{0817641742}{Proposition 5.72}
  }
\label{thm:op_inprodiso}
\label{thm:isometric_equiv}
\index{operator!isometric}
\index{isometric operator}
%---------------------------------------
Let $\clBxx $ be the space of bounded linear operators on a normed linear space $\spX\eqd\normspaceX$.
Let $\opN$ be a bounded linear operator in $\clLxx$, and
$\opI$ the identity operator in $\clLxx$.
Let $\norm{\vx}\eqd\sqrt{\inprod{\vx}{\vx}}$.
\thmbox{\begin{array}{>{\qquad}FrclCDc}
  \mc{7}{M}{The following conditions are all {\bf equivalent}:}
  \\
     1. & \opMa\opM &=& \opI
        &
        & %($\opM$ is isometric)
        & \iff
   \\2. & \inprod{\opM\vx}{\opM\vy} &=& \inprod{\vx}{\vy}
        & \forall\vx,\vy\in\setX
        & ($\opM$ is surjective)
        & \iff
   \\3. & \norm{\opM\vx-\opM\vy} &=& \norm{x-y}
        & \forall\vx,\vy\in\setX
        & (isometric in distance)
        & \iff
   \\4. & \norm{\opM\vx} &=& \norm{\vx}
        & \forall\vx\in\setX
        & (isometric in length)
        &
  \end{array}}
\end{theorem}
\begin{proof}
\begin{enumerate}
  \item Proof that (1)$\implies$(2):
    \begin{align*}
      \inprod{\opM\vx}{\opM\vy}
        &= \inprod{\vx}{\opMa \opM\vy} && \text{by \prefp{prop:op_adjoint} (definition of adjoint)}
      \\&= \inprod{\vx}{\opI\vy}       && \text{by (1)}
      \\&= \inprod{\vx}{\vy}            && \text{by \prefp{def:opI} (definition of $\opI$)}
    \end{align*}

  %\item Proof that (1)$\implies$(3):
  %  \begin{align*}
  %    \inprod{\opMa\vx}{\opMa\vy}
  %      &= \inprod{\vx}{\opMa^\ast \opMa\vy}
  %      && \text{by \prefp{prop:op_adjoint} (definition of adjoint)}
  %    \\&= \inprod{\vx}{\opM \opMa\vy}
  %      && \text{by \prefp{thm:op_star} (involution)}
  %    \\&= \inprod{\vx}{\opI\vy}
  %      && \text{by (1)}
  %    \\&= \inprod{\vx}{\vy}
  %      && \text{by \prefp{def:opI} (definition of $\opI$)}
  %  \end{align*}
  %
  %\item Proof that (3)$\implies$(4):
  %  \begin{align*}
  %    \norm{\opM\vx - \opM\vy}^2
  %      &=    \inprod{\opM\vx - \opM\vy}{\opM\vx - \opM\vy}
  %      &&    \text{by definition of induced norm $\normn$}
  %    \\&=    \inprod{\opM\vx}{\opM\vx} - \inprod{\opM\vx}{\opM\vy}
  %           -\inprod{\opM\vy}{\opM\vx} + \inprod{\opM\vy}{\opM\vy}
  %      &&    \text{by definition of $\inprodn$ \prefp{def:inprod}}
  %    \\&=    \inprod{\vx}{\vx} - \inprod{\vx}{\vy} -\inprod{\vy}{\vx} + \inprod{\vy}{\vy}
  %      &&    \text{by left hypothesis}
  %    \\&=    \inprod{\vx-\vy}{\vx-\vy}
  %      &&    \text{by definition of $\inprodn$ \prefp{def:inprod}}
  %    \\&=    \norm{\vx - \vy}^2
  %      &&    \text{by definition of induced norm $\normn$}
  %  \end{align*}

  \item Proof that (2)$\implies$(4):
    \begin{align*}
      \norm{\opM\vx}
        &= \sqrt{\inprod{\opM\vx}{\opM\vx}}
        && \text{by definition of $\normn$}
      \\&= \sqrt{\inprod{\vx}{\vx}}
        && \text{by right hypothesis}
      \\&= \norm{\vx}
        && \text{by definition of $\normn$}
    \end{align*}

  \item Proof that (2)$\impliedby$(4):
    \begin{align*}
      4\inprod{\opM\vx}{\opM\vy}
        &= \norm{\opM\vx+ \opM\vy}^2 -\norm{\opM\vx- \opM\vy}^2 +i\norm{\opM\vx+i\opM\vy}^2 -i\norm{\opM\vx-i\opM\vy}^2
        && \text{by polarization id.}% \prefpo{thm:polar_id}}
      \\&= \norm{\opM(\vx+\vy)}^2 -\norm{\opM(\vx-\vy)}^2 +i\norm{\opM(\vx+i\vy)}^2 -i\norm{\opM(\vx-i\vy)}^2
        %&& \text{by \prope{lineararity} of $\opM$ (\prefp{def:linop})}
        && \text{by \pref{def:linop}}
      \\&= \norm{\vx+\vy}^2 -\norm{\vx-\vy}^2 +i\norm{\vx+i\vy}^2 -i\norm{\vx-i\vy}^2
        && \text{by left hypothesis}
    \end{align*}

  \item Proof that (3)$\iff$(4): by \prefp{thm:op_normiso}

  \item Proof that (4)$\implies$(1):
    \begin{align*}
      \inprod{\opMa\opM\vx}{\vx}
        &=    \inprod{\opM\vx}{\opM^{\ast\ast}\vx}
        &&    \text{by \prefp{prop:op_adjoint} (definition of adjoint)}
      \\&=    \inprod{\opM\vx}{\opM\vx}
        &&    \text{by \prefp{thm:op_star} (property of adjoint)}
      \\&=    \norm{\opM\vx}^2
        &&    \text{by definition}
      \\&=    \norm{\vx}^2
        &&    \text{by left hypothesis with $y=0$}
      \\&=    \inprod{\vx}{\vx}
        &&    \text{by definition}
      \\&=    \inprod{\opI\vx}{\vx}
        &&    \text{by \prefp{def:opI} (definition of $\opI$)}
      \\\implies\qquad\opMa\opM &= \opI && \forall\vx\in\setX
    \end{align*}

\end{enumerate}
\end{proof}




%---------------------------------------
\begin{theorem}
\label{thm:isometric_prop}
\footnote{
  \citerpg{michel1993}{432}{048667598X}
  }
\index{operator!isometric}
\index{isometric operator}
%---------------------------------------
Let $\clBxy $ be the space of bounded linear operators on normed linear spaces $\spX$ and $\spY$.
Let $\opM$ be a bounded linear operator in $\clBxy$, and
$\opI$ the identity operator in $\clLxx$.
Let $\Lambda$ be the set of eigenvalues of $\opM$.
Let $\norm{\vx}\eqd\sqrt{\inprod{\vx}{\vx}}$.
\thmbox{
  \mcom{\opMa\opM=\opI}{$\opM$ is isometric}
  \qquad\implies\qquad
  \brbl{\begin{array}{rclCDD}
    \normop{\opM} &=& 1   &                     & (\prope{unit length}) & and \\
    \abs{\lambda} &=& 1   & \forall \lambda\in\Lambda & &
  \end{array}}
  }
\end{theorem}
\begin{proof}
\begin{align*}
  \intertext{1. Proof that $\opMa\opM=\opI \implies \normop{\opM} = 1$:}
  \normop{\opM}
    &= \sup_{\vx\in\setX}\set{\norm{\opM\vx}}{\norm{\vx}=1}
    && \text{by \prefp{def:op_norm}}
  \\&= \sup_{\vx\in\setX}\set{\norm{\vx}}{\norm{\vx}=1}
    && \text{by \prefp{thm:isometric_equiv}}
  \\&= \sup_{\vx\in\setX}\{1\}
  \\&= 1
  \\
  \intertext{2. Proof that $\abs{\lambda}=1$: Let $\opair{\vx}{\lambda}$ be an eigenvector-eigenvalue pair.}
  1 &= \frac{1}{\norm{\vx}}\; \norm{\vx}
  \\&= \frac{1}{\norm{\vx}}\; \norm{\opM\vx}
    && \text{by \prefp{thm:isometric_equiv}}
  \\&= \frac{1}{\norm{\vx}}\; \norm{\lambda\vx}
    && \text{by definition of $\lambda$}
  \\&= \frac{1}{\norm{\vx}}\; \abs{\lambda}\;\norm{\vx}
    && \text{by homogeneous property of $\normn$}
  \\&= \abs{\lambda}
\end{align*}
\end{proof}

%---------------------------------------
\begin{example}[One sided shift operator]
\footnote{
  \citerpg{michel1993}{441}{048667598X}
  }
\index{shift operator}
\index{one sided shift operator}
\index{operator!shift}
%---------------------------------------
Let $\spX$ be the set of all sequences with range $\Znn$ ($0,1,2,\ldots$)
and shift operators defined as
\[\begin{array}{lrcl@{\qquad}D}
  1. & \opS_r\seqn{x_0,x_1,x_2,\ldots} &\eqd& \seqn{0,x_0,x_1,x_2,\ldots}
     & \text{(right shift operator)}
     \\
  2. & \opS_l\seqn{x_0,x_1,x_2,\ldots} &\eqd& \seqn{x_1,x_2,x_3,\ldots}
     & \text{(left shift operator)}
\end{array}\]
\exbox{\begin{array}{>{\scy}rl}
  1. & \opS_r \text{ is an isometric operator.}  \\
  2. & \opS_r^\ast = \opS_l
\end{array}}
\end{example}
\begin{proof}
\begin{align*}
  \intertext{1. Proof that $\opS_r^\ast = \opS_l$:}
  \inprod{\opS_r\seqn{x_0,x_1,x_2,\ldots}}{\seqn{y_0,\vy_1,\vy_2,\ldots}}
    &= \inprod{\seqn{0,x_0,x_1,x_2,\ldots}}{\seqn{y_0,\vy_1,\vy_2,\ldots}}
  \\&= \sum_{n=1}^\infty\vx_{n-1} \;\vy_n^\ast
  \\&= \sum_{n=0}^\infty\vx_{n} \;\vy_{n+1}^\ast
  \\&= \sum_{n=0}^\infty\vx_{n} \;\vy_{n+1}^\ast
  \\&= \inprod{\seqn{x_0,x_1,x_2,\ldots}}{\seqn{y_1,\vy_2,\vy_3,\ldots}}
  \\&= \inprod{\seqn{x_0,x_1,x_2,\ldots}}{\mcom{\opS_l}{$\opS_r^\ast$}\seqn{y_0,\vy_1,\vy_2,\ldots}}
  %
  \intertext{2. Proof that $\opS_r$ is isometric ($\opS_r^\ast\opS_r=\opI$):}
  \opS_r^\ast \opS_r
    &= \opS_l \opS_r
    && \text{by 1.}
  \\&= \opI
\end{align*}
\end{proof}


%=======================================
\subsection{Unitary operators}
%=======================================
%---------------------------------------
\begin{definition}
\footnote{
  \citerpg{rudinf}{312}{0070542252},
  \citerpg{michel1993}{431}{048667598X},
  \citorp{autonne1901}{209},
  \citor{autonne1902},
  \citor{schur1909},
  \citer{steen1973}
  }
\label{def:op_unitary}
\index{operator!normal}      \index{normal operator}
\index{operator!isometric}   \index{isometric operator}
\index{operator!unitary}     \index{unitary operator}
%---------------------------------------
Let $\clBxy $ be the space of bounded linear operators on normed linear spaces $\spX$ and $\spY$.
Let $\opU$ be a bounded linear operator in $\clBxy$,
and $\spI$ the identity operator in $\clBxx$.
\defbox{\begin{array}{M rcl}
  The operator $\opU$ is \hid{unitary} if & \opUa\opU &=& \opU\opUa = \opI .
\end{array}}
\end{definition}


%---------------------------------------
\begin{proposition}
\label{prop:op_unitary_UV}
%---------------------------------------
Let $\clBxy $ be the space of bounded linear operators on normed linear spaces $\spX$ and $\spY$.
Let $\opU$ and $\opV$ be bounded linear operators in $\clBxy$.
\propbox{
  \brbr{\begin{array}{MD}
    $\opU$ is \prope{unitary} & and\\
    $\opV$ is \prope{unitary}
  \end{array}}
  \qquad\implies\qquad
  \text{$\brp{\opU\opV}$ is \prope{unitary}.}
  }
\end{proposition}
\begin{proof}
  \begin{align*}
    \brp{\opU\opV}\brp{\opU\opV}^\ast
      &= \brp{\opU\opV}\brp{\opVa\opUa}
      && \text{by \prefp{thm:op_adjoint}}
    \\&= \opU\brp{\opV\opVa}\opUa
      && \text{by \prop{associative} property\ifsxref{relation}{thm:op_associative}}
    \\&= \opU\opI\opUa
      && \text{by definition of \prop{unitary} operators---\prefp{def:op_unitary}}
    \\&= \opI
      && \text{by definition of \prop{unitary} operators---\prefp{def:op_unitary}}
    \\ \\
    \brp{\opU\opV}^\ast\brp{\opU\opV}
      &= \brp{\opVa\opUa}\brp{\opU\opV}
      && \text{by \prefp{thm:op_adjoint}}
    \\&= \opVa\brp{\opUa\opU}\opV
      && \text{by \prop{associative} property\ifsxref{relation}{thm:op_associative}}
    \\&= \opVa\opI\opV
      && \text{by definition of \prop{unitary} operators---\prefp{def:op_unitary}}
    \\&= \opI
      && \text{by definition of \prop{unitary} operators---\prefp{def:op_unitary}}
  \end{align*}
\end{proof}

%---------------------------------------
\begin{theorem}
\label{thm:unitary_equiv}
\footnote{
  \citerppgc{rudinf}{313}{314}{0070542252}{Theorem 12.13},
  \citerpgc{knappa2005}{45}{0817643826}{Proposition 2.6}
  }
\index{operator!unitary}
%---------------------------------------
Let $\clBhh $ be the space of bounded linear operators on a Hilbert space $\spH$.
Let $\opU$ be a bounded linear operator in $\clBhh$,
and $\oppI(\opU)$ the \structe{image set}\ifsxref{relation}{def:rel_image} of $\opU$.
\thmbox{\begin{array}{>{\qquad}FlDlDc}
  \mc{6}{M}{The following conditions are {\bf equivalent}:}
  \\
     1. & \opU\opUa=\opUa\opU=\opI
        &
        &
        & (unitary)
        & \iff
   \\2. & \inprod{\opU\vx}{\opU\vy}  = \inprod{\opUa\vx}{\opUa\vy} =  \inprod{\vx}{\vy}
        & and
        & \oppI(\opU)=\setX
        & (surjective)
        & \iff
   \\3. & \norm{\opU\vx-\opU\vy} = \norm{\opUa\vx-\opUa\vy} = \norm{\vx-\vy}
        & and
        & \oppI(\opU)=\setX
        & (isometric in distance)
        & \iff
   \\4. & \norm{\opU\vx} = \norm{\vx}
        & and
        & \oppI(\opU)=\setX
        & (isometric in length)
        &
\end{array}}
\end{theorem}
\begin{proof}
\begin{enumerate}
  \item Proof that (1) $\implies$ (2):
  \begin{enumerate}
    \item $\inprod{\opU\vx}{\opU\vy}  = \inprod{\opUa\vx}{\opUa\vy} =  \inprod{\vx}{\vy}$
          by \prefpp{thm:isometric_equiv}.
    \item Proof that $\oppI(\opU)=\setX$:
      \begin{align*}
        \setX
          &\supseteq \oppI(\opU)
          &&         \text{because $\opU\in\clFxx$}
        \\&\supseteq \oppI(\opU\opUa)
        \\&=         \oppI(\opI)
          &&         \text{by left hypothesis ($\opUa\opU=\opU\opUa=\opI$)}
        \\&=         \setX
          &&         \text{by \prefp{def:opI} (definition of $\opI$)}
        %\\
        %\\
        %\setX
        %  &\supseteq \oppI(\opUa)
        %\\&\supseteq \oppI(\opUa\opU)
        %\\&=         \oppI(\opI)
        %  &&         \text{by left hypothesis ($\opUa\opU=\opU\opUa=\opI$)}
        %\\&=         \setX
        %  &&         \text{by \prefp{def:opI} (definition of $\opI$)}
      \end{align*}
    \end{enumerate}
  \item Proof that (2) $\iff$ (3) $\iff$ (4):    by \prefp{thm:isometric_equiv}.
  \item Proof that (3) $\implies$ (1):
    \begin{enumerate}
      \item Proof that $\norm{\opU\vx-\opU\vy}=\norm{\vx-\vy}$ $\implies$ $\opUa\opU=\opI$:  by \prefp{thm:isometric_equiv}
      \item Proof that $\norm{\opUa\vx-\opUa\vy}=\norm{\vx-\vy}$ $\implies$ $\opU\opUa=\opI$:
        \begin{align*}
          \norm{\opUa\vx-\opUa\vy}=\norm{\vx-\vy}
            &\implies & \opU^{\ast\ast}\opUa &= \opI    && \text{by \prefp{thm:isometric_equiv}}
          \\&         & \opU\opUa            &= \opI    && \text{by \prefp{thm:op_star}}
        \end{align*}
    \end{enumerate}
\end{enumerate}
\end{proof}

%---------------------------------------
\begin{theorem}
%\footnote{
%  \citerpg{lax}{372}{0471556041}
%  }
\label{thm:unitary_prop}
\index{operator!unitary}
%---------------------------------------
Let $\clBhh $ be the space of bounded linear operators on a Hilbert space $\spH$.
Let $\opU$ be a bounded linear operator in $\clBhh$,
    $\oppN(\opU)$ the \structe{null space} of $\opU$,
and $\oppI(\opU)$ the \structe{image set} of $\opU$.\ifdochas{relation}{\footnote{%
  \begin{tabular}{ll}
    \structe{null space}: & \prefp{def:rel_null} \\
    \structe{image set}:  & \prefp{def:rel_image}
  \end{tabular}}}
%(\prefp{def:rel_null}).
\thmbox{\begin{array}{l}
  \mcom{\opU\opUa=\opUa\opU=\opI}{$\opU$ is unitary}
  \implies
  \left\{\begin{array}{rclclD}
  \opUi                    &=& \opUa                     & &                     &  \\
  \oppI(\opU)              &=& \oppI(\opUa)              &=& \setX               &  \\
  \oppN(\opU)              &=& \oppN(\opUa)              &=& \setn{\vzero}    &  \\
  \normop{\opU}            &=& \normop{\opUa}            &=&  1                  & (\prope{unit length})
  \end{array}\right.
\end{array}}
\end{theorem}
\begin{proof}
\begin{enumerate}
  \item Note that $\opU$, $\opUa$, and $\opUi$ are all both
        {\bf isometric} and {\bf normal}:
    \[\begin{array}{lclclll}
      \opUa\opU & &           &=& \opI &\implies &\text{$\opU$ is isometric}  \\
      \opU\opUa &=& \opUa\opU &=& \opI &\implies &\text{$\opUa$ is isometric} \\
      \opUi     &=& \opUa     & &      &\implies &\text{$\opUi$ is isometric} \\
      \\
      \opUa\opU &=& \opU\opUa &=&      &\implies &\text{$\opU$ is normal}   \\
      \opU\opUa &=& \opUa\opU &=&      &\implies &\text{$\opUa$ is normal}  \\
      \opUi     &=& \opUa     & &      &\implies &\text{$\opUi$ is normal}
    \end{array}\]

  \item Proof that
    $\opUa\opU=\opU\opUa=\opI
     \implies
     \oppI(\opU)=\oppI(\opUa)=\spH
    $: by \prefp{thm:unitary_equiv}.

  \item Proof that
    $\opUa\opU=\opU\opUa=\opI
     \implies
     \oppN(\opU)=\oppN(\opUa)=\oppN(\opUi)
    $:
    \begin{align*}
      \oppN(\opUa)
        &= \oppN(\opU)
        && \text{because $\opU$ and $\opUa$ are both normal
                 and by \prefp{thm:op_norm_prop}}
      \\&= \oppI(\opU)^\perp
        && \text{by \prefp{thm:oppN_oppR}}
      \\&= \setX^\perp
        && \text{by above result}
      \\&= \setn{\vzero}
        && \ifdochas{vsinprod}{\text{by \prefp{prop:inprod_orthog}}}
    \end{align*}

  \item Proof that
    $\opUa\opU=\opU\opUa=\opI
     \implies
     \norm{\opUi}=\norm{\opUa}=\norm{\opU}=1
    $: \\
    Because $\opU$, $\opUa$, and $\opUi$ are all isometric
    and by \prefp{thm:isometric_prop}.
\end{enumerate}
\end{proof}

%%---------------------------------------
%\begin{theorem}
%\index{group}
%%---------------------------------------
%Let $\mathcal{U} \eqd \set{\opU\in\clOxx}{\mcom{\opUa\opU=\opU\opUa=\opI}{$\opU$ is unitary}}$
%be the set of all unitary operators on space $\spX$.
%\thmbox{
%  \text{$(\mathcal{U},\cdot)$ is a group.}
%  }
%\end{theorem}
%\begin{proof}
%\[\begin{array}{lll>{$}l<{$}}
%  1. & \opU\opI = \opI\opU = \opU
%     & \forall \opU\in\mathcal{U}
%     & ($\opI\in\mathcal{U}$ is the identity element)
%     \\
%  2. & \opUa\opU = \opU\opUa = \opI
%     & \forall \opU\in\mathcal{U}
%     & ($\opUa$ is the inverse of $\opU$)
%     \\
%  3. & (\opU\opV)\opW = \opU(\opV\opW)
%     & \forall \opU,\opV,\opW\in\mathcal{U}
%     & ($(\mathcal{U},\cdot)$ is associative)
%\end{array}\]
%\end{proof}

%---------------------------------------
\begin{example}
%---------------------------------------
Examples of \hie{Fredholm integral operators} include
\[
\begin{array}{llrclrcl}
  1. & \mbox{Fourier Transform}
     & [\opFT \fx](f) &=& \int_t \fx(t) e^{-i2\pi ft}\dt
     & \kappa(t,f)    &=& e^{-i2\pi ft}
\\
  2. & \mbox{Inverse Fourier Transform}
     & [\opFTi \Fx](t) &=& \int_f \Fx(f) e^{i2\pi ft}\df
     & \kappa(f,t)    &=& e^{i2\pi ft}
\\
  3. & \mbox{Laplace operator}
     & [\opL \fx](s) &=& \int_t \fx(t) e^{-st}\dt
     & \kappa(t,s)   &=& e^{-st}
\end{array}
\]
\end{example}



%---------------------------------------
\begin{example}[Translation operator]
\index{translation}
\index{operator!translation}
%---------------------------------------
Let $\spX=\spLLR$ and $\opT\in\clOxx$ be defined as
\[\begin{array}{rcl @{\qquad}C @{\qquad}D}
  \opT\ff(x)  &\eqd& \ff(x-1)
              &\forall \ff\in\spLLR
              &(translation operator)
\end{array}\]
\exbox{\begin{array}{l rcl @{\qquad}C @{\qquad}D}
  1. & \opTi\ff(x) &=& \ff(x+1)
     & \forall \ff\in\spLLR
     & (inverse translation operator)
     \\
  2. & \opTa &=& \opTi
     &
     & ($\opT$ is invertible)
     \\
  3. & \opTa\opT &=& \opT\opTa = \opI
     &
     & ($\opT$ is unitary)
\end{array}}
\end{example}
\begin{proof}
\begin{align*}
  \intertext{1. Proof that $\opTi\ff(x)=\ff(x+1)$:}
  \opTi\opT &= \opI \\
  \opT\opTi &= \opI
  %
  \intertext{2. Proof that $\opT$ is unitary:}
  \inprod{\opT\ff(x)}{\fg(x)}
    &= \inprod{\ff(x-1)}{\fg(x)}
    && \text{by definition of $\opT$}
  \\&= \int_x\ff(x-1) \fg^\ast(x) \dx
  \\&= \int_x\ff(x) \fg^\ast(x+1) \dx
  \\&= \inprod{\ff(x)}{\fg(x+1)}
  \\&= \inprod{\ff(x)}{\mcom{\opTi}{$\opTa$}\fg(x)}
    && \text{by 1.}
\end{align*}
\end{proof}


%---------------------------------------
\begin{example}[Dilation operator]
\index{dilation}
\index{operator!dilation}
%---------------------------------------
Let $\spX=\spLLR$ and $\opT\in\clOxx$ be defined as
\[\begin{array}{rcl @{\qquad}C @{\qquad}D}
  \opD\ff(x)  &\eqd& \sqrt{2}\ff(2x)
              &\forall \ff\in\spLLR
              &(dilation operator)
\end{array}\]
\exbox{\begin{array}{l rcl @{\qquad}C @{\qquad}D}
  1. & \opDi\ff(x) &=& \frac{1}{\sqrt{2}} \ff\left(\frac{1}{2}x\right)
     & \forall \ff\in\spLLR
     & (inverse dilation operator)
     \\
  2. & \opDa &=& \opDi
     &
     & ($\opD$ is invertible)
     \\
  3. & \opDa\opD &=& \opD\opDa = \opI
     &
     & ($\opD$ is unitary)
\end{array}}
\end{example}
\begin{proof}
\begin{align*}
  \intertext{1. Proof that $\opDi\ff(x)=\frac{1}{\sqrt{2}} \ff\left(\frac{1}{2}x\right)$:}
  \opDi\opD &= \opI \\
  \opD\opDi &= \opI
  %
  \intertext{2. Proof that $\opD$ is unitary:}
  \inprod{\opD\ff(x)}{\fg(x)}
    &= \inprod{\sqrt{2}\ff(2x)}{\fg(x)}
    && \text{by definition of $\opD$}
  \\&= \int_x  \sqrt{2} \ff(2x) \fg^\ast(x) \dx
  \\&= \int_{u\in\R} \sqrt{2} \ff(u) \fg^\ast\left(\frac{1}{2}u\right) \frac{1}{2}\du
    && \text{let $u\eqd 2x \quad\implies\quad \dx=\frac{1}{2}\du$}
  \\&= \int_{u\in\R} \ff(u) \left[\frac{1}{\sqrt{2}}\fg\left(\frac{1}{2}u\right)\right]^\ast \du
  \\&= \inprod{\ff(x)}{\frac{1}{\sqrt{2}}\fg\left(\frac{1}{2}x\right)}
  \\&= \inprod{\ff(x)}{\mcom{\opDi}{$\opDa$}\fg(x)}
    && \text{by 1.}
\end{align*}
\end{proof}

%---------------------------------------
\begin{example}[Delay operator]
\index{delay}
\index{operator!delay}
%---------------------------------------
Let $\spX$ be the set of all sequences and $\opD\in\clOxx$ be
a delay operator.
\exbox{
  \text{The delay operator $\opD\seq{x_n}{n\in\Z}\eqd\seq{x_{n-1}}{n\in\Z}$
  is unitary.}
  }
\end{example}
\begin{proof}
The inverse $\opDi$ of the delay operator $\opD$ is
  \[\opDi\seq{x_n}{n\in\Z}\eqd\seq{x_{n+1}}{n\in\Z}.\]
\begin{align*}
  \inprod{\opD\seqn{x_n}}{\seqn{y_n}}
    &= \inprod{\seqn{x_{n-1}}}{\seqn{y_n}}
    && \text{by definition of $\opD$}
  \\&= \sum_n\vx_{n-1}\;\vy_n^\ast
  \\&= \sum_n\vx_n\;\vy_{n+1}^\ast
  \\&= \inprod{\seqn{x_n}}{\seqn{y_{n+1}}}
  \\&= \inprod{\seqn{x_n}}{\mcom{\opDi}{$\opDa$} \seqn{y_n}}
\end{align*}
Therefore,
$\opDa=\opDi$. This implies that $\opD\opDa=\opDa\opD=\opI$
which implies that $\opD$ is unitary.
\end{proof}

%---------------------------------------
\begin{example}[Fourier transform]
\index{Fourier transform}
\index{unitary}
\label{ex:operator_opFT_unitary}
%---------------------------------------
Let $\opFT$ be the \ope{Fourier Transform} and
$\opFTi$ the \ope{inverse Fourier Transform} operator \ifxref{harPoly}{thm:opFSi}
  \[
    [\opFT \fx](f) \eqd \int_t \fx(t) \mcom{e^{-i2\pi ft}}{$\kappa(t,f)$} \dt
    \qquad\qquad
    \left[\opFTi \Fx\right](t) \eqd \int_f \Fx(f) \mcom{e^{i2\pi ft}}{$\kappa^\ast(t,f)$} \df.
  \]
\exbox{
  \opFTa = \opFTi
  \qquad
  \text{(the Fourier Transform operator $\opFT$ is unitary)}
  }
\end{example}
\begin{proof}
  \begin{align*}
    \inprod{\opFT \fx}{\Fy}
      &= \inprod{\int_t \fx(t) e^{-i2\pi ft} \dt }{\Fy(f)}
    \\&= \int_t \fx(t) \inprod{e^{-i2\pi ft} }{\Fy(f)} \dt
    \\&= \int_t \fx(t) \int_f e^{-i2\pi ft} \Fy^\ast(f) \df \dt
    \\&= \int_t \fx(t) \left[\int_f e^{i2\pi ft} \Fy(f) \df \right]^\ast \dt
    \\&= \inprod{\fx(t)}{\int_f \Fy(f) e^{i2\pi ft} \df }
    \\&= \inprod{\fx}{\mcom{\opFTi}{$\opFTa$} \Fy}
  \end{align*}
This implies that $\opFT$ is unitary ($\opFTa = \opFTi$).
\end{proof}



%---------------------------------------
\begin{example}[Rotation matrix]
\citep{noble}{311}
\index{rotation matrix}
\index{matrix!rotation}
\label{ex:operator_rotation_unitary}
%---------------------------------------
Let the rotation matrix $\opR_\theta:\R^2\to\R^2$ be defined as
\[ \opR_\theta \eqd
   \left[\begin{array}{rr}
     \cos\theta & -\sin\theta   \\
     \sin\theta &  \cos\theta
   \end{array}\right]
\]
\exbox{\begin{array}{l rcl @{\qquad}D}
  1. & \opRi_\theta &=& \opR_{-\theta}  \\
  2. & \opRa_\theta &=& \opRi_\theta   & ($\opR$ is unitary)
\end{array}}
\end{example}
\begin{proof}
\begin{align*}
  \opRa
    &= \opR^H
  \\&= \left[\begin{array}{rr}
         \cos\theta & -\sin\theta   \\
         \sin\theta &  \cos\theta
       \end{array}\right]^H
    && \text{by definition of $\opR$}
  \\&= \left[\begin{array}{rr}
          \cos\theta &  \sin\theta   \\
         -\sin\theta &  \cos\theta
       \end{array}\right]
    && \text{by definition of Hermetian transpose operator $H$}
  \\&= \left[\begin{array}{rr}
          \cos(-\theta) & -\sin(-\theta)   \\
          \sin(-\theta) &  \cos(-\theta)
       \end{array}\right]
    && \text{because $\cos(-\theta)=\cos\theta$ and $\sin(-\theta)=-\sin\theta$}
  \\&= \opR_{-\theta}
    && \text{by definition of $\opR$}
  \\&= \opRi
    && \text{by 1.}
\end{align*}
\end{proof}


\ifexclude{wsd}{

%=======================================
\section{Operator order}
%=======================================
%---------------------------------------
\begin{definition}
\label{def:op_ge}
\footnote{
  \citerpgc{michel1993}{429}{048667598X}{Definition 7.4.12}
  %\citerpg{ab2006}{2}{1402050070}
  %\citerp{pedersen2000}{87}
  }
\index{operator!positive}
%---------------------------------------
Let $\opP\in\clOxy$ be an operator.
\defbox{\begin{array}{M}
  $\opP$ is \hid{positive} if \qquad $\inprod{\opP\vx}{\vx} \ge 0$ $\forall\vx\in\spX$.
  \\
  This condition is denoted \qquad$\opP\ge0$.
\end{array}}
\end{definition}

%%---------------------------------------
%\begin{proposition}
%\footnote{
%  \citerpg{ab2006}{2}{1402050070}
%  }
%%---------------------------------------
%Let $\opP\in\clOxy$ be an operator.
%\propbox{\begin{array}{>{\qquad\scy}rl@{\qquad}D}
%  \mc{3}{M}{The following statements are equivalent:}\\
%    1. & \opP \ge 0  & ($\opP$ is \prope{positive}) \\
%    2. & \opP(\spXp) \subseteq \spYp \\
%    3. & \vx\orel \vy \implies \opP\vx \orel \opP\vy
%\end{array}}
%\end{proposition}

%---------------------------------------
\begin{theorem}
\footnote{
  \citerpg{michel1993}{429}{048667598X}
  }
%---------------------------------------
\thmbox{
  \mcom{\opP\ge0 \text{ and } \opQ\ge0}{$\opP$ and $\opQ$ are both positive}
  \qquad\implies\qquad
  \left\{\begin{array}{lcl@{\qquad}C@{\qquad}D}
    (\opP+\opQ)    &\ge& 0 &                                 & ($(\opP+\opQ)$ is positive)\\
    \opAa\opP\opA  &\ge& 0 & \forall \opA\in\clBxx  & ($\opAa\opP\opA$ is positive)\\
    \opAa\opA      &\ge& 0 & \forall \opA\in\clBxx  & ($\opAa\opA$ is positive)\\
  \end{array}\right.
  }
\end{theorem}
\begin{proof}
  \begin{align*}
    \inprod{(\opP+\opQ)\vx}{\vx}
      &=   \inprod{\opP\vx}{\vx} + \inprod{\opQ\vx}{\vx}
      &&   \text{by additive property of $\inprodn$ (\prefp{def:inprod})}
    \\&\ge \inprod{\opP\vx}{\vx}
      &&   \text{by left hypothesis}
    \\&\ge 0
      &&   \text{by left hypothesis}
    \\
    \inprod{\opAa\opP\opA\vx}{\vx}
      &=   \inprod{\opP\opA\vx}{\opA\vx}
      &&   \text{by definition of adjoint (\prefp{prop:op_adjoint})}
    \\&=   \inprod{\opP\vy}{\vy}
      &&   \text{where $\vy\eqd\opA\vx$}
    \\&\ge 0
      &&   \text{by left hypothesis}
    \\
    \inprod{\opI\vx}{\vx}
      &=   \inprod{\vx}{\vx}
      &&   \text{by definition of $\opI$ (\prefp{def:opI})}
    \\&\ge 0
      &&   \text{by non-negative property of $\inprodn$ (\prefp{def:inprod})}
    \\&\implies \text{$\opI$ is positive}
    \\
    \inprod{\opAa\opA\vx}{\vx}
      &=   \inprod{\opAa\opI\opA\vx}{\vx}
      &&   \text{by definition of $\opI$ (\prefp{def:opI})}
    \\&\ge 0
      &&   \text{by two previous results}
  \end{align*}
\end{proof}

%---------------------------------------
\begin{definition}
\footnote{
  \citerpg{michel1993}{429}{048667598X}
  }
\index{operator!positive}
\label{def:op_A>B}
%---------------------------------------
Let $\opA,\opB\in\clBxy$ be bounded operators.
\defbox{\indxs{\ge}
  \mcom{\opA \ge \opB}{``$\opA$ is greater than or equal to $\opB$"}
  \quad\iff\quad
  \mcom{(\opA-\opB)\ge 0}{``$(\opA-\opB)$ is positive"}
  }
\end{definition}

} % end wsd exclude


%\begin{minipage}{5\tw/16}
%  \color{figcolor}
%  \begin{center}
%  \begin{fsL}
%  \setlength{\unitlength}{\textwidth/700}
%  \begin{picture}(600,500)(-300,-200)
%    %{\color{graphpaper}\graphpaper[100](-300,-200)(600,500)}
%    \thinlines
%    \put(-200,   0){\line( 1, 1){120} }
%    \put(   0, 200){\line(-1,-1){ 40} }
%    \put(-100,   0){\line(-1, 1){ 50} }
%    \put(   0,   0){\line(-1, 1){100} }
%    \put( 200,   0){\line(-1, 1){200} }
%
%    \put(-200,   0){\line( 1,-1){200} }
%    \put(-100,   0){\line( 1,-2){100} }
%    \put(   0,   0){\line( 0,-1){200} }
%    \put( 200,   0){\line(-1,-1){200} }
%
%    \put(   0, 200){\circle*{15}}  %maximum element
%    \put(   0,-200){\circle*{15}}  %minimum element
%
%    \put(-150,  50){\circle*{15}}
%    \put(-100, 100){\circle*{15}}
%
%    \put(-200, 0){\circle*{15}}
%    \put(-100, 0){\circle*{15}}
%    \put(   0, 0){\circle*{15}}
%    \put( 100, 0){$\cdots$}
%    \put( 200, 0){\circle*{15}}
%
%    \put(   0, 210){\makebox(0,0)[b] {$\spX$}}
%    \put(-110, 100){\makebox(0,0)[r] {$\spV_3$}}
%    \put(-160,  50){\makebox(0,0)[r] {$\spV_2$}}
%    \put(-210,   0){\makebox(0,0)[r] {$\spV_1$}}
%    \put(-110,   0){\makebox(0,0)[r] {$\spW_1$}}
%    \put(  10,   0){\makebox(0,0)[l] {$\spW_2$}}
%    \put( 210,   0){\makebox(0,0)[l] {$\spW_{n-1}$}}
%    \put(  . ,-210){\makebox(0,0)[t] {$0$}}
%
%    \put( -50, 15){\makebox(0,0)[c] {$\ddots$}}
%    {\color{red}
%      \put(-300,- 50){\dashbox{10}(600,80){}}
%      \put(-300, 260){\makebox(0,0)[bl] {scaling subspaces}}
%      \put( 300, 110){\makebox(0,0)[br] {wavelet subspaces}}
%      \put( 200, 100){\vector(0,-1){70}}
%      \put(-250, 250){\vector(1,-1){70}}
%      \thicklines
%      \qbezier[30](-300,  50)(-175, 175)( -50, 300)
%      \qbezier[30](-200, -50)( -75,  75)(  50, 200)
%      \qbezier[15](-200,-50)(-250,0)(-300,50)
%      \qbezier[15](  50,200)(0,250)(-50,300)
%    }
%  \end{picture}
%  \end{fsL}
%  \end{center}
%\end{minipage}
%\begin{minipage}{11\tw/16}
%%---------------------------------------
%\begin{example}
%%---------------------------------------
%A \hie{wavelet transform} partitions a space $\spX$ into a pair of sequences of subspaces:
%\begin{liste}
%  \item \hie{scaling subspace sequence} $\seqn{\spV_n}$
%  \item \hie{wavelet subspace sequence} $\seqn{\spW_n}$
%\end{liste}
%Let $\opP_n$ be the projection operator onto subspace $\spV_n$ and
%let $\opQ_n$ be the projection operator onto subspace $\spW_n$.
%Then
%\begin{liste}
%  \item $\opP_n\ge\opP_m$ for $n\ge m$
%  \item $\opP_{n+1} \ge \opQ_n$
%  \item $\opQ_n$ is \hie{not comparable} to $\opQ_m$ for $n\ne m$
%\end{liste}
%Note that the tuple
%$(\setn{\seq{\opP_n}{n\in\Z},\seq{\opQ_n}{n\in\Z}},\,\ge,\,+,\,\cdot)$
%is a \hie{lattice}.
%\end{example}
%\end{minipage}






%%=======================================
%\section{Literature}
%%=======================================
%\begin{survey}
%\begin{enumerate}
%  \item Algebra of projection operators:
%    \\\citer{murray1936}
%
%  \item John von Neumann's contributions to and use of lattice theory:
%    \\\citer{birkhoff1958}
%
%\end{enumerate}
%\end{survey}
%