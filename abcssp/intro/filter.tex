%============================================================================
% Daniel J. Greenhoe
% LaTeX file
%============================================================================
%=======================================
\subsection{Filtering}
%=======================================
%---------------------------------------
\begin{definition}
\label{def:filter}
\label{def:filtern}
%---------------------------------------
Let $\seq{x_n}{\Dom_1}$ and $\seq{y_n}{\Dom_2}$ be \structe{sequence}s \xref{def:sequence}. % over a \structe{field} $\Field$.
\\The \structe{sequence} $\seqnD{z_n}$ is said to be $\seqn{x_n}$ \opd{filtered} by 
$\seqn{y_n}$ if $\seqn{z_n}\eqd\seqn{x_n}\conv\seqn{y_n}$ \xref{def:conv}.
\\Moreover, in this case, the operation $\conv\seqn{y_n}$ is a \opd{filter} on the \structe{sequence} $\seqn{x_n}$.
\end{definition}

%%---------------------------------------
%\begin{definition}
%\label{def:filtern}
%%---------------------------------------
%Let $\seq{h_n}{\Dom_2}$ be a \structe{sequence} over a \structe{field} $\Field$ 
%and $\vx\eqd\seq{\oquad{x_{n,1}}{x_{n,2}}{\cdots}{x_{n,\xN}}}{\Dom_1}$    a \structe{sequence} over $\Field^\xN$.
%Let $\seq{x_n}{\Dom_1}^m \eqd \seq{x_{n,m}}{\Dom_1}$, the $m$th dimensional subsequence of $\vx$.
%The \structe{sequence} $\vz\eqd\seq{\oquad{z_{n,1}}{z_{n,2}}{\cdots}{z_{n,\xN}}}{\Dom}$
%is said to be $\vx$ \opd{filtered} by $\seqn{h_n}$ if 
%$\seq{z_{n,m}}{\Dom}\eqd\seq{x_{n,m}}{\Dom_1}\conv\seq{h_n}{\Dom_2}$.
%Moreover, in this case, the operation $\conv\seqn{h_n}$ is a \opd{filter} on the \structe{sequence} $\vx$.
%\end{definition}

%%--------------------------------------
%\begin{example}
%%--------------------------------------
%Let $\seq{x_n}{\intcc{0}{1}}\eqd\seqn{\opair{1}{11},\opair{2}{22}}$ be a \structe{sequence} over $\R^2$ 
%and $\seq{h_n}{\intcc{0}{2}}\eqd\seqn{3,5,7}$ a \structe{sequence} over $\R$.
%Then the \structe{domain} $\Dom$ of the \ope{convolution} $\seqnD{z_n}\eqd\seqn{x_n}\conv\seqn{h_n}$ 
%is $\Dom\eqd\intoo{0+0-1}{1+2+1}=\intcc{0}{3}=\setn{0,1,2,3}$ 
%and 
%\begin{align*}
%  %\seqnD{z_n} 
%   %&\eqd \seq{\opair{\sum_{m\in\setn{0,1}}\f_1(0,m)}{\sum_{m\in\setn{0,1}}\f_2(0,m)},\, \sum_{m\in\setn{0,1}}\f(1,m),\, \sum_{m\in\setn{0,1}}\f(2,m),\, \sum_{m\in\setn{0,1}}\f(3,m)}{\setn{0,1,2,3}}
%  \seqnD{z_{n,1}} 
%    &=    \seq{(1\times3+0),\, (1\times5+2\times3),\, (1\times7+2\times5),\, (0+2\times7)}{\setn{0,1,2,3}}
%  \\&=    \seq{\underbrace{  3}_{z_{0,1}},\,
%               \underbrace{ 11}_{z_{1,1}},\,
%               \underbrace{ 17}_{z_{2,1}},\,
%               \underbrace{ 14}_{z_{3,1}}
%              }{\setn{0,1,2,3}}
%  \\
%  \seqnD{z_{n,2}} 
%    &=    \seq{(11\times3+0),\, (11\times5+22\times3),\, (11\times7+22\times5),\, (0+22\times7)}{\setn{0,1,2,3}}
%  \\&=    \seq{\underbrace{ 33}_{z_{0,2}},\,
%               \underbrace{121}_{z_{1,2}},\,
%               \underbrace{187}_{z_{2,2}},\,
%               \underbrace{154}_{z_{3,2}}
%              }{\setn{0,1,2,3}}
%  \\
%  \seqnD{z_n} 
%    &=    \seq{\underbrace{\opair{ 3}{ 33}}_{z_{0}},\,
%               \underbrace{\opair{11}{121}}_{z_{1}},\,
%               \underbrace{\opair{17}{187}}_{z_{2}},\,
%               \underbrace{\opair{14}{154}}_{z_{3}}
%              }{\setn{0,1,2,3}}
%\end{align*}
%%Where $\f_1$ is the \fncte{convolution function} \xref{def:convf} on the 1st dimension of $\seqn{x_n}$ and $\f_2$ the \fncte{convolution function} on the 2nd dimension.
%\end{example}


%---------------------------------------
\begin{definition}
\label{def:lp_rect}
%---------------------------------------
A \fnctd{length $\xM$ low pass rectangular sequence} $\seq{h_n}{n\in\intcc{0}{\xM-1}}$ is here defined as 
\\\indentx$h_n=\sfrac{1}{\xM}$ for $n\in\intcc{0}{\xM-1}$.
\end{definition}

%---------------------------------------
\begin{definition}
\label{def:hp_rect}
%---------------------------------------
A \fnctd{length $\xM$ high pass rectangular sequence} $\seq{h_n}{n\in\intcc{0}{\xM}}$ is here defined as 
\\\indentx$h_n \eqd 
  \brb{\begin{array}{lM}
    0          & for $n=0$\\
    (-1)^{n+1} & for $n=1,2,\ldots,\xM$
  \end{array}}$
\end{definition}
Note that in this definition, the sequence has been offset by 1 on the x-axis from what might normally be expected.
This is for the purpose of computational convenience used in \prefpp{sec:hp}.

%---------------------------------------
\begin{example}
\label{ex:lp_rect}
\label{ex:hp_rect}
%---------------------------------------
A   \fncte{length 16 low pass rectangular sequence} 
and \fncte{length 16 high pass rectangular sequence} are 
illustrated below:
\\\indentx
  \begin{tabular}{|*{2}{>{\scs}c|}}
    \hline
      \includegraphics{../common/math/graphics/pdfs/seq_lprect16.pdf} 
     &\includegraphics{../common/math/graphics/pdfs/seq_hprect16.pdf}
    \\\fncte{length 16 low  pass rectangular sequence} 
     &\fncte{length 16 high pass rectangular sequence}
    \\\hline
  \end{tabular}
\end{example}

%---------------------------------------
\begin{definition}
\footnote{
  \citePpc{blackman1958mar}{502}{B.5 Particular Pairs of Windows},
  \citerpc{blackman1959}{98}{B.5 Particular Pairs of Windows},
  \citerpg{oppenheim1999}{763}{0137549202},
  \citerpg{prabhu2013}{148}{1466515848}
  }
\label{def:lp_hann}
%---------------------------------------
A \fnctd{length $\xM$ low pass Hanning sequence} $\seq{h_n}{n\in\intcc{0}{\xM-1}}$ is here defined as 
\\\indentx$h_n \eqd \frac{1}{2}\brs{1-\cos\brp{\frac{2\pi n}{\xM-1}}}$ for $n=0,1,2,\ldots,\xM-1$.
\end{definition}

%---------------------------------------
\begin{example}
\label{ex:lp_hann}
%---------------------------------------
A \fncte{length 50 low pass Hanning sequence} is illustrated below:
\\\mbox{}\hfill\includegraphics{../common/math/graphics/pdfs/hanninglp50.pdf}\hfill\mbox{}
\end{example}

%---------------------------------------
\begin{definition}
\label{def:hp_hann}
%---------------------------------------
A \fnctd{length $\xM$ high pass Hanning sequence} $\seq{h_n}{n\in\intcc{0}{\xM-1}}$ is here defined as 
\\\indentx$h_n \eqd (-1)^{n} \frac{1}{2}\brs{1-\cos\brp{\frac{2\pi n}{\xM-1}}}$ for $n=0,1,2,\ldots,\xM-1$
\end{definition}

%---------------------------------------
\begin{example}
\label{ex:hp_hann}
%---------------------------------------
A \fncte{length 50 high pass Hanning sequence} is illustrated below:
\\\mbox{}\hfill\includegraphics{../common/math/graphics/pdfs/hanninghp50.pdf}\hfill\mbox{}
\end{example}

%---------------------------------------
\begin{definition}
\label{def:shaar}
%---------------------------------------
A \fnctd{length $\xM$ Haar scaling sequence} $\seq{h_n}{n\in\intcc{0}{\xM-1}}$ is here defined as 
\\\indentx$h_n=\sqrt{\sfrac{1}{\xM}}$ \quad for $n\in\intcc{0}{\xM-1}$.
\end{definition}

%---------------------------------------
\begin{definition}
\label{def:whaar}
%---------------------------------------
A \fnctd{length $\xM$ Haar wavelet sequence} $\seq{h_n}{n\in\intcc{0}{\xM-1}}$ is here defined as 
\\\indentx$h_n \eqd 
  \brb{\begin{array}{rM}
    +\sqrt{\sfrac{1}{\xM}} & for $n=0,\,1,\,\ldots,\,\floor{\sfrac{\xM}{2}}-1$\\
    -\sqrt{\sfrac{1}{\xM}} & for $n=\ceil{\sfrac{\xM}{2}},\,\ceil{\sfrac{\xM}{2}}+1,\,\ldots,\,\xM-1$\\
     0                     & otherwise
  \end{array}}$
\end{definition}

%---------------------------------------
\begin{example}
\label{ex:haar}
%---------------------------------------
A   \fncte{length 8 Haar scaling sequence},
a   \fncte{length 8 Haar wavelet sequence},
and \fncte{length 9 Haar wavelet sequence}
are illustrated below:
%\\\indentx
%  \begin{tabular}{|*{2}{>{\scs}c|}}
%    \hline
%      \includegraphics{../common/math/graphics/pdfs/seq_shaar16.pdf} 
%     &\includegraphics{../common/math/graphics/pdfs/seq_whaar16.pdf}
%    \\\fncte{length 16 Haar scaling sequence} 
%     &\fncte{length 16 Haar wavelet sequence}
%    \\\hline
%  \end{tabular}
\\\indentx
  \begin{tabular}{|*{3}{>{\scs}c|}}
    \hline
      \includegraphics{../common/math/graphics/pdfs/seq_shaar8.pdf} 
     &\includegraphics{../common/math/graphics/pdfs/seq_whaar8.pdf}
     &\includegraphics{../common/math/graphics/pdfs/seq_whaar9.pdf}
    \\\fncte{length 8 Haar scaling sequence} 
     &\fncte{length 8 Haar wavelet sequence}
     &\fncte{length 9 Haar wavelet sequence}
    \\\hline
  \end{tabular}
\end{example}



