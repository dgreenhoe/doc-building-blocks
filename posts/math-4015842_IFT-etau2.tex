Daniel J. Greenhoe
https://math.stackexchange.com/questions/4015842/


$\newcommand{\eqd}{\triangleq}$
$\newcommand{\eqa}{\approx}$
$\newcommand{\abs}[1]{{\left\lvert #1 \right\rvert}}$
$\newcommand{\brp}[1]{{\left(#1\right)}}$
$\newcommand{\brs}[1]{{\left[#1\right]}}$
$\newcommand{\brlr}[1]{\left.#1\right|}$
$\newcommand{\deriv} [2]   {{\frac{\mathrm{d}#1}{\mathrm{d}#2} }}$
$\newcommand{\R}{\Bbb{R}}$
$\newcommand{\intcc} [2]  {{\left[#1:#2\right]}}$
$\newcommand{\intoo} [2]  {{\left(#1:#2\right)}}$
$\newcommand{\intoc} [2]  {{\left(#1:#2\right]}}$
$\newcommand{\intco} [2]  {{\left[#1:#2\right)}}$
$\newcommand{\ff}{\mathrm{f}}$
$\newcommand{\fF}{\mathrm{F}}$
$\newcommand{\fG}{\mathrm{G}}$
$\newcommand{\fg}{\mathrm{g}}$
$\newcommand{\fphi}{\mathrm{\phi}}$
$\newcommand{\dx}{\mathrm{dx}}$
$\newcommand{\dtau}{\mathrm{d\tau}}$
$\newcommand{\du}{\mathrm{du}}$
$\newcommand{\dv}{\mathrm{dv}}$
$\newcommand{\ds}{\displaystyle}$
$\newcommand{\thme}[1]{#1}$
$\newcommand{\opFT}{\mathrm{FT}}$
$\newcommand{\opIFT}{\mathrm{IFT}}$
$\newcommand{\thme}[1]{#1}$

Part 1: $\opIFT\brs{\frac{2\alpha}{\alpha^2+\omega^2}} = e^{-\alpha\abs{\tau}}$
-------------------------------------------------------------------------------
Assuming the definition of the Fourier Transform used is
$\fG(\omega)\eqd \int_{-\infty}^{\infty}\fg(\tau)e^{-j\omega\tau}\dtau$,
we can check the relation as follows:
\begin{align}
  \opFT\brs{e^{-\alpha\abs{\tau}}}
    &\eqd \int_{-\infty}^{\infty} e^{-\alpha\abs{\tau}} e^{-j\omega\tau}\dtau
    && \text{by local definition Fourier Transform}
  \\&= \int_{-\infty}^{0} e^{-\alpha(-\tau)} e^{-j\omega\tau}\dtau
     + \int_{0}^{\infty} e^{-\alpha(\tau)} e^{-j\omega\tau}\dtau
    && \text{by definition of $\abs{\tau}$}
  \\&= \int_{-\infty}^{0} e^{\tau(\alpha-j\omega)}\dtau
     + \int_{0}^{\infty} e^{\tau(-\alpha-j\omega)}\dtau
    && \text{by Distributive Property}
  \\&= \brlr{\frac{e^{\tau(\alpha-j\omega)}}{\alpha-j\omega}}_{-\infty}^{0}
     + \brlr{\frac{e^{\tau(-\alpha-j\omega)}}{-\alpha-j\omega}}_{0}^{\infty}
    && \text{by Fundamental Theorem of Calculus}
  \\&= \brs{\frac{1}{\alpha-j\omega} - 0}
     + \brs{0- \frac{1}{-\alpha-j\omega}}
  \\&= \brs{\frac{1}{\alpha-j\omega}}\brs{\frac{\alpha-j\omega}{\alpha-j\omega}}
     + \brs{\frac{1}{\alpha+j\omega}}\brs{\frac{\alpha+j\omega}{\alpha+j\omega}}
  \\&= \frac{\alpha-j\omega}{\alpha^2+\omega^2}
     + \frac{\alpha+j\omega}{\alpha^2+\omega^2}
  \\&= \boxed{\frac{2\alpha}{\alpha^2+\omega^2}}
\end{align}

And so the Inverse Fourier Transform (IFT) of $\frac{2\alpha}{\alpha^2+\omega^2}$
is $e^{-\alpha\abs{\tau}}$.

However, your question does not ask for the IFT of
$\frac{2\alpha}{\alpha^2+\omega^2}=\frac{2\alpha}{\alpha^2+(2\pi f)^2}$,
but rather for the IFT of
$$
  \frac{2\alpha}{\alpha^2+(j2\pi f)^2}
    = \frac{2\alpha}{\alpha^2+(j\omega)^2}
    =\frac{2\alpha}{\alpha^2-\omega^2}
$$

Part 2: $\opIFT\brs{\frac{\alpha}{\alpha^2-\omega^2}} = \brs{\cos(\omega)\cosh(\omega)-j\sin(\omega)\sinh(\omega)}e^{-\alpha\abs{\tau}}$
-------------------------------------------------------------------------------

To prove that, first we need a couple lemmas.
And before the keenly anticipated lemmas, a couple [definitions](http://books.google.com/books?vid=ISBN9789383746460&pg=PA81):
$$\begin{array}{rc>{\ds}ll}
  \cosh(x) &\eqd& \frac{e^{x}+e^{-x}}{2} & \text{(hyperbolic cosine)}\\
  \sinh(x) &\eqd& \frac{e^{x}-e^{-x}}{2} & \text{(hyperbolic sine)}
\end{array}$$

lemma 1:
\begin{align*}
  \boxed{\cos(x+jy)}
    &= \cos(x)\cos(jy) - \sin(x)\sin(jy)
    && \text{by double angle formulas}
  \\&= \cos(x)\brs{\frac{e^{jjy}+e^{-jjy}}{2}} - \sin(x)\brs{\frac{e^{jjy}-e^{-jjy}}{2j}}
    &&  \text{by Euler's identity}
  \\&= \cos(x)\brs{\frac{e^{-y}+e^{y}}{2}} -  \sin(x)\brs{\frac{e^{-y}-e^{y}}{2j}}
  \\&= \cos(x)\brs{\frac{e^{-y}+e^{y}}{2}} +  \sin(x)\brs{\frac{e^{y}-e^{-y}}{2j}}\frac{j}{j}
  \\&= \cos(x)\brs{\frac{e^{y}+e^{-y}}{2}} - j\sin(x)\brs{\frac{e^{y}-e^{-y}}{2}}
  \\&\eqd \boxed{\cos(x)\cosh(y) - j\sin(x)\sinh(y)}
    && \text{by definitions of $\cosh$ and $\sinh$}
\end{align*}
References: 
  [Saxelby(1920)](https://archive.org/details/courseinpractica00saxerich/page/416/),
  [ProofWiki](https://proofwiki.org/wiki/Cosine_of_Complex_Number)

lemma 2:
\begin{align*}
  \opFT\brs{\cos(\beta\tau)e^{-\alpha\abs{\tau}}}
    &\eqd \int_{-\infty}^{\infty} \cos(\beta x)e^{-\alpha\abs{\tau}} e^{-j\omega\tau}\dtau
    && \text{by definition Fourier Transform}
  \\&= \int_{-\infty}^{\infty} \brs{\frac{e^{j\beta\tau}+e^{-j\beta\tau}}{2}} e^{-\alpha\abs{\tau}} e^{-j\omega\tau}\dtau
    &&  \text{by Euler's identity}
  \\&= \frac{1}{2}\int_{-\infty}^{\infty} \brs{e^{-\alpha\abs{\tau}+j\beta\tau-j\omega\tau}+e^{-\alpha\abs{\tau}-j\beta\tau-j\omega\tau}} \dtau
  \\&= \frac{1}{2}\int_{-\infty}^{0} \brs{e^{ \alpha\tau +j\beta\tau-j\omega\tau}+e^{ \alpha\tau -j\beta\tau-j\omega\tau}} \dtau
     + \frac{1}{2}\int_{0}^{\infty}  \brs{e^{-\alpha\tau +j\beta\tau-j\omega\tau}+e^{-\alpha\tau -j\beta\tau-j\omega\tau}} \dtau
  \\&= \frac{1}{2} \brs{\frac{e^{ \alpha\tau +j\tau(\beta-\omega)}}{ \alpha +j(\beta-\omega)} + \frac{e^{ \alpha\tau -j\tau(\beta-\omega)}}{ \alpha -j(\beta-\omega}}_{-\infty}^{0}
     + \frac{1}{2} \brs{\frac{e^{-\alpha\tau +j\tau(\beta-\omega)}}{-\alpha +j(\beta-\omega)} + \frac{e^{-\alpha\tau -j\tau(\beta-\omega)}}{-\alpha -j(\beta-\omega}}_{0}^{\infty} 
    && \text{by Fundamental Theorem of Calculus}
  \\&= \frac{1}{2} \brs{\frac{e^{ \alpha\tau} e^{j\tau(\beta-\omega)}}{ \alpha +j(\beta-\omega)} + \frac{e^{ \alpha\tau} e^{-j\tau(\beta-\omega)}}{ \alpha -j(\beta-\omega)}}_{-\infty}^{0}
     + \frac{1}{2} \brs{\frac{e^{-\alpha\tau} e^{j\tau(\beta-\omega)}}{-\alpha +j(\beta-\omega)} + \frac{e^{-\alpha\tau} e^{-j\tau(\beta-\omega)}}{-\alpha -j(\beta-\omega)}}_{0}^{\infty} 
  \\&= \frac{1}{2} \brs{\frac{1}{ \alpha +j(\beta-\omega)} + \frac{1}{ \alpha -j(\beta-\omega)} - (0+0)}
     + \frac{1}{2} \brs{(0+0)-\brp{\frac{1}{-\alpha +j(\beta-\omega)} + \frac{1}{-\alpha -j(\beta-\omega)}}}
  \\&= \brs{\frac{1}{ \alpha +j(\beta-\omega)}} 
     + \brs{\frac{1}{ \alpha -j(\beta-\omega)}} 
  \\&= \brs{\frac{1}{ \alpha +j(\beta-\omega)}} \brs{\frac{ \alpha -j(\beta-\omega)}{ \alpha -j(\beta-\omega)}}
     + \brs{\frac{1}{ \alpha -j(\beta-\omega)}} \brs{\frac{ \alpha +j(\beta-\omega)}{ \alpha +j(\beta-\omega)}}
    && \text{(Rationalizing the Denominator)}
  \\&= \frac{2\alpha}{\alpha^2+(\beta-\omega)^2}
\end{align*}

Then we can use the above 2 lemmas to finish the proof...
\begin{align*}
\opFT\brs{\brp{\cos(\omega)\cosh(\omega)-j\sin(\omega)\sinh(\omega)}e^{-\alpha\abs{\tau}}}
  &= \opFT\brs{\cos(\omega+j\omega) e^{-\alpha\abs{\tau}}}
  && \text{by lemma 1}
  \\&= \frac{2\alpha}{\alpha^2+([\omega+j\omega]-\omega)^2}
  && \text{by lemma 2}
  \\&= \frac{2\alpha}{\alpha^2-\omega^2}
\end{align*}
And so, the Inverse Fourier Transform of $\frac{\alpha}{\alpha^2-\omega^2}$ is (? or maybe not ?)
$$\brs{\cos(\omega)\cosh(\omega)-j\sin(\omega)\sinh(\omega)}e^{-\alpha\abs{\tau}}$$
