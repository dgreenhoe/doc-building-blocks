%============================================================================
% Daniel J. Greenhoe
% LaTeX file
%============================================================================
%============================================================================
\chapter{System Identification}
%============================================================================

%---------------------------------------
\begin{definition}[\fnctd{Least Squares Technique}]
\footnote{
  \citerpgc{shin2008}{293}{0470725648}{
    $H_1(f)=\frac{\Swxy(f)}{\Swxx(f)}$ (9.63);
    which differs from \pref{def:H1}, but see \prefp{rem:Rxym}
    },
  \citerpgc{bendat2011}{185}{1118210824}{$H_1(f) = \frac{G_{xy}(f)}{G_{xx}(f)}$ (6.37)},
  \citeP{bendat1978}{cited by Cobb(1988)---variance estimate for $\fH_1$},
  \citePc{allemang1979}{cited by Shin(2008)},
  \citePpc{leuridan1986}{910}{\fncte{Least Squares Technique}; (8) $[G_{xx}](H)=[G_{xy}]$},
  \citeP{abom1986}{cited by Cobb(1988)---variance estimate for $\fH_1$},
  \citePppc{allemang1987}{54}{55}{5.3.1 $H_1$ Technique; $[H]=[G_{XF}][G_{FF}]^{-1}$ (11)},
  \citePpc{cobb1988}{2}{$^1\hat{H}(f)=\frac{\hat{G}_{yx}(f)}{\hat{G}_{xx}(f)}$ (1)}
  }
\label{def:H1}
%---------------------------------------
\defbox{\begin{array}{Mrc>{\ds}l}
  The transfer function $\fH_1(\omega)$ is defined as & \fH_1(\omega) &\eqd& \frac{\Swyx(\omega)}{\Swxx(\omega)}
\end{array}}
\end{definition}

%---------------------------------------
\begin{definition}
\footnote{
  \citerpgc{shin2008}{293}{0470725648}{$H_2(f)=\frac{\Swyy(f)}{\Swyx(f)}$ (9.65);
    which differs from \pref{def:H2}, but see \prefp{rem:Rxym}
    },
  \citerpgc{bendat2011}{186}{1118210824}{$H_2(f) = \frac{G_{yy}(f)}{G_{yx}(f)}$ (6.42)},

  \citePc{mitchell1980}{cited by Cobb(1988)},
  \citePc{mitchell1982}{cited by Cobb(1988)},
  \citePpc{cobb1988}{3}{$^2\hat{H}(f)=\frac{\hat{G}_{yy}(f)}{\hat{G}_{xy}(f)}$ (1)}
  }
\label{def:H2}
%---------------------------------------
\defbox{\begin{array}{Mrc>{\ds}l}
  The transfer function $\fH_2(\omega)$ is defined as & \fH_2(\omega) &\eqd& \frac{\Swyy(\omega)}{\Swxy(\omega)}
\end{array}}
\end{definition}

%---------------------------------------
\begin{proposition}
\footnote{
  \citerpg{shin2008}{293}{0470725648}
  }
%---------------------------------------
\propbox{
  \brb{\begin{array}{rc>{\ds}l}
    \fy'[m] &=& \fy[m] + \fv[m]
  \end{array}}
  \implies
  \brb{\begin{array}{FrclD}
          &\fH_1(\omega) = \fH(\omega)
%    \\(2).&\fH_2(\omega) &=& 
  \end{array}}
  }
\end{proposition}
\begin{proof}
\begin{align*}
  \fH_1(\omega)
    &\eqd \frac{\Swxx[y'x](\omega)}{\Swxx(\omega)}
    && \text{by definition of $\fH_1$}
    && \text{\xref{def:H1}}
  \\&\eqd \frac{\opDTFT\Rxx[y'x][m]}{\Swxx(\omega)}
  \\&\eqd \frac{\opDTFT\pE\brs{\rvy'[m]\rvx^\ast[0]}}{\Swxx(\omega)}
  \\&\eqd \frac{\opDTFT\pE\brs{(\rvy[m]+\fv[m])\rvx^\ast[0]}}{\Swxx(\omega)}
  \\&=    \frac{\opDTFT\pE\brs{\rvy[m]\rvx^\ast[0]}}{\Swxx(\omega)}
     +    \frac{\opDTFT\pE\brs{\fv[m]\rvx^\ast[0]}}{\Swxx(\omega)}
  \\&=    \frac{\opDTFT\pE\brs{\rvy[m]\rvx^\ast[0]}}{\Swxx(\omega)}
     +    0
  \\&=    \frac{\Swyx(\omega)}{\Swxx(\omega)}
  \\&\eqd \fH(\omega)
\end{align*}
\end{proof}
