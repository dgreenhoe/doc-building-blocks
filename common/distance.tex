%============================================================================
% LaTeX File
% Daniel J. Greenhoe
%============================================================================

%======================================
\chapter{Distance Spaces}
\label{chp:dspace}
%======================================

A \structe{distance space} \xref{def:dspace} can be defined as a \structe{metric space} \xref{def:metric}
without the \prope{triangle inequality} constraint.
Much of the material in this chapter about \structe{distance space}s is
standard in \structe{metric space}s.
However, this chapter revisits what may commonly be associated with metric spaces 
to demonstrate ``how far we can go", and can't go, 
without the \prope{triangle inequality}.

%======================================
\section{Fundamental structure of distance spaces}
%======================================
%======================================
%\subsection{Definitions}
%======================================
%---------------------------------------
\begin{definition}
\footnote{
  \citePpc{menger1928}{76}{``Abstand $a$ $b$ definiert ist\ldots" (distance from $a$ to $b$ is defined as\ldots")},
  \citePpc{wilson1931a}{361}{\textsection1., ``distance", ``semi-metric space"},
  \citePp{blumenthal1938}{38},
  \citerpgc{blumenthal1953}{7}{0828402426}{``{\scshape Definition 5.1.} A distance space is called semimetric provided\ldots"},
  \citePpc{galvin1984}{67}{``distance function"},
  \citerpgc{laos1998}{118}{9810231806}{``distance space"},
  \citerpgc{khamsi2001}{13}{0471418250}{``semimetric space"},
  \citePpc{bessenyei2014}{2}{``semimetric space"}, %{{\scshape Conventions And Basic Notions}},
  \citerpgc{deza2014}{3}{3662443422}{``\textbf{distance} (or \textbf{dissimilarity})"}
  }
\label{def:distance}
\label{def:dspace}
%---------------------------------------
%Let $\setX$ be a set and $\Rnn$ the set of non-negative real numbers.
\defboxt{
  A function $\hxs{\distancen}$ in the set $\clF{\setX\times\setX}{\R}$\ifsxref{relation}{def:clFxy} is a \fnctd{distance} if
  \\\indentx$\begin{array}{F rcl CDD}
        1. & \distance{x}{y} &\ge& 0               & \forall x,y   \in\setX & (\prope{non-negative})   & and 
      \\2. & \distance{x}{y} &=  & 0  \iff x=y     & \forall x,y   \in\setX & (\prope{nondegenerate})  & and 
      \\3. & \distance{x}{y} &=  & \distance{y}{x} & \forall x,y   \in\setX & (\prope{symmetric})      & 
    \end{array}$
  \\The pair $\dspaceX$ is a \structd{distance space} if $\distancen$ is a \fncte{distance} on a set $\setX$.
  \\A \fncte{distance} is also called a \fnctd{dissimilarity}.
  }
\end{definition}

In a \structe{metric space} \xref{def:metric}\index{space!metric},
it is sometimes useful to know the maximum distance between any two points in the set.
This maximum distance is called the \fncte{diameter} of the set (next definition).

%---------------------------------------
\begin{remark}
%---------------------------------------
The \fncte{diameter} is an example of a broader class of functions called 
\fncte{set functions}.\footnote{
  \citerpg{pap1995}{7}{0792336585},
  \citer{hahn1948},
  \citor{choquet1954}
  }
Some \structe{distance space}s \xref{def:distance} and all \structe{metric space}s \xref{def:metric} induce 
\structe{topological space}s \xref{def:topology}.
However the \fncte{set function} %\xref{def:setf}
\fncte{set diameter} %(next definition) 
and the related property of \prope{boundedness} \xref{def:bounded} 
are fundamentally \structe{distance space} concepts, 
not topological ones.\footnote{in \structe{metric space}: \citerpg{munkres2000}{121}{0131816292}}
\end{remark}

%---------------------------------------
\begin{definition}
\footnote{
  in \structe{metric space}:
  %\citerpg{davis2005}{28}{0071243399} \\
  \citerp{hausdorff1937e}{166},
  \citerp{copson1968}{23},
  \citerp{michel1993}{267},
  \citerpg{molchanov2005}{389}{185233892X}
  }
\label{def:diam}
%---------------------------------------
Let $\dspaceX$ be a \structe{distance space} and $\psetX$ be the \structe{power set} of $\setX$ \xref{def:pset}.
\defboxt{
  The \vald{diameter} $\diam\setA$ of a set $\setA\in\psetX$ is
  \quad$\ds
  \diam\setA \eqd
    \brbl{\begin{array}{ll}
      0                            & \text{for } \setA=\emptyset \\
      \sup\set{\distance{x}{y}}{x,y\in\setA} & \text{otherwise}
    \end{array}}$
  }
\end{definition}

%The \fncte{diameter} of a set is a special case of a  
%\prefpp{sec:topmetric} presents some topological properties of metric spaces.
%---------------------------------------
\begin{definition}
\footnote{
  in \structe{metric space}:
  \citerpc{thron1966}{154}{definition 19.5},
  \citerpg{bruckner1997}{356}{013458886X}
  }
\label{def:bounded}
%---------------------------------------
Let $\dspaceX$ be a \structe{distance space}.
Let $\psetX$ be the \structe{power set} of $\setX$.
\defboxt{
  A set $\setA$ is \propd{bounded} in $\dspaceX$ if
  \\\indentx$\setA\in\psetX$ and $\diam\setA<\infty$.
  }
\end{definition}

%%---------------------------------------
%\begin{remark}
%%---------------------------------------
%Although the function $\diam$ is a \fncte{set function}, it is not a \fncte{measure}.
%\end{remark}
%\begin{proof}
%Let $\setA$ and $\setB$ be sets that are \prope{disjoint} and \prope{disconnected}
%with respect to each other.
%\begin{align*}
%  \diam{\setA \setu \setB}
%    &= \diam\setA + \diam\setB + \inf\set{\distance{x}{y}}{x,y\in\setA}
%  \\&\ge \diam\setA + \diam\setB
%  \\\implies & \text{$\diam$ is not a measure}
%    && \text{by definition of measure \ifdochas{measure}{\prefpo{def:measure}}}
%\end{align*}
%\end{proof}


%%---------------------------------------
%\begin{definition}
%\label{def:ods}
%%---------------------------------------
%\defboxp{
%  The triple $\omsD$ is an \structd{ordered distance space} if
%  $\dspaceX$ is a \structe{distance space} \xref{def:dspace}
%  and $\osetX$ is an \structe{ordered set} \xref{def:poset}.
%  }
%\end{definition}

%=======================================
%\subsection{Properties}
%=======================================
%%---------------------------------------
%\begin{remark}
%\footnote{
%  \citePpc{heath1961}{810}{{\scshape Theorem}},
%  \citePpc{galvin1984}{71}{{\scshape 2.3 Lemma}}
%  }
%%---------------------------------------
%Although every \structe{metric space} \xref{def:metric} induces a \structe{topology} \xref{def:topology},
%a \structe{distance space} does \textbf{not} in general induce a topology.
%\end{remark}
%
%---------------------------------------
\begin{remark}
%---------------------------------------
%Let $\dspaceX$ be a \structb{distance space} \xref{def:dspace}.
Let $\seqxZ{x_n}$ be a \fncte{sequence} in a \structe{distance space} $\dspaceX$.
The \structe{distance space} $\dspaceX$ does not necessarily have all the nice properties that a 
\structe{metric space} \xref{def:metric} has.
In particular, note the following:
\\\indentx\remboxp{$\begin{array}{FMcMM}
  1. & $\distancen$   is a \fncte{distance} in $\dspaceX$ &\notimplies& $\distancen$ is \prope{continuous} in $\dspaceX$                  & \xref{ex:dspace_21}.\\
  2. & $\balln$ is an \structe{open ball} in $\dspaceX$ &\notimplies& $\balln$ is \prope{open} in $\dspaceX$                              & \xref{ex:dspace_1n}.\\
  3. & $\baseB$ is the set of all                                    &\notimplies& $\baseB$ is a \structe{base} for a                     & \xref{ex:dspace_1n}.\footnotemark\\
     &                            \structe{open ball}s in $\dspaceX$ &           &                                    topology on $\setX$ & \\
  4. & $\seqn{x_n}$ is \prope{convergent} in $\dspaceX$ &\notimplies& limit is \prope{unique}                                             & \xref{ex:dspace_01}.\\
  5. & $\seqn{x_n}$ is \prope{convergent} in $\dspaceX$ &\notimplies& $\seqn{x_n}$ is \prope{Cauchy} in $\dspaceX$                        & \xref{ex:dspace_1n}.\\
\end{array}$}
%In a \structe{metric space} \xref{def:metric}, if a \fncte{sequence} has a limit, that limit is unique.
%This is not true in general for \structe{distance space}s \xref{def:distance}.
%See \prefpp{ex:dspace_01} for a distance space which converges to two distinct limits.
\footnotetext{
  \citePpc{heath1961}{810}{{\scshape Theorem}},
  \citePpc{galvin1984}{71}{{\scshape 2.3 Lemma}}
  }
\end{remark}


%=======================================
\section{Open sets in distance spaces}
%=======================================
%=======================================
%\subsection{Definitions}
%=======================================
%\pref{def:ball} (next) defines the \structe{open ball}.
%In a \structe{metric space} \xref{def:metric}\index{space!metric}, 
%sets are often specified in terms of an \prope{open ball};
%and an open ball is specified in terms of a metric.
%---------------------------------------
\begin{definition}
\footnote{
  in \structe{metric space}:
  \citerp{ab}{35}
  }
\label{def:ball}
\label{def:ballc}
%---------------------------------------
Let $\dspaceX$ be a \structe{distance space} \xref{def:distance}.
Let $\Rp$ be the \structe{set of positive real numbers}\ifsxref{randprocess}{def:Rx}.
\\\defboxt{$\begin{array}{Mrcl}
  An \structd{open ball}   centered at $x$ with radius $r$ is the set & \ball{x}{r}  &\eqd& \set{y\in\setX}{\distance{x}{y}<r}.\\
  A  \structd{closed ball} centered at $x$ with radius $r$ is the set & \ballc{x}{r} &\eqd& \set{y\in\setX}{\distance{x}{y}\le r}.
\end{array}$}
\end{definition}

%Open balls will often ``appear" different in different metric spaces. 
%Some examples include the following \xref{ex:taxicab}:\\
%\setlength{\unitlength}{\tw/6000}
%\begin{tabular}{cll>{\footnotesize(}l<{\footnotesize)}}
%%\begin{tabular}{cl}%<{:}l>{\footnotesize(}l<{\footnotesize)}}
%  \begin{picture}(300,350)(-130,-130)
%    \thicklines
%    \color{axis}%
%      \put(-130,   0){\line(1,0){260} }%
%      \put(   0,-130){\line(0,1){260} }%
%    \color{blue}%
%      \qbezier( 100,0)( 50, 50)(0, 100)%
%      \qbezier(-100,0)(-50, 50)(0, 100)%
%      \qbezier(-100,0)(-50,-50)(0,-100)%
%      \qbezier( 100,0)( 50,-50)(0,-100)%
%  \end{picture}
%  & \prope{taxi-cab metric}%  & \pref{ex:ms_taxi} & \prefpo{ex:ms_taxi} 
%  \\
%  \begin{picture}(300,300)(-130,-130)
%    \thicklines
%    \color{axis}%
%      \put(-130,   0){\line(1,0){260} }%
%      \put(   0,-130){\line(0,1){260} }%
%    \color{blue}%============================================================================
% NCTU - Hsinchu, Taiwan
% LaTeX File
% Daniel Greenhoe
%
% Unit circle with radius 100
%============================================================================

\qbezier( 100,   0)( 100, 41.421356)(+70.710678,+70.710678) % 0   -->1pi/4
\qbezier(   0, 100)( 41.421356, 100)(+70.710678,+70.710678) % pi/4-->2pi/4
\qbezier(   0, 100)(-41.421356, 100)(-70.710678,+70.710678) %2pi/4-->3pi/4
\qbezier(-100,   0)(-100, 41.421356)(-70.710678,+70.710678) %3pi/4--> pi 
\qbezier(-100,   0)(-100,-41.421356)(-70.710678,-70.710678) % pi  -->5pi/4
\qbezier(   0,-100)(-41.421356,-100)(-70.710678,-70.710678) %5pi/4-->6pi/4
\qbezier(   0,-100)( 41.421356,-100)( 70.710678,-70.710678) %6pi/4-->7pi/4
\qbezier( 100,   0)( 100,-41.421356)( 70.710678,-70.710678) %7pi/4-->2pi


%
%  \end{picture}
%  & \prope{Euclidean metric}% & \pref{ex:ms_euclidean} & \prefpo{ex:ms_euclidean} 
%  \\
%  \begin{picture}(300,300)(-130,-130)
%    \thicklines
%    \color{axis}%
%      \put(-130,   0){\line(1,0){260} }%
%      \put(   0,-130){\line(0,1){260} }%
%    \color{blue}%
%      \put(-100,-100){\line( 1, 0){200} }%
%      \put(-100,-100){\line( 0, 1){200} }%
%      \put( 100, 100){\line(-1, 0){200} }%
%      \put( 100, 100){\line( 0,-1){200} }%
%  \end{picture}
%  & \prope{sup metric}% & \pref{ex:ms_sup} & \prefpo{ex:ms_sup} 
%%  \\
%%  \begin{picture}(300,300)(-130,-130)%
%%    %{\color{graphpaper}\graphpaper[10](-150,-150)(300,300)}%
%%    \thicklines%
%%    \color{axis}%
%%      \put(-130,   0){\line(1,0){260} }%
%%      \put(   0,-130){\line(0,1){260} }%
%%    \color{blue}%
%%      \qbezier( 100,0)(0,0)(0, 100)%
%%      \qbezier( 100,0)(0,0)(0,-100)%
%%      \qbezier(-100,0)(0,0)(0,-100)%
%%      \qbezier(-100,0)(0,0)(0, 100)%
%%  \end{picture}
%%  & \prope{parabolic metric} & \pref{ex:ms_parabolic} & \prefpo{ex:ms_parabolic}
%%  \\ 
%%  \begin{picture}(400,400)(-200,-200)%
%%    \thicklines%
%%    \color{axis}%
%%      \put(-200,   0){\line(1,0){400} }%
%%      \put(   0,-200){\line(0,1){400} }%
%%    \color{blue}%
%%      \qbezier( 100,0)(100,100)(0, 100)%
%%      \qbezier(-171,0)(-50,50)(0, 100)%
%%      \qbezier(-171,0)(-50,-50)(0,-171)%
%%      \qbezier( 100,0)(50,-50)(0,-171)%
%%  \end{picture}
%%  & \prope{exponential metric} & \pref{ex:ms_32x} & \prefpo{ex:ms_32x}
%%  \\ 
%%  \begin{picture}(300,300)(-150,-150)%
%%    \thicklines%
%%    \color{axis}%
%%      \put(-130,   0){\line(1,0){260} }%
%%      \put(   0,-130){\line(0,1){260} }%
%%    \color{blue}%
%%      \qbezier( 100,0)(70,70)(0, 100)%
%%      \qbezier(-100,0)(-70,70)(0, 100)%
%%      \qbezier(-100,0)(-70,-70)(0,-100)%
%%      \qbezier( 100,0)(70,-70)(0,-100)%
%%  \end{picture}
%%  & \prope{tangential metric} & \pref{ex:ms_tan} & \prefpo{ex:ms_tan}
%\end{tabular}

%%---------------------------------------
%\begin{theorem}
%\footnote{
%  \citerpp{isham1999}{10}{11} \\
%  \citor{birkhoff1933}
%  }
%%---------------------------------------
%\thmbox{\begin{array}{rc>{\ds}l}
%  \distancesub{1}{x}{y} \join \distancesub{2}{x}{y} 
%    &\eqd& \max\brb{\distancesub{1}{x}{y},\, \distancesub{2}{x}{y}} 
%    \\
%  \distancesub{1}{x}{y} \meet \distancesub{2}{x}{y} 
%    &\eqd& \inf_{\seqn{x=x_1,x_2,\ldots,x_n=y}} \sum_{i=1}^{n-1} \brb{\distancesub{1}{x_i}{x_{i+1}},\, \distancesub{2}{x_i}{x_{i+1}}}
%\end{array}}
%\end{theorem}

%\begin{figure}[h]
%  \centering%
%  \psset{unit=6mm}%
%  %============================================================================
% Daniel J. Greenhoe
% LaTeX file
%============================================================================
{\psset{unit=0.30mm}
\begin{pspicture}(-152,-80)(152,90)
  %-------------------------------------
  % options
  %-------------------------------------
  \psset{
    dotsize=5pt,
    linestyle=dashed,
    fillstyle=none,
    %labelsep=5pt,
    }
  %-------------------------------------
  % design support
  %-------------------------------------
  %\psgrid[unit=100\psunit](-1,-1)(1,1)
  %-------------------------------------
  % axes
  %-------------------------------------
  %\psline[linecolor=axis]{<->}(-120,0)(120,0)% x-axis
  %\psline[linecolor=axis]{<->}(0,-120)(0,120)% y-axis
  %-------------------------------------
  % nodes
  %-------------------------------------
  \pnode(  0,  7){o}% origin (and center of outer ball)
  \pnode( 41, 25){p1}% a point p (and center of an inner ball)
  \pnode(  0, 55){p2}% a point p (and center of an inner ball)
  \pnode( 83, 19){p3}% a point p (and center of an inner ball)
  \pnode( 67,-37){p4}% a point p (and center of an inner ball)
  \pnode(-23,-29){p5}% a point p (and center of an inner ball)
  \pnode(-59, 17){p6}% a point p (and center of an inner ball)
  \pnode(-13, 11){p7}% a point p (and center of an inner ball)
  \pnode( 20,-53){p8}% a point p (and center of an inner ball)
  %-------------------------------------
  % objects
  %-------------------------------------
  \psccurve[linecolor=blue](110,0)(80,80)(0,70)(-110,0)(-50,-30)(0,-70)% open set
  \psdot(p1)\pscircle[linecolor=red](p1){33}\uput{4pt}[0](p1){$p_1$}%
  \psdot(p2)\pscircle[linecolor=red](p2){15}\uput{4pt}[90](p2){$p_2$}%
  \psdot(p3)\pscircle[linecolor=red](p3){20}\uput{4pt}[0](p3){$p_3$}%
  \psdot(p4)\pscircle[linecolor=red](p4){11}\uput{4pt}[45](p4){$p_4$}%
  \psdot(p5)\pscircle[linecolor=red](p5){19}\uput{4pt}[0](p5){$p_5$}%
  \psdot(p6)\pscircle[linecolor=red](p6){30}\uput{4pt}[90](p6){$p_6$}%
  \psdot(p7)\pscircle[linecolor=red](p7){31}\uput{4pt}[0](p7){$p_7$}%
  \psdot(p8)\pscircle[linecolor=red](p8){15}\uput{4pt}[90](p8){$p_8$}%
\end{pspicture}
}%
%}%


%  \caption{\structe{open set} \xref{def:dspace_open} \label{fig:ms_open}}
%\end{figure}
%---------------------------------------
\begin{definition}
\label{def:dspace_open}
%\label{def:dspace_closed}
%---------------------------------------
Let $\dspaceX$ be a \structe{distance space}. % \xref{def:dspace}.
Let $\setX\setd\setA$ be the \ope{set difference} of $\setX$ and a set $\setA$.
\defboxp{
  A set $\setU$ is \propd{open} in $\dspaceX$ if $\setU\in\psetX$ and 
  %\\\indentx$x\in\setU \qquad\implies\qquad$there exists $r\in\Rp$ such that $\ball{x}{r}\subseteq \setU$.
  for every $x$ in $\setU$ there exists $r\in\Rp$ such that $\ball{x}{r}\subseteq \setU$.
  A set $\setU$ is an \structd{open set} in $\dspaceX$ if $\setU$ is \prope{open} in $\dspaceX$. 
  A set $\setD$ is \propd{closed} in $\dspaceX$ if $\brp{\setX\setd\setD}$ is \prope{open}.
  A set $\setD$ is a \structd{closed set} in $\dspaceX$ if $\setD$ is \prope{closed} in $\dspaceX$. 
  }
\end{definition}

%=======================================
%\subsection{Properties}
%=======================================
%\pref{thm:dspace_open} (next) identifies four fundamental properties of open sets in
%distance spaces.
%These properties are the same as those defining a topology\ifsxref{topology}{def:topology}.
%---------------------------------------
\begin{theorem}
\footnote{
  in \structe{metric space}:
  \citerpp{dieudonne1969}{33}{34},
  \citerpg{rosenlicht}{39}{0486650383}
  %\citerp{giles1987}{215} \\
  %\citerpg{davis2005}{19}{0071243399}
  %\citerpg{ab}{35}{0120502577}
  }
\label{thm:dspace_open}
\index{space!metric}
%---------------------------------------
Let $\dspaceX$ be a \structe{distance space}. % \xref{def:dspace}.
Let $\xN$ be any (finite) positive integer.
Let $\Gamma$ be a \structe{set} possibly with an uncountable number of elements.
\thmbox{\begin{array}{F Mll>{\ds}l l}
    1. &                                                                &                         &          & \setX                                   & \text{is \prope{open}.}\\
    2. &                                                                &                         &          & \emptyset                               & \text{is \prope{open}.}\\
    3. & each element in $\setxn{\setU_n}$                              & \text{is \prope{open}} & \implies & \setopi_{n=1}^\xN \setU_n               & \text{is \prope{open}.}\\
    4. & each element in $\set{\setU_\gamma\in\psetX}{\gamma\in\Gamma}$ & \text{is \prope{open}} & \implies & \setopu_{\gamma\in\Gamma} \setU_\gamma  & \text{is \prope{open}.} 
  \end{array}}
\end{theorem}
\begin{proof}
\begin{enumerate}
  \item Proof that $\setX$ is \prope{open} in $\dspaceX$:
    \begin{enumerate}
      \item By definition of \structe{open set} \xref{def:dspace_open}, 
            $\setX$ is \prope{open} $\iff \forall x\in\setX \quad\exists r \st \ball{x}{r}\subseteq \setX$.
      \item By definition of \structe{open ball} \xref{def:ball}, it is always true that $\ball{x}{r}\subseteq\setX$ in $\dspaceX$.
      \item Therefore, $\setX$ is \prope{open} in $\dspaceX$.
    \end{enumerate}

  \item Proof that $\emptyset$ is \prope{open} in $\dspaceX$:
    \begin{enumerate}
      \item By definition of \structe{open set} \xref{def:dspace_open}, 
            $\emptyset$ is \prope{open} $\iff \forall x\in\setX \quad\exists r \st \ball{x}{r}\subseteq \emptyset$.
      \item By definition of \structe{empty set} $\emptyset$, %\xref{def:emptyset}, 
            this is always true because no $x$ is in $\emptyset$.
      \item Therefore, $\emptyset$ is \prope{open} in $\dspaceX$.
    \end{enumerate}

  \item Proof that $\setopu\setU_\gamma$ is \prope{open} in $\dspaceX$:
    \begin{enumerate}
      \item By definition of \structe{open set} \xref{def:dspace_open}, 
            $\setopu\setU_\gamma$ is \prope{open} $\iff \forall x\in\setopu\setU_\gamma \quad\exists r \st \ball{x}{r}\subseteq\setopu\setU_\gamma$.
      \item If $x\in\setopu\setU_\gamma$, then there is at least one $\setU\in\setopu\setU_\gamma$ that contains $x$.
      \item By the left hypothesis in (4), that set $\setU$ is open and so for that $x$,
            $\exists r \st \ball{x}{r}\subseteq\setU\subseteq\setopu\setU_\gamma$.
      \item Therefore, $\setopu\setU_\gamma$ is \prope{open} in $\dspaceX$.
    \end{enumerate}

  \item Proof that $\setU_1$ and $\setU_2$ are \prope{open} $\implies$ $\setU_1\seti\setU_2$ is \prope{open}: \label{ilem:ms_open_seti}
    \begin{enumerate}
      \item By definition of \structe{open set} \xref{def:dspace_open}, 
            $\setU_1\seti\setU_2$ is \prope{open} $\iff \forall x\in\setU_1\seti\setU_2\quad\exists r \st \ball{x}{r}\subseteq\setU_1\seti\setU_2$.
      \item By the left hypothesis above, $\setU_1$ and $\setU_2$ are \prope{open};
            and by the definition of \structe{open set}s \xref{def:dspace_open}, 
            there exists $r_1$ and $r_2$ such that
            $\ball{x}{r_1}\subseteq\setU_1$ and $\ball{x}{r_2}\subseteq\setU_2$.
      \item Let $r\eqd\min\setn{r_1,r_2}$. Then $\ball{x}{r}\subseteq\setU_1$ and $\ball{x}{r}\subseteq\setU_2$.
      \item By definition of \ope{set intersection} $\seti$ then, $\ball{x}{r}\subseteq\setU_1\seti\setU_2$.
      \item By definition of \ope{open set} \xref{def:dspace_open}, $\setU_1\seti\setU_2$ is \prope{open}.
    \end{enumerate}

  \item Proof that $\setopi_{n=1}^\xN\setU_n$ is \prope{open} (by induction):
    \begin{enumerate}
      \item Proof for $\xN=1$ case: $\setopi_{n=1}^\xN\setU_n=\setopi_{n=1}^1\setU_n=\setU_1$ is \prope{open} by hypothesis.
      \item Proof that $\xN$ case $\implies$ $\xN+1$ case: 
        \begin{align*} 
          \setopi_{n=1}^{\xN+1}\setU_n
            &= \brp{\setopi_{n=1}^{\xN}\setU_n}\seti\setU_{\xN+1}
            && \text{by property of $\setopi$}
          \\&\implies \text{\prope{open}} 
            && \text{by ``$\xN$ case" hypothesis and \prefp{ilem:ms_open_seti}}
        \end{align*}
    \end{enumerate}
\end{enumerate}
\end{proof}

%---------------------------------------
\begin{corollary}
\label{cor:dspace_open}
%---------------------------------------
Let $\dspaceX$ be a \structe{distance space}.
%Let an $\prope{open}$ set in $\psetX$ be defined as in \prefpp{def:dspace_open}.
\corboxt{
  The set 
  \quad$\topT\eqd\set{\setU\in\psetX}{\text{$\setU$ is \prope{open} in $\dspaceX$}}$\quad
  is a \structe{topology} on $\setX$, 
  \\and $\topspaceX$ is a \structe{topologogical space}.
  }
\end{corollary}
\begin{proof}
This follows directly from the definition of an \prope{open} set \xref{def:dspace_open},
\prefpp{thm:dspace_open}, and the definition of \structe{topology} \xref{def:topology}.
\end{proof}

%%---------------------------------------
%\begin{corollary}
%\label{cor:dspace_open}
%%---------------------------------------
%Let $\dspaceX$ be a \structe{distance space} \xref{def:dspace}.
%Let $\Gamma$ be a \structe{set} possibly with an uncountable number of elements.
%Let $\baseU\eqd\set{\setU_\gamma\in\psetX}{\gamma\in\Gamma}$.
%Let $\baseB$ be a family of sets such that $\baseB\subseteq\psetX$.
%\corbox{
%  \brb{\begin{array}{FMD}
%    1. & each set in $\baseU$ is \prope{open} in $\dspaceX$ & and
%  \\2. & $\baseU\subseteq\baseB$ &and 
%  \\3. & $\setX\in\baseB$ & and
%  \\4. & $\setU_1,\setU_2\in\baseU\implies\setU_1\seti\setU_2\in\baseB$ & and
%  \end{array}}
%  \quad\implies\quad
%  \brb{\begin{array}{M}
%    $\baseB$ is a \structe{base} for a\\\structe{topology} \xref{def:topology} on $\setX$
%  \end{array}}
%  }
%\end{corollary}
%\begin{proof}
%This follows directly from \prefpp{thm:dspace_open}, the definition of \structe{topology} \xref{def:topology},
%and the definition of \structe{base} \xref{def:base}.
%\end{proof}

Of course it is possible to define a very large number of topologies even on a finite set with just a handful of elements;\footnote{
  For a finite set $\setX$ with $n$ elements, there are 
  29 topologies on $\setX$ if $n=3$, 
  6942 topologies on $\setX$ if $n=5$, and
  and 8,977,053,873,043 (almost 9 trillion) topologies on $\setX$ if $n=10$.
  References:  \citeoeis{A000798},
  \citerp{brown1996}{31},
  \citerpg{comtet1974}{229}{9027704414},
  \citer{comtet1966},
  \citerp{chatterji1967}{7},
  \citer{evans1967},
  \citerp{krishnamurthy1966}{157}
  }
and it is possible to define an infinite number of topologies even on a \prope{linearly ordered} infinite set
like the \structe{real line} $\osetR$.\footnote{%
  For examples of topologies on the real line, see the following:
      \citerpgc{adams2008}{31}{0131848690}{"six topologies on the real line"},
      \citerppgc{salzmann2007}{64}{70}{0521865166}{Weird topologies on the real line},
      \citerpgc{murdeshwar1990}{53}{8122402461}{``often used topologies on the real line"},
      \citerppgc{joshi1983}{85}{91}{0852264445}{\textsection4.2 Examples of Topological Spaces}
  }
Be that as it may, \pref{def:dspacetop} (next definition) defines a single but convenient 
\structe{topological space} in terms of a \structe{distance space}.
Note that every \structe{metric space} conveniently and naturally induces a \structe{topological space}
because the \structe{open ball}s of the metric space form a \structe{base} for the \structe{topology}.
This is not the case for all {distance space}s.
But if the open balls of a \structe{distance space} are all \prope{open}, 
then those open balls induce a topology (next theorem).\footnote{
  \structe{metric space}: \prefp{def:metric}; 
  \structe{open ball}: \prefp{def:ball};
  \structe{base}: \prefp{def:base};  
  \structe{topology}: \prefp{def:topology};
  not all open balls are open in a distance space: \prefpp{ex:dspace_01} and \prefpp{ex:dspace_1n};
  }
%---------------------------------------
\begin{definition}
\label{def:dspacetop}
%---------------------------------------
Let $\dspaceX$ be a \structe{distance space}. % \xref{def:dspace}.
%Let $\psetX$ be the \structe{power set} \xref{def:psetx} of $\setX$.
\defboxt{
  The set 
  \quad$\topT\eqd\set{\setU\in\psetX}{\text{$\setU$ is \prope{open} in $\dspaceX$}}$\quad
  is the \structd{topology induced by $\dspaceX$ on $\setX$}. 
  \\The pair $\topspaceX$ is called the \structd{topological space induced by $\dspaceX$}. 
  }
\end{definition}

For any \structe{distance space} $\dspaceX$, no matter how strange, 
there is guaranteed to be at least one \structe{topological space induced by $\dspaceX$}---and that 
is the \structe{indiscrete topological space} \xref{ex:idts} because for any distance space $\dspaceX$, 
$\emptyset$ and $\setX$ are \prope{open set}s in $\dspaceX$ \xref{thm:dspace_open}.



%In a \structe{metric space}, the set of all \structe{open ball}s 
%is a \structe{base} for a \structe{topology}.
%This is largely due to the fact that in a \structe{metric space}, all \structe{open ball}s are \prope{open}.
%However, in a \structe{distance space}, \structe{open ball}s are not always \prope{open} \xref{ex:dspace_1n}.
%But if they all are, then the open balls induce a topology (next theorem).\footnote{
%  \structe{metric space}: \prefp{def:metric}; 
%  \structe{open ball}: \prefp{def:ball};
%  \structe{base}: \prefp{def:base};  
%  \structe{topology}: \prefp{def:topology};
%  not all \structe{open ball}s are \prope{open} in a \structe{distance space}: \prefp{ex:dspace_1n};
%  }
%---------------------------------------
\begin{theorem}
\label{thm:baseoball}
%---------------------------------------
Let $\baseB$ be the set of all \structe{open ball}s in a \structe{distance space}
$\dspaceX$. % \xref{def:dspace}.
\thmbox{
  \brb{\text{every \structe{open ball} in $\baseB$ is \prope{open}}}
  \quad\iff\quad
  \brb{\text{$\baseB$ is a \structe{base} for a \structe{topology}}}
  }
\end{theorem}
\begin{proof}
    \begin{align*}
      \mathrlap{\boxed{\text{every \structe{open ball} in $\baseB$ is \prope{open}}}}
      \\&\implies\text{for every $x$ in $\setB_y\in\baseB$ there exists $r\in\Rp$ such that $\ball{x}{r}\subseteq \setB_y$}
        && \text{by definition of \prope{open} \xref{def:dspace_open}}
      \\&\implies \brb{\begin{array}{M}
                    for every $x\in\setX$ and for every $\setB_y\in\baseB$ containing $x$,\\
                    there exists $\setB_x\in\baseB$ such that\qquad $x\in\setB_x\subseteq\setB_y$.
                  \end{array}}
        && \text{because $\forall\opair{x}{r}\in\setX\times\Rp$, $\ball{x}{r}\subseteq\setX$}
      \\&\implies \boxed{\text{$\baseB$ is a \structe{base} for $\topT$}}
        && \text{by \prefp{thm:basex}}
      %
      \\&\implies \brb{\begin{array}{M}
                    for every $x\in\setX$ and for every $\setU\subseteq\topT$ containing $x$,\\
                    there exists $\setB_x\in\baseB$ such that\qquad $x\in\setB_x\subseteq\setU$.
                  \end{array}}
        && \text{by \prefp{thm:basex}}
      \\&\implies \brb{\begin{array}{M}
                    for every $x\in\setX$ and for every $\setB_y\in\baseB\subseteq\topT$ containing $x$,\\
                    there exists $\setB_x\in\baseB$ such that\qquad $x\in\setB_x\subseteq\setB_y$.
                  \end{array}}
        && \text{by definition of \structe{base} \xref{def:base}}
      \\&\implies \brb{\begin{array}{M}
                    for every $x\in\setB_y\in\baseB\subseteq\topT$,\\
                    there exists $\setB_x\in\baseB$ such that\qquad $x\in\setB_x\subseteq\setB_y$.
                  \end{array}}
      \\&\implies \boxed{\text{every \structe{open ball} in $\baseB$ is \prope{open}}}
        && \text{by definition of \prope{open} \xref{def:dspace_open}}
    \end{align*}
\end{proof}

%%---------------------------------------
%\begin{remark}
%\label{rem:dspace_tspace}
%%---------------------------------------
%In a \structe{metric space} \xref{def:metric}, the set of all \structe{open ball}s \xref{def:ball} 
%is a \structe{base} \xref{def:base} for a \structe{topology} \xref{def:topology}.
%That is, the open balls induce a topology on the base set.
%In a \structe{distance space} \xref{def:dspace}, however, this is \emph{not} the case \xxref{thm:baseoball}{ex:dspace_1n}.
%But this by no means implies that all the beautiful structure of topological spaces is unavailable to us in a distance space.
%Rather, any set of \structe{open set}s \xref{def:dspace_open} in a \structe{distance space} $\dspaceX$ that
%satifies the four relations in \prefpp{thm:dspace_open} \emph{is}, by \prefpp{def:topology}, a \structe{topology} on $\setX$.
%For example, regardless of how strange the \fncte{distance function} $\distancen$ may be, 
%of course the \structe{discrete topology} $\psetX$
%and \structe{indiscrete topology} $\setn{\emptyset,\setX}$ \xref{ex:discretetop}
%are still legitimate topologies on $\setX$.
%And in any \structe{topological space} we may construct on a distance space, 
%all the structure of topological space, such as \structe{derived sets} \xxref{def:clsA}{def:bndA},
%and fundamental properties, such as the \thme{closed set theorem} \xref{thm:cst}, 
%is still available to us.
%\end{remark}

