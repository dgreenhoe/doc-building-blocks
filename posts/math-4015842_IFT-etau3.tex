Daniel J. Greenhoe
https://math.stackexchange.com/questions/4015842/


$\newcommand{\eqd}{\triangleq}$
$\newcommand{\eqa}{\approx}$
$\newcommand{\abs}[1]{{\left\lvert #1 \right\rvert}}$
$\newcommand{\brp}[1]{{\left(#1\right)}}$
$\newcommand{\brs}[1]{{\left[#1\right]}}$
$\newcommand{\brlr}[1]{\left.#1\right|}$
$\newcommand{\deriv} [2]   {{\frac{\mathrm{d}#1}{\mathrm{d}#2} }}$
$\newcommand{\R}{\Bbb{R}}$
$\newcommand{\intcc} [2]  {{\left[#1:#2\right]}}$
$\newcommand{\intoo} [2]  {{\left(#1:#2\right)}}$
$\newcommand{\intoc} [2]  {{\left(#1:#2\right]}}$
$\newcommand{\intco} [2]  {{\left[#1:#2\right)}}$
$\newcommand{\ff}{\mathrm{f}}$
$\newcommand{\fF}{\mathrm{F}}$
$\newcommand{\fG}{\mathrm{G}}$
$\newcommand{\fg}{\mathrm{g}}$
$\newcommand{\fphi}{\mathrm{\phi}}$
$\newcommand{\dx}{\mathrm{dx}}$
$\newcommand{\dS}{\mathrm{ds}}$
$\newcommand{\dtau}{\mathrm{d\tau}}$
$\newcommand{\du}{\mathrm{du}}$
$\newcommand{\dv}{\mathrm{dv}}$
$\newcommand{\step}{\mu}$
$\newcommand{\ds}{\displaystyle}$
$\newcommand{\thme}[1]{#1}$
$\newcommand{\opFT}{\mathrm{F}}$
$\newcommand{\opFT}{\mathrm{L}}$
$\newcommand{\opIFT}{\mathrm{IFT}}$
$\newcommand{\thme}[1]{#1}$
$\newcommand{\Real}{\mathrm{Re}}$


I'll be honest with you---it looks like there is no Inverse Fourier Transform for $2\alpha^2/(\alpha^2-\omega^2)$. 
I'd be happy to be proven wrong...but at least consider the following.

(1) Let $s\eqd j\omega$. Then the Fourier Transform $\int_{-\infty}^{\infty} \fg(\tau)e^{-j\omega\tau}\dtau$ of $\fg(\tau)$
becomes the (Two-Sided) Laplace Transform $\int_{-\infty}^{\infty} \fg(\tau)e^{-s\tau}\dS$ of $\fg(\tau)$.

(2) Note that what we want is a function $\fg(\tau)$ with Laplace Transform $\frac{2\alpha}{\alpha^2+s^2}$. Why?
Because then when we substitute back in $s=j\omega$ we would get 
$$\brlr{\frac{2\alpha}{\alpha^2+s^2}}_{s=j\omega} = \frac{2\alpha}{\alpha^2-\omega^2}$$
and the world would be a happier place.

(3) If only there was such a function $\fg(\tau)$ ... hmmmmm ... But wait! There _is_ a function like that! In fact, if you happened to guess that the function $\fg(\tau)$ is
$$\fg(\tau)\eqd 2\sin(\omega\tau)\step(\tau)$$
where $\step(\tau)=1$ for $\tau\geq1$ and $0$ otherwise (the step function), then you would be right! And you can check that [here](https://books.google.com/books?vid=ISBN9780738672458&pg=PA3) (where the One-Sided Laplace is used and so $\step(\tau)$ is implied).

(4) So the Inverse Laplace Transform of $\fG(s)\eqd\frac{2\alpha}{\alpha^2+s^2}$ is $\fg(\tau)\eqd 2\sin(\omega\tau)\step(\tau)$. So we just set $s=j\omega$ and we're all done here, right? Well not quite. That Laplace Transform $\fG(s)$ has a Region of Convergence of $\Re(s)>0$, where $\Re(s)$ is the real component of the complex $s$. The problem here is that the Fourier Transform is the imaginary axis of $s$---that is, $s=j\omega$, which means $\Re(s)=0$ ... which is *not* in the Region of Convergence of $\fG(s)$ (rather, it **diverges** or "goes to infinity" or "blows up" at $s=j\omega$).

And so, while the Laplace Transform of $\fg(\tau)$ _does_ exist (for $\Re(s)>0$), it does _not_ exist (it diverges) for $\Re(s)\leq0$. And so by extension, it looks like that the Fourier Transform $\fG(\omega)=\frac{2\alpha}{\alpha^2-\omega^2}$ does _not_ exist (because it is _not_ in the Region of Convergence of $\fG(s)$).
