%============================================================================
% Daniel J. Greenhoe
% XeLaTeX file
% Relations on lattices with negation
%============================================================================


%=======================================
\chapter{Relations on lattices with negation}
%=======================================


%=======================================
%\subsection{Orthogonality}
%=======================================
The relations in this chapter are typically defined on 
an \structe{orthocomplemented lattice} \xref{def:latoc}.
Here, some relations are generalized to a \structe{lattice with negation} \xref{def:latn}.
A \structe{lattice} \xref{def:lattice} with an \fncte{ortho negation} negation successfully defined on it 
is an \structe{orthocomplemented lattice} \xref{def:latoc}.
In many cases, these relations only work well on an \structe{orthocomplemented lattice},
and thus many results are restricted to orthocomplemented lattices.
%=======================================
\section{Orthogonality}
%=======================================
%---------------------------------------
\begin{proposition}
\label{prop:latoc_x_orel_y}
%---------------------------------------
Let $\latocd$ be an \structe{orthocomplemented lattice} \xref{def:latoc}.
\propbox{
  x \orel y
  \qquad\implies\qquad 
  \brb{\begin{array}{rclccD}
    x^\ocop &\join& y       &=& \lid   & and \\
    x       &\meet& y^\ocop &=& \lzero &  \\
  \end{array}}
  \qquad\scy\forall x,y\in\setX
  }
\end{proposition}
\begin{proof}
\begin{align*}
  x \orel y
    &\implies x\join x^\ocop \orel y\join x^\ocop
    && \text{by \prope{monotone} property of \structe{lattice}s \xref{prop:latmono}}
  \\&\implies \lid \orel y\join x^\ocop
    && \text{by \prope{excluded middle} property of \structe{ortho lattice}s \xref{def:latoc}}
  \\&\implies x^\ocop\join y = \lid
    && \text{by \prope{upper bounded} property of \structe{bounded lattices} \xref{def:latb}}
  \\
  x \orel y
    &\implies x\meet y^\ocop \orel y\meet y^\ocop
    && \text{by \prope{monotone} property of \structe{lattice}s \xref{prop:latmono}}
  \\&\implies x\meet y^\ocop \orel \lzero
    && \text{by \prope{non-contradiction} property of \structe{ortho lattice}s \xref{def:latoc}}
  \\&\implies x\meet y^\ocop = \lzero
    && \text{by \prope{lower bounded} property of \structe{bounded lattices} \xref{def:latb}}
\end{align*}
\end{proof}

%---------------------------------------
\begin{definition}
\footnote{
  \citerpg{stern1999}{12}{0521461057},
  %\citerp{holland1963}{67}\\
  \citorp{loomis1955}{3}
  }
\label{def:latoc_orthog}
%---------------------------------------
%Let $\latocd$ be an \structe{orthocomplemented lattice} \xref{def:latoc}.
Let $\latnX$ be a \structe{lattice with negation} \xref{def:latn}.
\defbox{\begin{array}{M}%
  The \hid{orthogonality} relation $\symxd{\orthog}\in\clRxx$ is defined as
  \\\indentx$
      x\orthog y 
      \qquad\iffdef\qquad
      x \orel \negat{y}
    $
  \\
  If $x\orthog y$, we say that $x$ is \hid{orthogonal} to $y$.
\end{array}}
\end{definition}

%---------------------------------------
\begin{lemma}
%\footnote{
%  \citerp{holland1963}{67}{ortho case}
%  }
\label{lem:latn_orthog}
%---------------------------------------
%Let $\latocd$ be an \structe{orthocomplemented lattice} \xref{def:latoc}.
Let $\latnX$ be a \structe{lattice with negation} \xref{def:latn}.
\lembox{
  \brb{\begin{array}{lD}
    x \orthog y & (\prope{orthogonal} \xrefnp{def:latoc_orthog})
  \end{array}}
  \qquad\implies\qquad 
  \brb{\begin{array}{lD}
    y \orthog x & (\prope{symmetric})
  \end{array}}
  }
\end{lemma}
\begin{proof}
\begin{align*}
  x \orthog y
    &\implies x \orel \negat{y}
    && \text{by definition of $\orthog$ \xref{def:latoc_orthog}}
  \\&\implies \brp{\negat\negat{y}} \orel \negat{x}
    && \text{by \prope{antitone} property \xref{def:latoc}}
  \\&\implies y \orel \negat{x}
    && \text{by \prope{weak double negation} property of \fncte{negation} \xref{def:negat}}
  \\&\implies y \orthog x
    && \text{by definition of $\orthog$ \xref{def:latoc_orthog}}
\end{align*}
\end{proof}

%---------------------------------------
\begin{lemma}
\footnote{
  \citerp{holland1963}{67}
  }
\label{lem:latoc_orthog}
%---------------------------------------
Let $\latocd$ be an \structe{orthocomplemented lattice} \xref{def:latoc}.
\lembox{
  %\brb{\begin{array}{M}
    \mcom{x \orthog y}{\prope{orthogonal} \xref{def:latoc_orthog}}
  %\end{array}}
  \qquad\implies\qquad 
  \brb{\begin{array}{FlclD}
    1. & x \meet y             &=& \lzero   & and\\
    2. & x^\ocop \join y^\ocop &=& \lid
  \end{array}}
  }
\end{lemma}
\begin{proof}
\begin{align*}
  x \orthog y
    &\implies x \orel y^\ocop
    && \text{by definition of $\orthog$ \xref{def:latoc_orthog}}
  \\&\implies x \meet y \orel y^\ocop \meet y
    && \text{by \prope{monotone} property of \structe{lattice}s \xref{prop:latmono}}
  \\&\implies x \meet y \orel y \meet y^\ocop 
    && \text{by \prope{commutative} property of lattices \ifxref{lattice}{thm:lattice}}
  \\&\implies x \meet y \orel \lzero
    && \text{by \prope{non-contradiction} property of \structe{ortho negation} \xref{def:negor}}
  \\&\implies x \meet y = \lzero
    && \text{by \prope{lower bound} property of \structe{bounded lattices} \xref{def:latb}}
  \\
  \\
  x \orthog y
    &\implies x \orel y^\ocop
    && \text{by definition of $\orthog$ \xref{def:latoc_orthog}}
  \\&\implies x^\ocop \join x  \orel x^\ocop \join y^\ocop
    && \text{by \prope{monotone} property of \structe{lattice}s \xref{prop:latmono}}
  \\&\implies x \join x^\ocop  \orel x^\ocop \join y^\ocop
    && \text{by \prope{commutative} property of \structe{lattices} \xref{thm:lattice}}
  \\&\implies \lid \orel x^\ocop \join y^\ocop 
    && \text{by \prope{excluded middle} property of \structe{ortho lattice}s \xref{thm:latn_ortho}}
  \\&\implies x^\ocop \join y^\ocop
    && \text{by \prope{upper bound} property of \structe{bounded lattice}s \xref{def:latb}}
\end{align*}
\end{proof}

%---------------------------------------
\begin{remark}
%---------------------------------------
{
In an \structe{orthocomplemented lattice} $\latL$,
the \rele{orthogonality} relation $\orthog$ is in general \prope{non-associative}. That is
\\\indentx
  $\brb{\begin{array}{cccD}
    x&\orthog&y & and\\ 
    y&\orthog&z &
  \end{array}}
  \quad\notimplies\quad
  x\orthog z$
}
\end{remark}
\begin{proof}
Consider the $\latL_2^4$ \structe{Boolean lattice} in \prefpp{ex:latoc}.
\\\begin{tabular}{cl}
    \imark & $a^\ocop \orthog p$ because $a^\ocop\orel p^\ocop$.
  \\\imark & $p \orthog r$  because $p\orel r^\ocop$.
  \\\imark & But yet $a^\ocop$ is \emph{not} orthogonal to $r$ because $a^\ocop\nleq r^\ocop$.
\end{tabular}
\end{proof}


%---------------------------------------
\begin{example}
%---------------------------------------
\exboxp{
In the \structe{$O_6$ lattice} \xref{def:o6}, there are a total of $\bcoef{6}{2}=\frac{6!}{(6-2)!2!}=\frac{6\times5}{2}=15$ 
distinct unordered (the $\orthog$ relation is \prope{symmetric} by \prefp{lem:latn_orthog} so the order doesn't matter) pairs of elements.
\\\begin{tabular}{m{86mm}l}
  Of these 15 pairs, 8 are orthogonal to each other, and $\lzero$ is orthogonal to itself, making a total of 9 orthogonal pairs: &
  $\begin{array}{|lcl|lcl|lcl|}
    \hline
    x &\orthog& y          & x &\orthog& \lzero        & y^\ocop &\orthog& \lzero\\
    x &\orthog& x^\ocop    & y &\orthog& \lzero        & \lid&\orthog&\lzero\\
    y &\orthog& y^\ocop    & x^\ocop &\orthog& \lzero  & \lzero&\orthog&\lzero\\
    \hline
  \end{array}$
\end{tabular}
  }
\end{example}

%---------------------------------------
\begin{example}
%---------------------------------------
\exboxp{
In lattice 5 of \prefpp{ex:latoc}, 
there are a total of $\bcoef{10}{2}=\frac{10!}{(10-2)!2!}=\frac{10\times9}{2}=45$ 
distinct unordered pairs of elements.
\\\begin{tabular}{m{66mm}l}
  Of these 45 pairs, 18 are orthogonal to each other, and $\lzero$ is orthogonal to itself, making a total of 19 orthogonal pairs: &
  $\begin{array}{|lcl|lcl|lcl|lcl|}
    \hline
    p &\orthog& p^\ocop    & x &\orthog& x^\ocop    & y       &\orthog& z          & x^\ocop &\orthog& \lzero\\
    p &\orthog& x^\ocop    & x &\orthog& y          & y       &\orthog& \lzero     & y^\ocop &\orthog& \lzero\\
    p &\orthog& y          & x &\orthog& z          & z       &\orthog& z^\ocop    & z^\ocop &\orthog& \lzero\\
    p &\orthog& z          & x &\orthog& \lzero     & z       &\orthog& \lzero     & \lzero  &\orthog& \lzero\\
    p &\orthog& \lzero     & y &\orthog& y^\ocop    & p^\ocop &\orthog& \lzero     &         &       &       \\
    \hline
  \end{array}$
\end{tabular}
  }
\end{example}

%---------------------------------------
\begin{example}
%---------------------------------------
\exboxt{
In the \structb{$\R^3$ Euclidean space} illustrated in \prefpp{ex:latoc_r3},
\\\indentx$\begin{array}{lcl@{\qquad}lcl}
    \spX\subseteq\spY^\ocop &\implies& \spX\orthog\spY & \spY\subseteq\spX^\ocop &\implies& \spY\orthog\spX \\
    \spX\subseteq\spZ^\ocop &\implies& \spX\orthog\spZ & \spY\subseteq\spZ^\ocop &\implies& \spY\orthog\spZ \\
    \mc{6}{l}{\spX\meet\spY=\spX\meet\spZ=\spY\meet\spZ=\spZero}
  \end{array}$
}
\end{example}%



%=======================================
\section{Commutativity}
%=======================================
The \rele{commutes} relation is defined next.
Motivation for the name ``commutes" is provided by \prefpp{prop:latoc_pxypyx}
which shows that if $x$ commutes with $y$ in a lattice $\latL$, 
then $x$ and $y$ commute in the \ope{Sasaki projection} $\sasaki{x}{y}$ on $\latL$.
%---------------------------------------
\begin{definition}
\footnote{
  %\citorp{nakamura1957}{158},
  \citerpg{kalmbach1983}{20}{0123945801},
  \citerpc{holland1970}{79}{A. Commutativity},
  \citerpc{maeda1958}{227}{Hilfssatz (Lemma) XII.1.2},
  \citePpc{sasaki1954}{301}{Def.5.2, cf Foulis 1962}, % page 66
  \citePpc{birkhoff1936oct}{833}{``$a=(a\seti x)\setu(a\seti x')$"} % holland p.79
  }
\label{def:latoc_commutes}
\label{def:latn_commutes}
%---------------------------------------
%Let $\latL\eqd\latocd$ be an \structe{orthocomplemented lattice} \xref{def:latoc}.
Let $\latL\eqd\latnX$ be a \structe{lattice with negation} \xref{def:latn}.
\defboxt{
  %For any two elements $x,y\in\setX$, 
  The \reld{commutes} relation $\hxs{\commutes}$ is defined as 
    \\\indentx$x\commutes y \qquad\iffdef\qquad x = \brp{x \meet y} \join \brp{x \meet \negat{y}}\qquad{\scy\forall x,y\in\setX}$,\\
  in which case we say, ``$x$ \hid{commutes} with $y$ in $\latL$".\\
  That is, $\commutes$ is a relation in $\clRxx$ such that
  \\\indentx$
      \commutes \eqd \set{\opair{x}{y}\in\setX^2}
                         {x = \brp{x \meet y} \join \brp{x \meet \negat{y}}} 
    $
% \\A pair $\opair{x}{y}\in\commutes$  is alternatively denoted as either $\opair{x}{y}\commutes$ or $x\commutes y$.
  %and is called a \propd{modular} pair.
  }
\end{definition}
%That is, $x\commutes y \quad\iffdef\quad x = \brp{x \meet y} \join \brp{x \meet y^\ocop}$.

%---------------------------------------
\begin{proposition}
\footnote{
  \citerp{holland1963}{67}
  }
\label{prop:commutes}
%---------------------------------------
Let $\latL\eqd\latocd$ be an \structe{orthocomplemented lattice}. % \xref{def:latoc}.
\propbox{
  \begin{array}{lDlC@{\qquad}|@{\qquad}lclC}
    x \commutes \lzero & and & \lzero \commutes x & \forall x   \in\setX  & x \commutes y      &\iff&     x \commutes y^\ocop & \forall x,y \in\setX\\
    x \commutes \lid   & and & \lid \commutes x   & \forall x   \in\setX  & x \orel y          &\implies& x \commutes y       & \forall x,y \in\setX\\
    x \commutes x      &     &                    & \forall x   \in\setX  & x \orthog y        &\implies& x \commutes y       & \forall x,y \in\setX  
  \end{array}
  }
\end{proposition}
\begin{proof}
%\begin{enumerate}
%  \item Proof that $x \commutes \lzero$, $\lzero \commutes x$, $x \commutes \lid$, $\lid\commutes x$, and $x\commutes x$:
    \begin{align*}
      \brp{x\meet \lzero} \join \brp{x \meet \lzero^\ocop}
        &= \lzero \join \brp{x \meet \lzero^\ocop}
        && \text{by \prope{lower bound} property of \structe{bounded lattice}s \xref{def:latb}}
      \\&= \lzero \join \brp{x \meet \lid}
        && \text{by \prope{boundary condition} of \structe{ortho negation} \xref{thm:latn_ortho}}
      \\&= \lzero \join \brp{x}
        && \text{by \prope{upper bound} property of \structe{bounded lattice}s \xref{def:latb}}
      \\&= x
        && \text{by \prope{lower bound} property of \structe{bounded lattice}s \xref{def:latb}}
      \\\implies& x \commutes \lzero
        && \text{by definition of $\commutes$ relation \xref{def:latoc_commutes}}
  \\
  %  \end{align*}
  %
  %\item Proof that $\lzero \commutes x$:
  %  \begin{align*}
      \brp{\lzero \meet x} \join \brp{\lzero \meet x^\ocop}
        &= \lzero \join \brp{\lzero}
        && \text{by \prope{lower bound} property of \structe{bounded lattice}s \xref{def:latb}}
      \\&= \lzero
        && \text{by \prope{lower bound} property of \structe{bounded lattice}s \xref{def:latb}}
      \\\implies& \lzero \commutes x
        && \text{by definition of $\commutes$ relation \xref{def:latoc_commutes}}
  \\
  %  \end{align*}
  %
  %\item Proof that $x \commutes \lid$:
  %  \begin{align*}
      \brp{x\meet \lid} \join \brp{x \meet \lid^\ocop}
        &= x \join \brp{x \meet \lid^\ocop}
        && \text{by \prope{lower bound} property of \structe{bounded lattice}s \xref{def:latb}}
      \\&= x \join \brp{x \meet \lzero}
        && \text{by \prope{boundary condition} of \structe{ortho negation} \xref{thm:latn_ortho}}
      \\&= \brp{x} \join \brp{\lzero}
        && \text{by \prope{lower bound} property of \structe{bounded lattice}s \xref{def:latb}}
      \\&= x
        && \text{by \prope{lower bound} property of \structe{bounded lattice}s \xref{def:latb}}
      \\\implies& x \commutes \lid
        && \text{by definition of $\commutes$ relation \xref{def:latoc_commutes}}
  \\
  %  \end{align*}
  %
  %\item Proof that $\lid \commutes x$:
  %  \begin{align*}
      \brp{\lid \meet x} \join \brp{\lid \meet x^\ocop}
        &= \brp{x} \join \brp{x^\ocop}
        && \text{by \prope{non-contradiction} prop. of \fncte{ortho negation} \xref{def:negor}}
      \\&= \lid
        && \text{by \prope{excluded middle} property of \fncte{ortho negation} \xref{thm:latn_ortho}}
      \\\implies&\lid \commutes x
        && \text{by definition of $\commutes$ relation \xref{def:latoc_commutes}}
  \\
  %  \end{align*}
  %
  %\item Proof that $x \commutes x$:
  %  \begin{align*}
      \brp{x\meet x} \join \brp{x \meet x^\ocop}
        &= x \join \brp{x \meet x^\ocop}
        && \text{by \prope{idempotent} property of \structe{lattice}s \xref{thm:lattice}}
      \\&= x \join \brp{\lzero}
        && \text{by \prope{non-contradiction} prop. of \fncte{ortho negation} \xref{def:negor}}
      \\&= x
        && \text{by \prope{lower bound} property of \structe{bounded lattice}s \xref{def:latb}}
      \\\implies &x \commutes x
        && \text{by definition of $\commutes$ relation \xref{def:latoc_commutes}}
  \\
  %  \end{align*}
  %  
  %\item Proof that $x \commutes y \implies x \commutes y^\ocop$:
  %  \begin{align*}
      x\commutes y
        &\implies\brp{x\meet y^\ocop} \join \brp{x \meet y^{\ocop\ocop}}
        && \text{by definition of $\commutes$ \xref{def:latn_commutes}}
      \\&= \brp{x\meet y^\ocop} \join \brp{x \meet y}
        && \text{by \prope{involution} property of $\ocop$ \xref{def:latoc}}
      \\&= \brp{x\meet y} \join \brp{x \meet y^\ocop}
        && \text{by \prope{commutative} property of \structe{lattice}s \xref{def:lattice}}
      \\&= x
        && \text{by $x \commutes y$ hypothesis and \prefp{def:latoc_commutes}}
      \\\implies& x\commutes y^\ocop
        && \text{by definition of $\commutes$ relation \xref{def:latoc_commutes}}
  \\
 %   \end{align*}
 %
 % \item Proof that $x \commutes y \impliedby x \commutes y^\ocop$:
 %   \begin{align*}
      x\commutes y^\ocop
        &\implies\brp{x\meet y} \join \brp{x \meet y^{\ocop}}
        && \text{by definition of $\commutes$ \xref{def:latn_commutes}}
      \\&= \brp{x\meet y^{\ocop\ocop}} \join \brp{x \meet y^\ocop}
        && \text{by \prope{involution} property of $\ocop$ \xref{def:latoc}}
      \\&= \brp{x\meet y^\ocop} \join \brp{x \meet y^{\ocop\ocop}}
        && \text{by \prope{commutative} property of \structe{lattice}s \xref{def:lattice}}
      \\&= x
        && \text{by $x \commutes y^\ocop$ hypothesis and \prefp{def:latoc_commutes}}
      \\\implies& x\commutes y
        && \text{by definition of $\commutes$ relation \xref{def:latoc_commutes}}
  \\
  %  \end{align*}
  %
  %\item Proof that $x \orel y     \implies x \commutes y$:
  %  \begin{align*}
      x \orel y
        &\implies\brp{x \meet y} \join \brp{x \meet y^\ocop}
        && \text{by definition of $\commutes$ \xref{def:latn_commutes}}
      \\&=        x \join \brp{x \meet y^\ocop}
        &&        \text{by $x\orel y$ hypothesis}
      \\&=        x
        &&        \text{by \prope{absorptive} property \ifxref{lattice}{thm:lattice}}
      \\&\implies x\commutes y
        &&        \text{by definition of $\commutes$ \xref{def:latoc_commutes}}
  \\
 %   \end{align*}
 %
 % \item Proof that $x \orthog y   \implies x \commutes y$:
 %   \begin{align*}
      x \orthog y
        &\implies\brp{x \meet y} \join \brp{x \meet y^\ocop}
        && \text{by definition of $\commutes$ \xref{def:latn_commutes}}
      \\&= \lzero \join \brp{x \meet y^\ocop}
        && \text{by \prefp{lem:latoc_orthog}}
      \\&= \lzero \join x
        && \text{by $x\orthog y$ hypothesis ($x\orthog y\implies x\orel y^\ocop$)} % \xref{def:latoc_orthog}
      \\&= x \join \lzero
        && \text{by \prope{commutative} property \ifxref{lattice}{thm:lattice}}
      \\&= x
        && \text{by \prope{identity} property of \structe{bounded lattice}s\ifsxref{latvar}{prop:latb_prop}}
      \\&\implies x\commutes y
        &&        \text{by definition of $\commutes$ \xref{def:latoc_commutes}}
    \end{align*}
%  \end{enumerate}
\end{proof}

%%=======================================
%\subsection{Symmetry}
%%=======================================
%---------------------------------------
\begin{definition}
%---------------------------------------
%Let $\latL\eqd\latocd$ be an \structe{orthocomplemented lattice} \xref{def:latoc}.
Let $\commutes$ be the \rele{commutes} relation \xref{def:latn_commutes} on a 
\structe{lattice with negation} $\latL\eqd\latnX$ \xref{def:latn}.
\defboxt{
  $\latL$ is \propd{symmetric} if
  \\\indentx$x \commutes y\quad\implies\quad y \commutes x \qquad{\scy\forall x,y\in\setX}$
  }
\end{definition}

In general, the commutes relation is not \prope{symmetric}.
But \pref{prop:latoc_symmetry} (next) describes some conditions under which it 
\emph{is} symmetric.
%---------------------------------------
\begin{proposition}
\footnote{
  \citorp{holland1963}{68},
  \citePp{nakamura1957}{158}
  }
\label{prop:latoc_symmetry}
%---------------------------------------
Let $\latocd$ be an \structe{orthocomplemented lattice} \xref{def:latoc}.
\propbox{
  \mcom{\brb{x\commutes y \implies y\commutes x}}{$\commutes$ is \prope{symmetric} at $\opair{x}{y}$ (1)}
  \begin{array}[t]{@{\quad}l@{\quad}lDD}
    \iff & \brb{x \orel y \implies y = x \join \brp{x^\ocop \meet y}}      & (\prope{orthomodular identity})                & (2)\\
    \iff & \brb{x \orel y \implies x = y \meet \brp{x \join y^\ocop}}      & ($x=\sasaki{y}{x}$ (\ope{Sasaki projection}) ) & (3)\\
    \iff & \brb{y = \brp{x\meet y} \join \brs{y\meet\brp{x\meet y}^\ocop}} &                                                & (4)\\
    \iff & \brb{x = \brp{x\join y} \meet \brs{x\join\brp{x\join y}^\ocop}} &                                                & (5)
  \end{array}
  }
\end{proposition}
\begin{proof}
%Let $+$ represent $\join$ and juxtaposition represent $\meet$, with 
%$\meet$ taking algebraic precedence over $\join$.
\begin{enumerate}
  \item Proof that (2) $\iff$ (3):
      %$\brb{x \orel y \implies y = x \join \brp{x^\ocop \meet y}}$
      %$\iff$
      %$\brb{x \orel y \implies x = y \meet \brp{x \join y^\ocop}}$:
    \begin{align*}
      x\orel y
        &\implies y^\ocop \orel x^\ocop
        && \text{by \prope{antitone} property \xref{def:latoc}}
      \\&\implies x^\ocop = y^\ocop \join \brp{y^{\ocop\ocop} \meet x^\ocop}
        && \text{by left hypothesis}
      \\&\implies \brp{x^\ocop}^\ocop = \brs{y^\ocop \join \brp{y^{\ocop\ocop} \meet x^\ocop}}^\ocop
      \\&\implies x = \brs{y^\ocop \join \brp{y^{\ocop\ocop} \meet x^\ocop}}^\ocop
        && \text{by \prope{involutory} property \xref{def:latoc}}
      \\&\qquad= y^{\ocop\ocop} \meet \brp{y^{\ocop\ocop} \meet x^\ocop}^\ocop
        && \text{by \prope{de Morgan} property \xref{thm:latoc_prop}}
      \\&\qquad= y \meet \brp{y \meet x^\ocop}^\ocop
        && \text{by \prope{involutory} property \xref{def:latoc}}
      \\&\qquad= y \meet \brp{y^\ocop \join x^{\ocop\ocop}}
        && \text{by \prope{de Morgan} property \xref{thm:latoc_prop}}
      \\&\qquad= y \meet \brp{y^\ocop \join x}
        && \text{by \prope{involutory} property \xref{def:latoc}}
      \\&\qquad= y \meet \brp{x \join y^\ocop}
        && \text{by \prope{commutative} property \ifxref{lattice}{thm:lattice}}
      \\
      \\
      x\orel y
        &\implies y^\ocop \orel x^\ocop
        && \text{by \prope{antitone} property \xref{def:latoc}}
      \\&\implies y^\ocop = x^\ocop \meet \brp{y^\ocop \join x^{\ocop\ocop}}
        && \text{by right hypothesis}
      \\&\implies \brp{y^\ocop}^\ocop = \brs{x^\ocop \meet \brp{y^\ocop \join x^{\ocop\ocop}}}^\ocop
      \\&\implies y = \brs{x^\ocop \meet \brp{y^\ocop \join x^{\ocop\ocop}}}^\ocop
        && \text{by \prope{involutory} property \xref{def:latoc}}
      \\&\qquad= x^{\ocop\ocop} \join \brp{y^\ocop\join x^{\ocop\ocop}}^\ocop
        && \text{by \prope{de Morgan} property \xref{thm:latoc_prop}}
      \\&\qquad= x \join \brp{y^\ocop\join x}^\ocop
        && \text{by \prope{involutory} property \xref{def:latoc}}
      \\&\qquad= x \join \brp{y^{\ocop\ocop}\meet x^\ocop}
        && \text{by \prope{de Morgan} property \xref{thm:latoc_prop}}
      \\&\qquad= x \join \brp{y \meet x^\ocop}
        && \text{by \prope{involutory} property \xref{def:latoc}}
      \\&\qquad= x \join \brp{x^\ocop\meet y}
        && \text{by \prope{commutative} property \ifxref{lattice}{thm:lattice}}
    \end{align*}

  \item Proof that (2) $\iff$ (4):
        %$ \brb{x \orel y \implies y = x \join \brp{x^\ocop \meet y}}
        %  \iff
        %  \brb{y = \brp{x\meet y} \join \brs{y\meet\brp{x\meet y}^\ocop}}
        %$:\label{item:latoc_symmetry_iffeq}
    \begin{align*}
      \brp{xy} \join  \brs{y\brp{xy}^\ocop}
        &= u \join  \brs{yu^\ocop}
        && \text{where $u\eqd xy \orel y$}
      \\&= u \join  \brs{u^\ocop y}
        && \text{by \prope{commutative} property of lattices \ifxref{lattice}{thm:lattice}}
      \\&= y
        && \text{by left hypothesis}
      \\
      \\
      x \orel y \implies x \join  \brp{x^\ocop y} 
        &= xy \join  \brs{(xy)^\ocop y}
        && \text{by $x\orel y$ hypothesis}
      \\&= xy \join  \brs{y(xy)^\ocop}
        && \text{by \prope{commutative} property of lattices \ifxref{lattice}{thm:lattice}}
      \\&= y
        && \text{by right hypothesis}
    \end{align*}

  \item Proof that (3) $\iff$ (5):
        %$ \brb{x \orel y \implies x = y \meet \brp{x \join y^\ocop}}
        %  \iff
        %  \brb{x = \brp{x\join y} \meet \brs{x\join\brp{x\join y}^\ocop}}
        \label{item:latoc_symmetry_iffeqd}
    \begin{align*}
      \brp{x\join y}\brs{x\join \brp{x\join y}^\ocop}
        &= u\brs{x\join u^\ocop}
        && \text{where $x\orel u\eqd x\join y$}
      \\&= x
        && \text{by left hypothesis}
      \\
      \\
      x \orel y \implies y\brp{x \join  y^\ocop} 
        &= \brp{x\join y} \brs{x \join  \brp{x\join y}^\ocop}
        && \text{by $x\orel y$ hypothesis}
      \\&= x
        && \text{by right hypothesis}
    \end{align*}

  \item Proof that (1) $\implies$ (2): %\prope{symmetric} property $\implies \brb{x \orel y \implies y = x \join \brp{x^\ocop \meet y}}$:
        \label{item:latoc_symmetry_iffineq}
    \begin{align*}
      x \orel y
        &\implies x\commutes y
        &&        \text{by \prefp{prop:commutes}}
      \\&\implies y\commutes x
        &&        \text{by \prope{symmetry} hypothesis (left hypothesis)}
      \\&\implies y = \brp{y \meet x} \join \brp{y \meet x^\ocop}
        &&        \text{by definition of $\commutes$ \xref{def:latoc_commutes}}
      \\&\implies y = x \join \brp{y \meet x^\ocop}
        &&        \text{by $x\orel y$ hypothesis}
      \\&\implies y = x \join \brp{x^\ocop \meet y}
        &&        \text{by \prope{commutative} property of lattices \ifxref{lattice}{thm:lattice}}
    \end{align*}

  \item Proof that (2) $\implies$ (4): 
        %$\brb{y = \brp{x\meet y} \join \brs{y\meet\brp{x\meet y}^\ocop}}$ $\implies$ \prope{symmetric} property:
    \begin{enumerate}
      \item lemma: proof that $x \commutes y \implies x^\ocop y = \brp{xy}^\ocop y$: \label{item:latoc_symmetry_xyxyy}
        \begin{align*}
          x \commutes y \implies
          x^\ocop y
            &= \brp{xy \join  xy^\ocop}^\ocop y
            && \text{by definition of $\commutes$ \xref{def:latoc_commutes}}
          \\&= \brp{xy}^\ocop \brp{xy^\ocop}^\ocop y
            && \text{by \prope{de Morgan}'s law \xref{thm:latn_demorgan}}
          \\&= \brp{xy}^\ocop \brs{\brp{x^\ocop\join y^{\ocop^\ocop}} y}
            && \text{by \prope{de Morgan}'s law \xref{thm:latn_demorgan}}
          \\&= \brp{xy}^\ocop \brs{\brp{x^\ocop\join y} y}
            &&        \text{by \prope{involutory}'s property \xref{def:latoc}}
          \\&= \brp{xy}^\ocop y
            &&        \text{by \prope{absorptive} property of lattices \ifxref{lattice}{thm:lattice}}
        \end{align*}

      \item Completion of proof for (2) $\implies$ (4): %Proof that $x\commutes y\implies y\commutes x$:
        \begin{align*}
          x\commutes y\implies
          xy \join  y\brp{xy}^\ocop
            &= xy \join  \brp{xy}^\ocop y
            && \text{by \prope{commutative} property \ifxref{lattice}{thm:lattice}}
          \\&= xy \join  x^\ocop y
            && \text{by $x\commutes y$ hypothesis and \pref{item:latoc_symmetry_xyxyy}}
          \\&= (yx) \join  \brs{y x^\ocop }
            && \text{by \prope{commutative} property \ifxref{lattice}{thm:lattice}}
          \\&\implies y \commutes x
            && \text{by definition of $\commutes$ \xref{def:latoc_commutes}}
      \end{align*}
  \end{enumerate}
\end{enumerate}
\end{proof}


%---------------------------------------
\begin{theorem}
\footnote{
  \citerpg{kalmbach1983}{20}{0123945801},
  \citeP{maclaren1964}
  }
%---------------------------------------
Let $\latL\eqd\latocd$ be an \structe{orthocomplemented lattice} \xref{def:latoc}.
\thmbox{\begin{array}{M}
  $\brb{x\commutes c \quad \forall x\in\setX}$
  $\qquad\iff\qquad$
  $\brb{\text{$\latL$ is \emph{isomorphic} to $\intcc{\lzero}{c}\times\intcc{\lzero}{c^\ocop}$}}$
  \\\indentx with isomorphism 
    $\ftheta(x) \eqd \opair{\intcc{\lzero}{c}}{\intcc{\lzero}{c^\ocop}}$.
\end{array}}
\end{theorem}


%---------------------------------------
\begin{proposition}
\footnote{
  \citePp{foulis1962}{66},
  \citePc{sasaki1954}{cf Foulis 1962}
  }
\label{prop:latoc_pxypyx}
%---------------------------------------
Let $\latocX$ be an \prope{orthomodular} lattice.
\propbox{
  x\commutes y 
  \qquad\iff\qquad
  \sasaki{x}{y} = \sasaki{y}{x} = x \meet y
  \qquad\scy\forall x,y\in\setX
  }
\end{proposition}
%\begin{proof}
%\begin{enumerate}
%  \item Proof that $x\commutes y \implies   \sasaki{x}{y} = \sasaki{y}{x} = x \meet y$:
%  \item Proof that $x\commutes y \impliedby \sasaki{x}{y} = \sasaki{y}{x} = x \meet y$:
%\end{enumerate}
%\end{proof}

%=======================================
\section{Center}
%=======================================
An element in an \structe{orthocomplemented lattice} \xref{def:latoc} is in the \structe{center} of the lattice
if that element \rele{commutes} \xref{def:latoc_commutes} with every other element in the lattice (next definition).
\emph{All} the elements of an \structe{orthocomplemented lattice} are in the \structe{center} 
if and only if that lattice is \prope{Boolean} \xref{prop:latoc_center_boolean}.
%---------------------------------------
\begin{definition}
\footnote{
  \citerp{holland1970}{80}
  }
\label{def:center}
%---------------------------------------
%Let $\latL\eqd\latocd$ be an \structe{orthocomplemented lattice} \xref{def:latoc}.
Let $\commutes$ be the \rele{commutes} relation \xref{def:latn_commutes} on a 
\structe{lattice with negation} $\latL\eqd\latnX$ \xref{def:latn}.
\defboxt{
  The \structd{center} of $\latL$ is defined as
  \\\indentx$\ds\set{x\in\setX}{x\commutes y\quad\forall y\in\setX}$
  %An element $x\in\setX$ is in the \structd{center} of $\latL$ if
  %\\\indentx$x\commutes y$ for every $y\in\setX$
  }
\end{definition}

%---------------------------------------
\begin{proposition}
\label{prop:center}
%---------------------------------------
Let $\latL\eqd\latocd$ be an \structe{orthocomplemented lattice} \xref{def:latoc}.
\propboxt{
  $\lzero$ and $\lid$ are in the \structb{center} of $\latL$.
  }
\end{proposition}
\begin{proof}
This follows directly from \prefpp{def:latn_commutes} and \prefpp{prop:commutes}.
\end{proof}

%---------------------------------------
\begin{theorem}
\footnote{
  \citePpc{jeffcott1972}{645}{\textsection5. Main theorem}
  }
\label{thm:center_boolean}
%---------------------------------------
Let $\latL\eqd\latocd$ be an \structe{orthocomplemented lattice} \xref{def:latoc}.
\thmboxt{
  The \structe{center} of $\latL$ is \prope{boolean}\ifsxref{boolean}{def:boolean}.
  }
\end{theorem}

%---------------------------------------
\begin{example}
%---------------------------------------
\exbox{\begin{tabular}{m{\tw-53mm}}
    The \structb{center} of the \structb{$O_6$ lattice} \xref{def:o6} is the set $\setn{0,\,x,\,z,\,1}$.
    The elements $x^\ocop$ and $z^\ocop$ are \textbf{not} in the center of $\latL$.
    The $O_6$ lattice is illustrated to the right, with the center elements as solid dots.
    Note that the center is the \prope{Boolean} lattice $\latL_2^2$ \xref{prop:latoc_center_boolean}.
  \end{tabular}\qquad\tbox{\includegraphics{../common/math/graphics/pdfs/lat6_o6_center.pdf}}}
\end{example}
\begin{proof}
\begin{enumerate}
  \item Proof that $\lzero$ and $\lid$ are in the \structe{center} of $\latL$: by \prefpp{prop:center}.

  \item Proof that $x$ is in the \structe{center} of $\latL$: 
    \begin{align*}
      (x \meet x)       \join (x \meet x^\ocop)      &= x      \join \lzero &&= x  &&\implies x \commutes x       \\
      (x \meet z)       \join (x \meet z^\ocop)      &= \lzero \join x      &&= x  &&\implies x \commutes z       \\
    \end{align*}
    $x\commutes x$, $x\commutes x^\ocop$, $x \commutes z^\ocop$, $x\commutes\lzero$, and $x\commutes\lid$ by \prefpp{prop:commutes}.

  \item Proof that $z$ is in the \structe{center} of $\latL$: 
    \begin{align*}
      (z \meet z)       \join (z \meet z^\ocop)      &= z\join \lzero       &&= z &&\implies z \commutes z       \\
      (z \meet x)       \join (z \meet x^\ocop)      &= \lzero \join z      &&= z &&\implies z \commutes x       \\
    \end{align*}
    $z\commutes z$, $z\commutes x^\ocop$, $z \commutes z^\ocop$, $z\commutes\lzero$, and $z\commutes\lid$ by \prefpp{prop:commutes}.

  \item Proof that $x^\ocop$ and $z^\ocop$ are \emph{not} in the \structe{center} of $\latL$:
    \begin{align*}
      (x^\ocop \meet y) \join (x^\ocop \meet y^\ocop)  &= y \join \lzero &&= y &&\implies x^\ocop \notcommutes y   \\
      (z^\ocop \meet x) \join (z^\ocop \meet x^\ocop)  &= x \join \lzero &&= x &&\implies z^\ocop \notcommutes x   
    \end{align*}

\end{enumerate}
\end{proof}


%---------------------------------------
\begin{example}
%---------------------------------------
\exbox{\begin{tabular}{m{\tw-55mm}}
    The \structb{center} the lattice illustrated to the right \xref{ex:latoc}, 
    with center elements as solid dots, is the set $\setn{\lzero,\lid,p,y,z,x^\ocop,y^\ocop,z^\ocop,}$.
    The elements $x$ and $p^\ocop$ are \emph{not} in the \structe{center} of $\latL$.
    Note that the center is the \prope{Boolean} lattice $\latL_2^3$ \xref{prop:latoc_center_boolean}.
  \end{tabular}\quad\tbox{\includegraphics{../common/math/graphics/pdfs/lat10_n5n5oc_xyzp_center.pdf}}}%
\end{example}
\begin{proof}
\begin{enumerate}
  \item Proof that $\lzero$ and $\lid$ are in the \structe{center} of $\latL$: by \prefpp{prop:center}.

  \item Proof that $x$ is in the \structe{center} of $\latL$:
    \begin{align*}
      (x \meet p)       \join (x \meet p^\ocop)      &= x      \join \lzero &&= x  &&\implies x \commutes p       \\
      (x \meet y)       \join (x \meet y^\ocop)      &= \lzero \join x      &&= x  &&\implies x \commutes y       \\
      (x \meet z)       \join (x \meet z^\ocop)      &= \lzero \join x      &&= x  &&\implies x \commutes z       
    \end{align*}
    $x\commutes x$, $x\commutes x^\ocop$, $x\commutes p^\ocop$, $x\commutes y^\ocop$, $x \commutes z^\ocop$, $x\commutes\lzero$, and $x\commutes\lid$ by \prefpp{prop:commutes}.

  \item Proof that $y$ is in the \structe{center} of $\latL$:
    \begin{align*}
      (y \meet x)       \join (y \meet x^\ocop)      &= \lzero \join y      &&= y  &&\implies y \commutes x       \\
      (y \meet p)       \join (y \meet p^\ocop)      &= \lzero \join y      &&= y  &&\implies y \commutes p       \\
      (y \meet z)       \join (y \meet z^\ocop)      &= \lzero \join y      &&= y  &&\implies y \commutes z       
    \end{align*}
    $y\commutes y$, $y\commutes x^\ocop$, $y\commutes p^\ocop$, $y\commutes y^\ocop$, $y \commutes z^\ocop$, $y\commutes\lzero$, and $y\commutes\lid$ by \prefpp{prop:commutes}.

  \item Proof that $z$ is in the \structe{center} of $\latL$:
    \begin{align*}
      (z \meet x)       \join (z \meet x^\ocop)      &= \lzero \join z      &&= z  &&\implies z \commutes x       \\
      (z \meet p)       \join (z \meet p^\ocop)      &= \lzero \join z      &&= z  &&\implies z \commutes p       \\
      (z \meet y)       \join (z \meet y^\ocop)      &= \lzero \join z      &&= z  &&\implies z \commutes y       
    \end{align*}
    $z\commutes z$, $z\commutes x^\ocop$, $z\commutes p^\ocop$, $z\commutes y^\ocop$, $z \commutes z^\ocop$, $z\commutes\lzero$, and $z\commutes\lid$ by \prefpp{prop:commutes}.

  \item Proof that $x^\ocop$ is in the \structe{center} of $\latL$:
    \begin{align*}
      (p^\ocop \meet x)   \join (p^\ocop \meet x^\ocop)      &= \lzero \join p^\ocop      &&= p^\ocop  &&\implies p^\ocop \commutes x       \\
      (p^\ocop \meet y)   \join (p^\ocop \meet y^\ocop)      &= y \join z                 &&= p^\ocop  &&\implies p^\ocop \commutes y       \\
      (p^\ocop \meet z)   \join (p^\ocop \meet z^\ocop)      &= z \join y                 &&= p^\ocop  &&\implies p^\ocop \commutes z       \\
    \end{align*}
    $p^\ocop\commutes x^\ocop$, $p^\ocop\commutes p^\ocop$, $p^\ocop\commutes y^\ocop$, $p^\ocop\commutes z^\ocop$, $p^\ocop \commutes \lzero$, and $p^\ocop\commutes\lid$ by \prefpp{prop:commutes}.

  \item Proof that $y^\ocop$ is in the \structe{center} of $\latL$:
    \begin{align*}
      (y^\ocop \meet x)   \join (y^\ocop \meet x^\ocop)      &= x \join z  &&= y^\ocop  &&\implies y^\ocop \commutes x       \\
      (y^\ocop \meet p)   \join (y^\ocop \meet p^\ocop)      &= p \join z  &&= y^\ocop  &&\implies y^\ocop \notcommutes p       \\
      (y^\ocop \meet z)   \join (y^\ocop \meet z^\ocop)      &= z \join p  &&= y^\ocop  &&\implies y^\ocop \commutes z       \\
    \end{align*}
    $p^\ocop\commutes x^\ocop$, $p^\ocop\commutes p^\ocop$, $p^\ocop\commutes y^\ocop$, $p^\ocop\commutes z^\ocop$, $p^\ocop \commutes \lzero$, and $p^\ocop\commutes\lid$ by \prefpp{prop:commutes}.

  \item Proof that $z^\ocop$ is in the \structe{center} of $\latL$:
    \begin{align*}
      (z^\ocop \meet x)   \join (z^\ocop \meet x^\ocop)      &= x \join y  &&= z^\ocop  &&\implies z^\ocop \commutes x       \\
      (z^\ocop \meet p)   \join (z^\ocop \meet p^\ocop)      &= p \join y  &&= z^\ocop  &&\implies y^\ocop \notcommutes p       \\
      (z^\ocop \meet y)   \join (z^\ocop \meet y^\ocop)      &= z \join p  &&= z^\ocop  &&\implies y^\ocop \commutes z       \\
    \end{align*}
    $z^\ocop\commutes x^\ocop$, $z^\ocop\commutes p^\ocop$, $z^\ocop\commutes y^\ocop$, $z^\ocop\commutes z^\ocop$, $z^\ocop \commutes \lzero$, and $z^\ocop\commutes\lid$ by \prefpp{prop:commutes}.

  \item Proof that $p$ and $x^\ocop$ are \emph{not} in the \structe{center} of $\latL$:
    \begin{align*}
      (p \meet x)       \join (p \meet x^\ocop)        &= x      \join \lzero   &&= x        &&\implies p \notcommutes x         \\
      (x^\ocop \meet p) \join (x^\ocop \meet p^\ocop)  &= \lzero \join p^\ocop  &&= p^\ocop  &&\implies x^\ocop \notcommutes p       \\
    \end{align*}
\end{enumerate}
\end{proof}


%---------------------------------------
\begin{example}
%---------------------------------------
\exbox{\begin{tabular}{m{\tw-55mm}}
    The \structb{center} of the lattice illustrated to the right is illustrated with solid dots.
    Note that the center is the \prope{Boolean} lattice $\latL_2^2$ \xref{prop:latoc_center_boolean}.
  \end{tabular}\tbox{\includegraphics{../common/math/graphics/pdfs/lat10_o8m2_center_xypqoc.pdf}}}
\end{example}

%---------------------------------------
\begin{example}
%---------------------------------------
\exbox{\begin{tabular}{m{\tw-55mm}}
    In a \prope{Boolean} lattice, such as the one illustrated to the right,
    every element is in the center \xref{prop:latoc_center_boolean}.
  \end{tabular}\tbox{\includegraphics{../common/math/graphics/pdfs/latoc2xyz_center.pdf}}}
\end{example}

