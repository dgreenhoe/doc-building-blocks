%============================================================================
% Daniel J. Greenhoe
% XeLaTeX file
%============================================================================
{\sffamily
This text was typeset using \hi{\XeLaTeX},
which is part of the \TeX family of typesetting engines, 
which is arguably the greatest development since the \hi{Gutenberg Press}.
Graphics were rendered using the \hie{pstricks} and related packages, and \hi{\LaTeX} graphics support.
%and the \hie{pgfplots} package (for rendering the Pollen 3-d wavelet appearing on the cover and in \prefp{fig:pollen4}).
%Data for some figures was generated using \hie{GNU Octave}---a free alternative to {\footnotesize MATLAB}.

%\begin{sloppypar} 
The main {\rmfamily roman}, {\itshape italic}, and {\bfseries bold} font typefaces used 
are all from the \hie{Heuristica} family of typefaces
(based on the \hie{Utopia} typeface, released by \hie{Adobe Systems Incorporated}).
The math font is {{\fntXits XITS}} from the {{\fntXits XITS font project}}.
The font used in quotation boxes is adapted from {\fntZapf Zapf Chancery Medium Italic},
originally from URW++ Design and Development Incorporated.
The font used for the text in the title is {\fntAdventor Adventor} (similar to \hie{Avant-Garde}) from the \hie{\TeX-Gyre Project}.
The font used for the ISBN in the footer of individual pages is 
\mbox{\fntDigital\footnotesize Liquid} \mbox{\fntDigital\footnotesize Crystal} 
%{\fntDigital Liq}-{\fntDigital uid Crystal} 
(\hie{Liquid Crystal}) from \hie{FontLab Studio}.
The Latin handwriting font is {\fntLavi Lavi} from the \hie{Free Software Foundation}.
%The traditional Chinese font used for the author's Chinese name 
%is {{\fntzhthw ???????}}.\footnote{\sffamily%
%pinyin: \hi{W{/'a}ng H{/`an} Z{/=o}ng Zh/=ong F{/va}ng S{/`o}ng F{/'an}}; 
%translation: H/`an Z/=ong W/'ang's Medium-weight S/`ong-style Traditional Characters;
%literal:
%{\fntzhthw ???}$\sim$font designer's name;
%{\fntzhthw ?}$\sim$medium;
%{\fntzhthw ?}$\sim$to imitate;
%{\fntzhthw ?}$\sim$S/`ong (a dynasty);
%{\fntzhthw ?}$\sim$traditional
%}
%Chinese glyphs appearing elsewhere in the text are from the font {{\fntzht ???????}}.\footnote{\sffamily%
%pinyin: \hi{W{/'a}ng H{/`an} Z{/-o}ng Zh/-ong Mi{/'n}g T{/>i} F{/'an}}; 
%translation: H/`an Z/=ong W/'ang's Medium-weight Mi{/'n}g-style Traditional Characters;
%literal:
%{\fntzhthw ???}$\sim$font designer's name;
%{\fntzhthw ?}$\sim$medium;
%{\fntzhthw ?}$\sim$Mi{/'n}g (a dynasty);
%{\fntzhthw ?}$\sim$style;
%{\fntzhthw ?}$\sim$traditional
%}
%%\end{sloppypar}

\textimgr[11mm]{../common/graphics/watercraft/ghind_blue.pdf}{%
  The ship 
  %$\ds\brp{\text{\raisebox{-2mm}{\includegraphics*[height=8mm]{../common/graphics/watercraft/ghindgray_djg.eps}}}}$
  appearing throughout this text is loosely based on the \hie{Golden Hind}, a sixteenth century English galleon 
  famous for circumnavigating the globe.\footnotemark
  }
\footnotetext{\citerpgc{paine2000}{63}{0395984149}{Golden Hind}}
}
