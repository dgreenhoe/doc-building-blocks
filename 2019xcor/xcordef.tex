%============================================================================
% LaTeX File
% Daniel J. Greenhoe
%============================================================================
%======================================
\section{Definitions}
%======================================
Here is a very limited overview of the definitions of $\Rxy(m)$:
%\\\defbox{\begin{array}{lM>{\ds}rc>{\ds}l}
\begin{longtable}{ll>{$\ds}r<{$}>{$\ds}c<{$}>{$\ds}l<{$}}
  \mc{5}{l}{References that put conjugate $\conj$ on $\rvy$:}
  \\&\citerpg{papoulis1984}{263}{0070484686} & R_{xy}(m) &\eqd& E\brb{\rvx(m)\rvy^\ast(0)}
  \\&\citerpg{cadzow}{341}{0023180102}       & r_{xy}(m) &\eqd& E\brs{\rvx(m)\rvy^\ast(0)}
  \\&\citeP{matlab_cpsd},\citeP{matlab_xcorr}& R_{xy}(m) &\eqd& E\setn{x_{n+m}y_n^\ast}
 %\\&                     & R_{xy}(m) &\eqd& E\setn{x_{n+m}y_n^\ast}
  \\
  \mc{5}{l}{References that put conjugate $\conj$ on $\rvx$:}
  \\&\citerpg{kay1988}{52}{8131733564}                    & r_{xy}[m] &\eqd& \mathcal{E}\brb{x^\ast[0]y[m]}
  \\&\citerpg{weisstein2002}{594}{1420035223}\footnotemark& f\star g  &\eqd& \int_{-\infty}^{\infty}\bar{f}(\tau)g(t+\tau)\dtau
  \\&\citePpc{leuridan1986}{2}{(7)}                       & GXY_1     &\eqd& \sum X_1^\ast Y
  \\
  \mc{5}{l}{References that use no conjugate $\conj$:}
  \\&\citerpg{bendat2010}{111}{1118210824}          & R_{xy}(m)            &\eqd& E\brs{\rvx(0)\rvy(m)}
  \\&\citerpg{helstrom1991}{369}{0023535717}        & \Rxy(t_1,t_2)        &\eqd& E[\rvx(t_1)\rvy(t_2)]
  \\&\citerpg{proakis1996}{A4}{0133737624}          & \gamma_{xy}(t_1,t_2) &\eqd& E(X_{t_1}Y_{t_2})
  \\&\citerpg{shin2008}{280}{0470725648}            & R_{xy}(\tau)         &\eqd& E[x(t)y(t+\tau)]
  \\&\citerpg{bracewell1978}{46}{007007013X}\footnotemark & g^\ast\star h  &\eqd& \int_{-\infty}^{\infty}g^\ast(u)h(u+x)\du   %{Pentagram notation for cross correlation}
\end{longtable}
%\end{array}}
\addtocounter{footnote}{-1}
\footnotetext{
  Bracewell and Weisstein here use the \ope{integral operator} $\int_{\R}\!\dx$ rather than the
  \ope{expectation operator} $\pE$.
  That is, they use a \ope{time average} rather than an \ope{ensemble average}.
  But in essence, the two types of operators are ``the same" because both types represent
  \ope{inner product}s.
  That is, $\int_{x\in\R}\ff(x)\fg^\ast(x)\dx\eqd\inprod{\ff(x)}{\fg(x)}_1$ and
  $\pE\brs{\rvx(t)\rvy^\ast(t)}\eqd\inprod{\rvx(t)}{\rvy(t)}_2$
  (both are inner products, but operate in perpendicular orientations across the ensemble plane).
  }
\stepcounter{footnote}
\footnotetext{
  Note that Bracewell's ``\ope{Pentagram notation for cross correlation}"
  $g^\ast\star h =\int_{-\infty}^{\infty}g^\ast(u)h(u+x)\du$
  implies
  $g\star h =\int_{-\infty}^{\infty}g(u)h(u+x)\du$
  (and hence in the ``References that use no conjugate" category).
  }
%\end{definition}

In terms of the expectation operator $\pE$,
there are a total of eight different ways of defining the cross-correlation $\Rxy(m)$ of
\prope{complex-valued} \prope{wide-sense stationary} sequences $\rvx(n)$ and $\rvy(n)$.
There are eight because each sequence may be defined with or without the conjugate operator $\ast$,
and $\rvx$ may lead or lag $\rvy$
($2\times2\times2=8$).
Here is a formalized list of the eight definitions.

%---------------------------------------
\begin{definition}
\label{def:Rxym}
%---------------------------------------
\mbox{}\\$\begin{array}{|*{2}{FM   >{\ds}r  c     l *{4}{@{\hspace{0pt}}l}  |}}
  \hline
    (1).&\fnctd{Papoulis}:        & \Rxy(m) &\eqd& \pE[\rvx     &(m)&\rvy^\ast&(0&)]  &(5).&\fnctd{Bendat-Piersol}:\footnotemark  & \Rxy(m) &\eqd& \pE[\rvx     &(0)&\rvy     &(m&)]
  \\(2).&\fnctd{Kay}:             & \Rxy(m) &\eqd& \pE[\rvx^\ast&(0)&\rvy     &(m&)]  &(6).&\fnctd{alt-BP}:          & \Rxy(m) &\eqd& \pE[\rvx     &(m)&\rvy     &(0&)]
  \\(3).&\fnctd{y-star-m}:        & \Rxy(m) &\eqd& \pE[\rvx     &(0)&\rvy^\ast&(m&)]  &(7).&\fnctd{BP-star}:         & \Rxy(m) &\eqd& \pE[\rvx^\ast&(0)&\rvy^\ast&(m&)]
  \\(4).&\fnctd{x-star-m}:        & \Rxy(m) &\eqd& \pE[\rvx^\ast&(m)&\rvy     &(0&)]  &(8).&\fnctd{alt-BP-star}:     & \Rxy(m) &\eqd& \pE[\rvx^\ast&(m)&\rvy^\ast&(0&)]
  \\\hline
\end{array}$
\end{definition}
\footnotetext{
  Note that Bendat and Piersol are well known and highly cited for their work related to 
  random vibration testing. 
  In this field, data samples are customarily collected using an analog-to-digital converter (ADC)
  and as such, for this application, are customarily \prope{real-valued}.
  Therefore, it is very understandable that these authors would define $\Rxy(m)$ 
  \emph{without} any conjugate operator.
  }

%=======================================
\section{Results}
%=======================================
The 8 different definitions of $\Rxy(m)$ listed in \pref{def:Rxym} yield \ldots
%Depending on which one of the eight definitions of $\Rxy(m)$ is selected, we get \ldots
\\\indentx$\begin{array}{FMlM}
    \imark & \mc{2}{M}{only 4 cases in which $\Swxx(\omega)$ is guarenteed to be \prope{real-valued}}  & \xref{prop:Swxx_real}
  \\\imark & 2 different relations on the pair   & \opair{\Ryx(m)}{\Rxy(m)}                            & \xref{prop:Rxy}
  \\\imark & 2 different relations on the pair   & \opair{\Rxx(-m)}{\Rxx(m)}                           & \xref{prop:Rxy}
  \\\imark & 4 different relations on the triple & \otriple{\Szxy(z)     }{\ZH(z)     }{\Szxx(z)     } & \xref{prop:RxySzxy}
  \\\imark & 4 different relations on the triple & \otriple{\Szxy(z)     }{\ZH(z)     }{\Szyy(z)     } & \xref{prop:RxySzxy}
  \\\imark & 4 different relations on the triple & \otriple{\Swxy(\omega)}{\FH(\omega)}{\Swxx(\omega)} & \xref{prop:RxySwxy}
  \\\imark & 3 different relations on the triple & \otriple{\Swxy(\omega)}{\FH(\omega)}{\Swyy(\omega)} & \xref{prop:RxySwxy}
  \\\imark & with $\rvx$ and $\rvy$ \prope{real-valued}, 2 relations on & \otriple{\Swxy(\omega)}{\FH(\omega)}{\Swxx(\omega)} & \xref{prop:RxySwxy_real}
  \\\imark & with $\rvx$ and $\rvy$ \prope{real-valued}, 2 relations on & \otriple{\Swxy(\omega)}{\FH(\omega)}{\Swyy(\omega)} & \xref{prop:RxySwxy_real}
\end{array}$


%---------------------------------------
\begin{proposition}
\label{prop:Rxy}
%---------------------------------------
%\propbox{
Let (1)--(8) below correspond to the eight definitions of $\Rxy(m)$ in \pref{def:Rxym}.
\\
$\begin{array}{|Mc        l       c l@{\hspace{0pt}}c D   l         c l@{\hspace{0pt}}c D|}
  \hline
    (1), (2), (3), or (4) & \implies& \Ryx(m) &=&\Rxy^\ast&(-m) &and&  \Rxx(-m) &=&\Rxx^\ast&(m) & (\prope{conjugate symmetric})
  \\(5), (6), (7), or (8) & \implies& \Ryx(m) &=&\Rxy     &(-m) &and&  \Rxx(-m) &=&\Rxx     &(m) & (\prope{symmetric})
  \\\hline
\end{array}$
%}
\end{proposition}
\begin{proof}
\begin{align*}
    (1).\quad\Ryx(m)
      &\eqd \pE\brs{\rvy(m)\rvx^\ast(0)}
      && \text{by Papoulis' definition of $\Rxy(m)$}                           && \text{\xref{def:Rxym}}
    \\&= \brp{\pE\brs{\rvx(0)\rvy^\ast(m)}}^\ast
      && \text{by \prope{antiautomorphic} property of *-algebras}              && \text{\xref{def:staralg}}
    \\&= \brp{\pE\brs{\rvx(0-m)\rvy^\ast(m-m)}}^\ast
      && \text{by \prope{wide sense stationary} property}
    \\&\eqd \Rxy^\ast(-m)
      && \text{by Papoulis' definition of $\Rxy(m)$}                           && \text{\xref{def:Rxym}}
    \\
    \Rxx(-m)
      &\eqd \brlr{\Rxy(-m)}_{\rvy=\rvx}
     &&= \brlr{\Ryx^\ast(m)}_{\rvy=\rvx}
      %&& \text{by previous result}
     &&= \Rxx^\ast(m)
\end{align*}

\begin{align*}
    (2).\quad\Ryx(m)
      &\eqd \pE\brs{\rvy^\ast(0)\rvx(m)}
     &&=\brp{\pE\brs{\rvx^\ast(m)\rvy(0)}}^\ast
     &&=\brp{\pE\brs{\rvx^\ast(m-m)\rvy(0-m)}}^\ast
     &&\eqd \Rxy^\ast(-m)
    \\
    (3).\quad\Ryx(m)
      &\eqd \pE\brs{\rvy(0)\rvx^\ast(m)}
     &&=\brp{\pE\brs{\rvx(m)\rvy^\ast(0)}}^\ast
     &&=\brp{\pE\brs{\rvx(m-m)\rvy^\ast(0-m)}}^\ast
     &&\eqd \Rxy^\ast(-m)
    \\
    (4).\quad\Ryx(m)
      &\eqd \pE\brs{\rvy^\ast(m)\rvx(0)}
     &&=\brp{\pE\brs{\rvx^\ast(0)\rvy(m)}}^\ast
     &&=\brp{\pE\brs{\rvx^\ast(-m)\rvy(0)}}^\ast
     &&\eqd \Rxy^\ast(-m)
    \\
    (5).\quad\Ryx(m)
      &\eqd  \pE\brs{\rvy(0)\rvx(m)}
     &&=     \pE\brs{\rvx(m)\rvy(0)}
     &&=     \pE\brs{\rvx(m-m)\rvy(0-m)}
     &&\eqd \Rxy(-m)
    \\
    (6).\quad\Ryx(m)
      &\eqd  \pE\brs{\rvy(m)\rvx(0)}
     &&=     \pE\brs{\rvx(0)\rvy(m)}
     &&=     \pE\brs{\rvx(0-m)\rvy(m-m)}
     &&\eqd \Rxy(-m)
    \\
    (7).\quad\Ryx(m)
      &\eqd  \pE\brs{\rvy^\ast(0)\rvx^\ast(m)}
     &&=     \pE\brs{\rvx^\ast(m)\rvy^\ast(0)}
     &&=     \pE\brs{\rvx^\ast(m-m)\rvy^\ast(0-m)}
     &&\eqd \Rxy(-m)
    \\
    (8).\quad\Ryx(m)
      &\eqd  \pE\brs{\rvy^\ast(m)\rvx^\ast(0)}
     &&=     \pE\brs{\rvx^\ast(0)\rvy^\ast(m)}
     &&=     \pE\brs{\rvx^\ast(0-m)\rvy^\ast(m-m)}
     &&\eqd \Rxy(-m)
\end{align*}

\end{proof}

%---------------------------------------
\begin{proposition}
\label{prop:Szxy}
%---------------------------------------
%\propbox{
Let (1)--(8) below correspond to the eight definitions of $\Rxy(m)$ in \pref{def:Rxym}.
\\
$\begin{array}{|M           c         >{\ds}l   c >{\ds}l@{\hspace{0pt}}c            D     >{\ds}l                      c  >{\ds}l@{\hspace{0pt}}c|}
  \hline
    (1), (2), (3), or (4) & \implies& \Szyx(z) &=&\Szxy^\ast&\brp{\frac{1}{z^\ast}} &and&  \Szxx\brp{\frac{1}{z^\ast}} &=& \Szxx^\ast&(z)
  \\(5), (6), (7), or (8) & \implies& \Szyx(z) &=&\Szxy     &\brp{\frac{1}{z}}      &and&  \Szxx\brp{\frac{1}{z}}      &=& \Szxx     &(z)
  \\\hline
\end{array}$
%}
\end{proposition}
\begin{proof}
\begin{align*}
  (1)-(4):\,\Szyx(z)
     &\eqd \opZ \Ryx(m)
    && \text{by definition of $\Szxy(z)$}                          && %\text{\xref{def:csd}}
  \\&\eqd \sum_{m\in\Z}\Ryx(m) z^{-m}
    && \text{by definition of $\opZ$}                              && \text{\xref{def:opZ}}
  \\&\eqd \sum_{m\in\Z} \Rxy^\ast(-m) z^{-m}
    && \text{by \prope{conjugate symmetry} property}               && \text{\xref{prop:Rxy}}
  \\&= \brs{\sum_{m\in\Z} \Rxy(-m) (z^\ast)^{-m}}^\ast
    && \text{by \prope{antiautomorphic} property of *-algebras}    && \text{\xref{def:staralg}}
  \\&= \brs{\sum_{-p\in\Z} \Rxy(p) (z^\ast)^{p}}^\ast
    && \text{where $p\eqd -m$}                                     && \text{$\implies$ $m=-p$}
  \\&= \brs{\sum_{p\in\Z} \Rxy(p) (z^\ast)^{p}}^\ast
    && \text{by \prope{absolutely summable} property}             %&& \text{\xref{def:spllR}}
  \\&= \brs{\sum_{p\in\Z} \Rxy(p) \brp{\frac{1}{z^\ast}}^{-p}}^\ast
  \\&\eqd \Szxy^\ast\brp{\frac{1}{z^\ast}}
    && \text{by definition of $\Szxy(z)$}                          && \text{\xref{def:Szxy}}
\end{align*}
\begin{align*}
  (1)-(4):\,\Szxx^\ast(z)
    &\eqd \brs{\Szyx(z)}^\ast_{\rvy=\rvx}
   &&= \brs{\Szxy^\ast\brp{\frac{1}{z^\ast}}}^\ast_{\rvy=\rvx}
   &&= \brs{\Szxx^\ast\brp{\frac{1}{z^\ast}}}^\ast
   &&= \Szxx\brp{\frac{1}{z^\ast}}
  \\
  (5)-(8):\,\Szyx(z)
    &= \sum_{m\in\Z} \Rxy(-m) z^{-m}
   &&= \sum_{-p\in\Z} \Rxy(p) z^{p}
   &&= \sum_{p\in\Z} \Rxy(p) \brp{\frac{1}{z}}^{-p}
   &&\eqd \Szxy\brp{\frac{1}{z}}
  \\
  (5)-(8):\,\Szxx(z)
    &= \brlr{\Szyx(z)}_{\rvy=\rvx}
   &&= \brlr{\Szxy\brp{\frac{1}{z}}}_{\rvy=\rvx}
   &&= \brlr{\Szxx\brp{\frac{1}{z}}}
   &&= \Szxx\brp{\frac{1}{z}}
\end{align*}
\end{proof}

%---------------------------------------
\begin{proposition}
\label{prop:Swxx_real}
%---------------------------------------
Let (1)--(8) correspond to the eight definitions of $\Rxy(m)$ in \pref{def:Rxym}.
\\
$\begin{array}{|M           c         l               Ml|}
  \hline
    \{(1), (2), (3), or (4)\} & \implies& \{\Swxx^\ast(\omega)=\Swxx(\omega) & ($\Swxx(\omega)$ is \prope{real-valued})&\}
  \\\hline
\end{array}$
\end{proposition}
\begin{proof}
{\begin{align*}
  \Swxx^\ast(\omega)
    &= \brlr{\Szxx^\ast\brp{z}}_{z=e^{i\omega}}
  \\&= \brlr{\Szxx^\ast\brp{\frac{1}{z^\ast}}}_{z=e^{i\omega}}
  \\&= \brlr{\Szxx\brp{z}}_{z=e^{i\omega}}
    && \text{by \prefp{prop:Szxy}}
  \\&= \Swxx(\omega)
\end{align*}}
\end{proof}

%---------------------------------------
\begin{proposition}
\label{prop:RxySzxy}
%---------------------------------------
%\propbox{
Let (1)--(8) below correspond to the eight definitions of $\Rxy(m)$ in \pref{def:Rxym}.
\\
$\begin{array}{|Fc        l         c l       @{\hspace{0pt}}c@{\hspace{0pt}}l  D    l         c  l@{\hspace{0pt}}c@{\hspace{0pt}}l     c  l@{\hspace{0pt}}c@{\hspace{0pt}}l@{\hspace{0pt}}c@{\hspace{0pt}}l|}
  \hline
    (1) &      \implies& \Szxy(z) &=&\ZH^\ast&\brp{\frac{1}{z^\ast}} &\Szxx(z) &and& \Szyy(z) &=& \ZH     &(z)               &\Szxy(z) &=& \ZH           &(z)               &\ZH^\ast&\brp{\frac{1}{z^\ast}} &\Szxx(z)
  \\(2) &      \implies& \Szxy(z) &=&\ZH     &(z)                    &\Szxx(z) &and& \Szyy(z) &=& \ZH     &(z)               &\Szyx(z) &=& \ZH           &(z)               &\ZH^\ast&\brp{\frac{1}{z^\ast}} &\Szxx(z)
  \\(3) &      \implies& \Szxy(z) &=&\ZH^\ast&\brp{z^\ast}           &\Szxx(z) &and& \Szyy(z) &=& \ZH     &\brp{\frac{1}{z}} &\Szxy(z) &=& \ZH^\ast      &\brp{z^\ast}      &\ZH     &\brp{\frac{1}{z}}      &\Szxx(z)
  \\(4) &      \implies& \Szxy(z) &=&\ZH     &\brp{\frac{1}{z}}      &\Szxx(z) &and& \Szyy(z) &=& \ZH     &\brp{\frac{1}{z}} &\Szyx(z) &=& \ZH           &\brp{\frac{1}{z}} &\ZH^\ast&\brp{z^\ast}           &\Szxx(z)
  \\(5) &      \implies& \Szxy(z) &=&\ZH     &(z)                    &\Szxx(z) &and& \Szyy(z) &=& \ZH     &(z)               &\Szyx(z) &=& \ZH           &(z)               &\ZH     &\brp{\frac{1}{z}}      &\Szxx(z)
  \\(6) &      \implies& \Szxy(z) &=&\ZH     &\brp{\frac{1}{z}}      &\Szxx(z) &and& \Szyy(z) &=& \ZH     &\brp{\frac{1}{z}} &\Szyx(z) &=& \ZH           &\brp{\frac{1}{z}} &\ZH     &(z)                    &\Szxx(z)
  \\(7) &      \implies& \Szxy(z) &=&\ZH^\ast&\brp{z^\ast}           &\Szxx(z) &and& \Szyy(z) &=& \ZH^\ast&(z^\ast)          &\Szyx(z) &=& \ZH^\ast      &\brp{z^\ast}      &\ZH^\ast&\brp{\frac{1}{z^\ast}} &\Szxx(z)
  \\(8) &      \implies& \Szxy(z) &=&\ZH     &\brp{\frac{1}{z}}      &\Szxx(z) &and& \Szyy(z) &=& \ZH     &\brp{\frac{1}{z}} &\Szyx(z) &=& \ZH           &\brp{\frac{1}{z}} &\ZH     &(z)                    &\Szxx(z)
  \\\hline
\end{array}$
%}
\end{proposition}
\begin{proof}
\begin{align*}
%\intertext{(1). If we follow Papoulis $\brp{\Rxy(m)\eqd\pE\brs{\rvx(m)\rvy^\ast(0)}}$, then\ldots}
    (1).\quad\Szyt(z)
      &\eqd \opZ\Ryt(m)
      && \text{by definition of $\Szty(z)$}                                    && \text{\xref{def:Szxy}}
    \\&\eqd \opZ\pE\brs{\rvy(m)\rvt^\ast(0)}
      && \text{by Papoulis' definition of $\Rty(m)$}                           && \text{\xref{def:Rxym}}
    \\&=    \opZ\pE\brs{\brp{\sum_{k\in\Z} \fh(k)\rvx(m-k)}\rvt^\ast(0)}
      && \mathrlap{\text{by \prope{linear time-invariant} property of $\opH$}}
    \\&=    \opZ        \sum_{k\in\Z} \fh(k) \pE\brs{\rvx(m-k)\rvt^\ast(0)}
      && \text{by \prope{linearity} of $\opE$}                                %&& \text{\xref{thm:pE_linop}}
    \\&\eqd \opZ        \sum_{k\in\Z} \fh(k) \Rxt(m-k)
      && \text{by Papoulis' definition of $\Rty(m)$}                           && \text{\xref{def:Rxym}}
    \\&\eqd \opZ\brs{\fh(m) \convd \Rxt(m)}
      && \text{by definition of \ope{convolution}}                             && \text{\xref{def:convd}}
    \\&= \brs{\opZ\fh(m)} \brs{\opZ\Rxt(m)}
      && \text{by \thme{convolution theorem}}                                  && \text{\xref{thm:conv}}
    \\&\eqd \ZH(z) \Szxt(z)
      && \text{by definitions of $\ZH(z)$ and $\Szxt(z)$}                      && \text{\xref{def:Szxy}}
    \\
    \boxed{\Szxy(z)}
      &=\Szyx^\ast\brp{\frac{1}{z^\ast}}\quad\text{(by \pref{prop:Szxy})}
     &&\eqd \brlr{\Szyt^\ast\brp{\frac{1}{z^\ast}}}_{\rvt\eqd\rvx}
       =\ZH^\ast\brp{\frac{1}{z^\ast}} \Szxx^\ast\brp{\frac{1}{z^\ast}}
     &&=\boxed{\ZH^\ast\brp{\frac{1}{z^\ast}} \Szxx(z)}
     \\
    \boxed{\Szyy(z)}
      &\eqd \brlr{\Szyt(z)}_{\rvt\eqd\rvy}
     &&=    \brlr{ \ZH(z) \Szxt\brp{z}}_{\rvt\eqd\rvy}
     =    \boxed{\ZH(z) \Szxy\brp{z}}
     &&=    \boxed{\ZH(z) \ZH^\ast\brp{\frac{1}{z^\ast}} \Szxx(z)}
\\
    (2).\quad\Szty(z)
      &\eqd \opZ\Rty(m)
     &&\eqd \opZ\pE\brs{\rvt^\ast(0)\rvy(m)}
     &&=    \opZ\pE\brs{\rvt^\ast(0)\brp{\fh(m)\convd\rvx(m)}}
    \\&=    \opZ\pE\brs{\rvt^\ast(0)\brp{\sum_{k\in\Z} \fh(k)\rvx(m-k)}}
     &&=    \opZ\brp{\sum_{k\in\Z} \fh(k)\pE\brs{\rvt^\ast(0)\rvx(m-k)}}
     &&\eqd \opZ\brp{\sum_{k\in\Z} \fh(k)\Rtx(m-k)}
    \\&\eqd \opZ\brp{\fh(m)\convd\Rtx(m)}
     &&=    \brs{\opZ\fh(m)}\brs{\opZ\Rtx(m)}
     &&\eqd \ZH(z) \Sztx(z)
    \\
    \boxed{\Szxy(z)}
      &\eqd \brlr{\Szty}_{\rvt=\rvx}
     &&=    \brlr{\ZH(z) \Sztx(z)}_{\rvt=\rvx}
     &&=    \boxed{\ZH(z) \Szxx(z)}
     \\
    \boxed{\Szyy(z)}
      &\eqd
            \brlr{\Szty(z)}_{\rvt\eqd\rvy}
       =    \brlr{\ZH(z) \Sztx(z)}_{\rvt\eqd\rvy}
     &&=\mathrlap{\boxed{\ZH(z) \Szyx(z)}
       =    \ZH(z) \ZH^\ast\brp{\frac{1}{z^\ast}} \Szxx^\ast\brp{\frac{1}{z^\ast}}
     =    \boxed{\ZH(z) \ZH^\ast\brp{\frac{1}{z^\ast}} \Szxx(z)}}
\\
    (3).\quad\Szty(z)
      &\eqd \opZ\Rty(m)
       \eqd \opZ\pE\brs{\rvt(0)\rvy^\ast(m)}
     &&=    \opZ\pE\brp{\rvt(0)\brs{\sum_{k\in\Z} \fh(k)\rvx(m-k)}^\ast}
    \\&=    \opZ\pE\brs{\rvt(0) \sum_{k\in\Z} \fh^\ast(k)      \rvx^\ast(m-k)}
     &&=    \opZ        \sum_{k\in\Z} \fh^\ast(k) \pE\brs{\rvt(0)\rvx^\ast(m-k)}
     &&\eqd \opZ        \sum_{k\in\Z} \fh^\ast(k) \Rtx(m-k)
    \\&\eqd \opZ\brs{\fh^\ast(m) \convd \Rtx(m)}
     &&=\mathrlap{\brs{\opZ\fh^\ast(m)} \brs{\opZ\Rtx(m)}
       = \ZH^\ast\brp{z^\ast} \Sztx\brp{z} \quad\text{by \prefp{thm:opZ}}}
    \\
    \boxed{\Szxy(z)}
      &\eqd \brlr{\Szty(z)}_{\rvt=\rvx}
     &&= \brlr{\ZH^\ast\brp{z^\ast} \Sztx(z)}_{\rvt=\rvx}
     &&= \boxed{\ZH^\ast\brp{z^\ast} \Szxx(z)}
    \\
    \boxed{\Szyy(z)}
      &\eqd \brlr{\Szty(z)}_{\rvt=\rvy}
       = \brlr{\ZH^\ast\brp{z^\ast} \Sztx\brp{z}}_{\rvt=\rvy}
     &&\mathrlap{= \boxed{\ZH^\ast\brp{z^\ast} \Szyx\brp{z}}
       = \ZH^\ast\brp{z^\ast} \ZH\brp{\frac{1}{z}} \Szxx^\ast\brp{\frac{1}{z^\ast}}
       = \boxed{\ZH^\ast\brp{z^\ast} \ZH\brp{\frac{1}{z}} \Szxx(z)}}
\\
    (4).\quad\Szyt(z)
      &\eqd \opZ\Ryt(m)
     &&\eqd \opZ\pE\brs{\rvy^\ast(m)\rvt(0)}
    \\&=    \opZ\pE\brs{\brs{\sum_{k\in\Z} \fh(k)\rvx(m-k)}^\ast \rvt(0)}
     &&=    \opZ\pE\brs{\brs{\sum_{k\in\Z} \fh^\ast(k)\rvx^\ast(m-k) \rvt(0)}}
     &&=    \opZ\sum_{k\in\Z} \fh^\ast(k)\pE\brs{\rvx^\ast(m-k) \rvt(0)}
    \\&=    \opZ\sum_{k\in\Z} \fh^\ast(k)\Rxt(m-k)
     &&\eqd \opZ\brs{\fh^\ast(m)\convd\Rxt(m)}
     &&=    \opZ\fh^\ast(m)\opZ\Rxt(m)
    \\&=    \ZH^\ast\brp{z^\ast}\Szxt(z)
    && \text{by \prefp{thm:opZ}}
    \\
    \boxed{\Szxy(z)}
      &= \Szyx^\ast\brp{\frac{1}{z^\ast}}
     &&\eqd \brlr{\Szyt^\ast\brp{\frac{1}{z^\ast}}}_{\rvt=\rvx}
     &&=  \brlr{\ZH\brp{\frac{1}{z}} \Szxt^\ast\brp{\frac{1}{z^\ast}}}_{\rvt=\rvx}
   \\&\eqd \ZH\brp{\frac{1}{z}} \Szxx^\ast\brp{\frac{1}{z^\ast}}
     &&=  \boxed{\ZH\brp{\frac{1}{z}} \Szxx(z)}
     &&   \text{by \pref{prop:Szxy}}
    \\
    \boxed{\Szyy(z)}
      &\eqd \brlr{\Szyt(z)}_{\rvt=\rvy}
       = \brlr{\ZH^\ast\brp{z^\ast}\Szxt(z)}_{\rvt=\rvy}
     &&= \boxed{\ZH^\ast\brp{z^\ast}\Szxy(z)}
     &&= \boxed{\ZH^\ast\brp{z^\ast}\ZH\brp{\frac{1}{z}} \Szxx(z)}
    \\
    (5).\quad\Szty(z)
      &\eqd \opZ\Rty(m)
     &&\eqd \opZ\pE\brs{\rvt(0)\rvy(m)}
    \\&=    \opZ\pE\brs{\rvt(0)\brp{\sum_{k\in\Z} \fh(k) \rvx(m-k)}}
     &&=    \opZ                    \sum_{k\in\Z} \fh(k) \pE\brs{\rvt(0)\rvx(m-k)}
     &&\eqd \opZ                    \sum_{k\in\Z} \fh(k) \Rtx(m-k)
    \\&\eqd \opZ\brs{\fh(m) \convd \Rtx(m)}
      &&
      &&= \ZH(z) \Sztx(z)
    \\
    \boxed{\Szxy(z)}
      &\eqd \brlr{\Szty(z)}_{\rvt=\rvx}
     &&= \brlr{\ZH(z) \Sztx(z)}_{\rvt=\rvx}
     &&= \boxed{\ZH(z) \Szxx(z)}
    \\
    \boxed{\Szyy(z)}
      &\eqd \brlr{\Szty(z)}_{\rvt=\rvy}
     &&= \brlr{\ZH(z) \Sztx(z)}_{\rvt=\rvy}
     &&= \ZH(z) \Szyx(z)
    \\&= \boxed{\ZH(z) \Szxy\brp{\frac{1}{z}}}
     &&= \boxed{\ZH(z) \ZH\brp{\frac{1}{z}} \Szxx\brp{\frac{1}{z}}}
\\
   (6).\quad\Szty(z)
      &\eqd \opZ\Rty(m)
     &&\eqd \opZ\pE\brs{\rvt(m)\rvy(0)}
       \mathrlap{\qquad
       =    \opZ\pE\brs{\rvt(m)\brp{\sum_{k\in\Z} \fh(k) \rvx(0-k)}}}
    \\&=    \opZ\sum_{k\in\Z} \fh(k) \pE\brs{\rvt(m)\rvx(-k)}
     &&=    \opZ\sum_{k\in\Z} \fh(k) \pE\brs{\rvt(m+k)\rvx(0)}
     &&\eqd \opZ\sum_{k\in\Z} \fh(k) \Rtx(m+k)
    \\&\mcom{\eqd \opZ\brs{\fh(-m) \convd \Rtx(m)}}{by \prefp{prop:conv_knk}}
     &&= \opZ\brs{\fh(-m)} \brs{\opZ\Rtx(m)}
     &&= \ZH\brp{\frac{1}{z}} \Sztx(z)
    \\
    \boxed{\Szxy(z)}
      &\eqd \brlr{\Szty(z)}_{\rvt=\rvx}
     &&= \brlr{\ZH\brp{\frac{1}{z}} \Sztx(z)}_{\rvt=\rvx}
     &&= \boxed{\ZH\brp{\frac{1}{z}} \Szxx(z)}
    \\
    \boxed{\Szyy(z)}
      &\eqd \brlr{\Szty(z)}_{\rvt=\rvy}
     &&= \brlr{\ZH\brp{\frac{1}{z}} \Sztx(z)}_{\rvt=\rvy}
     &&= \ZH\brp{\frac{1}{z}} \Szyx(z)
    \\&= \boxed{\ZH\brp{\frac{1}{z}} \Szxy\brp{\frac{1}{z}}}
     &&= \boxed{\ZH\brp{\frac{1}{z}} \ZH\brp{z} \Szxx\brp{\frac{1}{z}}}
\\
    (7).\quad\Szty(z)
      &\eqd \opZ\Rty(m)
     &&\eqd \opZ\pE\brs{\rvt^\ast(0)\rvy^\ast(m)}
     &&=    \opZ\pE\brs{\rvt^\ast(0)\rvy^\ast(m)}
    \\&=    \opZ\pE\brs{\rvt^\ast(0)\brp{\sum_{k\in\Z} \fh(k) \rvx(m-k)}^\ast}
     &&=    \opZ                    \sum_{k\in\Z} \fh^\ast(k) \pE\brs{\rvt^\ast(0)\rvx^\ast(m-k)}
     &&\eqd \opZ                    \sum_{k\in\Z} \fh^\ast(k) \Rtx(m-k)
    \\&\eqd \opZ\brs{\fh^\ast(m) \convd \Rtx(m)}
     &&= \brs{\opZ\fh^\ast(m)} \brs{\opZ\Rtx(m)}
     &&= \ZH^\ast\brp{z^\ast} \Sztx(z)
    \\
    \boxed{\Szxy(z)}
      &\eqd \brlr{\Szty(z)}_{\rvt=\rvx}
     &&= \brlr{\ZH^\ast\brp{z^\ast} \Sztx(z)}_{\rvt=\rvx}
     &&= \boxed{\ZH^\ast\brp{z^\ast} \Szxx(z)}
    \\
    \boxed{\Szyy(z)}
      &\eqd \brlr{\Szty(z)}_{\rvt=\rvy}
     &&= \brlr{\ZH^\ast\brp{z^\ast} \Sztx(z)}_{\rvt=\rvy}
     &&= \ZH^\ast\brp{z^\ast} \Szyx(z)
    \\&= \boxed{\ZH^\ast\brp{z^\ast} \Szxy\brp{\frac{1}{z}}}
     &&= \boxed{\ZH^\ast\brp{z^\ast} \ZH^\ast\brp{\frac{1}{z^\ast}} \Szxx\brp{\frac{1}{z}}}
\\
    (8).\quad\Szty(z)
      &\eqd \opZ\Rty(m)
     &&\eqd \opZ\pE\brs{\rvt^\ast(m)\rvy^\ast(0)}
       \mathrlap{\qquad=    \opZ\pE\brs{\rvt^\ast(m)\brp{\sum_{k\in\Z} \fh^\ast(k) \rvx^\ast(0-k)}}}
    \\&=    \opZ\sum_{k\in\Z} \fh^\ast(k) \pE\brs{\rvt^\ast(m)\rvx^\ast(-k)}
     &&=    \opZ\sum_{k\in\Z} \fh^\ast(k) \pE\brs{\rvt^\ast(0)\rvx^\ast(-m-k)}
    \\&=    \opZ\sum_{k\in\Z} \fh^\ast(k) \pE\brs{\rvt^\ast(-m-k)\rvx^\ast(0)}
     &&\eqd \opZ                    \sum_{k\in\Z} \fh^\ast(k) \Rtx^\ast(-m-k)
     &&=    \opZ                    \sum_{k\in\Z} \fh^\ast(k) \Rtx(m+k)
    \\&\eqd \opZ\brs{\fh(-m) \convd \Rtx(m)}
     &&= \opZ\brs{\fh(-m) \convd \Rtx(m)}
     &&= \brs{\opZ\fh(-m)} \brs{\opZ\Rtx(m)}
    \\&= \ZH\brp{\frac{1}{z}} \Sztx(z)
    \\
    \boxed{\Szxy(z)}
      &\eqd \brlr{\Szty(z)}_{\rvt=\rvx}
     &&= \brlr{\ZH\brp{\frac{1}{z}} \Sztx(z)}_{\rvt=\rvx}
     &&= \boxed{\ZH\brp{\frac{1}{z}} \Szxx(z)}
    \\
    \boxed{\Szyy(z)}
      &\eqd \brlr{\Szty(z)}_{\rvt=\rvy}
     &&= \brlr{\ZH\brp{\frac{1}{z}} \Sztx(z)}_{\rvt=\rvy}
     &&= \ZH\brp{\frac{1}{z}} \Szyx(z)
    \\&= \boxed{\ZH\brp{\frac{1}{z}} \Szxy\brp{\frac{1}{z}}}
     &&= \boxed{\ZH\brp{\frac{1}{z}} \ZH\brp{z} \Szxx\brp{\frac{1}{z}}}
  \end{align*}
\end{proof}

%---------------------------------------
\begin{proposition}
\label{prop:Rxym}
\label{prop:RxySwxy}
%---------------------------------------
Let (1)--(8) below correspond to the eight definitions of $\Rxy(m)$ in \pref{def:Rxym}.
\\
$\begin{array}{|Fc        l              c l       *{2}{@{\hspace{0pt}}r}       D    l              c  l@{\hspace{0pt}}r *{2}{@{\hspace{0pt}}l} c  l@{\hspace{0pt}}r@{\hspace{0pt}}l@{\hspace{0pt}}r|}
  \hline
    (1) &      \implies& \Swxy(\omega) &=&\FH^\ast&( \omega) &\Swxx(\omega)   &and& \Swyy(\omega) &=& \FH     &( \omega)&\Swxy     &( \omega) &=& |\FH&( \omega)|^2 &\Swxx     &(\omega)
  \\(2) &      \implies& \Swxy(\omega) &=&\FH     &( \omega) &\Swxx(\omega)   &and& \Swyy(\omega) &=& \FH     &( \omega)&\Swxy^\ast&( \omega) &=& |\FH&( \omega)|^2 &\Swxx     &(\omega)
  \\(3) &      \implies& \Swxy(\omega) &=&\FH^\ast&(-\omega) &\Swxx(\omega)   &and& \Swyy(\omega) &=& \FH^\ast&(-\omega)&\Swxy^\ast&( \omega) &=& |\FH&(-\omega)|^2 &\Swxx     &(\omega)
  \\(4) &      \implies& \Swxy(\omega) &=&\FH     &(-\omega) &\Swxx(\omega)   &and& \Swyy(\omega) &=& \FH     &(-\omega)&\Swxy^\ast&( \omega) &=& |\FH&(-\omega)|^2 &\Swxx     &(\omega)
  \\(5) &      \implies& \Swxy(\omega) &=&\FH     &( \omega) &\Swxx(\omega)   &and& \Swyy(\omega) &=& \FH     &( \omega)&\Swxy^\ast&( \omega) &=& |\FH&( \omega)|^2 &\Swxx^\ast&(\omega)
  \\(6) &      \implies& \Swxy(\omega) &=&\FH     &(-\omega) &\Swxx(\omega)   &and& \Swyy(\omega) &=& \FH     &(-\omega)&\Swxy^\ast&( \omega) &=& |\FH&(-\omega)|^2 &\Swxx^\ast&(\omega)
  \\(7) &      \implies& \Swxy(\omega) &=&\FH^\ast&(-\omega) &\Swxx(\omega)   &and& \Swyy(\omega) &=& \FH^\ast&(-\omega)&\Swxy^\ast&( \omega) &=& |\FH&(-\omega)|^2 &\Swxx^\ast&(\omega)
  \\(8) &      \implies& \Swxy(\omega) &=&\FH     &(-\omega) &\Swxx(\omega)   &and& \Swyy(\omega) &=& \FH     &(-\omega)&\Swxy^\ast&( \omega) &=& |\FH&(-\omega)|^2 &\Swxx^\ast&(\omega)
  \\\hline
\end{array}$
\end{proposition}
\begin{proof}
\begin{align*}
  (1).\quad\boxed{\Swxy(\omega)}
      &= \brlr{\Szxy(z)}_{z=e^{i\omega}}
    \\&= \brlr{\ZH^\ast\brp{\frac{1}{z^\ast}} \Szxx(z)}_{z=e^{i\omega}}
      && \text{by $\Szxy(z)$ result}         &&    \text{\xref{prop:RxySzxy}}
    \\&= \ZH^\ast\brp{e^{i\omega}} \Szxx\brp{e^{i\omega}}
      && \text{(evaluation around unit circle in $z$-plane)}
    \\&= \boxed{\FH^\ast(\omega) \Swxx(\omega)}
      && \text{by definition of \ope{DTFT}}  &&    \text{\xref{def:dtft}}
    \\
    \boxed{\Swyy(\omega)}
      &= \brlr{\Szyy(z)}_{z=e^{i\omega}}
    \\&= \brlr{\ZH(z) \Szxy(z)}_{z=e^{i\omega}}
      && \text{by $\Szxy(z)$ result}         &&    \text{\xref{prop:RxySzxy}}
    \\&= \ZH\brp{e^{i\omega}} \Szxy\brp{e^{i\omega}}
      && \text{(evaluation around unit circle in $z$-plane)}
    \\&= \boxed{\FH(\omega) \Swxy(\omega)}
      && \text{by definition of \ope{DTFT}}  &&    \text{\xref{def:dtft}}
    \\
    \boxed{\Swyy(\omega)}
      &= \brlr{\Szyy(z)}_{z=e^{i\omega}}
    \\&= \brlr{\ZH(z) \ZH^\ast\brp{\frac{1}{z^\ast}} \Szxx^\ast\brp{\frac{1}{z^\ast}}}_{z=e^{i\omega}}
      && \text{by $\Szxy(z)$ result}         &&    \text{\xref{prop:RxySzxy}}
    \\&= \ZH\brp{e^{i\omega}} \ZH^\ast\brp{\frac{1}{e^{-i\omega}}} \Szxx^\ast\brp{\frac{1}{e^{-i\omega}}}
    \\&=\mathrlap{ 
         \ZH\brp{e^{i\omega}} \ZH^\ast\brp{e^{i\omega}} \Szxx^\ast\brp{e^{i\omega}}
       = \FH\brp{\omega} \FH^\ast(\omega) \Swxx^\ast(\omega)
       = \abs{\FH\brp{\omega}}^2 \Swxx^\ast(\omega)}
    \\&= \boxed{\abs{\FH\brp{\omega}}^2 \Swxx(\omega)}
      && \text{because $\Swxx(\omega)$ is \prope{real-valued}}  
      && \text{\xref{cor:Swxy_sym}}
\end{align*}

The other seven sets of proofs follow in like manner.
\end{proof}


%---------------------------------------
\begin{proposition}
\label{prop:RxySwxy_real}
%---------------------------------------
Let (1)--(8) below correspond to the eight definitions of $\Rxy(m)$ in \pref{def:Rxym}.
\\\textbf{If} $\rvx(n)$ and $\rvy(n)$ are \propb{real-valued}, then the following hold:
\\
$\begin{array}{|Fc        l              c l       *{2}{@{\hspace{0pt}}r}       D    l              c  l@{\hspace{0pt}}r *{2}{@{\hspace{0pt}}l} c  l*{2}{@{\hspace{0pt}}r}|}
  \hline
    (1) &      \implies& \Swxy(\omega) &=&\FH^\ast&( \omega) &\Swxx(\omega)   &and& \Swyy(\omega) &=& \FH     &( \omega)&\Swxy     &(\omega) &=& |\FH&(\omega)|^2 &\Swxx(\omega)
  \\(2) &      \implies& \Swxy(\omega) &=&\FH     &( \omega) &\Swxx(\omega)   &and& \Swyy(\omega) &=& \FH     &( \omega)&\Swxy^\ast&(\omega) &=& |\FH&(\omega)|^2 &\Swxx(\omega)
  \\(3) &      \implies& \Swxy(\omega) &=&\FH     &( \omega) &\Swxx(\omega)   &and& \Swyy(\omega) &=& \FH     &( \omega)&\Swxy^\ast&(\omega) &=& |\FH&(\omega)|^2 &\Swxx(\omega)
  \\(4) &      \implies& \Swxy(\omega) &=&\FH^\ast&( \omega) &\Swxx(\omega)   &and& \Swyy(\omega) &=& \FH^\ast&( \omega)&\Swxy^\ast&(\omega) &=& |\FH&(\omega)|^2 &\Swxx(\omega)
  \\(5) &      \implies& \Swxy(\omega) &=&\FH     &( \omega) &\Swxx(\omega)   &and& \Swyy(\omega) &=& \FH     &( \omega)&\Swxy^\ast&(\omega) &=& |\FH&(\omega)|^2 &\Swxx(\omega)
  \\(6) &      \implies& \Swxy(\omega) &=&\FH^\ast&( \omega) &\Swxx(\omega)   &and& \Swyy(\omega) &=& \FH^\ast&( \omega)&\Swxy^\ast&(\omega) &=& |\FH&(\omega)|^2 &\Swxx(\omega)
  \\(7) &      \implies& \Swxy(\omega) &=&\FH     &( \omega) &\Swxx(\omega)   &and& \Swyy(\omega) &=& \FH     &( \omega)&\Swxy^\ast&(\omega) &=& |\FH&(\omega)|^2 &\Swxx(\omega)
  \\(8) &      \implies& \Swxy(\omega) &=&\FH^\ast&( \omega) &\Swxx(\omega)   &and& \Swyy(\omega) &=& \FH^\ast&( \omega)&\Swxy^\ast&(\omega) &=& |\FH&(\omega)|^2 &\Swxx(\omega)
  \\\hline
\end{array}$
\end{proposition}
\begin{proof}
These results follow directly from \pref{prop:RxySwxy} and \prefpp{thm:dtft_conjneg_real}.
\end{proof}

%=======================================
\section{Which one?}
%=======================================
Which definition of $\Rxy(m)$ should we use?
Any one of them is perfectly acceptable---as long as a clear definition is provided and that definition is used consistently.
That being said, note the following:

\begin{enumerate}
\item The \ope{expectation} operator $\pE\brp{\rvX\rvY^\ast}$ is an \fncte{inner product}.
As such, it would seem the most natural to follow the convention of other inner product definitions
and thus put the conjugate $\conj$ on $\rvy$ (i.e. follow Papoulis):
\\\indentx$\begin{array}{c>{\ds}rc>{\ds}l}
    \imark & \inprod{\fx(t)}{\fy(t)} &\eqd& \int_{t\in\R} \fx(t)\fy^\conj(t) \dt
  \\\imark & \inprod{\fx(n)}{\fy(n)} &\eqd& \sum_{n\in\Z} \fx(n)\fy^\conj(n)
  \\\imark & \inprod{\rvX}{\rvY}     &\eqd& \pE\brp{\rvX\rvY^\conj}
\end{array}$

\item If we view $\Rxy(m)$ as an \ope{analysis} of $\rvy$ in terms of $\rvx$
      (or as a \ope{projection} of $\rvy$ onto $\rvx$),
      then it would seem more natural to put the conjugate on $\rvx$ (i.e. follow Kay).
      This is what is done in Fourier analysis when projecting a function $\ff(t)$ onto the
      set of basis functions $\set{e^{i\omega n}}{\omega\in\R}$, as in
      \\\begin{align*}
        \opDTFT\brs{\rvy(n)}(\omega)
          &\eqd \inprod{\rvy(n)}{e^{i\omega n}}
          && \text{(\ope{project} $\rvy(n)$ onto $e^{i\omega n}$ for some $\omega\in\R$)}
        \\&\eqd \sum_{n\in\Z} \rvy(n) \brs{e^{+i\omega n}}^\ast
        \\&\eqd \sum_{n\in\Z} \rvy(n) e^{-i\omega n}
      \end{align*}
      But arguably, a ``projection of $\rvy$ onto $\rvx$" would better be served by the use of $\Ryx(m)$ rather than $\Rxy(m)$.

\item If we follow Kay, then there is the advantage that you also end up with the Bendat-Piersol result for $\Swxy(\omega)$.
\end{enumerate}


