%============================================================================
% LaTeX File
% Daniel J. Greenhoe
%============================================================================
%======================================
\chapter{Inner Product Spaces}
\label{sec:inprod}
%======================================
%======================================
\section{Definition and basic results}
%======================================
%--------------------------------------
\begin{definition}
\footnote{
  \citerpgc{istratescu1987}{111}{9027721823}{Definition 4.1.1},
  \citerppg{bollobas1999}{130}{131}{0521655773},
  \citerpg{haaser1991}{277}{0486665097},
  \citerp{ab}{276},
  \citorp{peano1888e}{72}
  }
\label{def:inprod}
\index{$\inprodn$}
\index{space!inner product}
%--------------------------------------
Let $\spO\eqd\linearspaceX$ be a \structe{linear space} \xref{def:vspace}.
\defbox{\begin{array}{>{\qquad}F >{\ds}rc>{\ds}l @{\qquad}C @{\qquad}D @{\qquad}D}
  \mc{7}{M}{A \structe{functional}\ifsxref{functionals}{def:functional}
  $\inprodn\in\clFxxf$ is an \fnctd{inner product} on $\spO$ if}
  \\
   1. & \inprod{\alpha\vx}{\vy}    &=& \alpha\inprod{\vx}{{\vy}}
      & \forall \vx,\vy\in X,\;\forall\alpha\in\C
      & (\prope{homogeneous})
      & and
      \\
   2. & \inprod{\vx+\vy}{\vu} &=& \inprod{\vx}{{\vu}} + \inprod{\vy}{{\vu}}
      & \forall \vx,\vy,\vu\in X
      & (\prope{additive})
      & and
      \\
   3. & \inprod{\vx    }{\vy} &=& \inprod{\vy}{\vx}^\ast
      & \forall \vx,\vy\in X
      & (\prope{conjugate symmetric}).
      & and
      \\
   4. & \inprod{\vx    }{\vx} &\ge& 0
      & \forall \vx\in X
      & (\prope{non-negative})
      & and
      \\
   5. & \inprod{\vx    }{\vx} &=& 0 \iff \vx=\vzero
      & \forall \vx\in X
      & (\prope{non-isotropic})
  \\
  \mc{7}{M}{An inner product is also called a \fnctd{scalar product}.}\\
  \mc{7}{M}{The tuple $\inprodspaceX$ is called an \structd{inner product space}.}
  %A function $\inprodn:X\times X\to\C$ is a \hid{hermitian form} on $\spO$ if it satisfies conditions 1--3.
  %A function $\inprodn:X\times X\to\C$ is a \hid{pre-inner product} on $\spO$ if it satisfies conditions 1--4.
\end{array}}
\end{definition}

%---------------------------------------
\begin{theorem}
\footnote{
  \citerpg{berberian1961}{27}{0821819127},
  \citerpg{haaser1991}{277}{0486665097}
  }
\label{thm:inprod_prop}
%\label{lem:inprod0}
%---------------------------------------
Let $\spO\eqd\inprodspaceX$ be a \structe{linear space} \xref{def:vspace}.
\thmbox{\begin{array}{Frcl@{\quad}Cl}
    1. & \inprod{\vx}{\vy+\vz}   &=& \inprod{\vx}{\vy} + \inprod{\vx}{\vz} & \forall \vx,\vy,\vz\in\setX
  \\2. & \inprod{\vx}{\alpha\vy} &=& \alpha^\ast \inprod{\vx}{\vy}         & \forall \vx,\vy\in\setX,\; \alpha\in\F
  \\3. & \inprod{\vx}{\vzero}    &=& \inprod{\vzero}{\vx} = 0              & \forall \vx\in\setX
  \\4. & \inprod{\vx-\vy}{\vz}   &=& \inprod{\vx}{\vz}-\inprod{\vy}{\vz}   & \forall \vx,\vy,\vz\in\setX
  \\5. & \inprod{\vx}{\vy-\vz}   &=& \inprod{\vx}{\vy}-\inprod{\vx}{\vz}   & \forall \vx,\vy,\vz\in\setX
  \\6. & \inprod{\vx}{\vz}       &=& \inprod{\vy}{\vz}                     & \forall \vz\in\setX\neq\setn{\vzero}  & \iff \vx=\vy
  \\7. & \inprod{\vx}{\vy}       &=& 0                                     & \forall \vx\in\setX                   &\iff \vy=\vzero
\end{array}}
\end{theorem}
\begin{proof}
{\begin{align*}
  \inprod{\vx}{\vy+\vz}
    &= \inprod{\vy+\vz}{\vx}^\ast
    && \text{by \prope{conjugate symmetric} property of $\inprodn$}
    && \text{\xref{def:inprod}}
  \\&= \brp{\inprod{\vy}{\vx} + \inprod{\vz}{\vx}}^\ast
    && \text{by \prope{additive} property of $\inprodn$}
    && \text{\xref{def:inprod}}
  \\&= \inprod{\vy}{\vx}^\ast + \inprod{\vz}{\vx}^\ast
    && \text{by \prope{distributive} property of $\invo$}
    && \text{\ifxref{normalg}{def:staralg}}
   %&& \text{\ifxref{numsys}{thm:conj}}
  \\&= \inprod{\vx}{\vy} + \inprod{\vx}{\vz}
    && \text{by \prope{conjugate symmetric} property of $\inprodn$}
    && \text{\xref{def:inprod}}
  \\
  \inprod{\vx}{\alpha\vy}
    &= \inprod{\alpha\vy}{\vx}^\ast
    && \text{by \prope{conjugate symmetric} property of $\inprodn$}
    && \text{\xref{def:inprod}}
  \\&= \brp{\alpha\inprod{\vy}{\vx}}^\ast
    && \text{by \prope{homogeneous} property of $\inprodn$}
    && \text{\xref{def:inprod}}
  \\&= \alpha^\ast \inprod{\vy}{\vx}^\ast
    && \text{by \prope{antiautomorphic} property of $\invo$}
    && \text{\ifxref{normalg}{def:staralg}}
  \\&= \alpha^\ast \inprod{\vx}{\vy}
    && \text{by \prope{conjugate symmetric} property of $\inprodn$}
    && \text{\xref{def:inprod}}
  \\
  \inprod{\vx}{\vzero}
    &= \inprod{\vzero}{\vx}^\ast
    && \text{by \prope{conjugate symmetric} property of $\inprodn$}
    && \text{\xref{def:inprod}}
  \\&= \inprod{0\cdot\vy}{\vx}^\ast
  \\&= \brp{0\cdot\inprod{\vy}{\vx}}^\ast
    && \text{by \prope{homogeneous} property of $\inprodn$}
    && \text{\xref{def:inprod}}
  \\&= 0
  \\
  \inprod{\vzero}{\vx}
    &= \inprod{0\cdot\vy}{\vx}
  \\&= \brp{0\cdot\inprod{\vy}{\vx}}
    && \text{by \prope{homogeneous} property of $\inprodn$}
    && \text{\xref{def:inprod}}
  \\&= 0
  \\
  \inprod{\vx-\vy}{\vz}
    &= \inprod{\vx+(-\vy)}{\vz}
    && \text{by definition of $+$}
  \\&= \inprod{\vx}{\vz} + \inprod{-\vy}{\vz}
    && \text{by \prope{additive} property of $\inprodn$}
    && \text{\xref{def:inprod}}
  \\&= \inprod{\vx}{\vz} - \inprod{\vy}{\vz}
    && \text{by \prope{homogeneous} property of $\inprodn$}
    && \text{\xref{def:inprod}}
  \\
  \inprod{\vx}{\vy-\vz}
    &= \inprod{\vy-\vz}{\vx}^\ast
    && \text{by \prope{conjugate symmetric} property of $\inprodn$}
    && \text{\xref{def:inprod}}
  \\&= \brp{\inprod{\vy}{\vx}-\inprod{\vz}{\vx}}^\ast
    && \text{by 4.}
  \\&= \inprod{\vy}{\vx}^\ast-\inprod{\vz}{\vx}^\ast
    && \text{by \prope{distributive} property of $\invo$}
    && \text{\ifxref{normalg}{def:staralg}}
    %&& \text{by property of complex numbers}
  \\&= \inprod{\vx}{\vy}-\inprod{\vx}{\vz}
    && \text{by \prope{conjugate symmetric} property of $\inprodn$}
    && \text{\xref{def:inprod}}
\end{align*}}

\begin{align*}
         && \inprod{\vx}{\vz}  = \inprod{\vy}{\vz} &        & \forall \vz \qquad
  \\\iff && \inprod{\vx}{\vz}  - \inprod{\vy}{\vz} &=0      & \forall \vz \qquad & \text{by property of complex numbers}
  \\\iff && \inprod{\vx-\vy}{\vz}                  &=0      & \forall \vz \qquad & \text{by 4.}
  \\\iff && \vx-\vy                                &=\vzero & \forall \vz \qquad & \text{by \prope{non-isotropic} property of $\inprodn$ \xref{def:inprod}}
\end{align*}

Proof that $\inprod{\vx}{\vy}=0  \implies \vy=\vzero$:
    \begin{enume}
      \item Suppose $\vy\neq\vzero$;
      \item Then $\inprod{\vy}{\vy}\neq\vzero$ by the \prope{non-isotropic} property of $\inprodn$ \xref{def:inprod}
      \item But because $\vy\in\setX$, the left hypothesis implies that $\inprod{\vy}{\vy}=0$.
      \item This is a \emph{contradiction}.
      \item Therefore $\vy\neq\vzero$ must be incorrect and $\vy=\vzero$ must be correct.
    \end{enume}

Proof that $\inprod{\vx}{\vy}=0  \impliedby \vy=\vzero$:
    \begin{align*}
      \inprod{\vx}{\vy}
        &= \inprod{\vx}{\vzero}
        && \text{by right hypothesis}
      \\&= 0
        && \text{by \prefp{thm:inprod_prop}}
    \end{align*}

\end{proof}

One of the most useful and widely used inequalities in analysis is the \ineqe{Cauchy-Schwarz Inequality}
(sometimes also called the \ineqe{Cauchy-Bunyakovsky-Schwarz Inequality}).
In fact, we will use this inequality shortly to prove that
every \structe{inner product space} {\bf has} a \fncte{norm}
and therefore every \structe{inner product space} {\bf is} a \structe{normed linear space}.
%---------------------------------------
\begin{theorem}[\ineqd{Cauchy-Schwarz Inequality}]
\footnote{
  \citerpg{haaser1991}{278}{0486665097},
  \citerp{ab}{278},
  %\citerpp{pedersen2000}{34}{35},
  \citorp{cauchy1821}{455},
  \citorp{bunyakovsky}{6},
  \citor{schwarz}
  }
\label{thm:cs}
\index{inequalities!Cauchy-Schwarz}
\index{inequalities!Cauchy-Bunyakovsky-Schwarz}
%---------------------------------------
Let $\inprodspaceX$ be an \structe{inner product space}
and $\abs{\cdot}\in\clFcr$ an \fncte{absolute value} function\ifsxref{algebra}{def:abs}.
Let $\normn$ be a function in $\clFfr$ such that $\norm{\vx}\eqd\sqrt{\inprod{\vx}{\vx}}$.\footnote{
The function $\normn$ is a \structe{norm} \xrefP{thm:norm=inprod}
and is called the \structe{norm induced by the inner product $\inprodn$} \xref{def:innorm}.}
\thmbox{\begin{array}{rclclC}
  \abs{\inprod{\vx}{\vy}}^2 &\le& \inprod{\vx}{\vx} \; \inprod{\vy}{\vy}
    &&
    & \forall \vx,\vy\in\setX
    \\
  \abs{\inprod{\vx}{\vy}}^2 &=& \inprod{\vx}{\vx} \; \inprod{\vy}{\vy}
    &\iff& \exists \alpha\in\F \st \vy=\alpha\vx
    & \forall \vx,\vy\in\setX
    \\
  \abs{\inprod{\vx}{\vy}}   &\le& \norm{\vx}\,\norm{\vy}
    &&
    & \forall \vx,\vy\in\setX
    \\
  \abs{\inprod{\vx}{\vy}}   &=& \norm{\vx}\,\norm{\vy}
    &\iff& \exists \alpha\in\F \st \vy=\alpha\vx
    & \forall \vx,\vy\in\setX
\end{array}
}
\end{theorem}
\begin{proof}
\begin{enumerate}
  \item Proof that $\abs{\inprod{\vx}{\vy}} \le \norm{\vx}\norm{\vy}$:
        \cittrpg{haaser1991}{278}{0486665097}  %\citepp{pedersen2000}{34}{35}
  \begin{enumerate}
    \item $\vy=\vzero$ case:
      \begin{align*}
         \abs{\inprod{\vx}{\vy}}^2
           &=   \abs{\inprod{\vx}{\vzero}}^2
           &&   \text{by $\vy=\vzero$ hypothesis}
         \\&=   \abs{\inprod{\vzero}{\vx}}^2
           &&   \text{by \prefp{def:inprod}}
         \\&=   \abs{\inprod{0\vzero}{\vx}}^2
           &&   \ifdochas{vector}{\text{by \prefp{def:vspace}}}
         \\&=   \abs{0\inprod{\vzero}{\vx}}^2
           &&   \text{by \prefp{def:inprod}}
         \\&=   0
         \\&=   \inprod{\vx}{\vx}\inprod{\vzero}{\vzero}
         \\&=   \inprod{\vx}{\vx}\inprod{\vy}{\vy}
           &&   \text{by $\vy=\vzero$ hypothesis}
      \end{align*}

    \item $\vy\ne \vzero$ case:
          Let $\lambda\eqd\frac{\inprod{\vx}{\vy}}{\inprod{\vy}{\vy}}$.
      \begin{align*}
         0
           &\le \inprod{\vx-\lambda\vy}{\vx-\lambda\vy}
           &&   \text{by \pref{def:inprod}}
         \\&=   \inprod{\vx}{\vx-\lambda\vy}
            +   \inprod{-\lambda\vy}{\vx-\lambda\vy}
           &&   \text{by \pref{def:inprod}}
         \\&=   \inprod{\vx-\lambda\vy}{\vx}^\ast
               -\lambda\inprod{\vy}{\vx-\lambda\vy}
           &&   \text{by \pref{def:inprod}}
         \\&=   \inprod{\vx}{\vx}^\ast
               +\inprod{-\lambda\vy}{\vx}^\ast
               -\lambda\inprod{\vx-\lambda\vy}{\vy}^\ast
           &&   \text{by \pref{def:inprod}}
         \\&=   \inprod{\vx}{\vx}^\ast
               -\lambda^\ast\inprod{\vy}{\vx}^\ast
               -\lambda\inprod{\vx}{\vy}^\ast
               -\lambda\inprod{-\lambda\vy}{\vy}^\ast
           &&   \text{by \pref{def:inprod}}
         \\&=   \inprod{\vx}{\vx}
               -\lambda^\ast\inprod{\vx}{\vy}
               -\lambda\inprod{\vx}{\vy}^\ast
               +\lambda\lambda^\ast\inprod{\vy}{\vy}^\ast
           &&   \text{by \pref{def:inprod}}
         \\&=   \inprod{\vx}{\vx}
            +   \cancelto{0}{\left[
                  \frac{\inprod{\vx}{\vy}}{\inprod{\vy}{\vy}}\lambda^\ast\inprod{\vy}{\vy}
                  - \lambda^\ast\inprod{\vx}{\vy}
                \right]}
            -   \frac{\inprod{\vx}{\vy}}{\inprod{\vy}{\vy}}\inprod{\vx}{\vy}^\ast
           &&   \text{by definition of $\lambda$}
         \\&=   \inprod{\vx}{\vx}
            -   \frac{1}{\inprod{\vy}{\vy}}\; \left| \inprod{\vx}{\vy}\right|^2
         \\\implies
         \left|\inprod{\vx}{\vy}\right|^2
           &\le   \inprod{\vx}{\vx} \; \inprod{\vy}{\vy}
      \end{align*}

  \end{enumerate}

  \item Proof that
      $\abs{\inprod{\vx}{\vy}}^2 = \inprod{\vx}{\vx}\, \inprod{\vy}{\vy}
       \iff
       \vy=a\vx
      $:\\
          Let $\frac{1}{a}\eqd\lambda\eqd\frac{\inprod{\vx}{\vy}}{\inprod{\vy}{\vy}}$. Then\ldots
      \begin{align*}
           &&
         \vy
           &= a\vx
         \\&\iff&
         \vx
           &= \lambda\vy
         \\&\iff&
         \vx-\lambda\vy
           &= \vzero
         \\&\iff&
         0
           &=   \inprod{\vx-\lambda\vy}{\vx-\lambda\vy}
           &&   \text{by \prefp{def:inprod}}
         \\&&&=   \inprod{\vx}{\vx-\lambda\vy}
            +   \inprod{-\lambda\vy}{\vx-\lambda\vy}
           &&   \text{by \prefp{def:inprod}}
         \\&&&=   \inprod{\vx-\lambda\vy}{\vx}^\ast
               -\lambda\inprod{\vy}{\vx-\lambda\vy}
           &&   \text{by \prefp{def:inprod}}
         \\&&&\quad\vdots
           &&   \text{(same steps as in 1(b)}
         \\&&&=   \inprod{\vx}{\vx}
            -   \frac{1}{\inprod{\vy}{\vy}}\; \left| \inprod{\vx}{\vy}\right|^2
         \\&\iff&
         \left|\inprod{\vx}{\vy}\right|^2
           &=   \inprod{\vx}{\vx} \; \inprod{\vy}{\vy}
      \end{align*}


\item Alternate proof for $\abs{\inprod{\vx}{\vy}} \le \norm{\vx}\norm{\vy}$:
      (Note: This is essentially the same proof as used by Schwarz).
      \footnote{
        \citerp{ab}{278},
        \citerp{steele}{11}
        }

  \begin{enumerate}

    \item Proof that
            $\ds\brb{a\lambda^2+b\lambda+c\ge0\qquad\forall \lambda\in\R} \quad\implies\quad \brb{b^2\le4ac}$
            (\ineqd{quadratic discriminate inequality}):

        {\color{figcolor}
        \begin{center}
        \begin{fsL}
        \setlength{\unitlength}{0.1mm}
        \begin{picture}(220,130)(-100,-30)
          %\graphpaper[10](0,0)(200,200)
          \thicklines
          \put(-100 ,   0 ){\line(1,0){200} }
          \put(   0 ,-30 ){\line(0,1){130} }
          \thicklines
          \put( 110,   0 ){\makebox(0,0)[l]{$\lambda$}}
          \qbezier(- 30,  80)(  20, -40)(  70,  80)
          {\color{red}
            \put(   0, 120 ){\makebox(0,0)[b]{$a\lambda^2+b\lambda+c$}}
            \put(  20, 110 ){\vector(1,-1){40}}
          }
        \end{picture}
        \end{fsL}
        \end{center}
        }

                Let $k\in(0,\infty)$, and $r_1,r_2\in\C$ be the roots of
           $a\lambda^2 + b\lambda + c = 0$.
        Then
        \begin{align*}
           0
             &\le a\lambda^2 + b\lambda + c
             &&   \text{by left hypothesis}
           \\&=   k(\lambda-r_1)(\lambda-r_2)
             &&   \text{by definition of $r_1$ and $r_2$}
           \\&=   k(\lambda^2 -r_1\lambda -r_2\lambda + r_1r_2)
           \\\implies\qquad
           \lambda^2 -r_1\lambda -r_2\lambda + r_1r_2
             &\ge 0
           \\\implies\qquad
           r_1 &= r_2^\ast
             && \text{because $r_1r_2\ge0$ for $\lambda=0$}
        \end{align*}

        The \hie{quadratic equation} places another constraint on $r_1$ and $r_2$:
        \begin{align*}
          \frac{b^2 + \sqrt{b^2-4ac}}{2a}
            &= r_1
            && \text{by quadratic equation}
          \\&= r_2^\ast
            && \text{by previous result}
          \\&= \left(\frac{b^2 - \sqrt{b^2-4ac}}{2a}\right)^\ast
            && \text{by quadratic equation}
        \end{align*}

        The only way for this to be true is if $b^2 \le 4ac$
        (the \hib{discriminate} is \prop{non-positive}).

    \item Proof that
      $\inprod{\vy}{\vy } \lambda^2 +2\abs{\inprod{\vx }{\vy}} \lambda + \inprod{\vx}{\vx}\ge0
       \quad\forall\lambda\in\R$:
      \begin{align*}
         0
           &\le \inprod{ \vx + \alpha \vy }{ \vx + \alpha \vy }
           &&   \text{by \prefp{def:inprod}}
         \\&=   \inprod{\vx}{\vx + \alpha \vy }
            +   \inprod{\alpha \vy }{ \vx + \alpha \vy }
           &&   \text{by \prefp{def:inprod}}
         \\&=   \inprod{\vx}{\vx + \alpha \vy }
            +   \alpha \inprod{\vy }{ \vx + \alpha \vy }
           &&   \text{by \prefp{def:inprod}}
         \\&=   \inprod{\vx + \alpha \vy }{\vx}^\ast
            +   \alpha \inprod{ \vx + \alpha \vy }{\vy }^\ast
           &&   \text{by \prefp{def:inprod}}
         \\&=   \inprod{\vx}{\vx}^\ast
            +   \inprod{\alpha \vy }{\vx}^\ast
            +   \alpha \inprod{\vx}{\vy }^\ast
            +   \alpha \inprod{\alpha \vy }{\vy }^\ast
           &&   \text{by \prefp{def:inprod}}
         \\&=   \inprod{\vx}{\vx}^\ast
            +   \alpha^\ast \inprod{\vy }{\vx}^\ast
            +   \alpha \inprod{\vx}{\vy }^\ast
            +   \alpha\alpha^\ast \inprod{\vy}{\vy }^\ast
           &&   \text{by \prefp{def:inprod}}
         \\&=   \inprod{\vx}{\vx}
            +   \alpha^\ast \inprod{\vx }{\vy}
            +   \left(\alpha^\ast \inprod{\vx}{\vy }\right)^\ast
            +   |\alpha|^2 \inprod{\vy}{\vy }
           &&   \text{by \prefp{def:inprod}}
         \\&=   \inprod{\vx}{\vx}
            +   2\Re\left(\alpha^\ast \inprod{\vx }{\vy}\right)
            +   |\alpha|^2 \inprod{\vy}{\vy }
         \\&\le \inprod{\vx}{\vx}
            +   2\abs{\alpha^\ast \inprod{\vx }{\vy}}
            +   \abs{\alpha}^2 \inprod{\vy}{\vy }
         \\&=   \inprod{\vx}{\vx}
            +   2\abs{\inprod{\vx }{\vy}} \abs{\alpha}
            +   \inprod{\vy}{\vy } \abs{\alpha}^2
         \\&=   \inprod{\vy}{\vy } \abs{\alpha}^2
            +   2\left|\inprod{\vx }{\vy}\right| \abs{\alpha}
            +    \inprod{\vx}{\vx}
         \\&=   \mcom{\inprod{\vy}{\vy }}{$a$} \lambda^2
            +   \mcom{2\abs{\inprod{\vx }{\vy}}}{$b$} \lambda
            +   \mcom{\inprod{\vx}{\vx}}{$c$}
           &&   \text{because $\lambda\eqd\abs{\alpha}\in\R$}
      \end{align*}

    \item The above equation is in the quadratic form used in the lemma of part (a).
      \begin{align*}
        \left(\mcom{2 \abs{\inprod{\vx}{\vy}} }{$b$}\right)^2
          &\le 4\mcom{\inprod{\vy}{\vy}}{$a$}\; \mcom{\inprod{\vx}{\vx}}{$c$}
          &&   \text{by the results of parts (a) and (b)}
          \\\implies
          \abs{\inprod{\vx}{\vy}}^2 \le \inprod{\vx}{\vx}\, \inprod{\vy}{\vy}
      \end{align*}
  \end{enumerate}

  \item Proof that $\abs{\inprod{\vx}{\vy}} \le \norm{\vx}\,\norm{\vy}$:\\
        This follows directly from the definition $\norm{\vx}\eqd\sqrt{\inprod{\vx}{\vx}}$.

  \item Proof that $\abs{\inprod{\vx}{\vy}} = \norm{\vx}\,\norm{\vy}$ $\quad\iff\quad \exists \alpha\in\C \st \vy=\alpha\vx$:\\
        This follows directly from the definition $\norm{\vx}\eqd\sqrt{\inprod{\vx}{\vx}}$.
\end{enumerate}
\end{proof}

%--------------------------------------
\begin{corollary}
\footnote{
  \citerpg{bollobas1999}{132}{0521655773},
  \citerpgc{ab}{279}{0120502577}{Lemma 32.4},
  \citerpg{haaser1991}{278}{0486665097}
  }
%--------------------------------------
Let $\inprodspaceX$ be an \structe{inner product space}.
\corbox{
  \text{$\inprod{\vx}{\vy}$ is \prope{continuous}\ifsxref{topology}{def:continuous} in both $\vx$ and $\vy$.}
  }
\end{corollary}
\begin{proof}
  \begin{enumerate}
    \item Let $\norm{\vx} \eqd \sqrt{\inprod{\vx}{\vx}}$.
    \item Proof:
      \begin{align*}
        \abs{\inprod{\vx+\vepsilon}{\vy} - \inprod{\vx}{\vy}}^2
          &= \abs{\inprod{\vx+\vepsilon-\vx}{\vy}}^2
          && \text{by \prope{additivity} of $\inprodn$}&&\text{\xref{def:inprod}}
        \\&= \abs{\inprod{\vepsilon}{\vy}}^2
        \\&\le \norm{\vepsilon}^2\norm{\vy}
          && \text{by \ineqe{Cauchy-Schwarz Inequality}}&&\text{\xref{thm:cs}}
      \end{align*}
    \item Alternate proof (see \citerpg{haaser1991}{278}{0486665097})
      \begin{align*}
        \abs{\inprod{\vx}{\vy}-\inprod{\vx_0}{\vy_0}}
          &= \abs{\inprod{\vx-\vx_0}{\vy-\vy_0}
                  \mcom{-\inprod{\vx_0}{\vy_0}+\inprod{\vx}{\vy_0}}{$\inprod{\vx-\vx_0}{\vy_0}$}
                  \mcom{+\inprod{\vx_0}{\vy}-\inprod{\vx_0}{\vy_0}}{$\inprod{\vx_0}{\vy-\vy_0}$}
                 }
        \\&= \abs{\inprod{\vx-\vx_0}{\vy-\vy_0} +\inprod{\vx-\vx_0}{\vy_0} +\inprod{\vx_0}{\vy-\vy_0}}
        \\&\le \abs{\inprod{\vx-\vx_0}{\vy-\vy_0}}
             + \abs{\inprod{\vx-\vx_0}{\vy_0}}
             + \abs{\inprod{\vx_0}{\vy-\vy_0}}
        \\&\le \norm{\vx-\vx_0} \; \norm{\vy-\vy_0}
             + \norm{\vx-\vx_0} \; \norm{\vy_0}
             + \norm{\vx_0} \; \norm{\vy-\vy_0}
          \quad \text{by Cauchy-Swartz ineq. \prefpo{thm:cs}}
        \\&\le \brp{\max\setn{\norm{\vx-\vx_0},\, \norm{\vy-\vy_0}}}^2
             + \brp{\norm{\vx_0}+\norm{\vy_0}}\max\setn{\norm{\vx-\vx_0},\,\norm{\vy-\vy_0}}
        \\&=   \max\setn{\norm{\vx-\vx_0},\, \norm{\vy-\vy_0}}
               \brp{1 + \max\setn{\norm{\vx-\vx_0},\,\norm{\vy-\vy_0}} }
        \\&=   \metric{\opair{\vx}{\vy}}{\opair{\vx_0}{\vy_0}}
               \brs{1 + \metric{\opair{\vx}{\vy}}{\opair{\vx_0}{\vy_0}} }
        \\&\implies \inprod{\vx}{\vy}\to\inprod{\vx_0}{\vy_0} \text{ as } \opair{\vx}{\vy}\to\opair{\vx_0}{\vy_0}
      \end{align*}
  \end{enumerate}
\end{proof}

%======================================
\section{Relationship between norms and inner products}
%======================================
%======================================
\subsection{Norms induced by inner products}
%======================================
%--------------------------------------
\begin{lemma}[\thmd{Polar Identity}]
\footnote{
  \citerpg{conway1990}{4}{0387972455},
  \citerpgc{heil2011}{27}{9780817646868}{Lemma 1.36(a)}
  }
\label{lem:||x+y||}
\label{lem:polarid}
%--------------------------------------
Let $\inprodspaceX$ be an \structe{inner product space} \xref{def:inprod}.
Let $\Re z$ represent the real part of $z\in\C$.
Let $\normn$ be a function in $\clFfr$ such that $\norm{\vx}\eqd\sqrt{\inprod{\vx}{\vx}}$.\footnote{
The function $\normn$ is a \structe{norm} \xrefP{thm:norm=inprod} and is called the \structe{norm induced by the inner product $\inprodn$}
\xref{def:innorm}.}
\lembox{
  \norm{\vx+\vy}^2 = \norm{\vx}^2 +2\Reb{\inprod{\vx}{\vy}} + \norm{\vy}^2
  \qquad\scriptstyle
  \forall \vx,\vy\in\setX
  }
\end{lemma}
\begin{proof}
\begin{align*}
   \norm{\vx+\vy}^2
     &= \inprod{\vx+\vy}{\vx+\vy}
     && \text{by definition of \fncte{induced norm}}
     && \text{\xref{thm:inducednorm}}
   \\&= \inprod{\vx}{\vx+\vy} + \inprod{\vy}{\vx+\vy}
     && \text{by \prefp{def:inprod}}
   \\&= \inprod{\vx+\vy}{\vx}^\ast + \inprod{\vx+\vy}{\vy}^\ast
     && \text{by \prefp{def:inprod}}
   \\&= \inprod{\vx}{\vx}^\ast + \inprod{\vy}{\vx}^\ast + \inprod{\vx}{\vy}^\ast + \inprod{\vy}{\vy}^\ast
     && \text{by \prefp{def:inprod}}
   \\&= \inprod{\vx}{\vx} + \inprod{\vx}{\vy} + \inprod{\vx}{\vy}^\ast + \inprod{\vy}{\vy}
     && \text{by definition of \fncte{inner product}}
     && \text{\xref{def:inprod}}
   \\&= \norm{\vx}^2 + 2\Re\inprod{\vx}{\vy} + \norm{\vy}^2
     && \text{by definition of \fncte{induced norm}}
     && \text{\xref{thm:inducednorm}}
\end{align*}
\end{proof}



%--------------------------------------
\begin{theorem}[\ineqd{Minkowski's Inequality}]
\footnote{
  \citerppgc{ab}{278}{279}{0120502577}{Theorem 32.3},
  \citer{maligranda1995},
  \citorp{minkowski1910}{115}
  }
%--------------------------------------
Let $\inprodspaceX$ be an \structe{inner product space}.
Let $\normn$ be a function in $\clFfr$ such that $\norm{\vx}\eqd\sqrt{\inprod{\vx}{\vx}}$.\footnote{
The function $\normn$ is a \structe{norm} \xrefP{thm:norm=inprod} and is called the \structe{normed induced by the inner product $\inprodn$}
\xref{def:innorm}.}
%$\norm{\vx} \eqd \sqrt{\inprod{\vx}{\vx}}$.
\thmbox{
  \norm{\vx+\vy} \le \norm{\vx}+\norm{\vy}
  \quad\sst
  \forall \vx,\vy\in\setX
  }
\end{theorem}
\begin{proof}
  \begin{align*}
    \norm{\vx+\vy}^2
      &=    \norm{\vx}^2 +2\Re{\inprod{\vx}{\vy}} + \norm{\vy}^2
      &&    \text{by \thme{Polar Identity}}
      &&    \text{\xref{lem:||x+y||}}
    \\&\le  \norm{\vx}^2 + 2\abs{\inprod{\vx}{\vy}} +\norm{\vy}^2
    \\&\le  \norm{\vx}^2 + 2\sqrt{\inprod{\vx}{\vx}} \sqrt{\inprod{\vy}{\vy}} + \norm{\vy}^2
      &&    \text{by \ineqe{Cauchy-Schwarz Inequality}}
      &&    \text{\xref{thm:cs}}
    \\&=    \brp{\norm{\vx} + \norm{\vy}}^2
  \end{align*}
\end{proof}

%======================================
\subsection{Inner products induced by norms}
%======================================
%--------------------------------------
\begin{theorem}[\thmd{induced norm}]
\footnote{
  \citerpp{ab}{278}{279},
  \citerpg{haaser1991}{278}{0486665097}
  }
\label{thm:norm=inprod}
\label{thm:inducednorm}
\index{norm!induced by inner product}
%--------------------------------------
Let $\spO\eqd\inprodspaceX$ be an \structe{inner product space} \xref{def:inprod}.
\thmbox{
  \norm{\vx}\eqd\sqrt{\inprod{\vx}{\vx}}
  \qquad\implies\qquad
  \text{$\normn$ is a \structe{norm}}
  }
\end{theorem}
\begin{proof}
For a function to be a norm, it must satisfy the four properties listed in
\prefpp{def:norm}.
\begin{enumerate}
  \item Proof that $\normn$ is a norm:
\begin{enumerate}
  \item Proof that $\norm{\vx}>0$ for $\vx\ne 0$ (\prop{non-negative}): \\
    By \prefp{def:inprod}, all inner products have this property.

  \item Proof that $\norm{\vx}=0 \iff \vx=0$ (\prop{non-isometric}): \\
    By \pref{def:inprod}, all inner products have this property.

  \item Prove $\norm{a\vx}=\abs{a}\norm{\vx}$ (\prop{homogeneous}):
  \begin{eqnarray*}
    \norm{a\vx}
      &\eqd& \sqrt{\inprod{a\vx}{a\vx}}
      = \sqrt{a a^\ast \inprod{\vx}{\vx}}
      = \sqrt{|a|^2 \inprod{\vx}{\vx}}
      = |a| \norm{\vx}
  \end{eqnarray*}

  \item Proof that $\norm{\vx+\vy}\le\norm{\vx}+\norm{\vy}$ (\prop{subadditive}):
        This is true by \ineqe{Minkowski's Inequality}\ifsxref{sums}{thm:lp_minkowski}.
\end{enumerate}

  \item Proof that every inner product space is a normed linear space:\\
    Since every inner product induces a norm, so every inner product space
    has a norm (the norm induced by the inner product)
    and is therefore a normed linear space.
\end{enumerate}
\end{proof}

\pref{thm:norm=inprod} (previous theorem)
demonstrates that in any inner product space\\$\inprodspaceX$,
the function $\norm{\vx}\eqd\sqrt{\inprod{\vx}{\vx}}$ is a norm.
That is, $\norm{\vx}$ is the \fncte{norm induced by the inner product}.
This norm is formally defined next.
%--------------------------------------
\begin{definition}
\index{norm!induced by inner product}
\label{def:norm=inprod}
\label{def:innorm}
%--------------------------------------
Let $\inprodspaceX$ be an \structe{inner product space} \xref{def:inprod}.
\defbox{\begin{array}{M}
  The \hid{norm induced by the inner product} $\inprodn$ is defined as
  \\\qquad$\norm{\vx} \eqd \sqrt{\inprod{\vx}{\vx}}$
\end{array}}
%\defbox{\begin{array}{FMD}
%  \cnto & The \hid{norm induced by the inner product} $\inprodn$ is defined as $\norm{\vx} \eqd \sqrt{\inprod{\vx}{\vx}}$ & and
%  \cntn & $\normspaceX$ is a \hid{normed linear space}.
%\end{array}}
\end{definition}

\prefpp{thm:norm=inprod} demonstrates that if a \structe{linear space} \xref{def:vspace} has an \structe{inner product} \xref{def:inprod},
then that inner product always induces a \structe{norm} \xref{def:norm},
and the relationship between the two is simply $\norm{\vx}=\sqrt{\inprod{\vx}{\vx}}$ \xref{def:innorm}.
But what about the converse?
What if a linear space has a norm---can that norm also induce an inner product?
The answer in general is ``no":
Not all norms can induce an inner product.
But a less harsh answer is ``sometimes":
Some norms {\bf can} induce inner products.
This leads to some important and interesting questions:
\begin{enumerate}
  \item How many different inner products can be induced from a single norm?
    The answer turns out to be {\bf at most} one, but maybe none \xref{thm:inprod_unique}.

  \item When a norm \emph{can} induce an inner product, what is that (unique) inner product?
  The inner product expressed in terms of the norm is given by the
  \hie{Polarization Identity} \xref{thm:polar_id}.

  \item Which norms can induce an inner product and which ones cannot?
    The answer is that norms that satisfy the \hie{parallelogram law}
    \xref{thm:parallelogram} {\bf can} induce an inner product;
    and the ones that don't, cannot \xref{thm:parallelogram}.
\end{enumerate}


%--------------------------------------
\begin{theorem}
\label{thm:inprod_unique}
\index{inner product!uniqueness}
\footnote{
  \citerp{ab}{280},
  \citerp{bollobas1999}{132},
  \citorp{jordan1935}{721}
  }
%--------------------------------------
Let $\normspaceX$ be a \structe{normed linear space} \xref{def:norm}.
%Then the norm $\normn$ is induced by {\bf at most one} inner product:
  \thmbox{
    \mcomr{\left.\begin{array}{l}
      \exists \hspace{1ex} \inprodn
      \text{ and }  \inproda{\cdot}{\cdot}
      \st \\
      \norm{\vx}^2=\inprod{\vx}{\vx}=\inproda{\vx}{\vx}
      \quad
      {\sst \forall\vx\in\setX}
    \end{array}\right\}}
    {If a norm is induced by two inner products\ldots}
    \quad\implies\quad
    \mcoml{\inprod{\vx}{\vy} = \inproda{\vx}{\vy} \quad {\scriptstyle\forall \vx,\vy\in\setX}}
          {\ldots then those two inner products are equivalent.}
  }
\end{theorem}
\begin{proof}
  \begin{align*}
    2\inprod{\vx}{\vy}
      &= \left[\, \inprod{\vx}{\vy} + \inprod{\vy}{\vx} \,\right]
       + \left[\, \inprod{\vx}{\vy} - \inprod{\vy}{\vx} \,\right]
    \\&= \left[\, \inprod{\vx}{\vy} + \inprod{\vy}{\vx} \,\right]
        +i\left[\,-i\inprod{\vx}{\vy} +i\inprod{\vy}{\vx} \,\right]
    \\&= \left[\, \inprod{\vx}{\vy} + \inprod{\vy}{\vx} \,\right]
       +i\left[\,-\inprod{i\vx}{\vy} -\inprod{\vy}{i\vx} \,\right]
    \\&= \left(
           \mcom{\left[\, \inprod{\vx}{\vy} + \inprod{\vy}{\vx} + \inprod{\vx}{\vx} + \inprod{\vy}{\vy} \,\right]}
                {$\inprod{\vx+\vy}{\vx+\vy}$}
          -\mcom{\left[\, \inprod{\vx}{\vx} + \inprod{\vy}{\vy} \,\right]}
                {residue}
         \right)
         \\&\quad
       -i\left(
           \mcom{\left[\, \inprod{i\vx}{\vy} + \inprod{\vy}{i\vx} + \inprod{i\vx}{i\vx} + \inprod{\vy}{\vy} \,\right]}
                {$\inprod{i\vx+\vy}{i\vx+\vy}$}
          -\mcom{\left[\, \inprod{i\vx}{i\vx} + \inprod{\vy}{\vy} \,\right]}
                {residue}
         \right)
    \\&= \left(
           \left[\, \inprod{\vx+\vy}{\vx+\vy} \,\right]
          -\left[\, \inprod{\vx}{\vx} + \inprod{\vy}{\vy} \,\right]
         \right)
       -i\left(
           \left[\, \inprod{i\vx+\vy}{i\vx+\vy} \,\right]
          -\left[\, \inprod{i\vx}{i\vx} + \inprod{\vy}{\vy} \,\right]
         \right)
    \\&= \left(
           \left[\, \inproda{\vx+\vy}{\vx+\vy} \,\right]
          -\left[\, \inproda{\vx}{\vx} + \inproda{\vy}{\vy} \,\right]
         \right)
       -i\left(
           \left[\, \inproda{i\vx+\vy}{i\vx+\vy} \,\right]
          -\left[\, \inproda{i\vx}{i\vx} + \inproda{\vy}{\vy} \,\right]
         \right)
    \\&= \left(
           \mcom{\left[\, \inproda{\vx}{\vy} + \inproda{\vy}{\vx} + \inproda{\vx}{\vx} + \inproda{\vy}{\vy} \,\right]}
                {$\inproda{\vx+\vy}{\vx+\vy}$}
          -\mcom{\left[\, \inproda{\vx}{\vx} + \inproda{\vy}{\vy} \,\right]}
                {residue}
         \right)
         \\&\quad
       -i\left(
           \mcom{\left[\, \inproda{i\vx}{\vy} + \inproda{\vy}{i\vx} + \inproda{i\vx}{i\vx} + \inproda{\vy}{\vy} \,\right]}
                {$\inproda{i\vx+\vy}{i\vx+\vy}$}
          -\mcom{\left[\, \inproda{i\vx}{i\vx} + \inproda{\vy}{\vy} \,\right]}
                {residue}
         \right)
    \\&= \left[\, \inproda{\vx}{\vy} + \inproda{\vy}{\vx} \,\right]
       +i\left[\,-\inproda{i\vx}{\vy} -\inproda{\vy}{i\vx} \,\right]
    \\&= \left[\, \inproda{\vx}{\vy} + \inproda{\vy}{\vx} \,\right]
        +i\left[\,-i\inproda{\vx}{\vy} +i\inproda{\vy}{\vx} \,\right]
    \\&= \left[\, \inproda{\vx}{\vy} + \inproda{\vy}{\vx} \,\right]
       + \left[\, \inproda{\vx}{\vy} - \inproda{\vy}{\vx} \,\right]
    \\&= 2\inproda{\vx}{\vy}
  \end{align*}

\end{proof}

%--------------------------------------
\begin{theorem}[\thm{Polarization Identities}]
\label{thm:polar_id}
\index{inner product!Polarization Identity}
\index{norm!Polarization Identity}
\footnote{
  \citerppgc{berberian1961}{29}{30}{0821819127}{Theorem~II.3.3},
  \citerpgc{istratescu1987}{110}{9027721823}{Proposition 4.1.5},
  \citerp{bollobas1999}{132},
  \citorp{jordan1935}{721}
  %\cithrp{ab}{280}
  }
%--------------------------------------
Let $\spO\eqd\linearspaceX$ be a linear space,
$\inprodn\in\clFxxf$ a function, and $\norm{\vx}\eqd\sqrt{\inprod{\vx}{\vx}}$.
\thmbox{\begin{array}{M}
  $\inprodspaceX$ is an inner product space  $\implies$
  \\\quad
  $\ds\mcom{4\inprod{\vx}{\vy} =
    \left\{\begin{array}{llC}
      \norm{\vx+ \vy}^2 -\norm{\vx- \vy}^2 +i\norm{\vx+i\vy}^2 -i\norm{\vx-i\vy}^2 & \text{for } \F=\C & \forall \vx,\vy\in\setX \\
      \norm{\vx+ \vy}^2 -\norm{\vx- \vy}^2                                         & \text{for } \F=\R & \forall \vx,\vy\in\setX
    \end{array}\right.
    }{inner product induced by norm}$
\end{array}}
\end{theorem}
\begin{proof}
\begin{enumerate}
  \item These follow directly from properties of \structe{bilinear functionals} \xref{thm:bform_polar}.
  \item Alternative proof for $\F=\C$ case:
\begin{align*}
  &\norm{\vx+ \vy}^2 -\norm{\vx- \vy}^2 +i\norm{\vx+i\vy}^2  -i\norm{\vx-i\vy}^2
  \\&= \mcom{\norm{\vx}^2+\norm{\vy}^2+2\Re\inprod{\vx}{\vy}}
            {$\inprod{\vx+ \vy}{\vx+ \vy}$}
     - \mcom{\left(\norm{\vx}^2+\norm{-\vy}^2+2\Re\inprod{\vx}{-\vy}\right)}
            {$\inprod{\vx- \vy}{\vx- \vy}$}
    \\\qquad&+ \mcom{i\left(\norm{\vx}^2+\norm{i\vy}^2+2\Re\inprod{\vx}{i\vy}\right)}
            {$i\inprod{\vx+i\vy}{\vx+i\vy}$}
     - \mcom{i\left(\norm{\vx}^2+\norm{-i\vy}^2+2\Re\inprod{\vx}{-i\vy}\right)}
            {$i\inprod{\vx-i\vy}{\vx-i\vy}$}
    && \text{by \prefp{lem:||x+y||}}
  \\&= \mcom{\norm{\vx}^2+\norm{\vy}^2+2\Re\inprod{\vx}{\vy}}
            {$\inprod{\vx+ \vy}{\vx+ \vy}$}
     - \mcom{\left(\norm{\vx}^2+\norm{\vy}^2-2\Re\inprod{\vx}{\vy}\right)}
            {$\inprod{\vx- \vy}{\vx- \vy}$}
    \\\qquad& + \mcom{i\left(\norm{\vx}^2+\norm{\vy}^2+2\Re\inprod{\vx}{i\vy}\right)}
                     {$i\inprod{\vx+i\vy}{\vx+i\vy}$}
     - \mcom{i\left(\norm{\vx}^2+\norm{\vy}^2-2\Re\inprod{\vx}{i\vy}\right)}
            {$i\inprod{\vx-i\vy}{\vx-i\vy}$}
    && \text{by \prefp{def:inprod}}
  \\&= 4\Re\inprod{\vx}{\vy} + 4i\Re\inprod{\vx}{i\vy}
  \\&=   \mcom{ 2\left(\inprod{\vx}{\vy}+\inprod{\vx}{\vy}^\ast \right) }
              {$4\Re\inprod{\vx}{\vy}$ }
       + \mcom{ 2i\left(\inprod{\vx}{i\vy} + \inprod{\vx}{i\vy}^\ast \right)}
              {$4i\Re\inprod{\vx}{i\vy}$}
  \\&=   2\left(\inprod{\vx}{\vy}+\inprod{\vx}{\vy}^\ast \right)
       + 2i\left(i^\ast\inprod{\vx}{\vy} +(i^{\ast\ast}) \inprod{\vx}{\vy}^\ast \right)
  \\&=   2 \left(\inprod{\vx}{\vy}+\inprod{\vx}{\vy}^\ast \right)
       + 2i\left(-i\inprod{\vx}{\vy} + i \inprod{\vx}{\vy}^\ast \right)
    && \text{by \prefp{def:inprod}}
  \\&=   2\inprod{\vx}{\vy} + 2\inprod{\vx}{\vy}^\ast
       + 2\inprod{\vx}{\vy} - 2\inprod{\vx}{\vy}^\ast
  \\&=   4\inprod{\vx}{\vy}
\end{align*}
\end{enumerate}
\end{proof}

\begin{minipage}{\tw/3}%
  \color{figcolor}
  \begin{center}
  \begin{fsL}
  \setlength{\unitlength}{\tw/340}
  \begin{picture}(300,120)(0,0)%
    %\graphpaper[10](0,0)(300,120)%
    \thicklines%
    {\color{uvect}%
      \put(   0,   0){\vector( 1, 1){100} }%
      \put( 100, 100){\vector( 1, 0){200} }%
      \put(   0,   0){\vector( 1, 0){200} }%
      \put( 200,   0){\vector( 1, 1){100} }%
      \put(  40,  50 ){\makebox(0,0)[br]{$\vy$}}%
      \put( 240,  50 ){\makebox(0,0)[br]{$\vy$}}%
      \put( 200, 100 ){\makebox(0,0)[b ]{$\vx$}}%
      \put( 100,   0 ){\makebox(0,0)[b ]{$\vx$}}%
      }%
    {\color{vector}%
      \put(   0,   0){\vector( 3, 1){300} }%
      \put( 100, 100){\vector( 1,-1){100} }%
      \put( 120,  80 ){\makebox(0,0)[bl]{$\vx-\vy$}}%
      \put( 105,  35 ){\makebox(0,0)[tl]{$\vx+\vy$}}%
      }%
  \end{picture}%
  \end{fsL}
  \end{center}
\end{minipage}%
\begin{minipage}{2\tw/3}
  In plane geometry ($\R^2$), the \hie{parallelogram law}
  states that the sum of the squares of the lengths of the sides of a \prop{parallelogram}
  is equal to the sum of the squares of the lengths of its diagonals.
  This is illustrated in the figure to the left.
\end{minipage}
\\[0.5ex]
Acutally, the parallelogram law can be generalized to
\emph{any inner product space} (not just in the plane).
And if the parallelogram law happens to hold true in a normed linear space,
then that normed linear space is actually an \structe{inner product space}.
The parallelogram law and its relation to inner product spaces is stated in the
next theorem.

%--------------------------------------
\begin{theorem}[\thmd{Parallelogram law}]
\label{thm:parallelogram}
\footnote{
  \citerp{amir1986}{8},
  \citerpg{istratescu1987}{110}{9027721823},
  \citerp{day}{151},
  \citerp{halmos}{14},
  \citerppgc{ab}{280}{281}{0120502577}{Theorem 32.6},
  \citorp{riesz1934}{36?},
  \citorpp{jordan1935}{721}{722}
  %\cithrp{bollobas1999}{132}
  }
\index{equations!parallelogram law}
%--------------------------------------
Let $\spO\eqd\inprodspaceX$
%Let there be two functions defined on $\spO$ of the form
%$\inprodn:\spO^2\to\F$
and $\norm{\vx}\eqd\sqrt{\inprod{\vx}{\vx}}$.
\thmbox{
  %\left.
  %\parbox[c][][c]{\textwidth/5}{\raggedright
    \text{$\spO$ is an inner product space}
  %  }
  %\right\}
  \quad\iff\quad
  \mcom{2\norm{\vx}^2 + 2\norm{\vy}^2 = \norm{\vx+\vy}^2 + \norm{\vx-\vy}^2}
       {\thme{parallelogram law} / \prope{von Neumann-Jordan condition}}
  \qquad\scriptstyle
  \forall \vx,\vy \in \spO
  }
\end{theorem}
\begin{proof}
\begin{enumerate}
\item Proof that
    [$\exists\inprod{\vx}{\vy}\st\norm{\vx}^2=\inprod{\vx}{\vx}$ ]
     $\implies$
    [parallelogram law is true]:

  \begin{align*}
    \norm{\vx+\vy}^2 + \norm{\vx-\vy}^2
      &=  \left[ \norm{\vx}^2 + \norm{\vy}^2 + 2\Reb{2\inprod{\vx}{\vy}}  \right]
      &&+ \left[ \norm{\vx}^2 + \norm{-\vy}^2 + 2\Reb{2\inprod{\vx}{-\vy}} \right]
      \\&\qquad\text{by \prefp{lem:||x+y||}}
    \\&=  \left[ \norm{\vx}^2 + \norm{\vy}^2 + 2\Reb{2\inprod{\vx}{\vy}}  \right]
      &&+ \left[ \norm{\vx}^2 + \norm{\vy}^2 - 2\Reb{2\inprod{\vx}{\vy}} \right]
    \\&=  2\norm{\vx}^2 + 2\norm{\vy}^2
  \end{align*}

\item Proof that
  [$\exists\inprod{\vx}{\vy}\st\norm{\vx}^2=\inprod{\vx}{\vx}$ ]
   $\impliedby$
  [parallelogram law is true]:\\
  Note that  {\bf if} an inner product exists in the norm linear
  space $(\spO,\normn )$, {\bf then} that norm linear space is actually
  an inner product space $\inprodspaceX$.
  And if it is an inner product space, then by \prefp{thm:polar_id}
  that inner product must be given by the \hib{Polarization Identity}
  \\\indentx
  $\inprod{\vx}{\vy}=\norm{a\vx+\vy}^2 - \norm{a\vx-\vy}^2 + i\norm{a\vx+i\vy}^2 -i\norm{a\vx-i\vy}^2$.
  \\
  Therefore, here we must use the parallelogram law to show that the bilinear function
  $\ff(\vx,\vy)\eqd\inprod{\vx}{\vy}$
  given on the left hand side of the ``$=$" relation is indeed an inner product---
  that is, that it satisfies the requirements of \prefp{def:inprod}.

  \begin{enumerate}
    \item Proof that $\inprod{\vx}{\vx} \ge 0$ (\prop{non-negative}):
      \begin{align*}
        4\inprod{\vx}{\vx}
          &\eqd \norm{\vx+\vx}^2 - \cancelto{0}{\norm{\vx-\vx}^2} + i\norm{\vx+i\vx}^2 -i\norm{\vx-i\vx}^2
          &&    \text{by \thme{Polarization Identity}}
        \\&=    \norm{2\vx}^2 - \vzero + i\left(\, \norm{\vx+i\vx}^2 -\norm{\vx-i\vx}^2\,\right)
          &&    \text{by \prefp{def:norm}}
        \\&=    \abs{2}^2\norm{\vx}^2 + i\left(\, \norm{\vx+i\vx}^2 -\abs{i}\norm{\vx-i\vx}^2\,\right)
        \\&=    4\norm{\vx}^2 + i\cancelto{0}{\left(\, \norm{\vx+i\vx}^2 -\norm{i\vx+\vx}^2\,\right)}
          &&    \text{by \prefp{def:norm}}
        \\&=    4\norm{\vx}^2
        \\&\ge  0
          &&    \text{by \prefp{def:norm}}
      \end{align*}

    \item Proof that $\inprod{\vx}{\vx} = 0 \,\iff\, \vx=\vzero$ (\prop{non-isotropic}):
      \begin{align*}
        4\inprod{\vx}{\vx}
          &=    4\norm{\vx}^2
          &&    \text{by result of part (a)}
        \\&=    0 \qquad\iff\qquad \vx=\vzero
          &&    \text{by \prefp{def:norm}}
      \end{align*}

    \item Proof that $\inprod{\vx+\vu}{\vy}=\inprod{\vx}{\vy}+\inprod{\vu}{\vy}$ (additive):\cittrpg{ab}{281}{0120502577}
      \begin{align*}
        4\inprod{\vx+\vy}{\vz}
          &= 8\inprod{\frac{\vx+\vy}{2}}{\vz}
          && \text{by \prefp{def:inprod}}
          %
        \\&= 2\norm{\frac{\vx+\vy}{2}+\vz}^2 -  2\norm{\frac{\vx+\vy}{2}-\vz}^2
          \\&\quad+ 2i\norm{\frac{\vx+\vy}{2}+\vz}^2 - 2i\norm{\frac{\vx+\vy}{2}-i\vz}^2
          && \text{by \thme{Polarization Identity}}
        \\&= \left(2\norm{\frac{\vx+\vy}{2}+\vz}^2 + 2\norm{\frac{\vx-\vy}{2}}^2 \right)
          \\&\quad -\left(2\norm{\frac{\vx+\vy}{2}-\vz}^2 + 2\norm{\frac{\vx-\vy}{2}}^2 \right)
          \\&\quad +\left(2i\norm{\frac{\vx+\vy}{2}+\vz}^2 + 2i\norm{\frac{\vx-\vy}{2}}^2 \right)
          \\&\quad -\left(2i\norm{\frac{\vx+\vy}{2}-i\vz}^2 + 2i\norm{\frac{\vx-\vy}{2}}^2 \right)
        \\&= \left(\norm{\vx+\vz}^2 + \norm{\vy+\vz}^2 \right)
            -\left(\norm{\vx-\vz}^2 + \norm{\vy-\vz}^2 \right)
          \\&\quad +\left(i\norm{\vx+\vz}^2 + i\norm{\vy+\vz}^2 \right)
            -\left(i\norm{\vx-i\vz}^2 + i\norm{\vy-i\vz}^2 \right)
          && \text{by parallelogram law}
        \\&= \left(\norm{\vx+\vz}^2 -\norm{\vx-\vz}^2 + i\norm{\vx+\vz}^2 -i\norm{\vx-i\vz}^2 \right)
          \\&\quad +\left(\norm{\vy+\vz}^2 -\norm{\vy-\vz}^2 + i\norm{\vy+\vz}^2 -i\norm{\vy-i\vz}^2 \right)
        \\&= 4\inprod{\vx}{\vz} + 4\inprod{\vy}{\vz}
          && \text{by Polarization Identity}
      \end{align*}

    \item Proof that $\inprod{\vx}{\vy}=\inprod{\vy}{\vy}^\ast$ (\prope{conjugate symmetric}):
      \begin{align*}
        4\inprod{\vx}{\vy}
          &\eqd \norm{\vx+\vy}^2 - \norm{\vx-\vy}^2 + i\norm{\vx+i\vy}^2 -i\norm{\vx-i\vy}^2
          &&    \text{by Polarization Identity}
        \\&=    \norm{\vy+\vx}^2 - \norm{\vy-\vx}^2 + i\norm{i(\vy-i\vx)}^2 -i\norm{-i(\vy+i\vx)}^2
          &&    \ifdochas{vector}{\text{by \prefp{def:vspace}}}
        \\&=    \norm{\vy+\vx}^2 - \norm{\vy-\vx}^2 + i\norm{\vy-i\vx}^2 -i\norm{\vy+i\vx}^2
          &&    \text{by \prefp{def:norm}}
        \\&=    \left( \norm{\vy+\vx}^2 - \norm{\vy-\vx}^2 - i\norm{\vy-i\vx}^2 +i\norm{\vy+i\vx}^2 \right)^\ast
        \\&=    \left( \norm{\vy+\vx}^2 - \norm{\vy-\vx}^2 + i\norm{\vy+i\vx}^2  - i\norm{\vy-i\vx}^2 \right)^\ast
        \\&\eqd 4\inprod{\vy}{\vx}^\ast
          &&    \text{by \thme{Polarization Identity}}
      \end{align*}

    \item Proof that $\inprod{\alpha\vx}{\vy}=\alpha\inprod{\vx}{\vy}$ (\prope{homogeneous}): \cittrpg{ab}{138}{0120502577}

      \begin{enumerate}
        \item Proof that $\inprod{\alpha\vx}{\vy}$ is linear in $\alpha$:
          \begin{align*}
            0
              &\le \abs{\norm{\alpha\vx+\vy}-\norm{\beta\vx+\vy}}
              &&   \ifdochas{algebra}{\text{by \prefp{def:abs}}}
            \\&\le \abs{\norm{ (\alpha\vx+\vy) -(\beta\vx+\vy) }}
              &&   \ifdochas{vsnorm}{\text{by \prefp{thm:shortest_dist}}}
            \\&\le \abs{\norm{ (\alpha-\beta)\vx }}
          \end{align*}
          This implies that as $a\rightarrow \beta$,
          $\norm{\alpha\vx+\vy} \rightarrow \norm{\beta\vx+\vy}$,
          which by definition implies that $\norm{\alpha\vx+\vy}$ linear in $\alpha$.
          And by the parallelogram law, $\inprod{\alpha\vx}{\vy}$ is also linear in $\alpha$.

        \item Proof that $\inprod{n\vx}{\vy}=n\inprod{\vx}{\vy}$ for $n\in\Z$ (integer case):
          \begin{enumerate}
            \item Proof for $n=\pm 1$:
              \begin{align*}
                \inprod{n\vx}{\vy}
                  &= \inprod{\pm1\vx}{\vy}
                  && \text{by $n=\pm1$ hypothesis}
                \\&= \pm1\inprod{\vx}{\vy}
                  && \text{by definition of \fncte{inner product}}
                  && \text{\xref{def:inprod}}
                \\&= n\inprod{\vx}{\vy}
                  && \text{by $n=\pm1$ hypothesis}
              \end{align*}

            \item Proof for $n=0$:
              \begin{align*}
                \inprod{n\vx}{\vy}
                  &= \inprod{0\vx}{\vy}
                  && \text{by $n=0$ hypothesis}
                \\&= \inprod{\vx-\vx}{\vy}
                \\&= \inprod{\vx}{\vy} + \inprod{-1\vx}{\vy}
                \\&= \inprod{\vx}{\vy} -1 \inprod{\vx}{\vy}
                \\&= \inprod{\vx}{\vy} - \inprod{\vx}{\vy}
                \\&= 0\inprod{\vx}{\vy}
                \\&= n\inprod{\vx}{\vy}
                  && \text{by $n=0$ hypothesis}
              \end{align*}

            \item Proof for $n=\pm 2$:
              \begin{align*}
                \inprod{n\vx}{\vy}
                  &= \inprod{\pm2\vx}{\vy}
                  && \text{by $n=\pm1$ hypothesis}
                \\&= \inprod{\pm(\vx+\vx)}{\vy}
                \\&= \pm\inprod{\vx+\vx}{\vy}
                  && \text{by definition of \fncte{inner product}}
                  && \text{\xref{def:inprod}}
                \\&= \pm\left(\inprod{\vx}{\vy} + \inprod{\vx}{\vy}\right)
                  && \text{by additive property}
                \\&= \pm2\inprod{\vx}{\vy}
                \\&= n\inprod{\vx}{\vy}
                  && \text{by $n=\pm2$ hypothesis}
              \end{align*}

            \item Proof that [$n$ case] $\implies$ [$n\pm1$ case]:
              \begin{align*}
                \inprod{(n\pm1)\vx}{\vy}
                  &= \inprod{n\vx \pm1\vx}{\vy}
                \\&= \inprod{n\vx}{\vy} + \inprod{\pm1\vx}{\vy}
                  && \text{by additive property}
                \\&= n\inprod{\vx}{\vy} \pm1 \inprod{\vx}{\vy}
                  && \text{by left hypothesis}
                \\&= (n\pm1)\inprod{\vx}{\vy}
              \end{align*}
          \end{enumerate}

        \item Proof that $\inprod{q\vx}{\vy}=q\inprod{\vx}{\vy}$ for $q\in\Q$ (rational number case):
          \begin{align*}
            \frac{n}{m}\inprod{\vx}{\vy}
              &= \frac{n}{m}\inprod{\frac{m}{m}\vx}{\vy}
              && \text{where $n,m\in\Z$ and $m\ne 0$}
            \\&= \frac{nm}{m}\inprod{\frac{1}{m}\vx}{\vy}
              && \text{by previous result}
            \\&= \frac{m}{m}\inprod{\frac{n}{m}\vx}{\vy}
              && \text{by previous result}
            \\&= \inprod{\frac{n}{m}\vx}{\vy}
          \end{align*}

        \item Proof that $\inprod{r\vx}{\vy}=r\inprod{\vx}{\vy}$ for all $r\in\R$ (real number case):\\
          Because $\Q$ is dense in $\R$ and because $\norm{\alpha\vx+\vy}$ is
          continuous in $\alpha$,
          so $\inprod{\alpha\vx}{\vy}=\alpha\inprod{\vx}{\vy}$ for all $\alpha\in\R$.

        \item Proof that $\inprod{c\vx}{\vy}=c\inprod{\vx}{\vy}$ for all $c\in\C$ (complex number case):\\
          No proof at this time. \attention
      \end{enumerate}
    \end{enumerate}
  \end{enumerate}
\end{proof}

%--------------------------------------
\begin{remark}
\footnote{
  \citerpp{loomis1953}{23}{24},
  \citerpg{kubrusly2001}{317}{0817641742}
  }
%--------------------------------------
The inner product has already been defined in \prefpp{def:inprod} as a bilinear function that
is \prope{non-negative}, \prope{non-isotropic}, \prope{homogeneous}, \prope{additive}, and \prope{conjugate symmetric}.
However, given a normed linear space, we could alternatively define the inner product using
the \hie{parallelogram law} \xrefP{thm:parallelogram} together with
the \thme{Polarization Identity} \xrefP{thm:polar_id}.
Under this new definition, an inner product \emph{exists} if the \prop{parallelogram law} is satisfied,
and is \emph{specified}, in terms of the norm, by the \prop{Polarization Identity}.
\end{remark}

Of the uncountably infinite number of $\splpF$ norms, only the norm for $p=2$ induces an inner product
(\pref{prop:lp_norm2ip}, next).
%--------------------------------------
\begin{proposition}
\footnote{
  \citerppgc{kubrusly2001}{318}{319}{0817641742}{Example 5B}
  }
\label{prop:lp_norm2ip}
%--------------------------------------
Let $\norm{\seqxZ{x_n}}_p$ be the $\splpF$ norm\ifsxref{seq}{def:lp_norm} of the sequence $\seqn{x_n}$ in the space $\splpF$.
\propbox{
  \text{$\ds\norm{\seqn{x_n}}_p$ induces an inner product}  \qquad\iff\qquad p=2
  }
\end{proposition}
\begin{proof}
\begin{enumerate}
  \item Proof that $\normn_p$ induces an inner product $\impliedby$ $p=2$
        (using the \hie{Parallelogram law} \prefpo{thm:parallelogram}):
    \begin{align*}
      &\norm{\vx+\vy}_p^2 + \norm{\vx-\vy}_p^2
      \\&= \norm{\vx+\vy}_2^2 + \norm{\vx-\vy}_2^2
        && \text{by right hypothesis}
      \\&= \brp{\sum_{n\in\Z} \abs{x_n+y_n}^2}^\frac{2}{2} +
           \brp{\sum_{n\in\Z} \abs{x_n-y_n}^2}^\frac{2}{2}
        && \text{by definition of $\normn_p$}
      \\&= \sum_{n\in\Z} (x_n+y_n)(x_n+y_n)^\ast +
           \sum_{n\in\Z} (x_n-y_n)(x_n-y_n)^\ast
      \\&= \sum_{n\in\Z} \brp{\abs{x_n}^2 + \abs{y_n}^2 + 2\Re(x_ny_n)} +
           \sum_{n\in\Z} \brp{\abs{x_n}^2 + \abs{y_n}^2 - 2\Re(x_ny_n)}
      \\&= 2\sum_{n\in\Z} \abs{x_n}^2 + 2\sum_{n\in\Z}\abs{y_n}^2
      \\&= 2\norm{\vx}_2^2 + 2\norm{\vy}_2^2
        && \text{by definition of $\normn_p$}
      \\&= 2\norm{\vx}_p^2 + 2\norm{\vy}_p^2
        && \text{by right hypothesis}
      \\&\implies \text{$\normn_2$ induces an inner product}
        && \text{by \prefp{thm:parallelogram}}
    \end{align*}

  \item Proof that $\normn_p$ induces an inner product $\implies$ $p=2$:
    \begin{enumerate}

      \item Let $\vx\eqd\opair{1}{0}$ and $\vy\eqd\opair{0}{1}$. Then
            \footnote{\url{http://groups.google.com/group/sci.math/msg/531b1173f08871e9}}\\
        \begin{align*}
          \norm{\vx+\vy}_p^2 + \norm{\vx-\vy}_p^2
            &= \brp{\sum_{n\in\Z} \abs{x_n+y_n}^p}^\frac{2}{p} +
               \brp{\sum_{n\in\Z} \abs{x_n-y_n}^p}^\frac{2}{p}
            && \text{by definition of $\normn_p$}
          \\&= \brp{\abs{1+0}^p + \abs{0+1}^p}^\frac{2}{p} +
               \brp{\abs{1-0}^p + \abs{0-1}^p}^\frac{2}{p}
            && \text{by definitions of $\vx$ and $\vy$}
          \\&= 2^\frac{2}{p} + 2^\frac{2}{p}
          \\&= 2\cdot2^\frac{2}{p}
          \\
          2\norm{\vx}_p^2 + 2\norm{\vy}_p^2
            &= 2\brp{\sum_{n\in\Z} \abs{x_n}^p}^\frac{2}{p} +
               2\brp{\sum_{n\in\Z} \abs{y_n}^p}^\frac{2}{p}
            && \text{by definition of $\normn_p$}
          \\&= 2\brp{\abs{1}^p + \abs{0}^p}^\frac{2}{p} +
               2\brp{\abs{1}^p + \abs{0}^p}^\frac{2}{p}
            && \text{by definitions of $\vx$ and $\vy$}
          \\&= 2 + 2
          \\&= 4
          \\
          2\cdot2^\frac{2}{p} = 4
            &= \iff 2^\frac{2}{p} = 2
          \\&= \implies p=2
        \end{align*}

      \item Proof that $2^{2/p}$ is monotonic decreasing in $p$ (and so $p=2$ is the only solution):\\
        \begin{minipage}{8\tw/16}
        \begin{align*}
          \deriv{}{p} 2^\frac{2}{p}
            &= \deriv{}{p} e^{\ln 2^\frac{2}{p}}
          \\&= \brp{e^{\ln 2^\frac{2}{p}}} \deriv{}{p} \ln 2^\frac{2}{p}
          \\&= \brp{2^\frac{2}{p}} \deriv{}{p} \brp{2\ln2} \frac{1}{p}
          \\&= \brp{2^\frac{2}{p}} 2\ln2 \brp{ -\frac{1}{p^2}}
          \\&< 0 \qquad \forall p\in\opair{0}{\infty}
        \end{align*}
        \end{minipage}
        \begin{minipage}{7\tw/16}
          \begin{center}%
          \begin{fsL}%
          \setlength{\unitlength}{\tw/240}%
          \begin{picture}(140,140)(-20,-20)%
          %{\color{graphpaper}\graphpaper[10](0,0)(100,100)}%
            \thinlines%
            \color{axis}%
              \put(0,0){\line(0,1){110}}%
              \put(0,0){\line(1,0){110}}%
              \put(115,0){\makebox(0,0)[l]{$p$}}%
              %\put(0,115){\makebox(0,0)[b]{$v$}}%
            \thicklines%
            \color{blue}%
              \qbezier(10,100)(10,30)(100,30)%
              \put(50,40){\makebox(0,0)[bl]{$\ds 2^\frac{2}{p}$}}%
            \color{red}%
              \qbezier[20](10,0)(10,50)(10,100)%
              \qbezier[5](0,100)(5,100)(10,100)%
              \put(-5,100){\makebox(0,0)[r]{$4$}}%
              \put(10,-5){\makebox(0,0)[t]{$1$}}%
              %
              \qbezier[20](20,0)(20,30)(20,60)%
              \qbezier[5](0,60)(10,60)(20,60)%
              \put(-5,60){\makebox(0,0)[r]{$2$}}%
              \put(20,-5){\makebox(0,0)[t]{$2$}}%
              %
              \qbezier[25](0,25)(50,25)(100,25)%
              \put(-5,25){\makebox(0,0)[r]{$1$}}%
          \end{picture}%
          \end{fsL}%
          \end{center}%
        \end{minipage}
  \end{enumerate}
\end{enumerate}
\end{proof}

%--------------------------------------
\begin{example}
\footnote{
  \citerpg{bachman2002}{23}{9780387988993}
  }
\index{norm!$l_1$}  \index{norm!taxi cab}
\index{norm!$l_2$}  \index{norm!Euclidean}
\index{norm!$l_\infty$} \index{norm:sup}
%--------------------------------------
\exbox{\begin{array}{>{\text{The norm}}ll>{$\bf}l<{$}l<{\text{induce an inner product.}}}
    & \mcom{\norm{\vx} \eqd \sum_{i=1}^n \abs{x_i}                        }{\prop{$l_1$-norm} or \prop{taxi cab norm} } & does not &
  \\& \mcom{\norm{\vx} \eqd \left(\sum_{i=1}^n \abs{x_i}^2\right)^\frac{1}{2} }{\prop{$l_2$-norm} or \prop{Euclidean norm}} & does     &
  \\& \mcom{\norm{\vx} \eqd \max\set{\abs{x_i}}{\scriptstyle i=1,\dots,n} }{\prop{$l_\infty$-norm} or \prop{sup norm} } & does not &
\end{array}}
\end{example}
\begin{proof}
\begin{align*}
  \intertext{1. Taxi-cab norm case (counter-example): $\vx=[1,\,1],\;\vy=[1,\,-1]$}
  \norm{\vx+\vy}^2 + \norm{\vx-\vy}^2
    &= \norm{[1,\,1]+[1,\,-1]}^2 + \norm{[1,\,1]-[1,\,-1]}^2
    && \text{by definition of $\vx$ and $\vy$}
  \\&= \norm{[2,\,0]}^2 + \norm{[0,\,2]}^2
  \\&= (\abs{2}+\abs{0})^2 + (\abs{0}+\abs{2})^2
    && \text{by definition of $\normn$}
  \\&= 8
  \\&\ne  16
  \\&= 8 + 8
  \\&= 2(\abs{1}+\abs{1})^2 + 2(\abs{1}+\abs{-1})^2
  \\&= 2\norm{[1,\,1]}^2 + 2\norm{[1,\,-1]}^2
    && \text{by definition of $\normn$}
  \\&= 2\norm{\vx}^2 + 2\norm{\vy}^2
    && \text{by definition of $\vx$ and $\vy$}
  \\
  \intertext{2. Euclidean norm case:}
  \norm{\vx+\vy}^2 + \norm{\vx-\vy}^2
    &= \left(\sum_{i=1}^n \abs{x_i + y_i}^2 \right) +
       \left(\sum_{i=1}^n \abs{x_i - y_i}^2 \right)
    && \text{by definition of $\normn$}
  \\&= \sum_{i=1}^n \Big( \abs{x_i + y_i}^2 + \abs{x_i - y_i}^2 \Big)
  \\&= \sum_{i=1}^n \Big[ (x_i + y_i)(x_i + y_i)^\ast + (x_i - y_i)(x_i - y_i)^\ast \Big]
    && \ifdochas{numsys}{\text{by \prefp{thm:C_abs}}}
  \\&= \sum_{i=1}^n \Big[ (x_ix_i^\ast + y_iy_i^\ast + x_iy_i^\ast + x_i^\ast y_i)
     +                    (x_ix_i^\ast + y_iy_i^\ast - x_iy_i^\ast - x_i^\ast y_i) \Big]
  \\&= \sum_{i=1}^n \Big[ 2\abs{x_i}^2 + 2\abs{y_i}^2 \Big]
  \\&= 2\sum_{i=1}^n \abs{x_i}^2 + 2 \sum_{i=1}^n \abs{y_i}^2
    && \ifdochas{numsys}{\text{by \prefp{thm:C_abs}}}
  \\&= 2\norm{\vx}^2 + 2\norm{\vy}^2
    && \text{by definition of $\normn$}
  \\
  \intertext{3. Sup norm case(counter-example): $\vx=[1,\,1],\;\vy=[1,\,-1]$}
  \norm{\vx+\vy}^2 + \norm{\vx-\vy}^2
    &= \norm{[1,\,1]+[1,\,-1]}^2 + \norm{[1,\,1]-[1,\,-1]}^2
    && \text{by definition of $\vx$ and $\vy$}
  \\&= \norm{[2,\,0]}^2 + \norm{[0,\,2]}^2
  \\&= \left(\max\setn{\abs{2},\,\abs{0}}\right)^2 +
       \left(\max\setn{\abs{0},\,\abs{2}}\right)^2
    && \text{by definition of $\normn$}
  \\&= 2^2 + 2^2
  \\&= 8
  \\&\ne  4
  \\&= 2\cdot 1^2 + 2\cdot 1^2
  \\&= 2\left(\max\setn{\abs{1},\,\abs{1}}\right)^2
     + 2\left(\max\setn{\abs{1},\,\abs{-1}}\right)^2
  \\&= 2\norm{[1,\,1]}^2 + 2\norm{[1,\,-1]}^2
    && \text{by definition of $\normn$}
  \\&= 2\norm{\vx}^2 + 2\norm{\vy}^2
    && \text{by definition of $\vx$ and $\vy$}
\end{align*}
\end{proof}

%=======================================
\section{Orthogonality}
%=======================================
%---------------------------------------
\begin{definition}
\label{def:kdelta}
%---------------------------------------
\defbox{\begin{array}{M}
The \fnctd{Kronecker delta function} $\hxs{\kdelta_n}$ is defined as
  \qquad
  $\ds\kdelta_n\eqd
    \brbl{\begin{array}{cMD}
      1 & for $n=0$ & and \\
      0 & otherwise
    \end{array}}
  \qquad\scy
  \forall n\in\Z$
\end{array}}
\end{definition}

%--------------------------------------
\begin{definition}
\label{def:orthog}
%\footnote{
%  \citer{james1945}
%  }
\index{orthogonality!inner product space}
%--------------------------------------
Let $\inprodspaceX$ be an \structe{inner product space} \xref{def:inprod}.
\defboxt{
  Two vectors $\vx$ and $\vy$ in $\setX$ are \propd{orthogonal} if
  \\\indentx$\ds
    \inprod{\vx}{\vy}=
      \brbl{\begin{array}{lM}
        0            & for $\vx\neq\vy$\\
        c\in\F\setd0 & for $\vx=\vy$
      \end{array}}$
  \\
  The notation $\vx\orthog\vy$ implies $\vx$ and $\vy$ are \prope{orthogonal}.
  \\
  A set $\setY\in\psetX$ is \propd{orthogonal} if $\vx\orthog\vy\quad{\scy\forall\vx,\vy\in\setY}$.
  \\
  A set $\setY$ is \propd{orthonomal} if it is \prope{orthogonal} and $\inprod{\vy}{\vy}=1\quad{\scy\forall\vy\in\setY}$.
  \\
  A sequence $\seqxZ{\vx_n\in\setX}$ is \propd{orthogonal} if $\inprod{\vx_n}{\vx_m}=c\kdelta_{nm}$ for some $c\in\R\setd0$.
  \\
  A sequence $\seqxZ{\vx_n\in\setX}$ is \propd{orthonormal} if $\inprod{\vx_n}{\vx_m}=\kdelta_{nm}$.
  }
\end{definition}

The definition of the orthogonality relation $\orthog$
has several immediate consequences (next theorem):
%--------------------------------------
\begin{theorem}
\footnote{
  \citerp{james1945}{292},
  \citerpg{drljevic1989}{232}{9971506661}
  }
%--------------------------------------
Let $\inprodspaceX$ be an \structe{inner product space}.
\thmbox{\begin{array}{FlCD}
  1. & \vx \orthog \vx \iff \vx=\vzero
     & \forall \vx\in\setX
     \\
  2. & \vx \orthog \vy \implies \alpha\vx\orthog\vy
     & \forall \alpha\in\R, \vx,\vy\in\setX
     & (\prope{homogeneous})
     \\
  3. & \vx \orthog \vy \iff \vy \orthog \vx
     & \forall \vx,\vy\in\setX
     & (\prope{symmetry})
     \\
  4. & \vx \orthog \vy \text{ and } \vy \orthog \vz \implies \vx\orthog(\vy + \vz)
     & \forall \vx,\vy,\vz\in\setX
     & (\prope{additive})
     \\
  5. & \exists \beta\in\R \st \vx \orthog (\beta\vx+\vy)
     & \forall \vx\in\setX\setd\vzero,\, \vy\in\setX
\end{array}}
\end{theorem}

%--------------------------------------
\begin{theorem}
\label{thm:vsinprod_zero}
%\footnote{
  %djg
 % }
%--------------------------------------
Let $\inprodspaceX$ be an \structe{inner product space}.
\thmbox{
  \brbr{\begin{array}{>{\scy}rrclD}
    1. & \inprod{\vx}{\vy} &=& 0 & and\\
    2. & \vx + \vy &=& \vzero
  \end{array}}
  \qquad\iff\qquad
  \brbl{\begin{array}{>{\scy}rrclD}
    1. & \vx &=& \vzero & and\\
    2. & \vy &=& \vzero
  \end{array}}
  \qquad\scy
  \forall \vx,\vy\in\setX
  }
\end{theorem}
\begin{proof}
\begin{enumerate}
  \item Proof that $\vx=\vy=\vzero$:
    \begin{align*}
      0
        &= \inprod{\vzero}{\vzero}
        && \text{by \prope{non-isotropic} property of $\inprodn$ \xref{def:inprod}}
      \\&= \inprod{\vx+\vy}{\vx+\vy}
        && \text{by left hypothesis 2}
      \\&= \inprod{\vx}{\vx+\vy}+\inprod{\vy}{\vx+\vy}
        && \text{by \prope{additive} property of $\inprodn$ \xref{def:inprod}}
      \\&= \inprod{\vx}{\vx}+\inprod{\vx}{\vy}+\inprod{\vx}{\vy}^\ast+\inprod{\vy}{\vy}
        && \text{by \prope{conjugate symmetric} and \prope{additive} properties of $\inprodn$} % \xref{def:inprod}}
      \\&= \mcom{\inprod{\vx}{\vx}}{$\ge0$}+0+0+\mcom{\inprod{\vy}{\vy}}{$\ge0$}
        && \text{by left hypothesis 1}
      \\&\implies \vx=\vzero \text{ and } \vy=\vzero
        && \text{by \prope{non-negative} and \prope{non-isotropic} properties of $\inprodn$} % \xref{def:inprod}}
    \end{align*}

  \item Proof that $\inprod{\vx}{\vy}=0$:
    \begin{align*}
      \inprod{\vx}{\vy}
        &= \inprod{\vzero}{\vzero}
        && \text{by right hypotheses}
      \\&= 0
        && \text{by \prope{non-isotropic} property of $\inprodn$ \xref{def:inprod}}
    \end{align*}

  \item Proof that $\vx + \vy = \vzero$:
    \begin{align*}
      \vx + \vy
        &= \vzero + \vzero
        && \text{by right hypotheses}
      \\&= \vzero
    \end{align*}
\end{enumerate}
\end{proof}

The \ineqe{triangle inequality for vectors}
in a \structe{normed linear space}\ifsxref{vsnorm}{thm:norm_tri}
demonstrates that
  \\\indentx\tbox{$\ds\norm{\sum_{n=1}^\xN \vx_n} \le \sum_{n=1}^\xN \norm{\vx_n}.$}
\hfill\begin{minipage}{\tw-50mm}
  The \thme{Pythagorean Theorem} (next) demonstrates that this
  {\em inequality} becomes {\em equality} when the set $\setn{\vx_n}$ is \prope{orthogonal}.
\end{minipage}\\
%--------------------------------------
\begin{theorem}[\thmd{Pythagorean Theorem}]
\footnote{
  %\citerp{pinsky2002}{305},
  \citerppgc{ab}{282}{283}{0120502577}{Theorem 32.7},
  \citerpgc{kubrusly2001}{324}{0817641742}{Proposition 5.8},
  \citerppgc{bollobas1999}{132}{133}{0521655773}{Theorem 3}
  }
\label{thm:pythag}
%--------------------------------------
Let $\setxn{\vx_n\in\setX}$ be a set of vectors in an \structe{inner product space} \xref{def:inprod} $\inprodspaceX$ and let
$\norm{\vx}\eqd\sqrt{\inprod{\vx}{\vx}}$\ifsxref{vsinprod}{def:innorm}.
\thmbox{\begin{array}{>{\ds}c}
  \text{$\setn{\vx_n}$ is \prope{orthogonal}}
  \qquad\iff\qquad
  \norm{\sum_{n=1}^\xN \vx_{n}}^2 = \sum_{n=1}^\xN \norm{\vx_n}^2
  \qquad\scriptstyle
  \forall \xN\in\Zp
  \end{array}}
\end{theorem}
\begin{proof}
  1. Proof for ($\implies$) case:
    \begin{align*}
      \norm{\sum_{n=1}^\xN \vx_n}^2
        &= \inprod{\sum_{n=1}^\xN \vx_n}{\sum_{m=1}^\xN \vx_m}
        && \text{by def. of $\normn$}
        && \text{\ifxref{vsnorm}{def:norm}}
      \\&= \sum_{n=1}^\xN \sum_{m=1}^\xN \inprod{\vx_n}{\vx_m}
        && \text{by def. of $\inprodn$}
        && \text{\ifxref{vsinprod}{def:inprod}}
      \\&= \sum_{n=1}^\xN \sum_{m=1}^\xN \inprod{\vx_n}{\vx_m}\kdelta_{n-m}
        && \text{by left hypothesis}
      \\&= \sum_{n=1}^\xN \inprod{\vx_n}{\vx_n}
        && \text{by def. of $\kdelta$}
        && \text{\xref{def:kdelta}}
      \\&= \sum_{n=1}^\xN \norm{\vx_n}^2
        && \text{by def. of $\normn$}
        && \text{\ifxref{vsnorm}{def:norm}}
    \end{align*}

  2. Proof for ($\impliedby$) case:
    \begin{align*}
      &4\inprod{\vx}{\vy}
      \\&= \norm{\vx+ \vy}^2 -\norm{\vx- \vy}^2 +i\norm{\vx+i\vy}^2 -i\norm{\vx-i\vy}^2
        \quad\text{by \thme{polarization identity} \xref{thm:polar_id}}
      \\&= \left(\norm{\vx}^2 + \norm{  \vy}^2\right)
         - \left(\norm{\vx}^2 + \norm{- \vy}^2\right)
         +i\left(\norm{\vx}^2 + \norm{ i\vy}^2\right)
         -i\left(\norm{\vx}^2 + \norm{-i\vy}^2\right)
        \quad\text{by right hypothesis}
      \\&= \left(\norm{\vx}^2 +           \norm{\vy}^2\right)
         - \left(\norm{\vx}^2 + \abs{-1}^2\norm{\vy}^2\right)
         +i\left(\norm{\vx}^2 + \abs{ i}^2\norm{\vy}^2\right)
         -i\left(\norm{\vx}^2 + \abs{-i}^2\norm{\vy}^2\right)
        \quad\text{by definition of $\normn$}
        %\quad\text{by def. of $\normn$ \xref{def:norm}}
      \\&= \left(\norm{\vx}^2 + \norm{\vy}^2\right)
         - \left(\norm{\vx}^2 + \norm{\vy}^2\right)
         +i\left(\norm{\vx}^2 + \norm{\vy}^2\right)
         -i\left(\norm{\vx}^2 + \norm{\vy}^2\right)
        \quad\text{by def. of $\abs{\cdot}$\ifsxref{algebra}{def:abs}}
      \\&= 0
    \end{align*}
\end{proof}

%=======================================
\section{Literature}
%=======================================
\begin{survey}
\begin{enumerate}
  \item Inner product spaces: \citer{istratescu1987}

  \item Characterizations of inner product spaces
    \begin{enumerate}
      \item specific characterizations:
        \citer{jordan1935},
        \citer{james1947bams}

      \item surveys:
        \citer{day1947},
        \citer{amir1986},
        \citer{mendoza2003}
    \end{enumerate}
\end{enumerate}
\end{survey}

