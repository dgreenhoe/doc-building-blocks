%============================================================================
% LaTeX File
% Daniel J. Greenhoe
%============================================================================
%======================================
\section{Which one?}
%======================================
Which definition of $\Rxy(m)$ should we use?
Any one of them is perfectly acceptable---as long as a clear definition is provided and that definition is used consistently.
That being said, note the following:

\begin{enumerate}
\item The \ope{expectation} operator $\pE\brp{\rvX\rvY^\ast}$ is an \fncte{inner product}.
As such, it would seem the most natural to follow the convention of other inner product definitions
and thus put the conjugate $\conj$ on $\rvy$ (i.e. follow Papoulis):
\\\indentx$\begin{array}{c>{\ds}rc>{\ds}l}
    $\imark$ & \inprod{\fx(t)}{\fy(t)} &\eqd& \int_{t\in\R} \fx(t)\fy^\conj(t) \dt
  \\$\imark$ & \inprod{\fx(n)}{\fy(n)} &\eqd& \sum_{n\in\Z} \fx(n)\fy^\conj(n)
  \\$\imark$ & \inprod{\rvX}{\rvY}     &\eqd& \pE\brp{\rvX\rvY^\conj}
\end{array}$

\item If we view $\Rxy(m)$ as an \ope{analysis} of $\rvy$ in terms of $\rvx$
      (or as a \ope{projection} of $\rvy$ onto $\rvx$),
      then it would seem more natural to put the conjugate on $\rvx$ (i.e. follow Kay).
      This is what is done in Fourier analysis when projecting a function $\ff(t)$ onto the
      set of basis functions $\set{e^{i\omega n}}{\omega\in\R}$, as in
      \\\begin{align*}
        \opDTFT\brs{\rvy(n)}(\omega)
          &\eqd \inprod{\rvy(n)}{e^{i\omega n}}
          && \text{(\ope{project} $\rvy(n)$ onto $e^{i\omega n}$ for some $\omega\in\R$)}
        \\&\eqd \sum_{n\in\Z} \rvy(n) \brs{e^{+i\omega n}}^\ast
        \\&\eqd \sum_{n\in\Z} \rvy(n) e^{-i\omega n}
      \end{align*}
      But arguably, a ``projection of $\rvy$ onto $\rvx$" would better be served by the use of $\Ryx(m)$ rather than $\Rxy(m)$.

\item As demonstrated in \prefpp{sec:case_systemid_nonlinear}, the Papoulis definition (1)
is arguably more convenient for performing least-squares-like optimization.
\end{enumerate}
