%============================================================================
% Daniel J. Greenhoe
% XeLaTeX file
% graph theory
%============================================================================
%=======================================
\subsection{Graph Theory}
%=======================================
%---------------------------------------
\begin{definition}
\footnote{
  \citerpc{bangjensen2007}{2}{\textsection1.2},
  \citerp{haray1969}{9}
  }
\label{def:graph}
\label{def:dgraph}
%---------------------------------------
%Let $\setX$ be a \structe{set}.
Let $\clRxx$ be the set of all \structe{relation}s \xref{def:clRxy} on a set $\setX$.
\defboxp{
  The pair $\opair{\setX}{\setE}$ is a \structd{graph} if $\setE\subseteq\clRxx$.
  A graph $\opair{\setX}{\setE}$ is \propd{undirected} if $\opair{x}{y}\in\setE\implies\opair{y}{x}\in\setE$.
  A graph $\opair{\setX}{\setE}$ is \propd{directed} if it is \emph{not} undirected.
  A \structe{graph} that is \prope{directed} is a \structd{directed graph}.
  A \structe{graph} that is \prope{undirected} is an \structd{undirected graph}.
  The elements of $\setX$ are the \structd{vertices} and 
  the ordered pairs of $\setE$ are the \structd{arcs} of a \structe{graph} $\opair{\setX}{\setE}$.
  The element $x$ is the \vald{tail} and $y$ is the \vald{head} of an arc $\opair{x}{y}$.
  }
\end{definition}

%---------------------------------------
\begin{definition}
\label{def:wgraph}
%---------------------------------------
\defboxp{
  The tuple $\oquad{\setX}{\setE}{\metricn}{\fw}$ is a \structd{weighted graph}
  if $\opair{\setX}{\setE}$ is a \structe{graph} \xref{def:graph},
        $\metric{x}{y}$ is a \structe{function} in $\clF{\setX\times\setX}{(\Rnn)}$ \xref{def:clFxy},
  and   $\fw(x)$ is a function in $\clF{\setX}{(\Rnn)}$.
  \\The function $\metricn$ is called the \fnctd{edge weight},
  and the function $\fw$ is called the \fnctd{vertice weight}.
  }
\end{definition}

%---------------------------------------
\begin{definition}
\label{def:gcenter}
\label{def:gcen}
%---------------------------------------
Let $\grG\eqd\oquad{\setX}{\setE}{\metricn}{\fw}$ be a \structe{weighted graph} \xref{def:wgraph}.
\\\defboxt{
  The \structd{center} $\gcen(\grG)$ of $\grG$ is
  \quad$\ds\gcen(\grG)\eqd\argmin_{x\in\setX}\max_{y\in\setX} \metric{x}{y} \fw(y)$.
  }
\end{definition}

%This paper introduces a new %type of %center for graphs (next definition)
%%and a new 
%quantity called \structe{graph variance} \xref{def:gvar}.
%%%---------------------------------------
%%\begin{definition}
%%\label{def:gcens}
%%%---------------------------------------
%%Let $\grG\eqd\opair{\setX}{\setE}$ be a \structe{graph} \xref{def:graph}.
%%\defboxt{
%%  The \structd{summed center} $\gcens(\grG)$ of $\grG$ is
%%  \\\indentx$\ds\gcen(\grG)\eqd\argmin_{x\in\setX}\sum_{y\in\setX} \fa(x,y)\fv(y)$
%%  }
%%\end{definition}
%%
%%---------------------------------------
%\begin{definition}
%\label{def:gvar}
%%---------------------------------------
%Let $\grG\eqd\opair{\setX}{\setE}$ be a \structe{graph} \xref{def:graph}.
%\defboxt{
%  The \structd{graph variance} $\gvar(\grG)$ of $\grG$ is
%  \\\indentx$\ds\gvar(\grG)\eqd\argmin_{x\in\setX}\sum_{y\in\setX} \brs{\fa(x,y)}^2\fv(y)$
%  }
%\end{definition}



