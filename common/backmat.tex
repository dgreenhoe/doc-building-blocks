%============================================================================
% Daniel J. Greenhoe
% XeLaTeX file
%============================================================================


\backmatter
\section*{Back Matter}
\addcontentsline{toc}{chapter}{Back Matter}

\qboxnps
  {
    \href{http://en.wikipedia.org/wiki/Niels_Henrik_Abel}{Niels Henrik Abel}
   (1802--1829),
   \href{http://www-history.mcs.st-andrews.ac.uk/BirthplaceMaps/Places/Russia.html}{Norwegian mathematician}
    \index{Abel, Niels Henrik}
    \index{quotes!Abel, Niels Henrik}
    \footnotemark
  }
  {../common/people/small/abel.jpg}
  {It appears to me that if one wants to make progress in mathematics,
    one should study the masters and not the pupils.}
  \citetblt{
    quote: \citerp{simmons2007}{187}.\\ 
    image: \scs\url{http://en.wikipedia.org/wiki/Image:Niels_Henrik_Abel.jpg}, public domain
    }

\qboxnpq
  {
    \href{http://en.wikipedia.org/wiki/Niccol\%C3\%B2_Machiavelli}{Niccol\`o Machiavelli}
    (1469--1527), Italian political philosopher,
    in a 1513 letter to friend Francesco Vettori.
    \index{Machiavelli, Niccol\`o}
    \index{quotes!Machiavelli, Niccol\`o}
    \footnotemark
  }
  {../common/people/Machiavelli_Tito_headcrop_pdomain_bw.jpg}
  {When evening comes, I return home and go to my study.
    On the threshold I strip naked, taking off my muddy, sweaty workaday clothes,
    and put on the robes of court and palace,
    and in this graver dress I enter the courts of the ancients and am welcomed by them,
    and there I taste the food that alone is mine, and for which I was born.
    And there I make bold to speak to them and ask the motives of their actions,
    and they, in their humanity reply to me.
    And for the space of four hours I forget the world, remember no vexation,
    fear poverty no more, tremble no more at death;
    I pass indeed into their world.}
  \citetblt{
    quote: \citerp{machiavelli}{139?}.\\
    image: \parbox[t]{\tw-25mm}{\scs\url{http://commons.wikimedia.org/wiki/File:Santi_di_Tito_-_Niccolo_Machiavelli\%27s_portrait_headcrop.jpg}, public domain}
    }

\begin{center}
\begin{tabular}{*{2}{>{\scs}c}}
    \includegraphics*[height=100mm]{../common/graphics/paintings/Ancientlibraryalex_wkp_pdomain.jpg}
  & \includegraphics*[height=100mm]{../common/graphics/paintings/Spitzweg1850_TheBookWorm_wkp_pdomain.jpg}
 %& \includegraphics*[height=50mm]{../common/graphics/paintings/Spitzweg1850_TheBookWorm_wkp_pdomain_bw.jpg}
  \\
    ancient library of Alexandria
  & \hie{The Book Worm} by \hie{Carl Spitzweg}, circa 1850
\end{tabular}
\citetbl{\raggedright
  \scs\url{http://en.wikipedia.org/wiki/File:Ancientlibraryalex.jpg}, public domain
  \scs\url{http://en.wikipedia.org/wiki/File:Carl_Spitzweg_021.jpg}, public domain
  }
\end{center}

\qboxnps
  {
    \href{http://en.wikipedia.org/wiki/Yoshida_Kenko}{Yoshida Kenko (Urabe Kaneyoshi)}
    (1283? -- 1350?),
    Japanese author and Buddhist monk
    \index{Kenko, Yoshida}  \index{Kaneyoshi, Urabe}
    \index{quotes!Kenko, Yoshida}  \index{quotes!Kaneyoshi, Urabe}
    \footnotemark
  }
  {../common/people/small/kenko.jpg}
  %{The pleasantest of all diversions is to sit alone under the lamp,
  % a book spread out before you,
  % and to make friends with people of a distant past you have never known. (Keene translation page 12)
  %}
  {To sit alone in the lamplight with a book spread out before you,
   and hold intimate converse with men of unseen generations---such is a pleasure beyond compare.}
  \citetblt{
    quote: & \citer{kenko_sansom} \\
    image: & \url{http://en.wikipedia.org/wiki/Yoshida_Kenko}
    }

%---------------------------------------
% References
%---------------------------------------
\addcontentsline{toc}{section}{References}
\bibliography{../common/bib/analysis,../common/bib/language,../common/bib/algebra,../common/bib/biology,../common/bib/found,../common/bib/order,../common/bib/mathmisc,../common/bib/mathhist,../common/bib/probstat,../common/bib/wavelets,../common/bib/approx,../common/bib/digcom,../common/bib/em,../common/bib/english,../common/bib/unicode}

%---------------------------------------
\chapter*{Reference Index}
%---------------------------------------
%\fancyhf[HOL,HER]{\scshape{Reference Index}}
\markboth{Reference Index}{Reference Index}%
\addcontentsline{toc}{section}{Reference Index}
\begin{multicols}{3}
\input{xref.ind}
\end{multicols}

%---------------------------------------
\chapter*{Subject Index}
%---------------------------------------
%\fancyhf[HOL,HER]{\scshape{Subject Index}}
\markboth{Subject Index}{Subject Index}%
\addcontentsline{toc}{section}{Subject Index}
\begin{multicols}{3}
  %============================================================================
% LaTeX File
% Daniel J. Greenhoe
%============================================================================
\documentclass [12pt,twoside,final]{book}
%--------------------------------------
% version
%--------------------------------------
\newcommand{\version}{0.00x}
%--------------------------------------
% Style Packages
%--------------------------------------
 \usepackage{natbib}                  % Bibliographic style: must be loaded before {imakeidx}?
 \usepackage{../common/sty/packages}  % common packages
 \usepackage{../common/sty/switches}  %
 \usepackage{../common/sty/fonts}     % fonts (using fontspec environment)
 \usepackage{../common/sty/dan}       % common definitions
 \usepackage{../common/sty/colors_rgb}% colors
 \usepackage{../common/sty/math}      % common mathematics definitions
 \usepackage{../common/sty/wavelets}  % wavelets
 \usepackage{../common/sty/xsd}       %
 \usepackage[ISBN=000-0-0000000-0-0]{ean13isbn}
 \usepackage{../common/sty/defaults}  %

%--------------------------------------
% flags
%--------------------------------------
\setboolean{wsd}{true}
%--------------------------------------
% color model
%--------------------------------------
%\selectcolormodel{cmyk}% use cmyk color model
 \selectcolormodel{rgb}%  use rgb  color model
%\selectcolormodel{gray}% use gray-scale color model

%--------------------------------------
% page layout
%--------------------------------------
\geometry{
  showframe=false,showcrop=false,
  }

%--------------------------------------
% Headers and Footers
%--------------------------------------
\footerinfo{0}{Operations on Sequeces}{}
%---------------------------------------
% pdf metadata
%---------------------------------------
\hypersetup{%
  pdfkeywords={DSP},
  pdftitle={Operations on sequences},
  pdfdisplaydoctitle=false,
  %pdfsubject={(msc2010)03B50,03B47;03B60,03G05,03G10,03B52.}
  }
%--------------------------------------
% index support with \usepackage{index}
%--------------------------------------
%\newindex{xref}{rdx}{rnd}{Index of References } % index of references
%\newindex{xsym}{sdx}{snd}{Index of Symbols } % index of symbols
%\makeindex         % make the index
%--------------------------------------
% index support with \usepackage{imakeidx}
%--------------------------------------
 \makeindex             % General   index
 \makeindex[name=xref]  % Reference index
 \makeindex[name=xsym]  % Symbol    index
%--------------------------------------
% document content
%--------------------------------------
\dochas{frontmat}
\dochas{backmat}

% main wavelet material
%-----------------------------
 \dochas{dtft}
 \dochas{transop}
 \dochas{mra}
 \dochas{wavstrct}
 \dochas{partuni}
 \dochas{pounity}
 \dochas{vanish}
 \dochas{ortho}
 \dochas{compactp}
 \dochas{symlets}

% Foundational Structures
%-----------------------------
 \dochas{relation}
%\dochas{order}
%\dochas{lattice}
% \dochas{measure}
% \dochas{setstrct}

% Topological Structure
%-----------------------------
% \dochas{topology}
% \dochas{vstopo}
% \dochas{vector}
% \dochas{metric}
% \dochas{vsnorm}
% \dochas{vsinprod}
% \dochas{functionals}
  \dochas{operator}
  \dochas{integrat}

% Basis Theory
%-----------------------------
%\dochas{subspace}
%\dochas{seq}
\dochas{sums}
\dochas{series}
\dochas{frames}

% Fourier Structure
%-----------------------------
 \dochas{harTrig}
 \dochas{harFour}
 \dochas{fs}
 \dochas{harPoly}

% Algebraic Tools
%-----------------------------
 \dochas{algebra}
 \dochas{binomial}
 \dochas{polynom}
 \dochas{spline}
%\dochas{bwav}
 \dochas{interpo}

% Flotation Devices
%-----------------------------
 \dochas{pollen}
 \dochas{fwt}
 \dochas{dsp}
 \dochas{pwrspec}
 \dochas{src_code}

% other
%-----------------------------
%\dochas{logic}
%\dochas{latd}
%\dochas{latm}
%\dochas{latvar}
%\dochas{boolean}
%\dochas{ortholat}
%\dochas{convex}
 \dochas{morph}
%\dochas{found}
%\dochas{numsys}
%\dochas{metricex}

%----------------------------------------------------------------------------
% Body
%----------------------------------------------------------------------------
\begin{document}
\VerbatimFootnotes   % macro to allow verbatim in footnotes (used with fancyvrb package)
%--------------------------------------
% front matter
%--------------------------------------
%============================================================================
% Daniel J. Greenhoe
% XeLaTeX file
%============================================================================
%--------------------------------------
% Page setup
%--------------------------------------
\setlength{\parindent}{0pt}
\pagenumbering{roman}
%---------------------------------------
% cover
%---------------------------------------
%\cleartooddpage%
%============================================================================
% XeLaTeX File
% Daniel J. Greenhoe
%============================================================================
{%
\newgeometry{noheadfoot,nomarginpar,bindingoffset=0mm,centering,inner=0mm,outer=0mm,top=0mm,bottom=0mm}%
\thispagestyle{empty}%
\psset[pst-grad]{cmyk=true}%
\definecolor{clrbegin}{cmyk}{0,0,1,0}% 
\definecolor{clrend}{cmyk}{ 0,1,1,0}%
\psset{unit=1mm}%
\begin{pspicture}(-105,-148.5)(105,148.5)%
  %-------------------------------------
  % global settings
  %-------------------------------------
  \color{richblack}%
  \centering%
  \bfseries%
  \fntFreeSans%
  \large%
  \psset{%
    fillstyle=solid,%
    linecolor=black,%
    gridcolor=graph,%
    gridwidth=0.6pt,%
    gridlabels=14pt,%
    subgriddiv=5,%
    gradbegin=clrbegin,%
    gradend=clrend,%
    gradlines=1024,%
   %gradlines=16384,%
    gradmidpoint=0.50,%
    %gradmidpoint=0,%
    PstDebug=0,%
    }%
  %-------------------------------------
  % background
  %-------------------------------------
  \psframe[fillstyle=gradient,linestyle=none](-105,-148.5)(105,148.5)% a4 paper size
  %-------------------------------------
  % title
  %-------------------------------------
  \rput[t](0,125){\begin{tabular}{c}%
    \Huge\fontsize{16mm}{16mm}\fntAdventor\bfseries{A Book Concerning}\\
    \Huge\fontsize{16mm}{16mm}\fntAdventor\bfseries{Digital Signal Processing}\\\\
    \Large\fntAdventor\versionstamp
  \end{tabular}}%
  %-------------------------------------
  % star data 
  % raw data reference: https://en.wikipedia.org/wiki/List_of_stars_in_Ursa_Major
  % 1 degree = 1 psunit (1mm)
  % r=visible magnitude
  % brightness def= e^{-r}*12
  % Right ascension increases towards the East;
  % if you are standing at the center of the earth, this means right ascension increases towards your left.
  % The negated x coordinates in the \psdot commands is because of this.
  %reference: https://en.wikipedia.org/wiki/List_of_stars_in_Orion
  %-------------------------------------
  \rput(7.35,-10.84){% reference coordinate (x,y) for star plotting
    \psset{linecolor=white,unit=3mm}%
    \rput(180,-30){% star data:  (x,y) on book cover is celestial coordinates 180°right ascension,30°declination
      \psdot[dotsize=1.963850mm](-165.932667,61.751111)%  Alpha   Ursae Majoris  11h 03m 43.84s  +61°45'04.0" 1.81 124 (165.932666...,61.751111...)
      \psdot[dotsize=1.155932mm](-165.459958,54.925417)%  Beta    Ursae Majoris  11h 01m 50.39s +56°22'56.4" 2.34 79 (165.459958333...,54.92541666...)
      \psdot[dotsize=1.077784mm](-178.457250,53.694722)%  Gamma   Ursae Majoris  11h 53m 49.74s +53°41' 41.0" 2.41 84 (178.45725,53.6947222...)
      \psdot[dotsize=0.433834mm](-183.856042,57.032611)%  Delta   Ursae Majoris  12h 15m 25.45s +57°01'57.4" 3.32 81 (183.856041666...,57.0326111...)
      \psdot[dotsize=2.064538mm](-193.506792,55.959833)%  Epsilon Ursae Majoris  12h 54m 01.63s  +55°57'35.4" 1.76  81 (193.50679166..., 55.959833...)
      \psdot[dotsize=1.290341mm](-200.980917,54.925417)%  Mizar                  13h 23m 55.42s  +54°55' 31.5" 2.23  78 (200.98091666...,54.92541666...)
      \psdot[dotsize=1.886846mm](-206.885625,49.313306)%  Eta     Ursae Majoris  13h 47m 32.55s  +49°18'47.9" 1.85 101 (206.885625,49.31330555...
      %
      %\rput(-75,45){\psdot[dotsize=12pt,linecolor=cyan](0,0)\psline(-5,0)(5,0)\psline(0,-5)(0,5)\uput[0]{0}{\color{cyan}5h,$45^\circ$}}% center (comment out before printing)
      %\psdot[dotsize=10.02324mm](-78.6344583,8.201639)%   Rigel                  05h 14m 32.27s  -08° 12'05.9" 0.18 (78.634458333...,8.20163888...)
      }}%
  %-------------------------------------
  % middle graphics
  %-------------------------------------
  \rput   (15.5,35){\includegraphics[width=50mm]{../common/math/graphics/pdfs/iir2n.pdf}}%
  \rput[r]( 93,35){\includegraphics{../common/math/graphics/pdfs/iir2rphi.pdf}}%
  \rput[l](-93,35){\includegraphics{../common/math/graphics/pdfs/vecres.pdf}}%
  %-------------------------------------
  % lower graphics
  %-------------------------------------
  \rput[t](0,-20){\begin{tabular}{c}
    \tbox{\includegraphics{../common/math/graphics/pdfs/pz_unstable2.pdf}}
    \tbox{$\times$}
    \tbox{\includegraphics{../common/math/graphics/pdfs/pz_allpass.pdf}}
    \tbox{$=$}
    \tbox{\includegraphics{../common/math/graphics/pdfs/pz_unall.pdf}}
      \\
    Daniel J. Greenhoe
      \\
    \includegraphics[height=50mm]{../common/math/graphics/pdfs/sinclat.pdf}%
      \\
    {\itshape Signal Processing ABCs} series
      \\
    volume \framebox{1}%
  \end{tabular}%
  }%
  \rput[b]{33}(-25,-85){\reflectbox{\includegraphics*[height=15mm]{../common/graphics/watercraft/ghind_blue.pdf}}}% ship
  %-------------------------------------
  % guide (comment out before publishing)
  %-------------------------------------
  %\psline[linecolor=cyan](-10,0)(10,0)%
  %\psline[linecolor=cyan](0,-10)(0,10)%
\end{pspicture}%
\restoregeometry%
%---------------------------------------
% inside front cover
%---------------------------------------
\newpage
\thispagestyle{empty}%
\mbox{}%
}%

\newpage%
%---------------------------------------
% title page (v)
%---------------------------------------
\cleartooddpage
%\thispagestyle{empty}%
\mbox{}\vfill
\markboth{Title page}{Title page}
\addcontentsline{toc}{section}{Title page}
\begin{tabular}{ll}
    title:              & \fntAdventor\itshape{A Book Concerning Symbolic Sequence Processing}
  \\document type:      & \ttfamily{book}
  \\series:             & {\itshape Signal Processing ABCs}
  \\volume:             & \framebox{3}
  \\author:             & \fntHeros{Daniel J. Greenhoe}
  \\version:            & \versionstamp
  \\time stamp:         & \timestamp
  \\copyright:          & \copyrightstamp
  \\license:            & \CCBYNCNDstatement
  \\typesetting engine: & \XeLaTeX
  \\document url:       & \url{https://www.researchgate.net/project/Signal-Processing-ABCs}
 %\\document url:       & \url{https://www.researchgate.net/project/Symbolic-Sequence-Processing}
 %\\document url:       & \url{https://www.researchgate.net/profile/Daniel_Greenhoe/publications}
 % \\document url:       & \footnotesize\url{https://www.researchgate.net/project/Mathematical-Structure-and-Design}
 % \\Octave/MatLab code: & \footnotesize\url{https://github.com/dgreenhoe/wavelets}
\end{tabular}
\vfill\mbox{}
%---------------------------------------
% copyright page (page ii)
%---------------------------------------
\vfill
{\sffamily%============================================================================
% Daniel J. Greenhoe
% XeLaTeX file
%============================================================================
This text was typeset using \hi{\XeLaTeX},
which is part of the \TeX family of typesetting engines, 
which is arguably the greatest development since the \hi{Gutenberg Press}.
Graphics were rendered using the \hie{pstricks} and related packages, and \hi{\LaTeX} graphics support.
%and the \hie{pgfplots} package (for rendering the Pollen 3-d wavelet appearing on the cover and in \prefp{fig:pollen4}).
%Data for some figures was generated using \hie{GNU Octave}---a free alternative to {\footnotesize MATLAB}.

%\begin{sloppypar} 
The main {\rmfamily roman}, {\itshape italic}, and {\bfseries bold} font typefaces used 
are all from the \hie{Heuristica} family of typefaces
(based on the \hie{Utopia} typeface, released by \hie{Adobe Systems Incorporated}).
The math font is {{\fntXits XITS}} from the {{\fntXits XITS font project}}.
The font used in quotation boxes is adapted from {\fntZapf Zapf Chancery Medium Italic},
originally from URW++ Design and Development Incorporated.
The font used for the text in the title is {\fntAdventor Adventor} (similar to \hie{Avant-Garde}) from the \hie{\TeX-Gyre Project}.
The font used for the version in the footer of individual pages is 
\mbox{\fntDigital\footnotesize Liquid} \mbox{\fntDigital\footnotesize Crystal} 
(\hie{Liquid Crystal}) from \hie{FontLab Studio}.
The Latin handwriting font is {\fntLavi Lavi} from the \hie{Free Software Foundation}.
%The traditional Chinese font used for the author's Chinese name in the header is {{\fntzhthw 王漢宗中仿宋繁}}.\footnote{\sffamily%
%pinyin: \hi{W{/'a}ng H{/`an} Z{/=o}ng Zh/=ong F{/va}ng S{/`o}ng F{/'an}}; 
%translation: H/`an Z/=ong W/'ang's Medium-weight S/`ong-style Traditional Characters;
%literal:
%{\fntzhthw 王漢宗}$\sim$font designer's name;
%{\fntzhthw 中}$\sim$medium;
%{\fntzhthw 仿}$\sim$to imitate;
%{\fntzhthw 宋}$\sim$S/`ong (a dynasty);
%{\fntzhthw 繁}$\sim$traditional
%}
%Chinese glyphs appearing in the text are from the font {{\fntzht 王漢宗中明體繁}}.\footnote{\sffamily%
%pinyin: \hi{W{/'a}ng H{/`an} Z{/-o}ng Zh/-ong Mi{/'n}g T{/>i} F{/'an}}; 
%translation: H/`an Z/=ong W/'ang's Medium-weight Mi{/'n}g-style Traditional Characters;
%literal:
%{\fntzhthw 王漢宗}$\sim$font designer's name;
%{\fntzhthw 中}$\sim$medium;
%{\fntzhthw 明}$\sim$Mi{/'n}g (a dynasty);
%{\fntzhthw 體}$\sim$style;
%{\fntzhthw 繁}$\sim$traditional
%}

\textimgr[11mm]{../common/graphics/watercraft/ghind_blue.pdf}{%
  The ship 
  appearing throughout this text %(usually tinted blue)
  is loosely based on the \hie{Golden Hind}, a sixteenth century English galleon 
  famous for circumnavigating the globe.
  }




}
\vfill
%---------------------------------------
% abstract
%---------------------------------------
\cleartooddpage
%\thispagestyle{empty}%
\setlength{\parskip}{1ex}%
%============================================================================
% Daniel J. Greenhoe
% XeLaTeX / LaTeX file
% abstract
%============================================================================
\begin{abstract}
The literature offers varying and in general incompatible definitions of the
\fncte{cross-correlation} function $\Rxy(n,m)$ and its \ope{jointly wide-sense stationary} special case $\Rxy(m)$.
The choice of definition has consequences for results involving the \fncte{cross-spectral density} function 
$\Swxy(\omega)$, and the more general Z-transform density $\Szxy(z)$.
\end{abstract}

%\begin{asciiabstract}
The literature offers varying and in general incompatible definitions of the cross-correlation functions Rxy(n,m) and its jointly wide-sense stationary special case Rxy(m). The choice of definition has consequences for results involving the cross-spectral density function Sxy(w), and the more general Z-transform density Sxy(z).
%
%\end{asciiabstract}


%---------------------------------------
% quote page pair (odd page iii then even page iv)
%---------------------------------------
\cleartooddpage
%============================================================================
% Daniel J. Greenhoe
% XeLaTeX file
%============================================================================
%---------------------------------------
% quote page odd
%---------------------------------------
\cleartooddpage%
\thispagestyle{empty}%
\mbox{}\vfill
\addcontentsline{toc}{section}{Quotes}
\qboxnpqt
  {\href{http://en.wikipedia.org/wiki/Housman}{Alfred Edward Housman}, English poet (1859--1936)
    \index{Housman, Alfred Edward}
    \index{quotes!Housman, Alfred Edward}
    \footnotemark
  }{../common/people/small/housman.jpg}
  {Here, on the level sand, \\
    Between the sea and land, \\
    What shall I build or write \\
    Against the fall of night?
  }{Tell me of runes to grave \\
    That hold the bursting wave, \\
    Or bastions to design \\
    For longer date than mine.
  }
  %\citetblt{
  %  quote: \citerpc{housman1936}{64}{``Smooth Between Sea and Land"}, 
  %         \citerc{hardy1940}{section 7}.
  %  image: \url{http://en.wikipedia.org/wiki/Image:Housman.jpg}
  %  }
\vfill
\qboxnps
  {
    \href{http://en.wikipedia.org/wiki/Igor_Stravinsky}{Igor Fyodorovich Stravinsky}
    (1882--1971), Russian-born composer
    \index{Stravinsky, Igor}
    \index{quotes!Stravinsky, Igor}
    \footnotemark
  }{../common/people/small/stravinsky.jpg}
  {The uninitiated imagine that one must await inspiration in order to create.
   That is a mistake.
   I am far from saying that there is no such thing as inspiration;
   quite the opposite.
   It is found as a driving force in every kind of human activity, 
   and is in no wise peculiar to artists.
   But that force is brought into action by an effort, 
   and that effort is work.
   Just as appetite comes by eating so work brings inspiration, 
   if inspiration is not discernible at the beginning.
   %But it is not simply inspiration that counts:
   %it is the result of inspiration---
  }
  %\citetblt{
  %  quote:  \citerp{ewen1961}{408},
  %          \citer{ewen1950}.
  %  %      & \citerp{deonaraine}{11} \\
  %  %      & \url{http://www3.thinkexist.com/quotes/igor_stravinsky/} \\
  %  image:  \url{http://en.wikipedia.org/wiki/Image:Igor_Stravinsky.jpg}.
  %}
\vfill
\qboxnpq
  { \href{http://en.wikipedia.org/wiki/Bertrand_Russell}{Bertrand Russell}
    \href{http://www-history.mcs.st-andrews.ac.uk/Timelines/TimelineF.html}{(1872--1970)},
    \href{http://www-history.mcs.st-andrews.ac.uk/BirthplaceMaps/Places/UK.html}{British mathematician},
    in a 1962 November 23 letter to Dr. van Heijenoort.
    \index{Russull, Bertrand}
    \index{quotes!Russull, Bertrand}
    \footnotemark
  }{../common/people/small/russell.jpg}
  {As I think about acts of integrity and grace,
   I realise that there is nothing in my knowledge to compare with Frege's dedication to truth.
   His entire life's work was on the verge of completion,
   much of his work had been ignored to the benefit of men infinitely less capable,
   his second volume was about to be published, and upon finding that his fundamental assumption
   was in error, 
   he responded with intellectual pleasure clearly submerging any feelings of personal disappointment.
   It was almost superhuman and a telling indication of that of which men are capable if their
   dedication is to creative work and knowledge instead of cruder efforts to dominate and be 
   known.}
  %\citetblt{
  %  quote:  \citerp{heijenoort}{127}.
  %  image:  \url{http://www-history.mcs.st-andrews.ac.uk/PictDisplay/Russell.html}.
  %  }
\vfill
\qboxns{Translated note written in Czech found in a World War II Nazi fired shell, one of eleven
in the fuel tanks of the B-17 bomber ``Tondelayo", that somehow made it safely back to base
instead of bursting into flames over Kassel, Germany---all the shells 
being devoid of explosive charge.\footnotemark\hspace{1ex}
In like manner, perhaps this text in some small way may help someone find success in accomplishing a task 
with significant returns to the well-being of humanity.}
{This is all we can do for you now.}
%\footnotetext{
%  \citerpg{Bendiner1980}{139}{9780399123726}
%  \url{https://www.truthorfiction.com/fall-of-fortresses-bendiner/}
%}
\vfill
\hfill\includegraphics*[height=10mm]{../common/graphics/watercraft/ghind_blue.pdf}%

%---------------------------------------
% page even: references to quotes and impages
%---------------------------------------
\cleartoevenpage
%\thispagestyle{empty}%
\mbox{}%
%\markboth{Quotes}{Quotes}
\addtocounter{footnote}{-3}%
\footnotetext{\begin{tabular}[t]{ll}% Housman quote
  quote: & \citerpc{housman1936}{64}{``Smooth Between Sea and Land"}, \citerc{hardy1940}{section 7}\\
  image: & \url{http://en.wikipedia.org/wiki/Image:Housman.jpg}
  \end{tabular}}%
  %
\stepcounter{footnote}%
\footnotetext{\begin{tabular}[t]{ll}% Ewen quote
  quote: & \citerp{ewen1961}{408}, \citer{ewen1950}
  %      & \citerp{deonaraine}{11} \\
  %      & \url{http://www3.thinkexist.com/quotes/igor_stravinsky/} \\
    \\
  image: & \url{http://en.wikipedia.org/wiki/Image:Igor_Stravinsky.jpg}
  \end{tabular}}%
  %
\stepcounter{footnote}%
\footnotetext{\begin{tabular}[t]{ll}% Russell quote
  quote: & \citerp{heijenoort}{127}\\
  image: & \url{http://www-history.mcs.st-andrews.ac.uk/PictDisplay/Russell.html}
  \end{tabular}}%

\stepcounter{footnote}%
\footnotetext{%"This is all we can do for you now." quote
  \citerpg{Bendiner1980}{139}{9780399123726},
  \url{https://www.truthorfiction.com/fall-of-fortresses-bendiner/}
  }

%%============================================================================
% LaTeX File
% Daniel J. Greenhoe
%============================================================================

%=======================================
\section*{Preface}
\addcontentsline{toc}{section}{Preface}
\markboth{{Preface}}{{Preface}}
%=======================================
\qboxnps
  {
    \href{http://en.wikipedia.org/wiki/Carl_Friedrich_Gauss}{Karl Friedrich Gauss}
    (1777--1855), German mathematician
    \index{Gauss, Karl Friedrich}
    \index{quotes!Gauss, Karl Friedrich}
    \footnotemark
  }
  {../common/people/small/gauss.jpg}
  {You know that I write slowly.
    This is chiefly because I am never satisfied until I have said as much as
    possible in a few words,
    and writing briefly takes far more time than writing at length.}
  \citetblt{
    quote:  \citerp{simmons2007}{177} \\
    image:  \url{http://en.wikipedia.org/wiki/Karl_Friedrich_Gauss}
    }



%\qboxnps
%  {
%    \href{http://en.wikipedia.org/wiki/Carl_Friedrich_Gauss}{Karl Friedrich Gauss}
%    (1777--1855), German mathematician
%    \index{Gauss, Karl Friedrich}
%    \index{quotes!Gauss, Karl Friedrich}
%    \footnotemark
%  }
%  %{../common/people/small/gauss1828.jpg}
%  {../common/people/gauss.jpg}
%  {I mean the word proof not in the sense of lawyers,
%    who set two half proofs equal to a whole one,
%    but in the sense of the mathematician, where \textonehalf proof = 0
%    and it is demanded for proof that every doubt becomes impossible.}
%  \citetblt{
%    quote:  \citerp{simmons2007}{177} \\
%    %image:  \url{http://www-history.mcs.st-andrews.ac.uk/PictDisplay/Gauss.html}
%    image:  \url{http://en.wikipedia.org/wiki/Karl_Friedrich_Gauss}
%    }

%
%\qboxnps
%  {
%    \href{http://en.wikipedia.org/wiki/Enrico_Fermi}{Enrico Fermi}
%    \href{http://www-history.mcs.st-andrews.ac.uk/Timelines/TimelineG.html}{(1777--1855)},
%    \href{http://www-history.mcs.st-andrews.ac.uk/BirthplaceMaps/Places/Italy.html}{Italian physicist}
%    \index{Fermi, Enrico}
%    \index{quotes!Fermi, Enrico}
%    \footnotemark
%  }
%  {../common/people/small/fermi.jpg}
%  {If it is true, it can be proved.}
%  \citetblt{
%    quote:  \citerp{benedetto}{85} \\
%    image:  \url{http://www-history.mcs.st-andrews.ac.uk/Mathematicians/Fermi.html}
%    }


%Aubrey, John (1626-1697)
%[About Thomas Hobbes:]
%He was 40 years old before he looked on geometry; which happened accidentally. Being in a gentleman's library, Euclid's Elements lay open, and "twas the 47 El. libri I" [Pythagoras' Theorem]. He read the proposition . "By God", sayd he, "this is impossible:" So he reads the demonstration of it, which referred him back to such a proposition; which proposition he read. That referred him back to another, which he also read. Et sic deinceps, that at last he was demonstratively convinced of that trueth. This made him in love with geometry.
%In O. L. Dick (ed.) Brief Lives, Oxford: Oxford University Press, 1960, p. 604.
%http://math.furman.edu/~mwoodard/ascquota.html

This book is largely based on four papers written by the same author as this text:

\begin{enumerate}
  \item \citeP{greenhoe2015sto}: \emph{Order and metric geometry compatible stochastic processing}.
    \\\url{https://peerj.com/preprints/844/}
    \\\url{https://www.researchgate.net/publication/303057738}
    \\This paper is the basis for \pref{chp:sto}.
  \item \citeP{greenhoe2015ssp}: \emph{Order and Metric Compatible Symbolic Sequence Processing}.
    \\\url{https://peerj.com/preprints/2052/}
    \\\url{https://www.researchgate.net/publication/302947227}
    \\This paper is the basis for \pref{chp:ssp}.
    \\The C++ source code used to generate the 128 or so TeX files for the data plots in the paper can be downloaded from here:
    \\\url{https://www.researchgate.net/publication/302953844}
  \item \citeP{greenhoe2015larc}: \emph{An extension to the spherical metric using polar linear interpolation}.
    \\\url{https://www.researchgate.net/publication/303984226}
    \\This paper is the basis for \pref{app:larc}.
  \item \citeP{greenhoe2015pds}: \emph{Properties of distance spaces with power triangle inequalities}.
    \\\url{http://www.journals.pu.if.ua/index.php/cmp/article/view/483}
    \\\url{https://peerj.com/preprints/2055}
    \\\url{https://www.researchgate.net/publication/281831459}
    \\This paper is the basis for \pref{app:distance} and \pref{app:pds}.
    It has been published in the 2016 volume 8 number 1 edition of the 
    journal \emph{Carpathian Mathematical Publications}.
\end{enumerate}




%---------------------------------------
% Acknowledgement (page v)
%---------------------------------------
\cleartooddpage
%\thispagestyle{empty}%
\setlength{\parskip}{1ex}%
%============================================================================
% Daniel J. Greenhoe
% XeLaTeX file
% introduction
%============================================================================
%=======================================
\chapter*{Acknowledgements}
\addcontentsline{toc}{section}{Acknowledgements}
%=======================================
It is not far from the truth to say that much of the research presented in this text started with this email from 
%my academic advisor, 
Professor Po-Ning Chen of National Chiao-Tung University, Taiwan:\footnote{%
  \begin{tabular}[t]{lll}%
      Po-Ning Chen:                          & \zht{陳伯寧}       &(pinyin: Ch/'en B/'o Ni/'ng)
    \\National Chiao-Tung University:        & \zht{國立交通大學} &(G/'uo L/`i Ji/-ao T/-ong D/`a X/'ue).
    \\Taiwan:                                & \zht{台灣}         &(pinyin: T/'ai W/-an).
    %\\Communications Engineering Department: & \zht{電信工程學系} &(Di/`an X/`in G/-ong Ch/'eng X/'ue X/`i)
    %\\address: Hsinchu, Taiwan:              & \zht{新竹,台灣}   &(X/-in Zh/'u, T/'ai W/-an)
  \end{tabular}}
\begin{center}\fbox{\begin{minipage}{\tw-20mm}\ttfamily\raggedright %\fntmono\raggedright
\hfill Sat, 25 Jan 2014 03:45:51 -0800 (PST)

Dear Dan:
So far people are mostly (and are used to) dealing with ``frequency-transform" of a numerical sequences. 
However, sometimes, we need to find out the "occurrence frequency" of a symbolic sequence. 
I am wondering ``Can we design a wavelet transform for a symbolic sequence like DNA?" 
What do you think? See the attached paper as an example.

Po-Ning
\end{minipage}}\end{center}

%I responded the same day saying, ``Thank you for the paper. So far I have read the introduction. I have
%never heard of ``Genomic Signal Processing" before. That is very
%interesting, and maybe something that could help a lot of people some
%day.\ldots I will investigate Genomic Signal Processing more. Thank you for
%introducing me to the topic."\footnote{The ``attached paper" referred to by Professor Chen's email is \citeP{galleani2010}.}
%
%This text is not about genomic signal processing in particular. 
%But it is about symbolic sequence processing, of which genomic signal processing is a special 
%and motivational case.

So I would like to say one more time to %my academic advisor, 
Professor Po-Ning Chen, 
``Thank you for introducing me to the topic"(!)
which has to a large extent resulted in the research described in this text.\footnote{%
  The ``attached paper" referred to by Professor Chen's email is \citeP{galleani2010}.}

%Moreover, it would be difficult to effectly communicate the ideas in this text without the use of 
%the typesetting software {\XeLaTeX}, and {\LaTeX} upon which it is based.
%Even though I am from the United States and {\LaTeX} has its origins in the United States,
%it seems I never really knew what it was before coming to Taiwan and being introduced to it by Professor Chen.
%And likewise {\XeLaTeX} has it's origins in the United States as well, but again I was introduced to it in Taiwan,
%this time by Professor Peng-Hua Wang.\footnote{%
%  Peng-Hua Wang (Chinese: \zht{王鵬華}, pinyin: W/'ang P/'eng Hu/'a).
%  }
%So I would like to thank Professor Chen for introducing me to \LaTeX,
%and I would like to thank Professor Wang for introducing me to \XeLaTeX.
%
%I would also like to thank 
%Professor Po-Ning Chen,\footnote{Po-Ning Chen (Chinese: \zht{陳伯寧}, pinyin: Ch/'en B/'o N/'ing)}
%Professor Yuan-Pei Lin,\footnote{Yuan-Pei Lin (Chinese: \zht{林源倍}, pinyin: L/'in Y/'uan B/`ei)}
%Professor Hsiao-Feng Lu,\footnote{Hsiao-Feng Lu (Chinese: \zht{陸曉峰}, pinyin: L/`u Xi/>ao F/-eng)}
%and
%Professor Zhong-Jin Lu\footnote{Zhong-Jin Lu (Chinese: \zht{呂忠津}, pinyin: L/>:u Zh/-ong J/-in)}
%who served on the committee for my oral defense, which was largely based on the material in this text.
%
%Finally, I would like to thank National Chiao-Tung University (NCTU), the Communications Engineering Department of NCTU, 
%and the people of Taiwan.\footnote{\begin{tabular}[t]{lll}%
%    Taiwan:                                & \zht{台灣}         &(pinyin: T/'ai W/-an).
%  \\National Chiao-Tung University:        & \zht{國立交通大學} &(G/'uo L/`i Ji/-ao T/-ong D/`a X/'ue).
%  \\Communications Engineering Department: & \zht{電信工程學系} &(Di/`an X/`in G/-ong Ch/'eng X/'ue X/`i)
%  \\address: Hsinchu, Taiwan:              & \zht{新竹,台灣}   &(X/-in Zh/'u, T/'ai W/-an)
%\end{tabular}}
%Although born and raised in the United States,
%Taiwan has graciously allowed me to live, study, and work here for several years now.
%To live in a country not one's own is always a privilege, an honor, and a great opportunity to learn about life.
%I am very happy to have had and to continue to have this privilege.
%\\[2ex]
%{\color{blue}---\enghw{Daniel J. Greenhoe} (\zhthw{柯晨光})}






%%---------------------------------------
%% Dedication (page iii)
%%---------------------------------------
%---------------------------------------
% blank page (page iv)
%---------------------------------------
%\cleartoevenpage
%\markboth{}{}
%\fancyhf[HOL,HER]{}
%---------------------------------------
% Symbol table 
%---------------------------------------
\clearpage
%============================================================================
% NCTU - Hsinchu, Taiwan
% LaTeX File
% Daniel Greenhoe
%============================================================================

%--------------------------------------
\chapter*{Symbols}
\addcontentsline{toc}{section}{Symbols}
%--------------------------------------
\qboxnpqt
  {
    Ren\'e Descartes (1596--1650), philosopher and mathematician
    \index{Descartes, Ren\'e}
    \footnotemark
  }
  {../common/people/descart.eps}
  {{\bf rugula XVI.} 
      Quae vero praesentem mentis attentionem non requirunt, 
      etiamsi ad conclusionem necessaria sint, 
      illa melius est per brevissimas notas designare quam per integras figuras: 
      ita enim memoria non poterit falli, nec tamen interim cogitatio distrahetur 
      ad haec retinenda, dum aliis deducendis incumbit.}
  {{\bf Rule XVI.} 
      As for things which do not require the immediate attention of the mind, 
      however necessary they may be for the conclusion, 
      it is better to represent them by very concise symbols rather than by 
      complete figures. 
      It will thus be impossible for our memory to go wrong, 
      and our mind will not be distracted by having to retain these 
      while it is taken up with deducing other matters.}
  \footnotetext{\begin{tabular}[t]{ll}
    quote:       & \citei[rugula XVI]{descartes_rules} \\
    translation: & \citei[rule XVI]{descartes_rules_eng} \\
    image:       & \url{http://en.wikiquote.org/wiki/Image:Descartes.jpg}
  \end{tabular}}



%--------------------------------------
\subsubsection*{System parameters}
%--------------------------------------
\begin{tabular}{lll}
  $\Rsym $   & symbol rate    & $\frac{\mbox{symbols         }}{\mbox{second   }}          $ \\
  $\Rsam $   & sampling rate  & $\frac{\mbox{samples         }}{\mbox{second   }}          $ \\
  $\Rdata$   & data rate      & $\frac{\mbox{data bits       }}{\mbox{second   }}          $ \\
  $\rcode$   & coding ratio   & $\frac{\mbox{information bits}}{\mbox{code bits}}          $ \\
  $\rsym $   & symbol ratio   & $\frac{\mbox{1               }}{\mbox{number of signal waveforms}} $ \\
  $\rchan$   & channel ratio  & $\frac{\mbox{information bits}}{\mbox{symbol}}             $ 
\end{tabular}

%--------------------------------------
\subsubsection*{Sets of Numbers}
%\citec{weis}{\url{\ldots/NaturalNumber.html}}
%--------------------------------------
\begin{tabular}{lll}
$\Z$       & integers              & $\{\ldots,-3,-2,-1,0,1,2,3,\ldots\}$ \\
$\Znn$     & non-negative integers & $\{0,1,2,3,\ldots\}$ \\
$\Znp$     & non-positive integers & $\{\ldots,-3,-2,-1,0\}$ \\
$\Zp$      & positive integers     & $\{1,2,3,\ldots\}$ \\
$\Zn$      & negative integers     & $\{\ldots,-3,-2,-1\}$ \\
$\Ze$      & even integers         & $\{\ldots,-4,-2,0,2,4,\ldots\}$ \\
$\Zo$      & odd  integers         & $\{\ldots,-3,-1,1,3,\ldots\}$ \\
\\
$\R$       & real numbers          &               \\
$\Rnn$     & non-negative real     & $[0,\infty)$  \\
$\Rnp$     & non-positive real     & $(-\infty,0]$ \\
$\Rp$      & positive real         & $(0,\infty)$  \\
$\Rn$      & negative real         & $(-\infty,0)$ \\
\\
$\Q$       & field of rational numbers \\        
$\C$       & field of complex numbers          
\end{tabular}

%--------------------------------------
\subsubsection*{Operations on sets}
%--------------------------------------
\begin{tabular}{ll}
   $A^c     $ & complement   \\
   $A \cup B$ & union        \\
   $A \cap B$ & intersection \\
   $A -    B$ & difference
\end{tabular}

%--------------------------------------
\subsubsection*{Spaces}
%--------------------------------------
\begin{tabular}{lll}
$\spLL$  & $\set{f:\R\to\C}{\int_x |\ff(x)|^2 \dx < \infty }$ \\
$\spLp$  & $\set{f:\R\to\C}{\int_x |\ff(x)|^p \dx < \infty }$ \\
$\spII$  & $\set{f:\Z\to\C}{\sum_n |\ff(n)|^2     < \infty }$ \\
$\spIp$  & $\set{f:\Z\to\C}{\sum_n |\ff(n)|^p     < \infty }$ \\
$\spPP$  & $\set{f:\R\to\C}{\ff(x)=\ff(x+T), \int_0^T |\ff(x)|^2     < \infty }$ 
\end{tabular}

%--------------------------------------
\subsubsection*{Spaces}
%--------------------------------------
\begin{tabular}{ll}
$\vZn$     & $\set{(a_1, a_2, \ldots, a_n)}{a_i\in\Z} $ \\
$\vRn$     & $\set{(a_1, a_2, \ldots, a_n)}{a_i\in\R} $ \\
$\vCn$     & $\set{(a_1, a_2, \ldots, a_n)}{a_i\in\C} $ 
%$\vRc$     & $\set{set of all continuous real-valued functions    \\
%$\vCc$     & $\set{set of all continuous complex-valued functions
\end{tabular}

%--------------------------------------
\subsubsection*{Relations}
%--------------------------------------
\begin{tabular}{cll}
$=$          & equality relation \\
$\eqd$       & equality by definition \\
$\eqi$       & identity \\
$\eqa$       & approximately equal to \\
$\to$        & maps to \\
$\in$        & is a member of; is in  \\
$\subseteq$  & is a subset of   \\
$<$          & irreflexive ordering relation &``less than"   \\
$\le$        & reflexive ordering relation   &``less than or equal to"   \\
$>$          & irreflexive ordering relation &``greater than"   \\
$\ge$        & reflexive ordering relation   &``greater than or equal to"   \\
\end{tabular}


%--------------------------------------
\subsubsection*{Propositional and predicate logic}
%--------------------------------------
\begin{tabular}{cl}
$\lnot$        & logical NOT operation \\
$\land$        & logical AND operation \\
$\lor$         & logical inclusive OR operation \\
$\implies$     & ``implies"; ``only if" \\
$\Longleftarrow$      & ``is implied by"; ``if" \\
$\iff$     & ``if and only if"; ``implies and is implied by" \\
$\forall$     & universal quantifier: ``for every" or ``for all" \\
$\exists$     & existential quantifier: ``there exists" \\
\end{tabular}

%--------------------------------------
\subsubsection*{Operators\footnotemark}
%--------------------------------------
\footnotetext{
   \begin{tabular}[t]{lll}
      An & {\bf operator  } & is a mapping from a function to a function.\\
      A  & {\bf functional} & is a mapping from a function to a single value.  \\
      A  & {\bf function  } & is a mapping from a single value to a single value. 
   \end{tabular}
}
\begin{tabular}[t]{lll}
   $\opFT$   & Fourier transform           & see Appendix~\ref{app:fourier} page~\pageref{app:fourier}\\
   $\opFS$   & Fourier Series   \\
   $\opDFT$  & Discrete Fourier Transform   \\
   $\opDTFT$ & Discrete Time Fourier Transform   \\
   $\opFi$   & inverse Fourier transform  \\
   $\opL $   & Laplace transform           \\
   $\opZ $   & Z-transform                 & see Definition~\ref{def:opZ} page~\pageref{def:opZ}\\
   $\ft{x}$  & Fourier transform of a function $x$ \\
   $\FS{x}$  & Fourier series of a function $x$ \\
   $\opSmp$  & sampling \\
   $\opSys$  & system   \\
   $\opT$    & transmit \\
   $\opR$    & receive \\
   $\opI$    & identity \\
   $\opC$    & channel  \\
   $\opCan$  & additive noise channel  \\
   $\opCawn$ & additive white noise channel  \\
   $\opCagn$ & additive gaussian noise channel  \\
   $\opCawgn$& additive white Gaussian noise channel  \\
\end{tabular}

%--------------------------------------
\subsubsection*{Probability}
%--------------------------------------
\begin{tabular}{ll}
   $\pso$ & set of outcomes of an experiment           \\
   $\pse$ & set of all the sets of events from $\pso$  \\
   $\psp$ & probability measure on $\pse$   \\
   $\pp$  & probability density function \\
   $\Mx$  & mean of random variable $\vx$ \\
   $\Q{x}$ & normal probability measure function \\
   $\E   $ & expectation                 \\
   $\var$  & variance \\
   $\cov{}{}$  & covariance \\
   $\Rxx$ & autocorrelation of random variable $\vx$ \\
   $\Sxx$ & spectral density of $\vx$ \\
\end{tabular}

%--------------------------------------
\subsubsection*{Estimation}
%\dotfill Appendix~\ref{app:est} page~\pageref{app:est}
%--------------------------------------
\begin{tabular}{ll}
   $\estn   $ & estimate of parameter $\vp$          \\
   $\estMAP $ & maximum apriori probability estimate \\
   $\estML  $ & maximum likelihood estimate          \\
   $\estB   $ & Bayesian estimate                    \\
   $\estMM  $ & minimax estimate                     \\
   $\estNP  $ & Neyman-Pearson estimate              \\
   $\estMS  $ & minimum mean square estimate         \\
   $\estLS  $ & least square estimate                \\
   $\fCost  $ & cost function                        \\
\end{tabular}

%--------------------------------------
\subsubsection*{Electromagnetics}
%\dotfill Appendix~\ref{app:em} page~\pageref{app:em}
%--------------------------------------
\begin{tabular}{ll}
   $\E          $ & electric field           \\
   $\H          $ & magnetic field           \\
   $\D          $ & electric field density   \\
   $\B          $ & magnetic field density   \\
   $\J          $ & current                  \\
   $\diver{}    $ & divergence               \\
   $\curl{}     $ & curl                     \\
   $\DE         $ & permittivity operator    \\
   $\BH         $ & permeability operator    \\
   $\laplacian{}$ & laplacian operator       \\
   $\unit{}     $ & unit vector
\end{tabular}

%--------------------------------------
\subsubsection*{Calculus}
%--------------------------------------
\begin{tabular}{lll}
   $\grad        $ & gradient                                      \\
   $\deriv{y}{x} $ & derivative of $y$ with respect to $x$         \\
   $\pderiv{y}{x}$ & partial derivative of $y$ with respect to $x$ \\
   $\int_x \ff(x)\dx$ & Lebesgue integral over $(-\infty,\infty)$    \\
   $\int_A \ff(x) \mathrm{d}A$ & Lebesgue integral over set $A$              \\
\end{tabular}


%--------------------------------------
\subsubsection*{Vector space}
%\dotfill Appendix~\ref{app:inprod} page~\pageref{app:inprod}
%--------------------------------------
\begin{tabular}{lll}
   $\inprod{x}{y}$ & innerproduct of $x$ and $y$   \\
   $\norm{x}     $ & norm of $x$                   \\
   $\metric{x}{y}  $ & distance between $x$ and $y$  \\
\end{tabular}

%--------------------------------------
\subsubsection*{Complex numbers}
%\dotfill Appendix~\ref{app:complex} page~\pageref{app:complex}
%--------------------------------------
\begin{tabular}{lll}
   $x^\ast$        & complex conjugate of $x$ where $x\in\C$ & see Section~\ref{sec:conjugate} page~\pageref{sec:conjugate} \\
   $\Re  $         & real part                   \\
   $\Im  $         & imaginary part              \\
   $i$             & imaginary number $i\eqd(0,1)\in\C$ & see Section~\ref{sec:i=(0,1)} page~\pageref{sec:i=(0,1)}
\end{tabular}

%--------------------------------------
\subsubsection*{Miscellaneous operations}
%--------------------------------------
\begin{tabular}{lll}
   $\sinc{\theta}$ & sinc function & $\sinc{\theta}\eqd\frac{\sin\theta}{\theta}$ \\
   $\snr$          & signal to noise ratio \\
   $\conv$         & convolution                & see Section~\ref{sec:conv} page~\pageref{sec:conv} \\
   $\flE $         & energy                      \\
   $\delta$        & Dirac delta distribution   & see Section~\ref{sec:dirac} page~\pageref{sec:dirac} \\
   $\kdelta$       & kronecker delta function   & $\ds \kdelta_{mn}\eqd \left\{ \begin{array}{lll}1&:&m=n\\0&:&m\ne n\end{array}\right.$ \\
   $\sym$          & signalling waveform         \\
   $\chi$          & characteristic function    & $
                                                  \chi_A(x) \eqd
                                                  \left\{
                                                  \begin{tabular}{llll}
                                                     1 & : $x\in A$ \\
                                                     0 & : $x\notin A$
                                                  \end{tabular}
                                                  \right.$ \\
   $\mod[M]$       & modulo (M)           \\
\end{tabular}

%--------------------------------------
\subsubsection*{Acronyms}
%--------------------------------------
\begin{tabular}{ll}
   DSP        & Digital Signal Processing \\
   LTI        & Linear Time Invariant     \\
   WSS        & wide sense stationary     \\
   pn         & pseudo-noise
\end{tabular}









%---------------------------------------
% Symbol Index (page xi)
%---------------------------------------
%\cleartooddpage
%\section*{Symbol index}
%\thispagestyle{empty}%
%\markboth{{Symbol index}}{{Symbol index}}
%\addcontentsline{toc}{section}{Symbol Index}
%\begin{multicols}{4}
%  %============================================================================
% LaTeX File
% Daniel J. Greenhoe
%============================================================================
\documentclass [12pt,twoside,final]{book}
%--------------------------------------
% version
%--------------------------------------
\newcommand{\version}{0.00x}
%--------------------------------------
% Style Packages
%--------------------------------------
 \usepackage{natbib}                  % Bibliographic style: must be loaded before {imakeidx}?
 \usepackage{../common/sty/packages}  % common packages
 \usepackage{../common/sty/switches}  %
 \usepackage{../common/sty/fonts}     % fonts (using fontspec environment)
 \usepackage{../common/sty/dan}       % common definitions
 \usepackage{../common/sty/colors_rgb}% colors
 \usepackage{../common/sty/math}      % common mathematics definitions
 \usepackage{../common/sty/wavelets}  % wavelets
 \usepackage{../common/sty/xsd}       %
 \usepackage[ISBN=000-0-0000000-0-0]{ean13isbn}
 \usepackage{../common/sty/defaults}  %

%--------------------------------------
% flags
%--------------------------------------
\setboolean{wsd}{true}
%--------------------------------------
% color model
%--------------------------------------
%\selectcolormodel{cmyk}% use cmyk color model
 \selectcolormodel{rgb}%  use rgb  color model
%\selectcolormodel{gray}% use gray-scale color model

%--------------------------------------
% page layout
%--------------------------------------
\geometry{
  showframe=false,showcrop=false,
  }

%--------------------------------------
% Headers and Footers
%--------------------------------------
\footerinfo{0}{Operations on Sequeces}{}
%---------------------------------------
% pdf metadata
%---------------------------------------
\hypersetup{%
  pdfkeywords={DSP},
  pdftitle={Operations on sequences},
  pdfdisplaydoctitle=false,
  %pdfsubject={(msc2010)03B50,03B47;03B60,03G05,03G10,03B52.}
  }
%--------------------------------------
% index support with \usepackage{index}
%--------------------------------------
%\newindex{xref}{rdx}{rnd}{Index of References } % index of references
%\newindex{xsym}{sdx}{snd}{Index of Symbols } % index of symbols
%\makeindex         % make the index
%--------------------------------------
% index support with \usepackage{imakeidx}
%--------------------------------------
 \makeindex             % General   index
 \makeindex[name=xref]  % Reference index
 \makeindex[name=xsym]  % Symbol    index
%--------------------------------------
% document content
%--------------------------------------
\dochas{frontmat}
\dochas{backmat}

% main wavelet material
%-----------------------------
 \dochas{dtft}
 \dochas{transop}
 \dochas{mra}
 \dochas{wavstrct}
 \dochas{partuni}
 \dochas{pounity}
 \dochas{vanish}
 \dochas{ortho}
 \dochas{compactp}
 \dochas{symlets}

% Foundational Structures
%-----------------------------
 \dochas{relation}
%\dochas{order}
%\dochas{lattice}
% \dochas{measure}
% \dochas{setstrct}

% Topological Structure
%-----------------------------
% \dochas{topology}
% \dochas{vstopo}
% \dochas{vector}
% \dochas{metric}
% \dochas{vsnorm}
% \dochas{vsinprod}
% \dochas{functionals}
  \dochas{operator}
  \dochas{integrat}

% Basis Theory
%-----------------------------
%\dochas{subspace}
%\dochas{seq}
\dochas{sums}
\dochas{series}
\dochas{frames}

% Fourier Structure
%-----------------------------
 \dochas{harTrig}
 \dochas{harFour}
 \dochas{fs}
 \dochas{harPoly}

% Algebraic Tools
%-----------------------------
 \dochas{algebra}
 \dochas{binomial}
 \dochas{polynom}
 \dochas{spline}
%\dochas{bwav}
 \dochas{interpo}

% Flotation Devices
%-----------------------------
 \dochas{pollen}
 \dochas{fwt}
 \dochas{dsp}
 \dochas{pwrspec}
 \dochas{src_code}

% other
%-----------------------------
%\dochas{logic}
%\dochas{latd}
%\dochas{latm}
%\dochas{latvar}
%\dochas{boolean}
%\dochas{ortholat}
%\dochas{convex}
 \dochas{morph}
%\dochas{found}
%\dochas{numsys}
%\dochas{metricex}

%----------------------------------------------------------------------------
% Body
%----------------------------------------------------------------------------
\begin{document}
\VerbatimFootnotes   % macro to allow verbatim in footnotes (used with fancyvrb package)
%--------------------------------------
% front matter
%--------------------------------------
\include{front}
%--------------------------------------
% main matter
%--------------------------------------
 \ifDocHasCommonInclude{dsp}
 \ifDocHasCommonInclude{dtft}
\include{../common/src}
\include{../common/ztricks}
\include{../common/zchar}
\include{../common/zcalc}
\include{../common/dspcalculus}




%--------------------------------------
% appendixes
%--------------------------------------
\part{Appendices}
\begin{appendix}

  \ifDocHasCommonInclude{transop}
  \ifDocHasCommonInclude{mra}
  \ifDocHasWaveletsInclude{fwt}
%  \ifDocHasWaveletsInclude{pounity}
%  \ifDocHasWaveletsInclude{wavstrct}
%  \ifDocHasWaveletsInclude{vanish}
%  \ifDocHasWaveletsInclude{ortho}
%  \ifDocHasWaveletsInclude{compactp}
%  \ifDocHasWaveletsInclude{symlets}
%
%
%  %\renewcommand*{\thechapter}{\AlphAlph{\value{chapter}}}
%  %\part{Foundational Structures}
    \ifDocHasCommonInclude{relation}
%    \ifDocHasCommonInclude{order}
%    \ifDocHasCommonInclude{setstrct}
%    \ifDocHasCommonInclude{algebra}
%    \ifDocHasCommonInclude{measure}
%    \ifDocHasCommonInclude{binomial}
%    \ifDocHasCommonInclude{polynom}
%
%  %\part{Topological Structure}% \vfill\mbox{}\hfill\includegraphics*[height=15mm]{../common/graphics/watercraft/ghindgray_djg.eps}%
%    \ifDocHasCommonInclude{topology}
%    \ifDocHasCommonInclude{metric}
%    \ifDocHasCommonInclude{vector}
%    \ifDocHasCommonInclude{vstopo}
%    \ifDocHasCommonInclude{functionals}
%    \ifDocHasCommonInclude{vsnorm}
%    \ifDocHasCommonInclude{vsinprod}
%    \ifDocHasCommonInclude{operator}
    \ifDocHasCommonInclude{integrat}
    \include{../common/calculus}
%
%  %\part{Basis Theory}% \vfill\mbox{}\hfill\includegraphics*[height=15mm]{../common/graphics/watercraft/ghindgray_djg.eps}%
%    \ifDocHasCommonInclude{subspace}
%    \ifDocHasCommonInclude{ortholat}
%    \ifDocHasCommonInclude{seq}
%    \ifDocHasCommonInclude{sums}
%    \ifDocHasCommonInclude{series}
% 
%  %\part{Fourier Structure}% \vfill\mbox{}\hfill\includegraphics*[height=15mm]{../common/graphics/watercraft/ghindgray_djg.eps}%
    \ifDocHasCommonInclude{harTrig}
    \ifDocHasCommonInclude{harPoly}
    \ifDocHasCommonInclude{fs}
    \ifDocHasCommonInclude{harFour}
%
%   \ifDocHasCommonInclude{frames}
%
%%  \part{Measure Structure}% \vfill\mbox{}\hfill\includegraphics*[height=15mm]{../common/graphics/watercraft/ghindgray_djg.eps}%
%
%  %\part{Algebraic Tools}% \vfill\mbox{}\hfill\includegraphics*[height=15mm]{../common/graphics/watercraft/ghindgray_djg.eps}%
%    \ifDocHasCommonInclude{spline}
%    \ifDocHasCommonInclude{bwav}
%    \ifDocHasCommonInclude{interpo}
%    \ifDocHasCommonInclude{partuni}
%    \ifDocHasCommonInclude{pwrspec}
%
%  %\part{Support Tools}
%  %\part{Riptide}
%  %\part{Flotation Devices}% \vfill\mbox{}\hfill\includegraphics*[height=15mm]{../common/graphics/watercraft/ghindgray_djg.eps}%
%    \ifDocHasWaveletsInclude{pollen}
%    \ifDocHasCommonInclude{morph}
    \ifDocHasInclude{src_code}
\end{appendix}

\ifDocHasCommonInclude{backmat}
%--------------------------------------
% end document
%--------------------------------------

\end{document}


%\end{multicols}
%---------------------------------------
% blank page (page xii)
%---------------------------------------
%\cleartoevenpage
%\markboth{}{}
%---------------------------------------
% Contents (pages xiii--xix)
%---------------------------------------
\cleartooddpage
\markboth{{Contents}}{{Contents}}
\addcontentsline{toc}{section}{Contents}%
{
  %
  \setlength{\parskip}{0em}%
  {\sffamily\footnotesize\tableofcontents}%
  %
  %\listoffigures
  %\listoftables
}
%---------------------------------------
% preface
%---------------------------------------
\clearpage
%============================================================================
% LaTeX File
% Daniel J. Greenhoe
%============================================================================

%=======================================
\section*{Preface}
\addcontentsline{toc}{section}{Preface}
\markboth{{Preface}}{{Preface}}
%=======================================
\qboxnps
  {
    \href{http://en.wikipedia.org/wiki/Carl_Friedrich_Gauss}{Karl Friedrich Gauss}
    (1777--1855), German mathematician
    \index{Gauss, Karl Friedrich}
    \index{quotes!Gauss, Karl Friedrich}
    \footnotemark
  }
  {../common/people/small/gauss.jpg}
  {You know that I write slowly.
    This is chiefly because I am never satisfied until I have said as much as
    possible in a few words,
    and writing briefly takes far more time than writing at length.}
  \citetblt{
    quote:  \citerp{simmons2007}{177} \\
    image:  \url{http://en.wikipedia.org/wiki/Karl_Friedrich_Gauss}
    }



%\qboxnps
%  {
%    \href{http://en.wikipedia.org/wiki/Carl_Friedrich_Gauss}{Karl Friedrich Gauss}
%    (1777--1855), German mathematician
%    \index{Gauss, Karl Friedrich}
%    \index{quotes!Gauss, Karl Friedrich}
%    \footnotemark
%  }
%  %{../common/people/small/gauss1828.jpg}
%  {../common/people/gauss.jpg}
%  {I mean the word proof not in the sense of lawyers,
%    who set two half proofs equal to a whole one,
%    but in the sense of the mathematician, where \textonehalf proof = 0
%    and it is demanded for proof that every doubt becomes impossible.}
%  \citetblt{
%    quote:  \citerp{simmons2007}{177} \\
%    %image:  \url{http://www-history.mcs.st-andrews.ac.uk/PictDisplay/Gauss.html}
%    image:  \url{http://en.wikipedia.org/wiki/Karl_Friedrich_Gauss}
%    }

%
%\qboxnps
%  {
%    \href{http://en.wikipedia.org/wiki/Enrico_Fermi}{Enrico Fermi}
%    \href{http://www-history.mcs.st-andrews.ac.uk/Timelines/TimelineG.html}{(1777--1855)},
%    \href{http://www-history.mcs.st-andrews.ac.uk/BirthplaceMaps/Places/Italy.html}{Italian physicist}
%    \index{Fermi, Enrico}
%    \index{quotes!Fermi, Enrico}
%    \footnotemark
%  }
%  {../common/people/small/fermi.jpg}
%  {If it is true, it can be proved.}
%  \citetblt{
%    quote:  \citerp{benedetto}{85} \\
%    image:  \url{http://www-history.mcs.st-andrews.ac.uk/Mathematicians/Fermi.html}
%    }


%Aubrey, John (1626-1697)
%[About Thomas Hobbes:]
%He was 40 years old before he looked on geometry; which happened accidentally. Being in a gentleman's library, Euclid's Elements lay open, and "twas the 47 El. libri I" [Pythagoras' Theorem]. He read the proposition . "By God", sayd he, "this is impossible:" So he reads the demonstration of it, which referred him back to such a proposition; which proposition he read. That referred him back to another, which he also read. Et sic deinceps, that at last he was demonstratively convinced of that trueth. This made him in love with geometry.
%In O. L. Dick (ed.) Brief Lives, Oxford: Oxford University Press, 1960, p. 604.
%http://math.furman.edu/~mwoodard/ascquota.html

This book is largely based on four papers written by the same author as this text:

\begin{enumerate}
  \item \citeP{greenhoe2015sto}: \emph{Order and metric geometry compatible stochastic processing}.
    \\\url{https://peerj.com/preprints/844/}
    \\\url{https://www.researchgate.net/publication/303057738}
    \\This paper is the basis for \pref{chp:sto}.
  \item \citeP{greenhoe2015ssp}: \emph{Order and Metric Compatible Symbolic Sequence Processing}.
    \\\url{https://peerj.com/preprints/2052/}
    \\\url{https://www.researchgate.net/publication/302947227}
    \\This paper is the basis for \pref{chp:ssp}.
    \\The C++ source code used to generate the 128 or so TeX files for the data plots in the paper can be downloaded from here:
    \\\url{https://www.researchgate.net/publication/302953844}
  \item \citeP{greenhoe2015larc}: \emph{An extension to the spherical metric using polar linear interpolation}.
    \\\url{https://www.researchgate.net/publication/303984226}
    \\This paper is the basis for \pref{app:larc}.
  \item \citeP{greenhoe2015pds}: \emph{Properties of distance spaces with power triangle inequalities}.
    \\\url{http://www.journals.pu.if.ua/index.php/cmp/article/view/483}
    \\\url{https://peerj.com/preprints/2055}
    \\\url{https://www.researchgate.net/publication/281831459}
    \\This paper is the basis for \pref{app:distance} and \pref{app:pds}.
    It has been published in the 2016 volume 8 number 1 edition of the 
    journal \emph{Carpathian Mathematical Publications}.
\end{enumerate}




% %============================================================================
% LaTeX File
% Daniel J. Greenhoe
%============================================================================

%=======================================
%\chapter{Introduction to the structure and design series}
\chapter{Introduction}
%\addcontentsline{toc}{section}{Introduction to the structure and design series}
%=======================================

%=======================================
\section*{Mathematics as art}
%=======================================
\qboxnps
  {\href{http://en.wikipedia.org/wiki/Aristotle}{Aristotle}
   \href{http://www-history.mcs.st-andrews.ac.uk/Timelines/TimelineA.html}{384 BC -- 322 BC},
   \href{http://www-history.mcs.st-andrews.ac.uk/BirthplaceMaps/Places/Greece.html}{Greek philosopher}
   \index{Aristotle}
   \index{quotes!Aristotle}
   \footnotemark
  }
  {../common/people/small/aristotle.jpg}
  {\ldots those who assert that the mathematical sciences
     say nothing of the beautiful or the good are in error.
     For these sciences say and prove a great deal about them;
     if they do not expressly mention them, but prove attributes
     which are their results or definitions, it is not true that they tell
     us nothing about them.
     The chief forms of beauty are order and symmetry and definiteness,
     which the mathematical sciences demonstrate in a special degree.}
  \citetblt{
    %quote: & \citerc{aristotle}{paragraphs 34--35?} \\
    quote: & \citerc{aristotle_metaphysics}{Book XIII Part 3} \\
    %       & \url{http://en.wikiquote.org/wiki/Aristotle} \\
    %image: & \url{http://en.wikipedia.org/wiki/Aristotle}
    image: & \url{http://upload.wikimedia.org/wikipedia/commons/9/98/Sanzio_01_Plato_Aristotle.jpg}
    }

\qboxnpq
  {\href{http://en.wikipedia.org/wiki/Norbert_Wiener}{Norbert Wiener}
   \href{http://www-history.mcs.st-andrews.ac.uk/Timelines/TimelineG.html}{(1894--1964)},
   \href{http://www-history.mcs.st-andrews.ac.uk/BirthplaceMaps/Places/USA.html}{American mathematician}
    \index{Wiener, Norbert}
    \index{quotes!Wiener, Norbert}
    \footnotemark
  }
  {../common/people/small/wiener.jpg}
  {The musician regarded me as heavy-handed and Philistine.
   This was partly because of my actual social ineptitude and bad manners,
   but it was also due to the fact that he considered that mathematics
   by its own nature stood in direct opposition to the arts.
   On the other hand, I maintained the thesis of this book:
   that mathematics is essentially one of the arts;}
   %and I dingdonged on this theme far too much for the patience of a man
   %initially disposed to hate mathematics for its own sake.
   %Later on we got into an explicit quarrel,
   %in which we really said the unpleasant things we thought of one another,
   %and this finally cleared up into a certain degree of understanding
   %and even of a limited friendship.}
  \citetblt{
    quote: & \citerp{wiener}{65} \\
    image: & \url{http://www-history.mcs.st-andrews.ac.uk/PictDisplay/Wiener_Norbert.html}
    }


\qboxnps
  {\href{http://en.wikipedia.org/wiki/G._H._Hardy}{G.H. Hardy}
   \href{http://www-history.mcs.st-andrews.ac.uk/Timelines/TimelineG.html}{(1877--1947)},
   \href{http://www-history.mcs.st-andrews.ac.uk/BirthplaceMaps/Places/UK.html}{English mathematician}
    \index{Hardy, G.H.}
    \index{quotes!Hardy, G.H.}
    \footnotemark
  }
  {../common/people/small/hardy.jpg}
  {A mathematician, like a painter or a poet, is a maker of patterns. 
   If his patterns are more permanent than theirs, 
   it is because they are made with \emph{ideas}.}
  \citetblt{
    quote: & \citerc{hardy1940}{section 10} \\
    image: & \url{http://www-history.mcs.st-andrews.ac.uk/PictDisplay/Hardy.html}
    }




Just as Plato pointed out that rhetoric is
one of the arts\citetblt{\citerp{plato1875}{320}},
so Norbert Wiener pointed out that mathematics is also
one of the arts.
And as in all the arts, value is not primarily measured by
the measure of utility in real world applications,
but by the measure of beauty that it helps create.



%=======================================
\section*{Structure of mathematics}
%=======================================
\qboxnpqt
  {Ren\'e Descartes, philosopher and mathematician (1596--1650)
   \index{Descartes, Ren\'e}
   \index{quotes!Descartes, Ren\'e}
   \footnotemark}
  {../common/people/small/descart.jpg}
  {Je me plaisois surtout aux math\'ematiques,
    \`a cause de la certitude et de l'\'evidence de leurs raisons:
    mais je ne remarquois point encore leur vrai usage;
    et, pensant qu'elles ne servoient qu'aux arts m\'ecaniques,
    je m'\'etonnois de ce que leurs fondements \'etant si fermes et si solides,
    on n'avoit rien b\^ati dessus de plus relev\'e:}
  {I was especially delighted with the mathematics,
    on account of the certitude and evidence of their reasonings;
    but I had not as yet a precise knowledge of their true use;
    and thinking that they but contributed to the advancement of the mechanical arts,
    I was astonished that foundations, so strong and solid,
    should have had no loftier superstructure reared on them.}
  \citetblt{
    quote: & \citer{descartes_method} \\
    translation: & \citerc{descartes_method_eng}{part I, paragraph 10} \\
    image: & \url{http://en.wikipedia.org/wiki/Image:Descartes_Discourse_on_Method.png}
    }

\qboxnpqt
  { Jules Henri Poincar\'e (1854-1912), physicist and mathematician
    \index{Poincar\'e, Jules Henri}
    \index{quotes!Poincar\'e, Jules Henri}
    \footnotemark
  }
  {../common/people/small/poincare.jpg}
  {\ldots on fait la science avec des faits comme une maison avec des pierres ; 
   mais une accumulation de faits n'est pas plus une science qu'un tas de 
   pierres n'est une maison.}
  {Science is built up of facts, as a house is built of stones;
   but an accumulation of facts is no more a science than a heap of stones is a house.}
  \citetblt{
    quote:       & \citerc{poincare_sah}{Chapter IX, paragraph 7} \\
    translation: & \citerp{poincare_sah_eng}{141} \\
    image:       & \url{http://www-groups.dcs.st-and.ac.uk/~history/PictDisplay/Poincare.html}
    }


\qboxnps
  {
    Freeman Dyson (1923--2020), physicist and mathematician  %(January 1994)
    \index{Dyson, Freeman}
    \index{quotes!Dyson, Freeman}
    \footnotemark
  }
  {../common/people/small/dyson.jpg}
  {The bottom line for mathematicians is that the architecture has to be right.
    In all the mathematics that I did, the essential point was to find
    the right architecture.
    It's like building a bridge.
    Once the main lines of the structure are right,
    then the details miraculously fit.
    The problem is the overall design.}
  \citetblt{
    quote: & \citerp{dyson1994}{20}  \\
    image: & \url{http://en.wikipedia.org/wiki/Image:FreemanDysonOSCON2004.jpg}
    }

Just as in paintings and architectural works, 
the art that is mathematics is often demonstrated in the structure inherent in it.
This text tries to show some of that structure.

%=======================================
\section*{Connected concepts}
%=======================================

\qboxnpqt
  { \href{http://en.wikipedia.org/wiki/Joseph_Louis_Lagrange}{Joseph-Louis Lagrange}
    (\href{http://www-history.mcs.st-andrews.ac.uk/Timelines/TimelineD.html}{1736--1813},
     \href{http://www-history.mcs.st-andrews.ac.uk/BirthplaceMaps/Places/Italy.html}{Italian-French mathematician and astronomer}
    \index{Langrange, Joseph-Louis}
    \index{quotes!Langrange, Joseph-Louis}
    \footnotemark
  }
  {../common/people/small/lagrange.jpg}
  {Tant que l'Alg\`ebre et la G\'eom\'etrie ont \'et\'e s\'epar\'ees,
   leurs progr\`es ont \'et\'e lents et leurs usages born\'es;
   mais lorsque ces deux sciences se sont r\'eunies, elles vers la perfection.}
  {As long as algebra and geometry have been separated,
   their progress have been slow and their uses limited;
   but when these two sciences have been united, they have lent each other mutual forces,
   and have marched together with a rapid step towards perfection.
  }
  \citetblt{
    quote:       & \citerp{lagrange1795}{271} \\
    translation: & \citerpg{grattan1990}{254}{3764322373} \\
%                 & \url{http://www-groups.dcs.st-and.ac.uk/~history/Quotations/Lagrange.html} \\
    image:       & \url{http://en.wikipedia.org/wiki/Joseph_Louis_Lagrange}
    }


\qboxnps
  {\href{http://en.wikipedia.org/wiki/G._H._Hardy}{G.H. Hardy}
   \href{http://www-history.mcs.st-andrews.ac.uk/Timelines/TimelineG.html}{(1877--1947)},
   \href{http://www-history.mcs.st-andrews.ac.uk/BirthplaceMaps/Places/UK.html}{English mathematician}
    \index{Hardy, G.H.}
    \index{quotes!Hardy, G.H.}
    \footnotemark
  }
  {../common/people/small/hardy.jpg}
  {The ``seriousness" of a mathematical theorem lies,
    not in its practical consequences,
    which are usually negligible,
    but in the {\em significance} of the mathematical ideas which it connects.
    We may say, roughly, that a mathematical idea is ``significant" if it can be
    connected, in a natural illuminating way,
    with a large complex of other mathematical ideas.}
  \citetblt{
    quote: & \citerc{hardy1940}{section 11} \\
    image: & \url{http://www-history.mcs.st-andrews.ac.uk/PictDisplay/Hardy.html}
    }


Mathematics is not simply a collection of definitions and equations.
Rather it is a carefully built \emph{structure} of \emph{connected} concepts.
And the purpose of this text is to present the foundations of mathematics
in a {\em structured} and {\em connected} manner.

That is, we start with the most basic and fundamental concepts,
and build upon them other structures.
But critical to any structure is the connections between components.
As pointed out by Hardy, the value of a mathematical concept is not primarily in
it's applicability to real world problems,
but rather in how well it is connected too and helps connect other mathematical concepts.

%=======================================
\section*{Abstraction}
%=======================================
\qboxnps
  { attributed to Karl Gustav Jakob Jacobi (1804--1851), mathematician
    \index{Jacobi, Karl Gustav Jakob}
    \index{quotes!Jacobi, Karl Gustav Jakob}
    \footnotemark
  }
  {../common/people/small/jacobi.jpg}
  {Man muss immer generalisieren.\quotec \\
   \quoteo One should always generalize.}
  \citetblt{
    %quote: & \url{http://en.wikiquote.org/wiki/Gustav_Jacobi} \\
     quote: & \citerpg{davis1999}{134}{0395929687} \\
     image: & \url{http://en.wikipedia.org/wiki/Carl_Gustav_Jacobi}
    }

\qboxnpqt
  { Jules Henri Poincar\'e (1854-1912), physicist and mathematician
    \index{Poincar\'e, Jules Henri}
    \index{quotes!Poincar\'e, Jules Henri}
    \footnotemark
  }
  {../common/people/small/poincare.jpg}
  {Les math\'ematiciens n'\'etudient pas des objets, 
   mais des relations entre les objets; 
   il leur est donc indiff\'erent de remplacer ces objets par d'autres, 
   pourvu que les relations ne changent pas. 
   La mati\`ere ne leur importe pas, la forme seule les int\'eresse.}
  {Mathematicians do not study objects, but the relations between objects;
   to them it is a matter of indifference if these objects are replaced by others,
   provided that the relations do not change.
   Matter does not engage their attention,
   they are interested in form alone.}
  \citetblt{
    quote:       & \citerc{poincare_sah}{Chapter 2} \\
    translation: & \citerp{poincare_sah_eng}{20} \\
    %image:       & \url{http://en.wikipedia.org/wiki/Image:Poincare_jh.jpg}
    image:       & \url{http://www-groups.dcs.st-and.ac.uk/~history/PictDisplay/Poincare.html}
    }

\qboxnqt
  {
    Jules Henri Poincar\'e (1854-1912), physicist and mathematician
    \index{Poincar\'e, Jules Henri}
    \index{quotes!Poincar\'e, Jules Henri}
    \footnotemark
  }
  %{../common/people/small/poincare.jpg}
  {Je ne sais si je n'ai d\'ej\`a` dit quelque part que la Math\'ematique est 
  l'art de donner le m\^eme nom \`a des choses diff\'erentes. 
  Il convient que ces choses, diff\'erentes par la mati\`ere, 
  soient semblables par la forme, qu'elles puissent, 
  pour ainsi dire, se couler dans le m\^eme moule. 
  Quand le langage a \'et\'e bien choisi, on est tout \'etonn\'e 
  de voir que toutes les d\'emonstrations, faites pour un objet connu, 
  s'appliquent imm\'ediatement \`a beaucoup d'objets nouveaux ; 
  on n'a rien \`a y changer, pas m\^eme les mots, puisque les noms sont devenus les m\^emes.}
  %
  {I think I have already said somewhere that mathematics is the art
   of giving the same name to different things. 
   It is enough that these things, though differing in matter, 
   should be similar in form, to permit of their being, so to speak,
   run in the same mould.
   When language has been well chosen, one is astonished to find that all
   demonstrations made for a known object apply immediately to many new objects:
   nothing requires to be changed, not even the terms,
   since the names have become the same.}
  \citetblt{
    quote:   & \citerc{poincare_sam}{book 1, chapter 2, paragraph 20} \\
             & \url{http://fr.wikisource.org/wiki/Science_et_m\%C3\%A9thode_-_Livre_premier\%2C_\%C2\%A7_II} \\
    trans.:  & \citerp{poincare_sam_eng}{34} \\
   %image:        & \url{http://en.wikipedia.org/wiki/Image:Poincare_jh.jpg}
    }

The twentieth century was the century of abstraction in mathematics.%
\footnote{This concept of generalization so prevelant in 
twentieth century mathematics was well described by Van de Vel:
``It is typical of an axiomatic approach not to emphasize what an object \emph{\bf represents},
but rather how it \emph{\bf behaves}."---\citerp{vel1993}{3}
}
Here are some examples:\citetbl{
  \citor{peano1888} \\
  \citor{weber1893} \\
  \citor{dedekind1900} \\
  \citor{frechet1906} \\
  \citor{hausdorff1914} \\
  \citor{banach1922} \\
  \citor{vonNeumann1929} \\
  \citor{birkhoff1948}
  }
\begin{liste}
  \item In 1888, Giuseppe Peano introduced the \hie{vector space}, 
        a generalization of real functions.

  \item In 1893, Heinrich Weber introduced the algebraic \hie{field}, 
        a generalization of the real numbers and associated arithmetic.

  \item In 1900, Richard Dedekind introduced \hie{modularity}, a generalization of
        \prop{distributivity}.

  \item In 1906, Maurice Ren\'e Fr\'echet
        introduced the concepts of the \hie{abstract space} as well as the 
        \hie{metric space}, which generalize concepts in analysis involving 
        convergence.

  \item In 1914, Felix Hausdorff introduced the modern concept of the 
        \hie{topological space}, a further generalization of the 
        already very general metric space.

  \item In 1922, Stephen Banach introduced the \hie{normed linear space},
        a generalization of operations involving integration.

  \item In 1929, John von Neumann introduced the \hie{Hilbert space}, a 
        generalization of earlier work by Hilbert involving integral operations.

  \item In 1948, Garrett Birkhoff introduced the \hie{lattice},
        a generalization involving partially ordered sets.
\end{liste}

This text presents the most general and abstract structures first,
followed by progressively more specific structures.
The idea is, that we try to prove as much as possible in the most general setting;
and then we only need to prove that a specific structure is a special case of the general
structure and hence can simply \hie{inherit} the properties of that general structure
without having to prove everything all over again.


%=======================================
\section*{Concrete and specific cases}
%=======================================

\qboxnps
  {\href{http://en.wikipedia.org/wiki/Paul_Halmos}{Paul R. Halmos}
   \href{http://www-history.mcs.st-andrews.ac.uk/Timelines/TimelineG.html}{(1916--2006)},
   \href{http://www-history.mcs.st-andrews.ac.uk/BirthplaceMaps/Places/Germany.html}{Hungarian-born American mathematician}
   \index{Halmos, Paul R.}
   \index{quotes!Halmos, Paul R.}
   \footnotemark
  }
  {../common/people/small/halmos.jpg}
  {\ldots the source of all great mathematics is the special case,
    the concrete example.
    It is frequent in mathematics that every instance of a concept of seemingly
    great generality is in essence the same as a small and concrete special case.}
  \citetblt{
    quote: & \citer{halmos1985} \\
    image: & \url{http://en.wikipedia.org/wiki/Image:Paul_Halmos.jpeg}
    }



\qboxnps
  {attributed to \href{http://en.wikipedia.org/wiki/Hilbert}{David Hilbert}
   \href{http://www-history.mcs.st-andrews.ac.uk/Timelines/TimelineF.html}{(1862--1943)},
   \href{http://www-history.mcs.st-andrews.ac.uk/BirthplaceMaps/Places/Germany.html}{German mathematician}
    \index{Hilbert, David}
    \index{quotes!Hilbert, David}
    \footnotemark
  }
  {../common/people/small/hilbert.jpg}
  {The art of doing mathematics consists in finding
    that special case which contains all the germs of generality.}
  \citetblt{
    quote: & \citer{rose1988} \\
    image: & \url{http://en.wikipedia.org/wiki/Image:Hilbert.JPG}
    }

%2014jan05sunday
%Hermann Weyl:
%    ``Nevertheless I should not pass over in silence the fact that today the feeling among mathematicians is beginning to spread that the fertility of these abstracting methods is approaching exhaustion. The case is this: that all these nice general concepts do not fall into our laps by themselves. But definite concrete problems were first conquered in their undivided complexity, singlehanded by brute force, so to speak. Only afterwards the axiomaticians came along and stated: Instead of breaking the door with all your might and bruising your hands, you should have constructed such and such a key of skill, and by it you would have been able to open the door quite smoothly. But they can construct the key only because they are able, after the breaking in was successful, to study the lock from within and without. Before you can generalize, formalize, and axiomatize, there must be a mathematical substance."
%Pioneers of Representation Theory: Frobenius, Burnside, Schur, and Brauer, by Charles W. Curtis, pg 210.
%http://www.plambeck.org/archives/001272.html

\qboxnpq
  {
    Hermann Weyl (1885--1955); mathematician, theoretical physicist, and philosopher
    \index{Weyl, Hermann}
    \index{quotes!Weyl, Hermann}
    \footnotemark
  }
  {../common/people/weylhermann_wkp_free.jpg}
  {Nevertheless I should not pass over in silence the fact that today the 
   feeling among mathematicians is beginning to spread that the fertility 
   of these abstracting methods is approaching exhaustion. 
   The case is this: that all these nice general concepts do not fall into our laps by themselves. 
   But definite concrete problems were first conquered in their undivided complexity, 
   singlehanded by brute force, so to speak. Only afterwards the axiomaticians came along and stated: 
   Instead of breaking the door with all your might and bruising your hands, 
   you should have constructed such and such a key of skill, 
   and by it you would have been able to open the door quite smoothly. 
   But they can construct the key only because they are able, after the breaking in was successful, 
   to study the lock from within and without. 
   Before you can generalize, formalize, and axiomatize, there must be a mathematical substance.}
  \footnotetext{
    quote: \citePpc{weyl1935}{14}{H. Weyl, quoting himself in ``a conference on topology and abstract algebra as two ways of mathematical understanding, in 1931"}. 
    image: \url{https://en.wikipedia.org/wiki/File:Hermann_Weyl_ETH-Bib_Portr_00890.jpg}: ``This work is free and may be used by anyone for any purpose."
    }

It is general abstract concepts that allows us to easily see the connectivity
between multiple mathematical ideas and to more easily develop new ideas.
However, it is concrete specific examples that often make a general abstract concept clear
and more easily remembered in the future; and it may be argued that it is the concrete
examples that give the general abstract concepts meaning and significance.
This text provides many concrete examples that will hopefully greatly clarify
the more general concepts;
and will hopefully also help give value to the abstract concepts.

%=======================================
\section*{Applications}
%=======================================

\qboxnpq
  {
    Joseph Louis Lagrange (1736--1813), mathematician
    \index{Lagrange, Joseph Louis}
    \index{quotes!Lagrange, Joseph Louis}
    \footnotemark
  }
  {../common/people/small/lagrange.jpg}
  {I regard as quite useless the reading of large treatises of pure analysis:
    too large a number of methods pass at once before the eyes.
    It is in the works of applications that one must study them;
    one judges their ability there and one apprises the manner of making use of them.}
  \citetblt{
    quote: &  \citerp{stopple2003}{xi} \\
          %&  \url{http://www.math.okstate.edu/~wli/teach/fmq.html} \\
          %&  \url{http://www-groups.dcs.st-and.ac.uk/~history/Quotations/Lagrange.html} \\
    image: & \url{http://en.wikipedia.org/wiki/Image:Langrange_portrait.jpg}
    }

\qboxnps
  {
    \href{http://en.wikipedia.org/wiki/Eric_temple_bell}{Eric Temple Bell}
    (1883--1960), Scotish-American mathematician
    \index{Bell, Eric Temple}
    \index{quotes!Bell, Eric Temple}
    \footnotemark
  }
  {../common/people/small/bell.jpg}
  {The pursuit of pretty formulas and neat theorems
    can no doubt quickly degenerate into a silly vice,
    but so can the quest for austere generalities which are so very general indeed
    that they are incapable of application to any particular.}
  \citetblt{
    %quote: & \citer{eves1972} \\
    quote: & \citerp{bell1986}{488} \\
    image: & \url{http://www-history.mcs.st-andrews.ac.uk/PictDisplay/Bell.html}
    }



\begin{minipage}{4\tw/16}%
  \includegraphics*[width=\tw, keepaspectratio=true, clip=true]
  {../common/people/small/plimpton322.jpg}\footnotemark
\end{minipage}%
\footnotetext{\hie{Plimpton 322}: One of the most famous mathematical tablets from Ancient Babylon.
  Image source: \url{http://en.wikipedia.org/wiki/Plimpton_322}
  }%
\hfill
\begin{minipage}{10\tw/16}%
  Mathematics is a very old art form.
  Archeologists have found thousands of mathematical tablets among the ruins of the ancient
  Babylonians, buried in the sand for 4000 years but still readable (by some) today.
  Mathematics has for centuries been held in esteem because of its ability to
  solve practical problems.
\end{minipage}
Before the century of generalization---the twentieth century---and the 
century of precision---the nineteenth century---came several centuries of 
applications.
In fact, mathematics came to enjoy one of its finest hours during the
European Renaissance.
During this time, mathematics was brought to bear by so many to solve so many
extremely practical problems.
One of the most famous examples is Isaac \prop{Newton},
who demonstrated that the motion of objects
as small as an apple to as large as a planet could be accurately expressed and
(more importantly) accurately predicted by the analytical power of the calculus.
\hie{Peter the Great} of Russia, even though he himself did not
know too much about mathematics,
kept \prop{Euler} as a member of his court to work on the solution of very
practical problems. Peter the Great even referred to Euler,
at least in the beginning, as {\em My Professor}
(later, as the relationship became strained, he referred to Euler as
{\em My Cyclops}---Euler only had one good eye at the time).


\begin{figure}
\color{figcolor}
\begin{center}
\begin{fsL}
\input{../common/math/graphics/history/notablemath.tex}
\end{fsL}
\end{center}
\caption{
   Number of notable mathematicians alive over time
   \label{fig:intro_timeline}
   }
\end{figure}

But it is not that applications are no longer important to mathematics,
but rather applications drive mathematics forward.
Evidence of this hypothesis is given in 
\prefpp{fig:intro_timeline}.\footnote{Data for \prefpp{fig:intro_timeline} extracted from \\
  \url{http://www-history.mcs.st-andrews.ac.uk/Timelines/WhoWasThere.html}}
This graph shows the number of ``notable" mathematicians alive during the last 3000 years;
And a ``notable mathematician" is here defined as one whose name appears
in \hie{Saint Andrew's University's} \hie{Who Was There} 
website.\footnote{\url{http://www-history.mcs.st-andrews.ac.uk/Timelines/WhoWasThere.html}}
Note the following:
\begin{liste}

\item The number of mathematicians starts to exponentially increase at 
about the time of Gutenberg's invention of the printing press---
that is, when information of discoveries and results could be widely and economically 
circulated.\footnote{This point is also made by Resnikoff and Wells in\\
\citerp{resnikoff}{9}.}

\item There is another increase after the invention of the slide rule in the early 1600s---
that is, when computational power increased.

\item There are huge increases around the time of the first and second industrial revolutions---
that is, when there were many {\bf applications} that called for mathematical solutions.

\item After the invention of the pocket scientific calculator in 1972 
and home IBM PC in 1981---machines that could 
often make hard-core mathmatical analysis unnecessary in real-world applications---
there was a huge drop in the number of mathematicians.

\end{liste}

The point here is, that however little some mathematicians may think of 
real-world applications,
historically it would seem that applications are critical to 
how well mathematics thrives and develops. 
When mathematics is needed for the development of applications, 
mathematics prospers.
But when mathematics is not viewed as critical to those applications
(such as after the introduction of the personal computer), 
mathematics withers.


%=======================================
\section*{Writing style}
%=======================================
\qboxnpqt
  { %\href{http://www-history.mcs.st-andrews.ac.uk/Biographies/Poincare.html}{Jules Henri Poincar\'e} 
    \href{http://en.wikipedia.org/wiki/Henri_Poincar\%C3\%A9}{Jules Henri Poincar\'e} 
    \href{http://www-history.mcs.st-andrews.ac.uk/Timelines/TimelineF.html}{(1854--1912)}, 
    \href{http://www-history.mcs.st-andrews.ac.uk/BirthplaceMaps/Places/France.html}{French physicist and mathematician}
    \index{Poincar\'e, Jules Henri}
    \index{quotes!Poincar\'e, Jules Henri}
    \footnotemark
  }
  {../common/people/small/poincare.jpg}
  {Ainsi, la logique et l'intuition ont chacune leur r\^ole n\'ecessaire.
    Toutes deux sont indispensables.
    La logique qui peut seule donner la certitude est l'instrument de la d\'emonstration:
    l'intuition est l'instrument de l'invention.}
  {Thus, logic and the intuition each have their necessary r\^ole.
    Each is indispensable.
    Logic, which alone can give certainty, is the instrument of demonstration;
    intuition is the instrument of invention.}
  \citetblt{
    quote:       & \citerc{poincare_vos}{chapter 1 \S V}  \\
    translation: & \citerp{poincare_vos_e}{23} \\
    image:       & \url{http://en.wikipedia.org/wiki/Image:Poincare_jh.jpg}
    }

  \paragraph{Dual structure writing style.}
  Mathematician \href{http://en.wikipedia.org/wiki/Steenrod}{Norman E. Steenrod}
  proposed a mathematical style of writing
  in which there is a distinction between
  ``the {\em formal} or {\em logical} structure consisting of definitions, theorems,
  and proofs, and the complementary
  {\em informal} or {\em introductory} material consisting of motivations, analogies,
  examples, and metamathematical explanations."\citep{steenrod}{1}
  This is the style that largely characterizes this text.
  The overall development is in the definition-lemma-theorem style,
  which seems to be a reliable way to impose rigor on the development.
  The motivation and general discussion are in the more informal style,
  which helps give an intuitive understanding.

\qboxnps
  {
    \href{http://en.wikipedia.org/wiki/Carl_Friedrich_Gauss}{Karl Friedrich Gauss}
    (1777--1855), German mathematician
    \index{Gauss, Karl Friedrich}
    \index{quotes!Gauss, Karl Friedrich}
    \footnotemark
  }
  {../common/people/small/gauss.jpg}
  {You know that I write slowly.
    This is chiefly because I am never satisfied until I have said as much as
    possible in a few words,
    and writing briefly takes far more time than writing at length.}
  \citetblt{
    quote: & \citerp{simmons2007}{177} \\
    image: & \url{http://en.wikipedia.org/wiki/Karl_Friedrich_Gauss}
    }


  \paragraph{Informal structure.}
  More specifically, my guidelines for the informal portions of text are as follows:
  \begin{dingautolist}{"C0}
    \item The text should be terse rather than verbose.
    %\item The text should not be cluttered with supporting background
    %      mathematical material.
    %      Most all such material is placed in appendices at the end of the text.
    \item Key points are often indicated by enclosure in boxes---
          red double boxes for definitions, blue single boxes for theorems,
          curved-corner boxes for examples.
          This makes parsing key points faster and more efficient---
          allowing you to easily distinguish between key results, detailed proofs,
          and general discussion.
    \item Examples will be explicitly worked out, giving readers
          confidence that they really understand the material.
  \end{dingautolist}


\qboxnps
  {
    \href{http://en.wikipedia.org/wiki/Carl_Friedrich_Gauss}{Karl Friedrich Gauss}
    (1777--1855), German mathematician
    \index{Gauss, Karl Friedrich}
    \index{quotes!Gauss, Karl Friedrich}
    \footnotemark
  }
  %{../common/people/small/gauss1828.jpg}
  {../common/people/gauss.jpg}
  {I mean the word proof not in the sense of lawyers,
    who set two half proofs equal to a whole one,
    but in the sense of the mathematician, where \textonehalf proof = 0
    and it is demanded for proof that every doubt becomes impossible.}
  \citetblt{
    quote: & \citerp{simmons2007}{177} \\
    %image: & \url{http://www-history.mcs.st-andrews.ac.uk/PictDisplay/Gauss.html}
    image: & \url{http://en.wikipedia.org/wiki/Karl_Friedrich_Gauss}
    }

\qboxnpq
  {
    \href{http://en.wikipedia.org/wiki/Carl_Gustav_Jakob_Jacobi}{Carl Gustav Jacob Jacobi}
    (1804--1851), Jewish-German mathematician
    \index{Jacobi, Carl Gustav Jacob}
    \index{quotes!Jacobi, Carl Gustav Jacob}
    \footnotemark
  }
  {../common/people/small/jacobic.jpg}
  {Dirichlet alone, not I, nor Cauchy, nor Gauss knows what a completely rigorous proof is.
   Rather we learn it first from him.
   When Gauss says he has proved something it is clear;
   when Cauchy says it, one can wager as much pro as con;
   when Dirichlet says it, it is certain.}
  %{Dirichlet alone, not I, nor Cauchy, nor Gauss knows what a completely rigorous proof is,
  % and we are learning it from him.
  % When Gauss says he has proved something, it is very probable to me;
  % when Cauchy says it, it is more likely than not,
  % when Dirichlet says it, it is \emph{proved}.}
  \citetblt{
    quote: & \url{http://lagrange.math.trinity.edu/aholder/misc/quotes.shtml} \\
           & \citerp{schubring2005}{558} \\
          %& \citerp{biermann1988}{46} \\
    image: & \url{http://en.wikipedia.org/wiki/Carl_Gustav_Jakob_Jacobi}
    }

\qboxnps
  {
    \href{http://en.wikipedia.org/wiki/Enrico_Fermi}{Enrico Fermi}
    \href{http://www-history.mcs.st-andrews.ac.uk/Timelines/TimelineG.html}{(1777--1855)},
    \href{http://www-history.mcs.st-andrews.ac.uk/BirthplaceMaps/Places/Italy.html}{Italian physicist}
    \index{Fermi, Enrico}
    \index{quotes!Fermi, Enrico}
    \footnotemark
  }
  {../common/people/small/fermi.jpg}
  {If it is true, it can be proved.}
  \citetblt{
    quote: & \citerp{benedetto}{85} \\
    image: & \url{http://www-history.mcs.st-andrews.ac.uk/Mathematicians/Fermi.html}
    }


%Aubrey, John (1626-1697)
%[About Thomas Hobbes:]
%He was 40 years old before he looked on geometry; which happened accidentally. Being in a gentleman's library, Euclid's Elements lay open, and "twas the 47 El. libri I" [Pythagoras' Theorem]. He read the proposition . "By God", sayd he, "this is impossible:" So he reads the demonstration of it, which referred him back to such a proposition; which proposition he read. That referred him back to another, which he also read. Et sic deinceps, that at last he was demonstratively convinced of that trueth. This made him in love with geometry.
%In O. L. Dick (ed.) Brief Lives, Oxford: Oxford University Press, 1960, p. 604.
%http://math.furman.edu/~mwoodard/ascquota.html

  \paragraph{Formal structure.}
  Guidelines in writing the proofs include:
  \begin{dingautolist}{"C0}
    \item Proofs should be very detailed (verbose rather than terse).
          This could tend to obscure the points the proofs are proving
          (``can't see the forest for all of the trees");
          but placing these points in boxes helps remedy this situation
          (see previous discussion).
    \item Every step that relies on another definition or theorem
          should be justified immediately to the right of the step,
          with cross referencing information.
          Many times these references include page numbers for more convenient access.
          If you happen to be viewing the pdf version of this text,
          then you can simply click on a particular reference and your pdf viewer will
          take you immediately to that location.
    \item Proofs, whenever possible, should be {\em direct proofs}
          clearly linking statement to statement with nothing more than
          the equality relation ($=$).
          They should not be cluttered with extensive explanations
          between steps. Let the mathematics speak for itself instead
          of me constantly jumping in waving my hands about in an effort to
          make clear what the mathematical equations make clear
          themselves.
    \item The development of proofs should rely fundamentally on
          propositional logic(logical AND $\land$, logical OR $\lor$, and logical NOT $\lnot$)
          and predicate logic (``for all" universal quantifier $\forall$ and
          ``there exists" existential quantifier $\exists$.)
  \end{dingautolist}

%Although countless practical problems were solved in the eighteenth century
%by the mathematical framework available,
%some began to see that this framework, howbeit sufficiently useful,
%was not sufficiently rigorous to withstand the scrutiny of rigorous logic
%and not sufficiently general to  support the most general functions.
%The nineteenth century was the century of analysis for mathematics.
%Thus with the arrival of the nineteenth century came also the arrival of
%the analytic mathematical samurais such as Cantor, Dirichlet, Lebesgue, and Weirstrass.
%These were men given to detail and who established mathematics on a detailed
%and solid mathematical
%footing through intense and thorough analysis.
%In the end Cantor lost his mind; but the world gained a mindset
%of a solid mathematical
%foundation on which new mathematical structures could be built in the
%twentieth century.



%---------------------------------------
\section*{External reference support}
%---------------------------------------
\qboxnps
  {
    \href{http://en.wikipedia.org/wiki/Niels_Henrik_Abel}{Niels Henrik Abel}
   (1802--1829),
   \href{http://www-history.mcs.st-andrews.ac.uk/BirthplaceMaps/Places/Russia.html}{Nowegian mathematician}
    \index{Abel, Niels Henrik}
    \index{quotes!Abel, Niels Henrik}
    \footnotemark
  }
  {../common/people/small/abel.jpg}
  {It appears to me that if one wants to make progress in mathematics,
    one should study the masters and not the pupils.}
  \citetblt{
    quote: & \citerp{simmons2007}{187} \\
    image: & \url{http://en.wikipedia.org/wiki/Image:Niels_Henrik_Abel.jpg}
    }
%
I have tried to include extensive reference information throughout the text.
These refences appear as brief footnotes at the bottom of the page where
the reference is relevant.
More information about a reference is given in the \hie{bibliography} at the
end of this text.
In most cases, each reference has an associated web link.
That link contains more information about the reference---in some
cases this means a full text download, in other cases a partial viewing,
in some cases the contents of the refence can be searched online,
and in some cases the location where that reference can be found
in libraries scattered throughout the world.
If you happen to be viewing the pdf version of this text,
then you can simply click on a particular footnote number or
footnote reference and your pdf viewer will
immediately take you to that location.
In providing easier access to references,
you the reader can ``study the masters and not the pupils."


%---------------------------------------
\section*{Historical viewpoint}
%---------------------------------------
\qboxnps
  {
    \href{http://en.wikipedia.org/wiki/Yoshida_Kenko}{Yoshida Kenko (Urabe Kaneyoshi)}
    (1283? -- 1350?),
    Japanese author and Buddhist monk
    \index{Kenko, Yoshida}  \index{Kaneyoshi, Urabe}
    \index{quotes!Kenko, Yoshida}  \index{quotes!Kaneyoshi, Urabe}
    \footnotemark
  }
  {../common/people/small/kenko.jpg}
  %{The pleasantest of all diversions is to sit alone under the lamp,
  % a book spread out before you,
  % and to make friends with people of a distant past you have never known. (Keene translation page 12)
  %}
  {To sit alone in the lamplight with a book spread out before you,
   and hold intimate converse with men of unseen generations---
   such is a pleasure beyond compare.}
  \citetblt{
    quote: & \citer{kenko_sansom} \\
    image: & \url{http://en.wikipedia.org/wiki/Yoshida_Kenko}
    }


\qboxnpq
  {
    \href{http://en.wikipedia.org/wiki/Niccol\%C3\%B2_Machiavelli}{Niccol\`o Machiavelli}
    (1469--1527), Italian political philosopher,
    in a 1513 letter to friend Francesco Vettori.
    \index{Machiavelli, Niccol\`o}
    \index{quotes!Machiavelli, Niccol\`o}
    \footnotemark
  }
  {../common/people/small/mach.jpg}
  {When evening comes, I return home and go to my study.
    On the threshold I strip naked, taking off my muddy, sweaty workaday clothes,
    and put on the robes of court and palace,
    and in this graver dress I enter the courts of the ancients and am welcomed by them,
    and there I taste the food that alone is mine, and for which I was born.
    And there I make bold to speak to them and ask the motives of their actions,
    and they, in their humanity reply to me.
    And for the space of four hours I forget the world, remember no vexation,
    fear poverty no more, tremble no more at death;
    I pass indeed into their world.}
  \citetblt{
    quote: & \citerp{machiavelli}{139?} \\
    image: & \url{http://en.wikipedia.org/wiki/Niccol\%C3\%B2_Machiavelli}
    }

\qboxnps
  {
    \href{http://www-history.mcs.st-andrews.ac.uk/Biographies/Poincare.html}
         {Henri Poincar\'e}
    \index{Henri Poincar\'e}
    \index{quotes!Henri Poincar\'e}
    (\href{http://www-history.mcs.st-andrews.ac.uk/Timelines/TimelineF.html}{1854--1912}),
    \href{http://www-history.mcs.st-andrews.ac.uk/BirthplaceMaps/Places/France.html}{French}
    mathematician and physicist,
    in an address to the Fourth International Congress of Mathematicians at Rome, 1908
    \footnotemark
  }
  {../common/people/small/poincare.jpg}
  {The true method of foreseeing the future of mathematics is to study its history
   and its actual state.}
  \citetblt{
    quote: & \citerp{bottazzini}{1} \\
    %quote: & \citer{poincare_sam} ??????  maybe not  \\
    image: & \url{http://www-groups.dcs.st-and.ac.uk/~history/PictDisplay/Poincare.html}
    }

I have tried to include information that gives readers an understanding of where
concepts came from in the history of mathematics.
In some cases, the original source is given as a reference.
In some cases, you can download the source text for free if you have an internet
connection. Web link addresses are provided in the bibliography.
Besides references,
sometimes I have also included quotes from famous mathematicians that
influenced mathematical thinking at the time when a mathematical idea was developing.
These quotes normally appear as
\shadowbox{``shadow boxes"} in the text.

In providing such information from notable mathematicians, you the reader can
``make bold to speak to them and ask the motives of their actions,
        and they, in their humanity reply"---and all this without having to change clothes.

%---------------------------------------
\section*{Hyper-link support}
%---------------------------------------
As already mentioned, most of the references in the bibliography feature web links
for further information and in some cases full text download capability.
Also web links are given for most of the images of
famous mathematicians appearing in this book.
%Lastly, this book itself is available at \\
%  \url{http://banyan.cm.nctu.edu.tw/~dgreenhoe/msd/index.html}\\

The pdf (portable document format) version of this text has
been embedded with an extensive number of hyper-links.
These hyperlinks are highlighted by a yellow box.
For example, in the table of contents,
you can click on a chapter or section title and immediately jump to that chapter or section.
In a proof statement, you can click on a reference to a previous result or definition
and jump to that result.
In a footnote reference, you can click on that reference and immediately jump
to the bibliography for more information about that reference.
In the bibliography, most of the references have links to the world wide web.
If your computer is online and you click on one of those links,
your default browser will display that web page after securing permission from you to do so.




%---------------------------------------
% prepare for main text
%---------------------------------------
\cleartooddpage
\markboth{}{}














%--------------------------------------
% main matter
%--------------------------------------
 \ifDocHasCommonInclude{dsp}
 \ifDocHasCommonInclude{dtft}
%============================================================================
% LaTeX File
% Daniel J. Greenhoe
%============================================================================

%======================================
\chapter{Sample Rate Conversion}
\label{app:src}
%======================================


%=======================================
%\section{Sample rate conversion}
%=======================================
%---------------------------------------
\begin{theorem}[\thmd{upsampling}]
%\footnote{
%  \citerpgc{strang1996}{101}{0961408871}{Second Nobel Identity}
%  }
\label{thm:upsample}
%---------------------------------------
Let $\seqxZ{x_n}$ and $\seqxZ{y_n}$ be \structe{sequences} \xref{def:seq} 
in $\spllF$ \xref{def:spllF} over a \structe{field} $\F$.
\thmbox{
  y_n = 
  \brb{\begin{array}{lM}
    x_{(n/\xN)} & for $n\mod \xN=0$  \\
    0         & otherwise
  \end{array}}
  \qquad\implies\qquad
  {\Zy(z) = \Zx\brp{z^\xN}}
  }
\end{theorem}
\begin{proof}
\begin{align*}
  \Zy(z)
    &\eqd \sum_{n\in\Z} y_n z^{-n}
    &&    \text{by definition of \fncte{z-transform}}
    &&    \text{\xref{def:opZ}}
  \\&=    \sum_{n\mod\xN=0}   y_n z^{-n} +
          \sum_{n\mod\xN\ne0} y_n z^{-n}
  \\&=    \sum_{n\mod\xN=0} x_{n/\xN} z^{-n} +
          \cancelto{0}{\sum_{n\mod \xN\ne 0} 0 z^{-n}}
    &&    \text{by definition of $\seqn{y_n}$}
  \\&=    \sum_{m\in\Z} x_m z^{-m\xN}
    &&    \text{where $m\eqd n/\xN\,\implies\,n=m\xN$}
  \\&=    \sum_{m\in\Z} x_m \brp{z^\xN}^{-m}
  \\&\eqd \Zx\brp{z^\xN}
    &&    \text{by definition of \fncte{z-transform}}
    &&    \text{\xref{def:opZ}}
\end{align*}
\end{proof}



%---------------------------------------
\begin{theorem}[\thmd{downsampling}]
\index{decimation}
\label{thm:downsample}
%---------------------------------------
Let $\seqxZ{x_n}$ and $\seqxZ{y_n}$ be \structe{sequences} \xref{def:seq} 
in $\spllF$ \xref{def:spllF} over a \structe{field} $\F$.
\thmbox{
  \brb{y_n = x_{(\xN n)}}
  \qquad\implies\qquad
  \brb{\Zy(z)=\frac{1}{\xN}\sum_{m=0}^{\xN-1}\Zx\left( e^{i\frac{2\pi m}{\xN}} z^{\frac{1}{\xN}} \right)}
  }
\end{theorem}
\begin{proof}
\begin{align*}
  \Zy(z)
    &\eqd \sum_{n\in\Z} y_n z^{-n}
    &&    \text{by definition of \fncte{z-transform}}
    &&    \text{\xref{def:opZ}}
  \\&=    \sum_{n\in\Z} x_{(n\xN)} z^{-n}
    &&    \text{by definition of $\seqn{y_n}$}
  \\&=    \sum_{n\in\Z} x_n \Big[ \kdelta_{(n\mod \xN)} \Big] z^{-\frac{n}{\xN}}
  \\&=    \sum_{n\in\Z} x_n \brs{\frac{1}{\xN}\sum_{m=0}^{\xN-1} e^{-i\frac{2\pi nm}{\xN}}} z^{-\frac{n}{\xN}}
    &&    \text{by \thme{Summation around unit circle}}
    &&    \text{\xref{cor:trig_circle}}
  \\&=    \frac{1}{\xN}\sum_{m=0}^{\xN-1} \sum_{n\in\Z} x_n 
                 \brp{e^{i\frac{2\pi m}{\xN}}}^{-n} 
                 \brp{ z^\frac{1}{\xN}}^{-n}
  \\&=    \frac{1}{\xN}\sum_{m=0}^{\xN-1} \sum_{n\in\Z} x_n 
                 \brp{e^{i\frac{2\pi m}{\xN}} z^\frac{1}{\xN}}^{-n} 
  \\&\eqd \frac{1}{\xN}\sum_{m=0}^{\xN-1}\Zx\brp{e^{i\frac{2\pi m}{\xN}} z^\frac{1}{\xN}}
    &&    \text{by definition of \fncte{z-transform}}
    &&    \text{\xref{def:opZ}}
\end{align*}
\end{proof}






%============================================================================
% XeLaTeX File
% Daniel J. Greenhoe
%============================================================================

%=======================================
\chapter{Magic Tricks in the World of z}
%=======================================

%---------------------------------------
\section*{The Magic Wand: The z-domain IIR filter}
%---------------------------------------

What we want:

\indentx\begin{tabular}{|cll|}
    \hline
    \imark & linear         & (double the input means double the output)
  \\\imark & time-invariant & (doesn't change with time)
  \\\imark & tractable      & (we can build it)
  \\\hline
\end{tabular}

%\includegraphics{../common/graphics/misc/wizard_dragon_4953570.jpg}



\includegraphics[width=\tw-50mm]{graphics/linear.pdf}

Which one is \prope{linear}?



\includegraphics[width=\tw-50mm]{graphics/nonti.pdf}

Is the filter \prope{time-invariant}?



\includegraphics[width=\tw-50mm]{graphics/cosLcos.pdf}

\prope{linear} + \prope{time-invariant} = \prope{LTI}


%---------------------------------------
\section*{Linear operators}
%---------------------------------------
Note:

\fbox{\begin{tabular}{llll}
    A  & \opb{filter}    & in   \hie{engineering} & is an
  \\   & \opb{operator}  & in   \hie{mathematics} & is a
  \\   & \opb{black box} & in a \hie{system} diagram.
\end{tabular}}



\qboxnqt
  {
    Jules Henri Poincar\'e (1854-1912), physicist and mathematician
    \index{Poincar\'e, Jules Henri}
    \index{quotes!Poincar\'e, Jules Henri}
    \footnotemark
  }
  %{../common/people/small/poincare.jpg}
  {Je ne sais si je n'ai d\'ej\`a` dit quelque part que la Math\'ematique est 
  l'art de donner le m\^eme nom \`a des choses diff\'erentes. 
  Il convient que ces choses, diff\'erentes par la mati\`ere, 
  soient semblables par la forme, 
  \ldots
  %qu'elles puissent, 
  %pour ainsi dire, se couler dans le m\^eme moule. 
  %Quand le langage a \'et\'e bien choisi, 
  on est tout \'etonn\'e 
  de voir que toutes les d\'emonstrations, faites pour un objet connu, 
  s'appliquent imm\'ediatement \`a beaucoup d'objets nouveaux ; 
  \ldots
  %on n'a rien \`a y changer, pas m\^eme les mots, puisque les noms sont devenus les m\^emes.
  }
  %
  {I think I have already said somewhere that mathematics is the art
   of giving the same name to different things. 
   It is enough that these things, though differing in matter, 
   should be similar in form, 
   \ldots
   %to permit of their being, so to speak,
   %run in the same mould.
   %When language has been well chosen, 
   one is astonished to find that all
   demonstrations made for a known object apply immediately to many new objects:
   \ldots
   %nothing requires to be changed, 
   %not even the terms,
   %since the names have become the same.
   }
  \citetblt{
    quote:   & \citerc{poincare_sam}{book 1, chapter 2, paragraph 20} \\
             & \url{http://fr.wikisource.org/wiki/Science_et_m\%C3\%A9thode_-_Livre_premier\%2C_\%C2\%A7_II} \\
    trans.:  & \citerp{poincare_sam_eng}{34} \\
   %image:        & \url{http://en.wikipedia.org/wiki/Image:Poincare_jh.jpg}
    }



\fbox{\begin{tabular}{llll}
    A  & \opb{relation} & is a \ope{map}      & from one set to another.
  \\A  & \opb{function} & is a \ope{relation} & where 1 point cannot map to 2 points.
  \\An & \opb{operator} & is a \ope{function} & that maps vectors to vectors.
\end{tabular}}

$\begin{array}{|c|c|}
   \hline
   \includegraphics{graphics/123le12.pdf}
  &\includegraphics{graphics/fnctsq.pdf}
  \\\ope{relation}&\ope{function}
  \\\hline
\end{array}$




%---------------------------------------
%\begin{definition}
\label{def:linop}
\label{def:clL}
%---------------------------------------
%Let $\spX\eqd\linearspaceX$ and $\spY\eqd\linearspaceY$ be linear spaces.\\
  \defbox{%
    \index{operator!linear}
    \begin{array}{>{\scy\qquad}r rcl @{\qquad}C @{\qquad}D@{\qquad}D}
      \mc{7}{M}{An operator $\opL$ is \hid{linear} if \footnotemark}
        \\1. & \opL(\vx + \vy)  &=&  \opL \vx + \opL \vy & %\forall \vx,\vy\in\spX                      
                                                         & (\prope{additive}) & and
        \\2. & \opL(\alpha \vx) &=&  \alpha\opL \vx      & %\forall \vx\in\spX,\quad \forall\alpha\in\F 
                                                         & (\prope{homogeneous}).
    %\\\mc{7}{M}{The set of all linear operators from $\spX$ to $\spY$ is denoted $\clLxy$ such that}
    %\\\mc{7}{l}{\qquad\clLxy \eqd \set{\opL\in\clOxy}{\text{$\opL$ is linear}}}.
    \end{array}
  }
\footnotetext{
  \citerpg{kubrusly2001}{55}{0817641742},
  \citerpg{ab}{224}{0120502577},
  %\citerpg{michel1993}{95,407}{048667598X} \\
  \citorp{hilbert1927}{6},
  \citorp{stone1932}{33}
  }
%\end{definition}


%---------------------------------------
%\begin{theorem}
\label{thm:L_prop}
%---------------------------------------
%Let $\opL$ be an operator from a linear space $\spX$ to a linear space $\spY$, both over a field $\F$.
\thmbox{
  \text{$\opL$ is \prope{linear}}
  \qquad\implies\qquad
  \brbl{\begin{array}{>{\scy}r>{\ds}lc>{\ds}lC}
      1. & \opL\vzero                            &=& \vzero                                & %
    \\2. & \opL(-\vx)                            &=& -(\opL\vx)                            & %\forall \vx\in\spX
    \\3. & \opL(\vx-\vy)                         &=& \opL\vx - \opL\vy                     & %\forall \vx,\vy\in\spX
    \\4. & \opL\brp{\sum_{n=1}^N \alpha_n\vx_n}  &=& \sum_{n=1}^N \alpha_n\brp{\opL\vx_n}  & %\vx_n\in\spX,\,\alpha_n\in\F
  \end{array}}\footnotemark
}
\footnotetext{
  \citerpgc{berberian1961}{79}{0821819127}{Theorem~IV.1.1}
  }


%---------------------------------------
%\begin{theorem}
%---------------------------------------
Let $\spW$, $\spX$, $\spY$, and $\spZ$ be linear spaces over a field $\F$.
\thmbox{\begin{array}{>{\scy}rlcl@{\qquad}D}
    1. & \opL\brp{\opM\opN}      &=& \brp{\opL\opM}\opN                          & (\prope{associative})
  \\2. & \opL\brp{\opM\addo\opN} &=& \brp{\opL\opM}\addo\brp{\opL\opN}           & (\prope{left distributive})
  \\3. & \brp{\opL\addo\opM}\opN &=& \brp{\opL\opN}\addo\brp{\opM\opN}           & (\prope{right distributive})
  \\4. & \alpha\brp{\opL\opM}    &=& \brp{\alpha\opL}\opM = \opL\brp{\alpha\opM} & (\prope{homogeneous})
\end{array}\footnotemark}
\footnotetext{
  \citerpgc{berberian1961}{88}{0821819127}{Theorem~IV.5.1}
  }
%\end{theorem}





\tbox{$\ds X(z)\eqd\sum_{n\in\Z} \fx[n] z^{-n}$}
\tbox{\includegraphics[width=100mm]{graphics/ziir2.pdf}}
\tbox{$\ds Y(z)\eqd\sum_{n\in\Z} \fy[n] z^{-n}$}

\indentx\begin{tabular}{cl}
    \imark&Is it \prope{linear}?
  \\\imark&Is it \prope{time-invariant}?
  \\\imark&Is it \prope{tractable}?
\end{tabular}



\tbox{$\ds X(z)$}
\tbox{\includegraphics[width=100mm]{graphics/ziir2.pdf}}
\tbox{$\ds Y(z)$}
\begin{align*}
  H(z)\eqd \frac{Y(z)}{X(z)}
    &=\frac{b_0 z^2 + b_1 z + b_2}{a_0 z^2 + a_1 z + a_2}
    \qquad=\frac{z^{-2}}{z^{-2}}\times\frac{b_0 z^2 + b_1 z + b_2}{a_0 z^2 + a_1 z + a_2}
  \\&=\frac{b_0 + b_1 z^{-1} + b_2z^{-2}}{a_0 + a_1 z^{-1} + a_2z^{-2}}
  \\
  \\Y(z)\brs{a_0 + a_1 z^{-1} + a_2z^{-2}}
    &= X(z) \brs{b_0 + b_1 z^{-1} + b_2z^{-2}}
\end{align*}



If $a_0=1$ then

\begin{align*}
  Y(z) &=  b_0X(z) + b_1z^{-1}X(z)  + b_2z^{-2}X(z) - a_1 z^{-1}Y(z) + a_2z^{-2}Y(z)
\end{align*}

\begin{align*}
  a X(z)
    &\eqd a \opZ \seqn{\fx[n]}                && \text{by definition of $X(z)$}
  \\&\eqd a \sum_{n\in\Z} \fx[n] z^{-n}       && \text{by definition of $\opZ$ operator}
  \\&\eqd \sum_{n\in\Z} \brp{a \fx[n]} z^{-n} && \text{by \prope{distributive} property}
  \\&\eqd \opZ\seqn{a\fx[n]}                  && \text{by definition of $\opZ$ operator}
\end{align*}




\begin{align*}
  z^{-2}X(z) 
    &= z^{-2} \opZ\seqn{\fx[n]}
   &&\eqd z^{-2}\sum_{n=-\infty}^{n=+\infty} \fx[n] z^{-n}
  \\&=          \sum_{n=-\infty}^{n=+\infty} \fx[n] z^{-n-2}
  \\&=          \sum_{m-2=-\infty}^{m-2=+\infty} \fx[m-2] z^{-m}
    && \text{where $m\eqd n+2$ $\implies$ $n=m-2$}
  \\&=          \sum_{m=-\infty}^{m=+\infty} \fx[m-2] z^{-m}
  \\&=          \sum_{n=-\infty}^{n=+\infty} \fx[n-2] z^{-n}
    && \text{by change of free variable $m\rightarrow n$}
  \\&\eqd \opZ\seqn{\fx[n-2]}
    && \text{by definition of $\opZ$ operator}
\end{align*}
\\



%---------------------------------------
\section*{Trick \# 1: From the Domain of z Back to the real world}
%---------------------------------------

\begin{align*}
  Y(z) &=  b_0X(z) + b_1z^{-1}X(z)  + b_2z^{-2}X(z) - a_1 z^{-1}Y(z) + a_2z^{-2}Y(z)
  \\\\
  \fy[n] &= b_0\fx[n] + b_1\fx[n-1] + b_2\fx[n-2] - a_1\fy[n-1] - a_2\fy[n-2]
\end{align*}

\includegraphics{graphics/iir2n.pdf}



Example

$\ds{\frac{3z^2 + 5z + 7}{2z^2 + 10z + 12}}$
=
$\ds{\frac{3z^2 + 5z + 7}{2\brp{z^2 + 5z + 6}}}$
=
$\ds{\frac{\brp{\sfrac{3}{2}z^2 + \sfrac{5}{2}z + \sfrac{7}{2}}}
               {z^2 + 5z + 6}}$
=
$\ds{\frac{\brp{\sfrac{3}{2} + \sfrac{5}{2}z^{-1} + \sfrac{7}{2}z^{-2}}}
               {1 + 5z^{-1} + 6z^{-2}}}$

\includegraphics{graphics/dfI_order2_156.pdf}

%---------------------------------------
\section*{Trick \# 2: The mirror trick}
%---------------------------------------
  \includegraphics{graphics/pz_minphase.pdf}%

If you want real coefficients, choose poles and zeros in conjugate pairs.

%\begin{align*}
%  \brp{z-p_1}\brp{z-p_1^*}
%    &= \brp{z-re^{i\theta}}\brp{z-re^{-i\theta}}
%  \\&= z^2 -r\brs{e^{-i\theta}+e^{i\theta}}z + 1
%  \\&= z^2 -r\brp{\brs{\cos\theta - i\sin\theta}+\brs{\cos\theta + i\sin\theta}}z + 1
%  \\&= z^2 -2r\cos(\theta) z + 1
%\end{align*}
\begin{align*}
  \brp{z-p_1}\brp{z-p_1^*}
    &= \brs{z-\brp{a+ib}} \brs{z-\brp{a-ib}}
  \\&= z^2 +\brs{-a+ib-ib-a}z - \brs{ib}^2
  \\&= z^2 -2a z + b^2
\end{align*}

Example: 

\includegraphics{graphics/pz_realcoefs_11.pdf}

\begin{align*}
  H(z)   &= G\frac{\brs{z-z_1}\brs{z-z_2}}
                  {\brs{z-p_1}\brs{z-p_2}}
          = G\frac{\brs{z-\brp{1+i}}\brs{z-\brp{1-i}}}
                  {\brs{z-\brp{-\sfrac{2}{3}+i\sfrac{1}{2}}}\brs{z-\brp{-\sfrac{2}{3}-i\sfrac{1}{2}}}}
       \\&= G\frac{z^2 - z\brs{\brp{1-i}+\brp{1+i}} + \brp{1-i}\brp{1+i}}
                  {z^2 - z\brs{\brp{-\sfrac{2}{3}+i\sfrac{1}{2}}+\brp{-\sfrac{2}{3}+i\sfrac{1}{2}}} + \brp{-\sfrac{2}{3}+i\sfrac{1}{2}}\brp{-\sfrac{2}{3}+i\sfrac{1}{2}}}
       \\&= G\frac{z^2 - 2z + 2}
                  {z^2 - \sfrac{4}{3}z + \brp{\sfrac{4}{3}+\sfrac{1}{4}}}
          = G\frac{z^2 - 2z + 2}
                  {z^2 - \sfrac{4}{3}z + \sfrac{19}{12}}
\end{align*}

%---------------------------------------
\section*{Trick \# 3: The 0Hz Gain}
%---------------------------------------
\tbox{$\ds X(z)$}
\tbox{\includegraphics[width=100mm]{graphics/ziir2.pdf}}
\tbox{$\ds Y(z)$}

$\ds\begin{array}{rclM}
  \ds\brb{\opDTFT \fx[n]}(\omega) &\ds\eqd& \ds\sum_{n\in\Z} \fx[n] e^{-i\omega n}  & DTFT\\
  \ds\brb{\opZ    \fx[n]}(z)      &\ds\eqd& \ds\sum_{n\in\Z} \fx[n] z^{-n}          & z-Transform
\end{array}$

\[\implies\qquad z=e^{i\omega} \qquad z=1 \text{ at $\omega=0$}\]

\begin{align*}
  \brlr{H(z)}_{z=e^{i0}=1}
    &= \brlr{\frac{b_0 z^2 + b_1 z + b_2}{a_0 z^2 + a_1 z + a_2}}_{z=e^{i0}=1}%
     = \frac{b_0     + b_1   + b_2}{a_0     + a_1   + a_2}
\end{align*}


%---------------------------------------
\section*{Trick \# 4: The Fs/2 Gain}
%---------------------------------------
\tbox{$\ds X(z)$}
\tbox{\includegraphics[width=100mm]{graphics/ziir2.pdf}}
\tbox{$\ds Y(z)$}


\[z=e^{i\omega} \qquad z=-1 \text{ at $\omega=\pi$}\]

\begin{align*}
  \brlr{H(z)}_{z=e^{i\pi}}
    &= \brlr{\frac{b_0 z^2 + b_1 z + b_2}{a_0 z^2 + a_1 z + a_2}}_{z=e^{i\pi}=-1}%
    &= \brlr{\frac{b_0 (-1)^2 + b_1 (-1) + b_2}{a_0 (-1)^2 + a_1 (-1) + a_2}}_{z=e^{i\pi}=-1}%
  \\&= \frac{b_0     - b_1   + b_2}{a_0     - a_1   + a_2}
\end{align*}

%---------------------------------------
\section*{Trick \# 5: For a stable filter, put all the poles inside the unit circle}
%---------------------------------------

\begin{tabular}{cc}
    \includegraphics{graphics/pz_unstable.pdf}%
   &\includegraphics{graphics/pz_stable.pdf}%
  \\\prope{unstable}&\prope{stable}
  \\\includegraphics{../common/graphics/emotions/bear_surprisedL_4955791.jpg}
   &\includegraphics{../common/graphics/emotions/bear_happy_4954937.jpg}
\end{tabular}


%---------------------------------------
\section*{Trick \# 6: For minimum phase, put all the zeros inside the unit circle}
%---------------------------------------
\begin{tabular}{cc}
  \includegraphics{graphics/pz_realcoefs.pdf}%
  &\includegraphics{graphics/pz_minphase.pdf}%
  \\\emph{not} minimum phase & \prope{minimum phase}
\end{tabular}


The impulse response of a minimum phase filter has most of its energy concentrated
near the beginning of its support

%--------------------------------------
\thmd{Robinson's Energy Delay Theorem}
\footnote{
  \citerpg{dumitrescu2007}{36}{1402051247},
  \citor{robinson1962},  % referenced by claerbout1976
  \citorc{robinson1966}{???},  % referenced by online thesis
  \citerpp{claerbout1976}{52}{53}
  %\citerp{os}{291}\\
  %\citerp{mallat}{253}
  }
\label{thm:ztr_redp}
\index{minimum phase!energy}
\index{energy}
%--------------------------------------
\\
%Let $\fp(z)\eqd\sum_{n=0}^\xN a_n z^{-n}$ 
%and $\fq(z)\eqd\sum_{n=0}^\xN b_n z^{-n}$ 
%be polynomials.
\thmbox{
  \brb{\begin{array}{lMD}
    \fp & is \prope{minimum phase} & and\\
    \fq & is \emph{not} minimum phase & 
  \end{array}}
  \implies
  \mcom{\sum_{n=0}^{m-1} \abs{a_n}^2}{\parbox{20mm}{``energy" of the first $m$ coefficients of $\fp(z)$}} \ge 
  \mcom{\sum_{n=0}^{m-1} \abs{b_n}^2}{\parbox{20mm}{``energy" of the first $m$ coefficients of $\fq(z)$}} 
  %\qquad \forall 0\le m\le\xN
  }




Minimum phase filters are easy to invert: each zero becomes a pole and each pole becomes a zero

$\begin{array}{ccccc}
     \tbox{\includegraphics{graphics/pz_minphase.pdf}}
    &\tbox{$\times$}&
     \tbox{\includegraphics{graphics/pz_minphase_inv.pdf}}
    &\tbox{$=$}&
     \tbox{$1$}
  \\
  \ds\frac{\brp{z-z_1}\brp{z-z_2}\brp{z-z_3}\brp{z-z_4}}
          {\brp{z-p_1}\brp{z-p_2}\brp{z-p_3}\brp{z-p_4}}
  &\times&
  \ds\frac{\brp{z-p_1}\brp{z-p_2}\brp{z-p_3}\brp{z-p_4}}
          {\brp{z-z_1}\brp{z-z_2}\brp{z-z_3}\brp{z-z_4}}
    &=& 1
\end{array}$



%---------------------------------------
\section*{Trick \# 7: But for more symmetry, put some inside and some outside}
%---------------------------------------

  \exboxt{\begin{tabular}{c|c}
    Daubechies-8 & Symlet-8
    \\\hline
    \includegraphics{graphics/D8_pz.pdf}&\includegraphics{graphics/S8_pz.pdf}\\
    \includegraphics{graphics/d8_phi_h.pdf}&\includegraphics{graphics/s8_phi_h.pdf}
    %\includegraphics{graphics/d8_psi_g.pdf}&\includegraphics{graphics/s8_psi_g.pdf}
  \end{tabular}}





%---------------------------------------
\section*{Trick \# 8: Really must invert a non-minimum phase filter? All-pass filter to save the day!}
%---------------------------------------

\begin{tabular}{ccccc}
     \tbox{\includegraphics{graphics/pz_unstable2.pdf}}
    &\tbox{$\times$}&
     \tbox{\includegraphics{graphics/pz_allpass.pdf}}
    &\tbox{$=$}&
     \tbox{\includegraphics{graphics/pz_unall.pdf}}
  \\unstable&&all-pass&&stable!
  \\\includegraphics{../common/graphics/emotions/bear_surprisedL_4955791.jpg}
    &&
    \includegraphics{../common/graphics/superhero/rush_superhero_4954908.jpg}
    &&
    \includegraphics{../common/graphics/emotions/bear_happy_4954937.jpg}
\end{tabular}



\begin{align*}
  \abs{A\brp{z}}_{z=e^{i\omega}}
    &= \frac{1}{r}\abs{\frac{z-r          e^{i\phi}}
                            {z-\frac{1}{r}e^{i\phi}}}_{z=e^{i\omega}}
   &&= \abs{\frac{ z- re^{i\phi}}
                 {rz-  e^{i\phi}}}_{z=e^{i\omega}}
  \\&= \abs{e^{i\phi}\brp{
            \frac{e^{-i\phi}z-r}
                 {rz- e^{i\phi}}}}_{z=e^{i\omega}}
   &&= \abs{z\brp{
            \frac{e^{-i\phi}-rz^{-1}}
                 {rz- e^{i\phi}}}}_{z=e^{i\omega}}
  \\&= \abs{-z\brp{
            \frac{rz^{-1}- e^{-i\phi}}
                 {rz     - e^{ i\phi}}}}_{z=e^{i\omega}}
   &&= \abs{\mcom{e^{i\pi}}{$-1$}e^{i\omega}\brp{
            \frac{re^{-i\omega} - e^{-i\phi}}
                 {re^{ i\omega} - e^{ i\phi}}}}
  \\&= \abs{\frac{1}{e^{-iv}}\brp{
            \frac{re^{-i\omega} - e^{-i\phi}}
                 {\brp{re^{ i\omega} - e^{ i\phi}}^\ast}}}
   &&= \abs{\frac{re^{-i\omega} - e^{-i\phi}}
                 {re^{-i\omega} - e^{-i\phi}}}
  \\&= 1
\end{align*}





%---------------------------------------
\section*{Trick \# 9: The Reappearing Roots: Now you don't see them, now you do.}
%---------------------------------------

True or False? This filter has no poles:

  $\ds H(z)= b_0 + b_1 z^{-1} + b_2 z^{-2}$

\includegraphics{graphics/dfI_order2_fir.pdf}


\begin{align*}
  H(z)
    &= b_0 + b_1 z^{-1} + b_2 z^{-2}
     = \frac{z^2}{z^2} \times \frac{b_0 + b_1 z^{-1} + b_2 z^{-2}}{1}
     = \frac{b_0 z^2 + b_1 z^{1} + b_2 }{z^2}
\end{align*}

\includegraphics{graphics/pz_pole00.pdf}




%---------------------------------------
\section*{Trick \# 10: Have a low-pass but a high-pass need? Just put a minus in front of z!}
%---------------------------------------

Suppose $\Zh(z)$ is a \propb{low-pass} filter. \qquad Then what is  $\Zh(-z)$?

{\begin{align*}
  \abs{\Fg(\omega)}^2
    &\eqd \abs{\Zh(-z)}_{z=e^{i\omega}}
    && \text{by definition of $\Fg(\omega)$}
  \\&= \abs{\Zh(e^{-i\pi}z)}_{z=e^{i\omega}}
  \\&= \abs{\Zh(z)}_{z=e^{i\omega}e^{-i\pi}}
  \\&= \abs{\Zh(z)}_{z=e^{i(\omega-\pi)}}
  \\&\eqd \abs{\Fh(\omega-\pi)}^2
    && \text{by definition of $\Fh(\omega)$}
\end{align*}}



%============================================================================
% XeLaTeX File
% Daniel J. Greenhoe
%============================================================================

%=======================================
\chapter{Magnitude characteristics of z-filters}
%=======================================
%---------------------------------------
\section{The 0Hz and $F_s/2$ Gain}
%---------------------------------------
%---------------------------------------
\begin{proposition}
%---------------------------------------
\propbox{
  \brp{\Zh(z) = \frac{\ds\sum_{n=0}^{\xN}b_n z^{-n}}
                     {\ds\sum_{n=0}^{\xN}a_n z^{-n}}}
  \quad\implies\quad
  \brp{\Fh(0) = \frac{\ds\sum_{n=0}^{\xN}b_n }
                          {\ds\sum_{n=0}^{\xN}a_n}}
  }
\tbox{$\ds\Zx(z)$}
\tbox{\includegraphics{graphics/ziir2.pdf}}
\tbox{$\ds\Zy(z)$}
\end{proposition}
\begin{proof}
%$\ds\begin{array}{rclM}
%  \ds\brb{\opDTFT \fx[n]}(\omega) &\ds\eqd& \ds\sum_{n\in\Z} \fx[n] e^{-i\omega n}  & DTFT\\
%  \ds\brb{\opZ    \fx[n]}(z)      &\ds\eqd& \ds\sum_{n\in\Z} \fx[n] z^{-n}          & z-Transform
%\end{array}$
%
\begin{align*}\
  \Fh(0) 
     = \brlr{\Fh(\omega)}_{\omega=0}
     = \brlr{\ZH\brp{e^{i\omega}}}_{\omega=0}
     = \brlr{\Zh(z)}_{z=1}
     = \brlr{\frac{\ds\sum_{n=0}^{\xN}b_n z^{-n}}
            {\ds\sum_{n=0}^{\xN}a_n z^{-n}}}_{z=1}
     = \frac{\ds\sum_{n=0}^{\xN}b_n }
            {\ds\sum_{n=0}^{\xN}a_n }
\end{align*}
\end{proof}

%---------------------------------------
\begin{proposition}
%---------------------------------------
\propbox{
  \brp{\Zh(z) = \frac{\ds\sum_{n=0}^{\xN}b_n z^{-n}}
                     {\ds\sum_{n=0}^{\xN}a_n z^{-n}}}
  \quad\implies\quad
  \brp{\brlr{\Fh(\omega)}_{\omega=\frac{F_s}{2}} = 
         \frac{\ds\sum_{n=0}^{\xN}(-1)^n b_n }
              {\ds\sum_{n=0}^{\xN}(-1)^n a_n}}
      }
\end{proposition}
\begin{proof}
\begin{align*}
  \brlr{\Fh(\omega)}_{\omega=\frac{F_s}{2}} 
    &= \brlr{\Zh(z)}_{z=e^{i\pi}}
     = \brlr{\Zh(z)}_{z=-1}
     = \brlr{\frac{\ds\sum_{n=0}^{\xN}(-1)^n b_n}
                  {\ds\sum_{n=0}^{\xN}(-1)^n a_n}}_{z=-1}
     =       \frac{\ds\sum_{n=0}^{\xN}(-1)^n b_n }
                  {\ds\sum_{n=0}^{\xN}(-1)^n a_n}
\end{align*}
\end{proof}

%=======================================
\section{Pole and zero location analysis}
%=======================================
Note the following:
\\\indentx\begin{tabular}{cll}
    \imark & The frequency response of $\Zh(z)$ \propb{repeats} every $2\pi$.
           & \prefp{prop:dtft_periodic}
  \\\imark & If the coefficients are \propb{real}, &
         \\& then the magnitude response is \propb{symmetric}
           & \prefp{prop:dtft_real}
  \\\imark & Moments and derivatives are related:
           & \prefp{thm:dtft_ddw}
\end{tabular}

The pole zero locations of a digital filter determine the magnitude and 
phase frequency response of the digital filter.\footnote{%
  \citerpp{cadzow}{90}{91},
  \citerppgc{ifeachor1993}{134}{136}{020154413X}{\textsection ``1.5.3 Geometric evaluation of frequency response"}, 
  \citerppgc{ifeachor2002}{201}{203}{0201596199}{\textsection ``4.5.3 Geometric evaluation of frequency response"}
  }
This can be seen by representing the pole and zero vectors in the complex z-plane.
Each of these vectors has a magnitude $M$ and a direction $\theta$.
Also, each factor $(z-z_i)$ and $(z-p_i)$ can be represented as vectors as well
(the difference of two vectors).
Each of these factors can be represented by a magnitude/phase factor
$M_ie^{i\theta_i}$.  The overall magnitude and phase of $H(z)$ can then 
be analyzed.

\begin{figure}[ht]
  \centering%
  \includegraphics{graphics/vecres.pdf}
  \caption{
     Vector response of digital filter \xref{ex:zchar_vector}
     \label{fig:zchar_vector}
     }
\end{figure}
%---------------------------------------
\begin{example}
\label{ex:zchar_vector}
%---------------------------------------
Take the following filter for example.
\begin{align*}
   H(z) &= \frac{b_0 + b_1z^{-1} + b_2z^{-2} }
                {1   + a_1z^{-1} + a_2z^{-2} }
   \\   &= \frac{(z-z_1)(z-z_2)}
                {(z-p_1)(z-p_2)}
   \\   &= \frac{M_1e^{i\theta_1} \; M_2e^{i\theta_2} \; }
                {M_3e^{i\theta_3} \; M_4e^{i\theta_4} \; }
   \\   &= \left(\frac{M_1M_2}{M_3M_4} \right)
           \left(\frac{e^{i\theta_1} e^{i\theta_2} }
                      {e^{i\theta_3} e^{i\theta_4} }\right)
\end{align*}

This is illustrated in \prefpp{fig:zchar_vector}.
The unit circle represents frequency in the Fourier domain.
The frequency response of a filter is just a rotating vector on this circle.
The magnitude response of the filter is just then a {\em vector sum}.
For example, the magnitude of any $H(z)$ is as follows:

\begin{align*}
   \abs{H(z)} &=& \frac{|(z-z_1)|\;|(z-z_2)|}
                 {|(z-p_1)|\;|(z-p_2)|}
\end{align*}
\end{example}

%=======================================
\section{Coefficient analysis}
%=======================================
%---------------------------------------
\begin{lemma}
\label{lem:aacos}
%---------------------------------------
\lembox{
  \sum_{n=0}^{\xN}\ds\sum_{m=0}^{\xN} a_n a_m e^{-i\omega(n-m)}
    =\sum_{n=0}^{\xN}\abs{a_n}^2 + 2\sum_{n=0}^{\xN}\ds\sum_{m=n+1}^{\xN} \Re\brs{a_n a_m^\ast} \cos\brs{\omega(n-m)}
  }
\end{lemma}

%---------------------------------------
\begin{example}
This example graphically illustrates \prefpp{lem:aacos} for the case $\xN=4$.
%---------------------------------------
\begin{align*}
  \sum_{n=0}^{4}\ds\sum_{m=0}^{4} a_n a_m e^{-i\omega(n-m)}
    &= \brp{\begin{array}{|c||*{5}{>{\ds}c|}}
           \hline          &{\ccg}$m=0$                 &{\ccg}$m=1$                 &{\ccg}$m=2$                 &{\ccg}$m=3$                 &{\ccg}$m=4$
         \\\hline
           \hline\ccg$n=0$ &{\ccc}a_0 a_0^\ast               &      a_0 a_1^\ast e^{+i \omega} &      a_0 a_2^\ast e^{+i2\omega} &      a_0 a_3^\ast e^{+i3\omega} &      a_0 a_4^\ast e^{+i4\omega} 
         \\\hline\ccg$n=1$ &      a_1 a_0^\ast e^{-i \omega} &{\ccc}a_1 a_1^\ast               &      a_1 a_2^\ast e^{+i \omega} &      a_1 a_3^\ast e^{+i2\omega} &      a_1 a_4^\ast e^{+i3\omega} 
         \\\hline\ccg$n=2$ &      a_2 a_0^\ast e^{-i2\omega} &      a_2 a_1^\ast e^{-i \omega} &{\ccc}a_2 a_2^\ast               &      a_2 a_3^\ast e^{+i \omega} &      a_2 a_4^\ast e^{+i2\omega} 
         \\\hline\ccg$n=3$ &      a_3 a_0^\ast e^{-i3\omega} &      a_3 a_1^\ast e^{-i2\omega} &      a_3 a_2^\ast e^{-i \omega} &{\ccc}a_3 a_3^\ast               &      a_3 a_4^\ast e^{+i \omega} 
         \\\hline\ccg$n=4$ &      a_4 a_0^\ast e^{-i4\omega} &      a_4 a_1^\ast e^{-i3\omega} &      a_4 a_2^\ast e^{-i2\omega} &      a_4 a_3^\ast e^{-i \omega} &{\ccc}a_4 a_4^\ast                  
         \\\hline
       \end{array}}
  \\&=  \sum_{n=0}^{4}a_n a_n^\ast
     + 2\sum_{n=0}^{4}\ds\sum_{m=n+1}^{4} \brs{\brp{a_n a_m^\ast e^{i\omega}} + \brp{a_n^\ast a_m e^{-i\omega}}}
  \\&=  \sum_{n=0}^{4}a_n a_n^\ast
     + 2\sum_{n=0}^{4}\ds\sum_{m=n+1}^{4} \brs{\brp{a_n a_m^\ast e^{i\omega}} + \brp{a_n a_m^\ast e^{i\omega}}^\ast}
  \\&=  \sum_{n=0}^{4}a_n a_n^\ast
     +  \sum_{n=0}^{4}\ds\sum_{m=n+1}^{4} 2\Re\brs{\brp{a_n a_m^\ast e^{i\omega}}}
  \\&=  \sum_{n=0}^{4}\abs{a_n}^2 
     + 2\sum_{n=0}^{4}\ds\sum_{m=n+1}^{4}  \Re\brs{a_n a_m^\ast} \Re\brs{e^{i\omega(n-m)}}
  \\&=  \sum_{n=0}^{4}\abs{a_n}^2 
     + 2\sum_{n=0}^{4}\ds\sum_{m=n+1}^{4}  \Re\brs{a_n a_m^\ast} \cos\brs{\omega(n-m)}
\end{align*}
\end{example}

%---------------------------------------
\begin{lemma}
\label{lem:qzcos}
%---------------------------------------
\lembox{
  \brb{\Zq(z) \eqd \sum_{n=0}^{\xN} a_n z^{-n}}
  \quad\implies\quad
  \brb{\abs{\Fq(\omega)}^2
    = \sum_{n=0}^{\xN}\abs{a_n}^2 + 2\sum_{n=0}^{\xN} \sum_{m=n+1}^{\xN} a_n a_m^\ast \cos\brs{\omega(n-m)}
    }
  }
\end{lemma}
\begin{proof}
\begin{align*}
  \boxed{\abs{\Fq(\omega)}^2}
    &= \brs{\abs{\Zq(z)}^2}_{z=e^{i\omega}}
  \\&= \brs{\Zq(z)\Zq^\ast(z)}_{z=e^{i\omega}}
  \\&= \brs{\brp{\sum_{n=0}^{\xN} a_n      z^{-n}}
            \brp{\sum_{n=0}^{\xN} a_n^\ast z^{ n}}}_{z=e^{i\omega}}
  \\&= \brs{\sum_{m=0}^{\xN}\sum_{n=0}^{\xN} a_m a_n^\ast z^{n-m}}_{z=e^{i\omega}}
  \\&= \sum_{m=0}^{\xN}\sum_{n=0}^{\xN} a_m a_n^\ast e^{i\omega(n-m)}
  \\&= \sum_{n=0}^{\xN}a_n^2 + 2\sum_{n=0}^{\xN}\ds\sum_{m=n+1}^{\xN} a_n a_m \cos\brs{\omega(n-m)}
    && \text{by \prefp{lem:aacos}}
\end{align*}
\end{proof}

%---------------------------------------
\begin{theorem}
%---------------------------------------
\thmbox{
  \abs{\Fh(\omega)}^2
    =\frac{\ds\sum_{n=0}^{\xN}b_n^2 + 2\sum_{n=0}^{\xN}\ds\sum_{m=n+1}^{\xN} b_n b_m \cos\brs{\omega(n-m)}}
          {\ds\sum_{n=0}^{\xN}a_n^2 + 2\sum_{n=0}^{\xN}\ds\sum_{m=n+1}^{\xN} a_n a_m \cos\brs{\omega(n-m)}}
  }
\end{theorem}
\begin{proof}
\begin{align*}
  \boxed{\abs{\Fh(\omega)}^2}
    &= \abs{\Zh(z)}^2_{z=e^{i\omega}}
  \\&= \brs{\Zh(z)\Zh^\ast(z)}_{z=e^{i\omega}}
  \\&= \abs{\frac{\ds\sum_{n=0}^{\xN}b_n z^{-n}}{\ds\sum_{n=0}^{\xN}a_n z^{-n}}}^2_{z=e^{i\omega}}  
  \\&=\boxed{\frac{\ds\sum_{n=0}^{\xN}b_n^2 + 2\sum_{n=0}^{\xN}\ds\sum_{m=n+1}^{\xN} b_n b_m \cos\brs{\omega(n-m)}}
                  {\ds\sum_{n=0}^{\xN}a_n^2 + 2\sum_{n=0}^{\xN}\ds\sum_{m=n+1}^{\xN} a_n a_m \cos\brs{\omega(n-m)}}}
    && \text{by \prefp{lem:qzcos}}
\end{align*}
\end{proof}


%---------------------------------------
\begin{theorem}
%---------------------------------------
\thmbox{\ddw\abs{\Fh(\omega)}^2_{\omega=0} = 0}
\end{theorem}
\begin{proof}
\begin{align*}
  \ddw\abs{\Zh(z)}^2_{z=e^{i\omega},\omega=0}  
    &= \ddw\brs{\Zh(z)\Zh^\ast(z)}_{z=e^{i\omega},\omega=0}
  \\&= \ddw\brs{\frac{\ds\sum_{n=0}^{\xN}b_n^2 + 2\sum_{n=0}^{\xN}\ds\sum_{m=n}^{\xN} b_n b_m \cos\brs{\omega(n-m)}}
                 {\ds\sum_{n=0}^{\xN}a_n^2 + 2\sum_{n=0}^{\xN}\ds\sum_{m=n}^{\xN} a_n a_m \cos\brs{\omega(n-m)}}}_{\omega=0}
  \\&\eqd \ddw\brs{\frac{\ff(\omega)}{\fg(\omega)}}_{\omega=0}
  \\&= \brs{\frac{\ff'(\omega)\fg(\omega) - \ff(\omega)\fg'(\omega)}{\fg^2(\omega)}}_{\omega=0}
    \qquad\text{by the \thmb{Quotient Rule}}
  \\&= 0
    \qquad\brp{\begin{array}[t]{MMM}
      because & $\ddw$constant$=0$                   & and \\
              & $\ddw\cos(k\omega)=-\sin(k\omega)=0$ & at $\omega=0,\,\pi$
    \end{array}}
\end{align*}
\end{proof}


%---------------------------------------
\section{Conversion from low-pass to high-pass}
%---------------------------------------
%---------------------------------------
\begin{theorem}
%---------------------------------------
\thmbox{
  \brb{\begin{array}{M}
    $\Zh(z)$ is \propb{low-pass}
  \end{array}}
  \quad\implies\quad
  \brb{\begin{array}{M}
    $\Zh(-z)$ is \propb{high-pass}
  \end{array}}
  }
\end{theorem}
\begin{proof}
  \begin{align*}
    \abs{\Fg(\omega)}^2
      &\eqd \abs{\Zh(-z)}_{z=e^{i\omega}}
      && \text{by definition of $\Fg(\omega)$}
    \\&= \abs{\Zh(e^{-i\pi}z)}_{z=e^{i\omega}}
    \\&= \abs{\Zh(z)}_{z=e^{i\omega}e^{-i\pi}}
    \\&= \abs{\Zh(z)}_{z=e^{i(\omega-\pi)}}
    \\&\eqd \abs{\Fh(\omega-\pi)}^2
      && \text{by definition of $\Fh(\omega)$}
  \end{align*}
\end{proof}


%============================================================================
% Daniel J. Greenhoe
% GNU Ocatave file
% 1st order low pass filter design
%============================================================================
%=======================================
\chapter{Coefficient Calculation}
%=======================================
%--------------------------------------
1st Order Low-Pass calculation
%--------------------------------------
\vfill
\qquad$\ds\boxed{ H(z) = \frac{a + bz^{-1}}
                        {1 + cz^{-1}}
          }$

\begin{align*}
  0 &= \abs{H(z)}_{z=e^{i\pi}=-1}
   &&= \frac{a + bz^{-1}}{1 + cz^{-1}}_{z=-1}
   &&= \frac{a - b}{1 - c}
   &&\implies \boxed{a=b}
  \\
  \\
  1 &= \abs{H(z)}_{z=e^{i0}=1}
   &&= \frac{a + bz^{-1}}{1 + cz^{-1}}_{z=1}
   &&= \frac{a + b}{1 + c}
   &&\implies \boxed{a=\frac{c+1}{2}}
\end{align*}
\mbox{}\vfill
\begin{align*}
  \frac{1}{2}
    &= \abs{H(z)}^2_{z=e^{i\omega_c}}
   &&= \brlr{H(z)H^\ast(z)}_{z=e^{i\omega_c}}
  \\&= \brp{\frac{a + be^{-i\omega_c}}{1 + ce^{-i\omega_c}}}
       \brp{\frac{a + be^{i\omega_c}}{1 + ce^{i\omega_c}}}
   &&= \frac{a^2 + abe^{-i\omega_c} + abe^{i\omega_c} + b^2}
            {1   +  ce^{i\omega_c} +  ce^{-i\omega_c} + c^2}
  \\&= \frac{a^2 + 2ab\cos\brp{\omega_c} + b^2}
            {1   + 2 c\cos\brp{\omega_c} + c^2}
    && \begin{array}[t]{MM}because $e^{i\theta}+e^{-i\theta}=2\cos(\theta)$\\
                       (\thme{Euler formulas})\footnotemark\end{array}
  \\&= \frac{2a^2\brs{1 +  \cos\brp{\omega_c}}}
            {c^2 + 2 c\cos\brp{\omega_c} + 1}
    && \text{because $\ds a=b$}
  \\&= \frac{2\brp{\frac{c+1}{2}}^2\brs{1 +  \cos\brp{\omega_c}}}
            {c^2 + 2 c\cos\brp{\omega_c} + 1}
    && \text{because $\ds a=\frac{c+1}{2}$}
\end{align*}
\footnotetext{
  \citor{euler1748v1},
  \citerp{bottazzini}{12}
  }



{\begin{align*}
    \implies &c^2 + 2 c\cos\brp{\omega_c} + 1 
             = \brp{c+1}^2\brs{1 +  \cos\brp{\omega_c}}
  \\\\\implies & c^2\brs{1-1-\cos\brp{\omega_c}} 
             + c  \brs{2\cos\brp{\omega_c}-2-2\cos\brp{\omega_c}}
             +    \brs{1-1-\cos\brp{\omega_c}}
             \\&= 0
  \\\\\implies & c^2\brs{\cos\brp{\omega_c}} 
             + c  \brs{2}
             +    \brs{\cos\brp{\omega_c}}
             = 0
\end{align*}}



{\begin{align*}
  \\\implies c &= \frac{-2 \pm \sqrt{(2)^2 -4\cos^2\brp{\omega_c}}}
                       {2\cos\brp{\omega_c}}
               && \text{by \thme{Quadratic Equation}}
             \\&= \frac{-1 \pm \sqrt{1 -\cos^2\brp{\omega_c}}}
                       {\cos\brp{\omega_c}}
             \\&= \frac{-1 \pm \sin\brp{\omega_c}}
                       {\cos\brp{\omega_c}}
               && \text{because $\sin^2x + \cos^2x = 1$ for all $x\in\R$}
  \\
  \\\implies &\ds\boxed{c = \frac{-1 + \sin\brp{\omega_c}}
                    {\cos\brp{\omega_c}}}
\end{align*}}

%---------------------------------------
Where is the zero? Where is the pole?
%---------------------------------------
The zero is at z=-1. \qquad The pole is at 
$\ds\boxed{z=-c = \frac{1 - \sin\brp{\omega_c}}
                    {\cos\brp{\omega_c}}}$

%---------------------------------------
\begin{example}[\exmd{1st order low-pass with corner frequency $\omega_c=\frac{2}{3}\pi$}]
%---------------------------------------
{\begin{align*}
  c &= \frac{-1 + \sin\brp{\omega_c}}{\cos\brp{\omega_c}}
     = \frac{-1 + \sin\brp{\sfrac{2}{3}\pi}}{\cos\brp{\sfrac{2}{3}\pi}}
     = \frac{-1 + \frac{\sqrt{3}}{2}}{-\sfrac{1}{2}}
     = 2 - \sqrt{3}
  \\
  \\
  \abs{H(\omega)}^2
    &= \abs{H(z)}^2_{z=e^{i\omega}}
     = \brlr{H(z)H^\ast(z)}_{z=e^{i\omega}}
     = \frac{2\brp{\frac{c+1}{2}}^2\brs{1 +  \cos\brp{\omega}}}
            {c^2 + 2 c\cos\brp{\omega} + 1}
\end{align*}}

\includegraphics{../common/math/graphics/pdfs/lpfwc.pdf}
\end{example}



%--------------------------------------
1st Order High-Pass calculation
%--------------------------------------
$\ds\boxed{ H(z) = \frac{a + bz^{-1}}
                        {1 + cz^{-1}}
          }$

\begin{align*}
  0 &= {\abs{H(z)}}_{z=e^{i0}=1}    
    &&=\brlr{\frac{a + bz^{-1}}{1 + cz^{-1}}}_{z=1} 
    &&=\frac{a + b}{1 + c}                   
    && \implies \boxed{a=-b}
  \\\\
  1 &= \abs{H(z)}_{z=e^{i\pi}=-1} 
   &&= \brlr{\frac{a + bz^{-1}}{1 + cz^{-1}}}_{z=-1} 
   &&= \frac{a - b}{1 - c} = \frac{2a}{1-c}                   
   &&  \implies \boxed{a=\frac{1-c}{2}}
\end{align*}


%
\begin{align*}
  \frac{1}{2}
    &= \abs{H(z)}^2_{z=e^{i\omega_c}}
   &&= \brs{H(z)H^\ast(z)}_{z=e^{i\omega_c}}
  \\&= \brp{\frac{a + be^{-i\omega_c}}{1 + ce^{-i\omega_c}}}
       \brp{\frac{a + be^{i\omega_c}}{1 + ce^{i\omega_c}}}
   &&= \frac{a^2 + abe^{-i\omega_c} + abe^{i\omega_c} + b^2}
            {1   +  ce^{i\omega_c} +  ce^{-i\omega_c} + c^2}
  \\&= \frac{a^2 + 2ab\cos\brp{\omega_c} + b^2}
            {1   + 2 c\cos\brp{\omega_c} + c^2}
  \\&= \frac{2a^2\brs{1 -  \cos\brp{\omega_c}}}
            {c^2 + 2 c\cos\brp{\omega_c} + 1}
    && \text{because $a=-b$}
  \\&= \frac{2\brp{\frac{1-c}{2}}^2\brs{1 -  \cos\brp{\omega_c}}}
            {c^2 + 2 c\cos\brp{\omega_c} + 1}
    && \text{because $a=\frac{1-c}{2}$}
\end{align*}


{\begin{align*}
    \implies &c^2 + 2 c\cos\brp{\omega_c} + 1 
             = \brp{1-c}^2\brs{1 -  \cos\brp{\omega_c}}
  \\\\\implies & c^2\brs{1-1+\cos\brp{\omega_c}} 
             + c  \brs{2\cos\brp{\omega_c}+2-2\cos\brp{\omega_c}}
             +    \brs{1-1+\cos\brp{\omega_c}}
             \\&= 0
  \\\\\implies & c^2{\cos\brp{\omega_c}} 
             + 2c 
             +    \cos\brp{\omega_c}
             = 0
\end{align*}}



{\begin{align*}
  \\\implies c &= \frac{-2 \pm \sqrt{(2)^2 -4\cos^2\brp{\omega_c}}}
                       {2\cos\brp{\omega_c}}
               && \text{by \thme{Quadratic Equation}}
             \\&= \frac{-1 \pm \sqrt{1 -\cos^2\brp{\omega_c}}}
                       {\cos\brp{\omega_c}}
             \\&= \frac{-1 \pm \sin\brp{\omega_c}}
                       {\cos\brp{\omega_c}}
               && \text{because $\sin^2x + \cos^2x = 1$ for all $x\in\R$}
  \\\implies c &=
      \boxed{\frac{-1 + \sin\brp{\omega_c}}
                    {\cos\brp{\omega_c}}}
               && \text{because want pole inside unit circle}
\end{align*}}




%---------------------------------------
\begin{example}[\exmd{1st order high-pass with corner frequency $\omega_c=\frac{1}{3}\pi$}
%---------------------------------------

{\begin{align*}
  c &= \frac{-1 + \sin\brp{\omega_c}}{\cos\brp{\omega_c}}
     = \frac{-1 + \sin\brp{\sfrac{1}{3}\pi}}{\cos\brp{\sfrac{1}{3}\pi}}
     = \frac{-1 + \frac{\sqrt{3}}{2}}{\frac{1}{2}}
     = \sqrt{3} - 2
  \\
  \\
  \abs{H(\omega)}^2
    &= \abs{H(z)}^2_{z=e^{i\omega}}
     = \brlr{H(z)H^\ast(z)}_{z=e^{i\omega}}
     = \frac{2\brp{\frac{1-c}{2}}^2\brs{1 -  \cos\brp{\omega}}}
            {c^2 + 2 c\cos\brp{\omega} + 1}
\end{align*}}

\includegraphics{../common/math/graphics/pdfs/hpfwc.pdf}
\end{example}


So $c=\sqrt{3}-2$,\qquad $a=\frac{1-c}{2}=\frac{3-\sqrt{3}}{2}$, \qquad $b=-a=\frac{\sqrt{3}-3}{2}$
\begin{align*}
  H_{hp}(z)
    &= \frac{a + bz^{-1}}{1+cz^{-1}}
  \\&= \frac{\brp{\frac{3-\sqrt{3}}{2}} + \brp{\frac{\sqrt{3}-3}{2}}z^{-1}}
            {1+\brp{\sqrt{3}-2}z^{-1}}
  \\&= \frac{\brp{\frac{[2-\sqrt{3}]+1}{2}} + \brp{\frac{[2-\sqrt{3}]+1}{2}}(-z)^{-1}}
            {1+\brp{2-\sqrt{3}}(-z)^{-1}}
  \\&= H_{lp}(-z)
\end{align*}







%============================================================================
% XeLaTeX File
% Daniel J. Greenhoe
%============================================================================
%=======================================
\chapter{DSP Calculus}
%=======================================

%=======================================
\section{Why it's important}
%=======================================
For modeling real-world processes above the quantum level, measurements are \prope{continuous} in time---that is,
the first derivative of a function over time representing the measurement \prope{exists}.

But even for ``simple" physical systems, it is not just the first derivative that matters.
For example, the classical ``vibrating string" vertical displacement $\fu(x,t)$ wave equation can be described as
        \\\indentx$\ds\ppderiv{u}{x^2} - \frac{1}{c^2}\ppderiv{u}{t^2} = 0$

\begin{figure}
  \centering
  \begin{tabular}{c}
    \includegraphics[width=\tw-10mm]{graphics/sunspots.pdf}\\
    \includegraphics[width=\tw-10mm,height=50mm]{graphics/earthquake_kapi_20180928.pdf}
  \end{tabular}
  \caption{Sunspot and earthquake measurements\label{fig:sunspot}}
\end{figure}
Not only do physical systems demonstrate heavy dependence on the derivatives of their measurement functions,
but also commonly exhibit \prope{oscillation}, as demonstrated by sunspot activity over the last 300 years or
earthquake activity \xref{fig:sunspot}.

In fact, derivatives and oscillations are fundamentally linked
as demonstrated by the fact that 
all solutions of homogeneous second order differential equations
        are linear combinations of sine and cosine functions\ifsxref{harTrig}{thm:D2f_cos_sin}:
        \\\indentx$\ds  \brb{\opDiff\ff + \ff=0}
  %     {\fncte{2nd order homogeneous differential equation}}
  \quad\iff\quad
  \brb{\ff(x) = \ff(0)\,\cos(x) + \ff^\prime(0)\,\sin(x)}
  \qquad\msizes\forall\ff\in\spC,\,\forall x\in\R$

Derivatives are calculated \prope{locally} about a point.
Oscillations are observed \prope{globally} over a range,
and analyzed (decomposed) by projecting the function onto a sequence of basis functions---sinusoids 
in the case of Fourier Transform family.
Projection is accomplished using inner products, and often these are calculated using \ope{integration}.
Note that derivatives and integrals are also fundamentally linked as demonstrated by the
\thme{Fundamental Theorem of Calculus}\ldots which shows that integration 
can be calculated using anti-differentiation:
\\\indentx$\ds\int_a^b \ff(x)\dx = \fF(b) - \fF(a)$\qquad where $\fF(x)$ is the \fncte{antiderivative} of $\ff(x)$.

%In nature, take displacement, velocity, and acceleration for example:
%\\\indentx$\begin{array}{rc>{\ds}l}
%  \fx(t) &=& \int_{u=0}^t \fv(u) \du + \mcom{\fx(0)}{initial condition}
%  \\
%  \fv(t) &=& \int_{u=0}^t \fa(u) \du + \mcom{\fv(0)}{initial condition}
%\end{array}$

Brook Taylor showed that for \prope{analytic} functions,\footnote{
  \prope{analytic} functions: Functions for which all their derivatives exist.}
knowledge of the derivatives of a function at a location $x=a$
%\\\indentx$\seqn{\ff(a),\, \frac{1}{1!}\ff'(a),\,  \frac{1}{2!}\ff''(a),\,  \frac{1}{3!}\ff'''(a),\cdots}$\\
%of the Taylor polynomial at the point $x=a$ 
allows you to determine (predict) arbitrarily closely all the points $\ff(x)$ in the vicinity of $x=a$:\footnote{
\citePp{robinson1982}{886}
}
\\\indentx$\ds\ff(x) = \ff(a) + \frac{1}{1!}\ff'  (a)\brs{x-a}
                              + \frac{1}{2!}\ff'' (x)\brs{x-a}^2
                              + \frac{1}{3!}\ff'''(x)\brs{x-a}^3
                              + \cdots
                  $

On the other hand, the \ope{Fourier Transform} is a kind of counter-part of the Taylor expansion:\footnote{
        \citePp{robinson1982}{886}
        }
        %\begin{tabular}{|@{2}{p{\tw/2-5mm}|}}
        \\\begin{tabular}{|c|l|l|}
            \hline
              & \mc{1}{|c|}{Taylor coefficients} & \mc{1}{c|}{Fourier coefficients}
            \\\hline
              \imark&Depend on derivatives $\ds\dndxn\ff(x)$        &Depend on integrals   $\ds\int_{x\in\R} \ff(x)e^{-i\omega x} \dx$
            \\\imark&Behavior in the vicinity of a point.           &Behavior over the entire function.
            \\\imark&Demonstrate trends locally.                    &Demonstrate trends globally, such as oscillations.
            \\\imark&Admits \prope{analytic} functions only.        &Admits \prope{non-analytic} functions as well.
            \\\imark&Function must be \prope{continuous}.           &Function can be \prope{discontinuous}.
            \\\hline
        \end{tabular}

%=======================================
\section{Fourier Transform calculus}
%=======================================
%---------------------------------------
\begin{theorem}
\label{thm:opLT_diff}
%---------------------------------------
Let $\opLT$ be the \ope{Laplace Transform} operator.
\mbox{}\\
\thmbox{
    \brb{\lim_{t\to-\infty}\fx(t)=0}
    \qquad\implies\qquad
    \brb{\opLT \brs{\ddt \fx(t)} = s \brs{\opLT\fx}(s)}
  }
\end{theorem}
\begin{proof}
\begin{align*}
  \boxed{\opLT \brs{\ddt \fx(t)}} 
    &\eqd \int_{t\in\R} \mcom{\brs{\ddt \fx(t)}}{$\dv$} \mcom{e^{-st}}{$u$} \dt
    && \text{by definition of $\opLT$}
  \\&= \brlr{\mcom{e^{-st}}{$u$} \mcom{\fx(t)}{$v$}}_{t=-\infty}^{t=+\infty}
      -\int_{t\in\R} \mcom{\fx(t)}{$v$} \mcom{(-s)e^{-s t}}{$\du$} \dt
    && \text{by \thme{Integration by Parts}}
  \\&= \cancelto{0}{e^{-s \infty}}\fx(\infty) - e^{s \infty}\cancelto{0}{\fx(-\infty)} 
      -(-s)\mcom{\int_{t\in\R} \fx(t) e^{-st} \dt}{\ope{Laplace Transform} of $\fx(t)$}
    && \text{by left hypothesis}
  \\&= \boxed{s \brs{\opLT\fx}(s)}
\end{align*}
\end{proof}

%---------------------------------------
\begin{corollary}
%---------------------------------------
Let $\opFT$ be the \ope{Fourier Transform} operator.
\corbox{
    \brb{\lim_{t\to-\infty}\fx(t)=0}
    \qquad\implies\qquad
    \brb{\opFT \brs{\ddt \fx(t)} = i\omega \brs{\opFT\fx}(\omega)}
  }
\end{corollary}
\begin{proof}
\begin{align*}
  \opFT \brs{\ddt \fx(t)}
    &\eqd \brlr{\opLT\brs{\ddt \fx(t)}(s)}_{s=i\omega}
    && \text{by definitions of $\opLT$ and $\opFT$}
  \\&= \brlr{s \brs{\opLT\fx(t)}(s)}_{s=i\omega}
    && \text{by \prefp{thm:opLT_diff}}
  \\&= i\omega \brs{\opFT\fx}(\omega)
\end{align*}
\end{proof}

%---------------------------------------
\begin{theorem}
\label{thm:opLT_int}
%---------------------------------------
Let $\opLT$ be the \ope{Laplace Transform} operator.
\thmbox{
    %\brb{\lim_{t\to-\infty}\fx(t)=0}
    %\qquad\implies\qquad
    \opLT \int_{u=-\infty}^{u=t} \fx(u) \du = \frac{1}{s} \brs{\opLT\fx}(s)
  }
\end{theorem}
\begin{proof}
\begin{enumerate}
  \item Define the \fncte{Heaviside function} $\fh(t)$ as\qquad \label{idef:heaviside}
    $\ds \fh(t)\eqd\brbl{\begin{array}{lM}
                           1 & for $t\ge0$
                         \\0 & otherwise
                         \end{array}}$

  \item Remainder of proof\ldots
    \begin{align*}
      \boxed{\opLT \int_{u=-\infty}^{u=t} \fx(u) \du}
        &\eqd \int_{t=-\infty}^{t=+\infty} \brs{\int_{u=-\infty}^{u=t} \fx(u) \du} e^{-s t} \dt
        && \text{by definition of $\opLT$}
      \\&= \int_{t=-\infty}^{t=+\infty} \brs{\int_{u=-\infty}^{u=+\infty} \fx(u) \fh(t-u) \du} e^{-s t} \dt
        && \brp{\begin{array}{M}
             by definition of \fncte{Heaviside function}\\
             \pref{idef:heaviside}
           \end{array}}
      \\&= \int_{v=-\infty}^{v=+\infty} \int_{u=-\infty}^{u=+\infty} \fx(u) \fh(v)  e^{-s (u+v)} \du \dv
        && \brp{\begin{array}{Mr@{\hspace{2pt}}c@{\hspace{2pt}}l}
             where      & v&\eqd&t-u\\ 
             $\implies$ & t&=&u+v
           \end{array}}
      \\&= \brs{\int_{v=-\infty}^{v=+\infty} \fh(v) e^{-s v} \dv} 
            \mcom{\brs{\int_{u=-\infty}^{u=+\infty} \fx(u)   e^{-s u} \du }}
                 {\ope{Laplace Transform} of $\fx(t)$}
      \\&= \brs{\int_{v=0}^{v=\infty} e^{-s v} \dv} \brs{\opLT\fx}(s)
        && \brp{\begin{array}{M}
             by definition of \fncte{Heaviside function}\\
             \pref{idef:heaviside}
           \end{array}}
      \\&= \brlr{\frac{1}{-s}e^{-s v}}_{v=0}^{v=\infty} \brs{\opLT\fx}(s)
        && \text{by \thme{Fundamental Theorem of Calculus}}
      \\&= \boxed{\frac{1}{s} \brs{\opLT\fx}(s)}
    \end{align*}
\end{enumerate}
\end{proof}

%---------------------------------------
\begin{corollary}
%---------------------------------------
Let $\opFT$ be the \ope{Fourier Transform} operator.
\corbox{
    \opFT \int_{u=-\infty}^{u=t} \fx(u) \du = \frac{1}{i\omega} \brs{\opFT\fx}(\omega)
  }
\end{corollary}
\begin{proof}
\begin{align*}
  \opFT \int_{u=-\infty}^{u=t} \fx(u) \du
    &\eqd \brlr{\opLT \int_{u=-\infty}^{u=t} \fx(u) \du}_{s=i\omega}
  \\&=    \brlr{\frac{1}{s} \brs{\opLT\fx(t)}(s)}_{s=i\omega}
    && \text{by \prefp{thm:opLT_int}}
  \\&=    \frac{1}{i\omega} \brs{\opFT\fx(t)}(\omega)
\end{align*}
\end{proof}

%=======================================
\section{Digital differentiation methods}
%=======================================
%=======================================
%\subsection{Method 1: Difference}
%=======================================


Digital Differentiation Method \#1: \ope{Difference}\footnote{\citerpgc{williams1986}{69}{9780132018562}{Difference}}
 
\begin{align*}
  \fy[n]
    &\eqd \fx[n] - \fx[n-1]
  \\\\
  \opZ\brb{\fy[n]} &= \opZ\brb{\fx[n] - \fx[n-1]}
  \\
  \ZY(z) &= \ZX(z) + z^{-1}\ZX(z)
  \\\\
  \frac{\ZY(z)}{\ZX(z)} &= 1-z^{-1} \quad=\quad \boxed{\frac{z-1}{z}}
  \qquad\brb{\begin{array}{MM}
    How many zeros? & Where?\\
    How many poles? & Where?
  \end{array}}
\end{align*}

%---------------------------------------
%\newpage\mbox{} 
%---------------------------------------
 
\begin{figure}[h]
  \centering
  \begin{tabular}{|c|c|}
      \hline
      \includegraphics{graphics/diff.pdf} & \includegraphics{graphics/cendiff.pdf}
    \\Difference method & Central Difference method
    \\\hline
  \end{tabular}
  \caption{Digital differentiation methods\label{fig:differentiation}}
\end{figure}

Is digital differentiation equivalent to continuous differentiation?\footnote{\citerpgc{williams1986}{70}{9780132018562}{Figure 2.14(a)}}
 
\begin{align*}
  \abs{\frac{z-1}{z}}_{z=e^{i\omega}}
    &= \abs{\frac{e^{i\omega}-1}
                 {e^{i\omega}}
           }
  \\&= \abs{\frac{e^{i\omega/2}\brp{e^{i\omega/2}-e^{-i\omega/2}}}
                 {e^{i\omega}}
           }
   &&= \abs{\mcom{e^{-i\omega/2}}{phase}\,\mcom{{2\sin\brp{\frac{\omega}{2}}}}{magnitude}}
  \\&= \boxed{{2\sin\brp{\frac{\omega}{2}}}} 
    && \text{for $0\le\omega\le\pi$}
\end{align*}

%=======================================
\subsection{Digital Differentiation Method \#2: \ope{Central Difference}}
\footnote{\citerpgc{williams1986}{69}{9780132018562}{Difference}}
%=======================================
 
\begin{align*}
  \fy[n]
    &\eqd \frac{\fx[n] - \fx[n-2]}{2}
  \\\\
  Y(z) &= \frac{X(z) + z^{-1}X(z)}{2}
  \\\\
  \frac{Y(z)}{X(z)} &= \frac{1-z^{-1}}{2} \quad=\quad {\frac{z^2-1}{2z^2}}
  \\\\
                    &= \boxed{\frac{(z+1)(z-1)}{2z^2}} 
  \qquad\brb{\begin{array}{MM}
    How many zeros? & Where?\\
    How many poles? & Where?
  \end{array}}
\end{align*}


Central Difference = Continuous Differentiation?\footnote{\citerpgc{williams1986}{70}{9780132018562}{Figure 2.14(b)}}
 
\begin{align*}
  \abs{\frac{z^2-1}{2z^2}}_{z=e^{i\omega}}
    &= \abs{\frac{e^{2i\omega}-1}
                 {2e^{2i\omega}}
           }
     = \abs{\brp{\frac{e^{i\omega}}{e^{2i\omega}}}
            \frac{\brp{e^{i\omega}-e^{-i\omega}}}{2}
           }
  \\&= \abs{\brp{e^{-i\omega}}
            \frac{\brs{cos(\omega)+i\sin(\omega)}-\brs{\cos(\omega)+i\sin(-\omega)}}{2}
           }
  \\&= \abs{\brp{e^{-i\omega}}
            \frac{\brs{cos(\omega)+i\sin(\omega)}-\brs{\cos(\omega)-i\sin(\omega)}}{2}
           }
  \\&= \abs{\brp{e^{-i\omega+\pi/2}}
            \frac{2\sin(\omega)}{2}
           }
     = \boxed{\abs{\sin(\omega)}}
\end{align*}


%=======================================
\section{Digital integration}
%=======================================
%=======================================
\subsection{Digital Integration Method \#1: \opd{Summation}}
%=======================================

{\begin{align*}
  \fy[n]
    &\eqd \fx[n] + \mcom{\fx[n-1] + \fx[n-2] + \fx[n-3] + \fx[n-4] + \fx[n-5] + \cdots}{{$\fy[n-1]$}}
  \\
  \fy[n] &=    \fx[n] + \fy[n-1]
  \\\\
  \opZ\brb{\fy[n]} &= \opZ\brb{\fx[n] + \fy[n-1]}
  \\
  Y(z) &= X(z) + z^{-1}Y(z)
  \\
  Y(z)\brs{1-z^{-1}} &= X(z)
  \\\\
  \frac{Y(z)}{X(z)} &= \frac{1}{1-z^{-1}} \quad=\quad \boxed{\frac{z}{z-1}}
  \qquad\brb{\begin{array}{MM}
    How many zeros? & Where?\\
    How many poles? & Where?
  \end{array}}
\end{align*}}


%---------------------------------------
%\newpage\mbox{} 
%---------------------------------------
%=======================================
\subsection{Digital Integration Method \#2: \opd{Trapezoid}}
%=======================================
 
{\begin{align*}
  \fy[n]
    &\eqd \frac{\fx[n]+\fx[n-1]}{2} + \frac{\fx[n-1]+\fx[n-2]}{2} + \frac{\fx[n-2]+\fx[n-3]}{2} + \cdots
  \\&=    \sfrac{1}{2}\fx[n] + \mcom{\fx[n-1] + \fx[n-2] + \fx[n-3] + \fx[n-4] + \fx[n-5] + \cdots}{{$\fy[n-1]+\sfrac{1}{2}\fx[n-1]$}}
  \\&=    \sfrac{1}{2}\fx[n] + \fy[n-1]+\sfrac{1}{2}\fx[n-1]
  \\\\
  \fy[n]-\fy[n-1]&=    \sfrac{1}{2}\brs{\fx[n] + +\fx[n-1]} 
  \\\\
  Y(z)\brs{1-z^{-1}} &= \sfrac{1}{2} X(z)\brs{1+z^{-1}}
  \\\\
  \frac{Y(z)}{X(z)} 
    &= \brp{\frac{1}{2}} \frac{1+z^{-1}}{1-z^{-1}}
     = \boxed{\brp{\frac{1}{2}} \frac{z+1}{z-1}} 
  \qquad\brb{\begin{array}{MM}
    How many zeros? & Where?\\
    How many poles? & Where?
  \end{array}}
\end{align*}}

%=======================================
\subsection{Digital Integration Method \#3:  \opd{Simpson's Rule}}
%=======================================

% 
%{\begin{align*}
%  \fy[n]
%    &\eqd \frac{\fx[n  ]+4\fx[n -1]+\fx[n -2]}{3}   
%     +    \frac{\fx[n-1]+4\fx[n -2]+\fx[n -3]}{3}   
%     +    \frac{\fx[n-2]+4\fx[n -3]+\fx[n -4]}{3} 
%     +    \frac{\fx[n-3]+4\fx[n -4]+\fx[n -5]}{3} 
%     +    \frac{\fx[n-4]+4\fx[n -5]+\fx[n -6]}{3} 
%     +    \frac{\fx[n-5]+4\fx[n -6]+\fx[n -7]}{3} 
%     +    \frac{\fx[n-6]+4\fx[n -7]+\fx[n -8]}{3} 
%     +    \frac{\fx[n-7]+4\fx[n -8]+\fx[n -9]}{3} 
%     +    \frac{\fx[n-8]+4\fx[n -9]+\fx[n-10]}{3} 
%     +    \frac{\fx[n-9]+4\fx[n-10]+\fx[n-11]}{3} 
%     +    \cdots  
%  \\&=    \sfrac{1}{3}\fx[n]+\sfrac{5}{3}\fx[n-1] 
%        + \mcom{2\fx[n-2] + 2\fx[n-3] + 2\fx[n-4] + \cdots}
%               {$\fy[n-2]+\sfrac{5}{3}\fx[n-2]+\sfrac{1}{3}\fx[n-3]$}
%
%
%  \\&=    \sfrac{1}{3}\fx[n]+\sfrac{5}{3}\fx[n-1] 
%        + \mcom{2\fx[n-2] + 2\fx[n-3] + 2\fx[n-4] + \cdots}
%               {$\fy[n-2]+\sfrac{5}{3}\fx[n-2]+\sfrac{1}{3}\fx[n-3]$}
%
%
%  \\&=    \sfrac{1}{3}\fx[n]+\mcom{\sfrac{5}{3}\fx[n-1] + 2\fx[n-2] + 2\fx[n-3] + 2\fx[n-4] + \cdots}
%                                  {$\fy[n-1]+\sfrac{2}{3}\fx[n-1]+\sfrac{1}{3}\fx[n-2]$}
%  \\&=    \fy[n-1] - \sfrac{1}{3}\brp{\fx[n]+ 2\fx[n-1]+\fx[n-2]}
%
%
%
%
%     +    \frac{\fx[n-1]+4\fx[n-2]+\fx[n-3]}{3}   
%     +    \frac{\fx[n-2]+4\fx[n-3]+\fx[n-4]}{3} 
%     +    \cdots  
%  \\&=    \sfrac{1}{2}\fx[n] + \mcom{\fx[n-1] + \fx[n-2] + \fx[n-3] + \fx[n-4] + \fx[n-5] + \cdots}{{$\fy[n-1]+\sfrac{1}{2}\fx[n-1]$}}
%  \\&=    \sfrac{1}{2}\fx[n] + \fy[n-1]+\sfrac{1}{2}\fx[n-1]
%  \\\\
%  \fy[n]-\fy[n-1]&=    \sfrac{1}{2}\brs{\fx[n] + +\fx[n-1]} 
%  \\\\
%  Y(z)\brs{1-z^{-1}} &= \sfrac{1}{2} X(z)\brs{1+z^{-1}}
%  \\\\
%  \frac{Y(z)}{X(z)} 
%    &= \brp{\frac{1}{2}} \frac{1+z^{-1}}{1-z^{-1}}
%     = \boxed{\brp{\frac{1}{2}} \frac{z+1}{z-1}} 
%  \qquad\brb{\begin{array}{MM}
%    How many zeros? & Where?\\
%    How many poles? & Where?
%  \end{array}}
%\end{align*}}
%

%---------------------------------------
%\newpage\mbox{} 
%---------------------------------------
 
\begin{figure}[h]
  \centering
  \begin{tabular}{|c|c|}
    \hline
    \includegraphics{graphics/IntSum.pdf}&\includegraphics{graphics/IntTrap.pdf}
  \\summation integration & trapezoid integration
  \\\hline
  \end{tabular}
  \caption{Comparison of digital integration methods to analytic integration\label{fig:dspint}}
\end{figure}
Is digital summation integration equivalent to continuous integration?
Not really \xref{fig:dspint}.

\begin{align*}
  \abs{\frac{z}{z-1}}_{z=e^{i\omega}}
    &= \abs{\frac{e^{i\omega}}{e^{i\omega}-1}}
  \\&= \abs{\frac{e^{i\omega}}{e^{i\omega/2}\brp{e^{i\omega/2}-e^{-i\omega/2}}}}
   &&= \abs{\mcom{e^{i\omega/2}}{phase}\,\mcom{\frac{1}{2\sin\brp{\frac{\omega}{2}}}}{magnitude}}
  \\&= \boxed{\frac{1}{2\sin\brp{\frac{\omega}{2}}}} 
    && \text{for $0\le\omega\le\pi$}
\end{align*}


Is digital trapezoid integration equivalent to continuous integration? 
Not really \xref{fig:dspint}.
 
\begin{align*}
  \abs{\frac{1}{2}\brp{\frac{z+1}{z-1}}}_{z=e^{i\omega}}
    &= \frac{1}{2}
       \abs{\frac{e^{i\omega}+1}{e^{i\omega}-1}}
  \\&= \frac{1}{2}
       \abs{\frac{e^{i\omega/2}\brp{e^{i\omega/2}+e^{-i\omega/2}}}
                 {e^{i\omega/2}\brp{e^{i\omega/2}-e^{-i\omega/2}}}
           }
   &&= \frac{1}{2}
       \abs{\frac{2\cos\brp{\frac{\omega}{2}}}
                 {2\sin\brp{\frac{\omega}{2}}}
           }
  \\&= \boxed{\frac{1}{2} \abs{\cot\brp{\frac{\omega}{2}}}}
    && \text{for $0\le\omega\le\pi$}
\end{align*}










%--------------------------------------
% appendixes
%--------------------------------------
\part{Appendices}
\begin{appendix}

  \ifDocHasCommonInclude{transop}
  \ifDocHasCommonInclude{mra}
  \ifDocHasWaveletsInclude{fwt}
%  \ifDocHasWaveletsInclude{pounity}
%  \ifDocHasWaveletsInclude{wavstrct}
%  \ifDocHasWaveletsInclude{vanish}
%  \ifDocHasWaveletsInclude{ortho}
%  \ifDocHasWaveletsInclude{compactp}
%  \ifDocHasWaveletsInclude{symlets}
%
%
%  %\renewcommand*{\thechapter}{\AlphAlph{\value{chapter}}}
%  %\part{Foundational Structures}
    \ifDocHasCommonInclude{relation}
%    \ifDocHasCommonInclude{order}
%    \ifDocHasCommonInclude{setstrct}
%    \ifDocHasCommonInclude{algebra}
%    \ifDocHasCommonInclude{measure}
%    \ifDocHasCommonInclude{binomial}
%    \ifDocHasCommonInclude{polynom}
%
%  %\part{Topological Structure}% \vfill\mbox{}\hfill\includegraphics*[height=15mm]{../common/graphics/watercraft/ghindgray_djg.eps}%
%    \ifDocHasCommonInclude{topology}
%    \ifDocHasCommonInclude{metric}
%    \ifDocHasCommonInclude{vector}
%    \ifDocHasCommonInclude{vstopo}
%    \ifDocHasCommonInclude{functionals}
%    \ifDocHasCommonInclude{vsnorm}
%    \ifDocHasCommonInclude{vsinprod}
%    \ifDocHasCommonInclude{operator}
    \ifDocHasCommonInclude{integrat}
    %============================================================================
% Daniel J. Greenhoe
% LaTeX file
%============================================================================
%=======================================
\chapter{Calculus}
%=======================================

%---------------------------------------
\begin{definition}
\label{def:spLLR}
\label{def:spLLRBu}
%---------------------------------------
Let $\R$ be the set of real numbers, $\borel$ the set of \structe{Borel sets} on $\R$, and
$\msm$ the standard \fncte{Borel measure} on $\borel$.
Let $\clFrr$ be as in \prefp{def:spXY}.
\defbox{\begin{array}{M}
  The \structd{space of Lebesgue square-integrable functions} $\spLLRBu$ (or $\spLLR$) is defined as
  \\\indentx$\ds\hxs{\spLLR}\eqd\hxs{\spLLRBu}\eqd\set{\ff\in\clFrr}{\brp{\int_\R \abs{\ff}^2}^\frac{1}{2} \dmu < \infty}$.\\
  The \hid{standard inner product} $\inprodn$ on $\spLLR$ is defined as
  \\\indentx$\ds\inprod{\ff(x)}{\fg(x)} \eqd \int_{\R}\ff(x)\fg^\ast(x)\dx$.\\
  The \hid{standard norm} $\normn$ on $\spLLR$ is defined as $\norm{\ff(x)}\eqd\inprod{\ff(x)}{\ff(x)}^\frac{1}{2}$
  %and $\opair{\spLLRBu}{\inprodn}$ is a \structe{Hilbert space}.
\end{array}}
\end{definition}

%---------------------------------------
\begin{definition}
\label{def:ddx}
%---------------------------------------
Let $\ff(x)$ be a \structe{function} in $\clFrr$.
\defbox{
  \ddx\ff(x) \eqd \ffp(x)\eqd \lim_{\varepsilon\to0}\frac{\ff(x+\varepsilon)-\ff(x)}{\varepsilon}
  }
\end{definition}

%---------------------------------------
\begin{proposition}
\label{prop:ddx_symmetry}
%---------------------------------------
\propbox{
  \brb{\begin{array}{FMD}
    (1). & $\ff(x)$ is \prope{continuous} & and\\
    (2). & $\mcom{\ff(a+x)=\ff(a-x)}{\prope{symmetric} about a point $a$}$
  \end{array}}
  \implies
  \brb{\begin{array}{FrclD}
    (1). & \ff'(a+x)&=& -\ff'(a-x) & (\prope{anti-symmetric} about $a$)\\
    (2). & \ff'(a)  &=& 0
  \end{array}}
  }
\end{proposition}
\begin{proof}
\begin{align*}
  \ff'(a+x)
    &= \lim_{\varepsilon\to0}\frac{1}{\varepsilon}\brs{\ff(a+x+\varepsilon)-\ff(a+x-\varepsilon)}
  \\&= \lim_{\varepsilon\to0}\frac{1}{\varepsilon}\brs{\ff(a-x-\varepsilon)-\ff(a-x+\varepsilon)}
    && \text{by hpothesis (2)}
  \\&= -\lim_{\varepsilon\to0}\frac{1}{\varepsilon}\brs{\ff(a-x+\varepsilon)-\ff(a-x-\varepsilon)}
  \\&= -\ff(a-x)
  \\
  \\
  \ff'(a)
    &= \frac{1}{2}\ff'(a+0) + \frac{1}{2}\ff'(a-0)
  \\&= \frac{1}{2}\brs{\ff'(a+0) - \ff'(a+0)}
    && \text{by previous result}
  \\&= 0
\end{align*}
\end{proof}

%---------------------------------------
\begin{lemma}
\label{lem:ddyffi}
%---------------------------------------
\lembox{
  \text{$\ff(x)$ is \prope{invertible}}
  \implies
  \brb{\ddy\ffi(y)=\frac{1}{\ddx\ff\brs{\ffi(y)}}}
  }
\end{lemma}
\begin{proof}
\begin{align*}
  \ddy\ffi(y)
    &\eqd \lim_{\varepsilon\to0}\frac{\ffi(y+\varepsilon)-\ffi(y)}{\varepsilon}
    &&    \text{by definition of $\ddy$} && \text{\xref{def:ddx}}
  \\&=    \lim_{\delta\to0}\brlr{\frac{1}{\ds\brs{\frac{\ff(x+\delta)-\ff(x)}{\delta}}}}_{x\eqd\ffi(y)}
    &&    \text{because in the limit, $\frac{\Delta y}{\Delta x}=\brp{\frac{\Delta x}{\Delta y}}^{-1}$}
  \\&\eqd \brlr{\frac{1}{\ddx\ff(x)}}_{x\eqd\ffi(y)}
    &&    \text{by definition of $\ddx$} && \text{\xref{def:ddx}}
  \\&=    \frac{1}{\ddx\ff\brs{\ffi(y)}}
    &&    \text{because $x\eqd\ffi(y)$}
\end{align*}
\end{proof}

%---------------------------------------
\begin{theorem}
\footnote{
  \citerpgc{chui}{86}{0121745848}{item (ii)},
  \citerppgc{prasad}{145}{146}{0849331692}{Theorem 6.2 (b)}
  }
\label{thm:int01}
%---------------------------------------
Let $\ff$ be a continuous function in $\spLLR$ and $\ff^{(n)}$ the $n$th derivative of $\ff$.
\thmbox{
  \int_{\intco{0}{1}^n} \ff^{(n)}\brp{\sum_{k=1}^n x_k} \dx_1\dx_2\cdots\dx_n = \sum_{k=0}^n (-1)^{n-k}\bcoef{n}{k}\ff(k)
  \qquad\forall n\in\Zp
  }
\end{theorem}
\begin{proof}
Proof by induction:
  \begin{enumerate}
    \item Base case \ldots proof for $n=1$ case:
      \begin{align*}
        \int_\intco{0}{1} \ff^{(1)}\brp{x} \dx
          &= \ff(1)-\ff(0)
          && \text{by \thme{Fundamental theorem of calculus}}
        \\&= (-1)^{1+1}\bcoef{1}{1}\ff(1) + (-1)^{1+0}\bcoef{1}{0}\ff(0)
        \\&= \sum_{k=0}^1 (-1)^{n-k}\bcoef{n}{k}\ff(k)
      \end{align*}

    \item Induction step \ldots proof that $n$ case $\implies$ $n+1$ case:
      \begin{align*}
          &\int_{\intco{0}{1}^{n+1}} \ff^{(n+1)}\brp{\sum_{k=1}^{n+1} x_k} \dx_1\dx_2\cdots\dx_{n+1}
        \\&= \int_{\intco{0}{1}^{n}}\brs{\int_0^1 \ff^{(n+1)}\brp{x_{n+1}+\sum_{k=1}^{n} x_k} \dx_{n+1}}\dx_1\dx_2\cdots\dx_{n}
        \\&= \mathrlap{%
             \int_{\intco{0}{1}^n}\brs{\left. \ff^{(n)}\brp{x_{n+1}+\sum_{k=1}^n x_k}\right|_{x_{n+1}=0}^{x_{n+1}=1}} \dx_1\dx_2\cdots\dx_n
             \qquad\text{by \thme{Fundamental theorem of calculus}}}
        \\&= \int_{\intco{0}{1}^n}\brs{\ff^{(n)}\brp{1+\sum_{k=1}^n x_k}-\ff^{(n)}\brp{0+\sum_{k=1}^n x_k}} \dx_1\dx_2\cdots\dx_n
        \\&= \sum_{k=0}^n (-1)^{n-k}\bcoef{n}{k}\ff(k+1) - \sum_{k=0}^n (-1)^{n-k}\bcoef{n}{k}\ff(k)
          && \text{by induction hypothesis}
        \\&= \sum_{m=1}^{m=n+1} (-1)^{n-m+1}\bcoef{n}{m-1}\ff(m) + \sum_{k=0}^n (-1)(-1)^{n-k}\bcoef{n}{k}\ff(k)
          && \text{where $m\eqd k+1\implies k=m-1$}
        \\&=\mathrlap{% 
              \brs{ \ff(n+1)+ \sum_{k=1}^{n} (-1)^{n-k+1}\bcoef{n}{k-1}\ff(k) }
            + \brs{ (-1)^{n+1} \ff(0) + \sum_{k=1}^n (-1)^{n-k+1}\bcoef{n}{k}\ff(k)}
            }
         %&& \text{by change of dummy variable ($m\rightarrow k$)}
        \\&= \ff(n+1) + (-1)^{n+1}\ff(0) 
            + \sum_{k=1}^{n} (-1)^{n-k+1}\mcom{\brs{\bcoef{n}{k-1}+\bcoef{n}{k}}}{use \thme{Stifel formula}}\ff(k) 
        \\&= (-1)^0\bcoef{n+1}{n+1}\ff(n+1) + (-1)^{n+1}\bcoef{n+1}{0}\ff(0) 
            + \sum_{k=1}^{n} (-1)^{n-k+1}\bcoef{n+1}{k}\ff(k) 
          && \text{\begin{tabular}{l}by \thme{Stifel formula}\\\ifxref{binomial}{thm:stifel}\end{tabular}}
        \\&= \sum_{k=0}^{n+1} (-1)^{n-k+1}\bcoef{n+1}{k}\ff(k) 
      \end{align*}
  \end{enumerate}
\end{proof}

Some proofs invoke differentiation multiple times.
This is simplified thanks to the \thme{Leibniz rule}, also called the 
\hie{generalized product rule} (\hie{GPR}, next lemma).
The Leibniz rule is remarkably similar in form to the \thme{binomial theorem}.
%--------------------------------------
\begin{lemma}[\thmd{Leibniz rule} / \thmd{generalized product rule}]
%\footnote{\url{http://en.wikipedia.org/wiki/Leibniz_rule_(generalized_product_rule)}}
\footnote{
  \citerpg{benisrael2002}{154}{3211829245},
  \citor{leibniz1710}
  }
\label{lem:LGPR}
%--------------------------------------
Let $\ff(x),\fg(x)\in\spLLR$ with derivatives
$\ff^{(n)}(x)\eqd\deriv{^n}{x^n}\ff(x)$ and
$\fg^{(n)}(x)\eqd\deriv{^n}{x^n}\fg(x)$ for $n=0,1,2,\ldots$,
and ${n\choose k}\eqd\frac{n!}{(n-k)!k!}$ (binomial coefficient).
Then
\lembox{
  \deriv{^n}{x^n}[\ff(x)\fg(x)] =
  \sum_{k=0}^n {n\choose k} \ff^{(k)}(x) \fg^{(n-k)}(x)
  }
\end{lemma}

%--------------------------------------
\begin{example}
\exbox{
  \deriv{^3}{x^3}\brs{\ff(x)\fg(x)} = \ff'''(x)\fg(x) + 3\ff''(x)\fg'(x) + 3\ff'(x)\fg''(x) + \ff(x)\fg'''(x)
  }
\end{example}

%---------------------------------------
\begin{theorem}[\thmd{Leibniz integration rule}]
\footnote{
  \citePpc{flanders1973}{615}{(1.1)}
  \citer{talvila2001},
  \citerpgc{knappb2005}{389}{0817632506}{Chapter VII},
  \citerpgc{protter2012}{422}{1461210860}{Leibniz Rule. Theorem 1.},
  \url{http://planetmath.org/encyclopedia/DifferentiationUnderIntegralSign.html}
  }
\label{thm:lir}
%---------------------------------------
  \thmbox{
    \ddx \int_{\fa(x)}^{\fb(x)} \fg(t) \dt  
      = \fg\brs{\fb(x)}\fb'(x) - \fg\brs{\fa(x)}\fa'(x)
    }
\end{theorem}











%
%  %\part{Basis Theory}% \vfill\mbox{}\hfill\includegraphics*[height=15mm]{../common/graphics/watercraft/ghindgray_djg.eps}%
%    \ifDocHasCommonInclude{subspace}
%    \ifDocHasCommonInclude{ortholat}
%    \ifDocHasCommonInclude{seq}
%    \ifDocHasCommonInclude{sums}
%    \ifDocHasCommonInclude{series}
% 
%  %\part{Fourier Structure}% \vfill\mbox{}\hfill\includegraphics*[height=15mm]{../common/graphics/watercraft/ghindgray_djg.eps}%
    \ifDocHasCommonInclude{harTrig}
    \ifDocHasCommonInclude{harPoly}
    \ifDocHasCommonInclude{fs}
    \ifDocHasCommonInclude{harFour}
%
%   \ifDocHasCommonInclude{frames}
%
%%  \part{Measure Structure}% \vfill\mbox{}\hfill\includegraphics*[height=15mm]{../common/graphics/watercraft/ghindgray_djg.eps}%
%
%  %\part{Algebraic Tools}% \vfill\mbox{}\hfill\includegraphics*[height=15mm]{../common/graphics/watercraft/ghindgray_djg.eps}%
%    \ifDocHasCommonInclude{spline}
%    \ifDocHasCommonInclude{bwav}
%    \ifDocHasCommonInclude{interpo}
%    \ifDocHasCommonInclude{partuni}
%    \ifDocHasCommonInclude{pwrspec}
%
%  %\part{Support Tools}
%  %\part{Riptide}
%  %\part{Flotation Devices}% \vfill\mbox{}\hfill\includegraphics*[height=15mm]{../common/graphics/watercraft/ghindgray_djg.eps}%
%    \ifDocHasWaveletsInclude{pollen}
%    \ifDocHasCommonInclude{morph}
    \ifDocHasInclude{src_code}
\end{appendix}

\ifDocHasCommonInclude{backmat}
%--------------------------------------
% end document
%--------------------------------------

\end{document}


\end{multicols}

%---------------------------------------
\subsection*{License}
%---------------------------------------
\markboth{License}{License}
\addcontentsline{toc}{section}{License}
This document is provided under the terms of 
the \href{https://creativecommons.org/}{Creative Commons} license \href{https://creativecommons.org/licenses/}{CC BY-NC-ND 4.0}.
\\For an exact statement of the license, see 
\\\indentx\url{https://creativecommons.org/licenses/by-nc-nd/4.0/legalcode}
\\
The icon 
\tbox{\href{https://creativecommons.org/licenses/}{\ccbyncnd}}
appearing throughout this document is based on one that was once at 
\\\indentx\url{https://creativecommons.org/}\\
where it was stated, 
``Except where otherwise noted, content on this site is licensed under a Creative Commons Attribution 4.0 International license."

%---------------------------------------
% last page
%---------------------------------------
\cleartooddpage
%\EANisbn\mbox{}\\\vfill
\mbox{}\\\vfill
%\subsubsection*{last page}
%\fancyhf[HOL,HER]{\scshape{last page}}
\markboth{last page}{last page}%
\begin{center}
\includegraphics*[height=10mm]{../common/graphics/watercraft/ghind_blue.pdf}%
%\textcolor{blue}{\usebox{\robotBoyR}} 
{\sffamily\ldots last page \ldots please stop reading \ldots} 
%\textcolor{red}{\usebox{\robotGirlL}}
\includegraphics*[height=10mm]{../common/graphics/watercraft/ghind_blue.pdf}%
\label{doc:end}
\addcontentsline{toc}{section}{End of document}
%\addcontentsline{toc}{chapter}{(last page of document)}
\end{center}













