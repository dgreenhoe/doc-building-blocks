%============================================================================
% Daniel J. Greenhoe
% LaTeX file
%============================================================================





%======================================
\chapter{Reports}
%======================================
%======================================
\section{Report for tomorrow's 2007-June-15-Friday meeting}
%======================================

\begin{enumerate}

\item A commutative relationship between the 
      translation operator\footnote{\hie{translation operator}: \prefp{def:wav_opT}} $\opT$ 
      and dilation operator\footnote{\hie{dilation operator}: \prefp{def:wav_opD}} $\opD$ 
      is given by \prefpp{thm:TD_DT} and shown below.
   \formbox{
     \opT^{2n}\opD = \opD \opT^n 
     \qquad\scriptstyle
     \forall n\in\Z
     }

\item Relationships between dilation operator $\opD$ and Fourier Transform operator $\opFT$
      are given by \prefpp{thm:wav_FD} and \prefpp{cor:wav_DFD}.
      \formbox{\begin{array}{rcl cl}
        \opFT\opD &=& \opDi\opFT  \\
        \opD\opFT &=& \opFT\opDi  \\
        \opD      &=& \opFT\opDi\opFTi &=& \opFTi\opDi\opFT \\
        \opFT     &=& \opD\opFT\opD &=& \opDi\opFT\opDi  
      \end{array}}

\item A relationship between translation operator $\opT$ and Fourier Transform operator $\opFT$
      is given by \prefpp{thm:wav_FT}.
      \formbox{\begin{array}{rcl}
        \opFT\opT^n  &=& e^{-i\omega n} \opFT
      \end{array}}

\item Necessary conditions 
\begin{enumerate}
  \item Admissibility condition---\prefpp{thm:admiss}.
    \formbox{
      \inprod{\vphi}{\vunit}\ne 0
      \qquad\implies\qquad
      \mcom{\sum_n  h_n  = \sqrt{2}}{admissibility in time}
      }

  \item Neumann condition---\prefpp{thm:wav_net}
      \formbox{
      \sum_n \abs{h_n} \ge 1
      }

  \item Quadrature conditon---\prefpp{cor:wds_quadcon}.
      \formbox{
        \left.\begin{array}{llD}
          1. & \opT^{2n}\opD = \opD\opT^n & (property of $\spLL$)\\
          2. & \inprod{\phi}{\opT^n\phi} = \kdelta_n & ($\seqn{\phi_n}$ is orthonormal)
        \end{array}\right\}
        \qquad\implies\qquad
        \inprod{\fphi(t)}{\opT^n \fphi(t)}
          = \sum_m h_m h_{m-2n}^\ast 
      }

  \item Partition of unity condition---\prefpp{thm:wav_unity}.
        \formbox{
          \mcom{\sum_n \opT^n \vphi = 1}
               {partition of unity}
          \qquad\iff\qquad
          \mcom{\sum_n (-1)^n h_n=0}
               {zero at $z=1$}
          }

\end{enumerate}

\item A continuing problem for generalizing wavelets to arbitrary vector space is the 
      Poisson Summation Formula and it's dependence on $e^{-i\omega t}$.
      PSF can be drived using the Fourier Series.
      I have made an attempt to rederive PSF based on the spectral theorem.
      This should be possible because the summation $\sum_n \fx(t+n)$ is a 
      self-adjoint operator on $\fx$---\prefpp{thm:psf}.

      \formbox{\begin{array}{>{\ds}l}
         \mcom{\sum_n \ff(t+nT)}
              {summation in ``time"}
         = 
         \mcom{\sqrt{\frac{2\pi}{T}}\: \opFSi \opS\opFT\brs{\ff(t)}}
              {\parbox{6\tw/16}{proportional to the inverse Fourier series 
               of the Fourier transform of $\ff(t)$ 
               sampled at $\frac{2\pi}{T}$ intervals}
              }
         = 
         \mcom{\frac{\sqrt{2\pi}}{T} \sum_n \Ff\brp{\frac{2\pi}{T}n} e^{i\frac{2\pi}{T}nt}}
              {summation in ``frequency"}
        \\\qquad
          \begin{array}{>{$}l<{$}>{\ds}l}
            where & \text{$\opS:\spLL\to\spII$ is the \hie{sampling operator} defined as}\\
                  & \brs{\opS\ff(t)}(n) \eqd \ff\brp{\frac{2\pi}{T} n}\qquad\scriptstyle \forall \ff\in\spLL
          \end{array}
      \end{array}}

\item Wavelets developed using necessary conditions---- 
      \prefpp{sec:sim_two} (2 coefficients), \prefpp{sec:sim_four} (four coefficients),
      \prefpp{sec:sim_six} (6 coefficients).

\end{enumerate}





















