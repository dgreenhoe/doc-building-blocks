%============================================================================
% Daniel J. Greenhoe
% LaTeX file
%============================================================================
%=======================================
\chapter{Order}
\label{chp:order}
%=======================================
Equivalence relations \ifxref{found}{def:eq_rel}
require \prope{symmetry} ($x\eqcirc y\iff y\eqcirc x$).
However another very important type of relation,
the \rele{order relation}, actually requires \prope{anti-symmetry}.
This chapter presents some useful structures regarding order relations.
Ordering relations on a set allow us to \emph{compare} 
some pairs of elements in a set
and determine whether or not one element is \emph{less than} another.
In this case, we say that those two elements are \prope{comparable};
otherwise, they are \prope{incomparable}.
A set together with an order relation is called an \structe{ordered set},
a \structe{partially ordered set}, or a \structe{poset} \xref{def:order_rel}.

%=======================================
\section{Preordered sets}
%=======================================
%---------------------------------------
\begin{definition}
\footnote{
  \citerpg{schroder2003}{115}{0817641289},
  \citerp{brown1991}{317}
  }
\label{def:preorder}
%---------------------------------------
Let $\setX$ be a set.
\defbox{\begin{array}{M}
  A relation $\hxsd{\preorderrel}$ is a \reld{preorder relation} on $\setX$ if 
  \\$\begin{array}{@{\qquad}l l@{\qquad}C@{\qquad}DD}
    1. & x \preorderrel x
       & \forall x\in\setX
       & (\prope{reflexive})
       & and
       \\
    2. & x \preorderrel y \text{ and } y \preorderrel z \implies x \preorderrel z
       & \forall x,y,z\in\setX
       & (\prope{transitive})
  \end{array}$
  \\
  A \structd{preordered set} is the pair $\hxs{\opair{\setX}{\preorderrel}}$.
\end{array}}
\end{definition}

%---------------------------------------
\begin{example}
\footnote{
  \citerpg{shen2002}{43}{0821827316}
  }
%---------------------------------------
\exbox{\begin{array}{M}
  $\preorderrel$ is a \rele{preorder relation} on the set of \sete{positive integers} $\Zp$ if\\
  $n \preorderrel m 
   \qquad\iff\qquad
   \brp{ \text{$p$ is a prime factor of $n$}\implies \text{$p$ is a prime factor of $m$} }$
\end{array}}
\end{example}



%=======================================
\section{Order relations}
%=======================================

%---------------------------------------
\begin{definition}
\label{def:order_rel}
\label{def:orel}
\label{def:poset}
\label{def:order}
\footnote{
  %\citerpp{menini2004}{44}{45} \\
  \citerpg{maclane1999}{470}{0821816462},
  \citerpg{beran1985}{1}{902771715X},
  %\citerpg{halmos1960}{54}{0387900926}\\
  %\citorp{kuratowski1921}{164}\\
  \citePpc{korselt1894}{156}{I, II, (1)},
  \citePpc{dedekind1900}{373}{I--III}
  %\citorc{harriot1631}{\S 1}
  }
\index{order relation}
\index{ordered set}
%---------------------------------------
Let $\setX$ be a set. %\hid{partial order relation} 
Let $\clRxx$ be the set of all relations on $\setX$.
\defboxp{
  A relation $\hxsd{\orel}$ is an \reld{order relation} in $\clRxx$ if
  \\$\begin{array}{@{\qquad}Fl@{\qquad}C@{\qquad}DD@{}r@{}D}
    \cline{6-6}
    1. & x \orel x
       & \forall x\in\setX
       & (\prope{reflexive})
       & and \hspace{2ex}
       & \text{ }\vline
       & \text{\hspace{2ex}preorder}
       \\
    2. & x \orel y \text{ and } y \orel z \implies x \orel z
       & \forall x,y,z\in\setX
       & (\prope{transitive})
       & and
       & \vline
    \\\cline{6-6}
    3. & x \orel y \text{ and } y \orel x \implies x=y
       & \forall x,y\in\setX
       & (\prope{anti-symmetric})
       & 
       & 
  \end{array}$
  \\
  An \structd{ordered set} is the pair $\hxsd{\opair{\setX}{\orel}}$.
  The set $\setX$ is called the \structd{base set} of $\opair{\setX}{\orel}$.
  If $x\orel y$ or $y\orel x$, then elements $x$ and $y$ are said to be \propd{comparable}, denoted $x\sim y$. %$\hxsd{x\sim y}$. !!!problem indexing
  Otherwise they are \propd{incomparable}, denoted $x||y$. % !!! seems to cause indexing problem: $\hxsd{x||y}$.
  The relation $\hxsd{\lneq}$ is the relation $\orel\setd=$ (``less than but not equal to"), 
  where $\setd$ is the \ope{set difference} operator, and $=$ is the equality relation.
  An order relation is also called a \reld{partial order relation}.
  An ordered set is also called a \reld{partially ordered set} or \structd{poset}.
  }
\end{definition}

The familiar relations $\oreld$, $<$, and $>$ (next) 
can be defined in terms of the 
order relation $\orel$ (\pref{def:order_rel}---previous).
%---------------------------------------
\begin{definition}
\label{def:<}
\label{def:>}
\label{def:ge}
\label{def:oreld}
\footnote{
  \citePp{peirce1880ajm}{2}
  }
%---------------------------------------
Let $\opair{\setX}{\orel}$ be an ordered set.
\defbox{\begin{array}{M}
  The relations $\symxd{\oreld},\,\symxd{<},\,\symxd{>}\in\clRxx$ are defined as follows:
  \\\indentx$\begin{array}{lclMlC}
    x\hxsd{\oreld} y & \iffdef & y\orel x &     &         & \forall x,y\in\setX\\
    x\hxsd{\lneq} y       & \iffdef & x\orel  y & and & x\ne y  & \forall x,y\in\setX\\
    x\hxsd{\gneq} y       & \iffdef & x\oreld y & and & x\ne y  & \forall x,y\in\setX
  \end{array}$
  \\
  The relation $\oreld$ is called the \relxd{dual} of $\orel$.
\end{array}}
%\ifdochas{relation}{\footnotetext{
%  \relxe{inverse} relation: \prefp{def:rel_inverse}
%  }}
\end{definition}

%---------------------------------------
\begin{theorem}
\footnote{
  \citerp{gratzer1998}{3}
  }
%---------------------------------------
Let $\setX$ be a set.
\thmbox{
  \text{$\opair{\setX}{\orel}$ is an ordered set}
  \qquad\iff\qquad
  \text{$\opair{\setX}{\oreld}$ is an ordered set}
  }
\end{theorem}

%---------------------------------------
\begin{example}
%---------------------------------------
\exbox{
  \begin{array}{l>{$}l<{$} | l>{$}l<{$}}
     \mc{2}{c|}{$order relation$} & \mc{2}{c}{$dual order relation$}
  \\\hline
     \orel       & (integer less than or equal to) & \oreld & (integer greater than or equal to)
  \\ \subseteq & (subset)  & \supseteq   & (super set)
  \\ |         & (divides) &        & (divided by)
  \\ \implies  & (implies) & \impliedby  & (implied by)
  \end{array}
  }
\end{example}


%\begin{minipage}[c]{40mm}
%  \begin{center}%
%  %============================================================================
% Daniel J. Greenhoe
% LaTeX file
% lattice (2^{x,y,z}, subseteq)
% recommended unit = 10mm
%============================================================================
{\psset{unit=0.75\psunit}%
\begin{pspicture}(-2.4,-.3)(2.4,3.3)
  %---------------------------------
  % settings
  %---------------------------------
  \psset{%
    labelsep=1.5mm,
    }%
  %---------------------------------
  % nodes
  %---------------------------------
  \Cnode(0,3){t}
  \Cnode(-1,2){xy} \Cnode(0,2){xz} \Cnode(1,2){yz}
  \Cnode(-1,1){x}  \Cnode(0,1){y}  \Cnode(1,1){z}
  \Cnode(0,0){b}
  %---------------------------------
  % node connections
  %---------------------------------
  \ncline{t}{xy}\ncline{t}{xz}\ncline{t}{yz}
  \ncline{x}{xy}\ncline{x}{xz}
  \ncline{y}{xy}\ncline{y}{yz}
  \ncline{z}{xz}\ncline{z}{yz}
  \ncline{b}{x} \ncline{b}{y} \ncline{b}{z}
  %---------------------------------
  % node labels
  %---------------------------------
  \uput[180](t) {$\setn{x,y,z}$}%
  \uput[180](xy){$\setn{x,y}$}%   
 %\uput{1pt}[ 70](xz){$\setn{x,z}$} 
  \uput[0](yz){$\setn{y,z}$}%
  \uput[180](x) {$\setn{x}$}%     
 %\uput{1pt}[-45](y) {$\setn{y}$}   
  \uput[0](z) {$\setn{z}$}%
  \uput[180](b) {$\szero$}%
  \uput[0](1,3){\rnode{xzlabel}{$\setn{x,z}$}}% 
  \uput[0](1,  0){\rnode{ylabel}{$\setn{y}$}}%
  \ncline[linestyle=dotted,linecolor=red,nodesep=1pt]{->}{xzlabel}{xz}%
  \ncline[linestyle=dotted,linecolor=red,nodesep=1pt]{->}{ylabel}{y}%
\end{pspicture}
}%
%  \end{center}%
%\end{minipage}%
\begin{tabular}{c}%
  \includegraphics{../common/math/graphics/pdfs/lat8_2e3_setxyz.pdf}%
\end{tabular}
\hfill%
\begin{minipage}{\tw-85mm}\raggedright
%---------------------------------------
\begin{example}
%---------------------------------------
The Hasse diagram to the left illustrates the ordered set 
$\opair{\pset{\setn{x,y,z}}}{\subseteq}$
and the Hasse diagram to the right illustrates its dual
$\opair{\pset{\setn{x,y,z}}}{\supseteq}$.
\end{example}
\end{minipage}%
\hfill
\begin{tabular}{c}%
  \includegraphics{../common/math/graphics/pdfs/lat8_2e3_setxyz.pdf}%
\end{tabular}
%\begin{minipage}[c]{40mm}
%  \begin{center}%
%  %============================================================================
% Daniel J. Greenhoe
% LaTeX file
% lattice (2^{x,y,z}, subseteq)
%============================================================================
%\latmatl{4}{%
%                 & [name=1]\null                   \\
%  [name=zc]\null & [name=yc]\null & [name=xc]\null \\
%  [name=x]\null  & [name=y] \null & [name=z ]\null \\
%                 & [name=0]\null
%  }{%
%  \ncline{1}{zc}\ncline{1}{yc}\ncline{1}{xc}
%  \ncline{x}{zc}\ncline{x}{yc}
%  \ncline{y}{zc}\ncline{y}{xc}
%  \ncline{z}{yc}\ncline{z}{xc}
%  \ncline{0}{x} \ncline{0}{y} \ncline{0}{z}
%  }{%
%  \nput{ 90}{1} {$\setn{x,y,z}$}
%  \nput{135}{zc}{$\setn{x,y}$} \nput{ 45}{yc}{$\setn{x,z}$} \nput{0}{xc}{$\setn{y,z}$}
%  \nput{180}{x} {$\setn{x}$}   \nput{-45}{y} {$\setn{y}$}   \nput{0}{z}{$\setn{z}$}
%  \nput{-90}{0} {$\szero$}
%  }%
%\fbox{
{\psset{unit=0.075mm}
\begin{pspicture}(-270,-70)(270,370)
  %---------------------------------
  % settings
  %---------------------------------
  \psset{%
    labelsep=1.5mm,
    }%
  %---------------------------------
  % nodes
  %---------------------------------
  \Cnode(0,300){t}
  \Cnode(-100,200){xy} \Cnode(0,200){xz} \Cnode(100,200){yz}
  \Cnode(-100,100){x}  \Cnode(0,100){y}  \Cnode(100,100){z}
  \Cnode(0,0){b}
  %---------------------------------
  % node connections
  %---------------------------------
  \ncline{t}{xy}\ncline{t}{xz}\ncline{t}{yz}
  \ncline{x}{xy}\ncline{x}{xz}
  \ncline{y}{xy}\ncline{y}{yz}
  \ncline{z}{xz}\ncline{z}{yz}
  \ncline{b}{x} \ncline{b}{y} \ncline{b}{z}
  %---------------------------------
  % node labels
  %---------------------------------
  \uput[-90](b) {$\setn{x,y,z}$}%
  \uput[180](x){$\setn{x,y}$}%   
 %\uput{1pt}[ 70](xz){$\setn{x,z}$} 
  \uput[0](z){$\setn{y,z}$}%
  \uput[180](xy) {$\setn{x}$}%     
 %\uput{1pt}[-45](y) {$\setn{y}$}   
  \uput[0](yz) {$\setn{z}$}%
  \uput[90](t) {$\szero$}%
  \uput[0](100,  0){\rnode{xzlabel}{$\setn{x,z}$}}% 
  \uput[0](100,300){\rnode{ylabel}{$\setn{y}$}}%
  \ncline[linestyle=dotted,nodesep=1pt]{->}{xzlabel}{y}%
  \ncline[linestyle=dotted,nodesep=1pt]{->}{ylabel}{xz}%
\end{pspicture}
}%
%}%
%  \end{center}%
%\end{minipage}







%=======================================
%\section{Special sets}
%=======================================
%=======================================
\section{Linearly ordered sets}
%=======================================

In an ordered set we can say that some element
is less than or equal to some other element.
That is, we can say that these two elements are \prope{comparable}---we 
can \emph{compare} them to see which one is lesser or equal to
the other.
But it is very possible that there are two elements
that are not comparable, or \prope{incomparable}.
That is, we cannot say that one element is less than the other---it is 
simply not possible to compare them because their ordered pair is not 
an element of the order relation.

For example, in the ordered set $(\pset{\setn{x,y,z}},\subseteq$)
of \pref{ex:poset_xyz},
we can say that $\setn{x}\subseteq\setn{x,z}$ (we can compare these two sets
with respect to the order relation $\subseteq$),
but we cannot say $\setn{y}\subseteq\setn{x,z}$, nor can we say
$\setn{x,z}\subseteq\setn{y}$.
Rather, these two elements $\setn{y}$ and $\setn{x,z}$ are simply \prope{incomparable}.

However, there are some ordered sets 
in which every element is comparable with every other element;
and in this special case we say that this ordered set is a 
\prope{totally ordered} set or is \prope{linearly ordered} (next definition).

%---------------------------------------
\begin{definition}
\footnote{
  \citerpg{maclane1999}{470}{0821816462},
  \citePp{ore1935}{410}
  }
\label{def:toset}
\label{def:chain}
\index{ordered set!totally}
\index{ordered set!linearly}
%---------------------------------------
\defboxt{
  A relation $\hxsd{\orel}$ is a \reld{linear order relation} on $\setX$ if
  \\\indentx$\begin{array}{FlCDD}
    1. & \mc{2}{M}{$\orel$ is an \structe{order relation}}      & (\prefp{def:orel})    & and \\
    2. & x\orel y \text{ or } y\orel x  & \forall x,y\in\setX   & (\prope{comparable}).
  \end{array}$
  \\
  A \structd{linearly ordered set} is the pair $\hxsd{\opair{\setX}{\orel}}$.\\
  A linearly ordered set is also called a \structd{totally ordered set}, 
  a \structd{fully ordered set}, and a \structd{chain}.
  }
\end{definition}

%%---------------------------------------
%\begin{definition}
%\footnote{
%  \citerpg{maclane1999}{470}{0821816462}\\
%  \citePp{ore1935}{410}
%  }
%\label{def:toset}
%\label{def:chain}
%\index{ordered set!totally}
%\index{ordered set!linearly}
%%---------------------------------------
%\defbox{\begin{array}{M}
%  A relation $\orel$ is a \hid{linear order relation} on $\setX$ if
%  \\$\begin{array}{@{\qquad}Fl@{\qquad}C@{\qquad}DD@{}r@{}D}
%    \cline{6-6}
%    1. & x \orel x
%       & \forall x\in\setX
%       & (\prope{reflexive})
%       & and \hspace{1ex}
%       & \text{ }\vline
%       & 
%       \\
%    2. & x \orel y \text{ and } y \orel z \implies x \orel z
%       & \forall x,y,z\in\setX
%       & (\prope{transitive})
%       & and
%       & \vline
%       & \hspace{1ex}\text{order relation \xref{def:order}}
%       \\
%    3. & x \orel y \text{ and } y \orel x \implies x=y
%       & \forall x,y\in\setX
%       & (\prope{antisymmetric})
%       & and
%       & \vline
%       & 
%    \\\cline{6-6}
%    4. & x\orel y \text{ or } y\orel x 
%       & \forall x,y\in\setX
%       & (\prope{comparable}).
%       & 
%       & 
%  \end{array}$
%  \\
%  An \structd{linearly ordered set} is the pair $\opair{\setX}{\orel}$.\\
%  A linearly ordered set is also called a \structd{totally ordered set}, 
%  a \structd{fully ordered set}, and a \structd{chain}.
%\end{array}}
%\end{definition}


%=======================================
%\subsection{Products}
%=======================================
%---------------------------------------
\begin{definition}[poset product]
\footnote{
  \citerpg{birkhoff1948}{7}{3540120440},
  \citerp{maclane1967}{489}
  }
\label{def:order_product}
\index{poset product}
\index{product!poset}
%---------------------------------------
\defboxp{
  The \opd{product} $P\hxsd{\cprod} Q$ of ordered pairs $P\eqd\opair{\setX}{\orela}$ and
  $Q\eqd\opair{\setY}{\orelb}$ is the ordered pair
  $\opair{\cprodXY}{\orel}$ where
  \\\indentx
  $\opair{x_1}{y_1}\orel\opair{x_2}{y_2} \qquad\iffdef\qquad  x_1\orela x_2 \text{ and } y_1\orelb y_2
  \qquad
  {\scriptstyle \forall x_1,x_2\in\setX;\; y_1,y_2\in\setY}$
  }
\end{definition}


%=======================================
\section{Representation}
%=======================================
%---------------------------------------
\begin{definition}
\footnote{
  \citePp{birkhoff1933}{445}
  }
\label{def:cover}
%---------------------------------------
\defbox{\begin{array}{M}\indxs{\coverrel}
  $y$ \propd{covers} $x$ in the ordered set $\opair{\setX}{\orel}$ if
  \\\indentx$\begin{array}{@{}ll@{\qquad}D D}
    1. & x\orel y  
       & ($y$ is greater than $x$)
       & and
       \\
    2. & \brp{x \orel z \orel y} \quad\implies\quad \brp{z=x \text{ or } z=y}
       & (there is no element between $x$ and $y$).
  \end{array}$
  \\
  The case in which $y$ covers $x$ is denoted
  \\\indentx%
    $x\hxsd{\coverrel} y$.
\end{array}}
\end{definition}


%---------------------------------------
\begin{example}
%---------------------------------------
Let $\opair{\setn{x,y,z}}{\le}$ be an ordered set with cover relation $\coverrel$.
\exbox{ 
  \brb{x < y < z }
  \qquad\implies\qquad 
  \brb{\begin{array}{rMl}
    y & covers & x \\
    z & covers & y \\
    z & does {\bf not} cover & x
  \end{array}}
  }
\end{example}



An ordered set can be represented in four ways:
\begin{enume}
  \item \hi{Hasse diagram}
  \item tables
  \item set of ordered pairs of order relations
  \item set of ordered pairs of cover relations
\end{enume}

%---------------------------------------
\begin{definition}
\label{def:hasse}
%---------------------------------------
Let $\opair{\setX}{\orel}$ be an ordered pair.
\defboxp{
  A diagram is a \structd{Hasse diagram} of $\opair{\setX}{\orel}$ if it satisfies the following criteria:
  \begin{liste}
    \item Each element in $\setX$ is represented by a dot or small circle.
    \item For each $x,y\in\setX$, if $x\coverrel y$, then $y$ appears at a higher position than $x$ 
          and a line connects $x$ and $y$.
  \end{liste}
  }
\end{definition}

%---------------------------------------
\begin{example}
\label{ex:order_oset_rep}
%---------------------------------------
Here are three ways of representing the ordered set $\opair{\pset{\setn{x,y}}}{\subseteq}$;
\begin{dingautolist}{"AC}
  %\item \begin{tabular}{m{\tw-38mm}m{23mm}}%
  %        \hib{Hasse diagrams}: If two elements are comparable, then the lesser of the two is drawn lower
  %        on the page than the other with a line connecting them.%
  %        &\centering\psset{unit=5mm}%============================================================================
% Daniel J. Greenhoe
% LaTeX file
% lattice M2
%============================================================================
\begin{pspicture}(-1.5,-0.5)(1.5,2.5)%
  %---------------------------------
  % settings
  %---------------------------------
  \psset{%
    labelsep=1.5mm,%
    }%
  %---------------------------------
  % nodes
  %---------------------------------
  \Cnode(0,2){t}%
  \Cnode(-1,1){x}\Cnode(1,1){y}%
  \Cnode(0,0){b}%
  %---------------------------------
  % node connections
  %---------------------------------
  \ncline{t}{x}\ncline{t}{y}%
  \ncline{b}{x}\ncline{b}{y}%
  %---------------------------------
  % node labels
  %---------------------------------
  \uput[ 90](t) {$\bid$}%
  \uput[180](x) {$x$}%     
  \uput[0](y){$y$}%   
  \uput[-90](b) {$\bzero$}%
  %\ncline[linestyle=dotted,nodesep=1pt]{->}{xzlabel}{xz}%
  %\ncline[linestyle=dotted,nodesep=1pt]{->}{ylabel}{y}%
\end{pspicture}%%
  %      \end{tabular}
  \item \begin{tabular}{m{\tw-38mm}}%
          \hib{Hasse diagrams}: If two elements are comparable, then the lesser of the two is drawn lower
          on the page than the other with a line connecting them.%
          %&\centering\psset{unit=5mm}%============================================================================
% Daniel J. Greenhoe
% LaTeX file
% lattice M2
%============================================================================
\begin{pspicture}(-1.5,-0.5)(1.5,2.5)%
  %---------------------------------
  % settings
  %---------------------------------
  \psset{%
    labelsep=1.5mm,%
    }%
  %---------------------------------
  % nodes
  %---------------------------------
  \Cnode(0,2){t}%
  \Cnode(-1,1){x}\Cnode(1,1){y}%
  \Cnode(0,0){b}%
  %---------------------------------
  % node connections
  %---------------------------------
  \ncline{t}{x}\ncline{t}{y}%
  \ncline{b}{x}\ncline{b}{y}%
  %---------------------------------
  % node labels
  %---------------------------------
  \uput[ 90](t) {$\bid$}%
  \uput[180](x) {$x$}%     
  \uput[0](y){$y$}%   
  \uput[-90](b) {$\bzero$}%
  %\ncline[linestyle=dotted,nodesep=1pt]{->}{xzlabel}{xz}%
  %\ncline[linestyle=dotted,nodesep=1pt]{->}{ylabel}{y}%
\end{pspicture}%%
        \end{tabular}
        \hfill\begin{tabular}{c}%
          \includegraphics{../common/math/graphics/pdfs/lat4_m2_1xy0.pdf}%
        \end{tabular}%

  \item Sets of ordered pairs specifying \rele{order relation}s \xref{def:orel}:
    \[ \subseteq 
         = \left\{\begin{array}{cccc}
             \opair{\emptyset}{\emptyset},      & \opair{\setn{x}}{\setn{x}}, & \opair{\setn{y}}{\setn{y}},   & \opair{\setn{x,y}}{\setn{x,y}}, \\
                    \opair{\emptyset}{\setn{x}},& \opair{\emptyset}{\setn{y}},& \opair{\emptyset}{\setn{x,y}},& \opair{\setn{x}}{\setn{x,y}},\, \opair{\setn{y}}{\setn{x,y}} 
           \end{array}\right\}
    \]
  \item Sets of ordered pairs specifying \rele{covering relation}s:
    \[ \coverrel
         = \left\{\begin{array}{cccc}
                  \opair{\emptyset}{\setn{x}},& \opair{\emptyset}{\setn{y}},& \opair{\setn{x}}{\setn{x,y}},\, \opair{\setn{y}}{\setn{x,y}} 
           \end{array}\right\}
    \]
\end{dingautolist}
\end{example}




%\begin{minipage}[c]{3\tw/16}
%  \begin{center}%
%  \latmat{4}{
%          &       & \null         \\
%          & \null &       & \null \\
%    \null &       & \null         \\
%          & \null                 
%    }{
%    \ncline{1,3}{2,2}\ncline{1,3}{2,4}
%    \ncline{2,2}{3,1}\ncline{2,2}{3,3}
%    \ncline{4,2}{3,1}\ncline{4,2}{3,3}
%    \ncline{3,3}{2,4}
%    }
%  \end{center}%
%\end{minipage}%
\begin{tabular}{c}\includegraphics{../common/math/graphics/pdfs/lat6_m2m2.pdf}\end{tabular}\hfill%
\begin{minipage}{\tw-50mm}\raggedright
%---------------------------------------
\begin{example}
%---------------------------------------
The Hasse diagrams to the left and right represent 
\emph{equivalent} ordered sets.
They are simply drawn differently.
\end{example}
\end{minipage}%
\hfill\begin{tabular}{c}\includegraphics{../common/math/graphics/pdfs/lat6_o6bslash.pdf}\end{tabular}%

\begin{tabular}{c}\includegraphics{../common/math/graphics/pdfs/lat7_m2inm2_alt.pdf}\end{tabular}\hfill%
\begin{minipage}{\tw-50mm}\raggedright
%---------------------------------------
\begin{example}
%---------------------------------------
The Hasse diagrams to the left and right represent 
\emph{equivalent} ordered sets.
They are simply drawn differently.
\end{example}
\end{minipage}%
\hfill\begin{tabular}{c}\includegraphics{../common/math/graphics/pdfs/lat7_m2inm2.pdf}\end{tabular}

%---------------------------------------
\begin{example}
%---------------------------------------
The Hasse diagrams to the left and right represent 
\emph{equivalent} ordered sets.\\
\begin{tabular}{c}\includegraphics{../common/math/graphics/pdfs/lat5_pentbslash_1xyz0.pdf}\end{tabular}\hfill%
\begin{minipage}{\tw-50mm}\raggedright
In particular, the line extending from $1$ to $y$ in the diagram to the left is
redundant because other lines already indicate that
$z\orel 1$ and $y\orel z$; 
and thus by the \prope{transitive} property \xref{def:orel},
these two relations imply $1\orel y$.
A more concise explanation is that both have the same convering relation:
\\\indentx$\coverrel = \setn{\opair{z}{1},\, \opair{x}{z},\, \opair{\bzero}{x},\, \opair{y}{z},\, \opair{\bzero}{y}}$
\end{minipage}%
\hfill\begin{tabular}{c}\includegraphics{../common/math/graphics/pdfs/lat5_l2onm2_1xyz0.pdf}\end{tabular}%
\end{example}

%=======================================
\section{Examples}
%=======================================
Examples of order relations include the following:\\
\begin{longtable}{@{\qquad}>{$\imark$ }lll}
    set inclusion order relation:       & \pref{ex:poset_xyz}            & \prefpo{ex:poset_xyz} 
  \\integer divides order relation:     & \pref{ex:poset_532}            & \prefpo{ex:poset_532}  
  \\linear operator order relation:     & \pref{ex:order_operator}       & \prefpo{ex:order_operator} 
  \\projection operator order relation: & \pref{ex:order_projection_op}  & \prefpo{ex:order_projection_op} 
 %\\logical implication order relation  & \pref{ex:order_implies}        & \prefpo{ex:order_implies}
  \\integer order relation:             & \pref{ex:order_int}            & \prefpo{ex:order_int} 
  \\metric order relation:              & \pref{ex:order_metric}         & \prefpo{ex:order_metric} 
  \\coordinatewise order relation       & \pref{ex:order_coordinatewise} & \prefpo{ex:order_coordinatewise}
  \\lexicographical order relation      & \pref{ex:order_lex}            & \prefpo{ex:order_lex}
  %\\pointwise order relation            & \pref{ex:order_function}       & \prefpo{ex:order_function}
\end{longtable}






\begin{minipage}{\tw-43mm}\raggedright%
%---------------------------------------
\begin{example}[Set inclusion order relation]
\label{ex:poset_xyz}
\footnotemark
%---------------------------------------
Let $\setX$ be a set, $\psetx$ the power set of $\setX$,
and $\subseteq$ the set inclusion relation.
Then, $\subseteq$ is an \rele{order relation} on the set $\psetx$
and the pair $\opair{\psetx}{\subseteq}$ is an ordered set.
The ordered set $\opair{\pset{\setn{x,y,z}}}{\subseteq}$ is illustrated 
to the right by its \structe{Hasse diagram}.
\end{example}%
\end{minipage}%
\footnotetext{\citerppg{menini2004}{56}{57}{0824709853}}%
\hfill\begin{tabular}{c}\includegraphics{../common/math/graphics/pdfs/lat8_2e3_setxyz.pdf}\end{tabular}\hfill\mbox{}\\%
%\begin{minipage}[c]{40mm}%
%  \begin{center}%
%  %============================================================================
% Daniel J. Greenhoe
% LaTeX file
% lattice (2^{x,y,z}, subseteq)
% recommended unit = 10mm
%============================================================================
{\psset{unit=0.75\psunit}%
\begin{pspicture}(-2.4,-.3)(2.4,3.3)
  %---------------------------------
  % settings
  %---------------------------------
  \psset{%
    labelsep=1.5mm,
    }%
  %---------------------------------
  % nodes
  %---------------------------------
  \Cnode(0,3){t}
  \Cnode(-1,2){xy} \Cnode(0,2){xz} \Cnode(1,2){yz}
  \Cnode(-1,1){x}  \Cnode(0,1){y}  \Cnode(1,1){z}
  \Cnode(0,0){b}
  %---------------------------------
  % node connections
  %---------------------------------
  \ncline{t}{xy}\ncline{t}{xz}\ncline{t}{yz}
  \ncline{x}{xy}\ncline{x}{xz}
  \ncline{y}{xy}\ncline{y}{yz}
  \ncline{z}{xz}\ncline{z}{yz}
  \ncline{b}{x} \ncline{b}{y} \ncline{b}{z}
  %---------------------------------
  % node labels
  %---------------------------------
  \uput[180](t) {$\setn{x,y,z}$}%
  \uput[180](xy){$\setn{x,y}$}%   
 %\uput{1pt}[ 70](xz){$\setn{x,z}$} 
  \uput[0](yz){$\setn{y,z}$}%
  \uput[180](x) {$\setn{x}$}%     
 %\uput{1pt}[-45](y) {$\setn{y}$}   
  \uput[0](z) {$\setn{z}$}%
  \uput[180](b) {$\szero$}%
  \uput[0](1,3){\rnode{xzlabel}{$\setn{x,z}$}}% 
  \uput[0](1,  0){\rnode{ylabel}{$\setn{y}$}}%
  \ncline[linestyle=dotted,linecolor=red,nodesep=1pt]{->}{xzlabel}{xz}%
  \ncline[linestyle=dotted,linecolor=red,nodesep=1pt]{->}{ylabel}{y}%
\end{pspicture}
}%%
%  \end{center}%
%\end{minipage}%

\begin{minipage}{\tw-43mm}%
%---------------------------------------
\begin{example}[Integer divides order relation]%
\label{ex:poset_532}
\footnotemark
%---------------------------------------
Let $|$ be the ``divides" relation on the set $\Zp$ of positive integers such that
$n|m$ represents $m$ divides $n$.
Then $|$ is an \rele{order relation} on $\Zp$
and the pair $\opair{\Zp}{|}$ is an \sete{ordered set}.
The ordered set
$\opair{\set{n\in\Zp}{n|2 \text{ or } n|3 \text{ or } n|5}}{|}$
is illustrated by a \structe{Hasse diagram} to the right.
\end{example}%
\end{minipage}%
\footnotetext{
  \citerpg{maclane1999}{484}{0821816462},
  %\citerpg{menini2004}{60}{0824709853}\\
  %\citerp{huntington1933}{278}\\
  \citePpc{sheffer1920}{310}{footnote 1}
  }%
\hfill\begin{tabular}{c}\includegraphics{../common/math/graphics/pdfs/lat8_2e3_set235.pdf}\end{tabular}\hfill\mbox{}\\%

%---------------------------------------
\begin{example}[Operator order relation]
\footnote{
  \citerpg{michel1993}{429}{048667598X},
  \citerpg{pedersen2000}{87}{0849371694}
  }
\label{ex:order_operator}
%---------------------------------------
Let $\spX$ be an inner-product space.
We can define the order relation $\orela$ on the linear operators 
$\opL_1,\opL_2,\opL_3\ldots\in\clFxx$ as follows:
\exbox{ \opL_1 \orela \opL_2 \qquad\iffdef\qquad \inprod{\opL_2\vx-\opL_1\vx}{\vx} \ge 0
        \qquad\forall \vx\in\spX}
\end{example}




\begin{minipage}[c]{\tw-43mm}%
  %---------------------------------------
  \begin{example}[Projection operator order relation]
  \label{ex:order_projection_op}
  \footnotemark
  %---------------------------------------
  Let $\seqn{\spV_n}$ be a sequence of subspaces in a Hilbert space $\spX$.
  We can define a projection operator $\opP_n$ for every subspace $\spV_n\subseteq\spX$ 
  in a \prop{subspace lattice} such that 
  \\\indentx
    $\ds \spV_n = \opP_n\spX \qquad \forall n\in\Z.$
  \\
  Each projection operator $\opP_n$ in the lattice ``projects" the range space $\spX$
  onto a subspace $\spV_n$.
  We can define an order relation on the projection operators as follows:
  \exbox{ \opP_1 \orel \opP_2 \qquad\iffdef\qquad \opP_1\opP_2=\opP_2\opP_1=\opP_1}
  \end{example}
\end{minipage}%
\footnotetext{%
  \citerppg{isham1999}{21}{22}{9810235623},
  \citerp{dunford1957}{481},
  \citerpg{svozil}{72}{981020809X}
  }%
\hfill\begin{tabular}{c}\includegraphics{../common/math/graphics/pdfs/lat6_plat_P12345.pdf}\end{tabular}\hfill\mbox{}\\%

\begin{minipage}{\tw-20mm}%
  %---------------------------------------
  \begin{example}[Integer order relation]
  \label{ex:order_int}
  %\citepp{menini2004}{56}{57}
  %---------------------------------------
  Let $\orel$ be the standard order relation on the set of integers $\Z$.
  Then the ordered pair $\opair{\Z}{\orel}$ is a totally ordered set.
  The totally ordered set $\opair{\setn{1,2,3,4}}{\orel}$ is illustrated to the 
  right.
  Other familiar examples of totally ordered sets include the 
  pair $\opair{\Q}{\orel}$ (where $\Q$ is the set of rational numbers)
  and  $\opair{\R}{\orel}$ (where $\R$ is the set of real numbers).
  \end{example}
\end{minipage}%
\hfill\begin{tabular}{c}\includegraphics{../common/math/graphics/pdfs/lat4_l4_1234.pdf}\end{tabular}\hfill\mbox{}\\%

%---------------------------------------
% Metrics: l_1, l_2, l_infty, sqrtabs
%---------------------------------------
\begin{figure}[ht] \color{figcolor}
\begin{center}
\begin{fsL}
\setlength{\unitlength}{\tw/1000}
\begin{picture}(500,300)(-130,-130)%
  %{\color{graphpaper}\graphpaper[10](-150,-150)(300,300)}%
  \thicklines%
  \color{axis}%
    \put(-130,   0){\line(1,0){260} }%
    \put(   0,-130){\line(0,1){260} }%
    \put( 140,   0){\makebox(0,0)[l]{$x$}}%
    \put(   0, 140){\makebox(0,0)[b]{$y$}}%
    \put(-100, -10){\line(0,1){20} }%
    \put( 100, -10){\line(0,1){20} }%
    \put( -10,-100){\line(1,0){20} }%
    \put( -10, 100){\line(1,0){20} }%
    \put(  10, 110){\makebox(0,0)[bl]{$1$} }%
    \put(  10,-110){\makebox(0,0)[tl]{$1$} }%
    \put(-110,  10){\makebox(0,0)[br]{$1$} }%
    \put( 110,  10){\makebox(0,0)[bl]{$1$} }%
  \color{red}%
    \put(-100,   0){\line( 1, 1){100} }%
    \put(-100,   0){\line( 1,-1){100} }%
    \put( 100,   0){\line(-1, 1){100} }%
    \put( 100,   0){\line(-1,-1){100} }%
    \put(150,80){\makebox(0,0)[l]{unit ball in sup metric space}}%
    \put(140,80){\vector(-1, 0){40}}%
  \color{blue}%
    %============================================================================
% NCTU - Hsinchu, Taiwan
% LaTeX File
% Daniel Greenhoe
%
% Unit circle with radius 100
%============================================================================

\qbezier( 100,   0)( 100, 41.421356)(+70.710678,+70.710678) % 0   -->1pi/4
\qbezier(   0, 100)( 41.421356, 100)(+70.710678,+70.710678) % pi/4-->2pi/4
\qbezier(   0, 100)(-41.421356, 100)(-70.710678,+70.710678) %2pi/4-->3pi/4
\qbezier(-100,   0)(-100, 41.421356)(-70.710678,+70.710678) %3pi/4--> pi 
\qbezier(-100,   0)(-100,-41.421356)(-70.710678,-70.710678) % pi  -->5pi/4
\qbezier(   0,-100)(-41.421356,-100)(-70.710678,-70.710678) %5pi/4-->6pi/4
\qbezier(   0,-100)( 41.421356,-100)( 70.710678,-70.710678) %6pi/4-->7pi/4
\qbezier( 100,   0)( 100,-41.421356)( 70.710678,-70.710678) %7pi/4-->2pi


%
    \put(150,60){\makebox(0,0)[l]{unit ball in Euclidean metric space}}%
    \put(140,60){\vector(-1, 0){55}}%
  \color{red}%
    \put(-100,-100){\line( 1, 0){200} }%
    \put(-100,-100){\line( 0, 1){200} }%
    \put( 100, 100){\line(-1, 0){200} }%
    \put( 100, 100){\line( 0,-1){200} }%
    \put(150, 40){\makebox(0,0)[l]{unit ball in taxi-cab metric space}}%
    \put(140, 40){\vector(-1, 0){80}}%
  \color{blue}%
    \qbezier( 100,0)(0,0)(0, 100)%
    \qbezier( 100,0)(0,0)(0,-100)%
    \qbezier(-100,0)(0,0)(0,-100)%
    \qbezier(-100,0)(0,0)(0, 100)%
    \put(150,-40){\makebox(0,0)[l]{unit ball in square-root metric space}}%
    \put(140,-40){\vector(-1, 0){125}}%
  \color{red}%
    \put(0,0){\latdot}%
    \put(-20,120){\makebox(0,0)[r]{unit ball in discrete metric space}}%
    \put(-60,110){\vector(1,-2){48}}%
\end{picture}
\end{fsL}
\end{center}
\caption{
   Balls on the set $\R^2$ using different metrics
   \label{fig:lat_metric_balls}
   }
\end{figure}

\begin{minipage}[c]{3\tw/4}
%---------------------------------------
\begin{example}[Metric order relation]
\label{ex:order_metric}
\index{order!metric}
\footnotemark
%---------------------------------------
Let $\fd_n$ be a metric on the set $\setX$
and $\balln_n$ be the unit ball centered at ``$0$" in the metric space $\opair{\setX}{\fd_n}$.
Define an order relation $\orel$ on the set of metric spaces
$\set{\opair{\setX}{\fd_n}}{n=1,2,\ldots}$
such that
  \[ \opair{\setX}{\fd_n} \orel \opair{\setX}{\fd_m} \qquad\iff\qquad \balln_n \subseteq \balln_m. \]
The the tuple $(\set{\opair{\setX}{\fd_n}}{n=1,2,\ldots},\, \orel)$
is an ordered set.
The ordered set of several common metric spaces is a \prope{totally ordered} set,
as illustrated to the right and with associated unit balls illustrated in 
\prefpp{fig:lat_metric_balls}.
\end{example}
\end{minipage}%
\footnotetext{%
  \citerpg{michel1993}{354}{048667598X},
  \citerpg{giles1987}{29}{0521359287}
  }%
\begin{minipage}[c]{\tw/4}%
%\begin{figure}[th]
\begin{center}
\footnotesize
\setlength{\unitlength}{\tw/530}%
\begin{picture}(530,850)(-70,0)%
  \thicklines
  %{\color{graphpaper}\graphpaper[50](-50,0)(100,520)}%
  \color{picbox}%
    \put( -50,750){\framebox(100,100){\color{blue}$\spX$}}%
    \put( -50,600){\framebox(100,100){}}%
    \put( -50,450){\framebox(100,100){}}%
    \put( -50,300){\framebox(100,100){}}%
    \put( -50,150){\framebox(100,100){}}%
    \put( -50,  0){\framebox(100,100){}}%
    %
  \color{black}%
    \put(   0,700){\line( 0, 1){ 50}}%
    \put(   0,550){\line( 0, 1){ 50}}%
    \put(   0,400){\line( 0, 1){ 50}}%
    \put(   0,250){\line( 0, 1){ 50}}%
    \put(   0,100){\line( 0, 1){ 50}}%
    %
  \color{blue}%
    \put( 110, 800){\makebox(0,0)[l]{bounded metric}}%
    \put( 110, 650){\makebox(0,0)[l]{sup metric}}%
    \put( 110, 500){\makebox(0,0)[l]{Euclidean metric}}%
    \put( 110, 350){\makebox(0,0)[l]{taxi-cab metric}}%
    \put( 110, 200){\makebox(0,0)[l]{parabolic metric}}%
    \put( 110,  50){\makebox(0,0)[l]{discrete metric}}%
    %
  \put(0,650){%
    \setlength{\unitlength}{1\tw/(400*3)}%
    \begin{picture}(0,0)(0,0)
      \thicklines
      \color{axis}%
        \put(-130,   0){\line(1,0){260} }%
        \put(   0,-130){\line(0,1){260} }%
        %\put( 140,   0){\makebox(0,0)[l]{$x$}}%
        %\put(   0, 140){\makebox(0,0)[b]{$y$}}%
        \put(-100, -10){\line(0,1){20} }%
        \put( 100, -10){\line(0,1){20} }%
        \put( -10,-100){\line(1,0){20} }%
        \put( -10, 100){\line(1,0){20} }%
        %\put(  10, 110){\makebox(0,0)[bl]{$1$} }%
        %\put(  10,-110){\makebox(0,0)[tl]{$-1$} }%
        %\put(-110,  10){\makebox(0,0)[br]{$-1$} }%
        %\put( 110,  10){\makebox(0,0)[bl]{$1$} }%
      \color{blue}%
        \put(-100,-100){\line( 1, 0){200} }%
        \put(-100,-100){\line( 0, 1){200} }%
        \put( 100, 100){\line(-1, 0){200} }%
        \put( 100, 100){\line( 0,-1){200} }%
    \end{picture}
  }
    \put(0,500){%
      \setlength{\unitlength}{1\tw/(400*3)}%
      \begin{picture}(0,0)(0,0)
        \thicklines
        \color{axis}%
          \put(-130,   0){\line(1,0){260} }%
          \put(   0,-130){\line(0,1){260} }%
          %\put( 140,   0){\makebox(0,0)[l]{$x$}}%
          %\put(   0, 140){\makebox(0,0)[b]{$y$}}%
          \put(-100, -10){\line(0,1){20} }%
          \put( 100, -10){\line(0,1){20} }%
          \put( -10,-100){\line(1,0){20} }%
          \put( -10, 100){\line(1,0){20} }%
          %\put(  10, 110){\makebox(0,0)[bl]{$1$} }%
          %\put(  10,-110){\makebox(0,0)[tl]{$-1$} }%
          %\put(-110,  10){\makebox(0,0)[br]{$-1$} }%
          %\put( 110,  10){\makebox(0,0)[bl]{$1$} }%
        \color{blue}%
          %============================================================================
% NCTU - Hsinchu, Taiwan
% LaTeX File
% Daniel Greenhoe
%
% Unit circle with radius 100
%============================================================================

\qbezier( 100,   0)( 100, 41.421356)(+70.710678,+70.710678) % 0   -->1pi/4
\qbezier(   0, 100)( 41.421356, 100)(+70.710678,+70.710678) % pi/4-->2pi/4
\qbezier(   0, 100)(-41.421356, 100)(-70.710678,+70.710678) %2pi/4-->3pi/4
\qbezier(-100,   0)(-100, 41.421356)(-70.710678,+70.710678) %3pi/4--> pi 
\qbezier(-100,   0)(-100,-41.421356)(-70.710678,-70.710678) % pi  -->5pi/4
\qbezier(   0,-100)(-41.421356,-100)(-70.710678,-70.710678) %5pi/4-->6pi/4
\qbezier(   0,-100)( 41.421356,-100)( 70.710678,-70.710678) %6pi/4-->7pi/4
\qbezier( 100,   0)( 100,-41.421356)( 70.710678,-70.710678) %7pi/4-->2pi


%
      \end{picture}
    }
  \put(0,350){%
    \setlength{\unitlength}{1\tw/(400*3)}%
    \begin{picture}(0,0)(0,0)%
      %{\color{graphpaper}\graphpaper[10](-150,-150)(300,300)}%
      \thicklines%
      \color{axis}%
        \put(-130,   0){\line(1,0){260} }%
        \put(   0,-130){\line(0,1){260} }%
        %\put( 140,   0){\makebox(0,0)[l]{$x$}}%
        %\put(   0, 140){\makebox(0,0)[b]{$y$}}%
        \put(-100, -10){\line(0,1){20} }%
        \put( 100, -10){\line(0,1){20} }%
        \put( -10,-100){\line(1,0){20} }%
        \put( -10, 100){\line(1,0){20} }%
        %\put(  10, 110){\makebox(0,0)[bl]{$1$} }%
        %\put(  10,-110){\makebox(0,0)[tl]{$-1$} }%
        %\put(-110,  10){\makebox(0,0)[br]{$-1$} }%
        %\put( 110,  10){\makebox(0,0)[bl]{$1$} }%
      \color{blue}%
        \qbezier( 100,0)(50,50)(0, 100)%
        \qbezier( 100,0)(50,-50)(0,-100)%
        \qbezier(-100,0)(-50,-50)(0,-100)%
        \qbezier(-100,0)(-50,50)(0, 100)%
    \end{picture}
  }
  \put(0,200){%
    \setlength{\unitlength}{1\tw/(400*3)}%
    \begin{picture}(0,0)(0,0)%
      %{\color{graphpaper}\graphpaper[10](-150,-150)(300,300)}%
      \thicklines%
      \color{axis}%
        \put(-130,   0){\line(1,0){260} }%
        \put(   0,-130){\line(0,1){260} }%
        %\put( 140,   0){\makebox(0,0)[l]{$x$}}%
        %\put(   0, 140){\makebox(0,0)[b]{$y$}}%
        \put(-100, -10){\line(0,1){20} }%
        \put( 100, -10){\line(0,1){20} }%
        \put( -10,-100){\line(1,0){20} }%
        \put( -10, 100){\line(1,0){20} }%
        %\put(  10, 110){\makebox(0,0)[bl]{$1$} }%
        %\put(  10,-110){\makebox(0,0)[tl]{$-1$} }%
        %\put(-110,  10){\makebox(0,0)[br]{$-1$} }%
        %\put( 110,  10){\makebox(0,0)[bl]{$1$} }%
      \color{blue}%
        \qbezier( 100,0)(0,0)(0, 100)%
        \qbezier( 100,0)(0,0)(0,-100)%
        \qbezier(-100,0)(0,0)(0,-100)%
        \qbezier(-100,0)(0,0)(0, 100)%
    \end{picture}
  }
  \put(0,50){%
    \setlength{\unitlength}{1\tw/(400*3)}%
    \begin{picture}(0,0)(0,0)%
      %{\color{graphpaper}\graphpaper[10](-150,-150)(300,300)}%
      \thicklines%
      \color{axis}%
        \put(-130,   0){\line(1,0){260} }%
        \put(   0,-130){\line(0,1){260} }%
        %\put( 140,   0){\makebox(0,0)[l]{$x$}}%
        %\put(   0, 140){\makebox(0,0)[b]{$y$}}%
        \put(-100, -10){\line(0,1){20} }%
        \put( 100, -10){\line(0,1){20} }%
        \put( -10,-100){\line(1,0){20} }%
        \put( -10, 100){\line(1,0){20} }%
        %\put(  10, 110){\makebox(0,0)[bl]{$1$} }%
        %\put(  10,-110){\makebox(0,0)[tl]{$-1$} }%
        %\put(-110,  10){\makebox(0,0)[br]{$-1$} }%
        %\put( 110,  10){\makebox(0,0)[bl]{$1$} }%
      \color{blue}%
        \put(0, 0){\latdot}%
    \end{picture}
  }
\end{picture}
\end{center}
\end{minipage}




%---------------------------------------
\begin{example}[\exmd{Coordinatewise order relation}]
\footnote{
  \citerpg{shen2002}{43}{0821827316}
  }
\label{ex:order_coordinatewise}
\index{order relations!coordinatewise}
%---------------------------------------
Let $\opair{\setX}{\orel}$ be an ordered set.\\
Let $\vx\eqd\oquad{x_1}{x_2}{\ldots}{x_n}$ and $\vy\eqd\oquad{y_1}{y_2}{\ldots}{y_n}$.
\exbox{\begin{array}{M}
  The \reld{coordinatewise order relation} $\orela$ on the Cartesian product $\setX^n$
  \\is defined for all $\vx,\vy\in\setX^n$ as
  \\\indentx
  $\vx\orela\vy
  \qquad\iffdef\qquad
  \brb{ x_1\orel y_1 \text{ and } x_2\orel y_2 \text{ and } \ldots \text{ and } x_n\orel y_n }$
\end{array}}
\end{example}

%---------------------------------------
\begin{example}[\exmd{Lexicographical order relation}]
\footnote{
  \citerpg{shen2002}{44}{0821827316},
  \citerpg{halmos1960}{58}{0387900926},
  \citerpg{hausdorff1937e}{54}{0828401195}
  }
\label{ex:order_lex}
\index{order relations!lexicographical}
\index{order relations!dictionary}
\index{order relations!alphabetic}
%---------------------------------------
Let $\opair{\setX}{\orel}$ be an ordered set.\\
Let $\vx\eqd\oquad{x_1}{x_2}{\ldots}{x_n}$ and $\vy\eqd\oquad{y_1}{y_2}{\ldots}{y_n}$.
\exbox{\begin{array}{M}
  The \reld{lexicographical order relation} $\orela$ on the Cartesian product $\setX^n$
  \\is defined forall $\vx,\vy\in\setX^n$ as
  \\
  $\vx\orela\vy
  \iffdef
  \left\{\begin{array}{>{\big(}l l >{$}l<{$} l l<{\big)} >{$}l<{$} }
      & x_1 < y_1 &     &                                     & & or
    \\& x_2 < y_2 & and & x_1 = y_1                           & & or
    \\& x_3 < y_3 & and & \opair{x_1}{x_2} = \opair{y_1}{y_2} & & or
    \\& \ldots        & \ldots & \ldots                           & & or 
    \\& x_{n-1} < y_{n-1} & and & \oquad{x_1}{x_2}{\ldots}{x_{n-2}} = \oquad{y_1}{y_2}{\ldots}{y_{n-2}} & & or
    \\& x_{n} \leq y_{n} & and & \oquad{x_1}{x_2}{\ldots}{x_{n-1}} = \oquad{y_1}{y_2}{\ldots}{y_{n-1}} & &
  \end{array}\right\}$
  \\The lexicographical order relation is also called the \reld{dictionary order relation} 
  \\or \reld{alphabetic order relation}.
\end{array}}
\end{example}



%---------------------------------------
\begin{definition}
%---------------------------------------
\defbox{\begin{array}{M}
  An ordered set is \propd{labeled} if the labels on the elements are significant.\\
  An ordered set is \propd{unlabeled} if the labels on the elements are not significant.
\end{array}}
\end{definition}



%---------------------------------------
\begin{proposition}
\footnote{
  \citeoeis{A001035},
  \citeoeis{A000112},
  \citerpg{comtet1974}{60}{9027704414},
  \citer{brinkmann2002}
  }
\label{prop:num_posets}
\index{number of posets}
\index{posets!number}
\index{exponential numbers}
\index{Euler numbers}
%---------------------------------------
Let $\setX_n$ be a finite set with order $n=\seto{\setX_n}$.
Let $P_n$ be the number of labeled ordered sets on $\setX_n$
and $p_n$ the number of unlabeled  ordered sets.
\propbox{\begin{array}{l|*{10}{|c}}
  n   & 0 & 1 & 2 & 3 &  4 &  5 &   6 &   7 &    8 &      9 \\
  \hline
  P_n      & 1 & 1 & 3  & 19  & 219  & 4231    & 130,023   & 6,129,859   & 431,723,379 & 44,511,042,511  \\
  p_n      & 1 & 1 & 2  & 5   & 16   & 63      & 318       & 2045        & 16,999      & 183,231  \\
\end{array}}
\end{proposition}


\begin{figure}[th]
\begin{center}
%\begin{fsL}
%\setlength{\unitlength}{\tw/(9*300)}
%\input{../common/order3.inp}
%\end{fsL}
\includegraphics{../common/math/graphics/pdfs/orderingsxyz.pdf}%
%%============================================================================
% Daniel J. Greenhoe
% LaTeX File
% lattice of topologies over the set {x,y,z}
%============================================================================
{\psset{xunit=0.20mm,yunit=0.20mm}%
%\fbox%
  {\begin{pspicture}(-300,-50)(300,350)%
  %---------------------------------
  % settings
  %---------------------------------
  %\fns%
  %\psset{labelsep=1.5mm,radius=75\psunit}
  \psset{
    labelsep=8mm,
    radius=7.5mm,
    linearc=18\psxunit,
    }
  %---------------------------------
  % developement tools
  %---------------------------------
  %\psgrid[xunit=100\psxunit,yunit=100\psyunit](-5,-1)(5,5)%
  %---------------------------------
  % nodes
  %---------------------------------
  %
  \Cnode( 250,100){T16}%
  \Cnode(  50,100){T14}%
  \Cnode(-150,100){T12}%
  \Cnode( 150,100){T15}%
  \Cnode( -50,100){T13}%
  \Cnode(-250,100){T11}%
  \Cnode(   0,0)  {T00}%
  %
  \Cnode( 250,300){Tzyx}%
  \Cnode(  50,300){Tyzx}%
  \Cnode(-150,300){Txzy}%
  \Cnode( 150,300){Tzxy}%
  \Cnode( -50,300){Tyxz}%
  \Cnode(-250,300){Txyz}%
  %
  \Cnode( 250,200){T26}%
  \Cnode(  50,200){T24}%
  \Cnode(-150,200){T22}%
  \Cnode( 150,200){T25}%
  \Cnode( -50,200){T23}%
  \Cnode(-250,200){T21}%
  %
  \Cnode( 250,100){T16}%
  \Cnode(  50,100){T14}%
  \Cnode(-150,100){T12}%
  \Cnode( 150,100){T15}%
  \Cnode( -50,100){T13}%
  \Cnode(-250,100){T11}%
  %
  %\Cnode(   0,0)  {T00}%
  %---------------------------------
  % node connections
  %---------------------------------
  %\ncline{T26}{Txyz}\ncline{T26}{Tyxz}%
  %\ncline{T25}{Txzy}\ncline{T25}{Tzxy}%
  %\ncline{T24}{Tzyx}\ncline{T24}{Tzyx}%
  %\ncline{T23}{Tzxy}\ncline{T23}{Tzyx}%
  %\ncline{T22}{Tyxz}\ncline{T22}{Tzyx}%
  %\ncline{T21}{Txyz}\ncline{T21}{Txzy}%
  %%
  %\ncline{T16}{T24}%
  %\ncline{T16}{T22}%
  %\ncline{T15}{T25}%
  %\ncline{T15}{T21}%
  %\ncline{T14}{T24}%
  %\ncline{T14}{T23}%
  %\ncline{T13}{T26}%
  %\ncline{T13}{T21}%
  %\ncline{T12}{T25}%
  %\ncline{T12}{T23}%
  %\ncline{T11}{T26}%
  %\ncline{T11}{T21}%
  %%
  %\ncline{T00}{T11}%
  %\ncline{T00}{T12}%
  %\ncline{T00}{T13}%
  %\ncline{T00}{T14}%
  %\ncline{T00}{T15}%
  %\ncline{T00}{T16}%
  %---------------------------------
  % node labels
  %---------------------------------
  %
  %\uput[-20](T00){$\topT_{00}$}%
  %---------------------------------
  % discriptions
  %---------------------------------
  %\rput[bl](-450,0){%left M5 lattice
  %  \psframe[linestyle=dashed,linecolor=red](0,0)(200,450)%
  %  \uput[-45](200,0){$M5$ lattice}
  %  }%
  %\rput[br](450,0){%right M5 lattice
  %  \psframe[linestyle=dashed,linecolor=red](0,0)(-200,450)%
  %  \uput[-45](-200,0){$M5$ lattice}
  %  }%
  \ncbox[nodesep=9\psyunit,boxsize=45\psxunit,linestyle=dashed,linecolor=red]{Txyz}{Tzyx}%
  \ncbox[nodesep=9\psyunit,boxsize=45\psxunit,linestyle=dashed,linecolor=red]{T24}{T26}%
  \ncbox[nodesep=9\psyunit,boxsize=45\psxunit,linestyle=dashed,linecolor=red]{T21}{T23}%
  \ncbox[nodesep=9\psyunit,boxsize=45\psxunit,linestyle=dashed,linecolor=red]{T11}{T16}%
  \ncbox[nodesep=9\psyunit,boxsize=45\psxunit,linestyle=dashed,linecolor=red]{T00}{T00}%
  %\rput[t](-350,10){M5 lattice}
  %\rput[t]( 350,10){M5 lattice}
  %---------------------------------
  % node inner lattices
  %---------------------------------
  \psset{
    unit=0.04mm,
    labelsep=3.5pt,
    radius=1mm,
    dotsep=0.5pt,
    linecolor=blue,
    }%
  %-------------------------------------
  % row 3 (totally ordered/linearly ordered)
  %-------------------------------------
  \rput(Tzyx){\begin{pspicture}(-100,-100)(100,100)
      \Cnode(0, 100){x}% 
      \Cnode(0,   0){y}% 
      \Cnode(0,-100){z}%
      \ncline{z}{y}\ncline{y}{x}%
      \uput[180](x){$x$}\uput[180](y){$y$}\uput[180](z){$z$}
    \end{pspicture}}%
  \rput(Tzxy){\begin{pspicture}(-100,-100)(100,100)
      \Cnode(0, 100){y}% 
      \Cnode(0,   0){x}% 
      \Cnode(0,-100){z}%
      \ncline{z}{x}\ncline{x}{y}%
      \uput[180](x){$x$}\uput[180](y){$y$}\uput[180](z){$z$}
    \end{pspicture}}%
  \rput(Tyzx){\begin{pspicture}(-100,-100)(100,100)
      \Cnode(0, 100){x}% 
      \Cnode(0,   0){z}% 
      \Cnode(0,-100){y}%
      \ncline{y}{z}\ncline{z}{x}%
      \uput[180](x){$x$}\uput[180](y){$y$}\uput[180](z){$z$}
    \end{pspicture}}%
  \rput(Tyxz){\begin{pspicture}(-100,-100)(100,100)
      \Cnode(0, 100){z}% 
      \Cnode(0,   0){x}% 
      \Cnode(0,-100){y}%
      \ncline{y}{x}\ncline{x}{z}%
      \uput[0](x){$x$}\uput[0](y){$y$}\uput[0](z){$z$}
    \end{pspicture}}%
  \rput(Txzy){\begin{pspicture}(-100,-100)(100,100)
      \Cnode(0, 100){y}% 
      \Cnode(0,   0){z}% 
      \Cnode(0,-100){x}%
      \ncline{x}{z}\ncline{y}{z}%
      \uput[0](x){$x$}\uput[0](y){$y$}\uput[0](z){$z$}
    \end{pspicture}}%
  \rput(Txyz){\begin{pspicture}(-100,-100)(100,100)
      \Cnode(0, 100){z}% 
      \Cnode(0,   0){y}% 
      \Cnode(0,-100){x}%
      \ncline{x}{y}\ncline{y}{z}%
      \uput[0](x){$x$}\uput[0](y){$y$}\uput[0](z){$z$}
    \end{pspicture}}%
  %-------------------------------------
  % row 2
  %-------------------------------------
  \rput(T26){\begin{pspicture}(-100,-100)(100,100)
                   \Cnode(0,100){z}% 
      \Cnode(-100,-100){x} \Cnode(100,-100){y}%
      \ncline{z}{x}\ncline{z}{y}%
      \uput[135](x){$x$}\uput[45](y){$y$}\uput[90](z){$z$}
    \end{pspicture}}%
  \rput(T25){\begin{pspicture}(-100,-100)(100,100)
                   \Cnode(0,100){y}% 
      \Cnode(-100,-100){x} \Cnode(100,-100){z}%
      \ncline{y}{x}\ncline{y}{z}%
      \uput[135](x){$x$}\uput[90](y){$y$}\uput[45](z){$z$}
    \end{pspicture}}%
  \rput(T24){\begin{pspicture}(-100,-100)(100,100)
                   \Cnode(0,100){x}% 
      \Cnode(-100,-100){y} \Cnode(100,-100){z}%
      \ncline{x}{y}\ncline{x}{z}%
      \uput[90](x){$x$}\uput[135](y){$y$}\uput[45](z){$z$}
    \end{pspicture}}%
  \rput(T23){\begin{pspicture}(-100,-100)(100,100)
      \Cnode(-100,100){x} \Cnode(100,100){y}%
                \Cnode(0,-100){z}% 
      \ncline{x}{z}\ncline{y}{z}%
      \uput[225](x){$x$}\uput[-45](y){$y$}\uput[-90](z){$z$}
    \end{pspicture}}%
  \rput(T22){\begin{pspicture}(-100,-100)(100,100)
      \Cnode(-100,100){x} \Cnode(100,100){z}%
                \Cnode(0,-100){y}% 
      \ncline{x}{y}\ncline{y}{z}%
      \uput[225](x){$x$}\uput[-90](y){$y$}\uput[-45](z){$z$}
    \end{pspicture}}%
  \rput(T21){\begin{pspicture}(-100,-100)(100,100)
      \Cnode(-100,100){y} \Cnode(100,100){z}%
                \Cnode(0,-100){x}% 
      \ncline{x}{y}\ncline{x}{z}%
      \uput[-90](x){$x$}\uput[225](y){$y$}\uput[-45](z){$z$}
    \end{pspicture}}%
  %-------------------------------------
  % row 1
  %-------------------------------------
  \rput(T16){\begin{pspicture}(-100,-100)(100,100)
      \Cnode(0, 100){x}%
                        \Cnode(100,0){z} 
      \Cnode(0,-100){y}%
      \ncline{y}{x}%
      \uput[180](x){$x$}\uput[180](y){$y$}\uput[0](z){$z$}
    \end{pspicture}}%
  \rput(T15){\begin{pspicture}(-100,-100)(100,100)
      \Cnode(0, 100){y}%
                        \Cnode(100,0){z} 
      \Cnode(0,-100){x}%
      \ncline{y}{x}%
      \uput[180](x){$x$}\uput[180](y){$y$}\uput[0](z){$z$}
    \end{pspicture}}%
  \rput(T14){\begin{pspicture}(-100,-100)(100,100)
      \Cnode(0, 100){x}%
                        \Cnode(100,0){y} 
      \Cnode(0,-100){z}%
      \ncline{z}{x}%
      \uput[180](x){$x$}\uput[0](y){$y$}\uput[180](z){$z$}
    \end{pspicture}}%
  \rput(T13){\begin{pspicture}(-100,-100)(100,100)
                        \Cnode(0, 100){z}%
      \Cnode(-100,0){y} 
                        \Cnode(0,-100){x}%
      \ncline{x}{z}%
      \uput[0](x){$x$}\uput[180](y){$y$}\uput[0](z){$z$}
    \end{pspicture}}%
  \rput(T12){\begin{pspicture}(-100,-100)(100,100)
                        \Cnode(0, 100){y}%
      \Cnode(-100,0){x} 
                        \Cnode(0,-100){z}%
      \ncline{y}{z}%
      \uput[180](x){$x$}\uput[0](y){$y$}\uput[0](z){$z$}
    \end{pspicture}}%
  \rput(T11){\begin{pspicture}(-100,-100)(100,100)
                        \Cnode(0, 100){z}%
      \Cnode(-100,0){x} 
                        \Cnode(0,-100){y}%
      \ncline{y}{z}%
      \uput[180](x){$x$}\uput[0](y){$y$}\uput[0](z){$z$}
    \end{pspicture}}%
  %-------------------------------------
  % row 0
  %-------------------------------------
  \rput(T00){\begin{pspicture}(-100,-100)(100,100)
      \Cnode(-100,0){x} \Cnode(0,0){y} \Cnode(100,0){z}%
      %\ncline{x}{x}\ncline{x}{z}
      \uput[-90](x){$x$}\uput[-90](y){$y$}\uput[-90](z){$z$}
    \end{pspicture}}%
\end{pspicture}%
}%
}%
\end{center}
\caption{
  All possible orderings of the set $\setn{x,y,z}$ \xref{ex:num_posets_xyz}.
  \label{fig:num_posets_xyz}
  }
\end{figure}

%---------------------------------------
\begin{example}  %[Order relations on $\setn{x,y,z}$]
\label{ex:num_posets_xyz}
%---------------------------------------
\prefpp{prop:num_posets} indicates that there are exactly
19 labeled order relations on the set $\setn{x,y,z}$ and 5 unlabeled order relations.
\exboxt{
  The 19 labeled order relations on $\setn{x,y,z,}$ are represented here using three methods:
  \\\indentx$\begin{array}{FMMM}
    1. & Hasse diagrams:     & \pref{fig:num_posets_xyz} & \prefpo{fig:num_posets_xyz}\\
    2. & order relations:    & \pref{tbl:order_xyz}      & \prefpo{tbl:order_xyz}     \\
    3. & covering relations: & \pref{tbl:cover_xyz}      & \prefpo{tbl:cover_xyz}     \\
  \end{array}$
  }\\
%In \prefpp{fig:num_posets_xyz}, 
In each of these three methods, the 19 \emph{labeled} order relations are arranged into 5 groups,
each group representing one of the 5 \emph{unlabeled} order relations.
\end{example}

\begin{table}\centering
\begin{tabular}{|MNM *{4}{M} M|}
  \hline
  \mc{8}{G}{labeled order relations on $\setn{x,y,z}$}\\
  \hline
  \orel_{ 1} &=& \{ &\opair{x}{x},\, \opair{y}{y},\, \opair{z}{z}  &               &               &              & \}  \\\hline
  \orel_{ 2} &=& \{ &\opair{x}{x},\, \opair{y}{y},\, \opair{z}{z}, & \opair{y}{z}  &               &              & \}  \\
  \orel_{ 3} &=& \{ &\opair{x}{x},\, \opair{y}{y},\, \opair{z}{z}, & \opair{z}{y}  &               &              & \}  \\
  \orel_{ 4} &=& \{ &\opair{x}{x},\, \opair{y}{y},\, \opair{z}{z}, & \opair{x}{z}  &               &              & \}  \\
  \orel_{ 5} &=& \{ &\opair{x}{x},\, \opair{y}{y},\, \opair{z}{z}, & \opair{z}{x}  &               &              & \}  \\
  \orel_{ 6} &=& \{ &\opair{x}{x},\, \opair{y}{y},\, \opair{z}{z}, & \opair{x}{y}  &               &              & \}  \\
  \orel_{ 7} &=& \{ &\opair{x}{x},\, \opair{y}{y},\, \opair{z}{z}, & \opair{y}{x}  &               &              & \}  \\\hline
  \orel_{ 8} &=& \{ &\opair{x}{x},\, \opair{y}{y},\, \opair{z}{z}, & \opair{x}{y}, & \opair{x}{z}  &              & \}  \\
  \orel_{ 9} &=& \{ &\opair{x}{x},\, \opair{y}{y},\, \opair{z}{z}, & \opair{x}{y}, & \opair{y}{z}  &              & \}  \\
  \orel_{10} &=& \{ &\opair{x}{x},\, \opair{y}{y},\, \opair{z}{z}, & \opair{z}{x}, & \opair{z}{y}  &              & \}  \\\hline
  \orel_{11} &=& \{ &\opair{x}{x},\, \opair{y}{y},\, \opair{z}{z}, & \opair{y}{x}, & \opair{z}{x}  &              & \}  \\
  \orel_{12} &=& \{ &\opair{x}{x},\, \opair{y}{y},\, \opair{z}{z}, & \opair{x}{y}, & \opair{z}{y}  &              & \}  \\
  \orel_{13} &=& \{ &\opair{x}{x},\, \opair{y}{y},\, \opair{z}{z}, & \opair{x}{z}, & \opair{y}{z}  &              & \}  \\\hline
  \orel_{14} &=& \{ &\opair{x}{x},\, \opair{y}{y},\, \opair{z}{z}, & \opair{x}{y}, & \opair{y}{z}, & \opair{x}{z} & \}  \\
  \orel_{15} &=& \{ &\opair{x}{x},\, \opair{y}{y},\, \opair{z}{z}, & \opair{x}{z}, & \opair{x}{y}, & \opair{z}{y} & \}  \\
  \orel_{16} &=& \{ &\opair{x}{x},\, \opair{y}{y},\, \opair{z}{z}, & \opair{y}{x}, & \opair{y}{z}, & \opair{x}{z} & \}  \\
  \orel_{17} &=& \{ &\opair{x}{x},\, \opair{y}{y},\, \opair{z}{z}, & \opair{y}{z}, & \opair{y}{x}, & \opair{z}{x} & \}  \\
  \orel_{18} &=& \{ &\opair{x}{x},\, \opair{y}{y},\, \opair{z}{z}, & \opair{z}{x}, & \opair{z}{y}, & \opair{x}{y} & \}  \\
  \orel_{19} &=& \{ &\opair{x}{x},\, \opair{y}{y},\, \opair{z}{z}, & \opair{z}{y}, & \opair{z}{x}, & \opair{y}{x} & \}  \\
  \hline
\end{tabular}
\caption{labeled order relations on $\setn{x,y,z}$
  \label{tbl:order_xyz}
  }
\end{table}

\begin{table}\centering
\begin{tabular}{|*{2}{MNMMMM|}}
  \hline
  \mc{12}{G}{labeled cover relations on $\setn{x,y,z}$}
  \\\hline
  \tblx             \coverrel_{ 1} &=& \mc{4}{M|}{\emptyset}                     \tblc \coverrel_{11} &=& \{ & \opair{y}{x}, & \opair{z}{x}  & \}
  \tbln\cline{1-6}  \coverrel_{ 2} &=& \{ & \opair{y}{z}  &               & \}   \tblc \coverrel_{12} &=& \{ & \opair{x}{y}, & \opair{z}{y}  & \}
  \tbln             \coverrel_{ 3} &=& \{ & \opair{z}{y}  &               & \}   \tblc \coverrel_{13} &=& \{ & \opair{x}{z}, & \opair{y}{z}  & \}
  \tbln\cline{7-12} \coverrel_{ 4} &=& \{ & \opair{x}{z}  &               & \}   \tblc \coverrel_{14} &=& \{ & \opair{x}{y}, & \opair{y}{z}  & \}
  \tbln             \coverrel_{ 5} &=& \{ & \opair{z}{x}  &               & \}   \tblc \coverrel_{15} &=& \{ & \opair{x}{z}, & \opair{x}{y}  & \}
  \tbln             \coverrel_{ 6} &=& \{ & \opair{x}{y}  &               & \}   \tblc \coverrel_{16} &=& \{ & \opair{y}{x}, & \opair{y}{z}  & \}
  \tbln             \coverrel_{ 7} &=& \{ & \opair{y}{x}  &               & \}   \tblc \coverrel_{17} &=& \{ & \opair{y}{z}, & \opair{y}{x}  & \}
  \tbln\cline{1-6}  \coverrel_{ 8} &=& \{ & \opair{x}{y}, & \opair{x}{z}  & \}   \tblc \coverrel_{18} &=& \{ & \opair{z}{x}, & \opair{z}{y}  & \}
  \tbln             \coverrel_{ 9} &=& \{ & \opair{x}{y}, & \opair{y}{z}  & \}   \tblc \coverrel_{19} &=& \{ & \opair{z}{y}, & \opair{z}{x}  & \}
  \tbln             \coverrel_{10} &=& \{ & \opair{z}{x}, & \opair{z}{y}  & \}   \tblc                & &    &               &               &   
  \\\hline 
\end{tabular}
\caption{labeled cover relations on $\setn{x,y,z}$
  \label{tbl:cover_xyz}
  }
\end{table}

%\ifexclude{mssa}{
%=======================================
\section{Functions on ordered sets}
%=======================================
%---------------------------------------
\begin{definition}
\footnote{
  \citerp{burris2000}{10}
  }
\label{def:poset_orderpre}
\index{poset!order preserving}
%---------------------------------------
Let $\opair{\setX}{\orel}$ and $\opair{\setY}{\orela}$ be ordered sets.
\defbox{\begin{array}{M}%
  A function $\ds\ftheta\in\clFxy$ is \propd{order preserving} 
  with respect to $\orel$ and $\orela$ if
  \\\indentx
    $\ds x \orel y \qquad\implies\qquad \ftheta(x)\orela\ftheta(y) \qquad\forall x,y\in\setX$.
\end{array}}
\end{definition}

\begin{minipage}{\tw-40mm}%
%---------------------------------------
\begin{example}
\footnotemark
\label{ex:order_M2_L4}
%---------------------------------------
In the diagram to the right, the function $\ftheta\in\clFxy$ 
is \prope{order preserving} with respect to $\orel$ and $\orela$.
Note that $\ftheta^{-1}$ is \emph{not} order preserving.
This example also illustrates the fact that 
that order preserving does not imply \prope{isomorphic}.
\end{example}
\end{minipage}%
\footnotetext{%
  \citerp{burris2000}{10}%
  }%
\hfill%
\tbox{\includegraphics{../common/math/graphics/pdfs/m2tol4.pdf}}%

\begin{minipage}{\tw-40mm}%
%---------------------------------------
\begin{example}
\label{ex:order_M3_N5}
%---------------------------------------
In the diagram to the right, the function $\ftheta\in\clFxy$ 
is \prope{order preserving} with respect to $\orel$ and $\orela$.
Note that $\ftheta^{-1}$ is \emph{not} order preserving.
Like \prefpp{ex:order_M2_L4},
this example also illustrates the fact that 
that order preserving does not imply \prope{isomorphic}.
\end{example}
\end{minipage}%
\hfill%
\tbox{\includegraphics{../common/math/graphics/pdfs/m3ton5.pdf}}%

\begin{minipage}{\tw-40mm}%
%---------------------------------------
\begin{example}
\label{ex:order_N5_N5}
%---------------------------------------
In the diagram to the right, the function $\ftheta\in\clFxy$ 
is \prope{order preserving} with respect to $\orel$ and $\orela$.
Note that $\ftheta^{-1}$ \emph{is also} order preserving.
In this case, $\ftheta$ is an \prope{isomorphism} and 
the ordered sets $\opair{\setX}{\orel}$ and $\opair{\setY}{\orela}$ are \prope{isomorphic}.
\end{example}
\end{minipage}%
\hfill%
\tbox{\includegraphics{../common/math/graphics/pdfs/n5ton5.pdf}}%

%---------------------------------------
\begin{example}
\footnote{
  \citerpgc{munkres2000}{25}{0131816292}{Example 1\textsection3.9}
  }
%---------------------------------------
\exbox{\text{
  The function $\ff(x)\eqd\frac{x}{1-x^2}$ in $\clF{\intoo{-1}{1}}{\R}$ is \prope{bijective} and \prope{order preserving}.
  }}
\end{example}

%---------------------------------------
\begin{theorem}[\thm{Pointwise ordering relation}]
\label{thm:f_order}
\footnote{
  \citerp{shen2002}{43},
  \citerp{giles2000}{252}
  }%
%---------------------------------------
Let $\setX$ be a set, $\opair{\setY}{\orel}$ an ordered set, and
$\ff,\fg\in\clFxy$.
\thmbox{\begin{array}{M}
  $\ff(x)\orel\fg(x) \forall x\in\setX
  \qquad\implies\qquad
  \opair{\clFxy}{\orela}$ is an ordered set.
  \\
  In this case we say $\ff$ is ``dominated by" $\fg$ in $\setX$,
  or we say $\fg$ ``dominates" $\ff$ in $\setX$.
\end{array}}
\end{theorem}


\begin{minipage}[c]{\tw/2}%
%---------------------------------------
\begin{example}[Pointwise ordering relation]
\label{ex:order_function}
\footnotemark
%---------------------------------------
Let $\ff\orela\fg$ represent that $\ff(x)\orel\fg(x)$ for all $0\orel x\orel 1$
(we say $\ff$ is ``dominated by" $\fg$ in the region $[0,1]$,
 or we say $\fg$ ``dominates" $\ff$ in the region $[0,1]$).
  The pair $(\set{\ff_n(x)=1-x^n}{n\in\Zp},\; \orela)$ is a totally ordered set.
\end{example}
\end{minipage}%
\footnotetext{
  \citerp{shen2002}{43},
  \citerp{giles2000}{252},
  \citerpg{ab2006}{2}{1402050070}
  }%
\begin{minipage}{\tw/2}%
  \begin{center}%
  \begin{fsL}%
  \setlength{\unitlength}{\tw/180}%
  \begin{picture}(140,140)(-20,-20)%
  {\color{graphpaper}\graphpaper[25](0,0)(100,100)}%
    \thinlines%
    \color{axis}%
      \put(0,0){\line(0,1){110}}
      \put(0,0){\line(1,0){110}}
      \put(115,0){\makebox(0,0)[l]{$x$}}
      %\qbezier[40](50,0)(50,50)(50,100)
    \thicklines%
    \color{blue}%
      \put( 0, 0){\line(1, 0){100}}%
      \put( 0,100){\line(1,-1){100}}%
      \qbezier(0,100)(50,100)(100,0)%
      \qbezier(0,100)(66.7,100)(100,0)%
      \qbezier(0,100)(75,100)(100,0)%
      \qbezier(0,100)(80,100)(100,0)%
      \qbezier(0,100)(83.3,100)(100,0)%
      \qbezier(0,100)(85.7,100)(100,0)%
      \qbezier(0,100)(87.5,100)(100,0)%
      \put(50,75){\makebox(0,0)[tr]{$n=2$}}%
      \put(25,75){\makebox(0,0)[rt]{$n=1$}}%
      \put(30, 1){\makebox(0,0)[lb]{$n=0$}}%
    \color{red}%
      \qbezier(0,100)(100,100)(100,0)%
      \qbezier(12.5,10)(12.5,25)(25,37.5)%
      \put(25,37.5){\vector(1,1){30}}%
      \put(20,23){\makebox(0,0)[tl]{increasing $n$}}%
      \put(75,75){\makebox(0,0)[bl]{$n=\infty$}}%
  \end{picture}%
  \end{fsL}%
  \end{center}%
\end{minipage}






%=======================================
\section{Decomposition}
%=======================================
%=======================================
\subsection{Subposets}
%=======================================
%---------------------------------------
\begin{definition}
\footnote{
  \citerpg{gratzer2003}{2}{3764369965}
  }
\label{def:subposet}
%---------------------------------------
\defbox{\begin{array}{M}
  The tupple $\opair{\setY}{\orela}$ is a \structd{subposet} of the ordered set $\opair{\setX}{\orel}$ if 
  \\$\ds\begin{array}{@{\qquad}llDD}
    1. & \setY\subseteq\setX            & ($\setY$ is a subset of $\setX$) & and  \\
    2. & \orela = \orel \seti\setY^2 & ($\orela$ is the relation $\orel$ restricted to $\setY\cprod\setY$)
  \end{array}$
\end{array}}
\end{definition}


%---------------------------------------
\begin{example}
%---------------------------------------
\mbox{}\\%
%Examples of subposets of the ordered set $\opair{\setn{x,y,z}}{\subseteq}$ \ldots\\
\begin{tabular}{ccccccc}
  \parbox[c][\tw/8][c]{3\tw/16}{Subposets of\\\vspace{3ex}}
  &\includegraphics{../common/math/graphics/pdfs/lat8_l2e3.pdf}
  & 
  \parbox[c][\tw/8][c]{2\tw/16}{include\\\vspace{3ex}}
  &\includegraphics{../common/math/graphics/pdfs/lat4_l4.pdf}
  &\includegraphics{../common/math/graphics/pdfs/lat4_m2.pdf}
  &\includegraphics{../common/math/graphics/pdfs/lat5_l2onm2.pdf}
  &\includegraphics{../common/math/graphics/pdfs/lat5_m2onl2.pdf}
\end{tabular}
\end{example}

%---------------------------------------
\begin{example}
%---------------------------------------
Let\\
\begin{minipage}{12\tw/16}%
\begin{align*}
  \opair{\setX}{\orel}
    &\eqd \Big( \setn{0,a,b,c,p,1},
    &&    \Big\{\opair{0}{0},\,\opair{a}{a},\,\opair{b}{b},\,\opair{c}{c},\,\opair{p}{p},\,\opair{1}{1},
  \\&&&   \opair{0}{a},\,\opair{0}{b},\,\opair{0}{c},\,\opair{0}{p},\,\opair{0}{1},
  \\&&&   \opair{a}{b},\,\opair{a}{c},\,\opair{a}{1},\,\opair{p}{1},
  \\&&&   \opair{b}{c},\,\opair{b}{1},\,\opair{c}{1},\,\opair{p}{1} \Big\}\Big)
  \\
  \opair{\setY}{\orela}
    &\eqd \Big( \setn{0,a,c,p,1},
    &&    \Big\{\opair{0}{0},\,\opair{a}{a},\,\opair{c}{c},\,\opair{p}{p},\,\opair{1}{1},
  \\&&&   \opair{0}{a},\,\opair{0}{c},\,\opair{0}{p},\,\opair{0}{1},
  \\&&&   \opair{a}{c},\,\opair{a}{1},\,\opair{p}{1},\,\opair{c}{1},\,\opair{p}{1} \Big\}\Big).
\end{align*}
\end{minipage}%
\begin{minipage}{4\tw/16}%
\begin{center}%
\begin{tabular}{c}%
   \includegraphics{../common/math/graphics/pdfs/lat6_n6_0abcp1.pdf}
\\\\
  \includegraphics{../common/math/graphics/pdfs/lat5_n5_0acp1.pdf}
\end{tabular}
\end{center}%
\end{minipage}
\\Then $\opair{\setY}{\orela}$ is a subposet of $\opair{\setX}{\orel}$ because
$\setY\subseteq\setX$ and $\orela=(\orel\seti\setY^2)$.
\end{example}

A \structe{chain} is an ordered set in which every pair of elements is \prope{comparable} \xref{def:chain}.
An \structe{antichain} is just the opposite---it is an ordered set in which \emph{no} pair of elements is comparable
(next definition).\\
\begin{minipage}{\tw-43mm}
%---------------------------------------
\begin{definition}
\footnotemark
\label{def:antichain}
%---------------------------------------
\defboxt{
  %The subposet $\opair{\setC}{\orela}$ in the ordered set $\opair{\setX}{\orel}$ 
  %is a \hid{chain} if it is linearly ordered.
  The subposet $\opair{\setA}{\orel}$ in the ordered set $\opair{\setX}{\orel}$ is an \structd{antichain} if 
  \\\indentx$a||b\qquad\forall a,b\in\setA$\\ 
  (all elements in $\setA$ are \prope{incomparable}).
  }
\end{definition}
\end{minipage}%
\footnotetext{
  \citerpg{gratzer2003}{2}{3764369965}
  }%
\hfill\tbox{\includegraphics{../common/math/graphics/pdfs/lat2xyz_antichains.pdf}}%

%---------------------------------------
\begin{definition}
\footnote{
  \citerpg{gratzer2003}{2}{3764369965},
  \citerpg{birkhoff1967}{5}{0821810251}
  }
\label{def:poset_length}
\label{def:poset_width}
\label{def:length}
\label{def:width}
%---------------------------------------
\defboxp{
  \begin{liste}
    \item The \propd{length} of a chain $\opair{\setC}{\orel}$ equals $\seto{\setC}-1$.
    \item The \propd{length} of a poset $\opair{\setX}{\orel}$ is the length of the longest chain in the ordered set.
    \item The \propd{width} of a poset $\opair{\setX}{\orel}$ is number of elements in the largest antichain in the ordered set.
  \end{liste}
  }
\end{definition}


%---------------------------------------
\begin{theorem}[Dilworth's theorem]
\footnote{
  \citePp{dilworth1950}{161},
  \citer{dilworth1950r},
  \citePp{farley1997}{4}
  }
\label{thm:dilworth}
\index{Dilworth's theorem}
\index{theorems!Dilworth}
%---------------------------------------
Let $\opair{\setX}{\orel}$ be an ordered set with width $n$.
\thmbox{
  \brb{\begin{array}{M}
    \prope{width} $n$ of $\opair{\setX}{\orel}$\\
    is \prope{finite}
  \end{array}}
  \implies
  \brb{\begin{array}{FMD}
    1. & there exists a \structe{partition} of $\opair{\setX}{\orel}$ into $n$ chains & and \\
    2. & there does not exist any \structe{partition}  \\
       & of $\opair{\setX}{\orel}$ into less than $n$ chains
  \end{array}}
  }
\end{theorem}

%=======================================
\subsection{Operations on posets}
%=======================================
%---------------------------------------
\begin{definition}
\footnote{
  \citerpg{stanley1997}{100}{0521663512}
  }
%---------------------------------------
Let $\setX$ and $\setY$ be disjoint sets.
Let $\osetP\eqd\oset{\setX}{\orela}$ and $\osetQ\eqd\oset{\setY}{\orelb}$ 
be ordered sets on $\setX$ and $\setY$.
\defbox{\begin{array}{M}
  The \opd{direct sum} of $\osetP$ and $\osetQ$ is defined as
  \\\indentx
  $\ds \osetP\hxsd{+}\osetQ \eqd \oset{\setX\setu\setY}{\orel}$
  \\where $\ds x\orel y$ if
  \\$\begin{array}{@{\qquad}llMlM}
    1. & x,y\in\setX & and & x\orela y & or \\
    2. & x,y\in\setY & and & x\orelb y
  \end{array}$
  \\
  The direct sum operation is also called the \opd{disjoint union}.
  The notation $n\osetP$ is defined as
  \\\indentx$\ds n\osetP \eqd \mcom{\osetP + \osetP + \cdots + \osetP}{$n-1$ ``$+$" operations}$.
\end{array}}
\end{definition}

%---------------------------------------
\begin{definition}
\footnote{
  \citerppg{stanley1997}{100}{101}{0521663512},
  \citerpg{shen2002}{43}{0821827316}
  }
%---------------------------------------
Let $\setX$ and $\setY$ be disjoint sets.
Let $\osetP\eqd\oset{\setX}{\orela}$ and $\osetQ\eqd\oset{\setY}{\orelb}$ 
be ordered sets on $\setX$ and $\setY$.
\defboxp{
  The \opd{direct product} of $\osetP$ and $\osetQ$ is defined as
  \\\indentx
  $\ds \osetP\times\osetQ \eqd \oset{\setX\cprod\setY}{\orel}$
  \\where $\ds \opair{x_1}{y_1}\orel\opair{x_2}{y_2}$ if
  $\ds x_1 \orela x_2 \quad\text{and}\quad y_1 \orela y_2.$
  \\
  The direct product operation is also called the \opd{cartesian product}.
  The order relation $\orel$ is called a \reld{coordinate wise order relation}.
  The notation $\osetP^n$ is defined as
  \\\indentx$\ds \osetP^n \eqd \mcom{\osetP \times \osetP \times \cdots \times \osetP}{$n-1$ ``$\times$" operations}$.
  }
\end{definition}

%---------------------------------------
\begin{definition}
\footnote{
  \citerpg{stanley1997}{100}{0521663512}
  }
%---------------------------------------
Let $\setX$ and $\setY$ be disjoint sets.
Let $\osetP\eqd\oset{\setX}{\orela}$ and $\osetQ\eqd\oset{\setY}{\orelb}$ 
be ordered sets on $\setX$ and $\setY$.
\defbox{\begin{array}{M}
  The \opd{ordinal sum} of $\osetP$ and $\osetQ$ is defined as
  \\\indentx
  $\ds \osetP\hxsd{\oplus}\osetQ \eqd \oset{\setX\setu\setY}{\orel}$
  \\where $\ds x\orel y$ if
  \\$\begin{array}{@{\qquad}llMlM}
    1. & x,y\in\setX & and & x\orela y & or \\
    2. & x,y\in\setY & and & x\orelb y & or \\
    3. & x\in\setX   & and & y\in\setY.
  \end{array}$
\end{array}}
\end{definition}

%---------------------------------------
\begin{definition}
\footnote{
  \citerpg{stanley1997}{101}{0521663512},
  \citerpg{shen2002}{44}{0821827316},
  \citerpg{halmos1960}{58}{0387900926},
  \citerpg{hausdorff1937e}{54}{0828401195}
  }
%---------------------------------------
Let $\setX$ and $\setY$ be disjoint sets.
Let $\osetP\eqd\oset{\setX}{\orela}$ and $\osetQ\eqd\oset{\setY}{\orelb}$ 
be ordered sets on $\setX$ and $\setY$.
\defboxt{
  The \opd{ordinal product} of $\osetP$ and $\osetQ$ is defined as
  \\\indentx
  $\ds \osetP\hxsd{\otimes}\osetQ \eqd \oset{\setX\cprod\setY}{\orel}$
  \\where $\ds \opair{x_1}{y_1}\orel\opair{x_2}{y_2}$ if
  \\\indentx$\begin{array}{FlMlM}
    1. & x_1\ne x_2   & and & x_1\orela x_2 & or \\
    2. & x_1=x_2      & and & y_1\orelb y_2 
  \end{array}$
  \\
  The order relation $\orel$ is called a \reld{lexicographical} order relation,
  \reld{dictionary order relation},\\
  or \reld{alphabetic order relation}.
  }
\end{definition}

%---------------------------------------
\begin{definition}
\footnote{
  \citerpg{stanley1997}{101}{0521663512}
  }
%---------------------------------------
Let $\osetP\eqd\oset{\setX}{\orel}$ be an ordered set.
Let $\oreld$ be the dual order relation of $\orel$.
\defbox{\begin{array}{M}
  The \structd{dual} of $\osetP$ is defined as
  \\\indentx
  $\ds \hxsd{\osetP^\ast} \eqd \oset{\setX}{\oreld}$
\end{array}}
\end{definition}

%---------------------------------------
\begin{definition}
\footnote{
  \citerpg{stanley1997}{101}{0521663512}
  }
%---------------------------------------
Let $\setX$ and $\setY$ be disjoint sets.
Let $\osetP\eqd\oset{\setX}{\orela}$ and $\osetQ\eqd\oset{\setY}{\orelb}$ 
be ordered sets on $\setX$ and $\setY$.
\defbox{\begin{array}{M}
  The \opd{ordinal product} of $\osetP$ and $\osetQ$ is defined as
  \\\indentx
  $\ds 
    \hxsd{\clF{\osetP}{\osetQ}} 
    \eqd 
    \oset{\set{\ff\in\clFxy}{\text{$\ff$ is \prope{order preserving}}}}{\orel}
  $
  \\where $\ds \ff\orel\fg$ if $\ff(x)\orel\fg(x)\quad\forall x\in\setX$.
  \\
  The order relation $\orel$ is called a \reld{pointwise order relation} \xref{ex:order_function}.
\end{array}}
\end{definition}

%---------------------------------------
\begin{theorem}[cardinal arithmetic]
\footnote{
  \citerpg{stanley1997}{102}{0521663512}
  }
\index{cardinal arithmetic}
%---------------------------------------
Let $\osetP\eqd\oset{\setX}{\orel}$ be an ordered set.
\thmbox{
  \begin{array}{llclD}
    1. & \osetP + \osetQ &=& \osetQ+\osetP & \prop{commutative} \\
    2. & \osetP \times \osetQ &=& \osetQ\times\osetP & \prop{commutative} \\
    3. & \brp{\osetP + \osetQ}+\osetR &=& \osetP+\brp{\osetQ+\osetR} & \prop{associative} \\
    4. & \brp{\osetP \times \osetQ}\times\osetR &=& \osetP\times\brp{\osetQ\times\osetR} & \prop{associative} \\
    5. & \osetP\times\brp{\osetQ+\osetR}&=& \brp{\osetP\times\osetQ}+\brp{\osetP\times\osetR} & \prop{distributive}\\
    6. & \osetR^{\osetP+\osetQ} &=& \osetR^\osetP \times \osetR^\osetQ \\
    7. & \brp{\osetP^\osetQ}^\osetR &=& \osetP^{\osetQ\times\osetR}
  \end{array}
  }
\end{theorem}


%=======================================
\subsection{Primitive subposets}
%=======================================
\begin{minipage}{13\textwidth/16}
%---------------------------------------
\begin{definition}
\label{def:lat_1}
%---------------------------------------
\defbox{\begin{array}{M}
  The ordered set $\hxsd{\latL_1}$ is defined as $\opair{\setn{x}}{\orel}$, for some value $x$.
\end{array}}
\\The $\latL_1$ ordered set is illustrated by the Hasse diagram to the right.
\end{definition}
\end{minipage}%
\\\hfill\tbox{\includegraphics{../common/math/graphics/pdfs/lat1.pdf}}

\begin{minipage}{\tw-10mm}
%---------------------------------------
\begin{definition}
\label{def:lat_2}
%---------------------------------------
\defbox{\begin{array}{M}
  The ordered set $\hxsd{\latTwo}$ is defined as $\latTwo\eqd\latOne^2$.
  %$\opair{\setn{\bzero,\bid}}{\orel}$.
  %with cover relation
  %\\\indentx$\coverrel = \setn{\opair{\bzero}{\bid}}$
\end{array}}
\\The $\latTwo$ ordered set is illustrated by the Hasse diagram to the right.
\end{definition}
\end{minipage}%
\\\hfill\tbox{\includegraphics{../common/math/graphics/pdfs/lat2_l2.pdf}}

%=======================================
\subsection{Decomposition examples}
%=======================================
\begin{figure}
  \centering%
  \begin{tabular}{|cNc N c|}
    \hline
    %-------------------------------------
    % direct sum
    %-------------------------------------
    \tbox{\includegraphics{../common/math/graphics/pdfs/oset4_Nref.pdf}}
      &+&\tbox{\includegraphics{../common/math/graphics/pdfs/oset3_wedge.pdf}}%
      &=& \tbox{\includegraphics{../common/math/graphics/pdfs/oset4_Nref.pdf}}%
          \quad%
          \tbox{\includegraphics{../common/math/graphics/pdfs/oset3_wedge.pdf}}%
      \\\hline
    %-------------------------------------
    % ordinal sum
    %-------------------------------------
    \tbox{\includegraphics{../common/math/graphics/pdfs/oset4_Nref.pdf}}
      &\oplus&\tbox{\includegraphics{../common/math/graphics/pdfs/oset3_wedge.pdf}}%
      &=&\tbox{\includegraphics{../common/math/graphics/pdfs/oset7_wedge_oplus_Nref.pdf}}%
      \\\hline
    %-------------------------------------
    % direct product
    %-------------------------------------
    \tbox{\includegraphics{../common/math/graphics/pdfs/oset4_Nref.pdf}}
      &\times&\tbox{\includegraphics{../common/math/graphics/pdfs/oset3_wedge.pdf}}%
      &=&\tbox{\includegraphics{../common/math/graphics/pdfs/oset12_wedge_times_Nref.pdf}}% 
      \\\hline
    %-------------------------------------
    % ordinal product
    %-------------------------------------
    \tbox{\includegraphics{../common/math/graphics/pdfs/oset4_Nref.pdf}}
      &\otimes&\tbox{\includegraphics{../common/math/graphics/pdfs/oset3_wedge.pdf}}%
      &=&\tbox{\includegraphics{../common/math/graphics/pdfs/oset12_wedge_otimes_Nref.pdf}}% 
      \\\hline
  \end{tabular}
  \caption{Operations on ordered sets \xref{ex:posetops}
    \label{fig:posetops}
    }
\end{figure}
%---------------------------------------
\begin{example}
\label{ex:posetops}
%---------------------------------------
\prefpp{fig:posetops} illustrates the four ordered set operations
$+$, $\times$, $\oplus$, and $\otimes$.
\end{example}

\begin{minipage}{\tw-35mm}
%---------------------------------------
\begin{example}
\footnotemark
%---------------------------------------
The ordered set $n\latOne$ is the \structe{anti-chain} with $n$ elements.
The ordered set $4\latOne$ is illustrated to the right.
\end{example}
\end{minipage}%
\footnotetext{
  \citerpg{stanley1997}{100}{0521663512}
  }%
\begin{minipage}{3\tw/16}\center%
  \latmatw{2}{0.25}{\null & \null & \null & \null}{}
\end{minipage}

%---------------------------------------
\begin{minipage}{\tw-35mm}
\begin{example}
%---------------------------------------
The ordered set $\latOne^n$ is the \structe{chain} with $n$ elements.
The ordered set $\latOne^4$ is illustrated to the right.
\end{example}
\end{minipage}%
\hfill\includegraphics{../common/math/graphics/pdfs/lat4_l4.pdf}\hfill\mbox{}\\%

\begin{minipage}{\tw-35mm}
%---------------------------------------
\begin{example}
%---------------------------------------
The ordered set $\latTwo^2$ is the 4 element \structe{Boolean algebra} illustrated to the right.
\end{example}
\end{minipage}%
\begin{minipage}{3\tw/16}\center%
  \latmat{3}{
                  & [name=1]\null                 \\
    [name=x]\null &               & [name=y]\null \\
                  & [name=0]\null
    }{
    \ncline{1}{y}\ncline{0}{x}
    \psset{linecolor=red}
    \ncline{0}{y}\ncline{1}{x}
    }
\end{minipage}

\begin{minipage}{\tw-35mm}
%---------------------------------------
\begin{example}
%---------------------------------------
The ordered set $\latTwo^3$ is the 8 element \structe{Boolean algebra} illustrated to the right.
\end{example}
\end{minipage}%
\begin{minipage}{3\tw/16}\center%
\latmat{4}{
                 & [name=1]\null                   \\
  [name=zc]\null & [name=yc]\null & [name=xc]\null \\
  [name=x]\null  & [name=y] \null & [name=z ]\null \\
                 & [name=0]\null
  }{
  \ncline{1}{zc}\ncline{1}{xc}
  \ncline{x}{yc}
  \ncline{y}{zc}\ncline{y}{xc}
  \ncline{z}{yc}
  \ncline{0}{x} \ncline{0}{z}
  \psset{linecolor=red}
  \ncline{1}{yc}\ncline{0}{y}\ncline{x}{zc}\ncline{z}{xc}
  }
\end{minipage}





\begin{minipage}{\tw-55mm}%
%---------------------------------------
\begin{example}
\footnotemark
%---------------------------------------
The longest \structe{antichain} \xref{def:antichain} in the figure to the right has 4 elements
giving this ordered set a \prope{width} \xref{def:poset_width} of 4.
The longest chain also has 4 elements, giving the ordered set a \prope{length} \xref{def:poset_length} of 3.
By \thme{Dilworth's theorem} \xref{thm:dilworth}, the smallest \structe{partition} consists of four \structe{chain}s \xref{def:chain}.
One such \structe{partition} is
  \\\indentx$\setn{\setn{0,a,p,1},\,\setn{b},\,\setn{c,q},\,\setn{r}}.$
\end{example}
\end{minipage}%
\footnotetext{
  \citePp{farley1997}{4}
  }%
\hfill\tbox{\includegraphics{../common/math/graphics/pdfs/m3m3_antichain_abcpqr.pdf}}%

%%---------------------------------------
%\begin{example}
%\label{ex:poset_xyz_dilworth}
%%---------------------------------------
%The lattice illustrated below has width 3 and by \thme{Dilworth's theorem} \xref{thm:dilworth} 
%can be partitioned into 3 disjoint chains.
%One such \structe{partition} is illustrated below.
%{\centering%
% \gsize%
% \psset{unit=8mm}%
% \begin{tabular}{c}%============================================================================
% Daniel J. Greenhoe
% LaTeX file
% 
%============================================================================
{%\psset{unit=0.5\psunit}%
\begin{pspicture}(-1.5,-\latbot)(1.5,3.2)
  %---------------------------------
  % nodes
  %---------------------------------
  \Cnode(0,3){t}
  \Cnode(-1,2){xy} \Cnode(0,2){xz} \Cnode(1,2){yz}
  \Cnode(-1,1){x}  \Cnode(0,1){y}  \Cnode(1,1){z}
  \Cnode(0,0){b}
  %---------------------------------
  % node connections
  %---------------------------------
  \ncline{t}{xy}\ncline{t}{xz}\ncline{t}{yz}%
  \ncline{x}{xz}\ncline{x}{xy}%
  \ncline{y}{xy}\ncline{y}{yz}%
  \ncline{z}{xz}\ncline{z}{yz}%
  \ncline{b}{x} \ncline{b}{y} \ncline{b}{z}%
  %---------------------------------
  % node labels
  %---------------------------------
  \uput[0](t){$\bid$}%
  \uput[135](xy){$p$}\uput{1pt}[135](xz){$q$}\uput[45](yz){$r$}%
  \uput[-135](x){$a$}\uput[-45](y) {$b$}\uput[-45](z) {$c$}%
  \uput[180](b) {$\bzero$}%
\end{pspicture}
}%\end{tabular}  \begin{tabular}{c}{\huge=}\end{tabular}
% \begin{tabular}{c}%============================================================================
% Daniel J. Greenhoe
% LaTeX file
% 
%============================================================================
{%\psset{unit=0.5\psunit}%
\begin{pspicture}(-1.5,-\latbot)(1.5,3.2)
  %---------------------------------
  % nodes
  %---------------------------------
  \Cnode(0,3){t}
  \Cnode(-1,2){xy} %\Cnode(0,2){xz} \Cnode(1,2){yz}
  \Cnode(-1,1){x}  %\Cnode(0,1){y}  \Cnode(1,1){z}
  \Cnode(0,0){b}
  %---------------------------------
  % node connections
  %---------------------------------
  \ncline{t}{xy}%\ncline{t}{xz}\ncline{t}{yz}%
  %\ncline{x}{xz}
  \ncline{x}{xy}%
  %\ncline{y}{xy}\ncline{y}{yz}%
  %\ncline{z}{xz}\ncline{z}{yz}%
  \ncline{b}{x} %\ncline{b}{y} \ncline{b}{z}%
  %---------------------------------
  % node labels
  %---------------------------------
  \uput[0](t){$\bid$}%
  \uput[135](xy){$p$}%\uput{1pt}[135](xz){$q$}\uput[45](yz){$r$}%
  \uput[-135](x){$a$}%\uput[-45](y) {$b$}\uput[-45](z) {$c$}%
  \uput[180](b) {$\bzero$}%
\end{pspicture}
}%\end{tabular}     \begin{tabular}{c}{\huge+}\end{tabular}
% \begin{tabular}{c}%============================================================================
% Daniel J. Greenhoe
% LaTeX file
% 
%============================================================================
{%\psset{unit=0.5\psunit}%
\begin{pspicture}(-1.5,-\latbot)(1.5,3.2)
  %---------------------------------
  % nodes
  %---------------------------------
  \Cnode(0,3){t}
  %\Cnode(-1,2){xy} 
  %\Cnode(0,2){xz} 
  \Cnode(1,2){yz}
  %\Cnode(-1,1){x}  
  \Cnode(0,1){y}  
  %\Cnode(1,1){z}
  \Cnode(0,0){b}
  %---------------------------------
  % node connections
  %---------------------------------
  %\ncline{t}{xy}%\ncline{t}{xz}
  \ncline{t}{yz}%
  %\ncline{x}{xz}
  %\ncline{x}{xy}%
  %\ncline{y}{xy}
  \ncline{y}{yz}%
  %\ncline{z}{xz}\ncline{z}{yz}%
  %\ncline{b}{x} 
  \ncline{b}{y} 
  %\ncline{b}{z}%
  %---------------------------------
  % node labels
  %---------------------------------
  \uput[0](t){$\bid$}%
  %\uput[135](xy){$p$}
  %\uput{1pt}[135](xz){$q$}
  \uput[45](yz){$r$}%
  %\uput[-135](x){$a$}
  \uput[-45](y) {$b$}
  %\uput[-45](z) {$c$}%
  \uput[180](b) {$\bzero$}%
\end{pspicture}
}%\end{tabular}     \begin{tabular}{c}{\huge+}\end{tabular}
% \begin{tabular}{c}%============================================================================
% Daniel J. Greenhoe
% LaTeX file
% 
%============================================================================
{%\psset{unit=0.5\psunit}%
\begin{pspicture}(-1.5,-\latbot)(1.5,3.2)
  %---------------------------------
  % nodes
  %---------------------------------
  \Cnode(0,3){t}
  %\Cnode(-1,2){xy} 
  \Cnode(0,2){xz} 
  %\Cnode(1,2){yz}
  %\Cnode(-1,1){x}  
  %\Cnode(0,1){y}  
  \Cnode(1,1){z}
  \Cnode(0,0){b}
  %---------------------------------
  % node connections
  %---------------------------------
  %\ncline{t}{xy}
  \ncline{t}{xz}
  %\ncline{t}{yz}%
  %\ncline{x}{xz}
  %\ncline{x}{xy}%
  %\ncline{y}{xy}\ncline{y}{yz}%
  \ncline{z}{xz}%\ncline{z}{yz}%
  %\ncline{b}{x} %\ncline{b}{y} 
  \ncline{b}{z}%
  %---------------------------------
  % node labels
  %---------------------------------
  \uput[0](t){$\bid$}%
  %\uput[135](xy){$p$}
  \uput{1pt}[135](xz){$q$}
  %\uput[45](yz){$r$}%
  %\uput[-135](x){$a$}
  %\uput[-45](y) {$b$}
  \uput[-45](z) {$c$}%
  \uput[180](b) {$\bzero$}%
\end{pspicture}
}%\end{tabular}
% }
%\end{example}%
%}% end exclude mssa


%=======================================
\section{Bounds on ordered sets}
%=======================================
%\pref{chp:order} introduces the \hie{ordered set}.
In an \structe{ordered set} \xref{def:poset}, a pair of elements $\setn{x,y}$ may not be \prope{comparable}.
Despite this, we may still be able to find elements that {\em are}
comparable to both $x$ and $y$ and are ``{\em greater}" than both of them.
Such a greater element is called an \hie{upper bound} of $x$ and $y$.
There may be many elements that are upper bounds of $x$ and $y$.
But if one of these upper bounds is comparable with and is smaller than all the other upper bounds,
than this ``smallest" of the ``greater" elements is called the
\hie{least upper bound} (\hie{lub}) of $x$ and $y$, and is denoted $x \join y$ \xref{def:lub}.
Likewise, we may also be able to find elements that are
comparable to $\setn{x,y}$ and are ``{\em lesser}" than both of them.
Such a lesser element is called a \hie{lower bound} of $x$ and $y$.
If one of these lower bounds is comparable with and is larger than all the other lower bounds,
than this ``largest" of the ``lesser" elements is called the
\hie{greatest lower bound} (\prope{glb}) of $\setn{x,y}$ and is denoted $x \meet y$ \xref{def:glb}.
If every pair of elements in an ordered set has both a least upper bound and a greatest lower bound in the ordered set,
then that ordered set is a \structe{lattice} \xref{def:lattice}.

%---------------------------------------
\begin{definition}
\label{def:lub}\label{def:sup}\label{def:join}
\index{bound!least upper bound}
\index{bound!supremum}
%---------------------------------------
Let $\opair{\setX}{\orel}$ be an ordered set and $\psetx$ the power set of $\setX$.
\defboxp{\indxs{\join}\indxs{\supA}\indxs{\joinop\setA}
  For any set $\setA\in\psetx$, $c$ is an \vald{upper bound} of $\setA$ in $\opair{\setX}{\orel}$ if
  \\\indentx$\ds\begin{array}{Fll}
    1. & x \orel c & \forall x\in\setA .
  \end{array}$\\
  An element $b$ is the \vald{least upper bound}, or \vald{\lub}, of $\setA$ in $\opair{\setX}{\orel}$ if
  \\\indentx$\ds\begin{array}{FMcl}
    2. & $b$ and $c$ are \vale{upper bound}s of $\setA$ &\implies& b\orel c .
  \end{array}$\\
  %For any set $\setA\in\psetx$, $b$ is the \structd{least upper bound} of $\setA$ in $\opair{\setX}{\orel}$ if
  %\\\indentx$\ds\begin{array}{FlllDD}
  %  1. & x \orel b & \forall x\in\setA &                             & ($b$ is an upper bound of $\setA$) & and \\
  %  2. & x \orel c & \forall x\in\setA & \quad\implies\quad b\orel c & ($b$ is the least upper bound of $\setA$). &
  %\end{array}$
  \\
  The least upper bound of the set $\setA$ is denoted $\hxsd{\joinop\setA}$.
  It is also called the \structd{supremum} of $\setA$, which is denoted $\hxsd{\supA}$.
  The \opd{join} $x\hxsd{\join} y$ of $x$ and $y$ is defined as $x\join y\eqd\joinop\setn{x,y}$.
  }
\end{definition}

%---------------------------------------
\begin{definition}
\label{def:glb}
\label{def:inf}
\label{def:meet}
\index{bound!greatest lower bound}
\index{bound!infimum}
%---------------------------------------
Let $\opair{\setX}{\orel}$ be an ordered set and $\psetx$ the power set of $\setX$.
\defboxp{\indxs{\join}\indxs{\supA}\indxs{\joinop\setA}
  For any set $\setA\in\psetx$, $p$ is a \vald{lower bound} of $\setA$ in $\opair{\setX}{\orel}$ if
  \\\indentx$\ds\begin{array}{Fll}
    1. & p \orel x & \forall x\in\setA .
  \end{array}$\\
  An element $a$ is the \vald{greatest lower bound}, or \vald{\glb}, of $\setA$ in $\opair{\setX}{\orel}$ if
  \\\indentx$\ds\begin{array}{FMcl}
    2. & $a$ and $p$ are \vale{lower bound}s of $\setA$ &\implies& p\orel a .
  \end{array}$\\
  %For any set $\setA\in\psetx$, $b$ is the \structd{least upper bound} of $\setA$ in $\opair{\setX}{\orel}$ if
  %\\\indentx$\ds\begin{array}{FlllDD}
  %  1. & x \orel b & \forall x\in\setA &                             & ($b$ is an upper bound of $\setA$) & and \\
  %  2. & x \orel c & \forall x\in\setA & \quad\implies\quad b\orel c & ($b$ is the least upper bound of $\setA$). &
  %\end{array}$
  \\
  The greatest lower bound of the set $\setA$ is denoted $\hxsd{\meetop\setA}$.
  It is also called the \vald{infimum} of $\setA$, which is denoted $\hxsd{\infA}$.
  The \opd{meet} $x\hxsd{\meet} y$ of $x$ and $y$ is defined as $x\meet y\eqd\meetop\setn{x,y}$.
  }
\end{definition}

%---------------------------------------
\begin{definition}[least upper bound property]
\label{def:lsb_prop}
\footnote{
  \citerpg{pugh2002}{13}{0387952977},
  \citerpg{rudinp}{4}{007054235X}
  }
\index{completeness axiom}
\index{least upper bound property}
%---------------------------------------
Let $\setX$ be a set. Let $\supA$ be the supremum (least upper bound) of a set $\setA$.
\defbox{\begin{array}{M}
  A set $\setX$ satisfies the \propd{least upper bound property} if
  \\\indentx
  $\ds\brbr{\begin{array}{FlD}
    1. & \setA\subseteq\setX & and     \\
    2. & \setA \ne  \emptyset  & and \\
    3. & \exists b\in\setX \st \forall a\in\setA,\; a\le b  & \text{($\setA$ is bounded above in $\setX$)}
  \end{array}}
  \quad\implies\quad
  \exists\supA\in\setX
  $\\
  A set $\setX$ that satisfies the least upper bound property is also said to be \propd{complete}.
\end{array}}
\end{definition}

%---------------------------------------
\begin{proposition}
\label{prop:orel_meetjoin}
%---------------------------------------
Let $\latX$ be an \structe{ordered set} \xref{def:poset}.
\propbox{
  \begin{array}{rclclC}
    x &\orel& y &\iff& 
      \brb{\begin{array}{FrclCD}
        1. & x\meet y &=& x & and\\
        2. & x\join y &=& y 
      \end{array}}
  & \forall x,y\in\setX
  \end{array}
  }
\end{proposition}

%---------------------------------------
\begin{proposition}
\label{prop:orel_isotone}
%---------------------------------------
Let $\psetx$ be the \structe{power set} of a set $\setX$.
\propbox{
  \setA\subseteq\setB \implies 
  \brb{\begin{array}{FrclD}
    1. & \joinop\setA &\orel& \joinop\setB & and\\
    2. & \meetop\setA &\orel& \meetop\setB
  \end{array}}
  \qquad{\scy\forall \setA,\setB\in\psetx}
  }
\end{proposition}








%  %=======================================
%  \section{Irreflexive order relations}
%  %=======================================
%  A reflexive order relation $\relation$ includes the element
%  $(x,x)$, as in $x\orel x$.
%  However there are also order relations that are \emph{irreflexive}
%  (not reflexive).
%  For example, the order relation $<$ in $\Z\cprod\Z$ is
%  irreflexive such that we can never say $n<n$.
%  An irreflexive ordering is defined next.
%  %---------------------------------------
%  \begin{definition}
%  \label{def:order_rel_irreflexive}
%  \footnote{
%    \citerp{wolf}{188} \\
%    \citerpp{bogart}{442}{445}
%    }
%  \index{order relation!partial}
%  \index{order relation!irreflexive}
%  \index{irreflexive}
%  \index{antisymmetric}
%  \index{transitive}
%  \index{comparable}
%  %---------------------------------------
%  Let $\setX$ be a set.
%  \defbox{\begin{array}{l@{\qquad}l l@{\qquad}C@{\qquad}D}
%    \mc{5}{l}{\text{
%      An \hid{irreflexive partial order relation} on $\setX$ is any
%      relation $\sqsubset$ that satisfies
%      }}
%        \\
%    &1. & x \nsqsubset x
%        & \forall x\in\setX
%        & (irreflexive)
%        \\
%    &2. & x \sqsubset y \text{ and } x\neq y \implies y\nsqsubset x
%        & \forall x,y\in\setX
%        & (antisymmetric)
%        \\
%    &3. & x \sqsubset y \text{ and } y \sqsubset z \implies x \sqsubset z
%        & \forall x,y,z
%        & (transitive)
%  \end{array}}
%  \end{definition}
%  
%  
%  And as one might expect, we can have irreflexive totally ordered sets as well,
%  as illustrated in the next example.
%  %---------------------------------------
%  \begin{example}
%  \citep{wolf}{183}
%  %---------------------------------------
%  \exbox{\begin{array}{l@{\qquad}l@{\qquad}D}
%    \mc{3}{l}{\text{Examples of \prop{irreflexive totally ordered set}s include}} \\
%    &(\Z,<) & (\prop{integers})\\
%    &(\Q,<) & (\prop{rational numbers}) \\
%    &(\R,<) & (\prop{real numbers})
%  \end{array}}
%  \end{example}
%  
%  
%  In this text, normally {\bf we will avoid irreflexive order relations
%  and only use reflexive order relations}.





