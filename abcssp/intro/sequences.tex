%============================================================================
% LaTeX File
% Daniel J. Greenhoe
%============================================================================

%======================================
\subsection{Sequences}
%======================================
%--------------------------------------
\begin{definition}
\footnote{
  %\citerp{bromwich1908}{1},
  %\citerpgc{thomson2008}{23}{143484367X}{Definition 2.1},
  %\citerpg{joshi1997}{31}{8122408265},
  \citePc{simmons2016}{``Formal definition"}
  }
\label{def:sequence}
\label{def:seq}
%\label{def:tuple}
%--------------------------------------
%Let $\clFyx$ be the set of all functions from a set $\setY$ to a set $\setX$.
%Let $\hxs{\Z}$ be the \structe{set of integers}.
\defboxp{
  %2014jan05 A function $\ff$ in $\clFyx$ is a \structd{sequence} over $\setX$ if\quad $\setY=\Z$.\\
  %A function $\ff$ in $\clFyx$ \xref{def:function} is an $\setX$-valued \structd{sequence} if\quad $\setY=\Z$ \xref{def:Z}.
  %
  A \structe{function} in $\clF{\Dom}{\setX}$ \xref{def:function} is an $\setX$-valued \structd{sequence} if
  $\Dom\neq\emptyset$ and $\Dom$ is a \prope{convex} \xref{def:convex} subset of $\Z$.
  %
  %A function $\ff$ in $\clF{\Dom}{\setX}$ \xref{def:function} is an $\setX$-valued \structd{sequence} if
  %$\Dom\neq\emptyset$ and $\Dom$ is an \structe{open interval} %$\prefpp{def:intcc}$
  %in $\Z$.
  %
  A sequence may be denoted in the form $\ds\seqnD{x_n}$, or simply as $\ds\seqn{x_n}$.
  %
  %2014jan05 A function $\ff$ in $\clFyx$ is an \hid{n-tuple} over $\setX$ if\quad $\setY=\setn{1,2,\ldots,\xN}$.\\
  %A function $\ff$ in $\clFyx$ is an $\setX$-valued \hid{n-tuple} if\quad $\setY=\setn{1,2,\ldots,\xN}$.\\
  %An n-tuple may be denoted in the form $\ds\tuplexn{x_n}$ or simply as $\ds\tuplen{x_n}$.
  }
\end{definition}

%---------------------------------------
\begin{definition}
\label{def:downsample}
%---------------------------------------
The sequence $\seq{y_n}{\Dom_2}$ is the sequence $\seq{x_n}{\Dom_1}$ \opd{down sampled by a factor of $\xM$},
where $\xM\in\Zp$, if $n\in\Dom_2\iff\xM n\in\Dom_1$ and 
$y_n=x_{\xM n}\quad{\scy\forall n\in\Dom_2}$.
\end{definition}

%---------------------------------------
\begin{definition}
\label{def:conv}
\label{def:convf}
%---------------------------------------
Let $\oplus$ be the \ope{addition operator} \xref{def:add}
and $\otimes$   the \ope{multiplication operator} \xref{def:mpy}.
Let $\Dom_1$ and $\Dom_2$ be \structe{convex subset}s of $\Z$.
\\Let $\Dom\eqd\intoo{\meetop\Dom_1+\meetop\Dom_2-1}{\joinop\Dom_1+\joinop\Dom_2+1}.$
\\Let $\seq{x_n}{\Dom_1}$ be a \structe{sequence} over a \structe{field} $\F_1$
  and $\seq{y_n}{\Dom_2}$    a \structe{sequence} over a \structe{field} $\F_2$.
\\\defboxp{
  The \opd{convolution} $\seq{z_n}{\Dom}\eqd\seq{x_n}{\Dom_1}\conv\seq{y_n}{\Dom_2}$ 
  of $\seqn{x_n}$ and $\seqn{y_n}$ is defined as 
  \\$\ds
    z_n \eqd \bigoplus_{m\in\Dom_1} \ff(n,m)$
  \quad where $\ff$ is defined as\quad
  $
  \ff(n,m) \eqd \brb{\begin{array}{lM}
    x_m \otimes y_{n-m} & if $m\in\Dom_1$ and $(n-m)\in\Dom_2$\\
    0                   & otherwise
  \end{array}}\quad{\scy\forall n,m\in\Dom}
  $
  }
\end{definition}

%---------------------------------------
\begin{proposition}
\label{prop:convN}
%---------------------------------------
Let $\seqn{x_n}$ and $\seqn{y_n}$ be finite \structe{sequence}s with 
lengths $\xN$ and $\xM$, respectively.\\
Then the length of $\seqn{x_n}\conv\seqn{y_n}$ is\quad$\xN+\xM-1$.
\end{proposition}

%--------------------------------------
\begin{example}
\footnote{historical references:
  \citerc{cauchy1821}{Chapter IV},
  \citerpgc{apostol1975}{204}{0201002884}{note that convolution is a single element in a series that is the ``Cauchy product"},
  \citePpc{dominguez2010}{20}{section 4.2: connection to the work of Cauchy},
  \citePc{dominguez2015}{history of the continuous convolution operation}
  }
%--------------------------------------
Let $\seqnZ{x_n}$ and $\seqnZ{y_n}$ be \structe{sequence}s over a \structe{field} $\Field$.\\
Then the \structe{domain} $\Dom$ of the \ope{convolution} $\seqnD{z_n}\eqd\seqn{x_n}\conv\seqn{y_n}$ is\\ 
$\Dom\eqd\intoo{\meetop\Z+\meetop\Z-1}{\joinop\Z+\joinop\Z+1}=\Z$ and $\ds z_n \eqd \sum_{m\in\Z} x_m y_{n-m}$.
\end{example}

%--------------------------------------
\begin{example}
%--------------------------------------
Let $\seq{x_n}{\intcc{0}{1}}\eqd\seqn{1,2}$ 
and $\seq{y_n}{\intcc{0}{2}}\eqd\seqn{10,20,50}$ be \structe{sequence}s over the \structe{field} $\fieldR$.\\
Then the \structe{domain} $\Dom$ of the \ope{convolution} $\seqnD{z_n}\eqd\seqn{x_n}\conv\seqn{y_n}$ is
\\\indentx$\Dom\eqd\intoo{0+0-1}{1+2+1}=\intcc{0}{3}=\setn{0,1,2,3}$ and 
\begin{align*}
  \seqnD{z_n} 
    &\eqd \seq{\sum_{m\in\setn{0,1}}\f(0,m),\, \sum_{m\in\setn{0,1}}\f(1,m),\, \sum_{m\in\setn{0,1}}\f(2,m),\, \sum_{m\in\setn{0,1}}\f(3,m)}{\setn{0,1,2,3}}
  \\&\eqd \seq{(1\times10+0),\, (1\times20+2\times10),\, (1\times50+2\times20),\, (0+2\times50)}{\setn{0,1,2,3}}
  \\&=    \seq{\underbrace{ 10}_{z_0},\,
               \underbrace{ 40}_{z_1},\,
               \underbrace{ 90}_{z_2},\,
               \underbrace{100}_{z_3}
              }{\setn{0,1,2,3}}
\end{align*}
\end{example}

%--------------------------------------
\begin{example}
%--------------------------------------
Let $\seq{x_n}{\intcc{0}{1}}\eqd\seqn{1,2}$ 
and $\seq{y_n}{\intcc{3}{5}}\eqd\seqn{10,20,50}$ be \structe{sequence}s over the field $\fieldR$.
Then the \structe{domain} $\Dom$ of the \ope{convolution} $\seqnD{z_n}\eqd\seqn{x_n}\conv\seqn{y_n}$ is
\\\indentx$\Dom\eqd\intoo{0+3-1}{1+5+1}=\intcc{3}{6}=\setn{3,4,5,6}$ and
\begin{align*}
  \seqnD{z_n} 
    &\eqd \seq{\sum_{m\in\setn{0,1}}\f(3,m),\, \sum_{m\in\setn{0,1}}\f(4,m),\, \sum_{m\in\setn{0,1}}\f(5,m),\, \sum_{m\in\setn{0,1}}\f(6,m)}{\setn{3,4,5,6}}
  \\&\eqd \seq{(1\times10+0),\, (1\times20+2\times10),\, (1\times50+2\times20),\, (0+2\times50)}{\setn{3,4,5,6}}
  \\&=    \seq{\underbrace{ 10}_{z_3},\,
               \underbrace{ 40}_{z_4},\,
               \underbrace{ 90}_{z_5},\,
               \underbrace{100}_{z_6}
              }{\setn{3,4,5,6}}
\end{align*}
\end{example}

%%--------------------------------------
%\begin{example}
%%--------------------------------------
%Let $\seq{x_n}{\intcc{0}{1}}\eqd\seqn{5,9}$ 
%and $\seq{y_n}{\intcc{0}{2}}\eqd\seqn{\opair{1}{11},\opair{2}{22},\opair{3}{33}}$ be \structe{sequence}s.
%\\Then the \structe{domain} $\Dom$ of the \ope{convolution} $\seqnD{z_n}\eqd\seqn{x_n}\conv\seqn{y_n}$ 
%is %$\Dom\eqd\intoo{0+0-1}{1+2+1}=\intcc{0}{3}=\setn{0,1,2,3}$ 
%$\Dom=\setn{0,1,2,3}$ 
%and 
%\begin{align*}
%  \seqnD{z_n} 
%    &\eqd \seq{\bigoplus_{m\in\setn{0,1}}\f(0,m),\, \bigoplus_{m\in\setn{0,1}}\f(1,m),\, \bigoplus_{m\in\setn{0,1}}\f(2,m),\, \bigoplus_{m\in\setn{0,1}}\f(3,m)}{\setn{0,1,2,3}}
%  \\&\eqd \seq{\brs{5\otimes\opair{1}{11}\oplus0},\, \brs{5\otimes\opair{2}{22}\oplus9\otimes\opair{1}{11}},\, \brs{5\otimes\opair{3}{33}\oplus9\otimes\opair{2}{22}},\, \brs{0\oplus9\otimes\opair{3}{33}}}{\setn{0,1,2,3}}
%  \\&\eqd \seq{\brs{\opair{5}{55}},\, \brs{\opair{10}{110}\oplus\opair{9}{99}},\, \brs{\opair{5}{165}\oplus\opair{18}{198}},\, \brs{\opair{27}{297}}}{\setn{0,1,2,3}}
%  \\&=    \seq{\underbrace{\opair{ 5}{ 55}}_{z_0},\,
%               \underbrace{\opair{19}{209}}_{z_1},\,
%               \underbrace{\opair{23}{363}}_{z_2},\,
%               \underbrace{\opair{27}{297}}_{z_3}
%              }{\setn{0,1,2,3}}
%\end{align*}
%\end{example}

%--------------------------------------
\begin{example}
%--------------------------------------
Let $\seq{x_n}{\intcc{0}{1}}\eqd\seqn{1,2}$ 
and $\seq{y_n}{\intcc{0}{2}}\eqd\seqn{\opair{3}{4},\opair{5}{6},\opair{7}{8}}$ be \structe{sequence}s.
\\Then the \structe{domain} $\Dom$ of the \ope{convolution} $\seqnD{z_n}\eqd\seqn{x_n}\conv\seqn{y_n}$ 
is %$\Dom\eqd\intoo{0+0-1}{1+2+1}=\intcc{0}{3}=\setn{0,1,2,3}$ 
$\Dom=\setn{0,1,2,3}$ 
and 
\begin{align*}
  \seqnD{z_n} 
    &\eqd \seq{\bigoplus_{m\in\setn{0,1}}\f(0,m),\, \bigoplus_{m\in\setn{0,1}}\f(1,m),\, \bigoplus_{m\in\setn{0,1}}\f(2,m),\, \bigoplus_{m\in\setn{0,1}}\f(3,m)}{\setn{0,1,2,3}}
  \\&\eqd \seq{\brs{1\otimes\opair{3}{4}\oplus0},\, \brs{1\otimes\opair{5}{6}\oplus2\otimes\opair{3}{4}},\, \brs{1\otimes\opair{7}{8}\oplus2\otimes\opair{5}{6}},\, \brs{0\oplus2\otimes\opair{7}{8}}}{\setn{0,1,2,3}}
  \\&\eqd \seq{\brs{\opair{3}{4}},\, \brs{\opair{5}{6}\oplus\opair{6}{8}},\, \brs{\opair{7}{8}\oplus\opair{10}{12}},\, \brs{\opair{14}{16}}}{\setn{0,1,2,3}}
  \\&=    \seq{\underbrace{\opair{ 3}{  4}}_{z_0},\,
               \underbrace{\opair{11}{ 14}}_{z_1},\,
               \underbrace{\opair{17}{ 20}}_{z_2},\,
               \underbrace{\opair{14}{ 16}}_{z_3}
              }{\setn{0,1,2,3}}
\end{align*}
\end{example}

%%--------------------------------------
%\begin{example}
%%--------------------------------------
%Let $\seq{x_n}{\intcc{0}{1}}\eqd\seqn{1,2}$ 
%and $\seq{y_n}{\intcc{0}{2}}\eqd\seqn{\opair{1}{11},\opair{2}{22},\opair{3}{35}}$ be \structe{sequence}s.
%\\Then the \structe{domain} $\Dom$ of the \ope{convolution} $\seqnD{z_n}\eqd\seqn{x_n}\conv\seqn{y_n}$ 
%is %$\Dom\eqd\intoo{0+0-1}{1+2+1}=\intcc{0}{3}=\setn{0,1,2,3}$ 
%$\Dom=\setn{0,1,2,3}$ 
%and 
%\begin{align*}
%  \seqnD{z_n} 
%    &\eqd \seq{\bigoplus_{m\in\setn{0,1}}\f(0,m),\, \bigoplus_{m\in\setn{0,1}}\f(1,m),\, \bigoplus_{m\in\setn{0,1}}\f(2,m),\, \bigoplus_{m\in\setn{0,1}}\f(3,m)}{\setn{0,1,2,3}}
%  \\&\eqd \seq{\brs{1\otimes\opair{1}{11}\oplus0},\, \brs{1\otimes\opair{2}{22}\oplus2\otimes\opair{1}{11}},\, \brs{1\otimes\opair{3}{33}\oplus2\otimes\opair{2}{22}},\, \brs{0\oplus2\otimes\opair{3}{35}}}{\setn{0,1,2,3}}
%  \\&\eqd \seq{\brs{\opair{1}{11}},\, \brs{\opair{2}{22}\oplus\opair{2}{22}},\, \brs{\opair{3}{35}\oplus\opair{4}{44}},\, \brs{\opair{6}{70}}}{\setn{0,1,2,3}}
%  \\&=    \seq{\underbrace{\opair{ 1}{11}}_{z_0},\,
%               \underbrace{\opair{4}{44}}_{z_1},\,
%               \underbrace{\opair{7}{79}}_{z_2},\,
%               \underbrace{\opair{6}{70}}_{z_3}
%              }{\setn{0,1,2,3}}
%\end{align*}
%\end{example}

%--------------------------------------
\begin{definition}
\label{def:axn}
%--------------------------------------
Let $\seqnD{x_n}$ and $\seqnD{y_n}$ be sequences over a \structe{field} $\F\eqd\fieldX$, and $\alpha$ an element in $\F$.
The operations $\alpha+\seqn{x_n}$, $\seqn{x_n}+\alpha$, 
               $\alpha\seqn{x_n}$, and $\seqn{x_n}\alpha$ are defined as
\defbox{\begin{array}{rclclCD}
  \alpha+\seqnD{x_n} &\eqd& \seqnD{x_n}+\alpha &\eqd& \seqnD{x_n+\alpha}  & \forall \alpha\in\F & and\\
  \alpha\seqnD{x_n}  &\eqd& \seqnD{x_n}\alpha  &\eqd& \seqnD{\alpha x_n}  & \forall \alpha\in\F & .
\end{array}}
\end{definition}



