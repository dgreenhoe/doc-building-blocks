%============================================================================
% Daniel J. Greenhoe
% XeLaTeX file
%============================================================================

%=======================================
\chapter{Modular Lattices}
%=======================================
%=======================================
\section{Modular relation}
%=======================================
%---------------------------------------
\begin{definition}
\citetbl{
  \citerpg{stern1999}{11}{0521461057},
  \citerpc{maeda1970}{1}{Definition (1.1)},
  \citorp{maeda1966}{248}
  }
\label{def:lat_Mrel}
%---------------------------------------
Let $\latticed$ be a lattice.
Let $\clRxx$ be the set of all \structe{relation}s in $\setX^2$.
\defboxp{%
  The \relxd{modularity}      relation $\modular \in\clRxx$ and
  the \relxd{dual modularity} relation $\modulard\in\clRxx$ are defined as
  \\$\begin{array}{@{\qquad}lcl}
    x \symxd{\modular}  y &\iffdef& \set{\opair{x}{y}\in\setX^2}{a \orel  y \quad\implies\quad y \meet(x\join a) = (y\meet x) \join a \quad \forall a\in\setX} \\
    x \symxd{\modulard} y &\iffdef& \set{\opair{x}{y}\in\setX^2}{a \oreld y \quad\implies\quad y \join(x\meet a) = (y\join x) \meet a \quad \forall a\in\setX}. 
  \end{array}$
  \\
  A pair $\opair{x}{y}\in\modular$  is alternatively denoted as 
  $\opair{x}{y}\modular$,
  and is called a \propd{modular} pair.
  A pair $\opair{x}{y}\in\modulard$  is alternatively denoted as
  $\opair{x}{y}\modulard$,
  and is called a \propd{dual modular} pair.
  A pair $\opair{x}{y}$ that is \emph{not} a modular pair 
  ($\opair{x}{y}\notin\modular$) is denoted $x\symxd{\notmodular} y$.
  A pair $\opair{x}{y}$ that is \emph{not} a 
  dual modular pair is denoted $x\symxd{\notmodulard} y$.
  }
\end{definition}

%---------------------------------------
\begin{proposition}
\citetbl{
  \citerpc{maeda1970}{1}{Lemma (1.2)}
  }
\label{prop:latm_xmy}
%---------------------------------------
Let $\latL\eqd\latticed$ be a lattice.
\propbox{
  \brb{x \modular y \quad\iff\quad x \modulard y} \qquad\sst\forall x,y\in\setX
  }
\end{proposition}
\begin{proof}
\begin{align*}
  x \modular y
    &\iff \brb{a\orel  y \quad\implies\quad y \meet (x\join a)=(y\meet x)\join a \quad \forall a\in\setX}
    && \text{by definition of $\modular$ \xref{def:lat_Mrel}}
  \\&\iff \brb{a\oreld y \quad\implies\quad a \meet (x\join y)=(a\meet x)\join y \quad \forall a\in\setX}
    && \text{by definition of $\oreld$ \xref{def:oreld}}
  \\&\iff \brb{a\oreld y \quad\implies\quad (a\meet x)\join y = a \meet (x\join y) \quad \forall a\in\setX}
    && \text{by \prope{symmetric} property of $=$ \ifxref{relation}{def:eq_rel}}
  \\&\iff \brb{a\oreld y \quad\implies\quad y\join(x\meet a) = (y\join x)\meet a \quad \forall a\in\setX}
    && \text{by \prope{commutative} prop. of lat. \xref{thm:lattice}}
  \\&\iff x \modulard y
    && \text{by definition of $\modulard$ \xref{def:lat_Mrel}}
\end{align*}
\end{proof}

%---------------------------------------
\begin{proposition}
\citetbl{
  \citerp{maeda1970}{1}
  }
\label{prop:latm_xyxmx}
%---------------------------------------
Let $\latL\eqd\latticed$ be a lattice.
\propbox{
  \mcom{\brbr{\begin{array}{c@{\quad}D}
    x\orel y & or \\
    y\orel x
  \end{array}}}{$x,y$ are \prope{comparable}}
  \qquad\implies\qquad
  \brbl{\begin{array}{clc@{\quad}D}
    x &\modular &  y & and \\ 
    y &\modular &  x & and \\
    x &\modulard&  y & and \\
    y &\modulard&  x.& 
  \end{array}}
  }
\end{proposition}
\begin{proof}
\begin{align*}
  x\orel y
    &\implies \brb{a \orel  y \implies y \meet(x\join a) = x \join a = (y\meet x) \join a \quad \forall a\in\setX}
  \\&\iff x\modular y
    \qquad\text{by definition of $\modular$ \xref{def:lat_Mrel}}
    %&& \text{by \prefp{def:lat_Mrel}}
  \\
  x\orel y
    &\implies \brb{a \orel  x \implies x \meet(y\join a) = x = x\join a = (x\meet y) \join a \quad \forall a\in\setX}
  \\&\iff y\modular x
    \qquad\text{by definition of $\modular$ \xref{def:lat_Mrel}}
  \\
  x\orel y
    &\implies x\modulard y
    \qquad\text{because $x\orel y\implies x\modular y$ and by \prefp{prop:latm_xmy}}
  \\
  x\orel y
    &\implies y\modulard x
    \qquad\text{because $x\orel y\implies y\modular x$ and by \prefp{prop:latm_xmy}}
\end{align*}
\end{proof}

%---------------------------------------
\begin{proposition}
\label{prop:relM_reflexive}
%---------------------------------------
Let $\latL\eqd\latticed$ be a lattice.
\propbox{
  \begin{array}{clc@{\qquad}C@{\qquad}D}
    x &\modular  & x & \forall x\in\setX & ($\modular$ is \prope{reflexive}) \\
    x &\modulard & x & \forall x\in\setX & ($\modulard$ is \prope{reflexive}) 
  \end{array}
  }
\end{proposition}
\begin{proof}
Because $x\orel x$ and by \prefpp{prop:latm_xyxmx}.
\end{proof}





%=======================================
\section{Semimodular lattices}
%=======================================
%---------------------------------------
\begin{definition}
\citetbl{
  \citerpc{maeda1970}{3}{Definition (1.7)}
  }
\label{def:latsm}
%---------------------------------------
\defboxt{
  A lattice $\latticed$ is \propd{semimodular} if
  \\\indentx
  $x\modular y \quad\implies\quad y\modular x$
  \\
  A semimodular lattice is also called \propd{M-symmetric}.
  }
\end{definition}


%=======================================
\section{Modular lattices}
%=======================================
Modular lattices are a generalization of the distributive lattice in the sense that
all distributive lattices are modular,
but not equivalent because not all modular lattices are distributive
\xref{thm:lat_dis=>mod}.

%---------------------------------------
\begin{definition}
\citetbl{
  \citerpg{birkhoff1967}{82}{0821810251},
  \citerpc{maeda1970}{3}{Definition (1.7)}
  %\citerpg{maclane1999}{479}{0821816462}\\
  %\citerpg{burris1981}{11}{0387905782}\\
  %\cithrp{burris2000}{13}
  }
\label{def:lat_mod}
\label{def:latm}
%---------------------------------------
\defboxt{
  A lattice $\latticed$ is \propd{modular} if
  \\\indentx
  $x\modular y \qquad \forall x,y\in\setX.$
  }
\end{definition}


%=======================================
\subsection{Characterizations}
%=======================================
This section describes some characterizations of modular lattices---
that is, sets of properties that are equivalent to the definition of modular lattices
\xref{def:latm}:
  \\\begin{tabular}{@{\qquad}Nllll}
      \imark & Ore 1935      & (order     characterization) & \pref{thm:latm_id}         & \prefpo{thm:latm_id}
    \\\imark & N5 lattice    & (order     characterization) & \pref{thm:latm_n5}  & \prefpo{thm:latm_n5}
    \\\imark & Riecan 1957   & (algebraic characterization) & \pref{thm:lat_char_r1957}  & \prefpo{thm:lat_char_r1957}
  \end{tabular}\\
Alternatively, any of the sets of properties listed in this section could be
used as the definition of modular lattices and the definition
would in turn become a theorem/proposition.

%=======================================
\subsubsection{Order characterizations}
%=======================================
%---------------------------------------
\begin{theorem}
\citetbl{
  \citerpg{padmanabhan2008}{39}{9812834540},
  \citorpc{ore1935}{413}{(2)}
  }
\label{thm:latm_id}
%---------------------------------------
Let $\latL\eqd\latticed$ be a lattice.
\thmbox{
  \begin{array}{M@{\qquad}c@{\qquad}lcl@{\qquad}C}
    $\latL$ is \prope{modular} 
      & \iff & \mc{3}{l}{\brb{x\orel y \quad\implies\quad x \join (z\meet y) = (x\join z)\meet y}} & \forall x,y,z\in\setX \\
      & \iff & x \join \brs{\brp{x\join y}\meet z} &=& \brp{x\join y}\meet\brp{x\join z} & \forall x,y,z\in\setX \\
      & \iff & x \meet \brs{\brp{x\meet y}\join z} &=& \brp{x\meet y}\join\brp{x\meet z} & \forall x,y,z\in\setX \\
  \end{array}
  }
\end{theorem}
\begin{proof}
\begin{enumerate}
  \item Proof that $\latL$ is \prope{modular} $\iff$ $\brb{x\orel y \quad\implies\quad x \join (z\meet y) = (x\join z)\meet y}$:
        \label{item:latm_id_def}
    \begin{align*}
      \brb{\text{$\latL$ is \prope{modular}}}
        &\iff \brb{x\orel y \quad\implies\quad y\meet(z\join x) = (y\meet z)\join x \quad \forall x,y,z\in\setX}
        && \text{by \prefp{def:latm}}
      \\&\iff \brb{a\orel y \quad\implies\quad y\meet(x\join a) = (y\meet x)\join a \quad \forall x,y,a\in\setX}
        && \text{by change of variables}
      \\&\iff \brb{x\modular y \quad \sst \forall x,y\in\setX}
        && \text{by \prefp{def:lat_Mrel}}
    \end{align*}

  \item Proof that $\latL$ is \prope{modular} $\iff$ $x \join \brs{\brp{x\join y}\meet z} = \brp{x\join y}\meet\brp{x\join z}$:
    \begin{enumerate}
      \item Proof that $\latL$ is \prope{modular} $\implies$ $x \join \brs{\brp{x\join y}\meet z} = \brp{x\join y}\meet\brp{x\join z}$:
        \\First note that $x \orel x \join y$.
        \begin{align*}
          x \join \brs{\brp{x\join y}\meet z}
            &= \left.x \join \brp{u\meet z} \right|_{u\eqd x\join y}
            && \text{by substitution $u\eqd x\join y$}
          \\&= \left.u\meet \brp{x \join z} \right|_{u\eqd x\join y}
            && \text{by modularity hypothesis}
          \\&= \brp{x \join y}\meet\brp{x\join z}
            && \text{because $u\eqd x\join y$}
        \end{align*}

      \item Proof that $\latL$ is \prope{modular} $\impliedby$ $x \join \brs{\brp{x\join y}\meet z} = \brp{x\join y}\meet\brp{x\join z}$:
        \begin{align*}
          x\orel y \;\implies\; x \join (y\meet z)
            &= x \join (y\meet z)
            && \text{by right hypothesis and $x\orel y$}
          \\&= x \join (z\meet y)
            && \text{by \prope{commutative} property \prefp{thm:lattice}}
          \\&= x \join \brs{z\meet \brp{x\join y}}
            && \text{because $x\orel y$}
          \\&= x \join \brs{\brp{x\join y}\meet z}
            && \text{by \prope{commutative} property \prefp{thm:lattice}}
          \\&= \brp{x\join y}\meet\brp{x\join z}
            && \text{by right hypothesis}
          \\&= y \meet \brp{x\join z}
            && \text{because $x\orel y$}
        \end{align*}
    \end{enumerate}

  \item Proof that $\latL$ is modular $\iff$
           $\brb{y\orel x \implies x\meet(y\join z) = y \join (x\meet z) }$:
        \begin{align*}
          \text{$\latL$ is modular}
            &\iff \mcom{\brb{x\orel y \implies x \join (y\meet z) = y\meet(x\join z)}}{modularity definition \xref{def:latm}}
            &&    \text{by definition of modular \prefpo{def:latm}}
          \\&\iff \brb{y\orel x \implies y \join (x\meet z) = x\meet(y\join z)}
            &&    \text{by change of variables: $x\leftrightarrow y$}
          \\&\iff \mcom{\brb{y\orel x \implies x\meet(y\join z) = y \join (x\meet z) }}{dual of \pref{def:latm}}
            &&    \text{by \prope{reflexive} property of $=$ \ifxref{relation}{def:eq_rel}}
        \end{align*}

  \item Proof that
          $\brb{y\orel x \implies x\meet(y\join z) = y \join (x\meet z) }
          \iff
          \brb{x \meet \brs{\brp{x\meet y}\join z} = \brp{x\meet y}\join\brp{x\meet z}}$:
    \begin{enumerate}
      \item Proof that
              $\brb{y\orel x \implies x\meet(y\join z) = y \join (x\meet z) }
              \implies
              \brb{x \meet \brs{\brp{x\meet y}\join z} = \brp{x\meet y}\join\brp{x\meet z}}$:
        \\First note that $x\meet y \orel x$.
        \begin{align*}
          x \meet \brs{\brp{x\meet y}\join z}
            &= \left.x \meet \brp{u\join z} \right|_{u\eqd x\meet y}
            && \text{by substitution $u\eqd x\meet y$}
          \\&= \left.u\join \brp{x \meet z} \right|_{u\eqd x\meet y}
            && \text{by left hypothesis}
          \\&= \brp{x \meet y}\join\brp{x\meet z}
            && \text{because $u\eqd x\meet y$}
        \end{align*}

      \item Proof that
              $\brb{y\orel x \implies x\meet(y\join z) = y \join (x\meet z) }
              \impliedby
              \brb{x \meet \brs{\brp{x\meet y}\join z} = \brp{x\meet y}\join\brp{x\meet z}}$:
        \begin{align*}
          y\orel x \;\implies\; x \meet (y\join z)
            &= x \meet (z\join y)
            && \text{by \prope{commutative} property \prefp{thm:lattice}}
          \\&= x \meet \brs{z\join \brp{x\meet y}}
            && \text{because $y\orel x$}
          \\&= x \meet \brs{\brp{x\meet y}\join z}
            && \text{by \prope{commutative} property \prefp{thm:lattice}}
          \\&= \brp{x\meet y}\join\brp{x\meet z}
            && \text{by right hypothesis}
          \\&= y \join \brp{x\meet z}
            && \text{because $y\orel x$}
        \end{align*}
    \end{enumerate}

\end{enumerate}
\end{proof}



\begin{minipage}{\tw-33mm}
%---------------------------------------
\begin{definition}[N5 lattice/pentagon]
\footnotemark
\label{def:lat_N5}
\label{def:n5}
\index{lattice!N5}
%---------------------------------------
\defboxt{
  The \structd{N5 lattice} is the ordered set $\opair{\setn{0,a,b,p,1}}{\orel}$
  with cover relation
  \\\indentx$\coverrel = \setn{\opair{0}{a},\, \opair{a}{b},\, \opair{b}{1},\, \opair{p}{1},\, \opair{0}{p}}$.
  %\\It is also as defined by the \structe{Hasse diagram} to the right.
  \\The N5 lattice is also called the \structd{pentagon}.
  }
%The N5 lattice is illustrated by the Hasse diagram to the right.
\end{definition}
\end{minipage}%
\citetblt{%
  \citerppg{beran1985}{12}{13}{902771715X},
  \citePppc{dedekind1900}{391}{392}{(44) and (45)}
  }%
\hspace{5mm}\tbox{\includegraphics{graphics/lat5_n5_0abp1.pdf}}

%---------------------------------------
\begin{lemma}
\label{lem:latm_N5}
\citetbl{
  \citerpg{burris1981}{11}{0387905782}
  }
\index{lattice!N5}
\index{N5 lattice}
%---------------------------------------
\lemboxt{The $N5$ lattice (\hi{pentagon} lattice) is \prope{non-modular}.}
\end{lemma}
\begin{proof}
  \begin{align*}
    x\orel y \implies y \meet ( z\join x)
      &= y \meet b
      && \text{by \prefp{def:lub} (lub)}
    \\&= y
      && \text{by \prefp{def:glb} (glb)}
    \\&\ne  x
    \\&= x \join a
      && \text{by \prefp{def:lub} (lub)}
    \\&= (y \meet z) \join x
      && \text{by \prefp{def:lub} (lub)}
  \end{align*}
\end{proof}

%---------------------------------------
\begin{theorem}
\label{thm:latm_n5}
\label{thm:lat_mod_iff_N5}
\footnote{
    \citerpg{burris1981}{11}{0387905782},
  \citerpg{gratzer1971}{70}{0716704420},
  %\\\cithrp{burris2000}{14}
  %\citerpg{maclane1999}{480}{0821816462}
  \citePc{dedekind1900}{cf Stern 1999 page 10}
  }
\index{lattice!N5}
\index{N5 lattice}
%---------------------------------------
Let $\latL$ be a \structe{lattice} \xref{def:lattice}.
\thmbox{\begin{array}{McM}%
  $\latL$ is \prope{modular}
  &\iff&
  $\latL$ does \emph{not} contain $N5$ as a sublattice.
  %&{\psset{unit=3mm}%============================================================================
% Daniel J. Greenhoe
% LaTeX file
% lattice N5
% nominal unit = 10mm
%============================================================================
{%
\begin{pspicture}(-1.3,-\latbot)(1.3,3.3)
  %---------------------------------
  % nodes
  %---------------------------------
  \Cnode(0,3){t}
  \Cnode(-1,2){d}%
  \Cnode( 1,1.5){m}%
  \Cnode(-1,1){c}
  \Cnode(0,0){b}
  %---------------------------------
  % node connections
  %---------------------------------
  \ncline{t}{d}\ncline{t}{m}%
  \ncline{c}{d}%
  \ncline{b}{c}\ncline{b}{m}%
  %---------------------------------
  % node labels
  %---------------------------------
  %\uput[ 90](t) {$\setn{x,y,z}$}%
  %\uput[180](xy){$\setn{x,y}$}%   
  %\uput[0](yz){$\setn{y,z}$}%
  %\uput[180](x) {$\setn{x}$}%     
  %\uput[0](z) {$\setn{z}$}%
  %\uput[-90](b) {$\szero$}%
  %\uput[0](100,300){\rnode{xzlabel}{$\setn{x,z}$}}% 
  %\uput[0](100,  0){\rnode{ylabel}{$\setn{y}$}}%
  %\ncline[linestyle=dotted,nodesep=1pt]{->}{xzlabel}{xz}%
  %\ncline[linestyle=dotted,nodesep=1pt]{->}{ylabel}{y}%
\end{pspicture}
}%}
\end{array}
\qquad
\tbox{\includegraphics[height=12mm]{graphics/lat5_n5.pdf}}
}
%\thmbox{
%  \text{$\latL$ is \prope{modular}}
%  \qquad\iff\qquad
%  \text{$\latL$ does \emph{not} contain $N5$ as a sublattice}
%    \setlength{\unitlength}{0.02mm}
%    \begin{picture}(300,340)(-150,0)
%      %\graphpaper[10](0,0)(600,200)
%      \thinlines
%      \color{latline}
%        \qbezier(-100,200)(-50,250)(0,300)%
%        \qbezier(100,150)(50,75)(0,0)%
%        \qbezier(100,150)(50,225)(0,300)%
%        \qbezier(-100,100)(-100,150)(-100,200)%
%        \qbezier(0,0)(-50,50)(-100,100)%
%        \qbezier(0,0)(50,75)(100,150)%
%      \color{latdot}%
%        \put(   0, 300){\circle*{60}}%
%        \put(-100, 200){\circle*{60}}%
%        \put( 100, 150){\circle*{60}}%
%        \put(-100, 100){\circle*{60}}%
%        \put(   0,   0){\circle*{60}}%
%    \end{picture}
%}
\end{theorem}
\begin{proof}
\begin{enumerate}
  \item Proof that $\latL$ is modular $\implies$ $\latL$ does \emph{not} contain $N5$:\\
    This is because $N5$ is a non-modular lattice. Proof: \prefp{lem:latm_N5}

  \item Proof that $\latL$ does not contain $N5$ $\implies$ $\latL$ is modular:
    \begin{enumerate}
      \item In what follows, we will prove the equivalent contrapositive statement:\\
            $ N5 \in L \qquad\impliedby\qquad L\text{ is not modular}$\\
            (every non-modular lattice \emph{must} contain $N5$).

      \item \label{item:lat_mod_iff_N5_def}
            We will show that for any choice of $x,y\in L$ such that $x\orel y$ and
            under the following definitions,
            all non-modular lattices contain the $N5$ lattice illustrated below:\\
        %\begin{minipage}{20mm}%
        %  \centering%
        %  \psset{unit=5mm}%
        %  %============================================================================
% Daniel J. Greenhoe
% LaTeX file
% lattice N5
% nominal unit = 5mm
%============================================================================
\begin{pspicture}(-1.3,-\latbot)(1.3,3.4)
  %---------------------------------
  % nodes
  %---------------------------------
                 \Cnode(0,3){t}%
  \Cnode(-1,2){d}%
                               \Cnode( 1,1.5){m}%
  \Cnode(-1,1){c}%
                 \Cnode(0,0){b}%
  %---------------------------------
  % node connections
  %---------------------------------
  \ncline{t}{d}\ncline{t}{m}%
  \ncline{c}{d}%
  \ncline{b}{c}\ncline{b}{m}%
  %---------------------------------
  % node labels
  %---------------------------------
  \uput[10](t){$1$}%
  \uput[180](m){$z$}%
  \uput[0](d){$y$}%
  \uput[0](c){$x$}%
  \uput[-10](b){$0$}%
\end{pspicture}%%
        %\end{minipage}\hfill%
        \tbox{\includegraphics{graphics/lat5_n5_xyz.pdf}}\hfill%
        \begin{minipage}{\tw-46mm-22mm}
          \begin{align*}
            a &\eqd x\join(y\meet z) \\
            b &\eqd y\meet(x\join z)
          \end{align*}
        \end{minipage}%
        \hfill\tbox{\includegraphics{graphics/lat5_n5_abz.pdf}}%
        %\begin{minipage}{20mm}%
        %  \centering%
        %  \psset{unit=5mm}%
        %  %============================================================================
% Daniel J. Greenhoe
% LaTeX file
% lattice N5
%============================================================================
\begin{pspicture}(-2,-\latbot)(2,3.5)%
  %---------------------------------
  % nodes
  %---------------------------------
  \Cnode(0,3){t}%
  \Cnode(-1,2){d}%
  \Cnode( 1,1.5){m}%
  \Cnode(-1,1){c}%
  \Cnode(0,0){b}%
  %---------------------------------
  % node connections
  %---------------------------------
  \ncline{t}{d}\ncline{t}{m}%
  \ncline{c}{d}%
  \ncline{b}{c}\ncline{b}{m}%
  %---------------------------------
  % node labels
  %---------------------------------
  \uput[0](t){$x\join z$}%
  \uput[0](m){$z$}%
  \uput[180](d){$b$}%
  \uput[180](c){$a$}%
  \uput[0](b){$y\meet z$}%
\end{pspicture}%%
        %\end{minipage}%
        %\begin{minipage}{1\tw/16}\center%
        %  \latmatls{5}{0.5}{0.5}{
        %                  & [name=1]\null                 \\
        %    [name=y]\null                                 \\
        %                  &               & [name=z]\null \\
        %    [name=x]\null                                 \\
        %                  & [name=0]\null
        %    }{
        %    \ncline{1}{y}
        %    \ncline{y}{x}
        %    \ncline{x}{0}
        %    \ncline{1}{z}
        %    \ncline{z}{0}
        %    }{
        %    \nput{90}{1}{$x\join z$}
        %    \nput{180}{y}{$b$}
        %    \nput{  0}{z}{$z$}
        %    \nput{180}{x}{$a$}
        %    \nput{-90}{0}{$y\meet z$}
        %    }
        %\end{minipage}

      \item Proofs for comparable elements:
        \begin{align*}
          b
            &= y\meet(x\join z)
            && \text{by definition of $b$ in \pref{item:lat_mod_iff_N5_def}}
          \\&\orel x \join z
            && \text{by definition of $\meet$ \prefpo{def:glb}}
          \\
          \\
          a
            &= x\join(y\meet z)
            && \text{by definition of $a$ in \pref{item:lat_mod_iff_N5_def}}
          \\&\orel y\meet(x\join z)
            && \text{by modularity inequality \pref{thm:lat_mod}}
          \\&= b
            && \text{by definition of $b$ in \pref{item:lat_mod_iff_N5_def}}
          \\
          \\
          y \meet z
            &\orel x\join(y\meet z)
            && \text{by definition of $\join$ \prefpo{def:lub}}
          \\&= a
            && \text{by definition of $a$ in \pref{item:lat_mod_iff_N5_def}}
          \\
          \\
          z
            &\orel x \join z
            && \text{by definition of $\meet$ \prefpo{def:glb}}
          \\
          \\
          y \meet z
            &\orel z
            && \text{by definition of $\meet$ \prefpo{def:glb}}
        \end{align*}

      \item Proofs for noncomparable elements:
        \begin{align*}
          a \join z
            &= [x\join(y\meet z)] \join z
            && \text{by definition of $a$}
          \\&= z\join[x\join(y\meet z)]
            && \text{by \prope{commutative property} of lattices (\prefpo{thm:lattice})}
          \\&= [z\join x]\join(y\meet z)
            && \text{by \prope{associative property} of lattices (\prefpo{thm:lattice})}
          \\&= [x\join z]\join(y\meet z)
            && \text{by \prope{commutative property} of lattices (\prefpo{thm:lattice})}
          \\&= x\join [z \join(y\meet z)]
            && \text{by \prope{associative property} of lattices (\prefpo{thm:lattice})}
          \\&= x\join z
            && \text{by \prope{absorptive property} of lattices (\prefpo{thm:lattice})}
          \\
          \\
          b \join z
            &= (b\join a)\join z
            && \text{by previous result}
          \\&= b\join (a\join z)
            && \text{by \prope{associative property} of lattices (\prefpo{thm:lattice})}
          \\&= b\join (x\join z)
            && \text{by previous result}
          \\&= x\join z
            && \text{by previous result}
          \\
          \\
          a \meet z
            &= (a \meet b) \meet z
            && \text{by previous result}
          \\&= a \meet (b \meet z)
            && \text{by \prope{associative property} of lattices (\prefpo{thm:lattice})}
          \\&= a \meet (y \meet z)
            && \text{by previous result}
          \\&= y \meet z
            && \text{by previous result}
          \\
          \\
          b \meet z
            &= [y\meet(x\join z)] \join z
            && \text{by definition of $a$}
          \\&= z\meet[y\meet(x\join z)]
            && \text{by \prope{commutative property} of lattices (\prefpo{thm:lattice})}
          \\&= [z\meet y]\meet(x\join z)
            && \text{by \prope{associative property} of lattices (\prefpo{thm:lattice})}
          \\&= [y\meet z]\meet(x\join z)
            && \text{by \prope{commutative property} of lattices (\prefpo{thm:lattice})}
          \\&= y\meet [z\meet(x\join z)]
            && \text{by \prope{associative property} of lattices (\prefpo{thm:lattice})}
          \\&= y\meet z
            && \text{by \prope{absorptive property} of lattices (\prefpo{thm:lattice})}
        \end{align*}

      \item Thus, \emph{all} non-modular lattices \emph{must} contain an $N5$ sublattice.
            That is,
            \[ \text{$\latL$ is a non-modular lattice} \qquad\implies\qquad \text{$\latL$ contains an $N5$ sublattice}.\]
            And this implies (by the contrapostive of the statement)
            \[ \text{$\latL$ does \emph{not} contain an $N5$ sublattice} \qquad\implies\qquad \text{$\latL$ is modular lattice}.\]
    \end{enumerate}

\end{enumerate}
\end{proof}


%=======================================
\subsubsection{Algebraic characterizations}
%=======================================
%---------------------------------------
\begin{theorem}
\citetbl{
  \citerppg{padmanabhan2008}{42}{43}{9812834540},
  \citor{riecan1957}
  }
\label{thm:lat_char_r1957}
%---------------------------------------
Let $\latA\eqd\latticed$ be an algebraic structure.
\thmbox{
  \brb{\begin{array}{lclCD}
    (x \meet y) \join (x \meet z)     &=& \brs{(z \meet x) \join y} \meet x  & \forall x,y,z\in\setX   & and \\
    \brs{x \join (y \join z)} \meet z &=& z                                  & \forall x,y,z\in\setX
  \end{array}}
  \quad\iff\quad
  \brb{\begin{array}{M}
    $\latA$ is a\\ 
    \structb{modular lattice}
  \end{array}}
  }
\end{theorem}





%%=======================================
%\subsection{Duality}
%%=======================================
%Because the more general case of lattices support
%the \thme{principle of duality} \xref{thm:lat_duality},
%and because the characterizations of modular lattices given in \prefpp{thm:latm_id}
%are \prope{self-dual},
%so modular lattices also support the principle of duality (next theorem).
%%---------------------------------------
%\begin{theorem}[\thmd{Principle of duality}]
%\label{thm:latm_duality}
%\citetbl{
%  \citerpg{padmanabhan2008}{39}{9812834540}
%  }
%%---------------------------------------
%Let $\latM\eqd\latticed$ be a \prope{modular} lattice.
%\thmboxt{
%  $\brb{\parbox{6\tw/16}{\raggedright
%    $\phi$ is an identity on $\latM$ in terms of the operations
%    $\join$ and $\meet$}}$
%  $\qquad\implies\qquad$
%  $\opT\phi$ is also an identity on $\latM$
%  \\[1ex]
%  where the operator $\opT$ performs the following mapping on the operations of $\phi$:
%  \\\indentx $\join\rightarrow\meet,\qquad \meet\rightarrow\join$
%  }
%\end{theorem}
%\begin{proof}
%\begin{enumerate}
%  \item The 4 pairs of identities characterizing lattices in \prefpp{thm:lattice}
%        are prope{self-dual}.
%  \item The relationships characterizing modular lattices in \prefpp{thm:latm_id} are duals of each other.
%  \item Since these 5 pairs of identities characterizing modular lattices
%        are \prope{self-dual}, the \hie{principle of duality} applies to modular lattices.
%\end{enumerate}
%\end{proof}






%=======================================
\subsection{Special cases}
%=======================================
%---------------------------------------
\begin{theorem}
\footnote{
  \citerpg{salii1988}{27}{0821845225},
  \citerppc{dilworth1982}{333}{353}{cf Stern 1999},
  \citerpg{stern1999}{11}{0521461057},
  \citor{mclaughlin1956}
  }
%---------------------------------------
Let $\latL\eqd\latbd$ be a bounded lattice.
\thmbox{
  \brb{\begin{array}{FMD}
    1. & $\latL$ is \prope{complemented} & and \\
    2. & $\latL$ is \prope{atomic} & and \\
    3. & $\latL$ does \emph{not} contain an N5 lattice \\
       & with elements $\bzero$ and $\bid$
  \end{array}}
  \implies
  \brb{\begin{array}{FMD}
    1. & $\latL$ does not contain \\
       & any N5 sublattice & and \\
    2. & $\latL$ is \hie{modular}
  \end{array}}
  }
\end{theorem}

%=======================================
\section{Examples}
%=======================================
\begin{minipage}{\tw-20mm}
%---------------------------------------
\begin{example}
\label{ex:lat_N5}
\index{lattice!N5}
%---------------------------------------
The lattice illustrated to the right is the \hie{N5 lattice} \xref{def:lat_N5}.
The N5 lattice has a total of $5\times 5=25$ pairs of elements of the form 
$\opair{x}{y}$ where $x,y\in\setX$.
Of these 25, \emph{all} are modular pairs \emph{except} for the pair $\opair{z}{y}$.
That is, $z\notmodular y$.
Therefore, the N5 lattice is \prope{non-semimodular} (and \prope{non-modular}).
\end{example}
\end{minipage}%
\hfill\tbox{\includegraphics{graphics/lat5_n5_xyz.pdf}}\hfill\mbox{}%
\\
\begin{proof}
\begin{enumerate}
  \item Five are of the form $\opair{x}{x}$ and are therefore modular pairs 
    by the \prope{reflexive} property and \prefp{prop:relM_reflexive}:
    \\$\bid\modular\bid$, $y\modular y$, $x\modular x$, $z\modular z$, $\bzero\modular \bzero$.
    
  \item Of the remaining 20, 16 more are modular pairs simply because they are 
   \prope{comparable} and by \prefpp{prop:latm_xyxmx}:
   \\$\begin{array}{@{\qquad}*{8}c}
     {1}\modular{y} & {1}\modular{x} & {1}\modular{0} & {y}\modular{x} & {y}\modular{0} & {x}\modular{0} & {1}\modular{z} & {z}\modular{0} \\
     {y}\modular{1} & {x}\modular{1} & {0}\modular{1} & {x}\modular{y} & {0}\modular{y} & {0}\modular{x} & {z}\modular{1} & {0}\modular{z} 
   \end{array}$

  \item Of the remaining 4, 3 are modular pairs and 1 is a nonmodular pair:
    \\$\begin{array}{@{\qquad}*{2}c}
      y\modular z    & x\modular z \\ 
      z\notmodular y & z\modular x 
    \end{array}$
    \begin{align*}
      x\orel y \implies y\meet(z\join x)
        &= y\meet \bid
       &&= y
       &&\ne  x
       &&= \bzero\join x
       &&= (y\meet z)\join x
       &&\implies z\notmodular y
      \\
      \bzero\orel z \implies z\meet(y\join\bzero)
        &= z\meet y
       &&= \bzero
       &&
       &&= \bzero\join\bzero
       &&= (z\meet y)\join\bzero
       &&\implies y\modular z
      \\
      \bzero\orel z \implies z\meet(x\join\bzero)
        &= z\meet x
       &&= \bzero
       &&
       &&= \bzero\join\bzero
       &&= (z\meet x)\join\bzero
       &&\implies x\modular z
      \\
      \bzero\orel x \implies x\meet(z\join\bzero)
        &= x\meet z
       &&= \bzero
       &&
       &&= \bzero\join\bzero
       &&= (x\meet z)\join\bzero
       &&\implies z\modular x
    \end{align*}
\end{enumerate}
\end{proof}


\begin{minipage}{\tw-33mm}
%---------------------------------------
\begin{example}
\label{ex:latm_6_smod1}
%---------------------------------------
Of the non-comparable pairs in the lattice illustrated to the right,
the following are \prope{modular} pairs:
\\\indentx$ x\modular y,\, y\modular x,\, x\modular a,\, a\modular x,\, y\modular a,\, a\modular y,\, b\modular x,\, b\modular y$
\\and the remaining non-comparable pairs are \prope{non-modular}:
\\\indentx$ x\notmodular b,\, y\notmodular b.$
\\
Therefore, although the Hasse diagram shown is horizontally and vertically symmetric,
the lattice itself is \emph{not} \prope{M-symmetric} (not semimodular),
and thus also not modular and not distributive.
\end{example}
\end{minipage}%
\hfill\tbox{\includegraphics{graphics/lat6_l4inm2_xyab01.pdf}}\hfill\mbox{}\\%
\begin{proof}
\begin{align*}
  y(x+0) &= yx &&= yx+0 &&        &&      &&       &&\implies x\modular y \\
  x(y+0) &= xy &&= xy+0 &&        &&      &&       &&\implies y\modular x \\
  a(x+0) &= ax &&= ax+0 &&        &&      &&       &&\implies x\modular a \\
  x(a+0) &= xa &&= xa+0 &&        &&      &&       &&\implies a\modular x \\
  a(y+0) &= ay &&= ay+0 &&        &&      &&       &&\implies y\modular a \\
  y(a+0) &= ya &&= ya+0 &&        &&      &&       &&\implies a\modular y \\
  b(x+a) &= b1 &&= b    &&\ne  a &&= 0+a &&= bx+a &&\implies x\notmodular b \\
  x(b+0) &= xb &&= xb+0 &&        &&      &&       &&\implies b\modular x \\
  b(y+a) &= b1 &&= b    &&\ne  a &&= 0+a &&= by+a &&\implies y\notmodular b \\
  y(b+0) &= yb &&= yb+0 &&        &&      &&       &&\implies b\modular y    
\end{align*}
\end{proof}

\tbox{\includegraphics{graphics/lat6_primo_xyza01.pdf}}\hfill%
\begin{minipage}{\tw-67mm}%
%---------------------------------------
\begin{example}
\label{ex:latm_6_diadic}
%---------------------------------------
The lattices illustrated to the right and left are duals of each other.
Both are \prope{non-modular} and both are \prope{non-semimodular}.
\end{example}
\end{minipage}%
\hfill\tbox{\includegraphics{graphics/lat6_primo_dual_xyza01.pdf}}\\
\begin{proof}
\begin{align*}
  \intertext{Left  hand side lattice:}
    a(z+x) &= a1 &&= a    &&\ne  x &&= 0+x &&= az + x &&\implies z \notmodular a \\
    z(a+0) &= za &&= za+0 &&        &&      &&         &&\implies a \modular z \\    
  \intertext{Right hand side lattice:}
    z(x+0) &= zx &&= zx+0 &&        &&      &&         &&\implies x \modular z \\
    x(z+a) &= x1 &&= x    &&\ne  a &&= 0+a &&= xz+a   &&\implies z \notmodular x
\end{align*}
\end{proof}

\tbox{\includegraphics{graphics/lat6_o6bslash_xyab01.pdf}}\hfill%
\begin{minipage}{\tw-66mm}%
%---------------------------------------
\begin{example}
\label{ex:latm_6_hexagon}
%---------------------------------------
The lattice illustrated to the left is \prope{modular}.
The lattice illustrated to the right is \prope{non-modular}
and \prope{non-semimodular}.
\end{example}
\end{minipage}%
\hfill\tbox{\includegraphics{graphics/lat6_o6_xyab01.pdf}}\\%
\begin{proof}
\begin{enumerate}
  \item Proof that the left hand side is \prope{modular}: 
        because it does not contain the N5 lattice and by \prefpp{thm:latm_n5}.

  \item Proof that the right hand side is \prope{non-modular} and \prope{non-semimodular}:
    \begin{align*}
      x(b+y) &= xb &&= 0    &&        &&= 0+y &&= xb+y &&\implies b\modular    x \\
      b(x+y) &= b1 &&= b    &&\ne  y &&= 0+y &&= bx+y &&\implies x\notmodular b \\
      y(a+x) &= ya &&= 0    &&        &&= 0+x &&= ya+x &&\implies a\modular    y \\
      a(y+x) &= a1 &&= a    &&\ne  x &&= 0+x &&= ay+x &&\implies y\notmodular a 
    \end{align*}
\end{enumerate}
\end{proof}


%---------------------------------------
\begin{proposition}
\footnote{
  $l_n$:   \citeoeis{A006966}  | % lattices
  $m_n$:   \citeoeis{A006981}  | % modular
  %$d_n$:  \citeoeis{A006982}  | \\% distributive
  $l_n$:   \citer{heitzig2002}    %"Counting Finite Lattices"
  %$m_n$:  \citer{thakare2002}? %"A structure theorem for dismantlable lattices and enumeration"
  %$d_n$:  \citerp{erne2002}{17} %"On the number of distributive lattices"
  }
\label{prop:lat_num_lm}
\index{number of lattices}
%---------------------------------------
Let $\setX_n$ be a finite set with order $n=\seto{\setX_n}$.
Let $l_n$ be the number of unlabeled lattices on $\setX_n$,
    $d_n$ the number of unlabeled distributive lattices on $\setX_n$,
and $m_n$ the number of unlabeled modular lattices on $\setX_n$.
\propbox{\begin{array}{l|*{14}{|c}}
  n   & 0 & 1 & 2 & 3 &  4 &  5 &   6 &   7 &    8 &      9  & 10 & 11 & 12 & 13\\
  \hline
  l_n  & 1 & 1 & 1 & 1 & 2 & 5 & 15 & 53 & 222 & 1078 & 5994 & 37622 & 262,776 & 2,018,305\\
  m_n  & 1 & 1 & 1 & 1 & 2 & 4 & 8  & 16 & 34  & 72   & 157  & 343   \\
\end{array}}
\end{proposition}



%---------------------------------------
\begin{example}[modularity in 5 element sets]
\label{ex:lat_set5_mod}
%---------------------------------------
There are a total of five unlabeled lattices on a five element set 
\xref{prop:num_lattices};
and of these five, four are modular, 
and three of the five are \prope{distributive} \xref{ex:lat_set5_distrib}.
%There are no six element non-modular lattices that are also \prope{semimodular}.
\exbox{\begin{tabular}{cccc}%
  \hline
   \mc{4}{c}{\structb{modular lattice}s on 5 element sets}
  \\\hline
   \includegraphics{graphics/lat5_m3.pdf}
 %&\includegraphics{graphics/lat5_n5.pdf}
  &\includegraphics{graphics/lat5_l2onm2.pdf}
  &\includegraphics{graphics/lat5_m2onl2.pdf}
  &\includegraphics{graphics/lat5_l5.pdf}
  \\\hline
\end{tabular}}
%\exbox{\begin{tabular}{cccc}%
%   \mc{4}{c}{\structb{modular lattice}s on 5 element sets}
%  \\\hline%
%   \psset{unit=7.5mm}\footnotesize%============================================================================
% Daniel J. Greenhoe
% LaTeX file
% lattice M3
% nominal unit = 10mm
%============================================================================
{%
\begin{pspicture}(-1.3,-\latbot)(1.3,2.3)%
  %---------------------------------
  % nodes
  %---------------------------------
  \Cnode(0,2){t}%
  \Cnode(-1,1){x}\Cnode(0,1){y}\Cnode(1,1){z}%
  \Cnode(0,0){b}%
  %---------------------------------
  % node connections
  %---------------------------------
  \ncline{t}{x}\ncline{t}{y}\ncline{t}{z}%
  \ncline{b}{x}\ncline{b}{y}\ncline{b}{z}%
  %---------------------------------
  % node labels
  %---------------------------------
  %\uput[ 90](t) {$\setn{x,y,z}$}%
  %\uput[180](xy){$\setn{x,y}$}%   
  %\uput[0](yz){$\setn{y,z}$}%
\end{pspicture}
}%%
%  &\psset{unit=5mm}\footnotesize%============================================================================
% Daniel J. Greenhoe
% LaTeX file
% lattice M2 on L2
% nominal unit = 10mm
%============================================================================
{%
\begin{pspicture}(-1.3,-\latbot)(1.3,3.3)
  %---------------------------------
  % nodes
  %---------------------------------
  \Cnode(0,3){t}%
  \Cnode(0,2){c}%
  \Cnode(-1,1){x}\Cnode(1,1){y}%
  \Cnode(0,0){b}
  %---------------------------------
  % node connections
  %---------------------------------
  \ncline{t}{c}%
  \ncline{c}{x}\ncline{c}{y}%
  \ncline{b}{x}\ncline{b}{y}%
\end{pspicture}
}%%
%  &\psset{unit=5mm}\footnotesize%============================================================================
% Daniel J. Greenhoe
% LaTeX file
% lattice M2 on L2
% nominal unit = 10mm
%============================================================================
{%
\begin{pspicture}(-1.3,-\latbot)(1.3,3.3)
  %---------------------------------
  % nodes
  %---------------------------------
  \Cnode(0,3){t}%
  \Cnode(-1,2){x}\Cnode(1,2){y}%
  \Cnode(0,1){c}%
  \Cnode(0,0){b}
  %---------------------------------
  % node connections
  %---------------------------------
  \ncline{t}{x}\ncline{t}{y}%
  \ncline{c}{x}\ncline{c}{y}%
  \ncline{b}{c}%
  %---------------------------------
  % node labels
  %---------------------------------
  %\uput[ 90](t) {$\setn{x,y,z}$}%
  %\uput[180](xy){$\setn{x,y}$}%   
  %\uput[0](yz){$\setn{y,z}$}%
  %\uput[180](x) {$\setn{x}$}%     
  %\uput[0](z) {$\setn{z}$}%
  %\uput[-90](b) {$\szero$}%
\end{pspicture}
}%%
%  &\psset{unit=3.75mm}\footnotesize%============================================================================
% Daniel J. Greenhoe
% LaTeX file
% lattice M2 on L2
% nominal unit = 10mm
%============================================================================
{%
\begin{pspicture}(-0.5,-\latbot)(0.5,4.5)
  %---------------------------------
  % nodes
  %---------------------------------
  \Cnode(0,4){t}%
  \Cnode(0,3){e}%
  \Cnode(0,2){d}%
  \Cnode(0,1){c}%
  \Cnode(0,0){b}
  %---------------------------------
  % node connections
  %---------------------------------
  \ncline{e}{t}%
  \ncline{d}{e}%
  \ncline{c}{d}%
  \ncline{b}{c}%
  %---------------------------------
  % node labels
  %---------------------------------
  %\uput[0](t) {$\setn{w,x,y,z}$}%
  %\uput[0](e) {$\setn{w,x,y}$}%
  %\uput[0](d) {$\setn{w,x}$}%
  %\uput[0](c) {$\setn{w}$}%
  %\uput[0](b) {$\emptyset$}%
\end{pspicture}
}%%
%\end{tabular}}
%\\\input{../common/lat5mod.inp}
\end{example}

%---------------------------------------
\begin{example}[modularity in 6 element sets]
\label{ex:lat_set6_mod}
%---------------------------------------
There are a total of 15 unlabeled lattices on a six element set 
(\prefp{prop:num_lattices} and \prefp{ex:lat_set6});
and of these 15, eight are modular, 
and five of the eight are distributive \xref{prop:lat_num_ldm}.
There are no six element non-modular lattices that are also \prope{semimodular}.
\exbox{\begin{tabular}{cccccccc}%
\mc{5}{c|}{modular and distributive lattices}&
\mc{3}{c}{modular but not distributive}
  \\\hline
   \includegraphics{graphics/lat6_o6slash.pdf}%
  &\includegraphics{graphics/lat6_l3onm2.pdf}%
  &\includegraphics{graphics/lat6_l2onm2onl2.pdf}%
  &\includegraphics{graphics/lat6_m2onl3.pdf}%
  &\includegraphics{graphics/lat6_l6.pdf}%
  %
  &\includegraphics{graphics/lat6_m4.pdf}%
  &\includegraphics{graphics/lat6_l2onm3.pdf}%
  &\includegraphics{graphics/lat6_m3onl2.pdf}%
  \\\hline\hline
  \mc{8}{c}{non-semimodular lattices (and non-modular and non-distributive)}
  \\\hline
    \includegraphics{graphics/lat6_l4inm2.pdf}%
   &\includegraphics{graphics/lat6_p3.pdf}%
   &\includegraphics{graphics/lat6_p3_dual.pdf}%
   &\includegraphics{graphics/lat6_o6.pdf}%
   &\includegraphics{graphics/lat6_n6.pdf}%
   &\includegraphics{graphics/lat6_l2onn5.pdf}%
   &\includegraphics{graphics/lat6_n5onl2.pdf}%
   &
  \\\hline
\end{tabular}}
%\input{../common/lat6mod.inp}
\end{example}

\tbox{\includegraphics{graphics/lat7_veeino6_xyzab01.pdf}}\hfill%
\begin{minipage}{\tw-46mm}%
%---------------------------------------
\begin{example}
\label{ex:latm_7}
%---------------------------------------
The lattices illustrated to the left and right are duals of each other.
Both are \prope{non-modular}.
The left hand side lattice is also \prope{non-semimodular},
however the right hand side lattice is \prope{semimodular}.\footnotemark
\end{example}
\end{minipage}\hfill%
\citetblt{%
  \citerpc{maeda1970}{5}{Exercise 1.1}%
  }%
\hfill\tbox{\includegraphics{graphics/lat7_wedgeino6_xyabc01.pdf}}\\
\begin{proof}
\begin{align*}
  \intertext{Proof for lattice on left hand side:}
    y(a+0) &= ya &&       &&        &&= ya+0 &&            && \implies a\modular y\\
    a(y+x) &= aa &&= a    &&= y+x   &&= ay+x &&            && \implies y\modular a \\
    b(a+z) &= b1 &&= b    &&= y+z   &&= ba+z &&            && \implies a\modular b \\
    a(b+x) &= a1 &&= a    &&= y+x   &&= ab+x &&            && \implies b\modular a \\
    b(x+z) &= b1 &&= b    &&\ne  z &&= 0+z  && bx+z       && \implies x\notmodular b \\
    x(b+0) &= xb &&       &&=       &&= xb+0 &&            && \implies b\modular x \\
  \intertext{Proof for lattice on right hand side:}
    c(x+y) &= cb &&= y    &&= 0+y   &&= cx+y &&            && \implies x\modular c \\
    x(c+0) &= xc &&= xc+0 &&        &&       &&            && \implies c\modular x \\
    b(a+x) &= ba &&= x    &&= x+x   &&= ba+x && \text{and} && \\
    b(a+y) &= b1 &&= b    &&= x+y   &&= ba+y &&            && \implies a\modular b\\
    a(b+x) &= ab &&= 1    &&= 1+x   &&= ab+x &&            && \implies b\modular a\\
    c(a+y) &= c1 &&= c    &&\ne  y &&= 0+y  &&= ca + y    && \implies a\notmodular c \\
    a(c+x) &= a1 &&= a    &&\ne  x &&= 0+x  &&= ac + x    && \implies c\notmodular a \\
    c(x+y) &= cb &&= y    &&        &&= 0+y  &&= cx+y      && \implies x\modular c \\
    x(c+0) &= xc &&= xc+0 &&        &&       &&            && \implies c\modular x
    \\\vdots
\end{align*}
\end{proof}



%---------------------------------------
\begin{example}[modular lattices on 7 element sets]
\label{ex:lat_set7_mod}
%---------------------------------------
There are a total of 53 unlabeled lattices on a seven element set \xref{ex:lat_set7}.
Of these 53, 16 are modular, and 8 of these 16 are distributive \xref{prop:lat_num_ldm}.
\exbox{\begin{tabular}{cccccccc}%
\mc{8}{c}{modular (and distributive) lattices on 7 element sets}
  \\\hline
   \includegraphics{graphics/lat7_m2onm2.pdf}%
  &\includegraphics{graphics/lat7_l2ono6slash.pdf}%
  &\includegraphics{graphics/lat7_o6slashonl2.pdf}%
  &\includegraphics{graphics/lat7_l4onm2.pdf}%
  &\includegraphics{graphics/lat7_l3onm2onl2.pdf}%
  &\includegraphics{graphics/lat7_l2onm2onl3.pdf}%
  &\includegraphics{graphics/lat7_m2onl4.pdf}%
  &\includegraphics{graphics/lat7_l7.pdf}%
  \\\hline
  %
\mc{8}{c|}{modular but non-distributive lattices on 7 element sets}
  \\\hline
   \mc{2}{c}{\includegraphics{graphics/lat7_m5.pdf}}%
  &\includegraphics{graphics/lat7_m2m3.pdf}%
  &\includegraphics{graphics/lat7_l2onm4.pdf}%
  &\includegraphics{graphics/lat7_l3onm3.pdf}%
  &\includegraphics{graphics/lat7_l2onm3onl2.pdf}%
  \\
  &
  &\includegraphics{graphics/lat7_m3m2.pdf}%
  &\includegraphics{graphics/lat7_m4onl2.pdf}%
  &\includegraphics{graphics/lat7_m3onl3.pdf}%
  \\\hline
\end{tabular}}
\end{example}



%2014jun14sat% %=======================================
%2014jun14sat% \section{Literature}
%2014jun14sat% %=======================================
%2014jun14sat% \begin{survey}
%2014jun14sat% \begin{enumerate}
%2014jun14sat%   \item Modular lattices:
%2014jun14sat%     \\\citorpc{ore1935}{412}{$\gamma_1$, \hie{Dedekind axiom}}
%2014jun14sat%     \\\citorp{birkhoff1933}{445}
%2014jun14sat%     \\\citePp{dedekind1900}{374}
%2014jun14sat%     \\\citorc{dedekind1893}{Section 169, cf Gratzer 2007}
%2014jun14sat% \end{enumerate}
%2014jun14sat% \end{survey}




