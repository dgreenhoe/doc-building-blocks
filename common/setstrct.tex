%============================================================================
% Daniel J. Greenhoe
% LaTeX file
%============================================================================


%=======================================
\chapter{Set Structures}
%=======================================

\begin{figure}
\centering
%============================================================================
% Daniel J. Greenhoe
% LaTeX file
% lattice ({factors of 30}, |)
%============================================================================
{\psset{xunit=40mm,yunit=15mm}%
  \begin{pspicture}(-1.5,-0.5)(1.5,5.5)%
     \footnotesize%
     \psset{%
       %linecolor=blue,
      %linewidth=1pt,
       gridcolor=graph,
       subgriddiv=1,
       cornersize=relative,
       framearc=0.25,
       gridcolor=graph,
       subgriddiv=1,
       gridlabels=4pt,
       gridwidth=0.2pt,
       }%
     \begin{tabstr}{0.75}%
       \rput( 0,   5){\rnode{ss}        {\psframebox{\begin{tabular}{c}\structe{set structure}s\\\xref{def:ss}\end{tabular}}}}%
       \rput( 0,   4){\rnode{paving}    {\psframebox{\begin{tabular}{c}\structe{paving}\\\xref{def:paving}\end{tabular}}}}%
       \rput( 1,   4){\rnode{partition} {\psframebox{\begin{tabular}{c}\structe{partition}\ifnxref{latb}{def:partition}\end{tabular}}}}%
       \rput(-1,   3){\rnode{top}       {\psframebox{\begin{tabular}{c}\structe{topology}\ifnxref{topology}{def:topology}\end{tabular}}}}%
       \rput( 1,   3){\rnode{sigring}   {\psframebox{\begin{tabular}{c}\structe{\txsigma-ring}\\\xref{def:sigring}\end{tabular}}}}%
       \rput(-1,   2){\rnode{fintop}    {\psframebox{\begin{tabular}{c}\structe{topology on finite set}\\\xref{def:fintop}\end{tabular}}}}%
       \rput( 0,   2){\rnode{sigalg}    {\psframebox{\begin{tabular}{c}\structe{\txsigma-algebra}\\\xref{def:sigalg}\end{tabular}}}}%
       \rput( 1,   2){\rnode{ringset}   {\psframebox{\begin{tabular}{c}\structe{ring of sets}\\\xref{def:ringsets}\end{tabular}}}}%
       \rput( 0,   1){\rnode{algset}    {\psframebox{\begin{tabular}{c}\structe{algebra of sets}\\\xref{def:algsets}\end{tabular}}}}%
       \rput( 0,   0){\rnode{pwrset}    {\psframebox{\begin{tabular}{c}\structe{power set}\\\xref{def:powerset}\end{tabular}}}}%
     \end{tabstr}%
     %
     \ncline{ss}{paving}\ncline{ss}{partition}%
     \ncline{paving}{top}\ncline{paving}{sigring}%
     \ncline{fintop}{top}\ncline{sigalg}{top}\ncline{sigalg}{sigring}\ncline{ringset}{sigring}%
     \ncline{algset}{fintop}\ncline{algset}{sigalg}\ncline{algset}{ringset}%
     \ncline{pwrset}{algset}%
     %
     %\psccurve[linestyle=dashed,linecolor=red,fillstyle=none]%
     %  (0,-5)(20,4)(70,60)(20,60)(15,55)(-5,38)(-20,35)(-10,20)(-25,5)%
     %\psline[linecolor=red]{->}(60,75)(60,68)%
     %\uput[135](60,75){complete spaces}%
     %\psline[linecolor=red]{->}(26,80)(15,74)%
     %\psline[linecolor=red]{->}(30,80)(30,62.5)%
     %\uput[90](28,80){analytic spaces}%
     %
     %\psgrid[unit=10mm](-8,-1)(8,9)%
  \end{pspicture}%
}%

\caption{some standard set structures\label{fig:setstructures}}
\end{figure}

%=======================================
\section{General set structures}
%=======================================
Similar to the definition of a \rele{relation} on a set $\sid$
as being any subset of the \ope{Cartesian product} $\cprodXX$\ifsxref{relation}{def:relation},
a \structe{set structure} on a set $\sid$ is simply any subset of the
\structe{power set} $\psetx$ (next) of the set $\sid$.
%(next definition and \prefp{fig:setstructures}).
%---------------------------------------
\begin{definition}
\label{def:pset}
\label{def:powerset}
\index{set!power}
%---------------------------------------
\defboxt{
  The \structd{power set} $\hxs{\psetx}$ on a set $\sid$ is defined as
    \\\indentx$\ds\psetx \eqd \set{\setA}{\setA\sorel\sid}$
      \qquad\scriptsize(the set of all subsets of $\sid$)
  }
\end{definition}

%---------------------------------------
\begin{definition}
\citetbl{
  \citerpg{molchanov2005}{389}{185233892X},
  \citerpg{pap1995}{7}{0792336585},
  \citerpg{hahn1948}{254}{111422295X}
  }
\label{def:ss}
\label{def:paving}
%---------------------------------------
Let $\psetx$ be the \structe{power set} \xref{def:pset} of a set $\setX$.
\defbox{
  \begin{array}{Ml}
  A set $\sssSx$ is a \structd{set structure} on $\sid$ if & \sssSx\sorel\psetx.\\
  A \structe{set structure} $\sssQx$ is a \structd{paving}  on $\sid$ if & \emptyset\in\sssQx.
  \end{array}
  }
\end{definition}

%---------------------------------------
\begin{definition}
\footnote{
  \citerpgc{pap1995}{8}{0792336585}{Definition 2.3: extended real-valued set function},
  \citerpgc{halmos1950}{30}{0387900888}{\textsection7. {\scshape measure on rings}},
  \citer{hahn1948},
  \citeP{choquet1954}
  }
\label{def:setf}
%---------------------------------------
Let $\sssQx$ be a \structe{paving} \xref{def:paving} on a set $\setX$.
Let $\setY$ be a set containing the element $0$.
\defboxt{
  A function $\fm\in\clF{\sssQx}{\setY}$ is a \fnctd{set function} if 
  \\\indentx$\fm(\emptyset)=0$.
  }
\end{definition}

%=======================================
\section{Operations on the power set}
%=======================================
%=======================================
\subsection{Standard operations}
%=======================================
%---------------------------------------
\begin{definition}
\citetbl{
  \citerpgc{tao2011}{12}{0821869191}{Example 3.6},
  \citerpgc{tao2010}{7}{0821852787}{Example 1.1.14}
  }
\label{def:seto}
%---------------------------------------
Let $\psetx$ be a set.
Let $\seto{\setX}$ be a function in the function space $\clF{\setX}{\intcc{0}{+\infty}}$\ifsxref{relation}{def:clFxy}.
\defbox{\begin{array}{M}
  $\seto{\setX}$ is the \vald{cardinality} or \vald{order} of $\setX$ if
  \\\indentx$\ds
    \hxs{\seto{\setX}} \eqd 
    \brbl{\begin{array}{MM}
      %$0$                           & if $\setX=\emptyset$\\
      number of elements in $\setX$  & if $\setX$ is \prope{finite}\\
      $+\infty$                      & otherwise %if $\setX$ is \emph{not} finite. %\prope{countable} or \prope{uncountable}.
    \end{array}}$
\end{array}}
\end{definition}



\begin{figure}
  \centering%
  $\begin{array}{*{4}{c}}
      \includegraphics{../common/math/graphics/pdfs/setop_0000.pdf}%
     &\includegraphics{../common/math/graphics/pdfs/setop_0011.pdf}%
     &\includegraphics{../common/math/graphics/pdfs/setop_0100.pdf}%
     &\includegraphics{../common/math/graphics/pdfs/setop_0101.pdf}%
    \\%
      \emptyset
     &\cmpA
     &\setA\setd\setB
     &\cmpB
    \\%
      \includegraphics{../common/math/graphics/pdfs/setop_0110.pdf}%
     &\includegraphics{../common/math/graphics/pdfs/setop_1000.pdf}%
     &\includegraphics{../common/math/graphics/pdfs/setop_1110.pdf}%
     &\includegraphics{../common/math/graphics/pdfs/setop_1111.pdf}%
    \\%
      \setA\sets\setB
     &\setA\seti\setB
     &\setA\setu\setB
     &\setX
  \end{array}$
  \caption{Venn diagrams for standard set operations \xref{def:setops} \label{fig:setops}}
\end{figure}
\pref{def:ss_setops} (next) introduces seven standard set operations: % classified according to the \prope{arity} \xref{def:arity} of each.
two \prope{nullary} operations, one \prope{unary} operation, and four \structe{binary operation}s\ifsxref{relation}{def:arity}.
%---------------------------------------
\begin{definition}
\citetbl{
  \citerppg{ab}{2}{4}{0120502577}
  }
\label{def:ss_setops}
\label{def:setops}
\index{sets!operations}
%---------------------------------------
Let $\psetx$ be the \structe{power set} \xref{def:pset} on a set $\sid$.
Let $\lnot$ represent the \ope{logical not} operation,
    $\lor$  represent the \ope{logical or} operation,
    $\land$ represent the \ope{logical and} operation\ifsxref{logic}{def:logic}, and
    $\lxor$ represent the \ope{logical exclusive-or} operation\ifsxref{logic}{def:lxor}.
\defbox{%
  \begin{array}{Mcc|l@{\,}c@{\,}l   @{\;}c@{\;}  l @{\,}r@{\,}c@{\,}r |C}
    \mc{2}{N}{name/symbol} & \mc{1}{N|}{arity}   & \mc{8}{N|}{definition} & \mc{1}{N}{domain}
    \\\hline
      \opd{emptyset}             & \hxs{\szero } & 0 &        &     & \szero&\eqd& \big\{x\in\sid\big| & x\ne x           &     &                  \big\} &
    \\\opd{universal set}        & \hxs{\sid   } & 0 &        &     & \sid  &\eqd& \big\{x\in\sid\big| & x=x              &     &                  \big\} &
    \\\opd{complement}           & \hxs{\setopc} & 1 &        &     & \cmpA &\eqd& \big\{x\in\sid\big| & \lnot(x\in\setA) &     &                  \big\} & \forall \setA\in\psetx
    \\\opd{union}                & \hxs{\setu  } & 2 & \setA  &\setu& \setB &\eqd& \big\{x\in\sid\big| &      (x\in\setA) &\lor &      (x\in\setB) \big\} & \forall \setA,\setB\in\psetx
    \\\opd{intersection}         & \hxs{\seti  } & 2 & \setA  &\seti& \setB &\eqd& \big\{x\in\sid\big| &      (x\in\setA) &\land&      (x\in\setB) \big\} & \forall \setA,\setB\in\psetx
    \\\opd{difference}           & \hxs{\setd  } & 2 & \setA  &\setd& \setB &\eqd& \big\{x\in\sid\big| &      (x\in\setA) &\land& \lnot(x\in\setB) \big\} & \forall \setA,\setB\in\psetx
    \\\opd{symmetric difference} & \hxs{\sets  } & 2 & \setA  &\sets& \setB &\eqd& \big\{x\in\sid\big| &      (x\in\setA) &\lxor&      (x\in\setB) \big\} & \forall \setA,\setB\in\psetx
  \end{array}%
  }
\end{definition}

With regards to the standard seven set operations only,
\pref{thm:ss_rel_gg} (next) expresses each of the set operations
in terms of pairs of other operations.
%---------------------------------------
\begin{theorem}
\label{thm:ss_rel_gg}
%\citetbl{
%  \citerpg{vaidyanathaswamy1960}{16}{0486404560}
%  }
%---------------------------------------
%Each of the seven set operations may be expressed in terms of pairs of other set operations as follows:
\thmbox{\begin{array}{r*{6}{cl}}
  \sid&=& \cmp{\szero}
  \\
  \szero
    &=& \cmp{\sid}
     =  \cmp{\brp{\setA\setu\cmpA}}
    &=& \setA\seti\cmpA
    &=& \setA\setd\setA
    &=& \setA\sets\setA
  \\
  \sid
    &=& \setA\setu\cmpA
    &=& \cmp{\brp{\setA\seti\cmpA}}
  \\
  \cmpA
    &=& \sid\setd\setA
    &=& \sid\sets\setA
  \\
  \setA\setu\setB
    &=& \cmp{\brp{\cmpA\seti\cmpB}}
    &=& \brp{\setA\sets\setB}\sets\brp{\setA\seti\setB}
    &=& \brp{\setA\setd\setB}\sets\setB
  \\
  \setA\seti\setB
    &=& \cmp{\brp{\cmpA\setu\cmpB}}
    &=& \brp{\setA\setu\setB}\sets\setA\sets\setB
    &=& \setA\setd\brp{\setA\setd\setB}
  \\
  \setA\setd\setB
    &=& \cmp{\brp{\cmpA\setu\setB}}
    &=& \setA\seti\cmpB
    &=& \brp{\setA\setu\setB}\sets\setB
    &=& \brp{\setA\sets\setB}\seti\setA
  \\
  \setA\sets\setB
    &=& \mc{3}{l}{\brs{\cmp{\brp{\cmpA\setu\setB}}} \setu \brs{\cmp{\brp{\setA\setu\cmpB}}}}
    &=& \mc{3}{l}{\brs{\cmp{\brp{\cmpA\seti\cmpB}}} \seti \cmp{\brp{\setA\seti\setB}}}
  \\&=& \brp{\setA\setd\setB}\setu\brp{\setB\setd\setA}
\end{array}}
\end{theorem}

%---------------------------------------
\begin{proposition}
\label{prop:ss_sets_group}
%---------------------------------------
Let $\sid$ be a set and $\psetx$ the power set of $\sid$.
Let $\ssetR\sorel\sid$ such that $\ssetR$ is closed with respect to the
set symmetric difference operator $\sets$.
\propboxt{%
  $\opair{\ssetR}{\sets}$ is a \hie{group}.
  In particular,
  \\\indentx$\begin{array}{Fl @{\qquad}C @{\qquad}D}
    1. & \szero\sets\setA = \setA\sets\szero = \setA
       & \forall \setA\in\ssetR
       & ($\szero$ is the \prope{identity} element)
       \\
    2. & \setA\sets\setA = \szero
       & \forall \setA\in\ssetR
       & ($\setA$ is the \prope{inverse} of $\setA$)
       \\
    3. & \setA\sets\brp{\setB\sets\setC} = \brp{\setA\sets\setB}\sets\setC
       & \forall \setA,\setB,\setC\in\ssetR
       & (\prope{associative})
  \end{array}$
  }
\end{proposition}
\begin{proof}
\ifdochas{algebra}{The definition of a group is given by \prefpp{def:alg_group}.}
  \begin{align*}
    \intertext{Proof that $\szero$ is the \prope{identity} element:}
    \intertext{1a. Proof that $\szero\in\ssetR$:}
      \szero
        &= \setA \sets \setA
        && \text{$\sets$ closed with respect to $\ssetR$}
      \\&\in\ssetR
    \intertext{1b. Proof that $\szero\sets\setA=\setA$:}
      \szero\sets\setA
        &= \set{x\in\sid}{\brp{x\in\szero}\lxor\brp{x\in\setA}}
        && \text{by definition of $\sets$ \prefpo{def:ss_setops}}
      \\&= \set{x\in\sid}{\brp{x\in\set{x\in\sid}{x\ne x}}\lxor\brp{x\in\setA}}
        && \text{by definition of $\sets$ \prefpo{def:ss_setops}}
      \\&= \set{x\in\sid}{\lfalse\lxor\brp{x\in\setA}}
      \\&= \set{x\in\sid}{\brp{x\in\setA}}
        && \text{by definition of $\lxor$ \ifxref{logic}{def:log_op}}
      \\&= \setA
    \intertext{1c. Proof that $\setA\sets\szero=\setA$:}
      \setA\sets\szero
        &= \set{x\in\sid}{\brp{x\in\setA}\lxor\brp{x\in\szero}}
        && \text{by definition of $\sets$ \prefpo{def:ss_setops}}
      \\&= \set{x\in\sid}{\brp{x\in\setA}\lxor\brp{x\in\set{x\in\sid}{x\ne x}}}
        && \text{by definition of $\sets$ \prefpo{def:ss_setops}}
      \\&= \set{x\in\sid}{\brp{x\in\setA}\lxor\lfalse}
      \\&= \set{x\in\sid}{\brp{x\in\setA}}
        && \text{by definition of $\lxor$ \ifxref{logic}{def:log_op}}
      \\&= \setA
    \intertext{2. Proof that $\setA\sets\setA$:}
      \setA\sets\setA
        &= \set{x\in\sid}{\brp{x\in\setA}\lxor\brp{x\in\setA}}
        && \text{by definition of $\sets$ \prefpo{def:ss_setops}}
      \\&= \set{x\in\sid}{\lfalse}
        && \text{by definition of $\sets$ \prefpo{def:ss_setops}}
      \\&= \szero
        && \text{by definition of $\sets$ \prefpo{def:ss_setops}}
    \intertext{3. Proof that $\setA\sets\brp{\setB\sets\setC}=\brp{\setA\sets\setB}\sets\setC$:}
      \setA\sets\brp{\setB\sets\setC}
        &= \set{x\in\sid}{\brp{x\in\setA}\lxor\brs{x\in\brp{\setB\sets\setC}}}
        && \text{by definition of $\sets$ \prefpo{def:ss_setops}}
      \\&= \set{x\in\sid}{\brp{x\in\setA}\lxor\brs{\brp{x\in\setB}\lxor\brp{x\in\setC}}}
        && \text{by definition of $\sets$ \prefpo{def:ss_setops}}
      \\&= \set{x\in\sid}{\brs{\brp{x\in\setA}\lxor\brp{x\in\setB}}\lxor\brp{x\in\setC}}
      \\&= \brp{\setA\sets\setB}\sets\setC
  \end{align*}
\end{proof}


%=======================================
\subsection{Non-standard operations}
%=======================================
\begin{figure}
  \centering
$\begin{array}{cccc}
    \includegraphics{../common/math/graphics/pdfs/setop_1000.pdf}
   &\includegraphics{../common/math/graphics/pdfs/setop_0100.pdf}
   &\includegraphics{../common/math/graphics/pdfs/setop_0010.pdf}
   &\includegraphics{../common/math/graphics/pdfs/setop_0001.pdf}
  \\\setA\seti\setB
   &\setA\seti\cmpB
   &\cmpA\seti\setB
   &\cmpA\seti\cmpB
  \\\text{(in both $\setA$ and $\setB$)}
   &\text{(in $\setA$ but not in $\setB$)}
   &\text{(not in $\setA$ but in $\setB$)}
   &\text{(neither in $\setA$ nor $\setB$)}
\end{array}$
\caption{
  The partition of a set $\setX$ into 4 regions by subsets $\setA$ and $\setB$
  \label{fig:AB4}
  }
\end{figure}
%\begin{center}
%  %============================================================================
% Daniel J. Greenhoe
% LaTeX File
% lattice of topologies over the set {x,y,z}
%============================================================================
{\psset{unit=27mm}%
\begin{pspicture}(-3,-.50)(3,0.50)%
  %---------------------------------
  % settings
  %---------------------------------
  \fns%
  %\psset{labelsep=1.5mm,radius=75\psunit}
  \psset{
    labelsep=8mm,
    radius=7.5mm,
    fillcolor=white,
    %fillstyle=solid,
    %linearc=1\psxunit,
    %framesize=16mm 12mm,
    %framearc=0,
    }
  %---------------------------------
  % developement tools
  %---------------------------------
  %\psgrid[xunit=1\psxunit,yunit=1\psyunit](-5,-1)(5,5)%
  %---------------------------------
  % nodes
  % partition of 4 regions induced by set A and set B in a set X
  %   r1000 = AB      (in both set A and set B)
  %   r0100 = AB'     (only set A, nothing from set B)
  %   r0010 = A'B     (only set B, nothing from set A)
  %   r0001 = A'B'    (area outside both set A and set B)
  %   
  %   r1111 = X
  %   r0000 = emptyset
  %---------------------------------
  %\rput(0,4){\rnode}{r1111}{\psframebox{\parbox[][12mm][]{18mm}{}}}%
  %\pnode( 0,4){r1111}%                                                                                            1 area  (4 choose 4)=4!/(4!)(4-4)!=1
  %\pnode(-1.5,3){r1101}\pnode(-0.5,3){r0111}\pnode(0.5,3){r1110}\pnode(1.5,3){r1011}%                             3 areas (4 choose 3)=4!/(3!)(4-3)!=4
  %\pnode(-2.5,2){r0101}\pnode(-1.5,2){r1100}\pnode(-0.5,2){r1001}\pnode(0.5,2){r0110}\pnode(1.5,2){r1010}\pnode(2.5,2){r0011}%   2 areas (4 choose 2)=4!/(2!)(4-2)!=6
  \pnode(-1.5,0){r0100}\pnode(-0.5,0){r0001}\pnode(0.5,0){r1000}\pnode(1.5,0){r0010}%                             1 area  (4 choose 1)=4!/(1!)(4-1)!=4
  %\pnode(   0,0){r0000}%                                                                                          0 areas (4 choose 0)=4!/(0!)(4-0)!=1
  %%---------------------------------
  %% node connections
  %%---------------------------------
  %\ncline{r1111}{r1110}\ncline{r1111}{r1101}\ncline{r1111}{r1011}\ncline{r1111}{r0111}%
  %%
  %\ncline{r1100}{r1110}\ncline{r1100}{r1101}%
  %\ncline{r1010}{r1110}\ncline{r1010}{r1011}%
  %\ncline{r1001}{r1101}\ncline{r1001}{r1011}%
  %\ncline{r0110}{r1110}\ncline{r0110}{r0111}%
  %\ncline{r0101}{r1101}\ncline{r0101}{r0111}%
  %\ncline{r0011}{r0111}\ncline{r0011}{r1011}%
  %%
  %\ncline{r1000}{r1100}\ncline{r1000}{r1010}\ncline{r1000}{r1001}%
  %\ncline{r0100}{r0110}\ncline{r0100}{r1100}\ncline{r0100}{r0101}%
  %\ncline{r0010}{r0011}\ncline{r0010}{r0110}\ncline{r0010}{r1010}%
  %\ncline{r0001}{r0011}\ncline{r0001}{r0101}\ncline{r0001}{r1001}%
  %%
  %\ncline{r0000}{r0001}\ncline{r0000}{r0010}\ncline{r0000}{r0100}\ncline{r0000}{r1000}%
  %%---------------------------------
  %% node labels
  %%---------------------------------
  %%\uput[  0](T77){$\algA_{77}$}%
  %%\uput[-90](T14){$\algA_{14}$}%
  %%\uput[-45](T22){$\algA_{22}$}%
  %%\uput[-90](T41){$\algA_{41}$}%
  %%\uput[-20](T00){$\algA_{00}$}%
  %---------------------------------
  % node inner lattices
  %---------------------------------
  \psset{
    unit=6mm,
    radius=1mm,
    dotsep=0.5pt,
    linecolor=blue,
    }%
\footnotesize%
%\rput(r0000){\begin{pspicture}(-1.5,-1.5)(3,1.5)% 0000 = emptyset
%  \psframe[linecolor=black,fillstyle=solid] (-1.5,-1.5)(2.5,1.5)%
%  \pscircle[linecolor=blue,linestyle=dotted](0,0){1}% A
%  \pscircle[linecolor=red,linestyle=dotted] (1,0){1}% B
%  \rput(0.5,1.25){$\emptyset$}%
%\end{pspicture}}%
%%
\rput(r0001){\begin{pspicture}(-1.5,-1.5)(3,1.5)% 0001  = A'B'
  \psframe[linecolor=black,fillstyle=solid,fillcolor=vennshade] (-1.5,-1.5)(2.5,1.5)%
  \pscircle[linecolor=blue,fillstyle=solid,fillcolor=white](0,0){1}% A
  \pscircle[linecolor=red,fillstyle=solid,fillcolor=white] (1,0){1}% B
  \psarcn(0,0){1}{60}{-60}%
  \rput(-1.25,0){$\setA$}%
  \rput(2.25,0){$\setB$}%
  \rput[tr](2.4,1.4){$\setX$}%
  %\rput(0.5,-1.25){$\setA\snor\setB$}%
  \rput(0.5,-1.25){$\cmpA\seti\cmpB$}%
\end{pspicture}}%
%
\rput(r0010){\begin{pspicture}(-1.5,-1.5)(3,1.5)% 0010 = A'B = A inhibitx B
  \psframe[linecolor=black,fillstyle=solid] (-1.5,-1.5)(2.5,1.5)%
  %\pscustom[linecolor=green,fillstyle=solid,fillcolor=vennshade]{% A intersect B
  %  \psarc(0,0){1}{-60}{60}%
  %  \psarc(1,0){1}{120}{240}%
  %  }
  %\pscustom[linecolor=blue,fillstyle=solid,fillcolor=vennshade]{% A \ B
  %  \psarc(0,0){1}{60}{-60}%
  %  \psarcn(1,0){1}{-120}{120}%
  %  }
  \pscustom[linecolor=red,fillstyle=solid,fillcolor=vennshade]{% B \ A
    \psarcn(0,0){1}{60}{-60}%
    \psarc(1,0){1}{-120}{120}%
    }
  \pscircle[linecolor=blue](0,0){1}% A
  \pscircle[linecolor=red] (1,0){1}% B
 %\psarcn[linecolor=blue](0,0){1}{60}{-60}%
 %\psarc [linecolor=red](1,0){1}{120}{-120}%
  \rput(-1.25,0){$\setA$}%
  \rput(2.25,0){$\setB$}%
  \rput[tr](2.4,1.4){$\setX$}%
  %\rput(0.5,-1.25){$\setA\sinx\setB\equiv\setB\siny\setA$}%
  \rput(0.5,-1.25){$\cmpA\seti\setB$}%
 %\psline[linecolor=black]{->}(0.5,-1)(0.5,0)%
\end{pspicture}}%
%%
%\rput(r0011){\begin{pspicture}(-1.5,-1.5)(3,1.5)% 0011 = A'B+A'B' = A'
%  \psframe[linecolor=black,fillstyle=solid,fillcolor=vennshade] (-1.5,-1.5)(2.5,1.5)%
%  \pscircle[linecolor=blue,fillstyle=solid,fillcolor=white](0,0){1}% A
%  \pscircle[linecolor=red] (1,0){1}% B
%  \rput(-1.25,0){$\setA$}%
%  \rput(2.25,0){$\setB$}%
%  \rput[tr](2.4,1.4){$\setX$}%
%  \rput(0.5,-1.25){$\setA\snotx\setB\equiv\cmpA$}%
%\end{pspicture}}%
%%
\rput(r0100){\begin{pspicture}(-1.5,-1.5)(3,1.5)% 0100 = AB'
  \psframe[linecolor=black,fillstyle=solid] (-1.5,-1.5)(2.5,1.5)%
  %\pscustom[linecolor=green,fillstyle=solid,fillcolor=vennshade]{% A intersect B
  %  \psarc(0,0){1}{-60}{60}%
  %  \psarc(1,0){1}{120}{240}%
  %  }
  \pscustom[linecolor=blue,fillstyle=solid,fillcolor=vennshade]{% A \ B
    \psarc(0,0){1}{60}{-60}%
    \psarcn(1,0){1}{-120}{120}%
    }
 %\pscustom[linecolor=red,fillstyle=solid,fillcolor=vennshade]{% B \ A
 %  \psarcn(0,0){1}{60}{-60}%
 %  \psarc(1,0){1}{-120}{120}%
 %  }
 %\pscircle[linecolor=blue](0,0){1}% A
  \pscircle[linecolor=red] (1,0){1}% B
  \psarcn[linecolor=blue](0,0){1}{60}{-60}%
 %\psarc [linecolor=red](1,0){1}{120}{-120}%
  \rput(-1.25,0){$\setA$}%
  \rput(2.25,0){$\setB$}%
  \rput[tr](2.4,1.4){$\setX$}%
  %\rput(0.5,-1.25){$\setA\setd\setB$}%
  \rput(0.5,-1.25){$\setA\seti\cmpB$}%
 %\psline[linecolor=black]{->}(0.5,-1)(0.5,0)%
\end{pspicture}}%
%%
%\rput(r0101){\begin{pspicture}(-1.5,-1.5)(3,1.5)% 0101 = AB'+A'B' = B'
%  \psframe[linecolor=black,fillstyle=solid,fillcolor=vennshade] (-1.5,-1.5)(2.5,1.5)%
%  \pscircle[linecolor=blue](0,0){1}% A
%  \pscircle[linecolor=red,fillstyle=solid,fillcolor=white] (1,0){1}% B
%  \psarcn  [linecolor=blue](0,0){1}{60}{-60}%
%  \rput(-1.25,0){$\setA$}%
%  \rput(2.25,0){$\setB$}%
%  \rput[tr](2.4,1.4){$\setX$}%
%  \rput(0.5,-1.25){$\setA\snoty\setB\equiv\cmpB$}%
%\end{pspicture}}%
%%
%\rput(r0110){\begin{pspicture}(-1.5,-1.5)(3,1.5)% 0110 = AB'+A'B = A symmetricdifference B
%  \psframe[linecolor=black,fillstyle=solid] (-1.5,-1.5)(2.5,1.5)%
%  %\pscustom[linecolor=green,fillstyle=solid,fillcolor=vennshade]{% A intersect B
%  %  \psarc(0,0){1}{-60}{60}%
%  %  \psarc(1,0){1}{120}{240}%
%  %  }
%  \pscustom[linecolor=blue,fillstyle=solid,fillcolor=vennshade]{% A \ B
%    \psarc(0,0){1}{60}{-60}%
%    \psarcn(1,0){1}{-120}{120}%
%    }
%  \pscustom[linecolor=red,fillstyle=solid,fillcolor=vennshade]{% B \ A
%    \psarcn(0,0){1}{60}{-60}%
%    \psarc(1,0){1}{-120}{120}%
%    }
% %\pscircle[linecolor=blue, fillstyle=solid, fillcolor=vennshade](0,0){1}% A
% %\pscircle[linecolor=red,  fillstyle=solid, fillcolor=vennshade] (1,0){1}% B
%  \psarcn[linecolor=blue](0,0){1}{60}{-60}%
%  \psarc [linecolor=red](1,0){1}{120}{-120}%
%  \rput(-1.25,0){$\setA$}%
%  \rput(2.25,0){$\setB$}%
%  \rput[tr](2.4,1.4){$\setX$}%
%  \rput(0.5,-1.25){$\setA\sets\setB$}%
% %\psline[linecolor=black]{->}(0.5,-1)(0.5,0)%
%\end{pspicture}}%
%%
%\rput(r0111){\begin{pspicture}(-1.5,-1.5)(3,1.5)% 0111 = AB'+A'B+A'B' = A|B = A Sheffer_stroke B
%  \psframe[linecolor=black,fillstyle=solid,fillcolor=vennshade] (-1.5,-1.5)(2.5,1.5)%
%  \pscustom[linecolor=green,fillstyle=solid,fillcolor=white]{% A intersect B
%    \psarc(0,0){1}{-60}{60}%
%    \psarc(1,0){1}{120}{240}%
%    }
% %\pscustom[linecolor=blue,fillstyle=solid,fillcolor=vennshade]{% A \ B
% %  \psarc(0,0){1}{60}{-60}%
% %  \psarcn(1,0){1}{-120}{120}%
% %  }
% %\pscustom[linecolor=red,fillstyle=solid,fillcolor=vennshade]{% B \ A
% %  \psarcn(0,0){1}{60}{-60}%
% %  \psarc(1,0){1}{-120}{120}%
% %  }
% \pscircle[linecolor=blue](0,0){1}% A
% \pscircle[linecolor=red] (1,0){1}% B
% % \psarcn[linecolor=blue](0,0){1}{60}{-60}%
% % \psarc [linecolor=red](1,0){1}{120}{-120}%
%  \rput(-1.25,0){$\setA$}%
%  \rput(2.25,0){$\setB$}%
%  \rput[tr](2.4,1.4){$\setX$}%
%  \rput(0.5,-1.25){$\setA\snand\setB$}%
% %\psline[linecolor=black]{->}(0.5,-1)(0.5,0)%
%\end{pspicture}}%
%%
\rput(r1000){\begin{pspicture}(-1.5,-1.5)(3,1.5)% 1000 = AB = A intersect B
  \psframe[linecolor=black,fillstyle=solid] (-1.5,-1.5)(2.5,1.5)%
  \pscustom[linecolor=green,fillstyle=solid,fillcolor=vennshade]{%
    \psarc(0,0){1}{-60}{60}%
    \psarc(1,0){1}{120}{240}%
    }
  \pscircle[linecolor=blue](0,0){1}% A
  \pscircle[linecolor=red] (1,0){1}% B
  \rput(-1.25,0){$\setA$}%
  \rput(2.25,0){$\setB$}%
  \rput[tr](2.4,1.4){$\setX$}%
  \rput(0.5,-1.25){$\setA\seti\setB$}%
 %\psline[linecolor=black]{->}(0.5,-1)(0.5,0)%
\end{pspicture}}%
%%
%\rput(r1001){\begin{pspicture}(-1.5,-1.5)(3,1.5)% 1001 = AB + A'B' = A equivalence B
%  \psframe[linecolor=black,fillstyle=solid,fillcolor=vennshade] (-1.5,-1.5)(2.5,1.5)%
%  \pscustom[linecolor=green,fillstyle=solid,fillcolor=vennshade]{% A intersect B
%    \psarc(0,0){1}{-60}{60}%
%    \psarc(1,0){1}{120}{240}%
%    }
%  \pscustom[linecolor=blue,fillstyle=solid,fillcolor=white]{% A \ B
%    \psarc(0,0){1}{60}{-60}%
%    \psarcn(1,0){1}{-120}{120}%
%    }
%  \pscustom[linecolor=red,fillstyle=solid,fillcolor=white]{% B \ A
%    \psarcn(0,0){1}{60}{-60}%
%    \psarc(1,0){1}{-120}{120}%
%    }
%  %\pscircle[linecolor=blue,fillstyle=solid,fillcolor=white](0,0){1}% A
%  %\pscircle[linecolor=red,fillstyle=solid,fillcolor=white] (1,0){1}% B
%  %\psarcn[linecolor=blue](0,0){1}{60}{-60}%
%  \rput(-1.25,0){$\setA$}%
%  \rput(2.25,0){$\setB$}%
%  \rput[tr](2.4,1.4){$\setX$}%
%  \rput(0.5,-1.25){$\setA\sequiv\setB$}%
% %\psline[linecolor=black]{->}(0.5,-1)(0.5,0)%
%\end{pspicture}}%
%%
%\rput(r1010){\begin{pspicture}(-1.5,-1.5)(3,1.5)% 1010 = AB+A'B = B
%  \psframe[linecolor=black,fillstyle=solid] (-1.5,-1.5)(2.5,1.5)%
%  \pscircle[linecolor=blue](0,0){1}% A
%  \pscircle[linecolor=red,fillstyle=solid,fillcolor=vennshade] (1,0){1}% B
%  \psarcn[linecolor=blue](0,0){1}{60}{-60}%
%  \rput(-1.25,0){$\setA$}%
%  \rput(2.25,0){$\setB$}%
%  \rput[tr](2.4,1.4){$\setX$}%
%  \rput(0.5,-1.25){$\setA\strany\setB=\setB$}%
%\end{pspicture}}%
%%
%\rput(r1011){\begin{pspicture}(-1.5,-1.5)(3,1.5)% 1011 = AB+A'B+A'B' = A impl B
%  \psframe[linecolor=black,fillstyle=solid,fillcolor=vennshade] (-1.5,-1.5)(2.5,1.5)%
%  %\pscustom[linecolor=green,fillstyle=solid,fillcolor=vennshade]{% A intersect B
%  %  \psarc(0,0){1}{-60}{60}%
%  %  \psarc(1,0){1}{120}{240}%
%  %  }
%  \pscustom[linecolor=blue,fillstyle=solid,fillcolor=white]{% A \ B
%    \psarc(0,0){1}{60}{-60}%
%    \psarcn(1,0){1}{-120}{120}%
%    }
% %\pscustom[linecolor=red,fillstyle=solid,fillcolor=vennshade]{% B \ A
% %  \psarcn(0,0){1}{60}{-60}%
% %  \psarc(1,0){1}{-120}{120}%
% %  }
% %\pscircle[linecolor=blue](0,0){1}% A
%  \pscircle[linecolor=red] (1,0){1}% B
%  \psarcn[linecolor=blue](0,0){1}{60}{-60}%
% %\psarc [linecolor=red](1,0){1}{120}{-120}%
%  \rput(-1.25,0){$\setA$}%
%  \rput(2.25,0){$\setB$}%
%  \rput[tr](2.4,1.4){$\setX$}%
%  \rput(0.5,-1.25){$\setA\simpl\setB$}%
% %\psline[linecolor=black]{->}(0.5,-1)(0.5,0)%
%\end{pspicture}}%
%% 
%\rput(r1100){\begin{pspicture}(-1.5,-1.5)(3,1.5)% 1100 = AB+AB' = A
%  \psframe[linecolor=black,fillstyle=solid] (-1.5,-1.5)(2.5,1.5)%
%  \pscircle[linecolor=blue,fillstyle=solid,fillcolor=vennshade](0,0){1}% A
%  \pscircle[linecolor=red] (1,0){1}% B
%  \rput(-1.25,0){$\setA$}%
%  \rput(2.25,0){$\setB$}%
%  \rput[tr](2.4,1.4){$\setX$}%
%  \rput(0.5,-1.25){$\setA\stranx\setB=\setA$}%
%\end{pspicture}}%
%%
%\rput(r1101){\begin{pspicture}(-1.5,-1.5)(3,1.5)% 1101
%  \psframe[linecolor=black,fillstyle=solid,fillcolor=vennshade] (-1.5,-1.5)(2.5,1.5)%
%  %\pscustom[linecolor=green,fillstyle=solid,fillcolor=vennshade]{% A intersect B
%  %  \psarc(0,0){1}{-60}{60}%
%  %  \psarc(1,0){1}{120}{240}%
%  %  }
%  \pscustom[linecolor=blue,fillstyle=solid,fillcolor=vennshade]{% A \ B
%    \psarc(0,0){1}{60}{-60}%
%    \psarcn(1,0){1}{-120}{120}%
%    }
% \pscustom[linecolor=red,fillstyle=solid,fillcolor=white]{% B \ A
%   \psarcn(0,0){1}{60}{-60}%
%   \psarc(1,0){1}{-120}{120}%
%   }
% %\pscircle[linecolor=blue](0,0){1}% A
%  \pscircle[linecolor=red] (1,0){1}% B
%  \psarcn[linecolor=blue](0,0){1}{60}{-60}%
% %\psarc [linecolor=red](1,0){1}{120}{-120}%
%  \rput(-1.25,0){$\setA$}%
%  \rput(2.25,0){$\setB$}%
%  \rput[tr](2.4,1.4){$\setX$}%
%  \rput(0.5,-1.25){$\setA\sadj\setB$}%
% %\psline[linecolor=black]{->}(0.5,-1)(0.5,0)%
%\end{pspicture}}%
%%
%\rput(r1110){\begin{pspicture}(-1.5,-1.5)(3,1.5)% 1110 = AB+AB'+A'B = A union B
%  \pscustom[linecolor=green,fillstyle=solid,fillcolor=vennshade]{% A intersect B
%    \psarc(0,0){1}{-60}{60}%
%    \psarc(1,0){1}{120}{240}%
%    }
%  \psframe[linecolor=black,fillstyle=solid] (-1.5,-1.5)(2.5,1.5)%
%  %\pscustom[linecolor=blue,fillstyle=solid,fillcolor=vennshade]{% A \ B
%  %  \psarc(0,0){1}{60}{-60}%
%  %  \psarcn(1,0){1}{-120}{120}%
%  %  }
%  %\pscustom[linecolor=red,fillstyle=solid,fillcolor=vennshade]{% B \ A
%  %  \psarcn(0,0){1}{60}{-60}%
%  %  \psarc(1,0){1}{-120}{120}%
%  %  }
%  \pscircle[linecolor=blue, fillstyle=solid, fillcolor=vennshade](0,0){1}% A
%  \pscircle[linecolor=red,  fillstyle=solid, fillcolor=vennshade] (1,0){1}% B
%  \psarcn[linecolor=blue](0,0){1}{60}{-60}%
%  %\pscircle[linecolor=blue](0,0){1}% A
%  %\pscircle[linecolor=red] (1,0){1}% B
%  \rput(-1.25,0){$\setA$}%
%  \rput(2.25,0){$\setB$}%
%  \rput[tr](2.4,1.4){$\setX$}%
%  \rput(0.5,-1.25){$\setA\setu\setB$}%
% %\psline[linecolor=black]{->}(0.5,-1)(0.5,0)%
%\end{pspicture}}%
%%
%\rput(r1111){\begin{pspicture}(-1.5,-1.5)(3,1.5)% 1111 = AB+AB'+A'B+A'B' = X
%  \psframe[linecolor=black,fillstyle=solid,fillcolor=vennshade] (-1.5,-1.5)(2.5,1.5)%
%  \pscircle[linecolor=blue](0,0){1}% A
%  \pscircle[linecolor=red] (1,0){1}% B
%  \rput(-1.25,0){$\setA$}%
%  \rput(2.25,0){$\setB$}%
%  \rput[tr](2.4,1.4){$\setX$}%
%\end{pspicture}}%
\end{pspicture}%
}%
%\end{center}
%\\\indentx
%  \begin{tabular}{>{\scriptsize}rMNMl}
%    \cnto & \cmpA  &\seti& \cmpB & (neither in $\setA$ nor $\setB$)
%    \cntn & \setA  &\seti& \cmpB & (in $\setA$ but not in $\setB$)
%    \cntn & \cmpA  &\seti& \setB & (not in $\setA$ but in $\setB$)
%    \cntn & \setA  &\seti& \setB & (in both $\setA$ and $\setB$)
%  \end{tabular}
%\\
Two subsets $\setA$ and $\setB$ of a set $\setX$ that are intersecting but
yet one is not contained in the other,
partition the set $\setX$ into four regions, as illustrated in \prefpp{fig:AB4}.
Because there are four regions, the number of ways we can select one or more of them
is $2^4=16$.
Therefore, a binary operator on sets $\setA$ and $\setB$ can likewise result in one of
$2^4=16$ possibilities.
\prefpp{def:ss_setops16} presents 7 set operations.
Therefore, there should be an additional $16-7=9$ operations.
\pref{def:ss_setops16} (next definition) attempts to define these additional operations.
Some definitions are adapted from logic\ifsxref{logic}{tbl:log_op}.
But in general these definitions are non-standard definitions with respect to set theory.
The 16 set operations under the inclusion relation $\subseteq$ form a lattice;
this lattice is illustrated by a \structe{Hasse diagram} in \prefpp{fig:setops16}.
\begin{figure}
  \centering
  \includegraphics{../common/math/graphics/pdfs/latsetops.pdf}
  \caption{lattice of set operations\label{fig:setops16}}
\end{figure}

%---------------------------------------
\begin{definition}
\label{def:ss_setops16}
\citetbl{
  standard ops: & \citerppu{ab}{2}{4}{0120502577}
  %rejection:    &
  }
\index{sets!operations}
%---------------------------------------
Let $\psetx$ be the power set on a set $\sid$.
For any sets $\setA,\setB\in\psetx$, let $\setA\setB\eqd\brp{\setA\seti\setB}$.
\defbox{%
  \begin{array}{Mcc|  l@{\,}c@{\,}l     c  *{3}{l@{\,}c@{\,}}l |C   llllllllllllllll}
  %             nsa   A     #     B     =  AB+AB'+A'B+       A'B'   forall xinX
    \mc{2}{N}{name/symbol} & \mc{1}{N|}{arity}   & \mc{11}{N|}{definition} & \mc{1}{N}{domain}
    \\\hline
      \opd{empty set}            & \symx{\sebo   }  & 2 & \setA & \sebo    & \setB &\eqd& \emptyset  &     &             &     &             &     &              & \forall \setA,\setB\in\psetx
    \\\opd{rejection}            & \symx{\snor   }  & 2 & \setA & \snor    & \setB &\eqd&            &     &             &     &             &     & \cmpA \cmpB  & \forall \setA,\setB\in\psetx
    \\\opd{inhibit $x$}          & \symx{\sinx   }  & 2 & \setA & \sinx    & \setB &\eqd&            &     &             &     & \cmpA \setB &     &              & \forall \setA,\setB\in\psetx
    \\\opd{complement $x$}       & \symx{\snotx  }  & 2 & \setA & \snotx   & \setB &\eqd&            &     &             &     & \cmpA \setB &\setu& \cmpA \cmpB  & \forall \setA,\setB\in\psetx
    \\\opd{difference}           & \symx{\siny   }  & 2 & \setA & \siny    & \setB &\eqd&            &     & \setA\cmpB  &     &             &     &              & \forall \setA,\setB\in\psetx
    \\\opd{complement $y$}       & \symx{\snoty  }  & 2 & \setA & \snoty   & \setB &\eqd&            &     & \setA\cmpB  &\setu&             &     & \cmpA \cmpB  & \forall \setA,\setB\in\psetx
    \\\opd{symmetric difference} & \symx{\sxor   }  & 2 & \setA & \sxor    & \setB &\eqd&            &     & \setA\cmpB  &\setu& \cmpA \setB &     &              & \forall \setA,\setB\in\psetx
    \\\opd{Sheffer stroke}       & \symx{\snand  }  & 2 & \setA & \snand   & \setB &\eqd&            &     & \setA\cmpB  &\setu& \cmpA \setB &\setu& \cmpA \cmpB  & \forall \setA,\setB\in\psetx
    \\\opd{intersection}         & \symx{\sand   }  & 2 & \setA & \sand    & \setB &\eqd& \setA\setB &\setu&             &     &             &     &              & \forall \setA,\setB\in\psetx
    \\\opd{equivalence}          & \symx{\sequiv }  & 2 & \setA & \sequiv  & \setB &\eqd& \setA\setB &\setu&             &     &             &     & \cmpA \cmpB  & \forall \setA,\setB\in\psetx
    \\\opd{projection $y$}       & \symx{\strany }  & 2 & \setA & \strany  & \setB &\eqd& \setA\setB &\setu&             &     & \cmpA \setB &     &              & \forall \setA,\setB\in\psetx
    \\\opd{implication}          & \symx{\simpl  }  & 2 & \setA & \simpl   & \setB &\eqd& \setA\setB &\setu&             &     & \cmpA \setB &\setu& \cmpA \cmpB  & \forall \setA,\setB\in\psetx
    \\\opd{projection $x$}       & \symx{\stranx }  & 2 & \setA & \stranx  & \setB &\eqd& \setA\setB &\setu& \setA\cmpB  &     &             &     &              & \forall \setA,\setB\in\psetx
    \\\opd{adjunction}           & \symx{\sadj}     & 2 & \setA & \sadj    & \setB &\eqd& \setA\setB &\setu& \setA\cmpB  &\setu&             &     & \cmpA \cmpB  & \forall \setA,\setB\in\psetx
    \\\opd{union}                & \symx{\sor    }  & 2 & \setA & \sor     & \setB &\eqd& \setA\setB &\setu& \setA\cmpB  &\setu& \cmpA \setB &     &              & \forall \setA,\setB\in\psetx
    \\\opd{universal set}        & \symx{\sibo   }  & 2 & \setA & \sibo    & \setB &\eqd& \setA\setB &\setu& \setA\cmpB  &\setu& \cmpA \setB &\setu& \cmpA \cmpB  & \forall \setA,\setB\in\psetx
  \end{array}%
  }
\end{definition}


%\begin{footnotesize}\label{ba_log_set_cs}%
%\begin{longtable}{|M| Nl| Nl| Nl| Nl|}
%  \hline
%  \mc{9}{|B|}{terminology}
%  \\\hline
%  \mc{1}{|B}{} & \mc{2}{|B}{Boolean algebra} & \mc{2}{|B}{logic} & \mc{2}{|B}{sets} & \mc{2}{|B|}{computer science}
%  \\\hline
%  0000 & \symx{\bzero  } & \ope{bottom}                   & \symx{\lzero  } &  \ope{false}                & \symx{\szero  } & \ope{empty set}            & \symxd{\cszero  } & \ope{zero}          \\
%  0001 & \symx{\bnor   } & \ope{rejection}                & \symx{\lnor   } &  \ope{joint denial}         & \symx{\snor   } & \ope{rejection}            & \symxd{\csnor   } & \ope{nor}           \\
%  0010 & \symx{\binx   } & \ope{inhibit $x$}              & \symx{\linx   } &  \ope{inhibit $x$}          & \symx{\sinx   } & \ope{inhibit $x$}          & \symxd{\csinx   } & \ope{inhibit $x$}   \\
%  0011 & \symx{\bnotx  } & \ope{complement $x$}           & \symx{\lnotx  } &  \ope{negation $x$}         & \symx{\snotx  } & \ope{complement $x$}       & \symxd{\csnotx  } & \ope{not $x$}       \\
%  0100 & \symx{\biny   } & \ope{exception}                & \symx{\liny   } &  \ope{inhibit $y$}          & \symx{\siny   } & \ope{difference}           & \symxd{\csiny   } & \ope{difference}    \\
%  0101 & \symx{\bnoty  } & \ope{complement $y$}           & \symx{\lnoty  } &  \ope{negation $y$}         & \symx{\snoty  } & \ope{complement $y$}       & \symxd{\csnoty  } & \ope{not $y$}       \\
%  0110 & \symx{\bxor   } & \ope{Boolean addition}         & \symx{\lxor   } &  \ope{complete disjunction} & \symx{\sxor   } & \ope{symmetric difference} & \symxd{\csxor   } & \ope{exclusive-or}  \\
%  0111 & \symx{\bnand  } & \ope{Sheffer stroke}           & \symx{\lnand  } &  \ope{alternate denial}     & \symx{\snand  } & \ope{Sheffer stroke}       & \symxd{\csnand  } & \ope{nand}          \\
%  1000 & \symx{\band   } & \ope{meet}                     & \symx{\land   } &  \ope{conjuction}           & \symx{\sand   } & \ope{intersection}         & \symxd{\csand   } & \ope{and}           \\
%  1001 & \symx{\bequiv } & \ope{biconditional}            & \symx{\lequiv } &  \ope{equivalence}          & \symx{\sequiv } & \ope{equivalence}          & \symxd{\csequiv } & \ope{equivalence}   \\
%  1010 & \symx{\btrany } & \ope{projection $y$}           & \symx{\ltrany } &  \ope{transfer $y$}         & \symx{\strany } & \ope{projection $y$}       & \symxd{\cstrany } & \ope{projection $y$}\\
%  1011 & \symx{\bimpl  } & \ope{implication}              & \symx{\limpl  } &  \ope{implication}          & \symx{\simpl  } & \ope{implication}          & \symxd{\csimpl  } & \ope{implication}   \\
%  1100 & \symx{\btranx } & \ope{projection $x$}           & \symx{\ltranx } &  \ope{transfer $x$}         & \symx{\stranx } & \ope{projection $x$}       & \symxd{\cstranx } & \ope{projection $x$}\\
%  1101 & \symx{\bimplby} & \ope{adjunction}               & \symx{\limplby} &  \ope{implied by}           & \symx{\sadj}    & \ope{adjunction}           & \symxd{\csimplby} & \ope{adjunction}    \\
%  1110 & \symx{\bor    } & \ope{join}                     & \symx{\lor    } &  \ope{disjunction}          & \symx{\sor    } & \ope{union}                & \symxd{\csor    } & \ope{or}            \\
%  1111 & \symx{\bid    } & \ope{top}                      & \symx{\lid    } &  \ope{true}                 & \symx{\sid    } & \ope{universal set}        & \symxd{\csid    } & \ope{one}           \\
%  \hline
%\end{longtable}
%\end{footnotesize}
%

% note: the following 16 set operations table has been replaced by ../common/math/graphics/latsetops.tex: a lattice of 16 set operations. 2013nov20 9:44pm
%{\psset{unit=8mm}
%\begin{tabular}{cccccccc}
%\begin{pspicture}(-1.5,-1.5)(3,1.5)% 0000 = emptyset
%  \small%
%  \psframe[linecolor=black] (-1.5,-1.5)(2.5,1.5)%
%  \pscircle[linecolor=blue,linestyle=dotted](0,0){1}% A
%  \pscircle[linecolor=red,linestyle=dotted] (1,0){1}% B
%  \rput(0.5,1.25){$\emptyset$}%
%\end{pspicture}
%&
%\begin{pspicture}(-1.5,-1.5)(3,1.5)% 0001  = A'B'
%  \small%
%  \psframe[linecolor=black,fillstyle=solid,fillcolor=vennshade] (-1.5,-1.5)(2.5,1.5)%
%  \pscircle[linecolor=blue,fillstyle=solid,fillcolor=white](0,0){1}% A
%  \pscircle[linecolor=red,fillstyle=solid,fillcolor=white] (1,0){1}% B
%  \psarcn(0,0){1}{60}{-60}%
%  \rput(-1.25,0){$\setA$}%
%  \rput(2.25,0){$\setB$}%
%  \rput(0.5,1.25){$\setX$}%
%  \rput(0.5,-1.25){$\setA\snor\setB$}%
%\end{pspicture}
%&
%\begin{pspicture}(-1.5,-1.5)(3,1.5)% 0010 = A'B = A inhibitx B
%  \small%
%  %\pscustom[linecolor=green,fillstyle=solid,fillcolor=vennshade]{% A intersect B
%  %  \psarc(0,0){1}{-60}{60}%
%  %  \psarc(1,0){1}{120}{240}%
%  %  }
%  %\pscustom[linecolor=blue,fillstyle=solid,fillcolor=vennshade]{% A \ B
%  %  \psarc(0,0){1}{60}{-60}%
%  %  \psarcn(1,0){1}{-120}{120}%
%  %  }
%  \pscustom[linecolor=red,fillstyle=solid,fillcolor=vennshade]{% B \ A
%    \psarcn(0,0){1}{60}{-60}%
%    \psarc(1,0){1}{-120}{120}%
%    }
%  \pscircle[linecolor=blue](0,0){1}% A
%  \pscircle[linecolor=red] (1,0){1}% B
% %\psarcn[linecolor=blue](0,0){1}{60}{-60}%
% %\psarc [linecolor=red](1,0){1}{120}{-120}%
%  \psframe[linecolor=black] (-1.5,-1.5)(2.5,1.5)%
%  \rput(-1.25,0){$\setA$}%
%  \rput(2.25,0){$\setB$}%
%  \rput(0.5,1.25){$\setX$}%
%  \rput(0.5,-1.25){$\setA\sinx\setB\equiv\setB\siny\setA$}%
% %\psline[linecolor=black]{->}(0.5,-1)(0.5,0)%
%\end{pspicture}
%&
%\begin{pspicture}(-1.5,-1.5)(3,1.5)% 0011 = A'B+A'B' = A'
%  \small%
%  \psframe[linecolor=black,fillstyle=solid,fillcolor=vennshade] (-1.5,-1.5)(2.5,1.5)%
%  \pscircle[linecolor=blue,fillstyle=solid,fillcolor=white](0,0){1}% A
%  \pscircle[linecolor=red] (1,0){1}% B
%  \rput(-1.25,0){$\setA$}%
%  \rput(2.25,0){$\setB$}%
%  \rput(0.5,1.25){$\setX$}%
%  \rput(0.5,-1.25){$\setA\snotx\setB\equiv\cmpA$}%
%\end{pspicture}
%\\
%\begin{pspicture}(-1.5,-1.5)(3,1.5)% 0100 = AB'
%  \small%
%  %\pscustom[linecolor=green,fillstyle=solid,fillcolor=vennshade]{% A intersect B
%  %  \psarc(0,0){1}{-60}{60}%
%  %  \psarc(1,0){1}{120}{240}%
%  %  }
%  \pscustom[linecolor=blue,fillstyle=solid,fillcolor=vennshade]{% A \ B
%    \psarc(0,0){1}{60}{-60}%
%    \psarcn(1,0){1}{-120}{120}%
%    }
% %\pscustom[linecolor=red,fillstyle=solid,fillcolor=vennshade]{% B \ A
% %  \psarcn(0,0){1}{60}{-60}%
% %  \psarc(1,0){1}{-120}{120}%
% %  }
% %\pscircle[linecolor=blue](0,0){1}% A
%  \pscircle[linecolor=red] (1,0){1}% B
%  \psarcn[linecolor=blue](0,0){1}{60}{-60}%
% %\psarc [linecolor=red](1,0){1}{120}{-120}%
%  \psframe[linecolor=black] (-1.5,-1.5)(2.5,1.5)%
%  \rput(-1.25,0){$\setA$}%
%  \rput(2.25,0){$\setB$}%
%  \rput(0.5,1.25){$\setX$}%
%  \rput(0.5,-1.25){$\setA\setd\setB$}%
% %\psline[linecolor=black]{->}(0.5,-1)(0.5,0)%
%\end{pspicture}
%&
%\begin{pspicture}(-1.5,-1.5)(3,1.5)% 0101 = AB'+A'B' = B'
%  \small%
%  \psframe[linecolor=black,fillstyle=solid,fillcolor=vennshade] (-1.5,-1.5)(2.5,1.5)%
%  \pscircle[linecolor=blue](0,0){1}% A
%  \pscircle[linecolor=red,fillstyle=solid,fillcolor=white] (1,0){1}% B
%  \psarcn  [linecolor=blue](0,0){1}{60}{-60}%
%  \rput(-1.25,0){$\setA$}%
%  \rput(2.25,0){$\setB$}%
%  \rput(0.5,1.25){$\setX$}%
%  \rput(0.5,-1.25){$\setA\snoty\setB\equiv\cmpB$}%
%\end{pspicture}
%&
%\begin{pspicture}(-1.5,-1.5)(3,1.5)% 0110 = AB'+A'B = A symmetricdifference B
%  \small%
%  %\pscustom[linecolor=green,fillstyle=solid,fillcolor=vennshade]{% A intersect B
%  %  \psarc(0,0){1}{-60}{60}%
%  %  \psarc(1,0){1}{120}{240}%
%  %  }
%  \pscustom[linecolor=blue,fillstyle=solid,fillcolor=vennshade]{% A \ B
%    \psarc(0,0){1}{60}{-60}%
%    \psarcn(1,0){1}{-120}{120}%
%    }
%  \pscustom[linecolor=red,fillstyle=solid,fillcolor=vennshade]{% B \ A
%    \psarcn(0,0){1}{60}{-60}%
%    \psarc(1,0){1}{-120}{120}%
%    }
% %\pscircle[linecolor=blue, fillstyle=solid, fillcolor=vennshade](0,0){1}% A
% %\pscircle[linecolor=red,  fillstyle=solid, fillcolor=vennshade] (1,0){1}% B
%  \psarcn[linecolor=blue](0,0){1}{60}{-60}%
%  \psarc [linecolor=red](1,0){1}{120}{-120}%
%  \psframe[linecolor=black] (-1.5,-1.5)(2.5,1.5)%
%  \rput(-1.25,0){$\setA$}%
%  \rput(2.25,0){$\setB$}%
%  \rput(0.5,1.25){$\setX$}%
%  \rput(0.5,-1.25){$\setA\sets\setB$}%
% %\psline[linecolor=black]{->}(0.5,-1)(0.5,0)%
%\end{pspicture}
%&
%\begin{pspicture}(-1.5,-1.5)(3,1.5)% 0111 = AB'+A'B+A'B' = A|B = A Sheffer_stroke B
%  \small%
%  \psframe[linecolor=black,fillstyle=solid,fillcolor=vennshade] (-1.5,-1.5)(2.5,1.5)%
%  \pscustom[linecolor=green,fillstyle=solid,fillcolor=white]{% A intersect B
%    \psarc(0,0){1}{-60}{60}%
%    \psarc(1,0){1}{120}{240}%
%    }
% %\pscustom[linecolor=blue,fillstyle=solid,fillcolor=vennshade]{% A \ B
% %  \psarc(0,0){1}{60}{-60}%
% %  \psarcn(1,0){1}{-120}{120}%
% %  }
% %\pscustom[linecolor=red,fillstyle=solid,fillcolor=vennshade]{% B \ A
% %  \psarcn(0,0){1}{60}{-60}%
% %  \psarc(1,0){1}{-120}{120}%
% %  }
% \pscircle[linecolor=blue](0,0){1}% A
% \pscircle[linecolor=red] (1,0){1}% B
% % \psarcn[linecolor=blue](0,0){1}{60}{-60}%
% % \psarc [linecolor=red](1,0){1}{120}{-120}%
%  \rput(-1.25,0){$\setA$}%
%  \rput(2.25,0){$\setB$}%
%  \rput(0.5,1.25){$\setX$}%
%  \rput(0.5,-1.25){$\setA\snand\setB$}%
% %\psline[linecolor=black]{->}(0.5,-1)(0.5,0)%
%\end{pspicture}
%\\
%\begin{pspicture}(-1.5,-1.5)(3,1.5)% 1000 = AB = A intersect B
%  \small%
%  \pscustom[linecolor=green,fillstyle=solid,fillcolor=vennshade]{%
%    \psarc(0,0){1}{-60}{60}%
%    \psarc(1,0){1}{120}{240}%
%    }
%  \pscircle[linecolor=blue](0,0){1}% A
%  \pscircle[linecolor=red] (1,0){1}% B
%  \psframe[linecolor=black] (-1.5,-1.5)(2.5,1.5)%
%  \rput(-1.25,0){$\setA$}%
%  \rput(2.25,0){$\setB$}%
%  \rput(0.5,1.25){$\setX$}%
%  \rput(0.5,-1.25){$\setA\seti\setB$}%
% %\psline[linecolor=black]{->}(0.5,-1)(0.5,0)%
%\end{pspicture}
%&
%\begin{pspicture}(-1.5,-1.5)(3,1.5)% 1001 = AB + A'B' = A equivalence B
%  \small%
%  \psframe[linecolor=black,fillstyle=solid,fillcolor=vennshade] (-1.5,-1.5)(2.5,1.5)%
%  \pscustom[linecolor=green,fillstyle=solid,fillcolor=vennshade]{% A intersect B
%    \psarc(0,0){1}{-60}{60}%
%    \psarc(1,0){1}{120}{240}%
%    }
%  \pscustom[linecolor=blue,fillstyle=solid,fillcolor=white]{% A \ B
%    \psarc(0,0){1}{60}{-60}%
%    \psarcn(1,0){1}{-120}{120}%
%    }
%  \pscustom[linecolor=red,fillstyle=solid,fillcolor=white]{% B \ A
%    \psarcn(0,0){1}{60}{-60}%
%    \psarc(1,0){1}{-120}{120}%
%    }
%  %\pscircle[linecolor=blue,fillstyle=solid,fillcolor=white](0,0){1}% A
%  %\pscircle[linecolor=red,fillstyle=solid,fillcolor=white] (1,0){1}% B
%  %\psarcn[linecolor=blue](0,0){1}{60}{-60}%
%  \rput(-1.25,0){$\setA$}%
%  \rput(2.25,0){$\setB$}%
%  \rput(0.5,1.25){$\setX$}%
%  \rput(0.5,-1.25){$\setA\sequiv\setB$}%
% %\psline[linecolor=black]{->}(0.5,-1)(0.5,0)%
%\end{pspicture}
%&
%\begin{pspicture}(-1.5,-1.5)(3,1.5)% 1010 = AB+A'B = B
%  \small%
%  \psframe[linecolor=black] (-1.5,-1.5)(2.5,1.5)%
%  \pscircle[linecolor=blue](0,0){1}% A
%  \pscircle[linecolor=red,fillstyle=solid,fillcolor=vennshade] (1,0){1}% B
%  \psarcn[linecolor=blue](0,0){1}{60}{-60}%
%  \rput(-1.25,0){$\setA$}%
%  \rput(2.25,0){$\setB$}%
%  \rput(0.5,1.25){$\setX$}%
%  \rput(0.5,-1.25){$\setA\strany\setB=\setB$}%
%\end{pspicture}
%&
%\begin{pspicture}(-1.5,-1.5)(3,1.5)% 1011 = AB+A'B+A'B' = A impl B
%  \small%
%  \psframe[linecolor=black,fillstyle=solid,fillcolor=vennshade] (-1.5,-1.5)(2.5,1.5)%
%  %\pscustom[linecolor=green,fillstyle=solid,fillcolor=vennshade]{% A intersect B
%  %  \psarc(0,0){1}{-60}{60}%
%  %  \psarc(1,0){1}{120}{240}%
%  %  }
%  \pscustom[linecolor=blue,fillstyle=solid,fillcolor=white]{% A \ B
%    \psarc(0,0){1}{60}{-60}%
%    \psarcn(1,0){1}{-120}{120}%
%    }
% %\pscustom[linecolor=red,fillstyle=solid,fillcolor=vennshade]{% B \ A
% %  \psarcn(0,0){1}{60}{-60}%
% %  \psarc(1,0){1}{-120}{120}%
% %  }
% %\pscircle[linecolor=blue](0,0){1}% A
%  \pscircle[linecolor=red] (1,0){1}% B
%  \psarcn[linecolor=blue](0,0){1}{60}{-60}%
% %\psarc [linecolor=red](1,0){1}{120}{-120}%
%  \rput(-1.25,0){$\setA$}%
%  \rput(2.25,0){$\setB$}%
%  \rput(0.5,1.25){$\setX$}%
%  \rput(0.5,-1.25){$\setA\simpl\setB$}%
% %\psline[linecolor=black]{->}(0.5,-1)(0.5,0)%
%\end{pspicture}
%\\
%\begin{pspicture}(-1.5,-1.5)(3,1.5)% 1100 = AB+AB' = A
%  \small%
%  \psframe[linecolor=black] (-1.5,-1.5)(2.5,1.5)%
%  \pscircle[linecolor=blue,fillstyle=solid,fillcolor=vennshade](0,0){1}% A
%  \pscircle[linecolor=red] (1,0){1}% B
%  \rput(-1.25,0){$\setA$}%
%  \rput(2.25,0){$\setB$}%
%  \rput(0.5,1.25){$\setX$}%
%  \rput(0.5,-1.25){$\setA\stranx\setB=\setA$}%
%\end{pspicture}
%&
%\begin{pspicture}(-1.5,-1.5)(3,1.5)% 1101
%  \small%
%  \psframe[linecolor=black,fillstyle=solid,fillcolor=vennshade] (-1.5,-1.5)(2.5,1.5)%
%  %\pscustom[linecolor=green,fillstyle=solid,fillcolor=vennshade]{% A intersect B
%  %  \psarc(0,0){1}{-60}{60}%
%  %  \psarc(1,0){1}{120}{240}%
%  %  }
%  \pscustom[linecolor=blue,fillstyle=solid,fillcolor=vennshade]{% A \ B
%    \psarc(0,0){1}{60}{-60}%
%    \psarcn(1,0){1}{-120}{120}%
%    }
% \pscustom[linecolor=red,fillstyle=solid,fillcolor=white]{% B \ A
%   \psarcn(0,0){1}{60}{-60}%
%   \psarc(1,0){1}{-120}{120}%
%   }
% %\pscircle[linecolor=blue](0,0){1}% A
%  \pscircle[linecolor=red] (1,0){1}% B
%  \psarcn[linecolor=blue](0,0){1}{60}{-60}%
% %\psarc [linecolor=red](1,0){1}{120}{-120}%
%  \rput(-1.25,0){$\setA$}%
%  \rput(2.25,0){$\setB$}%
%  \rput(0.5,1.25){$\setX$}%
%  \rput(0.5,-1.25){$\setA\sadj\setB$}%
% %\psline[linecolor=black]{->}(0.5,-1)(0.5,0)%
%\end{pspicture}
%&
%\begin{pspicture}(-1.5,-1.5)(3,1.5)% 1110 = AB+AB'+A'B = A union B
%  \small%
%  \pscustom[linecolor=green,fillstyle=solid,fillcolor=vennshade]{% A intersect B
%    \psarc(0,0){1}{-60}{60}%
%    \psarc(1,0){1}{120}{240}%
%    }
%  %\pscustom[linecolor=blue,fillstyle=solid,fillcolor=vennshade]{% A \ B
%  %  \psarc(0,0){1}{60}{-60}%
%  %  \psarcn(1,0){1}{-120}{120}%
%  %  }
%  %\pscustom[linecolor=red,fillstyle=solid,fillcolor=vennshade]{% B \ A
%  %  \psarcn(0,0){1}{60}{-60}%
%  %  \psarc(1,0){1}{-120}{120}%
%  %  }
%  \pscircle[linecolor=blue, fillstyle=solid, fillcolor=vennshade](0,0){1}% A
%  \pscircle[linecolor=red,  fillstyle=solid, fillcolor=vennshade] (1,0){1}% B
%  \psarcn[linecolor=blue](0,0){1}{60}{-60}%
%  %\pscircle[linecolor=blue](0,0){1}% A
%  %\pscircle[linecolor=red] (1,0){1}% B
%  \psframe[linecolor=black] (-1.5,-1.5)(2.5,1.5)%
%  \rput(-1.25,0){$\setA$}%
%  \rput(2.25,0){$\setB$}%
%  \rput(0.5,1.25){$\setX$}%
%  \rput(0.5,-1.25){$\setA\setu\setB$}%
% %\psline[linecolor=black]{->}(0.5,-1)(0.5,0)%
%\end{pspicture}
%&
%\begin{pspicture}(-1.5,-1.5)(3,1.5)% 1111 = AB+AB'+A'B+A'B' = X
%  \small%
%  \psframe[linecolor=black,fillstyle=solid,fillcolor=vennshade] (-1.5,-1.5)(2.5,1.5)%
%  \pscircle[linecolor=blue](0,0){1}% A
%  \pscircle[linecolor=red] (1,0){1}% B
%  \rput(-1.25,0){$\setA$}%
%  \rput(2.25,0){$\setB$}%
%  \rput(0.5,1.25){$\setX$}%
%\end{pspicture}
%\end{tabular}
%}

%=======================================
\subsection{Generated operations}
%=======================================
\prefpp{def:ss_setops} %and \prefpp{def:ss_setops16} define all the non-nullary set operations
defines set operations in terms of logical operations.
However, it is also possible to express set operations in terms of two or more other set operations.
When all the set operations can be expressed in terms of a set of operations,
then that set of operations is \hie{functionally complete} 
(next definition\ifdochas{boolean}{, but see also \prefp{def:ba_fcomplete}}).
%---------------------------------------
\begin{definition}
\citetbl{
  \citerpg{whitesitt1995}{69}{0486684830}
  }
\label{def:fcomplete}
%---------------------------------------
Let $\ssetS$ be a set structure.
\defbox{\begin{array}{M}
  A set of operations $\Phi$ is \propd{functionally complete} in $\ssetS$ if
  \\\indentx$\sor$, $\sand$, $\snot$, $\szero$, and $\sid$
  \\can all be expressed in terms of elements of $\Phi$.
\end{array}}
\end{definition}

%---------------------------------------
\begin{example}
%---------------------------------------
Here are some examples of \prope{functionally complete} sets:
\exboxt{\begin{tabular}{@{\qquad}>{\imark\quad }Mlll}
  \setn{\snor}                       & (\hie{rejection})                                                        \\%& \pref{thm:boo_fc_rejection}       & \prefpo{thm:boo_fc_rejection}      \\
  \setn{\snand}                      & (\hie{Sheffer stroke})                                                   \\%& \pref{thm:boo_fc_stroke}          & \prefpo{thm:boo_fc_rejection}      \\
  \setn{\sadj,\,\szero}           & (\hie{adjunction} and $\szero$)                                          \\%& \pref{thm:boo_fc_adjunction_0}    & \prefpo{thm:boo_fc_adjunction_0}   \\
  \setn{\setd,\,\sid}                & (\hie{set difference} and $\sid$)                                        \\%& \pref{thm:boo_fc_exception_1}     & \prefpo{thm:boo_fc_exception_1}    \\
  \setn{\sor,\,\snot}                & (\hie{union} and \hie{complement})                                       \\%& \pref{thm:boo_fc_join_not}        & \prefpo{thm:boo_fc_join_not}       \\
  \setn{\sand,\,\snot}               & (\hie{intersection} and \hie{complement})                                \\%& \pref{thm:boo_fc_meet_not}        & \prefpo{thm:boo_fc_meet_not}       \\
  \setn{\sxor,\, \sand,\,\sid}       & (\hie{symmetric difference}, \hie{intersection}, and $\sid$)             \\%& \pref{thm:boo_fc_xor_meet_1}      & \prefpo{thm:boo_fc_xor_meet_1}     \\
  \setn{\sxor,\, \sor,\,\sid}        & (\hie{symmetric difference}, \hie{union}, and $\sid$)                    \\%& \pref{thm:boo_fc_xor_join_1}      & \prefpo{thm:boo_fc_xor_join_1}     \\
  \setn{\sxor,\, \setd,\,\snot}      & (\hie{symmetric difference}, \hie{set difference}, and \hie{complement}) \\%& \pref{thm:boo_fc_xor_except_not}  & \prefpo{thm:boo_fc_xor_except_not}
\end{tabular}}
\end{example}

\ifexclude{mssa}{
The five theorems that follow demonstrate which operations can be generated by
sets of generating operations:
\\\begin{tabular}{@{\qquad}cr M@{\,}N@{\,}Ml ll}
  $\imark$ & 2 generators, & \binom{7}{2}&=&21 & possibilities, & \pref{prop:ss_gg}     & \prefpo{prop:ss_gg}     \\
  $\imark$ & 3 generators, & \binom{7}{3}&=&35 & possibilities, & \pref{prop:ss_ggg}    & \prefpo{prop:ss_ggg}    \\
  $\imark$ & 4 generators, & \binom{7}{4}&=&35 & possibilities, & \pref{prop:ss_gggg}   & \prefpo{prop:ss_gggg}   \\
  $\imark$ & 5 generators, & \binom{7}{5}&=&21 & possibilities, & \pref{prop:ss_ggggg}  & \prefpo{prop:ss_ggggg}  \\
  $\imark$ & 6 generators, & \binom{7}{6}&=& 7 & possibilities, & \pref{prop:ss_gggggg} & \prefpo{prop:ss_gggggg}
\end{tabular}

%In several cases, a handful of operations can generate \emph{all} of the other operations;
%in this case, the generating functions are \prope{functionally complete} \xref{def:fcomplete}. 
Starting with any two subsets $\setA$ and $\setB$ and using all the operations of a \prope{functionally complete}
set of operations, an \structe{algebra of sets} \xref{def:ss_algebra} is produced.
Thus, a \prope{functionally complete} set of set operations induces an \structe{algebra of sets}.
Other less powerful sets of operations generate fewer operations and induce only a \structe{ring of sets} \xref{def:ss_ring}.
And some sets of operations, such as $\setn{\setu,\seti}$, generate no set operations but themselves.

%---------------------------------------
\begin{proposition}[2 generators]
\label{prop:ss_gg}
%---------------------------------------
The following table demonstrates the ``standard" operations generated by sets of 2 operations.
%\begin{small}
\begin{longtable}{|rNN|NNNNNNN|l|}
%\begin{array}{rll|*{7}{c}|l}
  \hline
  \mc{3}{|B|}{\text{generators}} & \mc{7}{B|}{\text{generated operations}} & \mc{1}{B|}{\text{induced set structure}}\\
  \hline
   1. & \szero  & \sid    & \gc\szero & \gc\sid  &            &          &          &          &          &                 \\
   2. & \szero  & \setopc & \gc\szero &    \sid  & \gc\setopc &          &          &          &          &                 \\
   3. & \szero  & \setu   & \gc\szero &          &            & \gc\setu &          &          &          &                 \\
   4. & \szero  & \seti   & \gc\szero &          &            &          & \gc\seti &          &          &                 \\
   5. & \szero  & \setd   & \gc\szero &          &            &          &          & \gc\setd &          &                 \\
   6. & \szero  & \sets   & \gc\szero &          &            &          &          &          & \gc\sets &                 \\
   7. & \sid    & \setopc &    \szero & \gc\sid  & \gc\setopc &          &          &          &          &                 \\
   8. & \sid    & \setu   &           & \gc\sid  &            & \gc\setu &          &          &          &                 \\
   9. & \sid    & \seti   &           & \gc\sid  &            &          & \gc\seti &          &          &                 \\
  10. & \sid    & \setd   &    \szero & \gc\sid  &    \setopc &    \setu &    \seti & \gc\setd &    \sets & algebra of sets \\
  11. & \sid    & \sets   &    \szero & \gc\sid  &    \setopc &          &          &          & \gc\sets &                 \\
  12. & \setopc & \setu   &    \szero &    \sid  & \gc\setopc & \gc\setu &    \seti &    \setd &    \sets & algebra of sets \\
  13. & \setopc & \seti   &    \szero &    \sid  & \gc\setopc &    \setu & \gc\seti &    \setd &    \sets & algebra of sets \\
  14. & \setopc & \setd   &    \szero &    \sid  & \gc\setopc &          &          & \gc\setd &          &                 \\
  15. & \setopc & \sets   &    \szero &    \sid  & \gc\setopc &          &          &          & \gc\sets &                 \\
  16. & \setu   & \seti   &           &          &            & \gc\setu & \gc\seti &          &          &                 \\
  17. & \setu   & \setd   &    \szero &          &            & \gc\setu &    \seti & \gc\setd &    \sets & ring of sets    \\
  18. & \setu   & \sets   &    \szero &          &            & \gc\setu &    \seti &    \setd & \gc\sets & ring of sets    \\
  19. & \seti   & \setd   &    \szero &          &            &          & \gc\seti & \gc\setd &          &                 \\
  20. & \seti   & \sets   &    \szero &          &            &    \setu & \gc\seti &    \setd & \gc\sets & ring of sets    \\
  21. & \setd   & \sets   &    \szero &          &            &    \setu &    \seti & \gc\setd & \gc\sets & ring of sets    \\
  \hline
\end{longtable}
%\end{small}
\end{proposition}
%\begin{proof}
%\begin{align*}
%  \setA \sets \setB
%    &= \set{x\in\sid}{ x\in\setA \lxor x\in\setB}
%    && \text{by \prefp{def:ss_setops}}
%  \\&= \set{x\in\sid}{ \brp{x\in\setA \lor x\in\setB} \land\lnot\brp{x\in\setA \land x\in\setB}}
%    && \text{by \prefp{prop:lxor_or_and}}
%  \\&= \set{x\in\sid}{ \brp{x\in\setA \lor x\in\setB}} \seti \set{x\in\sid}{\lnot\brp{x\in\setA \land x\in\setB}}
%    && \text{by \prefp{def:ss_setops}}
%  \\&= \brp{\setA \setu \setB} \seti \cmp{x\in\sid}{\brp{x\in\setA \land x\in\setB}}
%    && \text{by \prefp{def:ss_setops}}
%  \\&= \brp{\setA \setu \setB} \seti \setopc\brp{\setA \seti \setB}
%    && \text{by \prefp{def:ss_setops}}
%    \\
%  \setA \setd \setB
%    &= \set{x\in\sid}{ x\in\setA \land \lnot\brp{x\in\setB}}
%    && \text{by \prefp{def:ss_setops}}
%  \\&= \set{x\in\sid}{ x\in\setA} \seti \set{x\in\sid}{\lnot\brp{x\in\setB}}
%    && \text{by \prefp{def:ss_setops}}
%  \\&= \set{x\in\sid}{ x\in\setA} \seti \cmp{x\in\sid}{\brp{x\in\setB}}
%    && \text{by \prefp{def:ss_setops}}
%  \\&= \setA \seti \cmpB
%\end{align*}
%\end{proof}


%---------------------------------------
\begin{proposition}[3 generators]
\label{prop:ss_ggg}
%---------------------------------------
The following table demonstrates the ``standard" operations generated by sets of 3 operations.
%\begin{small}
\begin{longtable}{|r|NNN|NNNNNNN|l|}
  \hline
  \mc{1}{|B|}{} & \mc{3}{B|}{\text{generators}} & \mc{7}{B|}{\text{generated operations}} & \mc{1}{B|}{\text{induced set structure}}\\
  \hline
   1. & \szero    & \sid    & \setopc  & \gc\szero & \gc\sid & \gc\setopc &          &          &          &          &                  \\
   2. & \szero    & \sid    & \setu    & \gc\szero & \gc\sid &            & \gc\setu &          &          &          &                  \\
   3. & \szero    & \sid    & \seti    & \gc\szero & \gc\sid &            &          & \gc\seti &          &          &                  \\
   4. & \szero    & \sid    & \setd    & \gc\szero & \gc\sid &    \setopc &    \setu &    \seti & \gc\setd &    \sets & algebra of sets  \\
   5. & \szero    & \sid    & \sets    & \gc\szero & \gc\sid &    \setopc &          &          &          & \gc\sets &                  \\
   6. & \szero    & \setopc & \setu    & \gc\szero &    \sid & \gc\setopc & \gc\setu &    \seti &    \setd &    \sets & algebra of sets  \\
   7. & \szero    & \setopc & \seti    & \gc\szero &    \sid & \gc\setopc &    \setu & \gc\seti &    \setd &    \sets & algebra of sets  \\
   8. & \szero    & \setopc & \setd    & \gc\szero &    \sid & \gc\setopc &          &          & \gc\setd &          &                  \\
   9. & \szero    & \setopc & \sets    & \gc\szero &    \sid & \gc\setopc &          &          &          & \gc\sets &                  \\
  10. & \szero    & \setu   & \seti    & \gc\szero &         &            & \gc\setu & \gc\seti &          &          &                  \\
  11. & \szero    & \setu   & \setd    & \gc\szero &         &            & \gc\setu &    \seti & \gc\setd &    \sets & ring of sets     \\
  12. & \szero    & \setu   & \sets    & \gc\szero &         &            & \gc\setu &    \seti &    \setd & \gc\sets & ring of sets     \\
  13. & \szero    & \seti   & \setd    & \gc\szero &         &            &          & \gc\seti & \gc\setd &          &                  \\
  14. & \szero    & \seti   & \sets    & \gc\szero &         &            &    \setu & \gc\seti &    \setd & \gc\sets & ring of sets     \\
  15. & \szero    & \setd   & \sets    & \gc\szero &         &            &    \setu &    \seti & \gc\setd & \gc\sets & ring of sets     \\
  16. & \sid      & \setopc & \setu    &    \szero & \gc\sid & \gc\setopc & \gc\setu &    \seti &    \setd &    \sets & algebra of sets  \\
  17. & \sid      & \setopc & \seti    &    \szero & \gc\sid & \gc\setopc &    \setu & \gc\seti &    \setd &    \sets & algebra of sets  \\
  18. & \sid      & \setopc & \setd    &    \szero & \gc\sid & \gc\setopc &    \setu &    \seti & \gc\setd &    \sets & algebra of sets  \\
  19. & \sid      & \setopc & \sets    &    \szero & \gc\sid & \gc\setopc &          &          &          & \gc\sets &                  \\
  20. & \sid      & \setu   & \seti    &           & \gc\sid &            & \gc\setu & \gc\seti &          &          &                  \\
  21. & \sid      & \setu   & \setd    &    \szero & \gc\sid &    \setopc & \gc\setu &    \seti & \gc\setd &    \sets & algebra of sets  \\
  22. & \sid      & \setu   & \sets    &    \szero & \gc\sid &    \setopc & \gc\setu &    \seti &    \setd & \gc\sets & algebra of sets  \\
  23. & \sid      & \seti   & \setd    &    \szero & \gc\sid &    \setopc &    \setu & \gc\seti & \gc\setd &    \sets & algebra of sets  \\
  24. & \sid      & \seti   & \sets    &    \szero & \gc\sid &    \setopc &    \setu & \gc\seti &    \setd & \gc\sets & algebra of sets  \\
  25. & \sid      & \setd   & \sets    &    \szero & \gc\sid &    \setopc &    \setu &    \seti & \gc\setd & \gc\sets & algebra of sets  \\
  26. & \setopc   & \setu   & \seti    &    \szero &    \sid & \gc\setopc & \gc\setu & \gc\seti &    \setd &    \sets & algebra of sets  \\
  27. & \setopc   & \setu   & \setd    &    \szero &    \sid & \gc\setopc & \gc\setu &    \seti & \gc\setd &    \sets & algebra of sets  \\
  28. & \setopc   & \setu   & \sets    &    \szero &    \sid & \gc\setopc & \gc\setu &    \seti &    \setd & \gc\sets & algebra of sets  \\
  29. & \setopc   & \seti   & \setd    &    \szero &    \sid & \gc\setopc &    \setu & \gc\seti & \gc\setd &    \sets & algebra of sets  \\
  30. & \setopc   & \seti   & \sets    &    \szero &    \sid & \gc\setopc &    \setu & \gc\seti &    \setd & \gc\sets & algebra of sets  \\
  31. & \setopc   & \setd   & \sets    &    \szero &    \sid & \gc\setopc &    \setu &    \seti & \gc\setd & \gc\sets & algebra of sets  \\
  32. & \setu     & \seti   & \setd    &    \szero &         &            & \gc\setu & \gc\seti & \gc\setd &    \sets & ring of sets     \\
  33. & \setu     & \seti   & \sets    &    \szero &         &            & \gc\setu & \gc\seti &    \setd & \gc\sets & ring of sets     \\
  34. & \setu     & \setd   & \sets    &    \szero &         &            & \gc\setu &    \seti & \gc\setd & \gc\sets & ring of sets     \\
  35. & \seti     & \setd   & \sets    &    \szero &         &            &    \setu & \gc\seti & \gc\setd & \gc\sets & ring of sets     \\
  \hline
\end{longtable}
%\end{small}
\end{proposition}

%---------------------------------------
\begin{proposition}[4 generators]
\label{prop:ss_gggg}
%---------------------------------------
The following table demonstrates the ``standard" operations generated by sets of 4 operations.
\begin{longtable}{|r|NNNN|NNNNNNN|l|}
  \mc{1}{|B|}{} & \mc{4}{B|}{\text{generators}} & \mc{7}{B|}{\text{generated operations}} & \mc{1}{B}{\text{induced set structure}}\\
  \hline
   1. & \szero & \sid  & \setopc  & \setu  & \gc\szero & \gc\sid& \gc\setopc & \gc\setu &    \seti &    \setd &    \sets & algebra of sets \\
   2. & \szero & \sid  & \setopc  & \seti  & \gc\szero & \gc\sid& \gc\setopc &    \setu & \gc\seti &    \setd &    \sets & algebra of sets \\
   3. & \szero & \sid  & \setopc  & \setd  & \gc\szero & \gc\sid& \gc\setopc &    \setu &    \seti & \gc\setd &    \sets & algebra of sets \\
   4. & \szero & \sid  & \setopc  & \sets  & \gc\szero & \gc\sid& \gc\setopc &          &          &          & \gc\sets &                 \\
   5. & \szero & \sid  & \setu    & \seti  & \gc\szero & \gc\sid&            & \gc\setu & \gc\seti &          &          & pre-topology    \\
   6. & \szero & \sid  & \setu    & \setd  & \gc\szero & \gc\sid&    \setopc & \gc\setu &    \seti & \gc\setd &    \sets & algebra of sets \\
   7. & \szero & \sid  & \setu    & \sets  & \gc\szero & \gc\sid&    \setopc & \gc\setu &    \seti &    \setd & \gc\sets & algebra of sets \\
   8. & \szero & \sid  & \seti    & \setd  & \gc\szero & \gc\sid&    \setopc &    \setu & \gc\seti & \gc\setd &    \sets & algebra of sets \\
   9. & \szero & \sid  & \seti    & \sets  & \gc\szero & \gc\sid&    \setopc &    \setu & \gc\seti &    \setd & \gc\sets & algebra of sets \\
  10. & \szero & \sid  & \setd    & \sets  & \gc\szero & \gc\sid&    \setopc &    \setu &    \seti & \gc\setd & \gc\sets & algebra of sets \\
  11. & \szero & \setopc & \setu    & \seti  & \gc\szero &    \sid& \gc\setopc & \gc\setu & \gc\seti &    \setd &    \sets & algebra of sets \\
  12. & \szero & \setopc & \setu    & \setd  & \gc\szero &    \sid& \gc\setopc & \gc\setu &    \seti & \gc\setd &    \sets & algebra of sets \\
  13. & \szero & \setopc & \setu    & \sets  & \gc\szero &    \sid& \gc\setopc & \gc\setu &    \seti &    \setd & \gc\sets & algebra of sets \\
  14. & \szero & \setopc & \seti    & \setd  & \gc\szero &    \sid& \gc\setopc &    \setu & \gc\seti & \gc\setd &    \sets & algebra of sets \\
  15. & \szero & \setopc & \seti    & \sets  & \gc\szero &    \sid& \gc\setopc &    \setu & \gc\seti &    \setd & \gc\sets & algebra of sets \\
  16. & \szero & \setopc & \setd    & \sets  & \gc\szero &    \sid& \gc\setopc &    \setu &    \seti & \gc\setd & \gc\sets & algebra of sets \\
  17. & \szero & \setu   & \seti    & \setd  & \gc\szero &          &            & \gc\setu & \gc\seti & \gc\setd &    \sets & ring of sets    \\
  18. & \szero & \setu   & \seti    & \sets  & \gc\szero &          &            & \gc\setu & \gc\seti &    \setd & \gc\sets & ring of sets    \\
  19. & \szero & \setu   & \setd    & \sets  & \gc\szero &          &            & \gc\setu &    \seti & \gc\setd & \gc\sets & ring of sets    \\
  20. & \szero & \seti   & \setd    & \sets  & \gc\szero &          &            &    \setu & \gc\seti & \gc\setd & \gc\sets & ring of sets    \\
  21. & \sid    & \setopc & \setu    & \seti  &    \szero & \gc\sid& \gc\setopc & \gc\setu & \gc\seti &    \setd &    \sets & algebra of sets \\
  22. & \sid    & \setopc & \setu    & \setd  &    \szero & \gc\sid& \gc\setopc & \gc\setu &    \seti & \gc\setd &    \sets & algebra of sets \\
  23. & \sid    & \setopc & \setu    & \sets  &    \szero & \gc\sid& \gc\setopc & \gc\setu &    \seti &    \setd & \gc\sets & algebra of sets \\
  24. & \sid    & \setopc & \seti    & \setd  &    \szero & \gc\sid& \gc\setopc &    \setu & \gc\seti & \gc\setd &    \sets & algebra of sets \\
  25. & \sid    & \setopc & \seti    & \sets  &    \szero & \gc\sid& \gc\setopc &    \setu & \gc\seti &    \setd & \gc\sets & algebra of sets \\
  26. & \sid    & \setopc & \setd    & \sets  &    \szero & \gc\sid& \gc\setopc &    \setu &    \seti & \gc\setd & \gc\sets & algebra of sets \\
  27. & \sid    & \setu   & \seti    & \setd  &    \szero & \gc\sid&    \setopc & \gc\setu & \gc\seti & \gc\setd &    \sets & algebra of sets \\
  28. & \sid    & \setu   & \seti    & \sets  &    \szero & \gc\sid&    \setopc & \gc\setu & \gc\seti &    \setd & \gc\sets & algebra of sets \\
  29. & \sid    & \setu   & \setd    & \sets  &    \szero & \gc\sid&    \setopc & \gc\setu &    \seti & \gc\setd & \gc\sets & algebra of sets \\
  30. & \sid    & \seti   & \setd    & \sets  &    \szero & \gc\sid&    \setopc &    \setu & \gc\seti & \gc\setd & \gc\sets & algebra of sets \\
  31. & \setopc   & \setu   & \seti    & \setd  &    \szero &    \sid& \gc\setopc & \gc\setu & \gc\seti & \gc\setd &    \sets & algebra of sets \\
  32. & \setopc   & \setu   & \seti    & \sets  &    \szero &    \sid& \gc\setopc & \gc\setu & \gc\seti &    \setd & \gc\sets & algebra of sets \\
  33. & \setopc   & \setu   & \setd    & \sets  &    \szero &    \sid& \gc\setopc & \gc\setu &    \seti & \gc\setd & \gc\sets & algebra of sets \\
  34. & \setopc   & \seti   & \setd    & \sets  &    \szero &    \sid& \gc\setopc &    \setu & \gc\seti & \gc\setd & \gc\sets & algebra of sets \\
  35. & \setu     & \seti   & \setd    & \sets  &    \szero &    \sid&    \setopc & \gc\setu & \gc\seti & \gc\setd & \gc\sets & algebra of sets \\
  \hline
\end{longtable}
\end{proposition}


%---------------------------------------
\begin{proposition}[5 generators]
\label{prop:ss_ggggg}
%---------------------------------------
The following table demonstrates the ``standard" operations generated by sets of 5 operations.
\begin{longtable}{|r|NNNNN|NNNNNNN|l|}
  \hline
  \mc{1}{B}{} & \mc{5}{B|}{\text{generators}} & \mc{7}{B|}{\text{generated operations}} & \mc{1}{B|}{\text{induced set structure}}\\
  \hline
   1. & \szero & \sid  & \setopc  & \setu  & \seti & \gc\szero & \gc\sid& \gc\setopc & \gc\setu & \gc\seti &    \setd &    \sets & algebra of sets \\
   2. & \szero & \sid  & \setopc  & \setu  & \setd & \gc\szero & \gc\sid& \gc\setopc & \gc\setu &    \seti & \gc\setd &    \sets & algebra of sets \\
   3. & \szero & \sid  & \setopc  & \setu  & \sets & \gc\szero & \gc\sid& \gc\setopc & \gc\setu &    \seti &    \setd & \gc\sets & algebra of sets \\
   4. & \szero & \sid  & \setopc  & \seti  & \setd & \gc\szero & \gc\sid& \gc\setopc &    \setu & \gc\seti & \gc\setd &    \sets & algebra of sets \\
   5. & \szero & \sid  & \setopc  & \seti  & \sets & \gc\szero & \gc\sid& \gc\setopc &    \setu & \gc\seti &    \setd & \gc\sets & algebra of sets \\
   6. & \szero & \sid  & \setopc  & \setd  & \sets & \gc\szero & \gc\sid& \gc\setopc &    \setu &    \seti & \gc\setd & \gc\sets & algebra of sets \\
   7. & \szero & \sid  & \setu    & \seti  & \setd & \gc\szero & \gc\sid&    \setopc & \gc\setu & \gc\seti & \gc\setd &    \sets & algebra of sets \\
   8. & \szero & \sid  & \setu    & \seti  & \sets & \gc\szero & \gc\sid&    \setopc & \gc\setu & \gc\seti &    \setd & \gc\sets & algebra of sets \\
   9. & \szero & \sid  & \setu    & \setd  & \sets & \gc\szero & \gc\sid&    \setopc & \gc\setu &    \seti & \gc\setd & \gc\sets & algebra of sets \\
  10. & \szero & \sid  & \seti    & \setd  & \sets & \gc\szero & \gc\sid&    \setopc &    \setu & \gc\seti & \gc\setd & \gc\sets & algebra of sets \\
  11. & \szero & \setopc & \setu    & \seti  & \setd & \gc\szero &    \sid& \gc\setopc & \gc\setu & \gc\seti & \gc\setd &    \sets & algebra of sets \\
  12. & \szero & \setopc & \setu    & \seti  & \sets & \gc\szero &    \sid& \gc\setopc & \gc\setu & \gc\seti &    \setd & \gc\sets & algebra of sets \\
  13. & \szero & \setopc & \setu    & \setd  & \sets & \gc\szero &    \sid& \gc\setopc & \gc\setu &    \seti & \gc\setd & \gc\sets & algebra of sets \\
  14. & \szero & \setopc & \seti    & \setd  & \sets & \gc\szero &    \sid& \gc\setopc &    \setu & \gc\seti & \gc\setd & \gc\sets & algebra of sets \\
  15. & \szero & \setu   & \seti    & \setd  & \sets & \gc\szero &          &            & \gc\setu & \gc\seti & \gc\setd & \gc\sets & ring of sets    \\
  16. & \sid    & \setopc & \setu    & \seti  & \setd &    \szero & \gc\sid& \gc\setopc & \gc\setu & \gc\seti & \gc\setd &    \sets & algebra of sets \\
  17. & \sid    & \setopc & \setu    & \seti  & \sets &    \szero & \gc\sid& \gc\setopc & \gc\setu & \gc\seti &    \setd & \gc\sets & algebra of sets \\
  18. & \sid    & \setopc & \setu    & \setd  & \sets &    \szero & \gc\sid& \gc\setopc & \gc\setu &    \seti & \gc\setd & \gc\sets & algebra of sets \\
  19. & \sid    & \setopc & \seti    & \setd  & \sets &    \szero & \gc\sid& \gc\setopc &    \setu & \gc\seti & \gc\setd & \gc\sets & algebra of sets \\
  20. & \sid    & \setu   & \seti    & \setd  & \sets &    \szero & \gc\sid&    \setopc & \gc\setu & \gc\seti & \gc\setd & \gc\sets & algebra of sets \\
  21. & \setopc   & \setu   & \seti    & \setd  & \sets &    \szero &    \sid& \gc\setopc & \gc\setu & \gc\seti & \gc\setd & \gc\sets & algebra of sets \\
  \hline
\end{longtable}
\end{proposition}

%---------------------------------------
\begin{proposition}[6 genererators]
\label{prop:ss_gggggg}
%---------------------------------------
The following table demonstrates the ``standard" operations generated by sets of 6 operations.
\begin{longtable}{|r|NNNNNN|NNNNNNN|l|}
  \mc{1}{|B|}{} & \mc{6}{B|}{\text{generators}} & \mc{7}{B|}{\text{generated operations}} & \mc{1}{B}{\text{induced set structure}}\\
  \hline
   1. & \szero & \sid  & \setopc  & \setu  & \seti & \setd & \gc\szero & \gc\sid& \gc\setopc & \gc\setu & \gc\seti & \gc\setd &    \sets & algebra of sets \\
   2. & \szero & \sid  & \setopc  & \setu  & \seti & \sets & \gc\szero & \gc\sid& \gc\setopc & \gc\setu & \gc\seti &    \setd & \gc\sets & algebra of sets \\
   3. & \szero & \sid  & \setopc  & \setu  & \setd & \sets & \gc\szero & \gc\sid& \gc\setopc & \gc\setu &    \seti & \gc\setd & \gc\sets & algebra of sets \\
   4. & \szero & \sid  & \setopc  & \seti  & \setd & \sets & \gc\szero & \gc\sid& \gc\setopc &    \setu & \gc\seti & \gc\setd & \gc\sets & algebra of sets \\
   5. & \szero & \sid  & \setu    & \seti  & \setd & \sets & \gc\szero & \gc\sid&    \setopc & \gc\setu & \gc\seti & \gc\setd & \gc\sets & algebra of sets \\
   6. & \szero & \setopc & \setu    & \seti  & \setd & \sets & \gc\szero &    \sid& \gc\setopc & \gc\setu & \gc\seti & \gc\setd & \gc\sets & algebra of sets \\
   7. & \sid    & \setopc & \setu    & \seti  & \setd & \sets &    \szero & \gc\sid& \gc\setopc & \gc\setu & \gc\seti & \gc\setd & \gc\sets & algebra of sets \\
  \hline
\end{longtable}
\end{proposition}
} %end mssa exclude


%%=======================================
%\subsection{Additional set operations}
%%=======================================
%Besides the seven set operations of \pref{def:ss_setops} and
%using the 16 possible logic operations as a model \xref{tbl:log_op},
%additional useful set operations are possible
%(see also the table on \pageref{ba_log_set_cs}),
%as presented in \pref{def:ss_rel} (next).
%%---------------------------------------
%\begin{definition}%[additional set operations]
%\label{def:ss_rel}
%%---------------------------------------
%%Let $\logsysd$ be a propositional logic system.
%%Let $x'\eqd \lnot x$ and $y'\eqd \lnot y$.%, and juxtaposition represent $\land$.
%The following table defines additional operations on $\psetx$ in terms of
%$\sor$, $\sand$, and $\snot$.
%\defbox{\begin{array}{Mc|l@{\,}c@{\,}l   @{\;}c@{\;}  l @{\,}r@{\,}c@{\,}r |C}
%  \mc{2}{N|}{name/symbol}    & \mc{8}{N|}{definition} & \mc{1}{N}{domain}
%  \\\hline
%  \hid{rejection}            & \indxsh{\snor}     & \setA &\snor   & \setB &\eqd& \big\{x\in\setX \big| & \lnot(x\in\setA) &\land& \lnot(x\in\setB)\big\} & \forall \setA,\setB\in\psetx \\ % 0001
%  \hid{inhibit x}            & \indxsh{\sinx}     & \setA &\sinx   & \setB &\eqd& \big\{x\in\setX \big| & \lnot(x\in\setA) &\land&      (x\in\setB)\big\} & \forall \setA,\setB\in\psetx \\ % 0010
%  \hid{Sheffer stroke}       & \indxsh{\snand}    & \setA &\snand  & \setB &\eqd& \big\{x\in\setX \big| & \lnot(x\in\setA) &\lor & \lnot(x\in\setB)\big\} & \forall \setA,\setB\in\psetx \\ % 0111
%  \hid{implication}          & \indxsh{\simpl}    & \setA &\simpl  & \setB &\eqd& \big\{x\in\setX \big| & \lnot(x\in\setA) &\lor &      (x\in\setB)\big\} & \forall \setA,\setB\in\psetx \\ % 1011
%  \hid{adjunction}           & \indxsh{\sadj}     & \setA &\sadj& \setB &\eqd& \big\{x\in\setX \big| &      (x\in\setA) &\lor & \lnot(x\in\setB)\big\} & \forall \setA,\setB\in\psetx \\ % 1101
%\end{array}}
%\end{definition}
%

%=======================================
\subsection{Set multiplication}
%=======================================
\ifdochaselse{found}{
  The \hie{Cartesian product} operation $\cprod$ \xref{def:AxB}
  is a kind of \emph{set multiplication} operation.
  }{
  The \hie{Cartesian product} operation $\cprod$ (next definition)
  is a kind of \emph{set multiplication} operation.
  %---------------------------------------
  \begin{definition}
  \label{def:AxB}
  \citetbl{
    \citerpg{halmos1960}{24}{0387900926}\\
    G. Frege, 2007 August 25, \url{http://groups.google.com/group/sci.logic/msg/3b3294f5ac3a76f0}
    }
  %---------------------------------------
  Let $\setX$ and $\setY$ be sets, and let $\opair{x}{y}$ be an \structe{ordered pair}.
  \defbox{\begin{array}{M}\indxs{\cprod}
    The \hid{Cartesian product} $\cprodXY$ of $\setX$ and $\setY$ is
    \\\indentx$\ds \cprodXY \eqd \set{\opair{x}{y}}{(x\in\setX)\; \text{and} \; (y\in\setY)}$
  \end{array}}
  \end{definition}
  }
\pref{thm:xui} (next theorem) demonstrates how this set operation interacts with
certain other set operations.
The Cartesian product is of critical importance in general because, for example,
relations \ifdochas{relation}{\xrefP{def:relation}} and functions \ifdochas{relation}{\xrefP{def:function}}
are subsets of Cartesian products.
%---------------------------------------
\begin{theorem}
\label{thm:xui}
\citetbl{
  \citerp{menini2004}{50},
  \citerpg{halmos1960}{25}{0387900926}
  }
%---------------------------------------
Let $\setX,\setY,\setZ$ be sets.
\thmbox{\begin{array}{rcl@{\qquad}D}
  \setX \cprod (\setY\setu\setZ)
    &=& (\cprodXY) \setu (\cprodXZ)
    &   ($\cprod$ distributes over $\setu$)
    \\
  \setX \cprod (Y\seti Z)
    &=& (\cprodXY) \seti (\cprodXZ)
    &   ($\cprod$ distributes over $\seti$)
    \\
  \setX \cprod (\setY\setd\setZ)
    &=& (\cprodXY) \setd (\cprodXZ)
    & ($\cprod$ distributes over $\setd$)
    \\
  (\cprodXY) \seti (\setY\cprod\setX)
    &=& (\setX\seti\setY) \cprod (Y\seti X)
    &
    \\
  (\setX\cprod\setX) \seti (\setY\cprod\setY)
    &=& (\setX\seti\setY) \cprod (\setX\seti\setY)
    &
  \end{array}}
\end{theorem}
\begin{proof}
\begin{align*}
  \setX \cprod (Y\setu\setZ)
    &= \set{\opair{a}{b}}{(a\in\setX) \land (b\in\setY\setu\setZ)}
    && \ifdochas{found}{\text{by \pref{def:(a,b)}}}
  \\&= \set{\opair{a}{b}}{(a\in\setX) \land \left[(b\in\setY)\lor(b\setu\setZ)\right]}
    && \text{by \pref{def:ss_setops}}
  \\&= \set{\opair{a}{b}}{\left[(a\in\setX) \land (b\in\setY)\right]\lor\left[(a\in\setX)\land(b\in Z)\right]}
    && \ifdochas{logic}{\text{by \pref{thm:logic}}}
  \\&= \mcom{\set{\opair{a}{b}}{\left[(a\in\setX) \land (b\in\setY)\right]}}{$\cprodXY$}
       \setu
       \mcom{\set{\opair{a}{b}}{\left[(a\in\setX) \land (b\in Z)\right]}}{$\cprodXZ$}
    && \text{by \pref{def:ss_setops}}
  \\&= (\cprodXY) \setu (\cprodXZ)
    && \ifdochas{found}{\text{by \pref{def:(a,b)}}}
  \\
  \\
  \setX \cprod (Y\seti Z)
    &= \set{\opair{a}{b}}{(a\in\setX) \land (b\in\setY\seti Z)}
    && \ifdochas{found}{\text{by \pref{def:(a,b)}}}
  \\&= \set{\opair{a}{b}}{(a\in\setX) \land \left[(b\in\setY)\land(b\setu\setZ)\right]}
    && \text{by \pref{def:ss_setops}}
  \\&= \set{\opair{a}{b}}{\left[(a\in\setX) \land (b\in\setY)\right]\land\left[(a\in\setX)\land(b\in Z)\right]}
    && %\text{by \pref{thm:logic_distrib}}
  \\&= \mcom{\set{\opair{a}{b}}{\left[(a\in\setX) \land (b\in\setY)\right]}}{$\cprodXY$} \seti
       \mcom{\set{\opair{a}{b}}{\left[(a\in\setX) \land (b\in Z)\right]}}{$\cprodXZ$}
    && \text{by \pref{def:ss_setops}}
  \\&= (\cprodXY) \seti (\cprodXZ)
    && \ifdochas{found}{\text{by \pref{def:(a,b)}}}
  \\
  \\
  \setX \cprod (\setY\setd\setZ)
    &= \set{\opair{a}{b}}{(a\in\setX) \land (b\in\setY\setd\setZ)}
    && \ifdochas{found}{\text{by \pref{def:(a,b)}}}
  \\&= \set{\opair{a}{b}}{(a\in\setX) \land (b\in\setY\seti \cmpZ)}
    && \ifdochas{found}{\text{by \pref{thm:ss_rel_gg}}}
  \\&= \set{\opair{a}{b}}{(a\in\setX) \land \left[(b\in\setY)\land(b\in \cmpZ)\right]}
    && \text{by \pref{def:ss_setops}}
  \\&= \set{\opair{a}{b}}{\left[(a\in\setX) \land (b\in\setY)\right]\land\left[(a\in\setX)\land(b\in \cmpZ)\right]}
    && %\text{by \pref{thm:logic_distrib}}
  \\&= \mcom{\set{\opair{a}{b}}{\left[(a\in\setX) \land (b\in\setY)\right]}  }{$\cprodXY$}\seti
       \mcom{\set{\opair{a}{b}}{\left[(a\in\setX) \land (b\in \cmpZ)\right]}}{$X \cprod \cmpZ$}
    && \text{by \pref{def:ss_setops}}
  \\&= (\cprodXY) \seti (\setX\cprod \cmpZ)
    && \ifdochas{found}{\text{by \pref{def:(a,b)}}}
  \\&\ne  (\cprodXY) \setd (\cprodXZ)
  \\
  \\
  (\cprodXY) \seti (\setY\cprod\setX)
    &= \set{\opair{a}{b}}{(a\in\setX) \land (b\in\setY)} \seti
       \set{\opair{a}{b}}{(a\in\setY) \land (b\in\setX)}
    && \ifdochas{found}{\text{by \pref{def:(a,b)}}}
  \\&= \set{\opair{a}{b}}{[(a\in\setX) \land (b\in\setY)] \land [(a\in\setY) \land (b\in\setX)]}
    && \text{by \pref{def:ss_setops}}
  \\&= \set{\opair{a}{b}}{[(a\in\setX) \land (a\in\setY)] \land [(b\in\setY) \land (b\in\setX)]}
    && %\text{by \pref{def:ss_setops}}
  \\&= \set{\opair{a}{b}}{(a\in\setX\seti\setY) \land (b\in\setY\seti X)}
  \\&= (\setX\seti\setY) \cprod (Y\seti X)
    && \ifdochas{found}{\text{by \pref{def:(a,b)}}}
  \\
  \\
  (\setX\cprod\setX) \seti (\setY\cprod\setY)
    &= \set{\opair{a}{b}}{(a\in\setX) \land (b\in\setX)} \seti
       \set{\opair{a}{b}}{(a\in\setY) \land (b\in\setY)}
    && \ifdochas{found}{\text{by \pref{def:(a,b)}}}
  \\&= \set{\opair{a}{b}}{[(a\in\setX) \land (b\in\setX)] \;\land\; [(a\in\setY) \land (b\in\setY)]}
    && \text{by \pref{def:ss_setops}}
  \\&= \set{\opair{a}{b}}{[(a\in\setX) \land (a\in\setY)] \;\land\; [(b\in\setX) \land (b\in\setY)]}
  \\&= \set{\opair{a}{b}}{(a\in\setX\seti\setY) \;\land\; (b\in\setX\seti\setY)}
  \\&= (\setX\seti\setY) \cprod (\setX\seti\setY)
    && \ifdochas{found}{\text{by \pref{def:(a,b)}}}
\end{align*}
\end{proof}

%=======================================
\section{Standard set structures}
%=======================================
%=======================================
%\subsection{Definitions}
%=======================================

%While the previous section discusses arbitrary set structures,
Set structures are typically designed to satisfy some special properties---
such as being closed with respect to certain set operations.
Examples of commonly occurring set structures include
\\\begin{tabular}{@{\qquad$\imark$\hspace{1ex}}l>{(}ll<{)}}
  \structe{power set}       & \pref{def:pset}         & \prefpo{def:pset}  \\
  \structe{topologies}      & \pref{def:topology}  & \prefpo{def:topology}\\
  \structe{algebra of sets} & \pref{def:ss_algebra}   & \prefpo{def:ss_algebra}  \\
  \structe{ring of sets}    & \pref{def:ss_ring}      & \prefpo{def:ss_ring}     \\
  \structe{partitions}      & \pref{def:ss_partition} & \prefpo{def:ss_partition} 
\end{tabular}

%=======================================
\subsection{Topologies}
%=======================================
\ifdochaselse{topology}{See \prefpp{chp:topology}}{
%---------------------------------------
\begin{definition}
\label{def:fintop}
%\label{def:ss_open}
%\label{def:ss_closed}
\label{def:topology}
\label{def:topspace}
\label{def:openset}
\label{def:closedset}
\footnote{
  %\citerpp{davis2005}{41}{42},
  \citerpg{munkres2000}{76}{0131816292},
  %\citerpu{ab}{57}{0120502577},
  %\citerp{vel1993}{3},
  \citor{riesz1909},
  \citor{hausdorff1914},
  \citePc{tietze1923}{cited by Thron page 18},
  \citerpg{hausdorff1937e}{258}{0828401195}
  %By modifying 3 to closure under finite intersections, 1 and 2 are implied by 
  }
\index{space!topological}
%---------------------------------------
Let $\Gamma$ be a set with an arbitrary (possibly uncountable) number of elements.
Let $\psetX$ be the \structe{power set} of a set $\setX$.
\defbox{\begin{array}{M}\indxs{\topT}\indxs{\sssTx}
  A family of sets $\topT\subseteq\psetX$ is a \structd{topology} on a set $\sid$ if
  \\\indentx
  $\begin{array}{F>{\ds}l>{\ds}lDD}
      1. & \szero \in \topT
         &
         & ($\szero$ is in $\topT$)
         & and
    \\2. & \sid \in \topT
         &
         & ($\sid$ is in $\topT$)
         & and
    \\3. & \setU,\setV\in\topT
         & \implies \setU\seti\setV\in\topT
         & (the intersection of a finite number of open sets is open)
         & and
    \\4. & \set{\setU_\gamma}{\gamma\in\Gamma} \subseteq \topT
         & \implies \setopu_{\gamma\in\Gamma}\setU_\gamma \in \topT
         & (the union of an arbitrary number of open sets is open).
  \end{array}$
  \\
  A \structd{topological space} is the pair $\topspace{\sid}{\topT}$.
  An \structd{open set} is any member of $\topT$.\\
  A \structd{closed set} is any set $\setD$ such that $\cmpD$ is \prope{open}.\\
  The set of topologies on a set $\sid$ is denoted $\sssT{\sid}$. That is,
  \\\indentx$\ds\sssT{\sid} \eqd \set{\topT\sorel\pset{\sid}}{\text{$\topT$ is a \structd{topology}}}$.
  \\If $\setX$ is \prope{finite}, then $\topT$ is a \structd{topology on a finite set}, and (4.) can be replaced by 
  \\\indentx$\setU,\setV\in\topT \quad\implies\quad \setU\setu\setV \in \topT$.
\end{array}}
\end{definition}



%Just as the power set $\psetx$ and the set $\setn{\szero,\sid}$ are algebras of sets
%on a set $\sid$ \xref{prop:ss_algebra_smallest}, 
%so also are these sets topologies on $\sid$ (next proposition).




%---------------------------------------
\begin{example}
\index{topologies!trivial}
\index{topologies!indiscrete}
\index{topologies!discrete}
\footnote{
  \citerp{munkres2000}{77},
  \citerpgc{kubrusly2011}{107}{0817649972}{Example 3.J},
  \citerppgc{steen1978}{42}{43}{0387903127}{II.4},
  \citerpg{dibenedetto}{18}{0817642315}
  }
\label{ex:discretetop}
\label{ex:indiscretetop}
%---------------------------------------
Let $\sssT{\sid}$ be the set of topologies on a set $\sid$ and
$\pset{\sid}$ the \structe{power set} \xref{def:pset} on $\sid$.
\exbox{\begin{array}{lMl@{\qquad}D}
  \setn{\szero,\,\sid} &is a \structe{topology} in& \sssTx & (\structe{indiscrete topology} or \structe{trivial topology}) \\
  \psetx               &is a \structe{topology} in& \sssTx & (\structe{discrete topology})
\end{array}}
\end{example}


%---------------------------------------
\begin{example}
\citetbl{
  \citerp{isham1999}{44},
  \citerp{isham1989}{1515}
  }%
\label{ex:top_xy}
\label{ex:top_closed_xy}
%---------------------------------------
There are four topologies on the set $\setX\eqd\setn{x,y}$:
\exbox{
  %\begin{tabular}{|>{$}l<{$} @{$\;=\;\{$} *{4}{>{$}l<{$}@{\,}} @{$\}$\quad}|@{\quad$\{$} *{4}{>{$}l<{$}@{\,}} @{$\}\quad$} c|}
  \begin{array}{l@{\;=\;\{} *{4}{l@{\,}} @{\}\quad}|@{\quad\{} *{4}{l@{\,}} @{\}\quad} c|}
    \hline
    \mc{5}{N|}{topologies on $\setn{x,y}$}&\mc{5}{N}{corresponding closed sets}%
    \\\hline
        \topT_{0} & \emptyset, &           &           &  \setX
                  & \emptyset, &           &           &  \setX  &
      \\\topT_{1} & \emptyset, & \setn{x}, &           &  \setX
                  & \emptyset, &           & \setn{y}, &  \setX  &
      \\\topT_{2} & \emptyset, &           & \setn{y}, &  \setX
                  & \emptyset, & \setn{x}, &           &  \setX  &
      \\\topT_{3} & \emptyset, & \setn{x}, & \setn{y}, &  \setX
                  & \emptyset, & \setn{x}, & \setn{y}, &  \setX  &
  \end{array}}
  \\
  The topologies $\topspace{\setX}{\topT_1}$ and $\topspace{\setX}{\topT_2}$, as well as their corresponding closed set topological spaces,
  are all \structe{Serpi/'nski spaces}.
\end{example}

%---------------------------------------
\begin{example}
\label{ex:top_xyz}
\label{ex:top_closed_xyz}
%---------------------------------------
There are a total of 29 \structe{topologies} \xref{def:topology} on the set $\setX\eqd\setn{x,y,z}$:
{\footnotesize\arrayrulecolor{ex}
\begin{longtable}{|>{$}l<{$} @{$\;=\;\{$} *{8}{>{$}l<{$}@{\,}} @{$\}$\quad}  | @{\quad$\{$} *{8}{>{$}l<{$}@{\,}} @{$\}\quad$} |}
  \hline
  \mc{9}{G}{topologies on $\setn{x,y,z}$}&\mc{8}{G}{corresponding closed sets}
  \\\hline
    \topT_{00} & \emptyset, &           &           &           &             &             &             & \setX
               & \emptyset, &           &           &           &             &             &             & \setX
  \\\topT_{01} & \emptyset, & \setn{x}, &           &           &             &             &             & \setX
               & \emptyset, &           &           &           &             &             & \setn{y,z}, & \setX
  \\\topT_{02} & \emptyset, &           & \setn{y}, &           &             &             &             & \setX
               & \emptyset, &           &           &           &             & \setn{x,z}  &             & \setX
  \\\topT_{04} & \emptyset, &           &           & \setn{z}, &             &             &             & \setX
               & \emptyset, &           &           &           & \setn{x,y}, &             &             & \setX
  \\\topT_{10} & \emptyset, &           &           &           & \setn{x,y}, &             &             & \setX
               & \emptyset, &           &           & \setn{z}, &             &             &             & \setX
  \\\topT_{20} & \emptyset, &           &           &           &             & \setn{x,z}, &             & \setX
               & \emptyset, &           & \setn{y}, &           &             &             &             & \setX
  \\\topT_{40} & \emptyset, &           &           &           &             &             & \setn{y,z}, & \setX
               & \emptyset, & \setn{x}, &           &           &             &             &             & \setX
  \\\topT_{11} & \emptyset, & \setn{x}, &           &           & \setn{x,y}, &             &             & \setX
               & \emptyset, &           &           & \setn{z}, &             &             & \setn{y,z}, & \setX
  \\\topT_{21} & \emptyset, & \setn{x}, &           &           &             & \setn{x,z}, &             & \setX
               & \emptyset, &           & \setn{y}  &           &             &             & \setn{y,z}, & \setX
  \\\topT_{41} & \emptyset, & \setn{x}, &           &           &             &             & \setn{y,z}, & \setX
               & \emptyset, & \setn{x}, &           &           &             &             & \setn{y,z}, & \setX
  \\\topT_{12} & \emptyset, &           & \setn{y}, &           & \setn{x,y}, &             &             & \setX
               & \emptyset, &           &           & \setn{z}, &             & \setn{x,z}  &             & \setX
  \\\topT_{22} & \emptyset, &           & \setn{y}, &           &             & \setn{x,z}, &             & \setX
               & \emptyset, &           & \setn{y}, &           &             & \setn{x,z}, &             & \setX
  \\\topT_{42} & \emptyset, &           & \setn{y}, &           &             &             & \setn{y,z}, & \setX
               & \emptyset, & \setn{x}, &           &           &             & \setn{x,z}, &             & \setX
  \\\topT_{14} & \emptyset, &           &           & \setn{z}, & \setn{x,y}, &             &             & \setX
               & \emptyset, &           &           & \setn{z}, & \setn{x,y}, &             &             & \setX
  \\\topT_{24} & \emptyset, &           &           & \setn{z}, &             & \setn{x,z}, &             & \setX
               & \emptyset, &           & \setn{y}, &           & \setn{x,y}, &             &             & \setX
  \\\topT_{44} & \emptyset, &           &           & \setn{z}, &             &             & \setn{y,z}, & \setX
               & \emptyset, & \setn{x}, &           &           & \setn{x,y}, &             &             & \setX
  \\\topT_{31} & \emptyset, & \setn{x}, &           &           & \setn{x,y}, & \setn{x,z}, &             & \setX
               & \emptyset, &           & \setn{y}, & \setn{z}, &             &             & \setn{y,z}, & \setX
  \\\topT_{52} & \emptyset, &           & \setn{y}, &           & \setn{x,y}, &             & \setn{y,z}, & \setX
               & \emptyset, & \setn{x}, &           & \setn{z}, &             & \setn{x,z}, &             & \setX
  \\\topT_{64} & \emptyset, &           &           & \setn{z}, &             & \setn{x,z}, & \setn{y,z}, & \setX
               & \emptyset, & \setn{x}, & \setn{y}, &           & \setn{x,y}, &             &             & \setX
  \\\topT_{13} & \emptyset, & \setn{x}, & \setn{y}, &           & \setn{x,y}, &             &             & \setX
               & \emptyset, &           &           & \setn{z}, &             & \setn{x,z}, & \setn{y,z}, & \setX
  \\\topT_{25} & \emptyset, & \setn{x}, &           & \setn{z}, &             & \setn{x,z}, &             & \setX
               & \emptyset, &           & \setn{y}, &           & \setn{x,y}, &             & \setn{y,z}, & \setX
  \\\topT_{46} & \emptyset, &           & \setn{y}, & \setn{z}, &             &             & \setn{y,z}, & \setX
               & \emptyset, & \setn{x}, &           &           & \setn{x,y}, & \setn{x,z}, &             & \setX
  \\\topT_{33} & \emptyset, & \setn{x}, & \setn{y}, &           & \setn{x,y}, & \setn{x,z}, &             & \setX
               & \emptyset, &           & \setn{y}, & \setn{z}, &             & \setn{x,z}, & \setn{y,z}, & \setX
  \\\topT_{53} & \emptyset, & \setn{x}, & \setn{y}, &           & \setn{x,y}, &             & \setn{y,z}, & \setX
               & \emptyset, & \setn{x}, &           & \setn{z}, &             & \setn{x,z}, & \setn{y,z}, & \setX
  \\\topT_{35} & \emptyset, & \setn{x}, &           & \setn{z}, & \setn{x,y}, & \setn{x,z}, &             & \setX
               & \emptyset, &           & \setn{y}, & \setn{z}, & \setn{x,y}, &             & \setn{y,z}, & \setX
  \\\topT_{65} & \emptyset, & \setn{x}, &           & \setn{z}, &             & \setn{x,z}, & \setn{y,z}, & \setX
               & \emptyset, & \setn{x}, & \setn{y}, &           & \setn{x,y}, &             & \setn{y,z}, & \setX
  \\\topT_{56} & \emptyset, &           & \setn{y}, & \setn{z}, & \setn{x,y}, &             & \setn{y,z}, & \setX
               & \emptyset, & \setn{x}, &           & \setn{z}, & \setn{x,y}, & \setn{x,z}, &             & \setX
  \\\topT_{66} & \emptyset, &           & \setn{y}, & \setn{z}, &             & \setn{x,z}, & \setn{y,z}, & \setX
               & \emptyset, & \setn{x}, & \setn{y}, &           & \setn{x,y}, & \setn{x,z}, &             & \setX
  \\\topT_{77} & \emptyset, & \setn{x}, & \setn{y}, & \setn{z}, & \setn{x,y}, & \setn{x,z}, & \setn{y,z}, & \setX
               & \emptyset, & \setn{x}, & \setn{y}, & \setn{z}, & \setn{x,y}, & \setn{x,z}, & \setn{y,z}, & \setX
  \\\hline
\end{longtable}
}
\end{example}

%---------------------------------------
\begin{theorem}
%\citetbl{
%  \citerpg{gratzer2003}{85}{3764369965}\\
%  \citerpg{gratzer1971}{75}{0716704420}\\
%  \citor{stone1936} \\
%  \citor{birkhoff1933}?
%  }
\label{thm:latd_top}
%---------------------------------------
Let $\latL\eqd\latticed$ be a \hie{lattice}.
\thmbox{
  \text{$\topT$ is a \structe{topology}}
  \quad\implies\quad
  \text{$\lattice{\topT}{\sorel}{\sor}{\sand}$ is a \structe{distributive lattice}}
  }
\end{theorem}
\begin{proof}
%\begin{enumerate}
%  \item Proof that \hie{topology} $\implies$ \prope{distributive} \hie{lattice}:
    \begin{enumerate}
      \item \ifdochas{setstrct}{By \prefpp{prop:ss_order}, }$\topspace{\topS}{\sorel}$ is an \hie{ordered set}.
      \item \ifdochas{setstrct}{By \prefpp{prop:ss_bounds}, }$\sor$ is \ope{least upper bound} operation on $\topspace{\topS}{\sorel}$.
            and $\sand$ is \ope{greatest lower bound} operation on $\topspace{\topS}{\sorel}$.
      \item Therefore, \ifdochas{lattice}{by \prefpp{def:lattice}, }$\lattice{\topS}{\sorel}{\sor}{\sand}$ is a lattice. \label{item:ss_ui_lat}
      \item \ifdochas{lattice}{By \prefpp{thm:lattice}, }$\lattice{\topS}{\sorel}{\sor}{\sand}$ is 
            \prope{idempotent}, \prope{commutative}, \prope{associative}, and \prope{absorptive}.
      \item Proof that $\lattice{\topS}{\sorel}{\sor}{\sand}$ is \prope{distributive}: \label{item:ss_ui_dis}
        \begin{enumerate}
          \item Proof that $\setA\seti(\setB\setu\setC) = (\setA\seti\setB) \setu (\setA\seti\setC)$:
            \begin{align*}
              &\setA\seti(\setB\setu\setC) 
              \\&= \set{x\in\setX}{x\in\setA \land x\in(\setB\setu\setC)}
                && \text{by definition of $\seti$ \ifxref{setstrct}{def:ss_setops}}
              \\&= \set{x\in\setX}{x\in\setA \land x\in\set{x\in\setX}{x\in\setB \lor x\in\setC}}
                && \text{by definition of $\setu$ \ifxref{setstrct}{def:ss_setops}}
              \\&= \set{x\in\setX}{x\in\setA \land (x\in\setB \lor x\in\setC)}
              \\&= \set{x\in\setX}{(x\in\setA \land x\in\setB) \lor (x\in\setA \land x\in\setC)}
                && \text{\ifdochas{logic}{by \prefp{thm:logic}}}
              \\&= \set{x\in\setX}{x\in\setA \land x\in\setB} \sor \set{x\in\setX}{x\in\setA \land x\in\setC}
                && \text{by definition of $\setu$ \ifxref{setstrct}{def:ss_setops}}
              \\&= (\setA\seti\setB) \sor (\setA\seti\setC)
                && \text{by definition of $\seti$ \ifxref{setstrct}{def:ss_setops}}
            \end{align*}
    
          \item Proof that $\setA\setu(\setB\seti\setC) = (\setA\setu\setB) \seti (\setA\setu\setC)$:\\
            This follows from the fact that $\lattice{\topS}{\sorel}{\sor}{\sand}$ is a lattice\ifsxref{lattice}{item:ss_ui_lat},
            that $\sand$ distributes over $\sor$\ifsxref{setstrct}{item:ss_ui_dis}\ifdochas{latd}{, and by \prefpp{thm:lat_dis}}.
        \end{enumerate}
    \end{enumerate}


%  \item Proof that \prope{distributive} \hie{lattice} $\implies$ \hie{topology}:


%\end{enumerate}
\end{proof}



%---------------------------------------
\begin{example}
\label{ex:ss_set5}
%---------------------------------------
There are five unlabeled lattices on a five element 
set\ifsxref{lattice}{prop:num_lattices}.
Of these five, three are 
\prope{distributive}\ifsxref{latd}{prop:lat_num_ldm}. % and illustrated in \prefpp{ex:lat_set5_distrib}.
The following illustrates that the distributive lattices are isomorphic
to topologies, while the non-distributive lattices are not.
\\
%\exbox{%============================================================================
% Daniel J. Greenhoe
% LaTeX file
% 5 element sets: non-distributive/distributive
%============================================================================
%
%\begin{tabular*}{\tw}{|c@{\extracolsep\fill}c|ccc|}
\begin{tabular}{cc|ccc}
%\begin{tabular*}{\tw}{|c@{\tfill}cccc|}
%\hline
%\mc{2}{|G|}{\prop{non-distributive}}&\mc{3}{G|}{\prop{distributive} lattices on 5 element set}
\mc{2}{c|}{\prope{non-distributive}}&\mc{3}{c}{\prope{distributive}}
\\\hline
\latmat{3}{
                & [name=1]\null                 \\
  [name=x]\null & [name=y]\null & [name=z]\null \\
                & [name=0]\null
  }{
  \ncline{1}{x} \ncline{1}{y} \ncline{1}{z}
  \ncline{0}{x} \ncline{0}{y} \ncline{0}{z}
  }
&
\latmat{5}{
                & [name=1]\null                 \\
  [name=y]\null                                 \\
                &               & [name=z]\null \\
  [name=x]\null                                 \\
                & [name=0]\null
  }{
  \ncline{1}{y}
  \ncline{y}{x}
  \ncline{x}{0}
  \ncline{1}{z}
  \ncline{z}{0}
  }
&
\latmat{4}{
                & [name=1]\null                 \\
                & [name=a]\null                 \\
  [name=x]\null &               & [name=y]\null \\
                & [name=0]\null
  }{
  \ncline{a}{1}
  \ncline{a}{x}\ncline{a}{y}
  \ncline{0}{x}\ncline{0}{y}
  }
&
\latmat{4}{
                & [name=1]\null                 \\
  [name=x]\null &               & [name=y]\null \\
                & [name=a]\null                 \\
                & [name=0]\null                 
  }{
  \ncline{1}{x}\ncline{1}{y}
  \ncline{a}{x}\ncline{a}{y}
  \ncline{0}{a}
  }
&
\latmat{5}{
  [name=1]\null \\
  [name=z]\null \\
  [name=y]\null \\
  [name=x]\null \\
  [name=0]\null \\
  }{
  \ncline{z}{1}
  \ncline{y}{z}
  \ncline{x}{y}
  \ncline{0}{x}
  }
%\\\hline
\end{tabular}



}
\exbox{\begin{tabular}{cc|ccc}%
   \mc{2}{c|}{\prope{non-distributive}/\emph{not} topologies}%
  &\mc{3}{c}{\prope{distributive}/\emph{are} topologies}%
  \\\hline%
   \includegraphics{../common/math/graphics/pdfs/lat5_m3_nontop.pdf}%
  &\includegraphics{../common/math/graphics/pdfs/lat5_n5_nontop.pdf}%
  &\includegraphics{../common/math/graphics/pdfs/lat5_l2onm2_top.pdf}%
  &\includegraphics{../common/math/graphics/pdfs/lat5_m2onl2_top.pdf}%
  &\includegraphics{../common/math/graphics/pdfs/lat5_l5_top.pdf}%
\end{tabular}}
\end{example}
\begin{proof}
\begin{enumerate}
  \item The first two lattices are non-distributive by
        \thme{Birkhoff distributivity criterion}\ifsxref{latd}{thm:latd_char_n5m3}.
    \begin{dingautolist}{"AC}
      \item This lattice is not a topology because, for example, 
            \\\indentx
            $\setn{x} \join \setn{y} = \setn{x,y,z} \ne  \setn{x,y} = \setn{x} \setu \setn{y}.$
            \\
            That is, the set union operation $\setu$ is \emph{not} equivalent to the 
            order join operation $\join$.
    
      \item This lattice is not a topology because, for example, 
            \\\indentx$\setn{x}\join\setn{y}=\setn{y}\ne \setn{x,y}=\setn{x}\setu\setn{y}$
    \end{dingautolist}
    
  \item The last three lattices are distributive by
        \thme{Birkhoff distributivity criterion}\ifsxref{latd}{thm:latd_char_n5m3}.
    \begin{dingautolist}{"AE}
      \item This lattice is the topology $\topT_{13}$ of \prefpp{ex:top_xyz}. 
            On the set $\setn{x,y,z}$, there are a total of three topologies 
            that have this order structure (see \pref{ex:top_xyz}):
            \\\indentx$\begin{array}{lcr *{5}{l} l}
                \topT_{13} &=& \{ & \emptyset, & \setn{x}, & \setn{y}, & \setn{x,y}, & \setn{x,y,z} & \}
              \\\topT_{25} &=& \{ & \emptyset, & \setn{x}, & \setn{z}, & \setn{x,z}, & \setn{x,y,z} & \}
              \\\topT_{46} &=& \{ & \emptyset, & \setn{y}, & \setn{z}, & \setn{y,z}, & \setn{x,y,z} & \}
            \end{array}$
    
      \item This lattice is the topology $\topT_{31}$ of \prefpp{ex:top_xyz}. 
            On the set $\setn{x,y,z}$, there are a total of three topologies 
            that have this order structure (see \pref{ex:top_xyz}):
            \\\indentx$\begin{array}{lcr *{5}{l} l}
                \topT_{31} &=& \{ & \emptyset, & \setn{x}, & \setn{x,y}, & \setn{x,z}, & \setn{x,y,z} & \}
              \\\topT_{52} &=& \{ & \emptyset, & \setn{y}, & \setn{x,y}, & \setn{y,z}, & \setn{x,y,z} & \}
              \\\topT_{64} &=& \{ & \emptyset, & \setn{z}, & \setn{x,z}, & \setn{y,z}, & \setn{x,y,z} & \}
            \end{array}$
    
      \item This lattice is a topology by \prefpp{def:topology}.
    
    \end{dingautolist}

\end{enumerate}
\end{proof}

} % end ifdochaselse topology



%=======================================
\subsection{Algebras of sets}
%=======================================
%An \hie{algebra of sets} (next definition) is a set structure that is closed
%with respect to the set complement operator $\setopc$ and the
%set intersection opeartor $\seti$.
%\prefpp{prop:ss_gg} demonstrates that because an algebra of sets on a set $\sid$ is closed
%with respect to $\setopc$ and $\seti$,
%it is also closed with respect to the remaining 5 standard
%set operations: $\szero$, $\sid$, $\setu$, $\setd$, and $\sets$.
%Closure under $\setopc$ and $\seti$ is
%not the only way to define an algebra of sets;
%rather, \prefpp{prop:ss_gg} demonstrates closure under
%the set complement operator $\setopc$ and the
%set union opeartor $\setu$ also generates an algebra of sets.
%And if we allow more than two set operations in the definition, then there are several
%more sets of set operations that generate algebras of sets;
%for example, \prefpp{prop:ss_ggg} demonstrates that
%the operations $\setn{\sid,\, \setd,\, \sets}$ generate an algebra of sets.
%---------------------------------------
\begin{definition}
\footnote{
  \citerpu{ab}{95}{0120502577},
  \citerpg{ab}{151}{0120502577},
  \citerpg{halmos1950}{21}{0387900888},
  \citerpg{hausdorff1937e}{91}{0828401195}
  }
\label{def:ss_algebra}
\label{def:algsets}
\label{def:sigalg}
%---------------------------------------
Let $\sid$ be a set with \structe{power set} $\psetx$ \xref{def:pset}.
\defbox{\indxs{\algA}\indxs{\sssAx}
  \begin{array}{@{\qquad}Flcl DD}
    \mc{6}{M}{$\algA\sorel\psetx$ is an \structd{algebra of sets} on $\sid$ if}\\
      1. & \setA\in\algA       &\implies& \cmpA\in\algA             & (closed under complement operation)   & and  \\
      2. & \setA,\setB\in\algA &\implies& \setA\seti\setB\in\algA   & (closed under $\seti$)\\
    \mc{6}{M}{The set of all algebra of sets on a set $\sid$ is denoted $\sssAx$ such that}\\
    \mc{6}{M}{\indentx$\sssAx \eqd \set{\algA\sorel\psetx}{\text{$\algA$ is an algebra of sets}}$.}\\
    \mc{6}{M}{An \structe{algebra of sets} $\algA$ on $\setX$ is a \structd{\txsigma-algebra} on $\setX$ if}\\
      3.& \setxZ{\setA_n}\subseteq\algA &\implies& \ds\setopu_{n\in\Z}\setA_n \in  \algA & (closed under countable union operations).
  \end{array}}
\end{definition}

%The number of algeba of sets on a finite set with $n$ elements is given by the
%\hie{Bell numbers}: 1, 1, 2, 5, 15, 52, 203, 877, \ldots
%(see \pref{prop:set_num_partitions}}.
On every set $\sid$ with at least 2 elements,
there are always two particular algebras of sets:
the \hie{smallest algebra} and the \hie{largest algebra},
as demonstrated by \pref{ex:ss_algebra_smallest} (next).
%---------------------------------------
\begin{example}
\citetbl{
  \citerpg{stroock1999}{33}{0817640738},
  \citerppu{ab}{95}{96}{0120502577}
  }
\label{ex:ss_algebra_smallest}
%---------------------------------------
Let $\sssAx$ be the set of \structe{algebras of sets} \xref{def:ss_algebra} on a set $\sid$ and
$\pset{\sid}$ the \structe{power set} \xref{def:pset} on $\sid$.
\exbox{\begin{array}{lcl@{\qquad}D}
  \setn{\szero,\,\sid} &\in& \sssA{\sid} & (\structe{smallest algebra})\\
  \psetx               &\in& \sssA{\sid} & (\structe{largest algebra})
\end{array}}
\end{example}


Isomorphically, all \structe{algebras of sets} are \structe{boolean algebras}\ifsxref{boolean}{def:booalg} 
and all boolean algebras are algebras of sets (next theorem).
%---------------------------------------
\begin{theorem}[\thmd{Stone Representation Theorem}]
\footnote{
  \citerpg{levy2002}{257}{0486420795},
  \citerpg{gratzer2003}{85}{3764369965},
  \citerpg{joshi1989}{224}{8122401201},
  \citerpgc{salii1988}{32}{0821845225}{``Stone's Theorem"},
  %\citerpg{gratzer1971}{76}{0716704420}\\
  \citor{stone1936}
  }
\label{thm:lat_algebra}
%---------------------------------------
Let $\latL\eqd\latticed$ be a \hie{lattice}.
\thmbox{
  \text{$\latL$  is \prope{Boolean}}
  \quad\iff\quad
  \brb{\begin{tabular}{l}
    $\latL$ is isomorphic to $\booalg{\ssetA}{\sorel}{\sor}{\sand}{\snot}{\szero}{\sid}$\\
    for some \structe{algebra of sets} \xref{def:algsets} $\ssetA$
  \end{tabular}}
  }
\end{theorem}
\begin{proof}
\begin{enumerate}
  \item Proof that \hie{algebra of sets} $\implies$ \hie{Boolean algebra}:
    \begin{enumerate}
      \item Proof that $\ssetS$ is closed under $\sor$ and $\sand$: by hypothesis.\label{item:ss_fc_ui}
      \item By \pref{item:ss_fc_dis} and by \prefpp{thm:topprop},
            $\latL$ is a \prope{distributive} lattice.
            \label{item:ss_fc_dis}
      \item By \pref{item:ss_fc_dis} and properties of \structe{lattices}\ifsxref{lattice}{thm:lattice},
            $\latL$ is
            \prope{idempotent}, \prope{commutative}, \prope{associative}, and \prope{absorptive}.
            \label{item:ss_fc_lat}
      \item Proof that $\latL$ has \prope{identity}: \label{item:ss_fc_id}
        \begin{align*}
          \setA \sor \szero
            &= \set{x\in\setX}{(x\in\setA) \lor (x\in\szero)}
            && \text{by definition of $\sor$ \prefp{def:ss_setops}}
          \\&= \set{x\in\setX}{x\in\setA}
            && \text{by definition of $\szero$ \prefp{def:ss_setops}}
          \\&= \setA
          \\
          \setA \sand \sid
            &= \set{x\in\setX}{(x\in\setA) \land (x\in\sid)}
            && \text{by definition of $\sand$ \prefp{def:ss_setops}}
          \\&= \set{x\in\setX}{x\in\setA}
            && \text{by definition of $\szero$ \prefp{def:ss_setops}}
          \\&= \setA
        \end{align*}
      \item Proof that $\latL$ is \prope{complemented}: \label{item:ss_fc_comp}
            by hypothesis.
      \item Because $\latL$ is
            \prope{commutative}     \xref{item:ss_fc_lat},
            \prope{distributive}    \xref{item:ss_fc_dis},
            has \prope{identity}    \xref{item:ss_fc_id}, and
            is \prope{complemented} \xref{item:ss_fc_comp},
            and by the definition of \structe{Boolean algebra}s\ifsxref{boolean}{def:booalg},
            $\latL$ is a \structe{Boolean algebra}.
    \end{enumerate}


  \item Proof that \hie{Boolean algebra} $\implies$ \hie{algebra of sets}: not included at this time.
\end{enumerate}
\end{proof}




%%---------------------------------------
%\begin{example}
%\label{ex:alg_xy}
%%---------------------------------------
%There are only 2 \structe{algebra of sets} \xref{def:algsets} on the set $\setX\eqd\setn{x,y}$:\\
%{\footnotesize\arrayrulecolor{ex}
%\begin{tabular}{|>{$}l<{$} @{$\;=\;\{$} *{4}{>{$}l<{$}@{\,}} @{$\}$\quad} |}
%  \hline
%    \algA_{3} & \emptyset, & \setn{x}, & \setn{y}, & \setX
%  \\\algA_{0} & \emptyset, &           &           & \setX
%  \\\hline
%\end{tabular}
%}
%\end{example}
%
%%---------------------------------------
%\begin{example}
%\label{ex:alg_xyz}
%%---------------------------------------
%There are a total of 5 \structe{algebra of sets} \xref{def:algsets} on the set $\setX\eqd\setn{x,y,z}$:\\
%{\footnotesize\arrayrulecolor{ex}
%%\begin{tabular}{|>{$}l<{$} @{$\;=\;\{$} *{8}{>{$}l<{$}@{\,}} @{$\}$\quad}  | @{\quad$\{$} *{8}{>{$}l<{$}@{\,}} @{$\}\quad$} |}
%\begin{tabular}{|>{$}l<{$} @{$\;=\;\{$} *{8}{>{$}l<{$}@{\,}} @{$\}$\quad}|}
%  \hline
%    \algA_{77} & \emptyset, & \setn{x}, & \setn{y}, & \setn{z}, & \setn{x,y}, & \setn{x,z}, & \setn{y,z}, & \setX
%  \\\algA_{41} & \emptyset, & \setn{x}, &           &           &             &             & \setn{y,z}, & \setX
%  \\\algA_{22} & \emptyset, &           & \setn{y}, &           &             & \setn{x,z}, &             & \setX
%  \\\algA_{14} & \emptyset, &           &           & \setn{z}, & \setn{x,y}, &             &             & \setX
%  \\\algA_{00} & \emptyset, &           &           &           &             &             &             & \setX
%  \\\hline
%\end{tabular}
%}
%%\end{example}
%\\

%%---------------------------------------
%\begin{example}
%\label{ex:set_lat_part_xyz}
%%---------------------------------------
%There are a total of \textbf{5} \structe{algebras of sets} on the set $\sid\eqd\set{x,y,z}$.
%These sets are listed in \prefpp{ex:sets_asets}
%and illustrated in \prefpp{fig:set_lat_alg_xyz}.
%\end{example}

%%---------------------------------------
%\begin{example}
%\label{ex:set_lat_alg_xyz}
%%---------------------------------------
%There are a total of \textbf{15} \structe{algebras of sets} on the set $\sid\eqd\setn{x,y,z}$.
%These sets are listed in \prefpp{ex:set_asets}
%and illustrated in \prefpp{fig:set_lat_alg_xyz}.
%\end{example}



%\begin{minipage}{12\tw/16}
%%---------------------------------------
%\begin{example}
%\label{ex:set_lat_alg_xy}
%%---------------------------------------
%There are \textbf{2} algebras of sets on the set $\sid\eqd\setn{x,y}$.
%These are listed in \prefpp{ex:set_asets} and illustrated in the figure to the right.
%\end{example}
%\end{minipage}%
%\begin{minipage}[c]{4\tw/16}
%%\begin{figure}
%\color{figcolor}
%\begin{center}
%\begin{fsL}
%\setlength{\unitlength}{\tw/400}
%\begin{picture}(240,240)(-100,0)
%  %\graphpaper[10](0,0)(600,200)
%  \thicklines
%  \color{latline}%
%    \put(   0,   0){\line( 0, 1){200} }%
%  \color{latdot}%
%    \put(   0, 200){\circle*{15}}%
%    \put(   0,   0){\circle*{15}}%
%    \put(   0, 210){\makebox(0,0)[b]{ $\setn{\;\szero, \setn{x}, \setn{y}, \setn{x,y} }$}}%
%    \put(   0, -10){\makebox(0,0)[t]{ $\setn{\;\szero, \setn{x,y} }$}}%
%  \color{red}%
%    \put(  65, 110){\makebox(0,0)[b] {finer }}%
%    \put(  65,  90){\makebox(0,0)[t] {coarser}}%
%    \put(  65, 130){\vector(0, 1){60} }%
%    \put(  65,  70){\vector(0,-1){60} }%
%\end{picture}%
%\end{fsL}
%\end{center}
%\end{minipage}



%=======================================
\subsection{Rings of sets}
%=======================================
A \hie{ring of sets} (next definition) is a family of subsets
that is closed under an ``addition-like" set union operator $\setu$
and ``subtraction-like" set difference operator $\setd$.
Using these two operations, it is not difficult to show that
a ring of sets is also closed under a
%another ``addition-like" set union operator $\setu$ and 
``multiplication-like" set intersection operator $\seti$.
Because of this, a ring of sets behaves like an \hie{algebraic ring}.
%And in fact, as demonstrated by \prefpp{thm:ros<=>ring},
%a ring of sets \emph{is} an algebraic ring.
Note however that a ring of sets is not necessarily a \structe{topology} \xref{def:topology} 
because it does not necessarily include $\sid$ itself.

%---------------------------------------
\begin{definition}
\footnote{
  \citerpg{bsu1996}{4}{3764353449},
  \citerpg{halmos1950}{19}{0387900888},
  \citerpg{hausdorff1937e}{90}{0828401195}
  }
\label{def:ss_ring}
\label{def:ringsets}
\label{def:sigring}
\index{ring of sets}
\index{set!ring}
%---------------------------------------
Let $\sid$ be a set with \structe{power set} $\psetx$ \xref{def:pset}.
\defbox{\indxs{\ringR}\indxs{\sssAx}
  \begin{array}{@{\qquad}Flcl DD}
    \mc{6}{M}{$\ringR\sorel\psetx$ is a \structd{ring of sets} on $\sid$ if}\\
      1. & \setA,\setB\in\ringR &\implies& \setA\setu\setB            & (closed under $\setu$)   & and  \\
      2. & \setA,\setB\in\ringR &\implies& \setA\setd\setB\in\ringR   & (closed under $\setd$)\\
    \mc{6}{M}{The set of all rings of sets on a set $\sid$ is denoted $\sssRx$ such that}\\
    \mc{6}{M}{\indentx$\sssRx \eqd \set{\ringR\sorel\psetx}{\text{$\ringR$ is a ring of sets}}$.}\\
    \mc{6}{M}{A \structe{ring of sets} $\ringR$ on $\setX$ is a \structd{\txsigma-ring} on $\setX$ if}\\
      3.& \setxZ{\setA_n}\subseteq\ringR &\implies& \ds\setopu_{n\in\Z}\setA_n \in  \ringR & (closed under countable union operations).
  \end{array}}
\end{definition}


%---------------------------------------
\begin{example}
\label{ex:set_rings}
%---------------------------------------
\prefpp{tab:set_rings} lists some \structe{rings of sets} on a finite set $\sid$.
\end{example}
\begin{table}
  \centering%
  \begin{tabular}{|MMM|}
    \hline
    \mc{3}{|G|}{rings $\sssR{\sid}$ on a set $\sid$}
    \\\hline
    \sssR{\szero}&=&
    \setn{\begin{array}{lcl}
      \ssR_{ 1} &=& \setn{\szero}\\
    \end{array}}
    \\&&\\
    \sssR{\setn{x}}&=&
    \setn{\begin{array}{lc l*{2}l l}
      \ssR_{ 1} &=& \{ & \szero, &          & \}\\
      \ssR_{ 2} &=& \{ & \szero, & \setn{x} & \}\\
    \end{array}}
    \\&&\\
    \sssR{\setn{x,y}}&=&
    \setn{\begin{array}{lc l*{4}l l}
      \ssR_{ 1} &=& \{ & \szero, &           &           &            & \}\\
      \ssR_{ 2} &=& \{ & \szero, & \setn{x}, &           &            & \}\\
      \ssR_{ 3} &=& \{ & \szero, &           & \setn{y}, &            & \}\\
      \ssR_{ 4} &=& \{ & \szero, &           &           & \setn{x,y} & \}\\
      \ssR_{ 5} &=& \{ & \szero, & \setn{x}, & \setn{y}, & \setn{x,y} & \}
    \end{array}}
    \\&&\\
    \sssR{\setn{x,y,z}}&=&
    \setn{\begin{array}{ll *{8}{l} l}
      \ssR_{ 1} &=\{ & \szero, &           &           &           &             &             &             &       & \}\\ % R_{ 1}
      \ssR_{ 2} &=\{ & \szero, & \setn{x}, &           &           &             &             &             &       & \}\\ % R_{ 2}
      \ssR_{ 3} &=\{ & \szero, &           & \setn{y}, &           &             &             &             &       & \}\\ % R_{ 3}
      \ssR_{ 4} &=\{ & \szero, &           &           & \setn{z}, &             &             &             &       & \}\\ % R_{ 4}
      \ssR_{ 5} &=\{ & \szero, &           &           &           & \setn{x,y}, &             &             &       & \}\\ % R_{ 5}
      \ssR_{ 6} &=\{ & \szero, &           &           &           &             & \setn{x,z}, &             &       & \}\\ % R_{ 6}
      \ssR_{ 7} &=\{ & \szero, &           &           &           &             &             & \setn{y,z}, &       & \}\\ % R_{ 7}
      \ssR_{ 8} &=\{ & \szero, & \setn{x}, & \setn{y}, &           & \setn{x,y}, &             &             &       & \}\\ % R_{ 8}
      \ssR_{ 9} &=\{ & \szero, & \setn{x}, &           & \setn{z}, &             & \setn{x,z}, &             &       & \}\\ % R_{ 9}
      \ssR_{10} &=\{ & \szero, &           & \setn{y}, & \setn{z}, &             &             & \setn{y,z}, &       & \}\\ % R_{10}
      \ssR_{11} &=\{ & \szero, &           &           &           &             &             &             & \sid& \}\\ % R_{11}
      \ssR_{12} &=\{ & \szero, & \setn{x}, &           &           &             &             & \setn{y,z}, & \sid& \}\\ % R_{12}
      \ssR_{13} &=\{ & \szero, &           & \setn{y}, &           &             & \setn{x,z}, &             & \sid& \}\\ % R_{13}
      \ssR_{14} &=\{ & \szero, &           &           & \setn{z}, & \setn{x,y}, &             &             & \sid& \}\\ % R_{14}
      \ssR_{15} &=\{ & \szero, & \setn{x}, & \setn{y}, & \setn{z}, & \setn{x,y}, & \setn{x,z}, & \setn{y,z}, & \sid& \}\\ % R_{15}
    \end{array}}
    \\\hline
  \end{tabular}
  \caption{%
    some \structe{rings of sets} on a finite set $\sid$ 
    \xref{ex:set_rings}
    \label{tab:set_rings}
    }
\end{table}




\begin{minipage}[c]{\tw-43mm}
%---------------------------------------
\begin{example}
\label{ex:set_ring_progessive}
%---------------------------------------
Let $\sid\eqd\setn{x,y,z}$ be a set
and $\ssR$ be the family of sets
  \\\indentx$\ssR\eqd\setn{\szero,\, \sid,\, \setn{x},\, \setn{y},\, \setn{z},\, \setn{x,y}}.$\\
Note that $(\ssR,\, \sorel,\, \setu,\, \seti)$ is a lattice as illustrated
in the figure to the right.
However, $\ssR$ is {\em not} a ring of sets on $\sid$ because, for example,
  \\\indentx$\setn{x,y,z} \setd \setn{x} = \setn{y,z}  \notin \ssR$.
\end{example}
\end{minipage}%
\hfill\tbox{\includegraphics{../common/math/graphics/pdfs/lat6_primo_ss_xyz.pdf}}\hfill\mbox{}\\%

\begin{minipage}[c]{\tw-40mm}
%---------------------------------------
\begin{example}
\label{ex:set_ring_primitive}
%---------------------------------------
Let $\sid\eqd\setn{x,y,z}$ be a set
and $\ssR$ be the family of sets
  \\\indentx$\ssR\eqd\setn{\szero,\, \sid,\, \setn{x},\, \setn{y},\, \setn{z}}.$
Note that $\opair{\topT}{\sorel}{\setu}{\seti}$ is a lattice as illustrated
in the figure to the right.
However, $\ssR$ is {\em not} a ring of sets on $\sid$ because, for example,
  \\\indentx$\setn{x,y,z} \setd \setn{x} = \setn{y,z}  \notin \ssR.$
\end{example}
\end{minipage}%
\hfill\tbox{\includegraphics{../common/math/graphics/pdfs/lat5_m3_ss_xyz.pdf}}\hfill\mbox{}\\%

%---------------------------------------
\begin{proposition}
\citetbl{
  \citerpg{kolmogorov1975}{32}{0486612260},
  \citerpg{bartle2001}{318}{0821808451}
  }
%---------------------------------------
Let $\sssR{\sid}$ be the set of \structe{rings of sets} \xref{def:ss_ring} on a set $\sid$.
\propbox{
  \brb{\begin{array}{M}
    $\ringR_1$ and $\ringR_2$\\
    are \structb{rings of sets}
  \end{array}}
  \qquad\implies\qquad
  \brb{\begin{array}{M}
    $\brp{\ringR_1\seti\ringR_2}$\\
    is a \structb{ring of sets}
  \end{array}}
  }
\end{proposition}


%=======================================
\subsection{Partitions}
%=======================================
The following definition is a special case of \structe{partition} defined on lattices\ifsxref{latb}{def:partition}.
%---------------------------------------
\begin{definition}
\citetbl{
  \citerp{munkres2000}{23},
  \citerp{rota1964}{498},
  \citerpg{halmos1950}{31}{0387900888}
  }
\label{def:ss_partition}
%---------------------------------------
\defboxt{%
  A \structe{set structure} $\setxn{\setP_n\in\psetx}$ is a \structd{partition} of the set $\sid$ if
  \\\indentx$\begin{array}{F >{\ds}rcl@{\qquad}C@{\qquad}D@{\qquad}D}
    1. & \setP_n                  &\ne& \szero & \forall n\in\setn{1,2,\ldots,\xN} & \prope{non-empty}          & and\\
    2. & \setP_n \seti \setP_m    &=&   \szero & \forall n\ne m                    & \prope{mutually exclusive} & and \\
    3. & \setopu_{n\in\Z} \setP_n &=&   \sid   &
  \end{array}$
  }
\end{definition}

%---------------------------------------
\begin{example}
%---------------------------------------
Let $\setA,\setB\subseteq\setX$, as illustrated in \prefpp{fig:AB4}.
There are a total of 15 partitions of $\setX$ induced by $\setA$ and $\setB$ \xref{prop:bell_e}.
Here are 5 of these partitions:
\exbox{\begin{array}{FlDD}
    1. & \setn{\setX}                                                                 & (1 region)  &
  \\2. & \setn{\setA,\,\cmpA}                                                         & (2 regions) &
  \\3. & \setn{\setA\setu\setB,\,\cmpA\seti\cmpB}                                     & (2 regions) &
  \\4. & \setn{\setA\seti\setB,\,\setA\sets\setB,\,\cmpA\seti\cmpB}                   & (3 regions) &
  \\5. & \setn{\setA\seti\setB,\,\setA\seti\cmpB,\,\cmpA\seti\setB,\,\cmpA\seti\cmpB} & (4 regions) & [see also \prefp{fig:AB4} and \prefp{fig:setops16}]
  %\\6. & \setn{\setA,\,\setB\setd\setA,\,\cmpp{\setA\setu\setB}}
  %\\7. & \setn{\setA\setd\setB,\,\setB\setd\setA,\,\setA\seti\setB,\,\cmpp{\setA\setu\setB}}
\end{array}}
\end{example}


%---------------------------------------
\begin{proposition}
\label{prop:partition_orel}
\citetbl{
  \citerpg{roman2008}{111}{0387789006},
  \citerpg{comtet1974}{220}{9027704414},
  \citerp{gratzer2007}{697}
  }
%---------------------------------------
Let $\sssPx$ be the set of partitions on a set $\sid$.
\propbox{\begin{array}{M}
  The relation $\porel\in\clR{\sssetP}{\sssetP}$ defined as
  \\\indentx$
      \ssetP\porel\ssetQ
      \qquad\iffdef\qquad
      \forall\setB\in\ssetQ,\, \exists\setA\in\ssetP \st \setB\sorel\setA
    $
  \\
  is an ordering relation on $\sssPx$.
\end{array}}
\end{proposition}


%\begin{minipage}[c]{\tw-43mm}
%%---------------------------------------
%\begin{example}
%\label{ex:set_ring_progessive}
%%---------------------------------------
%Let $\sid\eqd\setn{x,y,z}$ be a set
%and $\ssR$ be the family of sets
%  \\\indentx$\ssR\eqd\setn{\szero,\, \sid,\, \setn{x},\, \setn{y},\, \setn{z},\, \setn{x,y}}$.
%Note that $(\ssR,\, \sorel,\, \setu,\, \seti)$ is a lattice as illustrated
%in the figure to the right.
%However, $\ssR$ is {\em not} a ring of sets on $\sid$ because, for example,
%  \\\indentx$\setn{x,y,z} \setd \setn{x} = \setn{y,z}  \notin \ssR$.
%\end{example}
%\end{minipage}%
%\begin{minipage}[c]{40mm}
%  \centering%
%  \footnotesize%
%  %============================================================================
% Daniel J. Greenhoe
% LaTeX file
% 6 element lattice, primordial architecture
% nominal unit = 7.5mm
%============================================================================
\begin{pspicture}(-2,-\latbot)(2,3.4)
  %---------------------------------
  % settings
  %---------------------------------
  \psset{labelsep=1.5mm}%
  %---------------------------------
  % nodes
  %---------------------------------
  \Cnode(0,3){t}%
  \Cnode(-0.5,2){xy}%
  \Cnode(-1,1){x}\Cnode(0,1){y}\Cnode(1,1){z}%
  \Cnode(0,0){b}%
  %---------------------------------
  % node connections
  %---------------------------------
  \ncline{t}{xy}\ncline{t}{z}%
  \ncline{y}{xy}%
  \ncline{x}{xy}%
  \ncline{b}{x}\ncline{b}{y}\ncline{b}{z}%
  %---------------------------------
  % node labels
  %---------------------------------
  \uput[0](t) {$\setn{x,y,z}$}%     
  \uput[180](xy) {$\setn{x,y}$}%     
  \uput[180](x) {$\setn{x}$}%     
  \uput[60](y) {$\setn{y}$}%     
  \uput[0](z) {$\setn{z}$}%     
  \uput[0](b) {$\emptyset$}%     
\end{pspicture}%
%\end{minipage}



%=======================================
%\section{Examples}
%=======================================

%---------------------------------------
\begin{example}
\label{ex:set_partitions}
%---------------------------------------
\prefpp{tab:set_partitions} lists some partitions $\ssetP(\sid)$ on a finite set $\sid$.
\end{example}
\begin{table}
  \begin{tabular}{|MMM|}
    \hline
    \mc{3}{|G|}{partitions $\sssP{\sid}$ on a set $\sid$}\\
    \sssP{\szero}&=&
    \setn{\begin{array}{lcl *{2}{l} l}
      \ssetP_{ 1} &=& \szero
    \end{array}}
    \\&&\\
    \sssP{\setn{x}} &=&
    \setn{\begin{array}{lcl *{2}{l} l}
      \ssetP_{ 1} &=& \{ & & \setn{x} & \}
    \end{array}}
    \\&&\\
    \sssP{\setn{x,y}} &=&
    \setn{\begin{array}{lcl *{4}{l} l}
      \ssetP_{ 1} &=& \{ & & \setn{x}, & \setn{y}, &            & \}\\
      \ssetP_{ 2} &=& \{ & &           &           & \setn{x,y} & \}\\
    \end{array}}
    \\&&\\
    \sssP{\setn{x,y,z}} &=&
    \setn{\begin{array}{lcl *{8}{l} l}
      \ssetP_{ 1} &=& \{ & &           &           &           &             &             &               \setn{x,y,z} & \}\\
      \ssetP_{ 2} &=& \{ & & \setn{x}, &           &           &             &             & \setn{y,z}, &              & \}\\
      \ssetP_{ 3} &=& \{ & &           & \setn{y}, &           &             & \setn{x,z}, &             &              & \}\\
      \ssetP_{ 4} &=& \{ & &           &           & \setn{z}, & \setn{x,y}  &             &             &              & \}\\
      \ssetP_{ 5} &=& \{ & & \setn{x}, & \setn{y}, & \setn{z}  &             &             &             &              & \}
    \end{array}}
    \\&&\\
    \sssP{\setn{w,x,y,z}} &=&
    \setn{\begin{array}{lcl llll l}
      \ssetP_{ 1} &=& \{ &                                         &                                                                         &                                                         & \sid& \}\\
      \ssetP_{ 2} &=& \{ & \setn{w},                               &                                                                         &                                           \setn{x,y,z}  &       & \}\\
      \ssetP_{ 3} &=& \{ &           \setn{x},                     &                                                                         &                             \setn{w,y,z}                &       & \}\\
      \ssetP_{ 4} &=& \{ &                     \setn{y},           &                                                                         &               \setn{w,x,z}                              &       & \}\\
      \ssetP_{ 5} &=& \{ &                               \setn{z}, &                                                                         & \setn{w,x,y}                                            &       & \}\\
      \ssetP_{ 6} &=& \{ &                                         & \setn{w,x},                                                 \setn{y,z}  &                                                         &       & \}\\
      \ssetP_{ 7} &=& \{ &                                         &             \setn{w,y},                         \setn{x,z}              &                                                         &       & \}\\
      \ssetP_{ 8} &=& \{ &                                         &                         \setn{w,z}, \setn{x,y},                         &                                                         &       & \}\\
      \ssetP_{ 9} &=& \{ & \setn{w}, \setn{x}                      &                                                             \setn{y,z}  &                                                         &       & \}\\
      \ssetP_{10} &=& \{ & \setn{w},           \setn{y},           &                                                 \setn{x,z}              &                                                         &       & \}\\
      \ssetP_{11} &=& \{ & \setn{w},                     \setn{z}, &                                     \setn{x,y}                          &                                                         &       & \}\\
      \ssetP_{12} &=& \{ &           \setn{x}, \setn{y},           &                         \setn{w,z}                                      &                                                         &       & \}\\
      \ssetP_{13} &=& \{ &           \setn{x},           \setn{z}, &             \setn{w,y}                                                  &                                                         &       & \}\\
      \ssetP_{14} &=& \{ &                     \setn{y}, \setn{z}, & \setn{w,x}                                                              &                                                         &       & \}\\
      \ssetP_{15} &=& \{ & \setn{w}, \setn{x}, \setn{y}, \setn{z}, &                                                                         &                                                         &       & \}
    \end{array}}
    \\\hline
  \end{tabular}
  \caption{%
    some partitions $\ssetP(\sid)$ on a finite set $\sid$
    \xref{ex:set_partitions} 
    \label{tab:set_partitions} 
    }
\end{table}


\ifexclude{mssa}{
%=======================================
\section{Numbers of set structures}
%=======================================
%---------------------------------------
\begin{proposition}
\label{prop:top_num}
\footnote{
  \citeoeis{A000798},
  \citerp{brown1996}{31},
  \citerpg{comtet1974}{229}{9027704414},
  \citer{comtet1966},
  \citerp{chatterji1967}{7},
  \citer{evans1967},
  \citerp{krishnamurthy1966}{157}
  }
\index{topologies!number of}
%---------------------------------------
\propboxt{%
  The \hid{number of topologies} $t_n$ on a finite set $\sid_n$ with $n$ elements is
  \\
  $\begin{array}{|l||*{9}{r|}}
    \hline
    n   & 0 & 1 & 2 &  3 &   4 &    5 &       6 &         7 &           8   \\
    t_n & 1 & 1 & 4 & 29 & 355 & 6942 & 209,527 & 9,535,241 & 642,779,354   \\
    \hline
  \end{array}$
  \\
  $\begin{array}{|l||*{2}{r|}}
    \hline
    n   &              9 &                10  \\
    t_n & 63,260,289,423 & 8,977,053,873,043  \\
    \hline
  \end{array}$
  }
\end{proposition}

%---------------------------------------
\begin{proposition}
\citetbl{
  \citerpp{chatterji1967}{6}{7},
  \citer{kleitman1970}
  }
%---------------------------------------
Let $t_n$ be the number of topologies on a finite set with $n$ elements.
\propbox{\begin{array}{>{\ds}rcl@{\qquad}C@{\qquad}D}
  %c\cdot 2^{\frac{n^2}{4}} {n \choose [n2]} &<& T(n) < n! 2^{\frac{n(n-1)}{2} B_n} & \forall n>1,\, 0<c<1 \\
  \lim_{n\to\infty} \frac{t_n}{2^{\frac{n^2}{4}}}               &=& \infty
    &
    & (lower bound)
    \\
  \lim_{n\to\infty} \frac{t_n}{2^{\brp{\frac{1}{2} + \epsilon}n^2}}\, &=& 0
    & \forall \epsilon>0
    & (upper bound)
    \\
  t_n &>& n t_{n-1} 
    &
    & (rate of growth)
\end{array}}
\end{proposition}


Similar to the amazing relationship between $e$, $\pi$, $i$, $1$, and $0$ given by
$e^{i\pi}+1=0$,
we find another relationship between $e$ and the
number of partitions, rings of sets, and algebras of sets \xref{thm:num_P}.
%---------------------------------------
\begin{definition}
\label{def:Bn}
\footnote{
  \citerppg{comtet1974}{210}{211}{9027704414},
  \citerp{rota1964}{499},
  \citorp{bell1934}{417},
  \citorp{ocagne1887}{371}
  }
%---------------------------------------
\defbox{\begin{array}{M}
  The \hid{Bell numbers} are the elements of the sequence $\seq{B_n}{n\in\Znn}$ defined as\\
  the solution to the following equation:
  \\\indentx
    $\ds e^{e^x-1} = \sum_{n=0}^\infty \frac{B_n}{n!} x^n$.
  \\
  The Bell numbers are also called the \hid{exponential numbers}.
\end{array}}
\end{definition}


%---------------------------------------
\begin{proposition}
\citetbl{
  \citeoeis{A000110} 
  }
\index{Bell numbers}
\index{exponential numbers}
%\index{Euler numbers}
\label{prop:bell_e}
%---------------------------------------
Let $\seq{B_n}{n\in\Znn}$ be the sequence of Bell numbers.
Then $\seqn{B_n}$ has the following values:
\propbox{\begin{array}{c|*{12}{|c}}
    n   & 0 & 1 & 2 & 3 &  4 &  5 &   6 &   7 &    8 &      9 &      10 &      11 \\\hline
    B_n & 1 & 1 & 2 & 5 & 15 & 52 & 203 & 877 & 4140 & 21,147 & 115,975 & 678,570
  \end{array}}
\end{proposition}
\begin{proof}
By \prefpp{def:Bn}, the sequence $\seqn{B_n}$ is the solution to
  \[  e^{e^x-1} = \sum_{n=0}^\infty \frac{B_n}{n!} x^n. \]
Let $\ff^{(n)}(x)$ be the $n$th derivative of a function $\ff:\R\to\R$.
The Maclaurin expansion of $\ff(x)$ is
  \[ \ff(x) = \sum_{n=0}^\infty \frac{\ff^{n}(0)}{n!} x^n \]
Let $\ff(x)\eqd e^{e^x}$. Then
%\begin{multicols}{2}
\begin{align*}
  \ff^{(0)}(0)
    &= \left.\ff^{(0)}(x)\right|_{x=0}
  \\&= e^{e^0}
  \\&= e
  \\
  \ff^{(1)}(0)
    &= \left.\ff^{(1)}(x)\right|_{x=0}
  \\&= \left.\deriv{}{x} e^{e^x} \right|_{x=0}
  \\&= \left. e^{e^x} e^x \right|_{x=0}
  \\&= e
  \\
  \ff^{(2)}(0)
    &= \left.\deriv{}{x}\ff^{(1)}(x)\right|_{x=0}
  \\&= \left.\deriv{}{x} e^{e^x} e^x \right|_{x=0}
  \\&= \left.\brp{e^{e^x} e^x}e^x + e^{e^x}e^x \right|_{x=0}
  \\&= \left. e^{e^x} \brp{e^{2x} + e^x} \right|_{x=0}
  \\&= 2e
  \\
  \ff^{(3)}(0)
    &= \left.\deriv{}{x}\ff^{(2)}(x)\right|_{x=0}
  \\&= \left. \deriv{}{x} e^{e^x} \brp{e^{2x} + e^x} \right|_{x=0}
  \\&= \left. e^{e^x} e^x \brp{e^{2x} + e^x} + e^{e^x} \brp{2e^{2x} + e^x} \right|_{x=0}
  \\&= \left. e^{e^x} \brp{e^{3x} + 3e^{2x} + e^x}  \right|_{x=0}
  \\&= 5e
  \\
  \ff^{(4)}(0)
    &= \left.\deriv{}{x}\ff^{(3)}(x)\right|_{x=0}
  \\&= \left. \deriv{}{x} e^{e^x} \brp{e^{3x} + 3e^{2x} + e^x}  \right|_{x=0}
  \\&= \left. \brp{e^{e^x}e^x} \brp{e^{3x} + 3e^{2x} + e^x} + e^{e^x}\brp{3e^{3x} + 6e^{2x} + e^x}  \right|_{x=0}
  \\&= \left. e^{e^x} \brp{e^{4x} + 6e^{3x} + 7e^{2x} + e^{x}}   \right|_{x=0}
  \\&= 15e
  \\
  \ff^{(5)}(0)
    &= \left.\deriv{}{x}\ff^{(4)}(x)\right|_{x=0}
  \\&= \left. \deriv{}{x} e^{e^x} \brp{e^{4x} + 6e^{3x} + 7e^{2x} + e^{x}}   \right|_{x=0}
  \\&= \left. \deriv{}{x} \brp{e^{e^x}e^x} \brp{e^{4x} + 6e^{3x} + 7e^{2x} + e^{x}} + e^{e^x} \brp{4e^{4x} + 18e^{3x} + 14e^{2x} + e^{x}}   \right|_{x=0}
  \\&= \left. \deriv{}{x} e^{e^x} \brp{e^{5x} + 10e^{4x} + 25e^{3x} + 15e^{2x} + e^{x}}   \right|_{x=0}
  \\&= 52e
  \\
  \ff^{(6)}(0)
    &= \left.\deriv{}{x}\ff^{(5)}(x)\right|_{x=0}
  \\&= \left. \deriv{}{x} e^{e^x} \brp{e^{5x} + 10e^{4x} + 25e^{3x} + 15e^{2x} + e^{x}}   \right|_{x=0}
  \\&= \left. \brp{e^{e^x}e^x} \brp{e^{5x} + 10e^{4x} + 25e^{3x} + 15e^{2x} + e^{x}} + e^{e^x} \brp{5e^{5x} + 40e^{4x} + 75e^{3x} + 30e^{2x} + e^{x}}   \right|_{x=0}
  \\&= \left. e^{e^x} \brp{e^{6x} + 15e^{5x} + 65e^{4x} + 90e^{3x} + 31e^{2x} + e^{x}}   \right|_{x=0}
  \\&= 203e
\end{align*}
%\end{multicols}
Thus, $e^{e^x}$ has Maclaurin expansion
  \[ e^{e^x} = e\brp{1 + x + \frac{2}{2}x^2 + \frac{5}{3!}x^3 + \frac{15}{4!}x^4 + \frac{52}{5!}x^5 + \frac{203}{6!}x^6 + \cdots}
             = e \sum_{n=0}^\infty \frac{B_n}{n!} x^n \]
\end{proof}

%---------------------------------------
\begin{theorem}
\footnote{
  \url{http://groups.google.com/group/sci.math/browse_thread/thread/70a73e734b69a6ec/}
  }
\label{thm:num_P}
%---------------------------------------
Let $\setX_n$ be a finite set with $n$ elements.
Let $\seq{B_n}{n\in\Znn}$ be the sequence of Bell numbers.
\thmbox{\begin{array}{M}
  \begin{tabular}{lll}
    The number of & \hie{partitions}       & on $\setX_n$ is $B_n$. \\
    The number of & \hie{rings of sets}    & on $\setX_n$ is $B_{n+1}$. \\
    The number of & \hie{algebras of sets} & on $\setX_n$ is $B_n$.
  \end{tabular}
\end{array}}
\end{theorem}


} % end mssa exclude





%---------------------------------------

%%---------------------------------------
%\begin{remark}
%%---------------------------------------
%The converse of \prefpp{thm:set_ring==>lattice} is not true.
%That is, all rings of sets generate lattices,
%but not all lattices generate rings of sets.
%\rembox{
%  \text{$\otriple{\ssR}{\setu}{\setd}$ is a ring of sets}
%  \qquad\notimpliedby\qquad
%  \text{$\lattice{\ssR}{\sorel}{\setu}{\seti}$  is a lattice}
%  }
%\end{remark}
%\begin{proof}
%  The ring of sets $\ssR$ is \\
%    \begin{tabular}{llll}
%      $\imark$ & closed under $\setu$ & by \pref{def:ss_ring} & \prefpo{def:ss_ring} \\
%      $\imark$ & closed under $\seti$ & by \pref{thm:set_ring} & \prefpo{thm:set_ring}
%    \end{tabular}
%
%  The 4-tupple $\lattice{\ssR}{\sorel}{\setu}{\seti}$ satisfies the
%  conditions of a lattice \xref{def:lattice} with
%  \begin{align*}
%    \setA \join \setB &\eqd \setA \setu \setB  \\
%    \setA \meet \setB &\eqd \setA\seti \setB
%  \end{align*}
%\end{proof}

%Examples are provided by
%\\\begin{tabular}{@{\qquad}llll}
%  1. & \pref{ex:set_lat_ring_xy}  & \prefpo{ex:set_lat_ring_xy}  & and \\
%  2. & \pref{ex:set_lat_ring_xyz} & \prefpo{ex:set_lat_ring_xyz}.
%\end{tabular}
%
%
%Counter examples are provided by the following:
%\\\begin{tabular}{@{\qquad}lll}
%  1. & \pref{ex:set_ring_progessive} & \prefpo{ex:set_ring_progessive} \\
%  2. & \pref{ex:set_ring_primitive}  & \prefpo{ex:set_ring_primitive}
%\end{tabular}
%


%\ifdochasnot{topology}{
%%=======================================
%\section{Topological set structures}
%%=======================================
%\input{../common/topset_in.tex}
%}



%---------------------------------------
%\begin{example}
%---------------------------------------
%The next table illustrates some lattices that are Boolean algebras and algebras of sets.
%\end{example}
%
%\begin{tabular*}{\tw}{|c@{\extracolsep\fill}cc|}
%\arrayrulecolor{ex}
%\hline
%\mc{3}{|c|}{\cellcolor{ex}\color{white}\bfseries Boolean algebras as algebras of sets}
%\\\hline&&\\
%\latmatl{2}{
%  \null \\
%  \null
%  }{
%  \ncline{1,1}{2,1}
%  }{
%  \nput{ 90}{1,1}{$\setn{x,y,z}$}
%  \nput{-90}{2,1}{$\szero$}
%  }
%&
%\latmatl{3}{
%        & \null         \\
%  \null &       & \null \\
%        & \null
%  }{
%  \ncline{1,2}{2,1}\ncline{1,2}{2,3}
%  \ncline{3,2}{2,1}\ncline{3,2}{2,3}
%  }{
%  \nput{ 90}{1,2}{$\setn{x,y,z}$}
%  \nput{180}{2,1}{$\setn{x}$}
%  \nput{  0}{2,3}{$\setn{y,z}$}
%  \nput{-90}{3,2}{$\szero$}
%  }
%&
%%============================================================================
% Daniel J. Greenhoe
% LaTeX file
% lattice (2^{x,y,z}, subseteq)
% recommended unit = 10mm
%============================================================================
{\psset{unit=0.75\psunit}%
\begin{pspicture}(-2.4,-.3)(2.4,3.3)
  %---------------------------------
  % settings
  %---------------------------------
  \psset{%
    labelsep=1.5mm,
    }%
  %---------------------------------
  % nodes
  %---------------------------------
  \Cnode(0,3){t}
  \Cnode(-1,2){xy} \Cnode(0,2){xz} \Cnode(1,2){yz}
  \Cnode(-1,1){x}  \Cnode(0,1){y}  \Cnode(1,1){z}
  \Cnode(0,0){b}
  %---------------------------------
  % node connections
  %---------------------------------
  \ncline{t}{xy}\ncline{t}{xz}\ncline{t}{yz}
  \ncline{x}{xy}\ncline{x}{xz}
  \ncline{y}{xy}\ncline{y}{yz}
  \ncline{z}{xz}\ncline{z}{yz}
  \ncline{b}{x} \ncline{b}{y} \ncline{b}{z}
  %---------------------------------
  % node labels
  %---------------------------------
  \uput[180](t) {$\setn{x,y,z}$}%
  \uput[180](xy){$\setn{x,y}$}%   
 %\uput{1pt}[ 70](xz){$\setn{x,z}$} 
  \uput[0](yz){$\setn{y,z}$}%
  \uput[180](x) {$\setn{x}$}%     
 %\uput{1pt}[-45](y) {$\setn{y}$}   
  \uput[0](z) {$\setn{z}$}%
  \uput[180](b) {$\szero$}%
  \uput[0](1,3){\rnode{xzlabel}{$\setn{x,z}$}}% 
  \uput[0](1,  0){\rnode{ylabel}{$\setn{y}$}}%
  \ncline[linestyle=dotted,linecolor=red,nodesep=1pt]{->}{xzlabel}{xz}%
  \ncline[linestyle=dotted,linecolor=red,nodesep=1pt]{->}{ylabel}{y}%
\end{pspicture}
}%%
%\\&&\\\hline
%\end{tabular*}%


%Notes:
%\begin{liste}
%  \item As promised by \prefpp{thm:lat_algebra}, all of these five lattices
%        are \prope{complemented} and \prope{distributive}.
%  \item We know that they are distributive because they do not contain the
%        $N5$ or $M3$ lattices.\ifdochas{latvar}{\footnotemark}
%  %\item Even though each of the individual five lattices {\em is} distributive and complemented,
%  %      the lattice of lattices {\em is not} distributive
%  %      (it is the $M3$ lattice) and {\em not} complemented.
%\end{liste}
%\ifdochas{latvar}{\footnotetext{
%  \begin{tabular}[t]{lll}
%    \hie{distributive lattice}: & \pref{def:lat_distributive}     & \prefpo{def:lat_distributive}    \\
%    \hie{complemented lattice}: & \pref{def:latc}            & \prefpo{def:latc}           \\
%    \hie{$N5$-lattice}:         & \pref{def:lat_N5}               & \prefpo{def:lat_N5}              \\
%    \hie{Birkhoff distributivity criterion}: & \pref{thm:latd_char_n5m3} & \prefpo{thm:latd_char_n5m3}
%  \end{tabular}
%  }}%

%=======================================
\section{Operations on set structures}
%=======================================
%---------------------------------------
\begin{proposition}
\label{thm:set_closed_op}
%---------------------------------------
\propbox{\begin{array}{c||c|c|c|c}
  \mc{1}{|Z|}{closed under}&%
  \mc{1}{|Z|}{\structe{partition}}&%
  \mc{1}{|Z|}{\structe{ring of sets}}&%
  \mc{1}{|Z|}{\structe{algebra of sets}}&%
  \mc{1}{|Z|}{\structe{topology}}%
  \\\hline
  \szero     &            & \checkmark & \checkmark  & \checkmark  \\
  \sid       & \checkmark &            & \checkmark  & \checkmark  \\
  \setopc    &            &            & \checkmark  &             \\
  \setu      &            & \checkmark & \checkmark  & \checkmark  \\
  \seti      &            & \checkmark & \checkmark  & \checkmark  \\
  \sets      &            & \checkmark & \checkmark  &             \\
  \setd      &            & \checkmark & \checkmark  &
\end{array}}
\end{proposition}
\begin{proof}
\begin{enumerate}
  \item Proof for closure in a \hie{topology}:
        \prefpp{def:topology}
  \item Proof for closure in a \hie{ring of sets}:
        \prefpp{def:ss_ring} and \prefpp{thm:set_ring}
  \item Proof for closure in an \hie{algebra of sets}:
        \prefpp{def:ss_algebra} and \prefpp{thm:set_algebra}
\end{enumerate}
\end{proof}




%---------------------------------------
\begin{theorem}
%\label{cor:ss_ui}
\label{thm:topprop}
\index{pre-topology}
\index{set structures!pre-topology}
%---------------------------------------
Let $\topT$ be a \structe{set structure} \xref{def:ss} on a set $\setX$.
\thmboxt{%
  %\textbf{If} $\setu$ and $\seti$ are closed in $\topT$ \textbf{then}
  $\topT$ is a \structb{topology} \quad$\implies$\quad $\forall\setA,\setB,\setC\in\topT$
  \\%\footnotesize
  $\begin{array}{rcl|rcl|D}%
       \setA \setu \setA &=& \setA
     & \setA \seti \setA &=& \setA
     &(\prope{idempotent})
    \\ \setA \setu \setB &=& \setB \setu \setA
     & \setA \seti \setB &=& \setB \seti \setA
     &(\prope{commutative})
    \\ \setA\setu (\setB\setu\setC) &=& (\setA\setu\setB) \setu \setC
     & \setA\seti (\setB\seti\setC) &=& (\setA\seti\setB) \seti \setC
     &(\prope{associative})
    \\ \setA \setu  (\setA \seti \setB) &=& \setA
     & \setA \seti (\setA \setu  \setB) &=& \setA
     &(\prope{absorptive})
    \\ \setA\setu(\setB\seti\setC) &=& (\setA\setu\setB) \seti (\setA\setu\setC)%
     & \setA\seti(\setB\setu\setC) &=& (\setA\seti\setB) \setu (\setA\seti\setC)%
     &(\prope{distributive})%
    \\\hline
     \mc{3}{H|}{property with emphasis on $\setu$}%
    &\mc{3}{H|}{dual property with emphasis on $\seti$}%
    &\mc{1}{H}{property name}%
  \end{array}$%
  }%
\end{theorem}
\begin{proof}
\begin{enume}
  \item By \prefpp{def:topology}, $\topT$ is a \structe{topology}.
  \item By \prefpp{thm:lat_algebra}, $\lattice{\topT}{\sorel}{\sor}{\sand}$ is a \prope{distributive} \hie{lattice}.
  \item The properties listed are all properties of \structe{distributive lattice}s\ifdochas{latd}{, as provided by \prefpp{thm:lattice}, \prefpp{def:latd}, and \prefpp{thm:lat_dis}}.
\end{enume}
\end{proof}

%---------------------------------------
\begin{proposition}
%\citetbl{
  %\citerpg{bsu1996}{4}{3764353449}
  %}
\label{thm:set_algebra}
%---------------------------------------
%Let $\sssAx$ be the set of \structe{algebra of sets} \xref{def:algsets} on a set $\sid$.
Let $\algA$ be a \structe{set structure} \xref{def:ss} on a set $\setX$.
\propbox{
  \brb{\begin{array}{M}
    $\algA$ is an\\
    \structb{algebra of sets}
  \end{array}}
  %\mcoml{\algA\in\sssAx}{$\algA$ is an algebra of sets}
  \quad\implies\quad
  \brb{\begin{array}{Fll @{\qquad}C @{\qquad}D}
    1. & \szero & \in \algA
       &
       & ($\algA$ includes the $\setu$ identity element)
       \\
    2. & \sid& \in \algA
       &
       & ($\algA$ includes the $\seti$ identity element)
       \\
    3. & \cmpA  & \in  \algA
       & \forall \setA\in \algA
       & ($\algA$ is closed under $\setopc$)
       \\
    4. & \setA\setu\setB      & \in  \algA
       & \forall \setA,\setB\in \algA
       & ($\algA$ is closed under $\setu$)
       \\
    5. & \setA\seti\setB      & \in  \algA
       & \forall \setA,\setB\in \algA
       & ($\algA$ is closed under $\seti$)
       \\
    6. & \setA \setd \setB      & \in  \algA
       & \forall \setA,\setB\in \algA
       & ($\algA$ is closed under $\setd$)
       \\
    7. & \setA\sets\setB  & \in  \algA
       & \forall \setA,\setB\in \algA
       & ($\algA$ is closed under $\sets$)
  \end{array}}
  }
\end{proposition}
\begin{proof}
\begin{align*}
  \szero
    &= \setA\seti \cmpA
    \\
  \sid&= \setopc\szero
    \\
  \setA\setu\setB
    &= \setopc(\cmpA\seti \cmpB)
    && \text{by \thme{de Morgan's Law} \xref{thm:demorgan}}
    \\
  \setA\setd\setB
    &= \setA\seti \cmpB
    \\
  \setA\sets\setB
    &= (\setA \setd \cmpB)\setu(\setB\setd \setA)
\end{align*}
$\otriple{\algA}{\setu}{\setd}$ is a ring of sets because $\setu$ and $\setd$
are closed in $\algA$ (as shown above).
\end{proof}



%---------------------------------------
\begin{theorem}
\citetbl{
  \citerppg{dieudonne1969}{3}{4}{1406727911},
  \citerpg{copson1968}{9}{0521047226}
  }
%\label{cor:ss_fc}
\label{thm:algprop}
\index{algebra of sets}
\index{set structures!algebra of sets}
%---------------------------------------
Let $\algA$ be a \structe{set structure} \xref{def:ss} on a set $\setX$.
\thmboxt{
  %\textbf{If} $\setu$ and $\seti$ are closed in $\topT$ \textbf{then}
  $\algA$ is an \structb{algebra of sets} \quad$\implies$\quad $\forall\setA,\setB,\setC\in\algA$
  %\textbf{If} $\ssetS$ is closed under $\sor$, $\sand$, and $\snot$, \textbf{then} for all $\setA,\setB,\setC\in\ssetS$
  \\\footnotesize
  ${\begin{array}{rcl|rcl|D}
       \setA \setu \setA &=& \setA
     & \setA \seti \setA &=& \setA
     & (\prope{idempotent})
    \\ \setA \setu \setB &=& \setB \setu \setA
     & \setA \seti \setB &=& \setB \seti \setA
     & (\prope{commutative})
    \\ \setA\setu (\setB\setu\setC) &=& (\setA\setu\setB) \setu \setC
     & \setA\seti (\setB\seti\setC) &=& (\setA\seti\setB) \seti \setC
     & (\prope{associative})
    \\ \setA \setu  (\setA \seti \setB) &=& \setA
     & \setA \seti (\setA \setu  \setB) &=& \setA
     & (\prope{absorptive})
    \\ \setA\setu(\setB\seti\setC) &=& (\setA\setu\setB) \seti (\setA\setu\setC)
     & \setA\seti(\setB\setu\setC) &=& (\setA\seti\setB) \setu (\setA\seti\setC)
     & (\prope{distributive})
    \\ \setA \setu \szero      &=& \setA
     & \setA \seti \sid        &=& \setA
     & (\prope{identity})
    \\ \setA \setu \sid        &=& \sid
     & \setA \seti \szero      &=& \szero
     & (\prope{bounded})
    \\ \setA \setu \cmpA         &=& \sid
     & \setA \seti \cmpA         &=& \szero
     & (\prope{complemented})
    \\ \cmpp{\cmpA}          &=& \setA
     &                           &&
     & (\prope{uniquely complemented})
    \\ \cmpp{\setA\setu \setB} &=& \cmpA \seti \cmpB
     & \cmpp{\setA\seti \setB} &=& \cmpA \setu \cmpB
     & (\prope{de Morgan})
    \\\hline
      \mc{3}{H|}{property emphasizing $\setu$}
    & \mc{3}{H|}{dual property emphasizing $\seti$}
    & \mc{1}{H}  {property name}
  \end{array}}$
  }
\end{theorem}
\begin{proof}
\begin{enume}
  \item By \prefpp{def:ss_algebra}, $\ssetS$ is an \hie{algebra of sets}.
  \item By the \thme{Stone Representation Theorem} \xref{thm:lat_algebra}, 
        $\booalg{\ssetS}{\sorel}{\sor}{\sand}{\snot}{\szero}{\sid}$ is a \structe{Boolean algebra}.
  \item The properties listed are all properties of \structe{Boolean algebra}s\ifsxref{boolean}{thm:boo_prop}.
\end{enume}
\end{proof}



%---------------------------------------
\begin{theorem}
\footnote{
  \citerpg{michel1993}{12}{048667598X},
  \citerpu{ab}{4}{0120502577},
  \citerppg{vaidyanathaswamy1960}{3}{4}{0486404560}
 %\citerpp{prasad}{4}{5}
  }
\index{de Morgan's Laws}
\index{distributive laws}
\label{thm:demorgan}
\label{thm:set_distProp}
%---------------------------------------
Let $\algA$ be an \structe{algebra of sets} \xref{def:algsets} on a set $\sid$.
\thmboxt{
  $\algA$ is an \structb{algebra of sets} \quad$\implies$\quad $\forall\setA_1,\setA_2,\ldots,\setA_\xN,\setB\in\algA$ and $\forall\xN\in\Zp$
  \\$\begin{array}{>{\ds}lc>{\ds}l|>{\ds}lc>{\ds}l|D}
    \cmpp{\setopu_{n=1}^\xN\setA_n}
      &=& \setopi_{n=1}^\xN\cmpA_n
    &
    \cmpp{\setopi_{n=1}^\xN\setA_n}
      &=& \setopu_{n=1}^\xN\cmpA_n
    &
    (\prope{de Morgan})
    \\
    \brp{\setopu_{n=1}^\xN\setA_n} \seti\setB
      &=& \setopu_{n=1}^\xN \brp{\setA_n \seti \setB}
    &
    \brp{\setopi_{n=1}^\xN\setA_n} \setu\setB
      &=& \setopi_{n=1}^\xN \brp{\setA_n \setu \setB}
    & $\brp{\begin{array}{M}\prope{distributive}\\with respect to\\$\setu$ and $\seti$\end{array}}$
    \\
    \brp{\setopu_{n=1}^\xN\setA_n} \setd\setB
      &=& \setopu_{n=1}^\xN \brp{\setA_n \setd \setB}
    &
    \brp{\setopi_{n=1}^\xN\setA_n} \setd\setB
      &=& \setopi_{n=1}^\xN \brp{\setA_n \setd \setB}
    & $\brp{\begin{array}{M}\prope{distributive}\\with respect to\\$\setd$ and $\seti$\end{array}}$
    \\\hline
    \mc{3}{H|}{property emphasizing $\setu$} & \mc{3}{H|}{dual property emphasizing $\seti$} & \mc{1}{H}{property name} \\
  \end{array}$%
  }
\end{theorem}
\begin{proof}
  \begin{enumerate}
    \item By \prefpp{thm:lat_algebra}, the lattice $\lattice{\sid}{\sorel}{\setu}{\seti}$
          is \prope{Boolean}.
    \item The first four properties are true any Boolean system\ifdochas{boolean}{ \prefpp{thm:cdl_seq}}.
    \item Proof for the remaining two:

\begin{align*}
   \left( \setopi_{n=1}^\xN\setA_n \right) \setd\setB
     &= \left( \setopi_{n=1}^\xN\setA_n \right) \seti \cmpB
     && \text{by \prefp{thm:ss_rel_gg}}
   \\&= \setopi_{n=1}^\xN (\setA_n \seti \cmpB)
     && \text{by previous result}
   \\&= \setopi_{n=1}^\xN (\setA_n \setd \setB)
     && \text{by \prefp{thm:ss_rel_gg}}
\\
\\
   \left( \setopu_{n=1}^\xN\setA_n \right) \setd\setB
     &= \left( \setopu_{n=1}^\xN\setA_n \right) \seti \cmpB
     && \text{by \prefp{thm:ss_rel_gg}}
   \\&= \setopu_{n=1}^\xN (\setA_n \seti \cmpB)
     && \text{by previous result}
   \\&= \setopu_{n=1}^\xN (\setA_n \setd \setB)
     && \text{by \prefp{thm:ss_rel_gg}}
\end{align*}
  \end{enumerate}
\end{proof}


%---------------------------------------
\begin{proposition}
\citetbl{
  \citerpg{bsu1996}{4}{3764353449},
  \citerppg{halmos1950}{19}{20}{0387900888}
  }
\label{thm:set_ring}
%---------------------------------------
Let $\ssR$ be a \structe{set structure} \xref{def:ss} on a set $\setX$.
\propbox{
  \brb{\begin{array}{M}
    $\ssR$ is a\\ 
    \structb{ring of sets}\\
    on $\setX$
  \end{array}}
  \quad\implies\quad
  \brb{\begin{array}{FllCDD}
    1. & \szero  & \in  \ssR
       &
       & ($\ssR$ includes the $\setu$ identity element)
       & and
       \\
    2. & \setA\setu\setB  & \in  \ssR
       & \forall \setA,\setB\in  \ssR
       & ($\ssR$ is closed under $\setu$)
       & and
       \\
    3. & \setA\seti\setB  & \in  \ssR
       & \forall \setA,\setB\in \ssR
       & ($\ssR$ is closed under $\seti$)
       & and
       \\
    4. & \setA\setd\setB      & \in  \ssR
       & \forall \setA,\setB\in \ssR
       & ($\ssR$ is closed under $\setd$)
       & and
       \\
    5. & \setA\sets\setB      & \in  \ssR
       & \forall \setA,\setB\in \ssR
       & ($\ssR$ is closed under $\sets$)
       & 
  \end{array}}
  }
\end{proposition}
\begin{proof}
\begin{align*}
  \setA\sets\setB
    &= (\setA\setd\setB) \setu (\setB\setd\setA)
  \\
  \setA\seti\setB
    &= (\setA\setu\setB) \setd (\setA\sets\setB)
  \\
  \setA\setd\setA
    &= \szero
\end{align*}
\end{proof}

%---------------------------------------
\begin{theorem}%[algebraic ring properties of rings of sets]
\footnote{
  \citerppg{vaidyanathaswamy1960}{17}{18}{0486404560},
  \citerpg{kelley1988}{22}{0387966331},
  \citerp{wilker1982}{211},
  \citerpg{vaidyanathaswamy1960}{19}{0486404560}
 %\citerpg{haaser1991}{4}{0486665097}
  }
\label{thm:ss_si_aring}
\label{thm:ros<=>ring}
\index{algebraic ring properties of rings of sets}
\index{theorems!algebraic ring properties of rings of sets}
%---------------------------------------
Let $\ssR$ be a \structe{set structure} \xref{def:ss} on a set $\sid$.
\thmboxt{%
  %\textbf{If} $\sid$, $\sets$ and $\seti$ are closed in $\ssetR$, \textbf{then}
  {If} $\ssR$ is an \structb{ring of sets} on $\setX$, {then}
  $\otriple{\ssR}{\sets}{\seti}$ is an \prope{algebraic ring};
  in particular,
  \\\indentx$\begin{array}{lclC@{\qquad}|@{\qquad}lclC}
      \setA \sets \szero  &=& \setA  & \forall \setA\in\ssR       &
      \setA \seti \szero  &=& \szero & \forall \setA\in\ssR       \\
      \setA \sets \sid    &=& \cmpA  & \forall \setA\in\ssR       &
      \setA \seti \sid    &=& \setA  & \forall \setA\in\ssR       \\
      \setA \sets \szero  &=& \setA  & \forall \setA\in\ssR       &
      \setA \seti \setA   &=& \setA  & \forall \setA\in\ssR       \\
      \mc{7}{c}{\setA \seti \brp{\setB \sets \setC} = \brp{\setA \seti \setB} \sets \brp{\setA \seti \setC}} & \forall\setA,\setB,\setC\in\ssR
      \\\hline
      \mc{4}{H@{\qquad}|@{\qquad}}{properties emphasizing $\sets$}&
      \mc{4}{H}{properties emphasizing $\seti$}
  \end{array}$
  }
\end{theorem}
\begin{proof}
\begin{enumerate}
  \item Proof that $\otriple{\ssR}{\setu}{\setd}$ is an \hie{algebraic ring}:
        by \prefpp{thm:ss_si_aring}
  \item Proof that a ring of sets is equivalent to $\otriple{\ssR}{\setu}{\setd}$:
    This is proven simply by noting that $\setu$ and $\setd$
    (the two operations in a ring of sets $\otriple{\ssR}{\setu}{\setd}$)
    can be expressed in terms of
    $\sets$ and $\seti$
    (the two operations in the algebraic ring $\otriple{\ssR}{\sets}{\seti}$)
    and vice-versa.
    And this is demonstrated by \prefpp{thm:ss_rel_gg}.
\end{enumerate}
\ifdochas{algebra}{The definition of an algebraic ring is given in \prefpp{def:ring}.}
\begin{align*}
  \intertext{1. Proof that $\opair{\ssetS}{\sets}$ is a group: see \prefpp{prop:ss_sets_group}.}
  \intertext{2. Proof that $\setA\seti\brp{\setB\seti\setC}=\brp{\setA\seti\setB}\seti\setC$:}
    \setA\seti\brp{\setB\seti\setC}
      &= \set{x\in\sid}{\brp{x\in\setA}\land\brs{\brp{x\in\setB}\land\brp{x\in\setC}}}
      && \text{by definition of $\seti$ \prefpo{def:ss_setops}}
    \\&= \set{x\in\sid}{\brs{\brp{x\in\setA}\land\brp{x\in\setB}}\land\brp{x\in\setC}}
    \\&= \brp{\setA\seti\setB}\seti\setC
      && \text{by definition of $\seti$ \prefpo{def:ss_setops}}
    \\
  \intertext{3. Proof that $\setA\seti\brp{\setB\sets\setC}=\brp{\setA\seti\setB}\sets\brp{\setA\seti\setC}$:}
    \setA\seti\brp{\setB\sets\setC}
      &= \set{x\in\sid}{\brp{x\in\setA}\land\brs{\brp{x\in\setB}\lxor\brp{x\in\setC}}}
      && \text{by definition of $\seti,\sets$ \prefpo{def:ss_setops}}
    \\&= \set{x\in\sid}{\brs{\brp{x\in\setA}\land\brp{x\in\setB}}\lxor\brs{\brp{x\in\setA}\land\brp{x\in\setC}}}
    \\&= \brp{\setA\seti\setB}\sets\brp{\setA\seti\setC}
      && \text{by definition of $\seti,\sets$ \prefpo{def:ss_setops}}
    \\
  \intertext{4. Proof that $\brp{\setA\sets\setB}\seti\setC=\brp{\setA\seti\setC}\sets\brp{\setB\seti\setC}$:}
    \brp{\setA\sets\setB}\seti\setC
      &= \set{x\in\sid}{\brs{\brp{x\in\setA}\lxor\brp{x\in\setB}}\land\brp{x\in\setC}}
      && \text{by definition of $\seti,\sets$ \prefpo{def:ss_setops}}
    \\&= \set{x\in\sid}{\brs{\brp{x\in\setA}\land\brp{x\in\setC}}\lxor\brs{\brp{x\in\setB}\land\brp{x\in\setC}}}
    \\&= \brp{\setA\seti\setC}\sets\brp{\setB\seti\setC}
      && \text{by definition of $\seti,\sets$ \prefpo{def:ss_setops}}
\end{align*}
\end{proof}

%=======================================
\section{Lattices of set structures}
%=======================================
%=======================================
\ifexclude{mssa}{\subsection{Ordering relations}}
%=======================================

The \ope{set inclusion} relation $\sorel$ \xref{def:subset}
is an \hie{order relation}\ifsxref{order}{def:order_rel} on set structures,
as demonstrated by \pref{prop:ss_order} (next proposition).
%But it is not the only possible order relation on a set structure.
%Another example, different but related, is provided by \prefpp{prop:partition_orel}
%for use with the \structe{partition set structure} \xref{def:ss_partition}.

%---------------------------------------
\begin{definition}
\label{def:subset}
%\label{def:ss_subset}
%---------------------------------------
Let $\ssetS$ be a \structe{set structure} \xref{def:ss} on a set $\sid$.
\defbox{\begin{array}{M}
  The relation $\sorel\in\clR{\ssetS}{\ssetS}$ is defined as
  \\\indentx$ \setA \sorel \setB \qquad\text{if}\qquad x\in\setA \implies x\in\setB \qquad \forall x\in\sid$
\end{array}}
\end{definition}

%---------------------------------------
\begin{proposition}[order properties]
\label{prop:ss_order}
%---------------------------------------
Let $\ssetS$ be a \structe{set structure} \xref{def:ss} on a set $\sid$.
\propbox{\begin{array}{M}
  The pair $\opair{\ssetS}{\sorel}$ is an \hie{ordered set}. In particular,
  \\$\begin{array}{@{\qquad}rcl D rcl c rcl CDD}
    \setA &\sorel& \setA  &     &       &         &       &        &       &         &      & \forall \setA\in\ssetS             & (\prope{reflexive})     & and \\
    \setA &\sorel& \setB  & and & \setB &\sorel& \setC &\implies& \setA &\sorel&\setC & \forall \setA,\setB,\setC\in\ssetS & (\prope{transitive})    & and \\
    \setA &\sorel& \setB  & and & \setB &\sorel& \setA &\implies& \setA &=        &\setB & \forall \setA,\setB\in\ssetS       & (\prope{anti-symmetric}).
  \end{array}$
\end{array}}
\end{proposition}
\begin{proof}
\ifdochaselse{order}{By \prefpp{def:order_rel}, a}{A}
relation is an \hie{order relation} if it is
\prope{reflexive}, \prope{transitive}, and \prope{anti-symmetric}.

\begin{enumerate}
  \item Proof that $\sorel$ is \prope{reflexive} on $\psetx$:
    \begin{align*}
      x\in\setA
        &\implies x\in\setA
      \\&\implies \setA\sorel\setA
      %\setA
      %  &= \set{x\in\sid}{x\in\setA}
      %  && \text{by definition of set \ifdochas{found}{\prefpo{def:set}}}
      %\\&\sorel \set{x\in\sid}{x\in\setA}
      %  && \text{by definition of $\sorel$ \ifdochas{found}{\prefpo{def:subset}}}
      %\\&= \setA
      %  && \text{by definition of set \ifdochas{found}{\prefpo{def:set}}}
    \end{align*}

  \item Proof that $\sorel$ is \prope{transitive} on $\psetx$:
    \begin{align*}
      x\in\setA
        &\implies x\in\setB
        &&        \text{by first left hypothesis}
      \\&\implies x\in\setC
        &&        \text{by second left hypothesis}
      \\&\implies \setA\sorel\setC
        &&        \text{\ifdochas{found}{by \prefp{def:subset}}}
    \end{align*}

  \item Proof that $\sorel$ is \prope{anti-symmetric} on $\psetx$:
    \begin{align*}
      \setA\sorel\setB
        &\implies \brp{x\in\setA\implies x\in\setB}
        \\
      \setB\sorel\setA
        &\implies \brp{x\in\setB\implies x\in\setA}
        \\
      \setA\sorel\setB \text{ and } \setB\sorel\setA
        &\implies \brp{x\in\setA\iff x\in\setB}
      \\&\implies \setA=\setB
    \end{align*}
\end{enumerate}
\end{proof}

In a set structure that is \prope{closed} under the \ope{union} operation $\setu$ and 
\ope{intersection} operation $\seti$, 
the \structe{greatest lower bound} of any two elements $\setA$ and $\setB$ is simply $\setA\seti\setB$
and \structe{least upper bound} is simply $\setA\setu\setB$ \xref{prop:ss_bounds}.
However, this may not be true for a set structure that is \emph{not} closed under these operations \xref{ex:union_neq_join}.

%---------------------------------------
\begin{proposition}
\label{prop:ss_bounds}
%---------------------------------------
Let $\ssetS$ be a \structe{set structure} \xref{def:ss} on a set $\sid$.
\propbox{\begin{array}{M}
  \textbf{If} $\ssetS$ is closed under $\sor$ and $\sand$ \textbf{then}
  \\$\begin{array}{@{\quad}lMM@{\qquad}lD}
    \setA \sor  \setB & is the \hie{least upper bound   } & of $\setA$ and $\setB$ in $\opair{\ssetS}{\sorel}$ & (\sor=\join) & and \\
    \setA \sand \setB & is the \hie{greatest lower bound} & of $\setA$ and $\setB$ in $\opair{\ssetS}{\sorel}$ & (\sand=\meet).
  \end{array}$
\end{array}}
\end{proposition}
\begin{proof}
\begin{enumerate}
  \item Proof that $\setA\sor\setB$ is the least upper bound:
    \begin{align*}
      \setA
        &=      \set{x\in\setX}{x\in\setA}
      \\&\sorel \set{x\in\setX}{x\in\setA \text{ or } x\in\setB}
      \\&=      \setA\sor\setB
        &&      \text{by \prefp{def:ss_setops}}
      \\
      \setB
        &=      \set{x\in\setX}{x\in\setB}
      \\&\sorel \set{x\in\setX}{x\in\setA \text{ or } x\in\setB}
      \\&=      \setA\sor\setB
        &&      \text{by \prefp{def:ss_setops}}
      \\
      \setA\sorel\setC \text{ and } \setB\sorel\setC
        &\implies \brb{x\in\setA \text{ and } y\in\setB \quad\implies\quad x,y\in\setC}
      \\&\implies \brb{x\in\setA \text{ or }  x\in\setB \quad\implies\quad x\in\setC}
      \\&\implies \brb{x\in\setA\sor\setB \quad\implies\quad x\in\setC}
      \\&\implies \setA\sor\setB \sorel \setC
    \end{align*}

  \item Proof that $\setA\seti\setB$ is the greatest lower bound:
    \begin{align*}
      \setA\sand\setB
        &=      \set{x\in\setX}{x\in\setA \text{ and } x\in\setB}
        &&      \text{by \prefp{def:ss_setops}}
      \\&\sorel \set{x\in\setX}{x\in\setA}
      \\&=      \setA
      \\
      \setA\sand\setB
        &=      \set{x\in\setX}{x\in\setA \text{ and } x\in\setB}
        &&      \text{by \prefp{def:ss_setops}}
      \\&\sorel \set{x\in\setX}{x\in\setB}
      \\&=      \setB
      \\
      \setC\sorel\setA \text{ and } \setC\sorel\setB
        &\implies \brb{x\in\setC \implies x\in\setA \quad\text{and}\quad x\in\setC\implies x\in\setB}
      \\&\implies \brb{x\in\setC \quad\implies\quad x\in\setA \text{ or } x\in\setB}
      \\&\implies \brb{x\in\setC \quad\implies\quad x\in\setA\sand\setB}
      \\&\implies \setC\sorel\setA\sand\setB
    \end{align*}
\end{enumerate}
\end{proof}

\begin{minipage}{\tw-50mm}%
%---------------------------------------
\begin{example}
\label{ex:union_neq_join}
%---------------------------------------
  The set structure 
  \\\indentx
  $\topS\eqd\setn{\szero,\, \setn{x},\, \setn{y},\, \setn{z},\, \setn{x,y},\, \setn{x,y,z}}$
  \\
  ordered by the set inclusion relation $\sorel$
  is illustrated by the Hasse diagram to the right.
  Note that 
  \\\indentx
  $\setn{x} \join \setn{z} = \setn{x,y,z} \ne  \setn{x,z} = \setn{x} \setu \setn{z}.$
  \\
  That is, the set union operation $\setu$ is \emph{not} equivalent to the 
  order join operation $\join$.
\end{example}
\end{minipage}%
\hfill\tbox{\includegraphics{../common/math/graphics/pdfs/latwavxyz.pdf}}\hfill\mbox{}\\%

%\ifexclude{mssa}
{
%=======================================
\subsection{Lattices of topologies}
%=======================================
\ifdochaselse{topology}{See \prefpp{sec:lattop}}{
\begin{figure}
  \centering
  \includegraphics{../common/math/graphics/pdfs/latlattopxyz.pdf}%
  \caption{
    Lattice of \structe{topologies} on $\setX\eqd\setn{x,y,z}$ (see \prefp{ex:set_lat_top_xyz})
    \label{fig:set_latlat_top_xyz}
    }
\end{figure}

\begin{minipage}{\tw-78mm}%
%---------------------------------------
\begin{example}
\label{ex:set_lat_top_xy}
\footnotemark
\index{topology!trivial}
\index{topology!indiscrete}
\index{topology!discrete}
%---------------------------------------
\prefpp{ex:top_xy} lists the four topologies on the set $\sid\eqd\setn{x,y}$.
The lattice of these topologies
\\\indentx$\lattice{\setn{\topT_1,\, \topT_2,\, \topT_3,\, \topT_4}}{\sorel}{\setu}{\seti}$\\
is illustrated by the \structe{Hasse diagram} to the right.
%Note that there are only four valid topologies out of a total sixteen
%possible families of sets:
%($2^{\seto{\psetx}} = 2^{2^{\seto{X}}} = 2^{2^2} = 2^4 = 16$).
%Half of the sixteen families are not valid topologies because they do not contain
%$\szero$ and half of the remaining are not valid because they do not contain $\sid$.
%This leaves $16 \cprod \frac{1}{2} \cprod \frac{1}{2} = 4$.
\end{example}
\end{minipage}%
\citetblt{
  \citerp{isham1999}{44},
  \citerp{isham1989}{1515}
  }%
\hfill\tbox{\includegraphics{../common/math/graphics/pdfs/lattopxy.pdf}}%%


\begin{minipage}{\tw-45mm}
%---------------------------------------
\begin{example}
\label{ex:set_lat_top_xyz}
\footnotemark
%---------------------------------------
Let a given topology in $\sssT{\setn{x,y,z}}$ be represented by a Hasse diagram 
as illustrated to the right, where a circle present means the indicated set is in the topology,
and a circle absent means the indicated set is not in the topology.
\prefpp{ex:top_xyz} lists the 29 topologies $\sssT{\setn{x,y,z}}$.
The lattice of these 29 topologies $\lattice{\sssT{\setn{x,y,z}}}{\sorel}{\setu}{\seti}$
is illustrated in \prefpp{fig:set_latlat_top_xyz}. % and \prefpp{fig:set_lat_top_xyz}.
The five topologies
$\topT_{1}$, $\topT_{41}$, $\topT_{22}$, $\topT_{14}$, and $\topT_{77}$
are also \hie{algebras of sets}\ifsxref{setstrct}{def:algsets}; 
these five sets are shaded in \pref{fig:set_latlat_top_xyz}. % and represented as solid dots in \pref{fig:set_lat_top_xyz}.
%Furthermore, \prefpp{fig:set_lat_top_xyz_d} redraws each of the 29 topological lattices 
%in a simpler unlabeled form demonstrating the distributive property of the topologies.
\end{example}%
\end{minipage}%
%\addtocounter{footnote}{-1}%
\citetblt{%
  \citerp{isham1999}{44},
  \citerp{isham1989}{1516},
  \citerp{steiner1966}{386}
  }%
%\stepcounter{footnote}%
%\citetblt{%
%  \citer{freese_latdraw}
%  }%
\hfill\tbox{\includegraphics{../common/math/graphics/pdfs/lat2xyzdotted.pdf}}%

%---------------------------------------
\begin{theorem}
\citetbl{
  \citerp{steiner1966}{384}
  }
%---------------------------------------
Let $\sssT{\sid}$ be the \structb{lattice of topologies} on a set $\sid$ with $\seto{\sid}$ elements.
\thmbox{\begin{array}{rclcl}
  \seto{\sid} &\le& 2 & \implies & \text{$\sssT{\sid}$ is \emph{distributive}} \\
  \seto{\sid} &\ge& 3 & \implies & \text{$\sssT{\sid}$ is \emph{not modular} (and not distributive)}
\end{array}}
\end{theorem}

%---------------------------------------
\begin{theorem}
\citetbl{
  \citerp{steiner1966}{385}
  }
%---------------------------------------
Let $\sssT{\sid}$ be the \structb{lattice of topologies} on a set $\sid$.
\thmbox{
  \sssT{\sid} \text{ is \emph{self-dual}} \qquad\iff\qquad \seto{\sid}\le 3
  }
\end{theorem}

%---------------------------------------
\begin{theorem}
\footnote{
  \citer{vanrooij1968},
  \citerp{steiner1966}{397},
  \citer{gaifman1961},
  \citer{hartmanis1958}
  }
%---------------------------------------
\thmbox{
  \text{Every lattice of topologies is complemented.}
  }
\end{theorem}

%---------------------------------------
\begin{theorem}
\footnote{
  \citer{hartmanis1958},
  \citerp{schnare1968}{56},
  \citer{watson1994},
  \citerp{brown1996}{32}
  }
\label{thm:top_comp_num}
%---------------------------------------
\thmboxp{
  Every \structe{topology} \xref{def:topology} except the \structe{discrete topology} and 
  \structe{indiscrete topology} \xref{ex:indiscretetop} in the
  \structb{lattice of topologies} on a set $\sid$ has at least $\seto{\sid}-1$ 
  \structe{complements}.
  }
\end{theorem}
%Let $\hat{\Sigma}(\sid)$ be the set of all topologies on $\sid$ except for the discrete and indiscrete topologies on $\sid$.

%---------------------------------------
\begin{example}
\label{ex:top_xyz_cmp}
%---------------------------------------
\prefpp{ex:top_xyz} lists the 29 topologies on a set $\sid\eqd\setn{x,y,z}$.
By \prefpp{thm:top_comp_num}, with the exception of
$\topT_{00}$ (the indiscrete topology) and $\topT_{77}$ (the discrete topology),
each of those topologies has exactly $\seto{\sid}-1=3-1=2$ complements.
\prefpp{tab:top_xyz_cmp} lists the 29 topologies on $\setn{x,y,z}$ along with their respective complements.
\end{example}
\begin{table}
  \arrayrulecolor{ex}
  \begin{tabular}{|>{$}l<{=$}
                    @{$\{$}*{8}{>{$}l<{$}@{\;}}
                    @{$\}$\quad} |>{$}c<{$\quad}| >{\quad$}c<{$}|}
    \hline
    \rowcolor{ex}
    \mc{9}{|G|}{topologies on $\setn{x,y,z}$}
     & \mc{1}{G|}{1st complement}
     & \mc{1}{G|}{2nd compl.}
    \\\hline
    \topT_{00} & \szero, &&&&&&& \sid
                & \topT_{77}
                &
                \\
    \topT_{01} & \szero,&\setn{x},&&&&&&\sid
                & \topT_{56}
                & \topT_{66}
                \\
    \topT_{02} & \szero,&&\setn{y},&&&&&\sid
                & \topT_{65}
                & \topT_{35}
                \\
    \topT_{04} & \szero,&&&\setn{z},&&&&\sid
                & \topT_{53}
                & \topT_{33}
                \\
    \topT_{10} & \szero,&&&&\setn{x,y},&&&\sid
                & \topT_{65}
                & \topT_{66}
                \\
    \topT_{20} & \szero,&&&&&\setn{x,z},&&\sid
                & \topT_{53}
                & \topT_{56}
                \\
    \topT_{40} & \szero,&&&&&&\setn{y,z},&\sid
                & \topT_{33}
                & \topT_{35}
                \\
    \topT_{11} & \szero,&\setn{x},&&&\setn{x,y},&&&\sid
                & \topT_{64}
                & \topT_{46}
                \\
    \topT_{21} & \szero,&\setn{x},&&&&\setn{x,z},&&\sid
                & \topT_{52}
                & \topT_{46}
                \\
    \topT_{41} & \szero,&\setn{x},&&&&&\,\setn{y,z},&\sid
                & \topT_{22}
                & \topT_{14}
                \\
    \topT_{12} & \szero,&&\setn{y},&&\setn{x,y},&&&\sid
                & \topT_{64}
                & \topT_{25}
                \\
    \topT_{22} & \szero,&&\setn{y},&&&\setn{x,z},&&\sid
                & \topT_{41}
                & \topT_{14}
                \\
    \topT_{42} & \szero,&&\setn{y},&&&&\setn{y,z},&\sid
                & \topT_{31}
                & \topT_{25}
                \\
    \topT_{14} & \szero,&&&\setn{z},&\setn{x,y},&&&\sid
                & \topT_{41}
                & \topT_{22}
                \\
    \topT_{24} & \szero,&&&\setn{z},&&\setn{x,z},&&\sid
                & \topT_{52}
                & \topT_{13}
                \\
    \topT_{44} & \szero,&&&\setn{z},&&&\setn{y,z},&\sid
                & \topT_{31}
                & \topT_{13}
                \\
    \topT_{31} & \szero,&\setn{x},&&&\setn{x,y},&\setn{x,z},&&\sid
                & \topT_{42}
                & \topT_{44}
                \\
    \topT_{52} & \szero,&&\setn{y},&&\setn{x,y},&\setn{x,z},&&\sid
                & \topT_{21}
                & \topT_{24}
                \\
    \topT_{64} & \szero,&&&\setn{z},&&\setn{x,z},&\setn{y,z},&\sid
                & \topT_{11}
                & \topT_{12}
                \\
    \topT_{13} & \szero,&\setn{x},&\setn{y},&&\setn{x,y},&&&\sid
                & \topT_{24}
                & \topT_{44}
                \\
    \topT_{25} & \szero,&\setn{x},&&\setn{z},&&\setn{x,z},&&\sid
                & \topT_{12}
                & \topT_{42}
                \\
    \topT_{46} & \szero,&&\setn{y},&\setn{z},&&&\setn{y,z},&\sid
                & \topT_{11}
                & \topT_{21}
                \\
    \topT_{33} & \szero,&\setn{x},&\setn{y},&&\setn{x,y},&\setn{x,z},&&\sid
                & \topT_{04}
                & \topT_{40}
                \\
    \topT_{53} & \szero,&\setn{x},&\setn{y},&&\setn{x,y},&&\setn{y,z},&\sid
                & \topT_{04}
                & \topT_{20}
                \\
    \topT_{35} & \szero,&\setn{x},&&\setn{z},&\setn{x,y},&\setn{x,z},&&\sid
                & \topT_{02}
                & \topT_{40}
                \\
    \topT_{65} & \szero,&\setn{x},&&\setn{z},&&\setn{x,z},&\setn{y,z},&\sid
                & \topT_{02}
                & \topT_{10}
                \\
    \topT_{56} & \szero,&&\setn{y},&\setn{z},&\setn{x,y},&&\setn{y,z},&\sid
                & \topT_{01}
                & \topT_{20}
                \\
    \topT_{66} & \szero,&&\setn{y},&\setn{z},&&\setn{x,z},&\setn{y,z},&\sid
                & \topT_{01}
                & \topT_{10}
                \\
    \topT_{77} & \szero,&\setn{x},&\setn{y},&\setn{z},&\setn{x,y},&\setn{x,z},&\setn{y,z},&\sid
                & \topT_{00}
                &
                \\
    \hline
  \end{tabular}
  \caption{
    the 29 topologies on a set $\setn{x,y,z}$ along with their respective complements
    \xref{ex:top_xyz_cmp}
    \label{tab:top_xyz_cmp}
    }
\end{table}


%---------------------------------------
\begin{theorem}
\footnote{
  \citerp{larson1975}{179},
  \citor{frohlich1964},
  \citor{vaidyanathaswamy1960},
  \citor{vaidyanathaswamy1947}
  }
%---------------------------------------
\thmbox{
  \text{$\sssT{\sid}$ is a topology of sets}
  \qquad\implies\qquad
  \left\{\begin{tabular}{l}
    $\sssT{\sid}$ is atomic. \\
    $\sssT{\sid}$ is anti-atomic.
  \end{tabular}\right.
  }
\end{theorem}

%---------------------------------------
\begin{theorem}
\citetbl{
  \citerp{larson1975}{179},
  \citor{frohlich1964}
  }
%---------------------------------------
Let $\sssT{\sid}$ be the lattice of topologies on a set $\sid$ and let $n\eqd \seto{\sid}$.
\thmbox{\begin{tabular}{ll}
  $\sssT{\sid}$ contains $\ds 2^n-2$              atoms      & for finite $\sid$. \\
  $\sssT{\sid}$ contains $\ds 2^\seto{\sid}$     atoms      & for infinite $\sid$. \\
  $\sssT{\sid}$ contains $\ds n(n-1)$             anti-atoms & for finite $\sid$. \\
  $\sssT{\sid}$ contains $\ds 2^{2^\seto{\sid}}$ anti-atoms & for infinite $\sid$.
\end{tabular}}
\end{theorem}
}

%=======================================
\subsection{Lattices of algebra of sets}
%=======================================
%\begin{minipage}{\tw-45mm}
%%---------------------------------------
%%\begin{example}
%%\label{ex:algxyz_hasse}
%%---------------------------------------
%%\prefpp{ex:alg_xyz} lists the 5 algebras of sets on $\setn{x,y,z}$.
%The lattice of these 5 algebras of sets $\lattice{\sssA{\setn{x,y,z}}}{\sorel}{\setu}{\seti}$
%is illustrated by two Hasse diagrams in \prefpp{fig:latalgxyz}.
%Let a given \structe{algebra of sets} $\algA$ on $\setn{x,y,z}$ be represented by a \structe{Hasse diagram}\ifsxref{order}{def:hasse}
%as illustrated to the right, where a circle present means the indicated set is in $\algA$,
%and a circle absent means the indicated set is not in $\algA$.
%This notation is used in right side lattice in \pref{fig:latalgxyz}.
%%(rendered by Ralph Freese's \hie{LatDraw} software\footnotemark).
%The five algebra of sets also appear in \prefp{fig:set_latlat_top_xyz} as solid dots. %and 
%%in \pref{fig:set_latlat_top_xyz} as shaded circles and in both figures are labeled as 
%%$\topT_{1}$, $\topT_{41}$, $\topT_{22}$, $\topT_{14}$, and $\topT_{77}$.
%%\end{example}%
%\end{minipage}%
%\hfill%
%%\addtocounter{footnote}{-1}%
%%\stepcounter{footnote}%
%%\citetblt{%
%%  \citer{freese_latdraw}
%%  }%
%\begin{minipage}{42mm}%
%%============================================================================
% Daniel J. Greenhoe
% LaTeX file
% lattice (2^{x,y,z}, subseteq)
%============================================================================
\begin{pspicture}(-2.7,-\latbot)(2.7,3.4)
  %---------------------------------
  % settings
  %---------------------------------
  %\psset{%
  %  labelsep=1.5mm,
  %  dotsep=0.5pt,
  %  }%
  %---------------------------------
  % nodes
  %---------------------------------
  \Cnode(0,3){t}%
  \Cnode(-1,2){xy} \Cnode(0,2){xz} \Cnode(1,2){yz}%
  \Cnode(-1,1){x}  \Cnode(0,1){y}  \Cnode(1,1){z}%
  \Cnode(0,0){b}%
  %---------------------------------
  % node connections
  %---------------------------------
  \psset{linestyle=dotted}%
  \ncline{t}{xy}\ncline{t}{xz}\ncline{t}{yz}%
  \ncline{x}{xy}\ncline{x}{xz}%
  \ncline{y}{xy}\ncline{y}{yz}%
  \ncline{z}{xz}\ncline{z}{yz}%
  \ncline{b}{x} \ncline{b}{y} \ncline{b}{z}%
  %---------------------------------
  % node labels
  %---------------------------------
  \uput[180](t) {$\setn{x,y,z}$}%
  \uput[180](xy){$\setn{x,y}$}%   
 %\uput{1pt}[ 70](xz){$\setn{x,z}$} 
  \uput[0](yz){$\setn{y,z}$}%
  \uput[180](x) {$\setn{x}$}%     
 %\uput{1pt}[-45](y) {$\setn{y}$}   
  \uput[0](z) {$\setn{z}$}%
  \uput[180](b) {$\szero$}%
  \uput[0](1,3){\rnode{xzlabel}{$\setn{x,z}$}}% 
  \uput[0](1,0.25){\rnode{ylabel}{$\setn{y}$}}%
  \ncline[linestyle=solid,nodesep=1pt,linecolor=red]{->}{xzlabel}{xz}%
  \ncline[linestyle=solid,nodesep=1pt,linecolor=red]{->}{ylabel}{y}%
\end{pspicture}%

%\end{minipage}%
%\end{example}


%\begin{figure}
%\begin{center}
%  %============================================================================
% Daniel J. Greenhoe
% LaTeX File
% lattice of topologies over the set {x,y,z}
%============================================================================
{\psset{unit=0.20mm}%
\begin{pspicture}(-150,-40)(150,240)%
  %---------------------------------
  % settings
  %---------------------------------
  \fns%
  \psset{
    labelsep=1.5mm,
    linearc=75\psxunit,
    }
  %---------------------------------
  % developement tools
  %---------------------------------
  %\psgrid[unit=100\psunit](-5,-1)(5,5)%
  %---------------------------------
  % nodes
  %---------------------------------
  \Cnode(   0,200){T77}%
  \Cnode( 100,100){T14}%
  \Cnode(   0,100){T22}%
  \Cnode(-100,100){T41}%
  \Cnode(   0,0)  {T00}%
  %---------------------------------
  % node connections
  %---------------------------------
  \ncline{T77}{T14}%
  \ncline{T77}{T22}%
  \ncline{T77}{T41}%
  \ncline{T00}{T14}%
  \ncline{T00}{T22}%
  \ncline{T00}{T41}%
  %---------------------------------
  % node labels
  %---------------------------------
  \uput[ 90](T77){$\algA_{77}$}%
  \uput[-45](T14){$\algA_{14}$}%
  \uput[-45](T22){$\algA_{22}$}%
  \uput[225](T41){$\algA_{41}$}%
  \uput[-90](T00){$\algA_{00}$}%
\end{pspicture}%
}%%
%  \hspace{15mm}
%  \psset{unit=17mm}%
%  %============================================================================
% Daniel J. Greenhoe
% LaTeX File
% lattice of topologies over the set {x,y,z}
% nominal unit = 20mm
%============================================================================
\begin{pspicture}(-1.44,-.40)(1.44,2.50)%
  %---------------------------------
  % settings
  %--------------------------------
  \fns%
  %\psset{labelsep=1.5mm,radius=75\psunit}
  \psset{
    labelsep=8mm,
    radius=7.5mm,
    linearc=1\psxunit,
    }
  %---------------------------------
  % developement tools
  %---------------------------------
  %\psgrid[xunit=1\psxunit,yunit=1\psyunit](-5,-1)(5,5)%
  %---------------------------------
  % nodes
  %---------------------------------
  \Cnode(   0,2){T77}%
  \Cnode( 1,1){T14}%
  \Cnode(   0,1){T22}%
  \Cnode(-1,1){T41}%
  \Cnode(   0,0)  {T00}%
  %---------------------------------
  % node connections
  %---------------------------------
  \ncline{T77}{T14}%
  \ncline{T77}{T22}%
  \ncline{T77}{T41}%
  \ncline{T00}{T14}%
  \ncline{T00}{T22}%
  \ncline{T00}{T41}%
  %---------------------------------
  % node labels
  %---------------------------------
  \uput[  0](T77){$\algA_{77}$}%
  \uput[-90](T14){$\algA_{14}$}%
  \uput[-45](T22){$\algA_{22}$}%
  \uput[-90](T41){$\algA_{41}$}%
  \uput[-20](T00){$\algA_{00}$}%
  %---------------------------------
  % node inner lattices
  %---------------------------------
  \psset{
    unit=4mm,
    radius=1mm,
    dotsep=0.5pt,
    linecolor=blue,
    }%
  \rput(T77){\begin{pspicture}(-1,0)(1,3)
                           \Cnode(0,3){t}
      \Cnode(-1,2){xy} \Cnode(0,2){xz} \Cnode(1,2){yz}
      \Cnode(-1,1){x}  \Cnode(0,1){y}  \Cnode(1,1){z}
                           \Cnode(0,  0){b}
      \psset{linestyle=dotted}%
      \ncline{t}{xy}\ncline{t}{xz}\ncline{t}{yz}
      \ncline{x}{xy}\ncline{x}{xz}
      \ncline{y}{xy}\ncline{y}{yz}
      \ncline{z}{xz}\ncline{z}{yz}
      \ncline{b}{x} \ncline{b}{y} \ncline{b}{z}
    \end{pspicture}}%
  \rput(T14){\begin{pspicture}(-1,0)(1,3)
                           \Cnode(0,3){t}%
      \Cnode(-1,2){xy} \pnode(0,2){xz} \pnode(1,2){yz}%
      \pnode(-1,1){x}  \pnode(0,1){y}  \Cnode(1,1){z}%
                           \Cnode(0,  0){b}%
      \psset{linestyle=dotted}%
      \ncline{t}{xy}\ncline{t}{xz}\ncline{t}{yz}
      \ncline{x}{xy}\ncline{x}{xz}
      \ncline{y}{xy}\ncline{y}{yz}
      \ncline{z}{xz}\ncline{z}{yz}
      \ncline{b}{x} \ncline{b}{y} \ncline{b}{z}
    \end{pspicture}}%
  \rput(T22){\begin{pspicture}(-1,0)(1,3)
                           \Cnode(0,3){t}%
      \pnode(-1,2){xy} \Cnode(0,2){xz} \pnode(1,2){yz}%
      \pnode(-1,1){x}  \Cnode(0,1){y}  \pnode(1,1){z}%
                           \Cnode(0,  0){b}%
      \psset{linestyle=dotted}%
      \ncline{t}{xy}\ncline{t}{xz}\ncline{t}{yz}
      \ncline{x}{xy}\ncline{x}{xz}
      \ncline{y}{xy}\ncline{y}{yz}
      \ncline{z}{xz}\ncline{z}{yz}
      \ncline{b}{x} \ncline{b}{y} \ncline{b}{z}
    \end{pspicture}}%
  \rput(T41){\begin{pspicture}(-1,0)(1,3)
                           \Cnode(0,3){t}%
      \pnode(-1,2){xy} \pnode(0,2){xz} \Cnode(1,2){yz}%
      \Cnode(-1,1){x}  \pnode(0,1){y}  \pnode(1,1){z}%
                           \Cnode(0,  0){b}%
      \psset{linestyle=dotted}%
      \ncline{t}{xy}\ncline{t}{xz}\ncline{t}{yz}
      \ncline{x}{xy}\ncline{x}{xz}
      \ncline{y}{xy}\ncline{y}{yz}
      \ncline{z}{xz}\ncline{z}{yz}
      \ncline{b}{x} \ncline{b}{y} \ncline{b}{z}
    \end{pspicture}}%
  \rput(T00){\begin{pspicture}(-1,0)(1,3)
                           \Cnode(0,3){t}%
      \pnode(-1,2){xy} \pnode(0,2){xz} \pnode(1,2){yz}%
      \pnode(-1,1){x}  \pnode(0,1){y}  \pnode(1,1){z}%
                           \Cnode(0,  0){b}%
      \psset{linestyle=dotted}%
      \ncline{t}{xy}\ncline{t}{xz}\ncline{t}{yz}
      \ncline{x}{xy}\ncline{x}{xz}
      \ncline{y}{xy}\ncline{y}{yz}
      \ncline{z}{xz}\ncline{z}{yz}
      \ncline{b}{x} \ncline{b}{y} \ncline{b}{z}
    \end{pspicture}}%
\end{pspicture}%%
%\end{center}
%\caption{
%  Lattice of \structe{algebra of sets} on $\sid\eqd\setn{x,y,z}$
%  \label{fig:latalgxyz}
%  }
%\end{figure}

%\begin{figure}
%\begin{center}
%%============================================================================
% Daniel J. Greenhoe
% LaTeX File
% lattice of topologies over the set {x,y,z}
% nominal unit = 20mm
%============================================================================
\begin{pspicture}(-1.44,-.40)(1.44,2.50)%
  %---------------------------------
  % settings
  %--------------------------------
  \fns%
  %\psset{labelsep=1.5mm,radius=75\psunit}
  \psset{
    labelsep=8mm,
    radius=7.5mm,
    linearc=1\psxunit,
    }
  %---------------------------------
  % developement tools
  %---------------------------------
  %\psgrid[xunit=1\psxunit,yunit=1\psyunit](-5,-1)(5,5)%
  %---------------------------------
  % nodes
  %---------------------------------
  \Cnode(   0,2){T77}%
  \Cnode( 1,1){T14}%
  \Cnode(   0,1){T22}%
  \Cnode(-1,1){T41}%
  \Cnode(   0,0)  {T00}%
  %---------------------------------
  % node connections
  %---------------------------------
  \ncline{T77}{T14}%
  \ncline{T77}{T22}%
  \ncline{T77}{T41}%
  \ncline{T00}{T14}%
  \ncline{T00}{T22}%
  \ncline{T00}{T41}%
  %---------------------------------
  % node labels
  %---------------------------------
  \uput[  0](T77){$\algA_{77}$}%
  \uput[-90](T14){$\algA_{14}$}%
  \uput[-45](T22){$\algA_{22}$}%
  \uput[-90](T41){$\algA_{41}$}%
  \uput[-20](T00){$\algA_{00}$}%
  %---------------------------------
  % node inner lattices
  %---------------------------------
  \psset{
    unit=4mm,
    radius=1mm,
    dotsep=0.5pt,
    linecolor=blue,
    }%
  \rput(T77){\begin{pspicture}(-1,0)(1,3)
                           \Cnode(0,3){t}
      \Cnode(-1,2){xy} \Cnode(0,2){xz} \Cnode(1,2){yz}
      \Cnode(-1,1){x}  \Cnode(0,1){y}  \Cnode(1,1){z}
                           \Cnode(0,  0){b}
      \psset{linestyle=dotted}%
      \ncline{t}{xy}\ncline{t}{xz}\ncline{t}{yz}
      \ncline{x}{xy}\ncline{x}{xz}
      \ncline{y}{xy}\ncline{y}{yz}
      \ncline{z}{xz}\ncline{z}{yz}
      \ncline{b}{x} \ncline{b}{y} \ncline{b}{z}
    \end{pspicture}}%
  \rput(T14){\begin{pspicture}(-1,0)(1,3)
                           \Cnode(0,3){t}%
      \Cnode(-1,2){xy} \pnode(0,2){xz} \pnode(1,2){yz}%
      \pnode(-1,1){x}  \pnode(0,1){y}  \Cnode(1,1){z}%
                           \Cnode(0,  0){b}%
      \psset{linestyle=dotted}%
      \ncline{t}{xy}\ncline{t}{xz}\ncline{t}{yz}
      \ncline{x}{xy}\ncline{x}{xz}
      \ncline{y}{xy}\ncline{y}{yz}
      \ncline{z}{xz}\ncline{z}{yz}
      \ncline{b}{x} \ncline{b}{y} \ncline{b}{z}
    \end{pspicture}}%
  \rput(T22){\begin{pspicture}(-1,0)(1,3)
                           \Cnode(0,3){t}%
      \pnode(-1,2){xy} \Cnode(0,2){xz} \pnode(1,2){yz}%
      \pnode(-1,1){x}  \Cnode(0,1){y}  \pnode(1,1){z}%
                           \Cnode(0,  0){b}%
      \psset{linestyle=dotted}%
      \ncline{t}{xy}\ncline{t}{xz}\ncline{t}{yz}
      \ncline{x}{xy}\ncline{x}{xz}
      \ncline{y}{xy}\ncline{y}{yz}
      \ncline{z}{xz}\ncline{z}{yz}
      \ncline{b}{x} \ncline{b}{y} \ncline{b}{z}
    \end{pspicture}}%
  \rput(T41){\begin{pspicture}(-1,0)(1,3)
                           \Cnode(0,3){t}%
      \pnode(-1,2){xy} \pnode(0,2){xz} \Cnode(1,2){yz}%
      \Cnode(-1,1){x}  \pnode(0,1){y}  \pnode(1,1){z}%
                           \Cnode(0,  0){b}%
      \psset{linestyle=dotted}%
      \ncline{t}{xy}\ncline{t}{xz}\ncline{t}{yz}
      \ncline{x}{xy}\ncline{x}{xz}
      \ncline{y}{xy}\ncline{y}{yz}
      \ncline{z}{xz}\ncline{z}{yz}
      \ncline{b}{x} \ncline{b}{y} \ncline{b}{z}
    \end{pspicture}}%
  \rput(T00){\begin{pspicture}(-1,0)(1,3)
                           \Cnode(0,3){t}%
      \pnode(-1,2){xy} \pnode(0,2){xz} \pnode(1,2){yz}%
      \pnode(-1,1){x}  \pnode(0,1){y}  \pnode(1,1){z}%
                           \Cnode(0,  0){b}%
      \psset{linestyle=dotted}%
      \ncline{t}{xy}\ncline{t}{xz}\ncline{t}{yz}
      \ncline{x}{xy}\ncline{x}{xz}
      \ncline{y}{xy}\ncline{y}{yz}
      \ncline{z}{xz}\ncline{z}{yz}
      \ncline{b}{x} \ncline{b}{y} \ncline{b}{z}
    \end{pspicture}}%
\end{pspicture}%%
%\end{center}
%\caption{
%  Lattice of topologies on $\setX\eqd\setn{x,y,z}$ (see \prefp{ex:latalgxyz})
%  \label{fig:latlattopxyz}
%  }
%\end{figure}


%\begin{figure}
%\begin{center}
%\footnotesize
% %============================================================================
% Daniel J. Greenhoe
% LaTeX File
% lattice of topologies over the set {x,y,z}
% nominal unit = 20mm
%============================================================================
\begin{pspicture}(-1.44,-.40)(1.44,2.50)%
  %---------------------------------
  % settings
  %--------------------------------
  \fns%
  %\psset{labelsep=1.5mm,radius=75\psunit}
  \psset{
    labelsep=8mm,
    radius=7.5mm,
    linearc=1\psxunit,
    }
  %---------------------------------
  % developement tools
  %---------------------------------
  %\psgrid[xunit=1\psxunit,yunit=1\psyunit](-5,-1)(5,5)%
  %---------------------------------
  % nodes
  %---------------------------------
  \Cnode(   0,2){T77}%
  \Cnode( 1,1){T14}%
  \Cnode(   0,1){T22}%
  \Cnode(-1,1){T41}%
  \Cnode(   0,0)  {T00}%
  %---------------------------------
  % node connections
  %---------------------------------
  \ncline{T77}{T14}%
  \ncline{T77}{T22}%
  \ncline{T77}{T41}%
  \ncline{T00}{T14}%
  \ncline{T00}{T22}%
  \ncline{T00}{T41}%
  %---------------------------------
  % node labels
  %---------------------------------
  \uput[  0](T77){$\algA_{77}$}%
  \uput[-90](T14){$\algA_{14}$}%
  \uput[-45](T22){$\algA_{22}$}%
  \uput[-90](T41){$\algA_{41}$}%
  \uput[-20](T00){$\algA_{00}$}%
  %---------------------------------
  % node inner lattices
  %---------------------------------
  \psset{
    unit=4mm,
    radius=1mm,
    dotsep=0.5pt,
    linecolor=blue,
    }%
  \rput(T77){\begin{pspicture}(-1,0)(1,3)
                           \Cnode(0,3){t}
      \Cnode(-1,2){xy} \Cnode(0,2){xz} \Cnode(1,2){yz}
      \Cnode(-1,1){x}  \Cnode(0,1){y}  \Cnode(1,1){z}
                           \Cnode(0,  0){b}
      \psset{linestyle=dotted}%
      \ncline{t}{xy}\ncline{t}{xz}\ncline{t}{yz}
      \ncline{x}{xy}\ncline{x}{xz}
      \ncline{y}{xy}\ncline{y}{yz}
      \ncline{z}{xz}\ncline{z}{yz}
      \ncline{b}{x} \ncline{b}{y} \ncline{b}{z}
    \end{pspicture}}%
  \rput(T14){\begin{pspicture}(-1,0)(1,3)
                           \Cnode(0,3){t}%
      \Cnode(-1,2){xy} \pnode(0,2){xz} \pnode(1,2){yz}%
      \pnode(-1,1){x}  \pnode(0,1){y}  \Cnode(1,1){z}%
                           \Cnode(0,  0){b}%
      \psset{linestyle=dotted}%
      \ncline{t}{xy}\ncline{t}{xz}\ncline{t}{yz}
      \ncline{x}{xy}\ncline{x}{xz}
      \ncline{y}{xy}\ncline{y}{yz}
      \ncline{z}{xz}\ncline{z}{yz}
      \ncline{b}{x} \ncline{b}{y} \ncline{b}{z}
    \end{pspicture}}%
  \rput(T22){\begin{pspicture}(-1,0)(1,3)
                           \Cnode(0,3){t}%
      \pnode(-1,2){xy} \Cnode(0,2){xz} \pnode(1,2){yz}%
      \pnode(-1,1){x}  \Cnode(0,1){y}  \pnode(1,1){z}%
                           \Cnode(0,  0){b}%
      \psset{linestyle=dotted}%
      \ncline{t}{xy}\ncline{t}{xz}\ncline{t}{yz}
      \ncline{x}{xy}\ncline{x}{xz}
      \ncline{y}{xy}\ncline{y}{yz}
      \ncline{z}{xz}\ncline{z}{yz}
      \ncline{b}{x} \ncline{b}{y} \ncline{b}{z}
    \end{pspicture}}%
  \rput(T41){\begin{pspicture}(-1,0)(1,3)
                           \Cnode(0,3){t}%
      \pnode(-1,2){xy} \pnode(0,2){xz} \Cnode(1,2){yz}%
      \Cnode(-1,1){x}  \pnode(0,1){y}  \pnode(1,1){z}%
                           \Cnode(0,  0){b}%
      \psset{linestyle=dotted}%
      \ncline{t}{xy}\ncline{t}{xz}\ncline{t}{yz}
      \ncline{x}{xy}\ncline{x}{xz}
      \ncline{y}{xy}\ncline{y}{yz}
      \ncline{z}{xz}\ncline{z}{yz}
      \ncline{b}{x} \ncline{b}{y} \ncline{b}{z}
    \end{pspicture}}%
  \rput(T00){\begin{pspicture}(-1,0)(1,3)
                           \Cnode(0,3){t}%
      \pnode(-1,2){xy} \pnode(0,2){xz} \pnode(1,2){yz}%
      \pnode(-1,1){x}  \pnode(0,1){y}  \pnode(1,1){z}%
                           \Cnode(0,  0){b}%
      \psset{linestyle=dotted}%
      \ncline{t}{xy}\ncline{t}{xz}\ncline{t}{yz}
      \ncline{x}{xy}\ncline{x}{xz}
      \ncline{y}{xy}\ncline{y}{yz}
      \ncline{z}{xz}\ncline{z}{yz}
      \ncline{b}{x} \ncline{b}{y} \ncline{b}{z}
    \end{pspicture}}%
\end{pspicture}%
%\end{center}
%\caption{
%  Lattice of \structe{algebras of sets} on $\sid\eqd\setn{x,y,z}$ \xref{ex:set_lat_alg_xyz}
%  \label{fig:set_lat_alg_wxyz}
%  }
%\end{figure}

%---------------------------------------
\begin{example}
\label{ex:set_asets}
%---------------------------------------
The following table lists some algebras of sets on a finite set $\sid$.
Lattices of algebras of sets are illustrated in \prefpp{fig:set_lat_alg_xyz} and %\prefpp{fig:latalgxyz}, 
\prefpp{fig:algsets_wxyz}.

\begin{longtable}{|MMM|}
  \hline
  \mc{3}{|G|}{algebra of sets $\sssA{\sid}$ on a set $\sid$}\\
  \sssA{\szero}&=&
  \setn{\begin{array}{lcl}
    \algA_1 &=& \setn{\szero}   \\
  \end{array}}
  \\&&\\
  \sssA{\setn{x}}&=&
  \setn{\begin{array}{lcl}
    \algA_1 &=& \setn{\szero,\, \setn{x}}   \\
  \end{array}}
  \\&&\\
  \sssA{\setn{x,y}}&=&
  \setn{\begin{array}{lcl}
    \algA_1 &=& \setn{\szero,\, \sid}   \\
    \algA_2 &=& \setn{\szero,\, \setn{x},\, \setn{y},\, \sid}
  \end{array}}
  \\&&\\
  \sssA{\setn{x,y,z}}&=&
  \setn{\begin{array}{lc l*{8}l l}
    \algA_{ 1} &=& \{ & \szero, &           &           &           &             &             &             & \sid& \}\\
    \algA_{ 2} &=& \{ & \szero, & \setn{x}, &           &           &             &             & \setn{y,z}, & \sid& \}\\
    \algA_{ 3} &=& \{ & \szero, &           & \setn{y}, &           &             & \setn{x,z}, &             & \sid& \}\\
    \algA_{ 4} &=& \{ & \szero, &           &           & \setn{z}, & \setn{x,y}, &             &             & \sid& \}\\
    \algA_{ 5} &=& \{ & \szero, & \setn{x}, & \setn{y}, & \setn{z}, & \setn{x,y}, & \setn{x,z}, & \setn{y,z}, & \sid& \}
  \end{array}}
  \\&&\\
  \sssA{\setn{w,x,y,z}} &=&
  \\\mc{3}{|M|}{\indentx
  \setn{\begin{array}{lcl lllll l}
    \ssetA_{ 1} &=& \{ & \szero, &                                         &                                                                         &                                                         & \sid& \}\\
    \ssetA_{ 2} &=& \{ & \szero, & \setn{w},                               &                                                                         &                                           \setn{x,y,z}, & \sid& \}\\
    \ssetA_{ 3} &=& \{ & \szero, &           \setn{x},                     &                                                                         &                             \setn{w,y,z},               & \sid& \}\\
    \ssetA_{ 4} &=& \{ & \szero, &                     \setn{y},           &                                                                         &               \setn{w,x,z},                             & \sid& \}\\
    \ssetA_{ 5} &=& \{ & \szero, &                               \setn{z}, &                                                                         & \setn{w,x,y},                                           & \sid& \}\\
    \ssetA_{ 6} &=& \{ & \szero, &                                         & \setn{w,x},                                                 \setn{y,z}, &                                                         & \sid& \}\\
    \ssetA_{ 7} &=& \{ & \szero, &                                         &             \setn{w,y},                         \setn{x,z},             &                                                         & \sid& \}\\
    \ssetA_{ 8} &=& \{ & \szero, &                                         &                         \setn{w,z}, \setn{x,y},                         &                                                         & \sid& \}\\
    \ssetA_{ 9} &=& \{ & \szero, & \setn{w}, \setn{x},                     & \setn{w,x},                                                 \setn{y,z}, &                             \setn{w,y,z}, \setn{x,y,z}, & \sid& \}\\
    \ssetA_{10} &=& \{ & \szero, & \setn{w},           \setn{y},           &             \setn{w,y},                         \setn{x,z},             &               \setn{w,x,z}, \setn{x,y,z},               & \sid& \}\\
    \ssetA_{11} &=& \{ & \szero, & \setn{w},                     \setn{z}, &                         \setn{w,z}, \setn{x,y},                         & \setn{w,x,y},                             \setn{x,y,z}, & \sid& \}\\
    \ssetA_{12} &=& \{ & \szero, &           \setn{x}, \setn{y},           &                         \setn{w,z}, \setn{x,y},                         & \setn{w,x,z},                \setn{w,y,z},              & \sid& \}\\
    \ssetA_{13} &=& \{ & \szero, &           \setn{x},           \setn{z}, &             \setn{w,y},                         \setn{x,z},             & \setn{w,x,y},                \setn{w,y,z},              & \sid& \}\\
    \ssetA_{14} &=& \{ & \szero, &                     \setn{y}, \setn{z}, & \setn{w,x},                                                 \setn{y,z}, &                              \setn{w,x,y},\setn{w,x,z}, & \sid& \}\\
    \ssetA_{15} &=& \mc{7}{l}{\psetx}
   %\ssetA_{15} &=& \{ & \szero, & \setn{w}, \setn{x}, \setn{y}, \setn{z}, & \setn{w,x}, \setn{w,y}, \setn{w,z}, \setn{x,y}, \setn{x,z}, \setn{y,z}, & \setn{w,x,y},  \setn{w,y,z}, \setn{w,y,z},\setn{x,y,z}, & \sid& \}\\
  \end{array}}}
  \\\hline
\end{longtable}
\end{example}

\begin{figure}
\begin{center}
  \includegraphics{../common/math/graphics/pdfs/latlatalgwxyz.pdf}%
  %{\psset{unit=24mm}{%============================================================================
% Daniel J. Greenhoe
% LaAeX File
% lattice of topologies over the set {x,y,z}
% nominal unit = 20mm
%============================================================================
\begin{pspicture}(-3,-.40)(3,4)%
  %---------------------------------
  % settings
  %---------------------------------
  %\fns%
  %\psset{labelsep=1.5mm,radius=75\psunit}
  \psset{
    labelsep=8mm,
    radius=7.5mm,
    linearc=1\psxunit,
    }
  %---------------------------------
  % developement tools
  %---------------------------------
  %\psgrid[xunit=1\psxunit,yunit=1\psyunit](-5,-1)(5,5)%
  %---------------------------------
  % nodes
  %---------------------------------
  {\psset{linecolor=red}%\psset{radius=6mm,linecolor=red}%
    \Cnode(   0,3.5){A15}%
    \Cnode(-1.667,1){A06}\Cnode( 0,1){A07}\Cnode(1.667,1){A08}%
    \Cnode( 0,0){A01}%
    }%
  \Cnode(-2.5,2.5){A09}\Cnode(-1.5,2.5){A10}\Cnode(2.5,2.5){A11}\Cnode( 0.5,2.5){A12}\Cnode( 1.5,2.5){A13}\Cnode(-0.5,2.5){A14}%
  \Cnode(-2.5,1){A02}\Cnode(-0.833,1){A03}\Cnode(0.833,1){A04}\Cnode(2.5,1){A05}%
  %---------------------------------
  % node connections
  %---------------------------------
  %
  {\psset{linecolor=red}%\psset{radius=6mm,linecolor=red}%
    \ncline{A15}{A09}\ncline{A15}{A10}\ncline{A15}{A11}\ncline{A15}{A12}\ncline{A15}{A13}\ncline{A15}{A14}%
    \ncline{A06}{A09}\ncline{A06}{A14}%
    \ncline{A07}{A10}\ncline{A07}{A13}%
    \ncline{A08}{A11}\ncline{A08}{A12}%
    \ncline{A01}{A08}\ncline{A01}{A07}\ncline{A01}{A06}%
    }%
  %
  \ncline{A09}{A02}\ncline{A09}{A03}%
  \ncline{A10}{A02}\ncline{A10}{A04}%
  \ncline{A11}{A02}\ncline{A11}{A05}%
  \ncline{A12}{A03}\ncline{A12}{A04}%
  \ncline{A13}{A03}\ncline{A13}{A05}%
  \ncline{A14}{A04}\ncline{A14}{A05}%
  %
  \ncline{A01}{A02}\ncline{A01}{A03}\ncline{A01}{A04}\ncline{A01}{A05}%
  %---------------------------------
  % node labels
  %---------------------------------
  \uput[  0](A15){$\algA_{15}$}%
  \uput[-90](A14){$\algA_{14}$}%
  \uput[-90](A13){$\algA_{13}$}%
  \uput[ 45](A12){$\algA_{12}$}%
  \uput[-60](A11){$\algA_{11}$}%
  \uput[-90](A10){$\algA_{10}$}%
  \uput[-120](A09){$\algA_{09}$}%
  {%\psset{labelsep=6.5mm}%
    \uput[-60](A08){$\algA_{08}$}%
    \uput[-60](A07){$\algA_{07}$}%
    \uput[-90](A06){$\algA_{06}$}%
  }%
  \uput[-60](A05){$\algA_{05}$}%
  \uput[-45](A04){$\algA_{04}$}%
  \uput[-60](A03){$\algA_{03}$}%
  \uput[-90](A02){$\algA_{02}$}%
  \uput[  0](A01){$\algA_{01}$}%
  %---------------------------------
  % node inner lattices
  %---------------------------------
%  \psset{
%    unit=4mm,
%    radius=1mm,
%    dotsep=0.5pt,
%    linecolor=blue,
%    }%
  \begin{tabstr}{0.75}%
  \rput(A15){$\psetX$}%
  \rput(A14){${\begin{array}{c}%
                 \emptyset,y,z,wx,\\%
                 yz,wxy,\\%
                 xwz,\setX
               \end{array}}$}%
  \rput(A13){${\begin{array}{c}%
                 \emptyset,x,z,wy,\\%
                 xz,wxy,\\%
                 wyz,\setX
               \end{array}}$}%
  \rput(A12){${\begin{array}{c}%
                 \emptyset,x,y,wz,\\%
                 xy,wxz,\\%
                 wyz,\setX
               \end{array}}$}%
  \rput(A11){${\begin{array}{c}%
                 \emptyset,w,z,wz,\\%
                 xy,wxy,\\%
                 xyz,\setX
               \end{array}}$}%
  \rput(A10){${\begin{array}{c}%
                 \emptyset,w,y,wy,\\%
                 xz,wxz,\\%
                 xyz,\setX
               \end{array}}$}%
  \rput(A09){${\begin{array}{c}%
                 \emptyset,w,x,wx,\\%
                 yz,wyz,\\%
                 xyz,\setX
               \end{array}}$}%
  \rput(A08){${\begin{array}{c}
                 \emptyset,wz,\\
                 xy,\setX
               \end{array}}$}%
  \rput(A07){${\begin{array}{c}
                 \emptyset,wy,\\
                 xz,\setX
               \end{array}}$}%
  \rput(A06){${\begin{array}{c}
                 \emptyset,wx,\\
                 yz,\setX
               \end{array}}$}%
  \rput(A05){${\begin{array}{c}
                 \emptyset,z,\\
                 wxy,\setX
               \end{array}}$}%
  \rput(A04){${\begin{array}{c}
                 \emptyset,y,\\
                 wxz,\setX
               \end{array}}$}%
  \rput(A03){${\begin{array}{c}
                 \emptyset,x,\\
                 wyz,\setX
               \end{array}}$}%
  \rput(A02){${\begin{array}{c}
                 \emptyset,w,\\
                 xyz,\setX
               \end{array}}$}%
  \rput(A01){$\emptyset,\setX$}%
  %\rput(A15){\begin{pspicture}(-1,0)(1,3)
  %                         \Cnode(0,3){t}
  %    \Cnode(-1,2){xy} \Cnode(0,2){xz} \Cnode(1,2){yz}
  %    \Cnode(-1,1){x}  \Cnode(0,1){y}  \Cnode(1,1){z}
  %                         \Cnode(0,  0){b}
  %    \psset{linestyle=dotted}%
  %    \ncline{t}{xy}\ncline{t}{xz}\ncline{t}{yz}
  %    \ncline{x}{xy}\ncline{x}{xz}
  %    \ncline{y}{xy}\ncline{y}{yz}
  %    \ncline{z}{xz}\ncline{z}{yz}
  %    \ncline{b}{x} \ncline{b}{y} \ncline{b}{z}
  %  \end{pspicture}}%
  \end{tabstr}%
\end{pspicture}%}}
\caption{%
  lattice of \structe{algebras of sets} on $\setn{w,x,y,z}$ \xref{ex:set_asets}
  \label{fig:algsets_wxyz}
  }
\end{center}
\end{figure}

%=======================================
\subsection{Lattices of rings of sets}
%=======================================
\begin{figure}
  \centering%
  \includegraphics{../common/math/graphics/pdfs/latlatringxyz.pdf}%
  \caption{%
    Lattice of rings of sets on $\sid\eqd\setn{x,y,z}$ \xref{ex:set_lat_ring_xyz}%
    \label{fig:set_lat_ring_xyz}%
    }%
\end{figure}
%---------------------------------------
\begin{example}
\label{ex:set_lat_ring_xyz}
%---------------------------------------
There are a total of \textbf{15} rings of sets on the set $\sid\eqd\setn{x,y,z}$.
These rings of sets are listed in \prefpp{ex:set_rings}
and illustrated in \prefpp{fig:set_lat_ring_xyz}.
The five rings containing $\sid$ ($\ssR_{11}$--$\ssR_{15}$)
are also \hie{algebras of sets} \xref{prop:ss_rxrra},
and thus also \hie{Boolean algebras} \xref{thm:lat_algebra}.
The five algebras of sets are shaded \pref{fig:set_lat_ring_xyz}.
\end{example}


%=======================================
\subsection{Lattices of partitions of sets}
%=======================================
\begin{figure}
  \begin{center}
  %\psset{unit=17mm}%
  \begin{tabular}{|ccc|}
    \hline
    \mc{1}{|G|}{partitions on $\setn{x,y,z}$}&%
    \mc{1}{|G|}{rings on $\setn{x,y}$}&%
    \mc{1}{|G|}{algebra of sets on $\setn{x,y,z}$}
    \\
    $\opair{\sssP{\setn{x,y,z}}}{\porel}$ &
    $\opair{\sssR{\setn{x,y}  }}{\sorel}$ &
    $\opair{\sssA{\setn{x,y,z}}}{\sorel}$
    \\
    \includegraphics{../common/math/graphics/pdfs/latlatpartitionsxyz.pdf}&%
    \includegraphics{../common/math/graphics/pdfs/latlatringxy.pdf}&%
    \includegraphics{../common/math/graphics/pdfs/latlatalgxyz.pdf}%
    \\\hline
  \end{tabular}
  \end{center}
  \caption{\label{fig:set_partition_xyz}\label{fig:set_lat_ring_xy}\label{fig:set_lat_alg_xyz}
    Lattices of set structures
    (see \prefpp{ex:set_lat_part_xyz}, \prefpp{ex:set_rings}, and \prefpp{ex:set_asets})
    }
\end{figure}


%---------------------------------------
\begin{example}
\label{ex:set_lat_part_xyz}
%---------------------------------------
There are a total of \textbf{5} partitions of sets on the set $\sid\eqd\setn{x,y,z}$.
These sets are listed in \prefpp{ex:set_partitions}
and illustrated in \prefpp{fig:set_partition_xyz}.
\end{example}

\begin{figure}
  \centering
  \includegraphics{../common/math/graphics/pdfs/latlatpartwxyz.pdf}
  \caption{
    Lattice of partitions of sets on $\sid\eqd\setn{w,x,y,z}$ \xref{ex:set_lat_part_wxyz}
    \label{fig:set_lat_part_wxyz}
    }
\end{figure}
%---------------------------------------
\begin{example}
\label{ex:set_lat_part_wxyz}
%---------------------------------------
There are a total of \textbf{15} partitions of sets on the set $\sid\eqd\setn{w,x,y,z}$.
These sets are listed in \prefpp{ex:set_partitions}
and illustrated in \prefpp{fig:set_lat_part_wxyz}.
\end{example}


In 1946, Philip Whitman proposed an amazing conjecture---that all finite lattices are isomorphic to a lattice of partitions.
A proof for this was published some 30 years later by
Pavel Pudl{/'a}k and Ji\v{r}\'i T\r{u}ma (next theorem).
%---------------------------------------
\begin{theorem}
\footnote{
  \citerc{pudlak1980}{improved proof},
  \citorc{pudlak1977}{proof},
  \citorc{whitman1946}{conjecture},
  \citerpgc{salii1988}{vii}{0821845225}{list of lattice theory breakthroughs}
  }
\label{thm:latpart}
%---------------------------------------
Let $\latL$ be a lattice.
\thmbox{
  \text{$\latL$ is \prope{finite}}
  \qquad\implies\qquad
  \text{$\latL$ is isomorphic to a \structe{lattice of partitions}}
  }
\end{theorem}

%---------------------------------------
\begin{example}
\label{ex:ss_lat5part}
%---------------------------------------
There are five unlabeled lattices on a five element 
set\ifdochas{lattice}{ as stated in \prefpp{prop:num_lattices} and illustrated in \prefpp{ex:lat_set5}}.
All of these lattices are isomorphic to a lattice of partitions \xref{thm:latpart},
as illustrated next.
\\%============================================================================
% Daniel J. Greenhoe
% LaTeX file
% 5 element sets as lattices of partitions
%============================================================================
%---------------------------------
% settings
%---------------------------------
\begin{tabular}{|c|c|}
%\begin{tabular*}{\tw}{|c@{\extracolsep\fill}c|ccc|}
%\begin{tabular*}{\tw}{|c@{\tfill}cccc|}
\hline
\mc{2}{|G|}{\normalsize lattices on 5 element set as lattices of partitions}
\\\hline
\begin{pspicture}(-3.1,-0.5)(3.1,2.5)%
  %---------------------------------
  % developement tools
  %---------------------------------
  %\psgrid[unit=100\psunit](-5,-1)(5,5)%
  %---------------------------------
  % nodes
  %---------------------------------
  \Cnode(   0,2){T77}%
  \Cnode( 1,1){T14}%
  \Cnode(   0,1){T22}%
  \Cnode(-1,1){T41}%
  \Cnode(   0,0)  {T00}%
  %---------------------------------
  % node connections
  %---------------------------------
  \ncline{T77}{T14}%
  \ncline{T77}{T22}%
  \ncline{T77}{T41}%
  \ncline{T00}{T14}%
  \ncline{T00}{T22}%
  \ncline{T00}{T41}%
  %---------------------------------
  % node labels
  %---------------------------------
  \uput[ 90](T77){$\setn{\setn{x},\,\setn{y},\,\setn{z}}$}%
  \uput[  0](T14){$\setn{\setn{z},\,\setn{x,y}}$}%         
  \uput[-67](T22){$\setn{\setn{y},\,\setn{x,z}}$}%         
  \uput[180](T41){$\setn{\setn{x},\,\setn{y,z}}$}%         
  \uput[-90](T00){$\setn{\setn{x,y,z}}$}%                  
\end{pspicture}%
%\setlength{\unitlength}{\tw/(4*600)}
%\begin{picture}(900,400)(-400,-100)%
%  %\graphpaper[10](0,0)(600,200)%
%  \thicklines%
%  \color{latline}%
%    \put( 100, 100){\line(-1, 1){100} }%
%    \put(-100, 100){\line( 1, 1){100} }%
%    \put(   0,   0){\line(-1, 1){100} }%
%    \put(   0,   0){\line( 0, 1){200} }%
%    \put(   0,   0){\line( 1, 1){100} }%
%  \color{latdot}%
%    \put(   0, 200){\latdot}%
%    \put( 100, 100){\latdot}%
%    \put(   0, 100){\latdot}%
%    \put(-100, 100){\latdot}%
%    \put(   0,   0){\latdot}%
%  \color{latlabel}%
%    \put(   0, 230){\makebox(0,0)[b]{$\setn{\setn{x},\,\setn{y},\,\setn{z}}$}}%
%    \put( 130, 100){\makebox(0,0)[l]{$\setn{\setn{x},\,\setn{y,z}}$}}%
%    \put(  15,  70){\makebox(0,0)[lt]{$\setn{\setn{y},\,\setn{x,z}}$}}%
%    \put(-130, 100){\makebox(0,0)[r]{$\setn{\setn{z},\,\setn{x,y}}$}}%
%    \put(   0, -30){\makebox(0,0)[t]{$\setn{\setn{x,y,z}}$}}%
%\end{picture}%
&
\psset{yunit=0.667\psunit}%
\begin{pspicture}(-4,-.50)(4,3.5)%
  %---------------------------------
  % settings
  %---------------------------------
  %---------------------------------
  % developement tools
  %---------------------------------
  %\psgrid[unit=100\psunit](-5,-1)(5,5)%
  %---------------------------------
  % nodes
  %---------------------------------
  \Cnode( 0,3){t}%
  \Cnode( 1,1.5){x}%
  \Cnode(-1,2){d}%
  \Cnode(-1,1){c}%
  \Cnode( 0,0){b}%
  %---------------------------------
  % node connections
  %---------------------------------
  \ncline{x}{t}%
  \ncline{d}{t}%
  \ncline{c}{d}%
  \ncline{b}{x}%
  \ncline{b}{c}%
  %---------------------------------
  % node labels
  %---------------------------------
  \uput[90](t){$\setn{\setn{w},\,\setn{x},\,\setn{y},\,\setn{z}}$}%
  \uput[-45](x){$\setn{\setn{w},\,\setn{x},\,\setn{y,z}}$}%
  \uput[180](d){$\setn{\setn{w},\,\setn{x},\,\setn{y,z}}$}%
  \uput[180](c){$\setn{\setn{w,x},\,\setn{y,z}}$}%
  \uput[-90](b){$\setn{\setn{w,x,y,z}}$}%
\end{pspicture}%
%\setlength{\unitlength}{\tw/(5*600)}
%\begin{picture}(1100,500)(-550,-100)%
%  %\graphpaper[10](0,0)(600,200)%
%  \thicklines%
%  \color{latline}%
%    \put( 100, 150){\line(-2,-3){100} }%
%    \put(   0, 300){\line( 2,-3){100} }%
%    \put(-100, 200){\line( 1, 1){100} }%
%    \put(-100, 100){\line( 0, 1){100} }%
%    \put(   0,   0){\line(-1, 1){100} }%
%  \color{latdot}%
%    \put( 100, 150){\latdot}%
%    \put(   0, 300){\latdot}%
%    \put(-100, 200){\latdot}%
%    \put(-100, 100){\latdot}%
%    \put(   0,   0){\latdot}%
%  \color{latlabel}%
%    \put(   0, 330){\makebox(0,0)[b]{$\setn{\setn{w},\,\setn{x},\,\setn{y},\,\setn{z}}$}}%
%    \put( 130, 150){\makebox(0,0)[l]{$\setn{\setn{w},\,\setn{x},\,\setn{y,z}}$}}%
%    \put(-130, 200){\makebox(0,0)[r]{$\setn{\setn{w},\,\setn{x},\,\setn{y,z}}$}}%
%    \put(-130, 100){\makebox(0,0)[r]{$\setn{\setn{w,x},\,\setn{y,z}}$}}%
%    \put(   0, -30){\makebox(0,0)[t]{$\setn{\setn{w,x,y,z}}$}}%
%\end{picture}%
\\\hline
\psset{unit=0.667\psunit}%
\begin{pspicture}(-3.50,-.50)(3.5,3.5)%
  %---------------------------------
  % settings
  %---------------------------------
  %---------------------------------
  % developement tools
  %---------------------------------
  %\psgrid[unit=100\psunit](-5,-1)(5,5)%
  %---------------------------------
  % nodes
  %---------------------------------
  \Cnode( 0,3){t}%
  \Cnode( 0,2){xy}%
  \Cnode(-1,1){x}\Cnode(1,1){y}%
  \Cnode( 0,0){b}%
  %---------------------------------
  % node connections
  %---------------------------------
  \ncline{xy}{t}%
  \ncline{y}{xy}%
  \ncline{x}{xy}%
  \ncline{b}{y}%
  \ncline{b}{x}%
  %---------------------------------
  % node labels
  %---------------------------------
  \uput[0](t)  {$\setn{\setn{w},\,\setn{x},\,\setn{y},\,\setn{z}}$}%
  \uput[  0](xy){$\setn{\setn{w},\,\setn{x,y},\,\setn{z}}$}%         
  \uput[  0](y) {$\setn{\setn{w,x,y},\,\setn{z}}$}%                  
  \uput[180](x) {$\setn{\setn{w},\,\setn{x,y,z}}$}%                  
  \uput[0](b) {$\setn{\setn{w,x,y,z}}$}%                           
\end{pspicture}%
%\setlength{\unitlength}{\tw/(5*600)}
%\begin{picture}(750,500)(-350,-100)%
%  %\graphpaper[10](0,0)(600,200)%
%  \thicklines%
%  \color{latline}%
%    \put(   0, 200){\line( 0, 1){100} }%
%    \put( 100, 100){\line(-1, 1){100} }%
%    \put(-100, 100){\line( 1, 1){100} }%
%    \put(   0,   0){\line(-1, 1){100} }%
%    \put(   0,   0){\line( 1, 1){100} }%
%  \color{latdot}%
%    \put(   0, 300){\latdot}%
%    \put(   0, 200){\latdot}%
%    \put( 100, 100){\latdot}%
%    \put(-100, 100){\latdot}%
%    \put(   0,   0){\latdot}%
%  \color{latlabel}%
%    \put(   0, 330){\makebox(0,0)[b]{$\setn{\setn{w},\,\setn{x},\,\setn{y},\,\setn{z}}$}}%
%    \put(  30, 200){\makebox(0,0)[l]{$\setn{\setn{w},\,\setn{x,y},\,\setn{z}}$}}%
%    \put( 130, 100){\makebox(0,0)[l]{$\setn{\setn{w,x,y},\,\setn{z}}$}}%
%    \put(-130, 100){\makebox(0,0)[r]{$\setn{\setn{w},\,\setn{x,y,z}}$}}%
%    \put(   0, -30){\makebox(0,0)[t]{$\setn{\setn{w,x,y,z}}$}}%
%\end{picture}%
&
\psset{unit=0.667\psunit}%
\begin{pspicture}(-3.50,-.50)(3.5,3.5)%
  %---------------------------------
  % settings
  %---------------------------------
  %---------------------------------
  % developement tools
  %---------------------------------
  %\psgrid[unit=100\psunit](-5,-1)(5,5)%
  %---------------------------------
  % nodes
  %---------------------------------
  \Cnode( 0,3){t}%
  \Cnode(-1,2){x}\Cnode(1,2){y}%
  \Cnode( 0,1){c}%
  \Cnode( 0,0){b}%
  %---------------------------------
  % node connections
  %---------------------------------
  \ncline{y}{t}%
  \ncline{x}{t}%
  \ncline{c}{y}%
  \ncline{c}{x}%
  \ncline{b}{c}%
  %---------------------------------
  % node labels
  %---------------------------------
  \uput[ 0](t)  {$\setn{\setn{w},\,\setn{x},\,\setn{y},\,\setn{z}}$}%
  \uput[  0](y) {$\setn{\setn{w},\,\setn{x},\,\setn{y,z}}$}%
  \uput[180](x) {$\setn{\setn{w,x},\,\setn{y,z}}$}%
  \uput[0](b) {$\setn{\setn{w,x,y,z}}$}%
\end{pspicture}%
%\setlength{\unitlength}{\tw/(5*600)}
%\begin{picture}(1000,500)(-500,-100)%
%  %\graphpaper[10](0,0)(600,200)%
%  \thicklines%
%  \color{latline}%
%    \put( 100, 200){\line(-1, 1){100} }%
%    \put(-100, 200){\line( 1, 1){100} }%
%    \put(   0, 100){\line(-1, 1){100} }%
%    \put(   0, 100){\line( 1, 1){100} }%
%    \put(   0,   0){\line( 0, 1){100} }%
%  \color{latdot}%
%    \put(   0, 300){\latdot}%
%    \put( 100, 200){\latdot}%
%    \put(-100, 200){\latdot}%
%    \put(   0, 100){\latdot}%
%    \put(   0,   0){\latdot}%
%  \color{latlabel}%
%    \put(   0, 330){\makebox(0,0)[b]{$\setn{\setn{w},\,\setn{x},\,\setn{y},\,\setn{z}}$}}%
%    \put( 130, 200){\makebox(0,0)[l]{$\setn{\setn{w,x},\,\setn{y},\,\setn{z}}$}}%
%    \put(-130, 200){\makebox(0,0)[r]{$\setn{\setn{w},\,\setn{x},\,\setn{y,z}}$}}%
%    \put(  30, 100){\makebox(0,0)[l]{$\setn{\setn{w,x},\,\setn{y,z}}$}}%
%    \put(   0, -30){\makebox(0,0)[t]{$\setn{\setn{w,x,y,z}}$}}%
%\end{picture}%
\\\hline
\psset{yunit=0.5\psunit}%
\begin{pspicture}(-1.50,-.50)(1.5,4.5)%
  %---------------------------------
  % settings
  %---------------------------------
  \fns%
  %---------------------------------
  % developement tools
  %---------------------------------
  %\psgrid[unit=100\psunit](-5,-1)(5,5)%
  %---------------------------------
  % nodes
  %---------------------------------
  \Cnode( 0,4){t}%
  \Cnode( 0,3){e}
  \Cnode( 0,2){d}%
  \Cnode( 0,1){c}%
  \Cnode( 0,0){b}%
  %---------------------------------
  % node connections
  %---------------------------------
  \ncline{e}{t}%
  \ncline{d}{e}%
  \ncline{c}{d}%
  \ncline{b}{c}%
  %---------------------------------
  % node labels
  %---------------------------------
  \uput[0](t) {$\setn{\setn{w,x,y,z}}$}%
  \uput[0](e) {$\setn{\setn{x,y,z}}$}%  
  \uput[0](d) {$\setn{\setn{x,y}}$}%    
  \uput[0](c) {$\setn{\setn{x}}$}%      
  \uput[0](b) {$\setn{\szero}$}%        
\end{pspicture}%
%\setlength{\unitlength}{\tw/(6*600)}
%\begin{picture}(500,600)(-50,-100)%
%  %\graphpaper[10](0,0)(600,200)%
%  \thicklines%
%  \color{latline}%
%    \put(   0,   0){\line( 0, 1){400} }%
%  \color{latdot}%
%    \put(   0, 400){\latdot}%
%    \put(   0, 300){\latdot}%
%    \put(   0, 200){\latdot}%
%    \put(   0, 100){\latdot}%
%    \put(   0,   0){\latdot}%
%  \color{latlabel}%
%    \put(  40, 400){\makebox(0,0)[l]{$\setn{\setn{w,x,y,z}}$}}%
%    \put(  40, 300){\makebox(0,0)[l]{$\setn{\setn{x,y,z}}$}}%
%    \put(  40, 200){\makebox(0,0)[l]{$\setn{\setn{x,y}}$}}%
%    \put(  40, 100){\makebox(0,0)[l]{$\setn{\setn{x}}$}}%
%    \put(  40,   0){\makebox(0,0)[l]{$\setn{\szero}$}}%
%\end{picture}%
&%
\\\hline
\end{tabular}




\end{example}

} % end mssa exclude

%\ifexclude{mssa}
{
%=======================================
\section{Relationships between set structures}
%=======================================


%---------------------------------------
\begin{proposition}
\citetbl{
  \citerpg{bsu1996}{4}{3764353449},
  \citerpg{halmos1950}{21}{0387900888}
  }
\label{prop:set_ring_alg}
%---------------------------------------
\propbox{
  \brb{\begin{array}{M}
    $\ssR$ is a \structb{ring of sets}\\ 
    on a set $\sid$
  \end{array}}
  \qquad\implies\qquad
  \brb{\begin{array}{M}
    $\ssR \setu \sid$ is an \structb{algebra of sets}\\ 
    on $\sid$
  \end{array}}
  }
\end{proposition}




%---------------------------------------
\begin{theorem}
\label{thm:set_atr}
%---------------------------------------
Let $\sid$ be a set.
\thmbox{
  \brb{\begin{array}{M}
    $\algA$ is an \structb{algebra of sets}\\ 
    on $\sid$
  \end{array}}
  \qquad\iff\qquad
  \brb{\begin{array}{FMD}
    1. & $\algA$ is a \structb{topology} on $\sid$     & and \\
    2. & $\algA$ is a \structb{ring of sets} on $\sid$
  \end{array}}
  }
\end{theorem}
\begin{proof}
\begin{align*}
  \text{$\algA$ is an algebra of sets on $\sid$}
    &\implies \text{$\algA$ is closed under $\setu$, $\seti$, $\setopc$, $\setd$, $\szero$, $\sid$}
    &&        \text{by \prefp{thm:set_closed_op}}
  \\&\implies
       \left\{\begin{array}{ll}
         1. & \text{$\algA$ is a topology on $\sid$} \\
            & \text{\scshape and} \\
         2. & \text{$\algA$ is a ring of sets on $\sid$}
       \end{array}\right\}
  \\
  \\
  \left\{\begin{array}{ll}
    1. & \text{$\algA$ is a topology on $\sid$} \\
       & \text{\scshape and} \\
    2. & \text{$\algA$ is a ring of sets on $\sid$}
  \end{array}\right\}
    &\implies \text{$\algA$ is closed under $\setopc$ and $\seti$}
    &&        \text{by \prefp{thm:set_closed_op}}
  \\&\implies \text{$\algA$ is a ring of sets}
\end{align*}
\end{proof}


%---------------------------------------
\begin{corollary}
\label{cor:set_atr}
%---------------------------------------
Let $\sid$ be a set and $\psetx$ the power set of $\sid$.
\corbox{\begin{array}{l}
  \set{\algA\sorel\psetx}{\text{$\algA$ is an algebra of sets on $\sid$}}
  \\\qquad=
  \set{\topT\sorel\psetx}{\text{$\topT$ is a topology on $\sid$}}
  \seti
  \set{\ssR\sorel\psetx}{\text{$\ssR$ is a ring of sets on $\sid$}}
  \end{array}}
\end{corollary}
\begin{proof}
\begin{align*}
  &\set{\topT}{\text{$\topT$ is a topology}} \seti \set{\ssR}{\text{$\ssR$ is a ring of sets}}
  \\&= \set{\ssetY}{\text{$\ssetY$ is a topology {\scshape and} a ring of sets}}
    && \text{by \prefp{def:ss_setops}}
  \\&= \set{\ssetY}{\text{$\ssetY$ is an algebra of sets}}
    && \text{by \prefp{thm:set_atr}}
  \\&= \set{\algA}{\text{$\algA$ is an algebra of sets}}
    && \text{by change of variable}
\end{align*}
\end{proof}


%---------------------------------------
\begin{example}
%---------------------------------------
Note that the {\em intersection} of
the lattice of topologies on $\setn{x,y,z}$ \xref{fig:set_latlat_top_xyz}
{\em and}
the lattice of rings of sets on $\setn{x,y,z}$ \xref{fig:set_lat_ring_xyz}
is {\em equal to}
the lattice of algebras of sets on $\setn{x,y,z}$ \xref{fig:set_lat_alg_xyz}.
\end{example}

%---------------------------------------
\begin{proposition}
\label{prop:ss_rxrra}
%---------------------------------------
Let $\sssRx$ be the set of \structe{rings of sets} \xref{def:ss_ring} and
    $\sssAx$ the set of \structe{algebras of sets} \xref{def:ss_algebra} on a set $\sid$.
\propbox{
  \brb{\begin{array}{FMD}
    1. & $\ssR$ is a \structb{ring of sets} & and \\
    2. & $\sid\in\ssR$
  \end{array}}
  \qquad\iff\qquad
  \brb{\begin{array}{M}
    $\ssR$ is an \structb{algebra of sets}
  \end{array}}
  }
\end{proposition}
\begin{proof}
\begin{align*}
  \cmpA
    &= \sid\setd\setA
    && \text{by \prefp{thm:ss_rel_gg}}
    \\
  \setA\seti\setB
    &= \setA\setd\brp{\setA\setd\setB}
    && \text{by \prefp{thm:ss_rel_gg}}
\end{align*}
Therefore, $\brp{\ssR\setu\sid}$ is closed under $\setopc$ and $\seti$, and
thus by the definition of algebras of sets \xref{def:ss_algebra},
$\brp{\ssR\setu\sid}$ is an algebra of sets.
\end{proof}





%---------------------------------------
\begin{definition}
\citetbl{
  \citerpu{ab}{97}{0120502577}
  }
\index{Borel set}
%---------------------------------------
\defbox{\begin{tabular}{l}
  The \hid{Borel set} $\opB(\msx,\topT)$ generated by the topological space $\opair{\msx}{\topT}$ \\
  is the $\sigma$-algebra generated by the topology $\topT$.
\end{tabular}}
\end{definition}

%---------------------------------------
\begin{example}
\index{Borel set}
%---------------------------------------
Suppose we have a dice with the standard six possible outcomes $\msx$.
Suppose also we construct the following
topology $\topT$ on $\msx$, and this in turn generates the following
Borel set ($\sigma$-algebra) $B$ on $\msx$:
\exbox{\small\begin{array}{ll}
  \msx    &= \left\{ \text{\diceA,\diceB,\diceC,\diceD,\diceE,\diceF} \right\}
  \\
  \topT &= \left\{ \mcom{\setn{\quad}}{$\szero$},\;
                  \mcom{\setn{\text{\diceA,\diceB,\diceC,\diceD,\diceE,\diceF}}}{$\pso$},\;
                  \mcom{\setn{\text{\diceA,\diceB,\diceC,\diceD}}}{first four},\;
                  \mcom{\setn{\text{\diceD,\diceE,\diceF}}}{last three},\;
                  \mcom{\setn{\text{\diceD}}}{$\setn{1234}\cap\setn{456}$},\;
          \right\}
  \\
  B    &= \left\{ \mcom{\setn{\quad}}{$\szero$},\;
                  \mcom{\setn{\text{\diceA,\diceB,\diceC,\diceD,\diceE,\diceF}}}{$\pso$},\;
                  \mcom{\setn{\text{\diceA,\diceB,\diceC,\diceD}}}{first four},\;
                  \mcom{\setn{\text{\diceD,\diceE,\diceF}}}{last three},\;
                  \mcom{\setn{\text{\diceD}}}{$\setn{1234}\cap\setn{456}$},\;
                  \right.
                  \\&\qquad
                  \left.
                  \mcom{\setn{\text{\diceA,\diceB,\diceC,\diceE,\diceF}}}{$\setn{4}$},\;
                  \mcom{\setn{\text{\diceE,\diceF}}}{$(\setn{4})\cap\setn{456}$},\;
                  \mcom{\setn{\text{\diceA,\diceB,\diceC}}}{$\setn{1234}\cap\setn{4}$},\;
          \right\}
\end{array}}
\end{example}




\begin{figure}
\begin{center}
\footnotesize
\setlength{\unitlength}{\textwidth/2000}%
\begin{picture}(1500,1200)(-600,-100)%
  \thicklines
  %{\color[gray]{0.5}\graphpaper[100](-800,-300)(1600,2200)}%
  %{\color{graphpaper}\graphpaper[100](-600,-100)(1500,1200)}%
  \color{red}%
    \put(-600,500){\dashbox{10}(1200,600){}}%
    \put(-550,200){\dashbox{10}(300,200){}}%
    \put(-150,200){\dashbox{10}(300,200){}}%
    \put( 250,200){\dashbox{10}(300,200){}}%
    \put( 600,800){\vector(1,0){140}}%
    \put(-400,200){\vector(0,-1){120}}%
    \put(   0,200){\vector(0,-1){120}}%
    \put( 400,200){\vector(0,-1){120}}%
%
%
% Algebras that generate the power set(xyz)
% power set(xyz)
  \put(800,800){%
    \setlength{\unitlength}{1\tw/(3*1500)}%
    \begin{picture}(0,0)(0,150)%
    %{\color{graphpaper}\graphpaper[50](-100,0)(200,300)}%
    \thicklines%
    \color{red}%
      \put(   0, 300){\line(-1,-1){100} }%
      \put(   0, 300){\line( 0,-1){100} }%
      \put(   0, 300){\line( 1,-1){100} }%
      \put( 100, 100){\line( 0, 1){100} }%
      \put( 100, 100){\line(-1, 1){100} }%
      \put(   0, 100){\line(-1, 1){100} }%
      \put(   0, 100){\line( 1, 1){100} }%
      \put(-100, 100){\line( 0, 1){100} }%
      \put(-100, 100){\line( 1, 1){100} }%
      \put(   0,   0){\line(-1, 1){100} }%
      \put(   0,   0){\line( 0, 1){100} }%
      \put(   0,   0){\line( 1, 1){100} }%
    \color{latdot}%
      \put(   0, 300){\latdot}%
      \put( 100, 200){\latdot}%
      \put(   0, 200){\latdot}%
      \put(-100, 200){\latdot}%
      \put( 100, 100){\latdot}%
      \put(   0, 100){\latdot}%
      \put(-100, 100){\latdot}%
      \put(   0,   0){\latdot}%
    \end{picture}%
  }%
%
%
% Algebras that generate the power set(xyz)
% population 6
  \put(-500,1000){%  {X, xz, xy, z, 0}%
    \setlength{\unitlength}{1\tw/(3*1500)}%
    \begin{picture}(0,0)(0,150)%
    %{\color{graphpaper}\graphpaper[50](-100,0)(200,300)}%
    \thicklines%
    \color{latline}%
      \put(   0, 300){\line(-1,-1){100} }%
      \put(   0, 300){\line( 0,-1){100} }%
      \put(   0, 300){\line( 1,-1){100} }%
      \put( 100, 100){\line( 0, 1){100} }%
      \put( 100, 100){\line(-1, 1){100} }%
      \put(   0, 100){\line(-1, 1){100} }%
      \put(   0, 100){\line( 1, 1){100} }%
      \put(-100, 100){\line( 0, 1){100} }%
      \put(-100, 100){\line( 1, 1){100} }%
      \put(   0,   0){\line(-1, 1){100} }%
      \put(   0,   0){\line( 0, 1){100} }%
      \put(   0,   0){\line( 1, 1){100} }%
    \color{latdot}%
      \put(   0, 300){\latdot}%
      \put(   0, 200){\latdot}%
      \put(-100, 200){\latdot}%
      \put(   0, 100){\latdot}%
      \put(-100, 100){\latdot}%
      \put(   0,   0){\latdot}%
    \end{picture}%
  }
  \put(-300,1000){%
    \setlength{\unitlength}{1\tw/(3*1500)}%
    \begin{picture}(0,0)(0,150)%
    %{\color{graphpaper}\graphpaper[50](-100,0)(200,300)}%
    \thicklines%
    \color{latline}%
      \put(   0, 300){\line(-1,-1){100} }%
      \put(   0, 300){\line( 0,-1){100} }%
      \put(   0, 300){\line( 1,-1){100} }%
      \put( 100, 100){\line( 0, 1){100} }%
      \put( 100, 100){\line(-1, 1){100} }%
      \put(   0, 100){\line(-1, 1){100} }%
      \put(   0, 100){\line( 1, 1){100} }%
      \put(-100, 100){\line( 0, 1){100} }%
      \put(-100, 100){\line( 1, 1){100} }%
      \put(   0,   0){\line(-1, 1){100} }%
      \put(   0,   0){\line( 0, 1){100} }%
      \put(   0,   0){\line( 1, 1){100} }%
    \color{latdot}%
      \put(   0, 300){\latdot}%
      \put( 100, 200){\latdot}%
      \put(-100, 200){\latdot}%
      \put(   0, 100){\latdot}%
      \put(-100, 100){\latdot}%
      \put(   0,   0){\latdot}%
    \end{picture}%
  }
% Level 4 Group 2 of 3
  \put(-100,1000){%  {X, xz, xy, z, 0}%
    \setlength{\unitlength}{1\tw/(3*1500)}%
    \begin{picture}(0,0)(0,150)%
    %{\color{graphpaper}\graphpaper[50](-100,0)(200,300)}%
    \thicklines%
    \color{latline}%
      \put(   0, 300){\line(-1,-1){100} }%
      \put(   0, 300){\line( 0,-1){100} }%
      \put(   0, 300){\line( 1,-1){100} }%
      \put( 100, 100){\line( 0, 1){100} }%
      \put( 100, 100){\line(-1, 1){100} }%
      \put(   0, 100){\line(-1, 1){100} }%
      \put(   0, 100){\line( 1, 1){100} }%
      \put(-100, 100){\line( 0, 1){100} }%
      \put(-100, 100){\line( 1, 1){100} }%
      \put(   0,   0){\line(-1, 1){100} }%
      \put(   0,   0){\line( 0, 1){100} }%
      \put(   0,   0){\line( 1, 1){100} }%
    \color{latdot}%
      \put(   0, 300){\latdot}%
      \put(   0, 200){\latdot}%
      \put(-100, 200){\latdot}%
      \put( 100, 100){\latdot}%
      \put(-100, 100){\latdot}%
      \put(   0,   0){\latdot}%
    \end{picture}%
  }
  \put(100,1000){%
    \setlength{\unitlength}{1\tw/(3*1500)}%
    \begin{picture}(0,0)(0,150)%
    %{\color{graphpaper}\graphpaper[50](-100,0)(200,300)}%
    \thicklines%
    \color{latline}%
      \put(   0, 300){\line(-1,-1){100} }%
      \put(   0, 300){\line( 0,-1){100} }%
      \put(   0, 300){\line( 1,-1){100} }%
      \put( 100, 100){\line( 0, 1){100} }%
      \put( 100, 100){\line(-1, 1){100} }%
      \put(   0, 100){\line(-1, 1){100} }%
      \put(   0, 100){\line( 1, 1){100} }%
      \put(-100, 100){\line( 0, 1){100} }%
      \put(-100, 100){\line( 1, 1){100} }%
      \put(   0,   0){\line(-1, 1){100} }%
      \put(   0,   0){\line( 0, 1){100} }%
      \put(   0,   0){\line( 1, 1){100} }%
    \color{latdot}%
      \put(   0, 300){\latdot}%
      \put( 100, 200){\latdot}%
      \put(   0, 200){\latdot}%
      \put( 100, 100){\latdot}%
      \put(-100, 100){\latdot}%
      \put(   0,   0){\latdot}%
    \end{picture}%
  }%
  % Level 4 Group 3 of 3
  \put(300,1000){%
    \setlength{\unitlength}{1\tw/(3*1500)}%
    \begin{picture}(0,0)(0,150)%
    %{\color{graphpaper}\graphpaper[50](-100,0)(200,300)}%
    \thicklines%
    \color{latline}%
      \put(   0, 300){\line(-1,-1){100} }%
      \put(   0, 300){\line( 0,-1){100} }%
      \put(   0, 300){\line( 1,-1){100} }%
      \put( 100, 100){\line( 0, 1){100} }%
      \put( 100, 100){\line(-1, 1){100} }%
      \put(   0, 100){\line(-1, 1){100} }%
      \put(   0, 100){\line( 1, 1){100} }%
      \put(-100, 100){\line( 0, 1){100} }%
      \put(-100, 100){\line( 1, 1){100} }%
      \put(   0,   0){\line(-1, 1){100} }%
      \put(   0,   0){\line( 0, 1){100} }%
      \put(   0,   0){\line( 1, 1){100} }%
    \color{latdot}%
      \put(   0, 300){\latdot}%
      \put( 100, 200){\latdot}%
      \put(-100, 200){\latdot}%
      \put( 100, 100){\latdot}%
      \put(   0, 100){\latdot}%
      \put(   0,   0){\latdot}%
    \end{picture}%
  }
  \put(500,1000){%
    \setlength{\unitlength}{1\tw/(3*1500)}%
    \begin{picture}(0,0)(0,150)%
    %{\color{graphpaper}\graphpaper[50](-100,0)(200,300)}%
    \thicklines%
    \color{latline}%
      \put(   0, 300){\line(-1,-1){100} }%
      \put(   0, 300){\line( 0,-1){100} }%
      \put(   0, 300){\line( 1,-1){100} }%
      \put( 100, 100){\line( 0, 1){100} }%
      \put( 100, 100){\line(-1, 1){100} }%
      \put(   0, 100){\line(-1, 1){100} }%
      \put(   0, 100){\line( 1, 1){100} }%
      \put(-100, 100){\line( 0, 1){100} }%
      \put(-100, 100){\line( 1, 1){100} }%
      \put(   0,   0){\line(-1, 1){100} }%
      \put(   0,   0){\line( 0, 1){100} }%
      \put(   0,   0){\line( 1, 1){100} }%
    \color{latdot}%
      \put(   0, 300){\latdot}%
      \put( 100, 200){\latdot}%
      \put(   0, 200){\latdot}%
      \put( 100, 100){\latdot}%
      \put(   0, 100){\latdot}%
      \put(   0,   0){\latdot}%
    \end{picture}%
  }
%
%
% Algebras that generate the power set(xyz)
% population 5
  \put(-500,800){%
    \setlength{\unitlength}{1\tw/(3*1500)}%
    \begin{picture}(0,0)(0,150)%
    %{\color{graphpaper}\graphpaper[50](-100,0)(200,300)}%
    \thicklines%
    \color{latline}%
      \put(   0, 300){\line(-1,-1){100} }%
      \put(   0, 300){\line( 0,-1){100} }%
      \put(   0, 300){\line( 1,-1){100} }%
      \put( 100, 100){\line( 0, 1){100} }%
      \put( 100, 100){\line(-1, 1){100} }%
      \put(   0, 100){\line(-1, 1){100} }%
      \put(   0, 100){\line( 1, 1){100} }%
      \put(-100, 100){\line( 0, 1){100} }%
      \put(-100, 100){\line( 1, 1){100} }%
      \put(   0,   0){\line(-1, 1){100} }%
      \put(   0,   0){\line( 0, 1){100} }%
      \put(   0,   0){\line( 1, 1){100} }%
    \color{latdot}%
      \put(   0, 300){\latdot}%
      \put(   0, 200){\latdot}%
      \put(-100, 200){\latdot}%
      \put(-100, 100){\latdot}%
      \put(   0,   0){\latdot}%
    \end{picture}%
  }
  \put(-300,800){%
    \setlength{\unitlength}{1\tw/(3*1500)}%
    \begin{picture}(0,0)(0,150)%
    %{\color{graphpaper}\graphpaper[50](-100,0)(200,300)}%
    \thicklines%
    \color{latline}%
      \put(   0, 300){\line(-1,-1){100} }%
      \put(   0, 300){\line( 0,-1){100} }%
      \put(   0, 300){\line( 1,-1){100} }%
      \put( 100, 100){\line( 0, 1){100} }%
      \put( 100, 100){\line(-1, 1){100} }%
      \put(   0, 100){\line(-1, 1){100} }%
      \put(   0, 100){\line( 1, 1){100} }%
      \put(-100, 100){\line( 0, 1){100} }%
      \put(-100, 100){\line( 1, 1){100} }%
      \put(   0,   0){\line(-1, 1){100} }%
      \put(   0,   0){\line( 0, 1){100} }%
      \put(   0,   0){\line( 1, 1){100} }%
    \color{latdot}%
      \put(   0, 300){\latdot}%
      \put( 100, 200){\latdot}%
      \put(-100, 200){\latdot}%
      \put(   0, 100){\latdot}%
      \put(   0,   0){\latdot}%
    \end{picture}%
  }%
  \put(-100,800){%
    \setlength{\unitlength}{1\tw/(3*1500)}%
    \begin{picture}(0,0)(0,150)%
    %{\color{graphpaper}\graphpaper[50](-100,0)(200,300)}%
    \thicklines%
    \color{latline}%
      \put(   0, 300){\line(-1,-1){100} }%
      \put(   0, 300){\line( 0,-1){100} }%
      \put(   0, 300){\line( 1,-1){100} }%
      \put( 100, 100){\line( 0, 1){100} }%
      \put( 100, 100){\line(-1, 1){100} }%
      \put(   0, 100){\line(-1, 1){100} }%
      \put(   0, 100){\line( 1, 1){100} }%
      \put(-100, 100){\line( 0, 1){100} }%
      \put(-100, 100){\line( 1, 1){100} }%
      \put(   0,   0){\line(-1, 1){100} }%
      \put(   0,   0){\line( 0, 1){100} }%
      \put(   0,   0){\line( 1, 1){100} }%
    \color{latdot}%
      \put(   0, 300){\latdot}%
      \put(   0, 200){\latdot}%
      \put( 100, 200){\latdot}%
      \put( 100, 100){\latdot}%
      \put(   0,   0){\latdot}%
    \end{picture}%
  }%
% Level 3 group 2 of 2
  \put(100,800){%
    \setlength{\unitlength}{1\tw/(3*1500)}%
    \begin{picture}(0,0)(0,150)%
    %{\color{graphpaper}\graphpaper[50](-100,0)(200,300)}%
    \thicklines%
    \color{latline}%
      \put(   0, 300){\line(-1,-1){100} }%
      \put(   0, 300){\line( 0,-1){100} }%
      \put(   0, 300){\line( 1,-1){100} }%
      \put( 100, 100){\line( 0, 1){100} }%
      \put( 100, 100){\line(-1, 1){100} }%
      \put(   0, 100){\line(-1, 1){100} }%
      \put(   0, 100){\line( 1, 1){100} }%
      \put(-100, 100){\line( 0, 1){100} }%
      \put(-100, 100){\line( 1, 1){100} }%
      \put(   0,   0){\line(-1, 1){100} }%
      \put(   0,   0){\line( 0, 1){100} }%
      \put(   0,   0){\line( 1, 1){100} }%
    \color{latdot}%
      \put(   0, 300){\latdot}%
      \put(-100, 200){\latdot}%
      \put(   0, 100){\latdot}%
      \put(-100, 100){\latdot}%
      \put(   0,   0){\latdot}%
    \end{picture}%
  }
  \put(300,800){%
    \setlength{\unitlength}{1\tw/(3*1500)}%
    \begin{picture}(0,0)(0,150)%
    %{\color{graphpaper}\graphpaper[50](-100,0)(200,300)}%
    \thicklines%
    \color{latline}%
      \put(   0, 300){\line(-1,-1){100} }%
      \put(   0, 300){\line( 0,-1){100} }%
      \put(   0, 300){\line( 1,-1){100} }%
      \put( 100, 100){\line( 0, 1){100} }%
      \put( 100, 100){\line(-1, 1){100} }%
      \put(   0, 100){\line(-1, 1){100} }%
      \put(   0, 100){\line( 1, 1){100} }%
      \put(-100, 100){\line( 0, 1){100} }%
      \put(-100, 100){\line( 1, 1){100} }%
      \put(   0,   0){\line(-1, 1){100} }%
      \put(   0,   0){\line( 0, 1){100} }%
      \put(   0,   0){\line( 1, 1){100} }%
    \color{latdot}%
      \put(   0, 300){\latdot}%
      \put(   0, 200){\latdot}%
      \put( 100, 100){\latdot}%
      \put(-100, 100){\latdot}%
      \put(   0,   0){\latdot}%
    \end{picture}%
  }
  \put(500,800){%
    \setlength{\unitlength}{1\tw/(3*1500)}%
    \begin{picture}(0,0)(0,150)%
    %{\color{graphpaper}\graphpaper[50](-100,0)(200,300)}%
    \thicklines%
    \color{latline}%
      \put(   0, 300){\line(-1,-1){100} }%
      \put(   0, 300){\line( 0,-1){100} }%
      \put(   0, 300){\line( 1,-1){100} }%
      \put( 100, 100){\line( 0, 1){100} }%
      \put( 100, 100){\line(-1, 1){100} }%
      \put(   0, 100){\line(-1, 1){100} }%
      \put(   0, 100){\line( 1, 1){100} }%
      \put(-100, 100){\line( 0, 1){100} }%
      \put(-100, 100){\line( 1, 1){100} }%
      \put(   0,   0){\line(-1, 1){100} }%
      \put(   0,   0){\line( 0, 1){100} }%
      \put(   0,   0){\line( 1, 1){100} }%
    \color{latdot}%
      \put(   0, 300){\latdot}%
      \put(100, 200){\latdot}%
      \put(   0, 100){\latdot}%
      \put( 100, 100){\latdot}%
      \put(   0,   0){\latdot}%
    \end{picture}%
  }
%
%
% Algebras that generate the power set(xyz)
% population 4
  \put(-500,600){%  {X, xz, xy, z, 0}%
    \setlength{\unitlength}{1\tw/(3*1500)}%
    \begin{picture}(0,0)(0,150)%
    %{\color{graphpaper}\graphpaper[50](-100,0)(200,300)}%
    \thicklines%
    \color{latline}%
      \put(   0, 300){\line(-1,-1){100} }%
      \put(   0, 300){\line( 0,-1){100} }%
      \put(   0, 300){\line( 1,-1){100} }%
      \put( 100, 100){\line( 0, 1){100} }%
      \put( 100, 100){\line(-1, 1){100} }%
      \put(   0, 100){\line(-1, 1){100} }%
      \put(   0, 100){\line( 1, 1){100} }%
      \put(-100, 100){\line( 0, 1){100} }%
      \put(-100, 100){\line( 1, 1){100} }%
      \put(   0,   0){\line(-1, 1){100} }%
      \put(   0,   0){\line( 0, 1){100} }%
      \put(   0,   0){\line( 1, 1){100} }%
    \color{latdot}%
      \put(   0, 300){\latdot}%
      \put(-100, 200){\latdot}%
      \put(-100, 100){\latdot}%
      \put(   0,   0){\latdot}%
    \end{picture}%
  }%
  \put(-300,600){%  {X, xz, xy, z, 0}%
    \setlength{\unitlength}{1\tw/(3*1500)}%
    \begin{picture}(0,0)(0,150)%
    %{\color{graphpaper}\graphpaper[50](-100,0)(200,300)}%
    \thicklines%
    \color{latline}%
      \put(   0, 300){\line(-1,-1){100} }%
      \put(   0, 300){\line( 0,-1){100} }%
      \put(   0, 300){\line( 1,-1){100} }%
      \put( 100, 100){\line( 0, 1){100} }%
      \put( 100, 100){\line(-1, 1){100} }%
      \put(   0, 100){\line(-1, 1){100} }%
      \put(   0, 100){\line( 1, 1){100} }%
      \put(-100, 100){\line( 0, 1){100} }%
      \put(-100, 100){\line( 1, 1){100} }%
      \put(   0,   0){\line(-1, 1){100} }%
      \put(   0,   0){\line( 0, 1){100} }%
      \put(   0,   0){\line( 1, 1){100} }%
    \color{latdot}%
      \put(   0, 300){\latdot}%
      \put(   0, 200){\latdot}%
      \put(-100, 100){\latdot}%
      \put(   0,   0){\latdot}%
    \end{picture}%
  }%
  % Level 2 group 2 of 3
  \put(-100,600){%
    \setlength{\unitlength}{1\tw/(3*1500)}%
    \begin{picture}(0,0)(0,150)%
    %{\color{graphpaper}\graphpaper[50](-100,0)(200,300)}%
    \thicklines%
    \color{latline}%
      \put(   0, 300){\line(-1,-1){100} }%
      \put(   0, 300){\line( 0,-1){100} }%
      \put(   0, 300){\line( 1,-1){100} }%
      \put( 100, 100){\line( 0, 1){100} }%
      \put( 100, 100){\line(-1, 1){100} }%
      \put(   0, 100){\line(-1, 1){100} }%
      \put(   0, 100){\line( 1, 1){100} }%
      \put(-100, 100){\line( 0, 1){100} }%
      \put(-100, 100){\line( 1, 1){100} }%
      \put(   0,   0){\line(-1, 1){100} }%
      \put(   0,   0){\line( 0, 1){100} }%
      \put(   0,   0){\line( 1, 1){100} }%
    \color{latdot}%
      \put(   0, 300){\latdot}%
      \put(-100, 200){\latdot}%
      \put(   0, 100){\latdot}%
      \put(   0,   0){\latdot}%
    \end{picture}%
  }%
  \put(100,600){%
    \setlength{\unitlength}{1\tw/(3*1500)}%
    \begin{picture}(0,0)(0,150)%
    %{\color{graphpaper}\graphpaper[50](-100,0)(200,300)}%
    \thicklines%
    \color{latline}%
      \put(   0, 300){\line(-1,-1){100} }%
      \put(   0, 300){\line( 0,-1){100} }%
      \put(   0, 300){\line( 1,-1){100} }%
      \put( 100, 100){\line( 0, 1){100} }%
      \put( 100, 100){\line(-1, 1){100} }%
      \put(   0, 100){\line(-1, 1){100} }%
      \put(   0, 100){\line( 1, 1){100} }%
      \put(-100, 100){\line( 0, 1){100} }%
      \put(-100, 100){\line( 1, 1){100} }%
      \put(   0,   0){\line(-1, 1){100} }%
      \put(   0,   0){\line( 0, 1){100} }%
      \put(   0,   0){\line( 1, 1){100} }%
    \color{latdot}%
      \put(   0, 300){\latdot}%
      \put( 100, 200){\latdot}%
      \put(   0, 100){\latdot}%
      \put(   0,   0){\latdot}%
    \end{picture}%
  }%
  % Level 2 group 3 of 3
  \put(300,600){%
    \setlength{\unitlength}{1\tw/(3*1500)}%
    \begin{picture}(0,0)(0,150)%
    %{\color{graphpaper}\graphpaper[50](-100,0)(200,300)}%
    \thicklines%
    \color{latline}%
      \put(   0, 300){\line(-1,-1){100} }%
      \put(   0, 300){\line( 0,-1){100} }%
      \put(   0, 300){\line( 1,-1){100} }%
      \put( 100, 100){\line( 0, 1){100} }%
      \put( 100, 100){\line(-1, 1){100} }%
      \put(   0, 100){\line(-1, 1){100} }%
      \put(   0, 100){\line( 1, 1){100} }%
      \put(-100, 100){\line( 0, 1){100} }%
      \put(-100, 100){\line( 1, 1){100} }%
      \put(   0,   0){\line(-1, 1){100} }%
      \put(   0,   0){\line( 0, 1){100} }%
      \put(   0,   0){\line( 1, 1){100} }%
    \color{latdot}%
      \put(   0, 300){\latdot}%
      \put(   0, 200){\latdot}%
      \put( 100, 100){\latdot}%
      \put(   0,   0){\latdot}%
    \end{picture}%
  }%
  \put(500,600){%
    \setlength{\unitlength}{1\tw/(3*1500)}%
    \begin{picture}(0,0)(0,150)%
    %{\color{graphpaper}\graphpaper[50](-100,0)(200,300)}%
    \thicklines%
    \color{latline}%
      \put(   0, 300){\line(-1,-1){100} }%
      \put(   0, 300){\line( 0,-1){100} }%
      \put(   0, 300){\line( 1,-1){100} }%
      \put( 100, 100){\line( 0, 1){100} }%
      \put( 100, 100){\line(-1, 1){100} }%
      \put(   0, 100){\line(-1, 1){100} }%
      \put(   0, 100){\line( 1, 1){100} }%
      \put(-100, 100){\line( 0, 1){100} }%
      \put(-100, 100){\line( 1, 1){100} }%
      \put(   0,   0){\line(-1, 1){100} }%
      \put(   0,   0){\line( 0, 1){100} }%
      \put(   0,   0){\line( 1, 1){100} }%
    \color{latdot}%
      \put(   0, 300){\latdot}%
      \put( 100, 200){\latdot}%
      \put( 100, 100){\latdot}%
      \put(   0,   0){\latdot}%
    \end{picture}%
  }%
%
%
%
%
% Topologies that generate the algebra {0, x, yz, xyz}
  \put(-400,0){%
    \setlength{\unitlength}{1\tw/(3*1500)}%
    \begin{picture}(0,0)(0,150)%
    %{\color{graphpaper}\graphpaper[50](-100,0)(200,300)}%
    \thicklines%
    \color{red}%
      \put(   0, 300){\line(-1,-1){100} }%
      \put(   0, 300){\line( 0,-1){100} }%
      \put(   0, 300){\line( 1,-1){100} }%
      \put( 100, 100){\line( 0, 1){100} }%
      \put( 100, 100){\line(-1, 1){100} }%
      \put(   0, 100){\line(-1, 1){100} }%
      \put(   0, 100){\line( 1, 1){100} }%
      \put(-100, 100){\line( 0, 1){100} }%
      \put(-100, 100){\line( 1, 1){100} }%
      \put(   0,   0){\line(-1, 1){100} }%
      \put(   0,   0){\line( 0, 1){100} }%
      \put(   0,   0){\line( 1, 1){100} }%
    \color{latdot}%
      \put(   0, 300){\latdot}%
      \put( 100, 200){\latdot}%
      \put(-100, 100){\latdot}%
      \put(   0,   0){\latdot}%
    \end{picture}%
  }%
  \put(-500,300){%
    \setlength{\unitlength}{1\tw/(3*1500)}%
    \begin{picture}(0,0)(0,150)%
    %{\color{graphpaper}\graphpaper[50](-100,0)(200,300)}%
    \thicklines%
    \color{latline}%
      \put(   0, 300){\line(-1,-1){100} }%
      \put(   0, 300){\line( 0,-1){100} }%
      \put(   0, 300){\line( 1,-1){100} }%
      \put( 100, 100){\line( 0, 1){100} }%
      \put( 100, 100){\line(-1, 1){100} }%
      \put(   0, 100){\line(-1, 1){100} }%
      \put(   0, 100){\line( 1, 1){100} }%
      \put(-100, 100){\line( 0, 1){100} }%
      \put(-100, 100){\line( 1, 1){100} }%
      \put(   0,   0){\line(-1, 1){100} }%
      \put(   0,   0){\line( 0, 1){100} }%
      \put(   0,   0){\line( 1, 1){100} }%
    \color{latdot}%
      \put(   0, 300){\latdot}%
      \put(-100, 100){\latdot}%
      \put(   0,   0){\latdot}%
    \end{picture}%
  }%
  \put(-300,300){%
    \setlength{\unitlength}{1\tw/(3*1500)}%
    \begin{picture}(0,0)(0,150)%
    %{\color{graphpaper}\graphpaper[50](-100,0)(200,300)}%
    \thicklines%
    \color{latline}%
      \put(   0, 300){\line(-1,-1){100} }%
      \put(   0, 300){\line( 0,-1){100} }%
      \put(   0, 300){\line( 1,-1){100} }%
      \put( 100, 100){\line( 0, 1){100} }%
      \put( 100, 100){\line(-1, 1){100} }%
      \put(   0, 100){\line(-1, 1){100} }%
      \put(   0, 100){\line( 1, 1){100} }%
      \put(-100, 100){\line( 0, 1){100} }%
      \put(-100, 100){\line( 1, 1){100} }%
      \put(   0,   0){\line(-1, 1){100} }%
      \put(   0,   0){\line( 0, 1){100} }%
      \put(   0,   0){\line( 1, 1){100} }%
    \color{latdot}%
      \put(   0, 300){\latdot}%
      \put( 100, 200){\latdot}%
      \put(   0,   0){\latdot}%
    \end{picture}%
  }%
%
%
%
% Topologies that generate the algebra {0, y, xz, xyz}
  \put(0,0){%
    \setlength{\unitlength}{1\tw/(3*1500)}%
    \begin{picture}(0,0)(0,150)%
    %{\color{graphpaper}\graphpaper[50](-100,0)(200,300)}%
    \thicklines%
    \color{red}%
      \put(   0, 300){\line(-1,-1){100} }%
      \put(   0, 300){\line( 0,-1){100} }%
      \put(   0, 300){\line( 1,-1){100} }%
      \put( 100, 100){\line( 0, 1){100} }%
      \put( 100, 100){\line(-1, 1){100} }%
      \put(   0, 100){\line(-1, 1){100} }%
      \put(   0, 100){\line( 1, 1){100} }%
      \put(-100, 100){\line( 0, 1){100} }%
      \put(-100, 100){\line( 1, 1){100} }%
      \put(   0,   0){\line(-1, 1){100} }%
      \put(   0,   0){\line( 0, 1){100} }%
      \put(   0,   0){\line( 1, 1){100} }%
    \color{latdot}%
      \put(   0, 300){\latdot}%
      \put(   0, 200){\latdot}%
      \put(   0, 100){\latdot}%
      \put(   0,   0){\latdot}%
    \end{picture}%
  }%
  \put(-100,300){%
    \setlength{\unitlength}{1\tw/(3*1500)}%
    \begin{picture}(0,0)(0,150)%
    %{\color{graphpaper}\graphpaper[50](-100,0)(200,300)}%
    \thicklines%
    \color{latline}%
      \put(   0, 300){\line(-1,-1){100} }%
      \put(   0, 300){\line( 0,-1){100} }%
      \put(   0, 300){\line( 1,-1){100} }%
      \put( 100, 100){\line( 0, 1){100} }%
      \put( 100, 100){\line(-1, 1){100} }%
      \put(   0, 100){\line(-1, 1){100} }%
      \put(   0, 100){\line( 1, 1){100} }%
      \put(-100, 100){\line( 0, 1){100} }%
      \put(-100, 100){\line( 1, 1){100} }%
      \put(   0,   0){\line(-1, 1){100} }%
      \put(   0,   0){\line( 0, 1){100} }%
      \put(   0,   0){\line( 1, 1){100} }%
    \color{latdot}%
      \put(   0, 300){\latdot}%
      \put(   0, 100){\latdot}%
      \put(   0,   0){\latdot}%
    \end{picture}%
  }%
  \put( 100,300){%
    \setlength{\unitlength}{1\tw/(3*1500)}%
    \begin{picture}(0,0)(0,150)%
    %{\color{graphpaper}\graphpaper[50](-100,0)(200,300)}%
    \thicklines%
    \color{latline}%
      \put(   0, 300){\line(-1,-1){100} }%
      \put(   0, 300){\line( 0,-1){100} }%
      \put(   0, 300){\line( 1,-1){100} }%
      \put( 100, 100){\line( 0, 1){100} }%
      \put( 100, 100){\line(-1, 1){100} }%
      \put(   0, 100){\line(-1, 1){100} }%
      \put(   0, 100){\line( 1, 1){100} }%
      \put(-100, 100){\line( 0, 1){100} }%
      \put(-100, 100){\line( 1, 1){100} }%
      \put(   0,   0){\line(-1, 1){100} }%
      \put(   0,   0){\line( 0, 1){100} }%
      \put(   0,   0){\line( 1, 1){100} }%
    \color{latdot}%
      \put(   0, 300){\latdot}%
      \put(   0, 200){\latdot}%
      \put(   0,   0){\latdot}%
    \end{picture}%
  }%
%
%
%
% Topologies that generate the algebra {0, z, xy, xyz}
  \put( 400,0){%
    \setlength{\unitlength}{1\tw/(3*1500)}%
    \begin{picture}(0,0)(0,150)%
    %{\color{graphpaper}\graphpaper[50](-100,0)(200,300)}%
    \thicklines%
    \color{red}%
      \put(   0, 300){\line(-1,-1){100} }%
      \put(   0, 300){\line( 0,-1){100} }%
      \put(   0, 300){\line( 1,-1){100} }%
      \put( 100, 100){\line( 0, 1){100} }%
      \put( 100, 100){\line(-1, 1){100} }%
      \put(   0, 100){\line(-1, 1){100} }%
      \put(   0, 100){\line( 1, 1){100} }%
      \put(-100, 100){\line( 0, 1){100} }%
      \put(-100, 100){\line( 1, 1){100} }%
      \put(   0,   0){\line(-1, 1){100} }%
      \put(   0,   0){\line( 0, 1){100} }%
      \put(   0,   0){\line( 1, 1){100} }%
    \color{latdot}%
      \put(   0, 300){\latdot}%
      \put(-100, 200){\latdot}%
      \put( 100, 100){\latdot}%
      \put(   0,   0){\latdot}%
    \end{picture}%
  }%
  \put(300,300){%
    \setlength{\unitlength}{1\tw/(3*1500)}%
    \begin{picture}(0,0)(0,150)%
    %{\color{graphpaper}\graphpaper[50](-100,0)(200,300)}%
    \thicklines%
    \color{latline}%
      \put(   0, 300){\line(-1,-1){100} }%
      \put(   0, 300){\line( 0,-1){100} }%
      \put(   0, 300){\line( 1,-1){100} }%
      \put( 100, 100){\line( 0, 1){100} }%
      \put( 100, 100){\line(-1, 1){100} }%
      \put(   0, 100){\line(-1, 1){100} }%
      \put(   0, 100){\line( 1, 1){100} }%
      \put(-100, 100){\line( 0, 1){100} }%
      \put(-100, 100){\line( 1, 1){100} }%
      \put(   0,   0){\line(-1, 1){100} }%
      \put(   0,   0){\line( 0, 1){100} }%
      \put(   0,   0){\line( 1, 1){100} }%
    \color{latdot}%
      \put(   0, 300){\latdot}%
      \put( 100, 100){\latdot}%
      \put(   0,   0){\latdot}%
    \end{picture}%
  }%
  \put(500,300){%
    \setlength{\unitlength}{1\tw/(3*1500)}%
    \begin{picture}(0,0)(0,150)%
    %{\color{graphpaper}\graphpaper[50](-100,0)(200,300)}%
    \thicklines%
    \color{latline}%
      \put(   0, 300){\line(-1,-1){100} }%
      \put(   0, 300){\line( 0,-1){100} }%
      \put(   0, 300){\line( 1,-1){100} }%
      \put( 100, 100){\line( 0, 1){100} }%
      \put( 100, 100){\line(-1, 1){100} }%
      \put(   0, 100){\line(-1, 1){100} }%
      \put(   0, 100){\line( 1, 1){100} }%
      \put(-100, 100){\line( 0, 1){100} }%
      \put(-100, 100){\line( 1, 1){100} }%
      \put(   0,   0){\line(-1, 1){100} }%
      \put(   0,   0){\line( 0, 1){100} }%
      \put(   0,   0){\line( 1, 1){100} }%
    \color{latdot}%
      \put(   0, 300){\latdot}%
      \put( 100, 100){\latdot}%
      \put(   0,   0){\latdot}%
    \end{picture}%
  }%
%
%
%
% Algebra {0, xyz}
  \put(800,0){%
    \setlength{\unitlength}{1\tw/(3*1500)}%
    \begin{picture}(0,0)(0,150)%
    %{\color{graphpaper}\graphpaper[50](-100,0)(200,300)}%
    \thicklines%
    \color{red}%
      \put(   0, 300){\line(-1,-1){100} }%
      \put(   0, 300){\line( 0,-1){100} }%
      \put(   0, 300){\line( 1,-1){100} }%
      \put( 100, 100){\line( 0, 1){100} }%
      \put( 100, 100){\line(-1, 1){100} }%
      \put(   0, 100){\line(-1, 1){100} }%
      \put(   0, 100){\line( 1, 1){100} }%
      \put(-100, 100){\line( 0, 1){100} }%
      \put(-100, 100){\line( 1, 1){100} }%
      \put(   0,   0){\line(-1, 1){100} }%
      \put(   0,   0){\line( 0, 1){100} }%
      \put(   0,   0){\line( 1, 1){100} }%
    \color{latdot}%
      \put(   0, 300){\latdot}%
      \put(   0,   0){\latdot}%
    \end{picture}%
  }%
\end{picture}
\end{center}
\caption{
  Algebras of sets generated by topologies on the set $\sid\eqd\setn{x,y,z}$
  (see \prefp{ex:set_borel_xyz})
  \label{fig:set_borel_xyz}
  }
\end{figure}


%---------------------------------------
\begin{example}
\label{ex:set_borel_xyz}
%---------------------------------------
There are a total of 29 \hie{topologies} on the set $\setX\eqd\setn{x,y,z}$\ifsxref{topology}{thm:top_num};
and of these, 5 are also \hie{algebras of sets}, 24 are not.
\prefpp{fig:set_borel_xyz} illustrates the
24 topologies on the set $\setn{x,y,z}$ that are \emph{not} algebras of sets
and the 5 algebras of sets that they generate.
\end{example}

} % end mssa exclude

\ifexclude{mssa}{
%=======================================
\section{Literature}
%=======================================
\begin{survey}
\begin{enumerate}
  \item Origin of the symbols $\setu$ and $\seti$:
    \\\citor{peano1888}
    \\\citor{peano1888e}

  \item There is some difference in the definition of \hie{ring of sets}:
    \begin{enumerate}
      \item \hie{ring of sets} defined as closed under $\sets,\seti$:
        \\\citerp{stone1936}{38}
        \\\citerpg{kolmogorov1975}{31}{0486612260}
        \\\citerpg{kolmogorov1999}{20}{0486406830}
        \\\citerpg{constantinescu1984}{155}{3110087847}

      \item \hie{ring of sets} defined as closed under $\setu,\setd$ (compatible definition):
        \\\citerp{wilker1982}{211}
        \\\citerpg{kelley1988}{21}{0387966331}
        \\\citerpu{ab}{96}{0120502577}
        \\\citerpg{haaser1991}{2}{0486665097}
        \\\citerpg{hewitt1994}{118}{0387941908}

      \item \hie{ring of sets} defined as closed under $\setu,\setd,\szero$ (compatible definition):
        \\\citerpg{rao2004}{15}{0824754018}

      \item \hie{ring of sets} defined as closed under $\setu,\seti$ (incompatible definition):
        \\\citerc{hausdorff1927}{???,p.77?}
        \\\citerpg{hausdorff1937e}{90}{0828401195}
        \\\citerp{birkhoff1937}{443}
        \\\citerp{erdos1943}{315}
        \\\citerpg{maclane1999}{485}{0821816462}
    \end{enumerate}

\item Relationship to lattices (order theory):
  \\\citor{stone1936}

\item More references dealing with set structures \ldots
  \\\citer{vaidyanathaswamy1947}
  \\\citer{bagley1955}
  \\\citer{hartmanis1958}
  \\\citer{vaidyanathaswamy1960}
  \\\citer{gaifman1961}
  \\\citer{gaifman1966}
  \\\citer{steiner1966}
  \\\citer{vanrooij1968}
  \\\citer{schnare1968}
  \\\citer{rayburn1969}
  \\\citer{larson1975}
  \\\citer{pudlak1980}
  \\\citer{brown1991}
  \\\citer{watson1994}
  \\\citer{brown1996}

\item Partitions
  \\\citerpg{deza2006}{142}{0444520872}
  \\\citer{day1981}
  \\\citer{rota1964}

\item Distributive and modular properties in lattice of topologies
  \begin{enumerate}
    \item Remark that ``It can be shewn easily that the lattice of topologies is not distributive."
          \\\citer{vaidyanathaswamy1947}
          \\\citerpg{vaidyanathaswamy1960}{134}{0486404560}

    \item Proof that the lattice of $T1$ topologies is not modular:
          \\\citer{bagley1955}

    \item Proof that the lattice of topologies on any set with 3 or more elements
          is not modular (and thus also not distributive):
          \\\citerp{steiner1966}{384}
  \end{enumerate}

\item Complements in lattice of topologies:
  \begin{enumerate}
    \item Proof that every lattice of topologies over a \emph{finite} set is complemented:
          \\\citer{hartmanis1958}
    \item Proof that  every lattice of topologies over a \emph{countably infinite} set is complemented:
          \\\citer{gaifman1961}
    \item Proof that  every lattice of topologies over a \emph{any arbitrary} set is complemented:
          \\\citerp{steiner1966}{397}
    \item \citer{vanrooij1968}
    \item Every topology in $\hat{\Sigma}(\sid$) has at least 2 complements for $\seto{\sid}\ge3$:
          \\\citer{hartmanis1958}
    \item Every topology in $\hat{\Sigma}(\sid$) has at least $\seto{\sid}-1$ complements for $\seto{\sid}\ge2$:
          \\\citer{schnare1968}
    \item A large number of topologies in $\hat{\Sigma}(\sid$) have at least $2^\seto{\sid}$ complements for $\seto{\sid}\ge4$:
          \\\citer{brown1996}
  \end{enumerate}

\end{enumerate}
\end{survey}

} %end mssa exclude

%notes:

%\begin{enumerate}
%  \item trennung in German = separate \index{trennung}
%\end{enumerate}

