%============================================================================
% LaTeX File
% Daniel J. Greenhoe
%============================================================================
%======================================
\chapter{Sequences and Convergence}
\label{chp:seq}
%======================================
\qboxnpq{
  Joseph Fourier (1768--1830)
  \index{Fourier, Joseph}
  \index{quotes!Fourier, Joseph}
  \footnotemark
  }
  {../common/people/fourier_bunzil_wkp_pdomain_gray.jpg}
  {%
  It is necessary
  that the values at which we arrive on increasing continually the
  number of terms, should approach more and more a fixed limit,
  and should differ from it only by a quantity which becomes less
  than any given magnitude: this limit is the value of the series.%
  }
  \footnotetext{\begin{tabular}[t]{ll}
    quote: & \citerppc{fourier1822e}{196}{197}{\textsection 228}\\
    image: & \url{http://en.wikipedia.org/wiki/File:Fourier2.jpg}, public domain
  \end{tabular}}
%======================================
\section{Definitions}
%======================================
%--------------------------------------
\begin{definition}
\label{def:sequence}
\label{def:seq}
\footnote{
  \citerp{bromwich1908}{1},
  \citerpgc{thomson2008}{23}{143484367X}{Definition 2.1},
  \citerpg{joshi1997}{31}{8122408265}
  }
\label{def:tuple}
%--------------------------------------
Let $\clFyx$ be the set of all functions from a set $\setY$ to a set $\setX$.
Let $\hxs{\Z}$ be the \structe{set of integers}.
\defboxt{
  %2014jan05 A function $\ff$ in $\clFyx$ is a \structd{sequence} over $\setX$ if\quad $\setY=\Z$.\\
  A function $\ff$ in $\clFyx$ is an $\setX$-valued \structd{sequence} if\quad $\setY=\Z$.\\
  A sequence may be denoted in the form $\ds\seqxZ{x_n}$ or simply as $\ds\seqn{x_n}$.\\
  %2014jan05 A function $\ff$ in $\clFyx$ is an \hid{n-tuple} over $\setX$ if\quad $\setY=\setn{1,2,\ldots,\xN}$.\\
  A function $\ff$ in $\clFyx$ is an $\setX$-valued \hid{n-tuple} if\quad $\setY=\setn{1,2,\ldots,\xN}$.\\
  An n-tuple may be denoted in the form $\ds\tuplexn{x_n}$ or simply as $\ds\tuplen{x_n}$.
  }
\end{definition}

%--------------------------------------
\begin{definition}
\footnote{
  \citerpgc{haaser1991}{42}{0486665097}{2$\cdot$2 Proposition}
  }
%--------------------------------------
Let $\seqxZ{x_n}$ and $\seqxZ{y_n}$ be sequences over a field $\F$.\\
Let $\tuplexn{x_n}$ and $\tuplexn{y_n}$ be n-tuples over a field $\F$.
\defbox{\begin{array}{rc>{\ds}l@{\qquad\qquad}rc>{\ds}lC}
  \seqn{x_n} + \seqn{y_n}     &\eqd& \seqn{x_n+y_n}     & \alpha\seqn{x_n}            &\eqd& \seqn{\alpha x_n}   & \forall \alpha\in\F\\
  \tuplen{x_n} + \tuplen{y_n} &\eqd& \tuplen{x_n+y_n}   & \alpha\tuplen{x_n}          &\eqd& \tuplen{\alpha x_n} & \forall \alpha\in\F  
  
  
\end{array}}
\end{definition}

%======================================
\section{Sequences in topological spaces}
%======================================
A \structe{topological space} \xref{def:topology} provides sufficient structure to support the property 
of \prope{convergence} (next definition) of a sequence.
In a \structe{metric space}\ifsxref{metric}{def:metric}, a convergent sequence converges to an \prope{unique} limit \xref{thm:xn_to_xy}.
However in a topological space, a convergent sequence may converge to more than one limit \xref{ex:seq_xt31}.

%---------------------------------------
\begin{definition}
\footnote{
  \citerpgc{joshi1983}{83}{0852264445}{(3.1) Definition},
  %\citerpgc{munkres2000}{98}{0131816292}{Hausdorff Spaces},
  ``$\to$" symbol: \citorpc{leathem1905}{13}{section III.11}  % referenced by bromwich1955 page 3
  }
\label{def:converge}
\label{def:limit}
\label{def:limitpnt}
%---------------------------------------
Let $\topspaceX$ be a \structe{topological space} \xref{def:topology}.
%Let $\intA$ be the \structe{interior} of a set $\setA$ \xrefP{def:intA}.
\defboxp{%\begin{array}{M}
  A sequence $\seqxZ{x_n}$ \propd{converges} to a point $x$ if 
  for each \structe{open set} \xref{def:openset}
  $\setU$ of $x$ there exists $\xN\in\Zp$ such that 
  \\\indentx
      $x_n\in\setU$ for all $n>\xN$.
  \\This condition can be expressed in any of the following forms:
  \\$\indentx\begin{array}{>{\scy}rM@{\qquad}>{\scy}rM}
      1. & The \hid{limit} of the sequence $\seqn{x_n}$ is $x$.             & 3. & $\ds\lim_{n\to\infty} \seqn{x_n} = x$.
    \\2. & The sequence $\seqn{x_n}$ is \hid{convergent} with limit $x$.    & 4. & $\ds\seqn{x_n} \to x$.                
  \end{array}$
  %\\& 5. & \metric{x_n}{x}\to 0
  \\A sequence that converges is \hid{convergent}.
    A sequence that does not converge is said to \hid{diverge}, or is \hid{divergent}.
    An element $x\in\setA$ is a \hid{limit point} of $\setA$ if it is the limit of some $\setA$-valued sequence $\seqn{x_n\in\setA}$.
  }
%\end{array}}
\end{definition}

%---------------------------------------
\begin{example}
\footnote{
  \citerpgc{munkres2000}{98}{0131816292}{Hausdorff Spaces}
  }
\label{ex:seq_xt31}
%---------------------------------------
\exboxp{ %\begin{array}{M}
  Let $\topspace{\setX}{\topT_{31}}$ be a \structe{topological space} where $\setX\eqd\setn{x,y,z}$ and
  \\\indentx$\topT_{31}\eqd\setn{\emptyset,\, \setn{x}, \, \setn{x,y},\, \setn{x,z},\, \setn{x,y,z}}$. 
  \\In this space, the sequence $\seqn{x,x,x,\ldots}$ converges to $x$.
    But this sequence also converges to both $y$ and $z$ because $x$ is in every \structe{neighborhood} \xref{def:neighborhood} 
    of $y$
    and $x$ is in every neighborhood of $z$.
    That is, the \structe{limit} \xref{def:limit} of the sequence is \emph{not} unique.
  }
%\end{array}}
\end{example}

%---------------------------------------
\begin{example}
\label{ex:seq_xt56}
%---------------------------------------
In contrast to \pref{ex:seq_xt31}, note that the limit of the sequence $\seqn{x,x,x,\ldots}$ 
\emph{is} \prop{unique} in a \structe{topological space} with sufficiently high resolution with respect to $y$ and $z$ 
such as the following:
\exboxp{ %\begin{array}{M}
    Define a \structe{topological space} $\topspace{\setX}{\topT_{56}}$ where $\setX\eqd\setn{x,y,z}$ and
  \\\indentx$\ds \topT_{56}\eqd\setn{\emptyset,\, \setn{y},\, \setn{z},\, \setn{x,y},\, \setn{y,z},\, \setn{x,y,z}}$. 
  \\In this space, the sequence $\seqn{x,x,x,\ldots}$ converges to $x$ only. 
    The sequence does \emph{not} converge to $y$ or $z$ because there are \structe{open set}s \xref{def:openset} 
    containing $y$ or $z$
    that do not contain $x$ (the open sets $\setn{y}$, $\setn{z}$, and $\setn{y,z}$). 
  }
%\end{array}}
\end{example}


%---------------------------------------
\begin{theorem}[\thmd{The Closed Set Theorem}]
\footnote{
  \citerpgc{kubrusly2001}{118}{0817641742}{Theorem 3.30},
  \citerpgc{haaser1991}{75}{0486665097}{6$\cdot$9 Proposition},
  \citerppg{rosenlicht}{47}{48}{0486650383}
  }
\label{thm:insubset_closed}
\label{thm:cst}
%---------------------------------------
Let $\topspaceX$ be a \structe{topological space}.
Let $\setA$ be a subset of $\setX$ ($\setA\subseteq\setX$).
Let $\clsA$ be the \structe{closure} \xref{def:clsA} of $\setA$ in $\topspaceX$.
%Let $\seq{a_n\in\setA}{n\in\Z}$ be a sequence in $\setA$ 
%that converges to a limit $x$ in $\setX$ such that 
%\\\indentx$\seq{a_n\in\setA}{n\in\Z}\to x\in\setX$.
\thmbox{
  \mcom{\text{$\setA$ is \prope{closed} in $\topspaceX$}}{($\setA=\clsA$)}
  \qquad\iff\qquad
  \brb{\begin{array}{M}
    Every $\setA$-valued sequence $\seq{x_n\in\setA}{n\in\Z}$\\ 
    that \prope{converges} in $\topspaceX$\\ 
    has its \structe{limit} in $\setA$ %($\lim\seqn{x_n}\in\setA$)
  \end{array}}
  }
\end{theorem}
\begin{proof}
\begin{enumerate}
  \item Proof that $\lim\seqn{x_n}\in\setA$ $\implies$   $\setA=\clsA$:
    \begin{enumerate}
      \item Proof that $\setA\subseteq\clsA$: by \prefp{lem:intAAclsA}.
      \item Proof that $\clsA\subseteq\setA$: % ($x\in\clsA\implies x\in\setA$):
        \begin{align*}
          x\in\clsA
            &\implies \text{each open set $\setU$ containing $x$ intersects $\setA$}
            && \text{by \prefp{lem:xinclsA}}
          \\&\implies \text{$\exists \seqn{x_n\in\setA}$ that converges to $x$}
            && \text{by \prefp{def:converge}}
          \\&\implies x\in\setA
            && \text{by left hypothesis}
        \end{align*}
    \end{enumerate}

  \item Proof that $\setA=\clsA$ $\implies$ $\lim\seqn{x_n}\in\setA$:
    \begin{align*}
      \lim\seqn{x_n}=x
        &\iff     \brb{\begin{array}{M}
                    for each \structe{open set} $\setU$ containing $x$, there exists $\xN$\\
                    such that $x_n\in\setU$, $\forall n>\xN$.
                  \end{array}}
        &&        \text{by \prefp{def:converge}}
        %&&        \text{by definition of \hid{convergence}: \prefp{def:converge}}
      \\&\implies \text{each open set $\setU$ containing $x$ intersects $\setA$}
        &&        \text{because $x_n$ in $\setA$}
      \\&\iff     x\in\clsA
        && \text{by \prefp{lem:xinclsA}}
      \\&\iff     x\in\setA
        && \text{by $\setA=\clsA$ hypothesis}
    \end{align*}
\end{enumerate}

\begin{enumerate}
  \item Proof that $\lim\seqn{x_n}\in\setA$ $\implies$   $\setA=\clsA$:
  \item Proof that $\lim\seqn{x_n}\in\setA$ $\impliedby$ $\setA=\clsA$ (proof by contradiction):
    \begin{align*}
      \lim\seqn{x_n}\notin\setA
        &\implies \lim\seqn{x_n}\notin\clsA
        &&        \text{by $\setA=\clsA$ hypothesis}
      \\&\implies \lim\seqn{x_n}\in\cmpp{\clsA}
        &&        \text{by definition of set complement: \prefp{def:ss_setops}}
      \\&\implies \exists x_n \st x_n\in\cmpp{\clsA}
        &&        \text{because $\cmpp{\clsA}$ is \prope{open} and by \prefp{def:converge}}
      \\&\implies \exists x_n \st x_n\notin\clsA
        &&        \text{by definition of set complement: \prefp{def:ss_setops}}
      \\&\implies \exists x_n \st x_n\notin\setA
        &&        \text{by $\setA=\clsA$ hypothesis}
      \\&\implies \text{contradiction}
        &&        \text{by definition of $\seqn{x_n\in\setA}$}
      \\&\implies \lim\seqn{x_n}\in\setA
    \end{align*}
\end{enumerate}

\begin{enumerate}
  \item Proof that $x\in\setA \implies   \setA \text{ is closed}$ (proof by contradiction):
    \begin{enumerate}
      \item Suppose that $\setA$ is not closed.
      %\item Then\ifdochas{metric}{ by \prefpp{def:ms_open},} $\cmpA$ is \emph{not} open.
      \item Then\ifdochas{topology}{ by \prefpp{def:closedset},} $\cmpA$ is \emph{not} open.
      \item If $\cmpA$ is not open, then there is nothing to prevent
            $x$ to be located in $\cmpA$ and yet be so close to $\setA$ that 
            it is \emph{still} a limit point of $\seqn{a_n}$.
            ({\scriptsize Note: 
            It is the openness property of $\cmpA$ that prevents this
            catastrophe from happening. Without it, as is the case under the 
            ``$\setA$ is not closed supposition", a limit point can get too close
            to the border and pull in an infinite number of other points from the sequence
            to the wrong side of the border. 
            That is, the openness property provides a protective buffer on the border
            that keeps points from being sucked across the border by the limit point.})
      \item That is, choose an $r$ such that $\ball{x}{r}\cmpA$.
      \item Since $x$ is a limit point of the sequence $\seqn{a_n}$ and by 
            \prefpp{thm:ms_converge}, 
            \[ \text{for some } 0<\varepsilon<r,\; \exists\xN \st \metric{a_n}{x}<\varepsilon<r.\]
      \item That is, now all the infinite number of points $\seq{a_n}{n>\xN}$ are inside 
            the ball $\ball{x}{r}$, and thus \emph{inside} $\cmpA$. 
      \item Thus, there are points in $\seqn{a_n}$ that are in $\cmpA$, and 
            not in $\setA$ where they are by definition supposed to be.
      \item This is a contradiction, and therefore the original supposition is 
            impossible and $\setA$ must be closed.
    \end{enumerate}

 %\item Proof that $x\in\setA \impliedby \setA \text{ is closed}$ (proof by contradiction):
 %  \begin{enumerate}
 %    \item Suppose that $x\notin\setA$. Then $x\in\cmpA$.
 %    \item By the right hypothesis, $\setA$ is closed and thus 
 %          \ifdochas{metric}{by \prefpp{def:ms_open}, }
 %          $\cmpA$ must be open.
 %    \item If $x\in\cmpA$ and $\cmpA$ is open, 
 %          then \ifdochas{metric}{by \prefp{def:ms_open}} there is a ball around $x$
 %          with arbitrarily small radius $r$  
 %          such that $\ball{x}{r}\subseteq\cmpA$.
 %    \item But no matter how small we make $r$, by \prefpp{thm:ms_converge}
 %          there is always some $0<\varepsilon<r$ such that 
 %          \[ \exists\xN \st \forall n>\xN \; \metric{a_n}{x}<\varepsilon.\]
 %    \item That is, all the points $\seq{a_n}{n>\xN}$ are inside the ball $\ball{x}{r}$.
 %    \item Because they are inside this ball, they are also in $\cmpA$ 
 %          (because the ball in is $\cmpA$.
 %    \item This contradicts the original definition of $\seqn{a_n}$ being a sequence in $\setA$.
 %    \item Therefore, the original supposition must be in error and 
 %          it must be that $x\in\setA$.
 %  \end{enumerate}
\end{enumerate}
\end{proof}

%=======================================
\section{Sequences in distance spaces}
\label{sec:seq_distance}
%=======================================
One of the most important applications of \structe{metric space} \xref{def:metric} analysis, 
and more generally in \structe{distance space} analysis as well, is the concept of \prope{convergence}.
Loosely speaking, a sequence that converges somehow implies that its elements are ``getting closer and closer" to some value.
In a distance space (of which the metric space is a special case), 
there are two standard types of sequences often used to describe this:
\\\begin{tabular}{@{\quad}lll}
  \circOne & \prope{convergent} sequence\,: & The elements of the sequence approach a fixed value $x$ 
  \\       &                            & \xrefP{def:dspace_converge}
  \\
  \circTwo & \prope{Cauchy} sequence\,:     & The elements of the sequence approach each other        
  \\       &                            & \xref{def:cauchy}
\end{tabular}
\\
In a \structe{metric space}, it follows from the \prope{triangle inequality} that 
the \prope{convergent} condition is ``stronger" than the \prope{Cauchy} condition
in the sense that all convergent sequences are Cauchy
but not all Cauchy sequences are convergent \xref{thm:convergent==>cauchy}.
This is \emph{not} the case for all \structe{distance space}s (where the triangle inequality does not hold).
%If however all the Cauchy sequences in a metric space are also convergent sequences
%(each sequence converges to a specific point) {\em and}
%each of those convergent points is {\em inside} the metric space,
%then that metric space is said to be \prope{complete} \xref{def:ms_complete}.

%=======================================
\subsection{Definitions}
%=======================================
%---------------------------------------
\begin{definition}
\footnote{
  in \structe{metric space}:
  \citerpg{rosenlicht}{45}{0486650383},
  \citerpgc{giles1987}{37}{0521359287}{3.2 Definition},
  \citerpgc{khamsi2001}{13}{0471418250}{Definition 2.1}
  %\citerpgc{thomson2008}{30}{143484367X}{Definition 2.1}
  %\citor{cauchy1821}
  ``$\to$" symbol: \citorpc{leathem1905}{13}{section III.11}  % referenced by bromwich1955 page 3
  }
\label{def:dspace_converge}
\label{def:dspace_limit}
%---------------------------------------
%Let $\topspaceX$ be the \structe{topological space} induced by a \structe{distance space} $\dspaceX$ \xref{def:dspace}.
Let $\seqxZ{x_n\in\setX}$ be a  \fncte{sequence} in a \structe{distance space} $\dspaceX$. % \xref{def:dspace}.
\defboxp{
  The sequence $\seqn{x_n}$ \propd{converge}s to a \propd{limit} $x$ if
    for any $\varepsilon\in\Rp$, there exists $\xN\in\Z$
    such that for all $n>\xN$,
    $\distance{x_n}{x}<\varepsilon$.
  \\This condition can be expressed in any of the following forms:
  \\$\indentx\begin{array}{>{\scy}rM@{\qquad}>{\scy}rM}
      1. & The \opd{limit} of the sequence $\seqn{x_n}$ is $x$.             & 3. & $\ds\lim_{n\to\infty} \seqn{x_n} = x$.
    \\2. & The sequence $\seqn{x_n}$ is \propd{convergent} with limit $x$.    & 4. & $\ds\seqn{x_n} \to x$.                
  \end{array}$
  \\A \fncte{sequence} that converges is \propd{convergent}.
    %a \fncte{sequence} that does not converge is said to \propd{diverge}, or is \propd{divergent}.
    %An element $x\in\setA$ is a \vald{limit point} of $\setA$ if it is the limit of some $\setA$-valued sequence $\seqn{x_n\in\setA}$.
  }
\end{definition}

%---------------------------------------
\begin{definition}
\footnote{
  in \structe{metric space}:
  \citerpgc{apostol1975}{73}{0201002884}{4.7},
  \citerpg{rosenlicht}{51}{0486650383}
  }
\label{def:cauchy}
\index{Cauchy sequences}
\index{sequences!Cauchy}
%---------------------------------------
Let $\seqxZ{x_n\in\setX}$ be a \fncte{sequence} in a \structe{distance space} $\dspaceX$. % \xref{def:distance}.
\\\defboxp{
  The sequence $\seqn{x_n}$ is a \structd{Cauchy sequence} in $\dspaceX$ if
  \\\indentx for every $\varepsilon\in\Rp$, there exists $\xN\in\Z$ such that $\forall n,m>\xN,\; \distance{x_n}{x_m}<\varepsilon$\qquad{\scs(\prope{Cauchy condition})}.
  }
\end{definition}


%%---------------------------------------
%\begin{definition}
%\footnote{
%  \citerpgc{blumenthal1953}{9}{0828402426}{{\scshape Definition 6.3}}
%  %\citerpgc{berberian1961}{27}{0821819127}{Theorem~II.4.1}
%  %\citerp{pedersen2000}{4}
%  }
%\label{def:dspace_continuous}
%%---------------------------------------
%%Let $\omsR$ be an \prope{ordered distance space} \xref{def:ods}
%%on the \structe{set of real numbers} $\R$ with the \fncte{usual metric} $\distancea{x}{y}\eqd\abs{x-y}$.
%Let $\dspaceX$ be a \structe{distance space} \xref{def:dspace}.
%\\\defboxp{
%  The \fncte{distance function} $\distancen$ is \propd{continuous} at $\opair{x}{y}$ if
%  \\\indentx$
%  \mcom{\seqn{x_n} \to x \text{ and } \seqn{y_n} \to y}{convergence in $\dspaceX$}
%  \quad \implies \quad
%  \mcom{\seqn{\distance{x_n}{y_n}} \to \distance{x}{y}}{convergence in $\opair{\R}{\distancean}$}
%  $\\
%  The \fncte{distance function} $\distancen$ is \propd{continuous} if $\distancen$ is \prope{continuous} at each $\opair{x}{y}$ in $\setX^2$.
%  }
%\end{definition}


A \fncte{sequence} is said to be \prope{complete} in a \structe{distance space} $\dspaceX$
if every \prope{Cauchy sequence} in $\dspaceX$ \prope{converges} to a \struct{limit} in $\dspaceX$ (next definition).
%---------------------------------------
\begin{definition}
\footnote{
  in \structe{metric space}:
  \citerpg{rosenlicht}{52}{0486650383}
  }
\label{def:complete}
%---------------------------------------
Let $\seqxZ{x_n\in\setX}$ be a \fncte{sequence} in a \structe{distance space} $\dspaceX$.
\\\defboxt{
  The sequence $\seqxZ{x_n\in\setX}$ is \propd{complete} in $\dspaceX$ if
  \\\indentx$\text{$\seqn{x_n}$ is \prope{Cauchy} in $\dspaceX$}
    \quad \implies \quad
    \text{$\seqn{x_n}$ is \prope{convergent} in $\dspaceX$.}$
  }
\end{definition}


%=======================================
\subsection{Properties}
%=======================================
%---------------------------------------
\begin{proposition}
\footnote{
  in \structe{metric space}:
  \citerpgc{giles1987}{49}{0521359287}{Theorem 3.30}
  }
\label{prop:cauchy==>bounded}
%---------------------------------------
Let $\seqxZ{x_n\in\setX}$ be a \fncte{sequence} in a \structe{distance space} $\dspaceX$. % \xref{def:dspace}.
\propbox{
  \brb{\begin{array}{M}
    $\seqn{x_n}$ is \prope{Cauchy} 
    in $\dspaceX$
  \end{array}}
  \qquad\implies\qquad
  \brb{\begin{array}{M}
    $\seqn{x_n}$ is \prope{bounded} 
    in $\dspaceX$
  \end{array}}
}
\end{proposition}
\begin{proof}
\begin{align*}
  \text{$\seqn{x_n}$ is \prope{Cauchy}}
    &\implies \text{for every $\varepsilon\in\Rp$},\; \exists\xN\in\Z \st \forall n,m>\xN,\; \distance{x_n}{x_m}<\varepsilon
    \quad \text{(by \pref{def:cauchy})}
  \\&\implies \exists\xN\in\Z \st \forall n,m>\xN,\; \distance{x_n}{x_m}<1
    \qquad \text{(arbitrarily choose $\varepsilon\eqd1$)}
  \\&\implies \exists\xN\in\Z \st \forall n,m\in\Z,\; \distance{x_n}{x_{m+1}}<\max\setn{\setn{1}\setu\set{\distance{x_p}{x_q}}{p,q\ngtr N}}
  \\&\implies \text{$\seqn{x_n}$ is \prope{bounded}}
    \qquad \text{(by \prefp{def:bounded})}
\end{align*}
\end{proof}


%---------------------------------------
\begin{proposition}
\footnote{
  in \structe{metric space}:
  \citerpg{rosenlicht}{52}{0486650383}
  }
\label{prop:cauchy_subseq}
%---------------------------------------
Let $\seqxZ{x_n\in\setX}$ be a \fncte{sequence} in a \structe{distance space} $\dspaceX$. % \xref{def:dspace}.
Let $\ff\in\clFzz$\ifsxref{relation}{def:clFxy} be a \prope{strictly monotone} function such that $\ff(n)<\ff(n+1)$.
\propbox{
  \mcom{\text{$\seq{x_n}{n\in\Z}$ is \prope{Cauchy}}}{sequence is \prope{Cauchy}}
  \qquad\implies\qquad
  \mcom{\text{$\seq{x_{\ff(n)}}{n\in\Z}$ is \prope{Cauchy}}}{subsequence is also \prope{Cauchy}}
  }
\end{proposition}
\begin{proof}
\begin{align*}
  &\text{$\seq{x_n}{n\in\Z}$ is \prope{Cauchy}}
  \\&\implies \text{for any given } \varepsilon>0,\; \exists\xN \st \forall n,m>\xN,\; \distance{x_n}{x_m}<\varepsilon
    &&        \text{by \prefp{def:cauchy}}
  \\&\implies \text{for any given } \varepsilon>0,\; \exists\xN' \st \forall \ff(n),\ff(m)>\xN',\; \distance{x_{\ff(n)}}{x_{\ff(m)}}<\varepsilon
  \\&\implies \seq{x_{\ff(n)}}{n\in\Z} \text{ is \prope{Cauchy}}
    &&        \text{by \prefp{def:cauchy}}
\end{align*}
\end{proof}



%%---------------------------------------
%\begin{proposition}
%\footnote{
%  \citerppg{deza2014}{6}{7}{3662443422}
%  }
%%---------------------------------------
%\propboxt{
%  If $\fq$ is a \fncte{quasi-metric} \xref{def:qmetric}, then 
%  \\\indentx$\begin{array}{rcl MD}
%    \distancen_1(x,y) &\eqd& \min\setn{\fq(x,y),\fq(y,x)}                    & is a \fncte{metric} & (\fncte{bi-distance}).\\
%    \distancen_2(x,y) &\eqd& \brp{\fq^r(x,y),\fq^r(y,x)}^\frac{1}{r},\,r\ge1 & is a \fncte{metric}.
%  \end{array}$
%  }
%\end{proposition}

%---------------------------------------
\begin{theorem}
\footnote{
  in \structe{metric space}:
  \citerpgc{kubrusly2001}{128}{0817641742}{Theorem 3.40},
  \citerpgc{haaser1991}{75}{0486665097}{6$\cdot$10, 6$\cdot$11 Propositions},
  \citerpgc{bryant1985}{40}{0521318971}{Theorem 3.6, 3.7},
  \citerppg{sutherland1975}{123}{124}{0198531613} %{Proposition 9.23}
  %\citerpgc{rosenlicht}{52}{0486650383}{Proposition}\\
  }
\label{thm:comcls_dspace}
%---------------------------------------
%Let $\dspaceX$ be a \structe{metric space}.
%Let $\dspaceA$ be a \structe{subspace} of a \structe{distance space} $\dspaceX$ \xref{def:dspace}.
Let $\dspaceX$ be a \structe{distance space}. % \xref{def:dspace}.
%Let $\setA$ be a subset of $\setX$.
Let $\clsA$ be the \structe{closure} \xref{def:clsA} of a $\setA$ in a 
\structe{topological space induced by $\dspaceX$}. % \xref{def:dspacetop}.
\thmbox{
  \brb{\begin{array}{FMMD}
    1. & \vale{limit}s are \prope{unique} in $\dspaceX$ & \xref{def:dspace_converge} & and\\
    2. & $\dspaceA$ is \prope{complete} in $\dspaceX$ & \xref{def:complete}        &
  \end{array}}
  \quad\implies\quad
  \mcom{\text{$\setA$ is \prope{closed} in $\dspaceX$}}{$\setA=\clsA$}
  %\\
  %\brb{\begin{array}{FMD}
  %  1. & $\dspaceX$ is \prope{complete} in $\dspaceX$ &  and \\
  %  2. &  $\setA$ is \prope{closed} in $\dspaceX$       &($\setA=\clsA$)
  %\end{array}}
  %&\implies&
  %\brb{\text{$\dspaceA$ is \prope{complete} in $\dspaceX$}}
  }
\end{theorem}
\begin{proof}
\begin{enumerate}
  \item Proof that $\setA\subseteq\clsA$: by \prefp{lem:intAAclsA}
  \item Proof that $\clsA\subseteq\setA$ (proof that $x\in\clsA$ $\implies$ $x\in\setA$):
    \begin{enumerate}
      \item Let $x$ be a point in $\clsA$ ($x\in\clsA$).
      \item Define a \fncte{sequence} of open balls $\seqn{\ball{x}{\frac{1}{1}},\,\ball{x}{\frac{1}{2}},\,\ball{x}{\frac{1}{3}},\,\ldots}$.
      \item Define a \fncte{sequence} of points $\seqn{x_1,\,x_2,\,x_3,\,\ldots}$ such that $x_n\in\ball{x_n}{\frac{1}{n}}\seti\setA$.
      \item Then $\seqn{x_n}$ is \prope{convergent} in $\setX$ with limit $x$ by \prefp{def:dspace_converge}
      \item and  $\seqn{x_n}$ is \prope{Cauchy} in $\setA$ by \prefp{def:cauchy}.
      %\item By the left hypothesis ($\dspaceA$ is \prope{complete}), $\seqn{x_n}$ is therefore also \prope{convergent} in $\setA$.\\
      \item By the hypothesis 2, $\seqn{x_n}$ is therefore also \prope{convergent} in $\setA$.\\
            Let this limit be $y$. Note that $y\in\setA$.\label{item:com_cls_yY}
      %\item By \prefp{prop:xn_to_xy}, limits are \prope{unique}, so $y=x$. \label{item:com_cls_yx}
      \item By hypothesis 1, limits are \prope{unique}, so $y=x$. \label{item:com_cls_yx}
      \item Because $y\in\setA$ \xref{item:com_cls_yY} and $y=x$ \xref{item:com_cls_yx}, so $x\in\setA$.
      \item Therefore, $x\in\clsA\implies x\in\setA$ and $\clsA\subseteq\setA$.
    \end{enumerate}
\end{enumerate}
\end{proof}

%---------------------------------------
\begin{proposition}
\footnote{
  in \structe{metric space}:
  \citerpg{rosenlicht}{46}{0486650383}
  }
%---------------------------------------
Let $\seq{x_n}{n\in\Z}$ be a sequence in a \structe{distance space} $\dspaceX$.
Let $\ff:\Z\to\Z$ be a strictly increasing function such that $\ff(n)<\ff(n+1)$.
\propbox{
  \mcomr{\seq{x_n}{n\in\Z} \to x}{sequence converges to limit $x$}
  \qquad\implies\qquad
  \mcoml{\seq{x_{\ff(n)}}{n\in\Z} \to x}{subsequence converges to the same limit $x$}
  }
\end{proposition}
\begin{proof}
\begin{align*}
  \seq{x_n}{n\in\Z} \to x
    &\implies \forall \varepsilon>0,\; \exists\xN \st \forall n>\xN,\; \distance{x_n}{x}<\varepsilon
    &&        \text{by \prefp{thm:ms_converge}}
  \\&\implies \forall \varepsilon>0,\; \exists \ff(N) \st \forall \ff(n)>\ff(N),\; \distance{x_{\ff(n)}}{x}<\varepsilon
  \\&\implies \seq{x_{\ff(n)}}{n\in\Z} \to x
    &&        \text{by \prefp{thm:ms_converge}}
\end{align*}
\end{proof}



%%---------------------------------------
%\begin{theorem}
%\footnote{
%  \citerpgc{kubrusly2001}{128}{0817641742}{Corollary 3.41}
%  }
%\label{thm:comcomcls}
%%---------------------------------------
%%Let $\dspaceA$ be a \structe{subspace} of a metric space $\dspaceX$ ($\setA\subseteq\setX$).
%Let $\dspaceX$ be a \structe{metric space} \xref{def:metric}.
%Let $\setA$ be a subset of $\setX$.
%Let $\clsA$ be the \structe{closure} \xref{def:clsA} of $\setA$ in $\dspaceX$.
%\thmboxp{
%If $\dspaceX$ is \prope{complete} \xref{def:complete}, then
%  \\\indentx$
%  \brb{\text{$\dspaceA$ is \prope{complete}}}
%  \qquad\iff\qquad
%  \mcom{\text{$\setA$ is \prope{closed} in $\dspaceX$}}{$\setA=\clsA$}
%  $
%  }
%\end{theorem}
%\begin{proof}
%Note that in this corollary, the metric space $\dspaceX$ is assumed to be \prope{complete}.
%\begin{enumerate}
%  \item Proof that \prope{complete} $\implies$ \prope{closed}: by \pref{thm:comcls} (1).
%  \item Proof that \prope{complete} $\impliedby$ \prope{closed}: by \prope{complete} hypothesis and \pref{thm:comcls} (2).
%    \begin{enumerate}
%      \item By left hypothesis 2, $\setA$ is closed in $\dspaceX$.
%      \item By \prefpp{thm:insubset_closed} and because $\setA$ is closed in $\dspaceX$,
%            sequences converge in $\setA$.
%      \item Therefore by \prefpp{def:complete}, $\dspaceA$ is complete.
%    \end{enumerate}
%\end{enumerate}
%\end{proof}

%%---------------------------------------
%\begin{example}
%%---------------------------------------
%Let $\Q$ be the set of \sete{rational numbers}.
%\exbox{%\begin{array}{ll}
%  \text{The metric space $(\Q,\distance{x}{y}=\abs{x-y})$ is {\em not} \prope{complete}.}
%  %2. & \text{The metric space $(\R,\distance{x}{y}=\abs{x-y})$ {\em is} complete.    }\\
%}%\end{array}}
%\end{example}
%\begin{proofns}
%Let $\seq{x_n}{n\in\Znn}$ be the sequence of values approximating $\pi$ truncated to $n$
%decimal points:
%  \[ \seq{x_n}{n\in\Znn} \eqd \seqn{3,\, 3.1,\, 3.14,\, 3.141,\, 3.1415,\, 3.14159,\, 3.141592,\,\ldots} \]
%This is a Cauchy sequence.
%However, this sequence (and all sequences converging to an irrational number)
%does not converge to a rational number ($\Q$) and thus is not in
%the metric space $(\Q,\distancen)$ and thus $(\Q,\distancen)$ is {\em not complete}.
%%But of course this sequence (and all other like sequences) do converge to real numbers and thus
%%$(\R,\distancen)$ is a complete metric space.
%\end{proofns}

%%---------------------------------------
%\begin{example}[\exmd{Cauchy's convergence criterion}/\exmd{Cauchy's criterion}]
%\footnote{
%  \citerpgc{sohrab2003}{54}{0817642110}{Theorem 2.2.5}
%  }
%%---------------------------------------
%Let $\seqxZ{r_n\in\R}$ be a \textbf{real} sequence.
%\exboxp{
%  The metric space $\opair{\seq{r_n}}{\abs{r_n-r_m}}$ is \prope{complete}.
%  }
%\end{example}

%---------------------------------------
\begin{theorem}[\thmd{Cantor intersection theorem}]
\footnote{
  in \structe{metric space}:
  \citerp{davis2005}{28},
  \citerp{hausdorff1937e}{150}
  }
\label{thm:cantor_int_dspace}
\label{thm:cit}
\index{Cantor intersection theorem}
\index{theorems!Cantor intersection}
%---------------------------------------
Let $\dspaceX$ be a \prope{distance space} \xref{def:dspace},
$\seqnZ{\setA_n}$ a \fncte{sequence} with each $\setA_n\in\psetX$, and $\seto{\setA}$ the number of elements in $\setA$.
\thmbox{
  \brb{\begin{array}{FMCMD}
    1. & $\dspaceX$ is \prope{complete}                &                 & \xref{def:complete}       & and \\
    2. & $\setA_n$ is \prope{closed}                     & \forall n\in\Zp & \xref{def:closedset}    & and \\
    3. & $\diam \setA_{n} \ge \diam \setA_{n+1}$         & \forall n\in\Zp & \xref{def:diam}         & and \\
    4. & $\diam \seqxZ{\setA_n} \to 0$                   &                 & \xref{def:dspace_limit}
  \end{array}}
  \qquad\implies\qquad
  \brb{\seto{\ds\setopi_{n\in\Zp} \setA_n } = 1}
  }
\end{theorem}
\begin{proof}
\begin{enumerate}
\item Proof that $\seto{\setopi_{n\in\Z} \setA_n}<2$:
  \begin{enumerate}
    \item Let $\setA\eqd\seti \setA_n$.
    \item $x\ne y$ and $\{x,y\}\in \setA \implies \distance{x}{y}>0$ and $\{x,y\}\subseteq \setA_n \forall n$
    \item $\exists n \st \diam \setA_n < \distance{x}{y}$ by left hypothesis 4
    \item $\implies \exists n \st \sup\set{\distance{x}{y}}{x,y\in \setA_n}<\distance{x}{y}$
    \item This is a contradiction, so $\{x,y\}\notin \setA$ and $\seto{\setopi \setA_n}<2$.
  \end{enumerate}
                                                 
\item Proof that $\seto{\seti \setA_n}\ge1$:
  \begin{enumerate}
    \item Let $x_n\in \setA_n$ and $x_m\in \setA_m$
    \item $\forall \varepsilon,\; \exists\xN\in\Zp \st \setA_N < \varepsilon$
    \item $\forall m,n>\xN,\; x_n\in \setA_n\subseteq \setA_N$ and $x_m\in \setA_m\subseteq \setA_N$
    \item $\distance{x_n}{x_m}\le\diam \setA_N < \varepsilon \implies \{x_n\}$ is a Cauchy sequence
    \item Because $\{x_n\}$ is complete, $x_n\to x$.
    \item $\implies x\in \cls{\brp{\setA_n}} = \setA_n$
    \item $\implies \seto{\setA_n}\ge1$
  \end{enumerate}
\end{enumerate}
\end{proof}

%---------------------------------------
\begin{definition}
\footnote{
  \citerpgc{blumenthal1953}{9}{0828402426}{{\scshape Definition 6.3}}
  }
\label{def:dspace_cont}
%---------------------------------------
Let $\dspaceX$ be a \structe{distance space}.
Let $\setC$ be the set of all \prope{convergent} sequences in $\dspaceX$.
\thmboxp{
  The \fncte{distance function} $\distancen$ is \propd{continuous} in $\dspaceX$ if
  \\\indentx$\ds
    \seqn{x_n},\seqn{y_n}\in\setC
    \quad\implies\quad
    \lim_{n\to\infty}\seqn{\distance{x_n}{y_n}}=\distance{\lim_{n\to\infty}\seqn{x_n}}{\lim_{n\to\infty}\seqn{y_n}}
    $.\\
  A \fncte{distance function} is \propd{discontinuous} if it is not \prope{continuous}.
  }
\end{definition}

%---------------------------------------
\begin{remark}
%---------------------------------------
Rather than defining \prope{continuity} of a \structe{distance function} in terms of 
the \thme{sequential characterization of continuity} \xref{def:dspace_cont},
we could define continuity using an \thme{inverse image characterization of continuity}" \xref{def:dspacetop}.
Assuming an equivalent \structe{topological space} is used for both characterizations, the 
two characterizations are equivalent \xref{thm:limcont}.
In fact, one could construct an equivalence such as the following:
%%---------------------------------------
%\begin{corollary}
%\label{cor:limcont}
%%---------------------------------------
%Let $\dspaceX$ be a \structe{distance space}.
%Let $\topspaceX$ be the \structe{topological space induced by $\dspaceX$} \xref{def:dspacetop}.
%Let $\topspace{\R}{\topS}$ be the \structe{usual topological space over $\R$}.
%Let $\setC$ be the set of all \prope{convergent} sequences in $\dspaceX$.
\rembox{
  \brb{\begin{tabstr}{0.75}\begin{array}{M}
    \\
    $\distancen$ is \prope{continuous}  in $\clF{\setX^2}{\R}$\\
    \xref{def:continuous}\\
    {\scs(\thme{inverse image characterization of continuity})}
  \end{array}\end{tabstr}}
  \quad\iff\quad
  \brb{\begin{tabstr}{0.75}\begin{array}{M}
    $\ds\seqn{x_n},\seqn{y_n}\in\setC\quad\implies$\\
    $\ds\lim_{n\to\infty}\seqn{\distance{x_n}{y_n}}=\distance{\lim_{n\to\infty}\seqn{x_n}}{\lim_{n\to\infty}\seqn{y_n}}$\\
    \xref{def:converge}\\
    {\scs(\thme{sequential characterization of continuity})}
  \end{array}\end{tabstr}}
  }\\
%\end{corollary}
%\begin{proof}
Note that just as $\seqn{x_n}$ is a sequence in $\setX$, so the ordered pair $\opair{\seqn{x_n}}{\seqn{y_n}}$
is a sequence in $\setX^2$.
The remainder %of this proof 
follows from \prefpp{thm:limcont}.
%\end{proof}
However, use of the \thme{inverse image characterization} is somewhat troublesome
because we would need a topology on $\setX^2$, and we don't immediately have one defined and ready to use.
In fact, we don't even immediately have a distance space on $\setX^2$ defined or even open balls in such a distance space.
The result is, for the scope of this chapter, it is arguably not worthwhile constructing the extra structure, 
but rather this chapter instead
uses the \thme{sequential characterization} as a definition \xref{def:dspace_cont}.
\end{remark}

%=======================================
\subsection{Examples}
%=======================================
Similar distance functions and several of the observations for the examples 
in this section can be found in \citerppg{blumenthal1953}{8}{13}{0828402426}.

In a \structe{metric space}, all \structe{open ball}s are \prope{open},
the \structe{open ball}s form a \structe{base} for a \structe{topology}, 
the limits of \prope{convergent} sequences are \prope{unique}, 
and the \fncte{metric function} is \prope{continuous}.
In the \structe{distance space} of the next example, none of these properties hold.
%---------------------------------------
\begin{example}
\footnote{
  A similar distance function $\distancen$ and \prefp{item:dspace_01} 
  can in essence be found in \citerpg{blumenthal1953}{8}{0828402426}.
  %Definitions for \pref{ex:dspace_01}:
  %$\opair{x}{y}$: \prefpp{def:opair};
  %$\intoo{a}{b}$ and $\intoc{a}{b}$: \prefpp{def:intxx};
  %$\abs{x}$: \prefpp{def:abs};
  %$\clF{\R\times\R}{\R}$: \prefpp{def:clFxy};
  %\fncte{distance}: \prefpp{def:distance};
  %\structe{open ball}: \prefpp{def:ball};
  %\prope{open}: \prefpp{def:dspace_open};
  %\structe{base}: \prefpp{def:base};
  %\structe{topology}: \prefpp{def:topology};
  %\structe{open set}: \prefpp{def:dspace_open};
  %\structe{topological space induced by $\dspaceRd$}: \prefpp{def:dspacetop};
  %\prope{discontinuous}: \prefpp{def:dspace_cont};
  }
\label{ex:dspace_01}
%---------------------------------------
%Let $\setX\eqd\R$.
Let $\opair{x}{y}$ be an \structe{ordered pair} in $\R^2$.
Let $\intoo{a}{b}$ be an \structe{open interval} and $\intoc{a}{b}$ a \structe{half-open interval} in $\R$.
Let $\abs{x}$ be the \fncte{absolute value} of $x\in\R$.
The function $\distance{x}{y}\in\clF{\R\times\R}{\R}$ such that
\\\indentx$\distance{x}{y} \eqd \brb{\begin{array}{lMD}
      y         & $\forall \opair{x}{y}\in\setn{4}\times\intoc{0}{2}$ & (\structe{vertical half-open interval})\\
      x         & $\forall \opair{x}{y}\in\intoc{0}{2}\times\setn{4}$ & (\structe{horizontal half-open interval})\\
      \abs{x-y} & otherwise                                           & (\prope{Euclidean})
    \end{array}}$\quad is a \fncte{distance} on $\R$.
\\
Note some characteristics of the \structe{distance space} $\dspaceRd$:
\begin{enumerate}
  \item $\dspaceRd$ is not a \structe{metric space} because $\distancen$ does not satisfy the \prope{triangle inequality}:
    \\\indentx$\distance{0}{4}
        \eqd \abs{0-4} = 4
        \nleq 2
        = \abs{0-1}     + 1
        \eqd \distance{0}{1} + \distance{1}{4}
       $

  \item Not every \structe{open ball} in $\dspaceRd$ is \prope{open}.\label{item:dspace_01_oballo}\\
        For example, the \structe{open ball} $\ball{3}{2}$ is \prope{not open} 
        because $4\in\ball{3}{2}$ \emph{but} for all $0<\varepsilon<1$
        \\\indentx$\ball{4}{\varepsilon}=\intoo{4-\varepsilon}{4+\varepsilon}\setu\intoo{0}{\varepsilon}
           \nsubseteq\intoo{1}{5}
           = \ball{3}{2}$

  \item The \structe{open balls} of $\dspaceRd$ do not form a \structe{base} for a \structe{topology} on $\R$.\\
        This follows directly from \pref{item:dspace_01_oballo} and \prefpp{thm:baseoball}.

  \item In the \structe{distance space} $\dspaceRd$, limits are \prope{not unique};\\
        For example, the sequence $\seqn{\sfrac{1}{n}}_1^\infty$ converges both to the limit $0$ and the limit $4$ in $\dspaceRd$:
        \label{item:dspace_01}
    \\\indentx$\begin{array}{*{5}{>{\ds}l}}
      \lim_{n\to\infty}\distance{x_n}{0} 
        &\eqd \lim_{n\to\infty}\distance{\sfrac{1}{n}}{0}
        &\eqd \lim_{n\to\infty}\abs{\sfrac{1}{n}-0}
        &= 0
       %&\implies \lim_{n\to\infty}\seqn{\sfrac{1}{n}}=0
        &\quad\implies\quad \seqn{\sfrac{1}{n}}\to0
        \\
      \lim_{n\to\infty}\distance{x_n}{4} 
        &\eqd \lim_{n\to\infty}\distance{\sfrac{1}{n}}{4}
        &\eqd \lim_{n\to\infty}\seqn{\sfrac{1}{n}}
        &= 0
       %&\implies \lim_{n\to\infty}\seqn{\sfrac{1}{n}}=4
        &\quad\implies\quad \seqn{\sfrac{1}{n}}\to4
    \end{array}$

  \item The \structe{topological space $\topspaceX$ induced by $\dspaceRd$} also 
        yields limits of $0$ and $4$ for the sequence $\seqn{\sfrac{1}{n}}_1^\infty$, just 
        as it does in \pref{item:dspace_01}.
        This is largely due to the fact that, for small $\varepsilon$, 
        the open balls $\ball{0}{\varepsilon}$ and $\ball{4}{\varepsilon}$ are \prope{open}.
    \begin{align*}
        \text{$\ball{0}{\varepsilon}$ is \prope{open}}
          &\implies \text{for each $\setU\in\topT$ that contains $0$, $\exists\xN\in\Zp \st \sfrac{1}{n}\in\setU\quad\forall n>\xN$}
        \\&\iff \seqn{\sfrac{1}{n}}\to0
          \qquad\text{by definition of \prope{convergence} \xref{def:converge}}
        \\
        \text{$\ball{4}{\varepsilon}$ is \prope{open}}
          &\implies \text{for each $\setU\in\topT$ that contains $4$, $\exists\xN\in\Zp \st \sfrac{1}{n}\in\setU\quad\forall n>\xN$}
        \\&\iff \seqn{\sfrac{1}{n}}\to4
          \qquad\text{by definition of \prope{convergence} \xref{def:converge}}
    \end{align*}  

  %\item The distance function $\distancen\in\clF{\R\times\R}{\R}$ is \prope{discontinuous} \label{item:dspace_01_discon}
  %      with respect to the \structe{Euclidean topologies of $\R$ and $\R^2$}.
  %      Consider the following: 
  %      Let $\opair{x}{y}\in\R^2$ be an \structe{ordered pair}.
  %      Let $\ball{\opair{x}{y}}{\varepsilon}$ be an \structe{open ball} in the \structe{Euclidean metric space}
  %      $\metspace{\R^2}{\metrican}$ where
  %      \\$\metrica{\opair{x_1}{x_2}}{\opair{y_1}{y_2}}\eqd\sqrt{(x_1-x_2)^2+(y_1-y_2)^2}$\qquad%
  %        ${\scy\forall \opair{x_1}{x_2},\opair{y_1}{y_2}\in\R^2}$.\footnote{
  %      Note that $\C\eqd\R^2$, where $\C$ is the set of \structe{complex numbers}, and thus $\dspaceRd$ 
  %      can be viewed as a \structe{distance space} over $\C$ and the  
  %      \fncte{distance function} $\metrican$ somewhat like a \fncte{projection} 
  %      from $\C^2=\R^4$ onto a ``real axis" $\R$ in $\C$.
  %      It is like a \fncte{projection} in the sense that it maps from $\C^2=\R^4$ to $\R$ and also because 
  %      it has a property similar to the \prope{idempotent} property in the sense that 
  %      \\\indentx$\metrica{\opair{\metrica{\opair{x_1}{x_2}}{\opair{y_1}{y_2}}}{0}}{\opair{0}{0}}=
  %      \metrica{\opair{x_1}{x_2}}{\opair{y_1}{y_2}}$ .
  %      } 
  %      \qquad Then
  %      \begin{align*}
  %        \distancen^{-1}\brs{\intoo{0}{2}}
  %          &\supseteq \setn{4}\times\intoo{0}{2}
  %          && \text{by definition of \fncte{distance} $\distancen$}
  %        \\&\nsupseteq\ball{\opair{4}{1}}{\varepsilon}\quad\forall \varepsilon>0
  %          && \text{by definition of \structe{open ball} $\balln$}
  %        \\&\implies \text{$\distancen^{-1}\brs{\intoo{0}{2}}$ is \prope{not open}}
  %          &&\text{by \prefpp{def:dspace_open}}
  %        \\&\implies \text{$\distancen$ is \prope{discontinuous}}
  %          &&\text{by \prefpp{def:continuous}}
  %      \end{align*}

  \item The distance function $\distancen$ is \prope{discontinuous} \xref{def:dspace_cont}:
  %\item As an alternative to the \pref{item:dspace_01_discon} method of using \structe{open set}s to prove that $\distancen$ is \prope{discontinuous}, we could use convergent sequences to prove the same:
    \begin{align*}
      \lim_{n\to\infty}\seqn{\distance{1-\sfrac{1}{n}}{4-\sfrac{1}{n}}}
        &=\lim_{n\to\infty}\seqn{\abs{\brp{1-\sfrac{1}{n}}-\brp{4-\sfrac{1}{n}}}}
         = \abs{1-4} = 3 \neq 4 = \distance{0}{4}
      \\&= \distance{\lim_{n\to\infty}\seqn{1-\sfrac{1}{n}}}{\lim_{n\to\infty}\seqn{4-\sfrac{1}{n}}}
    \end{align*}
   %which by \prefpp{cor:limcont} implies that $\distancen$ is \prope{discontinuous}.
   %   \item Therefore, by \prefpp{cor:limcont}, $\distancen$ is \prope{discontinuous}.
   % \end{enumerate}
\end{enumerate}
\end{example}

In a \structe{metric space}, all \prope{convergent} sequences are also \prope{Cauchy}.
However, this is not the case for all \structe{distance space}s, as demonstrated next:
%---------------------------------------
\begin{example}
\footnote{
  The distance function $\distancen$ and \prefp{item:dspace_1n_cauchy} 
  can in essence be found in \citerpg{blumenthal1953}{9}{0828402426}
  }
\label{ex:dspace_1n}
%---------------------------------------
%Let $\setX\eqd\set{\frac{1}{n}}{n=1,2,3,\ldots}\setu\setn{0}$ be a \structe{set}.
%The function $\distance{x}{y}\in\clF{\setX^2}{\R}$ such that
The function $\distance{x}{y}\in\clF{\R\times\R}{\R}$ such that
\\\indentx$\distance{x}{y} \eqd \brb{\begin{array}{lMD}
      \abs{x-y} & for $x=0$ or $y=0$ or $x=y$ & (\prope{Euclidean})\\
      1         & otherwise                   & (\prope{discrete})
    \end{array}}$\quad is a \fncte{distance} on $\R$. 
\\
Note some characteristics of the \structe{distance space} $\dspaceRd$:
\begin{enumerate}
  \item $\dspaceX$ is not a \structe{metric space} because the \prope{triangle inequality} does not hold:
      \\\indentx
        $\distance{\frac{1}{4}}{\frac{1}{2}}
        = 1
        \nleq \frac{3}{4}
        = \abs{\frac{1}{4}-0}     + \abs{0-\frac{1}{2}}
        = \distance{\frac{1}{4}}{0} + \distance{0}{\frac{1}{2}}
        $

  \item The \structe{open ball} $\ball{\frac{1}{4}}{\frac{1}{2}}$ is \prope{not open}
        because for any $\varepsilon\in\Rp$, no matter how small,  \label{item:dspace_1n_oballo}
        \\\indentx
         $\ball{0}{\varepsilon} = \intoo{-\varepsilon}{+\varepsilon}
            \nsubseteq \setn{0,\,\frac{1}{4}}
            =\set{x\in\setX}{\distance{\frac{1}{4}}{x}<\frac{1}{2}}
            \eqd\ball{\frac{1}{4}}{\frac{1}{2}}$
  
  \item Even though not all the \structe{open ball}s are \prope{open}, 
        it is still possible to have an \structe{open set} in $\dspaceX$. 
        For example, the set $\setU\eqd\setn{1,\,2}$ is \prope{open}:
        \\\indentx$\begin{array}{rclclclcl}
            \ball{1}{1}
            &\eqd& \set{x\in\setX}{\distance{1}{x}<1}
            &=& \setn{1}
            &\subseteq& \setn{1,\,2}
            &\eqd& \setU
            \\
            \ball{2}{1}
            &\eqd& \set{x\in\setX}{\distance{2}{x}<1}
            &=& \setn{2}
            &\subseteq& \setn{1,\,2}
            &\eqd& \setU
          \end{array}$

  \item By \pref{item:dspace_1n_oballo} and \prefpp{thm:baseoball}, 
        the \structe{open ball}s of $\dspaceRd$ do not form a \structe{base} for a \structe{topology} on $\R$.

  \item Even though the open balls in $\dspaceRd$ do not induce a topology on $\setX$, it is still possible to 
        find a set of \structe{open set}s in $\dspaceX$ that \emph{is} a topology. 
        For example, the set
          $\setn{\emptyset,\,\setn{1,2},\,\R}$
        is a topology on $\R$.

  \item In $\dspaceRd$, limits of \prope{convergent} sequences are \prope{unique}:
    \\\indentx$\ds
      \seqn{x_n}\to x \quad\implies\quad
      \lim_{n\to\infty}\distance{x_n}{x}=\brb{\begin{array}{rclMD}
        \lim\abs{x_n-0} &=& 0 & for $x=0$ & OR\\
            \abs{x-x}   &=& 0 & for constant $\seqn{x_n}$ for $n>\xN$ & OR\\
                    1   &\neq& 0 & otherwise
      \end{array}}$\\
     which says that there are only two ways for a sequence to converge: either $x=0$ or the sequence eventually becomes constant
     (or both). Any other sequence will \prope{diverge}. Therefore we can say the following:
    \begin{enumerate}
      \item If $x=0$ and the sequence is not constant, then the limit is \prope{unique} and $0$.
      \item If $x=0$ and the sequence is constant, then the limit is \prope{unique} and $0$.
      \item If $x\neq0$ and the sequence is constant, then the limit is \prope{unique} and $x$.
      \item If $x\neq0$ and the sequence is not constant, then the sequence diverges and there is no limit.
    \end{enumerate}

  \item In $\dspaceRd$, a \prope{convergent} sequence is not necessarily \prope{Cauchy}.\label{item:dspace_1n_cauchy}
    For example,
    \begin{enumerate}
      \item the sequence $\seqnZp{\sfrac{1}{n}}$ is \prope{convergent} with limit $0$:
       $\ds\lim_{n\to\infty}\distance{\sfrac{1}{n}}{0}
        = \lim_{n\to\infty}\sfrac{1}{n}
        = 0$
      \item However, even though $\seqn{\sfrac{1}{n}}$ is \prope{convergent}, it is \prope{not Cauchy}:
       $\ds\lim_{n,m\to\infty}\distance{\sfrac{1}{n}}{\sfrac{1}{m}}
       = 1
       \neq 0
      $
    \end{enumerate}

  \item The \fncte{distance function} $\distancen$ is \prope{discontinuous} in $\dspaceX$:
    \begin{align*}
        \lim_{n\to\infty}\seqn{\distance{\sfrac{1}{n}}{2-\sfrac{1}{n}}} 
        &= 1 
      \\&\neq 2 = \distance{0}{2} = \distance{\lim_{n\to\infty}\seqn{\sfrac{1}{n}}}{\lim_{n\to\infty}\seqn{2-\sfrac{1}{n}}}
    \end{align*}
    %which by \prefpp{cor:limcont} implies that $\distancen$ is \prope{discontinuous}.
\end{enumerate}
\end{example}

%In any \structe{distance space} in which the \prope{triangle inequality} holds 
%(a \structe{metric space}),
%the \fncte{distance} function is always \prope{continuous}. % \xref{prop:metric_continuous}.
%This is even true in the case of the \fncte{discrete metric} % \xref{ex:dmetric}
%which is induced from any other metric using the \prope{non-continuous} 
%\fncte{discrete metric preserving function}. % \xref{ex:mpf_discrete}.
%However, distance functions are not all \prope{continuous}, as demonstrated next.
%---------------------------------------
\begin{example}
\footnote{
  The distance function $\distancen$ and \prefp{item:dspace_21_cont}
  can in essence be found in \citerpg{blumenthal1953}{9}{0828402426}
  }
\label{ex:dspace_21}
%---------------------------------------
%Let $\setX\eqd\set{\frac{1}{n}}{n=1,2,3,\ldots}\setu\setn{0}$ be a \structe{set}.
The function $\distance{x}{y}\in\clF{\R\times\R}{\R}$ such that
\\\indentx$\distance{x}{y} \eqd \brb{\begin{array}{rMD}
      2\abs{x-y} & $\forall \opair{x}{y}\in\setn{\opair{0}{1},\,\opair{1}{0}}$ & (\prope{dilated Euclidean})\\
       \abs{x-y} & otherwise                                                   & (\prope{Euclidean})
    \end{array}}$\quad is a \fncte{distance} on $\R$.
\\
Note some characteristics of the \structe{distance space} $\dspaceRd$:
\begin{enumerate}
  \item $\dspaceRd$ is \emph{not} a \structe{metric space} because $\distancen$ does \emph{not} 
        satisfy the \prope{triangle inequality}:
      \\\indentx$\ds\distance{0}{1}
        \eqd 2\abs{0-1}
        = 2
        \nleq 1
        = \abs{0-\sfrac{1}{2}}  + \abs{\sfrac{1}{2}-1}
        \eqd \distance{0}{\sfrac{1}{2}} + \distance{\sfrac{1}{2}}{1}
        $

  \item The function $\distancen$ is \prope{discontinuous}: \label{item:dspace_21_cont}
        %In particular, it is \prope{not continous} at the point $\opair{\lim\seqn{1-\frac{1}{n}}}{\lim\seqn{\frac{1}{n}}}$.
    \begin{align*}
      &\lim_{n\to\infty}\seqn{\distance{1-\sfrac{1}{n}}{\sfrac{1}{n}}}
        \eqd\lim_{n\to\infty}\seqn{\abs{1-\sfrac{1}{n}-\sfrac{1}{n}}}
        =1
      \\&\qquad\neq 2 
        =2\abs{0-1}
        \eqd\distance{0}{1}
        =\distance{\lim_{n\to\infty}\seqn{1-\sfrac{1}{n}}}{\lim_{n\to\infty}\seqn{\sfrac{1}{n}}}
    \end{align*}
    %which by \prefpp{cor:limcont} implies that $\distancen$ is \prope{discontinuous}.

  \item In $\dspaceX$, \structe{open ball}s are \prope{open}: \label{item:dspace_21_oballo}
    \begin{enumerate}
      \item $\metrica{x}{y}\eqd\abs{x-y}$ is a \fncte{metric} and thus all open balls in that do not contain both $0$ and $1$ are \prope{open}.
      \item By \prefpp{ex:mpf_ascaled}, $\metricb{x}{y}\eqd2\abs{x-y}$ is also a \fncte{metric} and thus all open balls containing $0$ and $1$ only are \prope{open}.
      \item The only question remaining is with regards to open balls that contain $0$, $1$ and some other element(s) in $\R$.
            But even in this case, open balls are still open. For example:
        \\\indentx$\ball{-1}{2} = \intoo{-1}{2} = \intoo{-1}{1}\setu\intoo{1}{2}$\\
        Note that both $\intoo{-1}{1}$ and $\intoo{1}{2}$ are \prope{open}, and thus by \prefpp{thm:dspace_open},
        $\ball{-1}{2}$ is \prope{open} as well.
    \end{enumerate}

  \item By \pref{item:dspace_21_oballo} and \prefpp{thm:baseoball}, 
        the \structe{open ball}s of $\dspaceRd$ \emph{do} form a \structe{base} for a \structe{topology} on $\R$.

  \item In $\dspaceX$, the limits of \prope{convergent} sequences are \prope{unique}.
        This is demonstrated in \prefpp{ex:pdspace_21} using additional structure developed in \pref{chp:pdspace}.

  \item In $\dspaceX$, \prope{convergent} sequences are \prope{Cauchy}.\\
        This is also demonstrated in \prefpp{ex:pdspace_21}.

\end{enumerate}
\end{example}

The \fncte{distance function}s in \prefpp{ex:dspace_01}--\prefpp{ex:dspace_21} were all \prope{discontinuous}.
In the absence of the \prope{triangle inequality} and in light of these examples, 
one might try replacing the \prope{triangle inequality} with the weaker requirement of \prope{continuity}.
However, as demonstrated by the next example, this also leads to an arguably disastrous result.
%---------------------------------------
\begin{example}
\footnote{
  \citerppg{blumenthal1953}{12}{13}{0828402426},
  \citerppg{laos1998}{118}{119}{9810231806}
  }
\label{ex:dspace_xy2}
%---------------------------------------
The function $\distancen\in\clF{\R\times\R}{\R}$ such that $\distance{x}{y}\eqd(x-y)^2$ is a \fncte{distance} on $\R$.
\\Note some characteristics of the \structe{distance space} $\dspaceRd$:
\begin{enumerate}
  \item $\dspaceRd$ is \emph{not} a \structe{metric space} because the \prope{triangle inequality} does not hold:
    \\\indentx$\ds\distance{0}{2} \eqd \brp{0-2}^2 = 4 \nleq 2 = \brp{0-1}^2 + \brp{1-2}^2 \eqd \distance{0}{1} + \distance{1}{2} $

  \item The \fncte{distance function} $\distancen$ is \prope{continuous} in $\dspaceX$.
        This is demonstrated in the more general setting of \pref{chp:pdspace} in \prefpp{ex:pdspace_xy2}.

  \item Calculating the length of curves in $\dspaceX$ leads to a paradox:\footnote{
        This is the method of ``inscribed polygons" for calculating the length of a curve and goes back to Archimedes:
        \citerpg{brunschwig2003}{26}{0674021568},
        \citerpc{walmsley1920}{200}{\textsection158},
        }
    \begin{enumerate}
      \item Partition $\intcc{0}{1}$ into $2^\xN$ consecutive line segments connected at the points 
            \\\indentx$\seqn{0,\,\frac{1}{2^\xN},\,\frac{2}{2^\xN},\,\frac{3}{2^\xN},\,\ldots,\,\frac{2^{\xN-1}1}{2^\xN},\,1}$
      \item Then the distance, as measured by $\distancen$, between any two consecutive points is
            \\\indentx$\distance{p_n}{p_{n+1}}\eqd\brp{p_n-p_{n+1}}^2=\brp{\frac{1}{2^\xN}}^2=\frac{1}{2^{2\xN}}$
      \item But this leads to the paradox that the total length of $\intcc{0}{1}$ is 0:
            \\\indentx$\ds %1=\brp{0-1}^2\eqd\distance{0}{1}
              \lim_{\xN\to\infty}\sum_{n=0}^{2^{\xN}-1}\frac{1}{2^{2\xN}}
              =\lim_{\xN\to\infty}\frac{2^\xN}{2^{2\xN}}
              =\lim_{\xN\to\infty}\frac{1}{2^{\xN}}
              =0
             $
    \end{enumerate}
\end{enumerate}
\end{example}

%======================================
\section{Sequences in metric spaces}
%======================================

%=======================================
\subsection{Cauchy sequences}
%=======================================
\ifdochasnot{distance}{
%---------------------------------------
\begin{definition}
\footnote{
  \citerpgc{apostol1975}{73}{0201002884}{4.7},
  \citerpg{rosenlicht}{51}{0486650383}
  }
\label{def:cauchy}
\index{Cauchy sequences}
\index{sequences!Cauchy}
%---------------------------------------
\defbox{\begin{array}{l>{\ds}l}
  \mc{2}{l}{\text{A sequence $\seqxZ{x_n\in\setX}$
                  is a \structd{Cauchy sequence} in the metric space $\metspaceX$ if}} 
  \\&  \mcom{\text{for every } \varepsilon\in\Rp,\; \exists\xN\in\Z \st \forall n,m>\xN,\; \metric{x_n}{x_m}<\varepsilon}{\prope{Cauchy condition}}.
\end{array}}
\end{definition}
}

%---------------------------------------
\begin{lemma}
\label{lem:cauchy==>bounded}
\footnote{
  \citerpgc{giles1987}{49}{0521359287}{Theorem 3.30}
  }
%---------------------------------------
Let $\seqxZ{x_n\in\setX}$ be a sequence in the metric space $\metspaceX$.
\lembox{
  \brb{\begin{array}{M}
    $\seqn{x_n}$ is \prope{Cauchy} \xref{def:cauchy}\\ 
    in $\metspaceX$
  \end{array}}
  \qquad\implies\qquad
  \brb{\begin{array}{M}
    $\seqn{x_n}$ is \prope{bounded}\\ 
    in $\metspaceX$
  \end{array}}
}
\end{lemma}
\begin{proof}
\begin{align*}
  \text{$\seqn{x_n}$ is \prope{Cauchy}}
    &\implies \text{for every $\varepsilon\in\Rp$},\; \exists\xN\in\Zp \st \forall n,m>\xN,\; \metric{x_n}{x_m}<\varepsilon
    \qquad \text{\xref{def:cauchy}}
  \\&\implies \exists\xN\in\Z \st \forall n,m>\xN,\; \metric{x_n}{x_m}<1
    \qquad \text{(arbitrarily choose $\varepsilon\eqd1$)}
  \\&\implies \exists\xN\in\Z \st \forall n,m\in\Zp,\; \metric{x_n}{x_{m+1}}<\max\setn{\setn{1}\setu\set{\metric{x_p}{x_q}}{p,q\ngtr N}}
  \\&\implies \text{$\seqn{x_n}$ is \prope{bounded}}
    \qquad \text{(by definition of \prope{bounded}\ifsxref{latb}{def:bounded})}
\end{align*}
\end{proof}

%---------------------------------------
\begin{proposition}
\footnote{
  \citerpg{rosenlicht}{52}{0486650383}
  }
%---------------------------------------
Let $\seq{x_n}{n\in\Z}$ be a sequence in a metric space $\metspaceX$.
Let $\ff:\Z\to\Z$ be a strictly increasing function such that $\ff(n)<\ff(n+1)$.
\propbox{
  \mcom{\text{$\seq{x_n}{n\in\Z}$ is \prope{Cauchy}}}{Cauchy sequence}
  \qquad\implies\qquad
  \mcom{\text{$\seq{x_{\ff(n)}}{n\in\Z}$ is \prope{Cauchy}}}{subsequence is also Cauchy}
  }
\end{proposition}
\begin{proof}
\begin{align*}
  &\text{$\seq{x_n}{n\in\Z}$ is \prope{Cauchy}}
  \\&\implies \text{for any given } \varepsilon>0,\; \exists\xN \st \forall n,m>\xN,\; \metric{x_n}{x_m}<\varepsilon
    &&        \text{by \prefp{def:cauchy}}
  \\&\implies \text{for any given } \varepsilon>0,\; \exists\xN' \st \forall \ff(n),\ff(m)>\xN',\; \metric{x_{\ff(n)}}{x_{\ff(m)}}<\varepsilon
  \\&\implies \seq{x_{\ff(n)}}{n\in\Z} \text{ is \prope{Cauchy}}
    &&        \text{by \prefp{def:cauchy}}
\end{align*}
\end{proof}

%=======================================
\subsection{Convergence in Metric Space}
%=======================================
%---------------------------------------
\begin{theorem}
\footnote{
  \citerpg{rosenlicht}{45}{0486650383},
  \citerpgc{giles1987}{37}{0521359287}{3.2 Definition}
  %\citerpgc{khamsi2001}{13}{0471418250}{Definition 2.1}
  %\citerpgc{thomson2008}{30}{143484367X}{Definition 2.1}
  %\citor{cauchy1821}
%  ``$\to$" symbol: \citorpc{leathem1905}{13}{section III.11}  % referenced by bromwich1955 page 3
  }
\label{thm:ms_converge}
\index{convergence!metric space}
%---------------------------------------
Let $\topspaceX$ be the \structe{topological space} induced by a metric space $\metspaceX$.
Let $\seqxZ{x_n\in\setX}$ be a  sequence in $\metspaceX$.
\thmbox{
  \mcom{\text{$\seqn{x_n}$ converges to a limit $x$}}{\xref{def:converge}}
  \quad\iff\quad
  \brb{\begin{array}{M}
    for any $\varepsilon\in\Rp$, there exists $\xN\in\Zp$\\ 
    such that for all $n>\xN$,\\
    \indentx
    $\metric{x_n}{x}<\varepsilon$
  \end{array}}
  }
\end{theorem}
\begin{proof}
%\begin{enumerate}
  %\item Proof that $\seqn{x_n}\to x$ $\implies$   $\metric{x_n}{x}<\varepsilon$:
    \begin{align*}
      \seqn{x_n}\to x
        &\iff x_n\in\setU \quad\forall\setU\in\setN_x,\,n>\xN
        && \text{by \prefp{def:converge}}
      \\&\iff \exists \ball{x}{\varepsilon} \st x_n\in\ball{x}{\varepsilon} \forall n>\xN
        && \text{\ifdochas{metric}{by \prefp{lem:ms_open}}}
      \\&\iff \metric{x_n}{x}<\varepsilon
        && \text{\ifdochas{metric}{by \prefp{def:ball}}}
    \end{align*}

%  \item Proof that $\seqn{x_n}\to x$ $\impliedby$ $\metric{x_n}{x}<\varepsilon$:
%    \begin{align*}
%      \metric{x_n}{x}<\varepsilon
%      \\&\implies \seqn{x_n}\to x
%    \end{align*}
%\end{enumerate}
\end{proof}

A sequence that is \prope{convergent} is always \prope{Cauchy} (next theorem).
However, in a metric space, the converse is not true---a sequence that is \prope{convergent} is not in general \prope{Cauchy}.
This is in contrast to the special case of a real sequence in the metric space $\metspace{\R}{\abs{x-y}}$. %with the usual metric ($\metric{x}{y}\eqd\abs{x-y}$).
In this case, all Cauchy sequences are convergent and the Cauchy property is referred to as the 
\hie{Cauchy condition}.\footnote{
  \citerppc{whittaker1915}{13}{15}{2.22}
  }
%---------------------------------------
\begin{theorem}
\footnote{
  \citerpgc{giles1987}{49}{0521359287}{Theorem 3.30},
  \citerpg{rosenlicht}{51}{0486650383},
  \citerppgc{apostol1975}{72}{73}{0201002884}{Theorem 4.6}
  }
\label{thm:convergent==>cauchy}
%---------------------------------------
Let $\seqxZ{x_n\in\setX}$ be a sequence in the metric space $\metspaceX$.
\thmbox{
  \brb{\begin{array}{M}
    $\seqn{x_n}$ is \prope{convergent}\\ 
    in $\metspaceX$
  \end{array}}
  \implies
  \brb{\begin{array}{M}
    $\seqn{x_n}$ is \prope{Cauchy}\\ 
    in $\metspaceX$
  \end{array}}
  \implies
  \brb{\begin{array}{M}
    $\seqn{x_n}$ is \prope{bounded}\\ 
    in $\metspaceX$
  \end{array}}
}
\end{theorem}
\begin{proof}
\begin{enumerate}
  \item Proof that \prope{convergent} $\implies$ \prope{Cauchy}:
    \begin{align*}
      \metric{x_n}{x_m}
        &\le \metric{x_n}{x} + \metric{x}{x_m}
        &&   \text{by \ifdochas{metric}{\prefp{def:metric}} (triangle inequality)}
      \\&<   \varepsilon + \varepsilon
        &&   \text{by left hypothesis}
      \\&=   2\varepsilon
    \end{align*}

  \item Proof that \prope{Cauchy} $\implies$ \prope{bounded}: by \prefpp{lem:cauchy==>bounded}.

\end{enumerate}
\end{proof}

%---------------------------------------
\begin{proposition}
\footnote{
  \citerpg{rosenlicht}{52}{0486650383}
  }
%---------------------------------------
Let $\seq{x_n}{n\in\Z}$ be a sequence in a metric space $\metspaceX$.
Let $\ff:\Z\to\Z$ be a strictly increasing function such that $\ff(n)<\ff(n+1)$.
\propbox{
  \brb{\begin{array}{FMD}
    1. & $\seq{x_n}{n\in\Z}$ is \prope{Cauchy}     & and \\
    2. & $\seq{x_{\ff(n)}}{n\in\Z}$ is \prope{convergent}
  \end{array}}
  \qquad\implies\qquad
  \seq{x_n}{n\in\Z} \text{ is \prope{convergent}.}
  }
\end{proposition}
\begin{proof}
\begin{align*}
  \metric{x_n}{x}
    &= \metric{x}{x_n}
  \\&\le \mcom{\metric{x}{x_{\ff(n)}}}{$<\varepsilon$ by left hypothesis 2} + 
         \mcom{\metric{x_{\ff(n)}}{x_n}}{$<\varepsilon$ by left hypothesis 1}
  \\&<   \varepsilon + \varepsilon
  \\&=   2\varepsilon
  \\\implies& \seqn{x_n} \text{ is convergent.}
\end{align*}
\end{proof}

%---------------------------------------
\begin{proposition}
\footnote{
  \citerpgc{berberian1961}{37}{0821819127}{Theorem~II.4.1}
  %\citerp{pedersen2000}{4}
  }
%---------------------------------------
Let $\metspaceX$ be a \prope{metric space}.
Let $\opair{\R}{\metrican}$ be a metric space of real numbers with the usual metric 
$\metrica{x}{y}\eqd\abs{x-y}$.
\propbox{\begin{array}{llll}
  \mcom{\seqn{x_n} \to x \text{ and } \seqn{y_n} \to y}{convergence in $\metspaceX$}
  \quad \implies \quad
  \mcom{\seqn{\metric{x_n}{y_n}} \to \metric{x}{y}}{convergence in $\opair{\R}{\metrican}$}
  \qquad\scriptstyle
  \forall  x,y,\seq{x_n}{n\in\Z},\seq{y_n}{n\in\Z}\in \metspaceX
\end{array}}
\end{proposition}
\begin{proof}
\begin{align*}
  \metrica{\metric{x}{y}}{\metric{x_n}{y_n}}
    &\eqd \abs{\metric{x}{y}-\metric{x_n}{y_n}}
  \\&\le \brs{\metric{x}{x_n} + \metric{x_n}{y}}-\metric{x_n}{y_n}
    &&   \text{by triangle inequality \ifdochas{metric}{\prefpo{def:metric}}}
  \\&\le \metric{x}{x_n} + \brs{ \metric{x_n}{y_n} + \metric{y_n}{y} } -\metric{x_n}{y_n}
    &&   \text{by triangle inequality \ifdochas{metric}{\prefpo{def:metric}}}
  \\&=   \metric{x}{x_n} + \metric{y}{y_n}
    &&   \text{by definition of metric\ifsxref{metric}{def:metric}}
  \\&<   \varepsilon + \varepsilon
    &&   \text{by left hypothesis}
  \\&=   2\varepsilon
  \\\implies&   \metric{x_n}{y_n} \to \metric{x}{y}
\end{align*}
\end{proof}

\pref{thm:xn_to_xy} (next) demonstrates that, in a \structe{metric space}\ifsxref{metric}{def:metric}, 
if a sequence \prope{converges} \xref{def:converge}, then the limit it converges to is 
\emph{unique}---a sequence cannot converge to more than one limit (in a metric space).
This is in contrast to the more general topological spaces where a sequence \hie{can} converge to more than one limit
\xref{ex:seq_xt31}.

%---------------------------------------
\begin{theorem}[\thmd{Uniqueness of limit}]
\footnote{
  \citerpg{rosenlicht}{46}{0486650383},
  \citerpgc{thomson2008}{32}{143484367X}{Theorem 2.8}
  }
\label{thm:xn_to_xy}
%---------------------------------------
Let $\metspaceX$ be a \prope{metric space}\index{space!metric}.
Let $x,y,\in \setX$ and let $\seqn{x_n}$ be an $\setX$-valued sequence.
\thmbox{
  \mcom{\brb{\seqn{x_n} \to x \text{ and }  \seqn{x_n} \to y}
  \qquad\implies\qquad
  \brb{x=y}}{the \structe{limit} of a \prope{convergent} sequence is \prope{unique}}
  }
\end{theorem}
\begin{proof}
  \begin{enumerate}
    \item Proof that $\metric{x}{y} <2\varepsilon$ for arbitrarily small $\varepsilon>0$:
      \begin{align*}
        \seqn{x_n} \to x \text{ and }  \seqn{x_n} \to y
          &\implies \exists\xN \st \forall n>\xN,\;  \metric{x}{x_n}<\varepsilon \text{ and }  \metric{x_n}{y}<\varepsilon
        \\&\implies \exists\xN \st \forall n>\xN,\;  \metric{x}{y} = \metric{x}{x_n}+\metric{y_n}{y}<2\varepsilon 
        \\&\implies \exists\xN \st \forall n>\xN,\;  \metric{x}{y} <2\varepsilon 
        \\&\implies \metric{x}{y} <2\varepsilon 
      \end{align*}
  
    \item Proof that $\metric{x}{y}=0$:
      \begin{enumerate}
        \item If $\metric{x}{y}>0$, then we could choose an arbitrarily small $\varepsilon$ such that
          \[ \metric{x}{y} >2\varepsilon.\]
        \item But this would contradict the earlier result of $\metric{x}{y} <2\varepsilon$.
        \item Therefore, $\metric{x}{y}=0$ (proof by contradiction).
      \end{enumerate}
  
    \item Therefore, $x=y$ because by the definition of metrics\ifdochas{metric}{ \xrefP{def:metric}}, 
      \[ \metric{x}{y}=0 \qquad\iff\qquad x=y.\]
  
  \end{enumerate}
\end{proof}

%---------------------------------------
\begin{proposition}
\footnote{
  \citerpg{rosenlicht}{46}{0486650383}
  }
%---------------------------------------
Let $\seq{x_n}{n\in\Z}$ be a sequence in a metric space $\metspaceX$.
Let $\ff:\Z\to\Z$ be a strictly increasing function such that $\ff(n)<\ff(n+1)$.
\propbox{
  \mcomr{\seq{x_n}{n\in\Z} \to x}{sequence converges to limit $x$}
  \qquad\implies\qquad
  \mcoml{\seq{x_{\ff(n)}}{n\in\Z} \to x}{subsequence converges to the same limit $x$}
  }
\end{proposition}
\begin{proof}
\begin{align*}
  \seq{x_n}{n\in\Z} \to x
    &\implies \forall \varepsilon>0,\; \exists\xN \st \forall n>\xN,\; \metric{x_n}{x}<\varepsilon
    &&        \text{by \prefp{thm:ms_converge}}
  \\&\implies \forall \varepsilon>0,\; \exists \ff(N) \st \forall \ff(n)>\ff(N),\; \metric{x_{\ff(n)}}{x}<\varepsilon
  \\&\implies \seq{x_{\ff(n)}}{n\in\Z} \to x
    &&        \text{by \prefp{thm:ms_converge}}
\end{align*}
\end{proof}

%=======================================
\subsection{Complete metric spaces}
%=======================================
Even though a convergent sequence is always Cauchy,
not all Cauchy sequences are convergent.
That is, the points of a sequence may diverge less and less from each other
as $n\to\infty$ (Cauchy sequence),
they still may not converge to a single point
(which if they did they would be a convergent sequence).
However, if all the Cauchy sequences in a \prope{metric space}\index{space!metric} do converge,
and they all converge to points inside the metric space,
then that \prope{metric space}\index{space!metric} is called a {\em complete} \prope{metric space}\index{space!metric}.
%---------------------------------------
\begin{definition}
\label{def:ms_complete}
%\label{def:complete}
\footnote{
  \citerpg{rosenlicht}{52}{0486650383}
  }
\index{complete sequences}
\index{sequences!complete}
%---------------------------------------
Let $\metspaceX$ be a \structe{metric space}\ifsxref{metric}{def:metric}.
\defbox{\begin{array}{l>{\ds}l}
  \mc{2}{l}{\text{A sequence $\seqxZ{x_n\in\setX}$
                  is \propd{complete} in $\metspaceX$ if}} 
  \\
  \mcom{
  \text{$\seqn{x_n}$ is \prope{Cauchy} in $\metspaceX$}
  \quad \implies \quad
  \text{$\seqn{x_n}$ is convergent in $\metspaceX$.}
  }{every \prope{Cauchy sequence} in $\metspaceX$ \prope{converges} to a \struct{limit} in $\metspaceX$}
\end{array}}
\end{definition}

%---------------------------------------
\begin{theorem}
\footnote{
  \citerpgc{kubrusly2001}{128}{0817641742}{Theorem 3.40},
  \citerpgc{haaser1991}{75}{0486665097}{6$\cdot$10, 6$\cdot$11 Propositions},
  \citerpgc{bryant1985}{40}{0521318971}{Theorem 3.6, 3.7},
  \citerppg{sutherland1975}{123}{124}{0198531613} %{Proposition 9.23}
  %\citerpgc{rosenlicht}{52}{0486650383}{Proposition}\\
  }
\label{thm:comcls}
%---------------------------------------
%Let $\metspaceX$ be a \structe{metric space}.
%Let $\metspaceA$ be a \structe{subspace} of a metric space $\metspaceX$.
Let $\metspaceX$ be a \structe{metric space}\ifsxref{metric}{def:metric}.
Let $\setA$ be a subset of $\setX$.
Let $\clsA$ be the \structe{closure} \xref{def:clsA} of $\setA$ in $\metspaceX$.
\thmbox{\begin{array}{rcl}
  \brb{\text{$\metspaceA$ is \prope{complete}}}
  &\implies&
  \mcom{\text{$\setA$ is \prope{closed} in $\metspaceX$}}{$\setA=\clsA$}
  \\
  \brb{\begin{array}{FMD}
    1. & $\metspaceX$ is \prope{complete} \xref{def:complete} &  and \\
    2. &  $\setA$ is \prope{closed} in $\metspaceX$ &($\setA=\clsA$)
  \end{array}}
  &\implies&
  \brb{\text{$\metspaceA$ is \prope{complete}}}
\end{array}}
\end{theorem}
\begin{proof}
\begin{enumerate}
\item Proof that \prope{complete} $\implies$ \prope{closed}:
\begin{enumerate}
  \item Proof that $\setA\subseteq\clsA$: \prefp{lem:intAAclsA}
  \item Proof that $\clsA\subseteq\setA$ (proof that $x\in\clsA$ $\implies$ $x\in\setA$): by \prefp{thm:comcls_dspace}
    %\begin{enumerate}
    %  \item Let $x$ be a point in $\clsA$ ($x\in\clsA$).
    %  \item Define a sequence of open balls $\seqn{\ball{x}{\frac{1}{1}},\,\ball{x}{\frac{1}{2}},\,\ball{x}{\frac{1}{3}},\,\ldots}$.
    %  \item Define a sequence of points $\seqn{x_1,\,x_2,\,x_3,\,\ldots}$ such that $x_n\in\ball{x_n}{\frac{1}{n}}\seti\setA$.
    %  \item Then $\seqn{x_n}$ is \prope{convergent} in $\setX$ with limit $x$ by \prefp{def:converge}
    %  \item and  $\seqn{x_n}$ is \prope{Cauchy} in $\setA$ by \prefp{def:cauchy}.
    %  \item By the left hypothesis ($\metspaceA$ is \prope{complete}), $\seqn{x_n}$ is therefore also \prope{convergent} in $\setA$.\\
    %        Let this limit be $y$. Note that $y\in\setA$.\label{item:com_cls_yY}
    %  \item By \prefp{thm:xn_to_xy}, limits are \prope{unique}, so $y=x$. \label{item:com_cls_yx}
    %  \item Because $y\in\setA$ (\pref{item:com_cls_yY}) and $y=x$ (\pref{item:com_cls_yx}), so $x\in\setA$.
    %  \item Therefore, $x\in\clsA\implies x\in\setA$ and $\clsA\subseteq\setA$.
    %\end{enumerate}
\end{enumerate}

\item Proof that \prope{complete} and \prope{closed} $\implies$ \prope{complete}:

  \begin{enumerate}
    \item By left hypothesis 2, $\setA$ is closed in $\metspaceX$.
    \item By \prefpp{thm:insubset_closed} and because $\setA$ is closed in $\metspaceX$,
          sequences converge in $\setA$.
    \item Therefore by \prefpp{def:complete}, $\metspaceA$ is complete.
  \end{enumerate}
\end{enumerate}
\end{proof}

%---------------------------------------
\begin{corollary}
\footnote{
  \citerpgc{kubrusly2001}{128}{0817641742}{Corollary 3.41}
  }
\label{cor:comcomcls}
%---------------------------------------
%Let $\metspaceA$ be a \structe{subspace} of a metric space $\metspaceX$ ($\setA\subseteq\setX$).
Let $\metspaceX$ be a \structe{metric space}\ifsxref{metric}{def:metric}.
Let $\setA$ be a subset of $\setX$.
Let $\clsA$ be the \structe{closure} \xref{def:clsA} of $\setA$ in $\metspaceX$.
\corbox{
  \brb{\text{$\metspaceA$ is \prope{complete}}}
  \qquad\iff\qquad
  \mcom{\text{$\setA$ is \prope{closed} in $\metspaceX$}}{$\setA=\clsA$}
  }
\end{corollary}
\begin{proof}
Note that in this corollary, the metric space $\metspaceX$ is assumed to be \prope{complete}.
\begin{enumerate}
  \item Proof that \prope{complete} $\implies$ \prope{closed}: by \pref{thm:comcls} (1).
  \item Proof that \prope{complete} $\impliedby$ \prope{closed}: by \prope{complete} hypothesis and \pref{thm:comcls} (2).
\end{enumerate}
\end{proof}

%---------------------------------------
\begin{example}
%---------------------------------------
Let $\Q$ be the set of \sete{rational numbers}.
\exbox{%\begin{array}{ll}
  \text{The metric space $(\Q,\metric{x}{y}=\abs{x-y})$ is {\em not} \prope{complete}.}
  %2. & \text{The metric space $(\R,\metric{x}{y}=\abs{x-y})$ {\em is} complete.    }\\
}%\end{array}}
\end{example}
\begin{proof}
Let $\seq{x_n}{n\in\Znn}$ be the sequence of values approximating $\pi$ truncated to $n$
decimal points:
  \[ \seq{x_n}{n\in\Znn} \eqd \seqn{3,\, 3.1,\, 3.14,\, 3.141,\, 3.1415,\, 3.14159,\, 3.141592,\,\ldots} \]
This is a Cauchy sequence.
However, this sequence (and all sequences converging to an irrational number)
does not converge to a rational number ($\Q$) and thus is not in
the metric space $(\Q,\metricn)$ and thus $(\Q,\metricn)$ is {\em not complete}.
%But of course this sequence (and all other like sequences) do converge to real numbers and thus
%$(\R,\metricn)$ is a complete metric space.
\end{proof}

%---------------------------------------
\begin{example}[\exmd{Cauchy's convergence criterion}/\exmd{Cauchy's criterion}]
\footnote{
  \citerpgc{sohrab2003}{54}{0817642110}{Theorem 2.2.5}
  }
%---------------------------------------
Let $\seqxZ{r_n\in\R}$ be a \textbf{real} sequence.
\exbox{
  \text{The metric space $\opair{\seq{r_n}}{\abs{r_n-r_m}}$ is \prope{complete}.}
  }
\end{example}

%---------------------------------------
\begin{theorem}[\thmd{Cantor intersection theorem}]
\footnote{
  \citerp{davis2005}{28},
  \citerp{hausdorff1937e}{150}
  }
\label{thm:cantor_int}
\index{Cantor intersection theorem}
\index{theorems!Cantor intersection}
%---------------------------------------
Let $\metspaceX$ be a complete \prope{metric space}\index{space!metric},
$\seqnZ{\setA_n}$ a sequence with each $\setA_n\in\psetX$, and $\seto{\setA}$ the number of elements in $\setA$.
\thmbox{
  \brbr{\begin{array}{FMCD}
    1. & $\metspaceX$ is \prope{complete}   &                 & and \\
    2. & $\setA_n$ is \prope{closed}                     & \forall n\in\Zp & and \\
    3. & $\diam \setA_{n+1} \le \diam \setA_n$           & \forall n\in\Zp & and \\
    4. & $\diam \setA_n \to 0$                           &
  \end{array}}
  \qquad\implies\qquad
  \brb{\seto{\ds\setopi_{n\in\Zp} \setA_n } = 1}
  }
\end{theorem}
\begin{proof}
By \prefp{thm:cantor_int_dspace}
%\begin{enumerate}
%\item Proof that $\seto{\setopi_{n\in\Z} \setA_n}<2$:
%  \begin{enumerate}
%    \item Let $\setA\eqd\seti \setA_n$.
%    \item $x\ne y \land \{x,y\}\in \setA \implies \metric{x}{y}>0 \land \{x,y\}\subseteq \setA_n \forall n$
%    \item $\exists n \st \diam \setA_n < \metric{x}{y}$ by left hypothesis
%    \item $\implies \exists n \st \sup\set{\metric{x}{y}}{x,y\in \setA_n}<\metric{x}{y}$
%    \item This is a contradiction, so $\{x,y\}\notin \setA$ and $\seto{\setopi \setA_n}<2$.
%  \end{enumerate}
%
%\item Proof that $|\seti \setA_n|\ge1$:
%  \begin{enumerate}
%    \item Let $x_n\in \setA_n$ and $x_m\in \setA_m$
%    \item $\forall \varepsilon,\; \exists\xN\in\Zp \st \setA_N < \varepsilon$
%    \item $\forall m,n>\xN,\; x_n\in \setA_n\subseteq \setA_N$ and $x_m\in \setA_m\subseteq \setA_N$
%    \item $\metric{x_n}{x_m}\le\diam \setA_N < \varepsilon \implies \{x_n\}$ is a Cauchy sequence
%    \item Because $\{x_n\}$ is complete, $x_n\to x$.
%    \item $\implies x\in cls\setA_n = \setA_n$
%    \item $\implies |\setA_n|\ge1$
%  \end{enumerate}
%\end{enumerate}
\end{proof}





%%=======================================
%\section{Metric spaces of sequences}
%%=======================================
%%--------------------------------------
%\begin{definition}[Baire space]
%\footnote{
%  \citerpp{hausdorff1937e}{117}{118}
%  }
%\index{Baire space}
%%--------------------------------------
%Let $\vx\eqd\seq{x_n}{n\in\Zp}$ and $\vy\eqd\seq{y_n}{n\in\Zp}$ be sequences.
%\defboxp{
%  $\metric{\vx}{\vy} \eqd \left\{\begin{array}{ll}
%    0  & \text{for } \vx=\vy \\
%    \frac{1}{m} & \text{for } \vx\ne \vy
%  \end{array}\right.$
%  \\
%  where
%  $x_m \ne  y_m$ and $x_n=y_n$ for all $1\le n \le m-1$
%  }
%\end{definition}

%=======================================
\section{Sequences on normed linear spaces}
%=======================================
%=======================================
\subsection{Convergence in normed linear spaces}
%=======================================
\prefpp{thm:ms_converge} defines convergence in a general \structe{metric space}\ifsxref{metric}{def:metric}.
All \structe{normed linear space}s \xref{def:norm} are metric spaces, so they inherit this definition with the 
\fncte{metric induced by the norm} \xref{def:d=norm}.

%---------------------------------------
\begin{definition}
\footnote{
  \citerpg{bachman1966}{247}{0486402517},
  \citerpgc{katznelson2004}{67}{0521543592}{section 1.1}
  }
\label{def:strong_converge}
%---------------------------------------
Let $\normspaceX$ be a normed linear space .
Let the metric $\metricn$ be defined as $\metric{\vx}{\vy}\eqd\norm{\vx-\vy}$.
\defboxp{ %\begin{array}{M}
    A sequence $\seqxZ{\vx_n\in\setX}$ \hid{converges in norm} or \hid{converges strongly} to the \hid{limit} $\vx$ if
    $\seqxZ{\vx_n}$ converges to the limit $\vx$ in the metric space $\metspaceX$.
    That is, a sequence $\seqxZ{x_n}$ \hid{converges strongly} in the normed linear space 
    $\normspaceX$ to the \hid{limit} $\vx\in\setX$ if
      for any $\varepsilon\in\Rp$ there exists $\xN\in\Z$ such that 
    \\\indentx$\ds\norm{\vx_n-\vx}<\varepsilon \quad\forall n>\xN.$
    \\This mode of convergence is called \hid{strong convergence}.
  }
%\end{array}}
\end{definition}

%---------------------------------------
\begin{definition}
\footnote{
  \citerpgc{bachman1966}{231}{0486402517}{Definition 14.1}
  }
\label{def:weak_converge}
%---------------------------------------
Let $\normspaceX$ be a normed linear space .
Let the metric $\metricn$ be defined as $\metric{\vx}{\vy}\eqd\norm{\vx-\vy}$.
\defboxp{ %\begin{array}{M}
    A sequence $\seqxZ{\vx_n\in\setX}$ \hid{converges weakly} to the \hid{limit} $\vx$ if for every functional $\ff\in\clFxf$,
    $\seqxZ{\ff(\vx_n)}$ converges to the limit $\ff(\vx)$ in the metric space $\metspaceX$.
    That is, a sequence $\seqxZ{x_n}$ \hid{converges weakly} in the normed linear space 
    $\normspaceX$ to the \hid{limit} $\ff(\vx)$ if
    for every functional $\ff\in\clFxf$ and for any $\varepsilon\in\Rp$ there exists $\xN\in\Z$ such that 
    \\\indentx$\ds\norm{\ff(\vx_n)-\ff(\vx)}<\varepsilon \quad\forall n>\xN.$
    \\This mode of convergence is called \hid{weak convergence}.
  }
%\end{array}}
\end{definition}




%=======================================
\subsection{Bounded sequences}
%=======================================

%%---------------------------------------
%\begin{definition}
%\footnote{
%  \citerpgc{heil2011}{5}{9780817646868}{Definition 1.4}
%  }
%%---------------------------------------
%Let $\seqxZ{\vx_n\in\setX}$ be a sequence in a normed linear space $\normspaceX$.
%\defbox{
%  \begin{array}{MrclC}
%    $\seqn{\vx_n}$ is \hid{bounded below} if & \inf\norm{\vx_n} &>& 0      & \forall n\in\Z\\
%    $\seqn{\vx_n}$ is \hid{bounded above} if & \sup\norm{\vx_n} &<& \infty & \forall n\in\Z\\
%    $\seqn{\vx_n}$ is \hid{normalized}    if & \norm{\vx_n}     &=& 1      & \forall n\in\Z
%  \end{array}
%  }
%\end{definition}
%
%%---------------------------------------
%\begin{lemma}
%\footnote{
%  \citerpgc{heil2011}{8}{9780817646868}{Exercise 1.2(b)}\\
%  \citerpc{webber1966}{141}{Theorem 6--13}
%  }
%%---------------------------------------
%Let $\seqxZ{\vx_n\in\setX}$ be a sequence in a normed linear space $\normspaceX$.
%\lembox{
%  \text{$\seqn{\vx_n}$ is \prope{Cauchy} $\implies$ $\seqn{\vx_n}$ is \prope{bounded above}.}
%  }
%\end{lemma}
%
%%---------------------------------------
%\begin{lemma}
%\footnote{
%  \citerpgc{amann2005}{179}{3764371536}{Lemma 6.11}
%  }
%%---------------------------------------
%Let $\seqxZ{\vx_n\in\setX}$ be a sequence in a normed linear space $\normspaceX$.
%\lembox{\begin{array}{McM}
%  $\seqn{\vx_n}$ is \prope{bounded above} and \prope{increasing} &\implies& $\seqn{\vx_n}$ is \prope{Cauchy}.\\
%  $\seqn{\vx_n}$ is \prope{bounded below} and \prope{decreasing} &\implies& $\seqn{\vx_n}$ is \prope{Cauchy}.
%\end{array}}
%\end{lemma}

%%--------------------------------------
%\begin{definition}
%\footnote{
%  \citerpgc{bachman1966}{231}{0486402517}{Definition 14.1}\\
%  %\citerpgc{bachman2000fa}{231}{0486402517}{Definition 14.1}\\
%  \citerpgc{michel1993}{373}{048667598X}{Definition 6.10.1}\\
%  \citerpgc{heil2011}{22}{0817646868}{Definition 1.28}\\
%  \citerpg{riesz1990}{150}{0486662896}
%  }
%%--------------------------------------
%Let $\seqxZ{\vx_n\in\setX}$ be a sequence of vectors in a normed linear space\\
%$\spO\eqd\normspaceX$.
%Let $\spOd$ be the dual space of $\spO$.
%\defbox{\begin{array}{M}
%  The sequence $\seqn{\vx_n}$ \hid{converges in the norm} to a vector $\vx$ if \quad$\ds\lim_{n\to\infty} \norm{\vx-\vx_n}=0$.\\
%  In this case, $\seqn{\vx_n}$ is also said to \hid{converge strongly} to $\vx$, and to be \hid{strongly convergent}.\\
%  The sequence $\seqn{\vx_n}$ \hid{converges weakly} to a vector $\vx$ if \quad
%  $\ds\lim_{n\to\infty} \ff(\vx_n)=\ff(\vx)$ $\forall \ff\in\spOd$.
%\end{array}}
%\end{definition}




%======================================
\subsection{Complete normed linear spaces}
\label{sec:vsBanach}
%======================================


\qboxnpq
  {\href{http://en.wikipedia.org/wiki/Norbert_Wiener}{Norbert Wiener}
   \href{http://www-history.mcs.st-andrews.ac.uk/Timelines/TimelineG.html}{(1894--1964)}, 
   \href{http://www-history.mcs.st-andrews.ac.uk/BirthplaceMaps/Places/USA.html}{American mathematician}
    \index{Wiener, Norbert}
    \index{quotes!Wiener, Norbert}
    \footnotemark
  }
  {../common/people/wiener.jpg}
  {At that time, however, the theory seemed to me to contain for the immediate future
   nothing but some decades of rather formal and thin work.
   By this I do not mean to reproach the work of Banach himself
   but that of the many inferior writers, hungry for easy doctors' theses,
   who were drawn to it.
   As I foresaw, it was this class of writers that was first attracted 
   to the theory of Banach spaces.}
  \footnotetext{\begin{tabular}[t]{ll}
    quote: & \citerpp{wiener}{63}{64}, \citerp{werner}{41} \\
    image: & \url{http://www-history.mcs.st-andrews.ac.uk/PictDisplay/Wiener_Norbert.html}
  \end{tabular}}




%Much of analysis is concerned with the topic of \hie{convergence}. % and \hie{continuity}.
%Convergence is defined in terms of sequences.
%%And much of the convergence and continuity theory is developed using the idea of
%%sequences---sequences of functions, sequences of numbers, etc.
%Convergence implies the idea of elements of a sequence getting arbitrarily close to some ``limit" value. 
%This idea of ``closeness" requires some way of determining just how close two elements are to each other.
%One of the most general ways to way to do this is in
%a \prope{metric space} $\metspaceX$ using that space's metric $\metricn$.
%%If every Cauchy sequence in $\metspaceX$ converges, then $\metspaceX$ is \prope{complete} \xrefP{def:ms_complete}.
%If the metric space is also a normed linear space $\opair{\spV}{\normn}$, then
%we can use the norm to generate the metric. % \xrefP{def:d=norm}.
%If the norm space is complete, then it is called a \hie{Banach space}.
%Banach spaces are one of the most fundamental and important structures in mathematics.

%---------------------------------------
\begin{theorem}
\footnote{
  \citerppgc{ponnusamy2002}{94}{96}{0849317177}{``2.59. Proposition."; in the context of \structe{metric space}s;
  includes the ``{\scs\thme{inverse image characterization of continuity}}"
  and ``{\scs\thme{sequential characterization of continuity}}" terminology; this terminology does not seem to be widely 
  used in the literature in general, but has been adopted for use in this text}
  }
\label{thm:limcont}
%---------------------------------------
Let $\topspaceX$ and $\topspaceYS$ be a \structe{topological space}s.
Let $\ff$ be a function in $\clF{\topspaceX}{\topspaceYS}$.
\thmbox{
  \mcom{\brb{\begin{tabstr}{0.75}\begin{array}{N}
    $\ff$ is \prope{continuous}  in $\clF{\topspaceX}{\topspaceYS}$\\
    \xref{def:continuous}
  \end{array}\end{tabstr}}}{\thme{inverse image characterization of continuity}}
  \quad\iff\quad
  \mcom{\brb{\begin{tabstr}{0.75}\begin{array}{N}
    $\seqn{x_n}\to x \implies \ff(\seqn{x_n})\to \ff(x)$\\
    \xref{def:converge}
  \end{array}\end{tabstr}}}{\thme{sequential characterization of continuity}}
  }
\end{theorem}
\begin{proof}
\begin{enumerate}
  \item Proof for the $\implies$ case (proof by contradiction):
    \begin{enumerate}
      \item Let $\setU$ be an \structe{open set} in $\topspaceY$ that contains $\ff(x)$ but for which 
            there exists no $\xN$ such that $\ff(x_n)\in\setU$ for all $n>\xN$.
      \item Note that the set $\ffi(\setU)$ is also \structe{open} by the \prope{continuity} hypothesis.
      \item If $\seqn{x_n}\to x$, then
      \begin{align*}
        &\ff(\seqn{x_n})\notto \ff(x)
        \\&\implies \text{there exists no $\xN$ such that $\ff(x_n)\in\setU$ for all $n>\xN$}
          &&\text{by \prefpp{def:converge}}
         %&&\text{by definition of \prope{convergence} \xref{def:converge}}
        \\&\implies \text{there exists no $\xM$ such that $x_n\in\ffi(\setU)$ for all $n>\xM$}
          &&\text{by definition of $\ffi$}
        \\&\mathrlap{\implies \seqn{x_n}\notto x 
          \qquad\text{by \prope{continuity} hypothesis and def. of \prope{convergence} \xref{def:converge}}}
        \\&\implies \text{contradiction of $\seqn{x_n}\to x$ hypothesis}
        \\&\implies \ff(\seqn{x_n})\to \ff(x)
      \end{align*}
    \end{enumerate}

  \item Proof for the $\impliedby$ case (proof by contradiction):
    \begin{enumerate}
      \item Let $\setD$ be a \prope{closed} set in $\topspaceYS$.
      \item Suppose $\ffi(\setD)$ is \prope{not closed}\ldots \label{item:limcont_suppose}
      \item then by the \thme{closed set theorem} \xref{thm:cst}, there must exist 
            a \prope{convergent} sequence $\seqn{x_n}$ in $\topspaceX$, 
            but with limit $x$ \emph{not} in $\ffi(\setD)$. \label{item:limcont_x}
          %\\&\implies \text{there exists convergent $\seqn{x_n}$ in $\topspaceX$ with limit $x$, \emph{but} $x\notin\ffi(\setD)$}
      \item Note that $\ff(x)$ must be in $\setD$. Proof:
        \begin{enumerate}
          \item by definition of $\setD$ and $\ff$, $\ff(\seqn{x_n})$ is in $\setD$
          \item by left hypothesis, the sequence $\ff(\seqn{x_n})$ is \prope{convergent} with limit $\ff(x)$
          \item by \thme{closed set theorem} \xref{thm:cst}, $\ff(x)$ must be in $\setD$.
        \end{enumerate}
      \item Because $\ff(x)\in\setD$, it must be true that $x\in\ffi(\setD)$.
      \item But this is a contradiction to \prefpp{item:limcont_x}, 
            and so \prefpp{item:limcont_suppose} must be wrong, and $\ffi(\setD)$ must be \prope{closed}.
      \item And so by \prefpp{thm:continuous}, $\ff$ is \prope{continuous}.
    \end{enumerate}
  \end{enumerate}
\end{proof}





%---------------------------------------
\begin{definition}
\footnote{
  \citerpgc{bachman1966}{112}{0486402517}{Definition 8.1},
  %\citerpgc{bachman2000fa}{112}{0486402517}{Definition 8.1},
  \citerpc{banach1932}{53}{``espace du type (B)" (space type (B))},
  \citerp{banach1932e}{33}
  }
\index{space!Banach}
\label{def:banach}
%---------------------------------------
Let $\spO\eqd\normspaceX$ be a normed linear space.
\defboxp{ %\begin{array}{M}
  The space \structe{normed linear space} $\spO$ is a \structd{Banach space} if it is \prope{complete} 
  with respect to the metric $\metric{\vx}{\vy}\eqd\norm{\vy-\vx}$.
  %A Banach space is also called a \hid{B-space} or an \hid{l.n.c space} (linear normed complete space).
  }
%\end{array}}
\end{definition}

%\ifexclude{wsd}{
%=======================================
\subsection{The $l_p$ spaces}
%=======================================

%--------------------------------------
\begin{definition}
\footnote{
  \citerp{carothers2000}{44}
  }
\label{def:lp}
%--------------------------------------
Let $\seq{x_n\in\R}{n\in\Z}$ be a real sequence.
\defbox{\begin{array}{l@{\qquad}rc>{\ds}lC}
  \mc{5}{l}{\text{The space $\splpF$ and space $\splp{\infty}{\F}$ are defined as}}
  \\&\splpF            &\eqd& \set{\seq{x_n}{n\in\Z}}{\sum_{n\in\Z} \abs{x_n}^p < \infty} & \forall 1\le p<\infty 
  \\&\splp{\infty}{\F} &\eqd& \set{\seq{x_n}{n\in\Z}}{\sup\seqn{x_n}<\infty}
\end{array}}
\end{definition}

%--------------------------------------
\begin{lemma}
%--------------------------------------
Let $\seq{x_n\in\F}{n\in\Z}$ and $\seq{x_n\in\F}{n\in\Z}$ be sequences over a field $\F$.
\lembox{
  \seqn{x_n},\seqn{y_n}\in\splpF \qquad\implies\qquad \brp{\seqn{x_n}+\seqn{y_n}}\in \splpF
  \qquad\forall p\in\intcc{1}{\infty}
  }
\end{lemma}
\begin{proof}
\begin{enumerate}
  \item Proof for $p=1$:
    \begin{align*}
      \sum_{n\in\Z} \abs{x_n+y_n}^1
        &= \sum_{n\in\Z} \abs{x_n+y_n}
      \\&\le \sum_{n\in\Z} \brp{\abs{x_n}+\abs{y_n}}
        && \text{by norm properties of $\normn$\ifdochas{numsys}{ \prefpo{thm:C_norm}}}
      \\&= \sum_{n\in\Z} \abs{x_n}^1+ \sum_{n\in\Z} \abs{y_n}^1
      \\&< \infty
        && \text{because $\vx,\vy\in \splF$}
    \end{align*}

  \item Proof for $1<p<\infty$: Let $\norm{\vx}_p\eqd\brp{\sum_{n\in\Z} \abs{x_n}^p}^\frac{1}{p}$
    \begin{align*}
      \sum_{n\in\Z} \abs{x_n+y_n}^p
        &= \mcom{\brp{\brp{\sum_{n\in\Z} \abs{x_n+y_n}^p}^\frac{1}{p}}}{$\normn_p$}^p
      \\&= \norm{\vx+\vy}_p^p
        && \text{by definition of $\normn_p$ \prefpo{def:lp_norm}}
      \\&\le \brp{\norm{\vx}_p + \norm{\vy}_p}^p
        && \text{by Minkowski's inequality \prefpo{thm:lp_minkowski}}
      \\&\le \infty
        && \text{by $\vx,\vy\in \splpF$ hypothesis}
    \end{align*}

  \item Proof for $p=\infty$:
    \begin{align*}
      \sup\set{x_n + y_n}{n\in\Z}
        &\le \sup\seqn{x_n} + \sup\seqn{y_n}
      \\&\le \infty
        && \text{by $\vx,\vy\in \splpF$ hypothesis}
    \end{align*}
\end{enumerate}
\end{proof}

%%--------------------------------------
%\begin{theorem}
%%--------------------------------------
%The set of all sequences with addition and multiplication as defined above
%forms a linear space.
%\end{theorem}


%--------------------------------------
\begin{definition}
\label{def:lp_norm}
%--------------------------------------
Let $\seqxZ{x_n}$ be a sequence in the space $\splpF$.
\defbox{
  \text{The \hid{$\splpF$ norm} $\hxs{\norm{\seqn{x_n}}_p}$ of $\seqn{x_n}$ is defined as}
  \norm{\seqn{x_n}}_p \eqd  \brp{\sum_{n\in\Z} \abs{x_n}^p}^\frac{1}{p}   
  \qquad\text{for $p\in\intcc{1}{\infty}$}
  }
\end{definition}

%--------------------------------------
\begin{proposition}
\label{prop:lp_norm}
%--------------------------------------
Let $\norm{\seqn{x_n}}_p$ be the $\splpF$ norm of a sequence $\seq{x_n}{n\in\Z}$ in the space $\splpF$.
\propbox{
  \text{$\norm{\seqn{x_n}}_p$ is a norm.}
  }
\end{proposition}
\begin{proof}
\begin{align*}
  \intertext{$\quad\imark$ Proof that $\normn_p\ge0$:}
    \norm{\vx}
      &\eqd \brp{\sum_{n\in\Z} \abs{x_n}}^\frac{1}{p}
      &&    \text{by definition of $\norm{\cdot}$}
    \\&\ge  \sum_{i=1}^n 0
      &&    \ifdochas{numsys}{\text{by \prefp{def:abs}}}
    \\&=    0
  \\
  \intertext{$\quad\imark$ Proof that $\norm{\vx}=0\implies\vx=\vzero$:}
    0
      &= \norm{\vx}
      && \text{by left hypothesis}
    \\&= \sum_{i=1}^n \abs{x_i}
      && \text{by definition of $\norm{\cdot}$}
    \\&\implies x_i=0\quad i=1,2,\ldots,n
    \\&\implies \vx=\vzero
      && \text{by definition of $\vx$}
  \\
  \intertext{$\quad\imark$ Proof that $\norm{\vx}=0\impliedby\vx=\vzero$:}
    \norm{\vx}
      &= \sum_{i=1}^n \abs{x_i}
      && \text{by definition of $\norm{\cdot}$}
    \\&= \sum_{i=1}^n \abs{0}
      && \text{by right hypothesis}
    \\&= 0
  \\
  \intertext{$\quad\imark$ Proof that $\norm{\alpha\vx}=\abs{\alpha}\norm{\vx}$:}
    \norm{\alpha\vx}
      &= \sum_{i=1}^n \abs{\alpha x_i}
      && \text{by definition of $\norm{\cdot}$}
    \\&= \sum_{i=1}^n \abs{\alpha}\abs{x_i}
      && \ifdochas{numsys}{\text{by \prefp{def:abs}}}
    \\&= \abs{\alpha}\sum_{i=1}^n \abs{x_i}
    \\&= \abs{\alpha}\norm{\vx}
      && \text{by definition of $\vx$}
  \\
  \intertext{$\quad\imark$ Proof that $\norm{\vx+\vy}\le\norm{\vx}+\norm{\vy}$: by \thme{Minkowski's Inequality} {\xrefP{thm:lp_minkowski}}} 
\end{align*}
\end{proof}

%} %end wsd exclude

%======================================
\section{Complete inner-product spaces}
\label{sec:vsHilbert}
%======================================
%%--------------------------------------
%\begin{definition}
%\label{def:hilbert}
%%--------------------------------------
%Let $\spH\eqd\inprodspaceX$ be an \structe{inner-product space}.
%Let $\norm{\vx}\eqd\sqrt{\inprod{\vx}{\vx}}$.
%\defbox{
%  \text{$\spH$ is a \hid{Hilbert space} if $\normspaceX$ is a \structe{Banach space}.\footnotemark}
%  }
%\footnotetext{\structe{Banach space}: \prefp{def:banach}}
%\end{definition}

%--------------------------------------
\begin{definition}
\label{def:H-space}
\label{def:hspace}
\label{def:hilbert}
\index{space!Hilbert}
\index{Hilbert space}
\footnote{
  \citorp{vonNeumann1929}{55},
  \hspace{2ex}``Den abstrakten hilbertschen raum nennen wir $\mathcal{H}$"
  \hspace{2ex}(``we call the abstract Hilbert space $\mathcal{H}$"),
  \citerp{ab}{288}
  }
%--------------------------------------
Let $\spO\eqd\inprodspaceX$ be an inner-product space.\footnote{\prope{complete}: \prefp{def:ms_complete}}
\defboxp{ %\begin{array}{M}
  The inner-product space $\spO$ is a \hid{Hilbert space} if it is \prope{complete} with respect to the metric
  $\ds\metric{\vx}{\vy}\eqd\norm{\vx-\vy}\eqd\sqrt{\inprod{\vx-\vy}{\vx-\vy}}.$
  }
%\end{array}}
\end{definition}

%--------------------------------------
\begin{theorem}[\thmd{Complemented-subspace theorem}]
\label{thm:comsub}
\footnote{
  \citer{linden1971},
  \citerp{day}{157}
  }
\index{complemented-subspace theorem}
\index{theorems!complemented-subspace theorem}
%--------------------------------------
Let $\spB\eqd\BspaceX$ be a Banach space.
%Let $\spD^c$ be the complement in $\spX$ of some linear subspace $\spD$ of $\spX$.
\thmbox{
  \brb{\begin{array}{M}
    Every closed linear subspace $\spD$ in $\spB$ \\
    has a complement $\spD^c$ in $\spB$
  \end{array}}
  \quad\implies\quad
  \brb{\begin{array}{M}
    $\spB$ is isomorphic \\
    to a Hilbert  space $\spH$
  \end{array}}
  }
\end{theorem}


%=======================================
\section{Sequences of functions}
%=======================================
For a sequence of real numbers in a metric space, the concept of convergence is well defined and unambiguous
\xref{thm:ms_converge}.
But for sequences of functions $\seqxZ{\ff_n(x)}$, on the other hand, there are several different types or ``modes" of convergence.
Two of the most common modes are \prope{pointwise convergence} \xref{def:conpnt} and \prope{uniform convergence} \xref{def:conuni}. 
Both of these are defined in a metric space.
In both of these, the value $\xN$ beyond which the sequence becomes sufficiently ``close" 
to the limit $\ff(x)$ depends on a distance parameter $\varepsilon$.
The difference between the two modes is that in pointwise convergence, the value $\xN$ also depends on the value $x$ of 
the limit $\ff(x)$;
whereas in uniform convergence, the value $\xN$ does not depend on $x$.
%However, there are other modes of convergence besides these two. 
%In fact, there are dozens of types of convergence that appear in the literature.\cittrppgc{tao2010}{117}{131}{0821852787}{section 1.9}

%---------------------------------------
\begin{definition}
\footnote{
  \citerppgc{tao2011}{94}{96}{0821869191}{section 1.5},
  \citerpgc{thomson2008}{368}{143484367X}{Definition 9.3},
  \citerpgc{tao2010}{117}{0821852787}{Example 1.9.3}
  }
\label{def:conpnt}
%---------------------------------------
Let $\seqxZ{\ff_n(x)}$ be a sequence of functions in a \structe{metric space} $\metspaceX$. 
\defboxp{ %\begin{array}{M}
  %For a fixed value of $x$, the sequence $\seqn{\ff_n(x)}$ \hid{converges pointwise at $x$} to a \hid{limit} $\ff(x)$ if
  The sequence $\seqn{\ff_n(x)}$ \hid{converges pointwise} to a \hid{limit} $\ff(x)$ if
  for each $\varepsilon\in\Rp$ and for each $x\in\setX$ there exists an $\xN\in\Zp$ (dependent on $x$) such that 
  \\\indentx$\metric{\ff_n(x)}{\ff(x)}<\varepsilon$.
  }
%\end{array}}
\end{definition}

%---------------------------------------
\begin{definition}
\footnote{
  \citerppgc{tao2011}{94}{96}{0821869191}{section 1.5},
  \citerppg{thomson2008}{373}{374}{143484367X}, %{Definition 9.9}\\
  \citerpgc{tao2010}{117}{0821852787}{Example 1.9.4}
  }
\label{def:conuni}
%---------------------------------------
Let $\seqxZ{\ff_n(x)}$ be a sequence of functions in a \structe{metric space} $\metspaceX$. 
\defbox{\begin{array}{M}
  The sequence $\seqn{\ff_n(x)}$ \hid{converges uniformly} to a \hid{limit} $\ff(x)$ if
  \\for each $\varepsilon\in\Rp$ there exists an $\xN\in\Zp$ (independent of $x$) such that 
  \\\indentx$\metric{\ff_n(x)}{\ff(x)}<\varepsilon\quad$ for all $x\in\setX$.
\end{array}}
\end{definition}

%---------------------------------------
\begin{theorem}
%---------------------------------------
Let $\seqxZ{\ff_n(x)}$ be a sequence of functions in a \structe{metric space} $\metspaceX$. 
\thmbox{
  \text{$\seqn{\ff_n(x)}$ \prope{converges uniformly}} 
  \qquad\implies\qquad
  \text{$\seqn{\ff_n(x)}$ \prope{converges pointwise}} 
  }
\end{theorem}
\begin{proof}
  This follows directly from the definition of \prope{uniform convergence} \xrefP{def:conuni}
  and the definition of \prope{pointwise convergence} \xrefP{def:conpnt}.
\end{proof}

%%---------------------------------------
%\begin{theorem}
%%---------------------------------------
%Let $\seqxZ{\ff_n(x)}$ be a sequence of functions in a \structe{metric space} $\metspaceX$. 
%\thmbox{
%  \text{\seqn{\ff_n(x)} \prope{converges uniformly}} 
%  \qquad\implies\qquad
%  \brb{\begin{array}{F>{\ds}lc>{\ds}lD}
%    1. &      \mcom{\ds\lim_{n\to\infty} \opddx \ff_n(x)}{limit of derivative} 
%       &\neq& \mcom{\ds\opddx \lim_{n\to\infty} \ff_n(x)}{derivative of limit}
%    \\
%  \end{array}}
%  }
%\end{theorem}
%%\begin{proof}
%%\end{proof}


