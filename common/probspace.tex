%============================================================================
% Daniel J. Greenhoe
% LaTeX file
%============================================================================
%======================================
\chapter{Probability Space}
\label{chp:prbspace}
%======================================
\begin{figure}[th]
  \centering
  %============================================================================
% Daniel J. Greenhoe
% LaTeX file
%============================================================================
\begin{pspicture}(-7,0.5)(5,9)%
  %-------------------------------------
  % settings
  %-------------------------------------
   \psset{%
    arrowsize=4mm,
    arrowlength=0.6,
    arrowinset=0.1,
    %linecolor=blue,
    %linewidth=1pt,
     cornersize=relative,
     framearc=0.25,
    % gridcolor=graph,
    % subgriddiv=1,
    % gridlabels=4pt,
    % gridwidth=0.2pt,
    xunit=1\latunit,
    yunit=1.25\latunit,
     }%
  %-------------------------------------
  % nodes
  %-------------------------------------
   \begin{tabstr}{0.75}%
     \rput(-1, 8){\rnode{spaces}    {\psframebox{\begin{tabular}{c}abstract spaces\end{tabular}}}}%
     \rput(-4, 7){\rnode{lin}       {\psframebox{\begin{tabular}{c}linear spaces\ifnxref{vector}{def:vspace}\end{tabular}}}}%
     \rput( 2, 7){\rnode{top}       {\psframebox{\begin{tabular}{c}topological spaces\ifnxref{vstopo}{def:toplinspace}\end{tabular}}}}%
     \rput( 2, 6){\rnode{metric}    {\psframebox{\begin{tabular}{c}metric spaces\ifnxref{metric}{def:metric}\end{tabular}}}}%
     \rput( 2, 4.5){\rnode{commetric} {\psframebox{\begin{tabular}{c}complete metric spaces\ifnxref{seq}{def:ms_complete}\end{tabular}}}}%
     \rput(-4, 5){\rnode{metriclin} {\psframebox{\begin{tabular}{c}metric linear spaces\end{tabular}}}}%
     \rput(-4, 4){\rnode{normlin}   {\psframebox{\begin{tabular}{c}normed linear spaces\ifnxref{vsnorm}{def:norm}\end{tabular}}}}%
     \rput( 2, 3){\rnode{banach}    {\psframebox{\begin{tabular}{c}Banach spaces\ifnxref{seq}{def:banach}\end{tabular}}}}%
     \rput(-4, 3){\rnode{inprod}    {\psframebox{\begin{tabular}{c}inner-product spaces\ifnxref{vsinprod}{def:inprod}\end{tabular}}}}%
     \rput( 2, 2){\rnode{hilbert}   {\psframebox{\begin{tabular}{c}Hilbert spaces\ifnxref{seq}{def:hilbert}\end{tabular}}}}%
     \rput( 2, 1){\rnode{zero}      {\psframebox{\begin{tabular}{c}$\spZero$\ifnxref{subspace}{prop:subspace_0X}\end{tabular}}}}%
   \end{tabstr}%
  %-------------------------------------
  % connecting lines/arrows
  %-------------------------------------
   %\ncline[doubleline=true]{->}{lin}{spaces} double arrow seems to cause trouble for xdvipdfmx
   \ncline{lin}{spaces}
   \ncline{top}{spaces}%
   \ncline{com}{spaces}%
   \ncline{metriclin}{lin}%
   \ncline{normlin}{metriclin}%
   \ncline{inprod}{normlin}
   \ncline{banach}{normlin}%
   \ncline{hilbert}{inprod}%
   \ncline{hilbert}{banach}%
   \ncline{zero}{hilbert}%
   \ncline{metric}{top}%
   \ncline{metriclin}{metric}
   \ncline{commetric}{metric}%
   \ncline{banach}{commetric}%
  %-------------------------------------
  % labeling
  %-------------------------------------
   %\psccurve[linestyle=dashed,linecolor=red,fillstyle=none]%
   %  (0,-5)(20,4)(70,60)(20,60)(15,55)(-5,38)(-20,35)(-10,20)(-25,5)%
   %\psline[linecolor=red]{->}(60,75)(60,68)%
   %\uput[135](60,75){complete spaces}%
   %\psline[linecolor=red]{->}(26,80)(15,74)%
   %\psline[linecolor=red]{->}(30,80)(30,62.5)%
   %\uput[90](28,80){analytic spaces}%
   %
   %\psgrid[unit=10mm](-8,-1)(8,9)%
\end{pspicture}%

  \caption{Lattice of mathematical spaces\label{fig:vector_spaces}}
\end{figure}%

\qboxnps
  {
    Blaise Pascal (1623--1662), mathematician
    \index{Pascal, Blaise}
    \index{quotes!Pascal, Blaise}
    \footnotemark
  }
  {../common/people/pascal.jpg}
  {It is not certain that everything is certain.}
  \footnotetext{\begin{tabular}[t]{ll}
    quote: & \url{http://en.wikiquote.org/wiki/Blaise_Pascal} \\
    image: & \url{http://en.wikipedia.org/wiki/Image:Blaise_pascal.jpg}
  \end{tabular}}

%=======================================
%\section{A generalized probability function}
%=======================================
%The traditional probability function $\psP$ is defined on a \structe{Boolean lattice}\ifsxref{boolean}{def:boolean}.
%=======================================
\section{Probability functions}
%=======================================
%%---------------------------------------
%\begin{definition}
%\footnote{
%  \citerppgc{pap1995}{8}{9}{0792336585}{Definition 2.3(13)}
%  }
%\label{def:prob}
%\label{def:ps}
%\label{def:Psp}
%%---------------------------------------
%%Let $\intcc{0}{1}$ be a \structe{real interval}.
%%Let $\psL\eqd\logicX$ be a \structe{logic} \xref{def:logic}.
%Let $\psL\eqd\latocX$ be a \structe{orthocomplemented lattice}\ifsxref{latoc}{def:latoc}.
%Let $\distrib$ be the \rele{distributivity relation}\ifsxref{latd}{def:Drel}.
%Let $\oset{\R}{\orela}$ be the \structe{ordered set} \xref{def:oset} of \structe{real numbers}.
%\defboxt{
%  A function $\psP$ in $\clFxr$ is a \fnctd{probability} function on $\psL$ if
%  \\$\begin{array}{>{\qquad}Flc lcl CDD}
%    1. &                   &        & \psP(\lzero)   &=&      0               &                     & (\prope{nondegenerate}) & and \\
%    2. &                   &        & \psP(\lid)     &=&      1               &                     & (\prope{normalized})    & and \\
%    3. & x\orel y          &\implies& \psP(x)        &\orela& \psP(y)         & \forall x,y\in\setX & (\prope{monotone})      & and \\
%   %3. & x\meet y = \lzero &\implies& \psP(x\join y) &=&      \psP(x)+\psP(y) & \forall x,y\in\setX & (\prope{additive})      & .      
%    4. & \brb{\begin{array}{lD}
%           x\meet y = \lzero            & and\\
%           \otriple{z}{x}{y}\in\distrib & $\forall z\in\setX$
%         \end{array}} 
%       &\implies& \psP(x\join y) &=&      \psP(x)+\psP(y) & \forall x,y\in\setX & (\prope{additive}).   &       
%  \end{array}$
%  \\
%  If $\psP$ is a \fncte{probability} on an \structe{orthocomplemented lattice} $\psL$, 
%  then $\opair{\psL}{\psP}$ is a \structd{probability space},\\
%  and any element $x$ in $\setX$ is an \structd{event} in $\ps$.
%  % over $\pso$,
%  %$\pso$ is the \structd{outcome space}, and $\pse$ is the \structd{event space}.
%  }
%\end{definition}

%On a \structe{Boolean lattice}, the \fnctd{measure-theoretic probability} function, due to A. N. Kolmogorov, is defined as follows:
%---------------------------------------
\begin{definition}
  \footnote{
    \citerppgc{billingsley1995}{22}{23}{0471007102}{Probability Measures},
    \citer{kolmogorov1933},
    \citerpc{kolmogorov1933e2}{16}{\structe{field of probability}},
    \citerppgc{pap1995}{8}{9}{0792336585}{Definition 2.3(13)}, %{,\prope{\txsigma-additive}},
    \citerpg{kalmbach1986}{27}{9971500094}%{\prope{\txsigma-additive},\prope{\txsigma-complete}}
    }
\label{def:prob}
\label{def:Psp}
\label{def:psp}
%---------------------------------------
Let $\latnX$ be a \structe{lattice with negation}\ifsxref{negation}{def:latn}.
\defboxt{
  The function $\psP$ is a \fnctd{probability function} if
  \\\indentx$\begin{array}{Flc lcl CDD}
      (1). &                   &        & \psP(\lid)     &=&      1               &                     & (\prope{normalized}) & and \\
      (2). &                   &        & \psP(x)        &\oreld& 0               & \forall x\in\setX   & (\prope{nonnegative})& and \\
      (3). & \ds\meetop_{n=1}^\infty x_n=\lzero &\implies& \ds\psP\brp{\joinop_{n=1}^\infty x_n} &=&  \ds\sum_{n=1}^\infty\psP(x_n) & \forall x_n\in\setX & (\prope{\txsigma-additive})   & .      
    \end{array}$
  }
\end{definition}

%---------------------------------------
\begin{remark}
%---------------------------------------
The advantage of this definition is that $\psP$ is a \fncte{measure}, and hence all the power of measure theory 
is subsequently at one's disposal in using $\psP$.
However, it has often been argued that the requirement of \prope{\txsigma-additivity} is unnecessary for a probability function.
Even as early as 1930, de Finetti argued against it, in what became a kind of polite running debate with Fr/'echet.\footnote{
  \citeP{definetti1930a},
  \citeP{frechet1930a},
  \citeP{definetti1930b},
  \citeP{frechet1930b},
  \citeP{definetti1930c},
  \citePpp{cifarelli1996}{258}{260}
  }
In fact, Kolmogorov himself provided some argument against \prope{\txsigma-additivity} when referring to the closely related 
\hie{Axiom of Continuity} saying,
``Since the new axiom is essential for infinite fields of probability only,
it is almost impossible to elucidate its empirical meaning\ldots
For, in describing any observable random process we can obtain only finite fields of probability.\ldots"
But in its support he added, ``This limitation has been found expedient in researches of the most diverse sort."\footnote{
  \citerp{kolmogorov1933e2}{15}
  }

There are several other definitions of probability that only require \prope{additivity} rather than \prope{\txsigma-additivity}.
On a \structe{Boolean lattice}, the \fnctd{traditional probability} function is defined as%
  \footnote{
    \citerppg{papoulis}{21}{22}{0070484775},
    \citerpc{kolmogorov1933e2}{2}{\textsection 1. Axioms I--V}
    }
  \\\indentx$\begin{array}{Flc lcl CDD}
    (1). &                   &        & \psP(\lid)     &=&      1               &                     & (\prope{normalized})     & and \\
    (2). &                   &        & \psP(x)        &\oreld& 0               & \forall x\in\setX   & (\prope{nonnegative})& and \\
    (3). & x\meet y = \lzero &\implies& \psP(x\join y) &=&      \psP(x)+\psP(y) & \forall x,y\in\setX & (\prope{additive})   & .      
  \end{array}$\\
This definition implies (on a \structe{Boolean lattice}) that 
  \\\indentx$\begin{array}{FrclCDD}
      (a). & \psP(\lzero)  &=&     0                              &                     & (\prope{nondegenerate})      & and
    \\(b). & \psP(x)       &\orel& 1                              & \forall x\in\setX   & (\prope{upper bounded})      & and 
    \\(c). & \psP(x)       &=&     1-\psP({x^\orthog})            & \forall x\in\setX   &                              & and 
    \\(d). & \psP(x\join y)&\orel& \psP(x)+\psP(y)                & \forall x,y\in\setX & (\prope{subadditive})        & and 
    \\(e). & \psP(x\join y)&=&     \psP(x)+\psP(y)-\psP(x\meet y) & \forall x,y\in\setX &                              & and
    \\(f). & x\orel y \implies \psP(x)&\orel&\psP(y)              & \forall x,y\in\setX & (\prope{monotone})           & . 
  \end{array}$
  \\
On a \structe{distributive pseudocomplemented lattice}, the \fnctd{generalized probability} function has been defined as%
  \footnote{
    \citePp{narens2014}{118},
    \citerg{narens2007}{9812708014}
    }
  \\$\begin{array}{>{\qquad}Frcl CDD}
    (1). &  \psP(\lzero)    &=&      0                                              & (\prope{nondegenerate}) & and \\
    (2). &  \psP(\lid)      &=&      1                                              & (\prope{normalized})    & and \\
    (3). &  0\orel\psP(\lid)&\orel&  1                                              &                         & and \\
    (4). &  \psP(x\join y)  &=&\psP(x)+\psP(y)-\psP(x\meet y) & \forall x,y\in\setX &                         &  .      
  \end{array}$
  \\
On an \structe{orthomodular lattice}, or a \structe{finite modular lattice}, 
the \fnctd{quantum probability} function is defined as%
  \footnote{
    \citePpc{greechie1971}{126}{{\scshape Definitions}},
    \citePp{narens2014}{118}
    }
  \\$\begin{array}{>{\qquad}Flc lcl CDD}
    (1). &                   &        & \psP(\lzero)   &=&      0               &                     & (\prope{nondegenerate})& and \\
    (2). &                   &        & \psP(\lid)     &=&      1               &                     & (\prope{normalized})  & and \\
    (3). & x\ocop y          &\implies& \psP(x\join y) &=&      \psP(x)+\psP(y) & \forall x,y\in\setX & (\prope{additive})& .      
  \end{array}$
  \\
However, for lattices that are not \prope{distributive}, \prope{modular}, or \prope{orthomodular}, 
none of these definitions work out so well.
Take for example the \structe{O$_6$ lattice} with the ``very reasonable" probability function given in \prefpp{ex:ps_o6}.
This probability space $\opair{\text{O$_6$}}{\psP}$ fails to be any of the 4 probability functions defined in this Remark.
It fails to be a \fncte{measure-theoretic} or \fncte{traditional probability} function because 
\\\indentx$a\meet b=0$\qquad{but}\qquad$\psP(a\join b)=\psP(\lid)=1\neq\frac{1}{3}+\frac{1}{2}=\psP(a)+\psP(b)$ .\\
It fails to be a \fncte{generalized probability} function because 
\\\indentx$\psP(a\join b)=\psP(\lid)=1\neq\frac{1}{3}+\frac{1}{2}-0=\psP(a)+\psP(b)-\psP(0)=\psP(a)+\psP(b)-\psP(a\meet b)$ .\\
It fails to be an \fncte{quantum probability} function because 
\\\indentx$a\ocop b=0$\qquad{but}\qquad$\psP(a\join b)=\psP(\lid)=1\neq\frac{1}{3}+\frac{1}{2}=\psP(a)+\psP(b)$ .\\
In each of these cases, the function $\psP$ fails to be \prope{additive}.
The solution of \prefpp{def:prob} is simply to ``switch off" \prope{additivity} when the lattice is not \prope{distributive}.
This method is a little ``crude", but at least it allows us to define probability on a very wide class of lattices,
while retaining compatibility with the \prope{Boolean} case. % \xxxref{prop:ps_01}{prop:ps_ortho_xy}{prop:ps_boa_xy}.
\end{remark}
%(end \pref{rem:prob})

%======================================
\section{Probability Space}
%======================================
In mathematics, a \structe{space} is simply a set and in the most general definition,
nothing else.
However, normally for a space to actually be useful, some additional structure is added.
One of the most general additional structures is a \structe{topology};
and a space together with a topology is called a
\hie{topological space}.\ifdochas{topology}{\footnote{\hie{topological space}: \prefp{def:top_space}}}
A topological space imposes additional structure on a space  in the form of subsets
and guarantees that these subsets are closed under such fundamental operations as
set \hie{union} and set \hie{intersection}.
With the additional structure available in a topological space, we are able to
analyze such basic concepts as
continuity,
convergence, and
connectivity.

However for a great number of mathematical applications,
we need to \fncte{measure} mathematical objects---the most general measurement being measures on subsets of some set.
Examples of measurement in mathematics include integration and probability.
Before measurement can be effectively performed on a set,
the set must be equipped with a subset structure.
In analysis, arguably the most fundamental subset structure is the humble \structe{topology} \xref{def:topology}.
However, a simple topology does not provide sufficient structure
for effective measurement.
For example,
often we would not only like to measure some subset $\setA$,
but also its complement $\cmpA$.
A topology is not closed under the complement operation.
So instead of a topology only, we equip the space with a more powerful (and thus less general) structure
called a \structe{\txsigma-algebra} (\structe{sigma-algebra}) \xref{def:sigalg}.
A \structe{\txsigma-algebra} is a subset structure that is closed under set complement.
A set together with a \structe{\txsigma-algebra} is called a \structe{measurable space}.
And a set together with a \structe{\txsigma-algebra} and a \ope{measure} on that \structe{\txsigma-algebra}
is called a \structe{measure space} \xref{def:mspace}.

The next definition presents a very important measure space---the \structe{probability space}.
%---------------------------------------
\begin{definition}
\label{def:ps}
\index{probability space}
\index{outcomes} \index{events} \index{probability measure}
%---------------------------------------
\defboxp{
  The triple $\ps$ is a \structd{probability space} if
  \\\indentx$\begin{array}{FMMD}
    (1).& $\pso$ is a \structe{set}                             &                   & and
  \\(2).& $\pse$ is a \structe{\txsigma-algebra} on $\pso$      & \xref{def:sigalg} & and
  \\(3).& $\psP:\pse\to[0,1]$ is a \structe{measure} on $\pse$  & \xref{def:measure}& .
  \end{array}$
  \\
  If $\spS\eqd\ps$ is a \structe{probability space} then
  $x$ is an \structd{outcome} in $\spS$ if $x\in\pso$, 
  $\setA$ is an \structd{event} in $\spS$ if $\setA\in\pse$, and
  $\psP\setA$ is the \fnctd{probability} of $\setA$ in $\spS$ if $\setA$ is an \structe{event} in $\spS$.
  }
\end{definition}

%---------------------------------------
\begin{definition}
\footnote{
  \citerpgc{papoulis1990}{52}{0137116985}{Independent Events}
  }
\label{def:independence}
\label{def:indepen}
\label{def:independent}
%---------------------------------------
Let $\ps$ be a \structe{probability space} \xref{def:ps}.
\defboxt{
  Two \structe{event}s $\setA$ and $\setB$ in $\psE$ are \propd{independent} if
  \\\indentx$\psP(\setA\seti\setB) = \psP(\setA)\psP(\setB)$
  }
\end{definition}

%---------------------------------------
\begin{definition}
\footnote{
  \citerpgc{papoulis1990}{45}{0137116985}{2-3 Conditional Probability and Indenpendence}
  }
\label{def:conprob}
\label{def:conditional}
\label{def:conP}
%---------------------------------------
Let $\ps$ be a \structe{probability space} \xref{def:ps}.
Let $x$ and $y$ be \structe{event}s in $\psL$.
\defboxt{
  The \structd{conditional probability of $x$ given $y$} is defined as
  \\\indentx$\ds\psP(\setA|\setB) \eqd \frac{\psP(\setA\seti\setB)}{\psP(\setB)}$
  }
\end{definition}

%=======================================
\section{Properties}
%=======================================
%---------------------------------------
\begin{proposition}
%---------------------------------------
\propbox{
  \mcom{
    \text{$\ps$ is a \structe{probability space}}
    \quad\implies\quad
    \text{$\ps$ is a \structe{measure space}}
  }{(every probability space is a measure space)}
  }
\end{proposition}

%---------------------------------------
\begin{theorem}
\footnote{
  property (1): \citerpgc{papoulis}{21}{0070484775}{(2-11)}
  }
\label{thm:ps}
%\label{prop:ps_01}
%\label{prop:ps_ortho_xy}
%\label{thm:ps}
%---------------------------------------
Let $\ps$ be a \structe{probability space} \xref{def:ps}.
\thmbox{
  \begin{array}{FrclCDD}
   %  &     & \psP(\lzero) &=& 0     &                   & \\
    (1). & 0        &\orel   & \psP(x)      \orel 1                  & \forall x\in\setX   & (\prope{bounded})            & and\\
    (2). & \psP(x)  &=       & 1 - \psP({x^\orthog})                 & \forall x\in\setX   & (\prope{partition of unity}) & and\\
    (3). & x\orel y &\implies& \psP(y^\orthog) \orel \psP(x^\orthog) & \forall x,y\in\setX & (\prope{antitone})           &
  \end{array}
  }
\end{theorem}
\begin{proof}
  \begin{enumerate}
        % \psP(\lzero)
        %   &= \brlr{\psP(x)}_{x=\lzero}
        % \\&= \brlr{\psP(x\join y)-\psP(y)}_{x=\lzero}
        %   && \text{because $x\meet\lzero=\lzero$ and by \prope{additive} property of $\psP$ \xref{def:ps}}
        % \\&= \psP(\lzero\join y)-\psP(y)
        % \\&= \psP(y)-\psP(y)
        %   && \text{by definition of $\lzero$}
        % \\&= 0
        % \\
    \item Proof for $0\orel\psP(x)\orel 1$:
      \begin{align*}
        \boxed{0}
          &= \psP(\lzero)
          && \text{by by \prope{nondegenerate} property of $\psP$ \xref{def:ps}}
        \\&\orel \boxed{\psP(x)}
          && \text{because $\lzero\orel x$\ifsxref{latb}{def:latb} and \prope{monotone} property of $\psP$}
        \\&\orel \psP(\lid)
          && \text{because $x\orel\lid$\ifsxref{latb}{def:latb} and \prope{monotone} property of $\psP$}
        \\&= \boxed{\lid}
          && \text{by \prope{normalized} property of $\psP$}
      \end{align*}
    \item Proof for $\psP(x)=1-\psP({x^\orthog})$:
      \begin{enumerate}
        \item Proof that $\psP$ is \prope{additive} \xref{def:ps} over $\setn{0,x,x^\orthog}\subseteq\setX$: \label{item:ps_additive}
          \begin{enumerate}
            \item $\setn{0,x,x^\orthog}$ is \prope{distributive}\ifsxref{latd}{lem:latoc_distrib}.
            \item $x\meet x^\orthog=0$ for all $x\in\setX$ by the \prope{non-contradiction} property of 
                  \structe{orthocomplemented lattice}s\ifsxref{latoc}{def:latoc}.
            \item Therefore, by \pref{def:ps}, $\psP$ is \prope{additive} over $\setn{0,x,x^\orthog}$.
          \end{enumerate}
      
        \item Then \ldots
          \begin{align*}
            1-\psP({x^\orthog})
              &= \psP(\lid)-\psP({x^\orthog})
              && \text{by \prope{normalized} property of $\psP$}
              && \text{\xref{def:ps}}
            \\&= \psP(x\join{x^\orthog})-\psP({x^\orthog})
              && \text{by \prope{excluded middle} property of ortho. lat.}
              && \text{\ifxref{latoc}{thm:latoc_prop}}
            \\&= \psP(x)+\psP({x^\orthog})-\psP({x^\orthog})
              && \text{by \prope{additive} property of $\ps$}
              && \text{\xref{item:ps_additive}}
            \\&= \psP(x)
              && \text{by field property of $\fieldR$}
          \end{align*}
        \end{enumerate}
          
    \item Proof for $x\orel y\implies\psP(y^\orthog) \orel \psP(x^\orthog)$:
      \begin{align*}
        x\orel y 
          &\implies y^\orthog \orel x^\orthog
          && \text{by \prope{antitone} property of \structe{orthocomplemented lattice}s}
          && \text{\ifxref{latoc}{def:latoc}}
        \\&\implies \psP(y^\orthog) \orel \psP(x^\orthog)
          && \text{by \prope{monotone} property of $\psP$}
          && \text{\xref{def:ps}}
      \end{align*}
  \end{enumerate}
\end{proof}

%---------------------------------------
\begin{theorem}
\footnote{
  \citerpgc{papoulis}{21}{0070484775}{(2-13)},
  \citerppgc{feller1970}{22}{23}{0471257087}{(7.4),(7.6)}
  }
\label{thm:ps_boa_xy}
\label{thm:ps_add}
%---------------------------------------
Let $\ps$ be a \structe{probability space} \xref{def:ps}.
\thmbox{
   \brbr{\begin{array}{M}
     $\psL$ is \prope{Boolean}\\
     \xref{def:boolean}
   \end{array}}
   \implies
   \brbl{\begin{array}{FrclCD}
     1. & \psP(x\join y) &=&     \psP(x) + \psP(y) - \psP(x\meet y) & \forall x,y\in\setX & and\\
     2. & \psP(x\join y) &\orel& \psP(x) + \psP(y)                  & \forall x,y\in\setX & (\prope{Boole's inequality})
   \end{array}}
   }
\end{theorem}
\begin{proof}
\begin{enumerate}
  \item lemma: Proof that $\psP((\negat x)\meet y)=\psP(y)-\psP(x\meet y)$: \label{item:ps_boa_xy}
    \begin{align*}
      \psP(y)-\psP(xy)
        &= \psP(\lid\meet{y})-\psP(xy)
        && \text{by definition of $\lid$ and $\meet$\ifsxref{order}{def:meet}}
      \\&= \psP\brs{(x\join{x^\orthog})y}-\psP(xy)
        && \text{by \prope{excluded middle} property of \structe{Boolean lattice}s}
      \\&= \psP(xy\join{x^\orthog}y)-\psP(xy)
        && \text{by \prope{distributive} property of \structe{Boolean lattice}s}
      \\&= \psP(xy)+\psP({x^\orthog}y)-\psP(xy)
        %&& \text{because $(xy)({x^\orthog}y)=\lzero$ and by \prope{additive} prop. \xref{def:ps}}
        && \text{because $(xy)({x^\orthog}y)=\lzero$ and by \prope{additive} property}
      \\&= \psP({x^\orthog}y)
    \end{align*}

  \item Proof that $\psP(x\join y) = \psP(x) + \psP(y) - \psP(x\meet y)$:
    \begin{align*}
      \psP(x\join y) 
        &= \psP(x\join {x^\orthog}y)
        && \text{by property of \prope{Boolean lattice}s} 
      \\&= \psP(x) + \psP({x^\orthog}y)
        %&& \text{because $(x)({x^\orthog}y)=\lzero$ and by \prope{additive} property \xref{def:ps}}
        && \text{because $(x)({x^\orthog}y)=\lzero$ and by \prope{additive} property}
      \\&= \psP(x) + \psP(y) - \psP(x\meet y)
        && \text{by \prefpp{item:ps_boa_xy}}
    \end{align*}
\end{enumerate}
\end{proof}

%---------------------------------------
\begin{theorem}[\thmd{sum of products}]
\label{thm:psp_sop}
%\label{thm:psp_soc}
%---------------------------------------
Let $\latbX$ be a \structe{bounded lattice}\ifsxref{latb}{def:latb},
    $\ps$ a \structe{probability space} \xref{def:ps},
and $\setn{y,x_1,x_2,x_3,\ldots}$ a subset of $\setX$.
\thmbox{
  \brb{\begin{array}{FMD}
    1. & $\latL$ is \prope{distributive}                           & and\\ %\xref{def:distributive} & and\\
    2. & $\setn{x_1,x_2,\ldots}$ is a \structe{partition} of $y$   & %\xref{def:partition}    &
  \end{array}}
  \implies
  \brb{\begin{array}{Frc>{\ds}lD}
    1. & \psP(y)        &=& \sum_n \psP(x_n) & and \\%= \sum_n \psP(y|x_n)\psP(x_n)
    2. & \psP(y)        &=& \sum_n \psP(y\meet x_n) & and \\
    3. & \psP(z\meet y) &=& \sum_n \psP(z\meet x_n) &
  \end{array}} 
  }
\end{theorem}
\begin{proof}
\begin{enumerate}
  \item Proof that $\psP$ is \prope{additive} \xref{def:ps} on $\ps$: \label{item:sop_additive}
    \begin{enumerate}
      \item Proof that $(yx_n) \meet (yx_m) = 0$ for $n\neq m$:
        \begin{align*}
          (yx_n) \meet (yx_m)
            &= y(x_n x_m)
            && \text{by definition of $\meet$}
            && \text{\ifxref{order}{def:meet}}
          \\&= y\meet0
            && \text{by \prope{mutually exclusive} property of \structe{partition}s}
            && \text{\ifxref{latb}{def:partition}}
          \\&= 0
            && \text{by \prope{lower bounded} property of \structe{bounded lattice}s}
            && \text{\ifxref{latb}{prop:latb_prop}}
        \end{align*}
      \item Proof that $\latL$ is \prope{distributive}\ifsxref{latd}{def:distributive}: by \prope{distributive} hypothesis
    \end{enumerate}

  \item Proof that $\psP(y) = \sum_n \psP(x_n)$ \label{item:psp_sop}
    \begin{align*}
      \psP(y)
        &= \psP\brp{yx_1 \join yx_2 \join \cdots \join yx_n}
        && \text{\ifdochas{latb}{by \prefpp{prop:latb_partition}}}
      \\&= \sum_n\psP\brp{yx_n}
        && \text{by \pref{item:sop_additive} and \prope{additive} property}
        && \text{\xref{def:ps}}
      \\&= \sum_n\psP\brp{y|x_n}\psP(x_n)
        && \text{by \structe{conditional probability}}
        && \text{\xref{def:conditional}}
    \end{align*}

  %\item Proof that $\psP(y) = \sum_n \psP(y|x_n)\psP(x_n)$: This follows from \pref{item:psp_sop} and the 
  %      definition of \fncte{conditional probability} \xref{def:conditional}.

\end{enumerate}
\end{proof}

As described in \prefpp{def:ps},
every \structe{probability space} $\ps$ contains a probability \fncte{measure} $\psP:\pse\to[0,1]$.
This probability \fncte{measure} has some basic properties as described in
\pref{thm:P} (next).
%---------------------------------------
\begin{theorem}
\label{thm:P}
%---------------------------------------
Let $\ps$ be a \structe{probability space}.
Let $\setB$ be a set and $\set{\setB_n}{n=1,2,\ldots,\xN}$ a set of sets.
\thmbox{
  \brb{\begin{array}{M}
    $\set{\setB_n}{n=1,2,\ldots,\xN}$ is a \\
    \structe{partition} of $\setB$.
  \end{array}}
  \quad\implies\quad
  \brb{\begin{array}{Frc>{\ds}l lD}
    (1).&\psP(\setB)      &=&  \sum_{n=1}^{\xN} \psP(\setB_n)        & \forall \setB\in\pse & and\\
    (2).&\psP(\setA\setB) &=&  \sum_{n=1}^{\xN} \psP(\setA\setB_n)   & \forall \setA,\setB\in\pse 
  \end{array}}
  }
\end{theorem}
\begin{proof}
$\psP$ is a \fncte{measure} and by \prefpp{def:measure}.
\end{proof}

%---------------------------------------
\begin{proposition}
\label{prop:cdf_monotone}
%---------------------------------------
Let $\ps$ be a probability space, and $\rvX$ a \fncte{random variable} \xref{def:randvar}
with \fncte{probability density function} $\ppx(x)$ and
\fncte{cumulative distribution function} $\pcx(x)$.
\propbox{\begin{array}{FMD}
    (1).& $\pcx(x)$ is \prope{monotone}                                                    &and\\
    (2).& $\ppx(x)$ is \prope{continuous} $\implies$ $\pcx(x)$ is \prope{strictly monotone}&and\\
    (3).& $\ppx(x)$ is \prope{continuous} $\implies$ $\pcx(x)$ is \prope{invertible}
  \end{array}}
\end{proposition}

%=======================================
\section{Examples}
\label{sec:probspace_examples}
%=======================================
%---------------------------------------
\begin{example}[\exmd{single coin toss}]
\label{ex:ps_coin}
%---------------------------------------
Let $\coinhead$ represent ``heads" and $\cointail$ represent ``tails" in a coin toss.
Let $0<p<1$ be the probability of a head.
A \structe{probability space} $\ps$ for a single coin toss is as follows:
\exbox{
  \begin{array}{rcl}
    \psO &=&   \setn{\coinhead,\, \cointail}
      \\
    \psE &=&   \setn{\mcom{\emptyset}{neither},\,
                     \mcom{\setn{\coinhead}}{heads},\,
                     \mcom{\setn{\cointail}}{tails},\,
                     \mcom{\setn{\coinhead,\,\cointail}}{heads or tails}
                    }
      \\
    \psP(\setX) &=& 
      \brb{\begin{array}{lMc>{\ds}lD}
        0   & for $\setX$ &=& \emptyset                  & (neither heads nor tails) \\
        p   & for $\setX$ &=& \setn{\coinhead}           & (heads)                   \\
        1-p & for $\setX$ &=& \setn{\text{\cointail}}    & (tails)                   \\
        1   & for $\setX$ &=& \setn{\coinhead,\cointail} & (either heads or tails)
      \end{array}}
  \end{array}
  \tbox{\includegraphics{graphics/lat4_m2_psE_coin.pdf}}
  }
\end{example}

%%---------------------------------------
%\begin{example}[Two coin flips]
%%---------------------------------------
%Let $\seto{\setA}$ be the number of elements in a set $\setA$.
%Let $\coinhead$ represent ``heads" and $\cointail$ represent ``tails"
%in a coin toss.
%Suppose we flip a coin two consecutive time and want to
%know how many ``heads" we will have.
%The following is a valid probability space $\ps$:
%\exbox{\begin{array}{rcl}
%  \pso &=& \setn{ (\cointail\cointail),\, (\cointail\coinhead),\, (\coinhead\cointail),\, (\coinhead\coinhead) }  \\
%  \pse &=& \setn{\begin{array}{l}
%    \mcom{\emptyset}{nothing},\,
%    \mcom{\setn{(\cointail\cointail)}}{0 heads},\,
%    \mcom{\setn{(\cointail\coinhead),\,(\coinhead\cointail)}}{1 heads},\,
%    \mcom{\setn{(\coinhead\coinhead)}}{2 heads},\,
%    \mcom{\setn{(\cointail\cointail),\,(\cointail\coinhead),\,(\coinhead\cointail)}}{0 heads or 1 heads},\, \\
%    \mcom{\setn{(\cointail\cointail),\,(\coinhead\coinhead)}}{0 heads or 2 heads},\,
%    \mcom{\setn{(\cointail\coinhead),\,(\coinhead\cointail),\,(\coinhead\coinhead)}}{1 heads or 2 heads},\,
%    \pso
%    %\mcom{\setn{(\cointail\cointail),\, (\cointail\coinhead),\, (\coinhead\cointail),\, (\coinhead\coinhead)} }{0 or 1 or 2 heads ($\pso$)}
%    \end{array}}
%  \\
%  \psP x &=& \frac{1}{4} \seto{x}
%%  \psP x &=& \left\{\begin{array}{*{2}{l>{\text{for }}l@{\qquad\qquad}}}
%%    0           & x=\emptyset         & \frac{3}{4} & x=\text{ 0 heads or 1 heads}            \\
%%    \frac{1}{4} & x=\text{ 0 heads}   & \frac{1}{2} & x=\text{ 0 heads or 2 heads}            \\
%%    \frac{1}{2} & x=\text{ 1 heads}   & \frac{3}{4} & x=\text{ 1 heads or 2 heads}            \\
%%    \frac{1}{4} & x=\text{ 2 heads}   & 1           & x=\text{ 0 or 1 or 2 heads}
%%    \end{array}\right.
%\end{array}
%}
%
%Note however that
%\[   \pse = \left\{ \emptyset, \mbox{0 heads}, \mbox{1 head}, \mbox{2 heads}, \pso \right\} \]
%is {\bf not} a valid event set because it is {\bf not} a \structe{\txsigma-algebra}.
%For example, $\mbox{1 head}\cup\mbox{2 heads}$ is not in $\pse$.
%\end{example}

%---------------------------------------
\begin{figure}% Double coin toss
%---------------------------------------
  \centering
  \exbox{
    \tbox{\includegraphics{graphics/lat8_l2e3_prob_coin.pdf}}
    \begin{array}{|rcl|rcl}
      \psP(\emptyset) &=&0        & \psP(\setO)          &=& 1               \\
      \psP(\setA)     &=&(1-p)^2  & \psP(\setA\setu\setB)&=&(1-p)^2+2p(1-p)  \\
      \psP(\setB)     &=&2p(1-p)  & \psP(\setA\setu\setC)&=& 2p^2-2p+1       \\
      \psP(\setC)     &=&p^2      & \psP(\setB\setu\setC)&=&-p^2+2p 
    \end{array}
    }
  \caption{\structe{Double coin toss} \xref{ex:ps_doublecoin} \label{fig:ps_doublecoin}}
\end{figure}
%---------------------------------------
\begin{example}[\exmd{Double coin toss}]
\label{ex:ps_doublecoin}
%---------------------------------------
    Let $\coinhead$ represent ``heads" and $\cointail$ represent ``tails"
    in a double coin toss in which each toss is \prope{independent} \xref{def:indepen} of the other.
    Let $0<p<1$ be the probability of a head.
    The \structe{probability space} $\ps$ is illustrated in \prefpp{fig:ps_doublecoin}.
\end{example}
\begin{proof}
\begin{align*}
  \psP(\setO)
    &= 1
    && \text{by \prope{normalized} property of $\psP$}
    && \text{\xref{def:Psp}}
  \\
  \psP(\setC)
    &= \psP\setn{\coinhead\coinhead}
    && \text{by definition of $\setC$}
  \\&= \psP(\coinhead)\psP(\coinhead)
    && \text{by definition of \prope{independence}}
    && \text{\xref{def:indepen}}
  \\&= p^2
    && \text{by definition of $p$}
  \\
  \psP(\setA)
    &= \psP\setn{\cointail\cointail}
    && \text{by definition of $\setA$}
  \\&= \psP(\cointail)\psP(\cointail)
    && \text{by definition of \prope{independence}}
    && \text{\xref{def:indepen}}
  \\&= \brb{1-\psP(\coinhead)}\brb{1-\psP(\coinhead)}
    && \text{by \prope{antitone} property of $\psP$}
    && \text{\xref{thm:ps}}
  \\&= (1-p)^2
    && \text{by definition of $p$}
  \\
  \psP(\setB)
    &= \psP\setn{(\cointail\coinhead),(\coinhead\cointail)}
    && \text{by definition of $\setB$}
  \\&= \psP\setn{\cointail\coinhead} + \psP\setn{\coinhead\cointail}
    && \text{by \prope{additive} property of $\psP$}
    && \text{\xref{def:Psp}}
  \\&= \psP\setn{\cointail}\psP\setn{\coinhead} + \psP\setn{\coinhead}\psP\setn{\cointail}
    && \text{by definition of \prope{independence}}
    && \text{\xref{def:indepen}}
  \\&= (1-p)p + p(1-p)
    && \text{by \prope{antitone} property of $\psP$}
    && \text{\xref{thm:ps} and definition of $p$}
  \\&= -2p^2 + p + 1
  \\
  \psP(\setA\setu\setB)
    &= \psP(\setA) + \psP(\setB) -\psP(\setA\seti\setB)
    && \text{by \pref{thm:ps_add}}
  \\&= \psP(\setA) + \psP(\setB) -\psP(\emptyset)
  \\&= (1-p)^2 + (-2p^2 + p + 1) + 0
    && \text{by previous results}
  \\&= -p^2-p+1
  \\
  \psP(\emptyset)
    &= 0
    && \text{by \prope{nondegenerate} property of $\psP$}
    && \text{\xref{def:Psp}}
\end{align*}
\end{proof}

\begin{figure}[h]
  \exbox{
    \begin{array}{rc>{\ds}l}
      \pso &=&\left\{ \text{\dieA,\dieB,\dieC,\dieD,\dieE,\dieF} \right\}
      \\
      \pse &=&\brb{\mcom{\left\{\hspace{1ex} \right\}}{$\emptyset$}, 
                   \mcom{\left\{\text{\dieA,\dieC,\dieE}\right\}}{odd}, 
                   \mcom{\left\{\text{\dieB,\dieD,\dieF}\right\}}{even}, 
                   \mcom{\left\{ \text{\dieA,\dieB,\dieC,\dieD,\dieE,\dieF} \right\}}{$\pso$}
                  }
      \\ 
      \psP(\setX) &=&
        \brb{\begin{array}{l@{\qquad} >{\text{for }e=}l @{\qquad}D}
          0   & \{\hspace{1ex}\}  
              & ($\emptyset$)
              \\
          1   & \left\{ \text{\dieA,\dieB,\dieC,\dieD,\dieE,\dieF} \right\}
              & ($\pso$)
              \\
          p   & \left\{\text{\dieA,\dieC,\dieE}\right\}
              & (odd)
              \\
          1-p & \left\{\text{\dieB,\dieD,\dieF}\right\}
              & (even)
      \end{array}}
  \end{array}
  \quad
  \tbox{\includegraphics{graphics/lat4_m2_prob_die.pdf}}
}
\caption{\structe{even/odd die probability space} \xref{ex:die_evenodd} \label{fig:die_evenodd}}
\end{figure}
%---------------------------------------
\begin{example}[\exmd{even/odd die toss}]
\label{ex:die_evenodd}
%---------------------------------------
The \structe{probability space} for an \structd{even/odd die toss},
with $0<p<1$ being the probability of the die toss being odd, 
is illustrated in \prefpp{fig:die_evenodd}.
\end{example}
\begin{proof}
\begin{align*}
  \psP(\setO)
    &= 1
    && \text{by \prope{normalized} property of $\psP$}
    && \text{\xref{def:Psp}}
  \\
  \psP(\setC)
    &= \psP\setn{\coinhead\coinhead}
    && \text{by definition of $\setC$}
  \\&= \psP(\coinhead)\psP(\coinhead)
    && \text{by definition of \prope{independence}}
    && \text{\xref{def:indepen}}
  \\&= p^2
    && \text{by definition of $p$}
  \\
  \psP(\setA)
    &= \psP\setn{\dieA,\dieC,\dieE}
    && \text{by definition of $\setA$}
  \\&= p
    && \text{by definition of $p$}
  \\
  \psP(\setB)
    &= \psP\setn{\dieB,\dieD,\dieF}
    && \text{by definition of $\setB$}
  \\&= \psP\setn{\dieA,\dieC,\dieE}^\setopc
    && \text{by definition of set complement $\setopc$}
  \\&= \psP\setA^\setopc
    && \text{by definition of $\setA$}
  \\&= \psP(\negat\setA)
    && \text{by definition of $\negat$}
  \\&= 1-\psP(\setA)
    && \text{by \prefp{thm:ps}}
  \\&= 1-p
    && \text{by definition of $p$}
  \\
  \psP(\emptyset)
    &= 0
    && \text{by \prope{nondegenerate} property of $\psP$}
    && \text{\xref{def:Psp}}
\end{align*}
\end{proof}


The two previous \exme{even/odd die} example \xref{ex:die_first4} is in essence the same as the 
\exme{single coin toss} \xref{ex:ps_coin}.
The next offers a little more complexity.
%---------------------------------------
\begin{figure}[h]%
%---------------------------------------
  \centering
  \exbox{
    \tbox{\includegraphics{graphics/lat8_l2e3_prob_die.pdf}}
    \begin{array}{rclrclrcl}
      \psP(\emptyset) &=&0        & \psP(\setO)          &=& 1               \\
      \psP(\setA)     &=&(1-p)^2  & \psP(\setA\setu\setB)&=&(1-p)^2+2p(1-p)  \\
      \psP(\setB)     &=&2p(1-p)  & \psP(\setA\setu\setC)&=& 2p^2-2p+1       \\
      \psP(\setC)     &=&p^2      & \psP(\setB\setu\setC)&=&-p^2+2p 
    \end{array}
    }
  \caption{3-4-2 die example \xref{ex:die_342} \label{fig:die_342}}
\end{figure}
%---------------------------------------
\begin{example}
\label{ex:die_342}
%---------------------------------------
Suppose we have a ``fair" die and we are primarily interested in the
events of the first three $\setn{\dieA,\dieB,\dieC}$, the next two, $\setn{\dieD,\dieE}$ and the final one $\setn{\dieF}$.
The resulting \structe{probability space} is illustrated in \prefpp{fig:die_342}.
\end{example}

The two previous examples \xxref{ex:die_first4}{ex:die_342} illustrate a
\structe{probabilty space}s in which the events %(ignorning $\emptyset$ and $\pso$)
are \prope{mutually exclusive}. The (next) illustrates one where events are {\em not}.

\begin{figure}
  \centering
  \exbox{\begin{array}{ll}
    \pso &= \left\{ \text{\dieA,\dieB,\dieC,\dieD,\dieE,\dieF} \right\}
    \\
    \pse &= \left\{ \mcom{\setn{\quad}}{$\emptyset$},\;
                    \mcom{\setn{\text{\dieA,\dieB,\dieC,\dieD,\dieE,\dieF}}}{$\pso$},\;
                    \mcom{\setn{\text{\dieA,\dieB,\dieC,\dieD}}}{first four},\;
                    \mcom{\setn{\text{\dieD,\dieE,\dieF}}}{last three},\;
                    \right.
                    \\&\qquad
                    \left.
                    \mcom{\setn{\text{\dieD}}}{$\setn{1234}\cap\setn{456}$},\;
                    \mcom{\setn{\text{\dieA,\dieB,\dieC,\dieE,\dieF}}}{$\setn{4}^c$},\;
                    \mcom{\setn{\text{\dieE,\dieF}}}{$\setn{4}^c\cap\setn{456}$},\;
                    \mcom{\setn{\text{\dieA,\dieB,\dieC}}}{$\setn{1234}\cap\setn{4}^c$},\;
            \right\}
    \\
    \psP(e) &=
      \left\{\begin{array}{l@{\qquad} >{\text{for }e=}l @{\qquad}D}
        0           & \setn{\quad}
                    & ($\emptyset$)\\
        1           & \setn{\text{\dieA,\dieB,\dieC,\dieD,\dieE,\dieF}}
                    & ($\pso$)\\
        \frac{2}{3} & \setn{\text{\dieA,\dieB,\dieC,\dieD}}
                    & (first four)\\
        \frac{1}{2} & \setn{\text{\dieD,\dieE,\dieF}}
                    & (last three)\\
        \frac{1}{6} & \setn{\text{\dieD}}
                    & ($\setn{1234}\cap\setn{456}$) \\
        \frac{5}{6} & \setn{\text{\dieA,\dieB,\dieC,\dieE,\dieF}}
                    & ($\setn{4}^c$) \\
        \frac{1}{3} & \setn{\text{\dieE,\dieF}}
                    & ($\setn{4}^c\cap\setn{456}$) \\
        \frac{1}{2} & \setn{\text{\dieA,\dieB,\dieC}}
                    & ($\setn{1234}\cap\setn{4}^c$)
      \end{array}\right.
  \end{array}}
  \caption{First 4 / last 3 die example \xref{ex:die_first4} \label{fig:die_first4}}
\end{figure}
%---------------------------------------
\begin{example}
\label{ex:die_first4}
%---------------------------------------
Suppose we have a ``fair" die and we are primarily interested in the
events of the first four
$\left(\setn{\text{\dieA,\dieB,\dieC,\dieD}}\right)$
(that is, whether one roll of the die will produce
a value in the set $\setn{\dieA,\dieB,\dieC,\dieD}$)
and the last three
$\left(\setn{\text{\dieD,\dieE,\dieF}}\right)$
However, these events do not by themselves form a \structe{\txsigma-algebra}.
Rather under the $\seti$ and $\setu$ operations, these two events generate
a total of eight possible events that together form a \structe{\txsigma-algebra}.
The resulting \structe{probability space} is illustrated in \prefpp{fig:die_first4}.
\end{example}

But why go through all the trouble of requiring a \txsigma-algebra?
Having a \txsigma-algebra in place ensures that anything we might possibly 
want to measure {\em can} be measured.
It makes sure all possible combinations are taken into account.
And why go through the additional trouble of requiring a measure space?
With a measure space available, expressing the measure over a complex
set is often greatly simplified because the measure space provides nice 
algebraic properties (namely the \prope{\txsigma-additive} property).
\pref{ex:prob_123456} (next) illustrates how a rather complex
\txsigma-algebra (64 elements) can be compactly represented in a measure space.
%---------------------------------------
\begin{example}
\label{ex:prob_123456}
%---------------------------------------
Suppose we have a ``fair" dice and we are interested in measuring over the
power set of events (largest possible algebra---$2^6=64$ events).
This leads to the probability space $\ps$ where
\exbox{\begin{array}{ll@{\qquad}D}
  \psO    &= \setn{ \text{\dieA,\dieB,\dieC,\dieD,\dieE,\dieF} }
          \\
  \psE    &= \mathcal{P}(\pso) & (the power-set of $\pso$)
          \\
  \psP(e) &= \frac{1}{6} |e|
          & ($\frac{1}{6}$ times the number of possible outcomes in event $e$)
\end{array}}
\end{example}

%---------------------------------------
\begin{example}[Gaussian distribution on $\R$]
%---------------------------------------
Let $\ssB$ be the \structe{Borel algebra on $\R$}.
Let $\psL\eqd\opair{\ssB}{\subseteq}$ be the lattice formed by the elements of $\ssB$---this
lattice is a \structe{Boolean algebra}.
Let 
\\\indentx$\ds\psP(\setA)\eqd\frac{1}{\sqrt{2\pi\sigma^2}}\int_\setA e^{\frac{x^2}{2\sigma^2}}\dx$ for $\setA\subseteq\ssB$\\
and where $\int$ is the \ope{Lebesgue integral} \xref{def:intL}.
Then $\opair{\psL}{\psP}$ is a \structd{probability space}. 
\end{example}

%---------------------------------------
\begin{example}[Gaussian noise]
%---------------------------------------
Let $X\sim\pN{0}{\sigma^2}$ be a random variable with Gaussian distribution.
We can construct the following probability space $\ps$:

\exbox{\begin{array}{rcl}
  \pso &=& \R \\
  \pse &=& \setn{ \emptyset, \pso } \setu \set{(a,b)}{a,b\in\R,a<b} \\
  \psP x &=& \left\{\begin{array}{ll}
    0      & \mbox{ for } x=\emptyset \\
    1      & \mbox{ for } x=\pso      \\
    \frac{1}{\sqrt{2\pi\sigma^2}}\int_a^b e^{\frac{x^2}{2\sigma^2}}\dx & \mbox{ otherwise}
    \end{array}\right.
\end{array}}
\end{example}

%---------------------------------------
\begin{example}
\label{ex:prob_1011}
%---------------------------------------
The set of outcomes $\pso$ can also be a set of waveforms:
\exbox{\begin{array}{ll}
  \pso    &= \setn{\begin{tabular}{llll}
               \textifsym{|H|LL|HHH|L} &
               \textifsym{L|H|LL|HHH}  &
               \textifsym{H|L|H|LL|HH} &
               \textifsym{HH|L|H|LL|H} \\
               \textifsym{HHH|L|H|LL}  &
               \textifsym{L|HHH|L|H|L} &
               \textifsym{LL|HHH|L|H}  &
             \end{tabular}}
          \\
  \pse    &= \mathcal{P}(\pso)
          \\
  \psP(e) &= \frac{1}{7} |e|
\end{array}}
\end{example}


%=======================================
\section{Probability subspaces}
%=======================================

%---------------------------------------
% Euclidean 3-space partitioned by power lattice
%---------------------------------------
\begin{figure}[th]
\begin{minipage}[c]{8\tw/16}
\begin{center}
\footnotesize
\setlength{\unitlength}{\tw/440}%
\begin{picture}(440,520)(-220,0)%
  \thicklines
  %{\color{graphpaper}\graphpaper[10](-200,0)(400,520)}%
  \put(-150,300){\makebox(0,0){%
    \includegraphics*[width=4\tw/16, height=4\tw/16, clip=true]{../common/normxy_00.eps}}}%
  \put(   0,300){\makebox(0,0){%
    \includegraphics*[width=4\tw/16, height=4\tw/16, clip=true]{../common/normxy_80.eps}}}%
  \put( 150,300){\makebox(0,0){%
    \includegraphics*[width=4\tw/16, height=4\tw/16, clip=true]{../common/normxy_95.eps}}}%
  \put(-150,180){\makebox(0,0){%
    \begin{picture}(200,120)(-100,0)
      \setlength{\unitlength}{2\tw/(8*200)}%
      %{\color{graphpaper}\graphpaper[10](-100,0)(200,120)}%
      \thicklines%
      \color{axis}%
        \put(-100,   0 ){\line(1,0){200} }%
        \qbezier[30](  0,0)(  0, 60)(  0,120)%
      \color{red}%
        \qbezier( -40,  60)(   0, 180)(  40,  60)%
        \qbezier(-100,   0)( -60,   0)( -40,  60)%
        \qbezier(  40,  60)(  60,   0)( 100,   0)%
    \end{picture}
  }}%
  \put(0,180){\makebox(0,0){%
    \begin{picture}(200,120)(-100,0)
      \setlength{\unitlength}{2\tw/(8*200)}%
      %{\color{graphpaper}\graphpaper[10](-100,0)(200,120)}%
      \thicklines%
      \color{axis}%
        \put(-100,   0 ){\line(1,0){200} }%
        \qbezier[30](  0,0)(  0, 60)(  0,120)%
      \color{green}%
        \qbezier( -40,  60)(   0, 180)(  40,  60)%
        \qbezier(-100,   0)( -60,   0)( -40,  60)%
        \qbezier(  40,  60)(  60,   0)( 100,   0)%
    \end{picture}
  }}%
  \put(150,180){\makebox(0,0){%
    \begin{picture}(200,120)(-100,0)
      \setlength{\unitlength}{2\tw/(8*200)}%
      %{\color{graphpaper}\graphpaper[10](-100,0)(200,120)}%
      \thicklines%
      \color{axis}%
        \put(-100,   0 ){\line(1,0){200} }%
        \qbezier[30](  0,0)(  0, 60)(  0,120)%
      \color{blue}%
        \qbezier( -40,  60)(   0, 180)(  40,  60)%
        \qbezier(-100,   0)( -60,   0)( -40,  60)%
        \qbezier(  40,  60)(  60,   0)( 100,   0)%
    \end{picture}
  }}%
  {\color{picbox}%
    %\put( -50,400){\framebox(100,100){}}%
    \put( 100,250){\framebox(100,100){}}%
    \put( -50,250){\framebox(100,100){}}%
    \put(-200,250){\framebox(100,100){}}%
    \put( 100,100){\framebox(100,100){}}%
    \put( -50,100){\framebox(100,100){}}%
    \put(-200,100){\framebox(100,100){}}%
    \put( -25,  0){\framebox( 50, 50){}}%
    }%
  {\color{black}%
    \put(   0,400){\line( 0,-1){ 50}}%
    \put(   0,400){\line(-3,-1){150}}%
    \put(   0,400){\line( 3,-1){150}}%
    \put(   0,200){\line(-3, 1){150}}%
    \put( 150,200){\line(-3, 1){150}}%
    \put(-150,200){\line( 3, 1){150}}%
    \put( 150,200){\line( 0, 1){ 50}}%
    \put(-150,200){\line( 0, 1){ 50}}%
    \put(   0,200){\line( 3, 1){150}}%
    \put(   0, 50){\line(-3, 1){150}}%
    \put(   0, 50){\line( 0, 1){ 50}}%
    \put(   0, 50){\line( 3, 1){150}}%
    }%
  \put(0,450){%
    \setlength{\unitlength}{1\tw/(450*2)}%
    \begin{picture}(0,0)(0,0)%
      %{\color{graphpaper}\graphpaper[10](-150,-150)(300,300)}%
      {\color{red}%
        \put( -50,  50){\line( 1, 1){100} }%
        \put(   0,   0){\line(-1,-1){ 50} }%
        \put(   0,-100){\line(-1,-1){ 50} }%
        \put( -50, -50){\line( 1, 1){ 50} }%
        \put(-150, -50){\line( 1, 1){ 50} }%
        \put(  50, -50){\line( 1, 1){100} }%
        }%
      {\color{green}%
        \put(-150, -50){\line( 1, 0){200} }%
       %\put( -75,  25){\line( 1, 0){ 50} }%
       %\put(  25,  25){\line( 1, 0){100} }%
        \put(   0,   0){\line( 1, 0){100} }%
        \put(-100,   0){\line( 1, 0){ 50} }%
        \put(-100, 100){\line( 1, 0){200} }%
        \put(-100,-100){\line( 1, 0){ 50} }%
        \put(   0,-100){\line( 1, 0){100} }%
        \put( 150,  50){\line(-1, 0){ 50} }%
        }%
      {\color{blue}%
        \put( -50,-150){\line( 0, 1){200} }%
       %\put(  25,  25){\line( 0, 1){100} }%
        \put(  50, 150){\line( 0,-1){ 50} }%
       %\put(  25, -75){\line( 0, 1){ 50} }%
        \put(-100,-100){\line( 0, 1){ 50} }%
        \put(-100,   0){\line( 0, 1){100} }%
        \put( 100,-100){\line( 0, 1){200} }%
        \put(   0,-100){\line( 0, 1){ 50} }%
        \put(   0,   0){\line( 0, 1){100} }%
        }%
    \end{picture}%
  }
  \put(-150,300){%
    \setlength{\unitlength}{1\tw/(400*3)}%
    \begin{picture}(0,0)(0,0)%
      %{\color{graphpaper}\graphpaper[10](-150,-150)(300,300)}%
      {\color{red}%
        \put(-150, -50){\line( 1, 1){100} }%
        \put(  50, -50){\line( 1, 1){100} }%
        }%
      {\color{green}%
        \put(-150, -50){\line( 1, 0){200} }%
        \put( 150,  50){\line(-1, 0){200} }%
        }%
    \end{picture}%
  }
  \put(0,300){%
    \setlength{\unitlength}{1\tw/(400*3)}%
    \begin{picture}(0,0)(0,0)%
      %{\color{graphpaper}\graphpaper[10](-150,-150)(300,300)}%
      {\color{red}%
        \put( -50,  50){\line( 1, 1){100} }%
        \put(  50, -50){\line(-1,-1){100} }%
        }%
      {\color{blue}%
        \put( -50,-150){\line( 0, 1){200} }%
        \put(  50, 150){\line( 0,-1){200} }%
        }%
    \end{picture}%
  }
  \put(150,300){%
    \setlength{\unitlength}{1\tw/(400*3)}%
    \begin{picture}(0,0)(0,0)%
      %{\color{graphpaper}\graphpaper[10](-150,-150)(300,300)}%
      {\color{green}%
        \put(-100, 100){\line( 1, 0){200} }%
        \put(-100,-100){\line( 1, 0){200} }%
        }%
      {\color{blue}%
        \put(-100,-100){\line( 0, 1){200} }%
        \put( 100,-100){\line( 0, 1){200} }%
        }%
    \end{picture}%
  }
  \put(-150,150){%
    \setlength{\unitlength}{1\tw/(400*3)}%
    \begin{picture}(0,0)(0,0)%
      %{\color{graphpaper}\graphpaper[10](-150,-150)(300,300)}%
      {\color{red}%
        \put(0,0){\vector( 1, 1){100} }%
        \put(0,0){\vector(-1,-1){100} }%
        }%
    \end{picture}%
  }
  \put(0,150){%
    \setlength{\unitlength}{1\tw/(400*3)}%
    \begin{picture}(0,0)(0,0)%
      %{\color{graphpaper}\graphpaper[10](-150,-150)(300,300)}%
      {\color{green}%
        \put(0,0){\vector( 1, 0){100} }%
        \put(0,0){\vector(-1, 0){100} }%
        }%
    \end{picture}%
  }
  \put(150,150){%
    \setlength{\unitlength}{1\tw/(400*3)}%
    \begin{picture}(0,0)(0,0)%
      %{\color{graphpaper}\graphpaper[10](-150,-150)(300,300)}%
      {\color{blue}%
        \put(0,0){\vector( 0, 1){100} }%
        \put(0,0){\vector( 0,-1){100} }%
        }%
    \end{picture}%
  }
  \put(0,25){%
    \setlength{\unitlength}{1\tw/(400*3)}%
    \begin{picture}(0,0)(0,0)%
      %{\color{graphpaper}\graphpaper[10](-150,-150)(300,300)}%
      {\color{black}%
        \put(0,0){\circle*{15}}%
        }%
    \end{picture}%
  }
\end{picture}
\end{center}
\end{minipage}
\caption{
  Euclidean 3-dimensional space partitioned as a power lattice
  \label{fig:psub_E3d_power_P}
  }
\end{figure}

%---------------------------------------
\begin{example}
\label{ex:psub_lat_alg_xyz}
%---------------------------------------
Suppose a random process is capable of producing three values
$\pso\eqd\setn{x,y,z}$.
There are five \hie{algebras of sets} on $\pso$
and therefore five probability spaces $(\pso,\,\pse_n,\, \psP)$ on $\pso$
with the five values of $\pse_n$ listed below:
\footnote{\hie{algebra of sets}: \prefp{def:set_algebra}} \\
\begin{minipage}[c]{\tw/3}
  %\begin{figure}[th]
  \begin{center}
  \footnotesize
  \setlength{\unitlength}{3\tw/900}%
  \begin{picture}(300,400)(-150,-50)%
    \thicklines
    %{\color{graphpaper}\graphpaper[50](-150,-50)(300,400)}%
    \color{red}%
      \put( 0,250){\line(-2,-1){100}}%
      \put( 0,250){\line( 0,-1){50}}%
      \put( 0,250){\line( 2,-1){100}}%
      \put( 0, 50){\line(-2,1){100}}%
      \put( 0, 50){\line( 0,1){50}}%
      \put( 0, 50){\line( 2,1){100}}%
  %
    \put(0,300){%
      \setlength{\unitlength}{3\tw/(3*1000)}%
      \begin{picture}(0,0)(0,150)%
      %{\color{graphpaper}\graphpaper[50](-100,0)(200,300)}%
      \thicklines%
      \color{black}%
        \put( 100, 300){\makebox(0,0)[tl]{$(\pso,\pse_5,\psP)$}}%
        \put(  15, 300){\makebox(0,0)[l]{$1$}}%
        \put(-115, 200){\makebox(0,0)[r]{$\frac{2}{3}$}}%
        \put(  15, 200){\makebox(0,0)[l]{$\frac{2}{3}$}}%
        \put( 115, 200){\makebox(0,0)[l]{$\frac{2}{3}$}}%
        \put(-115, 100){\makebox(0,0)[r]{$\frac{1}{3}$}}%
        \put(  15, 100){\makebox(0,0)[l]{$\frac{1}{3}$}}%
        \put( 115, 100){\makebox(0,0)[l]{$\frac{1}{3}$}}%
        \put(  15,   0){\makebox(0,0)[l]{$0$}}%
      \color{latline}%
        \put(   0, 300){\line(-1,-1){100} }%
        \put(   0, 300){\line( 0,-1){100} }%
        \put(   0, 300){\line( 1,-1){100} }%
        \put( 100, 100){\line( 0, 1){100} }%
        \put( 100, 100){\line(-1, 1){100} }%
        \put(   0, 100){\line(-1, 1){100} }%
        \put(   0, 100){\line( 1, 1){100} }%
        \put(-100, 100){\line( 0, 1){100} }%
        \put(-100, 100){\line( 1, 1){100} }%
        \put(   0,   0){\line(-1, 1){100} }%
        \put(   0,   0){\line( 0, 1){100} }%
        \put(   0,   0){\line( 1, 1){100} }%
      \color{latdot}%
        \put(   0, 300){\circle*{40}}%
        \put( 100, 200){\circle*{40}}%
        \put(   0, 200){\circle*{40}}%
        \put(-100, 200){\circle*{40}}%
        \put( 100, 100){\circle*{40}}%
        \put(   0, 100){\circle*{40}}%
        \put(-100, 100){\circle*{40}}%
        \put(   0,   0){\circle*{40}}%
      \end{picture}%
    }
    \put(-100,150){%
      \setlength{\unitlength}{3\tw/(3*1000)}%
      \begin{picture}(0,0)(0,150)%
      %{\color{graphpaper}\graphpaper[50](-100,0)(200,300)}%
      \thicklines%
      \color{black}%
        \put(   0, 330){\makebox(0,0)[b]{$(\pso,\pse_2,\psP)$}}%
        \put(  15, 300){\makebox(0,0)[l]{$1$}}%
        \put( 115, 200){\makebox(0,0)[l]{$\frac{2}{3}$}}%
        \put(-115, 100){\makebox(0,0)[r]{$\frac{1}{3}$}}%
        \put(  15,   0){\makebox(0,0)[l]{$0$}}%
      \color{latline}%
        \put(   0, 300){\line(-1,-1){100} }%
        \put(   0, 300){\line( 0,-1){100} }%
        \put(   0, 300){\line( 1,-1){100} }%
        \put( 100, 100){\line( 0, 1){100} }%
        \put( 100, 100){\line(-1, 1){100} }%
        \put(   0, 100){\line(-1, 1){100} }%
        \put(   0, 100){\line( 1, 1){100} }%
        \put(-100, 100){\line( 0, 1){100} }%
        \put(-100, 100){\line( 1, 1){100} }%
        \put(   0,   0){\line(-1, 1){100} }%
        \put(   0,   0){\line( 0, 1){100} }%
        \put(   0,   0){\line( 1, 1){100} }%
      \color{latdot}%
        \put(   0, 300){\circle*{40}}%
        \put( 100, 200){\circle*{40}}%
        \put(-100, 100){\circle*{40}}%
        \put(   0,   0){\circle*{40}}%
      \end{picture}%
    }
    \put(0,150){%
      \setlength{\unitlength}{3\tw/(3*1000)}%
      \begin{picture}(0,0)(0,150)%
      %{\color{graphpaper}\graphpaper[50](-100,0)(200,300)}%
      \thicklines%
      \color{black}%
        \put(   0, 330){\makebox(0,0)[b]{$(\pso,\pse_3,\psP)$}}%
        \put(  15, 300){\makebox(0,0)[l]{$1$}}%
        \put(  15, 200){\makebox(0,0)[l]{$\frac{2}{3}$}}%
        \put(  15, 100){\makebox(0,0)[l]{$\frac{1}{3}$}}%
        \put(  15,   0){\makebox(0,0)[l]{$0$}}%
      \color{latline}%
        \put(   0, 300){\line(-1,-1){100} }%
        \put(   0, 300){\line( 0,-1){100} }%
        \put(   0, 300){\line( 1,-1){100} }%
        \put( 100, 100){\line( 0, 1){100} }%
        \put( 100, 100){\line(-1, 1){100} }%
        \put(   0, 100){\line(-1, 1){100} }%
        \put(   0, 100){\line( 1, 1){100} }%
        \put(-100, 100){\line( 0, 1){100} }%
        \put(-100, 100){\line( 1, 1){100} }%
        \put(   0,   0){\line(-1, 1){100} }%
        \put(   0,   0){\line( 0, 1){100} }%
        \put(   0,   0){\line( 1, 1){100} }%
      \color{latdot}%
        \put(   0, 300){\circle*{40}}%
        \put(   0, 200){\circle*{40}}%
        \put(   0, 100){\circle*{40}}%
        \put(   0,   0){\circle*{40}}%
      \end{picture}%
    }
    \put(100,150){%
      \setlength{\unitlength}{3\tw/(3*1000)}%
      \begin{picture}(0,0)(0,150)%
      %{\color{graphpaper}\graphpaper[50](-100,0)(200,300)}%
      \thicklines%
      \color{black}%
        \put(   0, 330){\makebox(0,0)[b]{$(\pso,\pse_4,\psP)$}}%
        \put(  15, 300){\makebox(0,0)[l]{$1$}}%
        \put(-115, 200){\makebox(0,0)[r]{$\frac{2}{3}$}}%
        \put( 115, 100){\makebox(0,0)[l]{$\frac{1}{3}$}}%
        \put(  15,   0){\makebox(0,0)[l]{$0$}}%
      \color{latline}%
        \put(   0, 300){\line(-1,-1){100} }%
        \put(   0, 300){\line( 0,-1){100} }%
        \put(   0, 300){\line( 1,-1){100} }%
        \put( 100, 100){\line( 0, 1){100} }%
        \put( 100, 100){\line(-1, 1){100} }%
        \put(   0, 100){\line(-1, 1){100} }%
        \put(   0, 100){\line( 1, 1){100} }%
        \put(-100, 100){\line( 0, 1){100} }%
        \put(-100, 100){\line( 1, 1){100} }%
        \put(   0,   0){\line(-1, 1){100} }%
        \put(   0,   0){\line( 0, 1){100} }%
        \put(   0,   0){\line( 1, 1){100} }%
      \color{latdot}%
        \put(   0, 300){\circle*{40}}%
        \put(-100, 200){\circle*{40}}%
        \put( 100, 100){\circle*{40}}%
        \put(   0,   0){\circle*{40}}%
      \end{picture}%
    }
    \put(0,0){%
      \setlength{\unitlength}{3\tw/(3*1000)}%
      \begin{picture}(0,0)(0,150)%
      %{\color{graphpaper}\graphpaper[50](-100,0)(200,300)}%
      \thicklines%
      \color{black}%
        \put( 130, 150){\makebox(0,0)[l]{$(\pso,\pse_1,\psP)$}}%
        \put(  15, 300){\makebox(0,0)[l]{$1$}}%
        \put(  15,   0){\makebox(0,0)[l]{$0$}}%
      \color{latline}%
        \put(   0, 300){\line(-1,-1){100} }%
        \put(   0, 300){\line( 0,-1){100} }%
        \put(   0, 300){\line( 1,-1){100} }%
        \put( 100, 100){\line( 0, 1){100} }%
        \put( 100, 100){\line(-1, 1){100} }%
        \put(   0, 100){\line(-1, 1){100} }%
        \put(   0, 100){\line( 1, 1){100} }%
        \put(-100, 100){\line( 0, 1){100} }%
        \put(-100, 100){\line( 1, 1){100} }%
        \put(   0,   0){\line(-1, 1){100} }%
        \put(   0,   0){\line( 0, 1){100} }%
        \put(   0,   0){\line( 1, 1){100} }%
      \color{latdot}%
        \put(   0, 300){\circle*{40}}%
        \put(   0,   0){\circle*{40}}%
      \end{picture}%
    }
  \end{picture}
  \end{center}
\end{minipage}
\begin{minipage}[c]{2\tw/3}
  \begin{align*}
    \pse_{ 1} &=\{ && \emptyset, &&           &&           &&           &&             &&             &&             && \setX && \}\\
    \pse_{ 2} &=\{ && \emptyset, && \setn{x}, &&           &&           &&             &&             && \setn{y,z}, && \setX && \}\\
    \pse_{ 3} &=\{ && \emptyset, &&           && \setn{y}, &&           &&             && \setn{x,z}, &&             && \setX && \}\\
    \pse_{ 4} &=\{ && \emptyset, &&           &&           && \setn{z}, && \setn{x,y}, &&             &&             && \setX && \}\\
    \pse_{ 5} &=\{ && \emptyset, && \setn{x}, && \setn{y}, && \setn{z}, && \setn{x,y}, && \setn{x,z}, && \setn{y,z}, && \setX && \}\\
  \end{align*}
Suppose the samples of $\pso$ are generated by a physical process such that they are all
equally likely to occur.
Then by varying the \structe{\txsigma-algebra} $\pse_n$ effectively varies the probability
distribution of the probability space $(\pso,\, \pse_n,\, \psP)$.
This is illustrated by the figure to the left.
\end{minipage}
\end{example}













