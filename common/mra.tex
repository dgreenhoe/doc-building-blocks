%============================================================================
% Daniel J. Greenhoe
% LaTeX File
%============================================================================


%======================================
%\chapter{Multiresolution Analysis Structures}
\chapter{MRA Structures}
%======================================


%=======================================
\section{Introduction}
%=======================================
%In Fourier analysis, \prope{continuous} {dilations} \xref{def:opD} of the \fncte{complex exponential} \xref{def:exp}
%form a  \structe{basis} \xref{def:basis_schauder} for the \structe{space of square integrable functions} $\spLLR$ \xref{def:spLLR} 
%such that
%  \\\indentx$\ds\spLLR=\linspan\set{\opDil_\omega e^{ix}}{\scy\omega\in\R}$.
%
%In Fourier series analysis \xref{thm:opFSi}, \prope{discrete} dilations of the complex exponential 
%form a  basis for $\spLL{\intoo{0}{2\pi}}$ such that
%  \\\indentx$\ds\spLL{\intoo{0}{2\pi}}=\linspan\setjZ{\opDil_j e^{ix}}$.
%
%In Wavelet analysis, for some \fncte{mother wavelet} \xref{def:wavelet} $\fpsi(x)$,
%  \\\indentx$\ds\spLLR=\linspan\set{\opDil_\omega\opTrn_\tau \fpsi(x)}{\omega,\tau\in\R}$.
%
%However, the ranges of parameters $\omega$ and $\tau$ can be much reduced to the countable set $\Z$ resulting in
%a \prope{dyadic} wavelet basis such that for some mother wavelet $\fpsi(x)$,
%  \\\indentx$\ds\spLLR=\linspan\set{\opDil^j\opTrn^n \fpsi(x)}{j,n\in\Z}$.\\
%This text deals almost exclusively with dyadic wavelets. 
%Wavelets that are both \prope{dyadic} and \prope{compactly supported} have the attractive feature 
%that they can be easily implemented in hardware or software by use of the 
%\structe{Fast Wavelet Transform} \xref{fig:fwt}.
%
\begin{minipage}{\tw-65mm}
  In 1989, St{/'e}phane G. Mallat introduced the \structe{Multiresolution Analysis} (MRA, \prefp{def:mra})
  method for wavelet construction. The MRA has become the dominate wavelet construction method.
  This text uses the MRA method extensively, 
  and combines the MRA ``scaling subspaces" \xref{def:mra} with ``wavelet subspaces"\ifsxref{wavstrct}{def:seqWn} 
  to form a subspace structure as represented by the 
  \structe{Hasse diagram} to the right.
  The \structe{Fast Wavelet Transform} combines both sets of subspaces as well, 
  providing the results of projections onto both wavelet and MRA subspaces.
  %The MRA is not the only method of wavelet construction,
  Moreover, P.G. Lemari{/'e} has proved that all wavelets with \prope{compact support} are generated by an MRA.\footnotemark
\end{minipage}\hfill%
\tbox{\includegraphics{graphics/latwav.pdf}}
\footnotetext{
  \citor{lemarie1990},
  \citerpg{mallat}{240}{012466606X}
  }

%=======================================
%\section{Analyses}
%=======================================
The MRA is an \hib{analysis} of the linear space $\spLLR$.
An analysis of a linear space $\spX$ is any sequence $\seq{\spV_j}{j\in\Z}$ of linear subspaces of $\spX$.
%A sequence $\seq{\spV_j}{j\in\Z}$ of linear subspaces of a linear space $\spX$
%        is an \hib{analysis} of $\spX$.
        %if  $\seq{\spV_j}{j\in\Z}$ is a partition of $\spX$.
        The partial or complete reconstruction of $\spX$ from $\seq{\spV_j}{j\in\Z}$ is a \hib{synthesis}.%
        \footnote{%
          The word \hie{analysis} comes from the Greek word
          {\fntagreek{>av'alusis}},
          meaning ``dissolution" (\citerpc{perschbacher1990}{23}{entry 359}),
          which in turn means
          ``the resolution or separation into component parts"
          (\citer{collins2009}, \scs\url{http://dictionary.reference.com/browse/dissolution})
          }
  An analysis is completely \hie{characterized} by a \hie{transform}.
  For example, a Fourier analysis is a sequence of subspaces with sinusoidal bases.
  Examples of subspaces in a Fourier analysis include $\spV_1=\linspan\setn{e^{ix}}$, 
  $\spV_{2.3}=\linspan\setn{e^{i2.3x}}$, $\spV_{\sqrt{2}}=\linspan\setn{e^{i\sqrt{2}x}}$, etc.
  A \hib{transform} is loosely defined as a function that maps a family of functions
  into an analysis.
  A very useful transform (a ``\hie{Fourier transform}") for Fourier Analysis is \xref{def:opFT}
  \\\indentx$\ds\brs{\opFT\ff}(\omega) \eqd \frac{1}{\sqrt{2\pi}} \int_\R \ff(x) e^{-i\omega x} \dx$

%=======================================
%\section{Multiresolution analysis}
%=======================================
%=======================================
\section{Definition}
%=======================================
\ifdochasnot{transop}{%
Much of the wavelet theory developed in this section is constructed using the \opb{translation operator} $\opTrn$
and the \opb{dilation operator} $\opDil$ (next).
%These operators are illustrated below.
%and in \prefp{fig:haar-jn} for the \hie{Haar} MRA\cittr{haar1910}.
%
%---------------------------------------
\begin{definition}
\footnote{
  \citerppgc{walnut2002}{79}{80}{0817639624}{Definition 3.39},
  \citerppg{christensen2003}{41}{42}{0817642951},
  \citerpgc{wojtaszczyk1997}{18}{0521578949}{Definitions 2.3,2.4},
  \citerpg{kammler2008}{A-21}{0521883407},
  \citerpg{bachman2000}{473}{0387988998},
  \citerpg{packer2004}{260}{0821834029}, %{section 3.1}\\
  \citerpg{zayed2004}{}{0817643044},
  \citerpgc{heil2011}{250}{0817646868}{Notation 9.4},
  \citerpg{casazza1998}{74}{0817639594},
  \citerp{goodman1993}{639},
  \citerp{dai1996}{81},
  \citerpg{dai1998}{2}{0821808001}
  %\citerpg{dai1998}{21}{0821808001}
  }
\label{def:opT}
\label{def:opD}
\label{def:opTD}
%---------------------------------------
%Let $\spX\eqd\spLLR$ be the space of all \structe{square Lebesgue integrable functions} \xref{def:spLLR}.
%Let $\opTrn$ and $\opDil$ be operators in $\clOxx$.
%Let $\C$ be the set of complex numbers,
%and $\hxs{\spLLR}$ the set of all functions with range $\C$ and domain $\C$.
\defbox{\begin{array}{Frc>{\ds}lMlC}
    1. & \mc{6}{M}{$\hxs{\opTrn}$ is the \opd{translation operator} on $\spCC$ defined as}\\
       & \indentx\hxs{\opTrn_\tau}  \ff(x) &\eqd&         \ff(x-\tau) &and& \hxs{\opTrn}\eqd\opTrn_1 & \forall \ff\in\spCC
       \\
    2. & \mc{6}{M}{$\hxs{\opDil}$ is the \opd{dilation operator} on $\spCC$ defined as}\\
       & \indentx\hxs{\opDil_\alpha} \ff(x) &\eqd& \ff(\alpha x) &and& \opDil\eqd\sqrt{2}\opDil_2 & \forall \ff\in\spCC
\end{array}}
\end{definition}

\begin{center}\begin{tabular}{cc}
  \includegraphics{graphics/opTrn.pdf}&\includegraphics{graphics/opDil.pdf}%
\end{tabular}\end{center}
}



A multiresolution analysis provides ``coarse" approximations of a function in a linear space $\spLLR$ at multiple
``scales" or ``resolutions".
%\paragraph{Scaling function.}
Key to this process is a sequence of \hie{scaling functions}.
Most traditional transforms feature a single \hie{scaling function} $\fphi(x)$
set equal to one ($\fphi(x)=1$).
This allows for convenient representation of the most basic functions, such as constants.\footnote{\citePp{jawerth}{8}}
A multiresolution system, on the other hand, uses a generalized form of the scaling concept:
\begin{dingautolist}{"AC}
  \item Instead of the scaling function simply being set \emph{equal to unity} ($\fphi(x)=1$),
        a multiresolution system \xref{def:mrasys} is often constructed in such a way that the scaling function 
        $\fphi(x)$ forms a \hie{partition of unity}\ifsxref{partuni}{def:pun} such that
        $\sum_{n\in\Z} \opTrn^n\fphi(x) = 1$.
  \item Instead of there being \emph{just one} scaling function, there
        is an entire sequence of scaling functions $\seqjZ{\opDil^j\fphi(x)}$, 
        each corresponding to a different ``\hie{resolution}".
\end{dingautolist}

%--------------------------------------
\begin{definition}% [multiresolution system]
\footnote{
  \citerpg{hernandez1996}{44}{0849382742},
  \citerpgc{mallat}{221}{012466606X}{Definition 7.1},
  \citorp{mallat89}{70},
  \citorpgc{meyer1992}{21}{0521458692}{Definition 2.2.1},
  \citerpgc{christensen2003}{284}{0817642951}{Definition 13.1.1},
  \citerppgc{bachman2000}{451}{452}{0387988998}{Definition 7.7.6},
  \citerppgc{walnut2002}{300}{301}{0817639624}{Definition 10.16},
 %\citerppgc{vidakovic}{51}{52}{0471293652}{Riesz basis: footnote on page 52}\\
  \citerppgc{dau}{129}{140}{0898712742}{Riesz basis: page 139}
  %\citerppgc{christensen2003}{73}{74}{0817642951}{Definition 3.8.2}\\
  %\citerpgc{heil2011}{371}{0817646868}{Definition 12.8}\\
  %\citerpgc{walter}{38}{1584882271}{3.1 Multiresolution Analysis}
  }
\label{def:seqVn}
\label{def:mra}
\label{def:wavstrct_phi}
%--------------------------------------
%Let $\spLLR$ be the space of all \structe{square Lebesgue integrable functions} \xref{def:spLLR}.
Let $\seqjZ{\spV_j}$ be a sequence of subspaces on $\spLLR$ \xref{def:spLLR}.  %be a \prope{separable} \structe{Hilbert space}.
Let $\clsA$ be the \structe{closure} of a set $\setA$.
\\\defboxt{
  The sequence $\seqjZ{\spV_j}$ is a \structd{multiresolution analysis} on $\spLLR$ if
  \\
  $\begin{array}{@{\qquad}F>{\ds}lCDD}
   %1. & \spV_j \text{ is a linear subspace of $\spX$}\qquad \forall \spV_j\in\seqjZ{\spV_j}
    %\cnto & \mc{2}{M}{$\spLLR$ is \prope{complete}}                       & ($\spLLR$ is a \structe{Hilbert Space})         & and
    %\cntn & \mc{2}{M}{$\spLLR$ is \prope{separable}}                      &                                               & and
      1.  & \spV_j          = \cls{\spV_j}              & \forall j\in\Z                 & (\prope{closed})                              & and 
    \\2.  & \spV_j          \subset \spV_{j+1}          & \forall j\in\Z                 & (\prope{linearly ordered})                    & and 
    \\3.  & \clsp{\Setu_{j\in\Z} \spV_j} = \spLLR         &                                & (\prope{dense} in $\spLLR$)    & and 
   %\cntn & \Seti_{j\in\Z} \spV_j = \setn{\vzero}       &                                & (\structe{greatest lower bound} is $\spZero$) & and 
    \\4.  & \ff\in\spV_j \iff    \opDil\ff\in\spV_{j+1} & \forall j\in\Z,\,\ff\in\spLLR  & (\prope{self-similar})                        & and
   %\cntn & \ff\in\spV_j \iff    \opTrn\ff\in\spV_j     & \forall n\in\Z,\,\ff\in\spLLR  & (\prope{translation invariant})               & and 
    \\5.  & \mc{3}{l}{\ds\exists \fphi \st \setxZ{\opTrn^n\fphi} \text{ is a \structe{Riesz basis} for $\spV_0$.}}                       & 
  \end{array}$
  \\
  A \structe{multiresolution analysis} is also called an \hid{MRA}.\\
  An element $\spV_j$ of $\seqjZ{\spV_j}$ is a \hid{scaling subspace} of the space $\spLLR$.\\
  The pair $\hxs{\MRAspaceLLRV}$ is a \hid{multiresolution analysis space}, or \hid{MRA space}.\\
  The function $\hxs{\fphi}$ is the \hid{scaling function} of the \structe{MRA space}.
  }
\end{definition}

The traditional definition of the \structe{MRA} also includes the following:
  \\\indentx$\begin{array}{F>{\ds}lCD}
      \cntn & \ff\in\spV_j \iff    \opTrn^n\ff\in\spV_j     & \forall n,j\in\Z,\,\ff\in\spLLR  & (\prope{translation invariant})
      \cntn & \Seti_{j\in\Z} \spV_j = \setn{\vzero}         &                                  & (\structe{greatest lower bound} is $\spZero$)
  \end{array}$\\
However, \pref{prop:mra_transinvar} (next) demonstrates that
both of these follow from the \structe{MRA} as defined in \pref{def:mra}.

%--------------------------------------
\begin{proposition}
\footnote{
  \citerpgc{hernandez1996}{45}{0849382742}{Theorem 1.6},
  \citerppgc{wojtaszczyk1997}{19}{28}{0521578949}{Proposition 2.14},
  \citerppgc{pinsky2002}{313}{314}{0534376606}{Lemma 6.4.28}
  }
\label{prop:mra_transinvar}
\label{prop:mra_glb}
%--------------------------------------
\propbox{
  \brbr{\begin{array}{N}
    $\seqjZ{\spV_j}$ is an \structe{MRA}\\
    \xref{def:mra}
  \end{array}}
  \implies
  \brbl{\begin{array}{F>{\ds}lCD}
    1. & \ff\in\spV_j \iff    \opTrn^n\ff\in\spV_j   & \forall n,j\in\Z,\,\ff\in\spLLR  &(\prope{translation invariant}) and\\
    2. & \Seti_{j\in\Z} \spV_j = \setn{\vzero}       &                                  & (\structe{greatest lower bound} is $\spZero$)
  \end{array}}
  }
\end{proposition}
\begin{proof}
%\begin{enumerate}
Proof for (1):
\begin{align*}
  &\opTrn^n\ff\in\spV_j
  \\&\iff \opTrn^n\ff\in\linspan\set{\opDil^j\opTrn^m\fphi}{\scy m\in\Z}
          &&
          && \text{by definition of $\setn{\fphi}$} && \text{\xref{def:mra}}
  \\&\iff \exists \seqxZ{\alpha_n} \st \opTrn^n\ff(x)
          &&= \sum_{k\in\Z}\alpha_k\opDil^j\opTrn^k\fphi(x)
          && \text{by definition of $\setn{\fphi}$} && \text{\xref{def:mra}}
  \\&\iff \exists \seqxZ{\alpha_n} \st \ff(x)
          &&= \opTrn^{-n}\sum_{k\in\Z}\alpha_k\opDil^j\opTrn^k\fphi(x)
          && \text{by definition of $\opTrn$} && \text{\xref{def:opT}}
  \\&     &&= \sum_{k\in\Z}\alpha_k\opTrn^{-n}\opDil^j\opTrn^k\fphi(x)
          %&& \text{by \prefp{prop:opTD_sum}}
  \\&     &&= \sum_{k\in\Z}\alpha_k\opDil^j\opTrn^{k-2n}\fphi(x)
          && \text{by \thme{commutator relation}} && \text{\xref{prop:DTTD}}
  \\&     &&= \sum_{\ell\in\Z}\alpha_{\ell+2n}\opDil^j\opTrn^{\ell}\fphi(x)
          && \text{where $\ell\eqd k-2n\implies$} &&k=\ell+2n
  \\&     &&= \sum_{\ell\in\Z}\beta_{\ell}\opDil^j\opTrn^{\ell}\fphi(x)
          && \text{where $\beta_{\ell}\eqd\alpha_{\ell+2n}$}
  \\&     &&\iff \quad\ff\in\spV_j
          && \text{by def. of $\setn{\opTrn^n\fphi}$} && \text{\xref{def:mra}}
\end{align*}

Proof for (2):
\begin{enumerate}
  \item Let $\opP_j$ be the \ope{projection operator} that generates the scaling subspace $\spV_j$ such that \label{idef:mra_glb_Pj}
    \\\indentx$\ds\opV_j=\set{\opP_j\ff}{\ff\in\spLLR}$

  \item lemma: Functions with \prope{compact support} are \prope{dense} in $\spLLR$.
        \label{ilem:mra_glb_dense}
        Therefore, we only need to prove that the proposition is true for functions with support in $\intcc{-R}{R}$, for all $R>0$.

  \item For some function $\ff\in\spLLR$, let $\seqxZ{\ff_n}$ be a sequence of functions in $\spLLR$ 
        with \prope{compact support} such that
        \\\indentx
        $\support\ff_n\subseteq\intcc{-R}{R}$ for some $R>0$
        \quad and\quad
        $\ds\ff(x)=\lim_{n\to\infty}\seqn{\ff_n(x)}$.
        \label{idef:mra_glb_ffn}

  \item lemma: $\ds\Seti\spV_j=\setn{\vzero}\quad\iff\quad\lim_{j\to-\infty}\norm{\opP_j\ff}=0\quad{\scy\forall\ff\in\spLLR}$. Proof:  \label{ilem:mra_glb_VjPj}
    \begin{align*}
      \Seti_{j\in\Z}\spV_j 
        &= \Seti_{j\in\Z}\set{\opP_j\ff}{\ff\in\spLLR}
        && \text{by definition of $\spV_j$} && \text{\xref{idef:mra_glb_Pj}}
      \\&= \lim_{j\to-\infty}\set{\opP_j\ff}{\ff\in\spLLR}
        && \text{by definition of $\seti$} && \text{\ifxref{setstrct}{def:setop}}
      \\&= \vzero
      \iff \lim_{j\to-\infty}\norm{\opP_j\ff}=0
        && \text{by \prope{nondegenerate} property of $\normn$} && \text{\xref{def:norm}}
    \end{align*}
  
  \item lemma: $\ds\lim_{j\to-\infty}\norm{\opP_j\ff}=0\quad{\scy\forall\ff\in\spLLR}$. Proof:\\
        Let $\setindAx$ be the \fncte{set indicator function} \xref{def:setind} \label{ilem:mra_glb_norm}
    \begin{align*}
      &\lim_{j\to-\infty}\norm{\opP_j\ff}^2
      \\&=   \lim_{j\to-\infty}\norm{\opP_j\lim_{n\to\infty}\seqn{\ff_n}}^2
        &&   \text{by \prefp{idef:mra_glb_ffn}}
      \\&\le \lim_{j\to-\infty}B\sum_{n\in\Z}\abs{\inprod{\opP_j\lim_{n\to\infty}\seqn{\ff_n}}{\opDil^j\opTrn^n\fphi}}^2
        &&   \text{by \prope{frame property}} && \text{\ifxref{frames}{prop:rbasis_frame}}
      \\&=   \lim_{j\to-\infty}B\sum_{n\in\Z}\abs{\inprod{\lim_{n\to\infty}\seqn{\ff_n}}{\opDil^j\opTrn^n\fphi}}^2
        &&   \text{by definition of $\opP_j$} && \text{\xref{idef:mra_glb_Pj}}
      \\&=   \lim_{j\to-\infty}B\sum_{n\in\Z}\abs{\inprod{\setind_\intcc{-R}{R}(x)\lim_{n\to\infty}\seqn{\ff_n}}{\opDil^j\opTrn^n\fphi(x)}}^2
        &&   \text{by definition of $\seqn{\ff_n}$}&& \text{\xref{idef:mra_glb_ffn}}
      \\&=   \lim_{j\to-\infty}B\sum_{n\in\Z}\abs{\inprod{\lim_{n\to\infty}\seqn{\ff_n}}{\setind_\intcc{-R}{R}(x)\opDil^j\opTrn^n\fphi(x)}}^2
        &&   \text{prop. of $\inprodn$ in $\spLLR$}&& \text{ \xref{def:spLLR}}
      \\&\le \lim_{j\to-\infty}B\sum_{n\in\Z}\norm{\lim_{n\to\infty}\seqn{\ff_n}}^2\norm{\setind_\intcc{-R}{R}(x)\opDil^j\opTrn^n\fphi(x)}^2
        &&   \text{by \thme{CS Inequality}}&& \text{\ifxref{vsinprod}{thm:cs}}
      \\&=   \lim_{j\to-\infty}B\sum_{n\in\Z}\norm{\ff}^2\norm{\setind_\intcc{-R}{R}(x)\opDil^j\opTrn^n\fphi(x)}^2
        &&   \text{by definition of $\seqn{\ff_n}$}&& \text{\xref{idef:mra_glb_ffn}}
      \\&=   \lim_{j\to-\infty}B\sum_{n\in\Z}\norm{\ff}^2\norm{\brs{\mcom{\opDil^j\opDil^{-j}}{$\opI$}\setind_\intcc{-R}{R}(x)}\brs{\opDil^j\opTrn^n\fphi(x)}}^2
        &&   \text{by property of $\opDil$} && \text{\xref{prop:opDi}}
      \\&=   \lim_{j\to-\infty}B\sum_{n\in\Z}\norm{\ff}^2\norm{2^{j/2}\opDil^j\brb{\brs{\opDil^{-j}\setind_\intcc{-R}{R}(x)}\brs{\opTrn^n\fphi(x)}}}^2
        &&   \mathrlap{\text{by \prefp{prop:DjTnfg}}}
      \\&=   \lim_{j\to-\infty}B\sum_{n\in\Z}\norm{\ff}^2\norm{\opDil^j\brb{2^{j/2}2^{-j/2}\setind_\intcc{-R}{R}(2^{-j}x)\brs{\opTrn^n\fphi(x)}}}^2
        &&   \text{by property of $\opDil$} && \text{\xref{prop:opDi}}
      \\&=   \lim_{j\to-\infty}B\sum_{n\in\Z}\norm{\ff}^2\norm{\opDil^j\brb{\brs{\mcom{\opTrn^n\opTrn^{-n}}{$\opI$}\setind_\intcc{-R}{R}(2^{-j}x)}\brs{\opTrn^n\fphi(x)}}}^2
        &&   \text{by property of $\opTrn$} && \text{\xref{prop:opTi}}
      \\&=   \lim_{j\to-\infty}B\sum_{n\in\Z}\norm{\ff}^2\norm{\opDil^j\brb{\brs{\opTrn^n\setind_\intcc{-R}{R}(2^{-j}x+n)}\brs{\opTrn^n\fphi(x)}}}^2
        &&   \text{by property of $\opTrn$} && \text{\xref{prop:opTi}}
      \\&=   \lim_{j\to-\infty}B\sum_{n\in\Z}\norm{\ff}^2\norm{\opDil^j\opTrn^n\brb{\setind_\intcc{-R}{R}(2^{-j}x+n)\fphi(x)}}^2
        &&   \text{by property of $\opDil$} && \text{\xref{prop:opDi}}
      \\&=   \lim_{j\to-\infty}B\sum_{n\in\Z}\norm{\ff}^2\norm{\setind_\intcc{-R}{R}(2^{-j}x+n)\fphi(x)}^2
        &&   \text{by \prope{unitary} prop.} && \text{\xref{thm:TD_unitary}}
      \\&=   B\norm{\ff}^2\sum_{n\in\Z}\lim_{j\to-\infty}\norm{\setind_\intcc{-2^jR+n}{2^jR+n}(u)\fphi(2^{-j}(u-n))}^2
        &&   \text{$u\eqd 2^jx+n\implies$} && x=2^{-j}(u-n)
      \\&=   B\norm{\ff}^2\sum_{n\in\Z}\lim_{j\to-\infty}\int_{-2^jR+n}^{2^jR+n}\abs{\fphi(2^{-j}(u-n))}^2\du
      \\&=   B\norm{\ff}^2\sum_{n\in\Z}\int_{n}^{n}\abs{\fphi(0)}^2\du
      \\&=   0
    \end{align*}

  \item Final step in proof that $\ds\Seti\spV_j=\setn{\vzero}$: by \prefp{ilem:mra_glb_VjPj} and \prefp{ilem:mra_glb_norm}
%\end{enumerate}
\end{enumerate}
\end{proof}



%---------------------------------------
\begin{proposition}
\citetbl{
  \citerppgc{wojtaszczyk1997}{28}{31}{0521578949}{Proposition 2.15}
  }
\label{prop:mra_UVj}
%--------------------------------------
\mbox{}\\%Let a \structe{Riesz sequence} be defined as in \prefp{def:rieszseq}.
\propbox{
  \brb{\begin{array}{FMD}
    (1). & $\seqn{\opTrn^n\fphi}$ is a \structe{Riesz sequence} &  and \\
    (2). & $\Fphi(\omega)$ is \prope{continuous} at $0$         &  and \\
    (3). & $\Fphi(0)\neq0$                                      &  
  \end{array}}
  \implies
  \brb{\begin{array}{>{\ds}lD}
     \cls{\brp{\Setu_{j\in\Z} \spV_j}} = \spLLR  & (\prope{dense} in $\spLLR$) 
  \end{array}}
  }
\end{proposition}
\begin{proof}
\begin{enumerate}
  \item Let $\opP_j$ be the \ope{projection operator} that generates the scaling subspace $\spV_j$ such that \label{item:mra_UVj_Pj}
    \\\indentx$\ds\opV_j=\set{\opP_j\ff}{\ff\in\spH}$

  \item definition: Choose $\ff\in\spLLR$ such that $\ff\orthog\Setu_{j\in\Z}\spV_j$.
        Let $\Ff(\omega)$ be the \ope{Fourier Transform} \xref{def:opFT} of $\ff(x)$.
        \label{idef:mra_UVj_f}

  \item lemma: The function $\ff$ \xref{idef:mra_UVj_f} \emph{exists} because the set of functions that 
        can be chosen to be $\ff$ at least contains $0$ (it is not the emptyset). Proof:
        \label{ilem:mra_UVj_fexists}
        \begin{align*}
          \ff(x)=0
            &\implies \inprodr{\ff}{\set{\fh\in\spLLR}{\fh\in\Setu_{j\in\Z}\spV_j}}
          \\&= \inprodr{0}{\set{\fh\in\spLLR}{\fh\in\Setu_{j\in\Z}\spV_j}}
          \\&= 0
          \\&\implies\quad \ff\orthog\Setu_{j\in\Z}\spV_j
          \\&\implies\quad \text{$\ff$ exists}
        \end{align*}

  \item lemma: $\norm{\opP_j\ff}=0\quad{\scy\forall j\in\Z}$. Proof:
        \label{ilem:mra_UVj_Pf}
    \begin{align*}
      \norm{\opP_j\ff}
        &= \norm{0}
        && \text{by definition of $\ff$} &&\text{\xref{idef:mra_UVj_f}}
      \\&= 0
        && \text{by \prope{nondegenerate} property of $\normn$} && \text{\ifxref{vsnorm}{def:norm}}
    \end{align*}

  \item definition: Choose some function $\fg\in\spLLR$ such that $\Fg(\omega)=\Ff(\omega)\setind_\intcc{-R}{R}$ \xref{def:setind} 
        for some $R>0$ 
        and such that $\norm{\ff-\fg}<\varepsilon$.
        Let $\Fg(\omega)$ be the \ope{Fourier Transform} \xref{def:opFT} of $\fg(x)$.
        \label{idef:mra_UVj_g}

  \item lemma: The function $\fg$ \xref{idef:mra_UVj_g} \emph{exists}. Proof: For some (possibly very large) $R$,
        \label{idef:mra_UVj_gexists}
    \begin{align*}
      \varepsilon
        &> \norm{\Ff(\omega)-\Fg(\omega)}
        && \text{by definition of $\fg$} && \text{\xref{idef:mra_UVj_g}}
      \\&= \norm{\opFT\ff(x)-\opFT\fg(x)}
        && \text{by definition of $\Ff$ and $\Fg$} && \text{\xref{idef:mra_UVj_f}, \xref{idef:mra_UVj_g}}
      \\&= \norm{\opFT\brs{\ff(x)-\fg(x)}}
        && \text{by \prope{linearity} of $\opFT$} && \text{\xref{def:linop}}
      \\&= \norm{\ff(x)-\fg(x)}
        && \text{by \prope{unitary} property of $\opFT$} && \text{ \xref{thm:ft_unitary}}
      \\&\implies\quad\text{$\fg$ exists}
        && \mathrlap{\text{because it's possible to satisfy \prefp{idef:mra_UVj_g}}}
    \end{align*}

  \item lemma: $\norm{\opP_j\fg}<\varepsilon\quad{\scy\forall j\in\Z}$ for sufficiently large $R$. Proof:
        \label{ilem:mra_UVj_ge}
    \begin{align*}
      \varepsilon
        &>   \norm{\ff-\fg}
        &&   \text{by definition of $\fg$} && \text{\xref{idef:mra_UVj_g}}
      \\&\ge \norm{\opP_j\brs{\ff-\fg}}
        &&   \text{by property of \ope{projection operator}s} && \text{\xref{def:opP}}
      \\&=   \norm{\opP_j\ff-\opP_j\fg}
        &&   \text{by \prope{additive} property of $\opP_j$} && \text{\ifxref{operator}{def:linop}}
      \\&\ge \abs{\norm{\opP_j\ff}-\norm{\opP_j\fg}}
        &&   \text{by \thme{Reverse Triangle Inequality}} && \text{\ifxref{vsnorm}{thm:rti}}
      \\&=   \abs{0-\norm{\opP_j\fg}}
        &&   \text{by \xref{ilem:mra_UVj_Pf}}
      \\&=   \norm{\opP_j\fg}
        &&   \text{by \prope{strictly positive} property of $\normn$} && \text{\xref{def:norm}}
    \end{align*}

  \item  lemma: $\fg=0$. Proof: \label{ilem:mra_UVj_g0}
    \begin{align*}
       0
        &=   \lim_{j\to\infty}\norm{\opP_j\fg}^2
        &&   \text{by \prefp{ilem:mra_UVj_ge}}
      \\&\ge \lim_{j\to\infty}A\sum_{n\in\Z}\abs{\inprod{\opP_j\fg}{\opDil^j\opTrn^n\fphi}}^2
        &&   \text{by \prope{frame property}} && \text{\ifxref{frames}{prop:rbasis_frame}}
      \\&=   \lim_{j\to\infty}A\sum_{n\in\Z}\abs{\inprod{\fg}{\opDil^j\opTrn^n\fphi}}^2
        &&   \text{by definition of $\opP_j$} && \text{ \xref{item:mra_UVj_Pj}}
      \\&=   \lim_{j\to\infty}A\sum_{n\in\Z}\abs{\inprod{\opFT\fg}{\opFT\opDil^j\opTrn^n\fphi}}^2
        &&   \text{by \prope{unitary} property of $\opFT$} && \text{ \xref{thm:ft_unitary}}
      \\&=   \lim_{j\to\infty}A\sum_{n\in\Z}\abs{\inprod{\Fg(\omega)}{2^{-j/2}e^{-i2^{-j}\omega n}\Fphi(2^{-j}\omega)}}^2
        &&   \text{by \prefp{prop:FTDf}}
      \\&=   \lim_{j\to\infty}A\sum_{n\in\Z}\abs{\inprod{\Fg(\omega)\Fphi^\ast(2^{-j}\omega)}{2^{-j/2}e^{-i2^{-j}\omega n}}}^2
        &&   \text{by property of $\inprodn$ in $\spLLR$}
      \\&=   \lim_{j\to\infty}A\norm{\Fg(\omega)\Fphi^\ast(2^{-j}\omega)}^2
        &&   \text{by \thme{Parseval's Identity}} && \text{\ifxref{frames}{thm:fst}}
      \\&=   A\norm{\Fg(\omega)\Fphi^\ast(0)}^2
        &&   \text{by left hypothesis (2)}
      \\&=   A\abs{\Fphi^\ast(0)}^2\,\norm{\Fg(\omega)}^2
        &&   \text{by \prope{homogeneous} property of $\normn$} && \text{\ifxref{vsnorm}{def:norm}}
      \\&=   A\abs{\Fphi(0)}^2\,\norm{\fg}^2
        &&   \text{by \prope{unitary} property of $\opFT$} && \text{\xref{thm:ft_unitary}}
      \\&\implies \norm{\fg}=0
        &&   \text{by left hypothesis (3)}
      \\&\iff     \fg=0
        &&   \text{by \prope{nondegenerate} property of $\normn$} && \text{\ifxref{vsnorm}{def:norm}}
      %\\&\implies  \cls{\brp{\Setu_{j\in\Z}\spV_j}}=\spLLR
    \end{align*}

  \item Final step in proof that $\ds\cls{\brp{\Setu_{j\in\Z} \spV_j}} = \spLLR$:
    \begin{align*}
      \fg
        &=0
        && \text{by \prefp{ilem:mra_UVj_g0}}
      \\&\implies\ff=0
        && \text{by definition of $\fg$} && \text{\xref{idef:mra_UVj_g}}
      \\&\implies \cls{\brp{\Setu_{j\in\Z} \spV_j}} = \spLLR
    \end{align*}
\end{enumerate}
\end{proof}

%=======================================
%\subsection{Separable Hilbert Space}
%=======================================
\pref{def:mra} defines an MRA on the space $\spLLR$, which is a special case of a \structe{separable Hilbert space}.
A Hilbert space\ifsxrefs{seq}{def:hilbert}is a \structe{linear space}\ifsxrefs{vector}{def:vspace}that is 
equipped with an \structe{inner product}\ifsxref{vsinprod}{def:inprod},
is \prope{complete}\ifsxrefs{seq}{def:complete}with respect to the 
\structe{metric}\ifsxrefs{metric}{def:metric}induced by the inner product,
and contains a subset that is \prope{dense}\ifsxrefs{topology}{def:dense}in $\spLLR$.

An \structe{inner product} on a linear space endows the linear space with a \structe{topology}\ifsxref{topology}{def:topology}.
The sum such as $\sum_{n=1}^\xN \alpha_n \ff_n$ is finite and thus suitable for a finite linear space only.
An infinite space requires an infinite sum $\sum_{n=1}^\infty \alpha_n \fphi_n$, and an infinite sum is defined
in terms of a limit\ifsxref{series}{def:suminf}.
%  \\\indentx$\ds\sum_{n=1}^\infty \alpha_n \fphi_n \eqd \lim_{\xN\to\infty}\mcom{\ds\sum_{n=1}^\xN \alpha_n \fphi_n}{partial sum}$.\\
The limit, in turn, is defined in terms of a \structe{topology}\ifsxref{topology}{def:topology}.
The \structe{inner product}\ifsxrefs{vsinprod}{def:inprod} induces a \structe{norm} \xref{def:norm} which induces a 
\structe{metric}\ifsxrefs{metric}{def:metric} which induces a topology\ifsxref{metric}{thm:(X,d)->(X,t)}.

%A common example of a separable Hilbert space is the space of square integrable functions, $\spLLR$.
%And in fact, for the design examples in this book, the reader may simply set $\spLLR=\spLLR$.

%%---------------------------------------
%\begin{proposition}
%\label{prop:Vn_separable}
%%---------------------------------------
%Let $\MRAspaceLLRV$ be an \structe{MRA space}.
%\propbox{
%  \text{$\spV_j$ is \prope{separable}}\qquad\scy\forall j\in\Z
%  }
%\end{proposition}
%\begin{proof}
%\begin{enume}
%  \item By \pref{def:mra}, $\spLLR$ is \prope{separable}.
%  \item So by \prefp{thm:XdYd_separable}, each $\spV_j$ is \prope{separable} as well.
%\end{enume}
%\end{proof}
%


%=======================================
%\subsection{Closure properties}
%=======================================
\pref{def:mra} defines each subspace $\spV_j$ to be \prope{closed} ($\spV_j=\cls{\spV_j}$) in $\spLLR$.
As one might imagine, the properties of \prope{completeness}\ifsxrefs{seq}{def:complete}and 
\prope{closure}\ifsxrefs{topology}{def:clsA}%, \prefp{def:subspace_closed}
are closely related. % (see next proposition).
Moreover, Every \prope{complete} sequence is also \prope{bounded}\ifsxref{metric}{def:bounded},
and so each subspace $\spV_j$ is \prope{bounded} as well. % (see \prefp{prop:Vn_bounded}).
%Both are topological properties. Completeness is defined on sequences \xref{def:sequence}; %closure is defined on sets.



%%---------------------------------------
%\begin{proposition}
%%---------------------------------------
%Let $\MRAspaceLLRV$ be an \structe{MRA space}.
%\propbox{
%  \mcom{\spLLR=\cls\spLLR}{$\spLLR$ is \prope{closed}.}
%  }
%\begin{proof}
%        The limit of an expansion (if the limit exists) may be inside the linear space or outside. \label{item:mra_Hcomplete}
%        We would like it to be inside. That is, we would like the space $\spLLR$ to contain all its 
%        \structe{limit points} \xref{def:limitpnt}.
%        The space $\spLLR$ does contain all its limit points because by definition, it is \propb{complete} \xref{def:complete}.
%        Any metric space (which includes all inner product spaces) that is \prope{complete} is also \prope{closed}
%        \xref{thm:comcls}.
%        And a metric space is \prope{closed} if and only if it contains all its limit points \xref{thm:cst}.
%        An inner product space that is \prope{complete} is called a \structe{Hilbert space} \xref{def:hilbert}.
%\end{proof}

%---------------------------------------
\begin{proposition}
\label{prop:Vn_complete}
%---------------------------------------
Let $\MRAspaceLLRV$ be an \structe{MRA space}.
\propbox{
  \text{Each subspace $\spV_j$ is \prope{complete}.}
  }
\end{proposition}
\begin{proof}
\begin{enume}
  \item By definition \pref{def:mra}, $\spLLR$ is \prope{complete}. 
  \item In any metric space, (which includes all inner product spaces such as $\spLLR$),
        a \prope{closed} subspace of a \prope{complete} metric space is itself also \prope{complete}\ifsxref{seq}{thm:comcls}.
  \item In any \prope{complete} metric space $\spX$ (which includes all Hilbert spaces such as $\spLLR$), 
        the two properties coincide---that is, a subspace is complete \emph{if and only if} 
        it is closed in the space $\spX$\ifsxref{seq}{cor:comcomcls}.
  \item So because $\spLLR$ is \prope{complete} and each $\spV_j$ is \prope{closed}, then each $\spV_j$ is also \prope{complete}.
\end{enume}
\end{proof}

%%---------------------------------------
%\begin{proposition}
%\label{prop:Vn_bounded}
%%---------------------------------------
%Let $\MRAspaceLLRV$ be an \structe{MRA space}.
%\propbox{\begin{array}{MMC}
%  $\spLLR$   & is \prope{bounded}.\\
%  $\spV_j$ & is \prope{bounded} & \forall n\in\Z .
%\end{array}}
%\end{proposition}
%\begin{proof}
%\begin{enume}
%  \item Every \prope{complete} metric space is \prope{bounded}\ifsxref{seq}{thm:convergent==>cauchy}.
%  \item $\spLLR$ is \prope{complete}, so it is also \prope{bounded} \xref{def:mra}.
%  \item Each $\spV_j$ is \prope{complete}, so each $\spV_j$ is also \prope{bounded} \xref{prop:Vn_complete}.
%\end{enume}
%\end{proof}

%=======================================
\section{Order structure}
%=======================================

\begin{minipage}{\tw-58mm}%
  A \structe{multiresolution analysis} \xref{def:mra} together with the set inclusion relation $\subseteq$
  forms the \hie{linearly ordered set} \ifdochas{order}{\xref{def:toset}}
  $\hxs{\opair{\seqn{\spV_j}}{\subseteq}}$, illustrated to the right by a \structe{Hasse diagram}\ifsxref{order}{def:hasse}.
  Subspaces $\spV_j$ increase in ``size" with increasing $j$.
  That is, they contain more and more vectors (functions) for larger and larger $j$---%
  with the upper limit of this sequence being $\spLLR$.
  %and the subspace $\spZero$ (smallest $n$) containing only the $\vzero$ vector.
  Alternatively, we can say that approximation within a subspace $\spV_j$ 
  yields greater ``\hie{resolution}" for increasing $j$.
  %In general, the number of subspaces in such a sequence can be countably infinite (e.g. $n\in\Z$).
\end{minipage}%
\hfill\tbox{\includegraphics{graphics/latmra.pdf}}%

The \structe{least upper bound} (\structe{l.u.b.}) of the linearly ordered set $\opair{\seqn{\spV_j}}{\subseteq}$ is $\spLLR$ \xref{def:mra}:
  \\\indentx
   $\ds\clsp{\Setu_{j\in\Z} \spV_j} = \spLLR$.
   %$\ds\lim_{\xN\to\infty}\spV_j \eqd \clsp{\Setu_{j\in\Z} \spV_j} = \spLLR$.
  \\

%      \propb{upper bounded}:
%  Furthermore, the property $\clsp{\Setu_{j\in\Z} \spV_j} = \spLLR$
%  demonstrates that the sequence of scaling subspaces $\seqn{\spV_j}$ is \prope{upper bounded} by $\spLLR$. % \xref{def:complete_set}.
%  Because the subspaces are nested (or linearly ordered with respect to $\subset$) such that $\spV_j\subset\spV_{j+1}$,
%  we could define the least upper bound (or the limit) of such a sequence 
%  as\citetbl{Many thanks to William Elliot, David C. Ullrich, and Seymour J. Shmuel Metz for help with this topic.
%             %\url{https://groups.google.com/forum/\#!topic/sci.math/YD4N58JH5to}
%            }

The \structe{greatest lower bound} (\structe{g.l.b.}) of the linearly ordered set $\opair{\seqn{\spV_j}}{\subseteq}$ is $\spZero$ \xref{prop:mra_glb}:
  \\\indentx
    $\ds\Seti_{j\in\Z}\spV_j = \spZero$.
  \\

All linear subspaces contain the zero vector\ifsxref{subspace}{prop:subspace_prop}.
So the intersection of any two subspaces must at least contain $\vzero$.
If the intersection of any two linear subspaces $\spX$ and $\spY$ is exactly $\setn{\vzero}$, 
then for any vector in
the sum of those subspaces ($\vu\in\spX\adds\spY$) there are \propb{unique} vectors $\ff\in\spX$ and 
$\fg\in\spY$ such that $\vu=\ff+\fg$.
This is \emph{not} necessarily true if the intersection contains more than just $\setn{\vzero}$
\ifsxref{subspace}{thm:XY0_unique}.


%%=======================================
%\subsection{Bases for wavelet system}
%%=======================================
%%%A linear space is a separable Hilbert space if and only if it has a complete basis.
%%Note that \pref{def:mra} does not require $\lim_{\xN\to\infty}\spV_j$ to be equal to $\spLLR$, 
%%it is only requires it to be \prope{dense} in $\spLLR$
%%(just as the rationals are dense in the real numbers).
%%In the set of real numbers, a countable union of closed sets is called a $\symx{\setFsigma}$ set
%%($\setF$ stands for the French word \hie{ferm/'e} or \prope{closed}, and $\sigma$ stands for the French word \hie{somme} or sum).\citetbl{
%%  \citerpg{carothers2000}{130}{0521497566}\\
%%  \citerpg{givant2009}{270}{0387402934}
%%  }                                                                             
%
%\prefp{def:mra} defines an MRA on the space $\spLLR$.
%The space $\spLLR$ is an example of a Hilbert space. % $\spLLR$. 
%A Hilbert space is a linear space equipped with an inner product 
%and that is complete with respect to the topology induced by the inner product.
%
%\begin{enumerate}
%  \item A \structb{linear space} \ifsxrefs{vector}{def:vspace} supports the expansion of a vector $\ff$
%        (e.g. a function) in terms of a set of \structe{coordinates} $\setxn{\alpha_n}$ and a 
%        \structe{Hamel basis} $\setxn{\fphi_n}$\ifsxrefs{frames}{def:hamel}such that \label{item:mra_hamel}
%          \\\indentx$\ds \ff(x)=\sum_{n=1}^\xN \alpha_n \fphi_n(x)$.\\
%        If such coordinates exist for a vector $\ff$ and basis $\setxn{\fphi_n}$, 
%        then those coordinates are \prope{unique}\ifsxref{frames}{thm:hamel_unique}.
%
%  \item The Hamel basis described in \pref{item:mra_hamel} provides sufficient support for expansion in finite linear spaces, 
%        but is problematic in infinite spaces.                           
%        In an infinite linear space with a topology (such as a Banach space or a Hilbert space),
%        a \structe{Schauder basis}\ifsxrefs{frames}{def:schauder}is often used.
%        The Schauder basis is defined in terms of a special type of convergence called \prope{strong convergence}\ifsxref{seq}{def:strong_converge}.
%        Strong convergence is defined in terms of the norm induced by the inner product: \label{item:mra_strong}
%        \\\indentx$\ds  
%          \ff \eqs \sum_{n=1}^\infty\alpha_n\fphi_n
%              \eqd \lim_{\xN\to\infty}\sum_{n=1}^\xN\alpha_n\fphi_n
%              \implies
%              \mcom{\ds\lim_{\xN\to\infty}\norm{\ff-\sum_{n=1}^\xN\alpha_n\fphi_n}=0}{\prope{strong convergence}}
%        $.\\
%        %\\\indentx$\ds  
%        %  \ff \eqs \sum_{n=1}^\infty\inprod{\ff}{\fphi_n}\fphi_n
%        %      \eqd \lim_{\xN\to\infty}\sum_{n=1}^\xN\inprod{\ff}{\fphi_n}\fphi_n
%        %      \implies
%        %      \mcom{\ds\lim_{\xN\to\infty}\norm{\ff-\sum_{n=1}^\xN\inprod{\ff}{\fphi_n}\fphi_n}=0}{\prope{strong convergence}}
%        %$.\\
%        That is, the sum $\sum_{n=1}^\infty\alpha_n\fphi_n$ is by definition  
%        the limit of the partial sums
%        $\sum_{n=1}^\xN\alpha_n\fphi_n$\ifsxref{series}{def:suminf},
%        and that these sums \prope{converge strongly} (``$\eqs$",\ifsxref{seq}{def:strong_converge}) to a limit $\ff$
%        with respect to the topology induced by the norm $\normn$, which in turn is 
%        induced by the inner product $\inprodn$. % \xref{def:norm=inprod}.
%        The completeness property ensures that all of these limits $\ff$ are also in the space $\spLLR$.
% 
%  \item In an MRA space $\MRAspaceLLRV$, the space $\spLLR$ is separable \xref{def:mra}, and the subspaces $\spV_j$ are
%        separable as well. % \xref{prop:Vn_separable}.
%        The property of a space being separable is very important in analysis:
%    \begin{enumerate}
%      \item Every Banach space (which includes all Hilbert spaces such as $\spLLR$ and each $\spV_j$) with 
%            a Schauder basis is \prope{separable}\ifsxref{frames}{thm:Bschauder==>separable}.\label{item:mra_Bschauder_separable}
%      \item The converse is \emph{not} true---not every separable Banach space has a basis\ifsxrefpo{frames}{BasisProblem}.
%    \end{enumerate}
%
%  \item Besides providing a topology, the \structe{inner product} also supports 
%        the notion of a subspace geometry, 
%        including the property of \prope{orthogonality}\ifsxref{vsinprod}{def:orthog}.
%    \begin{enumerate}
%      \item Orthogonality supports the \structe{Fourier expansion}\ifsxrefs{frames}{def:hspace_fex}of a vector $\ff$ over an 
%        \structe{orthornormal basis} $\setxZp{\fphi_n}$ in the form \label{item:mra_inprod}
%        \\\indentx$\ds  
%          \ff \eqs \sum_{n=1}^\infty \mcom{\inprod{\ff}{\fphi_n}}{\structe{Fourier coefficient}}\fphi_n
%          $
%      \item In contrast to \pref{item:mra_Bschauder_separable}, life in Hilbert spaces is much simpler. 
%            A Hilbert space has a Schauder basis \emph{if and only if} it is 
%            separable \ifsxrefs{frames}{thm:schauder<==>separable}. 
%            And so $\spLLR$ and each $\spV_j$ \emph{have} Schauder bases\ifsxref{frames}{thm:schauder<==>separable}.
%      \item A special case of a Schauder basis is an orthonormal basis\ifsxref{frames}{def:basis_ortho}.
%            A Hilbert space has an orthonormal basis if and only if it is separable\ifsxref{frames}{thm:ortho<==>separable}.
%            And so $\spLLR$ and each $\spV_j$ \emph{have} orthonormal bases as well% 
%            \ifdochas{frames}{ (\xref{thm:schauder<==>separable}, \xref{thm:HV_orthobasis})}.
%      \item It is always possible to construct an orthonormal basis for a separable Hilbert space using the 
%            \thme{Graham Schmidt orthogonalization} procedure.
%    \end{enumerate} 
%
%
%
%\item 
%  %Note that the intersection of any two orthogonal subspaces contains the zero vector only \xref{thm:YoZ==>YZ0}.
%  A basis\ifsxrefs{frames}{sec:hspace_bases} $\setxn{\fphi_n}$ that is \prope{orthonormal}  possesses a number of useful properties
%  including the following:
%    \begin{enumerate}
%      \item The \thmb{Pathogorean Theorem} holds such that  $\norm{\sum_{n=1}^\xN \fphi_{n}}^2 = \sum_{n=1}^\xN \norm{\fphi_n}^2$\ifsxref{frames}{thm:pythag}.
%      \item The sequence $\setn{\fphi_n}$ is \prope{linearly independent}\ifsxref{frames}{thm:orthog==>linin}.
%      %\item \thmb{Bessel's equality} holds such that 
%      %      $\ds\norm{\ff-\sum_{i=n}^\xN \inprod{\ff}{\fphi_n} \fphi_n }^2 = \norm{\ff}^2 - \sum_{i=1}^\xN |\inprod{\ff}{\fphi_n}|^2$
%      \item \thmb{Bessel's inequality} holds such that 
%            $\ds\sum_{n=1}^\infty \abs{\inprod{\ff}{\fphi_n}}^2 \le \norm{\ff}^2$\ifsxref{frames}{thm:bessel_ineq}.
%      \item Every vector $\ff$ in $\spLLR$ has a \structe{Fourier expansion}\ifsxrefs{frames}{def:hspace_fex}such that \\
%            $\ds\ff\eqs\sum_{n=1}^\infty \inprod{\ff}{\fphi_n} \fphi_n$\ifsxref{frames}{prop:hspace_fex}.
%      \item \thmb{Parseval's identity} holds \textbf{if and only if} $\setn{\fphi_n}$ is an orthonormal basis:\\
%            $\ds\norm{\ff}^2 \eqs\sum_{n=1}^\infty \abs{\inprod{\ff}{\fphi_n}}^2  \quad\scy\forall\ff\in\setX$%
%            \ifsxref{frames}{thm:parsevalid}.
%      \item The \structe{Fourier expansion} of a vector $\ff$ in a Hilbert space $\spLLR$
%            on an orthonormal basis $\setn{\fphi_n}$ 
%            that spans a subspace $\spY\subseteq\spLLR$ is the best approximation of $\ff$ in $\spY$ with respect to
%            the metric induced by the inner product (\thme{Best Approximation Theorem}\ifsxref{frames}{thm:bat}).
%    \end{enumerate}
%
%
%  \item \structb{Riesz basis}: \pref{def:mra} does not explicitly require an orthonormal basis.
%         Instead, it only specifies the weaker (more general) constraint of a Riesz basis.
%         This constraint implies simply that there is a linear mapping between the Riesz basis and an orthonormal basis.
%         In particular, an orthonormal basis can be constructed from the Riesz basis.
%\end{enumerate}

%%--------------------------------------
%\begin{proposition}
%\label{thm:HV_orthobasis}
%%--------------------------------------
%Let $\MRAspaceLLRV$ be an MRA space.
%\propbox{\begin{array}{MMC}
%  $\spLLR$   & has an \structe{orthonormal basis}\\
%  $\spV_j$ & has an \structe{orthonormal basis} & \forall j\in\Z
%\end{array}}
%\end{proposition}
%\begin{proofns}
%\begin{enume}
%  %\item By \prefp{def:mra}, $\spLLR$ is \prope{separable}.
%  \item $\spLLR$ is \prope{separable}.
%  \item Therefore $\spLLR$ has an orthonormal basis\ifsxref{frames}{thm:ortho<==>separable}.
%  \item Therefore each $\spV_j$ is \prope{separable}\ifsxref{subspace}{prop:Vn_separable}.
%  \item So each $\spV_j$ has an orthonormal basis\ifsxref{frames}{thm:ortho<==>separable}.
%\end{enume}
%\end{proofns}

%=======================================
\section{Dilation equation}
%=======================================
Several functions in mathematics exhibit a kind of \prope{self-similar} or \prope{recursive} property:
\begin{listi}
  \item If a function $\ff(x)$ is \prope{linear}, then \xref{ex:TD_flinear}
        \\\indentx$\ds\ff(x) = \ff(1)x - \ff(0)\opTrn x$.   %{$\setn{x,\,\opTrn x}$ is a \structe{basis} for $\clLcc$}$.
  \item If a function $\ff(x)$ is sufficiently \prope{bandlimited}, then the \structe{Cardinal series} \xref{ex:TD_cardinalseries} demonstrates
        \\\indentx$\ds\ff(x) = \sum_{n=1}^\infty \ff(n) \opTrn^n\frac{\sin\brs{\pi(x)}}{\pi(x)}$.
  \item \fncte{B-splines}\ifsxref{spline}{thm:bspline_recursion} are another example:
        \\\indentx$\ds\fN_n(x)   = \frac{1}{n}x\fN_{n-1}(x) - \frac{1}{n}x\opTrn\fN_{n-1}(x) + \frac{n+1}{n}\opTrn\fN_{n-1}(x)  \qquad\scy\forall n\in\Znn\setd\setn{1},\,  \forall x\in\R$.
\end{listi}

The scaling function $\fphi(x)$ \xref{def:mra} also exhibits a kind of \prope{self-similar} property.
By \prefp{def:mra}, the dilation $\opDil\ff$ of each vector $\ff$ in $\spV_0$ is in $\spV_1$.
If $\setxZ{\opTrn^n\fphi}$ is a basis for $\spV_0$,
then $\setxZ{\opDil\opTrn^n\fphi}$ is a basis for $\spV_1$,
$\setxZ{\opDil^2\opTrn^n\fphi}$ is a basis for $\spV_2$, \ldots;
and in general $\set{\opDil^j\opTrn^m\fphi}{j\in\Z}$ is a basis for $\spV_j$.
Also, if $\fphi$ is in $\spV_0$, then it is also in $\spV_1$ (because $\spV_0\subset\spV_1$).
And because $\fphi$ is in $\spV_1$ and because $\setxZ{\opDil\opTrn^n\fphi}$ is a basis for $\spV_1$,
$\fphi$ is a linear combination of the elements in $\setxZ{\opDil\opTrn^n\fphi}$.
That is, $\fphi$ can be represented as a linear combination of translated and dilated versions of itself.
The resulting equation is called the \hie{dilation equation} (\pref{def:dilation_eq}, next).\footnote{%
The property of \prope{translation invariance} is of particular significance in the theory of 
\structe{normed linear spaces} (a Hilbert space is a complete normed linear space equipped with an inner product)%
\ifdochas{vsnorm}{---see \prefp{lem:vsn_ti} and \prefp{thm:vsn_d2norm}}.
}

%--------------------------------------
\begin{definition}
\citetbl{
  \citerp{jawerth}{7}
  }
\label{def:dilation_eq}
%--------------------------------------
Let $\MRAspaceLLRV$ be a \structe{multiresolution analysis space} with scaling function $\fphi$ \xref{def:mra}.
Let $\seqxZ{\fh_n}$ be a \structe{sequence} \xref{def:seq} in $\spllR$ \xref{def:spllR}.
\defboxt{
  The \structe{equation} 
    \quad$\ds\brb{\fphi(x)=\sum_{n\in\Z} \fh_n \opDil \opTrn^n \fphi(x)\qquad\scy\forall x\in\R}$\quad
  is called the \equd{dilation equation}.\\
  It is also called the \equd{refinement equation},
  \equd{two-scale difference equation},\\and \equd{two-scale relation}.
  }
\end{definition}

%--------------------------------------
\begin{remark}
%--------------------------------------
\remboxt{
  The \eque{dilation equation} under the defintions of $\opT$ and $\opD$ evaluates to 
  \\\indentx$\ds\fphi(x)=\sum_{n\in\Z} \fh_n \fphi(2x-n)$.
  }
\end{remark}
\begin{proof}
  \begin{align*}
    \fphi(x)
      &= \sum_{n\in\Z} \fh_n \opDil \opTrn^n \fphi\brp{x}
    \\&= \sum_{n\in\Z} \fh_n \opDil \fphi\brp{x-n}
      && \text{by definition of $\opT$} && \text{\xref{def:opT}}
    \\&= \sum_{n\in\Z} \fh_n \fphi\brp{2x-n}
      && \text{by definition of $\opD$} && \text{\xref{def:opD}}
  \end{align*}
\end{proof}

%--------------------------------------
\begin{theorem}[\thmd{dilation equation}]
\label{thm:h->phi}
\label{thm:dilation_eq}
%--------------------------------------
%Let $\MRAspaceLLRV$ be a \structe{multiresolution analysis space} with scaling function $\fphi$ \xref{def:mra}.
%Let $\mrasys$ be an \structe{MRA system} \xref{def:mrasys}.
Let an \structe{MRA space} and \fncte{scaling function} be as defined in \prefp{def:mra}.
%Let $\Fphi(\omega)$ be the \fncte{Fourier transform}\ifsxrefs{harFour}{def:ft}of $\fphi(x)$.
%Let $\Dh(\omega)$ be the \fncte{Discrete time Fourier transform}\ifsxref{dsp}{def:dtft} of $\seqn{\fh_n}$.
%\\Let $\ds\prod_{n=1}^\infty x_n \eqd \lim_{\xN\to\infty}\prod_{n=1}^\xN x_n$.
\thmbox{
  \brb{\begin{array}{M}
    $\MRAspaceLLRV$ is an \structe{MRA space}\\ 
    with \structe{scaling function} $\fphi$
  \end{array}}
  \quad\implies\quad
  \mcom{\brb{\begin{array}{>{\ds}l}
    \scy\exists \seqxZ{\fh_n} \st\\
    \fphi(x)=\sum_{n\in\Z} \fh_n \opDil \opTrn^n \fphi(x)\qquad
    \scy\forall x\in\R
  \end{array}}}{\prope{dilation equation in ``time"}}
  }
\end{theorem}
\begin{proof}
  \begin{align*}
    \fphi &\in \spV_0
          && \text{by definition of \struct{MRA}} && \text{\xref{def:mra}}
        \\&\subseteq \spV_1
          && \text{by definition of \struct{MRA}} && \text{\xref{def:mra}}
        \\&\eqd \linspan\setxZ{\opDil\opTrn^n\fphi(x)}
          && \text{by definition of $\spV_j$}     && \text{\xref{def:mra}}
        \\&\implies 
           \exists \seqxZ{\fh_n} \st
              \fphi(x) = \sum_{n\in\Z}\fh_n \opDil \opTrn^n \fphi(x)
          && \text{by definition of $\linspan$}   && \text{\xref{def:linspan}}
  \end{align*}
\end{proof}

%--------------------------------------
\begin{lemma}
\citetbl{
  \citerpg{mallat}{228}{012466606X}
  }
\label{lem:Fphi}
%--------------------------------------
Let $\fphi(x)$ be a function in $\spLLR$ \xref{def:spLLR}.
Let $\Fphi(\omega)$ be the \fncte{Fourier transform}\ifsxrefs{harFour}{def:ft}of $\fphi(x)$.
Let $\Dh(\omega)$ be the \fncte{Discrete time Fourier transform}\ifsxref{dsp}{def:dtft} of a sequence $\seqnZ{\fh_n}$.
\lembox{
  \begin{array}{>{\ds}lc rc>{\ds}lCD}
  {\scy(A)}\quad\fphi(x)=\sum_{n\in\Z} \fh_n \opDil \opTrn^n \fphi(x)\quad{\scy \forall x\in\R} %{\prope{dilation equation in ``time" (A)}}
    &\iff&
    \Fphi\brp{\omega} &=& \cwt \: \Dh\brp{\frac{\omega}{2}}\: \Fphi\brp{\frac{\omega}{2}}
                        & \forall \omega\in\R
                        & (1)
  \\&\iff&
    \Fphi\brp{\omega} &=& \Fphi\brp{\frac{\omega}{2^\xN}} \prod_{n=1}^\xN \cwt\:\Dh\brp{\frac{\omega}{2^n}}
                        & \forall n\in\Zp,\,\omega\in\R
                        & (2)
  \end{array} 
  }
\end{lemma}
\begin{proof}
\begin{enumerate}
  \item Proof that (A)$\implies$(1): \label{item:Fphi_A1}
    \begin{align*}
      \Fphi\brp{\omega}
        &\eqd \opFT\fphi
      \\&= \opFT\sum_{n\in\Z}\fh_n \opDil \opTrn^n \fphi(x)
        && \text{by (A)}
      \\&= \sum_{n\in\Z}\fh_n \opFT\opDil\opTrn^n \fphi(x)
      \\&= \sum_{n\in\Z}\fh_n \mcom{\cwt e^{-i\frac{\omega}{2} n}\fphi\brp{\frac{\omega}{2}}}{$\opFT\opDil\opTrn^n \fphi(x)$}
        && \text{by \prefp{prop:FTDf}}
      \\&= \cwt \mcom{\brs{\sum_{n\in\Z}\fh_n e^{-i\frac{\omega}{2} n}}}{$\Dh(\omega/2)$} \Fphi\brp{\frac{\omega}{2}}
      \\&= \cwt\:\Dh\brp{\frac{\omega}{2}}\: \Fphi\brp{\frac{\omega}{2}}
        && \text{by definition of \ope{DTFT} \xref{def:dtft}}
    \end{align*}

  \item Proof that (A)$\impliedby$(1):
    \begin{align*}
      \fphi(x) 
        &= \opFTi\Fphi(\omega)
        && \text{by definition of $\Fphi(\omega)$}
      \\&= \opFTi\cwt\:\Dh\brp{\frac{\omega}{2}}\: \Fphi\brp{\frac{\omega}{2}}
        && \text{by (1)}
      \\&= \opFTi\cwt\:\sum_{n\in\Z}h_n e^{-i\frac{\omega}{2} n}\: \Fphi\brp{\frac{\omega}{2}}
        && \text{by definition of \ope{DTFT}}
        &&\text{\xref{def:dtft}}
      \\&= \cwt\:\sum_{n\in\Z}h_n \opFTi e^{-i\frac{\omega}{2} n}\: \Fphi\brp{\frac{\omega}{2}}
        && \text{by property of linear operators}
      \\&= \cwt\:\sum_{n\in\Z}h_n \opFTi \opFT\opDil\opTrn^n\fphi
        && \text{by \prefp{prop:FTDf}}
      \\&= \sum_{n\in\Z}\fh_n \opDil\opTrn^n \fphi(x)
        && \text{by definition of \ope{operator inverse}}
        && \text{\ifsxref{relation}{def:rel_inverse}}
    \end{align*}

  \item Proof that (1)$\implies$(2):
    \begin{enumerate}
      \item Proof for $\xN=1$ case:
        \begin{align*}
          \brlr{\Fphi\brp{\frac{\omega}{2^\xN}}\:\prod_{n=1}^\xN \cwt \Dh\brp{\frac{\omega}{2^n}}}_{\xN=1}
            &= \cwt\:\Dh\brp{\frac{\omega}{2}}\Fphi\brp{\frac{\omega}{2}}
          \\&= \Fphi(\omega)
            && \text{by (1)}
        \end{align*}

      \item Proof that [$\xN$ case]$\implies$[$\xN+1$ case]:
        \begin{align*}
          \Fphi\brp{\frac{\omega}{2^{\xN+1}}}\:\prod_{n=1}^{\xN+1} \cwt \Dh\brp{\frac{\omega}{2^n}}
            &= \brs{\prod_{n=1}^{\xN} \cwt \Dh\brp{\frac{\omega}{2^n}}}
               \mcom{\cwt \Dh\brp{\frac{\omega}{2^{N+1}}}\Fphi\brp{\frac{\omega}{2^{\xN+1}}}}{$\Fphi(\omega/2^\xN)$}
          \\&= \Fphi(\omega/2^\xN) \prod_{n=1}^{\xN} \cwt \Dh\brp{\frac{\omega}{2^n}}
          \\&= \Fphi(\omega)
            && \text{by [$\xN$ case] hypothesis}
        \end{align*}
    \end{enumerate}

  \item Proof that (1)$\impliedby$(2):
    \begin{align*}
      \Fphi(\omega)
        &= \brlr{\Fphi\brp{\frac{\omega}{2^{\xN}}}\:\prod_{n=1}^{\xN} \cwt \Dh\brp{\frac{\omega}{2^n}}}_{\xN=1}
        && \text{by (2)}
      \\&= \Fphi\brp{\frac{\omega}{2}}\cwt \Dh\brp{\frac{\omega}{2}}
      \\&= \cwt\Dh\brp{\frac{\omega}{2}}\Fphi\brp{\frac{\omega}{2}} 
    \end{align*}

\end{enumerate}
\end{proof}

%--------------------------------------
\begin{lemma}
\label{lem:Fphi_infty}
% 2013 August 09 Friday
% 2013 August 20 Monday: modified \implies relation
%--------------------------------------
Let $\fphi(x)$ be a function in $\spLLR$ \xref{def:spLLR}.
Let $\Fphi(\omega)$ be the \fncte{Fourier transform}\ifsxrefs{harFour}{def:ft}of $\fphi(x)$.
Let $\Dh(\omega)$ be the \fncte{Discrete time Fourier transform}\ifsxref{dsp}{def:dtft} of $\seqn{\fh_n}$.
Let $\ds\prod_{n=1}^\infty x_n \eqd \lim_{\xN\to\infty}\prod_{n=1}^\xN x_n$, with respect to the standard norm in $\spLLR$.

\lembox{
  \begin{array}{>{\ds}l c rc>{\ds}lCD}
    \brb{\begin{array}{M}
      $\Fphi\brp{\omega} = C\prod_{n=1}^\infty \cwt\:\Dh\brp{\frac{\omega}{2^n}}$\\
      $\scy\forall C>0,\,\omega\in\R$\qquad\qquad\scs(A)
    \end{array}}
      &\implies& \fphi(x)          &=& \sum_{n\in\Z} \fh_n \opDil \opTrn^n \fphi(x)
                                     & \forall x\in\R 
                                     & (1) %{\prope{dilation equation in ``time" (A)}}
    \\&\iff&     \Fphi\brp{\omega} &=& \cwt \: \Dh\brp{\frac{\omega}{2}}\: \Fphi\brp{\frac{\omega}{2}}
                                     & \forall \omega\in\R
                                     & (2)
  \\
      &\iff&     \Fphi\brp{\omega} &=& \Fphi\brp{\frac{\omega}{2^\xN}} \prod_{n=1}^\xN \cwt\:\Dh\brp{\frac{\omega}{2^n}}
                                     & \forall n\in\Zp,\,\omega\in\R
                                     & (3)
  \end{array}
  }
\end{lemma}
\begin{proof}
  \begin{enumerate}
    \item Proof that (1)$\iff$(2)$\iff$(3): by \prefp{lem:Fphi}
    \item Proof that (A)$\implies$(2):
      \begin{align*}
        \Fphi(\omega)
          &= C\:\prod_{n=1}^{\infty} \cwt \Dh\brp{\frac{\omega}{2^n}}
          && \text{by left hypothesis}
        \\&= C\:\cwt \Dh\brp{\frac{\omega}{2}} \prod_{n=1}^{\infty} \cwt \Dh\brp{\frac{\omega}{2^{n+1}}}
        \\&= C\:\cwt \Dh\brp{\frac{\omega}{2}} \prod_{n=1}^{\infty} \cwt \Dh\brp{\frac{\omega/2}{2^{n}}}
        \\&= \cwt \Dh\brp{\frac{\omega}{2}}\brs{C\prod_{n=1}^{\infty} \cwt \Dh\brp{\frac{\omega/2}{2^{n}}}}
        \\&= \cwt \Dh\brp{\frac{\omega}{2}} \Fphi\brp{\frac{\omega}{2}}
          && \text{by left hypothesis}
      \end{align*}
  \end{enumerate}
\end{proof}


%--------------------------------------
\begin{proposition}
\label{prop:Fphi}
%--------------------------------------
Let $\fphi(x)$ be a function in $\spLLR$ \xref{def:spLLR}.
Let $\Fphi(\omega)$ be the \fncte{Fourier transform}\ifsxrefs{harFour}{def:ft}of $\fphi(x)$.
Let $\Dh(\omega)$ be the \fncte{Discrete time Fourier transform}\ifsxref{dsp}{def:dtft} of $\seqn{\fh_n}$.
Let $\ds\prod_{n=1}^\infty x_n \eqd \lim_{\xN\to\infty}\prod_{n=1}^\xN x_n$, with respect to the standard norm in $\spLLR$.
\propbox{
  \brb{\begin{array}{M}
    $\Fphi(\omega)$ is\\
    \prope{continuous}\\ 
    at $\,\omega=0$
  \end{array}}
  \quad\implies\quad
  \brb{\begin{array}{c rc>{\ds}lCD}
        &
    \fphi(x)          &=& \sum_{n\in\Z} \fh_n \opDil \opTrn^n \fphi(x)
                        & \forall x\in\R 
                        & (1) %{\prope{dilation equation in ``time" (A)}}
  \\\iff&
    \Fphi\brp{\omega} &=& \cwt \: \Dh\brp{\frac{\omega}{2}}\: \Fphi\brp{\frac{\omega}{2}}
                        & \forall \omega\in\R
                        & (2)
  \\\iff&
    \Fphi\brp{\omega} &=& \Fphi\brp{\frac{\omega}{2^\xN}} \prod_{n=1}^\xN \cwt\:\Dh\brp{\frac{\omega}{2^n}}
                        & \forall n\in\Zp,\,\omega\in\R
                        & (3)
  \\\iff&
    \Fphi\brp{\omega} &=& \Fphi\brp{0} \prod_{n=1}^\infty \cwt\:\Dh\brp{\frac{\omega}{2^n}} 
                        & \omega\in\R
                        & (4)
  \end{array}}
  }
\end{proposition}
%\lembox{
%  \brb{\begin{array}{FMCD}
%    (A) & $\ds\fphi(x)=\sum_{n\in\Z} \fh_n \opDil \opTrn^n \fphi(x)$ & \forall x\in\R & and\\ % & {\prope{dilation equation}}\\
%    (B) & \mc{3}{M}{$\Fphi(\omega)$ is \prope{continuous} at $\omega=0$}
%  \end{array}}
%  \implies
%  \brb{\begin{array}{>{\ds}lCD}
%    \Fphi\brp{\omega} = \Fphi\brp{0} \prod_{n=1}^\infty \cwt\:\Dh\brp{\frac{\omega}{2^n}} & \forall \omega\in\R
%  \end{array}} 
%  }
%\end{lemma}
\begin{proof}
  \begin{enumerate}
    \item Proof that (1)$\iff$(2)$\iff$(3): by \prefp{lem:Fphi}
    \item Proof that (3)$\implies$(4):
      \begin{align*}
        \Fphi\brp{0}\:\prod_{n=1}^{\infty} \cwt \Dh\brp{\frac{\omega}{2^n}}
          &= \lim_{\xN\to\infty}\Fphi\brp{\frac{\omega}{2^{\xN}}}\:\prod_{n=1}^{\xN} \cwt \Dh\brp{\frac{\omega}{2^n}}
          && \text{by \prope{continuity} and definition of $\prod_{n=1}^\infty x_n$}
        \\&= \Fphi(\omega)
          && \text{by (3) and \prefp{lem:Fphi}}
      \end{align*}
    \item Proof that (2)$\impliedby$(4): by \prefp{lem:Fphi_infty}
      %\begin{align*}
      %  \Fphi(\omega)
      %    &= \Fphi\brp{0}\:\prod_{n=1}^{\infty} \cwt \Dh\brp{\frac{\omega}{2^n}}
      %    && \text{by (4)}
      %  \\&= \Fphi\brp{0}\:\cwt \Dh\brp{\frac{\omega}{2}} \prod_{n=1}^{\infty} \cwt \Dh\brp{\frac{\omega}{2^{n+1}}}
      %  \\&= \Fphi\brp{0}\:\cwt \Dh\brp{\frac{\omega}{2}} \prod_{n=1}^{\infty} \cwt \Dh\brp{\frac{\omega/2}{2^{n}}}
      %  \\&= \cwt \Dh\brp{\frac{\omega}{2}}\brs{ \Fphi\brp{0}\prod_{n=1}^{\infty} \cwt \Dh\brp{\frac{\omega/2}{2^{n}}}}
      %  \\&= \cwt \Dh\brp{\frac{\omega}{2}} \Fphi\brp{\frac{\omega}{2}}
      %    && \text{by (4)}
      %\end{align*}
  \end{enumerate}
\end{proof}



%\if 0


\pref{def:hn} (next) formally defines the coefficients that appear in \prefpp{thm:dilation_eq}.
%--------------------------------------
\begin{definition}%[subspace coefficients]
\label{def:hn}
\label{def:mrasys}
%--------------------------------------
Let $\MRAspaceLLRV$ be a multiresolution analysis space with scaling function $\fphi$.
Let $\seqxZ{\fh_n}$ be a sequence of coefficients such that $\fphi = \sum_{n\in\Z} \fh_n \opDil\opTrn^n \fphi$.
\defboxp{
  A \structd{multiresolution system} is the tuple $\mrasys$.
  The sequence $\seqxZ{\fh_n}$ is the \fnctd{scaling coefficient sequence}.
  A multiresolution system is also called an \structd{MRA system}.
  An \structe{MRA system} is an \structd{orthonormal MRA system} if $\setnZ{\opTrn^n\fphi}$ is \prope{orthonormal}.
  }
\end{definition}

%Examples of \hi{multiresolution analyses} are provided in
%\pref{ex:wavstrct_haar_sin} (next)
%-- \prefp{ex:sw_gh_bspline}.

%%--------------------------------------
%\begin{definition}
%\label{def:wavstrct_normcoef}
%%--------------------------------------
%Let $\mrasys$ be a {multiresolution system}, and $\opDil$ the dilation operator.
%\defboxp{
%  The \hid{normalization coefficient at resolution $n$} is the quantity 
%  \\\indentx$\norm{\opDil^j\fphi}$.
%  }
%\end{definition}


%--------------------------------------
\begin{theorem}
\label{thm:V0Vn}
%--------------------------------------
Let $\mrasys$ be an \structe{MRA system} \xref{def:mrasys}.\\
Let $\linspan\setA$ be the \structe{linear span}\ifsxref{frames}{def:span} of a set $\setA$.
\thmbox{
  \mcom{\linspan\setxZ{\opTrn^n\fphi}=\spV_0}
       {$\setxZ{\opTrn^n\fphi}$ is a \structe{basis} for $\spV_0$}
  \qquad\implies\qquad
  \mcom{\linspan\setxZ{\opDil^j\opTrn^n\fphi}=\spV_j\quad{\scy\forall j\in\Znn}}
       {$\setxZ{\opDil^j\opTrn^n\fphi}$ is a \structe{basis} for $\spV_j$}
  }
\end{theorem}
\begin{proof} Proof is by induction:\citetbl{\citerpg{smith2011}{4}{1420063723}}
\begin{enumerate}
\item induction basis (proof for $j=0$ case):
      %$\setxZ{\opDil^0\opTrn^n\fphi}$ is a basis for $\spV_0$
      %\\\indentx$\ds \spV_0 = \set{\ff(x)}{\ff(x) = \sum_{n\in\Z} \fc_{n} \fphi(x-n)}.$
  \begin{align*}
    \brlr{\linspan\setxZ{\opDil^j\opTrn^n\fphi}}_{j=0}
      &= \linspan\setxZ{\opTrn^n\fphi}
    \\&= \spV_0
      && \text{by left hypothesis}
  \end{align*}

\item induction step (proof that $j$ case $\implies$ $j+1$ case):
      %$\setxZ{\opDil^j\opTrn^n\fphi}$ is a basis for $\spV_j$
      %$\implies$ $\setxZ{\opDil^{j+1}\opTrn^n\fphi}$ is a basis for $\spV_{j+1}$:
  \begin{align*}
    &\linspan\setxZ{\opDil^{j+1}\opTrn^n\fphi}
    \\&= \set{\ff\in\spLLR}{\exists \seqn{\alpha_n} \st \ff(x)=\sum_{n\in\Z}\alpha_n \opDil^{j+1}\opTrn^n\fphi}
      && \text{by definition of $\linspan$} &&\text{\ifxref{frames}{def:span}}
    \\&= \set{\ff\in\spLLR}{\exists \seqn{\alpha_n} \st \ff(x)=\opDil\sum_{n\in\Z}\alpha_n \opDil^{j}\opTrn^n\fphi}
    \\&= \set{\ff\in\spLLR}{\exists \seqn{\alpha_n} \st \opDili\ff(x)=\sum_{n\in\Z}\alpha_n \opDil^{j}\opTrn^n\fphi}
    \\&= \mathrlap{
         \set{\brs{\opDil\ff}\in\spLLR}{\exists \seqn{\alpha_n} \st \opDili\brs{\opDil\ff(x)}=\sum_{n\in\Z}\alpha_n \opDil^{j}\opTrn^n\fphi}
         }
    \\&= \opDil\set{\ff\in\spLLR}{\exists \seqn{\alpha_n} \st \ff(x)=\sum_{n\in\Z}\alpha_n \opDil^{j}\opTrn^n\fphi}
    \\&= \opDil\linspan\setxZ{\opDil^j\opTrn^n\fphi}
      && \text{by definition of $\linspan$} &&\text{\ifxref{frames}{def:span}}
    \\&= \opDil\spV_j
      && \text{by induction hypothesis}
    \\&= \spV_{j+1}
      && \text{by \prope{self-similar} property} &&\text{\xref{def:mra}}
  \end{align*}
\end{enumerate}
\end{proof}

%--------------------------------------
\begin{example}
\label{ex:wavstrct_haar_sin}
\exmx{Haar scaling function}
%--------------------------------------
\exbox{\begin{array}{rclm{54mm}} 
  \mc{4}{M}{In the \hie{Haar} MRA, the scaling function $\fphi(x)$ is the \hie{pulse function}}
  \\
  \fphi(x) &=& \brbl{\begin{array}{lM}
                       1  & for $x\in\intco{0}{1}$ \\
                       0  & otherwise.
                     \end{array}}
  &
  \includegraphics{graphics/pulse.pdf}
  \\
  \mc{4}{M}{In the subspace $\spV_j$ ($j\in\Z$) the scaling functions are}
  \\
  \opDil^j\fphi(x) &=& \brbl{\begin{array}{lM}
                               \brp{2}^{j/2}   & for $x\in\intco{0}{\brp{2^{-j}}}$ \\
                               0                  & otherwise.
                              \end{array}}
  &
  \includegraphics{graphics/pulse2.pdf}%
\end{array}}

The scaling subspace $\spV_0$ is the span $\spV_0\eqd\linspan\setxZ{\opTrn^n\fphi}$.
The scaling subspace $\spV_j$ is the span $\spV_j\eqd\linspan\set{\opDil^j\opTrn^n\fphi}{n\in\Z}$.
  %$\opDil^j\fphi$ such that
Note that $\norm{\opDil^j\opTrn^n\fphi}$ for each resolution $j$ and shift $n$ is unity:
  \begin{align*}
    \norm{\opDil^j\opTrn^n\fphi}^2
      &= \norm{\fphi}^2  
      && \text{by \prope{unitary} properties of $\opTrn$ and $\opDil$} && \text{\xref{thm:TD_unitary}}
   %\\&= \int_{\intco{0}{\brp{2^{-j}}} \abs{\brp{\sqrt{2}}^j}^2 \dx
    \\&= \int_0^1 \abs{1}^2 \dx
      && \text{by definition of $\normn$ on $\spLLR$} && \text{\xref{def:spLLR}}
    %\\&= \brp{2^{-j}\brp{2^j}
    \\&= 1
  \end{align*}

\begin{minipage}{\tw-64mm}
Let $\ff(x)=\sin(\pi x)$.
Suppose we want to project $\ff(x)$ onto the subspaces $\spV_0$, $\spV_1$, $\spV_2$, \ldots.
\end{minipage}%
\hfill\tbox{\includegraphics{graphics/sin_pi_t.pdf}}%
\\
\begin{minipage}{\tw-68mm}
\ragr
The values of the transform coefficients for the subspace $\spV_j$ are %illustrated in \prefp{fig:wavstrct_Haar_sin}
given by
\end{minipage}%
\hfill%
\tbox{%
%\begin{figure}[t]
  %\mbox{}\\%
  %\psset{unit=8mm}%
  %%============================================================================
% Daniel J. Greenhoe
% LaTeX file
% sin(t)
%============================================================================
%  \psset{unit=1mm}
\begin{pspicture}(-40,-15)(40,15)%
  \footnotesize
  \psset{linecolor=blue}%
  %\rput(0,0){% axis
  %  \psset{linecolor=axis}
  %  \multirput(-30,0)(10,0){7}{\psline(0,-1)(0,1)}% markers on x axis
  %  \psline{<->}(-35,0)(35,0)% x axis
  %  \psline{<->}(0,-15)(0,15)%    y axis
  %  \psline(-1,10)(1,10)%
  %  \psline(-1,-10)(1,-10)%
  %  \uput[180](0,10){$\frac{1}{\pi}$}% y=1
  %  \uput[0](0,-10){$\frac{-1}{\pi}$}% y=1
  %  \multido{\ival=-3+1,\ipos=-30+10}{7}{%
  %    \uput[-90](\ipos,0){$\ival$}% x=
  %    }%
  %  \uput[0](40,0){$t$}%
  %  }%
  \psaxes[linecolor=axis,unit=10,labels=x]{<->}(0,0)(-3.5,-1.5)(3.5,1.5)%
  \multirput(-20,0)(20,0){3}{\psline{-o}(0,0)(0,10)}%
  \multirput(-30,0)(20,0){4}{\psline{-o}(0,0)(0,-10)}%
  \uput[180](0,10){$\frac{2}{\pi}$}% y=1
  \uput[0](0,-10){$\frac{-2}{\pi}$}% y=1
  \uput[0](35,0){$t$}%
  \rput[b](17.5,10){$\inprod{\ff(t-n)}{\sin(\pi t)}$}%
\end{pspicture}
%
  \includegraphics{graphics/haar0_sin_t.pdf}%  8mm
  }
\\
  \begin{align*}
    \brs{\opR_j\ff(x)}(n) 
      &=    \frac{1}{\norm{\opDil^j\opTrn^n\fphi}^2}\inprod{\ff(x)}{\opDil^j\opTrn^n\fphi} 
    %\\&=    \frac{1}{\cancelto{1}{\norm{\fphi}^2}}
    %        \inprod{\ff(x)}{\opDil^j\fphi\brp{x-n}}
    %  &&    \text{by definition of $\opTrn$ \xref{def:wav_opT}} 
    \\&=    \frac{1}{\cancelto{1}{\norm{\fphi}^2}}\inprod{\ff(x)}{2^{j/2}\fphi\brp{2^j x-n}} 
      &&    \text{by \prefp{prop:DjTn}}
    \\&=    2^{j/2} \inprod{\ff(x)}{\fphi\brp{2^j x-n}} 
    \\&=    2^{j/2} 
            \int_{2^{-j}n}^{2^{-j}(n+1)} \ff(x) \dx
    \\&=    2^{j/2} 
            \int_{2^{-j}n}^{2^{-j}(n+1)} \sin(\pi x) \dx
    \\&=    2^{j/2}
            \left. \brp{-\frac{1}{\pi}}\cos\brp{\pi x} \right|_{2^{-j} n}^{2^{-j}(n+1)}
    \\&=    \frac{2^{j/2}}{\pi}
            \brs{
              \cos\brp{{2^{-j}n\pi}} -
              \cos\brp{{2^{-j}(n+1)\pi}}  
              }
  \end{align*}



And the projection $\opA_n\ff(x)$ of the function $\ff(x)$ onto the subspace $\spV_j$ is
%(alternatively, the \hie{projection} of $\ff(x)$ \emph{onto} the space $\spV_j$ is) %\\
%\begin{minipage}{\tw-95mm}
  \begin{align*}
    \opA_j\ff(x) 
      &= \sum_{n\in\Z} \inprod{\ff(x)}{\opDil^j\opTrn^n\fphi} \opDil^j\opTrn^n\fphi 
    \\&= \frac{2^{j/2}}{\pi}
         \sum_{n\in\Z}
         \brs{
           \cos\brp{2^{-j}n\pi} -
           \cos\brp{2^{-j}(n+1)\pi}  
           } 2^{j/2}\fphi\brp{2^j x-n}
    \\&= \frac{2^j}{\pi}
         \sum_{n\in\Z}
         \brs{
           \cos\brp{2^{-j} n\pi} -
           \cos\brp{2^{-j}(n+1)\pi}  
           } \fphi\brp{2^j x-n}
  \end{align*}
%\end{minipage}%
%\hfill%
%\begin{minipage}{90mm}%
%  \mbox{}\\%
%  \psset{unit=8mm}%
%  %%============================================================================
% Daniel J. Greenhoe
% LaTeX file
% sin(t)
%============================================================================
%  \psset{unit=1mm}
  \begin{pspicture}(-40,-15)(40,15)%
    \footnotesize
    \psset{linecolor=blue}%
    \psaxes[linecolor=axis,unit=10,labels=x]{<->}(0,0)(-3.5,-1.5)(3.5,1.5)%
    \uput[180](0,10){$\frac{2}{\pi}$}% y=1
    \uput[0](0,-10){$\frac{-2}{\pi}$}% y=1
    \rput[r](-32,5){$\cdots$}% ... (left)
    \rput[l]( 32,5){$\cdots$}% ... (right)
    \multiput(-30,-10)(10,0){7}{\psline[linestyle=dotted,dotsep=0.5](0,0)(0,20)}% vertical dotted lines
    \multiput(-20, 10)(20,0){3}{\psline{*-o}(0,0)(10,0)}% upper horizontal lines
    \multiput(-30,-10)(20,0){3}{\psline{*-o}(0,0)(10,0)}% lower horizontal lines
    \rput(5,5){$\opA_0\ff(t)$}%
    \uput[0](35,0){$t$}%
  \end{pspicture}

%  %============================================================================
% Daniel J. Greenhoe
% LaTeX file
%
% approximation of a sin(pi t) in the Haar k=0 subspace 
%
%     2
% --------- = 0.63661977236758134307553505349006
%    pi
% nominal unit = 8mm
%============================================================================
\begin{pspicture}(-3.5,-1.5)(4,1.5)%
  \psaxes[linecolor=axis,labels=x]{<->}(0,0)(-3.5,-1.5)(3.5,1.5)%
  \multirput(-2,0)(1,0){6}{\psline[linestyle=dotted](0, 0.6366)(0,-0.6366)}% dotted vertical segments
  \multirput(-2,0)(2,0){3}{\psline{*-o}(0, 0.6366)(1, 0.6366)}%
  \multirput(-3,0)(2,0){3}{\psline{*-o}(0,-0.6366)(1,-0.6366)}%
  \uput[180](0,0.6366){$\frac{2}{\pi}$}% y=1
  \uput[0](0,-0.6366){$\frac{-2}{\pi}$}% y=1
  \uput[0](3.5,0){$x$}%
  \psplot[plotpoints=100,linestyle=dashed,linecolor=red,linewidth=1pt]{-3}{3}{x 180 mul sin}%
  %\rput[b](17.5,10){$\inprod{\ff(t-n)}{\sin(\pi t)}$}%
\end{pspicture}

%\end{minipage}

The transforms of $\sin(\pi x)$ into the subspaces $\spV_0$, $\spV_1$, and $\spV_2$,
as well as the approximations in those subspaces are as illustrated in \prefpp{fig:wavstrct_haar_sin}.
\end{example}
\begin{figure}
  \centering%
  \begin{tabular}{|l|l|l|}
    \hline
    \mc{1}{|c|}{subspace}&\mc{1}{c|}{transform}&\mc{1}{c|}{approximation}
    \\\hline\hline
    $\spV_0$
    & \includegraphics{graphics/haar0_sin_t.pdf}
    & \includegraphics{graphics/haar0_sin_a.pdf}
    \\\hline
    $\spV_1$
    & \includegraphics{graphics/haar1_sin_t.pdf}
    & \includegraphics{graphics/haar1_sin_a.pdf}
    \\\hline
    $\spV_2$
    & \includegraphics{graphics/haar2_sin_t.pdf}
    & \includegraphics{graphics/haar2_sin_a.pdf}
    \\\hline
  \end{tabular}
  \caption{
    Projections of $\sin(\pi x)$ on Haar subspaces
    \xref{ex:wavstrct_haar_sin}
    \label{fig:wavstrct_haar_sin}
    }
\end{figure}




%=======================================
\section{Necessary Conditions}
%=======================================
%Next we look at  two necessary conditions in the ``time domain" for scaling coefficient design.
%%They can be used in generating simultaneous equations for wavelet system design.
%\\\indentx
%  \begin{tabular}{@{\qquad}clll}
%    \imark & \hie{admissibility condition}: & \pref{thm:admiss}        & \xref{thm:admiss} \\
%    \imark & \hie{quadrature condition}:    & \pref{thm:wav_quadcon}   & \xref{thm:wav_quadcon}
%  \end{tabular}

%--------------------------------------
\begin{theorem}[\thmd{admissibility condition}]
\label{thm:admiss}
%--------------------------------------
%Let $\mrasys$ be a multiresolution system.
Let $\Zh(z)$ be the \fncte{Z-transform} \xref{def:opZ} and 
$\Dh(\omega)$ the \fncte{discrete-time Fourier transform} \xref{def:dtft} of a sequence $\seqxZ{\fh_n}$.
\thmbox{\begin{array}{M}
  %\brb{\begin{array}{M}$\mrasys$\\is an \structe{MRA system}\end{array}} &\impnotimpby&
  $\brb{\text{$\mrasys$ is an \structe{MRA system} \xref{def:mrasys}}}$
  \\$\ds\quad
  \impnotimpby \mcom{\brb{\sum_{n\in\Z} \fh_n  = \sqrt{2}}}{(1) \prope{admissibility} in ``time"}               
  \iff         \mcom{\brb{\Zh(z)\Big|_{z=1}   = \sqrt{2}}}{(2) \prope{admissibility} in ``z domain"}                  
  \iff         \mcom{\brb{\Dh(\omega)\Big|_{\omega=0} = \sqrt{2}}}{(3) \prope{admissibility} in ``frequency"}
  $
\end{array}}
\end{theorem}
\begin{proof}
\begin{enumerate}
  \item Proof that MRA system $\implies$ (1):
    \begin{align*}
      \sum_{n\in\Z}\fh_n
        &= \frac{\int_\R \fphi(x) \dx}{\int_\R \fphi(x) \dx} \sum_{n\in\Z}\fh_n
      \\&= \frac{1}{\int_\R \fphi(x) \dx} \int_\R \sum_{n\in\Z}\fh_n \fphi(x) \dx
      \\&= \frac{1}{\int_\R \fphi(x) \dx} \int_\R \sum_{n\in\Z}\fh_n \frac{\sqrt{2}}{\sqrt{2}}\fphi(2y-n) 2\dy
        && \text{let $y\eqd \frac{x+n}{2}\implies x=2y-n \implies \dx=2\dy$}
      \\&= \frac{2}{\sqrt{2}}\frac{1}{\int_\R \fphi(x) \dx} \int_\R \sum_{n\in\Z}\fh_n \opDil\opTrn^n\fphi(y)\dy
        && \text{by definitions of $\opTrn$ and $\opDil$ \xref{def:opT}}
      \\&= \sqrt{2} \frac{1}{\int_\R \fphi(x) \dx} \int_\R \fphi(y) \dy
        && \text{by \thme{dilation equation} \xref{thm:dilation_eq}}
      \\&= \sqrt{2}
    \end{align*}

  \item Alternate proof that MRA system $\implies$ (1):\\
    %Select a vector $\ff$ such that $\inprod{\fphi}{\ff}\ne 0$.
    Let $\ff(x)\eqd 1\quad\forall x\in\R$. % be a constant vector (e.g. $\ff(x)=1$). Then \ldots
    \begin{align*}
      \inprod{\fphi}{\ff}
        &= \inprod{\sum_{n\in\Z}\fh_n \opDil\opTrn^n \fphi}{\ff}
        && \text{by \thme{dilation equation}} && \text{\xref{thm:dilation_eq}}
      \\&= \sum_{n\in\Z}\fh_n \inprod{\opDil\opTrn^n \fphi}{\ff}
        && \text{by \prop{linearity} of $\inprodn$} && \text{\ifxref{vsinprod}{def:inprod}}
      \\&= \sum_{n\in\Z}\fh_n \inprod{ \fphi}{(\opDil\opTrn^n)^\ast\ff}
        && \text{by definition of operator adjoint} && \text{\ifxref{operator}{thm:op_star}}
      \\&= \sum_{n\in\Z}\fh_n \inprod{ \fphi}{(\opTrna)^n \opDila \ff}
        && \text{by property of operator adjoint} && \text{\ifxref{operator}{thm:op_star}}
      \\&= \sum_{n\in\Z}\fh_n \inprod{ \fphi}{(\opTrni)^n \opDili \ff}
        && \text{by unitary property of $\opTrn$ and $\opDil$} && \text{\xref{prop:TD_unitary}}
      \\&= \sum_{n\in\Z}\fh_n \inprod{ \fphi}{(\opTrni)^n \cwt  \ff}
        && \text{because $\ff$ is a constant hypothesis} &&\text{and by \prefp{prop:opDi}}
      \\&= \sum_{n\in\Z}\fh_n \inprod{ \fphi}{ \cwt \ff}
        && \text{by $\ff(x)=1$ definition}
      \\&= \sum_{n\in\Z}\fh_n \cwt \inprod{ \fphi}{ \ff}
        && \text{by property of $\inprodn$} && \text{\ifxref{vsinprod}{def:inprod}}
      \\&= \cwt \; \inprod{\fphi}{\ff}\; \sum_{n\in\Z}\fh_n
      \\&\implies \sum_{n\in\Z}\fh_n = \sqrt{2}
    \end{align*}

  \item Proof that (1) $\iff$ (2) $\iff$ (3): by \prefp{prop:tzf}.

  \item Proof for $\notimpliedby$ part: by \prefp{cnt:admiss}.
\end{enumerate}
\end{proof}

%--------------------------------------
\begin{counterex}
\label{cnt:admiss}
%--------------------------------------
Let $\mrasys$ be an \structe{MRA system} \xref{def:mrasys}.
\cntbox{\begin{array}{M}
  $\brb{\begin{array}{lm{28mm}}
     \seqn{\fh_n} \eqd \sqrt{2}\kdelta_{n-1} \eqd
       \brbl{\begin{array}{lD}
         \sqrt{2}     & for $n=1$  \\
         0            & otherwise.
       \end{array}}
    &\includegraphics{graphics/hn1.pdf}%
  \end{array}}
  \quad\implies\quad
  \brb{\fphi(x)=0}$
  \\
  which means
  \\
  $\ds\brb{\sum_{n\in\Z}\fh_n = \sqrt{2}} \quad\notimplies\quad \brb{\text{$\mrasys$ is an MRA system for $\spLLR$.}}$
\end{array}}
\end{counterex}
\begin{proof}
\begin{align*}
  \fphi(x)
    &= \sum_{n\in\Z}\fh_n \opDil\opTrn^n\fphi(x)
    && \text{by \thme{dilation equation}} &&\text{\xref{thm:dilation_eq}}
  \\&= \sum_{n\in\Z}\fh_n \fphi(2x-n)
    && \text{by definitions of $\opDil$ and $\opTrn$} &&\text{\xref{def:opT}}
  \\&= \sum_{n\in\Z} \mcom{\sqrt{2}\kdelta_{n-1}}{$\seqn{\fh_n}$} \fphi(2x-n)
    && \text{by definitions of $\seqn{\fh_n}$}
  \\&= \sqrt{2}\fphi(2x-1)
    && \text{by definition of $\fphi(x)$}
  \\\implies
  \fphi(x) &= 0
\end{align*}
This implies $\fphi(x)=0$, which implies that $\mrasys$ is \emph{not} an \structe{MRA system} for $\spLLR$ because
  \\\indentx$\ds \clsp{\Setu_{j\in\Z} \spV_j} = \clsp{\Setu_{j\in\Z} \linspan\set{\opDil^j\opTrn^n\fphi}{\scy n\in\Z}} \neq \spLLR$\\
(the \structe{least upper bound} is \emph{not} $\spLLR$).
\end{proof}




%--------------------------------------
\begin{theorem}[\thmd{Quadrature condition} in ``time"]
\label{thm:wav_quadcon}
\label{thm:wav_hh}
%--------------------------------------
Let $\mrasys$ be an \structe{MRA system} \xref{def:mrasys}.
\thmbox{
  \sum_{m\in\Z}\fh_m \sum_{k\in\Z}\fh_k^\ast \inprod{\fphi}{\opTrn^{2n-m+k} \fphi}
  =\inprod{\fphi}{\opTrn^n \fphi}
  \qquad\scy\forall n\in\Z
  }
\end{theorem}
\begin{proof}
\begin{align*}
  \inprod{\fphi}{\opTrn^n \fphi}
    &= \inprod{\sum_{m\in\Z}\fh_m \opDil \opTrn^m \fphi }{\opTrn^n \sum_{k\in\Z}\fh_k \opDil \opTrn^k \fphi}
    && \text{by \thme{dilation equation}} && \text{\xref{thm:dilation_eq}}
  \\&= \sum_{m\in\Z}\fh_m \sum_{k\in\Z}\fh_k^\ast \inprod{\opDil \opTrn^m \fphi }{\opTrn^n \opDil \opTrn^k \fphi}
    && \text{by properties of $\inprodn$} && \text{\ifxref{vsinprod}{def:inprod}}
  \\&= \sum_{m\in\Z}\fh_m \sum_{k\in\Z}\fh_k^\ast \inprod{\fphi }{\left(\opDil \opTrn^m \right)^\ast \opTrn^n \opDil \opTrn^k \fphi}
    && \text{by definition of operator adjoint} && \text{\ifxref{operator}{prop:op_adjoint}}
  \\&= \sum_{m\in\Z}\fh_m \sum_{k\in\Z}\fh_k^\ast \inprod{\fphi }{\left(\opDil \opTrn^m \right)^\ast \opDil \opTrn^{2n} \opTrn^k \fphi}
    && \text{by \prefp{prop:DTTD}}
  \\&= \sum_{m\in\Z}\fh_m \sum_{k\in\Z}\fh_k^\ast \inprod{\fphi }{\opTrna^m \opDila \opDil \opTrn^{2n} \opTrn^k \fphi}
    && \text{by operator star-algebra properties} && \text{\ifxref{operator}{thm:op_star}}
  \\&= \sum_{m\in\Z}\fh_m \sum_{k\in\Z}\fh_k^\ast \inprod{\fphi }{\opTrn^{-m} \opDil^{-1} \opDil \opTrn^{2n} \opTrn^k \fphi}
    && \text{by \prefp{prop:TD_unitary}}
  \\&= \sum_{m\in\Z}\fh_m \sum_{k\in\Z}\fh_k^\ast \inprod{\fphi }{\opTrn^{2n-m+k} \fphi}
\end{align*}
\end{proof}

%%---------------------------------------
%\begin{theorem}[Neumann Expansion Theorem]
%\index{Neumann Expansion Theorem}
%\thmx{Neumann Expansion Theorem}
%\label{thm:op_net2}
%\citep{michel1993}{415}
%%---------------------------------------
%Let $\opA\in\clFxx$ be an operator on a linear space $\spX$.
%Let $\opA^0\eqd \opI$.
%\thmbox{\begin{array}{ll}
%  \left.\begin{array}{lrclD}
%    1. & \opA          &\in& \oppB(\spX,\spX) & ($\opA$ is bounded) \\
%    2. & \normop{\opA} &<&   1
%  \end{array}\right\}
%  \implies
%  \left\{\begin{array}{lrc>{\ds}l}
%    1. & (\opI-\opA)^{-1} &&\text{ exists} \\
%    2. & \normop{(\opI-\opA)^{-1}} &\le& \frac{1}{1-\normop{\opA}} \\
%    3. & (\opI-\opA)^{-1} &=& \sum_{n=0}^\infty \opA^\xN  \\
%       & \mc{3}{c}{\text{ with uniform convergence}}
%  \end{array}\right.
%\end{array}}
%\end{theorem}
%
%
%
%
%
%
%
%%--------------------------------------
%\begin{theorem}
%\label{thm:wav_net}
%\thmx{$\sum_{n\in\Z} \abs{\fh_n} \ge 1$}
%%--------------------------------------
%Let $\wavsys$ be a \hi{wavelet system}.
%\thmbox{
%  \sum_{n\in\Z} \abs{\fh_n} \ge 1
%  }
%\end{theorem}
%\begin{proof}
%\begin{align*}
%  &&
%  \fphi &= \sum_{n\in\Z}\fh_n \opTrn^n \opDil \fphi
%  \\\implies&&
%  \left(\opI - \sum_{n\in\Z}\fh_n \opTrn^n \opDil \right)\fphi &= \vzero
%  \\\implies&&
%  \left(\opI - \sum_{n\in\Z}\fh_n \opTrn^n \opDil \right)^{-1} & \text{must not exist}
%  \\\implies&&
%  \normop{\sum_{n\in\Z}\fh_n \opTrn^n \opDil} & \ge 1
%    && \text{by Neumann Expansion Theorem \xref{thm:op_net2}}
%  \\\implies&&
%  1
%      &\le \normop{\sum_{n\in\Z}\fh_n \opTrn^n \opDil}
%     &&    %\text{by Neumann Expansion Theorem \xref{thm:op_net2}}
%  \\&&&\le \sum_{n\in\Z}  \normop{\fh_n \opTrn^n \opDil}
%     &&    \text{by generalized triangle inequality \ifdochas{vsnorm}{\xref{thm:norm_tri}}}
%  \\&&&=   \sum_{n\in\Z}  \abs{\fh_n}\; \cancelto{1}{\normop{ \opTrn^n \opDil}}
%     &&    \text{by homogeneous property of norm \ifdochas{vsnorm}{\xref{def:norm}}}
%  \\&&&=   \sum_{n\in\Z}  \abs{\fh_n}
%     &&    \ifdochas{operator}{\text{by \prefp{prop:op_unitary_UV} and \prefp{thm:unitary_prop}}}
%\end{align*}
%\end{proof}




%\pref{thm:gen_quadcon}
\pref{thm:Sphi} (next) presents the \structe{quadrature necessary conditions} of a \structe{wavelet system}.
These relations simplify dramatically in the special case of an
\structe{orthonormal wavelet system} \xref{thm:oquadcon}.
%--------------------------------------
\begin{theorem}[\thmd{Quadrature condition} in ``frequency"]
\citetbl{
  \citerp{chui}{135},
  \citerp{goswami}{110}
  }
\label{thm:Sphi}
%--------------------------------------
Let $\mrasys$ be an \structe{MRA system} \xref{def:mrasys}.
Let $\Fx(\omega)$ be the \fncte{discrete time Fourier transform}\ifsxrefs{dsp}{def:dtft}for a sequence $\seqxZ{x_n}$ in $\spllR$.
Let $\Swphi(\omega)$ be the \fncte{auto-power spectrum} \xref{def:Swfg} of $\fphi$.
\thmbox{\begin{array}{>{\ds}lc>{\ds}l}
   \abs{\Dh\left(\omega     \right)}^2 \rnode[b]{noteSphi1}{\Sphi}(\omega) + \abs{\Dh\brp{\omega+\pi }}^2 \rnode[b]{noteSphi2}{\Sphi}(\omega+\pi) &=& 2\rnode[b]{noteSphi3}{\Sphi}(2\omega)
\end{array}}
\hfill
\rnode[bl]{noteSphi}{\footnotesize$\brp{\begin{array}{N}
  Note: $\Sphi(\omega)=1$\\% $\iff$ $\setxZ{\opTrn^n\fphi}$\\
  for \prope{orthonormal} MRA\\
  \ifxref{ortho}{lem:oms_quadcon}
\end{array}}$}
\ncarc[arcangle=30,linewidth=0.5pt,linecolor=red,linestyle=dashed]{->}{noteSphi}{noteSphi1}
\ncarc[arcangle=30,linewidth=0.5pt,linecolor=red,linestyle=dashed]{->}{noteSphi}{noteSphi2}
\ncarc[arcangle=30,linewidth=0.5pt,linecolor=red,linestyle=dashed]{->}{noteSphi}{noteSphi3}
\end{theorem}
\begin{proof}
%\begin{enumerate}
%  \item First note that $\Dh(\omega)$ and $\Dg(\omega)$ are periodic with period $2\pi$ such that\ifsxrefs{dsp}{prop:dtft_2pi} 
%  \label{item:qc1}
%    \\\indentx$\begin{array}{rclC}
%         \Dh(\omega+2\pi n) &=& \Dh(\omega) & \forall n\in\Z   \\
%         \Dg(\omega+2\pi n) &=& \Dg(\omega) & \forall n\in\Z   
%       \end{array}$
%
%\item Proof for theorem statement:
\begin{align*}
   &2\Sphi(2\omega)
   \\&= 2\brp{2\pi}\sum_{n\in\Z} \left|\Fphi(2\omega+2\pi n)\right|^2
     && \ifdochas{pwrspec}{\text{by \prefp{thm:Swfg}}}
   \\&= 2\brp{2\pi}\sum_{n\in\Z} \left|\cwt \Dh\left(\frac{2\omega+2\pi n}{2}\right)\Fphi\left(\frac{2\omega+2\pi n}{2}\right)\right|^2
     && \text{by \prefp{lem:Fphi}}
   \\&= \mathrlap{
          {2\pi}\sum_{n\in\Ze} \left|\Dh\left(\frac{2\omega+2\pi n}{2}\right)\right|^2\left|\Fphi\left(\frac{2\omega+2\pi n}{2}\right)\right|^2 +
          {2\pi}\sum_{n\in\Zo} \left|\Dh\left(\frac{2\omega+2\pi n}{2}\right)\right|^2\left|\Fphi\left(\frac{2\omega+2\pi n}{2}\right)\right|^2
          }
   \\&= \mathrlap{
          2\pi\sum_{n\in\Z} \left|\Dh\left(\omega+2\pi n\right)\right|^2\left|\Fphi\left(\omega+2\pi n\right)\right|^2 +
          2\pi\sum_{n\in\Z} \left|\Dh\left(\omega+2\pi n+\pi \right)\right|^2\left|\Fphi\left(\omega+2\pi n+ \pi \right)\right|^2
          }
   \\&= 2\pi\sum_{n\in\Z} \left|\Dh\left(\omega\right)\right|^2\left|\Fphi\left(\omega+2\pi n\right)\right|^2 +
        2\pi\sum_{n\in\Z} \left|\Dh\left(\omega+\pi \right)\right|^2\left|\Fphi\left(\omega+2\pi n+ \pi \right)\right|^2
     && \text{by \prefp{prop:dtft_2pi}}
        %\indentx\text{by (\prefp{item:qc1})}
   \\&= \left|\Dh\left(\omega     \right)\right|^2 \brp{2\pi\sum_{n\in\Z} \left|\Fphi\left(\omega    +2\pi n\right)\right|^2} +
        \left|\Dh\left(\omega+\pi \right)\right|^2 \brp{2\pi\sum_{n\in\Z} \left|\Fphi\left(\omega+\pi+2\pi n\right)\right|^2}
   \\&= \left|\Dh\left(\omega     \right)\right|^2 \Sphi(\omega) +
        \left|\Dh\left(\omega+\pi \right)\right|^2 \Sphi(\omega+\pi)
     && \ifdochas{pwrspec}{\text{by \prefp{thm:Swfg}}}
\end{align*}

%\end{enumerate}
\end{proof}


%=======================================
\section{Sufficient conditions}
%=======================================
\pref{thm:mra_rdc} (next) gives a set of \emph{sufficient} conditions on the \fncte{scaling function} \xref{def:mra} 
$\fphi$ to generate an \structe{MRA}.
\ifdochas{ortho}{\prefpp{thm:h_ns} provides a set of sufficient conditions on the \fncte{scaling coefficients} \xref{def:hn} $\seqnZ{\fh_n}$ 
to generate an \structe{MRA}; howbeit, this set results in the more restrictive \prope{orthonormal} MRA.}
%---------------------------------------
\begin{theorem}
\citetbl{
  \citerpgc{wojtaszczyk1997}{28}{0521578949}{Theorem 2.13},
  \citerpgc{pinsky2002}{313}{0534376606}{Theorem 6.4.27}
  }
\label{thm:mra_rdc}
\label{thm:mra_sufficient_phi}
%--------------------------------------
%Let an \structe{MRA} be defined as in \prefp{def:mra}.
%Let a \structe{Riesz sequence} be defined as in \prefp{def:rieszseq}.
Let $\spV_j\eqd\linspan\setxZ{\opTrn\fphi(x)}$\ifsxref{frames}{def:span}.
\thmbox{
  \brb{\begin{array}{FMD}
    (1). & $\seqn{\opTrn^n\fphi}$ is a \structe{Riesz sequence}\ifsxref{frames}{def:rieszseq}    & and \\
    (2). & $\ds\exists \seqn{\fh_n} \st \fphi(x)=\sum_{n\in\Z}h_n\opDil\opTrn^n\fphi(x)$ & and \\
    (3). & $\Fphi(\omega)$ is \prope{continuous} at $0$                                & and \\
    (4). & $\Fphi(0)\neq0$
  \end{array}}
  \implies
  \brb{\begin{array}{N}
    $\seqjZ{\spV_j}$ is an \structe{MRA}\\
    \xref{def:mra}
  \end{array}}
  }
\end{theorem}
\begin{proof}
For this to be true, each of the conditions in the definition of an \structe{MRA} \xref{def:mra} must be satisfied:
\begin{enumerate}
  \item Proof that each $\spV_j$ is \prope{closed}: by definition of $\linspan$

  \item Proof that $\seqn{\spV_j}$ is \prope{linearly ordered}: 
    \begin{align*}
      \spV_j\subseteq\spV_{j+1}
        &\iff \linspan\setn{\opDil^j\opTrn^n\fphi}\subseteq\linspan\setn{\opDil^{j+1}\opTrn^n\fphi}
        &\iff (2)
    \end{align*}

  \item Proof that $\ds\Setu_{j\in\Z}\spV_j$ is \prope{dense} in $\spLLR$: by \prefp{prop:mra_UVj}
  
  \item Proof of \prope{self-similar} property:
    \begin{align*}
      \brb{\ff\in\spV_j\iff\opDil\ff\in\spV_{j+1}}
        &\iff \ff\in\linspan\setn{\opTrn^n\fphi}\iff\opDil\ff\in\linspan\setn{\opDil\opTrn^n\fphi}
        &\iff (2)
    \end{align*}

  \item Proof for \structe{Riesz basis}: by (1) and \prefp{prop:mra_UVj}.
\end{enumerate}
\end{proof}

%=======================================
\section{Support size}
%=======================================
The \hie{support} of a function is what it's non-zero part ``sits" on.
If the support of the scaling coefficients $\seqn{\fh_n}$ goes from say
$[0,3]$ in $\Z$, what is the support of the scaling function $\fphi(x)$?
The answer is $[0,3]$ in $\R$---essentially the same
as the support of $\seqn{\fh_n}$ except that the two functions have different
domains ($\Z$ versus $\R$).
This concept is defined in \pref{def:support} (next definition),
and proven in \pref{thm:mra_support} (next theorem).
%and illustrated in \prefpp{sec:examples_pounity}\ifdochas{compactp}{ and \prefpp{sec:examples_Dp}}.
%\pref{ex:sw_gh_d2} -- \pref{ex:sw_gh_bspline} (\prefpo{ex:sw_gh_d2} -- \prefpo{ex:sw_gh_bspline}).

%--------------------------------------
\begin{definition}
\index{support}
\label{def:support}
%--------------------------------------
Let $\mrasys$ be an \structe{MRA system} \xref{def:mrasys}.
Let $\cls{\setX}$ represent the \ope{closure} of a set $\setX$ in $\spLLR$, 
$\hxs{\join}\setX$ the \vale{least upper bound} of an \structe{ordered set} $\opair{\setX}{\orel}$,
$\hxs{\meet}\setX$ the \vale{greatest lower bound} of $\opair{\setX}{\orel}$,
and
\\\indentx$\begin{array}{lclCD}
  \hxs{\floor{x}} &\eqd& \joinop\set{n\in\Z}{n\le x} & \forall x\in\R & (\vale{floor} of $x$)\\
  \hxs{\ceil{x}}  &\eqd& \meetop\set{n\in\Z}{n\ge x} & \forall x\in\R & (\vale{ceiling} of $x$).
\end{array}$
\defboxt{
  The set $\support\ff$ of a function $\ff\in\clFxy$ is the \structd{support} of $\ff$ if
  \\\indentx$\ds
    \support\ff \eqd
    \brbl{\begin{array}{>{\ds}lMDD}
      \cls{\set{x\in\R}{\ff(x)\ne 0}}
        & for $\setX=\R$ 
        & (domain of $\ff$ is $\R$)
        & and
      \\
      \cls{\set{x\in\R}{\ff\brp{\floor{x}}\ne0 \text{ and } \ff\brp{\ceil{x}}\ne0}}
        & for $\setX=\Z$
        & (domain of $\ff$ is $\Z$)
        & .
    \end{array}}
  $
  }
\end{definition}

%--------------------------------------
\begin{theorem}[\thmd{support size}]
\citetbl{
  \citerppg{mallat}{243}{244}{012466606X}
  }
\label{thm:mra_support}
\thmx{support size}
%--------------------------------------
Let $\mrasys$ be an \structe{MRA system} \xref{def:mrasys}.\\
Let $\support\ff$ be the support of a function $\ff$ \xref{def:support}.
\thmbox{
  %\begin{array}{>{\ds}rc>{\ds}l}
  \support\fphi = \support\fh
  %\\
  %\xN\in\Zo \text{ and } \fg_n  = \pm(-1)^\xN\fh(\xN-n) \quad\implies\quad
  %\support\fpsi &=& \left[ \frac{\xN-(n_2-n_1)}{2}, \frac{\xN+(n_2-n_1)}{2} \right]
  %\end{array}$
  }
\end{theorem}
\begin{proof}
\begin{enumerate}

  \item Definitions:\quad \label{item:mra_support_def}
    $\begin{array}[t]{rcl}
      \support\fphi &\eqd& [a,b] \\
      \support\fh   &\eqd& [k,m].
     \end{array}$

  \item lemma:\quad \label{ilem:mra_support_lemma}
    $\support\fphi(x)=\brs{a,b} \quad\iff\quad \support\fphi(2x)=\brs{\frac{a}{2},\frac{b}{2}}$

  \item lemma:\quad \label{ilem:mra_support_mpy}
    $\support\brs{\lambda\fphi(x)}=\support\brs{\fphi(x)}\quad\forall\lambda\in\R\setd0$

  \item Proof that $k=a$:
    \begin{align*}
      a
        &= \meetop\support\fphi(x)
        && \text{by definition of $a$} && \text{\xref{item:mra_support_def}}
      \\&\eqd \meetop\support \brs{ \sum_{n\in\Z}\fh_n \opDil\opTrn^n\fphi(x) }
        && \text{by \thme{dilation equation}} && \text{\xref{thm:dilation_eq}}
      \\&= \meetop\support \brs{\sqrt{2}\sum_{n\in\Z} \fh_n \fphi(2x-n)}
        && \text{by definition of $\opTrn$ and $\opDil$} && \text{\xref{def:opTD}}
      \\&= \meetop\support \brs{\sum_{n\in\Z} \fh_n \fphi(2x-n)}
        && \text{by \pref{ilem:mra_support_mpy}}
      \\&= \meetop\support \brs{\fh_{k} \fphi\brp{2x-k}}
        && \mathrlap{\text{because $n={k}$ is the \vale{least value} of $n$ for which $\fh_n\neq0$}}
      \\&= \meetop\support \brs{\fphi\brp{2x-k}}
        && \text{by \pref{ilem:mra_support_mpy}}
        && \text{\mbox{}\hspace{45mm}\mbox{}}
      \\&= \meetop\support \brs{\fphi\brp{2\brs{x-\frac{k}{2}}}}
      \\&= \meetop\set{t}{\fphi\brp{2\brs{x-\frac{k}{2}}}\neq0}
        && \text{by definition of $\support$} && \text{\xref{def:support}}
      \\&= x \st  x-\frac{k}{2} = \frac{a}{2}
        && \text{by \pref{ilem:mra_support_lemma}}
      \\&= \frac{k}{2} + \frac{a}{2}
      \\&\implies &\qquad\frac{k}{2} = a - \frac{a}{2}
      \\&\iff &\qquad k = a
    \end{align*}

  \item Proof that $m=b$:
    \begin{align*}
      b
        &= \joinop\support\fphi(x)
        && \text{by definition of $b$} && \text{\xref{item:mra_support_def}}
      \\&\eqd \joinop\support \brs{ \sum_{n\in\Z}\fh_n \opDil\opTrn^n\fphi(x) }
        && \text{by \thme{dilation equation}} && \text{\xref{thm:dilation_eq}}
      \\&= \joinop\support \brs{\sqrt{2}\sum_{n\in\Z} \fh_n \fphi(2x-n)}
        && \text{by definition of $\opTrn$ and $\opDil$} && \text{\xref{def:opTD}}
      \\&= \joinop\support \brs{\sum_{n\in\Z} \fh_n \fphi(2x-n)}
        && \text{by \pref{ilem:mra_support_mpy}}
      \\&= \joinop\support \brs{\fh_{m} \fphi\brp{2x-m}}
        && \mathrlap{\text{because $n={m}$ is the \vale{greatest value} of $n$ for which $\fh_n\neq0$}}
      \\&= \joinop\support \brs{\fphi\brp{2x-m}}
        && \text{by \pref{ilem:mra_support_mpy}}
      \\&= \joinop\support \brs{\fphi\brp{2\brs{x-\frac{m}{2}}}}
      \\&= \joinop\set{t}{\fphi\brp{2\brs{x-\frac{m}{2}}}\neq0}
        && \text{by definition of $\support$} && \text{\xref{def:support}}
      \\&= x \st  x-\frac{m}{2} = \frac{b}{2}
        && \text{by \pref{ilem:mra_support_lemma}}
      \\&= \frac{m}{2} + \frac{b}{2}
      \\&\implies \qquad\frac{m}{2} = b - \frac{b}{2}
      \\&\iff \qquad m = b
    \end{align*}

\end{enumerate}
\end{proof}

%--------------------------------------
\begin{example}
%--------------------------------------
\exbox{\begin{array}{NNN}
    \includegraphics{graphics/n1_h.pdf}
   &\includegraphics{graphics/n2_h.pdf}
   %&\includegraphics{graphics/d2_phi_h.pdf}
   &\includegraphics{graphics/d3_phi_h.pdf}
  \\\fncte{B-spline $\fN_1(x)$}
   &\fncte{B-spline $\fN_2(x)$}
   %&\fncte{Daubechies-2 scaling function}
   &\fncte{Daubechies-3 scaling function}
  \\\xref{ex:bspline_N1}&\xref{ex:bspline_N2}&\ifxref{compactp}{ex:dau-p3}
\end{array}}
\end{example}

%=======================================
\section{Scaling functions with partition of unity}
%=======================================

%\ifdochaselse{sequence}{\prefpp{prop:twz} demonstrates that the}{The} 
The $Z$ transform \xrefP{def:opZ} of a sequence $\seqn{h_n}$ 
with sum $\sum_{n\in\Z} (-1)^n\fh_n=0$ has a zero at $z=-1$.
Somewhat surprisingly, the \prope{partition of unity} and
\prope{zero at $z=-1$} properties are actually equivalent (next theorem). 

\begin{center}
\begin{tabular}{cc}
  \includegraphics{graphics/pounityz.pdf}&\includegraphics{graphics/Fhw.pdf}
\end{tabular}
\end{center}


%--------------------------------------
\begin{theorem} %[Partition of unity]
\footnote{
  \citePp{jawerth}{8},
  \citerp{chui}{123}
  }
\label{thm:pun_zero}
\index{zero at $z=-1$}
\index{partition of unity}
\index{sum of even}
\index{sum of odd}
%--------------------------------------
Let $\mrasys$ be a \structe{multiresolution system} \xref{def:mrasys}. % wavelet system.
Let $\opFT\ff(\omega)$ be the \fncte{Fourier transform} \xref{def:opFT} of a function $\ff\in\spLLR$.
Let $\kdelta_n$ be the \fncte{Kronecker delta function}\ifsxref{frames}{def:kdelta}.
Let $c$ be some contant in $\R\setd0$.
\thmbox{
  \mcom{\sum_{n\in\Z} \opTrn^n \fphi = c}
       {(1) \prope{partition of unity}}
  \quad\iff\quad
  \mcom{\sum_{n\in\Z} (-1)^n\fh_n=0}
       {(2) \prope{zero at $z=-1$}}
  \quad\iff\quad
  \mcom{\sum_{n\in\Z}\fh_{2n} = \sum_{n\in\Z}\fh_{2n+1} = \cwt}
       {(3) sum of even = sum of odd = $\cwt$}
  }
\end{theorem}
\begin{proof}
Let $\Ze$ be the set of even integers and $\Zo$ the set of odd integers.
\begin{enumerate}
\item Proof that (1)$\impliedby$(2):
\begin{align*}
  \sum_{n\in\Z} \opTrn^n \fphi
    &= \sum_{n\in\Z} \opTrn^n \brs{\sum_{m\in\Z}\fh_m \opDil \opTrn^m \fphi}
    && \text{by \thme{dilation equation}} && \text{\xref{thm:dilation_eq}}
  \\&= \sum_{m\in\Z}\fh_m \sum_{n\in\Z} \opTrn^n \opDil \opTrn^m \fphi
  \\&= \sum_{m\in\Z}\fh_m \sum_{n\in\Z} \opDil \opTrn^{2n} \opTrn^m \fphi
    && \text{by \thme{commutator relation}} && \text{\xref{prop:DTTD}}
  \\&= \opDil \sum_{m\in\Z}\fh_m \sum_{n\in\Z} \opTrn^{2n} \opTrn^m \fphi
    %&& \text{by \prefp{prop:opTD_sum}}
  \\&= \opDil \sum_{m\in\Z}\fh_m \brs{\sqrt{\frac{2\pi}{2}}\opFSi\opS_2\opFT\brp{\opTrn^m \fphi}}
    && \text{by \thme{PSF}} && \text{\xref{thm:psf}}
  \\&= \sqrt{\pi} \opDil \sum_{m\in\Z}\fh_m \opFSi\opS_2 e^{-i\omega m}\opFT\fphi
    && \text{by \prefp{cor:FTD}}
  \\&= \sqrt{\pi} \opDil \sum_{m\in\Z}\fh_m \opFSi e^{-i\frac{2\pi}{2}k m}\opS_2 \opFT\fphi
    && \text{by definition of $\opS$}
    && \text{\xref{def:opS}}
  \\&= \sqrt{\pi} \opDil \sum_{m\in\Z}\fh_m \opFSi \brp{-1}^{km} \opS_2 \opFT\fphi
  \\&= \sqrt{\pi} \opDil \sum_{m\in\Z}\fh_m \brs{\cwt \sum_{k\in\Z} \brp{-1}^{km} \brp{\opS_2 \opFT\fphi} e^{i\frac{2\pi}{2}kx}}
    && \text{by definition of $\opFSi$} && \text{\xref{thm:opFSi}}
  \\&= \frac{\sqrt{2\pi}}{2} \opDil \sum_{k\in\Z}  \brp{\opS_2 \opFT\fphi} e^{i\pi kx}\sum_{m\in\Z} \brp{-1}^{km}\fh_m 
  \\&=\mathrlap{%
        \frac{\sqrt{2\pi}}{2} \opDil \sum_{k\in\Ze} \brp{\opS_2 \opFT\fphi} e^{i\pi kx}\sum_{m\in\Z} \brp{-1}^{km}\fh_m 
       +\frac{\sqrt{2\pi}}{2} \opDil \sum_{k\in\Zo}  \brp{\opS_2 \opFT\fphi} e^{i\pi kx}\sum_{m\in\Z} \brp{-1}^{km}\fh_m 
       }%
  \\&=\mathrlap{
        \frac{\sqrt{2\pi}}{2} \opDil \sum_{k\in\Ze} \brp{\opS_2 \opFT\fphi} e^{i\pi kx} \mcom{\sum_{m\in\Z}\fh_m}{$\sqrt{2}$}
       +\frac{\sqrt{2\pi}}{2} \opDil \sum_{k\in\Zo}  \brp{\opS_2 \opFT\fphi} e^{i\pi kx} \mcom{\sum_{m\in\Z} \brp{-1}^m\fh_m }{$0$}
       }%
  \\&= \sqrt{\pi} \opDil \sum_{k\in\Ze} \brp{\opS_2 \opFT\fphi} e^{i\pi kx} 
    && \text{by \prefpp{thm:admiss}}&& \text{and right hypothesis}
  \\&= \sqrt{\pi} \opDil \sum_{k\in\Ze} \Fphi\brp{\frac{2\pi}{2}k} e^{i\pi kx} 
    && \text{by definitions of $\opFT$ and $\opS_2$}
  \\&= \sqrt{\pi} \opDil \sum_{k\in\Z} \Fphi\brp{2\pi k} e^{i2\pi kx} 
    && \text{by definition of $\Ze$}
  \\&= \frac{1}{\sqrt{2}}\opDil \brb{\sqrt{2\pi} \sum_{k\in\Z} \Fphi\brp{2\pi k} e^{i2\pi kx}} 
  \\&= \frac{1}{\sqrt{2}}\opDil \sum_{n\in\Z}\fphi(x+n)
    && \text{by \thme{PSF}} && \text{\xref{thm:psf})}
  \\&= \frac{1}{\sqrt{2}}\opDil \sum_{n}\opTrn^n\fphi
    && \text{by definition of $\opTrn$} && \text{\xref{def:opT}}
\end{align*}

The above equation sequence demonstrates that 
  \\\indentx$\ds\opDil\sum_n\opTrn^n\fphi = \sqrt{2}\sum_n\opTrn^n\fphi$\\
(essentially that $\sum_n\opTrn^n\fphi$ is equal to it's own dilation).
This implies that $\sum_n\opTrn^n\fphi$ is a constant \xref{prop:opD_constant}.

\item Proof that (1)$\implies$(2):
\begin{align*}
  c &= \sum_{n\in\Z} \opTrn^n \fphi
    && \text{by left hypothesis}
  \\&= \sqrt{2\pi}\: \opFSi \opS \opFT \fphi
    && \text{by \thme{PSF}} && \text{\xref{thm:psf}}
  \\&= \sqrt{2\pi}\: \opFSi \opS 
       \mcom{\sqrt{2}
         \brp{\opDili\sum_{n\in\Z}\fh_n e^{-i\omega n}}\: 
         \brp{\opDili \opFT \fphi}
       }{$\opFT\fphi$}
    && \text{by \prefp{lem:Fphi}}
  \\&= 2\sqrt{\pi}\: \opFSi 
       \brp{\opS\opDili\sum_{n\in\Z}\fh_n e^{-i\omega n}}\: 
       \brp{\opS \opFT\opDil \fphi}
    && \text{by \prefp{cor:FTD}}
  \\&= 2\sqrt{\pi}\: \opFSi 
       \brp{\opS\frac{1}{\sqrt{2}}\sum_{n\in\Z}\fh_n e^{-i\frac{\omega}{2} n}}\: 
       \brp{\opS \opFT\opDil \fphi}
    && \text{by evaluation of $\opDili$} && \text{\xref{prop:opDi}}
  \\&= \sqrt{2\pi}\: \opFSi 
       \brp{\sum_{n\in\Z}\fh_n e^{-i\frac{2\pi k}{2} n}}\: 
       \brp{\opS \opFT\opDil \fphi}
    && \text{by definition of $\opS$}
    && \text{\xref{def:opS}}
  \\&= \sqrt{2\pi}\: \opFSi 
       \brp{\sum_{n\in\Z}\fh_n (-1)^{kn}}\: 
       \brp{\opS \opDili\opF \fphi}
  \\&= \sqrt{2\pi}\: \opFSi 
       \brp{\sum_{n\in\Z}\fh_n (-1)^{kn}}\: 
       \brp{\opS\frac{1}{\sqrt{2}}\Fphi\brp{\frac{\omega}{2}}}
    && \text{by definition of $\opS$} %&& \text{\xref{thm:psf}}
    && \text{\xref{def:opS}}
  \\&= \sqrt{2\pi}\: \opFSi 
       \brp{\sum_{n\in\Z}\fh_n (-1)^{kn}}\: 
       \brp{\frac{1}{\sqrt{2}}\Fphi\brp{\frac{2\pi k}{2}}}
  %\\&= \sqrt{\pi}\: \opFSi 
  %     \brp{\sum_{n\in\Z}\fh_n (-1)^{kn}}\: 
  %     \brp{\Fphi\brp{\frac{2\pi k}{2}}}
  \\&= \sqrt{\pi}\: \sum_{k\in\Z} 
       \sum_{n\in\Z}\fh_n (-1)^{kn}\: 
       \Fphi\brp{\pi k}
       e^{i 2\pi kx}
    && \text{by definition of $\opFSi$} && \text{\xref{thm:opFSi}}
  \\&= \mathrlap{%
       \sqrt{\pi}
       \sum_{\text{$k$ even}} 
       \sum_{n\in\Z}\fh_n (-1)^{kn}\: 
       \Fphi\brp{\pi k}
       e^{i 2\pi kx}
       +
       \sqrt{\pi}
       \sum_{\text{$k$ odd}} 
       \sum_{n\in\Z}\fh_n (-1)^{kn}\: 
       \Fphi\brp{\pi k}
       e^{i 2\pi kx}
       }
  \\&= \mathrlap{%
       \sqrt{\pi}
       \sum_{\text{$k$ even}} 
       \brp{\cancelto{\sqrt{2}}{\sum_{n\in\Z}\fh_n}} \: 
       \Fphi\brp{\pi k}
       e^{i 2\pi kx}
       +
       \sqrt{\pi}
       \sum_{\text{$k$ odd}} 
       \brp{\sum_{n\in\Z}\fh_n (-1)^{n}}\: 
       \Fphi\brp{\pi k}
       e^{i 2\pi kx}
       }
  \\&= \mathrlap{%
       \sqrt{\pi}\sum_{k\in\Z} 
       \sqrt{2} \: 
       \Fphi\brp{\pi 2k}
       e^{i 2\pi 2k x}
       +
       \sqrt{\pi}\sum_{k\in\Z}
       \brp{\sum_{n\in\Z}\fh_n (-1)^{n}}\: 
       \Fphi\brp{\pi [2k+1]}
       e^{i 2\pi [2k+1] x}
       \quad\text{by \prefp{thm:admiss}}
       }
  \\&= \frac{\sqrt{2\pi}}{\sqrt{2\pi}}
       \Fphi\brp{0}
       %e^{0}
       +
       \sqrt{\pi}e^{i 2\pi x}\sum_{n\in\Z}\fh_n (-1)^{n}\: 
       \sum_{k\in\Z}
       \Fphi\brp{\pi [2k+1]}
       e^{i 4\pi kx}
    && \mathrlap{\text{by left hypothesis and \prefp{thm:pounity_freq}}}
%
%
%
%  \\&= \sqrt{\pi}\sum_{k\in\Z}\brs{
%       \sqrt{2} 
%       \Fphi\brp{2\pi k}
%       +
%       \brp{\sum_{n\in\Z}\fh_n (-1)^{n}}\: 
%       \Fphi\brp{\pi [2k+1]}
%       e^{i 2\pi x}
%       }e^{i 4\pi kx}
%  \\&= \sqrt{\pi}\brs{
%       \sqrt{2} 
%       \Fphi(0)
%       +
%       \brp{\sum_{n\in\Z}\fh_n (-1)^{n}}\: 
%       \Fphi(\pi)
%       e^{i 2\pi x}
%       }
%    && \text{$k$ must be $0$ to make real constant for all $t$}
  \\&\implies \qquad \brp{\sum_{n\in\Z}\fh_n (-1)^{n}}=0
    && \mathrlap{\text{because the right side must equal $c$}}
\end{align*}

%\end{enumerate}
%\end{proof}
%
%
%%-------------------------------------
%\begin{theorem}
%\label{thm:zero_unity_evenodd}
%\footnote{
%  \citerp{chui}{123}
%  }
%\index{zero at $z=-1$}
%\index{partition of unity}
%\index{sum of even}
%\index{sum of odd}
%%-------------------------------------
%Let $\wavsys$ be a \hi{wavelet system}.
%\thmbox{
%  \mcom{\sum_{n\in\Z} \fphi(x-n)=\int_\R\fphi(x)\dx }
%       {(1) partition of unity}
%  \iff
%  \mcom{\sum_{n\in\Z} (-1)^n \fh_n  = 0 }
%       {(2) zero at $z=-1$}
%  \iff
%  \mcom{\sum_{n\in\Z} \fh_{2n} = \sum_{n\in\Z} \fh_{2n+1} = \frac{1}{\sqrt{2}}}
%       {(3) sum of even = sum of odd = $\cwt $}
%  }
%\end{theorem}
%\begin{proof}
%Let $\Fh(\omega)\eqd\sum_{n\in\Z} \fh_n \fkernea{x}{\omega}$.
%\begin{enumerate}
%
%%\item Proof that $\sum_n(-1)^n \fh_n =0 \implies \sum_n\fphi(x-n)=\int_t\fphi(x)\dt$:
%%\[\begin{array}{rclllllll}
%%  \Fphi(\omega) &=& \frac{1}{\sqrt{2}}\Fh(\omega/2) \Fphi(\omega/2) \\
%%  \Fphi(2\pi k) &=& \frac{1}{\sqrt{2}}\Fh(\pi k) \Fphi(\pi k) \\
%%  \Fphi(2\pi 0) &=& \frac{1}{\sqrt{2}}\Fh(\pi 0) \Fphi(\pi 0)
%%                &=& \frac{1}{\sqrt{2}}\sqrt{2}   \int_t\fphi(x)\dt
%%                &=& \int_t\fphi(x)\dt \\
%%  \Fphi(2\pi 1) &=& \frac{1}{\sqrt{2}}\Fh(\pi ) \Fphi(\pi )
%%                &=& \frac{1}{\sqrt{2}} 0 \Fphi(\pi )
%%                &=& 0 \\
%%  \Fphi(2\pi 2) &=& \frac{1}{\sqrt{2}}\Fh(2\pi ) \Fphi(2\pi )
%%                &=& \frac{1}{\sqrt{2}}\Fh(2\pi ) 0
%%                &=& 0 \\
%%  \Fphi(2\pi 3) &=& \frac{1}{\sqrt{2}}\Fh(3\pi ) \Fphi(3\pi )
%%                &=& \frac{1}{\sqrt{2}}\Fh(\pi ) \Fphi(3\pi )
%%                &=& \frac{1}{\sqrt{2}} 0 \Fphi(3\pi )
%%                &=& 0 \\
%%  \Fphi(2\pi 4) &=& \frac{1}{\sqrt{2}}\Fh(4\pi ) \Fphi(4\pi )
%%                &=& \frac{1}{\sqrt{2}}\Fh(4\pi ) 0
%%                &=& 0 \\
%%  \vdots \\
%%  \Fphi(2\pi k) &=& \kdelta_k \int_t\fphi(x)\dt
%%\end{array}\]
%%$\ds \implies \sum_{n\in\Z} \fphi(x-n)=\int_t\fphi(x)\dt$
%%(by \prefp{thm:pounity_freq}).
%
%\item Proof that (2) $\implies$ (1):
%\begin{align*}
%  \sum_{n\in\Z} \fphi(x-n)
%    &= \sum_{n\in\Z} \left[ \sqrt{2} \sum_{m\in\Z}\fh_m \fphi(2t-2n-m) \right]
%    && \text{by \thme{dilation equation} (\prefpo{thm:h->phi})}
%  \\&= \sqrt{2} \sum_{m\in\Z}\fh_m \sum_{n\in\Z} \fphi(2t-2n-m)
%  \\&= \sqrt{2}\sum_{m\in\Z}\fh_m \frac{1}{2}\sum_{k\in\Z} \Fphi\left(\frac{2\pi k}{2}\right) e^{i\frac{2\pi}{2}k(2x-m)}
%    && \text{by \thme{PSF} (\prefpo{thm:psf})}
%  \\&= \cwt  \sum_{k\in\Z} \Fphi(\pi k) e^{i2\pi kx} \sum_{m\in\Z}\fh_m e^{-i\pi km}
%  \\&= \cwt  \sum_{k\in\Z} \Fphi(\pi k) e^{i2\pi kx} \sum_{m\in\Z}\fh_m (-1)^{km}
%  \\&= \cwt  \sum_{\text{$k$ even}} \Fphi(\pi k) e^{i2\pi kx} \sum_{m\in\Z}\fh_m (-1)^{km}
%    \\&\qquad+ \cwt  \sum_{\text{$k$ odd}}  \Fphi(\pi k) e^{i2\pi kx} \sum_{m\in\Z}\fh_m (-1)^{km}
%  \\&= \cwt  \sum_{k\in\Z}  \Fphi(\pi 2k   ) e^{i2\pi 2kx}    \cancelto{\sqrt{2}}{\sum_{m\in\Z}\fh_m }
%     \\&\qquad+ \cwt  \sum_{k\in\Z}  \Fphi(\pi[2k+1]) e^{i2\pi(2k+1)x} \cancelto{0}{\sum_{m\in\Z} (-1)^m\fh_m }
%  \\&= \sum_{k\in\Z}  \Fphi(2\pi k) e^{i4\pi kx}
%    && \text{by left hypothesis}
%  \\&= \Fphi(0) + \mcom{\sum_{k\ne 0}  \Fphi(2\pi k) e^{i4\pi kx}}{0 because no imaginary part $\forall x$}
%    && \text{because $\sum_{n\in\Z} \fphi(x-n)$ has no imag. part}
%  \\&= \Fphi(0)
%  \\&= \left.\int_t \fphi(x) \fkernea{x}{\omega} \dt \right|_{\omega=0}
%  \\&= \int_t \fphi(x) \dt
%\end{align*}
%
%\item Proof that (2) $\impliedby$ (1):
%\begin{align*}
%  \sum_n(-1)^n \fh_n
%    &= \Fh(\pi)
%    && \text{by Proposition~\ref{prop:twz} }
%  \\&= \sqrt{2} \frac{\Fphi(2\pi)}{\Fphi(\pi)}
%    && \text{by \prefp{lem:Fphi}}
%  \\&= \sqrt{2} \frac{\kdelta_1}{\Fphi(\pi)}
%    && \text{by right hypothesis and \prefp{thm:pounity_freq}}
%  \\&= \sqrt{2} \frac{0}{\Fphi(\pi)}
%  \\&= 0
%\end{align*}

\item Proof that (2)$\implies$(3):
  \begin{align*}
    \sum_{n\in\Ze}\fh_n = \sum_{n\in\Zo}\fh_n 
      &= \frac{1}{2}\sum_{n\in\Z}\fh_n
      && \text{by (2) and \prefp{prop:dsp_zminone}}
    \\&= \cwt
      && \text{by \thme{admissibility condition} \xref{thm:admiss}}
  \end{align*}

\item Proof that (2)$\impliedby$(3):
\begin{align*}
  \cwt
    &= \mcom{\sum_{n\in\Ze} (-1)^n \fh_n}{even terms}
      +\mcom{\sum_{n\in\Zo} (-1)^n \fh_n}{odd terms}
    && \text{by (3)}
  \\&\implies \sum_{n\in\Z} (-1)^n\fh_n = 0
    && \text{by \prefp{prop:dsp_zminone}}
\end{align*}
\end{enumerate}

\end{proof}



Not every function that forms a \prope{partition of unity} is a \structe{basis} for an \structe{MRA},
as formerly stated next and demonstrated by 
\prefpp{cnt:pun_sin2} and \prefpp{cnt:pun_rsin}.
%---------------------------------------
\begin{proposition}
%---------------------------------------
\propbox{
  \text{$\fphi(x)$ generates a \prope{partition of unity}}
  \qquad\notimplies\qquad
  \text{$\fphi(x)$ generates an \structe{MRA} system.}
  }
\end{proposition}
\begin{proof}
  By \prefpp{cnt:pun_sin2} and \prefpp{cnt:pun_rsin}.
\end{proof}



%%=======================================
%\section{Examples}
%\label{sec:examples_pounity}
%%=======================================
%All of the scaling functions in the examples in this section form a \hie{partition of unity}, 
%but yet are not orthogonal.
%In particular, note that for all of them $\sum_{n\in\Z} (-1)^n\fh_n = 0$ (see \pref{thm:pun_zero}).
%
%

%-------------------------------------
\begin{counterex}
\label{cnt:pun_sin2}
%-------------------------------------
Let a function $\fphi$ be defined in terms of the sine function \xref{def:sin} as follows:
\cntbox{\begin{array}{m{\tw-85mm-18mm}m{85mm}}
      $\fphi(x) \eqd \brbl{%
       \begin{array}{>{\ds}lM}
          \sin^2\brp{\frac{\pi}{2}x}  & for $x\in\intcc{0}{2}$  \\
          0    & otherwise
       \end{array}}$
&\includegraphics{graphics/sinsin02.pdf}
\\Then $\int_\R\fphi(x)\dx=1$ and $\fphi$ induces a \structe{partition of unity}
&\includegraphics{graphics/sinsin02_pun.pdf}
\\\textbf{but} $\setxZ{\opTrn^n\fphi}$ does \textbf{not} generate an \structe{MRA}.
\end{array}}
\end{counterex}
\begin{proof}
Let $\setind_\setA(x)$ be the \fncte{set indicator function} \xref{def:setind} on a set $\setA$.
\begin{enumerate}
  \item Proof that $\int_\R\fphi(x)\dx=1$:  by \prefpp{ex:pun_sin2}
    %\begin{align*}
    %  \int_\R \fphi(x) \dx
    %    &= \int_\R \sin^2\brp{\frac{\pi}{2}x}\setind_\intcc{0}{2}(x) \dx
    %    && \text{by definition of $\fphi(x)$}
    %  \\&= \int_0^2 \sin^2\brp{\frac{\pi}{2}x} \dx
    %    && \text{by definition of $\setindAx$ \xref{def:setind}}
    %  \\&= \int_0^2 \frac{1}{2}\brs{1-\cos\brp{\pi x}} \dx
    %    && \text{by \prefp{thm:trig_sq}}
    %  \\&= \frac{1}{2}\brs{x-\frac{1}{\pi}\sin\brp{\pi x}}_0^2
    %  \\&= \frac{1}{2}\brs{2-0-0-0}
    %  \\&= 1
    %\end{align*}
  
  \item Proof that $\fphi(x)$ forms a \prope{partition of unity}: by \prefpp{ex:pun_sin2}
    %\begin{align*}
    %  \sum_{n\in\Z} \opTrn^n\fphi(x)
    %    &= \sum_{n\in\Z} \opTrn^n\sin^2\brp{\frac{\pi}{2}x}\setind_\intcc{0}{2}(x)
    %    && \text{by definition of $\fphi(x)$}
    %  \\&= \sum_{n\in\Z} \opTrn^n\sin^2\brp{\frac{\pi}{2}x}\setind_\intco{0}{2}(x)
    %    && \text{because $\sin^2\brp{\frac{\pi}{2}x}=0$ when $x=2$}
    %  \\&= \sum_{m\in\Z} \opTrn^{m-1}\sin^2\brp{\frac{\pi}{2}x}\setind_\intco{0}{2}(x)
    %    && \text{where $m\eqd n+1$ $\implies$ $n=m-1$}
    %  \\&= \sum_{m\in\Z} \sin^2\brp{\frac{\pi}{2}(x-m+1)}\setind_\intco{0}{2}(x-m+1)
    %    && \text{by definition of $\opTrn$ \xref{def:opT}}
    %  \\&= \sum_{m\in\Z} \sin^2\brp{\frac{\pi}{2}(x-m)+\frac{\pi}{2}}\setind_\intco{-1}{1}(x-m)
    %  \\&= \sum_{m\in\Z} \cos^2\brp{\frac{\pi}{2}(x-m)}\setind_\intco{-1}{1}(x-m)
    %    && \text{by \prefp{thm:trig_sq}}
    %  \\&= \sum_{m\in\Z} \opTrn^m\cos^2\brp{\frac{\pi}{2}x}\setind_\intco{-1}{1}(x)
    %    && \text{by definition of $\opTrn$ \xref{def:opT}}
    %  \\&= \sum_{m\in\Z} \opTrn^m\cos^2\brp{\frac{\pi}{2}x}\setind_\intcc{-1}{1}(x)
    %    && \text{because $\cos^2\brp{\frac{\pi}{2}x}=0$ when $x=1$}
    %  \\&= 1
    %    && \text{by \prefp{ex:pun_cos2}}
    %\end{align*}

  \item Proof that $\fphi(x)\notin\linspan\setxZ{\opDil\opTrn^n\fphi(x)}$ (and so does not generate an \structe{MRA}):
    \begin{enumerate}
      \item Note that the \prope{support} \xref{def:support} of $\fphi$ is $\support\fphi=\intcc{0}{2}$.
      \item Therefore, the \prope{support} of $\seqn{h_n}$ is $\support\seqn{h_n}=\setn{0,1,2}$ \xref{thm:mra_support}.
      \item So if $\fphi(x)$ \emph{is} an \structe{MRA}, we only need to compute $\setn{h_0,h_1,h_2}$ (the rest would be $0$).
            \\\indentx\includegraphics{graphics/sinsin02hn.pdf}\\
            Here would be the values of $\setn{h_1,h_2,h_3}$:\label{item:sinsin02hn_fig}
        \begin{align*}
          \fphi(x)
            &= \sum_{n\in\Z}\fh_n \opDil\opTrn^n\fphi(x)
          \\&= \sum_{n\in\Z}\fh_n \opDil\opTrn^n\sin^2\brp{\frac{\pi}{2}x}\setind_\intcc{0}{2}(x)
          \\&= \sum_{n\in\Z}\fh_n \sin^2\brp{\frac{\pi}{2}(2x-n)}\setind_\intcc{0}{2}(2x-n)
          \\&= \sum_{n=0}^2\fh_n  \sin^2\brp{\frac{\pi}{2}(2x-n)}\setind_\intcc{0}{2}(2x-n)
            && \text{by \prefp{thm:mra_support}}
        \end{align*}

      \item The values of $\seqn{h_0,h_1,h_2}$ can be conveniently calculated at the knot locations $x=\frac{1}{2}$, $x=1$, and $x=\frac{3}{2}$ (see the diagram in \prefp{item:sinsin02hn_fig}):
        \begin{align*}
          \cwt \cdot\frac{1}{2}
            &= \cwt \brp{\frac{1}{\sqrt{2}}}^2
          \\&= \cwt \sin^2\brp{\frac{\pi}{4}}
          \\&\eqd \cwt \fphi\brp{\frac{1}{2}}
          \\&= \cwt \sqrt{2}\sum_{n\in\Z}\fh_n \sin^2\brp{\frac{\pi}{2}(1-n)}\setind_\intcc{0}{2}(1-n)
          \\&=\fh_0 \sin^2\brp{\frac{\pi}{2}(1-0)}\setind_\intcc{0}{2}(1-0)
             +\fh_1 \sin^2\brp{\frac{\pi}{2}(1-1)}\setind_\intcc{0}{2}(1-1)
             \\&\qquad+\fh_2 \sin^2\brp{\frac{\pi}{2}(1-2)}\setind_\intcc{0}{2}(1-2)
          \\&=\fh_0 \cdot1\cdot1
             +\fh_1 \cdot0\cdot1
             +\fh_2 (-1)\cdot0
          \\&=\fh_0
          \\\\
          \cwt \cdot1
            &= \cwt \brp{1}^2
          \\&= \cwt \sin^2\brp{\frac{\pi}{2}}
          \\&\eqd \cwt \fphi\brp{1}
          \\&= \cwt \sqrt{2}\sum_{n\in\Z}\fh_n \sin^2\brp{\frac{\pi}{2}(2-n)}\setind_\intcc{0}{2}(2-n)
          \\&=\fh_0 \sin^2\brp{\frac{\pi}{2}(2-0)}\setind_\intcc{0}{2}(2-0)
             +\fh_1 \sin^2\brp{\frac{\pi}{2}(2-1)}\setind_\intcc{0}{2}(2-1)
             \\&\qquad+\fh_2 \sin^2\brp{\frac{\pi}{2}(2-2)}\setind_\intcc{0}{2}(2-2)
          \\&=\fh_0 \cdot0\cdot1
             +\fh_1 \cdot1\cdot1
             +\fh_2 \cdot0\cdot1
          \\&=\fh_1
          \\\\
          \cwt \cdot\frac{1}{2}
            &= \cwt \brp{\frac{1}{-\sqrt{2}}}^2
          \\&= \cwt \sin^2\brp{\frac{3\pi}{4}}
          \\&\eqd \cwt \fphi\brp{\frac{3}{2}}
          \\&= \cwt \sqrt{2}\sum_{n\in\Z}\fh_n \sin^2\brp{\frac{\pi}{2}(3-n)}\setind_\intcc{0}{2}(3-n)
          \\&=\fh_0 \sin^2\brp{\frac{\pi}{2}(3-0)}\setind_\intcc{0}{2}(3-0)
             +\fh_1 \sin^2\brp{\frac{\pi}{2}(3-1)}\setind_\intcc{0}{2}(3-1)
             \\&\qquad+\fh_2 \sin^2\brp{\frac{\pi}{2}(3-2)}\setind_\intcc{0}{2}(3-2)
          \\&=\fh_0 \cdot(-1)\cdot0
             +\fh_1 \cdot0\cdot1
             +\fh_2 1\cdot1
          \\&=\fh_2
        \end{align*}
      \item These values for $\seqn{h_0,h_1,h_2}$ are valid for the knot locations $x=\frac{1}{2}$, $x=1$, and $x=\frac{3}{2}$, 
            \textbf{but} they don't satisfy the \fncte{dilation equation} \xref{thm:dilation_eq}. In particular,
            \\\indentx$\ds\fphi(x) \neq \sum_{n\in\Z}\fh_n \opDil\opTrn^n \fphi(x)$
            \\(see the diagram in \prefp{item:sinsin02hn_fig})
    \end{enumerate}
\end{enumerate}
\end{proof}



%-------------------------------------
\begin{counterex}[\exmd{raised sine}]
\label{cnt:pun_rsin}
\footnote{
  \citerppg{proakis}{560}{561}{0-07-232111-3}
  }
%-------------------------------------
Let a function $\ff$ be defined in terms of a shifted cosine function \xref{def:cos} as follows:
%\exbox{\begin{array}{m{58mm}m{\tw-76mm}}
%\cntbox{\begin{array}{m{85mm-15mm}m{85mm}}
\cntbox{\begin{array}{l}
  \begin{array}{l}
  \fphi(x) \eqd \brbl{%
   \begin{array}{>{\ds}lM>{\ds}l}
      \left.\left.\frac{1}{2}\right\{1 + \cos\brs{\pi\brp{\abs{x-1}}}\right\}
        & for & 0 \le x < 2
        \\
      0    & \mc{2}{M}{otherwise}
   \end{array}}
   \end{array}\quad\tbox{\includegraphics{graphics/rsin.pdf}}%
   \\
   \begin{array}{M}
     Then $\fphi$ forms a \prope{partition of unity}:
   \end{array}\quad\tbox{\includegraphics{graphics/rsin_pun.pdf}}%
   \\
   \begin{array}{M}
     \textbf{but} $\setxZ{\opTrn^n\fphi}$ does \textbf{not}\\ 
     generate an \structe{MRA}.
   \end{array}\quad\tbox{\includegraphics{graphics/rsin02hn.pdf}}%
%\\\mc{2}{M}{Note that $\ds\Ff(\omega)= \frac{1}{2\sqrt{2\pi}}\Big[{\mcom{\frac{2\sin\omega}{\omega}}{$2\sinc(\omega)$}+\mcom{\frac{\sin(\omega-\pi)}{(\omega-\pi)}}{$\sinc(\omega-\pi)$} + \mcom{\frac{\sin(\omega+\pi)}{(\omega+\pi)}}{$\sinc(\omega+\pi)$}}\Big]$}
%\\and so $\Ff(2\pi n)=\frac{1}{\sqrt{2\pi}}\kdelta_n$:
%&\psset{xunit=7mm,yunit=14mm}%============================================================================
% Daniel J. Greenhoe
% LaTeX file
%
%
%            1      [ 2sin(w)   sin(w-pi)   sin(w+pi) ]
% F(w) = ---------  [ ------- + --------- + --------  ]
%        sqrt(2pi)  [    w        w-pi        w+pi    ]
%
% which is the Fourier transform of 
%   f(x)  = { cos^2(pi/2 x) for  -1 <= x <= 1
%           { 0             otherwise
% nominal xunit = 7.5mm
% nominal yunit = 15mm
%============================================================================
\begin{pspicture}(-4.75,-0.25)(5.25,0.58)%
  %-------------------------------------
  % labeling
  %-------------------------------------
  %-7.8748049728612098721453229972336
  %0.39894228040143267793994605993438
  %\uput{3.5pt}[0](0,0.395){$\frac{2}{2\sqrt{2\pi}}$}%
  \rput(4,0.39894228){\rnode{peakL}{$\frac{2}{2\sqrt{2\pi}}$}}%
  \pnode(0,0.39894228){peak}%
  \ncline[linestyle=dashed,linecolor=red,linewidth=0.75pt]{peak}{peakL}%
  %-------------------------------------
  % axes
  %-------------------------------------
  \psaxes[linecolor=axis,yAxis=false,linewidth=0.75pt,labels=none]{<->}(0,0)(-4.75,-0.5)(4.75,0.58)%
  \psaxes[linecolor=axis,xAxis=false,linewidth=0.75pt,ticks=none]{<->}(0,0)(0,-0.25)(0,0.58)%
  \uput{3.5pt}[0](4.75,0){$\omega$}%
  \rput[b]( 4,-4mm){$4\pi$}
  \rput[b]( 3,-4mm){$3\pi$}
  \rput[b]( 2,-4mm){$2\pi$}
  \rput[b]( 1,-4mm){$\pi$}
  \rput[b](-1,-4mm){$-\pi$}
  \rput[b](-2,-4mm){$-2\pi$}
  \rput[b](-3,-4mm){$-3\pi$}
  \rput[b](-4,-4mm){$-4\pi$}
  %-------------------------------------
  % plot functions
  %-------------------------------------
  %\psplot[plotpoints=64]{-3}{3}{180 x mul sin x 3 exp x 3.14159265 2 exp mul sub div -7.874805 mul 3.14159265 div}%
  %\psplot[plotpoints=64]{-3}{3}{180 x mul sin x div 180 x mul sin x 3.14159265 sub div sub 180 x mul sin x 3.14159265 add div sub -7.874805 mul 3.14159265 div}%
  %\psplot[plotpoints=64,linestyle=dotted]{-4.5}{4.5}{180 x mul sin x div 3.14159265 div}%
  %
  \psplot[plotpoints=64,linestyle=dotted,linecolor=purple]{-4.5}{4.5}{180 x mul sin x div 2 mul 2 3.14159265 mul sqrt div 2 div 3.14159265 div}%
  \psplot[plotpoints=64,linestyle=dotted,linecolor=purple]{-4.5}{4.5}{180 x 1 sub mul sin x 1 sub div 2 3.14159265 mul sqrt div 2 div 3.14159265 div}%
  \psplot[plotpoints=64,linestyle=dotted,linecolor=purple]{-4.5}{4.5}{180 x 1 add mul sin x 1 add div 2 3.14159265 mul sqrt div 2 div 3.14159265 div}%
  %\psplot[plotpoints=256]{-3}{3}{180 x mul sin}%
  %
  \psplot[plotpoints=256]{-4}{4}{180 x mul sin x div 2 mul 180 x 1 sub mul sin x 1 sub div add 180 x 1 add mul sin x 1 add div add 2 3.14159265 mul sqrt div 2 div 3.14159265 div}%
  \psplot[plotpoints=256,linestyle=dotted]{-4}{-4.5}{180 x mul sin x div 2 mul 180 x 1 sub mul sin x 1 sub div add 180 x 1 add mul sin x 1 add div add 2 3.14159265 mul sqrt div 2 div 3.14159265 div}%
  \psplot[plotpoints=256,linestyle=dotted]{4}{4.5}{180 x mul sin x div 2 mul 180 x 1 sub mul sin x 1 sub div add 180 x 1 add mul sin x 1 add div add 2 3.14159265 mul sqrt div 2 div 3.14159265 div}%
\end{pspicture}%
\end{array}}
\end{counterex}
\begin{proof}
Let $\setind_\setA(x)$ be the \fncte{set indicator function} \xref{def:setind} on a set $\setA$.
\begin{enumerate}
  \item Proof that $\fphi(x)$ forms a \prope{partition of unity}:
    \begin{align*}
      \sum_{n\in\Z} \opTrn^n\fphi(x)
        &= \sum_{n\in\Z} \opTrn^n\fphi(x+1)
        && \text{by \prefp{prop:opT_periodic}}
      \\&= \sum_{n\in\Z} \fphi(x+1-n)
        && \text{by \prefp{def:opT}}
        %&& \text{by definition of $\opTrn$ \xref{def:opT}}
      \\&= \sum_{n\in\Z} \frac{1}{2}\brb{1 + \cos\brs{\pi\brp{\abs{x-1+1-n}}}}\setind_\intco{0}{2}(x+1-n)
        && \text{by definition of $\fphi(x)$}
      \\&= \sum_{n\in\Z} \frac{1}{2}\brb{1 + \cos\brs{\pi\brp{\abs{x-n}}}}\setind_\intco{-1}{1}(x-n)
        && \text{by \prefp{def:setind}}
       %&& \text{by definition of $\setind$ \xref{def:setind}}
      \\&= \sum_{n\in\Z} \mcom{\left.\frac{1}{2}\brb{1 + \cos\brs{\frac{\pi}{\beta}\brp{\abs{x-n}-\frac{1-\beta}{2}}}}\setind_\intco{-1}{1}(x-n)\right|_{\beta=1}}
                {\exme{raised cosine} \xref{ex:pun_rcos} with $\beta=1$}
      \\&= 1
        && \text{by \prefp{ex:pun_rcos}}
        %&& \text{by \exme{raised cosine} example \xref{ex:rcos}}
    \end{align*}

  \item Proof that $\fphi(x)\notin\linspan\setxZ{\opDil\opTrn^n\fphi(x)}$ (and so does not generate an \structe{MRA}):
    \begin{enumerate}
      \item Note that the \prope{support} \xref{def:support} of $\fphi$ is $\support\fphi=\intcc{0}{2}$.
      \item Therefore, the \prope{support} of $\seqn{h_n}$ is $\support\seqn{h_n}=\setn{0,1,2}$ \xref{thm:mra_support}.
      \item So if $\fphi(x)$ \emph{is} an \structe{MRA}, we only need to compute $\setn{h_0,h_1,h_2}$ (the rest would be $0$).
            %\\\indentx\psset{unit=20mm}%============================================================================
% Daniel J. Greenhoe
% LaTeX file
% nominal unit = 20mm
%============================================================================
\begin{pspicture}(-2,-0.5)(4,1.5)%
  \psaxes[linecolor=axis,yAxis=false,linewidth=0.5pt]{<->}(0,0)(-2,0)(4,1.5)%
  \psaxes[linecolor=axis,xAxis=false,linewidth=0.5pt]{->}(0,0)(0.05,0)(0,1.5)%
  \normalsize
  \psline[linestyle=dashed,linecolor=red,linewidth=0.75pt](0,1)(1,1)%
  \psline[linestyle=dashed,linecolor=red,linewidth=0.75pt](1,0)(1,1)%
  %\psline[linestyle=dotted](1,1)(1,0)%
  \psline(-1.2, 0)( 0, 0)% left horizontal
  \psline(2,0)(3.2,0)% right horizontal
  \psline[linestyle=dotted](3.2,0)(3.75,0)%
  \psline[linestyle=dotted](-1.2,0)(-1.75,0)%
  \psplot[plotpoints=64]{0}{2}{90 x mul sin 2 exp}%
  \psplot[plotpoints=64]{0}{1}{90 x 2 mul 0 sub mul sin 2 exp 0.5 mul }%
  \psplot[plotpoints=64]{0.5}{1.5}{90 x 2 mul 1 sub mul sin 2 exp 1   mul }%
  \psplot[plotpoints=64]{1}{2}{90 x 2 mul 2 sub mul sin 2 exp 0.5 mul }%
  %
  \psplot[plotpoints=64]{0}{0.5}{90 x 2 mul 0 sub mul sin 2 exp 0.5 mul }%
  %
  \psplot[plotpoints=64]{0.5}{1}{90 x 2 mul 0 sub mul sin 2 exp 0.5 mul 90 x 2 mul 1 sub mul sin 2 exp 1   mul add}%
  %
  \psplot[plotpoints=64]{1}{1.5}{90 x 2 mul 1 sub mul sin 2 exp 1   mul 90 x 2 mul 2 sub mul sin 2 exp 0.5 mul add}%
  %
  \psplot[plotpoints=64]{1.5}{2}{90 x 2 mul 2 sub mul sin 2 exp 0.5 mul }%
  %
  %\rnode{fphi}{\rput[l](2,1){$\fphi(x)$}}
  %\rnode{fphihn}{\rput[l](2,0.5){$\ds\sum_{n\in\Z}h_n\opDil\opTrn^n\fphi(x)$}}
  %\rput[l](2,1){\rnode{fphi}{$\ds\fphi(x)$}}
  \rput[lt](1.75,1.25){$\ds\rnode{fphi}{\fphi(x)}\neq\rnode{fphihn}{\ds\sum_{n\in\Z}}h_n\opDil\opTrn^n\fphi(x)\quad\scy\forall x\in\R$}
  \pnode(1.3,0.8){sinsintop}
  \pnode(1.3,0.63){sinsinhn}
  \ncline[linewidth=0.75pt]{->}{fphi}{sinsintop}
  \ncline[linewidth=0.75pt]{->}{fphihn}{sinsinhn}
  %
  %\uput{3.5pt}[180](0.5,1){$1$}%
\end{pspicture}%
\\
            Here would be the values of $\setn{h_1,h_2,h_3}$:
        \begin{align*}
          \fphi(x)
            &= \sum_{n\in\Z}\fh_n \opDil\opTrn^n\fphi(x)
          \\&= \sum_{n\in\Z}\fh_n \opDil\opTrn^n\left.\left.\frac{1}{2}\right\{1 + \cos\brs{\pi\brp{\abs{x-1}}}\right\}\setind_\intcc{0}{2}(x)
            && \text{by definition of $\fphi(x)$}
          \\&= \sum_{n\in\Z}\fh_n \left.\left.\cwt \right\{1 + \cos\brs{\pi\brp{\abs{2x-1-n}}}\right\}\setind_\intcc{0}{2}(2x-n)
            && \text{by \prefp{def:opT}}
          \\&= \sum_{n=0}^2 \fh_n \left.\left.\cwt \right\{1 + \cos\brs{\pi\brp{\abs{2x-1-n}}}\right\}\setind_\intcc{0}{2}(2x-n)
            && \text{by \prefp{thm:mra_support}}
        \end{align*}

      \item The values of $\seqn{h_0,h_1,h_2}$ can be conveniently calculated at the knot locations $x=\frac{1}{2}$, $x=1$, and $x=\frac{3}{2}$ (see the diagram in \prefp{item:sinsin02hn_fig}):
        \begin{align*}
          \frac{1}{2}
            &= \left.\sum_{n=0}^2 \fh_n \left.\left.\cwt \right\{1 + \cos\brs{\pi\brp{\abs{2x-1-n}}}\right\}\setind_\intcc{0}{2}(2x-n)\right|_{x=\frac{1}{2}}
          \\&=\fh_0 \left.\left.\cwt \right\{1 + \cos\brs{1-1-0}\right\}
          \\&=\fh_0 \sqrt{2}
          \\&\implies\fh_0 = \frac{\sqrt{2}}{4}
          \\
          \\
          1
            &= \left.\sum_{n=0}^2 \fh_n \left.\left.\cwt \right\{1 + \cos\brs{\pi\brp{\abs{2x-1-n}}}\right\}\setind_\intcc{0}{2}(2x-n)\right|_{x=1}
          \\&=\fh_1 \left.\left.\cwt \right\{1 + \cos\brs{2-1-1}\right\}
          \\&=\fh_1 \sqrt{2}
          \\&\implies\fh_1 = \cwt 
          \\
          \\
          \frac{1}{2}
            &= \left.\sum_{n=0}^2 \fh_n \left.\left.\cwt \right\{1 + \cos\brs{\pi\brp{\abs{2x-1-n}}}\right\}\setind_\intcc{0}{2}(2x-n)\right|_{x=\frac{3}{2}}
          \\&=\fh_2 \left.\left.\cwt \right\{1 + \cos\brs{1-1-0}\right\}
          \\&=\fh_2 \sqrt{2}
          \\&\implies\fh_2 = \frac{\sqrt{2}}{4}
        \end{align*}

      \item These values for $\seqn{h_0,h_1,h_2}$ are valid for the knot locations $x=\frac{1}{2}$, $x=1$, and $x=\frac{3}{2}$, 
            \textbf{but} they don't satisfy the \fncte{dilation equation} \xref{thm:dilation_eq}. In particular (see diagram),
            \\\indentx$\ds\fphi(x) \neq \sum_{n\in\Z}\fh_n \opDil\opTrn^n \fphi(x)$ .
    \end{enumerate}
\end{enumerate}
\end{proof}

%--------------------------------------
\begin{example}[\exmd{2 coefficient case}/\exmd{Haar wavelet system}/\exmd{order 0 B-spline wavelet system}]
\footnote{
  \citor{haar1910},
  \citerppgc{wojtaszczyk1997}{14}{15}{0521578949}{``Sources and comments"}
  }
\label{ex:pun_n=2}
%--------------------------------------
%\prefpp{ex:ortho_n=2} used the admissibility condition (\prefp{thm:admiss})
%and orthonormal constrained quadrature condition (\prefp{thm:ortho_quadcon})
%to design a 2 coefficient wavelet analysis.
%In this example, we remove the orthnormality constraint and replace it with 
%the more general \hie{partition of unity} constraint
%(all orthonormal wavelet analyses possess the partition of unity property---
%\prefp{cor:pun_ortho}).
%But even under this relaxed constraint, the same coefficients are still produced.
%These coefficients are the \hie{Haar wavelet analysis}.
%They can also be produced using other systems of equations including the following:
%\begin{dingautolist}{"AC}
%  \item Admissibility condition and \hie{orthonormality}---\prefpp{ex:ortho_n=2}
%  \item \hie{Daubechies-$p1$} wavelets computed using spectral techniques---\prefpp{ex:dau-p1}
%\end{dingautolist}
%\\
%Then the scaling coefficients $h_0$ and $h_1$ must have the following values:
%Some wavelet coefficients, the scaling function, and the wavelet function are also illustrated next.
\\Let $\wavsys$ be an \structe{wavelet system}.
\exbox{
  \brb{\begin{array}{FMMD}
    1. & $\support\fphi(x)=\intcc{0}{1}$                             & \xref{thm:mra_support}  & and\\
    2. & \prope{admissibility condition}                             & \xref{thm:admiss}       & and \\
    3. & \prope{partition of unity}                                  & \xref{thm:pun_zero}     & 
   %4. & $\fg_n  = (-1)^n \fh_{\xN-n}^\ast$ $\scy\forall n\in\Z$      & \xref{thm:wavstrct_cqf} & %\xref{thm:cqf}
  \end{array}}
  \implies
  \brb{\begin{array}{r|r}%
    n  & \fh_{n}\\  %& g_n\\
    \hline
    0            & \ds \cwt\\   %& \ds \cwt   \\
    1            & \ds \cwt\\   %& \ds -\cwt  \\
    \text{other} & 0          %& 0
  \end{array}}
  %\brb{\begin{array}{r|r|r}%
  %  n  & \fh_{n}  & g_n\\
  %  \hline
  %  0  & \ds \cwt   & \ds \cwt   \\
  %  1  & \ds \cwt   & \ds -\cwt  \\
  %  \text{other} & 0 & 0
  %\end{array}}
  \tbox{\includegraphics{graphics/d1_phi_h.pdf}}
  }%
%\\
%  \begin{tabular}{cc}%
%    \includegraphics{graphics/d1_phi_h.pdf}&\includegraphics{graphics/d1_psi_g.pdf}%
%  \end{tabular}
\end{example}
\begin{proof}
\begin{enumerate}
  \item Proof that (1) $\implies$ that only $h_0$ and $h_1$ are non-zero: by \prefp{thm:mra_support}.

  \item Proof for values of $h_0$ and $h_1$:
    \begin{enumerate}
      \item Method 1: 
        Under the constraint of two non-zero scaling coefficients, 
        a scaling function design is fully constrained using the \hie{admissibility equation} \xref{thm:admiss} 
        and the \hie{partition of unity} constraint\ifsxref{partuni}{def:pun}.
        The partition of unity formed by $\fphi(x)$ is illustrated in \prefpp{ex:n0_pounity}. %\prefp{ex:pounity_squarepulse}.
    
        Here are the equations:
        \\\indentx$\ds\begin{array}{rcl@{\qquad}MMM}
         \fh_0 +\fh_1 &=& \sqrt{2}    & (admissibility equation          & \pref{thm:admiss}   & \prefpo{thm:admiss}) \\
         \fh_0 -\fh_1 &=& 0           & (partition of unity/zero at $-1$ & \pref{thm:pun_zero} & \prefpo{thm:pun_zero})
        \end{array}$
        \\
        Here are the calculations for the coefficients:
        \begin{align*}
          (h_0+h_1)+(h_0-h_1) &= 2h_0 &&= \sqrt{2}         &&\text{(add two equations together)}\\
          (h_0+h_1)-(h_0-h_1) &= 2h_1 &&= \sqrt{2}         &&\text{(subtract second from first)} \\
          \\
          %g_0 &=\fh_1 \\
          %g_1 &= -h_0
        \end{align*}
    
      \ifdochas{spline}{\item Method 2: By \prefp{thm:Bsplineh}.}
    \end{enumerate}

\ifdochas{compactp}{
  \item Note: $h_0$ and $h_1$ can also be produced using other systems of equations including the following:
    \begin{dingautolist}{"AC}
      \item \prope{admissibility condition} and \prope{orthonormality} \xref{ex:ortho_n=2}
      \item \fncte{Daubechies-$p1$} wavelets computed using spectral techniques\ifsxref{compactp}{ex:dau-p1}
    \end{dingautolist}
  }
  %\item Proof for values of $g_0$ and $g_1$: by \prefp{thm:wavstrct_cqf}.
\end{enumerate}
\end{proof}


%%--------------------------------------
%\begin{example}[\exm{order 1 B-spline wavelet system}]
%\footnote{
%  \citerp{strang89}{616},
%  \citerppgc{dau}{146}{148}{0898712742}{\textsection 5.4}
%  }
%\label{ex:sw_gh_tent}
%\exmx{tent function}
%%--------------------------------------
%The following figures illustrate scaling and wavelet coefficients and functions
%for the \hie{B-Spline $B_2$}, or \fncte{tent function}. % \xref{ex:pounity_tent}.
%The partition of unity formed by the scaling function $\fphi(x)$ is illustrated in \prefpp{ex:n1_pounity}. %\prefp{ex:pounity_tent}.
%\\\exbox{\begin{array}{m{40mm}m{51mm}m{51mm}}
%  $\begin{array}{>{\scy}c|r|r}
%    n & \mc{1}{c}{h_n} & \mc{1}{c}{g_n} \\
%    \hline
%      0   &  \brp{\frac{\sqrt{2}}{4}}  &   \brp{\frac{\sqrt{2}}{4}}  \\
%      1   & 2\brp{\frac{\sqrt{2}}{4}}  & -2\brp{\frac{\sqrt{2}}{4}}            \\
%      2   &  \brp{\frac{\sqrt{2}}{4}}  &   \brp{\frac{\sqrt{2}}{4}}
%  \end{array}$
%  &\includegraphics{graphics/tent_phi_h.pdf}&\includegraphics{graphics/tent_psi_g.pdf}
%\end{array}}
%\end{example}
%\begin{proof}
%These results follow from \prefp{thm:Bsplineh}.
%\\\indentx$\brp{\begin{array}{*{9}{c}}
%    &   &   &   & 1 &   &   &   &  \\
%    &   &   & 1 &   & 1 &   &   &  \\
%    &   & 1 &   & 2 &   & 1 &   &  
%\end{array}}$
%\end{proof}
%
%%--------------------------------------
%\begin{example}[\exm{order 3 B-spline wavelet system}]
%\footnote{
%  \citerp{strang89}{616}
%  }
%\label{ex:sw_gh_bspline}
%\exmx{B-spline}
%%--------------------------------------
%The following figures illustrate scaling and wavelet coefficients and functions
%for a \hie{B-spline}.
%\\\exbox{\begin{array}{m{40mm}m{51mm}m{51mm}}
%  $\begin{array}{>{\scy}c|c|r}
%    n & \mc{1}{c}{h_n} & \mc{1}{c}{g_n} \\
%    \hline
%      0   & { }\brp{\frac{\sqrt{2}}{16}}  &  { }\brp{\frac{\sqrt{2}}{16}}  \\
%      1   & {4}\brp{\frac{\sqrt{2}}{16}}  & -{4}\brp{\frac{\sqrt{2}}{16}}  \\
%      2   & {6}\brp{\frac{\sqrt{2}}{16}}  &  {6}\brp{\frac{\sqrt{2}}{16}}  \\
%      3   & {4}\brp{\frac{\sqrt{2}}{16}}  & -{4}\brp{\frac{\sqrt{2}}{16}}  \\
%      4   & { }\brp{\frac{\sqrt{2}}{16}}  &  { }\brp{\frac{\sqrt{2}}{16}}
%  \end{array}$
%  &\includegraphics{graphics/bspline_phi_h.pdf}&\includegraphics{graphics/bspline_psi_g.pdf}
%\end{array}}
%\end{example}
%\begin{proof}
%These results follow from \prefp{thm:Bsplineh}.
%\\\indentx$\brp{\begin{array}{*{9}{c}}
%    &   &   &   & 1 &   &   &   &  \\
%    &   &   & 1 &   & 1 &   &   &  \\
%    &   & 1 &   & 2 &   & 1 &   &  \\
%    & 1 &   & 3 &   & 3 &   & 1 &  \\
%  1 &   & 4 &   & 6 &   & 4 &   & 1
%\end{array}}$
%\end{proof}

