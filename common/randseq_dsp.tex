%============================================================================
% LaTeX File
% Daniel J. Greenhoe
%============================================================================
%======================================
\chapter{Random Sequences DSP}
%======================================
%=======================================
\section{Additive noise}
%=======================================
%---------------------------------------
\begin{proposition}
\footnote{
  \citerpg{shin2008}{293}{0470725648}
  }
\label{prop:Swxyv}
%---------------------------------------
Let $\fh[n]$ be the \fncte{impulse response} of a system 
with input $\fx[n]$ and output $\fy[n]$.
Let $\fv[n]$ be \fncte{noise sequence} and $\fy'[n]\eqd\fy[n]+\fv[n]$.
\propbox{
  \brb{\begin{array}{FMMD}
      (A).& $\fx[n]$ and $\fy[n]$ are &\prope{WSS}          & and
    \\(B).& $\fx[n]$ and $\fv[n]$ are &\prope{uncorrelated} &
  \end{array}}
  \implies
  \brb{\begin{array}{FrclD}
      (1).&\Swxx[y'y'](\omega) &=& \Swyx(\omega)                 & and
    \\(2).&\Swxx[y'x] (\omega) &=& \Swyy(\omega) + \Swvv(\omega) & and
    \\(3).&\Swxx[xy'] (\omega) &=& \Swxy(\omega) & 
  \end{array}}
  }
\end{proposition}
\begin{proof}
\begin{align*}
  \Swxx[y'y'](\omega)
    &\eqd \opDTFT\Rxx[y'y'][m]
    && \text{by definition of \ope{DTFT}}
    && \text{\xref{def:dtft}}
  \\&\eqd \opDTFT\pE\brs{\rvy'[m]\rvy'^\ast[0]}
    && \text{by definition of $\Rxy$}
    && \text{\xref{def:Rxym}}
  \\&\eqd \opDTFT\pE\brs{(\rvy[m]+\fv[m])(\rvy[0]+\fv[0])^\ast[0]}
    && \text{by definition of $\rvy'$}
  \\&= \opDTFT\pE\brs{\rvy[m]\rvy^\ast[0]}
     + \cancelto{0}{\opDTFT\pE\brs{\rvy[m]\rvv^\ast[0]}}
     + \cancelto{0}{\opDTFT\pE\brs{\fv[m]\rvy^\ast[0]} }
     + \opDTFT\pE\brs{\fv[m]\rvv^\ast[0]}
  \\&= \opDTFT\pE\brs{\rvy[m]\rvy^\ast[0]}
     + 0 + 0
     + \opDTFT\pE\brs{\fv[m]\rvv^\ast[0]}
    && \text{by \prope{uncorrelated} hypothesis}
    && \text{(C)}
  \\&= \opDTFT\Ryy[m] + \opDTFT\Rvv[m]
    && \text{by definition of $\Rxy$}
    && \text{\xref{def:Rxym}}
  \\&= \Swyy(\omega) + \Swvv(\omega)
    && \text{by definition of $\Swyy$}
    && \text{\xref{def:Swxy}}
  \\
  \Swxx[y'x](\omega)
    &\eqd \opDTFT\Rxx[y'x][m]
    && \text{by definition of \ope{DTFT}}
    && \text{\xref{def:dtft}}
  \\&\eqd \opDTFT\pE\brs{\rvy'[m]\rvx^\ast[0]}
    && \text{by definition of $\Rxy$}
    && \text{\xref{def:Rxym}}
  \\&\eqd \opDTFT\pE\brs{(\rvy[m]+\fv[m])\rvx^\ast[0]}
    && \text{by definition of $\rvy'$}
  \\&= \opDTFT\pE\brs{\rvy[m]\rvx^\ast[0]}
     + \opDTFT\pE\brs{\fv[m]\rvx^\ast[0]}
  \\&= \opDTFT\pE\brs{\rvy[m]\rvx^\ast[0]} +    0
    && \text{by \prope{uncorrelated} hypothesis}
    && \text{(B)}
  \\&= \Swyx(\omega)
    && \text{by definition of $\Swxy$}
    && \text{\xref{def:Swxy}}
  \\
  \Swxx[xy'](\omega)
    &= \Swxx^\ast[y'x](\omega)
    && \text{by \prefp{cor:Swxy}}
  \\&= \Swyx^\ast(\omega)
    && \text{by previous result}
  \\&= \Swxy(\omega)
    && \text{by \prefp{cor:Swxy}}
\end{align*}
\end{proof}

%=======================================
\section{Whitening}
\index{whitening filter}
\label{sec:d-whiten}
%=======================================
\begin{figure}[h]
  \centering
  \includegraphics{graphics/pz_minphase.pdf}
  \caption{
     Poles ($\times$) and zeros ($o$) of a \prope{minimum phase} filter
     \label{fig:w_pz_minphase}
     }
\end{figure}
%---------------------------------------
\begin{definition}
\index{minimum phase}
\index{rational expression}
%---------------------------------------
Let $\Zh(z)$ be the z-transform of the impulse response of a filter.
If $\Zh(z)$ can be expressed as a rational expression with poles and zeros
$r_ne^{i\theta_n}$,
then the filter is \textbf{minimum phase} if each $r_n<1$
(all roots lie inside the unit circle in the complex $z$-plane).
\end{definition}
See \prefp{fig:w_pz_minphase}.

Note that if $\fL(z)$ has a root at $z=re^{i\theta}$, then
$\fL^\ast(1/z^\ast)$ has a root at
\begin{align*}
   \frac{1}{z^\ast}
     &= \frac{1}{\brp{re^{i\theta}}^\ast}
      = \frac{1}{re^{-i\theta}}
      = \frac{1}{r} e^{i\theta}.
\end{align*}
That is, if $\fL(z)$ has a root inside the unit circle,
then $\fL^\ast(1/z^\ast)$ has a root directly opposite across the unit circle
boundary (see \prefp{fig:z-roots}).
A causal stable filter $\Zh(z)$ must have all of its poles inside
the unit circle.
A minimum phase filter is a filter with both its poles and zeros inside the
unit circle.
One advantage of a minimum phase filter is that its recipricol
(zeros become poles and poles become zeros)
is also causal and stable.

\begin{figure}[ht]
\begin{center}
\begin{fsL}
\setlength{\unitlength}{0.2mm}
\begin{picture}(300,300)(-130,-130)
  %\graphpaper[10](0,0)(200,200)
  \thicklines%
  \color{axis}%
    \put(-130 ,   0 ){\line(1,0){260} }%
    \put(   0 ,-130 ){\line(0,1){260} }%
    \put( 140 ,   0 ){\makebox(0,0)[l]{$\Re$}}%
    \put(   0 , 140 ){\makebox(0,0)[b]{$\Im$}}%
  \color{zero}%
    \qbezier[24](  0,  0)( 56.5,56.5)(113,113)
    %\put(   0 ,   0 ){\line(1,1){120}}%
    \put(  28 ,  28 ){\circle{10}}%
    \put( 113 , 113 ){\circle{10}}%
    \put(  38 ,  28 ){\makebox(0,0)[l]{zero of $\fL(z)$}}%
    \put( 123 , 113 ){\makebox(0,0)[l]{zero of $\fL^\ast\brp{\frac{1}{z^\ast}}$}}%
  \color{pole}%
    \qbezier[24](0,0)(-61.5,-26.5)(-123,-53)%
    %\put(   0 ,   0 ){\line(-3,-1){130}}%
    \put( -76 , -25 ){\makebox(0,0){$\times$}}%
    \put(-119 , -40 ){\makebox(0,0){$\times$}}%
    \put( -76 , -25 ){\makebox(0,0)[lt]{pole of $\fL(z)$}}%
    \put(-119 , -40 ){\makebox(0,0)[lt]{pole of $\fL^\ast\brp{\frac{1}{z^\ast}}$}}%
  \color{circle}%
    %============================================================================
% NCTU - Hsinchu, Taiwan
% LaTeX File
% Daniel Greenhoe
%
% Unit circle with radius 100
%============================================================================

\qbezier( 100,   0)( 100, 41.421356)(+70.710678,+70.710678) % 0   -->1pi/4
\qbezier(   0, 100)( 41.421356, 100)(+70.710678,+70.710678) % pi/4-->2pi/4
\qbezier(   0, 100)(-41.421356, 100)(-70.710678,+70.710678) %2pi/4-->3pi/4
\qbezier(-100,   0)(-100, 41.421356)(-70.710678,+70.710678) %3pi/4--> pi 
\qbezier(-100,   0)(-100,-41.421356)(-70.710678,-70.710678) % pi  -->5pi/4
\qbezier(   0,-100)(-41.421356,-100)(-70.710678,-70.710678) %5pi/4-->6pi/4
\qbezier(   0,-100)( 41.421356,-100)( 70.710678,-70.710678) %6pi/4-->7pi/4
\qbezier( 100,   0)( 100,-41.421356)( 70.710678,-70.710678) %7pi/4-->2pi



\end{picture}
\end{fsL}
\end{center}
\caption{
   Mirrored roots in complex-z plane
   \label{fig:z-roots}
   }
\end{figure}


\begin{figure}[ht]\color{figcolor}
\begin{fsK}
\begin{center}
  \setlength{\unitlength}{0.2mm}
  \begin{picture}(700,100)(-100,-60)
  \thicklines
  %\graphpaper[10](0,0)(160,80)
  \put(-100,  10 ){\makebox (100, 40)[b]{$\rvx(n)$}                  }
  \put(-100, -50 ){\makebox (100, 40)[t]{$\Rxx(m)$}                  }
  \put(-100, -50 ){\makebox (100, 40)[b]{$\Sxx(z)$}                  }
  \put(-100,   0 ){\vector  (  1,  0){100}                           }

  \put(   0, -50 ){\framebox(100,100)   {$\convd\gamma(n)$}           }
  \put(   0, -40 ){\makebox (100, 80)[t]{whitening}                  }
  \put(   0, -40 ){\makebox (100, 80)[b]{$\Gamma(z)$}                }
  \put( 100,   0 ){\vector  (  1,  0)   {200}                        }
  \put( 100,  10 ){\makebox (200, 40)[t]{white noise process}        }
  \put( 100,  10 ){\makebox (200, 40)[b]{$\vw(n)$}                 }
  \put( 100, -50 ){\makebox (200, 40)[t]{$\Rww(m)=\delta(m)$}  }
  \put( 100, -50 ){\makebox (200, 40)[b]{$\Sww(z)=1$}                }

  \put( 300, -50 ){\framebox(100,100)   {$\convd\fl(n)$}               }
  \put( 300, -40 ){\makebox (100, 80)[t]{innovations}                }
  \put( 300, -40 ){\makebox (100, 80)[b]{$\fL(z)$}                     }
  \put( 400,   0 ){\vector  (  1,  0)   {100}                        }
  \put( 400,  10 ){\makebox (100, 40)[b]{$\rvx(n)$}                  }
  \put( 400, -50 ){\makebox (200, 40)[t]{$\Rxx(m)=l(m)\convd\fl^\ast(-m)$}  }
  \put( 400, -50 ){\makebox (200, 40)[b]{$\Sxx(z)=L(z)\fL^\ast\brp{\frac{1}{z^\ast}}$}  }
  \end{picture}
\caption{
   Innovations and whitening filters
   \label{fig:d-innovations}
   }
\end{center}
\end{fsK}
\end{figure}


The next theorem demonstrates a method for ``whitening"
a \fncte{random sequence} $\fx(n)$ with a filter constructed from a decomposition
of $\Rxx(m)$.
The technique is stated precisely in \prefp{thm:d-innovations}
and illustrated in \prefp{fig:d-innovations}.
Both imply two filters with impulse responses $l(n)$ and $\gamma(n)$.
Filter $l(n)$ is referred to as the \textbf{innovations filter}
(because it generates or ``innovates" $\fx(n)$ from a white noise
process $\fw(n)$)
and $\gamma(n)$ is referred to as the \textbf{whitening filter}
because it produces a white noise sequence when the input sequence
is $\fx(n)$.\footnote{\citerpp{papoulis}{401}{402}}


%---------------------------------------
\begin{theorem}
\label{thm:d-innovations}
%---------------------------------------
Let $\fx(n)$ be a WSS \fncte{random sequence} with auto-correlation $\Rxx(m)$
and spectral density $\Sxx(z)$.
\textbf{If} $\Sxx(z)$ has a \textbf{rational expression},
then the following are true:

\begin{enume}
   \item There exists a rational expression $\fL(z)$ with minimum phase
         such that
         \[ \Sxx(z) =\fL(z)\fL^\ast\brp{\frac{1}{z^\ast}}. \]
   \item An LTI filter for which the Laplace transform of
         the impulse response $\gamma(n)$ is
         \[ \Gamma(z) = \frac{1}{\fL(z)} \]
         is both causal and stable.
   \item If $\fx(n)$ is the input to the filter $\gamma(n)$,
         the output $\fy(n)$ is a \textbf{white noise sequence} such that
         \[ \Syy(z)=1 \hspace{2cm} \Ryy(m)=\kdelta(m).\]
\end{enume}
\end{theorem}


\begin{proof}
\begin{align*}
   \Sww(z)
     &= \Gamma(z)\Gamma^\ast\brp{\frac{1}{z^\ast}} \Sxx(z)
   \\&= \frac{1}{\fL(z)} \frac{1}{\fL^\ast\brp{\frac{1}{z^\ast}}} \Sxx(z)
   \\&= \frac{1}{\fL(z)} \frac{1}{\fL^\ast\brp{\frac{1}{z^\ast}}}
        \fL(z)\fL^\ast\brp{\frac{1}{z^\ast}}
   \\&= 1
\end{align*}
\end{proof}

