%============================================================================
% LaTeX File
% Daniel J. Greenhoe
%============================================================================
%=======================================
\chapter{Model Estimation}
\label{chp:model_estimation}
%=======================================
A key point in model building is choosing what to model.
Defining a model with too many system properties (too many dimensions) can quickly lead to 
an intractable solution computationally and require more data than is available
(the ``curse of dimensionality").

For modeling real-world random processes above the quantum level, here are some possibly useful parameters to estimate and use:
\begin{enumerate}
  \item Derivatives $0,1,\ldots n$, where $n$ may be 2 or 3.
        Brook Taylor showed that for \prope{analytic} functions,\footnote{analytic function: a function for which all its derivatives exist}
        knowledge of the derivatives about a point at $x=a$
        %\\\indentx$\seqn{\ff(a),\, \frac{1}{1!}\ff'(a),\,  \frac{1}{2!}\ff''(a),\,  \frac{1}{3!}\ff'''(a),\cdots}$\\
        %of the Taylor polynomial at the point $x=a$ 
        allows you to determine (predict) arbitrarily closely all the points in the vicinity of $x=a$:\footnote{
        \citePp{robinson1982}{886}
        }
        \\\indentx$\ds\ff(x) = \ff(a) + \frac{1}{1!}\ff'  (a)\brs{x-a}
                                      + \frac{1}{2!}\ff'' (x)\brs{x-a}^2
                                      + \frac{1}{3!}\ff'''(x)\brs{x-a}^3
                                      + \cdots
                  $\\
        For linear or quasi-linear systems, $n=2$ may be sufficient. 
        For example, the classical ``vibrating string" vertical displacement $\fu(x,t)$ wave equation can be described as
        \\\indentx$\ppderiv{u}{x^2} - \frac{1}{c^2}\ppderiv{u}{t^2} = 0$\\
        Also note that all solutions of homogeneous second order differential equations
        are linear combinations of sine and cosine\ifsxref{harTrig}{thm:D2f_cos_sin}:
        \\\indentx$\ds  \brb{\opDiff\ff + \ff=0}
  %     {\fncte{2nd order homogeneous differential equation}}
  \quad\iff\quad
  \brb{\ff(x) = \ff(0)\,\cos(x) + \ff^\prime(0)\,\sin(x)}
  \qquad\msizes\forall\ff\in\spC,\,\forall x\in\R
$

  \item The \ope{Fourier Transform} is a kind of counter-part of the Taylor expansion demonstrated above:\footnote{
        \citePp{robinson1982}{886}
        }
        %\begin{tabular}{|@{2}{p{\tw/2-5mm}|}}
        \\\begin{tabular}{|c|l|l|}
            \hline
              & \mc{1}{|c|}{Taylor coefficients} & \mc{1}{c|}{Fourier coefficients}
            \\\hline
              \imark&Depend on derivatives $\ds\dndxn\ff(x)$        &Depend on integrals   $\ds\int_{x\in\R} \ff(x)e^{-i\omega x} \dx$
            \\\imark&Behavior in the vicinity of a point.           &Behavior over the entire function.
            \\\imark&Demonstrate trends locally.                    &Demonstrate trends globally. %, such as oscillations.
            \\\imark&Admits \prope{analytic} functions only---      &Admits \prope{non-analytic} functions as well---
            \\      &must be \prope{continuous}.                    &even if they are \prope{discontinuous}.
            \\\hline
        \end{tabular}
\end{enumerate}


