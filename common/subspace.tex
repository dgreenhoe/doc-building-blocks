%============================================================================
% Daniel J. Greenhoe
% LaTeX file
%============================================================================


%=======================================
\chapter{Linear Subspaces}
%=======================================

%======================================
\section{Subspaces of a linear space}
%======================================
\structe{Linear space}s \xref{def:vspace} can be decomposed into a collection of \structe{linear subspace}s \xref{def:subspace}.
%One of the most useful ways to analyze a linear space, is to decompose it into 
%a collection of linear subspaces \xrefP{def:subspace}.
Often such a collection along with an \structe{order relation}\ifsxref{order}{def:orel} forms a 
\structe{lattice}\ifsxref{lattice}{def:lattice}.
%\xrefP{thm:sub_lat}.

\begin{figure}[th]
  \centering%
  \includegraphics{../common/math/graphics/pdfs/latr3subspaces.pdf}%
  \caption{lattice of subspaces of $\R^3$ \xref{ex:r3subspaces}\label{fig:latr3subspaces}}%
\end{figure}
%--------------------------------------
%--------------------------------------
\exbox{\begin{tabular}{m{\tw-60mm}}
  \begin{example}\label{ex:r3subspaces}
    The 3-dimensional Euclidean space $\R^3$ contains 
    the 2-dimensional $xy$-plane and $xz$-plane subspaces,
    which in turn both contain the 1-dimensional $x$-axis subspace.
    These subspaces are illustrated in the figure to the right and in \prefpp{fig:latr3subspaces}.
  \end{example}\end{tabular}
  \tbox{\includegraphics{../common/math/graphics/pdfs/r3subspaces_xy.pdf}}%
  }

%--------------------------------------
\begin{definition}
\footnote{
  \citerpgc{michel1993}{81}{048667598X}{Definition 3.2.1},
  \citerpgc{berberian1961}{13}{0821819127}{Definition~I.5.1},
  \citerp{halmos1958}{16}
  }
\label{def:subspace}
\index{space!linear subspace}
%--------------------------------------
Let $\spO\eqd\linearspaceX$ be a \structe{linear space} \xref{def:vspace}.
\defboxt{%
  A tupple $\linearspaceY$ is a \structd{linear subspace} of $\spO$ if
  \\\indentx$\begin{array}{F>{\ds}lclDD}
      1. & \setY \ne  \emptyset                      &        &                       & ($\setY$ must contain at least one element)  & and
    \\2. & \setY \subseteq \setX                     &        &                       & ($\setY$ is a subset of $\setX$)             & and
    \\3. & \vx,\vy\in\setY                           &\implies& \vx \addv \vy\in\setY & (closed under vector addition)               & and
    \\4. & \vx\in\setY\text{ {\scs and} }\alpha\in\F &\implies& \alpha\vx \in\setY    & (closed under scalar-vector multiplication). & 
  \end{array}$%
  \\
  A linear subspace is also called a \hid{linear manifold}.%
  }
\end{definition}



Every \structe{linear space} \xref{def:vspace} $\spX$ has at least two 
\structe{linear subspace}s---itself and $\spZero$ \xref{prop:subspace_0X},
called the \structe{trivial linear space}.
The \structe{linear span} \xref{def:linspan} of every subset of a linear linear space is a subspace \xref{prop:subspace_span}.
Every \structe{linear subspace} contains the ``zero" vector $\vzero$, 
and is \prope{convex}\ifsxxref{convex}{def:convex_set}{prop:subspace_prop}.

%--------------------------------------
\begin{proposition}
\label{prop:subspace_0X}
\footnote{
  \citerppg{michel1993}{81}{83}{048667598X},
  \citerpg{haaser1991}{43}{0486665097}
  }
%--------------------------------------
Let $\spX\eqd\linearspaceX$ and $\spZero\eqd\linearspaceXYZ{\setn{\vzero}}$.
\propbox{
  \brb{\begin{array}{M}
    $\spX$ is a \structe{linear space}\\
    \xref{def:vspace}
  \end{array}}
  \quad\implies\quad
  \brb{\begin{array}{FcMD}
    1. & \spZero & is a \structe{linear subspace} of $\spX$ & and\\
    2. & \spX    & is a \structe{linear subspace} of $\spX$ & 
  \end{array}}
  }
\end{proposition}
\begin{proof}
For a structure to be a linear subspace of $\spX$, it must satisfy the
requirements of \prefpp{def:subspace}.
\begin{enumerate}
  \item Proof that $\{\vzero\}$ is a linear subspace:
    \begin{align*}
      \intertext{(a) Note that $\setn{\vzero}\ne \emptyset$.}
      %
      \intertext{(b) Proof that $\vx,\vy\in\setn{\vzero} \implies \vx+\vy\in\setn{\vzero}$:}
        \vx+\vy
          &=   \vzero + \vzero
          &&   \text{by $\vx,\vy\in\setn{\vzero}$ hypothesis}
        \\&=   \vzero
        \\&\in \setn{\vzero}
      %
      \intertext{(c) Proof that $\vx\in\setn{\vzero},\alpha\in \F \quad\implies\quad \alpha\vx\in\setn{\vzero}$:}
        \alpha\vx
          &=   \alpha\vzero
          &&   \text{by $\vx\in\setn{\vzero}$ hypothesis}
        \\&=   \vzero
          &&   \text{by definition of $\vzero$}
        \\&\in \setn{\vzero}
    \end{align*}

  \item Proof that $\spO$ is a linear subspace of itself:
    \begin{align*}
      \intertext{(a) Proof that $\setX\ne \emptyset$:}
        \setX &\ne  \emptyset
      %
      \intertext{(b) Proof that $\vx,\vy\in\setX \implies \vx+\vy\in\setX$:}
        \vx+\vy
          &\in \setn{\vzero}
          &&   \text{because $+:\setX\times\setX\to\setX$ ($\setX$ is closed under vector addition)}
      %
      \intertext{(c) Proof that $\vx\in\setX,\alpha\in \F \quad\implies\quad \alpha\vx\in\setX$:}
        \alpha\vx
          &\in \setX
          &&   \text{because $\cdot:\F\times\setX\to\setX$ ($\setX$ is closed under scalar-vector multiplication)}
    \end{align*}

\end{enumerate}
\end{proof}

\ifdochasnot{frames}{
%--------------------------------------
\begin{definition}
\footnote{
  \citerpgc{michel1993}{86}{048667598X}{3.3.7 Definition},
  \citerpg{kurdila2005}{44}{3764321989},
  \citerpgc{searcoid2002}{71}{185233424X}{Definition 3.2.5---more general definition}
  %\citerppgc{heil2011}{20}{21}{0817646868}{Definition 1.25}
  }
\label{def:span}
\label{def:linspan}
%--------------------------------------
Let $\linearspaceX$ be a linear space and $\setY$ be a subset of $\setX$.
\defbox{\begin{array}{Ml} 
  The \opd{linear span} of $\setY$ is defined as
  & \ds
  \hxs{\linspan}\setY
  \eqd 
  \setbigleft{\sum_{\gamma\in\Gamma} \alpha_\gamma \vy_\gamma}{\alpha_\gamma\in\F,\,\vy_\gamma\in\setY} .
  \\
  The set $\setY$ \hid{spans} a set $\setA$ if & \setA\subseteq\linspan\setY.
  %\\
  %{The \hid{closed span} $\hxs{\linspanc}\seqxZp{\vx_n}$ of $\seqxZp{\vx_n}$ in $\spO$ is the \hie{closure} of $\linspan\seqxZp{\vx_n}$ in $\spO$.}
  %\\
  %{The sequence $\seqxZp{\vx_n}$ is \hid{complete} in $\spO$ if $\linspanc\seqxZp{\vx_n}$ spans $\setX$.}
\end{array}}
\end{definition}
}

%--------------------------------------
\begin{proposition}
\footnote{
  \citorpg{michel1993}{86}{048667598X}
  }
\label{prop:subspace_span}
%--------------------------------------
Let $\spX\eqd\linearspaceX$ be a \structe{linear space} \xref{def:vspace}.
Let $\linspan$ be the \structe{linear span} of a set $\setY$ in $\spX$.
\propbox{
    \brb{\begin{array}{M}
      $\setY$ is a \structe{subset} of the set $\setX$\\
      ($\setY\subseteq\setX$)
      %\xref{def:subspace} of $\spX$
    \end{array}}
    \implies
    \brb{\begin{array}{cM}
      \linspan\setY & is a \structe{linear subspace} of $\spX$.
    \end{array}}
  %The linear subspace $\spV(\setY)$ is called the \hid{linear subspace generated} by the set $\setY$.
  }
\end{proposition}

%--------------------------------------
\begin{proposition}
\footnote{
  \citerpg{michel1993}{81}{048667598X}
  }
\label{prop:subspace_prop}
%--------------------------------------
Let $\spX\eqd\linearspaceX$ be a \structe{linear space} % \xref{def:vspace}.
and $\vzero$ the zero vector of $\spX$.
\propbox{
    \brb{\begin{array}{M}
      $\spY$ is a \structe{linear subspace} of $\spX$
      %\xref{def:subspace} of $\spX$
    \end{array}}
    \implies
    \brb{\begin{array}{FMD}
      1. & $\vzero\in\spY$ & and\\
      2. & $\spY$ is \prope{convex} in $\spX$ & 
    \end{array}}
  }
\end{proposition}
\begin{proof}
    \begin{align*}
      \text{$\spY$ is a \structe{subspace}}
        &\implies \exists (\alpha\vy)\in\spY\quad\forall\alpha\in\F
        && \text{by \prefp{def:subspace}}
      \\&\implies \exists \vzero\in\spY
        && \text{because $\alpha=0\in\F$}
      \\
      \\
          \text{$\spY$ is a linear subspace}
        &\implies \vx+\vy\in\setY \,\forall\vx,\vy\in\setY
      \\&\implies \lambda\vx+(1-\lambda)\vy\in\setY \,\forall\vx,\vy\in\setY
      \\&\implies \text{$\spY$ is \prope{convex}}
    \end{align*}
\end{proof}

%======================================
%\section{Operations on subspaces}
%======================================
%---------------------------------------
\begin{definition}
\footnote{
  \citorp{wedderburn1907}{79}
  %\citerp{rudinf}{6}
  }
\label{def:sub_add}
\index{space!Minkowski addition}
%---------------------------------------
Let $\spX\eqd\linearspaceX$ and $\spY\eqd\linearspaceY$ be \structe{linear subspace}s \xref{def:subspace} 
of a \structe{linear space} \xref{def:vspace} $\spO\eqd\linearspaceO$.
\defbox{\begin{array}{rcl@{\qquad}D}
  \spX \adds \spY &\eqd& \linearspace{\set{\vx+\vy}{\vx\in\setX\,\text{and}\,\vy\in\setY}}{+}{\cdot}{\F}{\addf}{\dottimes} & (\hib{Minkowski addition}\ifsxref{morph}{def:minkowski_add})\\
  \spX \spu  \spY &\eqd& \linearspace{\setX\setu\setY}{+}{\cdot}{\F}{\addf}{\dottimes} &(\hid{subspace union}) \\
  \spX \spi  \spY &\eqd& \linearspace{\setX\seti\setY}{+}{\cdot}{\F}{\addf}{\dottimes} & (\hid{subspace intersection})
  %\spX \sqsubseteq \spY &\implies& \setX\subseteq\setY
\end{array}}
\end{definition}



%---------------------------------------
\begin{example}
%---------------------------------------
Some examples of operations on subspaces in $\R^3$ are illustrated next:
%\\\includegraphics{../common/math/graphics/pdfs/minkowskiadd.tex}
\end{example}

%---------------------------------------
\begin{remark}\hspace{1pt}\\
%---------------------------------------
\begin{minipage}{\tw/2-1mm}
  Notice the similarities between the properties of linear subspaces
  in a linear space \xrefP{prop:lsub_add}
  and the properties of closed sets in a topological space\ifdochas{setstrct}{ \xrefP{thm:ts_closed}}:
\end{minipage}
\hfill
\begin{minipage}{\tw/2-1mm}
  \[\begin{array}{>{\ds}l | >{\ds}l}
       $\hie{linear subspaces}$         & $\hie{closed sets}$
    \\ \hline
       \vzero                     & \emptyset
    \\ \spO                       & \spO
    \\ \spX \adds \spY            & \setX \setu \setY
    \\ \setopi_{n=1}^\xN \spX_n   & \setopi_{\gamma\in\Gamma} \setX_\gamma
  \end{array}\]
\end{minipage}


\begin{minipage}{2\tw/3-5mm}%
  One key difference is that the union of two linear subspaces is not in general a
  linear subspace. For example, if $\vx$ is the vector $[1\,0]$ in
  the $x$ direction linear subspace of $\R^2$ and $\vy$ is the vector $[0\,1]$ in the
  $y$ direction linear subspace,
  then $\vx+\vy$ is not in the union of the two linear subspaces
  (it is not on the $x$ axis or $y$ axis but rather at $\opair{1}{1}$).\footnotemark
\end{minipage}%
\hfill
\begin{minipage}{\tw/3-5mm}%
  \color{figcolor}
  \begin{center}
  \begin{fsL}
  \setlength{\unitlength}{\tw/300}%
  \begin{picture}(270,270)(-130,-130)%
    %\graphpaper[10](0,0)(600,200)%
    {\color{axis}%
      \thicklines%
      \put(-120,   0){\line( 1, 0){240} }%
      \put(   0,-120){\line( 0, 1){240} }%
      \put( 130,   0){\makebox(0,0)[l]{$\spX$}}%
      \put(   0, 130){\makebox(0,0)[b]{$\spY$}}%
      \qbezier[20](100,0)(100,50)(100,100)%
      \qbezier[20](0,100)(50,100)(100,100)%
      }%
    {\color{uvect}%
      \thicklines%
      \put(   0,   0){\vector( 1, 0){100} }%
      \put(   0,   0){\vector( 0, 1){100} }%
      \put( 100, -10){\makebox(0,0)[t]{$\vx$}}%
      \put( -10, 100){\makebox(0,0)[r]{$\vy$}}%
      }%
    {\color{vector}%
      \thicklines%
      \put(   0,   0){\vector( 1, 1){100} }%
      \put( 10, 110){\makebox(0,0)[bl]{$\vx+\vy\notin\spX\spu \spY$}}%
      }%
  \end{picture}%
  \end{fsL}
  \end{center}
\end{minipage}%
\citetblt{\citerpg{michel1993}{82}{048667598X}}
\end{remark}



\begin{minipage}{3\tw/4}%
  In general, the set of all linear subspaces of a linear space $\spO$ is \emph{not} 
  closed under the subspace union ($\spu $) operation; that is, the union of 
  of two linear subspaces is \emph{not} necessarily a linear subspace.
  However the set \emph{is} closed under Minkowski sum ($\adds$) and subspace intersection ($\spi $).
  \pref{prop:lsub_add} (next) shows four useful objects are always subspaces.
  Some of these in Euclidean space $\R^3$ are illustrated to the right.
\end{minipage}%
\begin{minipage}{\tw/4}
  \color{figcolor}
  \begin{center}
  \begin{fsL}
  \setlength{\unitlength}{\tw/300}%
  \begin{picture}(250,250)(-100,-100)%
    %\graphpaper[10](-100,-100)(200,200)%
    {\color{axis}%
      \thicklines%
      \put(-100,   0){\line( 1, 0){200} }%
      \put(   0,-100){\line( 0, 1){200} }%
      \put(   0, 100){\line( 1, 1){ 50} }%
      \put(-100,   0){\line( 1, 1){ 50} }%
      \put( 100,   0){\line( 1, 1){ 50} }%
      \put(   0, 100){\line( 1, 1){ 50} }%
      \put(   0,-100){\line( 1, 1){ 50} }%
      \put(  50,  50){\line( 0, 1){100} }%
      \put(  50,  50){\line( 1, 0){100} }%
      \put( -50,  50){\line( 1, 0){ 50} }%
      \put(  50, -50){\line( 0, 1){ 50} }%
      \put( 130,   0){\makebox(0,0)[l]{$\spX$}}%
      \put(   0, 130){\makebox(0,0)[b]{$\spY$}}%
      }%
    {\color{vector}%
      \thicklines%
      \put(   0,   0){\line( 1, 1){ 50} }%
      \put( 100, -50){\vector( -1, 1){70} }%
      \put(  80, -60){\makebox(0,0)[tl]{$\spX\spi \spY$}}%
      }%
    {\color{figcolor}%
      \thicklines%
      \put(   0,   0){\line( 1, 1){ 50} }%
      \put( -50, -50){\vector( 1, 1){50} }%
      \put( -50, -60){\makebox(0,0)[tr]{$\vzero$}}%
      }%
  \end{picture}%
  \end{fsL}
  \end{center}
\end{minipage}%



%--------------------------------------
\begin{proposition}
\label{prop:lsub_add}
\footnote{
  \citerppg{michel1993}{81}{83}{048667598X}
  }
%--------------------------------------
Let $\spX$ be a \structe{linear space} \xref{def:vspace}.
\propbox{
  \brbr{\begin{array}{M}
    $\setxn{\spX_n}$ are\\
    \structe{linear subspace}s of $\spX$
  \end{array}}
  \implies
  \brbl{\begin{array}{FlM}
    1. & \spX_1\adds\spX_2\adds\cdots\adds\spX_\xN & is a \structe{linear subspace} of $\spX$ \\
       & \text{\scs and} \\
    2. & \spX_1\spi\spX_2\spi\cdots\spi\spX_\xN   & is a \structe{linear subspace} of $\spX$
  \end{array}}
  }
\end{proposition}
\begin{proof}
For a structure to be a linear subspace of $\spX$, it must satisfy the
requirements of \prefpp{def:subspace}.
\begin{enumerate}
  \item Proof that $\spX_1\adds\spX_2\adds\cdots\adds\spX_\xN$ is a \structe{linear subspace} (proof by induction):
    \begin{enumerate}
      \item proof for $\xN=1$ case: by left hypothesis.
      \item proof for $\xN=2$ case: \label{item:lsub_adds_n2}
        \begin{enumerate}
          \item proof that $\spX_1\adds\spX_2 \ne  \emptyset$:
            \begin{align*}
                \spX_1\adds\spX_2
                  &= \set{\vv+\vw}{\vv\in\spX_1 \text{ and } \vw\in\setY}
                  && \text{by \prefp{def:sub_add}}
                \\&\supseteq \set{\vv+\vw}{\vv\in\setn{\vzero}\subseteq\spX_1 \text{ and } \vw\in\setn{\vzero}\subseteq\spX_2}
                \\&= \setn{\vzero+\vzero}
                \\&= \setn{\vzero}
                \\&\ne  \emptyset
            \end{align*}
          %
          \item proof that $\vx,\vy\in\spX_1\adds\spX_2 \quad\implies\quad \vx+\vy\in\spX_1\adds\spX_2$:
            \begin{align*}
              \vx + \vy
                &= (\vv_1 + \vw_1) + (\vv_2 + \vw_2)
                && \text{by $\vx,\vy\in\spX_1\adds\spX_2$ hypothesis}
              \\&= \mcom{(\vv_1 + \vv_2)}{in $\spX_1$} +
                   \mcoml{(\vw_1 + \vw_2)}{in $\spX_2$ because $\spX_2$ is a linear subspace}
                && \ifdochas{vector}{\ifdochas{vector}{\text{by \prefp{def:vspace}}}}
              \\&\in \set{\vv+\vw}{\vv\in\spX_1 \text{ and } \vw\in\setY}
              \\&= \spX_1\adds\spX_2
                && \text{by \prefp{def:sub_add}}
            \end{align*}

          \item proof that $\vv\in\spX_1\adds\spX_2,\alpha\in F \quad\implies\quad \alpha\vv\in\spX_1\adds\spX_2$:
            \begin{align*}
              \alpha\vx
                &= \alpha(\vv_1 + \vw_1)
                && \text{by $\vx\in\spX_1\adds\spX_2$ hypothesis}
              \\&= \mcomr{\alpha\vv_1}{in $\spX_1$} +
                   \mcoml{\alpha\vw_1}{in $\spX_2$ because $\spX_2$ is a linear subspace}
                && \ifdochas{vector}{\text{by \prefp{def:vspace}}}
              \\&\in \set{\vv+\vw}{\vv\in\spX_1 \text{ and } \vw\in\setY}
              \\&= \spX_1\adds\spX_2
                && \text{by \prefp{def:sub_add}}
            \end{align*}
        \end{enumerate}

      \item Proof that [$\xN$ case] $\implies$ [$\xN+1$ case]:
        \begin{align*}
          \spX_1\adds\spX_2\adds\cdots\adds\spX_{\xN+1}
            &= \mcomr{\brp{\spX_1\adds\spX_2\adds\cdots\adds\spX_{\xN}}}{linear subspace by $\xN$ case hypothsis}\adds\spX_{\xN+1}
            && %\text{by \prefp{thm:set_ops}}
          \\&\implies \text{linear subspace by $\xN=2$ case \xref{item:lsub_adds_n2}}
        \end{align*}
    \end{enumerate}

  \item Proof that $\spX_1\spi\spX_2\spi\cdots\spi\spX_\xN$ is a \structe{linear subspace} (proof by induction):
    \begin{enumerate}
      \item proof for $\xN=1$ case: $\spX_1$ is a linear subspace by left hypothesis.

      \item Proof for $\xN=2$ case: \label{item:lsub_spi_n2}
        \begin{enumerate}
          \item proof that $\spX\spi \spY \ne  \emptyset$:
            \begin{align*}
                \spX\spi \spY
                  &= \set{\vx\in\setX}{\vx\in\setX \text{ and } \vw\in\setY}
                  && %\text{by \prefp{def:space_op}}
                \\&\supseteq \set{\vx\in\setX}{\vx\in\setn{\vzero}\subseteq\spX \text{ and } \vx\in\setn{\vzero}\subseteq\spY}
                \\&= \setn{\vzero+\vzero}
                \\&= \setn{\vzero}
                \\&\ne  \emptyset
            \end{align*}

          \item proof that $\vx,\vy\in\setX\spi \spY \quad\implies\quad \vx+\vy\in\setX\spi \spY$:
            \begin{align*}
              \vx,\vy\in \spX\spi \spY
                &\implies \vx,\vy\in \spX \text{ and } \vx,\vy\in\setY
                &&        \ifdochas{setstrct}{\text{by \prefp{def:ss_setops}}}
              \\&\implies \vx+\vy\in \spX \text{ and } \vx+\vy\in\setY
                &&        \text{because $\spX$ and $\spY$ are linear subspaces}
              \\&\implies \vx+\vy\in \spX\spi \spY
                &&        \ifdochas{setstrct}{\text{by \prefp{def:ss_setops}}}
            \end{align*}

          \item proof that $\vv\in\setX\spi \spY,\alpha\in F \quad\implies\quad \alpha\vv\in\setX\spi \spY$:
            \begin{align*}
              \vx\in \spX\spi \spY
                &\implies \vx\in \spX \text{ and } \vx\in\setY
                &&        \ifdochas{setstrct}{\text{by \prefp{def:ss_setops}}}
              \\&\implies \alpha\vx\in \spX \text{ and } \alpha\vx\in\setY
                &&        \text{because $\spX$ and $\spY$ are linear subspaces}
              \\&\implies \alpha\vx\in \spX\spi \spY
                &&        \ifdochas{setstrct}{\text{by \prefp{def:ss_setops}}}
            \end{align*}
        \end{enumerate}

      \item Proof that [$\xN$ case] $\implies$ [$\xN+1$ case]:
        \begin{align*}
          \spX_1\spi\spX_2\spi\cdots\spi\spX_{\xN+1}
            &= \mcomr{\brp{\spX_1\spi\spX_2\spi\cdots\spi\spX_{\xN}}}{linear subspace by $\xN$ case hypothsis}\spi\spX_{\xN+1}
          \\&\implies \text{linear subspace by $\xN=2$ case \xref{item:lsub_spi_n2}}
        \end{align*}
    \end{enumerate}
\end{enumerate}
\end{proof}

Every linear subspace contains the zero vector $\vzero$ \xref{prop:subspace_prop}.
But if a pair of linear subspaces of a linear space $\spX$ \emph{only} have $\vzero$ in commmon, then 
any vector in $\spX$ can be \emph{uniquely} represented by a single vector from each of the two subspaces (next).
%---------------------------------------
\begin{theorem}
\footnote{
  \citerpgc{michel1993}{83}{048667598X}{Theorem 3.2.12},
  \citerpgc{kubrusly2001}{67}{0817641742}{Theorem 2.14}
  }
\label{thm:XY0_unique}
%---------------------------------------
Let $\spX\eqd\linearspaceX$ and $\spY\eqd\linearspaceY$ be \structe{linear subspace}s \xref{def:subspace} 
of a \structe{linear space} \xref{def:vspace} $\spO\eqd\linearspaceO$.
\thmbox{
  \setX \seti \setY=\setn{\vzero} 
  \quad\iff\quad
  \brb{\begin{array}{>{\qquad}FMD}
    \mc{3}{M}{for every $\vu\in\spX\adds\spY$ there exist $\vx\in\setX$ and $\vy\in\setY$ such that}\\
      1. & $\vu=\vx+\vy$ & and \\
      2. & $\vx$ and $\vy$ are \prope{unique}.
  \end{array}}
  }
\end{theorem}
\begin{proof}
\begin{enumerate}
  \item Proof that $\setX\seti\setY=\setn{\vzero}$ $\implies$ \prope{unique} $\vx,\vy$:\\
    Suppose that $\vx$ and $\vy$ are not unique, but rather $\vu=\vx_1+\vy_1=\vx_2+\vy_2$ where
    $\vx_1,\vx_2\in\setX$ and $\vy_1,\vy_2\in\setY$.
    \begin{align*}
      \vu = \vx_1+\vy_1 = \vx_2+\vy_2
        &\implies \mcom{\vx_1-\vx_2}{$\in\setX$} = \mcom{\vy_2-\vy_1}{$\in\setY$}
        %&& \text{(suppose $\vx$ and $\vy$ are \prope{not unique})}
      \\&\implies {\vx_1-\vx_2},\,{\vy_2-\vy_1}\,\in\setX\seti \setY
      \\&\implies {\vx_1-\vx_2}={\vy_2-\vy_1}=\vzero
        && \text{by left hypothesis}
      \\&\implies {\vx_1=\vx_2} \quad\text{and}\quad {\vy_2=\vy_1}
      \\&\implies \text{$\vx$ and $\vy$ are \prope{unique}}
    \end{align*}

  \item Proof that $\setX\seti\setY=\setn{\vzero}$ $\impliedby$ \prope{unique} $\vx,\vy$:
    \begin{align*}
      \vu &= \vx + \vy
        \\&= \vx + \vy + \vy - \vy
          && \text{for some vector $\vy\in\setX\seti\setY$}
        \\&= \mcom{\brp{\vx + \vy}}{$\in\setX$} + \mcom{\brp{\vy - \vy}}{$\in\setY$}
          && \text{because $\vx\in\setX$ and $\vy\in\setX\seti\setY$ \ldots}
        \\&\implies \text{$\vx$ and $\vy$ are \prope{not unique} if $\vy\neq\vzero$}
        \\&\implies \vy=\vzero
          && \text{by right hypothesis}
        \\&\implies \setX\seti\setY=\setn{\vzero}
    \end{align*}
\end{enumerate}
\end{proof}

%---------------------------------------
\begin{theorem}
\label{thm:sub_lat_prop}
\footnote{%
  \citerppg{iturrioz1985}{56}{57}{0821850768}
  }%
%---------------------------------------
Let $\spO$ be a linear subspace and $\psetO$ the set of closed linear subspaces of $\spO$.
\thmboxt{
  $\latb{\psetO}{\subseteq}{\adds}{\spi }{\spZero}{\spO}$
  is a \structe{lattice}\ifsxref{lattice}{def:lattice}.
  In particular
  \\\indentx$\begin{array}{rcl @{\quad} rcl @{\quad}C}
      \spX \adds \spX &=& \spX
    & \spX \spi  \spX &=& \spX
    & \forall \spX\in\psetO
    %& (\prop{idempotent})
    \\
      \spX \adds \spY &=& \spY \adds \spX
    & \spX \spi  \spY &=& \spY \spi  \spX
    & \forall \spX,\spY\in\psetO
    %& (\prop{commutative})
    \\
      (\spX \adds \spY) \adds \spZ &=& \spX \adds (\spY \adds \spZ)
    & (\spX \spi  \spY) \spi  \spZ &=& \spX \spi  (\spY \spi  \spZ)
    & \forall \spX,\spY,\spZ\in\psetO
    %& (\prop{associative})
    \\
      \spX \adds (\spX \spi  \spY) &=& \spX
    & \spX \spi  (\spX \adds \spY) &=& \spX
    & \forall \spX,\spY\in\psetO
    %& (\prop{absorptive}/\prop{contractive})
\end{array}$}
\end{theorem}
\begin{proof}
These results follow directly from 
the properties of lattices\ifsxref{lattice}{thm:lattice}.
\end{proof}



%======================================
\section{Subspaces of an inner product space}
%======================================
%---------------------------------------
\begin{definition}
\label{def:sub_ocomp}
\label{def:set_orthog}
\footnote{
  \citerpgc{berberian1961}{59}{0821819127}{Definition~III.2.1},
  \citerpg{michel1993}{382}{048667598X},
  \citerpg{kubrusly2001}{328}{0817641742}
  %\citerpgc{heil2011}{30}{0817646868}{Definition 1.41}
  }
\index{space!orthogonal}
%---------------------------------------
Let $\spO\eqd\inprodspaceX$ be an \structe{inner product space} \xref{def:inprod}.
\defbox{\begin{array}{M}
  The \hid{orthogonal complement} $\setAo$ in $\spO$ of a set $\setA\subseteq\setX$ is
  \\\qquad
  $\ds\setAo \eqd \set{\vx\in\setX}{\inprod{\vx}{\vy}=0 \quad \forall \vy\in\setA}$.
  \\
  The expression $\setAoo$ is defined as $\brp{\setAo}^\orthog$.
    %$\setAooo$ as $\brp{\setAoo}^\orthog$, etcetera.
  \end{array}}
\end{definition}


%---------------------------------------
\begin{proposition}
\footnote{
  %\citerpg{michel1993}{383}{048667598X}\\
  \citerpgc{berberian1961}{60}{0821819127}{Theorem~III.2.2},
  \citerpg{kubrusly2011}{326}{0817649972}
  }
\label{prop:inprod_orthog_AB}
\label{prop:ABBoAo}
%---------------------------------------
Let $\inprodspaceX$ be an \structe{inner product space} \xref{def:inprod}.
%Let $\psetx$ be the power set of $\setA$.
\propbox{\begin{array}{rclCD}
      \setA \subseteq \setB &\implies& \setBo\subseteq\setAo &\forall\setA,\setB\in\psetX  & (\prope{antitone})
\end{array}}
\end{proposition}
\begin{proof}
\begin{align*}
  \setBo
    &\eqd \set{\vx\in\setX}{\inprod{\vx}{\vy}=0 \quad\forall \vy\in\setB}
    && \text{by definition of $\setBo$ \xrefP{def:sub_ocomp}}
  \\&\subseteq \set{\vx\in\setX}{\inprod{\vx}{\vy}=0 \quad\forall \vy\in\setA}
    && \text{by $\setA\subseteq\setB$ hypothesis}
  \\&= \setAo
    && \text{by definition of $\setAo$ \xref{def:sub_ocomp}}
\end{align*}
\end{proof}

Every \structe{linear space} $\spX$ 
contains $\spZero$ and $\spX$ as \structe{linear subspace}s \xref{prop:subspace_0X}.
If $\spX$ is also an \structe{inner product space}, then $\spZero$ and $\spX$ are \structe{orthogonal complement}s
of each other (next proposition).
%---------------------------------------
\begin{proposition}
\footnote{
  \citerpg{kubrusly2011}{326}{0817649972},
  \citerpg{michel1993}{383}{048667598X}
  }
\label{prop:inprod_orthog}
%---------------------------------------
Let $\spO\eqd\inprodspaceX$ be an \structe{inner product space} \xref{def:inprod}
and $\vzero$ the \structe{vector additive identity element} \xref{def:vspace} in $\spO$.
\propbox{
  %\brb{\begin{array}{M}
  %  $\spO$ is an \structe{inner product space}\\
  %  \xref{def:vspace}
  %\end{array}}
  %\qquad\implies\qquad
  {\begin{array}{Frcl}
    1. & \setn{\vzero}^\orthog &=& \setX                 \\
    2. & \setXo                &=& \setn{\vzero}   
  \end{array}}
  }
\end{proposition}
\begin{proof}
    \begin{align*}
      \setn{\vzero}^\orthog 
        &= \set{\vx\in\setX}{\inprod{\vx}{\vy}=0 \quad \forall \vy\in\setn{\vzero}}
        && \text{by definition of $\orthog$ \xrefP{def:sub_ocomp}}
      \\&= \set{\vx\in\setX}{\inprod{\vx}{\vzero}=0 }
      \\&= \setX
      \\
      \setXo
        &= \set{\vx\in\setX}{\inprod{\vx}{\vy}=0 \quad \forall \vy\in\setX}
        && \text{by definition of $\orthog$ \prefp{def:sub_ocomp}}
      \\&= \set{\vx\in\setX}{\inprod{\vx}{\vx}=0}
      \\&= \setn{\vzero}
    \end{align*}
\end{proof}


For any set $\setA$ contained in a linear space $\spX$, 
$\setAoo$ is a \structe{linear subspace}, and it 
is the smallest linear subspace containing the set $\setA$ ($\setAoo=\linspan\setA$, next theorem).
In the case that $\setA$ is a \structe{linear subspace} rather than just a subset,
results simplify significantly (next corollary).
%---------------------------------------
\begin{theorem}
\footnote{
  \citerpg{michel1993}{383}{048667598X},
  %\citerppg{kubrusly2001}{328}{329}{0817649972}
  \citerpg{kubrusly2011}{326}{0817649980}
  }
\label{thm:inprod_orthog}
%---------------------------------------
Let  $\spO\eqd\inprodspaceX$ be an \structe{inner product space} \xref{def:inprod}.
%Let $\spY\eqd\inprodspaceY$ be a subspace of an \structe{inner product space} $\spO\eqd\inprodspaceX$.
Let $\linspan\setA$ be the span of a set $\setA$ \xref{def:span}.
\thmbox{
    \brb{\begin{array}{M}
      $\setA$ is a \structb{subset} of $\setX$\\
      ($\setA\subseteq\setX$)
    \end{array}}
    \quad\implies\quad
    \brb{\begin{array}{FrclD}
      1. & \setA\seti \setAo      &=& 
           \brbl{\begin{array}{lMl}
             \setn{\vzero}  & if & \vzero\in\setA\\
             \emptyset      & if & \vzero\notin\setA
           \end{array}}
         & and
         \\
      2. & \setA &\subseteq& \setAoo = \linspan\setA    & and \\ %($\setAoo$ is the smallest linear subspace containing $\setA$)\\
      3. & \setAo &=& \setAooo = \clsAo = \oclsA = \brp{\linspan\setA}^\orthog    & and \\
      4. & \mc{3}{M}{$\setAo$ is a \structb{subspace} of $\spO$}
    \end{array}}
  }
\end{theorem}
\begin{proof}
\begin{enumerate}
  \item Proof that $\setA\seti \setAo = \cdots$:
    \begin{align*}
      \setA\seti\setAo      
        &= \set{\vx\in\setX}{\vx\in\setA} \spi  \set{\vx\in\setX}{\inprod{\vx}{\vy} \quad \forall \vy\in\setA}
        && \text{by definition of $\setAo$}
      \\&= \set{\vx\in\setX}{\vx\in\setA\quad\text{and}\quad \inprod{\vx}{\vy} \quad \forall \vy\in\setA}
      \\&= \brbl{\begin{array}{lMl}
             \setn{\vzero} & if & \vzero\in\setA\\
             \emptyset     & if & \vzero\notin\setA
           \end{array}}
    \end{align*}

  \item Proof that $\setA \subseteq \setAoo=\linspan\setA$: \label{item:inprod_orthog_oo}
    \begin{align*}
      \vx\in\setA 
        &\implies \setn{\vx}^{\orthog\orthog} \subseteq \setAoo
      \\&\implies \vx\in\setn{\vx}^{\orthog\orthog} \subseteq \setAoo
      \\&\implies \vx\in \setAoo
     \end{align*}
     but
     \begin{align*}
      \vx\in\setAoo
        &\notimplies \vx\in\setA
    \end{align*}
    Here is an example for the $\notimplies$ part using the linear space $\R^3$:
    \begin{enumerate}
      \item Let $\setA\eqd\setn{i}$, where $i$ is the unit vector on the x-axis.\\
      \item Then $\setAo = \set{\vx\in\setX}{\text{$\vx\in$ yz plane}}$.
      \item Then $\setAoo = \set{\vx\in\setX}{\text{$\vx\in$ x axis}}$.
      \item Therefore, $\setA\subsetneq\setAoo$
    \end{enumerate}

  \item Proof for $\setAo$ equivalent expressions:
    \begin{enumerate}
      \item Proof that $\setAo = \setAooo$:
        \begin{align*}
          \setAo &\subseteq \brp{\setAo}^{\orthog\orthog}
                 && \text{by \pref{item:inprod_orthog_oo}}
               \\&= \brp{\setAoo}^\orthog
               \\&= \setAooo
                 && \text{by \prefp{def:sub_ocomp}}
          \\
          \setAooo
            &= \brp{\setAoo}^\orthog
            && \text{by \prefp{def:sub_ocomp}}
          \\&\subseteq \setAo
            && \text{by \pref{item:inprod_orthog_oo} and \prefpp{prop:inprod_orthog_AB}}
        \end{align*}

      \item Proof that $\setAooo=\brp{\linspan\setA}^\orthog$: follows directly from \pref{item:inprod_orthog_oo} ($\setAoo=\linspan\setA$).

      \item Proof that $\setAo=\clsAo$: 
        \begin{enumerate}
          \item Let $\seqn{\vx_n}$ be an $\setAo$-valued sequence that converges to the limit $\vx$ in $\setX$.
          \item The limit point $\vx$ must be in $\setAo$ because for all $\vy\in\setA$
            \begin{align*}
              \inprod{\vx}{\vy} 
                &= \inprod{\lim\vx_n}{\vy}
                && \text{by definition of the sequence $\seqn{\vx_n}$}
              \\&= \lim\inprod{\vx_n}{\vy}
              \\&= 0
                && \text{because $\seqn{\vx_n}$ is $\setAo$-valued}
            \end{align*}
          \item Because $\inprod{\vx}{\vy}=0\quad\forall\vy\in\setA$, $\vx$ is in $\setAo$.
          \item Because $\setAo$ contains all its limit points, and by the \thme{Closed Set Theorem} \xref{thm:cst},
                it must be \prope{closed} ($\setAo=\clsAo$)
        \end{enumerate}

      \item Proof that $\setAo=\oclsA$: 
        \begin{enumerate}
          \item Let  $x\in\setAo$ and $y\in\clsA$.
          \item Let $\seqn{\vy_n}$ be an $\setAo$-valued sequence that converges in $\setX$ to $y$.
          \item Thus $\setAo\orthog\clsA$ because
            \begin{align*}
              \inprod{\vy}{\vx}
                &= \inprod{\lim\vy_n}{\vx}
                && \text{by definition of $\seqn{\vy_n}$}
              \\&= \lim\inprod{\vy_n}{\vx}
              \\&= 0
                && \text{because $\seqn{\vy_n}$ is $\setAo$-valued}
            \end{align*}
          \item Because $\setAo\orthog\clsA$, so $\setAo\subseteq\clsAo$.
          \item But $\clsAo\subseteq\setAo$ because
            \begin{align*}
              \setA
                &\subseteq \clsA \implies \clsAo \subseteq \setAo
                && \text{by \prope{antitone} property \xref{prop:ABBoAo}}
            \end{align*}
          \item And so $\setAo=\clsAo$.
        \end{enumerate}
    \end{enumerate}

  %\item Proof that $\setAoo =         \linspan\setA$:
  \item Proof that $\setAo$ is a \structb{subspace} of $\spO$ (must satisfy the conditions of \prefp{def:subspace}):
    \begin{enumerate}
      \item Proof that $\setAo\ne\emptyset$: $\setAo$ has at least one element, the element $\vzero$\ldots
        \begin{align*}
          \inprod{\vzero}{\vy} 
            &=0 \quad \forall \vy\in\setA       && \text{by definition of $\vzero$}
          \\&\implies \vzero\in\setAo           && \text{by definition of $\setAo$ \xref{def:set_orthog}}
        \end{align*}

      \item Proof that $\setAo\subseteq\setX$: 
        \begin{align*}
          \vu\in\setAo
            &\implies \vu\in\set{\vx\in\setX}{\inprod{\vx}{\vy}=0 \quad \forall \vy\in\setA}
            &&\text{by definition of $\setAo$ \xref{def:set_orthog}}
          \\&\implies \vu\in\setX
            &&\text{by definition of sets}
        \end{align*}

      \item Proof that $\vu,\vv\in\setAo\implies(\vu+\vv)\in\setAo$:
        \begin{align*}
          \vu,\vv\in\setAo
            &\implies \inprod{\vu}{\vy}=\inprod{\vv}{\vy}=0 && \forall \vy\in\setA  && \text{by definition of $\setAo$ \xref{def:set_orthog}}
          \\&\implies \inprod{\vu}{\vy}+\inprod{\vv}{\vy}=0 && \forall \vy\in\setA 
          \\&\implies \inprod{\vu+\vv}{\vy}=0               && \forall \vy\in\setA  && \text{by \prope{additive} property of $\inprodn$ \xref{def:inprod}}
          \\&\implies \vu+\vv\in\setAo                      &&                      && \text{by definition of $\setAo$ \xref{def:set_orthog}}
        \end{align*}

      \item Proof that $\vv\in\setO\implies\alpha\vv\in\setAo$:
        \begin{align*}
          \vv\in\setAo
            &\implies \inprod{\vv}{\vy}=0                  && \forall \vy\in\setA  && \text{by definition of $\setAo$ \xref{def:set_orthog}}
          \\&\implies \alpha\inprod{\vv}{\vy}=\alpha\cdot0 && \forall \vy\in\setA
          \\&\implies \inprod{\alpha\vv}{\vy}=0            && \forall \vy\in\setA  && \text{by \prope{homogeneous} property of $\inprodn$ \xref{def:inprod}}
          \\&\implies \alpha\vv\in\setAo                   &&                      && \text{by definition of $\setAo$ \xref{def:set_orthog}}
        \end{align*}
    \end{enumerate}
\end{enumerate}
\end{proof}

%---------------------------------------
\begin{corollary}
%\footnote{
%  \citerpg{michel1993}{383}{048667598X}\\
%  %\citerppg{kubrusly2001}{328}{329}{0817649972}
%  \citerpg{kubrusly2011}{326}{0817649972}
%  }
\label{cor:inprod_orthog}
%---------------------------------------
Let  $\spX\eqd\inprodspaceX$ and $\spY\eqd\inprodspaceY$ be \structe{inner product space}s.
%Let $\spY\eqd\inprodspaceY$ be a subspace of an \structe{inner product space} $\spO\eqd\inprodspaceX$.
Let $\linspan\setY$ be the span of the set $\setY$ \xref{def:span}.
\corbox{
    \brb{\begin{array}{M}
      $\spY$ is a \structb{linear subspace} of $\spX$
    \end{array}}
    \qquad\implies\qquad
    \brb{\begin{array}{FrclD}
      1. & \setY\seti \setYo      &=& \setn{\vzero}              & and \\
      2. & \setY                  &=& \setYoo = \linspan\setY    & and \\ 
      3. & \setYo                 &=& \setYooo                   & and \\
      4. & \mc{3}{M}{$\setYo$ is a \structb{subspace} of $\spX$} & 
    \end{array}}
  }
\end{corollary}
\begin{proof}
\begin{enumerate}
  \item Proof that $\spY\seti \setYo      = \setn{\vzero}               $: This follows from \prefpp{thm:inprod_orthog} and the fact that all subspaces contain the zero vector $\vzero$ \xref{prop:subspace_prop}.
  \item Proof that $\spY                  = \spY^{\orthog\orthog} = \linspan\spY$: This follows directly from \prefpp{thm:inprod_orthog}.
  \item Proof that $\spYo                 = \spY^{\orthog\orthog\orthog}$: This follows directly from \prefpp{thm:inprod_orthog}.
  \item Proof that $\setYo$ is a \structb{subspace} of $\spX$: This follows directly from \prefpp{thm:inprod_orthog}.
\end{enumerate}
\end{proof}




%\pref{def:space_relation} (next) defines relations between linear spaces.
%Note, however, that these relations are not operations on spaces
%(as in \pref{def:sub_add}),
%but are rather statements about the relationship between spaces.
%%---------------------------------------
%\begin{definition}
%\citep{pedersen2000}{50}
%\label{def:space_relation}
%\label{def:direct_sum}
%\label{def:ortho_spaces}
%\index{space!complement}
%\index{space!direct sum}
%\index{space!orthogonal}
%%---------------------------------------
%Let $\spX$ be a linear space, $\spV,\spW\subseteq\spX$,
%and $\vzero$ be the additive identity element in $\spX$.
%\begin{enumerate}
%\item ``$\spV$ is \hid{orthogonal} to $\spW$ in $\spX$"
%is denoted by $\spV\perp\spW$ and is defined as
%\defbox{
%  \mcom{\spV \perp \spW}{$\spV$ is \hid{orthogonal} to $\spW$ in $\spX$}
%  \quad \implies \quad
%  \inprod{\vx}{\vy}=0
%  \quad
%  \forall \vx\in\setV,\; \vy\in\setW
%  }
%
%\item ``$\spW$ is the \hid{complement} of $\spV$ in $\spX$"
%is denoted by $\spX=\spV\adds\spW$ and has the meaning
%\defbox{
%  \mcom{\spX=\spV\adds\spW}{$\spW$ is the \hid{complement} of $\spV$ in $\spX$}
%  \implies
%  \left\{\begin{array}{ll}
%    1. & \spX = \spV + \spW \\
%    2. & \spV \spi  \spW = \setn{\vzero}
%  \end{array}\right.
%  }
%
%\item Equivalent ways of expressing $\spX=\spV\adds\spW$ include
%\begin{enumerate}
%  \item $\spX$ is the \hid{direct sum} of $\spV$ and $\spW$.
%  \item $\spX = \spV \adds \spW$ (and ``$\adds$" is the \hid{direct sum operator})
%  \item $\spW$ is the \hid{complement} of $\spV$ in $\spX$.
%  \item $\spW$ \hid{complements} $\spV$ in $\spX$.
%\end{enumerate}
%\end{enumerate}
%\end{definition}

%---------------------------------------
\begin{theorem}
\footnote{
  \citerpg{kubrusly2001}{324}{0817649972}
  }
\label{thm:YoZ==>YZ0}
%---------------------------------------
Let $\spY\eqd\inprodspaceY$ and $\spZ\eqd\inprodspaceZ$ be \structe{linear subspace}s 
of an \structe{inner product space} $\spO\eqd\inprodspaceX$.
\thmbox{
  \spY \orthog \spZ 
  \qquad\implies\qquad
  \setY\seti\setZ=\setn{\vzero}
  }
\end{theorem}
\begin{proof}
\begin{align*}
  \vx\in\spY\seti\spZ
    & \implies \vx\in\spY \text{ and } \vx\in\spZ
    && \text{by definition of $\seti$}
  \\& \implies \inprod{\vx}{\vx}=0
    && \text{by hypothesis $\spY\orthog\spZ$}
  \\& \implies \vx=\vzero
    && \text{by \prope{non-isotropic}property of $\inprodn$ \xrefP{def:inprod}}
\end{align*}
\end{proof}

%---------------------------------------
\begin{theorem}
\footnote{
  \citerpgc{berberian1961}{61}{0821819127}{Theorem~III.2.3} 
  }
%---------------------------------------
Let $\spY\eqd\inprodspaceY$ and $\spZ\eqd\inprodspaceZ$ be linear subspaces 
of an \structe{inner product space} $\spO\eqd\inprodspaceX$.
\thmbox{
  \brb{\begin{array}{>{\scy}rlD}
    1. & \spY \orthog \spZ \text{ \scs and}\\
    2. & \vx \in \spY \adds \spZ
  \end{array}}
  \implies
  \brb{\begin{array}{>{\scy}rM}
    1. & There exists $\vy\in\spY$ and $\vz\in\spZ$ such that $\vx=\vy+\vz$ \scs and\\
    2. & $\vy$ and $\vz$ are \prope{unique}.
  \end{array}}
  }
\end{theorem}
\begin{proof}
\begin{enumerate}
  \item Proof that $\vy$ and $\vz$ exist: by definition of Minkowski addition operator $\adds$ \xrefP{def:sub_add}.
  \item Proof that $\vy$ and $\vz$ are \prope{unique}:
    \begin{enumerate}
      \item Suppose $\vx=\vy_1+\vz_1=\vy_1+ \vz_2$ for $\vy_1,\vy_2\in\spY$ and $\vz_1,\vz_2\in\spZ$.
      \item This implies
        \begin{align*}
          \vzero
            &= \vx-\vx
          \\&= \brp{\vy_1+\vz_1} - \brp{\vy_1+ \vz_2}
          \\&= \mcom{\brp{\vy_1-\vy_2}}{in $\spY$} + \mcom{\brp{\vz_1- \vz_2}}{in $\spZ$}
        \end{align*}
      \item Because $\vy_1-\vy_2\in\spY$, $\vz_1-\vz_2\in\spZ$, 
                    $(\vy_1-\vy_2)+(\vz_1+\vz_2)=\vzero$, and 
                    $\inprod{\vy_1-\vy_2}{\vz_1-\vz_2}=0$,
            then by \prefpp{thm:vsinprod_zero}, $\vy_1-\vy_2=\vzero$ and $\vz_1-\vz_2=\vzero$.
      \item This implies $\vy_1=\vy_2$ and $\vz_1=\vz_2$.
      \item This implies $\vy$ and $\vz$ are \prope{unique}.
    \end{enumerate}
\end{enumerate}
\end{proof}


%=======================================
\section{Subspaces of a Hilbert Space}
%=======================================
%---------------------------------------
\begin{theorem}
\footnote{
  \citerpgc{kubrusly2001}{330}{0817649972}{Theorem 5.13},
  \citerpgc{ab}{290}{0120502577}{Theorem 33.6},
  \citerpgc{berberian1961}{68}{0821819127}{Theorem~III.5.1}
  }
\label{thm:sub_convex_min}
%---------------------------------------
Let $\spH\eqd\HspaceX$ be a \structe{Hilbert space} \xref{def:hspace}.\\
Let $\setY$ be a \structe{subset} of $\setX$,
and let $\ds\metric{\vx}{\setY}\eqd\inf_{\vy\in\setY}\norm{\vx-\vy}$.
\thmbox{
  \brb{\begin{array}{FMDD}
      1. & $\setY\neq\emptyset$      &                                 & and
    \\2. & $\setY$ is \prope{closed} & \ifxref{topology}{def:clsA}     & and 
    \\3. & $\setY$ is \prope{convex} & \ifxref{convex}{def:convex_set} & 
  \end{array}}
  \quad\implies\quad
  \brb{\begin{array}{>{\qquad}FMD}
    \mc{3}{M}{There exists $\vp\in\setY$ such that} \\
        1. & $\metric{\vx}{\setY} = \norm{\vx-\vp}$ & and
      \\2. & $\vp$ is \prope{unique}.
  \end{array}}
  }
\end{theorem}
\begin{proof}
\begin{enumerate}
  \item Let $\delta\eqd\inf\set{\vx-\vy}{\vy\in\setY}$.\label{item:sub_convex_min_delta}
  \item Let $\seqxZ{\vy_n}$ be a sequence such that $\norm{\vx-\vy_n}\rightarrow\delta$.
  \item Proof that $\seqn{\vy_n}$ is \prope{Cauchy}: \label{item:sub_convex_min_cauchy}
    \begin{align*}
      &\lim_{m,n\rightarrow\infty}\norm{\vy_n-\vy_m}^2
      \\&= \lim_{m,n\rightarrow\infty}\norm{(\vy_n-x)+(x-\vy_m)}^2
      \\&= \lim_{m,n\rightarrow\infty}\brb{-\norm{(\vy_n-x)-(x-\vy_m)}^2 + 2\norm{\vy_n-x}^2 + 2\norm{x-\vy_m}^2}
        && \text{by \thme{parallelogram law} (\prefpo{thm:parallelogram})}
      \\&= \lim_{m,n\rightarrow\infty}\brb{-4\norm{\mcom{\brp{\frac{1}{2}\vy_n+\frac{1}{2}\vy_m}}{in $\spY$ by \prope{convexity}}-x}^2 + 2\norm{\vy_n-x}^2 + 2\norm{x-\vy_m}^2}
      \\&\le \lim_{m,n\rightarrow\infty}\brb{-4\delta^2 + 2\norm{\vy_n-x}^2 + 2\norm{x-\vy_m}^2}
        && \text{by definition of $\delta$ (\pref{item:sub_convex_min_delta})}
      \\&= -4\delta^2 + \lim_{m,n\rightarrow\infty}\brb{2\norm{\vy_n-x}^2} + \lim_{m,n\rightarrow\infty}\brb{2\norm{x-\vy_m}^2}
      \\&= -4\delta^2 + 2\delta^2 + 2\delta^2
        && \text{by definition of $\delta$ (\pref{item:sub_convex_min_delta})}
      \\&= 0
    \end{align*}

  \item Proof that $\metric{\vx}{\setY} = \norm{\vx-\vy}$: 
        because $\seqn{\vy_n}$ is \prope{Cauchy} (\pref{item:sub_convex_min_delta}) and by the \prope{closed} hypothesis.

  \item Proof that $\vy$ is \prope{unique}: 
        Because in a metric space, the limit of a convergent sequence is \prope{unique}\ifdochas{metric}{ by \prefp{thm:xn_to_xy}}.
\end{enumerate}
\end{proof}

%---------------------------------------
\begin{theorem}
\footnote{
  \citerpgc{kubrusly2001}{330}{0817649972}{Theorem 5.13}
  %\citerpgc{ab}{290}{0120502577}{Theorem 33.6}
  }
%\label{thm:inprod_orthog}
%---------------------------------------
Let $\spH\eqd\HspaceX$ be a \structe{Hilbert space} \xref{def:hspace}.
Let $\ds\metric{\vx}{\setY}\eqd\inf_{\vy\in\setY}\norm{\vx-\vy}$.
Let $\spY\eqd\HspaceY$  % be a subspace of $\spH$
and $\setYo$ the \structe{orthogonal complement} of $\setY$.
\thmbox{
  \brb{\begin{array}{M}
    $\spY$ is a \structe{subspace} of $\spH$
  \end{array}}
  \qquad\implies\qquad
  \brb{\begin{array}{>{\qquad}FMD}
    \mc{3}{M}{There exists $\vp\in\setY$ such that} \\
        1. & $\metric{\vx}{\setY} = \norm{\vx-\vp}$ & and
      \\2. & $\vp$ is \prope{unique}                & and
      \\3. & $\vx-\vp\in\setYo$.
  \end{array}}
  }
\end{theorem}
%\begin{proof}
%\begin{enumerate}
%
%\end{enumerate}
%\end{proof}

%---------------------------------------
\begin{theorem}[\thmd{Projection Theorem}]
\footnote{
  \citerpgc{bachman1966}{172}{0486402517}{Theorem 10.8},
  \citerpgc{kubrusly2001}{339}{0817649972}{Theorem 5.20}
  }
\label{thm:projection}
%---------------------------------------
Let $\spH\eqd\HspaceX$ be a Hilbert space.
\thmbox{
  \brb{\begin{array}{M}
    $\spY$ is a \structe{subspace} of $\spH$
  \end{array}}
  \qquad\implies\qquad
  \brb{\begin{array}{l}
  \spY \adds \spYo = \spH
  \end{array}}
  }
\end{theorem}
\begin{proof}
\begin{align*}
  \spY \adds \spYo
    &= \brs{\spY \adds \spYo}^{\orthog\orthog}
    && \text{by \prefp{cor:inprod_orthog}}
  \\&= \brs{\spYo \spi \spYoo}^{\orthog}
    && \text{by \prefpp{prop:ABBoAo} \ifdochas{ortholat}{and \prefpp{lem:latoc_demorgan}}}
  \\&= {\setn{\vzero}}^{\orthog}
    && \text{by \prefp{cor:inprod_orthog}}
  \\&= \spH
    && \text{by \prefp{prop:inprod_orthog}}
\end{align*}
\end{proof}

%=======================================
%\section{Subspace order structure}
%=======================================
%======================================
%\subsection{Ordered sets of subspaces}
%======================================
The inclusion relation $\subseteq$ is an order relation on the set of subspaces of 
a linear space $\spO$.

%--------------------------------------
\begin{proposition}
%--------------------------------------
Let $\ssetS$ be the set of subspaces of a linear space $\spO$.
Let $\subseteq$ be the inclusion relation.
\propboxt{$\opair{\ssetS}{\subseteq}$ is an \structb{ordered set}}
\end{proposition}
\begin{proof}
$\opair{\ssetS}{\subseteq}$ is an \structe{ordered set}\ifsxref{order}{def:order_rel} and because
  \\$\begin{array}{@{\qquad}ll@{\qquad}C@{\qquad}DD@{}r@{}D}
    \cline{6-6}
    1. & \spX \subseteq \spX
       & \forall \spX\in\ssetS
       & (\prop{reflexive})
       & and \hspace{2ex}
       & \text{ }\vline
       & \text{\hspace{2ex}preorder}
       \\
    2. & \spX \subseteq \spY \text{ and } \spY \subseteq \spZ \implies \spX \subseteq \spZ
       & \forall \spX,\spY,\spZ\in\ssetS
       & (\prop{transitive})
       & and
       & \vline
    \\\cline{6-6}
    3. & \spX \subseteq \spY \text{ and } \spY \subseteq \spX \implies \spX=\spY
       & \forall \spX,\spY\in\ssetS
       & (\prop{anti-symmetric})
       & 
       & 
  \end{array}$
\end{proof}

%======================================
%\subsection{Lattices of subspaces}
%======================================

%%---------------------------------------
%\begin{theorem}
%\label{thm:sub_lat}
%%---------------------------------------
%Let $\spO$ be a linear space and $\ssetS$ a set of linear subspaces of $\spO$.\footnote{
%  \begin{tabular}{lcl}
%    \structe{subspace addition operator}     & $\adds$ & \prefp{def:sub_add}\\
%    \structe{subspace intersection operator} & $\spi$  & \prefp{def:sub_add}
%  \end{tabular}}
%\thmbox{
%    \brb{\begin{array}{>{\scriptstyle}rllCD}
%    1. & \spZero       &\in \ssetS &                           & and \\ 
%    2. & \spO          &\in \ssetS &                           & and \\
%    3. & \spX\adds\spY &\in \ssetS & \forall\spX,\spY\in\ssetS & and \\
%    4. & \spX\spi \spY &\in \ssetS & \forall\spX,\spY\in\ssetS 
%    \end{array}}
%  \qquad\implies\qquad
%  \brb{\text{$\lattice{\ssetS}{\subseteq}{\adds}{\spi }$ is a \hib{lattice}.}}
%  }
%\end{theorem}
%
%In this document, we call the lattice described in \pref{thm:sub_lat} (previous theorem) a \hid{subspace lattice}.


%---------------------------------------
\begin{theorem}
\label{thm:sub_latoc}
\footnote{%
  \citerppg{iturrioz1985}{56}{57}{0821850768}
  }%
%---------------------------------------
Let $\spH$ be a Hilbert space and $\pset{\spH}$ the set of closed linear subspaces of $\spH$.
\thmboxt{
  $\latoc{\psetH}{\subseteq}{\adds}{\spi }{\orthog}{\spZero}{\spH}$
  is an \structe{orthomodular lattice}\ifsxref{ortholat}{def:latoc_omod}.
  In particular
  \\\indentx$\begin{array}{Frcl@{\quad}C @{\qquad}D}
  %    \cnto & \spX \adds \spX &=& \spX
  %  & \cntx & \spX \spi  \spX &=& \spX
  %  & \forall \spX\in\ssetS
  %  %& (\prop{idempotent})
  %  \\
  %    \cntx & \spX \adds \spY &=& \spY \adds \spX
  %  & \cntx & \spX \spi  \spY &=& \spY \spi  \spX
  %  & \forall \spX,\spY\in\ssetS
  %  %& (\prop{commutative})
  %  \\
  %    \cntx & (\spX \adds \spY) \adds \spZ &=& \spX \adds (\spY \adds \spZ)
  %  & \cntx & (\spX \spi  \spY) \spi  \spZ &=& \spX \spi  (\spY \spi  \spZ)
  %  & \forall \spX,\spY,\spZ\in\ssetS
  %  %& (\prop{associative})
  %  \\
  %    \cntx & \spX \adds (\spX \spi  \spY) &=& \spX
  %  & \cntx & \spX \spi  (\spX \adds \spY) &=& \spX
  %  & \forall \spX,\spY\in\ssetS
  %  %& (\prop{absorptive}/\prop{contractive})
  %  \\
      1. & \spX \adds \spXo     &=& \spH                & \forall\spX\in\spH & (\prope{complemented})
    \\2. & \spX \spi \spXo      &=& \spZero             & \forall\spX\in\spH & (\prope{complemented})
    \\3. & \brp{\spXo}^\orthog  &=& \spX                & \forall\spX\in\spH & (\prope{involutory})
    \\4. & \spX\orel \spY &\implies& \spYo\orel \spXo   & \forall\spX,\spY\in\spH & (\prope{antitone})
    \\5. & \spX\orel \spY &\implies& \spX \adds \brp{\spXo \spi \spY} = \spY &\forall \spX,\spY\in\setX & (\prope{orthomodular identity})
\end{array}$}
\end{theorem}
\begin{proof}
\begin{enume}
  \item Proof for \prope{complemented} (1) property: by \thme{Projection Theorem} \xref{thm:projection}.
  \item Proof for \prope{complemented} (2) property: by \prefpp{cor:inprod_orthog}.
  \item Proof for \prope{involutory}       property: by \prefpp{cor:inprod_orthog}.
  \item Proof for \prope{antitone}   property: by \prefpp{prop:ABBoAo}.
  \item Proof for \prope{orthomodular identity} property: 
  \item Proof that lattice is \prope{orthomodular}: by 5 properties and definition of \structe{orthomodular lattice}\ifsxref{ortholat}{def:latom}.
\end{enume}
\end{proof}

%%---------------------------------------
%\begin{theorem}
%\label{thm:sub_latc}
%%---------------------------------------
%Let $\spO$ be a \structe{linear space} \xref{def:vspace}.
%Let $\ssetS$ be a set of linear subspaces of $\spO$.
%\thmbox{
%  \brb{\begin{array}{FllCD}
%    \cnto & \spZero       &\in \ssetS &                           & and
%    \cntn & \spO          &\in \ssetS &                           & and
%    \cntn & \spX\adds\spY &\in \ssetS & \forall\spX,\spY\in\ssetS & and
%    \cntn & \spX\spi \spY &\in \ssetS & \forall\spX,\spY\in\ssetS & and
%    \cntn & \spXo         &\in \ssetS & \forall\spX     \in\ssetS 
%  \end{array}}
%  \quad\implies\quad
%  \brb{\begin{array}{M}
%    $\latc{\ssetS}{\subseteq}{\adds}{\spi }{\orthog}{\spZero}{\spO}$\\
%    is a \hib{complemented lattice}.
%  \end{array}}
%  }
%\end{theorem}
%
%In this document, the lattice described in \pref{thm:sub_latc} (previous theorem) will be called a \hid{complemented subspace lattice}.

%%---------------------------------------
%\begin{theorem}
%\label{thm:sub_latc_prop}
%%---------------------------------------
%Let $\latc{\ssetS}{\subseteq}{\adds}{\spi }{\orthog}{\spZero}{\spO}$ 
%be a \hib{complemented subspace lattice}.
%\thmbox{
%%\renewcommand{\arraystretch}{1.25}
%\begin{array}{rcl @{\qquad} rcl @{\qquad}C @{\qquad}D}
%    %  \spX \adds \spX &=& \spX
%    %& \spX \spi  \spX &=& \spX
%    %& \forall \spX\in\ssetS
%    %& (\prop{idempotent})
%    %\\
%    %  \spX \adds \spY &=& \spY \adds \spX
%    %& \spX \spi  \spY &=& \spY \spi  \spX
%    %& \forall \spX,\spY\in\ssetS
%    %& (\prop{commutative})
%    %\\
%    %  (\spX \adds \spY) \adds \spZ &=& \spX \adds (\spY \adds \spZ)
%    %& (\spX \spi  \spY) \spi  \spZ &=& \spX \spi  (\spY \spi  \spZ)
%    %& \forall \spX,\spY,\spZ\in\ssetS
%    %& (\prop{associative})
%    %\\
%    %  \spX \adds (\spX \spi  \spY) &=& \spX
%    %& \spX \spi  (\spX \adds \spY) &=& \spX
%    %& \forall \spX,\spY\in\ssetS
%    %& (\prop{absorptive}/\prop{contractive})
%    %\\
%      \spX \adds \spXo &=& \spO   
%    & \spX \spi  \spXo &=& \spZero   
%    & \forall \spX,\spY\in\ssetS
%    & (\prop{complemented})
%\end{array}}
%\end{theorem}
%\begin{proof}
%These results follow directly from the definition 
%of complemented lattices (\ifdochasni{latvar}{\prefp{def:latc}}).
%\end{proof}


%======================================
%\subsubsection{Subspaces of Hilbert spaces}
%======================================
\begin{minipage}{11\tw/16}%
%---------------------------------------
%\begin{theorem}
%\footnotemark
%---------------------------------------
%Let $\spH$ be a Hilbert space and $\pset{\spH}$ the set of closed linear subspaces of $\spH$.
%\thmboxt{
%  $\latoc{\psetH}{\subseteq}{\adds}{\spi }{\orthog}{\spZero}{\spH}$\\
%  is an \structe{orthomodular lattice} \xref{def:latoc_omod}.
%  }
%\\
This concept is illustrated to the right where $\spX,\spY\in\pset{\spH}$ 
are linear subspaces of the linear space $\spH$ and
\\\indentx$ \spX \subseteq \spY \quad\implies\quad \spY=\spX \adds \brp{\spX^\ocop \spi  \setY}$.
%\end{theorem}%
\end{minipage}%
%\citetblt{%
%  \citerppg{iturrioz1985}{56}{57}{0821850768}
%  }%
\begin{minipage}{5\tw/16}%
  \begin{center}
  \footnotesize
  \setlength{\unitlength}{\tw/250}%
  \begin{picture}(250,250)(-100,-100)%
    \thicklines%
    %{\color{graphpaper}\graphpaper[10](-100,-100)(250,250)}%
    \color{blue}%
      \put(  25,  25){\vector(1, 0){75}}%
      \put( 105,  25){\makebox(0,0)[l]{$\spX$}}%
    \color{green}%
      \put(   0,-100){\line( 0, 1){200} }%
      \put(   0, 100){\line( 1, 1){ 50} }%
      \put(   0,-100){\line( 1, 1){ 50} }%
      \put(  50,  50){\line( 0, 1){100} }%
      \put(  50, -50){\line( 0, 1){ 50} }%
      \put(  70,  90){\makebox(0,0)[l]{$\spX^\ocop$}}%
      \put(  60,  90){\vector(-1,0){35}}%
    \color{red}%
      \put(-100,   0){\line( 1, 0){200} }%
      \put(-100,   0){\line( 1, 1){ 50} }%
      \put( 100,   0){\line( 1, 1){ 50} }%
      \put( -50,  50){\line( 1, 0){ 50} }%
      \put(  50,  50){\line( 1, 0){100} }%
      \put( -50, 110){\makebox(0,0)[bl]{$\spY=\spX \adds \brp{\spX^\ocop \spi  \setY}$}}%
      \put( -30, 100){\vector(0,-1){75}}%
    \color{black}%
      \put(   0,   0){\line( 1, 1){ 50} }%
      \put(  70, -30){\makebox(0,0)[tl]{$\spX^\ocop\spi \spY$}}%
      \put(  65, -35){\vector(-1,1){50}}%
  \end{picture}%
  \end{center}
\end{minipage}%

%---------------------------------------
\begin{corollary}
%---------------------------------------
Let $\spH$ be a Hilbert space with orthogonality operation $\orthog$.
Let $\latoc{\psetH}{\subseteq}{\adds}{\spi }{\orthog}{\spZero}{\spH}$ be the lattice of subspaces of $\spH$.
\corbox{
  \begin{array}{rclCDD}
    (\spX\adds \spY)^\orthog &=& \spX^\orthog \spi  \spY^\orthog & \forall \spX,\spY\in\psetH & (\prope{de Morgan}) & and \\
    (\spX\spi  \spY)^\orthog &=& \spX^\orthog \adds \spY^\orthog & \forall \spX,\spY\in\psetH & (\prope{de Morgan}) & 
  \end{array}
  }
\end{corollary}
\begin{proof}
By properties of \structe{orthocomplemented lattice}s \ifsxref{ortholat}{thm:latoc_prop}.
\end{proof}




\if 0

%======================================
\subsection{Subspace lattice isomorphisms}
%======================================
There are three equivalent ways of representing a subspace lattice:
\\\begin{tabular}{>{\qquad}llll}
 %1. & subspace lattice             & \pref{def:subspace_lattice} & \prefpo{def:subspace_lattice} \\
  1. & subspace lattice             & \pref{thm:sub_lat}          & \prefpo{thm:sub_lat} \\
  2. & projection operator lattice  & \pref{thm:operator_lattice} & \prefpo{thm:operator_lattice} \\
  3. & bases lattice                & \pref{thm:basis_lattice}    & \prefpo{thm:basis_lattice}
\end{tabular}\\
And these three methods provide representations that are isomorphic to each other;
this is proven in \prefpp{thm:VPb_nsomorphic} and 
illustrated in \prefpp{fig:VPb_nsomorphic}.
Because they are isomorphic, they are essentially equivalent and
can be used interchangeably.


%%---------------------------------------
%% Progressive {x,y,z} lattice
%%---------------------------------------
%\begin{minipage}[c]{4\tw/16}
%  \begin{center}%
%  \begin{fsL}%
%  \setlength{\unitlength}{\tw/(400)}%
%  \begin{picture}(300,400)(-150,-50)%
%  %{\color{graphpaper}\graphpaper[10](-150,-50)(300,400)}%
%    \thicklines%
%    {\color{black}%
%      \put( 100, 100){\line(-1, 2){100}}%
%      \put(   0, 100){\line(-1, 2){ 50}}%
%      \put(-100, 100){\line( 1, 2){100}}%
%      \put(   0,   0){\line(-1, 1){100}}%
%      \put(   0,   0){\line( 0, 1){100}}%
%      \put(   0,   0){\line( 1, 1){100}}%
%      }%
%    {\color{blue}%
%      \put(   0, 300){\circle*{15}}%
%      \put(- 50, 200){\circle*{15}}%
%      \put( 100, 100){\circle*{15}}%
%      \put(   0, 100){\circle*{15}}%
%      \put(-100, 100){\circle*{15}}%
%      \put(   0,   0){\circle*{15}}%
%      }%
%    {\color{blue}%
%      \put(   0, 310){\makebox(0,0)[b] {$\spV_5$}}%
%      \put(- 60, 200){\makebox(0,0)[cr]{$\spV_4$}}%
%      \put( 110, 100){\makebox(0,0)[cl]{$\spV_3$}}%
%      \put(  10, 100){\makebox(0,0)[tl]{$\spV_2$}}%
%      \put(-110, 100){\makebox(0,0)[cr]{$\spV_1$}}%
%      \put(   0, -10){\makebox(0,0)[t] {$\spZero$}}%
%      }%
%  \end{picture}%
%  \end{fsL}%
%  \end{center}%
%\end{minipage}

%%---------------------------------------
%\begin{definition}
%\label{def:subspace_lattice}
%%---------------------------------------
%\defbox{\begin{array}{l@{\qquad}lll}
%  \mc{4}{l}{\text{A \hid{subspace lattice} is any 4-tupple
%  $(\seq{\spX_n}{n\in\Z},\, \subseteq,\, +,\, \spi )$ where \footnotemark
%  }}
%  \\ &1.& (\seq{\spX_n}{n\in\Z},\, \subseteq,\, +,\, \spi ) & \text{is a lattice}
%  \\ &2.& \seq{\spX_n}{n\in\Z} & \text{is a sequence of subspaces}
%  \\ &3.& \subseteq    & \text{is the set inclusion relation}
%  \\ &4.& \adds        & \text{is subspace addition}
%  \\ &5.& \mpys        & \text{is the set intersection operator.}
%\end{array}}
%\footnotetext{\begin{tabular}[t]{lll}
%  \hie{subspace}:          & \pref{def:subspace} & \prefpo{def:subspace}  \\
%  \hie{subspace addition}: & \pref{def:sub_add} & \prefpo{def:sub_add}  \\
%  \ifdochas{found}{\hie{set intersection}:  & \pref{def:ss_setops}  & \prefpo{def:ss_setops}}
%\end{tabular}}
%\end{definition}



%We can define a projection operator $\opP_n$ for every subspace $\spX_n\subseteq\spO$ 
%in a \prop{subspace lattice} such that 
%\[ \spX_n = \opP_n\spO \qquad \forall n\in\Z \]
%\xrefP{def:operator_lattice}.
%Each projection operator $\opP_n$ in the lattice ``projects" the range space $\spO$
%onto a subspace $\spX_n$.

%---------------------------------------
\begin{theorem}
\label{thm:operator_lattice}
%---------------------------------------
Let $\lattice{\ssetS}{\subseteq}{\adds}{\spi }$ be a \hib{subspace lattice}
on a linear space $\spO$.
Let $\ssetP$ be a set of projection operators in $\clF{\spO}{\spO}$,
$\addo$ operator addition, and $\mpyo$ operator multiplication\ifsxref{relation}{def:op+x}.
\thmbox{\begin{array}{>{\qquad\scriptstyle}rl@{\qquad}D}
  \mc{3}{M}{The tuple $\lattice{\ssetP}{\orelP}{\addo}{\mpyo}$ is a 
            \hib{lattice} and \hib{isomorphic} to $\lattice{\ssetS}{\subseteq}{\adds}{\mpys}$ if}
  \\ 1.& \forall \spX\in\ssetS \quad\exists\quad \opP\in\ssetP \st \opP\spO = \spX             & and 
  \\ 2.& \forall \opP\in\ssetP \quad\exists\quad \spX\in\ssetS \st \opP\spO = \spX             & and 
  \\ 3.& \opP_m\orelP\opP_n \quad\text{if}\quad \opP_m\opP_n=\opP_n\opP_m   =\opP_m            &
\end{array}}
\end{theorem}

%---------------------------------------
\begin{theorem}
\label{thm:basis_lattice}
%---------------------------------------
Let $\lattice{\ssetS}{\subseteq}{\adds}{\spi }$ be a \hib{subspace lattice}
on a linear space $\spO$.
Let $\ssetB$ be a set of bases on $\clF{\spO}{\spO}$.
\thmbox{\begin{array}{>{\qquad\scriptstyle}rl@{\qquad}D}
  \mc{3}{M}{The tuple $\lattice{\ssetB}{\orelB}{\addb}{\mpyb}$ is a 
            \hib{lattice} and \hib{isomorphic} to $\lattice{\ssetS}{\subseteq}{\adds}{\mpys}$ if}
  \\1.& \forall \spX\in\ssetS \quad\exists\quad \setA\in\ssetB \st \linspan\setA = \spX             & and 
  \\2.& \forall \setA\in\ssetB \quad\exists\quad \spX\in\ssetS \st \linspan\setA = \spX             & and 
  \\3.& \setA_1\orelB\setA_2 \implies \linspan\setA_1 \subseteq \linspan\setA_2
\end{array}}
\end{theorem}

%%---------------------------------------
%\begin{definition}
%\label{def:operator_lattice}
%%---------------------------------------
%Let $\spO$ be a linear space with \prop{subspace lattice}
%$(\seq{\spX_n}{n\in\Z},\, \subseteq,\, \adds,\, \spi )$.
%\defbox{\begin{array}{l@{\qquad}lll}
%  \mc{4}{l}{\text{A \hid{projection operator lattice} is any 4-tupple
%  $(\seq{\opP_n}{n\in\Z},\, \subseteq,\, \boxplus,\, \boxtimes)$ where
%  }}
%  \\ &1.& (\seq{\opP_n}{n\in\Z},\, \subseteq,\, \boxplus,\, \boxtimes) & \text{is a lattice}
%  \\ &2.& \seq{\opP_n}{n\in\Z} & \text{is a sequence of projection operators where } \opP_n\spX \eqd \spX_n \forall n\in\Z
%  \\ &3.& \orelP               & \text{where } \opP_m\orelP\opP_n \quad\text{if}\quad \opP_m\opP_n=\opP_n\opP_m=\opP_m
%  \\ &4.& \addo                & \text{is operator addition}
%  \\ &5.& \mpyo                & \text{is operator multiplication}
%\end{array}}
%\end{definition}
%
%
We can define a \hie{basis} $\seq{\psi_{nm}}{m\in\Z}$
for every subspace $\spX_n\subseteq\spO$ 
in a \prop{subspace lattice} such that 
$\spX_n = \linspan\seq{\psi_{nm}}{m\in\Z}$.
\xrefP{def:basis_lattice}.
Each basis sequence $\seqn{\psi_{nm}}$ \hie{spans} or \hie{generates} a subspace $\spX_n$.
%A basis lattice is illustrated in the figure to the left.
%Bases are available in various levels of sophistication---
%more sophistication meaning easier to  use but less general and less sophistication
%meaning more challenging to use but more general.

%\begin{minipage}{12\tw/16}%
%Examples of basis lattices are given by \\
%  \begin{tabular}{llll}
%    $\imark$ & \pref{ex:arch_d1}  & \prefpo{ex:arch_d1}  &---Haar basis / Daubechies-$p1$ basis  \\
%    $\imark$ & \pref{ex:arch_d2}  & \prefpo{ex:arch_d2}  &---Daubechechies-$p2$ basis \\
%    $\imark$ & \pref{ex:arch_cos4} & \prefpo{ex:arch_cos4} &---cosine basis
%  \end{tabular}
%\end{minipage}

%---------------------------------------
\begin{definition}
\label{def:basis_lattice}
%---------------------------------------
Let $\spO$ be a linear space with \prop{subspace lattice}
$(\seq{\spX_n}{n\in\Z},\, \subseteq,\, \adds,\, \spi )$.
\defboxt{
  A \hid{basis lattice} is any 4-tupple 
  $(\seq{\seq{\psi_{n,m}}{m\in\Z}}{n\in\Z},\, \lessdot,\, \spu ,\, \odot)$ where 
  \\\indentx$\begin{array}{FMD}
      1.& $(\seq{\seq{\psi_{n,m}}{m\in\Z}}{n\in\Z},\, \lessdot,\, \spu ,\, \odot)$ is a lattice & and
    \\2.& $\seq{\seq{\psi_{n,m}}{m\in\Z}}{n\in\Z}$ is a sequence of sequence of basis vectors   & and
    \\3.& $\linspan\seq{\psi_{n,m}}{m\in\Z}=\opV_n\quad \forall n\in\Z$                            & and
    %\\4.& \subseteq  & \text{where } 
    \\4.& $\left(\seq{\psi_{i,m}}{m\in\Z}\right)\lessdot\left(\seq{\psi_{j,m}}{m\in\Z}\right) \implies \spV_n\subseteq\spV_j$ & and
    \\5.& $\spu$ is set union & and 
    \\6.& $\odot$ is defined such that
    \\  & $\linspan(\seq{\psi_{i,m}}{m\in\Z}\odot\seq{\psi_{j,m}}{m\in\Z}) = (\linspan\seq{\psi_{i,m}}{m\in\Z})\,\odot\,(\linspan\seq{\psi_{j,m}}{m\in\Z})$.
  \end{array}$}
\end{definition}

%---------------------------------------
% isomorphic lattices
%---------------------------------------


\begin{figure}[th]
  \begin{center}%
  \begin{fsL}%
  \begin{minipage}[c]{12\tw/16}
  \begin{minipage}[c]{4\tw/16}%
  \center
  \latmatlw{4}{0.5}
    {
           &       & \null                 \\  
           & \null                         \\  
     \null &       & \null &       & \null \\  
           &       & \null                   
    }
    {\ncline{1,3}{2,2}\ncline{2,2}{3,1}
     \ncline{1,3}{3,5}
     \ncline{2,2}{3,3}
     \ncline{4,3}{3,1}\ncline{4,3}{3,3}\ncline{4,3}{3,5}
    }
    {\nput{ 90}{1,3}{$\spX_5$}
     \nput{135}{2,2}{$\spX_4$}
     \nput{180}{3,1}{$\spX_1$}
     \nput{ 67}{3,3}{$\spX_2$}
     \nput{  0}{3,5}{$\spX_3$}
     \nput{-90}{4,3}{$\spZero$}
    }
  \end{minipage}
  \hfill{\Huge$\thapprox$}\hfill
  \begin{minipage}[c]{4\tw/16}%
  \center
  \latmatlw{4}{0.5}
    {
           &       & \null                 \\  
           & \null                         \\  
     \null &       & \null &       & \null \\  
           &       & \null                   
    }
    {\ncline{1,3}{2,2}\ncline{2,2}{3,1}
     \ncline{1,3}{3,5}
     \ncline{2,2}{3,3}
     \ncline{4,3}{3,1}\ncline{4,3}{3,3}\ncline{4,3}{3,5}
    }
    {\nput{ 90}{1,3}{$\opP_5$}
     \nput{135}{2,2}{$\opP_4$}
     \nput{180}{3,1}{$\opP_1$}
     \nput{ 67}{3,3}{$\opP_2$}
     \nput{  0}{3,5}{$\opP_3$}
     \nput{-90}{4,3}{$\opZero$}
    }
  \end{minipage}
  \hfill{\Huge$\thapprox$}\hfill
  \begin{minipage}[c]{4\tw/16}%
  \center
  \latmatlw{4}{0.5}
    {
           &       & \null                 \\  
           & \null                         \\  
     \null &       & \null &       & \null \\  
           &       & \null                   
    }
    {\ncline{1,3}{2,2}\ncline{2,2}{3,1}
     \ncline{1,3}{3,5}
     \ncline{2,2}{3,3}
     \ncline{4,3}{3,1}\ncline{4,3}{3,3}\ncline{4,3}{3,5}
    }
    {\nput{ 90}{1,3}{$\seqn{\psi_{5,m}}$}
     \nput{135}{2,2}{$\seqn{\psi_{4,m}}$}
     \nput{180}{3,1}{$\seqn{\psi_{1,m}}$}
     \nput{ 67}{3,3}{$\seqn{\psi_{2,m}}$}
     \nput{  0}{3,5}{$\seqn{\psi_{3,m}}$}
     \nput{-90}{4,3}{$\opZero$}
    }
  \end{minipage}
\end{minipage}
\end{fsL}
\end{center}
\caption{
  Subspace, projection, basis lattice isomorphism
  \label{fig:VPb_nsomorphic}
  }
\end{figure}




%--------------------------------------
\begin{theorem}
\label{thm:VPb_nsomorphic}
%--------------------------------------
Let $\spX$ be a \structe{linear space} \xref{def:vspace}.
\thmbox{\begin{array}{>{\qquad\scriptstyle}r >{\ds}l @{\qquad}D}
  \mc{3}{M}{The following lattices are isomorphic to each other:}
  \\1.& (\seq{\spX_n}{n\in\Z},\,\subseteq,\,\adds,\,\spi )                    &     (\prop{subspace lattice})
  \\2.& (\seq{\opP_n}{n\in\Z},\,\sqsubseteq,\,\boxplus,\,\boxtimes)            &     (\prop{operator lattice})
  \\3.& (\seq{\seq{\psi_{n,m}}{m\in\Z}}{n\in\Z},\, \lessdot,\, \spu ,\, \odot) &     (\prop{basis lattice})
\end{array}}
\end{theorem}
\begin{proof}
\begin{enumerate}
\item Proof that (1) and (2) are isomorphic:
\begin{enumerate}
  \item Define a function
      $\ff:(\seq{\spX_n}{n\in\Z},\,\subseteq,\,\adds,\,\spi )
       \to (\seq{\opP_n}{n\in\Z},\,\sqsubseteq,\,\boxplus,\,\boxtimes)$:
      \[ \ff(\spX_n) = \opP_n  \]
      \label{item:wav_nso_lat}

  \item The function $\ff$ is an \hie{isomorphism} from the wavelet subspace lattice to the
        wavelet operator lattice. Proof:
    \begin{align*}
      \ff(\spX_n \adds \spV_m)
        &= \ff(\opP_n\spX \adds \opP_m\spX)
        && \text{by \prefp{thm:operator_lattice}}
      \\&= \ff([\opP_n \boxplus \opP_m]\spX )
        && \ifdochas{relation}{\text{by \prefp{def:op+x}}}
      \\&= \opP_n \boxplus \opP_m
        && \text{by \prefp{item:wav_nso_lat}}
      \\&= \ff(\spX_n) \boxplus \ff(\spV_m)
        && \text{by \prefp{item:wav_nso_lat}}
      \\
      \\
      \ff(\spX_n \spi  \spV_m)
        &= \ff(\opP_n\spX \spi  \opP_m\spX)
        && \text{by \prefp{thm:operator_lattice}}
      \\&= \ff(\opP_n[ \opP_m\spX] )
        &&
      \\&= \ff([\opP_n \boxtimes \opP_m]\spX )
        && \ifdochas{relation}{\text{by \prefp{def:op+x}}}
      \\&= \opP_n \boxtimes \opP_m
        && \text{by \prefp{item:wav_nso_lat}}
      \\&= \ff(\spX_n) \boxtimes \ff(\spV_m)
        && \text{by \prefp{item:wav_nso_lat}}
    \end{align*}
  \end{enumerate}

\item Proof that (2) and (3) are isomorphic:
\begin{enumerate}
  \item Define a function
      $\ff:(\seq{\spX_n}{n\in\Z},\,\subseteq,\,\adds,\,\spi )
       \to (\seq{\seq{\psi_{n,m}}{m\in\Z}}{n\in\Z},\, \lessdot,\, \spu ,\, \odot)$:
      \[ \ff(\spX_n) = \seq{\psi_{n,m}}{m\in\Z} \quad \forall n\in\Z \]
      \label{item:wav_nso_basis}

  \item The function $\ff$ is an \hie{isomorphism} from the wavelet subspace lattice to the
        wavelet basis lattice. Proof:
    \begin{align*}
      \ff(\spX_n + \spV_m)
        &= \ff(\linspan\seqn{\psi_n} \spu  \linspan\seq{\psi_m})
        && \text{by \prefp{def:basis_lattice}}
      \\&= \ff(\linspan[\seqn{\psi_n} \spu  \seq{\psi_m}])
      \\&= \seqn{\psi_n} \spu  \seq{\psi_m}]
        && \text{by \prefp{item:wav_nso_basis}}
      \\&= \ff(\spX_n) \spu  \ff(\spV_m)
        && \text{by \prefp{item:wav_nso_basis}}
      \\
      \\
      \ff(\spX_n \spi  \spV_m)
        &= \ff(\linspan\seqn{\psi_n} \spi  \linspan\seq{\psi_m})
        && \text{by \prefp{def:basis_lattice}}
      \\&= \ff(\linspan[\seqn{\psi_n} \odot \seq{\psi_m}])
        && \text{by definition of $\odot$ \prefpo{def:basis_lattice}}
      \\&= \seqn{\psi_n} \odot \linspan\seq{\psi_m}]
        && \text{by \prefp{item:wav_nso_basis}}
      \\&= \ff(\spX_n) \odot \ff(\spV_m)
        && \text{by \prefp{item:wav_nso_basis}}
    \end{align*}
  \end{enumerate}
\end{enumerate}
\end{proof}

%%======================================
%\subsubsection{Subspace architectures}
%%======================================
%A subspace lattice architecture is simply the general structure of the lattice.
%That is, two lattices have the same architecture if they are \hie{isomorphic} to each other.
%Examples of subspace architectures on the 3-dimensional Euclidean space are illustrated by 
%\\\begin{tabular}{>{\qquad}lllll}
%  $\imark$ & \pref{ex:lat_E3d_power}       & \prefpo{ex:lat_E3d_power}       &---power architecture \\
%  $\imark$ & \pref{ex:lat_E3d_progressive} & \prefpo{ex:lat_E3d_progressive} &---progressive architecture \\
%  $\imark$ & \pref{ex:lat_E3d_primitive}   & \prefpo{ex:lat_E3d_primitive}   &---primitive architecture \\
%  $\imark$ & \pref{ex:lat_E3d_linear}      & \prefpo{ex:lat_E3d_linear}      &---linear architecture
%\end{tabular}

%=======================================
\subsection{Analyses}
%=======================================
An \hid{analysis} of a space $\spX$ is any lattice of subspaces of $\spX$.
%A sequence $\seq{\spV_j}{j\in\Z}$ of linear subspaces of a linear space $\spX$ is an \hib{analysis} of $\spX$.
        %if  $\seq{\spV_j}{j\in\Z}$ is a partition of $\spX$.
        The partial or complete reconstruction of $\spX$ from this set is a \hid{synthesis}.%
\footnote{%
  The word \hie{analysis} comes from the Greek word
  {\fntagreek{>av'alusis}},
  meaning ``dissolution" (\citerpc{perschbacher1990}{23}{entry 359}),
  which in turn means
  ``the resolution or separation into component parts"
  (\citer{collins2009}, \scs\url{http://dictionary.reference.com/browse/dissolution})
  }

\begin{figure}[th]
  %\begin{tabular}{*{3}{>{\scs}c}}
  %   \includegraphics{../common/math/graphics/pdfs/latr3linear.tex}%
  %  &\includegraphics{../common/math/graphics/pdfs/latr3m3.tex}%
  %  &\includegraphics{../common/math/graphics/pdfs/latr3mra.tex}%
  %  \\linearly ordered analysis of $\R^3$
  %  &M-3 analysis of $\R^3$
  %  &wavelet-like analysis of $\R^3$
  %\end{tabular}
  \caption{some anayses of $\R^3$ \xref{ex:r3analyses} \label{fig:r3analyses}}
\end{figure}%
%---------------------------------------
\begin{example}
\label{ex:r3analyses}
%---------------------------------------
  The lattices of subspaces illustrated in \prefpp{fig:r3analyses} are all \structe{analyses} of 
  $\R^3$.
\end{example}



%=======================================
\subsection{Transform}
%=======================================
%---------------------------------------
\begin{definition}
%---------------------------------------
A \hid{transform} on a space $\spX$ is a sequence of projection operators that induces 
an \structe{analysis} on $\spX$.
\end{definition}

%---------------------------------------
\begin{example}
%---------------------------------------
A \structe{Fourier analysis} is a sequence of subspaces with sinusoidal bases.
  Examples of subspaces in a Fourier analysis include $\spV_1=\linspan\setn{e^{ix}}$, 
  $\spV_{2.3}=\linspan\setn{e^{i2.3x}}$, $\spV_{\sqrt{2}}=\linspan\setn{e^{i\sqrt{2}x}}$, etc.
  A \hib{transform} is a set of \structe{projection operators} that maps a family of functions (e.g. $\spLLR$)
  into an analysis.
  The \ope{Fourier transform}" for Fourier Analysis is \ifxref{harFour}{def:opFT}
  \\\indentx$\ds\brs{\opFT\ff}(\omega) \eqd \frac{1}{\sqrt{2\pi}} \int_\R \ff(x) e^{-i\omega x} \dx$
\end{example}

%  \item A sequence $\opT$ in $\clFxy$ is a \hib{transform} \label{item:wavstrct_T}
%        if each element in the sequence is a projection operator in $\clFxy$.
%        An example of a transform is the \hib{cosine transform} $\opT$ in $\clFrr$ such that
%        \begin{align*}
%          \opT\fx(t) &\eqd \seq{\opP_j}{j\in\Z}
%             \\&\eqd \seq{\int_{t\in\R} \fx(t)\,\mcom{\cos(nt)}{kernel} \dt}{n\in\Z}
%             \\&\eqd \seqn{\cdots,\,
%                           %\int_{t\in\R} \cos\brs{(-2)t}\,\fx(t) \dt,\,
%                           \int_{t\in\R} \fx(t)\,\cos\brs{(-1)t} \dt,\,
%                           \int_{t\in\R} \fx(t)\,                \dt,\,
%                           \int_{t\in\R} \fx(t)\,\cos\brs{( 1)t} \dt,\,
%                           %\int_{t\in\R} \cos\brs{(-2)t}\,\fx(t) \dt,\,
%                           \cdots
%                          }
%        \end{align*}
%        Further examples of transforms include the \hie{Fourier Transform} and various \hie{Wavelet Transforms}.
% it is a \hib{sequence} of projection operators on $\A function $\opT$ in $\clFxy$ is a \hib{transform} if with domain $\clFxy$ and range $\clF{\setA}{\setB}$ if


%---------------------------------------
\subsection{Properties of subspace order structures}
%---------------------------------------
The ordered set of all linear subspaces of a \structe{Hilbert space} is an \structe{orthomodular lattice}.
Orthomodular lattices (and hence Hilbert subspaces) have some special properties \xref{thm:latoc_char3}.
One is that they satisfy \prope{de Morgan's law}.

\begin{figure}
{\begin{center}%
  \begin{fsL}%
\psset{unit=8mm}%
\begin{tabular}{|c|c|}%
\hline%
\mc{1}{B}{Cosine analysis  (even Fourier series)} & \mc{1}{B}{Cosine polynomial analysis}%
\\%
  \includegraphics{../common/math/graphics/pdfs/pdfs/baslat_cosh.pdf}%
&%
  \includegraphics{../common/math/graphics/pdfs/baslat_cose.pdf}%
\\\hline%
\mc{1}{|B|}{Chebyshev polynomial analysis\cittrpg{rivlin1974}{4}{047172470X}}&\mc{1}{|B|}{Hadamard-3 analysis}%
\\%
  \includegraphics{../common/math/graphics/pdfs/baslat_cheby.pdf}%
&%
  \includegraphics{../common/math/graphics/pdfs/baslat_h3.pdf}%
\\\hline%
\mc{1}{|B|}{Haar/Daubechies-$p1$ wavelet analysis} & \mc{1}{B|}{Daubechies-$p2$ wavelet analysis}%
\\%
  \includegraphics{../common/math/graphics/pdfs/baslat_d1.pdf}%
&%
  \includegraphics{../common/math/graphics/pdfs/baslat_d2.pdf}%
\\\hline%
\end{tabular}%
  \end{fsL}%
\end{center}}%
\caption{some common transforms\label{fig:commontrans}}
\end{figure}

\mbox{}\\
\begin{minipage}{\tw-65mm}%
  %An analysis can also be partially characterized by its order structure with respect
  %to an order relation such as the set inclusion relation $\subseteq$.
  Most transforms have a very simple M-$n$ order structure,
  as illustrated to the right and in \prefp{fig:commontrans}.
  The M-$n$ lattices for $n\ge3$ are \prope{modular} but not \prope{distributive}.
  Analyses typically have one subspace that is a \hie{scaling} subspace;
  and this subspace is often simply a family of constants
  (as is the case with \structe{Fourier Analysis}).
  There is one noteable exception to this---MRA induced \structe{wavelet analysis}\ifsxref{wavstrct}{def:wavelet}.
\end{minipage}%
\hfill%
{\begin{minipage}{60mm}%
  \begin{center}
%  \mbox{}\\% force (just above?) top of graphic to be the top of the minipage
  \includegraphics{../common/math/graphics/pdfs/latmn.pdf}%
  \end{center}
\end{minipage}}

%%---------------------------------------
%% Euclidean 3-space partitioned by power lattice
%%---------------------------------------
%\begin{minipage}{10\tw/16}
%%---------------------------------------
%\begin{example}
%\label{ex:lat_E3d_power}
%%---------------------------------------
%The figure to the right illustrates the subspace lattice
%\[ \lattice{\pset{(\text{$x$-axis, $y$-axis, $z$-axis})}}{\subseteq}{+}{\spi }. \]
%The least upper bound of the lattice is the 3-dimensional Euclidean space.
%This lattice is isomorphic to the power set $\pset{\setn{x,y,z}}$.
%\end{example}
%\end{minipage}%
%\begin{minipage}{6\tw/16}%
%\begin{center}
%\footnotesize
%\setlength{\unitlength}{\tw/500}%
%\begin{picture}(440,520)(-220,0)%
%  \thicklines
%  %{\color{graphpaper}\graphpaper[10](-200,0)(400,520)}%
%  {\color{picbox}%
%    %\put( -50,400){\framebox(100,100){}}%
%    \put( 100,250){\framebox(100,100){}}%
%    \put( -50,250){\framebox(100,100){}}%
%    \put(-200,250){\framebox(100,100){}}%
%    \put( 100,100){\framebox(100,100){}}%
%    \put( -50,100){\framebox(100,100){}}%
%    \put(-200,100){\framebox(100,100){}}%
%    \put( -25,  0){\framebox( 50, 50){}}%
%    }%
%  {\color{black}%
%    \put(   0,400){\line( 0,-1){ 50}}%
%    \put(   0,400){\line(-3,-1){150}}%
%    \put(   0,400){\line( 3,-1){150}}%
%    \put(   0,200){\line(-3, 1){150}}%
%    \put( 150,200){\line(-3, 1){150}}%
%    \put(-150,200){\line( 3, 1){150}}%
%    \put( 150,200){\line( 0, 1){ 50}}%
%    \put(-150,200){\line( 0, 1){ 50}}%
%    \put(   0,200){\line( 3, 1){150}}%
%    \put(   0, 50){\line(-3, 1){150}}%
%    \put(   0, 50){\line( 0, 1){ 50}}%
%    \put(   0, 50){\line( 3, 1){150}}%
%    }%
%  \put(0,450){%
%    \setlength{\unitlength}{2\tw/(450*3)}%
%    \thicklines
%    \color{red}%
%      \put(  50,  50){\line( 1, 1){50} }%
%      \put(  50,-100){\line( 1, 1){50} }%
%      \put(-100,  50){\line( 1, 1){50} }%
%    \color{green}%
%      \put(- 50, 100){\line( 1, 0){150} }%
%      \put(-100,  50){\line( 1, 0){150} }%
%      \put(-100,-100){\line( 1, 0){150} }%
%    \color{blue}%
%      \put( 100,- 50){\line( 0, 1){150} }%
%      \put(  50,-100){\line( 0, 1){150} }%
%      \put(-100,-100){\line( 0, 1){150} }%
%    }
%  \put(-150,300){%
%    \setlength{\unitlength}{1\tw/(400*3)}%
%    \begin{picture}(0,0)(0,0)%
%      %{\color{graphpaper}\graphpaper[10](-150,-150)(300,300)}%
%      {\color{xaxis}%
%        \put(-150, -50){\line( 1, 1){100} }%
%        \put(  50, -50){\line( 1, 1){100} }%
%        }%
%      {\color{yaxis}%
%        \put(-150, -50){\line( 1, 0){200} }%
%        \put( 150,  50){\line(-1, 0){200} }%
%        }%
%    \end{picture}%
%  }
%  \put(0,300){%
%    \setlength{\unitlength}{1\tw/(400*3)}%
%    \begin{picture}(0,0)(0,0)%
%      %{\color{graphpaper}\graphpaper[10](-150,-150)(300,300)}%
%      {\color{xaxis}%
%        \put( -50,  50){\line( 1, 1){100} }%
%        \put(  50, -50){\line(-1,-1){100} }%
%        }%
%      {\color{zaxis}%
%        \put( -50,-150){\line( 0, 1){200} }%
%        \put(  50, 150){\line( 0,-1){200} }%
%        }%
%    \end{picture}%
%  }
%  \put(150,300){%
%    \setlength{\unitlength}{1\tw/(400*3)}%
%    \begin{picture}(0,0)(0,0)%
%      %{\color{graphpaper}\graphpaper[10](-150,-150)(300,300)}%
%      {\color{yaxis}%
%        \put(-100, 100){\line( 1, 0){200} }%
%        \put(-100,-100){\line( 1, 0){200} }%
%        }%
%      {\color{zaxis}%
%        \put(-100,-100){\line( 0, 1){200} }%
%        \put( 100,-100){\line( 0, 1){200} }%
%        }%
%    \end{picture}%
%  }
%  \put(-150,150){%
%    \setlength{\unitlength}{1\tw/(400*3)}%
%    \begin{picture}(0,0)(0,0)%
%      %{\color{graphpaper}\graphpaper[10](-150,-150)(300,300)}%
%      {\color{xaxis}%
%        \put(0,0){\vector( 1, 1){100} }%
%        \put(0,0){\vector(-1,-1){100} }%
%        }%
%    \end{picture}%
%  }
%  \put(0,150){%
%    \setlength{\unitlength}{1\tw/(400*3)}%
%    \begin{picture}(0,0)(0,0)%
%      %{\color{graphpaper}\graphpaper[10](-150,-150)(300,300)}%
%      {\color{yaxis}%
%        \put(0,0){\vector( 1, 0){100} }%
%        \put(0,0){\vector(-1, 0){100} }%
%        }%
%    \end{picture}%
%  }
%  \put(150,150){%
%    \setlength{\unitlength}{1\tw/(400*3)}%
%    \begin{picture}(0,0)(0,0)%
%      %{\color{graphpaper}\graphpaper[10](-150,-150)(300,300)}%
%      {\color{zaxis}%
%        \put(0,0){\vector( 0, 1){100} }%
%        \put(0,0){\vector( 0,-1){100} }%
%        }%
%    \end{picture}%
%  }
%  \put(0,25){%
%    \setlength{\unitlength}{1\tw/(400*3)}%
%    \begin{picture}(0,0)(0,0)%
%      %{\color{graphpaper}\graphpaper[10](-150,-150)(300,300)}%
%      {\color{origin}%
%        \put(0,0){\circle*{15}}%
%        }%
%    \end{picture}%
%  }
%\end{picture}
%\end{center}
%%\caption{
%%  Euclidean 3-dimensional space partitioned as a power lattice
%%  \label{fig:lat_E3d_power}
%%  }
%%\end{figure}
%\end{minipage}%
%
%\if 0
%%---------------------------------------
%% P({x,y,z}) lattice
%%---------------------------------------
%\begin{minipage}[c]{6\tw/16}
%  \begin{center}%
%  \begin{fsL}%
%  \setlength{\unitlength}{3\tw/1600}%
%  \begin{picture}(300,400)(-150,-50)%
%  %{\color{graphpaper}\graphpaper[10](-150,-50)(300,400)}%
%    \thicklines%
%    {\color{axis}%
%      \put( 100, 200){\line(-1, 1){100}}%
%      \put(   0, 200){\line( 0, 1){100}}%
%      \put(-100, 200){\line( 1, 1){100}}%
%      \put( 100, 100){\line( 0, 1){100}}%
%      \put( 100, 100){\line(-1, 1){100}}%
%      \put(   0, 100){\line(-1, 1){100}}%
%      \put(   0, 100){\line( 1, 1){100}}%
%      \put(-100, 100){\line( 0, 1){100}}%
%      \put(-100, 100){\line( 1, 1){100}}%
%      \put(   0,   0){\line(-1, 1){100}}%
%      \put(   0,   0){\line( 0, 1){100}}%
%      \put(   0,   0){\line( 1, 1){100}}%
%      }%
%    {\color{zaxis}%
%      \put(   0, 300){\circle*{15}}%
%      \put( 100, 200){\circle*{15}}%
%      \put(   0, 200){\circle*{15}}%
%      \put(-100, 200){\circle*{15}}%
%      \put( 100, 100){\circle*{15}}%
%      \put(   0, 100){\circle*{15}}%
%      \put(-100, 100){\circle*{15}}%
%      \put(   0,   0){\circle*{15}}%
%      }%
%    {\color{zaxis}%
%      \put(   0, 310){\makebox(0,0)[b] {$\spV_{xyz}$}}%
%      \put( 110, 200){\makebox(0,0)[cl]{$\spV_{yz}$}}%
%      \put(  10, 200){\makebox(0,0)[bl]{$\spV_{xz}$}}%
%      \put(-110, 200){\makebox(0,0)[cr]{$\spV_{xy}$}}%
%      \put( 110, 100){\makebox(0,0)[cl]{$\spV_z$}}%
%      \put(  10, 100){\makebox(0,0)[tl]{$\spV_y$}}%
%      \put(-110, 100){\makebox(0,0)[cr]{$\spV_x$}}%
%      \put(   0, -10){\makebox(0,0)[t] {$\spZero$}}%
%      }%
%  \end{picture}%
%  \end{fsL}%
%  \end{center}%
%\end{minipage}
%\fi
%
%
%%---------------------------------------
%% Euclidean 3-d subspace lattice with progressive architecture
%%---------------------------------------
%\begin{minipage}{10\tw/16}
%%---------------------------------------
%\begin{example}
%\label{ex:lat_E3d_progressive}
%%---------------------------------------
%The figure to the right illustrates the subspace lattice
%\[ (\setn{\text{0, $x$-axis, $y$-axis, $z$-axis, $xy$-plane, $xyz$-space}},\, \subseteq,\, +,\, \spi ). \]
%The least upper bound of the lattice is the 3-dimensional Euclidean space.
%The architecture of this subspace lattice is called the \hie{progressive architecture}.
%%\xrefP{def:arch_progressive}. 
%A progressive architecture is so named because it {\em progressively} adds 
%one dimension for each step up the linearly ordered portion
%(left hand side) of the lattice.
%\end{example}
%\end{minipage}%
%\begin{minipage}[c]{6\tw/16}
%  %\begin{figure}[th]
%  \begin{center}
%  \footnotesize
%  \setlength{\unitlength}{\tw/500}%
%  \begin{picture}(400,520)(-200,0)%
%    \thicklines
%    %{\color{graphpaper}\graphpaper[50](-200,0)(400,520)}%
%    {\color{picbox}%
%      %\put( -50,400){\framebox(100,100){}}%
%      \put(-125,250){\framebox(100,100){}}%
%      \put( 100,100){\framebox(100,100){}}%
%      \put( -50,100){\framebox(100,100){}}%
%      \put(-200,100){\framebox(100,100){}}%
%      \put( -25,  0){\framebox(50,50){}}%
%      }%
%    {\color{black}%
%      \put(   0,400){\line( 3,-4){150}}%
%      \put(   0,400){\line(-3,-2){ 75}}%
%      \put( -75,250){\line( 3,-2){ 75}}%
%      \put( -75,250){\line(-3,-2){ 75}}%
%      \put(   0, 50){\line(-3, 1){150}}%
%      \put(   0, 50){\line( 0, 1){ 50}}%
%      \put(   0, 50){\line( 3, 1){150}}%
%      }%
%  \put(0,450){%
%    \setlength{\unitlength}{2\tw/(450*3)}%
%    \thicklines
%    \color{red}%
%      \put(  50,  50){\line( 1, 1){50} }%
%      \put(  50,-100){\line( 1, 1){50} }%
%      \put(-100,  50){\line( 1, 1){50} }%
%    \color{green}%
%      \put(- 50, 100){\line( 1, 0){150} }%
%      \put(-100,  50){\line( 1, 0){150} }%
%      \put(-100,-100){\line( 1, 0){150} }%
%    \color{blue}%
%      \put( 100,- 50){\line( 0, 1){150} }%
%      \put(  50,-100){\line( 0, 1){150} }%
%      \put(-100,-100){\line( 0, 1){150} }%
%    }
%    \put(-75,300){%
%      \setlength{\unitlength}{1\tw/(400*3)}%
%      \begin{picture}(0,0)(0,0)%
%        %{\color{graphpaper}\graphpaper[10](-150,-150)(300,300)}%
%        {\color{xaxis}%
%          \put(-150, -50){\line( 1, 1){100} }%
%          \put(  50, -50){\line( 1, 1){100} }%
%          }%
%        {\color{yaxis}%
%          \put(-150, -50){\line( 1, 0){200} }%
%          \put( 150,  50){\line(-1, 0){200} }%
%          }%
%      \end{picture}%
%    }
%    \put(-150,150){%
%      \setlength{\unitlength}{1\tw/(400*3)}%
%      \begin{picture}(0,0)(0,0)%
%        %{\color{graphpaper}\graphpaper[10](-150,-150)(300,300)}%
%        {\color{xaxis}%
%          \put(0,0){\vector( 1, 1){100} }%
%          \put(0,0){\vector(-1,-1){100} }%
%          }%
%      \end{picture}%
%    }
%    \put(0,150){%
%      \setlength{\unitlength}{1\tw/(400*3)}%
%      \begin{picture}(0,0)(0,0)%
%        %{\color{graphpaper}\graphpaper[10](-150,-150)(300,300)}%
%        {\color{yaxis}%
%          \put(0,0){\vector( 1, 0){100} }%
%          \put(0,0){\vector(-1, 0){100} }%
%          }%
%      \end{picture}%
%    }
%    \put(150,150){%
%      \setlength{\unitlength}{1\tw/(400*3)}%
%      \begin{picture}(0,0)(0,0)%
%        %{\color{graphpaper}\graphpaper[10](-150,-150)(300,300)}%
%        {\color{zaxis}%
%          \put(0,0){\vector( 0, 1){100} }%
%          \put(0,0){\vector( 0,-1){100} }%
%          }%
%      \end{picture}%
%    }
%    \put(0,25){%
%      \setlength{\unitlength}{1\tw/(400*3)}%
%      \begin{picture}(0,0)(0,0)%
%        %{\color{graphpaper}\graphpaper[10](-150,-150)(300,300)}%
%        {\color{origin}%
%          \put(0,0){\circle*{15}}%
%          }%
%      \end{picture}%
%    }
%  \end{picture}
%  \end{center}
%  %\caption{
%  %  Euclidean 3-dimensional space partitioned as a progressive lattice
%  %  \label{fig:lat_E3d_power_progressive}
%  %  }
%  %\end{figure}
%\end{minipage}%
%
%
%
%\begin{minipage}{10\tw/16}
%%---------------------------------------
%\begin{example}
%\label{ex:lat_E3d_primitive}
%%---------------------------------------
%The figure to the left illustrates the subspace lattice
%\[ (\oquad{\text{0, $x$-axis, $y$-axis, $z$-axis, $xyz$-space}}{\subseteq}{+}{\spi }. \]
%The least upper bound of the lattice is the 3-dimensional Euclidean space.
%The architecture of this subspace lattice is called the 
%\hie{primitive architecture}.% \xrefP{def:arch_primitive}. 
%\end{example}
%\end{minipage}%
%\begin{minipage}{6\tw/16}
%  %\begin{figure}[th]
%  \begin{center}
%  \footnotesize
%  \setlength{\unitlength}{\tw/500}%
%  \begin{picture}(400,370)(-200,0)%
%    \thicklines
%    %{\color{graphpaper}\graphpaper[50](-200,0)(400,370)}%
%    {\color{picbox}%
%      %\put( -50,400){\framebox(100,100){}}%
%      %\put(-125,250){\framebox(100,100){}}%
%      \put( 100,100){\framebox(100,100){}}%
%      \put( -50,100){\framebox(100,100){}}%
%      \put(-200,100){\framebox(100,100){}}%
%      \put( -25,  0){\framebox(50,50){}}%
%      }%
%    {\color{black}%
%      \put(   0,250){\line(-3,-1){150}}%
%      \put(   0,250){\line( 0,-1){ 50}}%
%      \put(   0,250){\line( 3,-1){150}}%
%      \put(   0, 50){\line(-3, 1){150}}%
%      \put(   0, 50){\line( 0, 1){ 50}}%
%      \put(   0, 50){\line( 3, 1){150}}%
%      }%
%  \put(0,300){%
%    \setlength{\unitlength}{2\tw/(450*3)}%
%    \thicklines
%    \color{red}%
%      \put(  50,  50){\line( 1, 1){50} }%
%      \put(  50,-100){\line( 1, 1){50} }%
%      \put(-100,  50){\line( 1, 1){50} }%
%    \color{green}%
%      \put(- 50, 100){\line( 1, 0){150} }%
%      \put(-100,  50){\line( 1, 0){150} }%
%      \put(-100,-100){\line( 1, 0){150} }%
%    \color{blue}%
%      \put( 100,- 50){\line( 0, 1){150} }%
%      \put(  50,-100){\line( 0, 1){150} }%
%      \put(-100,-100){\line( 0, 1){150} }%
%    }
%    \put(-150,150){%
%      \setlength{\unitlength}{1\tw/(400*3)}%
%      \begin{picture}(0,0)(0,0)%
%        %{\color{graphpaper}\graphpaper[10](-150,-150)(300,300)}%
%        {\color{xaxis}%
%          \put(0,0){\vector( 1, 1){100} }%
%          \put(0,0){\vector(-1,-1){100} }%
%          }%
%      \end{picture}%
%    }
%    \put(0,150){%
%      \setlength{\unitlength}{1\tw/(400*3)}%
%      \begin{picture}(0,0)(0,0)%
%        %{\color{graphpaper}\graphpaper[10](-150,-150)(300,300)}%
%        {\color{yaxis}%
%          \put(0,0){\vector( 1, 0){100} }%
%          \put(0,0){\vector(-1, 0){100} }%
%          }%
%      \end{picture}%
%    }
%    \put(150,150){%
%      \setlength{\unitlength}{1\tw/(400*3)}%
%      \begin{picture}(0,0)(0,0)%
%        %{\color{graphpaper}\graphpaper[10](-150,-150)(300,300)}%
%        {\color{zaxis}%
%          \put(0,0){\vector( 0, 1){100} }%
%          \put(0,0){\vector( 0,-1){100} }%
%          }%
%      \end{picture}%
%    }
%    \put(0,25){%
%      \setlength{\unitlength}{1\tw/(400*3)}%
%      \begin{picture}(0,0)(0,0)%
%        %{\color{graphpaper}\graphpaper[10](-150,-150)(300,300)}%
%        {\color{origin}%
%          \put(0,0){\circle*{15}}%
%          }%
%      \end{picture}%
%    }
%  \end{picture}
%  \end{center}
%  %\caption{
%  %  Euclidean 3-dimensional space partitioned as a progressive lattice
%  %  \label{fig:lat_E3d_power_progressive}
%  %  }
%  %\end{figure}
%\end{minipage}%
%
%
%%---------------------------------------
%% Euclidean 3-space partitioned by power lattice
%%---------------------------------------
%\begin{minipage}[c]{\tw/2}
%%---------------------------------------
%\begin{example}
%\label{ex:lat_E3d_linear}
%%---------------------------------------
%The figure to the left illustrates the subspace lattice
%\[ \oquad{\setn{\text{$0$, $x$-axis, $xy$-plane, $xyz$-space}}}{\subseteq}{\spu }{\spi }. \]
%The least upper bound of the lattice is the 3-dimensional Euclidean space.
%%The architecture of this subspace lattice is called the \hie{linear architecture} \xref{def:arch_linear}.
%\end{example}
%\end{minipage}%
%\begin{minipage}[c]{7\tw/16}%
%%\begin{figure}[th]
%\begin{center}
%\footnotesize
%\setlength{\unitlength}{\tw/440}%
%\begin{picture}(100,520)(-50,0)%
%  \thicklines
%  %{\color[gray]{0.4}\graphpaper[50](-50,0)(100,520)}%
%  {\color{picbox}%
%    %\put( -50,400){\framebox(100,100){}}%
%    \put( -50,250){\framebox(100,100){}}%
%    \put( -50,100){\framebox(100,100){}}%
%    \put( -25,  0){\framebox( 50, 50){}}%
%    }%
%  {\color{black}%
%    \put(   0,400){\line( 0,-1){ 50}}%
%    \put(   0,200){\line( 0, 1){ 50}}%
%    \put(   0, 50){\line( 0, 1){ 50}}%
%    }%
%  \put(0,450){%
%    \setlength{\unitlength}{2\tw/(450*3)}%
%    \thicklines
%    \color{red}%
%      \put(  50,  50){\line( 1, 1){50} }%
%      \put(  50,-100){\line( 1, 1){50} }%
%      \put(-100,  50){\line( 1, 1){50} }%
%    \color{green}%
%      \put(- 50, 100){\line( 1, 0){150} }%
%      \put(-100,  50){\line( 1, 0){150} }%
%      \put(-100,-100){\line( 1, 0){150} }%
%    \color{blue}%
%      \put( 100,- 50){\line( 0, 1){150} }%
%      \put(  50,-100){\line( 0, 1){150} }%
%      \put(-100,-100){\line( 0, 1){150} }%
%    }
%    \put(0,300){%
%      \setlength{\unitlength}{1\tw/(400*3)}%
%      \begin{picture}(0,0)(0,0)%
%        %{\color{graphpaper}\graphpaper[10](-150,-150)(300,300)}%
%        {\color{xaxis}%
%          \put(-150, -50){\line( 1, 1){100} }%
%          \put(  50, -50){\line( 1, 1){100} }%
%          }%
%        {\color{yaxis}%
%          \put(-150, -50){\line( 1, 0){200} }%
%          \put( 150,  50){\line(-1, 0){200} }%
%          }%
%      \end{picture}%
%    }
%  \put(0,150){%
%    \setlength{\unitlength}{1\tw/(400*3)}%
%    \begin{picture}(0,0)(0,0)%
%      %{\color{graphpaper}\graphpaper[10](-150,-150)(300,300)}%
%      {\color{xaxis}%
%        \put(0,0){\vector( 1, 1){100} }%
%        \put(0,0){\vector(-1,-1){100} }%
%        }%
%    \end{picture}%
%  }
%  \put(0,25){%
%    \setlength{\unitlength}{1\tw/(400*3)}%
%    \begin{picture}(0,0)(0,0)%
%      %{\color{graphpaper}\graphpaper[10](-150,-150)(300,300)}%
%      {\color{origin}%
%        \put(0,0){\circle*{15}}%
%        }%
%    \end{picture}%
%  }
%\end{picture}
%\end{center}
%%\caption{
%%  Euclidean 3-dimensional space partitioned as a power lattice
%%  \label{fig:lat_E3d_power}
%%  }
%%\end{figure}
%\end{minipage}%
%
%
%
%
%Subspace lattice architectures can also be illustrated using basis lattices.
%Such examples include the following:\\
%  \begin{tabular}{llll}
%    $\imark$ & \pref{ex:arch_d1}  & \prefpo{ex:arch_d1}  &---Haar basis / Daubechies-$p1$ basis  \\
%    $\imark$ & \pref{ex:arch_d2}  & \prefpo{ex:arch_d2}  &---Daubechechies-$p2$ basis \\
%    $\imark$ & \pref{ex:arch_cos4} & \prefpo{ex:arch_cos4} &---cosine harmonic basis \\
%    $\imark$ & \pref{ex:arch_cos^0-cos^3} & \prefpo{ex:arch_cos^0-cos^3} &---cosine exponential basis \\
%    $\imark$ & \pref{ex:arch_cheby4} & \prefpo{ex:arch_cheby4} &---Chebyshev basis
%  \end{tabular}
%
%%---------------------------------------
%% Euclidean 3-space partitioned by progressive lattice
%%---------------------------------------
%\begin{minipage}[c]{7\tw/16}
%%\begin{figure}[th]
%\begin{center}
%\footnotesize
%\setlength{\unitlength}{\tw/(550)}%
%\includegraphics{../common/math/graphics/pdfs/baslat_d1.pdf}
%\end{center}
%\end{minipage}%
%\hfill
%\begin{minipage}[c]{\tw/2}
%  %---------------------------------------
%  \begin{example}
%  \label{ex:arch_d1}
%  %---------------------------------------
%  The figure to the left illustrates a subspace lattice 
%  %with \hie{progressive} architecture and 
%  with a \hie{Haar} basis (\hie{Daubechies-$p1$} basis).
%\end{example}
%\end{minipage}
%
%
%
%\begin{minipage}[c]{7\tw/16}
%  \begin{center}
%  \begin{fsL}
%  \setlength{\unitlength}{\tw/550}%
%  \includegraphics{../common/math/graphics/pdfs/baslat_d2.pdf}%
%  \end{fsL}
%  \end{center}
%\end{minipage}%
%\hfill
%\begin{minipage}[c]{\tw/2}
%%---------------------------------------
%\begin{example}
%\label{ex:arch_d2}
%%---------------------------------------
%The figure to the left illustrates a subspace lattice 
%%with \hie{progressive} architecture and 
%with a \hie{Daubechies-$p2$} basis.
%\end{example}
%\end{minipage}
%
%
%\begin{minipage}[c]{9\tw/16}
%  \begin{center}%
%  \begin{fsL}%
%  \setlength{\unitlength}{\tw/550}%
%  \includegraphics{../common/math/graphics/pdfs/baslat_cosh.pdf}%
%  \end{fsL}%
%  \end{center}%
%\end{minipage}%
%\hfill%
%\begin{minipage}[c]{6\tw/16}%
%%---------------------------------------
%\begin{example}
%\label{ex:arch_cos4}
%%---------------------------------------
%The figure to the left illustrates a subspace lattice 
%%with \hie{primitive} architecture and 
%with a \hie{cosine harmonic basis}.
%\end{example}
%\end{minipage}%
%
%\begin{minipage}[c]{9\tw/16}
%  \begin{center}%
%  \begin{fsL}%
%  \setlength{\unitlength}{\tw/550}%
%  \includegraphics{../common/math/graphics/pdfs/baslat_cose.pdf}%
%  \end{fsL}%
%  \end{center}%
%\end{minipage}%
%\hfill%
%\begin{minipage}[c]{6\tw/16}
%%---------------------------------------
%\begin{example}
%\label{ex:arch_cos^0-cos^3}
%%---------------------------------------
%The figure to the left illustrates a subspace lattice 
%%with \hie{primitive} architecture and 
%with a \hie{cosine exponential basis}.
%\end{example}
%\end{minipage}
%
%
%\begin{minipage}[c]{10\tw/16}
%  \begin{center}
%  \begin{fsL}
%  \setlength{\unitlength}{3\tw/(4*550)}
%  \includegraphics{../common/math/graphics/pdfs/baslat_cheby.pdf}%
%  \end{fsL}
%  \end{center}
%\end{minipage}%
%\hfill%
%\begin{minipage}[c]{6\tw/16}
%%---------------------------------------
%\begin{example}
%\label{ex:arch_cheby4}
%%---------------------------------------
%The figure to the left illustrates a subspace lattice 
%%with \hie{primitive} architecture and 
%with a \hie{Chebyshev polynomial basis}.
%\end{example}
%\end{minipage}


%\paragraph{Residual subspaces.}
%Let $\spO$ be a linear space with subspaces $\spV_{n},\spX_{n+1}\subseteq\spX$.
%If $\spX_n\subseteq\spX_{n+1}$, then we could view $\spX_{n+1}$ as
%a ``larger" subspace with more information or more ``detail" than the
%``smaller" subspace $\spX_n$.
%In this case, we could also say that $\spX_{n+1}$ is $\spX_n$ plus some
%\hie{residual} (or \hie{complementary}) subspace $\spW_n$ containing the missing information or detail.
%We can express this mathematically as
%\[ \spX_{n+1} = \spX_n + \spW_n \]
%We say that $\spW_n$ complements $\spX_n$ in $\spX_{n+1}$.
%For example, wavelet systems use a sequence of \hie{scaling subspaces} $\seq{\spX_n}{n\in\Z}$
%and complementary \hie{wavelet subspace} $\seq{\spW_n}{n\in\Z}$.
%
%\paragraph{Orthogonal subspaces.}
%If $\spX_{n+1} = \spX_n + \spW_n$ and $\spX_n \orthog \spW_n$
%($\spX_n$ is orthogonal to $\spW_n$), then this is denoted by
%  \[ \spX_{n+1} = \spX_n \adds \spW_n \]
%Many transforms make use of orthogonal subspaces, including the Fourier series transform
%and the wavelet transform.
%
%
%%=======================================
%\subsubsection{Subspace Partitioning}
%\label{sec:tran_sub_part}
%%=======================================
%We have already seen that a single space may be analyzed using several 
%different {\em subspace architectures} as illustrated by \\
%\begin{tabular}{lllll}
%  $\imark$ & \pref{ex:lat_E3d_power}       & \prefpo{ex:lat_E3d_power}       &---power architecture \\
%  $\imark$ & \pref{ex:lat_E3d_progressive} & \prefpo{ex:lat_E3d_progressive} &---progressive architecture \\
%  $\imark$ & \pref{ex:lat_E3d_primitive}   & \prefpo{ex:lat_E3d_primitive}   &---primitive architecture
%\end{tabular}
%
%\hie{Subspace partitioning} refers to the way in which elements of a mathematical space
%are distributed across a subspace architecture.
%A mathematical space may have the {\em same} subspace architecture but 
%yet have {\em different} subspace partitioning.
%This concept is illustrated by \\
%\begin{tabular}{lllll}
%  $\imark$ & \pref{ex:arch_cos4} & \prefpo{ex:arch_cos4} &---\prop{cosine harmonic basis} \\
%  $\imark$ & \pref{ex:arch_cos^0-cos^3} & \prefpo{ex:arch_cos^0-cos^3} &---\prop{cosine exponential basis}
%\end{tabular}\\
%In these two examples, the spaces (the least upper bounds) are the same,
%the subspace architectures are the same,
%but yet the subspace partitionings (the elements contained in the corresponding subspaces)
%are different. 
%The fact that the spaces are the same is because there is a one to one mapping between
%harmonic cosines of order $n$ and exponential cosines of order 
%$n$.%
%\ifdochas{harTrig}{\footnote{\begin{tabular}[t]{llll}
%  $\imark$ & harmonic to polynomial conversion: & \prefp{thm:cosnx}  & \prefpo{thm:cosnx} \\
%  $\imark$ & polynomial to harmonic conversion: & \prefp{thm:cos^\xNx} & \prefpo{thm:cos^\xNx} 
%\end{tabular}}}
%
%It is also worth noting the following two examples:\\
%\begin{tabular}{lllll}
%  $\imark$ & \pref{ex:arch_d1} & \prefpo{ex:arch_d1} &---progressive architecture partitioned by Daubechies-$p1$ basis \\
%  $\imark$ & \pref{ex:arch_d2} & \prefpo{ex:arch_d2} &---progressive architecture partitioned by Daubechies-$p2$ basis
%\end{tabular}\\
%In these two examples, the subspace architectures are the {\em same},
%and the subspace partitionings are {\em different}
%(as in the previous pair of examples), 
%but unlike the previous pair of examples, the spaces (the least upper bounds)
%are {\em different}
%(and therefore the previous pair of examples would seem a little more significant
%when discussing the significance of subspace partitioning).
%
%
%\paragraph{Partitioning methods.}
%There are several ways to partition a space.
%
%  \begin{tabular}{>{$\imark$}lll}
%     & frequency        & Fourier transform
%  \\ & scaling          & wavelet transform
%  \\ & randomness       & pn-sequences, KL expansion
%  \\ & order            & Chebyshev polynomials
%  \\ & Hamming weight   & Walsh functions
%  \end{tabular}
%
%
%
%
%
%
%
%
%%=======================================
%\subsection {Classifications of subspace architectures}
%%=======================================
%The subspaces of a space $\spX$ form the lattice $(\spX, \subseteq, +, \spi )$.
%Next we look at several subspace architectures in terms of lattices of subspaces.
%Names are assigned to these architectures.
%But note that these namings are not necessarily used generally in the literature
%and may be somewhat unique to this document.
%
%\begin{minipage}[t]{5\tw/16}
%  \color{figcolor}
%  \begin{center}
%  \begin{fsL}
%  \setlength{\unitlength}{\textwidth/500}
%  \begin{picture}(400,800)(-50,-350)
%    %{\color{graphpaper}\graphpaper[50](-50,-350)(400,800)}
%    \thicklines
%    \put(0,-150){\line( 0, 1){360} }
%    \put(0, 350){\line( 0, 1){ 50} }
%    \put(0,-300){\line( 0, 1){ 50} }
%
%    \put(0, 400){\circle*{15}}
%    \put(0, 200){\circle*{15}}
%    \put(0, 100){\circle*{15}}
%    \put(0,   0){\circle*{15}}
%    \put(0,-100){\circle*{15}}
%    \put(0,-300){\circle*{15}}
%
%
%
%    \put(10, 400){\makebox(0,0)[l] {$\spX$}}
%    \put( 0, 310){\makebox(0,0)[c] {$\vdots$}}
%    \put(10, 200){\makebox(0,0)[l] {$\spV_2$}}
%    \put(10, 100){\makebox(0,0)[l] {$\spV_1$}}
%    \put(10,   0){\makebox(0,0)[l] {$\spV_0$}}
%    \put(10,-100){\makebox(0,0)[l] {$\spV_{-1}$}}
%    \put( 0,-200){\makebox(0,0)[c] {$\vdots$}}
%    \put(10,-300){\makebox(0,0)[l] {$\spZero$}}
%
%
%    {\color{red}
%      \put( 110, 400){\makebox(0,0)[l] {\parbox{8\tw/16}{\raggedright entire linear space (least upper bound)}}}
%      \put( 100, 400){\vector(-1,0){60}}
%      \put( 110,-300){\makebox(0,0)[l] {\parbox{8\tw/16}{\raggedright greatest lower bound}}}
%      \put( 100,-300){\vector(-1,0){60}}
%      \put( 150, 150){\vector(0,1){50}}
%      \put( 110,  50){\makebox(0,0)[bl] {\parbox{6\tw/16}{\raggedright larger subspaces}}}
%      \put( 110,   0){\makebox(0,0)[tl] {\parbox{6\tw/16}{\raggedright smaller subspaces}}}
%      \put( 150,-100){\vector(0,-1){50}}
%      %\put(- 50,-275){\dashbox{10}(100,650){}}
%      %\put( 110,  50){\makebox(0,0)[bl] {scaling}}
%      %\put( 110,  45){\makebox(0,0)[tl] {subspaces}}
%      %\put( 100,  50){\vector(-1,0){50}}
%    }
%  \end{picture}
%  \end{fsL}
%  \end{center}
%\end{minipage}
%\begin{minipage}[t]{11\tw/16}
%\paragraph{Linear architecture.}
%The \hie{linear architecture} is illustrated by the figure to the left and 
%defined in \prefpp{def:arch_linear}.
%The scaling subspace sequence together with the set inclusion relation $\subseteq$
%form the \hie{linearly ordered set} $(\seqn{\spX_n}, \subseteq)$, illustrated
%to the left by its Hasse diagram.
%Subspaces $\spX_n$ increase in ``size" with increasing $n$.
%That is, they contain more and more functions for larger and larger $n$---
%with $\spX$ (largest $n$) containing all the vectors in the space
%and $\spZero$ (smallest $n$) containing only the $\vzero$ vector.
%Many lattices contain the linear architecture as one or more sublattices.
%\end{minipage}
%
%%--------------------------------------
%\begin{definition}
%\label{def:arch_linear}
%%--------------------------------------
%\defbox{\begin{array}{l@{\qquad}l>{\ds}l>{(}D<{)}}
%  \mc{4}{l}{\parbox{14\tw/16}{
%    A lattice subspace $(\seq{\spX_n}{n\in\Z},\, \subseteq,\, +,\, seti)$ 
%    has \hid{linear architecture} if it satisfies the following conditions:
%    }}
%  \\
%  & 1. & \spX_n          \subsetneq \spX_{n+1}
%       & linearly ordered wrt $\subseteq$  \\
%  & 2. & \sum_{n\in\Z} \spX_n = \spV_\infty
%       & least upper bound is $\spV_\infty$ \\
%  & 3. & \sum_{n\in\Z} \spX_n = \setn{\vzero}
%       & greatest lower bound is $\vzero$ \\
%\end{array}}
%\end{definition}
%
%An example of a subspace lattice with the linear architecture is
%\prefpp{ex:lat_E3d_linear}.
%%\begin{tabular}[t]{llll}
%%  $\imark$ & \pref{ex:lat_E3d_linear}  & \prefpo{ex:lat_E3d_linear}
%%            &---Euclidean 3-d space \\
%%  $\imark$ & \pref{ax:mra}  & \prefpo{ax:mra} &---Multiresolution Analysis 
%%\end{tabular}
%
%
%\begin{minipage}[c]{8\tw/16}
%  \color{figcolor}
%  \begin{center}
%  \begin{fsL}
%  \setlength{\unitlength}{\textwidth/800}
%  \begin{picture}(600,500)(-300,-250)
%    %{\color{graphpaper}\graphpaper[100](-300,-250)(600,500)}
%    \thicklines
%    \put(-200,   0){\line( 1, 1){200} }
%    \put(-100,   0){\line( 1, 2){100} }
%    \put(   0,   0){\line( 0, 1){200} }
%    \put( 200,   0){\line(-1, 1){200} }
%
%    \put(-200,   0){\line( 1,-1){200} }
%    \put(-100,   0){\line( 1,-2){100} }
%    \put(   0,   0){\line( 0,-1){200} }
%    \put( 200,   0){\line(-1,-1){200} }
%
%    \put(   0, 200){\circle*{15}}
%    \put(-200,   0){\circle*{15}}
%    \put(-100,   0){\circle*{15}}
%    \put(   0,   0){\circle*{15}}
%    \put( 100,   0){$\cdots$}
%    \put( 200,   0){\circle*{15}}
%    \put(   0,-200){\circle*{15}}
%
%    \put(  0, 210){\makebox(0,0)[b] {$\spO=\spX_0+\spX_1+\cdots+\spX_{n-1}$}}
%    \put(-210,   0){\makebox(0,0)[r] {$\spX_0$}}
%    \put(-110,   0){\makebox(0,0)[r] {$\spX_1$}}
%    \put(  10,   0){\makebox(0,0)[l] {$\spX_2$}}
%    \put( 210,   0){\makebox(0,0)[l] {$\spX_{n-1}$}}
%    \put( -10,-210){\makebox(0,0)[tl] {$\spZero=\spX_1 \cap \spX_2 \cap \cdots \cap \spX_{n-1}$}}
%    %\put( -10,-210){\makebox(0,0)[tl] {$\ds\spZero=\setopi_{n=1}^\xN \spW_n$}}
%    {\color{red}
%      \put(-300, -50){\dashbox{10}(600,100){}}
%      \put( 300, 110){\makebox(0,0)[br] {basis subspaces}}
%      \put( 200, 100){\vector(0,-100){50}}
%    }
%  \end{picture}
%  \end{fsL}
%  \end{center}
%\end{minipage}
%\begin{minipage}[c]{8\tw/16}
%  The \hie{primitive architecture} is illustrated by the Hasse diagram to the left.
%  %and defined in \prefpp{def:arch_primitive}.
%  Many transforms that use only orthogonal subspaces have the primitive architecture.
%\end{minipage}
%
% Examples of subspace lattices with the primitive architecture include
%\\\indentx\begin{tabular}[t]{llll}
%  $\imark$ & \pref{ex:lat_E3d_primitive}  & \prefpo{ex:lat_E3d_primitive} 
%            &---Euclidean 3-d space \\
%  $\imark$ & \pref{ex:arch_cos4} & \prefpo{ex:arch_cos4} &---cosine harmonic basis \\
%  $\imark$ & \pref{ex:arch_cos^0-cos^3} & \prefpo{ex:arch_cos^0-cos^3} &---cosine exponential basis \\
%  $\imark$ & \pref{ex:arch_cheby4} & \prefpo{ex:arch_cheby4} &---Chebyshev basis
%\end{tabular}
%
%%%---------------------------------------
%%\begin{definition}
%%\label{def:arch_primitive}
%%%---------------------------------------
%%\defbox{\begin{array}{l@{\qquad}l>{\ds}l>{(}D<{)}}
%%  \mc{4}{l}{\parbox{14\tw/16}{
%%    A lattice subspace $(\seq{\spX_n}{n\in\Z},\, \subseteq,\, +,\, seti)$ 
%%    has \hid{primitve architecture} if it satisfies the following conditions:
%%    }}
%%  \\
%%  & 1. & 
%%       & \\
%%  & 2. & 
%%       & 
%%\end{array}}
%%\problem
%%\end{definition}
%
%
%
%
%
%
%\begin{minipage}[c]{9\tw/16}
%  \color{figcolor}
%  \begin{center}
%  \begin{fsL}
%  \setlength{\unitlength}{\textwidth/700}
%  \begin{picture}(600,500)(-300,-200)
%    %{\color{graphpaper}\graphpaper[100](-300,-200)(600,500)}
%    \thicklines
%    \put(-200,   0){\line( 1, 1){120} }
%    \put(   0, 200){\line(-1,-1){ 40} }
%    \put(-100,   0){\line(-1, 1){ 50} }
%    \put(   0,   0){\line(-1, 1){100} }
%    \put( 200,   0){\line(-1, 1){200} }
%
%    \put(-200,   0){\line( 1,-1){200} }
%    \put(-100,   0){\line( 1,-2){100} }
%    \put(   0,   0){\line( 0,-1){200} }
%    \put( 200,   0){\line(-1,-1){200} }
%
%    \put(   0, 200){\circle*{15}}  %maximum element
%    \put(   0,-200){\circle*{15}}  %minimum element
%
%    \put(-150,  50){\circle*{15}}
%    \put(-100, 100){\circle*{15}}
%
%    \put(-200, 0){\circle*{15}}
%    \put(-100, 0){\circle*{15}}
%    \put(   0, 0){\circle*{15}}
%    \put( 100, 0){$\cdots$}
%    \put( 200, 0){\circle*{15}}
%
%    \put(   0, 210){\makebox(0,0)[b] {$\spX_n$}}
%    \put(-110, 100){\makebox(0,0)[r] {$\spX_3$}}
%    \put(-160,  50){\makebox(0,0)[r] {$\spX_2$}}
%    \put(-210,   0){\makebox(0,0)[r] {$\spX_1$}}
%    \put(-110,   0){\makebox(0,0)[r] {$\spY_1$}}
%    \put(  10,   0){\makebox(0,0)[l] {$\spY_2$}}
%    \put( 210,   0){\makebox(0,0)[l] {$\spY_{n-1}$}}
%    \put(  . ,-210){\makebox(0,0)[t] {$\spZero$}}
%
%    \put( -50, 15){\makebox(0,0)[c] {$\ddots$}}
%    {\color{red}
%      \put(-300,- 50){\dashbox{10}(600,80){}}
%      \put(-300, 260){\makebox(0,0)[bl] {linearly ordered subspaces}}
%      \put( 300, 110){\makebox(0,0)[br] {basis subspaces}}
%      \put( 200, 100){\vector(0,-1){70}}
%      \put(-250, 250){\vector(1,-1){70}}
%      \thicklines
%      \qbezier[30](-300,  50)(-175, 175)( -50, 300)
%      \qbezier[30](-200, -50)( -75,  75)(  50, 200)
%      \qbezier[15](-200,-50)(-250,0)(-300,50)
%      \qbezier[15](  50,200)(0,250)(-50,300)
%    }
%  \end{picture}
%  \end{fsL}
%  \end{center}
%\end{minipage}
%\begin{minipage}[c]{7\tw/16}
%  The \hie{progressive architecture} is illustrated by the Hasse diagram to the left.
%  %and defined in \prefpp{def:arch_progressive}.
%  Note that the left hand portion of the lattice contains a sublattice with 
%  \hie{linear architecture}.\footnotemark
%\end{minipage}
%\footnotetext{
%  The progressive architecture was essentially proposed by Burt and Adelson in 1983.
%  In their paper, the sequence $\seqn{\spX_n}$ is called a \hie{Gaussian pyramid}
%  and the sequence $\seqn{\spW_n}$ is called a 
%  \hie{Laplacian pyramid}. Reference:\cite{burt1983}.
%  }
%
% Examples of subspace lattices with the progressive architecture include
%\\\indentx\begin{tabular}[t]{llll}
%  $\imark$ & \pref{ex:lat_E3d_progressive}  & \prefpo{ex:lat_E3d_progressive} 
%            &---Euclidean 3-d space \\
%  $\imark$ & \pref{ex:arch_d1} & \prefpo{ex:arch_d1} &---Haar basis lattice \\
%  $\imark$ & \pref{ex:arch_d2} & \prefpo{ex:arch_d2} &---Daubechies-$p2$ basis lattice 
%\end{tabular}
%
%%%---------------------------------------
%%\begin{definition}
%%\label{def:arch_progressive}
%%%---------------------------------------
%%\defbox{\begin{array}{l@{\qquad}l>{\ds}l>{(}D<{)}}
%%  \mc{4}{l}{\parbox{14\tw/16}{
%%    A lattice subspace $(\seq{\spX_n}{n\in\Z},\, \subseteq,\, +,\, seti)$ 
%%    has \hid{progressive architecture} if it satisfies the following conditions:
%%    }}
%%  \\
%%  & 1. & 
%%       & \\
%%  & 2. & 
%%       & 
%%\end{array}}
%%\problem
%%\end{definition}
%
%
%
%
%\begin{minipage}{8\tw/16}
%  \color{figcolor}
%  \begin{center}
%  \begin{fsL}
%  \setlength{\unitlength}{\textwidth/6}
%  \begin{picture}(2,3)(-1,0)
%    %{\color{graphpaper}\graphpaper[1](-1,0)(2,3)}
%    \thicklines
%    \put( 1, 2){\line(-1, 1){1} }
%    \put( 0, 2){\line( 0, 1){1} }
%    \put(-1, 2){\line( 1, 1){1} }
%    \put( 1, 1){\line( 0, 1){1} }
%    \put( 1, 1){\line(-1, 1){1} }
%    \put( 0, 1){\line(-1, 1){1} }
%    \put( 0, 1){\line( 1, 1){1} }
%    \put(-1, 1){\line( 0, 1){1} }
%    \put(-1, 1){\line( 1, 1){1} }
%    \put( 0, 0){\line(-1, 1){1} }
%    \put( 0, 0){\line( 0, 1){1} }
%    \put( 0, 0){\line( 1, 1){1} }
%
%    \put( 0, 3){\circle*{0.15}}
%    \put( 1, 2){\circle*{0.15}}
%    \put( 0, 2){\circle*{0.15}}
%    \put(-1, 2){\circle*{0.15}}
%    \put( 1, 1){\circle*{0.15}}
%    \put( 0, 1){\circle*{0.15}}
%    \put(-1, 1){\circle*{0.15}}
%    \put( 0, 0){\circle*{0.15}}
%
%    {\color{red}
%      \put(-1.5, 0.75){\dashbox{0.10}(3,0.5){}}
%      \put( 1.5, 1.6){\makebox(0,0)[br] {basis subspaces}}
%      \put( 1.25, 1.55){\vector(0,-1){0.3}}
%    }
%
%    %\put(   0, 310){\makebox(0,0)[b] { $\setn{a,b,c}$ }}
%    %\put( 1, 2){\makebox(0,0)[bl]{ $\setn{  b,c}$ }}
%    %\put(   0, 2){\makebox(0,0)[bl]{ $\setn{a,  c}$ }}
%    %\put(-1, 2){\makebox(0,0)[br]{ $\setn{a,b  }$ }}
%    %\put( 1, 1){\makebox(0,0)[tl]{ $\setn{c    }$ }}
%    %\put(   0, 1){\makebox(0,0)[tl]{ $\setn{  b  }$ }}
%    %\put(-1, 1){\makebox(0,0)[tr]{ $\setn{    a}$ }}
%    %\put(   0, -10){\makebox(0,0)[t] { $\emptyset   $ }}
%  \end{picture}
%  \end{fsL}
%  \end{center}
%\end{minipage}
%\begin{minipage}{8\tw/16}
%  %---------------------------------------
%  %\begin{definition}
%  %---------------------------------------
%  The \hid{power architecture} is illustrated by the Hasse diagram to the left.
%  %\end{definition}
%  The power architecture is noteable because it is also the lattice formed by
%  the power set $\psetx$ of a set $\setX$.
%  Therefore it represents all the possible combinations of subspaces.
%\end{minipage}
%
%
%
%\begin{minipage}{8\tw/16}
%  \color{figcolor}
%  \begin{center}
%  \begin{fsL}
%  \setlength{\unitlength}{\textwidth/18}
%  \begin{picture}(16,9)(-8,-2)
%    %{\color{graphpaper}\graphpaper[1](-8,-2)(16,9)}
%    \thicklines
%    \put( 2, 1){\line( 1, 1){2} }
%    \put(-2, 1){\line(-1, 1){2} }
%
%    \put( 3,   0){\line(-1, 1){1} }
%    \put( 1,   0){\line( 1, 1){1} }
%    \put(-1,   0){\line(-1, 1){1} }
%    \put(-3,   0){\line( 1, 1){1} }
%
%    \put( 7,   0){\line(-1, 1){7} }
%    \put( 5,   0){\line( 1, 1){1} }
%    \put( 3,   0){\line(-1, 1){1} }
%    \put( 1,   0){\line( 1, 1){1} }
%    \put(-1,   0){\line(-1, 1){1} }
%    \put(-3,   0){\line( 1, 1){1} }
%    \put(-5,   0){\line(-1, 1){1} }
%    \put(-7,   0){\line( 1, 1){7} }
%
%    \qbezier(-7,0)(-3.5,-1)(0,-2)
%    \qbezier( 7,0)( 3.5,-1)(0,-2)
%
%    \put(-5,   0){\line( 5,-2){5} }
%    \put(-3,   0){\line( 3,-2){3} }
%    \put(-1,   0){\line( 1,-2){1} }
%    \put( 1,   0){\line(-1,-2){1} }
%    \put( 3,   0){\line(-3,-2){3} }
%    \put( 5,   0){\line(-5,-2){5} }
%
%    \put(   0, 7){\circle*{0.30}}  %maximum element
%
%    \put( 4, 3){\circle*{0.30}}
%    \put(-4, 3){\circle*{0.30}}
%    \put( 6, 1){\circle*{0.30}}
%    \put( 2, 1){\circle*{0.30}}
%    \put(-2, 1){\circle*{0.30}}
%    \put(-6, 1){\circle*{0.30}}
%
%    \put(-7,   0){\circle*{0.30}}
%    \put(-5,   0){\circle*{0.30}}
%    \put(-3,   0){\circle*{0.30}}
%    \put(-1,   0){\circle*{0.30}}
%    \put( 1,   0){\circle*{0.30}}
%    \put( 3,   0){\circle*{0.30}}
%    \put( 5,   0){\circle*{0.30}}
%    \put( 7,   0){\circle*{0.30}}
%
%    \put(   0,-2){\circle*{0.30}}  %minimum element
%
%  %  \put(   0, 210){\makebox(0,0)[b] {$\spLL$}}
%  %  \put(-210,   0){\makebox(0,0)[r] {$\spW_0$}}
%  %  \put(-110,   0){\makebox(0,0)[r] {$\spW_1$}}
%  %  \put(  10,   0){\makebox(0,0)[l] {$\spW_2$}}
%  %  \put( 210,   0){\makebox(0,0)[l] {$\spW_3$}}
%  %  \put(   0,-210){\makebox(0,0)[t] {$0$}}
%    {\color{red}
%      \put(-8, -0.5){\dashbox{0.2}(16,1){}}
%      \put( 0, 2.1){\makebox(0,0)[b] {bases subspaces}}
%      \put( 0, 2){\vector(0,-1){1.5}}
%    }
%  \end{picture}
%  \end{fsL}
%  \end{center}
%\end{minipage}
%\begin{minipage}{8\tw/16}
%  %---------------------------------------
%  %\begin{definition}
%  %---------------------------------------
%  The \hid{dyadic architecture} is illustrated by the Hasse diagram to the left.
%  %\end{definition}
%  In the dyadic architecture,
%  every subspace higher than the basis subspaces are the sum of two other subspaces
%  below it.
%  That is, these subspaces repeatedly split into two smaller subspaces.
%\end{minipage}
%
%
%\begin{minipage}{8\tw/16}
%  \color{figcolor}
%  \begin{center}
%  \begin{fsL}
%  \setlength{\unitlength}{\textwidth/18}
%  \begin{picture}(16,9)(-8,-2)
%    %{\color{graphpaper}\graphpaper[1](-8,-2)(16,9)}
%    \thicklines
%    \put( 7,   0){\line(-1, 1){7} }
%    \put( 5,   0){\line( 1, 1){1} }
%    \put( 3,   0){\line( 1, 3){1} }
%    \put( 1,   0){\line( 1, 5){1} }
%    \put(-1,   0){\line(-1, 5){1} }
%    \put(-3,   0){\line(-1, 3){1} }
%    \put(-5,   0){\line(-1, 1){1} }
%    \put(-7,   0){\line( 1, 1){7} }
%
%    \qbezier(-7,0)(-3.5,-1)(0,-2)
%    \qbezier( 7,0)( 3.5,-1)(0,-2)
%    \put(-5,   0){\line( 5,-2){5} }
%    \put(-3,   0){\line( 3,-2){3} }
%    \put(-1,   0){\line( 1,-2){1} }
%    \put( 1,   0){\line(-1,-2){1} }
%    \put( 3,   0){\line(-3,-2){3} }
%    \put( 5,   0){\line(-5,-2){5} }
%
%    \put(   0, 7){\circle*{0.30}}  %maximum element
%
%    \put( 6, 1){\circle*{0.30}}
%    \put( 4, 3){\circle*{0.30}}
%    \put( 2, 5){\circle*{0.30}}
%    \put(-2, 5){\circle*{0.30}}
%    \put(-4, 3){\circle*{0.30}}
%    \put(-6, 1){\circle*{0.30}}
%
%    \put(-7,   0){\circle*{0.30}}
%    \put(-5,   0){\circle*{0.30}}
%    \put(-3,   0){\circle*{0.30}}
%    \put(-1,   0){\circle*{0.30}}
%    \put( 1,   0){\circle*{0.30}}
%    \put( 3,   0){\circle*{0.30}}
%    \put( 5,   0){\circle*{0.30}}
%    \put( 7,   0){\circle*{0.30}}
%
%    \put(   0,-2){\circle*{0.30}}  %minimum element
%
%    {\color{red}
%      \put(-8, -0.5){\dashbox{0.2}(16,1){}}
%      \put( 0, 2.1){\makebox(0,0)[b] {bases subspaces}}
%      \put( 0, 2){\vector(0,-1){1.5}}
%    }
%  \end{picture}
%  \end{fsL}
%  \end{center}
%\end{minipage}
%\begin{minipage}{8\tw/16}
%  %---------------------------------------
%  %\begin{definition}
%  %---------------------------------------
%  The \hid{symmetric progressive architecture} is illustrated by the Hasse diagram to the left.
%  %\end{definition}
%  In the symmetric progressive architecture is similar to the progressive architecture
%  except that it is symmetric and complemented.
%\end{minipage}
%
%
%%---------------------------------------
%\begin{theorem}
%\label{thm:mra_lattice}
%%---------------------------------------
%\formbox{\begin{tabular}{p{\tw/8}|| *{6}{c|}}
%  architecture            & modular  & distributive & complemented & topology & ring         & algebra         \\
%                          &          &              &              &          & of sets      & of sets         \\
%  \hline
%  primitive               & $\checkmark$ &         & $\checkmark$  & $\checkmark$  &          &            \\
%  progressive             &         &         &          &          &          &            \\
%  power                   & $\checkmark$ & $\checkmark$ & $\checkmark$  & $\checkmark$  & $\checkmark$  & $\checkmark$    \\
%  dyadic                  & $\checkmark$ & $\checkmark$ & $\checkmark$  &          &          &            \\
%  symmetric progressive   & $\checkmark$ & $\checkmark$ & $\checkmark$  &          &          &            \\
%\end{tabular}}
%\end{theorem}
%\begin{proof}
%\begin{enumerate}
%\item The wavelet operator lattice contains the N5 sublattice.
%  \ifdochaselse{latvar}{By \prefpp{thm:latd_char_n5m3}}{Therefore}, 
%  the wavelet operator is non-modular and non-distributive.
%
%\item For the lattice to be complemented,
%  there must exist $\spW_{n-2}^c$ such that
%  $\spW_{n-2} \adds \spW_{n-2}^c = \spX_n$.
%  However, this subspace does not exist in the MRA.
%\end{enumerate}
%\end{proof}
%
%
%
%
%
%
%%======================================
%%\subsection{linear spaces of subspaces}
%%======================================
%
%

\fi

%=======================================
\section{Subspace Metrics}
%=======================================
%---------------------------------------
\begin{definition}[Hilbert space gap metric]
\footnote{
  \citerp{deza2006}{235},
  \citerp{akhiezer1e}{69},
  \citerp{berkson1963}{8},
  \citor{krein1947}
  %\citer{sznagy1946}
  }
\index{gap metric}
\index{metrics!gap}
\index{opening}
\index{aperature}
%---------------------------------------
Let $\spX$ be a {\bf Hilbert space} and $\ssetS$ the set of subspaces of $\spX$.
Then we define the following metric between subspaces of $\spX$.
\defbox{\begin{array}{rc>{\ds}lCD}
  \metric{\spV}{\spW} &\eqd& \norm{\opP - \opQ} 
    & \forall \spV,\spW\in\ssetS 
    & \parbox[t]{8\tw/16}{(the distance between subspaces $\spV$ and $\spW$ is the size of the difference of their projection operators)}
    \\
  \text{where }  \spV &\eqd& \opP\spX 
    & 
    & ($\opP$ is the projection operator that generates the subspace $\spV$) 
    \\
  \text{and }    \spW &\eqd& \opQ\spX 
    & 
    & ($\opQ$ is the projection operator that generates the subspace $\spW$).
\end{array}}
\end{definition}

%---------------------------------------
\begin{definition}[Banach space gap metric]
\footnote{
  \citerp{akhiezer1e}{70},
  \citerp{berkson1963}{8},
  \citor{krein1948}
  %\citer{sznagy1946}
  }
\index{gap metric}
\index{metrics!gap}
\index{aperature}
\index{opening}
%---------------------------------------
Let $\spX$ be a {\bf Banach space} and $\ssetS$ the set of subspaces of $\spX$.
%Let $\metrican$ be a metric from a point to a set of points.
Then we define the following metric between subspaces of $\spX$.
\defbox{\begin{array}{rc>{\ds}lC}
  \metric{\spV}{\spW} &\eqd& \max\setn{\sup_{\vv\in\setV,\norm{\vv}=1}\metrica{\vv}{\spW},\;\sup_{\vw\in\setW,\norm{\vw}=1}\metrica{\vw}{\spV}}
    & \forall \spV,\spW\in\ssetS 
    \\
  \text{where }  \metrica{\vv}{\spW} &\eqd& \inf_{\vw\in\setW} \norm{\vv-\vw}
      \qquad\text{\scriptsize(metric from the point $\vv$ to the subspace $\spW$)}
\end{array}}
\end{definition}


%---------------------------------------
\begin{definition}[Sch{/:a}ffer's metric]
\footnote{
  \citerpp{massera1958}{562}{563},
  \citerpp{berkson1963}{7}{8}
  }
\label{def:sub_schaffer}
\index{Sch{/:a}ffer's metric}
\index{metrics!Sch{/:a}ffer}
%---------------------------------------
\defbox{\begin{array}{rcl}
  \metric{\spV}{\spW} &=& \log\brp{1 + \max\setn{\fr(\spV,\spW),\,\fr(\spW,\spV)}} \qquad\text{ where}\\
  \fr(\spV,\spW)      &\eqd& 
    \brbl{\begin{array}{ll}
      \inf\set{\norm{\opA-\opI}}{\opA\opV = \opW}             & \text{if $\opA$ and $\opAi$ both exist}  \\
      1                                                       & \text{otherwise}
    \end{array}}
  \end{array}}
\end{definition}

%=======================================
\section{Literature}
%=======================================
\begin{survey}
\begin{enumerate}
  \item Lattice of subspaces
    \\\citer{birkhoffjvn1936}
    \\\citer{husimi1937}
    \\\citer{sasaki1954}
    \\\citer{loomis1955}
    \\\citer{vonNeumann1960}
    \\\citer{holland1970}
    \\\citer{halmos1998}
    \\\citer{amemiya1966}
    \\\citer{gudder1979}
    \\\citer{gudder2005}

  \item Characterizations of lattice of Hilbert subspaces (cf \citerpg{iturrioz1985}{60}{0821850768}):
    \\\citerc{kakutani1946}{using Banach spaces}
    \\\citerc{piron1964}{using pre-Hilbert spaces}
    \\\indentdr\citerc{piron1964e}{using pre-Hilbert spaces}
    \\\citerc{amemiya1966}{using pre-Hilbert spaces}
    \\\citerc{wilbur1975}{using locally convex spaces}

  %\item Orthoarguesian law:
    %\\\citer{day197x}

  \item Metrics on subspaces:\\
    \citer{burago}
    
\end{enumerate}
\end{survey}


