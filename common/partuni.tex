%============================================================================
% Daniel J. Greenhoe
% XeLaTeX file
%============================================================================



%=======================================
\chapter{Partition of Unity}
\label{chp:pounity}
%=======================================

%=======================================
\section{Definition and motivation}
%=======================================
\begin{figure}[ht] \color{figcolor}
\begin{center}
  \psset{
    doubleline=true,
    nodesep=2mm,
    }
  \begin{tabular}{ccc}
                                & \rnode{pounityp}{partition of unity} &\\
  \rnode{pounityq}{quadrature}  & \parbox[][30mm][]{30mm}{\mbox{}}     & \rnode{pounityz}{zero at $z=-1$}\\
                                & \rnode{pounityo}{orthonormality}
  \end{tabular} %{thm:pun_zero}
  \ncline{->}{pounityo}{pounityp}
  \ncline{->}{pounityo}{pounityq}
  \ncline{->}{pounityo}{pounityz}
  \ncline{->}{pounityq}{pounityz}
  \ncline{->}{pounityq}{pounityp}
  \ncline{<->}{pounityp}{pounityz}
\end{center}
\caption{
   Implications of scaling function properties
   \label{fig:phi_prop}
   }
\end{figure}

A very common property of scaling functions \xrefP{def:mra} is the \prope{partition of unity} property \xrefP{def:pun}.
The partition of unity is a kind of generalization of \prope{orthonormality}\ifsxref{vsinprod}{def:orthog};
that is, \emph{all} orthonormal scaling functions form a partition of unity\ifsxref{ortho}{thm:quad=>zero_unity}.
But the partition of unity property is not just a consequence of orthonormality, but
also a generalization of orthonormality, in that if you remove the
orthonormality constraint, the partition of unity is still a
reasonable constraint in and of itself.

There are two reasons why the partition of unity property is a reasonable
constraint on its own:
\begin{liste}
   \item Without a partition of unity, it is difficult to represent a function as simple as a
         constant.\footnote{\citePp{jawerth}{8}}

  \item For a multiresolution system $\mrasys$, the partition of unity property is equivalent to
        $\sum_{n\in\Z} (-1)^n h_n = 0$ \xrefP{thm:pun_zero}.
        As viewed from the perspective of \hi{discrete time signal processing} \xref{app:dsp},
        %\footnote{\hie{discrete time signal processing}: see \prefpp{app:dsp}.} 
        this implies that the scaling coefficients form a ``\structe{lowpass filter}"; 
        lowpass filters provide a kind of ``coarse approximation"
        of a function. And that is what the scaling function is ``supposed" to do---to provide a coarse approximation 
        at some resolution or ``scale" \xrefP{def:mra}.
\end{liste}

%-------------------------------------
\begin{definition}
\index{partition of unity}
\footnote{
  \citerpg{kelley1955}{171}{0387901256},
  \citerpg{munkres2000}{225}{0131816292},
  \citerpg{klaus1984}{116}{0387908927},
  \citerpc{willard1970}{152}{item 20C},
  \citerpgc{willard2004}{152}{0486434796}{item 20C}
  }
\label{def:pun}
%-------------------------------------
\defbox{\begin{array}{M} 
  {A function $\ff\in\spRR$ forms a \hid{partition of unity} if}
  \\\indentx$\ds\sum_{n\in\Z} \opTrn^n\ff(x) = 1 \qquad \forall x\in\R$.
\end{array}}
\end{definition}

%=======================================
\section{Results}
%=======================================
%--------------------------------------
\begin{theorem} %[Partition of unity]
\footnote{
  \citePp{jawerth}{8}
  }
\label{thm:pounity_freq}
%--------------------------------------
Let $\mrasys$ be a multiresolution system \xrefP{def:mrasys}. % wavelet system.
Let $\opFT\ff(\omega)$ be the \fncte{Fourier transform} \xref{def:opFT} of a function $\ff\in\spLLR$.
Let $\kdelta_n$ be the \fncte{Kronecker delta function}\ifsxref{relation}{def:kdelta}.
%Let $c$ be a contant in $\R$.
\thmbox{
  \mcom{\sum_{n\in\Z} \opTrn^n \ff = c}
       {\prope{partition of unity} in ``time"}
   \quad\iff\quad
   \mcom{\brs{\opFT\ff}(2\pi n) = \kdelta_n}{\prope{partition of unity} in ``frequency"}
  }
\end{theorem}
\begin{proof}
Let $\Ze$ be the set of even integers and $\Zo$ the set of odd integers.
\begin{enumerate}
  \item Proof for ($\implies$) case:
    \begin{align*}
       c
         &= \sum_{m\in\Z} \opTrn^m\ff(x)
         && \text{by left hypothesis}
       \\&= \sum_{m\in\Z} \ff(x-m)
         && \text{by definition of $\opTrn$} && \text{\xref{def:opT}}
       \\&= \sqrt{2\pi}\,\sum_{m\in\Z} \Ff(2\pi m) e^{i2\pi mx}
         && \text{by \thme{PSF}} && \text{\xref{thm:psf}}
       \\&= \mcom{\sqrt{2\pi}\,\Ff(2\pi n)e^{i2\pi nx}}{real and constant for $n=0$}
          + \mcom{\sqrt{2\pi}\,\sum_{m\in\Z\setd n} \Ff(2\pi m) e^{i2\pi mx}}{complex and non-constant}
       \\&\quad\implies
       \sqrt{2\pi}\Ff(2\pi n) = c\kdelta_n
         && \mathrlap{\text{because $c$ is real and constant for all $x$}}
    \end{align*}

  \item Proof for ($\impliedby$) case:
  %\item Proof that $\brs{\sum_{n\in\Z} \opTrn^n\ff(x)=c} \impliedby \brs{\Ff(2\pi n) = \frac{c}{\sqrt{2\pi}}\kdelta_n}$:
    \begin{align*}
      \sum_{n\in\Z} \opTrn^n\ff(x)
         &= \sum_{n\in\Z} \ff(x -n)
         && \text{by definition of $\opTrn$} && \text{ \xref{def:opT}}
       \\&= \sqrt{2\pi}\,\sum_{n\in\Z} \Ff(2\pi n) e^{-i2\pi nx}
         && \text{by \thme{PSF}} && \text{\xref{thm:psf}}
       \\&= \sqrt{2\pi}\,\sum_{n\in\Z} \frac{c}{\sqrt{2\pi}}\kdelta_n e^{-i2\pi nx}
         && \text{by right hypothesis}
       \\&= \sqrt{2\pi}\, \frac{c}{\sqrt{2\pi}} e^{-i2\pi 0x}
         && \text{by definition of $\kdelta_n$} && \text{\xref{def:kdelta}}
       \\&= c
    \end{align*}
\end{enumerate}
\end{proof}

%-------------------------------------
\begin{corollary}
\label{cor:pun_pulse}
%-------------------------------------
\corbox{
  \brb{\begin{array}{M}
    $\exists \fg\in\spLLR$ such that\\
    $\ff(x) = \setind_\intco{-1}{1}(x) \conv \fg(x)$
  \end{array}}
  %\qquad\impnotimpby\qquad
  \qquad\implies\qquad
  \brb{\begin{array}{M}
    $\ff(x)$ generates\\
    a \prope{partition of unity}
  \end{array}}
  }
\end{corollary}
\begin{proof}
%\begin{enumerate}
%  \item Proof for $\implies$ statement:
    \begin{align*}
      \ff(x) = \setind_\intco{0}{1}(x) \conv \fg(x)
        &\implies \Ff(\omega) = \opFT\brs{\setind_\intco{-1}{1}}(\omega) \Fg(\omega)
        && \text{by \thme{convolution theorem}} && \text{\xref{thm:conv}}
      \\&\iff \Ff(\omega) = \cft\frac{\sin(\omega)}{\omega} \Fg(\omega)
        && \text{by \exme{rectangular pulse} ex.} && \text{\xref{ex:ft_pulse}}
      \\&\implies \Ff(2\pi n) = 0
      \\&\iff \text{$\ff(x)$ generates a \prope{partition of unity}}
        &&    \text{by \prefp{thm:pounity_freq}}
    \end{align*}

%  \item Proof for $\notimpliedby$ statement: 
%\end{enumerate}
\end{proof}

%=======================================
\section{Examples}
%=======================================
%-------------------------------------
\begin{example}
\label{ex:pun_Bspline}
%-------------------------------------
All \fncte{B-splines} \xref{def:Bspline} form a partition of unity \xref{thm:bspline_Nprop}.
All B-splines of order $n=1$ or greater can be generated by convolution with a \fncte{pulse} function,
similar to that specified in \prefpp{cor:pun_pulse} and as illustrated below:
\\\indentx\tbox{\includegraphics{graphics/n0punity.pdf}}\quad\xref{ex:n0_pounity}
\\\indentx\tbox{\includegraphics{graphics/n1punity.pdf}}\quad\xref{ex:n1_pounity}
\\\indentx\tbox{\includegraphics{graphics/n2punity.pdf}}\quad\xref{ex:n2_pounity}
\end{example}

%-------------------------------------
\begin{example}
\label{ex:pun_cos2}
%-------------------------------------
Let a function $\ff$ be defined in terms of the cosine function \xref{def:cos} as follows:

%\exbox{\begin{array}{m{58mm}m{\tw-76mm}}
\exbox{\begin{array}{Mm{88mm}}
%\begin{minipage}{\tw/2-2mm}
  %\[ 
      $\ff(x) \eqd \brbl{%
       \begin{array}{>{\ds}lM}
          \cos^2\brp{\frac{\pi}{2}x}  & for $\abs{x}\le 1$  \\
          0    & otherwise
       \end{array}}$
  %\]
%\end{minipage}%
&\includegraphics{graphics/coscos11.pdf}
\\Then $\ff$ induces a \structe{partition of unity}: % because $\sum_{n\in\Z} \ff(x-n) = 1$.
&\includegraphics{graphics/sinsin02_pun.pdf}
\\\mc{2}{M}{Note that $\ds\Ff(\omega)= \frac{1}{2\sqrt{2\pi}}\Big[{\mcom{\frac{2\sin\omega}{\omega}}{$2\sinc(\omega)$}+\mcom{\frac{\sin(\omega-\pi)}{(\omega-\pi)}}{$\sinc(\omega-\pi)$} + \mcom{\frac{\sin(\omega+\pi)}{(\omega+\pi)}}{$\sinc(\omega+\pi)$}}\Big]$}
\\and so $\Ff(2\pi n)=\frac{1}{\sqrt{2\pi}}\kdelta_n$:
&\includegraphics{graphics/coscos11ft.pdf}
\end{array}}
\end{example}
\begin{proof}
Let $\setind_\setA(x)$ be the \fncte{set indicator function} \xref{def:setind} on a set $\setA$.
\begin{enumerate}
  
  \item Proof that $\sum_{n\in\Z} \opTrn^n\ff=1$ (time domain proof):
    \begin{align*}
      \sum_{n\in\Z} \opTrn^n\ff(x)
        &= \sum_{n\in\Z} \opTrn^n\cos^2(x)\setind_\intcc{-1}{1}(x)
        \qquad\text{by definition of $\ff(x)$}
      \\&= \sum_{n\in\Z} \opTrn^n\cos^2(x)\setind_\intco{-1}{1}(x)
        \qquad\text{because $\cos^2\brp{\frac{\pi}{2}x}=0$ when $x=1$}
      \\&= \sum_{n\in\Z} \cos^2\brp{\frac{\pi}{2}(x-n)}\setind_\intco{-1}{1}(x-n)
        \qquad\text{by definition of $\opTrn$ \xref{def:opT}}
      \\&= \mcom{\sum_{n\in\Zo} \cos^2\brp{\frac{\pi}{2}(x-n)}                   \setind_\intco{-1}{1}(x-n)}{odd part}
         + \mcom{\sum_{n\in\Ze} \cos^2\brp{\frac{\pi}{2}(x-n)}                   \setind_\intco{-1}{1}(x-n)}{even part}
      \\&= \sum_{n\in\Z} \cos^2\brp{\frac{\pi}{2}(x-2n)}                   \setind_\intco{-1}{1}(x-2n)
         + \sum_{n\in\Z} \cos^2\brp{\frac{\pi}{2}(x-2n-1)}                 \setind_\intco{-1}{1}(x-2n-1)
      \\&= \sum_{n\in\Z} \cos^2\brp{\frac{\pi}{2}x-n\pi}                   \setind_\intco{-1}{1}(x-2n)
         + \sum_{n\in\Z} \cos^2\brp{\frac{\pi}{2}x-n\pi-\frac{\pi}{2}}     \setind_\intco{-1}{1}(x-2n-1)
      \\&= \sum_{n\in\Z} (-1)^{2n}\cos^2\brp{\frac{\pi}{2}x}               \setind_\intco{-1}{1}(x-2n)  
         + \sum_{n\in\Z} (-1)^{2n}\cos^2\brp{\frac{\pi}{2}x-\frac{\pi}{2}} \setind_\intco{-1}{1}(x-2n-1)
      \\&= \sum_{n\in\Z} \cos^2\brp{\frac{\pi}{2}x}                        \setind_\intco{-1}{1}(x-2n)  
         + \sum_{n\in\Z} \sin^2\brp{\frac{\pi}{2}x}                        \setind_\intco{-1}{1}(x-2n-1)
        \quad\text{by \prefp{thm:trig_sq}}
      \\&= \cos^2\brp{\frac{\pi}{2}x}  \sum_{n\in\Z}                       \setind_\intco{-1}{1}(x-2n)  
         + \sin^2\brp{\frac{\pi}{2}x}  \sum_{n\in\Z}                       \setind_\intco{-1}{1}(x-2n-1)
      \\&= \cos^2\brp{\frac{\pi}{2}x}\cdot 1
         + \sin^2\brp{\frac{\pi}{2}x}\cdot 1
      \\&= 1
        \qquad\text{by \thme{square identity} \xref{thm:trig_sq}}
    \end{align*}

  \item Proof that $\Ff(\omega)=\cdots$: by \prefp{ex:ft_cos2}
\end{enumerate}
\end{proof}

%-------------------------------------
\begin{example}
\label{ex:pun_sin2}
%-------------------------------------
Let a function $\ff$ be defined in terms of the sine function \xref{def:sin} as follows:
\cntbox{\begin{array}{m{\tw-85mm-18mm}m{85mm}}
      $\ff(x) \eqd \brbl{%
       \begin{array}{>{\ds}lM}
          \sin^2\brp{\frac{\pi}{2}x}  & for $x\in\intcc{0}{2}$  \\
          0    & otherwise
       \end{array}}$
&\includegraphics{graphics/sinsin02.pdf}
\\Then $\int_\R\ff(x)\dx=1$ and $\ff$ induces a \structe{partition of unity}
&\includegraphics{graphics/sinsin02_pun.pdf}
%\\\textbf{but} $\setxZ{\opTrn^n\ff}$ does \textbf{not} generate an \structe{MRA}.
\end{array}}
\end{example}
\begin{proof}
%Let $\setind_\setA(x)$ be the \fncte{set indicator function} \xref{def:setind} on a set $\setA$.
\begin{enumerate}
  \item Proof that $\int_\R\ff(x)\dx=1$:
    \begin{align*}
      \int_\R \ff(x) \dx
        &= \int_\R \sin^2\brp{\frac{\pi}{2}x}\setind_\intcc{0}{2}(x) \dx
        && \text{by definition of $\ff(x)$}
      \\&= \int_0^2 \sin^2\brp{\frac{\pi}{2}x} \dx
        && \text{by definition of $\setindAx$ \xref{def:setind}}
      \\&= \int_0^2 \frac{1}{2}\brs{1-\cos\brp{\pi x}} \dx
        && \text{by \prefp{thm:trig_sq}}
      \\&= \frac{1}{2}\brs{x-\frac{1}{\pi}\sin\brp{\pi x}}_0^2
      \\&= \frac{1}{2}\brs{2-0-0-0}
      \\&= 1
    \end{align*}
  
  \item Proof that $\ff(x)$ forms a \prope{partition of unity}:
    \begin{align*}
      \sum_{n\in\Z} \opTrn^n\ff(x)
        &= \sum_{n\in\Z} \opTrn^n\sin^2\brp{\frac{\pi}{2}x}\setind_\intcc{0}{2}(x)
        && \text{by definition of $\ff(x)$}
      \\&= \sum_{n\in\Z} \opTrn^n\sin^2\brp{\frac{\pi}{2}x}\setind_\intco{0}{2}(x)
        && \text{because $\sin^2\brp{\frac{\pi}{2}x}=0$ when $x=2$}
      \\&= \sum_{m\in\Z} \opTrn^{m-1}\sin^2\brp{\frac{\pi}{2}x}\setind_\intco{0}{2}(x)
        && \text{where $m\eqd n+1$ $\implies$ $n=m-1$}
      \\&= \sum_{m\in\Z} \sin^2\brp{\frac{\pi}{2}(x-m+1)}\setind_\intco{0}{2}(x-m+1)
        && \text{by definition of $\opTrn$ \xref{def:opT}}
      \\&= \sum_{m\in\Z} \sin^2\brp{\frac{\pi}{2}(x-m)+\frac{\pi}{2}}\setind_\intco{-1}{1}(x-m)
      \\&= \sum_{m\in\Z} \cos^2\brp{\frac{\pi}{2}(x-m)}\setind_\intco{-1}{1}(x-m)
        && \text{by \prefp{thm:trig_sq}}
      \\&= \sum_{m\in\Z} \opTrn^m\cos^2\brp{\frac{\pi}{2}x}\setind_\intco{-1}{1}(x)
        && \text{by definition of $\opTrn$ \xref{def:opT}}
      \\&= \sum_{m\in\Z} \opTrn^m\cos^2\brp{\frac{\pi}{2}x}\setind_\intcc{-1}{1}(x)
        && \text{because $\cos^2\brp{\frac{\pi}{2}x}=0$ when $x=1$}
      \\&= 1
        && \text{by \prefp{ex:pun_cos2}}
    \end{align*}

\end{enumerate}
\end{proof}



%-------------------------------------
\begin{example}[\exmd{raised cosine}]
\label{ex:pun_rcos}
\footnote{
  \citerppg{proakis}{560}{561}{0-07-232111-3}
  }
%-------------------------------------
Let a function $\ff$ be defined in terms of the cosine function \xref{def:cos} as follows:
%\exbox{\begin{array}{m{58mm}m{\tw-76mm}}
\exbox{\begin{array}{Mm{88mm}}
%\begin{minipage}{\tw/2-2mm}
  %\[ 
      Let $\ff(x) \eqd \brbl{%
       \begin{array}{>{\ds}lM>{\ds}l}
          1 
            & for & 0 \le \abs{x} < \frac{1-\beta}{2}
            \\
          \frac{1}{2}\brb{1 + \cos\brs{\frac{\pi}{\beta}\brp{\abs{x} - \frac{1-\beta}{2}}}}
            & for & \frac{1-\beta}{2} \le \abs{x} < \frac{1+\beta}{2}
            \\
          0    & \mc{2}{M}{otherwise}
       \end{array}}$
  %\]
%\end{minipage}%
\\\includegraphics{graphics/rcos.pdf}
\\Then $\ff$ induces a \structe{partition of unity}: % because $\sum_{n\in\Z} \ff(x-n) = 1$.
\\\includegraphics{graphics/rcos_pun.pdf}
%\\\mc{2}{M}{Note that $\ds\Ff(\omega)= \frac{1}{2\sqrt{2\pi}}\Big[{\mcom{\frac{2\sin\omega}{\omega}}{$2\sinc(\omega)$}+\mcom{\frac{\sin(\omega-\pi)}{(\omega-\pi)}}{$\sinc(\omega-\pi)$} + \mcom{\frac{\sin(\omega+\pi)}{(\omega+\pi)}}{$\sinc(\omega+\pi)$}}\Big]$}
%\\and so $\Ff(2\pi n)=\frac{1}{\sqrt{2\pi}}\kdelta_n$:
%&\psset{xunit=7mm,yunit=14mm}%============================================================================
% Daniel J. Greenhoe
% LaTeX file
%
%
%            1      [ 2sin(w)   sin(w-pi)   sin(w+pi) ]
% F(w) = ---------  [ ------- + --------- + --------  ]
%        sqrt(2pi)  [    w        w-pi        w+pi    ]
%
% which is the Fourier transform of 
%   f(x)  = { cos^2(pi/2 x) for  -1 <= x <= 1
%           { 0             otherwise
% nominal xunit = 7.5mm
% nominal yunit = 15mm
%============================================================================
\begin{pspicture}(-4.75,-0.25)(5.25,0.58)%
  %-------------------------------------
  % labeling
  %-------------------------------------
  %-7.8748049728612098721453229972336
  %0.39894228040143267793994605993438
  %\uput{3.5pt}[0](0,0.395){$\frac{2}{2\sqrt{2\pi}}$}%
  \rput(4,0.39894228){\rnode{peakL}{$\frac{2}{2\sqrt{2\pi}}$}}%
  \pnode(0,0.39894228){peak}%
  \ncline[linestyle=dashed,linecolor=red,linewidth=0.75pt]{peak}{peakL}%
  %-------------------------------------
  % axes
  %-------------------------------------
  \psaxes[linecolor=axis,yAxis=false,linewidth=0.75pt,labels=none]{<->}(0,0)(-4.75,-0.5)(4.75,0.58)%
  \psaxes[linecolor=axis,xAxis=false,linewidth=0.75pt,ticks=none]{<->}(0,0)(0,-0.25)(0,0.58)%
  \uput{3.5pt}[0](4.75,0){$\omega$}%
  \rput[b]( 4,-4mm){$4\pi$}
  \rput[b]( 3,-4mm){$3\pi$}
  \rput[b]( 2,-4mm){$2\pi$}
  \rput[b]( 1,-4mm){$\pi$}
  \rput[b](-1,-4mm){$-\pi$}
  \rput[b](-2,-4mm){$-2\pi$}
  \rput[b](-3,-4mm){$-3\pi$}
  \rput[b](-4,-4mm){$-4\pi$}
  %-------------------------------------
  % plot functions
  %-------------------------------------
  %\psplot[plotpoints=64]{-3}{3}{180 x mul sin x 3 exp x 3.14159265 2 exp mul sub div -7.874805 mul 3.14159265 div}%
  %\psplot[plotpoints=64]{-3}{3}{180 x mul sin x div 180 x mul sin x 3.14159265 sub div sub 180 x mul sin x 3.14159265 add div sub -7.874805 mul 3.14159265 div}%
  %\psplot[plotpoints=64,linestyle=dotted]{-4.5}{4.5}{180 x mul sin x div 3.14159265 div}%
  %
  \psplot[plotpoints=64,linestyle=dotted,linecolor=purple]{-4.5}{4.5}{180 x mul sin x div 2 mul 2 3.14159265 mul sqrt div 2 div 3.14159265 div}%
  \psplot[plotpoints=64,linestyle=dotted,linecolor=purple]{-4.5}{4.5}{180 x 1 sub mul sin x 1 sub div 2 3.14159265 mul sqrt div 2 div 3.14159265 div}%
  \psplot[plotpoints=64,linestyle=dotted,linecolor=purple]{-4.5}{4.5}{180 x 1 add mul sin x 1 add div 2 3.14159265 mul sqrt div 2 div 3.14159265 div}%
  %\psplot[plotpoints=256]{-3}{3}{180 x mul sin}%
  %
  \psplot[plotpoints=256]{-4}{4}{180 x mul sin x div 2 mul 180 x 1 sub mul sin x 1 sub div add 180 x 1 add mul sin x 1 add div add 2 3.14159265 mul sqrt div 2 div 3.14159265 div}%
  \psplot[plotpoints=256,linestyle=dotted]{-4}{-4.5}{180 x mul sin x div 2 mul 180 x 1 sub mul sin x 1 sub div add 180 x 1 add mul sin x 1 add div add 2 3.14159265 mul sqrt div 2 div 3.14159265 div}%
  \psplot[plotpoints=256,linestyle=dotted]{4}{4.5}{180 x mul sin x div 2 mul 180 x 1 sub mul sin x 1 sub div add 180 x 1 add mul sin x 1 add div add 2 3.14159265 mul sqrt div 2 div 3.14159265 div}%
\end{pspicture}%
\end{array}}
\end{example}
\begin{proof}
\begin{enumerate}
  \item definition: Let $\setind_\setA(x)$ be the \fncte{set indicator function} \xref{def:setind} on a set $\setA$. \label{idef:pun_rcos_def}
    \\Let $\setA\eqd\intco{\frac{1+\beta}{-2}}{\frac{1-\beta}{-2}}$,\qquad
          $\setB\eqd\intco{\frac{1-\beta}{-2}}{\frac{1-\beta}{2}}$, and\qquad
          $\setC\eqd\intco{\frac{1-\beta}{2}}{\frac{1+\beta}{2}}$
  \item lemma: $\ds\setind_\setA(x-1) = \setind_\setC(x)$. Proof: \label{ilem:pun_rcos_Ax1C}
        \begin{align*}
          \setind_\setA(x-1)
            &\eqd \brbl{\begin{array}{cM}
                    1 & if $-\frac{1+\beta}{2} \le x-1 < -\frac{1-\beta}{2}$\\
                    0 & otherwise
                  \end{array}}
            && \text{by definition of $\setind$ \xref{def:setind} and $\setA$ \xref{ilem:pun_rcos_Ax1C}}
          \\&=    \brbl{\begin{array}{cM}
                    1 & if $1-\frac{1+\beta}{2} \le x < 1-\frac{1-\beta}{2}$\\
                    0 & otherwise
                  \end{array}}
          \\&=    \brbl{\begin{array}{cM}
                    1 & if $\frac{1-\beta}{2} \le x < \frac{1+\beta}{2}$\\
                    0 & otherwise
                  \end{array}}
           \\&\eqd \setind_\setC(x)
            && \text{by definition of $\setind$ \xref{def:setind} and $\setC$ \xref{ilem:pun_rcos_Ax1C}}
        \end{align*}
  %\item lemma: $\fg(x-1)\setind_\setA(x) = \fg(-x)\setind_\setC(x)$ \label{item:pun_rcos_xAC}
  %\item lemma: $\ds 1 - \frac{1-\beta}{2} = \frac{2-1+\beta}{2} = \frac{1+\beta}{2}$ \label{item:pun_rcos_lem11b}
  \item lemma: $\ds -1 + \frac{1-\beta}{2} = -\beta - \frac{1-\beta}{2}$. Proof: \label{ilem:pun_rcos_lem11b2}
    \begin{align*}
      -1 + \frac{1-\beta}{2}
        &= \frac{-2+1-\beta}{2}
        &= \frac{-1-\beta}{2}
        &= \brp{-\beta+\beta}-\brp{\frac{1+\beta}{2}}
        &= -\beta + \frac{2\beta-1-\beta}{2}
        &= -\beta - \frac{1-\beta}{2}
    \end{align*}
  %\item lemma: $\ds\beta - \frac{1+\beta}{2} = \frac{2\beta-1-\beta}{2} = \frac{1-\beta}{-2}$ \label{item:pun_rcos_lemb1b}

  \item Proof that $\sum_{n\in\Z} \opTrn^n\ff=1$:
    \begin{align*}
      &\sum_{n\in\Z} \opTrn^n\ff(x)
         = \sum_{n\in\Z} \ff(x-n)
        && \text{by \pref{def:opT}}
      \\&= \sum_{n\in\Z} \ff(x-n)\setind_\setC(x-n)
         + \sum_{n\in\Z} \ff(x-n)\setind_\setA(x-n)
         + \sum_{n\in\Z} \ff(x-n)\setind_\setB(x-n)
        && \text{by \prefp{idef:pun_rcos_def}}
      \\&= \sum_{n\in\Z} \ff(x-n)\setind_\setC(x-n)
         \\&\qquad+ \sum_{n\in\Z} \ff(x-n-1)\setind_\setA(x-n-1)
         + \sum_{n\in\Z} \ff(x-n)\setind_\setB(x-n)
        && \text{by \pref{prop:opT_periodic}}
      \\&= \sum_{n\in\Z} \ff(x-n)\setind_\setC(x-n)
         + \sum_{n\in\Z} \ff(x-n-1)\setind_\setC(x-n)
         + \sum_{n\in\Z} \ff(x-n)\setind_\setB(x-n)
        && \text{by \prefp{ilem:pun_rcos_Ax1C}}
      \\&=          \frac{1}{2}\sum_{n\in\Z} \brb{1 + \cos\brs{\frac{\pi}{\beta}\brp{\abs{x-n  } - \frac{1-\beta}{2}}}}\setind_\setC(x-n)
         \\&\qquad+ \frac{1}{2}\sum_{n\in\Z} \brb{1 + \cos\brs{\frac{\pi}{\beta}\brp{\abs{x-n-1} - \frac{1-\beta}{2}}}}\setind_\setC(x-n)
                  + \sum_{n\in\Z} \setind_\setB(x-n)
        && \text{by definition of $\ff(x)$}
      \\&=          \frac{1}{2}\sum_{n\in\Z} \brb{1 + \cos\brs{\frac{\pi}{\beta}\brp{\brp{x-n  } - \frac{1-\beta}{2}}}}\setind_\setC(x-n)
         \\&\qquad+ \frac{1}{2}\sum_{n\in\Z} \brb{1 + \cos\brs{\frac{\pi}{\beta}\brp{-\brp{x-n-1} - \frac{1-\beta}{2}}}}\setind_\setC(x-n)
                  + \sum_{n\in\Z} \setind_\setB(x-n)
        && \text{by def. of $\setind_\setC(x)$}
      \\&=          \frac{1}{2}\sum_{n\in\Z} \brb{1 + \cos\brs{\frac{\pi}{\beta}\brp{x-n - \frac{1-\beta}{2}}}}\setind_\setC(x-n)
         \\&\qquad+ \frac{1}{2}\sum_{n\in\Z} \brb{1 + \cos\brs{\frac{\pi}{\beta}\brp{x-n-1 + \frac{1-\beta}{2}}}}\setind_\setC(x-n)
                  + \sum_{n\in\Z} \setind_\setB(x-n)
     % \\&=          \frac{1}{2}\sum_{n\in\Z} \brb{1 + \cos\brs{\frac{\pi}{\beta}\brp{x-n - \frac{1-\beta}{2}}}}\setind_\setC(x-n)
     %    \\&\qquad+ \frac{1}{2}\sum_{n\in\Z} \brb{1 + \cos\brs{\frac{\pi}{\beta}\brp{x-n - \frac{1+\beta}{2}}}}\setind_\setC(x-n)
     %             + \sum_{n\in\Z} \setind_\setB(x-n)
     % \\&=          \frac{1}{2}\sum_{n\in\Z} \brb{1 + \cos\brs{\frac{\pi}{\beta}\brp{x-n - \frac{1-\beta}{2}}}}\setind_\setC(x-n)
     %    \\&\qquad+ \frac{1}{2}\sum_{n\in\Z} \brb{1 + \cos\brs{\frac{\pi}{\beta}\brp{x-n -\beta+\beta- \frac{1+\beta}{2}}}}\setind_\setC(x-n)
     %             + \sum_{n\in\Z} \setind_\setB(x-n)
      \\&=          \frac{1}{2}\sum_{n\in\Z} \brb{1 + \cos\brs{\frac{\pi}{\beta}\brp{x-n - \frac{1-\beta}{2}}}}\setind_\setC(x-n)
         \\&\qquad+ \frac{1}{2}\sum_{n\in\Z} \brb{1 + \cos\brs{\frac{\pi}{\beta}\brp{x-n -\beta- \frac{1-\beta}{2}}}}\setind_\setC(x-n)
                  + \sum_{n\in\Z} \setind_\setB(x-n)
        && \text{by \prefp{ilem:pun_rcos_lem11b2}}
      \\&=          \frac{1}{2}\sum_{n\in\Z} \brb{1 + \cos\brs{\frac{\pi}{\beta}\brp{x-n - \frac{1-\beta}{2}}}}\setind_\setC(x-n)
         \\&\qquad+ \frac{1}{2}\sum_{n\in\Z} \brb{1 + \cos\brs{\frac{\pi}{\beta}\brp{x-n - \frac{1-\beta}{2}-\frac{\pi\beta}{\beta}}}}\setind_\setC(x-n)
                  + \sum_{n\in\Z} \setind_\setB(x-n)
      \\&=          \frac{1}{2}\sum_{n\in\Z} \brb{1 + \cos\brs{\frac{\pi}{\beta}\brp{x-n - \frac{1-\beta}{2}}}}\setind_\setC(x-n)
         \\&\qquad+ \frac{1}{2}\sum_{n\in\Z} \brb{1 - \cos\brs{\frac{\pi}{\beta}\brp{x-n - \frac{1-\beta}{2}}}}\setind_\setC(x-n)
                  + \sum_{n\in\Z} \setind_\setB(x-n)
      \\&=          \frac{1}{2}\sum_{n\in\Z} \setind_\setC(x-n)
                  + \frac{1}{2}\sum_{n\in\Z} \setind_\setC(x-n)
                  + \sum_{n\in\Z} \setind_\setB(x-n)
      \\&=          \sum_{n\in\Z} \setind_{\setB\setu\setC}(x-n)
      \\&= 1
    \end{align*}
\end{enumerate}
\end{proof}


