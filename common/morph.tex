%============================================================================
% LaTeX File
% Daniel J. Greenhoe
%============================================================================

%======================================
\chapter{Translation Spaces}
\label{chp:morph}
%======================================

%=======================================
\section{Translation}
%=======================================
%=======================================
\subsection{Definitions}
%=======================================
%---------------------------------------
\begin{definition}
\label{def:mph_trans}
\label{def:mph_dilation}
%---------------------------------------
Let $\setX$ be a set and $\opI$ be the identity operator on $\setX$.
\defboxt{
  $\opT_x$ is a \hid{translation operator} on $\setX$ if
  \\\indentx$\begin{array}{F>{\ds}l@{\qquad}C@{\qquad}D@{\qquad}D}
    1. & \exists 0\in\setX \st \opT_0 = \opI
       & \forall \setA\in\pset{\setX}
       & (\prope{identity})
       & and
  \\2. & \opT_x \opT_y = \opT_y \opT_x
       & \forall x,y\in\setX
       & (\prope{commutative})
       & and
  \\3. & \opT_x \setopu_{i\in\setI} \setA_i = \setopu_{i\in\setI} \opT_x\setA_i
       & \forall \setA,\setY\in\pset{\setX},\,x\in\setX
       & (\prope{distributive} over $\setu$)
       & and
  \\4. & \setopu_{b\in\setB}\opT_b \setA = \setopu_{a\in\setA} \opT_a\setB
       & \forall \setA,\setB\in\pset{\setX}
       &
       & and
  \\5. & \opT_x\brp{\setA\seti\setB} = \brp{\opT_x\setA}\seti\brp{\opT_x\setB}
       & \forall \setA,\setB\in\pset{\setX},\,x\in\setX
       &
       & and
  \\6. & \opT_x\brp{\cmpA} = \setopc\brp{\opT_x\setA}
       & \forall \setA,\setB\in\pset{\setX},\,x\in\setX.
       &
       & 
  \end{array}$
  \\
  The pair $\opair{\setX}{\opT}$ is a \hid{translation space} on $\setX$.
  }
\end{definition}


%%---------------------------------------
%\begin{definition}
%\label{def:mph_trans}
%\label{def:mph_dilation}
%%---------------------------------------
%Let $\otriple{\setX}{+}{\cdot}$ be a ring and $\pset{\setX}$ the power set of $\setX$.
%\defbox{\begin{array}{lcl@{\qquad}C@{\qquad}D}
%  \opT_x\setA &\eqd& \set{a+x}{a\in\setA}
%    & \forall \setA\in\pset{\setX},\; x\in\setX
%    & ({\hid{translation}} of $\setA$ by $x$)
%  \\
%  \opD_\lambda\setA &\eqd& \set{\lambda a}{a\in\setA}
%    & \forall \setA\in\pset{\setX},\; \lambda\in\setX
%    & ({\hid{dilation}} of $\setA$ by $\lambda$)
%  \end{array}}
%\end{definition}

%---------------------------------------
\begin{definition}
\citetbl{
  %\citepg{pitas1991}{159}{0792390490}\\
  \citerpg{matheron1975}{17}{0471576212} \\
  \citerpg{lay1982}{7}{0471095842}
  }
\label{def:mph_add}
\label{def:minkowski_add}
\index[xsym]{$\oplus$}
%---------------------------------------
Let $\setX$ be a set on which is defined the translation opertor $\opT_x$.
\hid{Minkowski addition} $\oplus$ and \hid{Minkowski subtraction} $\ominus$ is defined as follows:
\defbox{\begin{array}{rc>{\ds}l @{\qquad}C@{\qquad}D}
  \setA\oplus\setB &=& \setopu_{ b\in\setB}\opT_b\setA
    & \forall\;\setA,\setB\in\pset{\setX}
    & (\prope{Minkowski addition})
    \\
  \setA\ominus\setB &=& \setopi_{ b\in\setB}\opT_b\setA
    & \forall\;\setA,\setB\in\pset{\setX}
    & (\prope{Minkowski subtraction})
\end{array}}
\end{definition}

\pref{thm:mph_addsub} (next) shows a relationship between
Minkowski addition and Minkowski subtraction.
%---------------------------------------
\begin{theorem}[de Morgan relations]
\label{thm:mph_addsub}
\citepg{pitas1991}{159}{0792390490}
%---------------------------------------
Let $\opair{\setX}{+}$ be a group with
Minokowski addition    operator $\oplus:\setX^2\to\setX$ and
Minokowski subtraction operator $\ominus:\setX^2\to\setX$.
\thmbox{\begin{array}{rcl>{\qquad}C}
    \setopc(\setA\oplus\setB) &=& \cmpA \ominus B
    & \forall\;\setA,\setB\in\pset{\setX}
    \\
    \setopc(\setA\ominus\setB) &=& \cmpA \oplus\setB
    & \forall\;\setA,\setB\in\pset{\setX}
  \end{array}}
\end{theorem}
\begin{proof}
\begin{align*}
  \setopc\brp{\setA\oplus\setB}
    &= \setopc\brp{\setopu_{ b\in\setB}\opT_b\setA}
    && \text{by \prefp{def:mph_add}}
  \\&= \setopi_{ b\in\setB}\setopc\brp{\opT_b\setA}
    && \text{by Demorgan relation \prefpo{thm:mph_addsub}}
  \\&= \setopi_{ b\in\setB}\opT_b\brp{\cmpA}
    && \text{by \prefp{def:mph_trans}}
  \\&= \cmpA \ominus \setB
    && \text{by \prefp{thm:mph_add}}
  \\
  \\
  \setopc\brp{\setA\ominus\setB}
    &= \setopc\brp{\setopi_{ b\in\setB}\opT_b\setA}
    && \text{by \prefp{def:mph_add}}
  \\&= \setopu_{ b\in\setB}\setopc\brp{\opT_b\setA}
    && \text{by Demorgan relation \prefpo{thm:mph_addsub}}
  \\&= \setopu_{ b\in\setB}\opT_b\brp{\cmpA}
    && \text{by \prefp{def:mph_trans}}
  \\&= \cmpA \oplus \setB
    && \text{by \prefp{thm:mph_add}}
\end{align*}
\end{proof}

%=======================================
\subsection{Examples}
%=======================================

%---------------------------------------
\begin{example}[Translation on groups]
\citetbl{
  \citerppg{matheron1975}{16}{17}{0471576212} \\
  \citerpg{pitas1991}{159}{0792390490}\\
  \citerpg{lay1982}{7}{0471095842}
  }
\label{ex:mph_group}
\index{addition!Minkowski}
\index{addition!Minkowski}
%---------------------------------------
Let $\oplus$ be the Minkowski addition operator defined in terms of the 
\ope{translation operator} $\opT$. Let $\opair{\setX}{+}$ be a \structe{group}. 
\exboxt{
  $\ds\brb{\begin{array}{lC}
    \opT_x\setA \eqd \set{a+x}{a\in\setA} & \forall \setA\in \pset{\setX}
  \end{array}}$
  $\implies$
  \\\indentx
  $\brb{\begin{array}{lD}
    \text{$\opT_x$ is a translation operator} & and \\
    \setA\oplus\setB = \set{a+b}{a\in\setA \;\text{and}\; b\in\setB}
      & $\forall\;\setA,\setB\in \pset{\setX}$
      %& (\prope{Minkowski addition})
    \end{array}}$
  }
\end{example}
\begin{proof}
\begin{align*}
  \intertext{1. Proof that $\exists 0\in\setX \st \opT_0 = \opI$:}
    \opT_0\setA
      &= \set{a+0}{a\in\setA}
      && \text{by definition of $\opT_x$}
    \\&= \set{a}{a\in\setA}
      && \text{by additive identity property of groups}
    \\&= \setA
    \\       
  \intertext{2. Proof that $\opT_x \opT_y = \opT_y \opT_x$:}
    \opT_x\opT_y\setA
      &= \opT_x\set{a+y}{a\in\setA}
      && \text{by definition of $\opT_y$}
    \\&= \set{a+y+x}{a\in\setA}
      && \text{by definition of $\opT_y$}
    \\&= \set{a+x+y}{a\in\setA}
      && \text{by commutative property of groups}
    \\&= \opT_y\set{a+x}{a\in\setA}
      && \text{by definition of $\opT_y$}
    \\&= \opT_y\opT_x\set{a}{a\in\setA}
      && \text{by definition of $\opT_x$}
    \\
  \intertext{3. Proof that $\opT_x \setopu_{i\in\setI} \setA_i = \setopu_{i\in\setI} \opT_x\setA_i$:}
    \opT_x\setopu_i \setA_i
      &= \set{y+x}{y\in\setopu_i\setA_i}
      && \text{by definition of $\opT_y$}
    \\&= \set{y+x}{\logopo_i y\in\setA_i}
    \\&= \setopu_i \set{y+x}{y\in\setA_i}
    \\&= \setopu_i \opT_x\set{y}{y\in\setA_i}
    \\&= \setopu_i \opT_x\setA
    \\
  \intertext{4. Proof that $\setopu_{b\in\setB}\opT_b \setA = \setopu_{a\in\setA} \opT_a\setB$:}
    \setopu_{b\in\setB}\opT_b \setA 
      &= \setopu_{b\in\setB} \set{a+b}{a\in\setA}
      && \text{by definition of $\opT_x$}
    \\&= \set{a+b}{a\in\setA \text{ and } b\in\setB}
    \\&= \set{b+a}{b\in\setB \text{ and } a\in\setA}
    \\&= \setopu_{a\in\setA} \set{b+a}{b\in\setB}
    \\&= \setopu_{a\in\setA} \opT_a \setB
    \\
  \intertext{5. Proof that $\opT_x \setopi_{i\in\setI} \setA_i = \setopi_{i\in\setI} \opT_x\setA_i$:}
    \opT_x\setopi_i \setA_i
      &= \set{y+x}{y\in\setopi_i\setA_i}
      && \text{by definition of $\opT_y$}
    \\&= \set{y+x}{\logopa_i y\in\setA_i}
    \\&= \setopi_i \set{y+x}{y\in\setA_i}
    \\&= \setopi_i \opT_x\set{y}{y\in\setA_i}
    \\&= \setopi_i \opT_x\setA
    \\
  \intertext{6. Proof that $\opT_x\brp{\cmpA} = \setopc\brp{\opT_x\setA}$:}
    \opT_x \setopc\setA
      &= \opT_x \set{a}{a\in\cmpA}
    \\&= \set{a+x}{a\in\cmpA}
    \\&= \set{a+x}{a\notin\setA}
    \\&= \set{a+x}{\lnot(a\in\setA)}
    \\&= \setopc\set{a+x}{a\in\setA}
    \\&= \setopc\opT_x\setA
\end{align*}

\begin{align*}
  \setA\oplus\setB
    &= \setopu_{ b\in\setB} \opT_b \setA
    && \text{by \prefp{def:mph_add}}
  \\&= \set{a+b}{a\in\setA \;\text{and}\; b\in\setB}
    && \text{by \prefp{def:mph_trans}}
%  \\
%  \setA\ominus\setB
%    &= \setopc(\cmpA \oplus\setB)
%    && \text{by \prefp{thm:mph_addsub}}
%  \\&= \setopc\left( \setopu_{ b\in\setB} \opT_b\cmpA \right)
%    && \text{by \prefp{thm:mph_addU}}
%  \\&= \setopi_{ b\in\setB}\opT_b\setA
%    && \text{by de Morgan's law \prefpo{thm:set_demorgan}}
\end{align*}
\end{proof}










\begin{figure}[ht]
\begin{center}%
\begin{fsL}
\setlength{\unitlength}{0.05mm}
\begin{tabular}{c>{\parbox[b][22mm][c]{3em}{\Large$\mathbf{\oplus}$}}c>{\parbox[b][22mm][c]{3em}{\Large$\mathbf{=}$}}c}
\begin{picture}(440,440)(-220,-220)%
  %\graphpaper[10](0,0)(600,200)%
  \thinlines%
  \color{axis}%
    \put(   0,-220){\line( 0, 1){440} }%
    \put(-220,   0){\line( 1, 0){440} }%
    \put( -10, 200){\makebox(0,0)[r]{ $2$ }}%
    \put( -10, 100){\makebox(0,0)[r]{ $1$ }}%
    \put( -10,-100){\makebox(0,0)[r]{ $-1$ }}%
    \put( -10,-200){\makebox(0,0)[r]{ $-2$ }}%
    \put( 200, -20){\makebox(0,0)[t]{ $2$ }}%
    \put( 100, -20){\makebox(0,0)[t]{ $1$ }}%
    \put(-100, -20){\makebox(0,0)[t]{ $-1$ }}%
    \put(-200, -20){\makebox(0,0)[t]{ $-2$ }}%
  \color{dot}%
    \put(   0,   0){\circle*{30}}%
    \put(   0, 100){\circle*{30}}%
    \put(   0,-100){\circle*{30}}%
    \put( 100,   0){\circle*{30}}%
\end{picture}
&
\begin{picture}(440,440)(-220,-220)%
  %\graphpaper[10](0,0)(600,200)%
  \thinlines%
  \color{axis}%
    \put(   0,-220){\line( 0, 1){440} }%
    \put(-220,   0){\line( 1, 0){440} }%
    \put( -10, 200){\makebox(0,0)[r]{ $2$ }}%
    \put( -10, 100){\makebox(0,0)[r]{ $1$ }}%
    \put( -10,-100){\makebox(0,0)[r]{ $-1$ }}%
    \put( -10,-200){\makebox(0,0)[r]{ $-2$ }}%
    \put( 200, -20){\makebox(0,0)[t]{ $2$ }}%
    \put( 100, -20){\makebox(0,0)[t]{ $1$ }}%
    \put(-100, -20){\makebox(0,0)[t]{ $-1$ }}%
    \put(-200, -20){\makebox(0,0)[t]{ $-2$ }}%
  \color{dot}%
    \put(   0,   0){\circle{30}}%
    \put(-200, 100){\makebox(0,0){$\boxempty$}}%
    \put( 100,-100){\makebox(0,0){$\Delta$}}%
\end{picture}
&
\begin{picture}(440,440)(-220,-220)%
  %\graphpaper[10](0,0)(600,200)
  \thinlines%
  \color{axis}%
    \put(   0,-220){\line( 0, 1){440} }%
    \put(-220,   0){\line( 1, 0){440} }%
    \put( -10, 200){\makebox(0,0)[r]{ $2$ }}%
    \put( -10, 100){\makebox(0,0)[r]{ $1$ }}%
    \put( -10,-100){\makebox(0,0)[r]{ $-1$ }}%
    \put( -10,-200){\makebox(0,0)[r]{ $-2$ }}%
    \put( 200, -20){\makebox(0,0)[t]{ $2$ }}%
    \put( 100, -20){\makebox(0,0)[t]{ $1$ }}%
    \put(-100, -20){\makebox(0,0)[t]{ $-1$ }}%
    \put(-200, -20){\makebox(0,0)[t]{ $-2$ }}%
  \color{dot}%
    \put(   0,   0){\circle{30}}%
    \put(   0, 100){\circle{30}}%
    \put(   0,-100){\circle{30}}%
    \put( 100,   0){\circle{30}}%
    \put(-200, 100){\makebox(0,0){$\boxempty$}}%
    \put(-200, 200){\makebox(0,0){$\boxempty$}}%
    \put(-200,   0){\makebox(0,0){$\boxempty$}}%
    \put(-100, 100){\makebox(0,0){$\boxempty$}}%
    \put( 100,-100){\makebox(0,0){$\Delta$}}%
    \put( 100,   0){\makebox(0,0){$\Delta$}}%
    \put( 100,-200){\makebox(0,0){$\Delta$}}%
    \put( 200,-100){\makebox(0,0){$\Delta$}}%
\end{picture}
\end{tabular}
\\ \vspace{2mm}
\begin{tabular}{c>{\parbox[b][22mm][c]{3em}{\Large$\mathbf{\ominus}$}}c>{\parbox[b][22mm][c]{3em}{\Large$\mathbf{=}$}}c}
\begin{picture}(440,440)(-220,-220)%
  %\graphpaper[10](0,0)(600,200)
  \thinlines%
  \color{axis}%
    \put(   0,-220){\line( 0, 1){440} }%
    \put(-220,   0){\line( 1, 0){440} }%
    \put( -10, 200){\makebox(0,0)[r]{ $2$ }}%
    \put( -10, 100){\makebox(0,0)[r]{ $1$ }}%
    \put( -10,-100){\makebox(0,0)[r]{ $-1$ }}%
    \put( -10,-200){\makebox(0,0)[r]{ $-2$ }}%
    \put( 200, -20){\makebox(0,0)[t]{ $2$ }}%
    \put( 100, -20){\makebox(0,0)[t]{ $1$ }}%
    \put(-100, -20){\makebox(0,0)[t]{ $-1$ }}%
    \put(-200, -20){\makebox(0,0)[t]{ $-2$ }}%
  \color{dot}%
    \put(   0,   0){\circle*{30}}%
    \put(   0, 100){\circle*{30}}%
    \put(   0,-100){\circle*{30}}%
    \put( 100,   0){\circle*{30}}%
\end{picture}%
&
\begin{picture}(440,440)(-220,-220)%
  %\graphpaper[10](0,0)(600,200)
  \thinlines%
  \color{axis}%
    \put(   0,-220){\line( 0, 1){440} }%
    \put(-220,   0){\line( 1, 0){440} }%
    \put( -10, 200){\makebox(0,0)[r]{ $2$ }}%
    \put( -10, 100){\makebox(0,0)[r]{ $1$ }}%
    \put( -10,-100){\makebox(0,0)[r]{ $-1$ }}%
    \put( -10,-200){\makebox(0,0)[r]{ $-2$ }}%
    \put( 200, -20){\makebox(0,0)[t]{ $2$ }}%
    \put( 100, -20){\makebox(0,0)[t]{ $1$ }}%
    \put(-100, -20){\makebox(0,0)[t]{ $-1$ }}%
    \put(-200, -20){\makebox(0,0)[t]{ $-2$ }}%
  \color{dot}%
    \put(   0,   0){\circle*{30}}%
    \put(-200, 100){\circle*{30}}%
    \put( 100,-100){\circle*{30}}%
\end{picture}
&
\begin{picture}(440,440)(-220,-220)
  %\graphpaper[10](0,0)(600,200)
  \thinlines%
  \color{axis}%
    \put(   0,-220){\line( 0, 1){440} }%
    \put(-220,   0){\line( 1, 0){440} }%
    \put( -10, 200){\makebox(0,0)[r]{ $2$ }}%
    \put( -10, 100){\makebox(0,0)[r]{ $1$ }}%
    \put( -10,-100){\makebox(0,0)[r]{ $-1$ }}%
    \put( -10,-200){\makebox(0,0)[r]{ $-2$ }}%
    \put( 200, -20){\makebox(0,0)[t]{ $2$ }}%
    \put( 100, -20){\makebox(0,0)[t]{ $1$ }}%
    \put(-100, -20){\makebox(0,0)[t]{ $-1$ }}%
    \put(-200, -20){\makebox(0,0)[t]{ $-2$ }}%
  \color{dot}%
    \put( 100,   0){\circle*{30}}%
\end{picture}
\end{tabular}
\end{fsL}
\end{center}
\caption{
   Illustration for \prefpp{ex:mph_T}
   \label{fig:mph_T}
   }
\end{figure}
%---------------------------------------
\begin{example}%[Minkowski addition]
\label{ex:mph_T}
%---------------------------------------
Let
\begin{align*}
  \setA &\eqd \setn{(0,0),\;(0,1),\;(0,-1),\;(1,1)}
  \\
  \setB & \eqd \setn{(0,0),\;(-2,1),\;\;(1,-1)}
\end{align*}
Then
{\footnotesize
\begin{align*}
  \setA\oplus\setB &=
  \setn{\opair{0}{0},\;\opair{0}{1},\;\opair{0}{-1},\;\opair{1}{1},\;
       \opair{-2}{1},\;\opair{-2}{2},\;\opair{-2}{0},\;\opair{-1}{2},\;
       \opair{1}{-1},\;\opair{1}{-2},\;\opair{2}{0}
      }
  \\
  \setA\ominus\setB &= \setn{\opair{1}{0}}
  \end{align*}
}
These relationships are illustrated in \prefpp{fig:mph_T}.
\end{example}


\begin{figure}[ht]
\begin{center}
\begin{fsL}
\setlength{\unitlength}{0.05mm}
\begin{picture}(800,400)(-50,-50)%
  %\graphpaper[10](0,0)(600,200)%
  \thinlines%
  \color{axis}%
    \put(   0,   0){\line( 0, 1){350} }%
    \put(   0,   0){\line( 1, 0){750} }%
    \put( -10, 300){\makebox(0,0)[r]{$3$}}%
    \put( -10, 200){\makebox(0,0)[r]{$2$}}%
    \put( -10, 100){\makebox(0,0)[r]{$1$}}%
    \put( 700, -20){\makebox(0,0)[t]{$7$}}%
    \put( 600, -20){\makebox(0,0)[t]{$6$}}%
    \put( 500, -20){\makebox(0,0)[t]{$5$}}%
    \put( 400, -20){\makebox(0,0)[t]{$4$}}%
    \put( 300, -20){\makebox(0,0)[t]{$3$}}%
    \put( 200, -20){\makebox(0,0)[t]{$2$}}%
    \put( 100, -20){\makebox(0,0)[t]{$1$}}%
  \thicklines%
  \color{red}%
    \put( 100, 100){\line(1, 1){100}}%
    \put( 160, 160){\makebox(0,0)[br]{$\setA$}}%
  \color{blue}%
    \put( 200, 100){\line(1, 0){300}}%
    \put( 350,  80){\makebox(0,0)[t]{$\setB$}}%
  \color{purple}%
    \put( 340, 250){\makebox(0,0)[br]{$\setA\oplus\setB$}}%
    \put( 600, 200){\line(1, 1){100}}%
    \put( 300, 200){\line(1, 1){100}}%
    \multiput( 300, 200)(10,10){11}{\line(1, 0){300}}%
\end{picture}
\hspace{1cm}
\begin{picture}(800,700)(-50,-50)%
  %\graphpaper[10](0,0)(600,200)%
  \thinlines%
  \color{axis}%
    \put(   0,   0){\line( 0, 1){650} }%
    \put(   0,   0){\line( 1, 0){750} }%
    \put( -10, 600){\makebox(0,0)[r]{$6$}}%
    \put( -10, 500){\makebox(0,0)[r]{$5$}}%
    \put( -10, 400){\makebox(0,0)[r]{$4$}}%
    \put( -10, 300){\makebox(0,0)[r]{$3$}}%
    \put( -10, 200){\makebox(0,0)[r]{$2$}}%
    \put( -10, 100){\makebox(0,0)[r]{$1$}}%
    \put( 700, -20){\makebox(0,0)[t]{$7$}}%
    \put( 600, -20){\makebox(0,0)[t]{$6$}}%
    \put( 500, -20){\makebox(0,0)[t]{$5$}}%
    \put( 400, -20){\makebox(0,0)[t]{$4$}}%
    \put( 300, -20){\makebox(0,0)[t]{$3$}}%
    \put( 200, -20){\makebox(0,0)[t]{$2$}}%
    \put( 100, -20){\makebox(0,0)[t]{$1$}}%
  \thicklines%
  \color{red}%
    \put( 100, 300){\circle*{100}}%
    \put( 100, 240){\makebox(0,0)[t]{$\setA$}}%
  \color{blue}%
    \put( 500,   0){\line(0, 1){200}}%
    \put( 200,   0){\line(0, 1){200}}%
    \multiput(200,0)(0,10){21}{\line(1, 0){300}}%
    \put( 180,  20){\makebox(0,0)[br]{$\setB$}}%
  \color{purple}%
    \put( 240, 530){\makebox(0,0)[tr]{$\setA\oplus\setB$}}%
    \cornersize{0.4}%
    \put( 450, 400){\fancyoval(400,300)}%
    \multiput( 300, 500)(25,0){13}{\circle{100}}%
    \multiput( 300, 400)(25,0){13}{\circle{100}}%
    \multiput( 300, 300)(25,0){13}{\circle{100}}%
\end{picture}
\end{fsL}
\end{center}
\caption{
   Illustration for \prefp{ex:mph_circle+square}
   \label{fig:mph_circle+square}
   }
\end{figure}
%---------------------------------------
\begin{example}
\label{ex:mph_circle+square}
\citetbl{
  \citerpg{lay1982}{7}{0471095842}
  }
%---------------------------------------
Two more examples are illustrated in \prefpp{fig:mph_circle+square}.
\end{example}








%=======================================
\subsection{Additive properties}
%=======================================


%---------------------------------------
\begin{theorem}
\label{thm:mph_add}
\citeppg{pitas1991}{163}{164}{0792390490}
%---------------------------------------
Let $\opair{\setX}{+}$ be a group with with Minokowski addition operator $\oplus:\setX^2\to\setX$.
\thmbox{\begin{array}{rcl>{\qquad}C>{\qquad}D}
    \setA \oplus \setn{0} &=& \setA
    & \forall \setA\subseteq\setX
    &
    \\
    \setA\oplus\setB &=& \setB\oplus \setA
    & \forall \setA,\setB\subseteq\setX
    & (\prope{commutative})
    \\
    \setA \oplus (\setB\oplus \setC) &=& (\setA\oplus\setB) \oplus \setC
    & \forall \setA,\setB,\setC\subseteq \setX
    & (\prope{associative})
    \\
    \opT_x\brp{\setA\oplus\setB} &=& \brp{\opT_x\setA} \oplus \setB
    & \forall \setA,\setB\subseteq\setX,\; x\in \setX
    & (\prope{translation invariant})
    %\\
    %\setA\subseteq \setB \implies \qquad
    %\setA \oplus \setC &\subseteq& \setB\oplus \setC
    %& \forall \setA,\setB,\setC\subseteq\setX
    %& (\prope{increasing})
  \end{array}}
\end{theorem}
\begin{proof}
\begin{align*}
  \setA \oplus \setn{0}
    &= \left.\setA \oplus \setB \right|_{\setB=\setn{0}}
  \\&= \left.\setopu_{b\in\setB} \opT_b \setA \right|_{\setB=\setn{0}}
    && \text{by \prefp{def:mph_add}}
  \\&= \opT_0 \setA 
  \\&= \setA
    && \text{by \prefp{def:mph_trans}}
  \\
  \\
  \setA\oplus\setB
    &= \setopu_{b\in\setB} \opT_b \setA
    && \text{by \prefp{def:mph_add}}
  \\&= \setopu_{a\in\setA} \opT_a \setB
    && \text{by \prefp{def:mph_trans}}
  \\&= \setB\oplus\setA
    && \text{by \prefp{def:mph_add}}
  \\
  \\
  \setA\oplus\brp{\setB\oplus\setC}
    &= \setopu_{y\in\setB\oplus\setC} \opT_y \setA
    && \text{by \prefp{def:mph_add}}
  \\&= \setopu_{a\in\setA} \opT_a \brp{\setB\oplus\setC}
    && \text{by \prefp{def:mph_trans}}
  \\&= \setopu_{a\in\setA} \opT_a \brp{\setopu_{c\in\setC}\opT_c\setB}
    && \text{by \prefp{def:mph_add}}
  \\&= \setopu_{a\in\setA}  \brp{\setopu_{c\in\setC}\opT_a\opT_c\setB}
    && \text{by \prefp{def:mph_trans}}
  \\&= \setopu_{a\in\setA}  \brp{\setopu_{c\in\setC}\opT_c\opT_a\setB}
    && \text{by \prefp{def:mph_trans}}
  \\&= \setopu_{c\in\setC}\opT_c \brp{\setopu_{a\in\setA} \opT_a\setB}
    && \text{by \prefp{def:mph_trans}}
  \\&= \setopu_{c\in\setC}\opT_c \brp{\setopu_{b\in\setB} \opT_b\setA}
    && \text{by \prefp{def:mph_trans}}
  \\&= \setopu_{c\in\setC}\opT_c \brp{\setA\oplus\setB}
    && \text{by \prefp{def:mph_add}}
  \\&= \brp{\setA\oplus\setB}\oplus\setC
    && \text{by \prefp{def:mph_add}}
  \\
  \\
  \opT_x\brp{\setA\oplus\setB}
    &= \opT_x \setopu_{b\in\setB} \opT_b \setA
    && \text{by \prefp{def:mph_add}}
  \\&= \setopu_{b\in\setB} \opT_x \opT_b \setA
    && \text{by \prefp{def:mph_trans}}
  \\&= \setopu_{b\in\setB} \opT_b \opT_x \setA
    && \text{by \prefp{def:mph_trans}}
  \\&= \brp{\opT_x\setA} \oplus \setB    
    && \text{by \prefp{def:mph_add}}
  %\\
  %\\
  %\setA\oplus\setC
  %  &= \set{a+c}{a\in\setA,\; c\in\setC}
  %  && \text{by \prefp{def:mph_add}}
  %\\&\subseteq \set{a+c}{a\in\setB,\; c\in\setC}
  %  && \text{by $A\subseteq B$ hypothesis}
  %\\&= \set{b+c}{b\in\setB,\; c\in\setC}
  %  && \text{by change of variable}
  %\\&= \setB\oplus \setC
  %  && \text{by \prefp{def:mph_add}}
\end{align*}
\end{proof}


%---------------------------------------
\begin{theorem}
\label{thm:mph_add_distrib}
\citepg{pitas1991}{163}{0792390490}
%---------------------------------------
Let $\opair{\setX}{+}$ be a group with with Minokowski addition operator $\oplus:\setX^2\to\setX$.
\thmbox{\begin{array}{rcl>{\qquad}C>{\qquad}D}
    \setA \oplus (\setB \setu \setC) &=& (\setA\oplus\setB) \setu (\setA\oplus \setC)
    & \forall \setA,\setB,\setC\subseteq\setX
    & ($\oplus$ is \prope{left distributive} over $\setu$)
    \\
    (\setA \setu \setB) \oplus \setC &=& (\setA\oplus\setC) \setu (\setB\oplus \setC)
    & \forall \setA,\setB,\setC\subseteq\setX
    & ($\oplus$ \prope{right distributive} over $\setu$)
    \\
    \setA \oplus (\setB \seti \setC) &\subseteq& (\setA\oplus\setB) \seti (\setA\oplus \setC)
    & \forall \setA,\setB,\setC\subseteq\setX
    & %($\oplus$ \prope{left distributive} over $\seti$)
    \\
    (\setA\seti\setB) \oplus\setC &\subseteq& (\setA\oplus\setC) \seti (\setB\oplus\setC)
    & \forall \setA,\setB,\setC\subseteq\setX
    & %($\oplus$ \prope{right distributive} over $\seti$)
  \end{array}}
\end{theorem}
\begin{proof}
\begin{align*}
  \brp{\setA\setu\setB} \oplus\setC
    &= \setopu_{c\in\setC} \opT_c \brp{\setA\setu\setB}
    && \text{by \prefp{def:mph_add}}
  \\&= \setopu_{c\in\setC} \brs{\brp{\opT_c\setA}\setu\brp{\opT_c\setB}}
    && \text{by \prefp{def:mph_trans}}
  \\&= \brp{\setopu_{c\in\setC} \opT_c\setA} \setu  \brp{\setopu_{c\in\setC} \opT_c\setB}
  \\&= \brp{\setA\oplus\setC} \setu  \brp{\setB\oplus\setC}
    && \text{by \prefp{def:mph_add}}
  \\
  \\
  \setA \oplus (\setB\setu\setC)
    &= (\setB\setu\setC) \oplus\setA
    && \text{by \prefp{thm:mph_add}}
  \\&= (\setB\oplus\setA) \setu (\setC\oplus\setA)
    && \text{by previous result}
  \\&= (\setA\oplus\setB) \setu (\setA\oplus\setC)
    && \text{by \prefp{thm:mph_add}}
  \\
  \\
  (\setA\seti\setB) \oplus\setC
    &= \setopu_{c\in\setC} \opT_c \brp{\setA\seti\setB}
    && \text{by \prefp{thm:mph_add}}
  \\&= \setopu_{c\in\setC}  \brs{\brp{\opT_c\setA}\seti\brp{\opT_c\setB}}
    && \text{by \prefp{def:mph_trans}}
  \\&\subseteq \brp{\setopu_{c\in\setC}\opT_c\setA}\seti\brp{\setopu_{c\in\setC}\opT_c\setB}
    && \text{by minimax inequality \ifdochas{lattice}{\prefpo{thm:minimax_ineq}}}
  \\&= \brp{\setA\oplus\setC}\seti\brp{\setB\oplus\setC}
    && \text{by \prefp{thm:mph_add}}
  \\
  \\
  \setA \oplus \brp{\setB \seti \setC}
    &= \brp{\setB \seti \setC}\oplus\setA
    && \text{by \prefp{thm:mph_add}}
  \\&\subseteq \brp{\setB\oplus\setA} \seti \brp{\setC\oplus\setA}
    && \text{by previous result}
  \\&= \brp{\setA\oplus\setB} \seti \brp{\setA\oplus\setC}
    && \text{by \prefp{thm:mph_add}}
\end{align*}
\end{proof}





%=======================================
\subsection{Subtractive properties}
%=======================================


%---------------------------------------
\begin{theorem}
\label{thm:mph_sub}
\citeppg{pitas1991}{164}{165}{0792390490}
%---------------------------------------
Let $\opair{\setX}{+}$ be a group with with Minokowski subtraction operator $\ominus:\setX^2\to\setX$.
\thmbox{\begin{array}{rcl>{\qquad}C>{\qquad}D}
    \setA \ominus \setn{0} &=& \setA
    & \forall \setA\subseteq\setX
    &
    \\
    \setA\ominus\setB &=& \cmpB \ominus \cmpA
    & \forall \setA,\setB\subseteq\setX
    &
    \\
    \opT_x\brp{\setA\ominus\setB} &=& \brp{\opT_x\setA} \ominus \setB
    & \forall \setA,\setB\subseteq\setX,\; x\in\setX
    & (\prope{translation invariant})
    \\
    \setA\subseteq\setB \implies \qquad
    \setA \ominus \setC &\subseteq& \setB\ominus \setC
    & \forall \setA,\setB,\setC\subseteq\setX
    & (\prope{increasing})
  \end{array}}
\end{theorem}
\begin{proof}
\begin{align*}
  \setA \ominus \setn{0}
    &= \setopc(\cmpA \oplus \setn{0})
    && \text{by \prefp{thm:mph_addsub}}
  \\&= \setopc(\cmpA)
    && \text{by \prefp{thm:mph_add}}
  \\&= \setA
    && %\text{by \prefp{thm:set_involutary}}
  \\
  \\
  \setA\ominus\setB
    &= \setopc\setopc(\setA\ominus\setB)
    && %\text{by \prefp{thm:set_involutary}}
  \\&= \setopc(\cmpA \oplus\setB)
    && \text{by \prefp{thm:mph_addsub}}
  \\&= \setopc(\setB\oplus \cmpA)
    && \text{by \prefp{thm:mph_add}}
  \\&= \cmpB \ominus \cmpA
    && \text{by \prefp{thm:mph_addsub}}
  \\
  \\
  \opT_x\brp{\setA\ominus\setB}
    &= \opT_x\setopc\brp{\cmpA\oplus\setB}
    && \text{by \prefp{thm:mph_addsub}}
  \\&= \setopc\opT_x\brp{\cmpA\oplus\setB}
    && \text{by \prefp{def:mph_trans}}
  \\&= \setopc\brp{\opT_x\cmpA\oplus\setB}
    && \text{by \prefp{thm:mph_add}}
  \\&= \setopc\brp{\setopc\opT_x\setA\oplus\setB}
    && \text{by \prefp{def:mph_trans}}
  \\&= \opT_x\setA\ominus\setB
    && \text{by \prefp{thm:mph_addsub}}
  \\
  \\
  \setA \ominus \setC
    &= \setopi_{c\in\setC} \setA_c
    && \text{by \prefp{thm:mph_add}}
  \\&\subseteq \setopi_{c\in\setC} \setB_c
    && \text{by $\setA\subseteq \setB$ hypothesis}
  \\&= \setB \ominus \setC
    && \text{by \prefp{def:mph_add}}
\end{align*}
\end{proof}



%---------------------------------------
\begin{theorem}
\label{thm:mph_sub_distrib}
\citepg{pitas1991}{165}{0792390490}
%---------------------------------------
Let $\opair{\setX}{+}$ be a group with with Minokowski subtraction operator
$\ominus:\setX^2\to\setX$.
\thmbox{\begin{array}{rcl>{\qquad}C>{\qquad}D}
    \setA \ominus (\setB\setu\setC) &=& (\setA\ominus\setB) \seti (\setA\ominus\setC)
    & \forall \setA,\setB,\setC\subseteq\setX
    & ($\ominus$ \prope{left distributive} over $\setu$)
    \\
    (\setA\seti\setB) \ominus\setC &=& (\setA \ominus\setC) \seti (\setB\ominus\setC)
    & \forall \setA,\setB,\setC\subseteq\setX
    & ($\ominus$ \prope{right distributive} over $\seti$)
    \\
    (\setA\setu\setB) \ominus\setC &\supseteq& (\setA \ominus\setC) \setu (\setB\ominus\setC)
    & \forall \setA,\setB,\setC\subseteq\setX
    &
    \\
    \setA \ominus (\setB\seti\setC) &\supseteq& (\setA\ominus\setB) \setu (\setA\ominus\setC)
    & \forall \setA,\setB,\setC\subseteq\setX
    &
  \end{array}}
\end{theorem}
\begin{proof}
\begin{align*}
  \setA \ominus (\setB\setu\setC)
    &= \setopc\setopc\Big[\; A \ominus (\setB\setu\setC) \;\Big]
    && %\text{by \prefp{thm:set_involutary}}
  \\&= \setopc\Big[\; \cmpA \oplus (\setB\setu\setC) \;\Big]
    && \text{by \prefp{thm:mph_addsub}}
  \\&= \setopc\Big[\; (\cmpA \oplus\setB) \setu (\cmpA\oplus\setC) \;\Big]
    && \text{by \prefp{thm:mph_add_distrib}}
  \\&= \brs{\setopc\brp{\cmpA \oplus\setB}} \seti \brs{\setopc\brp{\cmpA\oplus\setC}}
    && \text{by Demorgan relation \prefpo{thm:mph_addsub}}
  \\&= \brp{\setA\ominus\setB} \seti \brp{\setA \ominus\setC}
    && \text{by \prefp{thm:mph_addsub}}
  \\
  \\
  \brp{\setA\seti\setB} \ominus\setC
    &= \setopc\brs{\brp{\setA\seti\setB} \ominus\setC}
    && %\text{by \prefp{thm:set_involutary}}
  \\&= \setopc\brs{ \setopc(\setA\seti\setB) \oplus\setC }
    && \text{by \prefp{thm:mph_addsub}}
  \\&= \setopc\brs{(\cmpA \setu \cmpB) \oplus\setC }
    && %\text{by \prefp{thm:set_involutary}}
  \\&= \setopc\brs{(\cmpA\oplus\setC) \setu (\cmpB\oplus\setC)}
    && \text{by \prefp{thm:mph_add_distrib}}
  \\&= \setopc(\cmpA\oplus\setC) \seti \setopc(\cmpB\oplus\setC)
    && %\text{by \prefp{thm:set_involutary}}
  \\&= (\setA \ominus\setC) \seti (\setB\ominus\setC)
    && \text{by \prefp{thm:mph_addsub}}
  \\
  \\
  \setA \ominus (\setB\seti\setC)
    &= \setopc\setopc\Big[\; A \ominus (\setB\seti\setC) \;\Big]
    && %\text{by \prefp{thm:set_involutary}}
  \\&= \setopc\Big[\; \cmpA \oplus (\setB\seti\setC) \;\Big]
    && \text{by \prefp{thm:mph_addsub}}
  \\&\supseteq \setopc\brs{ (\cmpA \oplus\setB) \seti (\cmpA\oplus\setC) }
    && \text{by \prefp{thm:mph_add_distrib}}
  \\&= \brs{\setopc(\cmpA \oplus\setB)} \setu \brs{\setopc(\cmpA\oplus\setC)}
    && \text{by Demorgan relation \prefpo{thm:mph_addsub}}
  \\&= \brp{\setA\ominus\setB} \setu \brp{\setA \ominus\setC}
    && \text{by \prefp{thm:mph_addsub}}
  \\
  \\
  \brp{\setA\setu\setB} \ominus\setC
    &= \setopc\setopc\brs{\brp{\setA\setu\setB} \ominus\setC }
    && %\text{by \prefp{thm:set_involutary}}
  \\&= \setopc\brs{\setopc\brp{\setA\setu\setB} \oplus\setC }
    && \text{by \prefp{thm:mph_addsub}}
  \\&= \setopc\brs{\brp{\cmpA \seti \cmpB} \oplus\setC }
    && \text{by Demorgan relation \prefpo{thm:mph_addsub}}
  \\&\supseteq \setopc\brs{\brp{\cmpA\oplus\setC} \seti \brp{\cmpB\oplus\setC}}
    && %\text{by \prefp{thm:set_involutary}}
  \\&= \setopc\brp{\cmpA\oplus\setC} \setu \setopc\brp{\cmpB\oplus\setC}
    && %\text{by \prefp{thm:set_involutary}}
  \\&= \brp{\setA \ominus\setC} \setu \brp{\setB\ominus\setC}
    && \text{by \prefp{thm:mph_addsub}}
  \\
  \\
  %  (\setA\seti\setB) \ominus\setC
  %    &= \setopi_{c\in\setC} (\setA\seti\setB)_c
  %    && \text{by \prefp{thm:mph_addsub}}
  %  \\&= \setopi_{c\in\setC} \set{x}{x\in\setA\seti\setB}_c
  %    && \text{by \prefp{def:set_ops}}
  %  \\&= \setopi_{c\in\setC} \set{x+c}{x\in\setA\seti\setB}
  %    && \text{by \prefp{def:mph_trans}}
  %  \\&= \setopi_{c\in\setC} \set{x+c}{x\in\setA \;\land x\in\setB}
  %    && \text{by \prefp{def:set_ops}}
  %  \\&= \setopi_{c\in\setC}\left(\set{x+c}{x\in\setA} \;\cap\; \set{x+c}{x\in\setB}\right)
  %    && \text{by \prefp{def:set_ops}}
  %  \\&= \left(\setopi_{c\in\setC} \set{x+c}{x\in\setA} \right) \;\cap\;
  %       \left(\setopi_{c\in\setC} \set{x+c}{x\in\setB} \right)
  %    && \text{by \prefp{thm:set_ops}}
  %  \\&= \left(\setopi_{c\in\setC} \set{x}{x\in\setA}_c \right) \;\cap\;
  %       \left(\setopi_{c\in\setC} \set{x}{x\in\setB}_c \right)
  %    && \text{by \prefp{def:mph_trans}}
  %  \\&= \left(\setopi_{c\in\setC}\setA_c \right) \;\cap\;
  %       \left(\setopi_{c\in\setC}\setB_c \right)
  %  \\&= (\setA \ominus\setC) \;\cap\; (\setB\ominus\setC)
  %    && \text{by \prefp{thm:mph_addsub}}
\end{align*}
\end{proof}


%---------------------------------------
\begin{theorem}
\label{thm:mph_addsub_distrib}
\citepg{pitas1991}{166}{0792390490}
%---------------------------------------
Let $\opair{\setX}{+}$ be a group with with
Minokowski addition    operator $\oplus:\setX^2\to\setX$   and
Minokowski subtraction operator $\ominus:\setX^2\to\setX$.
\thmbox{\begin{array}{rcl>{\qquad}C>{\qquad}D}
    \setA \ominus (\setB\oplus\setC) &=& (\setA\ominus\setB) \ominus\setC
    & \forall \setA,\setB,\setC\subseteq\setX
    &
    \\
    A \oplus (\setB\ominus\setC) &\subseteq& (\setA\oplus\setB) \ominus\setC
    & \forall \setA,\setB,\setC\subseteq\setX
    &
  \end{array}}
\end{theorem}
\begin{proof}
\begin{align*}
  \setA \ominus (\setB\oplus\setC)
    &= \setopc\setopc\Big[\; A \ominus (\setB\oplus\setC) \;\Big]
    && %\text{by \prefp{thm:set_involutary}}
  \\&= \setopc\Big[\; \cmpA \oplus (\setB\oplus\setC) \;\Big]
    && \text{by \prefp{thm:mph_addsub}}
  \\&= \setopc\Big[\; (\cmpA \oplus\setB) \oplus\setC \;\Big]
    && \text{by \prefp{thm:mph_add}}
  \\&= \setopc(\cmpA \oplus\setB) \ominus\setC
    && \text{by \prefp{thm:mph_addsub}}
  \\&= (\setA\ominus\setB) \ominus\setC
    && \text{by \prefp{thm:mph_addsub}}
  \\
  \\
  \setA \oplus \brp{\setB\ominus\setC}
    &= \setA \oplus \brp{\setopi_{c\in\setC}\opT_c\setB}
    && \text{by \prefp{def:mph_add}}
  \\&= \brp{\setopi_{c\in\setC}\opT_c\setB} \oplus\setA
    && \text{by \prefp{thm:mph_add}}
  \\&= \setopu_{a\in\setA} \opT_a \brp{\setopi_{c\in\setC}\opT_c\setB }
    && \text{by \prefp{def:mph_add}}
  \\&= \setopu_{a\in\setA}\setopi_{c\in\setC}\opT_a \opT_c\setB
    && \text{by \prefp{def:mph_trans}}
  \\&\subseteq \setopi_{c\in\setC}\setopu_{a\in\setA}\opT_a \opT_c\setB
    && \text{by minimax inequality \ifdochas{lattice}{\prefpo{thm:minimax_ineq}}}
  \\&= \setopi_{c\in\setC}\setopu_{a\in\setA}\opT_c \opT_a\setB
    && \text{by \prefp{def:mph_trans}}
  \\&= \setopi_{c\in\setC}\opT_c \brp{\setopu_{a\in\setA} \opT_a\setB}
    && \text{by \prefp{def:mph_trans}}
  \\&= \setopi_{c\in\setC}\opT_c \brp{\setB\oplus\setA}
    && \text{by \prefp{def:mph_add}}
  \\&= \brp{\setB\oplus\setA}\ominus\setC
    && \text{by \prefp{def:mph_add}}
  \\&= \brp{\setA\oplus\setB}\ominus\setC
    && \text{by \prefp{thm:mph_add}}
\end{align*}
\end{proof}




%=======================================
\section{Operations}
%=======================================
%---------------------------------------
\begin{definition}
\label{def:mph_sym}
%\citeppg{pitas1991}{158}{159}{0792390490}
\citepg{matheron1975}{17}{0471576212}
\index{set!symmetric}
%---------------------------------------
Let $\opair{\setX}{+}$ be a group.
\defbox{\begin{array}{>{$}l<{$}rcl>{\qquad}C>{\qquad}D}
  The {\hid{symmetric set}} of $\setA$ is the set
    & \check{\setA} &\eqd& -\setA
    & \forall \setA\subseteq\setX
  \end{array}}
\end{definition}

%---------------------------------------
\begin{definition}
\citepg{pitas1991}{161}{0792390490}
\index[xsym]{$\check{D}$}
%---------------------------------------
Let $\opair{\setX}{+}$ be a group with
Minokowski addition    operator $\oplus:\setX^2\to\setX$,
Minokowski subtraction operator $\ominus:\setX^2\to\setX$, and
$D^s$ be the symmetric set of set $D$.
\defbox{\begin{array}{>{$}l<{$} r@{\;}c@{\;}l @{\qquad}C}
  The {\hid{dilation}} of $\setA$ by $D$ is the operation
    & \setA &\oplus& \check{D}
    & \forall \setA,\setD\subseteq\setX.
  \\
  The {\hid{erosion}} of $\setA$ by $\setE$ is the operation
    & \setA &\ominus& \check{\setE}
    & \forall \setA,\setE\subseteq\setX.
  \\
\end{array}}
\end{definition}


%---------------------------------------
\begin{definition}
%\citepg{pitas1991}{167}{0792390490}
\citepg{serra1982}{50}{0126372403}
\label{def:mph_open}
\label{def:mph_close}
%---------------------------------------
Let $\opair{\setX}{+}$ be a group with
Minokowski addition    operator $\oplus:\setX^2\to \setX$,
Minokowski subtraction operator $\ominus:\setX^2\to \setX$, and
$\setB^s$ be the symmetric set of a set $\setB$.
\defbox{\begin{array}{>{$}l<{$} rcl @{\qquad}C}
  The {\hid{opening}} of $\setA$ with respect to $\setB$ is the set
    & \setA_\setB &\eqd& \mcom{\mcom{(\setA\ominus\check{\setB})}{erosion} \oplus\setB}{dilation}
    & \forall \setA,\setB\subseteq\setX.
  \\
  The {\hid{closing}} of $\setA$ with respect to $\setB$ is the set
    & \setA^\setB &\eqd& \mcom{\mcom{(\setA\oplus\check{\setB})}{dilation} \ominus \setB}{erosion}
    & \forall \setA,\setB\subseteq\setX.
\end{array}}
\end{definition}


%---------------------------------------
\begin{theorem}
\label{thm:mph_open_close}
%\citepg{pitas1991}{167}{0792390490}
\citepg{serra1982}{51}{0126372403}
%---------------------------------------
Let $\opair{\setX}{+}$ be a group with
$\setA_\setB$ representing the opening of a set $\setA$ with respect to a set $\setB$ and
$\setA^\setB$ representing the closing of a set $\setA$ with respect to a set $\setB$.
\thmbox{\begin{array}{
  >{$\scriptsize(}r<{)$\ds\rightarrow}
  @{\qquad} rcl @{\qquad}
  >{\ds\leftarrow$\scriptsize(}l<{)$}@{\qquad}
  @{\qquad}C}
  complement of the opening
    & \setopc(\setA_\setB) &=& (\cmpA)^\setB
    & closing of the complement
    & \forall \setA,\setB\subseteq\setX
  \\
  complement of the closing
    & \setopc(\setA^\setB) &=& (\cmpA)_\setB
    & opening of the complement
    & \forall \setA,\setB\subseteq\setX
\end{array}}
\end{theorem}
\begin{proof}
\begin{align*}
  \setopc(\setA_\setB)
    &= \setopc\Big[\; (\setA \ominus \check{\setB}) \oplus\setB \;\Big]
    && \text{by \prefp{def:mph_open}}
  \\&= \setopc(\setA \ominus \check{\setB}) \ominus \setB
    && \text{by \prefp{thm:mph_addsub}}
  \\&= \setopc(\setA \ominus \check{\setB}) \ominus \setB
    && \text{by \prefp{thm:mph_addsub}}
  \\&= (\cmpA \oplus \check{\setB}) \ominus \setB
    && \text{by \prefp{thm:mph_addsub}}
  \\&= (\cmpA)^\setB
    && \text{by \prefp{def:mph_close}}
  \\
  \\
  \setopc(\setA^\setB)
    &= \setopc\Big[\; (\setA \oplus \check{\setB}) \ominus \setB \;\Big]
    && \text{by \prefp{def:mph_close}}
  \\&= \setopc(\setA \oplus \check{\setB}) \oplus\setB
    && \text{by \prefp{thm:mph_addsub}}
  \\&= \setopc(\setA \oplus \check{\setB}) \oplus\setB
    && \text{by \prefp{thm:mph_addsub}}
  \\&= (\cmpA \ominus \check{\setB}) \oplus\setB
    && \text{by \prefp{thm:mph_addsub}}
  \\&= (\cmpA)_\setB
    && \text{by \prefp{def:mph_open}}
\end{align*}
\end{proof}




%---------------------------------------
\begin{example}
\label{ex:mph_trans_sym}
%---------------------------------------
\exboxt{
  $\setA \eqd \setn{ \opair{0}{0},\; \opair{1}{0},\; \opair{2}{0},\; \opair{0}{1} }$
  $\implies$
  \\\indentx
  $\brb{\begin{array}{lclD}
    \setA_{(-2, 1)} &=& \setn{\opair{-2}{1},\;\opair{-1}{1},\;\opair{0}{1},\;\opair{-2}{2}} & and\\
    \check{\setA}   &=& \setn{\opair{0}{0},\;\opair{-1}{0},\;\opair{-2}{0},\;\opair{0}{-1}}
  \end{array}}$
  \\
  %These relationships are illustrated next:
  %\\
%\begin{center}
\begin{fsL}
\setlength{\unitlength}{0.05mm}
\begin{tabular}{c@{\hspace{10mm}}c@{\hspace{10mm}}c}
$\setA$ & $\opT_{(-2, 1)}\setA$ & $-\setA$
\\\vspace{2mm}
\begin{picture}(440,440)(-220,-220)
  %\graphpaper[10](0,0)(600,200)
  \thinlines
  \color{axis}%
    \put(   0,-220){\line( 0, 1){440} }%
    \put(-220,   0){\line( 1, 0){440} }%
    \put( -10, 200){\makebox(0,0)[r]{ $2$ }}%
    \put( -10, 100){\makebox(0,0)[r]{ $1$ }}%
    \put( -10,-100){\makebox(0,0)[r]{ $-1$ }}%
    \put( -10,-200){\makebox(0,0)[r]{ $-2$ }}%
    \put( 200, -20){\makebox(0,0)[t]{ $2$ }}%
    \put( 100, -20){\makebox(0,0)[t]{ $1$ }}%
    \put(-100, -20){\makebox(0,0)[t]{ $-1$ }}%
    \put(-200, -20){\makebox(0,0)[t]{ $-2$ }}%
  \color{dot}%
    \put(   0, 100){\circle{30}}%
    \put(   0,   0){\circle*{30}}%
    \put( 100,   0){\makebox(0,0){$\boxempty$}}%
    \put( 200,   0){\makebox(0,0){$\Delta$}}%
\end{picture}
&
\begin{picture}(440,440)(-220,-220)
  %\graphpaper[10](0,0)(600,200)
  \thinlines
  \color{axis}%
    \put(   0,-220){\line( 0, 1){440} }%
    \put(-220,   0){\line( 1, 0){440} }%
    \put( -10, 200){\makebox(0,0)[r]{ $2$ }}%
    \put( -10, 100){\makebox(0,0)[r]{ $1$ }}%
    \put( -10,-100){\makebox(0,0)[r]{ $-1$ }}%
    \put( -10,-200){\makebox(0,0)[r]{ $-2$ }}%
    \put( 200, -20){\makebox(0,0)[t]{ $2$ }}%
    \put( 100, -20){\makebox(0,0)[t]{ $1$ }}%
    \put(-100, -20){\makebox(0,0)[t]{ $-1$ }}%
    \put(-200, -20){\makebox(0,0)[t]{ $-2$ }}%
  \color{dot}%
    \put(-200, 100){\circle*{30}}
    \put(-100, 100){\makebox(0,0){$\boxempty$}}%
    \put(   0, 100){\makebox(0,0){$\Delta$}}%
    \put(-200, 200){\circle{30}}
\end{picture}
&
\begin{picture}(440,440)(-220,-220)
  %\graphpaper[10](0,0)(600,200)
  \thinlines
  \color{axis}%
    \put(   0,-220){\line( 0, 1){440} }%
    \put(-220,   0){\line( 1, 0){440} }%
    \put( -10, 200){\makebox(0,0)[r]{ $2$ }}%
    \put( -10, 100){\makebox(0,0)[r]{ $1$ }}%
    \put( -10,-100){\makebox(0,0)[r]{ $-1$ }}%
    \put( -10,-200){\makebox(0,0)[r]{ $-2$ }}%
    \put( 200, -20){\makebox(0,0)[t]{ $2$ }}%
    \put( 100, -20){\makebox(0,0)[t]{ $1$ }}%
    \put(-100, -20){\makebox(0,0)[t]{ $-1$ }}%
    \put(-200, -20){\makebox(0,0)[t]{ $-2$ }}%
  \color{dot}%
    \put(   0,   0){\circle*{30}}%
    \put(   0,-100){\circle{30}}%
    \put(-100,   0){\makebox(0,0){$\boxempty$}}%
    \put(-200,   0){\makebox(0,0){$\Delta$}}%
\end{picture}
\end{tabular}
\end{fsL}
%\end{center}
}
\end{example}


%---------------------------------------
\begin{example}
\label{ex:mph_trans_sym_solid}
%---------------------------------------
An example similar to \prefpp{ex:mph_trans_sym} but using
solid shapes is illustrated next:
\exboxt{
\begin{fsL}
\setlength{\unitlength}{0.05mm}
\begin{tabular}{c@{\hspace{10mm}}c@{\hspace{10mm}}c}
$\setA$ & $A_{(-3,-2)}$ & $\check{\setA}$
\\\vspace{2mm}
\begin{picture}(640,440)(-320,-220)%
  %\graphpaper[10](0,0)(600,200)%
  \thinlines%
  \color{axis}%
    \put(   0,-220){\line( 0, 1){440} }%
    \put(-320,   0){\line( 1, 0){640} }%
    \put( -10, 200){\makebox(0,0)[r]{ $2$ }}%
    \put( -10, 100){\makebox(0,0)[r]{ $1$ }}%
    \put( -10,-100){\makebox(0,0)[r]{ $-1$ }}%
    \put( -10,-200){\makebox(0,0)[r]{ $-2$ }}%
    \put( 300, -20){\makebox(0,0)[t]{ $3$ }}%
    \put( 200, -20){\makebox(0,0)[t]{ $2$ }}%
    \put( 100, -20){\makebox(0,0)[t]{ $1$ }}%
    \put(-100, -20){\makebox(0,0)[t]{ $-1$ }}%
    \put(-200, -20){\makebox(0,0)[t]{ $-2$ }}%
    \put(-300, -20){\makebox(0,0)[t]{ $-3$ }}%
  \color{dot}%
    \thicklines%
    \put(   0,   0){\line(1, 0){300}}%
    \put(   0,   0){\line(0, 1){200}}%
    \put(   0, 200){\line(1, 0){100}}%
    \put( 100, 200){\line(0,-1){100}}%
    \put( 100, 100){\line(1, 0){200}}%
    \put( 300, 100){\line(0,-1){100}}%
\end{picture}
&
\begin{picture}(640,440)(-320,-220)%
  %\graphpaper[10](0,0)(600,200)
  \thinlines%
  \color{axis}
    \put(   0,-220){\line( 0, 1){440} }%
    \put(-320,   0){\line( 1, 0){640} }%
    \put( -10, 200){\makebox(0,0)[r]{ $2$ }}%
    \put( -10, 100){\makebox(0,0)[r]{ $1$ }}%
    \put( -10,-100){\makebox(0,0)[r]{ $-1$ }}%
    \put( -10,-200){\makebox(0,0)[r]{ $-2$ }}%
    \put( 300, -20){\makebox(0,0)[t]{ $3$ }}%
    \put( 200, -20){\makebox(0,0)[t]{ $2$ }}%
    \put( 100, -20){\makebox(0,0)[t]{ $1$ }}%
    \put(-100, -20){\makebox(0,0)[t]{ $-1$ }}%
    \put(-200, -20){\makebox(0,0)[t]{ $-2$ }}%
    \put(-300, -20){\makebox(0,0)[t]{ $-3$ }}%
  \color{dot}%
    \thicklines%
    \put(-300,-200){\line(1, 0){300}}%
    \put(-300,-200){\line(0, 1){200}}%
    \put(-300,   0){\line(1, 0){100}}%
    \put(-200,   0){\line(0,-1){100}}%
    \put(-200,-100){\line(1, 0){200}}%
    \put(   0,-100){\line(0,-1){100}}%
\end{picture}
&
\begin{picture}(640,440)(-320,-220)%
  %\graphpaper[10](0,0)(600,200)
  \thinlines%
  \color{axis}%
    \put(   0,-220){\line( 0, 1){440} }%
    \put(-320,   0){\line( 1, 0){640} }%
    \put( -10, 200){\makebox(0,0)[r]{ $2$ }}%
    \put( -10, 100){\makebox(0,0)[r]{ $1$ }}%
    \put( -10,-100){\makebox(0,0)[r]{ $-1$ }}%
    \put( -10,-200){\makebox(0,0)[r]{ $-2$ }}%
    \put( 300, -20){\makebox(0,0)[t]{ $3$ }}%
    \put( 200, -20){\makebox(0,0)[t]{ $2$ }}%
    \put( 100, -20){\makebox(0,0)[t]{ $1$ }}%
    \put(-100, -20){\makebox(0,0)[t]{ $-1$ }}%
    \put(-200, -20){\makebox(0,0)[t]{ $-2$ }}%
    \put(-300, -20){\makebox(0,0)[t]{ $-3$ }}%
  \color{dot}%
    \thicklines%
    \put(   0,   0){\line(-1, 0){300}}%
    \put(   0,   0){\line( 0,-1){200}}%
    \put(   0,-200){\line(-1, 0){100}}%
    \put(-100,-200){\line( 0, 1){100}}%
    \put(-100,-100){\line(-1, 0){200}}%
    \put(-300,-100){\line( 0, 1){100}}%
\end{picture}%
\end{tabular}
\end{fsL}
%\end{center}
}
\end{example}




