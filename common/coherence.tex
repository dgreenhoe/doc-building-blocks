%============================================================================
% Daniel J. Greenhoe
% LaTeX file
%============================================================================
%============================================================================
\chapter{Coherence}
%============================================================================
%---------------------------------------
\begin{definition}
\footnote{
  \citePp{chen2012}{4699}{(1), (2)},
  \citerppgc{liang2015}{363}{365}{1498702341}{7.4.2 Coherence function},
  \citerpgc{ewins1986}{131}{0863800173}{$\gamma^2=\ffrac{H_1(\omega)}{H_2(\omega)}$ (3.8)}
  }
\label{def:Cxy}
\label{def:oCxy}
%---------------------------------------
Let $\opS$ be a \structe{system} with input $\rvx(n)$ and output $\rvy(n)$.
\defbox{\begin{array}{MM>{\ds}rc>{\ds}l}
  The \fnctd{complex coherence} function &is defined as&
    \Cxy(\omega) &\eqd& \frac{\Swxy^\ast(\omega)}{\sqrt{\Swxx(\omega)\Swyy(\omega)}}
    \\
  The \fnctd{ordinary coherence} function &is defined as&
    \oCxy(\omega) &\eqd& \frac{\abs{\Swxy(\omega)}^2}{\Swxx(\omega)\Swyy(\omega)}
\end{array}}
\end{definition}

%---------------------------------------
\begin{proposition}
\label{prop:oCxy}
%---------------------------------------
\propbox{\begin{array}{>{\ds}rc>{\ds}l}
  \oCxy(\omega) &=& \frac{\estHa(\omega)}{\estHb(\omega)}
\end{array}}
\end{proposition}
\begin{proof}
\begin{align*}
  \oCxy(\omega) 
    &\eqd \frac{\abs{\Swxy}^2}{\Swxx\Swyy}
    && \text{by definition of $\oCxy$}
    && \text{\xref{def:oCxy}}
  \\&= \frac{\ds\frac{\Swxy^\ast}{\Swxx}}
            {\ds\frac{\Swyy}     {\Swxy}}
  \\&\eqd \frac{\estHa}
               {\estHb}
    && \text{by definitions of $\estHa$ and $\estHb$}
    && \text{\xxref{def:H1}{def:H2}}
\end{align*}
\end{proof}

%---------------------------------------
\begin{remark}
\label{rem:HgmTxyCxy}
%---------------------------------------
Note that the \fncte{complex transmissibility} $\Twxy'$ of \pref{rem:HgmTxy_gen}
provides a nice mathematical symmetry (always a good sign of good direction) with \fncte{coherence}
in the system identification family tree.
In particular, note that the following:
\remboxt{
  $\Cxy \eqd \sqrt{\frac{\estHa^\ast}{\estHb}}$ whereas $\Twxy' \eqd \sqrt{\estHa \estHb}$
  }
\end{remark}
\begin{proof}
\begin{align*}
  \sqrt{\frac{\estHa^\ast(\omega)}{\estHb(\omega)}}
    && \text{by definition of $\estHgm$}
    && \text{\xref{def:Hgm}}
  %\\&\eqd \sqrt{\frac{\Swxy^\ast(\omega)}{\Swxx(\omega)}\frac{\Swyy(\omega)}{\Swxy(\omega)}}
  %  && \text{by definitions of $\estHa$ and $\estHb$}
  %  && \text{\xxref{def:H1}{def:H2}}
  %\\&= \sqrt{\frac{\Swyy(\omega)}{\Swxx(\omega)}\frac{\Swxy^\ast(\omega)}{\Swxy(\omega)}}
  %\\&= \Twxy(\omega)\sqrt{\frac{\Swxy^\ast(\omega)}{\Swxy(\omega)}}
  %  && \text{by definition of $\Twxy$}
  %  && \text{\xref{def:Txy}}
  %\\&= \Twxy(\omega)\sqrt{\frac{\abs{\Swxy(\omega)}e^{-i\phi(\omega)}}{\abs{\Swxy(\omega)}e^{i\phi(\omega)}}}
  %  && \text{where $\Swxy(\omega)\eqd{\abs{\Swxy(\omega)}e^{-i\phi(\omega)}}$}
  %\\&= \Twxy(\omega)\sqrt{e^{-i2\phi(\omega)}}
  %\\&= \Twxy(\omega)\,e^{-i\phi(\omega)}
\end{align*}
\end{proof}

%=======================================
\section{Beware of estimators}
%=======================================
Estimators yield, as the name implies, estimates.
These estimates in general contain some error.

%---------------------------------------
\begin{example}[The K=1 Welch estimate of coherence]
%---------------------------------------
Suppose we have two \prope{uncorrelated} stationary sequences $\rvx(n)$ and $\rvy(n)$. Then, there
CSD $\Sxy(\omega)$ should be $0$ because
\begin{align*}
  \Sxy(\omega)
    &\eqd \opDTFT\pE\Rxy(m)
  \\&=    \opDTFT\pE\brs{x(n)y[n+m]}
  \\&=    \opDTFT\brs{\pEx(n)} \brs{\pEy[n+m]}
  \\&=    \opDTFT\brs{0} \brs{0}
  \\&=    0
\end{align*}

This will give a coherence of $0$ also:
\[ C(\omega) = \frac{\Sxy}{\sqrt{\Sxx\Syy}} = 0\]

However, the Welch estimate with $K=1$ will yield
\begin{align*}
  \abs{C(\omega)}
    &= \abs{\frac{\ds\Sxy}{\sqrt{\ds\Sxx\Syy}}}
  \\&= \abs{\frac{\ds (\opFT x)(\opFT y)^\ast}{\sqrt{\ds\abs{\opFT x}^2\abs{\opFT y}^2}}}
  \\&= 1
\end{align*}

\end{example}
