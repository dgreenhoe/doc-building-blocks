%============================================================================
% Daniel J. Greenhoe
% GNU Ocatave file
% 1st order low pass filter design
%============================================================================
%=======================================
%\chapter{Coefficient Calculation}
%=======================================
%----------------------------------------------------------------------------
\section{2nd Order \propb{low-pass} calculation---polar form}
%----------------------------------------------------------------------------
\begin{align*}
  \boxed{\Fh(\omega)}
    &= \brlr{\Zh(z)}_{z=e^{i\omega}}
  \\&= G\brs{\frac{\brp{z+1}^2}
                  {\brp{z-p}\brp{z-p^\ast}}}_{z=e^{i\omega}}
  \\&= G\brs{\frac{\brp{z+1}^2}
                  {\brp{z-re^{i\phi}}\brp{z-(re^{i\phi})^\ast}}}_{z=e^{i\omega}}
  \\&= G\brs{\frac{z^2 + 2z + 1}
                  {z^2 - 2r\cos(\phi)z + r^2}
            }_{z=e^{i\omega}}
\end{align*}

We need 3 equations to solve for the 3 unkowns $G$, $r$, and $\phi$

%---------------------------------------

Equation 1: Gain=1 at DC
%---------------------------------------

\begin{align*}
  1 &= \abs{\Fh(0)}
  \\&= G\brs{\frac{z^2 + 2z            + 1}
                  {z^2 - 2r\cos(\phi)z + r^2}}_{z=e^{i\omega},\,\omega=0}
  \\&= G\brs{\frac{1   + 2             + 1}
                  {1   - 2r\cos(\phi)  + r^2}}
    %&& \text{(see \prefpo{equ:trickdc})}
  \\
  \\&\implies \boxed{G = \frac{r^2 - 2r\cos(\phi) + 1}{4}}
\end{align*}


%---------------------------------------

Equation 2: Gain = $\sfrac{1}{2}$ at corner frequency
%---------------------------------------
{ \begin{align*}
  \boxed{\abs{\Fh(\omega)}^2}
    &= \abs{\Zh(z)}^2_{z=e^{i\omega}}
     = G^2\brs{\frac{z^2 + 2z            + 1}
                    {z^2 - 2r\cos(\phi)z + r^2}}^2_{z=e^{i\omega}}
  \\&= G^2\brlr{
         \brs{\frac{z^2 + 2z + 1} {z^2 - 2r\cos(\phi)z + r^2}}
         \brs{\frac{z^2 + 2z + 1} {z^2 - 2r\cos(\phi)z + r^2}}^\ast
         }_{z=e^{i\omega}}
  \\&= G^2\brlr{
         \brs{\frac{z^2       + 2z      + 1} {z^2       - 2r\cos(\phi)z      + r^2}}
         \brs{\frac{z^{2\ast} + 2z^\ast + 1} {z^{2\ast} - 2r\cos(\phi)z^\ast + r^2}}
         }_{z=e^{i\omega}}
  \\&= G^2\brlr{
         \frac{
         \brs{ \abs{z}^4                    + 2 \abs{z}^2z               +  z^2} +
         \brs{2\abs{z}^2z^\ast              + 4 \abs{z}^2                + 2z  } +
         \brs{z^{2\ast}                     + 2      z^\ast              + 1   }
         }{\begin{array}{cl}
            &\brs{\abs{z}^4 - 2r\cos(\phi)z\abs{z}^2 + r^2z^2}
         \\+&\brs{-2r\cos(\phi)\abs{z}^2z^\ast +4r^2\cos^2(\phi)\abs{z}^2 -2r^3\cos(\phi)z}
         \\+&\brs{r^2z^{2\ast} - 2r^3\cos(\phi)z^\ast + r^4}
         \end{array}
         }}_{z=e^{i\omega}}
  \\&= G^2\brlr{
         \frac{\abs{z}^4 + 2           \abs{z}^2(z+z^\ast) +    (z^2+z^{2\ast}) + 4               \abs{z}^2 + 2             (z+z^\ast) + 1  }
              {\abs{z}^4 - 2r\cos(\phi)\abs{z}^2(z+z^\ast) + r^2(z^2+z^{2\ast}) + 4r^2\cos^2(\phi)\abs{z}^2 - 2r^3\cos(\phi)(z+z^\ast) + r^4}
         }_{z=e^{i\omega}}
  \\&= G^2\brs {
         \frac{1         + 4           \cos(\omega)        + 2   \cos(2\omega)  + 4                         +4             \cos(\omega) + 1  }
              {1         - 4r\cos(\phi)\cos(\omega)        + 2r^2\cos(2\omega)  + 4r^2\cos^2(\phi)          -4r^3\cos(\phi)\cos(\omega) + r^4}
             }
  \\&= G^2\brs {
         \frac{2   \boxed{\cos(2\omega)}  + 8                  \boxed{\cos(\omega)} +  \boxed{6}}
              {2r^2\boxed{\cos(2\omega)}  - 4r\cos(\phi)[1+r^2]\boxed{\cos(\omega)} +  \boxed{r^4 +  4r^2\cos^2(\phi) + 1}}
             }
\end{align*}


$\ds\boxed{
  G^2\brs {
         \frac{2   \boxed{\cos(2\omega_c)}  + 8                  \boxed{\cos(\omega_c)} +  \boxed{6}}
              {2r^2\boxed{\cos(2\omega_c)}  - 4r\cos(\phi)[1+r^2]\boxed{\cos(\omega_c)} +  \boxed{r^4 +  4r^2\cos^2(\phi) + 1}}
             }
  = \frac{1}{2}
  }$}


%---------------------------------------

Equation 3: For more smoothness in passband, set 2nd derivative to 0.
%---------------------------------------
%---------------------------------------
\begin{remark}
%---------------------------------------
Who cares about the second derivative?
In mathematics, \propd{smoothness} of a function $\ff(x)$ is the number of 
derivatives $\dndxn\ff(x)$ that are continuous.
For an example, consider \ope{Hermite interpolation} \xxref{sec:interpo_hermite}{thm:interpo_hermite}.
\end{remark}


\begin{align*}
  0 &= \ddwddw\abs{\Fh(\omega)}^2_{\omega=0}
  \\&= \ddwddw G^2 \brs{\frac{\ff(\omega)}
                             {\fg(\omega)}
                       }_{\omega=0}
  \\&= \ddw    G^2 \brs{\frac{\ff'\fg + \ff \fg'}
                             {\fg^2}
                       }_{\omega=0}
    && \text{by \thme{product rule}}
  \\&=         G^2 \brs{\frac{(\ff''\fg + \ff'\fg' - \ff'\fg' -\ff\fg'')\fg^2 - (\ff'\fg - \ff\fg')(2\fg\fg')}
                             {\fg^4}
                       }_{\omega=0}
  \\&=         G^2 \brs{\frac{\ff''\fg -\ff\fg''}
                             {\fg^2}
                       }_{\omega=0}
     \qquad \implies \boxed{\brlr{\ff''\fg = \ff\fg''}_{\omega=0}}
\end{align*}
\ldots because $\ff'(0) = \fg'(0) = 0$ %(see \prefpo{equ:mrflat})



\begin{align*}
  &\implies \mcom{\brs{-8\cos(2\omega)  - 8\cos(\omega)}}
                 { $\ff''$}
            \mcom{\brs{2r^2\cos(2\omega)  - 4r\cos(\phi)[1+r^2]\cos(\omega) +  r^4 +  4r^2\cos^2(\phi) + 1}}
                 { $\fg$}
 \\&\qquad= \brs{\mcom{\brs{2\cos(2\omega)  + 8 \cos(\omega) +  6}}
                      { $\ff$}
                 \mcom{\brs{-8r^2\cos(2\omega)  + 4r\cos(\phi)[1+r^2]\cos(\omega) }}
                      { $\fg''$}
                   }_{\omega=0}
  \\&\implies \brs{-8  - 8}      \brs{ 2r^2 - 4r\cos(\phi)[1+r^2] +  r^4 +  4r^2\cos^2(\phi) + 1}
       \\&\qquad= \brs{2  + 8  +  6} \brs{-8r^2 + 4r\cos(\phi)[1+r^2]}
  \\&\implies\boxed{r^4 + \brs{4\cos^2(\phi)-6} r^2 + 1 = 0 }
  \\
\end{align*}


\begin{figure}
  \centering
  \includegraphics{../common/math/graphics/pdfs/iir2rphi.pdf}
\end{figure}
$\begin{array}{rc>{\ds}l}
  r^2 &=& \frac{-\brs{4\cos^2\phi-6} \pm \sqrt{\brs{4\cos^2\phi-6}^2 - 4}}{2}
    \\&=& \frac{-\brs{4\cos^2\phi-6} \pm \sqrt{\brs{16\cos^4\phi-48\cos^2\phi+36} - 4}}{2}
    \\&=& \frac{-2\brs{2\cos^2\phi-3} \pm 2\sqrt{4\cos^4\phi-12\cos^2\phi+8}}{2}
    \\&=& \brs{3-2\cos^2\phi} \pm \sqrt{4\cos^4\phi-12\cos^2\phi+8}
    \\&=& \brs{3-2\cos^2\phi} \pm 2\sqrt{\cos^4\phi-3\cos^2\phi+2}
    \\&=& \brs{3-2\cos^2\phi} \pm 2\sqrt{(2-\cos^2\phi)(1-\cos^2)}
    \\&=& \brs{\sqrt{1-\cos^2\phi} \pm \sqrt{2-\cos^2\phi}}^2
    \\&=& \brs{\sin\phi \pm \sqrt{2-\cos^2\phi}}^2
\end{array}$

%---------------------------------------
\begin{example}2nd order \propb{low-pass} with corner frequency $\omega_c=\frac{2}{3}\pi$
%---------------------------------------
\begin{align*}
  1 &= \abs{\Fh(0)}
  \\&= G\brs{\frac{z^2 + 2z            + 1}
                  {z^2 - 2r\cos(\phi)z + r^2}}_{z=e^{i\omega},\,\omega=0}
  \\&= G\brs{\frac{1   + 2             + 1}
                  {1   - 2r\cos(\phi)  + r^2}}
    %&& \text{(see \prefpo{equ:trickdc})}
  \\
  \\&\implies \boxed{G = \frac{r^2 - 2r\cos(\phi) + 1}{4}}
  \\
  \frac{1}{2}
  &= G^2\brs {
         \frac{2   \boxed{\cos(2\omega_c)}  + 8                  \boxed{\cos(\omega_c)} +  \boxed{6}}
              {2r^2\boxed{\cos(2\omega_c)}  - 4r\cos(\phi)[1+r^2]\boxed{\cos(\omega_c)} +  \boxed{r^4 +  4r^2\cos^2(\phi + 1}}
             }
\\&= \boxed{G^2\brs {
         \frac{2-\sqrt{3}}
              {r^4 + 2\cos(\phi)r^3  + [1-\sqrt{3}]r^2  + 2r\cos(\phi) + 1}
              }}
\end{align*}

{  We can combine these equations to eliminate $G$}


\begin{align*}
1/2
  &= \brs{\frac{r^2 - 2r\cos(\phi) + 1}{4}}^2
     \brs{\frac{2-\sqrt{3}}
               {r^4 + 2\cos(\phi)r^3  + [1-\sqrt{3}]r^2  + 2r\cos(\phi) + 1}}
\\&= \brs{\frac{2-\sqrt{3}}{16}}
     \frac{\brs{r^2 - 2r\cos(\phi) + 1}^2}
          {r^4 + 2\cos(\phi)r^3  + [1-\sqrt{3}]r^2  + 2\cos(\phi)r + 1}
\end{align*}

$\ds{
8\brs{r^4 + 2\cos(\phi)r^3  + [1-\sqrt{3}]r^2  + 2\cos(\phi)r + 1}
   = \brs{2-\sqrt{3}} \brs{r^2 - 2r\cos(\phi) + 1}^2
}$

$\ds{
8\brs{r^4 + 2\cos(\phi)r^3  + [1-\sqrt{3}]r^2  + 2\cos(\phi)r + 1}
   = \brs{2-\sqrt{3}} \brs{r^4 - 4\cos(\phi)r^3 + 6\cos^2(\phi)r^2 -4\cos(\phi)r  + 1}
}$

$\ds\boxed{
    \brs{6+\sqrt{3}}r^4
 + 4\brs{6-\sqrt{3}}\cos(\phi)r^3
 + \brs{8-8\sqrt{3}- 6\brp{2-\sqrt{3}}\cos^2(\phi)}r^2
 + 4\brs{18-\sqrt{3}}\cos(\phi)r
 +\brs{6+\sqrt{3}}
 =0
}$

\end{example}
