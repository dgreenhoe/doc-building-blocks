%============================================================================
% Daniel J. Greenhoe
% XeLaTeX file
% Orthologics
%============================================================================


%=======================================
\chapter{Probability}
%=======================================


%%---------------------------------------
%\begin{definition}
%\citetbl{
%  \citerpg{narici1971}{2}{0824714849},
%  \citerpg{endler1972}{5}{0387060707}
%  }
%%---------------------------------------
%Let $\intcc{0}{1}$ be a \structe{real interval}.
%Let $\latL\eqd\logicX$ be a \structe{logic} \xref{def:logic}.
%\defboxt{
%  A function $\fv$ in $\clF{\setX}{\intcc{0}{1}}$ is a \fnctd{valuation} on $\setX$ if
%  \\\indentx$\begin{array}{FrclCD}
%    1. & \fv(x)        &\oreld& 0                    & \forall x\in\setX   & and \\
%    2. & \fv(x)        &=&      0 \quad\iff\quad x=0 & \forall x\in\setX   & and \\
%    3. & \fv(x\meet y) &=&      \fv(x)\fv(y)         & \forall x,y\in\setX & and \\
%    4. & \fv(x\join y) &\orel&  \fv(x) + \fv(y)      & \forall x,y\in\setX &     \\
%  \end{array}$
%  \\
%  A \fncte{valuation} is \prope{nonarchimedean} if 
%  $\quad\begin{array}{rclC}
%    \fv(x\join y) &\orel&  \max\brs{\fv(x) + \fv(y)}  & \forall x,y\in\setX; 
%  \end{array}$
%  \\otherwise, it is \prope{archimedean}.
%  }
%\end{definition}

%%---------------------------------------
%\begin{definition}
%\citetbl{
%  \citePpc{greechie1971}{126}{{\scshape Definitions}},
%  \citePp{narens2014}{118}
%  }
%\label{def:prob}
%\label{def:ps}
%%---------------------------------------
%Let $\intcc{0}{1}$ be a \structe{real interval}.
%Let $\pse\eqd\logicX$ be a \structe{logic} \xref{def:logic}.
%%Let $\pso$ be any lattice containing $\pse$.
%\defboxt{%
%  A function $\psp\in\clF{\pse}{\intcc{0}{1}}$ is a \fnctd{probability} function on $\pse$ if
%  \\\indentx$\begin{array}{FrclCD}
%    2. & \psp(\lzero)   &=& 0                 &             & and\\
%    3. & \psp(\lid)     &=& 1                 &             & and\\ 
%    4. & x\ocop y \implies \psp(x\join y) &=& \psp(x) + \psp(y) & x,y\in\setX
%  \end{array}$
%  \\
%  The triple $\ps$ is a \structd{probability space} over $\pso$,\\
%  $\pso$ is the \structd{outcome space}, and $\pse$ is the \structd{event space}.
%  }
%\end{definition}

%%---------------------------------------
%\begin{definition}
%\citetbl{
%  \citerppgc{pap1995}{8}{9}{0792336585}{Definition 2.3(13)}
%  }
%\label{def:prob}
%\label{def:ps}
%%---------------------------------------
%Let $\intcc{0}{1}$ be a \structe{real interval}.
%Let $\latL\eqd\logicX$ be a \structe{logic} \xref{def:logic}.
%\defboxt{
%  A function $\fm$ in $\clF{\setX}{\intcc{0}{1}}$ is a \fnctd{measure} on $\latL$ if
%  \\$\begin{array}{>{\qquad}F>{\ds}rc>{\ds}lCDD}
%    1. & \fm(\lzero)   &=&      0                                          &                     &                               & and \\
%    2. & \fm(x)        &\oreld& 0                                          & \forall x\in\setX   & (\prope{nonnegative})         & and \\
%    3. & \mc{3}{l}{\ds x_n\meet x_m=\lzero,\,n\neq m\quad\implies\quad\fm\brp{\joinop_{n=1}^\infty x_n} =  \sum_{n=1}^\infty\fm(x_n)}  & \forall x_n\in\setX & (\prope{\txsigma-additive})   & . \\
%    %\mc{7}{M}{A \fncte{measure} is a \fnctd{probability} on $\latL$ if}\\
%    %4. & \fm(\lid) &=&  1  
%  \end{array}$
%  \\
%  A \fncte{measure} is a \fnctd{probability} on $\latL$ if
%  \quad$\begin{array}{Frcl}
%    4. & \fm(\lid) &=&  1  
%  \end{array}$\quad
%  \\
%  If $\psp$ is a \fncte{probability} on a logic $\latL$, then $\opair{\latL}{\psp}$ is a \structd{probability space}. % over $\pso$,
%  %$\pso$ is the \structd{outcome space}, and $\pse$ is the \structd{event space}.
%  }
%\end{definition}

%---------------------------------------
\begin{definition}
%\citetbl{
%  \citerppgc{pap1995}{8}{9}{0792336585}{Definition 2.3(13)}
%  }
\label{def:prob}
\label{def:ps}
%---------------------------------------
%Let $\intcc{0}{1}$ be a \structe{real interval}.
%Let $\latL\eqd\logicX$ be a \structe{logic} \xref{def:logic}.
Let $\latL\eqd\latnX$ be a \structe{lattice with negation} \xref{def:latn}.
Let $\distrib$ be the \rele{distributivity} relation \xref{def:Drel}.
\defboxt{
  A function $\psp$ in $\clFxr$ is a \fnctd{probability} on $\latL$ if
  \\$\begin{array}{>{\qquad}Flc lcl CD}
    1. &                   &        & \psp(\lzero)   &=&      0               & \mc{2}{D}{(\prope{nondegenerate}) and} \\
    2. &                   &        & \psp(\lid)     &=&      1               & \mc{2}{D}{(\prope{normalized}) and} \\
    3. & x\orel y          &\implies& \psp(x)        &\orel&  \psp(y)         & \forall x,y\in\setX & (\prope{monotone}) and \\
   %3. & x\meet y = \lzero &\implies& \psp(x\join y) &=&      \psp(x)+\psp(y) & \forall x,y\in\setX & (\prope{additive})      & .      
    4. & \brb{\begin{array}{lD}
           x\meet y = \lzero            & and\\
           \otriple{z}{x}{y}\in\distrib & $\forall z\in\setX$
         \end{array}} 
       &\implies& \psp(x\join y) &=&      \psp(x)+\psp(y) & \forall x,y\in\setX & (\prope{additive}).      
  \end{array}$
  \\
  If $\psp$ is a \fncte{probability} on a \structe{lattice with negation} $\latL$, 
  then $\opair{\latL}{\psp}$ is a \structd{probability space}. % over $\pso$,
  %$\pso$ is the \structd{outcome space}, and $\pse$ is the \structd{event space}.
  }
\end{definition}

%---------------------------------------
\begin{remark}
%---------------------------------------
\prefp{def:prob} (previous) is not any standard definition of the \fncte{probability function}.
On a \structe{Boolean lattice}, the \fnctd{measure-theoretic probability} function, due to A. N. Kolmogorov, is defined as%
  \citetbl{
    \citerppgc{billingsley1995}{22}{23}{0471007102}{Probability Measures},
    \citer{kolmogorov1933},
    \citerpc{kolmogorov1933e2}{16}{\structe{field of probability}},
    \citerppgc{pap1995}{8}{9}{0792336585}{Definition 2.3(13)}, %{,\prope{\txsigma-additive}},
    \citerpg{kalmbach1986}{27}{9971500094}%{\prope{\txsigma-additive},\prope{\txsigma-complete}}
    }
  \\\indentx$\begin{array}{Flc lcl CDD}
    (1). &                   &        & \psp(\lid)     &=&      1               &                     & (\prope{normalized})     & and \\
    (2). &                   &        & \psp(x)        &\oreld& 0               & \forall x\in\setX   & (\prope{nonnegative})& and \\
    (3). & \ds\meetop_{n=1}^\infty x_n=\lzero &\implies& \ds\psp\brp{\joinop_{n=1}^\infty x_n} &=&  \ds\sum_{n=1}^\infty\psp(x_n) & \forall x_n\in\setX & (\prope{\txsigma-additive})   & .      
  \end{array}$\\
The advantage of this definition is that $\psp$ is a \fncte{measure}, and hence all the power of measure theory 
is subsequently at one's disposal in using $\psp$.
However, it has often been argued that the requirement of \prope{\txsigma-additivity} is unnecessary for a probability function.
Even as early as 1930, de Finetti argued against it, in what became a kind of polite running debate with Fr/'echet.\citetbl{
  \citeP{definetti1930a},
  \citeP{frechet1930a},
  \citeP{definetti1930b},
  \citeP{frechet1930b},
  \citeP{definetti1930c},
  \citePpp{cifarelli1996}{258}{260}
  }
In fact, Kolmogorov himself provided some argument against \prope{\txsigma-additivity} when referring to the closely related 
\hie{Axiom of Continuity} saying,
``Since the new axiom is essential for infinite fields of probability only,
it is almost impossible to elucidate its empirical meaning\ldots
For, in describing any observable random process we can obtain only finite fields of probability.\ldots"
But in its support he added, ``This limitation has been found expedient in researches of the most diverse sort."\citetbl{
  \citerp{kolmogorov1933e2}{15}
  }

There are several other definitions of probability that only require \prope{additivity} rather than \prope{\txsigma-additivity}.
On a \structe{Boolean lattice}, the \fnctd{traditional probability} function is defined as%
  \citetbl{
    \citerppg{papoulis}{21}{22}{0070484775},
    \citerpc{kolmogorov1933e2}{2}{\textsection 1. Axioms I--V}
    }
  \\\indentx$\begin{array}{Flc lcl CDD}
    (1). &                   &        & \psp(\lid)     &=&      1               &                     & (\prope{normalized})     & and \\
    (2). &                   &        & \psp(x)        &\oreld& 0               & \forall x\in\setX   & (\prope{nonnegative})& and \\
    (3). & x\meet y = \lzero &\implies& \psp(x\join y) &=&      \psp(x)+\psp(y) & \forall x,y\in\setX & (\prope{additive})   & .      
  \end{array}$\\
This definition implies (on a \structe{Boolean lattice}) that 
  \\\indentx$\begin{array}{FrclCDD}
      (a). & \psp(\lzero)  &=&     0                              &                     & (\prope{nondegenerate})      & and
    \\(b). & \psp(x)       &\orel& 1                              & \forall x\in\setX   & (\prope{upper bounded})      & and 
    \\(c). & \psp(x)       &=&     1-\psp(\negat{x})              & \forall x\in\setX   &                              & and 
    \\(d). & \psp(x\join y)&\orel& \psp(x)+\psp(y)                & \forall x,y\in\setX & (\prope{subadditive})        & and 
    \\(e). & \psp(x\join y)&=&     \psp(x)+\psp(y)-\psp(x\meet y) & \forall x,y\in\setX &                              & and
    \\(f). & x\orel y \implies \psp(x)&\orel&\psp(y)              & \forall x,y\in\setX & (\prope{monotone})           & . 
  \end{array}$
  \\
On a \structe{distributive pseudocomplemented lattice}, the \fnctd{generalized probability} function has been defined as%
  \citetbl{
    \citePp{narens2014}{118},
    \citerg{narens2007}{9812708014}
    }
  \\$\begin{array}{>{\qquad}Frcl CDD}
    (1). &  \psp(\lzero)    &=&      0                                              & (\prope{nondegenerate}) & and \\
    (2). &  \psp(\lid)      &=&      1                                              & (\prope{normalized}) & and \\
    (3). &  0\orel\psp(\lid)&\orel&  1                                              &                & and \\
    (4). &  \psp(x\join y)  &=&\psp(x)+\psp(y)-\psp(x\meet y) & \forall x,y\in\setX &                &  .      
  \end{array}$
  \\
On an \structe{orthomodular lattice}, or a \structe{finite modular lattice}, 
the \fnctd{quantum probability} function is defined as%
  \citetbl{
    \citePpc{greechie1971}{126}{{\scshape Definitions}},
    \citePp{narens2014}{118}
    }
  \\$\begin{array}{>{\qquad}Flc lcl CDD}
    (1). &                   &        & \psp(\lzero)   &=&      0               &                     & (\prope{nondegenerate})& and \\
    (2). &                   &        & \psp(\lid)     &=&      1               &                     & (\prope{normalized})  & and \\
    (3). & x\ocop y          &\implies& \psp(x\join y) &=&      \psp(x)+\psp(y) & \forall x,y\in\setX & (\prope{additive})& .      
  \end{array}$
  \\
However, for lattices that are not \prope{distributive}, \prope{modular}, or \prope{orthomodular}, 
none of these definitions work out so well.
Take for example the \structe{O$_6$ lattice} with the ``very reasonable" probability function given in \prefpp{ex:ps_o6}.
This probability space $\opair{\text{O$_6$}}{\psp}$ fails to be any of the 4 probability functions defined in this Remark.
It fails to be a \fncte{measure-theoretic} or \fncte{traditional probability} function because 
\\\indentx$a\meet b=0$\qquad{but}\qquad$\psp(a\join b)=\psp(\lid)=1\neq\frac{1}{3}+\frac{1}{2}=\psp(a)+\psp(b)$ .\\
It fails to be a \fncte{generalized probability} function because 
\\\indentx$\psp(a\join b)=\psp(\lid)=1\neq\frac{1}{3}+\frac{1}{2}-0=\psp(a)+\psp(b)-\psp(0)=\psp(a)+\psp(b)-\psp(a\meet b)$ .\\
It fails to be an \fncte{quantum probability} function because 
\\\indentx$a\ocop b=0$\qquad{but}\qquad$\psp(a\join b)=\psp(\lid)=1\neq\frac{1}{3}+\frac{1}{2}=\psp(a)+\psp(b)$ .\\
In each of these cases, the function $\psp$ fails to be \prope{additive}.
The solution of \prefpp{def:prob} is simply to ``switch off" \prope{additivity} when the lattice is not \prope{distributive}.
This method is a little ``crude", but at least it allows us to define probability on a very wide class of lattices,
while retaining compatibility with the \prope{Boolean} case \xxxref{prop:ps_01}{prop:ps_ortho_xy}{prop:ps_boa_xy}.
\end{remark}

%---------------------------------------
\begin{proposition}
\citetbl{
  \citerpgc{papoulis}{21}{0070484775}{(2-11)}
  }
\label{prop:ps_01}
%---------------------------------------
Let $\opair{\latL}{\psp}$ be a \structe{probability space} \xref{def:ps}.
\propbox{
  \begin{array}{rc rclCDD}
   %  &     & \psp(\lzero) &=& 0     &                   & \\
    0 &\orel& \psp(x)      &\orel& 1 & \forall x\in\setX & 
  \end{array}
  }
\end{proposition}
\begin{proof}
  \begin{align*}
   % \psp(\lzero)
   %   &= \brlr{\psp(x)}_{x=\lzero}
   % \\&= \brlr{\psp(x\join y)-\psp(y)}_{x=\lzero}
   %   && \text{because $x\meet\lzero=\lzero$ and by \prope{additive} property of $\psp$ \xref{def:ps}}
   % \\&= \psp(\lzero\join y)-\psp(y)
   % \\&= \psp(y)-\psp(y)
   %   && \text{by definition of $\lzero$}
   % \\&= 0
   % \\
    0
      &= \psp(\lzero)
      && \text{by previous result}
    \\&\orel \psp(x)
      && \text{because $\lzero\orel x$ and \prope{monotone} property \xref{def:ps}}
    \\
    \psp(x)
      &\orel \psp(\lid)
      && \text{because $x\orel\lid$ and \prope{monotone} property \xref{def:ps}}
    \\&= \lid
      && \text{by property of $\psp$ \xref{def:ps}}
  \end{align*}
\end{proof}

%---------------------------------------
\begin{proposition}
\citetbl{
  \citerpgc{papoulis}{21}{0070484775}{(2-12)}
  }
\label{prop:ps_ortho_xy}
%---------------------------------------
Let $\opair{\latL}{\psp}$ be a \structe{probability space} \xref{def:ps}.
\propbox{
   \brb{\begin{array}{M}
     $\latL$ is\\
     \prope{orthocomplemented}
   \end{array}}
   \qquad\implies\qquad
   \brb{\begin{array}{rclC}
     \psp(x) &=& 1 - \psp(\negat{x}) & \forall x\in\setX
   \end{array}}
   }
\end{proposition}
\begin{proof}
    \begin{align*}
      1-\psp(\negat{x})
        &= \psp(\lid)-\psp(\negat{x})
        && \text{by \prefp{def:ps}}
      \\&= \psp(x\join\negat{x})-\psp(\negat{x})
        && \text{by \prope{excluded middle} property of \fncte{ortho negation} \xref{def:negor}}
      \\&= \psp(x)+\psp(\negat{x})-\psp(\negat{x})
        && \text{because $(x)(\negat{x})=\lzero$ and \prope{additive} property \xref{def:ps}}
      \\&= \psp(x)
    \end{align*}
\end{proof}

%---------------------------------------
\begin{proposition}
\citetbl{
  \citerpgc{papoulis}{21}{0070484775}{(2-13)},
  \citerppgc{feller1970}{22}{23}{0471257087}{(7.4),(7.6)}
  }
\label{prop:ps_boa_xy}
%---------------------------------------
Let $\opair{\latL}{\psp}$ be a \structe{probability space} \xref{def:ps}.
\propbox{
   \brb{\begin{array}{M}
     $\latL$ is \\
     \prope{Boolean}
   \end{array}}
   \implies
   \brbl{\begin{array}{FrclCD}
     1. & \psp(x\join y) &=&     \psp(x) + \psp(y) - \psp(x\meet y) & \forall x,y\in\setX & and\\
     2. & \psp(x\join y) &\orel& \psp(x) + \psp(y)                  & \forall x,y\in\setX & (\prope{Boole's inequality})
   \end{array}}
   }
\end{proposition}
\begin{proof}
\begin{enumerate}
  \item lemma: Proof that $\psp((\negat x)\meet y)=\psp(y)-\psp(x\meet y)$: \label{item:ps_boa_xy}
    \begin{align*}
      \psp(y)-\psp(xy)
        &= \psp(\lid\meet{y})-\psp(xy)
        && \text{by definition of $\lid$ and $\meet$ \xref{def:meet}}
      \\&= \psp\brs{(x\join\negat{x})y}-\psp(xy)
        && \text{by \prope{excluded middle} property of \structe{Boolean lattice}s}
      \\&= \psp(xy\join\negat{x}y)-\psp(xy)
        && \text{by \prope{distributive} property of \structe{Boolean lattice}s}
      \\&= \psp(xy)+\psp(\negat{x}y)-\psp(xy)
        && \text{because $(xy)(\negat{x}y)=\lzero$ and by \prope{additive} property \xref{def:ps}}
      \\&= \psp(\negat{x}y)
    \end{align*}

  \item Proof that $\psp(x\join y) = \psp(x) + \psp(y) - \psp(x\meet y)$:
    \begin{align*}
      \psp(x\join y) 
        &= \psp(x\join \negat{x}y)
        && \text{by property of \prope{Boolean lattice}s} 
      \\&= \psp(x) + \psp(\negat{x}y)
        && \text{because $(x)(\negat{x}y)=\lzero$ and by \prope{additive} property \xref{def:ps}}
      \\&= \psp(x) + \psp(y) - \psp(x\meet y)
        && \text{by \prefpp{item:ps_boa_xy}}
    \end{align*}
\end{enumerate}
\end{proof}

%%---------------------------------------
%\begin{definition}
%\label{def:probtypes}
%%---------------------------------------
%Let $\ps$ be a \structe{probability space} \xref{def:ps}.
%\defbox{\begin{array}{MM}%
%  $\ps$ is a  \structd{minimal probability space}         &if $\pse$ is a     \fncte{minimal logic}.         \\% &\xref{def:latn}.\\
%  $\ps$ is a  \structd{fuzzy probability space}           &if $\pse$ is a     \fncte{fuzzy logic}.           \\% &\xref{def:latn}.\\
%  $\ps$ is an \structd{intuitionalistic probability space}&if $\pse$ is an    \fncte{intuitionalistic logic}.\\% &\xref{def:latn}.\\
%  $\ps$ is a  \structd{De Morgan probability space}       &if $\pse$ is a     \fncte{De Morgan logic}.       \\% &\xref{def:latn}.\\
%  $\ps$ is an \structd{ortho probability space}           &if $\pse$ is an    \fncte{ortho logic}.           \\% &\xref{def:latn}.\\
%  $\ps$ is a  \structd{probability space}                 &if $\pse$ is a     \fncte{classic logic}.           % &\xref{def:latn}.\\
%\end{array}}
%\end{definition}

%%---------------------------------------
%\begin{definition}
%\citetbl{
%  %\citePppc{randall1970}
%  \citePppc{jeffcott1972}{641}{642}{\textsection2. Definitions},
%  \citerppg{chiara2004}{155}{156}{1402019785}
%  }
%%---------------------------------------
%Let $\latL\eqd\latocd$ be an \structe{algebraic structure}.
%\defboxt{
%  $\latL$ is a \structd{orthologic} if
%  \\\indentx$\begin{array}{FMD}
%    1. & $\latticeX$ is a \structe{lattice} with \structe{least upper bound} $\bid$ and \structe{greatest lower bound} $\bzero$ & and \\
%    2. & 
%  \end{array}$
%  }
%\end{definition}

%---------------------------------------
\exboxt{
  \begin{tabular}{m{\tw-48mm}}
\begin{example}%[\exm{/-Lukasiewicz 3-valued logic}/\exm{Kleene 3-valued logic}/\exm{RM$_3$ logic}]
\label{ex:prob_L3_kl}
%---------------------------------------
    As stated in \prefpp{ex:negat_L3_kl}, the function $\negat$ on the lattice $\latL$
    as illustrated to the right is a \fncte{Kleene negation} \xref{def:negkl}.
    Together with the probability function $\psp$, also illustrated to the right, the pair $\opair{\latL}{\psp}$
    is a \structe{probability space}.
\end{example}
  \end{tabular}%
  \hfill%
  \begin{tabular}{M}
  \psset{unit=5mm}%
  \gsize%============================================================================
% Daniel J. Greenhoe
% LaTeX file
% nominal unit = 5mm
%============================================================================
\begin{pspicture}(-3,-\latbot)(3,2.4)%
  %---------------------------------
  % settings
  %---------------------------------
  %\psset{%
  %  %
  %  }%
  %---------------------------------
  % nodes
  %---------------------------------
  \Cnode(0,2){t}%
  \Cnode(0,1){n}%
  \Cnode(0,0){f}%
  %---------------------------------
  % node connections
  %---------------------------------
  \ncline{t}{n}%
  \ncline{n}{f}%
  %---------------------------------
  % node labels
  %---------------------------------
  \uput[180](t) {$\lid=\negat\lzero$}%
  \uput[180](n) {$a=\negat{a}$}%
  \uput[180](f) {$\lzero=\negat\lid$}%
  \uput[0](t) {$\psp(\lid)=1$}
  \uput[0](n) {$\psp(a)=\frac{1}{2}$}
  \uput[0](f) {$\psp(\lzero)=0$}
\end{pspicture}%
  \end{tabular}
  }

%---------------------------------------
\exboxt{
  \begin{tabular}{m{\tw-82mm}}
\begin{example}
\label{ex:limpx_m2}
%---------------------------------------
    The \structe{lattice with negation} $\latL$ \xref{def:latn} illustrated to the right is a
    \structe{Boolean lattice}.
    Together with the probability function $\psp$, also illustrated to the right, the pair $\opair{\latL}{\psp}$
    is a \structe{probability space}.
\end{example}
  \end{tabular}%
  \hfill%
  \begin{tabular}{M}
   \psset{unit=5mm}
   \gsize%============================================================================
% Daniel J. Greenhoe
% LaTeX file
% lattice M2
%============================================================================
\begin{pspicture}(-3.8,-\latbot)(6.4,2.5)
  %---------------------------------
  % nodes
  %---------------------------------
  \Cnode(0,2){t}%
  \Cnode(-1,1){x}\Cnode(1,1){y}%
  \Cnode(0,0){b}%
  %---------------------------------
  % node connections
  %---------------------------------
  \ncline{t}{x}\ncline{t}{y}%
  \ncline{b}{x}\ncline{b}{y}%
  %---------------------------------
  % node labels
  %---------------------------------
  %\uput[10](t) {$\lid=\negat\lzero\quad\psp(\lid)=1$}%
  %\uput[0](y) {$b=\negat{a}\quad\psp(b)=\frac{2}{3}$}%
  %\uput[180](x) {$a=\negat{b}$}%
  %\rput[rb](-1,0){$\psp(a)=\frac{1}{3}$}%
  %\uput[-10](b) {$\lzero=\negat\lid\quad\psp(\lzero)=0$}%
  %
  \uput[10](t) {$\lid=\negat\lzero$}%
  \uput[0](y) {$b=\negat{a}$}%
  \uput[135](x) {$a=\negat{b}$}%
  \uput[-10](b) {$\lzero=\negat\lid$}%
  %
  \rput[l](3.6,2){$\psp(\lid)=1$}%
  \rput[l](3.6,1) {$\psp(b)=\frac{2}{3}$}%
  \rput[l](3.6,0){$\psp(\lzero)=0$}%
  \uput[225](x){$\psp(a)=\frac{1}{3}$}%
\end{pspicture}%
  \end{tabular}
  }

%---------------------------------------
\exboxt{
  \begin{tabular}{m{\tw-83mm}}
\begin{example}
\label{ex:ps_o6}
%---------------------------------------
    The \structe{lattice with negation} $\latL$ \xref{def:latn} illustrated to the right is an
    \structe{orthocomplemented O$_6$ lattice} \xref{def:o6}.
    Together with the probability function $\psp$, also illustrated to the right, the pair $\opair{\latL}{\psp}$
    is a \structe{probability space}.
\end{example}
  \end{tabular}%
  \hfill%
  \begin{tabular}{M}
   \psset{unit=5mm}
   %\gsize\input{../common/math/graphics/prob/lat6_o6_1=-0_a=-d_b=-c_c=-b_d=-a_0=-1_prob.tex}
   \gsize%============================================================================
% Daniel J. Greenhoe
% LaTeX file
% lattice O6
%============================================================================
\begin{pspicture}(-6,-\latbot)(6,3.2)
  %---------------------------------
  % nodes
  %---------------------------------
  \Cnode(0,3){t}
  \Cnode(-1,2){c}\Cnode(1,2){d}%
  \Cnode(-1,1){x}\Cnode(1,1){y}%
  \Cnode(0,0){b}
  %---------------------------------
  % node connections
  %---------------------------------
  \ncline{t}{c}\ncline{t}{d}%
  \ncline{c}{x}\ncline{d}{y}%
  \ncline{b}{x}\ncline{b}{y}%
  %---------------------------------
  % node labels
  %---------------------------------
  \uput[0](t) {$\lid=\negat\lzero\quad\psp(\lid)=1$}%     
  \uput[0](d) {$d=\negat{a}\quad\psp(d)=\frac{2}{3}$}%     
  \uput[180](c) {$\psp(c)=\frac{1}{2}\quad c=\negat{b}$}%     
  \uput[0](y) {$b=\negat{c}\quad\psp(b)=\frac{1}{2}$}%     
  \uput[180](x) {$\psp(a)=\frac{1}{3}\quad a=\negat{d}$}%     
  \uput[0](b) {$\lzero=\negat\lid\quad\psp(\lzero)=0$}%     
\end{pspicture}%
  \end{tabular}
  }


%%---------------------------------------
%\begin{example}
%\citetbl{
%  \citeP{rawling2000}
%  }
%%---------------------------------------
%quantum computing
%\end{example}
%
%
%\begin{survey}
%%\begin{enumerate}
%  \citeP{birkhoffjvn1936}
%  \citeP{goldblatt1974}
%  \citeP{bell1983}
%  \citeP{bell1986qlogic}
%  \citer{chiara2004}
%%\end{enumerate}
%\end{survey}
