%============================================================================
% Daniel J. Greenhoe
% XeLaTeX file
%============================================================================
%=======================================
\section{Outcome subspace sequences}
%=======================================
%=======================================
%\subsection{Correlation operation on outcome subspace sequences}
%=======================================
%=======================================
\subsection{Definitions}
%=======================================
%---------------------------------------
\begin{definition}
\label{def:ocsseqmetric}
%---------------------------------------
Let $\Dom_1$ and $\Dom_2$ be \structe{convex subset}s \xref{def:convex} of $\Z$.\\
Let $\Dom\eqd\intoo{\meetop\Dom_1-\joinop\Dom_2-1}{\joinop\Dom_1-\meetop\Dom_2+1}.$
Let $\seq{x_n}{\Dom_1}$ and $\seq{y_n}{\Dom_2}$ be \structe{sequence}s over an \structe{outcome subspace} $\ocsD$.
\defboxp{
  The \fnctd{outcome subspace sequence metric} $\metrica{\seqn{x_n}}{\seqn{y_n}}$ is defined as
  \\\indentx
  $\ds\metrica{\seqn{x_n}}{\seqn{y_n}} \eqd \sum_{n\in\Dom} \ff(n)$
  \quad where \quad
  $
  \ff(n) \eqd \brb{\begin{array}{lM}
    \ocsd(x_n,y_n) & if $n\in   \Dom_1$ and $n\in   \Dom_2$\\
    1              & if $n\in   \Dom_1$ but $n\notin\Dom_2$\\
    1              & if $n\notin\Dom_1$ but $n\in   \Dom_2$\\
    0              & otherwise
  \end{array}}\qquad{\scy\forall n\in\Dom}
  $
  }
\end{definition}

%---------------------------------------
\begin{proposition}
%---------------------------------------
Let $\seq{x_n}{\Dom_1}$ and $\seq{y_n}{\Dom_2}$ be \structe{sequence}s over an \structe{outcome subspace} $\ocsD$.\\
Let $\metrican$ be the \fncte{outcome subspace sequence metric}.
\propbox{
  \Dom_1\seti\Dom_2\neq\emptyset \quad\implies\quad \text{$\metrican$ is a \fncte{metric}}
  }
\end{proposition}
\begin{proof}
This follows from the \thme{Fr{\'e}chet product metric} \xref{prop:fpm}.
In particular, $\metrican$ is a sum of metrics that include
the metrics $\ocsd(x_n,y_n)$ and the \fncte{discrete metric} \xref{def:dmetric}.
%This follows directly from either of the following two \fncte{product metric}s:
%\begin{liste}
%  \item Proof using \thme{Fr{\'e}chet product metric}:
%    \begin{align*}
%      \metrica{\seqn{x_n}}{\seqn{y_n}} 
%        &\eqd \sum_{n\in\Dom} \ocsd(x_n,y_n)
%        && \text{by \prefp{def:ocsseqmetric}}
%      \\&= \sum_{n\in\Dom} \alpha_n\ocsd(x_n,y_n)
%        && \text{for $\alpha_1=\alpha_2=\cdots=\alpha_\xN=1$}
%      \\&= \sum_{n\in\Dom} \alpha_n\ocsd_n(x_n,y_n)
%        && \text{for $\ocsd_1=\ocsd_2=\cdots=\ocsd_\xN=\ocsd$}
%      \\&\implies\text{$\metrican$ is a \fncte{metric}}
%        && \text{by \prope{Fr{\'e}chet product metric} \xref{prop:fpm}}
%    \end{align*}
%
%  \item Proof using \thme{Power mean metric}:
%    \begin{align*}
%      \metrica{\seqn{x_n}}{\seqn{y_n}} 
%        &\eqd \sum_{n\in\Dom} \ocsd(x_n,y_n)
%        && \text{by \prefp{def:ocsseqmetric}}
%      \\&= \xN\sum_{n\in\Dom} \lambda_n\ocsd(x_n,y_n)
%        && \text{for $\lambda_1=\lambda_2=\cdots=\lambda_\xN=\frac{1}{\xN}$}
%      \\&\implies\text{$\metrican$ is a \fncte{metric}}
%        && \text{by \prope{Power mean metric} \xref{prop:pmm}}
%      \\&&&\text{\qquad and \prope{$\alpha$-scaled metric} \xref{prop:asm}}
%    \end{align*}
%
%\end{liste}
\end{proof}

In standard signal processing, the \ope{autocorrelation} of a \structe{sequence} $\seqnD{x_n}$ 
is another \structe{sequence} $\seqnD{y_n}$ defined as 
$y_n\eqd\sum_{m\in\Z}x_m x_{m-n}$.
However, this definition requires that the sequence $\seqnD{x_n}$ be constructed over a \structe{field}.
In an \structe{outcome subspace sequence}, we in general do not have a \structe{field};
for example, in a \structe{die outcome subspace}, the expressions $\dieA+\dieB$ and $\dieA\times\dieB$ are undefined.
This paper offers an alternative definition (next) for \ope{autocorrelation} that uses the \fncte{distance} $\ocsd$ 
and that does not require a \structe{field}. 
%The autocorrelation function is used in \prefpp{ex:rdie_lp} and illustrated in \prefpp{fig:rdie_auto}.
%---------------------------------------
\begin{definition}
\label{def:ocsRxx}
%---------------------------------------
Let $\seqnD{x_n}$ and $\seqnD{y_n}$ be \fncte{sequence}s 
%\structe{outcome subspace sequence}s with \prope{finite support} \xref{def:support}
over the \structe{outcome subspace} $\ocsD$.
%Let $\xN$ be any whole number such that $\xN\ge\max(\seto{\support\seqn{x_n}},\seto{\support\seqn{y_n}})$, 
%where $\support\seqn{x_n}$ is the \ope{support} of $\seqn{x_n}$
%and   $\seto{\setX}$ is the \fncte{order} of a finite set $\setX$ \xref{def:seto}.
%Let $\metricbn$ be the \fncte{discrete metric} \xref{def:dmetric}.
%Let $\xN\eqd\sum_{n\in\Dom}\metricb{x_n}{\ocsz}$.
\\Let $\metrica{\seqn{x_n}}{\seqn{y_n}}$ be the \fnctd{outcome subspace sequence metric} \xref{def:ocsseqmetric}.
\\The \fnctd{cross-correlation} $\Rxy(n)$ of $\seqn{x_n}$ and $\seqn{y_n}$ and 
the \fnctd{autocorrelation}  $\Rxx(n)$ of $\seqn{x_n}$ 
are defined as
\defbox{\begin{array}{rclD}
  \Rxy(n) &\eqd& \ds - \sum_{m\in\Z}\metrica{\seqn{x_{m-n}}}{\seqn{y_m}} &\qquad (\fnctd{cross-correlation} $\Rxy(n)$ of $\seqn{x_n}$ and $\seqn{y_n}$)\\
  \Rxx(n) &\eqd& \ds - \sum_{m\in\Z}\metrica{\seqn{x_{m-n}}}{\seqn{x_m}} &\qquad (\fnctd{autocorrelation}  $\Rxx(n)$ of $\seqn{x_n}$)
\end{array}}\\
Moreover, the \opd{$\xM$-offset autocorrelation} of $\seqn{x_n}$ and $\seqn{y_n}$ is here defined 
as $\Rxx(n)+\xM$ \xref{def:axn}.
\end{definition}

%=======================================
\subsection{Examples of symbolic sequence statistics}
%=======================================
%---------------------------------------
\begin{example}[\exmd{fair die sequence}]
\label{ex:fdie_sha}
%---------------------------------------
Consider the pseudo-uniformly distributed \structe{fair die} \xref{def:fdie} 
sequence generated by the C code\footnotemark 
%\begin{lstlisting}
%\\\indentx\lstinline{#include<stdlib.h>} \ldots \lstinline!srand(0x5EED); for(n=0; n<N; n++){x[n] = 'A' + rand()%6;}!\\
%
\\\begin{minipage}{85mm}%
\begin{lstlisting}
#include<stdlib.h>
...
srand(0x5EED);
for(n=0; n<N; n++){x[n] = 'A' + rand()%6;}
\end{lstlisting}
\end{minipage}%
\hspace{10mm}%
\begin{tabular}{lclcl}
  where & \lstinline!'A'! &represents& $\dieA$ &,
     \\ & \lstinline!'B'! &represents& $\dieB$ &,
     %\\ & \lstinline!'C'! &represents& $\dieC$ &,
     %\\ & \lstinline!'D'! &represents& $\dieD$ &,
     %\\ & \lstinline!'E'! &represents& $\dieE$ &, and
     \\ & $\vdots$
     \\ & \lstinline!'F'! &represents& $\dieF$ &.
\end{tabular}
\\
%where \lstinline{'A'} represents $\dieA$, \lstinline{'B'} represents $\dieB$, etc.
The resulting sequence is partially displayed here:
   \\\includegraphics{../common/math/sspplots/fdie_5eed_51_seq.pdf}\\
\footnotetext{For a more complete source code listing, see \prefpp{sec:src_die}}
This sequence constrained to a length of $\xN=2667\times6=16002$ elements is approximately \prope{uniformly distributed} 
and \prope{uncorrelated}, as illustrated next:
%calculated over %$(16000-16)-(16-1)+1=15970$ elements:
\\\begin{tabular}{|>{\scs}c|>{\scs}c|}
     \hline
     \includegraphics{../common/math/sspplots/fdie_5eed_16002_histo.pdf}%
    &\includegraphics{../common/math/sspplots/fdie_5eed_16002_auto.pdf}
    \\histogram & \ope{$2\xN$-offset autocorrelation}
   \\\hline
\end{tabular}
\end{example}

%---------------------------------------
\begin{example}[\exmd{real die sequence}]
\label{ex:rdie_sha}
%---------------------------------------
Consider the pseudo-uniformly distributed \structe{real die} \xref{def:rdie}
sequence generated as in \prefpp{ex:fdie_sha},
but with the real die metric rather than the fair die metric.
This change will not affect the distribution of the sequence, but it does affect 
the autocorrelation, as illustrated next:
\\\begin{tabular}{|>{\scs}c|>{\scs}c|}
     \hline
     \includegraphics{../common/math/sspplots/rdie_5eed_16002_histo.pdf}%
     &\includegraphics{../common/math/sspplots/rdie_5eed_16002_auto.pdf}
   \\\hline
\end{tabular}
\end{example}


%---------------------------------------
\begin{example}[\exmd{spinner sequence}]
\label{ex:spinner_sha}
%---------------------------------------
Consider the pseudo-uniformly distributed \structe{spinner} \xref{def:spinner} 
sequence generated by the C code\footnotemark 
%\begin{lstlisting}
%\\\indentx\lstinline{#include<stdlib.h>} \ldots \lstinline!srand(0x5EED); for(n=0; n<N; n++){x[n] = 'A' + rand()%6;}!\\
%
\\\begin{minipage}{85mm}%
\begin{lstlisting}
#include<stdlib.h>
...
srand(0x5EED);
for(n=0; n<N; n++){x[n] = 'A' + rand()%6;}
\end{lstlisting}
\end{minipage}%
\hspace{10mm}%
\begin{tabular}{lclcl}
  where & \lstinline!'A'! &represents& $\spinA$ &,
     \\ & \lstinline!'B'! &represents& $\spinB$ &,
     %\\ & \lstinline!'C'! &represents& $\spinC$ &,
     %\\ & \lstinline!'D'! &represents& $\spinD$ &,
     %\\ & \lstinline!'E'! &represents& $\spinE$ &, and
     \\ & $\vdots$
     \\ & \lstinline!'F'! &represents& $\spinF$ &.
\end{tabular}
\\
%where \lstinline{'A'} represents $\dieA$, \lstinline{'B'} represents $\dieB$, etc.
The resulting sequence is partially displayed here:
   \\\includegraphics{../common/math/sspplots/spin_5eed_51_seq.pdf}\\
\footnotetext{For a more complete source code listing, see \prefpp{sec:src_spinner}}
This sequence is in essence identical to the fair die sequence \xref{ex:fdie_sha} and real die sequence \xref{ex:rdie_sha}
and thus yields what is essentially an identical histogram.
But because the metric is different, the autocorrelation  is also different.
In particular, because the nodes of the spinner metric are on average farther apart with respect to the spinner metric,
the sequence is less correlated (with respect to the metric), as illustrated next:
\\\begin{tabular}{|>{\scs}c|>{\scs}c|}
     \hline
     \includegraphics{../common/math/sspplots/spin_5eed_16002_histo.pdf}%
    &\includegraphics{../common/math/sspplots/spin_5eed_16002_auto.pdf}
   \\\hline
\end{tabular}
\end{example}


%---------------------------------------
\begin{example}[\exmd{weighted real die sequence}]
\label{ex:wrdie_sha}
%---------------------------------------
Consider the non-uniformly distributed \structe{weighted real die} \xref{def:wrdie} sequence with 
    \\\indentx$\psp(\dieE)=0.75$ and $\psp(\dieA)=\psp(\dieB)=\psp(\dieC)=\psp(\dieD)=\psp(\dieF)=0.05$,\\
    generated by the C code\footnote{For a more complete source code listing, see \prefpp{sec:src_rdie}}
\\\begin{minipage}{85mm}%
\begin{lstlisting}
srand(0x5EED);
for(n=0; n<N; n++){ u=rand()%100;
  if     (u< 5) x[n]='A'; /* 00-04 */ 
  else if(u<10) x[n]='B'; /* 05-09 */ 
  else if(u<15) x[n]='C'; /* 10-14 */ 
  else if(u<20) x[n]='D'; /* 15-19 */ 
  else if(u<95) x[n]='E'; /* 20-94 */ 
  else          x[n]='F'; /* 95-99 */ }
\end{lstlisting}
\end{minipage}%
\hspace{10mm}%
\begin{tabular}{lclcl}
  where & \lstinline!'A'! &represents& $\dieA$ &,
     \\ & \lstinline!'B'! &represents& $\dieB$ &,
     \\ & \lstinline!'C'! &represents& $\dieC$ &,
     \\ & \lstinline!'D'! &represents& $\dieD$ &,
     \\ & \lstinline!'E'! &represents& $\dieE$ &, and
     \\ & \lstinline!'F'! &represents& $\dieF$ &.
\end{tabular}
\\
     The resulting sequence is partially displayed here:
  \\\includegraphics{../common/math/sspplots/wrdie_5eed_51_seq.pdf}\\
  Of course the resulting histogram, as illustrated below on the left, reflects the non-uniform distribution.
  Also note, as illustrated below on the right, that the weighted sequence is much more correlated 
  (as defined by \prefp{def:ocsRxx})
  as compared to the uniformly distributed 
  \structe{real die} sequence of \prefpp{ex:rdie_sha}.
  %The following two illustrations, calculated over $(160000/50)-1=3199$ elements, 
  %are based on the above sequence downsampled by a factor of 50
  %(making them better comparable to the results obtained in \pref{item:rdie_rhp_R1map} and \pref{item:rdie_hp_R3map} below.
     \\\begin{tabular}{|>{\scs}c|>{\scs}c|}
          \hline
          \includegraphics{../common/math/sspplots/wrdie_5eed_16002_histo.pdf}%
         &\includegraphics{../common/math/sspplots/wrdie_5eed_16002_auto.pdf}
         % \includegraphics{../common/math/graphics/pdfs/wdie_5eed_160000_down50_histo.pdf}%
         %&\includegraphics{../common/math/graphics/pdfs/wdie_5eed_160000_down50_auto.pdf}
         %\\histogram & autocorrelation
         %\\\mc{2}{|>{\scs}c|}{calculated over $(160000/50)-1=3199$ elements}
        \\\hline
     \end{tabular}
\end{example}

%---------------------------------------
\begin{example}[\exmd{weighted die sequence}]
\label{ex:wdie_sha}
%---------------------------------------
Consider the non-uniformly distributed \structe{weighted die} \xref{def:wdie} sequence 
generated as in \pref{ex:wrdie_sha}.
Of course the resulting histogram is identical to that of \pref{ex:wrdie_sha},
but because the distance function is different, the autocorrelation sequence is also different.
     \\\begin{tabular}{|>{\scs}c|>{\scs}c|}
          \hline
          \includegraphics{../common/math/sspplots/wdie_5eed_16002_histo.pdf}%
         &\includegraphics{../common/math/sspplots/wdie_5eed_16002_auto.pdf}
        \\\hline
     \end{tabular}
\end{example}

%---------------------------------------
\begin{example}[\exmd{weighted spinner sequence}]
\label{ex:wspin_sha}
%---------------------------------------
Consider the non-uniformly distributed die sequence with 
    \\\indentx$\psp(\spinE)=0.75$ and $\psp(\spinA)=\psp(\spinB)=\psp(\spinC)=\psp(\spinD)=\psp(\spinF)=0.05$,\\
    generated by the C code\footnote{For a more complete source code listing, see \prefpp{sec:src_spinner}}
\\\begin{minipage}{85mm}%
\begin{lstlisting}
srand(0x5EED);
for(n=0; n<N; n++){ u=rand()%100;
  if     (u< 5) x[n]='A'; /* 00-04 */ 
  else if(u<10) x[n]='B'; /* 05-09 */ 
  else if(u<15) x[n]='C'; /* 10-14 */ 
  else if(u<20) x[n]='D'; /* 15-19 */ 
  else if(u<95) x[n]='E'; /* 20-94 */ 
  else          x[n]='F'; /* 95-99 */ }
\end{lstlisting}
\end{minipage}%
\hspace{10mm}%
\begin{tabular}{lclcl}
  where & \lstinline!'A'! &represents& $\spinA$ &,
     \\ & \lstinline!'B'! &represents& $\spinB$ &,
     \\ & \lstinline!'C'! &represents& $\spinC$ &,
     \\ & \lstinline!'D'! &represents& $\spinD$ &,
     \\ & \lstinline!'E'! &represents& $\spinE$ &, and
     \\ & \lstinline!'F'! &represents& $\spinF$ &.
\end{tabular}
\\
  The resulting sequence and histogram is in essence the same as in %\prefpp{ex:wrdie_sha}.
  the \exme{weighted die sequence} example \xref{ex:wrdie_sha}.
  %\\\includegraphics{../common/math/sspplots/wspin_5eed_51_seq.pdf}\\
  %Of course the resulting histogram, as illustrated below on the left, is in essence identical to that of the 
  %\exme{weighted die sequence} example \xref{ex:wrdie_sha}.
  But note, as illustrated below on the right, that the \exme{weighted spinner sequence} of this example is 
  significantly less correlated than the \exme{weighted die sequence} of \prefpp{ex:wrdie_sha},
  presumably due to the larger \structe{range} \xref{def:range} of the spinner metric ($\setn{0,1,2,3}$)
  as compared to the \structe{range} of the \structe{weighted real die} metric ($\setn{0,1,2}$) and 
  the \structe{weighted die} metric ($\setn{0,1}$).
     \\\begin{tabular}{|>{\scs}c|>{\scs}c|}
          \hline
          \includegraphics{../common/math/sspplots/wspin_5eed_16002_histo.pdf}%
         &\includegraphics{../common/math/sspplots/wspin_5eed_16002_auto.pdf}
        \\\hline
     \end{tabular}
\end{example}


%---------------------------------------
\begin{example}[\exmd{Random DNA sequence}]
\label{ex:dna_5eed}
%---------------------------------------
Consider the pseudo-uniformly distributed DNA sequence generated by the C code\footnotemark 
\\\begin{minipage}{85mm}%
\begin{lstlisting}
srand(0x5EED);
for(n=0; n<N; n++){  r=rand()%4;
  switch(r){ case 0: x[n]='A'; break;
             case 1: x[n]='T'; break;
             case 2: x[n]='C'; break;
             case 3: x[n]='G'; break; }}
\end{lstlisting}
\end{minipage}%
\hspace{10mm}%
\begin{tabular}{lclcl}
  where & \lstinline!'A'! &represents& $\symA$ &,
     \\ & \lstinline!'T'! &represents& $\symT$ &,
     \\ & \lstinline!'C'! &represents& $\symC$ &, and
     \\ & \lstinline!'G'! &represents& $\symG$ &.
\end{tabular}
\footnotetext{For a more complete source code listing, see \prefpp{sec:src_dna}}
  \\\includegraphics{../common/math/sspplots/dna_5eed_51_seq.pdf}\\
This sequence constrained to a length of $4000\times4=16000$ elements is approximately \prope{uniformly distributed} and \prope{uncorrelated}, 
as illustrated next:
  \\\begin{tabular}{|>{\scs}c|>{\scs}c|}
       \hline
       \includegraphics{../common/math/sspplots/dna_5eed_16000_histo.pdf}%
      &\includegraphics{../common/math/sspplots/dna_5eed_16000_auto.pdf}%
      %\\histogram & autocorrelation
     \\\hline
  \end{tabular}
\end{example}

%---------------------------------------
\begin{example}[\exmd{SARS coronavirus DNA sequence}]
\label{ex:dna_sars}
%---------------------------------------
Consider the genome sequence (DNA sequence) for the SARS coronavirus with \hie{GenBank} accession number NC\_004718.3.\footnote{\citer{ncbiSars}} %identifier gi30271926.
This sequence is of length $\xN=29751$ and is partially displayed here, followed by its histogram and 
\ope{$2\xN$-offset autocorrelation} plots.
  \\\includegraphics{../common/math/sspplots/dna_sars_51_seq.pdf}\\
     \\\begin{tabular}{|>{\scs}c|>{\scs}c|}
          \hline
          \includegraphics{../common/math/sspplots/dna_sars_histo.pdf}%
         &\includegraphics{../common/math/sspplots/dna_sars_auto.pdf}
         %\\histogram & autocorrelation
        \\\hline
     \end{tabular}
\end{example}

%---------------------------------------
\begin{example}[\exmd{Ebola virus DNA sequence}]
\label{ex:dna_ebola}
%---------------------------------------
Consider the genome sequence (DNA sequence) for the Ebola virus with GenBank accession \lstinline{AF086833.2}.\footnote{\citer{ncbiEbola}}
This sequence is of length 18959 and is partially displayed here:
  \\\includegraphics{../common/math/sspplots/dna_ebola_51_seq.pdf}\\
     \\\begin{tabular}{|>{\scs}c|>{\scs}c|}
          \hline
          \includegraphics{../common/math/sspplots/dna_ebola_histo.pdf}%
         &\includegraphics{../common/math/sspplots/dna_ebola_auto.pdf}
         %\\histogram & autocorrelation
        \\\hline
     \end{tabular}
\end{example}

%---------------------------------------
\begin{example}[\exmd{SARS-COV-2 virus DNA sequence}]
\label{ex:dna_SARSCoV2}
% common\symseq\fasta>GenBank_MT072688-1_SARS-COV-2.fasta
%---------------------------------------
Consider the genome sequence (DNA sequence) for the SARS-COV-2 virus with GenBank accession 
\lstinline{MT072688.1}.\footnote{\citeP{sah2020}, \citer{ncbiSARSCoV2}}
This sequence is of length 29811. % and is partially displayed here:
   \\\includegraphics{../common/math/graphics/pdfs/dna_SARSCoV2_51_seq.pdf}
     \\\begin{tabular}{|>{\scs}c|>{\scs}c|}
          \hline
          \includegraphics{../common/math/graphics/pdfs/dna_SARSCoV2_histo.pdf}%
         &\includegraphics{../common/math/graphics/pdfs/dna_SARSCoV2_auto.pdf}
         %\\histogram & autocorrelation
        \\\hline
     \end{tabular}
\end{example}

%---------------------------------------
\begin{example}[\exmd{Bacterium DNA sequence}]
\label{ex:dna_mpbacterium}
%---------------------------------------
Consider the genome sequence (DNA sequence) for the bacterium 
\hie{Melissococcus plutonius strain 49.3 plasmid pMP19}
with GenBank accession \lstinline{NZ_CM003360.1}.\footnote{\citer{ncbiMP}}
This sequence is of length 19430 and is partially displayed here:
  \\\includegraphics{../common/math/sspplots/dna_mpbacterium_51_seq.pdf}\\
     \\\begin{tabular}{|>{\scs}c|>{\scs}c|}
          \hline
          \includegraphics{../common/math/sspplots/dna_mpbacterium_histo.pdf}%
         &\includegraphics{../common/math/sspplots/dna_mpbacterium_auto.pdf}
         %\\histogram & autocorrelation
        \\\hline
     \end{tabular}
\end{example}

%---------------------------------------
\begin{example}[\exmd{Papaya DNA sequence}]
\footnote{
  \citer{ncbiPapaya}
  }
\label{ex:dna_papaya}
%---------------------------------------
Consider the genome sequence segment for the fruit
\hie{carica papaya} with GenBank accession \lstinline{DS982815.1}.
This sequence is not a complete genome, rather it is a ``genomic scaffold" \xref{def:dnan}. 
As such, there are some elements for which the content is not known.
For these locations, the symbol $\symN$ is used.
In this particular sequence, there are 144 $\symN$ symbols.
This sequence is of length 15495 and is partially displayed here:
  \\\includegraphics{../common/math/sspplots/dna_papaya1446_51_seq.pdf}\\
     \\\begin{tabular}{|>{\scs}c|>{\scs}c|}
          \hline
          \includegraphics{../common/math/sspplots/dna_papaya1446_histo.pdf}%
         &\includegraphics{../common/math/sspplots/dna_papaya1446_auto.pdf}
         %\\histogram & autocorrelation
        \\\hline
     \end{tabular}
\end{example}

%=======================================
\section{Extending to distance linear spaces}
%=======================================
%=======================================
\subsection{Motivation}
%=======================================
\prefpp{sec:ocs} demonstrated how a stochastic process could be defined as an \structe{outcome subspace}
with \prope{order} and \prope{metric} structures.
\prefpp{ex:realdieXRYZ} reviewed an example of a \structe{real die outcome subspace} 
that was mapped through 4 different \fncte{random variable}s to 4 different \structe{weighted graph}s.
Two of these random variables ($\rvY$ and $\rvZ$) mapped to structures (\structe{weighted graphs}) that are very similar 
to the \structe{real die} with respect to order and metric geometry.
Two other random variables ($\rvW$ and $\rvX$) mapped to structures 
(the \structe{real line} and the \structe{integer line}) that are very dissimilar.
The implication of this example is that if we want statistics that closely model the underlying stochastic process,
then we should map to a structure that that an order structure and distance geometry similar to 
that of the underlying stochastic process, and not simply the one that is the most convenient.
Ideally, we would like to map to a structure that is \prope{isomorphic} \xref{def:isomorphic}
and \prope{isometric} \xref{def:isometric} to the structure of the stochastic process.

However, for sequence processing using very basic methods such as FIR filtering, Fourier analysis,
or wavelet analysis,
we would very much like to map into the \structe{real line} $\R^1$ or possibly some higher dimensional space
$\R^n$.
Because the real line is often very dissimilar to the stochastic process,
we are motivated to find structures in $\R^n$ that \emph{are} similar.
And that is what this section presents---mapping from a stochastic process $\ocsD$ 
into an \structe{ordered distance linear space} $\omlsRnD$ in which $\omlo$ is an extension of $\ocso$ and
$\omld$ is an extension of $\ocsd$.

Thus, for sequence processing on an \structe{outcome subspace} $\ocsD$, we would like to 
define a \fncte{random variable} $\rvX$ and an \structe{ordered distance linear space} $\omlsRnD$ 
that satisfy the following constraints: 
\begin{enumerate}
  \item The random variable maps the elements of $\ocso$ into $\R^n$ and
  \item the order relation $\omlr$ is an \prope{extension} to $\omlo$ of the order relation $\ocsr$ on $\ocso$ and
  \item the \fncte{distance} function $\omld$ is an \prope{extension} to $\omlo$ of the distance function $\ocsd$ on $\ocso$.
\end{enumerate}

%=======================================
\subsection{Some random variables}
%=======================================
In this section, we first define some \fncte{random variable}s \xref{def:ocsrv} 
that are used later in this paper.
%---------------------------------------
\begin{definition}
\label{def:rv_dietrad}
%---------------------------------------
The \opd{traditional die random variable} $\rvX$ maps from the set %\xref{def:rvt} 
\\$\setn{\dieA,\dieB,\dieC,\dieD,\dieE,\dieF}$ into the set $\R^1$
and is defined as\footnotemark
\\\indentx
  $\rvX(\dieA)\eqd 1$, 
  $\rvX(\dieB)\eqd 2$, 
  $\rvX(\dieC)\eqd 3$, 
  $\rvX(\dieD)\eqd 4$, 
  $\rvX(\dieE)\eqd 5$, and 
  $\rvX(\dieF)\eqd 6$.
\end{definition}

%---------------------------------------
\begin{definition}
\label{def:rv_diepam}
%---------------------------------------
The \opd{PAM die random variable} $\rvX$ maps from the set $\setn{\dieA,\dieB,\dieC,\dieD,\dieE,\dieF}$ into the set $\R^1$
and is defined as\footnotemark
\\\indentx
  $\rvX(\dieA)\eqd-2.5$, 
  $\rvX(\dieB)\eqd-1.5$, 
  $\rvX(\dieC)\eqd-0.5$, 
  $\rvX(\dieD)\eqd+0.5$, 
  $\rvX(\dieE)\eqd+1.5$, and 
  $\rvX(\dieF)\eqd+2.5$.
\end{definition}
\footnotetext{PAM is an acronym for \ope{pulse amplitude modulation} and is a standard technique in 
the field of digital communications.}

%---------------------------------------
\begin{definition}
\label{def:rv_dieqpsk}
%---------------------------------------
The \opd{QPSK die random variable} $\rvX$ maps from the set $\setn{\dieA,\dieB,\dieC,\dieD,\dieE,\dieF}$ into the set $\C^1$
and is defined as\footnotemark 
\\\indentx$\begin{array}{rclrclrcl}
    \rvX(\dieA) &\eqd& \exp\brp{ 30\times\frac{\pi}{180}i}, 
   &\rvX(\dieB) &\eqd& \exp\brp{ 90\times\frac{\pi}{180}i}, 
   &\rvX(\dieC) &\eqd& \exp\brp{150\times\frac{\pi}{180}i}, 
  \\\rvX(\dieD) &\eqd& \exp\brp{210\times\frac{\pi}{180}i}, 
   &\rvX(\dieE) &\eqd& \exp\brp{270\times\frac{\pi}{180}i},
   &\rvX(\dieF) &\eqd& \exp\brp{330\times\frac{\pi}{180}i}.
\end{array}$
\end{definition}
\footnotetext{QPSK is an acronym for \ope{quadrature phase shift keying} and is a standard technique in 
the field of digital communications.}

%---------------------------------------
\begin{definition}
\label{def:rv_dieR3}
%---------------------------------------
The \opd{$\R^3$ die random variable} $\rvX$ maps from the set $\setn{\dieA,\dieB,\dieC,\dieD,\dieE,\dieF}$ into the set $\R^3$
and is defined as 
\\\indentx$\begin{array}{rclrclrcl}
    \rvX(\dieA) &\eqd& \otriple{+1}{ 0}{ 0}, 
   &\rvX(\dieB) &\eqd& \otriple{ 0}{+1}{ 0}, 
   &\rvX(\dieC) &\eqd& \otriple{ 0}{ 0}{+1}, 
  \\\rvX(\dieD) &\eqd& \otriple{ 0}{ 0}{-1}, 
   &\rvX(\dieE) &\eqd& \otriple{ 0}{-1}{ 0},
   &\rvX(\dieF) &\eqd& \otriple{-1}{ 0}{ 0}.
\end{array}$
\end{definition}

%---------------------------------------
\begin{definition}
\label{def:rv_dieR6}
%---------------------------------------
The \opd{$\R^6$ die random variable} $\rvX$ maps from the set $\setn{\dieA,\dieB,\dieC,\dieD,\dieE,\dieF}$ into the set $\R^6$
and is defined as 
\\\indentx$\begin{array}{rclrclrcl}
    \rvX(\dieA) &\eqd& \osix{1}{0}{0}{0}{0}{0}, 
   &\rvX(\dieB) &\eqd& \osix{0}{1}{0}{0}{0}{0}, 
   &\rvX(\dieC) &\eqd& \osix{0}{0}{1}{0}{0}{0}, 
  \\\rvX(\dieD) &\eqd& \osix{0}{0}{0}{1}{0}{0}, 
   &\rvX(\dieE) &\eqd& \osix{0}{0}{0}{0}{1}{0},
   &\rvX(\dieF) &\eqd& \osix{0}{0}{0}{0}{0}{1}.
\end{array}$
\end{definition}

%---------------------------------------
\begin{definition}
\label{def:rv_spinR1}
%---------------------------------------
The \opd{$\R^1$ spinner random variable} $\rvX$ maps from the set $\setn{\spinA,\spinB,\spinC,\spinD,\spinE,\spinF}$ into the set $\R^1$
and is defined as\footnotemark
\\\indentx
  $\rvX(\spinA)\eqd 1$, 
  $\rvX(\spinB)\eqd 2$, 
  $\rvX(\spinC)\eqd 3$, 
  $\rvX(\spinD)\eqd 4$, 
  $\rvX(\spinE)\eqd 5$, and 
  $\rvX(\spinF)\eqd 6$.
\end{definition}

%---------------------------------------
\begin{definition}
\label{def:rv_spinqpsk}
%---------------------------------------
The \opd{QPSK spinner random variable} $\rvX$ maps from the set $\setn{\spinA,\spinB,\spinC,\spinD,\spinE,\spinF}$ into the set $\C^1$
and is defined as 
\\\indentx$\begin{array}{rclrclrcl}
    \rvX(\spinA) &\eqd& \exp\brp{-90\times\frac{\pi}{180}i}, 
   &\rvX(\spinB) &\eqd& \exp\brp{-30\times\frac{\pi}{180}i}, 
   &\rvX(\spinC) &\eqd& \exp\brp{ 30\times\frac{\pi}{180}i}, 
  \\\rvX(\spinD) &\eqd& \exp\brp{ 90\times\frac{\pi}{180}i}, 
   &\rvX(\spinE) &\eqd& \exp\brp{150\times\frac{\pi}{180}i},
   &\rvX(\spinF) &\eqd& \exp\brp{210\times\frac{\pi}{180}i}.
\end{array}$
\end{definition}

%---------------------------------------
\begin{definition}
\footnote{
  \citePp{galleani2010}{772}
  }
\label{def:rv_dnapam}
%---------------------------------------
The \opd{PAM DNA random variable} $\rvX$ maps from the set $\setn{\dnaA,\dnaC,\dnaG,\dnaT}$ into the set $\R^1$
and is defined as\footnotemark
\\\indentx
  $\rvX(\dnaA)\eqd-1.5$, 
  $\rvX(\dnaC)\eqd-0.5$, 
  $\rvX(\dnaT)\eqd+0.5$, 
  $\rvX(\dnaG)\eqd+1.5$. 
\end{definition}

%---------------------------------------
\begin{definition}
\footnote{
  \citePp{galleani2010}{772}
  }
\label{def:rv_dnaqpsk}
%---------------------------------------
The \opd{QPSK DNA random variable} $\rvX$ maps from the set $\setn{\dnaA,\dnaC,\dnaG,\dnaT}$ into the set $\C^1$
and is defined as 
\\\indentx$\begin{array}{rclrcl}
     \rvX(\dnaA) &\eqd& \exp\brp{ 45\times\frac{\pi}{180}i}, 
    &\rvX(\dnaC) &\eqd& \exp\brp{135\times\frac{\pi}{180}i}, 
   \\\rvX(\dnaG) &\eqd& \exp\brp{225\times\frac{\pi}{180}i}, 
    &\rvX(\dnaT) &\eqd& \exp\brp{315\times\frac{\pi}{180}i}.
\end{array}$
\end{definition}
\footnotetext{QPSK is an acronym for \ope{quadrature phase shift keying} and is a standard technique in 
the field of digital communications.}

%---------------------------------------
\begin{definition}
\label{def:rv_dnaR4}
%---------------------------------------
The \opd{$\R^4$ DNA random variable} $\rvX$ maps from the set $\setn{\dnaA,\dnaC,\dnaG,\dnaT}$ into the set $\R^4$
and is defined as\footnotemark
\\\indentx$\begin{array}{rclrclrclrcl}
     \rvX(\dnaA) &\eqd& \oquad{1}{0}{0}{0}, 
    &\rvX(\dnaC) &\eqd& \oquad{0}{1}{0}{0}, 
    &\rvX(\dnaT) &\eqd& \oquad{0}{0}{1}{0}, 
    &\rvX(\dnaG) &\eqd& \oquad{0}{0}{0}{1}
\end{array}$
\end{definition}
\footnotetext{This type of mapping has previously been used by \citeP{voss1992} in calculating the \ope{Voss Spectrum}, 
  (a kind of Fourier analysis) of DNA sequences.
  See also \citePp{galleani2010}{772}.
  }


%=======================================
\subsection{Some ordered distance linear spaces}
%=======================================
%---------------------------------------
\begin{definition}
\label{def:dieR1oml}
%---------------------------------------
The structure $\oml{\R^1}{\omlr}{\omld}$ is the \structd{$\R^1$ die distance linear space}
if $\omlr$ is the \rele{standard ordering relation} on $\R$, 
and $\omld(x,y)\eqd\abs{x-y}$ (the \fncte{Euclidean metric} on $\R$, \xrefnp{def:emetric}).
\end{definition}

%---------------------------------------
\begin{definition}
\label{def:dieR3oml}
%---------------------------------------
%Let $\rvX$ be the \opd{$\R^3$ die random variable} \xref{def:dieR3rv}.
The structure $\oml{\R^3}{\omlr}{\omld}$ is the \structd{$\R^3$ die distance linear space}
if $\omlr=\emptyset$, and $\omld$ is the 
$2$-scaled \fncte{Lagrange arc distance} $\omld$ defined as follows:
\quad$\omld(p,q) \eqd 2\metrican(p,q)$\\
where $\metrican$ is the \fncte{Lagrange arc distance} \xref{def:larc}.
\end{definition}
%\begin{figure}[h]
%  \centering%
%  \gsize%
\mbox{}\hfill%
  \begin{tabular}{c}\includegraphics{../common/math/graphics/pdfs/rdie_r3met.pdf}\end{tabular}%
  \hspace{15mm}%
  $\begin{tabstr}{0.75}\begin{array}{|c|*{6}{@{\hspace{2pt}}c}|}
    \hline
    \ocsd(x,y) & \dieA &\dieB &\dieC &\dieD &\dieE &\dieF
    \\\hline
      \dieA &    0   &   1   &   1   &   1   &   1   &   2
    \\\dieB &    1   &   0   &   1   &   1   &   2   &   1
    \\\dieC &    1   &   1   &   0   &   2   &   1   &   1
    \\\dieD &    1   &   1   &   2   &   0   &   1   &   1
    \\\dieE &    1   &   2   &   1   &   1   &   0   &   1
    \\\dieF &    2   &   1   &   1   &   1   &   1   &   0
    \\\hline
  \end{array}\end{tabstr}$
\hfill\mbox{}\\
%  \caption{\structe{distance linear space} $\opair{\R^3}{\ocsd'}$ for \structe{real die}\label{fig:rdie3dmet}}
%\end{figure}
Used together with the \fncte{$\R^3$ die random variable} $\rvX$ \xref{def:rv_dieR3},
the distance $\omld$ in the \structe{$\R^3$ die distance linear space} \xrefr{def:dieR3oml}
is an extension of $\ocsd$ in the \structe{real die outcome subspace} 
\\$\ocsG\eqd\ocsrdie$ \xref{def:rdie}.
We can also say that $\rvX$ is an \fncte{isometry} \xref{def:isometry} 
and that the two structures are \prope{isometric}.
For example, %in $\ocsG$, $\ocsd(\dieA,\dieB)=1$ and $\ocsd(\dieA,\dieF)=2$; likewise in $\omlD$, 
\\\indentx$\begin{array}{lclclclD}
  \omld\brs{\rvX(\dieA),\rvX(\dieB)} &=& \omld\brs{\otriple{1}{0}{0},\otriple{ 0}{1}{0}} &=& 1 &=& \ocsd(\dieA,\dieB) & and\\
  \omld\brs{\rvX(\dieA),\rvX(\dieF)} &=& \omld\brs{\otriple{1}{0}{0},\otriple{-1}{0}{0}} &=& 2 &=& \ocsd(\dieA,\dieF) & .
\end{array}$
\\
As for order, the mapping $\rvX$ is also \prope{order preserving} \xref{def:opreserve},
but trivially, because the \structe{real die outcome subspace} is \prope{unordered} \xref{def:unordered}.
But if we still honor the standard ordering on each dimension $\R$ in $\R^3$,
then the two structures are \prope{not isomorphic} \xref{def:isomorphic} because\footnote{%
  Note that while $\rvX^{-1}$ \xref{def:inverse} does not exist as a \structe{function}, it does exist as a \structe{relation}.
  }
the inverse $\rvX^{-1}$ 
is \prope{not order preserving} \xref{thm:latiso}---%
for example, $\rvX(\dieD)=\otriple{0}{0}{-1}\orel\otriple{0}{0}{1}=\rvX(\dieC)$, but $\dieD$ and $\dieC$
are \prope{incomparable} \xref{def:incomparable} in $\ocsG$.

%---------------------------------------
\begin{definition}
\label{def:spinR2oml}
%---------------------------------------
The structure $\oml{\R^2}{\omlr}{\omld}$ is the \structd{$\R^2$ spinner distance linear space}
if $\omlr=\emptyset$, and $\omld$ is the 
$3$-scaled \fncte{Lagrange arc distance} $\omld$ defined as follows:
\quad$\omld(p,q) \eqd 3\metrican(p,q)$\\
where $\metrican$ is the \fncte{Lagrange arc distance} \xref{def:larc}.
\end{definition}
%\begin{figure}[h]
%  \centering%
%  \gsize%
\mbox{}\hfill%
  \begin{tabular}{c}\includegraphics{../common/math/graphics/pdfs/spinner_r2.pdf}\end{tabular}%
  \hspace{15mm}%
  $\begin{tabstr}{0.75}\begin{array}{|c|*{6}{@{\hspace{5pt}}c}|}
    \hline
    \ocsd(x,y)    & \spinA & \spinB & \spinC & \spinD & \spinE & \spinF
    \\\hline
      \spinA      &    0   &   1    &    2   &    3   &    2   &    1
    \\\spinB      &    1   &   0    &    1   &    2   &    3   &    2
    \\\spinC      &    2   &   1    &    0   &    1   &    2   &    3
    \\\spinD      &    3   &   2    &    1   &    0   &    1   &    2
    \\\spinE      &    2   &   3    &    2   &    1   &    0   &    1
    \\\spinF      &    1   &   2    &    3   &    2   &    1   &    0
    \\\hline
  \end{array}\end{tabstr}$
\hfill\mbox{}\\
%  \caption{\structe{distance linear space} $\opair{\R^2}{\ocsd'}$ for \structe{spinner}\label{fig:rdie3dmet}}
%\end{figure}
Used together with the \fncte{QPSK spinner random variable} $\rvX$ \xref{def:rv_spinqpsk},
the distance $\omld$ in the \structe{$\R^2$ spinner distance linear space} \xref{def:spinR2oml}
is an extension of $\ocsd$ in the \structe{spinner outcome subspace} 
$\ocsG\eqd\ocsspin$ \xref{def:spinner}.
We can again say that $\rvX$ is an \fncte{isometry} %\xref{def:isometry} 
and that the two structures are \prope{isometric}.
For example, %in $\ocsG$, $\ocsd(\spinA,\spinB)=1$, $\ocsd(\spinA,\spinC)=2$, and $\ocsd(\spinA,\spinD)=3$; likewise in $\omlD$, 
\\\indentx$\begin{array}{lclclclD}
  \omld\brs{\rvX(\spinA),\rvX(\spinB)} &=& \omld\brs{\opair{0}{-1},\opair{\sfrac{\sqrt{3}}{2}}{-\sfrac{1}{2}}} &=& 1 &=& \ocsd(\spinA,\spinB) & and\\
  \omld\brs{\rvX(\spinA),\rvX(\spinC)} &=& \omld\brs{\opair{0}{-1},\opair{\sfrac{\sqrt{3}}{2}}{+\sfrac{1}{2}}} &=& 2 &=& \ocsd(\spinA,\spinC) & and\\
  \omld\brs{\rvX(\spinA),\rvX(\spinD)} &=& \omld\brs{\opair{0}{-1},\opair{0}{1}}                               &=& 3 &=& \ocsd(\spinA,\spinD) & .
\end{array}$
\\
The mapping $\rvX$ is again trivally \prope{order preserving}.
And if we again honor the standard ordering on each dimension $\R$ in $\R^2$,
then the two structures are \prope{not isomorphic} \xref{def:isomorphic} because
the inverse $\rvX^{-1}$ 
is \prope{not order preserving}---%
for example, $\rvX(\spinA)=\opair{0}{-1}\orel\opair{0}{1}=\rvX(\spinD)$, but $\spinA$ and $\spinD$
are \prope{incomparable} in $\ocsG$.

%The next definition introduces a \structe{metric linear space} for \ope{fair die sequence processing}
%that unlike in the \structe{real die} structure, the random variable mapping 
%is fully \prope{order preserving} (trivally).
%---------------------------------------
\begin{definition}
\label{def:fdieR6oml}
%---------------------------------------
The structure $\oml{\R^6}{\omlr}{\omld}$ is the \structd{$\R^6$ die distance linear space}
if $\omlr=\emptyset$, and $\omld$ is defined as 
$\ds\omld(p,q) \eqd \sfrac{\sqrt{2}}{2}\metrica{p}{q}$,
where $\metrican$ is the \fncte{Euclidean metric} on $\R^6$ \xref{def:emetric}.
\end{definition}

Used together with the \fncte{$\R^6$ die random variable} $\rvX$ \xref{def:rv_dieR6},
the distance $\omld$ in the \structe{$\R^6$ fair die distance linear space} \xref{def:fdieR6oml}
is an extension of $\ocsd$ in the \structe{fair die outcome subspace} 
$\ocsG\eqd\ocsrdie$ \xref{def:rdie}.
We can again say that $\rvX$ is an \fncte{isometry} %\xref{def:isometry} 
and that the two structures are \prope{isometric}.
For example, %in $\ocsG$, $\ocsd(\dieA,\dieB)=1$, $\ocsd(\dieA,\dieC)=1$, and $\ocsd(\dieA,\dieF)=1$; likewise in $\omlD$, 
\\\indentx$\begin{array}{lclclclD}
  \omld\brs{\rvX(\dieA),\rvX(\dieB)} &=& \omld\brs{\osix{1}{0}{0}{0}{0}{0},\osix{0}{1}{0}{0}{0}{0}} &=& 1 &=& \ocsd(\dieA,\dieB) & and\\
  \omld\brs{\rvX(\dieA),\rvX(\dieC)} &=& \omld\brs{\osix{1}{0}{0}{0}{0}{0},\osix{0}{0}{1}{0}{0}{0}} &=& 1 &=& \ocsd(\dieA,\dieC) & and\\
  \omld\brs{\rvX(\dieA),\rvX(\dieF)} &=& \omld\brs{\osix{1}{0}{0}{0}{0}{0},\osix{0}{0}{0}{0}{0}{1}} &=& 1 &=& \ocsd(\dieA,\dieF) & .
\end{array}$
\\
The mapping $\rvX$ is again trivally \prope{order preserving},
and the inverse $\rvX^{-1}$ is trivally \prope{order preserving} as well.
And so unlike the \structe{$\R^3$ die distance linear space} \xrefr{def:dieR3oml}
and the \structe{$\R^2$ spinner distance linear space} \xrefr{def:spinR2oml},
this pair of structures is \prope{isomorphic}.



