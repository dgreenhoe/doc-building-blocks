%============================================================================
% LaTeX File
% Daniel J. Greenhoe
%============================================================================
%======================================
\chapter{Communication channels}
%======================================
%======================================
\section{System model}
%======================================
\begin{figure}[ht] \color{figcolor}
\centering%
\setlength{\unitlength}{0.20mm}
\begin{picture}(700,150)(-100,-50)
  \thicklines
  %\graphpaper[10](0,0)(500,100)
  \put(-100 ,  60 ){\makebox( 100,0)[b]{$\su$} }
  \put(-100 ,  50 ){\vector(1,0){100} }

  \put(  00 ,  10 ){\makebox( 100, 80)[t]{transmit} }
  \put(  00 ,  10 ){\makebox( 100, 80)[c]{operation} }
  \put(  00 ,  10 ){\makebox( 100, 80)[b]{$\opT$} }
  \put(  00 ,  00 ){\framebox( 100,100){} }

  \put( 100 ,  60 ){\makebox( 100,0)[b]{$\fs(t;\vu)$} }
  \put( 100 ,  50 ){\vector(1,0){100} }

  \put( 200 ,  10 ){\makebox( 100, 80)[t]{channel} }
  \put( 200 ,  10 ){\makebox( 100, 80)[c]{operation} }
  \put( 200 ,  10 ){\makebox( 100, 80)[b]{$\opC$} }
  \put( 200 ,  00 ){\framebox(100,100){} }

  \put( 300 ,  60 ){\makebox( 100,0)[b]{$\fr(t;\vu)$} }
  \put( 300 ,  50 ){\vector(1,0){100} }

  \put( 400 ,  00 ){\framebox(100,100){} }
  \put( 400 ,  10 ){\makebox( 100, 80)[t]{receive} }
  \put( 400 ,  10 ){\makebox( 100, 80)[c]{operation} }
  \put( 400 ,  10 ){\makebox( 100, 80)[b]{$\opR$} }

  \put( 500 ,  60 ){\makebox( 100,0)[b]{$\sue$} }
  \put( 500 ,  50 ){\vector(1,0){100} }

  \put(- 90 , -10 ){\makebox( 0, 0)[tl]{$\vu\eqd\su$} }
  \put( 110 , -10 ){\makebox( 0, 0)[tl]{$\fs(t;\vu)=\opT\vu$} }
  \put( 310 , -10 ){\makebox( 0, 0)[tl]{$\fr(t;\vu)=\opC\opT\vu$} }
  \put( 510 , -10 ){\makebox( 0, 0)[tl]{$\sue=\opSys\vu=\opR\opC\opT\vu$} }

\end{picture}
\caption{
   Communication system model
   \label{fig:sys_model}
   }
\end{figure}

A communication system is an operator.
%\footnote{
%   \begin{tabular}[t]{lll}
%      \hie{operator}:                & \pref{def:operator} & page~\pageref{def:operator}\\
%      \hie{operator multiplication}: & \pref{def:op+x}     & page~\pageref{def:op+x}
%   \end{tabular}
%   }
$\opSys$ over an information sequence $\su$ that generates an
estimated information sequence $\sue$.
The system operator factors into a
receive operator $\opR$, a channel operator $\opC$, and
a transmit operator $\opT$ such that
   \[ \opSys = \opR\opC\opT. \]
The transmit operator
operates on an information sequence $\su$ to generate
a channel signal $\fs(t;\vu)$.
The channel operator operates on the transmitted signal $\fs(t;\vu)$
to generate the received signal $\fr(t;\vu)$.
The receive operator operates on the received signal $\fr(t;\vu)$
to generate the estimate $\sue$
(see \prefpp{fig:sys_model}).


%---------------------------------------
\begin{definition}
\label{def:comsys}
\index{communication system}
%---------------------------------------
Let $U$ be the set of all sequences $\su$ and let
\defbox{
\begin{array}{lllll}
   \imark & \opSys &:U    &\to U    & \text{ (system   operator)  } \\
   \imark & \opT   &:U    &\to \vRc & \text{ (transmit operator)} \\
   \imark & \opC   &:\vRc &\to \vRc & \text{ (channel  operator) } \\
   \imark & \opR   &:\vRc &\to U    & \text{ (receive  operator) }
\end{array}
}\\
be operators.
A {\bf digital communication system} is the operation $\opSys$
on the set of information sequences $U$ such that
$\opSys \eqd \opR\opC\opT$.
\end{definition}

Communication systems can be continuous or discrete valued in
time and/or amplitude:

   \begin{tabular}{|c||c|c|}
      \hline
         $\fs(t)=a(t)\psi(t)$   &  continuous time $t$      & discrete time $t$   \\
      \hline
      \hline
         continuous amplitude $a(t)$ & analog communications   & discrete-time communications  \\
      \hline
         discrete amplitude $a(t)$   &---                    & digital communications \\
      \hline
   \end{tabular}

{\bf In this document, we normally take the approach that}
\begin{enume}
   \item $\opC$ is stochastic
         %\footnote{
         %An exception to this is
         %in \prefpp{chp:isi} (ISI) \prefpp{chp:isi},
         %where $\opC$ is deterministic
         %($\opC$ is simply a bandlimiting operation).
         %However in any real-world system, $\opC$ is always stochastic.
         %}
   \item There is no structural constraint on $\opR$.
   \item $\opR$ is optimum with respect to the ML-criterion.
\end{enume}
These characteristics are explained more fully below.

%======================================
\subsection{Channel operator}
%======================================
Real-world physical channels perform a number of operations on
a signal.
Often these operations are closely modeled by a channel
operator $\opC$.
Properties that characterize a particular
channel operator associated with some physical channel include
\begin{liste}
   \item linear or non-linear
   \item time-invariant or time-variant
   \item memoryless or non-memoryless
   \item deterministic or stochastic.
\end{liste}
Examples of physical channels include
free space, air, water, soil, copper wire, and fiber optic cable.
Information is carried through a channel using some physical process.
These processes include:
\\\indentx\begin{tabular}{cl|ll}
              & Process                            & Example
   \\\hline
     $\imark$ & electromagnetic waves              & free space, air    & \ifxref{em}{app:em}
   \\$\imark$ & acoustic waves                     & water, soil
   \\$\imark$ & electric field potential (voltage) & wire
   \\$\imark$ & light                              & fiber optic cable
   \\$\imark$ & quantum mechanics                  &
\end{tabular}

%======================================
\subsection{Receive operator}
%======================================
Let $\opI$ be the \ope{identity operator} \xref{def:opI}.
Ideally, $\opR$ is selected such that
   $\opR\opC\opT = \opI$.
In this case we say that $\opR$ is the \ope{left inverse}\footnote{
  \begin{tabular}[t]{llll}
    $\opXi{X}$ is the & \hie{left  inverse}& of $\opX$ if & $\opXi{X}\opX=\opI$. \\
    $\opXi{X}$ is the & \hie{right inverse}& of $\opX$ if & $\opX\opXi{X}=\opI$. \\
    $\opXi{X}$ is the & \hie{      inverse}& of $\opX$ if & $\opXi{X}\opX=\opX\opXi{X}=\opI$.
  \end{tabular}
   }
of $\opC\opT$ and denote this left inverse by $\opCTi$.
One example of a system where this inverse exists is the
noiseless ISI system.  %\xref{chp:isi}.
While this is quite useful for mathematical analysis and system design,
$\opCTi$ does not actually exist for any real-world system.

When $\opCTi$ does not exist, the ``ideal" $\opR$ is one that is
optimum
  \begin{enume}
    \item with respect to some \hie{criterion} (or cost function)
    \item and sometimes under some structural \hie{constraint}.
  \end{enume}
When a structural constraint is imposed on $\opR$,
the solution is called \prope{structured}; otherwise,
it is called \prope{non-structured}.\footnote{\citerpg{vantrees1}{12}{0471095176}}
A common example of a structured approach is the use of a
transversal filter (FIR filter in DSP) in which optimal coefficients
are found for the filter.
A structured $\opR$ is only optimal with respect to the
imposed constraint.
Even though $\opR$ may be optimal with respect to this structure,
$\opR$ may not be optimal in general;
that is, there may be another structure that would lead to a ``better"
solution.
In a non-structured approach, $\opR$ is free to take any form
whatsoever (practical or impractical) and therefore leads to the
best of the best solutions.

The nature of $\opR$ depends heavily on the nature of $\opC$.
If $\opCTi$ does not exist,
then the ideal $\opR$ is one that is optimal with respect to some criterion \xref{chp:est}
If $\opC$ is deterministic,
then appropriate optimization criterion may include
\begin{liste}
   \item least square error (LSE) criterion
   \item minimum absolute error criterion
   \item minimum peak distortion criterion.
\end{liste}
If $\opC$ is stochastic
then appropriate optimization criterion may include
\\\begin{tabular}{clll}
     $\imark$ & Bayes:                                  & pdf known and cost function defined
   \\$\imark$ & Maximum aposteriori probability (MAP):  & pdf known and uniform cost function
   \\$\imark$ & Maximum likelihood (ML):                & pdf known and no prior probability information
   \\$\imark$ & mini-max:                               & pdf not known but a cost function is defined
   \\$\imark$ & Neyman-Pearson:                         & pdf not known and no cost function defined.
\end{tabular}

Making $\opR$ optimum with respect to one of these criterion leads to
an \hie{estimate} $\sue = \opR\opC\opT\vu$ that is also optimum
with respect to the same criterion \xref{def:ML}.

%======================================
\section{Optimization in the case of additional operations}
%======================================
\begin{figure}[ht] \color{figcolor}
\centering%
\setlength{\unitlength}{0.15mm}
\begin{picture}(900,150)(-100,-50)
  \thicklines
  %\graphpaper[10](0,0)(500,100)
  \put(-100 ,  60 ){\makebox( 100,0)[b]{$\su$} }
  \put(-100 ,  50 ){\vector(1,0){100} }

  \put(  00 ,  10 ){\makebox( 100, 80)[t]{transmit} }
  \put(  00 ,  10 ){\makebox( 100, 80)[c]{operation} }
  \put(  00 ,  10 ){\makebox( 100, 80)[b]{$\opT$} }
  \put(  00 ,  00 ){\framebox( 100,100){} }

  \put( 100 ,  60 ){\makebox( 100,0)[b]{$\fs(t;\vu)$} }
  \put( 100 ,  50 ){\vector(1,0){100} }

  \put( 200 ,  10 ){\makebox( 100, 80)[t]{channel} }
  \put( 200 ,  10 ){\makebox( 100, 80)[c]{operation} }
  \put( 200 ,  10 ){\makebox( 100, 80)[b]{$\opC$} }
  \put( 200 ,  00 ){\framebox(100,100){} }

  \put( 300 ,  60 ){\makebox( 100,0)[b]{$\fr(t;\vu)$} }
  \put( 300 ,  50 ){\vector(1,0){100} }

  \put( 400 ,  00 ){\framebox(100,100){} }
  \put( 400 ,  10 ){\makebox( 100, 80)[t]{} }
  \put( 400 ,  10 ){\makebox( 100, 80)[c]{operation} }
  \put( 400 ,  10 ){\makebox( 100, 80)[b]{$\opX$} }

 %\put( 500 ,  60 ){\makebox( 100,0)[b]{$\sue$} }
  \put( 500 ,  50 ){\vector(1,0){100} }

  \put( 600 ,  00 ){\framebox(100,100){} }
  \put( 600 ,  10 ){\makebox( 100, 80)[t]{receive} }
  \put( 600 ,  10 ){\makebox( 100, 80)[c]{operation} }
  \put( 600 ,  10 ){\makebox( 100, 80)[b]{$\opR$} }

  \put( 700 ,  60 ){\makebox( 100,0)[b]{$\sue$} }
  \put( 700 ,  50 ){\vector(1,0){100} }

  \put(- 90 , -10 ){\makebox( 0, 0)[tl]{$\vu\eqd\su$} }
  \put( 110 , -10 ){\makebox( 0, 0)[tl]{$\fs(t;\vu)=\opT\vu$} }
  \put( 310 , -10 ){\makebox( 0, 0)[tl]{$\fr(t;\vu)=\opC\opT\vu$} }
  \put( 510 , -10 ){\makebox( 0, 0)[tl]{$\opX\opC\opT\vu$} }
  \put( 710 , -10 ){\makebox( 0, 0)[tl]{$\sue=\opR\opX\opC\opT\vu$} }

\end{picture}
\caption{
   Theorem of reversibility
   \label{fig:thm_rev}
   }
\end{figure}

Often in communication systems, an additional  operator $\opX$ is
inserted such that (see \prefpp{fig:thm_rev})
   \[ \opSys = \opR\opX\opC\opT.\]
An example of such an operator $\opX$ is a receive filter.
Is it still possible to find an $\opR$ that will perform as well as
the case where $\opX$ is not inserted?
In general, the answer is ``no".
For example, if $\opX r=0$, then all received information is lost
and obviously there is no $\opR$ that can recover from this event.
However, in the case where the right inverse $\opXi{X}$ of $\opX$ exists,
then the answer to the question is ``yes" and an optimum  $\opR$
still exists.
That is, it doesn't matter if an $\opX$ is inserted into system
as long as $\opX$ is invertible.
This is stated formally in the next theorem.
%--------------------------------------
\begin{theorem}[\thmd{Theorem of Reversibility}]
\footnote{
  \citerppg{vantrees1}{289}{290}{0471095176}
  }
\label{thm:reversibility}
%--------------------------------------
Let
\begin{liste}
   \item $\est = \opR\opC\opT\vu$ be the optimum estimate of $\vu$
   \item $\opX$ be an operator with right inverse $\opXi{X}$.
\end{liste}
Then there exists some $\opR'$ such that
   \thmbox{ \est = \opR'\opX\opC\opT\vu. }
\end{theorem}
\begin{proof}
  Let $\opR'=\opR\opXi{X}$.
  Then
  \[ \opR'\opX\opC\opT\vu = \opR\opXi{X}\opC\opT\vu = \opR\opC\opT\vu = \est. \]
\end{proof}

%======================================
\section{Alternative system partitioning}
%======================================
\begin{figure}[ht] \color{figcolor}
\begin{center}
\begin{fsK}
\setlength{\unitlength}{0.18mm}
\begin{picture}(600,400)(-300,-100)
  \thinlines
  %\graphpaper[10](0,0)(700,100)

  \put(-300, 210 ){\makebox (  50, 50)[b]{$\su$}        }
  \put(-300, 200 ){\vector  (   1,  0)   {100}          }
  \put(-200, 150 ){\framebox( 100,100)   {coding}       }
  \put(-150, 150 ){\vector  (   0, -1)   {100}          }
  \put(-200,- 50 ){\framebox( 100,100)   {modulator}    }
  \put(-100,   0 ){\vector  (   1,  0)   { 50}          }
  \put(- 50,- 50 ){\framebox( 100,100)   {channel}      }
  \put(  50,   0 ){\vector  (   1,  0)   { 50}          }
  \put( 100,- 50 ){\framebox( 100,100)   {demodulator}  }
  \put( 150,  50 ){\vector  (   0,  1)   {100}          }
  \put( 100, 150 ){\framebox( 100,100)   {decoding}     }
  \put( 250, 210 ){\makebox (  50, 50)[b]{$\sue$}        }
  \put( 200, 200 ){\vector  (   1,  0)   {100}          }

  \put(-250, 110 ){\dashbox ( 500,180)[tr]{outer transceiver}}
  \put(-250,- 90 ){\dashbox ( 500,180)[br]{inner transceiver}}
\end{picture}
\end{fsK}
\end{center}
\caption{
   Inner/outer transceiver
   \label{fig:inner_outer}
   }
\end{figure}

A communication system can be partitioned into two parts
(see \prefpp{fig:inner_outer}):\footnote{\citerp{meyr}{2}}

\begin{tabular}{lll}
   1.& outer transceiver: & data encoding/decoding \\
   2.& inner transceiver: & modulation/demodulation.
\end{tabular}

The outer transceiver can perform several types of coding
on the data sequence to be transmitted:
\\
\begin{tabular}{llp{10cm}}
   1.& source coding: &
       compress data sequence size
       (lower limit is Shannon Entropy $H$)
\\
   2.& channel coding: &
       modify data sequence such that errors induced by the channel can be
       detected and corrected
       (all errors can be theoretically corrected if the data
        rate is at or below the Shannon channel capacity $C$).
\\
   3.& modulation coding: &
       make sequence ``more suitable" for transmission through channel
\\
   4.& encryption: &
       increase the difficulty which an eavesdropper would need
       to be able to know the data sequence.
\end{tabular}

%======================================
\section{Channel Statistics}
%======================================
The receiver needs to make a decision as to what
sequence $\seqn{u}$ the transmitter has sent.
This decision should be optimal in some sense.
Very often the optimization criterion is chosen to be
the \hie{maximal likelihood (ML)} criterion.
The information that the receiver can use to make an optimal
decision is the received signal $\fr(t)$.

If the symbols in $\fr(t)$ are statistically \hie{independent},
then the optimal estimate of the current symbol depends only on the
current symbol period of $\fr(t)$.
Using other symbol periods of $\fr(t)$ has absolutely no
additional benefit.
Note that the AWGN channel is \hie{memoryless};
that is, the way the channel treats the current symbol has
nothing to do with the way it has treated any other symbol.
Therefore, if the symbols sent by the transmitter into the channel
are independent, the symbols coming out of the channel are also
independent.

However, also note that the symbols sent by the transmitter
are often very intentionally not independent;
but rather a strong relationship between symbols is intentionally
introduced. This relationship is called \hie{channel coding}.
With proper channel coding, it is theoretically possible
to reduce the probability of communication error to any
arbitrarily small value as long as the channel is operating below its
\hie{channel capacity}.

This chapter assumes that the received symbols are
statistically independent;
and therefore optimal decisions at the receiver
for the current symbol are made
only from the current symbol period of $\fr(t)$.

The received signal $\fr(t)$ over a single symbol period
contains an uncountably infinite number of points.
That is a lot.
It would be nice if the receiver did not have to look
at all those uncountably infinite number of points
when making an optimal decision.
And in fact the receiver does indeed not have to.
As it turns out, a single finite set of \hie{statistics}
$\setn{\fdotr_1,\;\fdotr_2,\ldots,\fdotr_N}$
is sufficient \xref{thm:sstat} for the receiver to make an optimal decision as to
which value the transmitter sent.

%%---------------------------------------
%\begin{definition}
%\label{def:chan_stats}
%\index{channel statistics}
%%---------------------------------------
%Let $\opC$ be an additive noise channel
%%such that $\fr(t)=[\opC\fs](t)=\fs(t)+\fn(t)$
%%and $\set{\fpsi_n}{n=1,2,\ldots,N}$ be a basis for $\fs(t)$.
%%\defbox{\begin{array}{lll}
%%  \fdotr_n    &\eqd& \inprod{\fr(t)}  {\psi_n(t)}  \\
%  %\fdots_n(u) &\eqd& \inprod{\fs(t;u)}{\psi_n(t)} \\
%%  \fdotn_n    &\eqd& \inprod{\fn(t)}  {\psi_n(t)}
%%\end{array}}
%\end{definition}




