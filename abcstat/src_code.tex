%============================================================================
% LaTeX File
% Daniel J. Greenhoe
%============================================================================

%======================================
\chapter{Source Code}
\label{app:src_code}
\index{source code}
\label{app:sourcecode}
%======================================
%The source code in this appendix for \hie{GNU Octave}.
%\footnote{ \hie{GNU Octave}: \url{http://www.octave.org/}}
%Octave is similar to \hie{MatLab} with some differences:
%\begin{enume}
%  \item GNU Octave is free.
%  \item GNU Octave is open-source.
%  \item GNU Octave uses a separate graphics engine called \hie{GNU-Plot}
%        for all graphing.
%\end{enume}
%Octave code can easily be adapted to MatLab code and vice-versa.
%
%
%
%%--------------------------------------
%%\section{GNU Octave script file}
%\label{sec:src_Ry}
%%--------------------------------------
%\lstinputlisting[language=Octave]{../common/wavelets/m/Ry.m}
%
%
%%--------------------------------------
%\section{GNU Plot script file}
%%\label{sec:src_pollen}
%%--------------------------------------
%\lstinputlisting{../common/wavelets/m/pollen4.plt}

The free and open source software package Maxima 
has been used to compute some of the algebraic expressions for \fncte{B-spline}s 
used in \prefpp{app:bspline}:
\\
\lstinputlisting[language=MuPAD]{../common/math/maxima/bsplines.max}

\begin{minipage}{\tw-65mm}
Once the polynomial expressions for a \fncte{B-spline} have been calculated, 
they can be plotted within a {\LaTeX} environment using the 
\href{http://www.ctan.org/pkg/pst-plot}{pst-plot package}
along with a {\LaTeX} source file such as the following:\footnotemark
\end{minipage}\hfill\tbox{\includegraphics{../common/math/graphics/pdfs/n3.pdf}}%
\footnotetext{
  For help with PostScript\puttrademark math operators, see 
  \citerppc{adobe1999}{508}{509}{Arithmetic and Math Operators}.
  }
\\
\lstinputlisting[language=TeX]{../common/math/graphics/bsplines/n3.tex}

\begin{minipage}{\tw-65mm}
Alternatively, one can plot $\fN_3(x)$ more or less directly from 
the equation given in \prefpp{thm:Nnx}
without first calculating the polynomial expressions:
\end{minipage}\hfill\tbox{\includegraphics{../common/math/graphics/pdfs/n3x.pdf}}%
\\
\lstinputlisting[language=TeX]{../common/math/graphics/bsplines/n3x.tex}

%\mbox{}\hfill\includegraphics{graphics/n5.pdf}\hfill\mbox{}\\
%The B-spline $\fN_5(x)$ illustrated above and in \prefpp{fig:N012}
%was generated using {\XeLaTeX} with the \hie{pstricks} and \hie{pst-plot} packages
%and the following script file:\footnote{
%  \begin{tabular}[t]{ll}
%    \hie{pstricks} documentation:& \url{http://www.ctan.org/pkg/pstricks-base}\\
%    \hie{pst-plot} documentation:& \url{http://www.ctan.org/pkg/pst-plot}\\
%    \hie{multido} documentation: & \url{http://www.ctan.org/pkg/multido}
%  \end{tabular}
%  }
%\\
%\lstinputlisting[language=TeX]{../common/math/graphics/bsplines/n5.tex}
