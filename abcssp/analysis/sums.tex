%============================================================================
% LaTeX File
% Daniel J. Greenhoe
%============================================================================





%======================================
\section{Sums}
%======================================

%=======================================
%\section{Combinitorial relations}
%=======================================
%``Some modern appraisals of the cavalier style of 18th-century mathematicians in handling 
%infinite series conveys the impression that these poor men set their brains aside when 
%confronted by them."
%--- \citerpg{grattan1990}{163}{3764322373}


\qboxnps
  { 
    \href{http://en.wikipedia.org/wiki/G.H._Hardy}{G.H. Hardy} 
    \href{http://www-history.mcs.st-andrews.ac.uk/Timelines/TimelineG.html}{(1877--1947)}
    in his ``Presidential Address" to the 
    \href{http://en.wikipedia.org/wiki/London_mathematical_society}{London Mathematical Society}
    on November 8, 1928, about a remark that he thought to be from 
    \href{http://en.wikipedia.org/wiki/Harald_Bohr}{Harald Bohr}
    \href{http://www-history.mcs.st-andrews.ac.uk/Timelines/TimelineG.html}{(1887--1951)}, 
    \href{http://www-history.mcs.st-andrews.ac.uk/Countries/Denmark.html}{Danish} 
    mathematician and pictured to the left.
    \index{Bohr, Harald} \index{Hardy, G.H.}
    \index{quotes!Bohr, Harald}
    \index{quotes!Hardy, G.H.}
    \footnotemark
  }
  {../common/people/bohrh.jpg}
  {I think that it was Harald Bohr who remarked to me that 
   ``all analysts spend half their time hunting through the literature 
   for inequalities which they want to use and cannot prove." }
  \citetbltbl{
    quote: & \citerp{hardy1928}{64} \\
    image: & \url{http://www-history.mcs.st-andrews.ac.uk/PictDisplay/Bohr_Harald.html}
    }


%=======================================
\subsection{Summation}
%=======================================
%--------------------------------------
\begin{definition}
\footnote{
  \citerpgc{berberian1961}{8}{0821819127}{Definition~I.3.1},
  \citePpc{fourier1820}{280}{``$\sum$" notation}  %{http://gallica.bnf.fr/ark:/12148/bpt6k33707/f285.image}
  }
\label{def:sum}
%--------------------------------------
Let $+$ be an addition operator on a tuple $\tuple{x_n}{m}{\xN}$.
\defbox{\begin{array}{M}
  The \hid{summation} of $\tuplen{x_n}$ from index $m$ to index $N$ with respect to $+$ is
  \\\qquad$\ds
  \sum_{n=m}^\xN x_n \eqd 
    \brbl{\begin{array}{>{\ds}l@{\qquad}M}
      0                            & for $\xN<m$\\
     %x_\xN                          & for $N=m$ \\
    \brp{\sum_{n=m}^{\xN-1} x_n}+x_\xN & for $\xN\ge m$
    \end{array}}
  $
\end{array}}
\end{definition}

%--------------------------------------
\begin{theorem}[\thmd{Generalized associative property}]
\footnote{
  \citerppgc{berberian1961}{9}{10}{0821819127}{Theorem I.3.1}
  }
\label{thm:sum_assoc}
%--------------------------------------
Let $+$ be an addition operator on a tuple $\tuple{x_n}{m}{\xN}$.
\thmbox{\begin{array}{M}
  $+$ is \prope{associative}
  $\quad\implies$
  \\$\ds
  \mcom{\ds
    \sum_{n=m}^{\xL} x_n + \brp{\sum_{n=\xL+1}^{\xM} x_n + \sum_{n=\xM+1}^{\xN} x_n}
    =\brp{\sum_{n=m}^{\xL} x_n + \sum_{n=\xL+1}^{\xM} x_n} + \sum_{n=\xM+1}^{\xN} x_n 
    \text{\qquad\scs for $m<\xL<\xM\le\xN$}
   }{$\ds\sum_{n=m}^\xN$ is \prope{associative}}
  $
\end{array}}
\end{theorem}
\begin{proof}
\begin{enumerate}
  \item Proof for $\xN<m$ case: $\ds \sum_{n=m}^\xN x_n = 0$.
  \item Proof for $N=m$ case: $\ds \sum_{n=m}^m x_n = \brp{\sum_{n=m}^{m-1} x_n}+x_m = 0+x_m = x_m$.
  \item Proof for $N=m+1$ case: \label{item:sum_assoc_m1}
        $\ds \sum_{n=m}^{m+1} x_n = \brp{\sum_{n=m}^{m} x_n}+x_{m+1} = x_{m} + x_{m+1}$

  \item Proof for $N=m+2$ case: 
    \begin{align*}
      \sum_{n=m}^{m+2} x_n 
        &= \brp{\sum_{n=m}^{m+1} x_n }  +  x_{m+2}
        && \text{by \prefp{def:sum}}
      \\&= \brp{x_{m} + x_{m+1}}        +  x_{m+2}
        && \text{by \pref{item:sum_assoc_m1}}
      \\&= x_{m} + \brp{x_{m+1} + x_{m+2}}
        && \text{by left hypothesis}
    \end{align*}

  \item Proof that $N$ case $\implies$ $N+1$ case:
    \begin{align*}
      \sum_{n=m}^{\xN+1} x_n 
        &= \mcom{\brp{\sum_{n=m}^{\xN} x_n}}{\prope{associative}} + x_{\xN+1}
        && \text{by \prefp{def:sum}}
      \\&= \brp{\sum_{n=m}^{\xL} x_n + \brp{\sum_{n=\xL+1}^{\xM} x_n + \sum_{n=\xM+1}^{\xN} x_n}} + x_{\xN+1}
      \\&= \brp{\brp{\sum_{n=m}^{\xL} x_n + \sum_{n=\xL+1}^{\xM} x_n} + \sum_{n=\xM+1}^{\xN} x_n} + x_{\xN+1}
      \\&= \brp{\sum_{n=m}^{\xL} x_n + \sum_{n=\xL+1}^{\xM} x_n} + \brp{\sum_{n=\xM+1}^{\xN} x_n + x_{\xN+1}}
      \\&= \brp{\sum_{n=m}^{\xL} x_n + \sum_{n=\xL+1}^{\xM} x_n} + \brp{\sum_{n=\xM+1}^{\xN+1} x_n}
    \end{align*}
\end{enumerate}
\end{proof}

%=======================================
\subsection{Convexity}
%=======================================
%--------------------------------------
\begin{definition}
\footnote{
  \citerpg{simon2011}{2}{1107007313},
  \citerpg{barvinok2002}{2}{0821872311},
  \citerp{bollobas1999}{3},
  \citerp{jensen1906}{176}
  }
\label{def:convexf}
\index{convex!functional}
\index{convex!strictly}
\index{strictly convex}
%--------------------------------------
%Let $\linearspaceX$ be a \structe{linear space}\ifsxref{vector}{def:vspace} 
%and $\setD$ a \structe{convex set} \xref{def:convex_set} in $\setX$.
\mbox{}\\\defboxt{$\begin{array}{rcl@{\qquad}l}
  \mc{4}{M}{A function $\ff\in\clFrr$ is \propd{convex} if}\\
  \indentx\ff\big(\lambda x + [1-\lambda]y\big) 
    &\le& \lambda \ff(x) + (1-\lambda)\:\ff(y)  
    &     \forall  x, y\in\R \text{ and } \forall\lambda\in\intoo{0}{1}
  \\ 
  \mc{4}{M}{A function $\fg\in\clFrr$ is \propd{strictly convex} if }\\
  \indentx\fg\big(\lambda  x + [1-\lambda] y\big) 
    &=&   \lambda \fg(x) + (1-\lambda)\:\fg(y)  
    &     \forall  x, y\in\setD,\; x\ne  y,\; \text{and } \forall\lambda\in\intoo{0}{1}
  \\ 
  \mc{4}{M}{A function $\ff\in\clFrr$ is \propd{concave} if $-\ff$ is \prope{convex}.}
  \\ 
  \mc{4}{M}{A function $\ff\in\clFrr$ is \propd{affine} if $\ff$ is \prope{convex} and \prope{concave}.}\\
  \end{array}$}
\end{definition}

%--------------------------------------
\begin{theorem}[\thmd{Jensen's Inequality}]
\footnote{
  \citerpg{mitrinovic2010}{6}{9048142253},
  \citerpg{bollobas1999}{3}{0521655773},
  %\citerpg{lay1982}{7}{0471095842}\\
  \citePpp{jensen1906}{179}{180}
  }
\label{thm:seq_jensen_ineq}
\index{inequalities!Jensen's}
%--------------------------------------
Let $\ff\in\clFrr$ be a function.
\thmbox{
  \brb{\begin{array}{FMD}
    1. & $\ff$ is \prope{convex}         & and \\%\xref{def:convexf} & and \\
    2. & $\ds\sum_{n=1}^\xN \lambda_n=1$ &     %(\structe{weights})
  \end{array}}
  \implies
  \brb{
    \ff\brp{\sum_{n=1}^\xN \lambda_n \: x_n} 
    \le \sum_{n=1}^\xN \lambda_n\:\ff\brp{x_n}
    \qquad
    \forall x_n \in \setD,\;\xN\in\Zp
    }
  }
\end{theorem}
%\begin{proof}
%Proof is by induction:
%\begin{align*}
%  \intertext{1. Proof that statement is true for $\xN=1$:} 
%    \ff\left(\sum_{n=1}^{\xN=1} \lambda_n \: x_n\right) 
%      &=   \ff(1\cdot  x_1)
%      &&   \text{by $\xN=1$ hypothesis}
%    \\&=   \sum_{n=1}^{\xN=1} \ff(x_n)
%      &&   \text{by $\xN=1$ hypothesis}
%    \\&=   \sum_{n=1}^{\xN=1} \lambda_n\:\ff(x_n)
%    \\
%  \intertext{2. Proof that statement is true for $\xN=2$:} 
%    \ff\left(\sum_{n=1}^{\xN=2} \lambda_n \:  x_n\right) 
%      &=   \ff(\lambda_1  x_1 + \lambda_2 x_2) 
%    \\&\le \lambda_1 \ff( x_1) + \lambda_2 \ff( x_2)
%      &&   \text{by convexity hypothesis}
%    \\&=   \sum_{n=1}^{\xN=2} \lambda_n\:\ff( x_n)
%    \\
%  \intertext{3. Proof that if the statement is true for $\xN$, then it is also true for $\xN+1$:} 
%    \ff\left(\sum_{n=1}^{\xN+1} \lambda_n \:  x_n\right) 
%      &=   \ff\left(\sum_{n=1}^{\xN} \lambda_n \:  x_n + \lambda_{\xN+1} x_{\xN+1}\right) 
%    \\&=   \ff\left([1-\lambda_{\xN+1}]\sum_{n=1}^{\xN} \frac{\lambda_n}{1-\lambda_{\xN+1}} \:  x_n + \lambda_{\xN+1} x_{\xN+1}\right) 
%    \\&\le [1-\lambda_{\xN+1}]\ff\left(\sum_{n=1}^{\xN} \frac{\lambda_n}{1-\lambda_{\xN+1}} \:  x_n \right) 
%         + \lambda_{\xN+1}\ff( x_{\xN+1}) 
%      &&   \text{by convexity hypothesis}
%    \\&\le [1-\lambda_{\xN+1}]\sum_{n=1}^{\xN} \frac{\lambda_n}{1-\lambda_{\xN+1}}\ff( x_n) 
%         + \lambda_{\xN+1}\ff( x_{\xN+1}) 
%      &&   \text{by ``true for $\xN$" hypothesis}
%    \\&=   \sum_{n=1}^{\xN} \lambda_n \ff( x_n) + \lambda_{\xN+1}\ff( x_{\xN+1}) 
%    \\&=   \sum_{n=1}^{\xN+1} \lambda_n \ff( x_n)
%    \\
%  \intertext{4. Since the statement is true for $\xN=1$, $\xN=2$, and 
%                true for $\xN$ $\implies$ true for $\xN+1$,
%                then it is true for $\xN=1,2,3,4,\ldots$}
%\end{align*}
%\end{proof}


%%=======================================
\subsection{Power means}
\label{app:pmean}
%%=======================================
%
%
%%=======================================
%\subsection{Weighted {\fntFreeSerif\textphi}-means}
%%=======================================

%---------------------------------------
\begin{definition}
\footnote{
  \citerpg{bollobas1999}{5}{0521655773}
  }
\label{def:seq_Mphi}
%---------------------------------------
\defbox{\begin{array}{M}
  The $\tuple{\lambda_n}{1}{\xN}$ weighted \opd{$\fphi$-mean} of a tuple $\tuple{x_n}{1}{\xN}$ is defined as
  \\\indentx$\ds \fM_\fphi\brp{\tuplen{x_n}} \eqd \fphi^{-1}\brp{\sum_{n=1}^\xN \lambda_n \fphi\brp{x_n}}$
  \\where $\fphi$ is a \prope{continuous} and \prope{strictly monotonic} function in $\clF{\Rnn}{\R}$
  \\and   $\tuple{\lambda_n}{n=1}{\xN}$ is a sequence of weights for which $\ds\sum_{n=1}^\xN\lambda_n=1$.
  \end{array}}
\end{definition}

%---------------------------------------
\begin{lemma}
\footnote{
  \citerpg{pecaric1992}{107}{0125492502},
  \citerpg{bollobas1999}{5}{0521655773},
  \citorpg{hardy1934}{75}{0521358809}
  }
\label{lem:seq_Mphi}
%---------------------------------------
Let $\fM_\fphi\brp{\tuplen{x_n}}$ be the $\tuple{\lambda_n}{1}{\xN}$ weighted $\fphi$-mean and
    $\fM_\fpsi\brp{\tuplen{x_n}}$    the $\tuple{\lambda_n}{1}{\xN}$ weighted $\fpsi$-mean of a tuple $\tuple{x_n}{1}{\xN}$.
%Let the property \prope{convex} be defined as in \prefpp{def:convexf}.
\lembox{
  \begin{array}{MMMc>{\ds}l}
    $\fphi\fpsi^{-1}$  is \prope{convex}  &and& $\fphi$ is \prope{increasing} &\implies& \fM_\fphi\brp{\tuplen{x_n}} \ge \fM_\fpsi\brp{\tuplen{x_n}}\\
    $\fphi\fpsi^{-1}$  is \prope{convex}  &and& $\fphi$ is \prope{decreasing} &\implies& \fM_\fphi\brp{\tuplen{x_n}} \le \fM_\fpsi\brp{\tuplen{x_n}}\\
    $\fphi\fpsi^{-1}$  is \prope{concave} &and& $\fphi$ is \prope{increasing} &\implies& \fM_\fphi\brp{\tuplen{x_n}} \le \fM_\fpsi\brp{\tuplen{x_n}}\\
    $\fphi\fpsi^{-1}$  is \prope{concave} &and& $\fphi$ is \prope{decreasing} &\implies& \fM_\fphi\brp{\tuplen{x_n}} \ge \fM_\fpsi\brp{\tuplen{x_n}}\\
  \end{array}
  }
\end{lemma}
%\begin{proof}
%\begin{align*}
%  \intertext{1. Case where $\fphi\fpsi^{-1}$  is \prope{convex} and $\fphi$ is \prope{increasing}:} 
%  \fM_\fphi\brp{\tuplen{x_n}} 
%    &\eqd \fphi^{-1}\brp{\sum_{n=1}^{\xN} \lambda_n \fphi\brp{x_n}}
%    &&    \text{by definition of $\fM_\fphi$ (\prefp{def:seq_Mphi})}
%  \\&=    \fphi^{-1}\brp{\sum_{n=1}^{\xN} \lambda_n \fphi\fpsi^{-1}\fpsi\brp{x_n}}
%    &&    \text{by definition of $\fpsi^{-1}$}
%  \\&\ge  \fphi^{-1}\brp{\fphi\fpsi^{-1}\sum_{n=1}^{\xN} \lambda_n \fpsi\brp{x_n}}
%    &&    \text{by \thme{Jensen's Inequality} \xref{thm:seq_jensen_ineq}}
%  \\&=    \brp{\fpsi^{-1}\sum_{n=1}^{\xN} \lambda_n \fpsi\brp{x_n}}
%    &&    \text{by definition of $\fpsi^{-1}$}
%  \\&\eqd \fM_\fpsi\brp{\tuplen{x_n}}
%    &&    \text{by definition of $\fM_\fpsi$ (\prefp{def:seq_Mphi})}
%  %
%  \intertext{2. Case where $\fphi\fpsi^{-1}$  is \prope{convex} and $\fphi$ is \prope{decreasing}:} 
%  \fM_\fphi\brp{\tuplen{x_n}} 
%    &\eqd \fphi^{-1}\brp{\sum_{n=1}^{\xN} \lambda_n \fphi\brp{x_n}}
%    &&    \text{by definition of $\fM_\fphi$ (\prefp{def:seq_Mphi})}
%  \\&=    \fphi^{-1}\brp{\sum_{n=1}^{\xN} \lambda_n \fphi\fpsi^{-1}\fpsi\brp{x_n}}
%    &&    \text{by definition of $\fpsi^{-1}$}
%  \\&\le  \fphi^{-1}\brp{\fphi\fpsi^{-1}\sum_{n=1}^{\xN} \lambda_n \fpsi\brp{x_n}}
%    &&    \text{by \thme{Jensen's Inequality} and because $\fphi^{-1}$ is \prope{decreasing}}
%  \\&=    \brp{\fpsi^{-1}\sum_{n=1}^{\xN} \lambda_n \fpsi\brp{x_n}}
%    &&    \text{by definition of $\fpsi^{-1}$}
%  \\&\eqd \fM_\fpsi\brp{\tuplen{x_n}}
%    &&    \text{by definition of $\fM_\fpsi$ (\prefp{def:seq_Mphi})}
%\end{align*}
%\end{proof}

One of the most well known inequalities in mathematics is \thme{Minkowski's Inequality}.
%(1910, \prefp{thm:lp_minkowski}). % which demonstrates
%\\\indentx$\ds
%  \brp{\sum_{n=1}^{\xN} \abs{x_n+y_n}^p}^\frac{1}{p}
%  \le
%  \brp{\sum_{n=1}^{\xN} \abs{x_n}^p}^\frac{1}{p}
%  +
%  \brp{\sum_{n=1}^{\xN} \abs{y_n}^p}^\frac{1}{p}
%  \qquad
%  \forall \; 1<p<\infty
%$\\
In 1946, H.P. Mulholland submitted a result that generalizes Minkowski's Inequality to
an equal weighted \textphi-mean.\footnote{
  \citePp{minkowski1910}{115},
  \citeP{mulholland1950},
  \citergc{hardy1934}{0521358809}{Theorem 24},
  \citePp{tolsted1964}{7},
  \citePp{maligranda1995}{258},
  \citerp{carothers2000}{44},
  \citerpg{bullen2003}{179}{1402015224}
  }
%\\\indentx$\ds
%  \fphi^{-1}\brp{\sum_{n=1}^{\xN} \fphi\brp{x_n+y_n}}
%  \le
%  \fphi^{-1}\brp{\sum_{n=1}^{\xN} \fphi\brp{x_n}}
%  +
%  \fphi^{-1}\brp{\sum_{n=1}^{\xN} \fphi\brp{y_n}}
%$\\
And Milovanovi/'c and Milovanov/'c (1979) generalized this even further to a \hie{weighted} \textphi-mean
(next).
%(\pref{thm:lp_wphi}, next).
%---------------------------------------
\begin{theorem}
\label{thm:lp_wphi}
\footnote{
  \citeP{milovanovic1979},
  \citerpgc{bullen2003}{306}{1402015224}{Theorem 9}
  }
%---------------------------------------
Let $\fphi$ be a function in $\clFrr$.
\\\thmboxt{$\ds
  \brb{\begin{array}{*{2}{FMD}}
    1. & $\fphi$ is \prope{convex}                   & and &%\\ %&                     & and \\
    2. & $\fphi$ is \prope{strictly monotone}        & and \\%\\ %&\xref{def:smono}     & and \\
    3. & $\fphi(0)=0$                                & and &%\\ %&                     & and \\
    4. & $\log\circ\fphi\circ\exp$ is \prope{convex} &        %&\xref{def:composit}
  \end{array}}$
  \\\indentx$\ds\implies
  \brb{
  \fphi^{-1}\brp{\sum_{n=1}^{\xN} \lambda_n \fphi\brp{x_n+y_n}}
  \le
  \fphi^{-1}\brp{\sum_{n=1}^{\xN} \lambda_n \fphi\brp{x_n}}
  +
  \fphi^{-1}\brp{\sum_{n=1}^{\xN} \lambda_n \fphi\brp{y_n}}
  }
  $}
\end{theorem}

%=======================================
%\subsection{Power means}
%=======================================
%---------------------------------------
\begin{definition}
\footnote{
  \citerpg{bullen2003}{175}{1402015224},
  \citerpg{bollobas1999}{6}{0521655773}
  %\citerpg{pecaric1992}{108}{0125492502}
  }
\label{def:seq_Mr}
\label{def:pmean}
%---------------------------------------
Let $\fM_{\fphi(x;p)}\brp{\tuplen{x_n}}$ be the $\tuple{\lambda_n}{1}{\xN}$ weighted $\fphi$-mean of 
a \prope{non-negative} tuple $\tuple{x_n}{1}{\xN}$.
%(\prefp{def:seq_Mphi}).
%Let $\Rx$ be the set of extended real numbers $\brp{\Rx\eqd\R\setu\setn{-\infty,\,\infty}}$.%
%\footnote{
%  \citerppgc{rana2002}{385}{388}{0821829742}{Appendix A}
%  }
\defboxt{
  A mean $\fM_{\fphi(x;p)}\brp{\tuplen{x_n}}$ is a \fnctd{power mean} with parameter $p$ if $\fphi(x)\eqd x^p$. That is,
  \\\indentx$\ds \fM_{\fphi(x;p)}\brp{\tuplen{x_n}} = \brp{\sum_{n=1}^{\xN} \lambda_n \brp{x_n}^p}^{\frac{1}{p}}$
  }
\end{definition}

%---------------------------------------
\begin{theorem}
\footnote{
  \citerppgc{bullen2003}{175}{177}{1402015224}{see also page 203},
  \citerppg{bollobas1999}{6}{8}{0521655773},
  %\citerpg{pecaric1992}{108}{0125492502},
  \citePp{bullen1990}{250},
  \citeP{besso1879},
  \citePp{bienayme1840}{68},
  \citePp{brenner1985}{160}
  %\citerg{greenhoe2014asp}{0983801126}
  }
\label{thm:pmean_continuous}
\label{thm:seq_Mr}
%---------------------------------------
Let $\fM_{\fphi(x;p)}\brp{\tuplen{x_n}}$ be the \fncte{power mean} with parameter $p$ of an $\xN$-tuple $\tuplexn{x_n}$
in which the elements are \emph{not} all equal.
\thmbox{\begin{array}{>{\ds}rc>{\ds}l}
  \ds\fM_{\fphi(x;p)}\brp{\tuplen{x_n}} &\eqd& \brp{\sum_{n=1}^{\xN} \lambda_n \brp{x_n}^p}^{\frac{1}{p}}
  \text{ is \prope{continuous} and \prope{strictly monotone} in $\Rx$.}
  \\
    \ds \fM_{\fphi(x;p)}\brp{\tuplen{x_n}} &=& 
    \brbl{\begin{array}{>{\ds}lM}
      \max_{n=1,2,\ldots,\xN}\tuplen{x_n}                     & for $p=+\infty$\\
      \prod_{n=1}^{\xN} x_n^{\lambda_n}  & for $p=0$\\
      \min_{n=1,2,\ldots,\xN}  \tuplen{x_n}          & for $p=-\infty$
    \end{array}}
  \end{array}}
\end{theorem}
\begin{proof}
\begin{enumerate}
  \item Proof that $M_{\fphi(x;p)}$ is \prope{strictly monotone} in $p$:
    \begin{enumerate}
      \item Let $p$ and $s$ be such that $-\infty<p<s<\infty$.
      \item Let $\fphi_p\eqd x^p$ and $\fphi_s\eqd x^s$. Then $\ds\fphi_p\fphi_s^{-1}=x^\frac{p}{s}$.
      \item The composite function $\ds\fphi_p\fphi_s^{-1}$ is \prope{convex} or \prope{concave} depending on the values of $p$ and $s$:
        \\\begin{tabular}{c||c|c}
                & $p<0$ ($\fphi_p$ decreasing) & $p>0$ ($\fphi_p$ increasing) 
        \\\hline\hline
          $s<0$ & \prope{convex}               & (not possible)
        \\\hline
          $s>0$ & \prope{convex}               & \prope{concave}
        \\\hline
        \end{tabular}
      \item Therefore by \prefpp{lem:seq_Mphi}, 
        \\\indentx$\ds-\infty<p<s<\infty\qquad\implies\qquad\fM_{\fphi(x;p)}\brp{\tuplen{x_n}}<\fM_{\fphi(x;s)}\brp{\tuplen{x_n}}$.
    \end{enumerate}
  \item Proof that $\fM_{\fphi(x;p)}$ is continuous in $p$ for $p\in\R\setd0$:
        The sum of continuous functions is continuous.
        \\For the cases of $p\in\setn{-\infty,\,0,\,\infty}$, see the items that follow. 

  \item Lemma: $\fM_{\fphi(x;-p)}\brp{\tuplen{x_n}}=\brb{\fM_{\fphi(x;p)}\brp{\tuplen{x_n^{-1}}}}^{-1}$.\label{item:Mr_lemma} Proof:
    \begin{align*}
      \brb{\fM_{\fphi(x;p)}\brp{\tuplen{x_n^{-1}}}}^{-1}
        &= \brb{\brp{\sum_{n=1}^{\xN} \lambda_n \brp{x_n^{-1}}^{p}}^{\frac{1}{p}}}^{-1}
        && \text{by definition of $\fM_\fphi$}
      \\&= \brp{\sum_{n=1}^{\xN} \lambda_n \brp{x_n}^{-p}}^{\frac{1}{-p}}
      \\&= \fM_{\fphi(x;-p)} \brp{\tuplen{x_n}}
        && \text{by definition of $\fM_\fphi$}
    \end{align*}

  \item Proof that $\ds\lim_{p\to\infty} \fM_\fphi\brp{\tuplen{x_n}} = \max_{n\in\Z} \tuplen{x_n}$: \label{item:Mr_max}
    \begin{enumerate}
      \item Let $\ds x_m\eqd \max_{n\in\Z}\tuplen{x_n}$ \label{item:Mr_xm}

      \item Note that $\ds\lim_{p\to\infty}\fM_\fphi \le \max_{n\in\Z} \tuplen{x_n}$ because
            \begin{align*}
              \lim_{p\to\infty} \fM_\fphi\brp{\tuplen{x_n}}
                &=   \lim_{p\to\infty} \brp{\sum_{n=1}^{\xN}\lambda_n x_n^p}^\frac{1}{p}
                &&   \text{by definition of $\fM_\fphi$}
              \\&\le \lim_{p\to\infty} \brp{\sum_{n=1}^{\xN}\lambda_n x_m^p}^\frac{1}{p}
                &&   \text{\parbox[t]{80mm}{by definition of $x_m$ in \pref{item:Mr_xm} and because
                          $\fphi(x)\eqd x^p$ and $\fphi^{-1}$ are both increasing or both decreasing}}
              \\&=   \lim_{p\to\infty} \brp{x_m^p\mcom{\sum_{n=1}^{\xN}\lambda_n}{$1$} }^\frac{1}{p}
                &&   \text{because $x_m$ is a constant}
              \\&=   \lim_{p\to\infty} \brp{x_m^p\cdot1 }^\frac{1}{p}
              \\&= x_m
              \\&= \max_{n\in\Z}\tuplen{x_n}
                &&   \text{by definition of $x_m$ in \pref{item:Mr_xm}}
            \end{align*}

      \item But also note that $\ds\lim_{p\to\infty}\fM_\fphi \ge \max_{n\in\Z} \tuplen{x_n}$ because
            \begin{align*}
              \lim_{p\to\infty} \fM_\fphi\brp{\tuplen{x_n}}
                &=   \lim_{p\to\infty} \brp{\sum_{n=1}^{\xN}\lambda_n x_n^p}^\frac{1}{p}
                &&   \text{by definition of $\fM_\fphi$}
              \\&\ge \lim_{p\to\infty} \brp{w_m x_m^p}^\frac{1}{p}
                &&   \text{\parbox[t]{80mm}{by definition of $x_m$ in \pref{item:Mr_xm} and because
                          $\fphi(x)\eqd x^p$ and $\fphi^{-1}$ are both increasing or both decreasing}}
              \\&=   \lim_{p\to\infty} w_m^\frac{1}{p} x_m^\frac{p}{p}
              \\&=   x_m
              \\&=   \max_{n\in\Z}\tuplen{x_n}
                &&   \text{by definition of $x_m$ in \pref{item:Mr_xm}}
            \end{align*}

      \item Combining items (b) and (c) we have
            $\ds\lim_{p\to\infty}\fM_\fphi = \max_{n\in\Z}\tuplen{x_n}$.
    \end{enumerate}

  \item Proof that $\ds\lim_{p\to-\infty} \fM_\fphi\brp{\tuplen{x_n}} = \min_{n\in\Z} \tuplen{x_n}$:
    \begin{align*}
      \lim_{p\to-\infty}\fM_{\fphi(x;p)}\brp{\tuplen{x_n}} 
      &= \lim_{p\to\infty} \fM_{\fphi(x;-p)}\brp{\tuplen{x_n}} 
      && \text{by change of variable $p$}
    \\&= \lim_{p\to\infty} \brb{\fM_{\fphi(x;p)}\brp{\tuplen{x_n^{-1}}}}^{-1}
      && \text{by Lemma in \prefp{item:Mr_lemma}}
    \\&= \lim_{p\to\infty} \frac{1}{\fM_{\fphi(x;p)}\brp{\tuplen{x_n^{-1}}}}
    \\&= \frac{\lim_{p\to\infty}1}{\lim_{p\to\infty}\fM_{\fphi(x;p)}\brp{\tuplen{x_n^{-1}}}}
      && \text{by property of $\lim$ \footnotemark}
    \\&= \frac{1}{\ds\max_{n\in\Z}\tuplen{x_n^{-1}}}
      && \text{by \pref{item:Mr_max}}
    \\&= \frac{1}{\ds\brp{\min_{n\in\Z}\tuplen{x_n}}^{-1}}
    \\&= \min_{n\in\Z}\tuplen{x_n}
    \end{align*}
    \citetblt{
      \citerpgc{rudinp}{85}{007054235X}{4.4 Theorem}
      }

  \item Proof that $\ds\lim_{p\to0} \fM_\fphi\brp{\tuplen{x_n}} = \prod_{n=1}^{\xN} x_n^{\lambda_n}$:
    \begin{align*}
      \lim_{p\to0} \fM_\fphi\brp{\tuplen{x_n}}
        &= \lim_{p\to0} \exp\brb{\ln\brb{\fM_\fphi\brp{\tuplen{x_n}}}} 
      \\&= \lim_{p\to0} \exp\brb{\ln\brb{\brp{\sum_{n=1}^{\xN}\lambda_n\brp{x_n^p}}^\frac{1}{p}}}
        && \text{by definition of $\fM_\fphi$}
      \\&= \exp\brb{\frac{\ds\pderiv{}{p}\ln\brp{\sum_{n=1}^{\xN}\lambda_n\brp{x_n^p}}}{\ds\pderiv{}{p}p}}_{p=0}
        && \text{by \thm{l'H{/<o}pital's rule}\footnotemark}
      \\&= \exp\brb{\frac{\ds\sum_{n=1}^{\xN}\lambda_n\pderiv{}{p}\brp{x_n^p}}
                         {\ds\sum_{n=1}^{\xN}\lambda_n\brp{x_n^p}}
                   }_{p=0}
       &&= \exp\brb{\frac{\ds\sum_{n=1}^{\xN}\lambda_n\pderiv{}{p}\exp\brp{\ln\brp{x_n^p}}}
                         {\ds\sum_{n=1}^{\xN}\lambda_n}
                   }_{p=0}
      \\&= \exp\brb{\frac{\ds\sum_{n=1}^{\xN}\lambda_n\pderiv{}{p}\exp\brp{r\ln\brp{x_n}}}
                         {1}
                   }_{p=0}
       &&= \exp\brb{\sum_{n=1}^{\xN}\lambda_n\pderiv{}{p}\exp\brp{p\ln\brp{x_n}}
                   }_{p=0}
      \\&= \exp\brb{\sum_{n=1}^{\xN}\lambda_n\exp\brb{p\ln{x_n}}\ln\brp{x_n}
                   }_{p=0}
       &&= \exp\brb{\sum_{n=1}^{\xN}\lambda_n\ln\brp{x_n}}
      \\&= \exp\brb{\sum_{n=1}^{\xN}\ln\brp{x_n^{\lambda_n}}}
       &&= \exp\brb{\ln\prod_{n=1}^{\xN}x_n^{\lambda_n}}
         = \prod_{n=1}^{\xN} x_n^{\lambda_n}
    \end{align*}
    \citetblt{
      \citerpgc{rudinp}{109}{007054235X}{5.13 Theorem}
      }
\end{enumerate}
\end{proof}






%--------------------------------------
\begin{corollary}
\footnote{
  \citerpg{bullen2003}{71}{1402015224},
  \citerpg{bollobas1999}{5}{0521655773},
  \citorppc{cauchy1821}{457}{459}{Note II, theorem 17},
  \citePp{jensen1906}{183},
  \citePp{hoehn1985}{151}
  %\citerg{greenhoe2014asp}{0983801126}
  }
\label{cor:means}
\index{AM-GM inequality}
\index{arithmetic mean geometric mean inequality}
\index{generalized arithmetic mean geometric mean inequality} 
\index{inequalities!AM-GM}
%--------------------------------------
Let $\tuplen{x_n}_1^\xN$ be a tuple. Let $\tuplen{\lambda_n}_1^\xN$ be a tuple of weighting values such that 
$\ds\sum_{n=1}^\xN\lambda_n=1$.
\corbox{
  \min\tuplen{x_n}
  \le
  \mcoml{\ds\brp{\sum_{n=1}^\xN \lambda_n \frac{1}{x_n}}^{-1}}{harmonic mean}
  \le 
  \mcomr{\ds\prod_{n=1}^\xN x_n^{\lambda_n}}{geometric mean} 
  \le 
  \mcoml{\ds\sum_{n=1}^\xN \lambda_n x_n}{arithmetic mean}
  \le
  \max\tuplen{x_n}
  }
\end{corollary}
\begin{proof}
\begin{enumerate}
  \item These five means are all special cases of the \hie{power mean} $\fM_{\fphi(x:p)}$ (\prefp{def:seq_Mr}):
    \\\begin{tabular}{ll}
        $p=\infty$:  & $\max\tuplen{x_n}$
      \\$p=1$:       & arithmetic mean
      \\$p=0$:       & geometric mean
      \\$p=-1$:      & harmonic mean
      \\$p=-\infty$: & $\min\tuplen{x_n}$
    \end{tabular}

  \item The inequalities follow directly from \prefpp{thm:seq_Mr}.

  \item \thm{Generalized AM-GM inequality}: If one is only concerned with the arithmetic mean and 
        geometric mean, their relationship can be established directly using \thme{Jensen's Inequality}:
    \begin{align*}
      \sum_{n=1}^\xN \lambda_n x_n
        &=   b^{\log_b\left(\sum_{n=1}^\xN \lambda_n x_n\right)}
      \\&\ge b^{\left(\sum_{n=1}^\xN \lambda_n \log_b x_n\right)}
        &&   \text{by \thme{Jensen's Inequality} \xref{thm:seq_jensen_ineq}}
      \\&= \mathrlap{  
          \prod_{n=1}^\xN b^{\left(\lambda_n \log_b x_n\right)}
         =\prod_{n=1}^\xN b^{\left(\log_b x_n\right)\lambda_n }
         =\prod_{n=1}^\xN x_n^{\lambda_n} 
         }
    \end{align*}
\end{enumerate}
\end{proof}



%=======================================
\subsection{Inequalities}
%=======================================
%--------------------------------------
\begin{lemma}[\thmd{Young's Inequality}]
\footnote{
  \citePp{young1912}{226},
  \citergc{hardy1934}{0521358809}{Theorem 24},
  \citePp{tolsted1964}{5},
  \citePp{maligranda1995}{257},
  \citerp{carothers2000}{43}
  }
\label{lem:lp_young}
\index{inequalities!Young}
%--------------------------------------
\lembox{\renewcommand{\arraystretch}{2}
  \begin{array}{rc>{\ds}l lll}
    xy  &<& \frac{x^p}{p} + \frac{y^q}{q} & \text{with $\frac{1}{p} + \frac{1}{q} = 1$} & \forall 1<p<\infty,\; x,y\ge0, & \text{but $y \ne  x^{p-1}$} \\
    xy  &=& \frac{x^p}{p} + \frac{y^q}{q} & \text{with $\frac{1}{p} + \frac{1}{q} = 1$} & \forall 1<p<\infty,\; x,y\ge0, & \text{and $y     = x^{p-1}$} 
  \end{array}
  }
\end{lemma}
%\begin{proof}
%\begin{enumerate}
%  \item Proof that $\frac{1}{p-1}=q-1$:  \label{item:lp_young_q-1}
%    \begin{align*}
%      \frac{1}{p} + \frac{1}{q} = 1
%        &\iff \frac{q}{q} + \frac{q}{p} = q
%      \\&\iff q\brp{1-\frac{1}{p}} = 1
%      \\&\iff q = \frac{1}{1-\frac{1}{p}}
%      \\&\iff q = \frac{p}{p-1}
%      \\&\iff q-1 = \frac{p}{p-1} - \frac{p-1}{p-1}
%      \\&\iff q-1 = \frac{p-(p-1)}{p-1}
%      \\&\iff q-1 = \frac{1}{p-1}
%    \end{align*}
%
%  \item Proof that $v=u^{p-1}$ $\iff$ $u=v^{q-1}$:
%    \begin{align*}
%      u
%        &= v^{\frac{1}{p-1}}
%        && \text{by left hypothesis}
%      \\&= v^{q-1}
%        && \text{by \pref{item:lp_young_q-1}}
%    \end{align*}
%
%  \item Proof that $v=u^{p-1}$ is monotonically increasing in $u$ and 
%                   $u=v^{q-1}$ is monotonically increasing in $v$:
%    \begin{align*}
%      \deriv{v}{u} 
%        &= \deriv{}{u} u^{p-1}
%      \\&= (p-1) u^{p-2}
%      \\&> 0
%      \\
%      \deriv{u}{v} 
%        &= \deriv{}{v} v^{q-1}
%      \\&= (q-1) v^{q-2}
%      \\&> 0
%    \end{align*}
%
%  \item Proof that $xy \le \frac{x^p}{p} + \frac{y^q}{q}$:\\
%    \begin{minipage}{8\tw/16}
%      \begin{align*}
%        xy
%          &\le \int_0^x u^{p-1} \du + \int_0^y v^{q-1} \dv
%        \\&=   \left. \frac{u^p}{p} \right|_0^x + \left. \frac{v^q}{q} \right|_0^y 
%        \\&=   \frac{x^p}{p}  + \frac{y^q}{q} 
%      \end{align*}
%    \end{minipage}
%    \begin{minipage}{7\tw/16}
%      \begin{center}%
%      \begin{fsL}%
%      \setlength{\unitlength}{\tw/240}%
%      \begin{picture}(140,140)(-20,-20)%
%      %{\color{graphpaper}\graphpaper[10](0,0)(100,100)}%
%        \thinlines%
%        \color{axis}%
%          \put(0,0){\line(0,1){110}}%
%          \put(0,0){\line(1,0){110}}%
%          \put(115,0){\makebox(0,0)[l]{$u$}}%
%          \put(0,115){\makebox(0,0)[b]{$v$}}%
%        \thicklines%
%        \color{blue}%
%          \qbezier(0,0)(80,0)(100,100)%
%          \put(75,85){\makebox(0,0)[r]{$v=u^{p-1},\;u=v^{q-1}$}}%
%          \put(80,85){\vector(1,0){16}}%
%        \color{red}%
%          \put(90,-5){\makebox(0,0)[t]{$x$}}%
%          \put(-5,30){\makebox(0,0)[t]{$y$}}%
%          \put(0,30){\line(1,0){90}}%
%          \put(90,0){\line(0,1){65}}%
%        \color{purple}%
%          \put(65,65){\makebox(0,0)[br]{error}}%
%          \put(60,60){\vector(1,-1){24}}%
%      \end{picture}%
%      \end{fsL}%
%      \end{center}%
%    \end{minipage}
%\end{enumerate}
%\end{proof}


%%--------------------------------------
%\begin{theorem}[\thmd{H/:older's Inequality}]
%\footnote{
%  \citerpgc{bullen2003}{178}{1402015224}{2.1}\\
%  \citerp{carothers2000}{44} \\
%  \citePp{tolsted1964}{6}\\
%  \citePp{maligranda1995}{257} \\
%  \citergc{hardy1934}{0521358809}{Theorem 11}\\
%  \citor{holder1889}
%  }
%\label{thm:lp_holder}
%\index{inequalities!H/:older}
%%--------------------------------------
%Let $\tuplexn{x_n\in\Cnum}$ and $\tuplexn{y_n\in\Cnum}$ be complex $\xN$-tuples. 
%\thmbox{
%  \mcom{\ds\sum_{n=1}^{\xN} \abs{x_n y_n}}{$\norm{\vx\cdot\vy}_1$} \le 
%  \mcom{\brp{\ds\sum_{n=1}^{\xN} \abs{x_n}^p}^\frac{1}{p}}{$\norm{\vx}_p$}
%  \mcom{\brp{\ds\sum_{n=1}^{\xN} \abs{y_n}^q}^\frac{1}{q}}{$\norm{\vy}_q$}
%  \qquad\text{with}\qquad
%  \frac{1}{p} + \frac{1}{q} = 1
%  \qquad
%  \forall \; 1<p<\infty
%  }
%\end{theorem}
%\begin{proof}
%  Let $\ds \norm{x_n}_p \eqd \brp{\sum_{n=1}^{\xN} \abs{x_n}^p}^\frac{1}{p}$.
%\begin{align*}
%  \sum_{n=1}^{\xN} \abs{x_n y_n}
%    &= \norm{\seqn{x_n}}_p\norm{\seqn{y_n}}_q
%       \sum_{n=1}^{\xN} \frac{\abs{x_n y_n}}{\norm{\seqn{x_n}}_p\norm{\seqn{y_n}}_q}
%  \\&= \norm{\seqn{x_n}}_p\norm{\seqn{y_n}}_q
%       \sum_{n=1}^{\xN} \frac{\abs{x_n}}{\norm{\seqn{x_n}}_p} \frac{\abs{y_n}}{\norm{\seqn{y_n}}_q}
%  \\&\le \norm{\seqn{x_n}}_p\norm{\seqn{y_n}}_q
%       \sum_{n=1}^{\xN} \brp{\frac{1}{p}\frac{\abs{x_n}^p}{\norm{\seqn{x_n}}_p^p} +
%                          \frac{1}{q}\frac{\abs{y_n}^q}{\norm{\seqn{y_n}}_q^q}}
%    && \text{by \thme{Young's Inequality} \xref{lem:lp_young}}
%  \\&= \norm{\seqn{x_n}}_p\norm{\seqn{y_n}}_q
%       \brp{
%         \frac{1}{p} \cdot \frac{\ds\sum\abs{x_n}^p}{\norm{\seqn{x_n}}_p^p} 
%         +
%         \frac{1}{q} \cdot \frac{\ds\sum\abs{y_n}^q}{\norm{\seqn{y_n}}_q^q}
%         }
%  \\&= \norm{\seqn{x_n}}_p \norm{\seqn{y_n}}_q
%       \brp{
%         \frac{1}{p} \frac{\norm{\seqn{x_n}}_p^p}{\norm{\seqn{x_n}}_p^p} 
%         +
%         \frac{1}{q} \frac{\norm{\seqn{y_n}}_q^q}{\norm{\seqn{y_n}}_q^q}
%         }
%    && \text{by definition of $\normn$}
%  \\&= \norm{\seqn{x_n}}_p \norm{\seqn{y_n}}_q \mcom{\brp{\frac{1}{p} + \frac{1}{q} }}{1}
%  \\&= \norm{\seqn{x_n}}_p \norm{\seqn{y_n}}_q 
%    && \text{by $\frac{1}{p}+\frac{1}{q}=1$ constraint}
%\end{align*}
%\end{proof}


%--------------------------------------
\begin{theorem}[\thmd{Minkowski's Inequality for sequences}]
\footnote{
  \citerp{minkowski1910}{115},
  \citergc{hardy1934}{0521358809}{Theorem 24},
  \citePp{maligranda1995}{258},
  \citerp{tolsted1964}{7},
  \citerp{carothers2000}{44},
  \citerpg{bullen2003}{179}{1402015224}
  }
\label{thm:lp_minkowski}
\index{inequalities!Minkowski (sequences)}
%--------------------------------------
Let $\tuplexn{x_n\in\Cnum}$ and $\tuplexn{y_n\in\Cnum}$ be complex $\xN$-tuples. 
\thmbox{
  \brp{\sum_{n=1}^{\xN} \abs{x_n+y_n}^p}^\frac{1}{p}
  \le
  \brp{\sum_{n=1}^{\xN} \abs{x_n}^p}^\frac{1}{p}
  +
  \brp{\sum_{n=1}^{\xN} \abs{y_n}^p}^\frac{1}{p}
  \qquad
  \forall \; 1<p<\infty
  }
\end{theorem}
\if 0
\begin{proof}
\mbox{}\\
\begin{enumerate}
  \item Define $q$ in terms of $p$ such that $\frac{1}{p}+\frac{1}{q}=1$

  \item Proof that $\frac{1}{q} = \frac{p-1}{p}$:  \label{item:lp_minkowski_qp}
    \begin{align*}
      \frac{1}{p}+\frac{1}{q}=1
        &\iff \frac{1}{q} = 1-\frac{1}{p}
      \\&\iff \frac{1}{q} = \frac{p-1}{p}
    \end{align*}

  \item Define $\normn$ as follows:
    \[ \norm{\vx}_p \eqd \brp{\sum_{n=1}^{\xN} \abs{x_n}^p}^\frac{1}{p} \]

  \item Proof that $\norm{x_n+y_n}_p \le \norm{x_n}_p + \norm{y_n}_p$:
    \begin{align*}
      \norm{x_n+y_n}_p^p
        &= \sum_{n=1}^{\xN} \abs{x_n+y_n}^p
        && \text{by definition of $\normn_p$}
      \\&= \sum_{n=1}^{\xN} \abs{x_n+y_n} \abs{x_n+y_n}^{p-1}
        && \text{by homogeneous property of $\absn$\ifdochas{numsys}{ \prefpo{thm:C_norm}}}
      \\&\le \sum_{n=1}^{\xN} \abs{x_n} \abs{x_n+y_n}^{p-1} +
                \sum_{n=1}^{\xN} \abs{y_n} \abs{x_n+y_n}^{p-1}
        && \text{by subadditive property of $\absn$\ifdochas{numsys}{ \prefpo{thm:C_norm}}}
      \\&= \sum_{n=1}^{\xN} \abs{x_n (x_n+y_n)^{p-1}} +
                \sum_{n=1}^{\xN} \abs{y_n (x_n+y_n)^{p-1}}
        && \text{by homogeneous property of $\absn$\ifdochas{numsys}{ \prefpo{thm:C_norm}}}
      \\&\le \norm{x_n}_p \norm{(x_n+y_n)^{p-1}}_q + \norm{y_n}_p \norm{(x_n+y_n)^{p-1}}_q
        && \text{by H/:older's Inequality \prefpo{thm:lp_holder}}
      \\&= \brp{\norm{x_n}_p+\norm{y_n}_p} \norm{(x_n+y_n)^{p-1}}_q
      \\&= \brp{\norm{x_n}_p+\norm{y_n}_p} \brp{\sum_{n=1}^{\xN}\abs{(x_n+y_n)^{p-1}}^q}^\frac{1}{q}
        && \text{by definition of $\normn_p$}
      \\&= \brp{\norm{x_n}_p+\norm{y_n}_p} \brp{\sum_{n=1}^{\xN}\abs{(x_n+y_n)^{p-1}}^\frac{p}{p-1}}^\frac{p-1}{p}
        && \text{by \pref{item:lp_minkowski_qp}}
      \\&= \brp{\norm{x_n}_p+\norm{y_n}_p} \brp{\sum_{n=1}^{\xN}\abs{(x_n+y_n)}^p}^\frac{p-1}{p}
      \\&= \brp{\norm{x_n}_p+\norm{y_n}_p} \norm{x_n+y_n}^{p-1}
      \\&\implies \norm{x_n+y_n}_p \le \norm{x_n}_p+\norm{y_n}_p
    \end{align*}

  \end{enumerate}
\end{proof}


%---------------------------------------
\qboxnps % 1826 quote by Abel
%---------------------------------------
  {
  in an 1826 letter written by Niels Henrik Abel
  \index{Abel}
  \index{quotes!Abel}
  \footnotemark
  }
  {../common/people/cauchy_wkp_pdomain.jpg}
  {Cauchy is the only one occupied with pure mathematics: Poisson, Fourier, Ampere, etc., 
   busy themselves exclusively with magnetism and other physical subjects.}
  \footnotetext{\begin{tabular}[t]{ll}
    quote: & \citerpg{boyer2011}{462}{0470525487}\\
    image: & \scs\url{http://en.wikipedia.org/wiki/File:Augustin-Louis_Cauchy_1901.jpg}, public domain
  \end{tabular}}

%---------------------------------------
\begin{theorem}[\thm{Cauchy-Schwarz Inequality}]
\footnote{
  %\citerpg{bullen2003}{183}{1402015224}\\ ???
  \citerp{ab}{278} \\
  \citor{schwarz1885}\\
  \citor{bouniakowsky1859} \\
  \citerpgc{hardy1934}{25}{0521358809}{Theorem 11}\\
  \citorpc{cauchy1821}{455}{???}
  }
\label{thm:seq_cs}
\index{inequalities!Cauchy-Schwarz}
\index{inequalities!Cauchy-Bunyakovsky-Schwarz}
%---------------------------------------
Let $\tuplexn{x_n\in\Cnum}$ and $\tuplexn{y_n\in\Cnum}$ be complex $\xN$-tuples. 
 \thmbox{\begin{array}{>{\ds}rc>{\ds}l@{\qquad}C}
   \abs{\sum_{n=1}^{\xN} x_n y_n^\ast }^2 &\le& \brp{\sum_{n=1}^{\xN} \abs{x_n}^2}\;\brp{\sum_{n=1}^{\xN} \abs{y_n}^2}
     & \forall \vx,\vy\in\spX
     \\
   \abs{\sum_{n=1}^{\xN} x_n y_n^\ast }^2 &=& \brp{\sum_{n=1}^{\xN} \abs{x_n}^2}\; \brp{\sum_{n=1}^{\xN} \abs{y_n}^2}
   \quad\iff\quad \exists a\in\Cnum \st \vy=a\vx
     & \forall \vx,\vy\in\spX
 \end{array}}
\end{theorem}
\begin{proof}
\begin{enumerate}
  \item The Cauchy-Schwarz inequality for sequences is a special case of the H/:older inequality (\pref{thm:lp_holder})
        for $p=q=2$.

  \item Alternatively, the Cauchy-Schwarz inequality for sequences is a special case of 
        the \hie{Cauchy-Schwarz inequality} in inner-product spaces:
    \begin{enumerate}
      \item $\inprod{x_n}{y_n}\eqd\sum_{n=1}^{\xN}x_n y_n$ is an inner-product and $\opair{\tuplen{x_n}}{\inprodn}$ is an inner-product space.
      \item By the \hie{Cauchy-Schwarz Inequality for inner-product spaces}\ifdochas{vsinprod}{ (\prefp{thm:cs})}, 
        \[\abs{\inprod{\vx}{\vy}}^2 \le \inprod{\vx}{\vx} \; \inprod{\vy}{\vy}\]
    \end{enumerate}

  \item Not only does the H/:older inequality imply the Cauchy-Schwarz inequality, but somewhat surprisingly,
        the converse is also true: The Cauchy-Schwarz inequality implies the H/:older inequality.%
        \footnote{
            \citerppgc{bullen2003}{183}{185}{1402015224}{Theorem 5}
          }
\end{enumerate}
\end{proof}


%--------------------------------------
\begin{proposition}
\footnote{
  \citerp{carothers2000}{43}
  }
%--------------------------------------
\propbox{
  (x+y)^p \le 2^p(x^p + y^p)
  \qquad
  \forall x,y\ge 0,\; 1<p<\infty
  }
\end{proposition}
\begin{proof}
\begin{align*}
  (x+y)^p
    &\le \brp{2\max\setn{x,y}}^p 
  \\&=   2^p \brp{\max\setn{x,y}}^p 
  \\&=   2^p \brp{\max\setn{x^p,y^p}} 
  \\&\le 2^p(x^p + y^p)
\end{align*}
\end{proof}


%=======================================
\section{Differentiation}
%=======================================
Some proofs require differentiation multiple times.
This is simplified thanks to the \hie{Leibniz rule}, also called the 
\hie{generalized product rule} (\hie{GPR}, next lemma).
The Leibniz rule is remarkably similar in form to the \hie{binomial theorem}.
%--------------------------------------
\begin{lemma}[Leibniz rule]
%\footnote{\url{http://en.wikipedia.org/wiki/Leibniz_pule_(generalized_product_pule)}}
\index{Leibniz rule}
\index{generalized product rule}
\index{binomial coefficient}
\label{lem:LGPR}
\footnote{
  \citerpg{benisrael2002}{154}{3211829245}\\
  \citor{leibniz1710}
  }
%--------------------------------------
Let $\ff(x),\fg(x)\in\spLL$ with derivatives
$\ff^{(n)}(x)\eqd\deriv{^n}{x^n}\ff(x)$ and
$\fg^{(n)}(x)\eqd\deriv{^n}{x^n}\fg(x)$ for $n=0,1,2,\ldots$,
and ${n\choose k}\eqd\frac{n!}{(n-k)!k!}$ (binomial coefficient).
Then
\lembox{
  \deriv{^n}{x^n}[\ff(x)\fg(x)] =
  \sum_{k=0}^n {n\choose k} \ff^{(k)}(x) \fg^{(n-k)}(x)
  }
\end{lemma}


%---------------------------------------
\begin{theorem}
\footnote{
  \citerpgc{chui}{86}{0121745848}{item (ii)}
  }
\label{thm:int01}
%---------------------------------------
Let $\ff$ be a continuous function in $\clFrc$ and $\ff^{(n)}$ the $n$th derivative of $\ff$.
\thmbox{
  \int_{\intcc{0}{1}^n} \ff^{(n)}\brp{\sum_{k=1}^n x_k} \dx_1\dx_2\cdots\dx_n = \sum_{k=0}^n (-1)^{n-k}\bcoef{n}{k}\ff(k)
  \qquad\forall n\in\Zp
  }
\end{theorem}
\begin{proof}
Proof by induction:
  \begin{enumerate}
    \item Proof for $n=1$ case:
      \begin{align*}
        \int_\intcc{0}{1} \ff^{(1)}\brp{x} \dx
          &= \left.\ff(x)\right|_0^1
        \\&= \ff(1)-\ff(0)
        \\&= (-1)^{1+1}\bcoef{1}{1}\ff(1) + (-1)^{1+0}\bcoef{1}{0}\ff(0)
        \\&= \sum_{k=0}^n (-1)^{n-k}\bcoef{n}{k}\ff(k)
      \end{align*}

    \item Proof that $n$ case $\implies$ $n+1$ case:
      \begin{align*}
        &\int_{\intcc{0}{1}^{n+1}} \ff^{(n+1)}\brp{\sum_{k=1}^n x_k} \dx_1\dx_2\cdots\dx_{n+1}
        \\&= \int_{\intcc{0}{1}^n}\brb{\left. \ff^{(n)}\brp{x_{n+1}+\sum_{k=1}^n x_k}\right|_0^1} \dx_1\dx_2\cdots\dx_n
        \\&= \int_{\intcc{0}{1}^n}{\ff^{(n)}\brp{1+\sum_{k=1}^n x_k}-\ff^{(n)}\brp{0+\sum_{k=1}^n x_k}} \dx_1\dx_2\cdots\dx_n
        \\&= \sum_{k=0}^n (-1)^{n-k}\bcoef{n}{k}\ff(k+1) - \sum_{k=0}^n (-1)^{n-k}\bcoef{n}{k}\ff(k)
          \qquad\text{by $n$ case hypothesis}
        \\&= \sum_{k=1}^{n+1} (-1)^{n-k+1}\bcoef{n}{k-1}\ff(k) + \sum_{k=0}^n (-1)(-1)^{n-k}\bcoef{n}{k}\ff(k)
        \\&= \brb{ \ff(n+1)+ \sum_{k=1}^{n} (-1)^{n-k+1}\bcoef{n}{k-1}\ff(k) }
           + \brb{ (-1)^{n+1} \ff(0) + \sum_{k=1}^n (-1)^{n-k+1}\bcoef{n}{k}\ff(k)}
        \\&= \ff(n+1) + (-1)^{n+1}\ff(0) 
            + \sum_{k=1}^{n} (-1)^{n-k+1}\brs{\bcoef{n}{k-1}+\bcoef{n}{k}}\ff(k) 
        \\&= \ff(n+1) + (-1)^{n+1}\ff(0) 
            + \sum_{k=1}^{n} (-1)^{n-k+1}\bcoef{n+1}{k}\ff(k) 
          \qquad\text{by \thme{Pascal's Recursion} \ifxref{binomial}{thm:stifel}}
        \\&= \sum_{k=0}^{n+1} (-1)^{n-k+1}\bcoef{n+1}{k}\ff(k) 
      \end{align*}
  \end{enumerate}
\end{proof}


%=======================================
\section{Power Sums}
%=======================================
%--------------------------------------
\begin{theorem}[Geometric Series]
\footnote{
  \citerpc{hall1894}{39}{article 55}
  }
\label{thm:series_geometric}
%--------------------------------------
\thmbox{
  \sum_{k=0}^{n-1} r^k = \frac{1-p^n}{1-p}
  \qquad
  \forall p\in\Cnum\setd0
  }
\end{theorem}
\begin{proof}
\begin{align*}
  \sum_{k=0}^{n-1} r^k
    &= \brp{\frac{1}{1-p}}\brs{\brp{1-p}\sum_{k=0}^{n-1} r^k}
  \\&= \brp{\frac{1}{1-p}}\brs{\sum_{k=0}^{n-1} r^k - r\sum_{k=0}^{n-1} r^k}
  \\&= \brp{\frac{1}{1-p}}\brs{\sum_{k=0}^{n-1} r^k - \brp{\sum_{k=0}^{n-1} r^k -1 + r^n}}
  \\&= \brp{\frac{1}{1-p}}\brs{1 - r^n}
  \\&= \frac{1-p^n}{1-p}
\end{align*}
\end{proof}

%--------------------------------------
\begin{lemma}
\label{lem:series_sumT}
%--------------------------------------
Let $\ff(t)$ be a function.
\lembox{
  \sum_{n\in\Z} \ff(t+nT) \qquad \text{is \prope{periodic} with period $T$.}
}
\end{lemma}
\begin{proof}
Let $\fh(t) \eqd \sum_{n\in\Z} \ff(t+nT)$.
\begin{align*} 
  \fh(t+T) 
    &\eqd \sum_{n\in\Z} \ff(t+T+nT) 
  \\&=    \sum_{n\in\Z} \ff(t+(n+1)T) 
  \\&=    \sum_{n\in\Z} \ff(t+nT) 
  \\&\eqd \fh(t)
\end{align*}
\end{proof}


%--------------------------------------
\begin{proposition}[{Power Sums}]
\footnote{
  \citerppg{amann2008}{51}{57}{3764374721}\\
  \citerpgc{menini2004}{91}{0824709853}{Exercises 5.36--5.39}
  %http://mathworld.wolfram.com/PowerSum.html
  }
%--------------------------------------
\propbox{\begin{array}{>{\ds}lc>{\ds}l @{\qquad}C}
  \sum_{k=1}^n k   &=& \frac{n(n+1)}{2}        &\forall n\in\Zp  \\
  \sum_{k=1}^n k^2 &=& \frac{n(n+1)(2n+1)}{6}  &\forall n\in\Zp  \\
  \sum_{k=1}^n k^3 &=& \frac{n^2(n+1)^2}{4}  &\forall n\in\Zp  \\
  \sum_{k=1}^n k^4 &=& \frac{n(n+1)(2n+1)(3n^2+3n-1)}{30}  &\forall n\in\Zp
\end{array}}
\end{proposition}
\begin{proof}
\begin{align*}
\intertext{1. Proof that $\sum_{k=1}^n k=\frac{n(n+1)}{2}$: (proof by induction)}
  \sum_{k=1}^{n=1} k
    &= 1
  \\&= \frac{1(1+1)}{2}
  \\&= \left.\frac{n(n+1)}{2}\right|_{n=1} 
  \\
  \sum_{k=1}^{n+1} k
    &= \Bigg(\sum_{k=1}^n k\Bigg) + \Bigg(n+1\Bigg)
  \\&= \Bigg(\frac{n(n+1)}{2}\Bigg) + \Bigg(n+1\Bigg)
    && \text{by left hypothesis}
  \\&= \Bigg(n+1\Bigg)\Bigg(\frac{n}{2}+1\Bigg)  
  \\&= \Bigg(n+1\Bigg)\Bigg(\frac{n+2}{2}\Bigg)  
  \\&= \frac{(n+1)(n+2)}{2} 
  \\
\intertext{2. Proof that $\sum_{k=1}^n k^2 = \frac{n(n+1)(2n+1)}{6}$: (proof by induction)}
  \sum_{k=1}^{n=1} k^2 
    &= 1
  \\&= \frac{1(1+1)(2+1)}{6}
  \\&= \left.\frac{n(n+1)(2n+1)}{6}\right|_{n=1} 
  \\
  \sum_{k=1}^{n+1} k^2
    &= \Bigg(\sum_{k=1}^n k^2\Bigg) + \Bigg(n+1\Bigg)^2
  \\&= \Bigg(\frac{n(n+1)(2n+1)}{6}\Bigg) + \Bigg(n+1\Bigg)^2
    && \text{by left hypothesis}
  \\&= \Bigg(n+1\Bigg)\Bigg(\frac{n(2n+1)+6(n+1)}{6}\Bigg) 
  \\&= \Bigg(n+1\Bigg)\Bigg(\frac{2n^2+7n+6}{6}\Bigg) 
  \\&= \Bigg(n+1\Bigg)\Bigg(\frac{(n+2)(2n+3)}{6}\Bigg) 
  \\&= \frac{(n+1)[(n+1)+1][2(n+1)+1]}{6} 
\end{align*}
\end{proof}


\fi