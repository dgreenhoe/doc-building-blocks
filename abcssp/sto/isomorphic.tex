%============================================================================
% Daniel J. Greenhoe
% LaTeX / XeLaTeX file
%============================================================================
%=======================================
\subsection{Isomorphic spaces}
%=======================================
%---------------------------------------
\begin{definition}
\citetbl{
  \citerp{burris2000}{10}
  }
\label{def:poset_orderpre}
\label{def:opreserve}
\index{poset!order preserving}
%---------------------------------------
Let $\spX\eqd\opair{\setX}{\orel}$ and $\spY\eqd\opair{\setY}{\orela}$ be ordered sets.
\\\defboxp{
  A function $\ds\ftheta\in\clFxy$ is \propd{order preserving} in $\opair{\spX}{\spY}$ if
  \indentx
    $\ds \brp{x \orel y} \implies \brp{\ftheta(x)\orela\ftheta(y)} \quad\scy\forall x,y\in\setX$.
  }
\end{definition}

%---------------------------------------
\begin{definition}
\label{def:lat_iso}
\label{def:isomorphic}
\index{lattice!isomorphic}
%---------------------------------------
%Let $\spX\eqd\opair{\setX}{\orel}$ be an \structe{ordered set} that induces the pair $\opair{\join}{\meet}$
%and $\spY\eqd\opair{\setY}{\orela}$ an \structe{ordered set} that induces the pair $\opair{\joina}{\meeta}$.
Let $\latL_1\eqd\latticed$ and
    $\latL_2\eqd\lattice{\setY}{\orela}{\joina}{\meeta}$ be \structe{lattice}s.%
\\\defboxp{ %\indxs{\iso}
  $\latL_1$ and $\latL_2$ are %\propd{algebraically isomorphic}, or simply \propd{isomorphic}, if
  \propd{isomorphic} on $\opair{\spX}{\spY}$ if
  there exists a function $\ftheta\in\clFxy$ such that
  \\\indentx
  $\begin{array}{Frcl @{\qquad}D@{\qquad}D@{\qquad}D}%
     1. & \ftheta(x \join y) &=& \ftheta(x) \joina \ftheta(y)
        & \forall x,y\in\setX
        & (\prope{preserves joins})
        & and \\
     2. & \ftheta(x \meet y) &=& \ftheta(x) \meeta \ftheta(y)
        & \forall x,y\in\setX
        & (\prope{preserves meets}).
   \end{array}$
  \\
  In this case, the function $\ftheta$ is said to be an \fnctd{isomorphism}
  from $\latL_1$ to $\latL_2$,
  and the isomorphic relationship between $\latL_1$ and $\latL_2$ is denoted as
  %\\\indentx
    $\latL_1\iso\latL_2$.
  }
\end{definition}

%---------------------------------------
\begin{theorem}
\citetbl{
  \citerp{burris2000}{10}
  }
\label{thm:latiso}
%---------------------------------------
Let $\latticed$ and
    $\lattice{\setY}{\orela}{\joina}{\meeta}$ be lattices
and $\ftheta\in\clFxy$ be a \prope{bijective} function with inverse $\ftheti\in\clFyx$.
%\footnote{\hie{order isomorphic}: \prefp{def:poset_iso}}.
%Let $\latticed \iso \lattice{\setY}{\orela}{\joina}{\meeta}$ represent the condition that the two lattices are \hie{isomorphic}.
\thmbox{
  \mcom{\brbr{\begin{array}{lcl c lcl CD}
    x_1 &\orel & x_2 &\implies& \ftheta(x_1)      &\orela& \ftheta(x_2)      & \forall x_1,x_2\in\setX & and\\
    y_1 &\orela& y_2 &\implies& \ftheta^{-1}(y_1) &\orela& \ftheta^{-1}(y_2) & \forall y_1,y_2\in\setY
  \end{array}}
  }{$\ftheta$ and $\ftheta^{-1}$ are \prope{order preserving}}
  \iff
  \mcom{\latticed \iso \lattice{\setY}{\orela}{\joina}{\meeta}}
       {\prope{isomorphic}}
  }
\end{theorem}
%\footnotetext{\prope{order preserving}: \prefp{def:poset_orderpre}}
\begin{proof}
Let $\ftheta\in\clFxy$ be the isomorphism between lattices
$\latticed$ and $\lattice{\setY}{\orela}{\joina}{\meeta}$.

\begin{enumerate}
\item Proof that \prope{order preserving} $\implies$ \prope{preserves joins}:
  \begin{enumerate}
    \item Proof that $\ftheta(x_1\join x_2) \orelad \ftheta(x_1)\joina\ftheta(x_2)$:
          \label{item:lat_iso_join_ge}
      \begin{enumerate}
        \item Note that
          \begin{align*}
            x_1 &\orel x_1\join x_2 \\
            x_2 &\orel x_1\join x_2.
          \end{align*}

        \item Because $\ftheta$ is \prope{order preserving}
          \begin{align*}
            \ftheta(x_1) &\orela \ftheta(x_1\join x_2) \\
            \ftheta(x_2) &\orela \ftheta(x_1\join x_2).
          \end{align*}

        \item We can then finish the proof of \pref{item:lat_iso_join_ge}:
          \begin{align*}
            \ftheta(x_1)\joina\ftheta(x_2)
              &\orela \mcom{\ftheta(x_1\join x_2)}{$x_1\orel x_1\join x_2$}
                   \joina
                   \mcom{\ftheta(x_1\join x_2)}{$x_2\orel x_1\join x_2$}
              &&   \text{by \prope{order preserving} hypothesis}
            \\&=   \ftheta(x_1\join x_2)
              &&   \text{by \prope{idempotent} property \prefpo{thm:lattice}}
          \end{align*}
      \end{enumerate}

    \item Proof that $\ftheta(x_1\join x_2)\orela\ftheta(x_1)\joina\ftheta(x_2)$: \label{item:lat_iso_join_le}
      \begin{enumerate}
        \item Just as in \pref{item:lat_iso_join_ge}, note that
              $\ftheti(y_1)\join\ftheti(y_2) \orel \ftheti(y_1\joina y_2)$:
              \label{item:lat_iso_join_ge_1}
          \begin{align*}
            \ftheti(y_1)\join\ftheti(y_2)
              &\orel \mcom{\ftheti(y_1\joina y_2)}{$y_1\orela y_1\joina y_2$}
                   \join
                   \mcom{\ftheti(y_1\joina y_2)}{$y_2\orela y_1\joina y_2$}
              &&   \text{by \prope{order preserving} hypothesis}
            \\&=   \ftheti(y_1\joina y_2)
              &&   \text{by \prope{idempotent} property \prefpo{thm:lattice}}
          \end{align*}

        \item Because $\ftheta$ is \prope{order preserving} \label{item:lat_iso_join_y1y2}
          \begin{align*}
            \ftheta\brs{\ftheti(y_1)\join\ftheti(y_2)}
              &\orela\ftheta\ftheti(y_1\joina y_2)
              &&   \text{by \prefp{item:lat_iso_join_ge_1}}
            \\&= y_1\joina y_2
              && \text{by definition of inverse function $\ftheti$}
          \end{align*}

        \item Let $u_1\eqd\ftheta(x_1)$ and $u_2\eqd\ftheta(x_2)$. \label{item:lat_iso_u1u2}
        \item We can then finish the proof of \pref{item:lat_iso_join_le}:
          \begin{align*}
            \ftheta(x_1\join x_2)
              &= \ftheta\brs{\ftheti\ftheta(x_1) \join \ftheti\ftheta(x_2)}
              && \text{by definition of inverse function $\ftheti$}
            \\&= \ftheta\brs{\ftheti(u_1) \join \ftheti(u_2)}
              && \text{by definition of $u_1,u_2$, \pref{item:lat_iso_u1u2}}
            \\&\orela u_1 \joina u_2
              && \text{by \pref{item:lat_iso_join_y1y2}}
            \\&= \ftheta(x_1) \joina \ftheta(x_2)
              && \text{by definition of $u_1,u_2$, \pref{item:lat_iso_u1u2}}
          \end{align*}
      \end{enumerate}

    \item And so, combining \pref{item:lat_iso_join_ge} and \pref{item:lat_iso_join_le}, we have
      \[ \brbr{\begin{array}{rclDD}
           \ftheta(x_1\join x_2) &\orelad& \ftheta(x_1)\joina\ftheta(x_2) & \xref{item:lat_iso_join_ge} & and \\
           \ftheta(x_1\join x_2) &\orela & \ftheta(x_1)\joina\ftheta(x_2) & \xref{item:lat_iso_join_le} &
         \end{array}}
         \qquad\implies\qquad
         \ftheta(x_1\join x_2) =\ftheta(x_1)\joina\ftheta(x_2)
      \]
  \end{enumerate}

\item Proof that \prope{order preserving} $\implies$ \prope{preserves meets}:
  \begin{enumerate}
    \item Proof that $\ftheta(x_1\meet x_2) \orela \ftheta(x_1)\meeta\ftheta(x_2)$:
          \label{item:lat_iso_meet_le}
      \begin{align*}
        \ftheta(x_1)\meeta\ftheta(x_2)
          &\orelad \mcom{\ftheta(x_1\meet x_2)}{$x_1\oreld x_1\meet x_2$}
               \meeta
               \mcom{\ftheta(x_1\meet x_2)}{$x_2\oreld x_1\meet x_2$}
          &&   \text{by \prope{order preserving} hypothesis}
        \\&=   \ftheta(x_1\meet x_2)
          &&   \text{by \prope{idempotent} property \prefpo{thm:lattice}}
      \end{align*}

    \item Proof that $\ftheta(x_1\meet x_2)\orelad\ftheta(x_1)\meeta\ftheta(x_2)$:
      \label{item:lat_iso_meet_ge}
      \begin{enumerate}
        \item Just as in \pref{item:lat_iso_meet_le}, note that
              $\ftheti(y_1)\meet\ftheti(y_2) \oreld \ftheti(y_1\meeta y_2)$:
              \label{item:lat_iso_meet_ge_1}
          \begin{align*}
            \ftheti(y_1)\meet\ftheti(y_2)
              &\oreld \mcom{\ftheti(y_1\meeta y_2)}{$y_1\orelad y_1\meeta y_2$}
                   \meeta
                   \mcom{\ftheti(y_1\meeta y_2)}{$y_2\orelad y_1\meeta y_2$}
              &&   \text{by \prope{order preserving} hypothesis}
            \\&=   \ftheti(y_1\meeta y_2)
              &&   \text{by \prope{idempotent} property \prefpo{thm:lattice}}
          \end{align*}

        \item Because $\ftheta$ is \prope{order preserving}
              \label{item:lat_iso_meet_y1y2}
          \begin{align*}
            \ftheta\brs{\ftheti(y_1)\meet\ftheti(y_2)}
              &\orelad\ftheta\ftheti(y_1\meeta y_2)
              &&   \text{by \pref{item:lat_iso_meet_ge_1}}
            \\&= y_1\meeta y_2
          \end{align*}

        \item Let $v_1\eqd\ftheta(x_1)$ and $v_2\eqd\ftheta(x_2)$. \label{item:lat_iso_meet_v1v2}
        \item We can then finish the proof of \pref{item:lat_iso_meet_le}:
          \begin{align*}
            \ftheta(x_1\meet x_2)
              &= \ftheta\brs{\ftheti\ftheta(x_1) \meet \ftheti\ftheta(x_2)}
            \\&= \ftheta\brs{\ftheti(v_1) \meet \ftheti(v_2)}
              && \text{by \pref{item:lat_iso_meet_v1v2}}
            \\&\orelad v_1 \meeta v_2
              && \text{by \pref{item:lat_iso_meet_y1y2}}
            \\&= \ftheta(x_1) \meeta \ftheta(x_2)
              && \text{by \pref{item:lat_iso_meet_v1v2}}
          \end{align*}
      \end{enumerate}

    \item And so, combining \pref{item:lat_iso_meet_le} and \pref{item:lat_iso_meet_ge}, we have
      \[ \brbr{\begin{array}{rclDD}
           \ftheta(x_1\meet x_2) &\orela  & \ftheta(x_1)\meeta\ftheta(x_2) & \xref{item:lat_iso_meet_le} & and \\
           \ftheta(x_1\meet x_2) &\orelad & \ftheta(x_1)\meeta\ftheta(x_2) & \xref{item:lat_iso_meet_ge} &
         \end{array}}
         \qquad\implies\qquad
         \ftheta(x_1\meet x_2) =\ftheta(x_1)\meeta\ftheta(x_2)
      \]

  \end{enumerate}


\item Proof that \prope{order preserving} $\impliedby$ \prope{isomorphic}:
  \begin{align*}
    x \orel y
      &\implies \ftheta(y) = \ftheta(x \join y) = \ftheta(x) \joina \ftheta(y)
      && \text{by right hypothesis}
    \\&\implies \ftheta(x) \orela \ftheta(y)
    \\
    x \orel y
      &\implies \ftheta(x) = \ftheta(x \meet y) = \ftheta(x) \meeta \ftheta(y)
      && \text{by right hypothesis}
    \\&\implies \ftheta(x) \orela \ftheta(y)
  \end{align*}

\end{enumerate}

\end{proof}


\begin{minipage}{\tw-40mm}%
%---------------------------------------
\begin{example}
\footnotemark
\label{ex:order_M2_L4}
%---------------------------------------
In the diagram to the right, the function $\ftheta\in\clFxy$ 
is \prope{order preserving} with respect to $\orel$ and $\orela$.
Note that $\ftheta^{-1}$ is \prope{not order preserving} and that
the ordered sets $\opair{\setX}{\orel}$ and $\opair{\setY}{\orela}$ are \prope{not isomorphic}---as already demonstrated 
by \pref{thm:latiso} that they cannot be.
%This example also illustrates the fact that 
%\prope{order preserving} does not imply \prope{isomorphic}.
\end{example}
\end{minipage}%
\citetblt{%
  \citerp{burris2000}{10}%
  }%
\hfill%
\begin{tabular}{c}
  \includegraphics{ssp/graphics/m2tol4.pdf}%
\end{tabular}

\begin{minipage}{\tw-40mm}%
%---------------------------------------
\begin{example}
\label{ex:order_M3_N5}
%---------------------------------------
In the diagram to the right, the function $\ftheta\in\clFxy$ 
is \prope{order preserving} with respect to $\orel$ and $\orela$.
Note that $\ftheta^{-1}$ is \emph{not} order preserving.
Like \prefpp{ex:order_M2_L4},
this example also illustrates the fact that 
that \prope{order preserving} does not imply \prope{isomorphic}.
\end{example}
\end{minipage}%
\hfill%
\begin{tabular}{c}
  \includegraphics{ssp/graphics/m3ton5.pdf}%
\end{tabular}

\begin{minipage}{\tw-40mm}%
%---------------------------------------
\begin{example}
\label{ex:order_N5_N5}
%---------------------------------------
In the diagram to the right, the function $\ftheta\in\clFxy$ 
is \prope{order preserving} with respect to $\orel$ and $\orela$.
Note that $\ftheta^{-1}$ is also \prope{order preserving} and that
the ordered sets $\opair{\setX}{\orel}$ and $\opair{\setY}{\orela}$
are \prope{isomorphic}---as already demonstrated 
by \pref{thm:latiso} that they must be.
\end{example}
\end{minipage}%
\hfill%
\begin{tabular}{c}
  \includegraphics{ssp/graphics/n5ton5.pdf}%
\end{tabular}

%%---------------------------------------
%\begin{example}
%\citetbl{
%  \citerpgc{munkres2000}{25}{0131816292}{Example 1\textsection3.9}
%  }
%%---------------------------------------
%\exbox{\text{
%  The function $\ff(x)\eqd\frac{x}{1-x^2}$ in $\clF{\intoo{-1}{1}}{\R}$ is \prope{bijective} and \prope{order preserving}.
%  }}
%\end{example}
%
%Lattices and \structe{ordered set} \xref{def:order}
%are examples of mathematical \structe{order structures}.
%Often we are interested in similarities between two lattices $\latL_1$ and $\latL_2$ 
%with respect to order.
%Similarities between lattices can be described by defining a function $\ftheta$ that maps 
%from the first lattice to the second.
%The degree of similarity can be roughly described in terms of the mapping $\ftheta$ as follows:
%\begin{dingautolist}{"AC}
%  \item If there exists a mapping that is \prope{bijective}\ifsxref{found}{def:f_types}
%        then the number of elements in $\latL_1$ and $\latL_2$ is the same. 
%        However, their order structure may still be very different.
%  \item Lattices $\latL_1$ and $\latL_2$ are more similar 
%        if there exists a mapping that is \prope{bijective} and 
%        \prope{order preserving} \xref{def:poset_orderpre}.
%        Despite having this property however, 
%        their order structure may still be remarkably different,
%        as illustrated by \prefpp{ex:order_M2_L4} and \prefpp{ex:order_M3_N5}.
%  \item Lattices $\latL_1$ and $\latL_2$ are essentially identical 
%        (except possibly for their labeling) 
%        if there exists a mapping $\ftheta$ that is not only 
%        \prope{bijective} and \prope{order preserving},
%        but whose \hie{inverse}\ifsxref{relation}{def:rel_inverse} 
%        is \emph{also} \prope{bijective} \xref{thm:lat_iso}.
%        In this case, the lattices $\latL_1$ and $\latL_2$ are \prope{isomorphic}
%        and the mapping $\ftheta$ is an \hie{isomorphism}.
%        An isomorphism between $\latL_1$ and $\latL_2$ implies that the two lattices
%        have an identical order structure. In particular,
%        the isomorphism $\ftheta$ preserves joins and meets (next definition).
%\end{dingautolist}
%



%\footnotetext{
%  \prope{order preserving}: \prefp{def:poset_orderpre}
%  }
%\begin{proof}
%Let $\ftheta\in\clFxy$ be the isomorphism between lattices
%$\latticed$ and $\lattice{\setY}{\orela}{\joina}{\meeta}$.
%
%\begin{enumerate}
%\item Proof that \prope{order preserving} $\implies$ \prope{preserves joins}:
%  \begin{enumerate}
%    \item Proof that $\ftheta(x_1\join x_2) \orelad \ftheta(x_1)\joina\ftheta(x_2)$:
%          \label{item:lat_iso_join_ge}
%      \begin{enumerate}
%        \item Note that
%          \begin{align*}
%            x_1 &\orel x_1\join x_2 \\
%            x_2 &\orel x_1\join x_2.
%          \end{align*}
%
%        \item Because $\ftheta$ is \prope{order preserving}
%          \begin{align*}
%            \ftheta(x_1) &\orela \ftheta(x_1\join x_2) \\
%            \ftheta(x_2) &\orela \ftheta(x_1\join x_2).
%          \end{align*}
%
%        \item We can then finish the proof of \pref{item:lat_iso_join_ge}:
%          \begin{align*}
%            \ftheta(x_1)\joina\ftheta(x_2)
%              &\orela \mcom{\ftheta(x_1\join x_2)}{$x_1\orel x_1\join x_2$}
%                   \joina
%                   \mcom{\ftheta(x_1\join x_2)}{$x_2\orel x_1\join x_2$}
%              &&   \text{by \prope{order preserving} hypothesis}
%            \\&=   \ftheta(x_1\join x_2)
%              &&   \text{by \prope{idempotent} property \prefpo{thm:lattice}}
%          \end{align*}
%      \end{enumerate}
%
%    \item Proof that $\ftheta(x_1\join x_2)\orela\ftheta(x_1)\joina\ftheta(x_2)$: \label{item:lat_iso_join_le}
%      \begin{enumerate}
%        \item Just as in \pref{item:lat_iso_join_ge}, note that
%              $\ftheti(y_1)\join\ftheti(y_2) \orel \ftheti(y_1\joina y_2)$:
%              \label{item:lat_iso_join_ge_1}
%          \begin{align*}
%            \ftheti(y_1)\join\ftheti(y_2)
%              &\orel \mcom{\ftheti(y_1\joina y_2)}{$y_1\orela y_1\joina y_2$}
%                   \join
%                   \mcom{\ftheti(y_1\joina y_2)}{$y_2\orela y_1\joina y_2$}
%              &&   \text{by \prope{order preserving} hypothesis}
%            \\&=   \ftheti(y_1\joina y_2)
%              &&   \text{by \prope{idempotent} property \prefpo{thm:lattice}}
%          \end{align*}
%
%        \item Because $\ftheta$ is \prope{order preserving} \label{item:lat_iso_join_y1y2}
%          \begin{align*}
%            \ftheta\brs{\ftheti(y_1)\join\ftheti(y_2)}
%              &\orela\ftheta\ftheti(y_1\joina y_2)
%              &&   \text{by \prefp{item:lat_iso_join_ge_1}}
%            \\&= y_1\joina y_2
%              && \text{by definition of inverse function $\ftheti$}
%          \end{align*}
%
%        \item Let $u_1\eqd\ftheta(x_1)$ and $u_2\eqd\ftheta(x_2)$. \label{item:lat_iso_u1u2}
%        \item We can then finish the proof of \pref{item:lat_iso_join_le}:
%          \begin{align*}
%            \ftheta(x_1\join x_2)
%              &= \ftheta\brs{\ftheti\ftheta(x_1) \join \ftheti\ftheta(x_2)}
%              && \text{by definition of inverse function $\ftheti$}
%            \\&= \ftheta\brs{\ftheti(u_1) \join \ftheti(u_2)}
%              && \text{by definition of $u_1,u_2$, \pref{item:lat_iso_u1u2}}
%            \\&\orela u_1 \joina u_2
%              && \text{by \pref{item:lat_iso_join_y1y2}}
%            \\&= \ftheta(x_1) \joina \ftheta(x_2)
%              && \text{by definition of $u_1,u_2$, \pref{item:lat_iso_u1u2}}
%          \end{align*}
%      \end{enumerate}
%
%    \item And so, combining \pref{item:lat_iso_join_ge} and \pref{item:lat_iso_join_le}, we have
%      \[ \brbr{\begin{array}{rclDD}
%           \ftheta(x_1\join x_2) &\orelad& \ftheta(x_1)\joina\ftheta(x_2) & \xref{item:lat_iso_join_ge} & and \\
%           \ftheta(x_1\join x_2) &\orela & \ftheta(x_1)\joina\ftheta(x_2) & \xref{item:lat_iso_join_le} &
%         \end{array}}
%         \qquad\implies\qquad
%         \ftheta(x_1\join x_2) =\ftheta(x_1)\joina\ftheta(x_2)
%      \]
%  \end{enumerate}
%
%\item Proof that \prope{order preserving} $\implies$ \prope{preserves meets}:
%  \begin{enumerate}
%    \item Proof that $\ftheta(x_1\meet x_2) \orela \ftheta(x_1)\meeta\ftheta(x_2)$:
%          \label{item:lat_iso_meet_le}
%      \begin{align*}
%        \ftheta(x_1)\meeta\ftheta(x_2)
%          &\orelad \mcom{\ftheta(x_1\meet x_2)}{$x_1\oreld x_1\meet x_2$}
%               \meeta
%               \mcom{\ftheta(x_1\meet x_2)}{$x_2\oreld x_1\meet x_2$}
%          &&   \text{by \prope{order preserving} hypothesis}
%        \\&=   \ftheta(x_1\meet x_2)
%          &&   \text{by \prope{idempotent} property \prefpo{thm:lattice}}
%      \end{align*}
%
%    \item Proof that $\ftheta(x_1\meet x_2)\orelad\ftheta(x_1)\meeta\ftheta(x_2)$:
%      \label{item:lat_iso_meet_ge}
%      \begin{enumerate}
%        \item Just as in \pref{item:lat_iso_meet_le}, note that
%              $\ftheti(y_1)\meet\ftheti(y_2) \oreld \ftheti(y_1\meeta y_2)$:
%              \label{item:lat_iso_meet_ge_1}
%          \begin{align*}
%            \ftheti(y_1)\meet\ftheti(y_2)
%              &\oreld \mcom{\ftheti(y_1\meeta y_2)}{$y_1\orelad y_1\meeta y_2$}
%                   \meeta
%                   \mcom{\ftheti(y_1\meeta y_2)}{$y_2\orelad y_1\meeta y_2$}
%              &&   \text{by \prope{order preserving} hypothesis}
%            \\&=   \ftheti(y_1\meeta y_2)
%              &&   \text{by \prope{idempotent} property \prefpo{thm:lattice}}
%          \end{align*}
%
%        \item Because $\ftheta$ is \prope{order preserving}
%              \label{item:lat_iso_meet_y1y2}
%          \begin{align*}
%            \ftheta\brs{\ftheti(y_1)\meet\ftheti(y_2)}
%              &\orelad\ftheta\ftheti(y_1\meeta y_2)
%              &&   \text{by \pref{item:lat_iso_meet_ge_1}}
%            \\&= y_1\meeta y_2
%          \end{align*}
%
%        \item Let $v_1\eqd\ftheta(x_1)$ and $v_2\eqd\ftheta(x_2)$. \label{item:lat_iso_meet_v1v2}
%        \item We can then finish the proof of \pref{item:lat_iso_meet_le}:
%          \begin{align*}
%            \ftheta(x_1\meet x_2)
%              &= \ftheta\brs{\ftheti\ftheta(x_1) \meet \ftheti\ftheta(x_2)}
%            \\&= \ftheta\brs{\ftheti(v_1) \meet \ftheti(v_2)}
%              && \text{by \pref{item:lat_iso_meet_v1v2}}
%            \\&\orelad v_1 \meeta v_2
%              && \text{by \pref{item:lat_iso_meet_y1y2}}
%            \\&= \ftheta(x_1) \meeta \ftheta(x_2)
%              && \text{by \pref{item:lat_iso_meet_v1v2}}
%          \end{align*}
%      \end{enumerate}
%
%    \item And so, combining \pref{item:lat_iso_meet_le} and \pref{item:lat_iso_meet_ge}, we have
%      \[ \brbr{\begin{array}{rclDD}
%           \ftheta(x_1\meet x_2) &\orela  & \ftheta(x_1)\meeta\ftheta(x_2) & \xref{item:lat_iso_meet_le} & and \\
%           \ftheta(x_1\meet x_2) &\orelad & \ftheta(x_1)\meeta\ftheta(x_2) & \xref{item:lat_iso_meet_ge} &
%         \end{array}}
%         \qquad\implies\qquad
%         \ftheta(x_1\meet x_2) =\ftheta(x_1)\meeta\ftheta(x_2)
%      \]
%
%  \end{enumerate}
%
%
%\item Proof that \prope{order preserving} $\impliedby$ \prope{isomorphic}:
%  \begin{align*}
%    x \orel y
%      &\implies \ftheta(y) = \ftheta(x \join y) = \ftheta(x) \joina \ftheta(y)
%      && \text{by right hypothesis}
%    \\&\implies \ftheta(x) \orela \ftheta(y)
%    \\
%    x \orel y
%      &\implies \ftheta(x) = \ftheta(x \meet y) = \ftheta(x) \meeta \ftheta(y)
%      && \text{by right hypothesis}
%    \\&\implies \ftheta(x) \orela \ftheta(y)
%  \end{align*}
%
%\end{enumerate}
%
%\end{proof}
%
%
%%---------------------------------------
%\begin{example}
%\label{ex:lat_xyz_235}
%%---------------------------------------
%Let $\latL\isomorphic\latM$ represent the condition that a lattice $\latL$ and
%a lattice $\latM$ are \hie{isomorphic}.
%\exbox{\begin{array}{M}
%  $\ds\lattice{\pset{\setn{x,y,z}}}{\subseteq}{\setu}{\seti}
%   \isomorphic
%   \lattice{\setn{1,2,3,5,6,10,15,30}}{|}{\lcm}{\gcd}$
%  \\\qquad
%  with isomorphism
%  \\
%  $\ds\ftheta(\setA)=5^{\setind_\setA(z)} \cdot 3^{\setind_\setA(y)} \cdot 2^{\setind_\setA(x)}
%     \qquad\scriptstyle\forall\setA\in \pset{\setn{a,b,c}}$
%\end{array}}
%\\
%Explicit cases are listed below and illustrated in
%\prefpp{ex:poset_xyz} and \prefpp{ex:poset_532}.\\
%\hfill
%\begin{minipage}{3\tw/8}
%\begin{align*}
%  \ftheta\left(\emptyset   \right) &= 5^0 \cdot 3^0 \cdot 2^0  &=  1 \\
%  \ftheta\left(\setn{x    }\right) &= 5^0 \cdot 3^0 \cdot 2^1  &=  2 \\
%  \ftheta\left(\setn{  y  }\right) &= 5^0 \cdot 3^1 \cdot 2^0  &=  3 \\
%  \ftheta\left(\setn{x,y  }\right) &= 5^0 \cdot 3^1 \cdot 2^1  &=  6 \\
%\end{align*}
%\end{minipage}
%\hfill
%\begin{minipage}{3\tw/8}
%\begin{align*}
%  \ftheta\left(\setn{    z}\right) &= 5^1 \cdot 3^0 \cdot 2^0  &=  5 \\
%  \ftheta\left(\setn{x,  z}\right) &= 5^1 \cdot 3^0 \cdot 2^1  &= 10 \\
%  \ftheta\left(\setn{  y,z}\right) &= 5^1 \cdot 3^1 \cdot 2^0  &= 15 \\
%  \ftheta\left(\setn{x,y,z}\right) &= 5^1 \cdot 3^1 \cdot 2^1  &= 30 \\
%\end{align*}
%\end{minipage}
%\hfill
%\end{example}
%\begin{proof}
%\begin{align*}
%  \ftheta(\setA\setu\setB)
%    &= 5^{\setind_{\setA\setu\setB}(a)} \cdot
%       3^{\setind_{\setA\setu\setB}(b)} \cdot
%       2^{\setind_{\setA\setu\setB}(c)}
%  \\&= 5^{\setind_A(a)\join\setind_B(a)} \cdot
%       3^{\setind_A(b)\join\setind_B(b)} \cdot
%       2^{\setind_A(c)\join\setind_B(c)}
%    && \ifdochas{relation}{\text{by \pref{thm:setind} \prefpo{thm:setind}}}
%  \\&= \lcm\left(5^{\setind_A(a)},\; 5^{\setind_B(a)} \right) \;\cdot\;
%       \lcm\left(3^{\setind_A(b)},\; 3^{\setind_B(b)} \right) \;\cdot\;
%       \lcm\left(2^{\setind_A(c)},\; 2^{\setind_B(c)} \right)
%    &&
%  \\&= \lcm\left(
%         5^{\setind_A(a)} \cdot 3^{\setind_A(b)} \cdot 2^{\setind_A(c)},\;
%         5^{\setind_B(a)} \cdot 3^{\setind_B(b)} \cdot 2^{\setind_B(c)}
%       \right)
%  \\&= \lcm\left( \ftheta(A),\;\ftheta(B) \right)
%  \\
%  \\
%  \ftheta(\setA\seti\setB)
%    &= 5^{\setind_{\setA\seti\setB}(a)} \cdot
%       3^{\setind_{\setA\seti\setB}(b)} \cdot
%       2^{\setind_{\setA\seti\setB}(c)}
%  \\&= 5^{\setind_A(a)\meet\setind_B(a)} \cdot
%       3^{\setind_A(b)\meet\setind_B(b)} \cdot
%       2^{\setind_A(c)\meet\setind_B(c)}
%    && \ifdochas{relation}{\text{by \pref{thm:setind} \prefpo{thm:setind}}}
%  \\&= \gcd\left(5^{\setind_A(a)},\; 5^{\setind_B(a)} \right) \;\cdot\;
%       \gcd\left(3^{\setind_A(b)},\; 3^{\setind_B(b)} \right) \;\cdot\;
%       \gcd\left(2^{\setind_A(c)},\; 2^{\setind_B(c)} \right)
%    &&
%  \\&= \gcd\left(
%         5^{\setind_A(a)} \cdot 3^{\setind_A(b)} \cdot 2^{\setind_A(c)},\;
%         5^{\setind_B(a)} \cdot 3^{\setind_B(b)} \cdot 2^{\setind_B(c)}
%       \right)
%  \\&= \gcd\left( \ftheta(A),\;\ftheta(B) \right)
%\end{align*}
%\end{proof}

%%---------------------------------------
%\begin{definition}
%\label{def:ceil}
%\label{def:floor}
%%---------------------------------------
%Let $\opair{\R}{\orel}$ be the \structe{standard ordered set of real numbers}.
%The \fncte{floor   function} $\floor{x}\in\clFrz$ and
%the \fncte{ceiling function} $\ceil{x}\in\clFrz$ are defined as
%$\ceil{x} \eqd\meetop\set{n\in\Z}{n\oreld x}$ and 
%$\floor{x}\eqd\joinop\set{n\in\Z}{n\orel  x}$.
%\end{definition}

