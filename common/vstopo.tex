%============================================================================
% LaTeX File
% Daniel J. Greenhoe
%============================================================================
%======================================
\chapter{Topological Linear Spaces}
%======================================
%======================================
\section{Definitions}
%======================================
A \hie{topologial linear space} (often called a \hie{topological vector space}) 
is basically a \structe{linear space} \xref{def:vspace} with a \structe{topology} \xref{def:topology}. 
If the topology is generated by a \structe{metric}\ifsxref{metric}{def:metric}, 
then it is a \hie{metric linear space} \xref{def:vs_metric}.
If the topology is generated by a \structe{norm} \xref{def:norm}, then it is a \structe{normed linear space}.
If the topology is generated by an \fncte{inner product} \xref{def:inprod}, then it is an \structe{inner product space}.

%--------------------------------------
\begin{definition}
\footnote{
  \citerpgc{schaefer1999}{12}{0387987266}{1. Vector Space Topologies},
  \citerpgc{robertson1980}{3}{0521298822}{3. Topological Vector Spaces}
  }
\label{def:vs_top}
\label{def:toplinspace}
%--------------------------------------
\defboxt{%
  The tuple $\toplinspaceX$ is a \structd{topological linear space} if%
  \\\indentx$\begin{array}{>{\qquad}FlMMD}%
        1. & \linearspaceX                            &is a \structe{linear space}                              &                       & and 
      \\2. & \topT                                    &is a \structe{topology} on $\psetx$                      &                       & and
      \\3. & \opair{\vx}{\vy}\rightarrow\vx+\vy       &is \prope{continuous}   on $\clF{\spX\times\spX}{\spX}$  & \xref{def:continuous} & and
      \\4. & \opair{\alpha}{\vx}\rightarrow\alpha\vx  &is \prope{continuous}   on $\clF{\F\times\spX}{\spX}$.   & \xref{def:continuous} & .
  \end{array}$%
  }%
\end{definition}

%--------------------------------------
\begin{definition}
\label{def:vs_Cxy}
%--------------------------------------
Let $\toplinspace{\setX}{+}{\cdot}{\F}{\dotplus}{\dottimes}{\topT_x}$ be a \structe{topological linear space} with topology $\topT_x$.\\
Let $\toplinspace{\setX}{+}{\cdot}{\F}{\dotplus}{\dottimes}{\topT_y}$ be a \structe{topological linear space} with topology $\topT_y$.
Let $\clFxy$ be the set of all functions (operators) from $\setX$ to $\setY$.
\defbox{\begin{array}{l}
  \text{The set $\clCxy$ is the \structd{space of continuous operators} from $\setX$ to $\setY$ and is defined as}
  \\\indentx\clCxy \eqd \set{\ff\in\clFxy}{\text{$\ff$ is \prop{continuous} with respect to $\opair{\topT_x}{\topT_y}$}}
  \end{array}
  }
\end{definition}


%=======================================
%\section{Closed Linear Spaces}
%=======================================
%--------------------------------------
\begin{definition}
\label{def:subspace_closed}
%--------------------------------------
Let $\spY\eqd\toplinspaceY$ be a subspace of a \structe{topological linear space} $\spX\eqd\toplinspaceX$.
Let $\clsY$ be the \structe{closure} of the set $\setY$ \xref{def:clsA} in the 
\structe{topological space} $\topspaceX$ \xref{def:topology}.
\defbox{\begin{array}{M}
  The subspace $\spY$ is \hid{closed} in $\spX$ if 
  \\\indentx
    $\setY=\clsY$ . %($\setY$ is \prope{closed}).
\end{array}}
%\crossreftblt{
%  \crossrefp{closed}           {def:ss_closed}\\
%  \crossrefp{topological space}{def:topology} 
%  }
\end{definition}

%%--------------------------------------
%\begin{example}
%\footnote{
%  \citerpgc{kolmogorov1975}{140}{0486612260}{Example 1}
%  }
%%--------------------------------------
%\exbox{\begin{array}{M}
%  Let $\setX$ be the set of all bounded sequences with a finite number of zeros.\\ %\eqd\mcom{\set{\intoo{a}{b}}{a<b,\,a,b\in\R}$.\\
%  Let $\topT$ be the metric topology on $\setX$ generated by the metric $\metric{x}{y}\eqd\abs{x-y}$.\\
%  Then $\spX\eqd\linearspaceX$ is a \structe{linear space} and $\opair{\spX}{\topT}$ is a \structe{topological linear space}.\\
%  However, $\opair{\spX}{\topT}$ is \prope{not closed}, because, for example\\
%  %\indentx $\ds \seqn{
%\end{array}}
%\end{example}

%--------------------------------------
\begin{example}
\footnote{
  \citerpgc{kolmogorov1975}{140}{0486612260}{Example 1}
  }
%--------------------------------------
Let $\clsA$ be the \structe{closure} \xref{def:clsA} of a set $\setA$ in a topological space.
Let $\setX$ be the set of all bounded sequences over $\R$.
Let $\setY$ be the set of all bounded sequences with a finite number of zeros.
Let $\topT$ be the standard topology on $\R$ generated by the metric $\metric{x}{y}\eqd\abs{x-y}$.
%\crossreftbl{
%  \crossrefs{closure}{def:clsA}
%  }
\exbox{\begin{array}{M}
  $\spX$ is a \structe{topological linear space}.\\
  $\spY$ is a \structe{topological linear space}.\\
  But $\setY$ is \prope{not closed} in $\opair{\spX}{\topT}$ ($\setY\neq\clsY$), because, for example,
  \\\indentx
  $\seqn{1,\,\frac{1}{2},\,\frac{1}{3},\,\cdots,\,\frac{1}{n},\,\cdots}$ \emph{is} in $\setY$,\\
  but its \structe{closure point}
  \\\indentx
  $\seqn{1,\,\frac{1}{2},\,\frac{1}{3},\,\cdots,\,\frac{1}{n},\,\cdots,\,0,\,0,\,0,\,\cdots}$ is \emph{not} in $\setY$ (but is in $\setX$).
\end{array}}
\end{example}
%\begin{proof}
%  $\clsp{P_{\intcc{a}{b}}}=C_{\intcc{a}{b}}$ by \thme{Weierstrass' Approximation Theorem}.
%\end{proof}

%--------------------------------------
\begin{example}
\footnote{
  \citerpgc{kolmogorov1975}{140}{0486612260}{Example 2}
  }
%--------------------------------------
Let $\clsA$ be the \structe{closure} of a set $\setA$ in a topological space.
Let $C_{\intcc{a}{b}}$ be the set of all \prope{continuous}   functions on the real interval $\intcc{a}{b}$.
Let $P_{\intcc{a}{b}}$ be the set of all \structe{polynomials} on the real interval $\intcc{a}{b}$.
Let $\topT$ be the standard topology on $\R$ generated by the metric $\metric{x}{y}\eqd\abs{x-y}$.
%\crossreftbl{
%  \crossrefs{closure}{def:clsA}
%  }
\exbox{\begin{array}{M}
  $\opair{C_{\intcc{a}{b}}}{\topT}$ is a \structe{topological linear space}.\\
  $\opair{P_{\intcc{a}{b}}}{\topT}$ is a \structe{topological linear space}.\\
  But $P_{\intcc{a}{b}}\neq\clsp{P_{\intcc{a}{b}}}=C_{\intcc{a}{b}}$, so $P_{\intcc{a}{b}}$ is \prope{not closed} in $\opair{C_{\intcc{a}{b}}}{\topT}$.
\end{array}}
\end{example}
\begin{proof}
  $\clsp{P_{\intcc{a}{b}}}=C_{\intcc{a}{b}}$ by \thme{Weierstrass' Approximation Theorem}.
\end{proof}

%=======================================
\section{Dual Spaces}
%=======================================




%By this definition, norms are functionals.
%And with fixed $\vy$, any inner product $\ff(\vx)\eqd\inprodxy$ is also a functional.%
%\footnote{\begin{tabular}{ll}
%  \hie{norm}:          & \prefp{def:norm} \\
%  \hie{inner product}: & \prefp{def:inprod}
%\end{tabular}}
%If we gather together all the functionals $\ff:X\to\F$, then these
%functionals are themselves a linear space. This new linear space
%is called the \hie{dual space} $\spXd$ of the linear space $\spX$.

%---------------------------------------
\begin{definition}
\footnote{
  \citerpgc{hunter2001}{116}{9810241917}{Definition 5.54},
  \citerpgc{kurdila2005}{76}{3764321989}{Definitions 2.2.3, 2.2.4},
  \citerpgc{hewitt1965}{211}{0387901388}{Definition 14.6}
  }
\label{def:spXd}
\index{dual space}
\index{space!dual}
%---------------------------------------
Let $\toplinspaceX$ be a topological linear space.
Let $\clFxf$ be the set of all functionals from $\setX$ to $\F$.
Let $\clCxf$ be the space of \prop{continuous} functionals from $\setX$ to $\F$.
\defboxp{
  $\begin{array}{Mrcl}
    The \hid{algebraic dual space}   $\hxs{\spXad}$ of $\spX$ is & \spXad&\eqd&\clFxf.\\
    The \hid{topological dual space} $\hxs{\spXd}$  of $\spX$ is & \spXd &\eqd&\clCxf.
  \end{array}$\\
  {The space $\spX$ is the \hid{predual} of $\spXd$.}
  {A topological dual space is also called a \hid{dual space}, \hid{conjugate space} or \hid{adjoint space}.} 
  }
\end{definition}

%---------------------------------------
\begin{theorem}
%---------------------------------------
Let $\spX=\linearspaceX$ be a linear space with dual space $\spXd$.
\thmbox{
  \text{$\spXd$ is a linear space.}
  }
\end{theorem}





%=======================================
\section{Metric Linear Spaces}
%=======================================
Metric space stucture can be added to a linear space resulting in a \structe{metric linear space} (next definition).
One key difference between metric linear spaces and normed linear spaces is that the balls in a 
\structe{normed linear space} \xref{def:norm}
are always \prope{convex}\ifsxref{convex}{def:convex_set}; 
this is not true for all metric linear spaces.\cittrpg{bruckner1997}{478}{013458886X}
%--------------------------------------
\begin{definition}
\footnote{
  \citerpg{maddox1988}{90}{052135868X},
  \citerpgc{bruckner1997}{477}{013458886X}{Definition 12.3},
  \citerpg{rolewicz1985}{1}{9027714800}
  }
\label{def:vs_metric}
\index{space!metric vector|textbf}
\index{metric linear space|textbf}
%--------------------------------------
Let $\spO\eqd\metlinspaceX$.
\defbox{\begin{array}{>{\qquad}FMD}
  \mc{3}{M}{The tuple $\spO$ is a \hid{metric linear space} if}\\
      1. & if $\linearspaceX$ is a \structe{linear space} & and 
    \\2. & $\metricn$ is a \structe{metric} in $\clFxr$   & and
    \\3. & $\metric{\vx+\vz}{\vy+\vz} = \metric{\vx}{\vy} \quad\sst\forall \vx,\vy,\vz\in\setX$  (\prope{translation invariant}) & and 
    \\4. & $\alpha_n\to\alpha$ and $\vx_n\to\vx$ $\implies$ $\alpha_n\vx_n\to\alpha\vx$          
\end{array}}
\end{definition}


%---------------------------------------
\begin{theorem}
\label{thm:vsm_convex}
\footnote{
  \citerp{norfolk}{5}
  }
%---------------------------------------
Let $\spO\eqd\metlinspaceX$ be a metric linear space.
\thmbox{
  \mcom{
    \metric{\vtheta}{\lambda\vx+(1-\lambda)\vy}
    \le \lambda\metric{\vtheta}{\vx} + (1-\lambda)\metric{\vtheta}{\vy}
    }{$\metricn$ is a \hie{convex} function}
  \qquad\implies\qquad
  \brb{\begin{array}{c}
    \ball{\theta}{r} \in\spO\\
    \text{is convex} \\
    \scriptstyle \forall\vtheta\in\setX,\,r\in\Rp
  \end{array}}
  }
\end{theorem}
\begin{proof}
\begin{align*}
  \metric{\vtheta}{\lambda\vx+(1-\lambda)\vy}
    &\le \lambda\metric{\vtheta}{\vx} + (1-\lambda)\metric{\vtheta}{\vy}
    &&   \text{by convexity hypothesis}
  \\&\le   \lambda r  + (1-\lambda) r
    &&   \forall \vx,\vy\in\ball{\theta}{r}
  \\&=   r
  \\\implies & \lambda\vx+(1-\lambda)\vy \in \ball{\theta}{r}
    &&   \forall \vx,\vy\in\ball{\theta}{r}
  \\\implies & \ball{\theta}{r} \in(\spX,\metricn) \text{ is convex}
    &&   \forall \theta\in\spX
\end{align*}
\end{proof}


%---------------------------------------
\begin{theorem}
\footnote{
  \citerpp{norfolk}{5}{6},
  \url{http://groups.google.com/group/sci.math/msg/a6f0a7924027957d}
  }
\label{thm:vsm_convex_invariant}
%---------------------------------------
Let $\metlinspaceXR$ be a real metric linear space.
\thmbox{\begin{array}{l}
  \brb{\begin{array}{F lcl CDD}
    1. & \metric{\vx+\vz}{\vy+\vz} &=& \metric{\vx}{\vy}
       & \forall \vx,\vy,\vz\in\spX
       & (\prope{translation invariant}) & and 
       \\
    2. & \metric{\lambda\vx}{\lambda\vy} &=& \lambda\,\metric{\vx}{\vy}
       & \forall \vx,\vy\in\spX,\; \lambda\in [0,1]
       & (\prope{homogeneous})
  \end{array}}
  \\\qquad\implies\qquad
  \brb{\text{$\ball{\theta}{r} \in \opair{\spX}{\metricn}$ \quad is \prope{convex} \quad $\scriptstyle \forall\vtheta\in\spX,\,r\in\Rp$}}
\end{array}}
\end{theorem}
\begin{proof}
\begin{align*}
  &\metric{\vtheta}{\lambda\vx+(1-\lambda)\vy}
  \\&=   \metric{\vzero}{\lambda\vx+(1-\lambda)\vy-\vtheta}
    &&   \text{by translation invariance hypothesis}
  \\&=   \metric{\vzero}{\lambda(\vx-\vtheta)+(1-\lambda)(\vy-\vtheta)}
  \\&\le \metric{\vzero}{\lambda(\vx-\vtheta)}
     +   \metric{\lambda(\vx-\vtheta)}{\lambda(\vx-\vtheta)+(1-\lambda)(\vy-\vtheta)}
    &&   \text{by \prope{subadditive} property\ifsxref{metric}{def:metric}}
  \\&=   \metric{\vzero}{\lambda(\vx-\vtheta)}
     +   \metric{\vzero}{\vzero+(1-\lambda)(\vy-\vtheta)}
    &&   \text{by translation invariance hypothesis}
  \\&=   \lambda \metric{\vzero}{\vx-\vtheta}
     +   (1-\lambda) \metric{\vzero}{\vy-\vtheta}
    &&   \text{by homogeneous hypothesis}
  \\&=   \lambda \metric{\vtheta}{\vx}
     +   (1-\lambda) \metric{\vtheta}{\vy}
    &&   \text{by translation invariance hypothesis}
  \\&\le   \lambda r  + (1-\lambda) r
    &&   \forall \vx,\vy\in\ball{\theta}{r}
  \\&=   r
  \\\implies & \lambda\vx+(1-\lambda)\vy \in \ball{\theta}{r}
    &&   \forall \vx,\vy\in\ball{\theta}{r}
  \\\implies & \ball{\theta}{r} \in(\spX,\metricn) \text{ is convex}
    &&   \forall \theta\in\spX
\end{align*}
\end{proof}

%%---------------------------------------
%\begin{example}
%\label{ex:vs_discrete}
%%---------------------------------------
%Let $X$ be a set and $\fd:X^2\to\Rnn$.
%\exbox{
%  \fd(x,y) \eqd \left\{\renewcommand{\arraystretch}{1}\begin{array}{ll}1 &\text{for } x\ne y \\0&\text{for }x=y \end{array} \right.
%  \qquad\implies\qquad
%  B(x,1) \text{ is convex in } (\spX,\fd)
%  }
%\end{example}
%\begin{proof}
%\begin{align*}
%  \metric{\vtheta}{\lambda\vx+(1-\lambda)\vy}
%    &=   \left\{\begin{array}{ll}
%           0 &\text{for } \vtheta=\lambda\vx+(1-\lambda)\vy \end{array} \\
%           1 &\text{otherwise}
%         \right.
%  \\&\le 1
%  \\\implies & \lambda\vx+(1-\lambda)\vy \in \ball{\theta}{1}
%    &&   \forall \vx,\vy\in\ball{\theta}{r}
%  \\\implies & \ball{\theta}{1} \in(\spX,\metricn) \text{ is convex}
%    &&   \forall \theta\in\spX
%\end{align*}
%\end{proof}

