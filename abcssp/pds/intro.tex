%============================================================================
% Daniel J. Greenhoe
% XeLaTeX file
% introduction for 2014flat
%============================================================================
%=======================================
\section{Introduction and summary}
%=======================================
\structe{Metric space}s provide a framework for analysis and have several very useful properties.
Many of these properties follow in part from the \rele{triangle inequality}.
However, there are several applications\footnote{references for applications in which the \rele{triangle inequality} may not hold:
  \citePpc{maligranda1987}{54}{``pseudonorm"},
  \citePc{lin1998}{``similarity measures", Table 6},
  \citePc{veltkamp2000}{``shape similarity measures"},
  \citePc{veltkamp2001}{``shape matching"},
  \citePc{costa2003}{``network distance estimation"},
  \citePpgc{burstein2005}{287}{3540258663}{distance matrices for ``genome phylogenies"},
  \citerpgc{jimenez2006}{224}{0387233946}{``statistical distances"},
  \citePpc{szirmai2007}{388}{``geodesic ball"},
  \citePpc{crammer2007}{326}{``decision-theoretic learning"},
  \citePpc{crammer2008}{1758}{``\emph{approximate} triangle inequality"},
  \citePpc{vitanyi2011}{2455}{``information distance"}
  }
in which the triangle inequality does not hold but in which we would still like to perform analysis.
So the questions that natually follow are:
\\\indentx$\begin{array}{FM}
    Q1. & What happens if we remove the \prope{triangle inequality} all together?
  \\Q2. & What happens if we replace the \prope{triangle inequality} with a generalized relation?
\end{array}$

A \structe{distance space} is a \structe{metric space} without the \structe{triangle inequality} constraint.
\pref{sec:dspace} introduces \structe{distance space}s and demonstrates that some properties 
commonly associated with \structe{metric space}s also hold in any \structe{distance space}:
%For example, the following are not only true in \structe{metric space}s, but in any \structe{distance space} $\dpspaceX$ as well:
\\\indentx$\begin{array}{FMM}
      D1. & $\emptyset$ and $\setX$ are \prope{open}                                          & \xref{thm:dspace_open}
    \\D2. & the intersection of a finite number of open sets is \prope{open}                  & \xref{thm:dspace_open}
    \\D3. & the union of an arbitrary    number of open sets is \prope{open}                  & \xref{thm:dspace_open}
    \\D4. & every \prop{Cauchy} sequence is \prope{bounded}                                   & \xref{prop:cauchy==>bounded}
    \\D5. & any subsequence of a \prope{Cauchy} sequence is also \prope{Cauchy}               & \xref{prop:cauchy_subseq}
    \\D6. & the \thme{Cantor Intersection Theorem} holds                                      & \xref{thm:cit}
\end{array}$

The following five properties (M1--M5) \emph{do} hold in any \structe{metric space}.
However, the examples from \pref{sec:dspace} listed below demonstrate that the five properties do \emph{not} 
hold in all \structe{distance spaces}: %\footnote{The examples referred to for M1--M5 can all in essense be found in \citerppgc{blumenthal1953}{8}{12}{0828402426}{6. Topology of semimetric spaces}}
\\\indentx$\begin{array}{FMDM}
      M1. & the \fncte{metric function} is \prope{continuous}                  &fails to hold in&\pref{ex:dspace_01}--\pref{ex:dspace_21}
    \\M2. & \structe{open ball}s are \prope{open}                              &fails to hold in&\pref{ex:dspace_01} and \pref{ex:dspace_1n}
    \\M3. & the \structe{open ball}s form a \structe{base} for a topology      &fails to hold in&\pref{ex:dspace_01} and \pref{ex:dspace_1n}
    \\M4. & the limits of \structe{convergent sequence}s are \prope{unique}    &fails to hold in&\pref{ex:dspace_01}
    \\M5. & \prope{convergent} sequences are \prope{Cauchy}                    &fails to hold in&\pref{ex:dspace_1n}
\end{array}$
\\
Hence, \pref{sec:dspace} answers question Q1.

\pref{app:trirel} begins to answer question Q2 by first introducing a new function, 
called the \fncte{power triangle function}
%\footnote{
%  A different but similar definition can be found in \citePp{bessenyei2014}{2}
%  where it is a special case of a function that the author calls a ``\fncte{triangle function}".
%  }
in a \structe{distance space} $\dspaceX$, as 
\quad$\ptfD\eqd 2\sigma\brs{\frac{1}{2}\distancen^p(x,z) + \frac{1}{2}\distancen^p(z,y)}^\frac{1}{p}$\quad
for some $\opair{p}{\sigma}\in\Rx\times\R$.
%\footnote{%
%  where $\Rx$ is the \structe{set of extended real numbers} and $\Rp$ is the 
%  \structe{set of positive real numbers} \xref{def:Rx}
%  }
\pref{app:trirel} then goes on to use this function to define a new relation, called the 
\rele{power triangle inequality} in $\dspaceX$, and defined as 
\\\indentx$\trirelD\eqd\set{\otriple{x}{y}{z}\in\setX^3}{\distance{x}{y}\le\ptfD}$.

The \rele{power triangle inequality} is a generalized form of the \prope{triangle inequality} in the sense that 
the two inequalities coincide at $\opair{p}{\sigma}=\opair{1}{1}$.
Other special values include $\opair{1}{\sigma}$ yielding the \rele{relaxed triangle inequality} 
(and its associated \structe{near metric space})
and $\opair{\infty}{\sigma}$ yielding the \rele{\txsigma-inframetric inequality} 
(and its associated \structe{\txsigma-inframetric space}).
Collectively, a distance space with a power triangle inequality is herein called a \structe{power distance space}
and denoted $\pdspaceX$.\footnote{
  \structe{power triangle inequality}: \prefp{def:trirels};
  \structe{power distance space}: \prefp{def:pdspace};
  examples of \structe{power distance space}: \prefp{def:pdspace_spaces};
  }

The \fncte{power triangle function}, at $\sigma=\frac{1}{2}$, is a special case of the \fncte{power mean} with $\xN=2$ 
and $\lambda_1=\lambda_2=\frac{1}{2}$.
\fncte{Power mean}s have the elegant properties of being \prope{continuous} and \prope{monontone} with respect
to a free parameter $p$.
From this it is easy to show that the \fncte{power triangle function} is also 
\prope{continuous} and \prope{monontone} with respect to both $p$ and $\sigma$.
Special values of $p$ yield operators coinciding with \ope{maximum}, \ope{minimum}, \ope{mean square}, \ope{arithmetic mean},
\ope{geometric mean}, and \ope{harmonic mean}.
\fncte{Power mean}s are briefly described in \pref{app:pmean}.\footnote{
  \fncte{power triangle function}: \prefpp{def:ptf};
  \fncte{power mean}: \prefpp{def:pmean};
  power mean is \prope{continuous} and \prope{monontone}: \prefpp{thm:pmean_continuous};
  power triangle function is \prope{continuous} and \prope{monontone}: \prefpp{cor:tri_mono};
  Special values of $p$: \prefpp{cor:ftri_means}, \prefpp{cor:means}
  }

\pref{sec:pdspace_prop} investigates the properties of \structe{power distance spaces}.
In particular, it shows for what values of $\opair{p}{\sigma}$ the properties M1--M5 hold. % in these spaces.
Here is a summary of the results
%for what values of $\opair{p}{\sigma}$ in $\Rx\times\Rp$
%the five basic properties listed earlier for metric spaces also hold 
in a \structe{power distance space} $\pdspaceX$, for all $x,y,z\in\setX$:
\\\indentx$\begin{array}{FMMM}
    (M1) & holds for any $\opair{p}{\sigma}\in(\Rx\setd\setn{0})\times\Rp$ such that $2\sigma =   2^{\frac{1}{p}}$ & \xref{thm:pdspace_continuous}
  \\(M2) & holds for any $\opair{p}{\sigma}\in(\Rx\setd\setn{0})\times\Rp$ such that $2\sigma \le 2^{\frac{1}{p}}$ & \xref{cor:oball_open}
  \\(M3) & holds for any $\opair{p}{\sigma}\in(\Rx\setd\setn{0})\times\Rp$ such that $2\sigma \le 2^{\frac{1}{p}}$ & \xref{cor:tspace_base}
  \\(M4) & holds for any $\opair{p}{\sigma}\in\Rx\times\Rp$                                                        & \xref{thm:xn_to_xy}
  \\(M5) & holds for any $\opair{p}{\sigma}\in\Rx\times\Rp$                                                        & \xref{thm:convergent==>cauchy}
\end{array}$

\pref{app:topology} briefly introduces \structe{topological space}s.
The \structe{open ball}s of any \structe{metric space} form a \structe{base} for a \structe{topology}.
This is largely due to the fact that in a metric space, open balls are \prope{open}.
Because of this, in metric spaces it is convenient to use topological structure to define and exploit
analytic concepts such as \prope{continuity}, \prope{convergence}, \structe{closed set}s, 
\structe{closure}, \structe{interior}, and \structe{accumulation point}. %, \structe{boundary}.
For example, in a metric space, the traditional definition of defining continuity using open balls
and the topological definition using open sets, coincide with each other.
Again, this is largely because the open balls of a metric space are open.\footnote{
  \structe{open ball}: \prefp{def:ball}; 
  \structe{metric space}: \prefp{def:mspace};
  \structe{base}: \prefp{def:base};
  \structe{topology}: \prefp{def:topology};
  \structe{open}: \prefp{def:dspace_open};
  %\prope{continuity} in \structe{distance space}: \prefp{def:dspace_continuous};
  \prope{continuity} in \structe{topological space}: \prefp{def:continuous};
  \prope{convergence} in \structe{distance space}: \prefp{def:dspace_converge};
  \prope{convergence} in \structe{topological space}: \prefp{def:converge};
  \prope{closed set}: \prefp{def:closedset};
  \structe{closure}, \structe{interior}, \structe{accumulation point}: \prefp{def:clsA};
 %\structe{boundary}: \prefp{def:bndA};
  coincidence in all \structe{metric space}s and some \structe{power distance space}s: \prefp{thm:ms_converge};
  }

However, this is not the case for all \structe{distance space}s. 
In general, the open balls of a distance space are not open, and they are not a base for a topology.
In fact, the open balls of a distance space are a base for a topology if and only if the open balls are open.
While the open sets in a distance space do 
induce a topology, it's open balls may not.%
%\ldots and distance space definitions using open balls and topological definitions using open sets may not coincide.
\footnote{
  %\structe{distance space}: \prefp{def:dspace};
  \emph{if and only if} statement: \prefp{thm:baseoball};
  open sets of a distance space induce a topology: \prefp{cor:dspace_open};
  }
