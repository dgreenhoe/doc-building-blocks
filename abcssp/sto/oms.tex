%============================================================================
% Daniel J. Greenhoe
% XeLaTeX file
% stochastic systems
%============================================================================

%=======================================
\section{Ordered metric spaces}
%=======================================
%=======================================
\subsection{Definitions}
%=======================================
%---------------------------------------
\begin{definition}
\label{def:oms}
\label{def:oqms}
%---------------------------------------
\defboxp{
A triple $\omsG\eqd\omsD$ is an \structd{ordered quasi-metric space}
if $\opair{\setX}{\metricn}$ is a \structe{quasi-metric space} \xref{def:qmetric}
and $\opair{\setX}{\orel}$ is an \structe{ordered set} \xref{def:oset}.
\\\indentx\begin{tabular}{ll}
      $\omsG$ is an \structd{ordered metric space}         &if $\metricn$ is a \structe{metric} \xref{def:metric}.
    \\$\omsG$ is an \structd{unordered quasi-metric space} &if $\orel=\emptyset$.
    \\$\omsG$ is an \structd{unordered metric space}       &if $\metricn$ is a \structe{metric} and $\orel=\emptyset$.
\end{tabular}
}
\end{definition}

%---------------------------------------
\begin{remark}
%---------------------------------------
Note that the four structures defined in \pref{def:oms} are not mutually exclusive.\\
For example, by \pref{def:oms},
\\{$\begin{array}{lclcl}
  \{\text{\structe{unordered metric space}}\}       &\subseteq&
  \{\text{\structe{unordered quasi-metric space}}\} &\subseteq&
  \{\text{\structe{ordered quasi-metric space}}\}  
  \\
  \{\text{\structe{unordered metric space}}\}       &\subseteq&
  \{\text{\structe{ordered   metric space}}\}       &\subseteq& 
  \{\text{\structe{ordered quasi-metric space}}\}.  
\end{array}$}
\end{remark}

%---------------------------------------
\begin{remark}
\label{rem:qmetric}
%---------------------------------------
The use of the \fncte{quasi-metric} rather than exclusive use of the more restrictive \fncte{metric} in \pref{def:oms} is motivated
by state machines, where metrics measuring distances between states are in some cases by nature \prope{non-symmetric}.
One such example is the \structe{linear congruential pseudo-random number generator} \xref{ex:lcg7x1m9_dgraph}.
%See \prefpp{ex:lcg7x1m9_xyz}--\prefpp{ex:lcg7x1m9_dgraph} for demonstration.
\end{remark}

\begin{figure}
\centering%
\gsize%
{\footnotesize\begin{tabular}{ccc}
   \includegraphics{sto/graphics/Zline.pdf}%   \psset{unit=3mm}%============================================================================
% Daniel J. Greenhoe
% LaTeX file
% ordered metric space
% real line (R, |.|, <=)
%============================================================================
{%\psset{unit=0.5\psunit}%
\begin{pspicture}(-0.5,-3)(0.5,3)%
  %---------------------------------
  % options
  %---------------------------------
  \psset{%
    %radius=1.25ex,
    %labelsep=2.5mm,
    linecolor=blue,%
    }%
    %\psline{<->}(0,-4)(0,4)%
    %\pnode(0,4){L8}%
    \pnode(0,3){L7}%
    \Cnode(0,2){L6}%
    \Cnode(0,1){L5}%
    %\Cnode[fillstyle=solid,fillcolor=snode](0,0){L4}%
    \Cnode(0,0){L4}%
    \Cnode(0,-1){L3}%
    \Cnode(0,-2){L2}%
    \pnode(0,-3){L1}%
    %\pnode(0,-4){L0}%
   %\Cnode[fillstyle=solid,linecolor=snode,fillcolor=snode,radius=0.5ex](0,0){LC}%
  %\ncline[linestyle=dotted]{L7}{L8}%
  \ncline[linestyle=dotted]{L6}{L7}%
  \ncline{L5}{L6}%
  \ncline{L4}{L5}%
  \ncline{L3}{L4}%
  \ncline{L2}{L3}%
  \ncline[linestyle=dotted]{L1}{L2}%
  %\ncline[linestyle=dotted]{L1}{L0}%
  %\rput(L7){\psline[linewidth=1pt](-0.1,0)(0.1,0)}%
  %\rput(L6){\psline[linewidth=1pt](-0.1,0)(0.1,0)}%
  %\rput(L5){\psline[linewidth=1pt](-0.1,0)(0.1,0)}%
  %\rput(L4){\psline[linewidth=1pt](-0.1,0)(0.1,0)}%
  %\rput(L3){\psline[linewidth=1pt](-0.1,0)(0.1,0)}%
  %\rput(L2){\psline[linewidth=1pt](-0.1,0)(0.1,0)}%
  %\rput(L1){\psline[linewidth=1pt](-0.1,0)(0.1,0)}%
  %
  %\rput(L7){$3$}%
  \uput[180](L6){$2$}%
  \uput[180](L5){$1$}%
  \uput[180](L4){$0$}%
  \uput[180](L3){$-1$}%
  \uput[180](L2){$-2$}%
  %\rput(L1){$-3$}%
\end{pspicture}
}%%   
  &\includegraphics{sto/graphics/realline.pdf}%\psset{unit=3mm}%============================================================================
% Daniel J. Greenhoe
% LaTeX file
% ordered metric space
% real line (R, |.|, <=)
%============================================================================
{%\psset{unit=0.5\psunit}%
\begin{pspicture}(-0.5,-3)(0.5,3)%
  %---------------------------------
  % options
  %---------------------------------
  \psset{%
    %radius=1.25ex,
    %labelsep=2.5mm,
    linecolor=blue,%
    }%
    \psline{<->}(0,-3)(0,3)%
    %\pnode(0,3){L7}%
    \pnode(0,2){L6}%
    \pnode(0,1){L5}%
    \pnode(0,0){L4}%
    \pnode(0,-1){L3}%
    \pnode(0,-2){L2}%
    %\pnode(0,-3){L1}%
   %\Cnode[fillstyle=solid,linecolor=snode,fillcolor=snode,radius=0.5ex](0,0){L4}%
  %\ncline{L5}{L6}%
  %\ncline{L4}{L5}%
  %\ncline{L3}{L4}%
  %\ncline{L2}{L3}%
  %\ncline{L1}{L2}%
  %\rput(L7){\psline[linewidth=1pt](-0.1,0)(0.1,0)}%
  \rput(L6){\psline[linewidth=1pt](-0.2,0)(0.2,0)}%
  \rput(L5){\psline[linewidth=1pt](-0.2,0)(0.2,0)}%
  \rput(L4){\psline[linewidth=1pt](-0.2,0)(0.2,0)}%
  \rput(L3){\psline[linewidth=1pt](-0.2,0)(0.2,0)}%
  \rput(L2){\psline[linewidth=1pt](-0.2,0)(0.2,0)}%
  %\rput(L1){\psline[linewidth=1pt](-0.1,0)(0.1,0)}%
  %
  %\uput[0](L7){$3$}%
  \uput[180](L6){$2$}%
  \uput[180](L5){$1$}%
  \uput[180](L4){$0$}%
  \uput[180](L3){$-1$}%
  \uput[180](L2){$-2$}%
  %\uput[0](L1){$-3$}%
\end{pspicture}
}%%
  &\includegraphics{sto/graphics/Cplane.pdf}%  \psset{unit=3mm}%============================================================================
% Daniel J. Greenhoe
% LaTeX file
% ordered metric space
% real line (R, |.|, <=)
%============================================================================
{%\psset{unit=0.5\psunit}%
\begin{pspicture}(-4,-3)(4.5,3)%
  %---------------------------------
  % options
  %---------------------------------
  \psset{%
    %radius=1.25ex,
    %labelsep=2.5mm,
    linecolor=blue,%
    }%
    \psline{<->}(0,-3)(0,3)%
    \pnode(0, 3){I7}%
    \pnode(0, 2){I6}%
    \pnode(0, 1){I5}%
    \pnode(0, 0){I4}%
    \pnode(0,-1){I3}%
    \pnode(0,-2){I2}%
    \pnode(0,-3){I1}%
    %
    \psline{<->}(-3,0)(3,0)%
    \pnode( 3,0){R7}%
    \pnode( 2,0){R6}%
    \pnode( 1,0){R5}%
    \pnode( 0,0){R4}%
    \pnode(-1,0){R3}%
    \pnode(-2,0){R2}%
    \pnode(-3,0){R1}%
   %\Cnode[fillstyle=solid,linecolor=snode,fillcolor=snode,radius=0.5ex](0,0){LC}%
  %\ncline{L5}{L6}%
  %\ncline{L4}{L5}%
  %\ncline{L3}{L4}%
  %\ncline{L2}{L3}%
  %\ncline{L1}{L2}%
  %\rput(R7){\psline[linewidth=1pt](0,-0.1)(0,0.1)}%
  \rput(R6){\psline[linewidth=1pt](0,-0.1)(0,0.1)}%
  \rput(R5){\psline[linewidth=1pt](0,-0.1)(0,0.1)}%
  %\rput(R4){\psline[linewidth=1pt](0,-0.1)(0,0.1)}%
  \rput(R3){\psline[linewidth=1pt](0,-0.1)(0,0.1)}%
  \rput(R2){\psline[linewidth=1pt](0,-0.1)(0,0.1)}%
  %\rput(R1){\psline[linewidth=1pt](0,-0.1)(0,0.1)}%
  %
  %\rput(I7){\psline[linewidth=1pt](-0.1,0)(0.1,0)}%
  \rput(I6){\psline[linewidth=1pt](-0.1,0)(0.1,0)}%
  \rput(I5){\psline[linewidth=1pt](-0.1,0)(0.1,0)}%
  %\rput(I4){\psline[linewidth=1pt](-0.1,0)(0.1,0)}%
  \rput(I3){\psline[linewidth=1pt](-0.1,0)(0.1,0)}%
  \rput(I2){\psline[linewidth=1pt](-0.1,0)(0.1,0)}%
  %\rput(I1){\psline[linewidth=1pt](-0.1,0)(0.1,0)}%
  %
  %\uput[-90](R7){$3$}%
  %\uput[-90](R6){$2$}%
  %\uput[-90](R5){$1$}%
  %\uput[-90](R4){$0$}%
  %\uput[-90](R3){$-1$}%
  %\uput[-90](R2){$-2$}%
  %\uput[-90](R1){$-3$}%
  %
  %\uput[180](I7){$3$}%
  %\uput[180](I6){$2$}%
  %\uput[180](I5){$1$}%
  %\uput[180](I4){$0$}%
  %\uput[180](I3){$-1$}%
  %\uput[180](I2){$-2$}%
  %\uput[180](I1){$-3$}%
  %
  \uput[0](0,2.5){$\Im$}%
  \uput[-90](2.5,0){$\Re$}%
\end{pspicture}
}%%  
  \\{\scs(A)} \structe{integer line} \xrefr{ex:Zline}
   &{\scs(B)} \structe{real line} \xrefr{ex:Rline}
   &{\scs(C)} \structe{complex plane} \xrefr{ex:Cplane}
 \\ linear ordered metric space
   &linear ordered metric space
   &ordered/unordered metric space
  \\
    %%============================================================================
% Daniel J. Greenhoe
% LaTeX file
% spinner 6 mapping to linearly ordered L6
%============================================================================
{%\psset{unit=0.5\psunit}%
\begin{pspicture}(-1.5,-1.5)(1.5,1.5)%
  %---------------------------------
  % options
  %---------------------------------
  \psset{%
    linecolor=blue,%
    radius=1.25ex,
    labelsep=2.5mm,
    }%
  %---------------------------------
  % spinner graph
  %---------------------------------
  \rput(0,0){%\psset{unit=2\psunit}%
    \Cnode(-0.8660,-0.5){D6}%
    \Cnode(-0.8660,0.5){D5}%
    \Cnode(0,1){D4}%
    \Cnode(0.8660,0.5){D3}%
    \Cnode(0.8660,-0.5){D2}%
    \Cnode(0,-1){D1}%
    }
  \rput[-150](D6){$5$}%
  \rput[ 150](D5){$4$}%
  \rput[  90](D4){$3$}%
  \rput[  30](D3){$2$}%
  \rput[   0](D2){$1$}%
  \rput[ -90](D1){$0$}%
  %
  \ncline{D6}{D1}%
  \ncline{D5}{D6}%
  \ncline{D4}{D5}%
  \ncline{D3}{D4}%
  \ncline{D2}{D3}%
  \ncline{D1}{D2}%
  %
  %\uput[ 210](D6){$\frac{1}{6}$}
  %\uput[ 150](D5){$\frac{1}{6}$}
  %\uput[  22](D4){$\frac{1}{6}$}
  %\uput[  30](D3){$\frac{1}{6}$}
  %\uput[ -30](D2){$\frac{1}{6}$}
  %\uput[ -22](D1){$\frac{1}{6}$}
  %
  %\uput[ 210](D6){${\scy\psp(\circSix)=}\frac{1}{6}$}
  %\uput[ 150](D5){${\scy\psp(\circFive)=}\frac{1}{6}$}
  %\uput[  22](D4){${\scy\psp(\circFour)=}\frac{1}{6}$}
  %\uput[  30](D3){${\scy\psp(\circThree)=}\frac{1}{6}$}
  %\uput[ -30](D2){${\scy\psp(\circTwo)=}\frac{1}{6}$}
  %\uput[-22](D1){${\scy\psp(\circOne)=}\frac{1}{6}$}
\end{pspicture}
}%%
    \includegraphics{sto/graphics/ring6.pdf}%
   %&%============================================================================
% Daniel J. Greenhoe
% LaTeX file
% discrete metric real dice mapping to linearly ordered L6
%============================================================================
\begin{pspicture}(-1.4,-1.4)(1.4,1.4)%
  %---------------------------------
  % options
  %---------------------------------
  \psset{%
    linecolor=blue,%
    radius=1.25ex,
    labelsep=2.5mm,
    }%
  %---------------------------------
  % dice graph
  %---------------------------------
  \rput(0,0){%\psset{unit=2\psunit}%
    \Cnode(-0.8660,-0.5){D4}%
    \Cnode(-0.8660,0.5){D5}%
    \Cnode(0,1){D6}%
    \Cnode(0.8660,0.5){D3}%
    \Cnode(0.8660,-0.5){D2}%
    \Cnode(0,-1){D1}%
    }
  \rput(D6){$5$}%
  \rput(D5){$4$}%
  \rput(D4){$3$}%
  \rput(D3){$2$}%
  \rput(D2){$1$}%
  \rput(D1){$0$}%
  %
  \ncline{D5}{D6}%
  \ncline{D4}{D5}\ncline{D4}{D6}%
  \ncline{D3}{D5}\ncline{D3}{D6}%
  \ncline{D2}{D3}\ncline{D2}{D4}\ncline{D2}{D6}%
  \ncline{D1}{D2}\ncline{D1}{D3}\ncline{D1}{D4}\ncline{D1}{D5}%
  \ncline{D3}{D4}%
  \ncline{D2}{D5}%
  \ncline{D1}{D6}%
  %
  %\uput[ 158](D6){$\frac{1}{6}$}
  %\uput[ 150](D5){$\frac{1}{6}$}
  %\uput[ 210](D4){$\frac{1}{6}$}
  %\uput[  22](D3){$\frac{1}{6}$}
  %\uput[ -45](D2){$\frac{1}{6}$}
  %\uput[-158](D1){$\frac{1}{6}$}
\end{pspicture}%
   &\includegraphics{sto/graphics/discretemetric6.pdf}%
   %&%============================================================================
% Daniel J. Greenhoe
% LaTeX file
% Boolean L_2^3 for integer divides relation
%============================================================================
\begin{pspicture}(-1.5,-1.5)(1.5,1.5)%
  %---------------------------------
  % options
  %---------------------------------
  \psset{%
    linecolor=blue,%
    radius=1.25ex,
    labelsep=2.5mm,
    }%
  %---------------------------------
  % spinner graph
  %---------------------------------
  \rput(0,0){%\psset{unit=2\psunit}%
    \Cnode(0,1){D30}%
    \Cnode(0.8660,0.5){D15}%
    \Cnode(0,0.333){D10}%
    \Cnode(-0.8660,0.5){D6}%
    \Cnode(0.8660,-0.5){D5}%
    \Cnode(0,-0.333){D3}%
    \Cnode(-0.8660,-0.5){D2}%
    \Cnode(0,-1){D1}%
    }
  \rput(D30){$30$}%
  \rput(D15){$15$}%
  \rput(D10){$10$}%
  \rput(D6){$6$}%
  \rput(D5){$5$}%
  \rput(D3){$3$}%
  \rput(D2){$2$}%
  \rput(D1){$1$}%
  %
  \ncline{D30}{D6}\ncline{D30}{D10}\ncline{D30}{D15}%
  \ncline{D5}{D10}\ncline{D5}{D15}%
  \ncline{D3}{D6}\ncline{D3}{D15}%
  \ncline{D2}{D6}\ncline{D2}{D10}%
  \ncline{D1}{D2}\ncline{D1}{D3}\ncline{D1}{D5}%
  %
  %\uput[ 210](D6){$\frac{1}{6}$}
  %\uput[ 150](D5){$\frac{1}{6}$}
  %\uput[  22](D4){$\frac{1}{6}$}
  %\uput[  30](D3){$\frac{1}{6}$}
  %\uput[ -30](D2){$\frac{1}{6}$}
  %\uput[ -22](D1){$\frac{1}{6}$}
\end{pspicture}%%
  %&%============================================================================
% Daniel J. Greenhoe
% LaTeX file
% linear congruential (LCG) pseudo-random number generator (PRNG) mappings
% x_{n+1} = (7x_n+5)mod 9
% y_{n+1} = (y_n+2)mod 5
%============================================================================
\begin{pspicture}(-1.2,-1.2)(1.2,1.2)%
  %---------------------------------
  % options
  %---------------------------------
  \psset{%
    radius=1.25ex,
    labelsep=2.5mm,
    linecolor=blue,%
    }%
  \rput{288}(0,0){\rput(1,0){\Cnode(0,0){S4}}}%
  \rput{216}(0,0){\rput(1,0){\Cnode(0,0){S2}}}%
  \rput{144}(0,0){\rput(1,0){\Cnode(0,0){S0}}}%
  \rput{ 72}(0,0){\rput(1,0){\Cnode[fillstyle=solid,fillcolor=snode](0,0){S3}}}%
  \rput{  0}(0,0){\rput(1,0){\Cnode(0,0){S1}}}%
  %
  \rput(S4){$4$}%
  \rput(S3){$3$}%
  \rput(S2){$2$}%
  \rput(S1){$1$}%
  \rput(S0){$0$}%
  %
  \ncline{->}{S4}{S1}\ncline{->}{S2}{S4}\ncline{->}{S3}{S4}%
  \ncline{->}{S2}{S4}%
  \ncline{->}{S0}{S2}%
  \ncline{->}{S3}{S0}%
  \ncline{->}{S1}{S3}%
  %
  %\uput[288](S4){$\frac{3}{9}$}
  %\uput[ 72](S3){$\frac{2}{9}$}
  %\uput[216](S2){$\frac{1}{9}$}
  %\uput[  0](S1){$\frac{1}{9}$}
  %\uput[144](S0){$\frac{2}{9}$}
  %\rput(0,0){$\ocsG$}%
\end{pspicture}%%
  &\includegraphics{sto/graphics/oms_wring5shortd.pdf}%
  \\{\scs(D)} \structe{6 element ring} \xrefr{ex:ring6}
   &{\scs(E)} \structe{6 element discrete metric} \xrefr{ex:discretemetric6}
  %&{\scs(F)} \structe{integer divides order relation} \xref{ex:ocs_532}
   &{\scs(F)} \structe{directed PRNG state machine} \xrefr{ex:ocs_prng}
 \\ unordered metric space
   &unordered metric space
   &unordered quasi-metric space
\end{tabular}}
\caption{Examples of \structe{ordered quasi-metric space}s \xref{def:oqms}
         %with \colorbox{snode}{shaded} \structe{center}s \xref{def:gcen} 
         \label{fig:oms}}   %{def:ocscen}
\end{figure}

%---------------------------------------
\begin{remark}
\label{rem:realtop}
%---------------------------------------
This text makes extensive reference to the \structe{real line} (next definition).
% and the \structe{integer line} (next two definitions).
There are several ways to define the real line.
In particular, there are many possible ordering relations on $\R$
and several possible topologies on $\R$.\footnote{
  \citerpgc{adams2008}{31}{0131848690}{"six topologies on the real line"},
  \citerppgc{salzmann2007}{64}{70}{0521865166}{Weird topologies on the real line},
  \citerpgc{murdeshwar1990}{53}{8122402461}{``often used topologies on the real line"},
  \citerppgc{joshi1983}{85}{91}{0852264445}{\textsection4.2 Examples of Topological Spaces}
  }
In fact, order and topology are closely related in that 
an order relation $\orel$ \xref{def:orel} on a set always induces a topology 
(called the \structe{order topology} / \structe{interval topology}\footnote{
  \citerpgc{salzmann2007}{23}{0521865166}{3.2 Topology induced by an ordering}
  \citerpgc{willard1970}{43}{0486434796}{6D. Ordered Spaces},
  \citerpgc{steen1978}{66}{0486319296}{39. Order Topology}
  }); 
and in the case of the real line, a topology induces an order structure up to the order relation's 
\rele{dual} \xref{def:oreld}.\footnote{
  \citerpcu{hocking1961}{52}{2--5 The interval and the circle}{https://archive.org/details/Topology_972},
  \citerppgc{salzmann2007}{69}{70}{0521865166}{5.75 Note: Ordering and topology on $\R$, see also 5.10 Theorem page 36}
  }
This text uses a fairly standard structure, as defined next.
% using standard linear ordering and the \fncte{usual metric}.
%The \structe{real line}, as used in this text, is defined next and illustrated in \prefp{fig:oms} (B).
\end{remark}

%---------------------------------------
%\begin{minipage}{\tw-13mm}
\begin{definition}
\label{def:Rline}
\label{def:rline}
%---------------------------------------
The triple $\omsR$ is the \structd{real line} 
%if $\omsR$ is an \structe{ordered metric space} \xref{def:oms},
if $\R$ is the \structe{set of real numbers} \xref{def:R}, 
$\metric{x}{y}\eqd\abs{x-y}$ is the \fncte{usual metric} on $\R$ \xref{def:d_usual},
and $\orel$ is the standard \structe{linear order relation} \xref{def:chain} on $\R$.
\end{definition}

%---------------------------------------
%\begin{minipage}{\tw-13mm}
\begin{definition}
\label{def:Zline}
\label{def:zline}
%---------------------------------------
The triple $\omsZ$ is the \structd{integer line} 
if $\Z$ is the \structe{set of integers} \xref{def:Z},
$\metric{m}{n}\eqd\abs{m-n}$ is the \fncte{usual metric} \xref{def:d_usual} on $\R$ restricted to $\Z$,
and $\orel$ is the standard \structe{linear order relation} on $\Z$
as induced by \structe{Peano's Axioms}.\footnote{
  \citerpg{landau1966}{2}{082182693X},
  \citerpg{halmos1960}{46}{0387900926},
  \citerpg{thurston1956}{51}{0486458067},
  \citer{peano1889},
  \citerpg{peano1889e}{94}{158348597X},
  \citeP{dedekind1888},
  \citePp{dedekind1888e}{67},
  \citerppgc{cori2001}{8}{15}{0198500513}{recursion theory}
  }
\end{definition}
%\end{minipage}\hfill%
%\begin{minipage}{10mm}
%  \gsize%
%  \centering%
%  \psset{unit=5mm}%
%  %============================================================================
% Daniel J. Greenhoe
% LaTeX file
% ordered metric space
% real line (R, |.|, <=)
%============================================================================
{%\psset{unit=0.5\psunit}%
\begin{pspicture}(-0.5,-3)(0.5,3)%
  %---------------------------------
  % options
  %---------------------------------
  \psset{%
    %radius=1.25ex,
    %labelsep=2.5mm,
    linecolor=blue,%
    }%
    %\psline{<->}(0,-4)(0,4)%
    %\pnode(0,4){L8}%
    \pnode(0,3){L7}%
    \Cnode(0,2){L6}%
    \Cnode(0,1){L5}%
    %\Cnode[fillstyle=solid,fillcolor=snode](0,0){L4}%
    \Cnode(0,0){L4}%
    \Cnode(0,-1){L3}%
    \Cnode(0,-2){L2}%
    \pnode(0,-3){L1}%
    %\pnode(0,-4){L0}%
   %\Cnode[fillstyle=solid,linecolor=snode,fillcolor=snode,radius=0.5ex](0,0){LC}%
  %\ncline[linestyle=dotted]{L7}{L8}%
  \ncline[linestyle=dotted]{L6}{L7}%
  \ncline{L5}{L6}%
  \ncline{L4}{L5}%
  \ncline{L3}{L4}%
  \ncline{L2}{L3}%
  \ncline[linestyle=dotted]{L1}{L2}%
  %\ncline[linestyle=dotted]{L1}{L0}%
  %\rput(L7){\psline[linewidth=1pt](-0.1,0)(0.1,0)}%
  %\rput(L6){\psline[linewidth=1pt](-0.1,0)(0.1,0)}%
  %\rput(L5){\psline[linewidth=1pt](-0.1,0)(0.1,0)}%
  %\rput(L4){\psline[linewidth=1pt](-0.1,0)(0.1,0)}%
  %\rput(L3){\psline[linewidth=1pt](-0.1,0)(0.1,0)}%
  %\rput(L2){\psline[linewidth=1pt](-0.1,0)(0.1,0)}%
  %\rput(L1){\psline[linewidth=1pt](-0.1,0)(0.1,0)}%
  %
  %\rput(L7){$3$}%
  \uput[180](L6){$2$}%
  \uput[180](L5){$1$}%
  \uput[180](L4){$0$}%
  \uput[180](L3){$-1$}%
  \uput[180](L2){$-2$}%
  %\rput(L1){$-3$}%
\end{pspicture}
}%%
%\end{minipage}


%=======================================
\subsection{Examples}
%=======================================
%---------------------------------------
\begin{example}
\label{ex:Zline}
%---------------------------------------
The \structe{integer line} \xref{def:Zline} is an \structe{ordered metric space} \xref{def:oms},
and is illustrated in \prefp{fig:oms} (A).
\end{example}

%---------------------------------------
\begin{example}
\label{ex:Rline}
%---------------------------------------
The \structe{real line} \xref{def:Rline} is an \structe{ordered metric space} \xref{def:oms},
and is illustrated in \prefp{fig:oms} (B).
\end{example}

%---------------------------------------
%\begin{minipage}{\tw-48mm}
\begin{example}
\label{ex:Cplane}
%---------------------------------------
The \structd{complex plane} $\otriple{\C}{\absn}{\orel}$ is an \structe{ordered metric space} \xref{def:oms}
where $\C\eqd\R^2$ is the \structe{set of complex numbers}, 
$\metric{x}{y}\eqd\abs{x-y}\eqd\sqrt{\Re x-\Re y)^2+(\Im x-\Im y)^2}$,
$\Re x\eqd\Re\opair{a}{b}\eqd a$ ${\scy\forall\opair{a}{b}\in\C}$ ($\Re x$ is the \vald{real part} of $x$),
$\Im x\eqd\Im\opair{a}{b}\eqd b$ ${\scy\forall\opair{a}{b}\in\C}$ ($\Im x$ is the \vald{imaginary part} of $x$),
and $\orel$ is any \rele{order relation} defined on $\C$.
Possible order relations include the \rele{coordinatewise order relation} \xref{ex:order_coordinatewise},
the \rele{lexicographical order relation} \xref{ex:order_lex},
and $\orel=\emptyset$ (in which case the \structe{complex plane} is \prope{unordered}).
The \structe{complex plane} is illustrated in \prefp{fig:oms} (C).
\end{example}
%\end{minipage}\hfill%
%\begin{minipage}{43mm}
%  \gsize%
%  \centering%
%  \psset{unit=3mm}%
%  %============================================================================
% Daniel J. Greenhoe
% LaTeX file
% ordered metric space
% real line (R, |.|, <=)
%============================================================================
{%\psset{unit=0.5\psunit}%
\begin{pspicture}(-4,-3)(4.5,3)%
  %---------------------------------
  % options
  %---------------------------------
  \psset{%
    %radius=1.25ex,
    %labelsep=2.5mm,
    linecolor=blue,%
    }%
    \psline{<->}(0,-3)(0,3)%
    \pnode(0, 3){I7}%
    \pnode(0, 2){I6}%
    \pnode(0, 1){I5}%
    \pnode(0, 0){I4}%
    \pnode(0,-1){I3}%
    \pnode(0,-2){I2}%
    \pnode(0,-3){I1}%
    %
    \psline{<->}(-3,0)(3,0)%
    \pnode( 3,0){R7}%
    \pnode( 2,0){R6}%
    \pnode( 1,0){R5}%
    \pnode( 0,0){R4}%
    \pnode(-1,0){R3}%
    \pnode(-2,0){R2}%
    \pnode(-3,0){R1}%
   %\Cnode[fillstyle=solid,linecolor=snode,fillcolor=snode,radius=0.5ex](0,0){LC}%
  %\ncline{L5}{L6}%
  %\ncline{L4}{L5}%
  %\ncline{L3}{L4}%
  %\ncline{L2}{L3}%
  %\ncline{L1}{L2}%
  %\rput(R7){\psline[linewidth=1pt](0,-0.1)(0,0.1)}%
  \rput(R6){\psline[linewidth=1pt](0,-0.1)(0,0.1)}%
  \rput(R5){\psline[linewidth=1pt](0,-0.1)(0,0.1)}%
  %\rput(R4){\psline[linewidth=1pt](0,-0.1)(0,0.1)}%
  \rput(R3){\psline[linewidth=1pt](0,-0.1)(0,0.1)}%
  \rput(R2){\psline[linewidth=1pt](0,-0.1)(0,0.1)}%
  %\rput(R1){\psline[linewidth=1pt](0,-0.1)(0,0.1)}%
  %
  %\rput(I7){\psline[linewidth=1pt](-0.1,0)(0.1,0)}%
  \rput(I6){\psline[linewidth=1pt](-0.1,0)(0.1,0)}%
  \rput(I5){\psline[linewidth=1pt](-0.1,0)(0.1,0)}%
  %\rput(I4){\psline[linewidth=1pt](-0.1,0)(0.1,0)}%
  \rput(I3){\psline[linewidth=1pt](-0.1,0)(0.1,0)}%
  \rput(I2){\psline[linewidth=1pt](-0.1,0)(0.1,0)}%
  %\rput(I1){\psline[linewidth=1pt](-0.1,0)(0.1,0)}%
  %
  %\uput[-90](R7){$3$}%
  %\uput[-90](R6){$2$}%
  %\uput[-90](R5){$1$}%
  %\uput[-90](R4){$0$}%
  %\uput[-90](R3){$-1$}%
  %\uput[-90](R2){$-2$}%
  %\uput[-90](R1){$-3$}%
  %
  %\uput[180](I7){$3$}%
  %\uput[180](I6){$2$}%
  %\uput[180](I5){$1$}%
  %\uput[180](I4){$0$}%
  %\uput[180](I3){$-1$}%
  %\uput[180](I2){$-2$}%
  %\uput[180](I1){$-3$}%
  %
  \uput[0](0,2.5){$\Im$}%
  \uput[-90](2.5,0){$\Re$}%
\end{pspicture}
}%%
%\end{minipage}

%---------------------------------------
%\begin{minipage}{\tw-48mm}
\begin{example}
\label{ex:ring6}
%---------------------------------------
A \structd{6 element ring} $\otriple{\setn{0,1,2,3,4,5}}{\metricn}{\emptyset}$ is an \structe{unordered metric space} \xref{def:oms}
where the metric $\metricn$ is defined on a ring as illustrated in \prefp{fig:oms} (D), with each line segment representing a 
distance of 1.
\end{example}
%\end{minipage}\hfill%
%\begin{minipage}{43mm}
%  \gsize%
%  \centering%
%  \psset{unit=7.5mm}%
%  %============================================================================
% Daniel J. Greenhoe
% LaTeX file
% spinner 6 mapping to linearly ordered L6
%============================================================================
{%\psset{unit=0.5\psunit}%
\begin{pspicture}(-1.5,-1.5)(1.5,1.5)%
  %---------------------------------
  % options
  %---------------------------------
  \psset{%
    linecolor=blue,%
    radius=1.25ex,
    labelsep=2.5mm,
    }%
  %---------------------------------
  % spinner graph
  %---------------------------------
  \rput(0,0){%\psset{unit=2\psunit}%
    \Cnode(-0.8660,-0.5){D6}%
    \Cnode(-0.8660,0.5){D5}%
    \Cnode(0,1){D4}%
    \Cnode(0.8660,0.5){D3}%
    \Cnode(0.8660,-0.5){D2}%
    \Cnode(0,-1){D1}%
    }
  \rput[-150](D6){$5$}%
  \rput[ 150](D5){$4$}%
  \rput[  90](D4){$3$}%
  \rput[  30](D3){$2$}%
  \rput[   0](D2){$1$}%
  \rput[ -90](D1){$0$}%
  %
  \ncline{D6}{D1}%
  \ncline{D5}{D6}%
  \ncline{D4}{D5}%
  \ncline{D3}{D4}%
  \ncline{D2}{D3}%
  \ncline{D1}{D2}%
  %
  %\uput[ 210](D6){$\frac{1}{6}$}
  %\uput[ 150](D5){$\frac{1}{6}$}
  %\uput[  22](D4){$\frac{1}{6}$}
  %\uput[  30](D3){$\frac{1}{6}$}
  %\uput[ -30](D2){$\frac{1}{6}$}
  %\uput[ -22](D1){$\frac{1}{6}$}
  %
  %\uput[ 210](D6){${\scy\psp(\circSix)=}\frac{1}{6}$}
  %\uput[ 150](D5){${\scy\psp(\circFive)=}\frac{1}{6}$}
  %\uput[  22](D4){${\scy\psp(\circFour)=}\frac{1}{6}$}
  %\uput[  30](D3){${\scy\psp(\circThree)=}\frac{1}{6}$}
  %\uput[ -30](D2){${\scy\psp(\circTwo)=}\frac{1}{6}$}
  %\uput[-22](D1){${\scy\psp(\circOne)=}\frac{1}{6}$}
\end{pspicture}
}%%
%\end{minipage}

%---------------------------------------
%\begin{minipage}{\tw-48mm}
\begin{example}
\label{ex:discretemetric6}
%---------------------------------------
A \structd{6 element discrete metric} $\otriple{\setn{0,1,2,3,4,5}}{\metricn}{\emptyset}$ is an \structe{unordered metric space} \xref{def:oms}
where the metric $\metricn$ is the \fncte{discrete metric} \xref{def:dmetric}.
This structure is illustrated in \prefp{fig:oms} (E).
\end{example}
%\end{minipage}\hfill%
%\begin{minipage}{43mm}
%  \gsize%
%  \centering%
%  \psset{unit=7.5mm}%
%  %============================================================================
% Daniel J. Greenhoe
% LaTeX file
% discrete metric real dice mapping to linearly ordered L6
%============================================================================
\begin{pspicture}(-1.4,-1.4)(1.4,1.4)%
  %---------------------------------
  % options
  %---------------------------------
  \psset{%
    linecolor=blue,%
    radius=1.25ex,
    labelsep=2.5mm,
    }%
  %---------------------------------
  % dice graph
  %---------------------------------
  \rput(0,0){%\psset{unit=2\psunit}%
    \Cnode(-0.8660,-0.5){D4}%
    \Cnode(-0.8660,0.5){D5}%
    \Cnode(0,1){D6}%
    \Cnode(0.8660,0.5){D3}%
    \Cnode(0.8660,-0.5){D2}%
    \Cnode(0,-1){D1}%
    }
  \rput(D6){$5$}%
  \rput(D5){$4$}%
  \rput(D4){$3$}%
  \rput(D3){$2$}%
  \rput(D2){$1$}%
  \rput(D1){$0$}%
  %
  \ncline{D5}{D6}%
  \ncline{D4}{D5}\ncline{D4}{D6}%
  \ncline{D3}{D5}\ncline{D3}{D6}%
  \ncline{D2}{D3}\ncline{D2}{D4}\ncline{D2}{D6}%
  \ncline{D1}{D2}\ncline{D1}{D3}\ncline{D1}{D4}\ncline{D1}{D5}%
  \ncline{D3}{D4}%
  \ncline{D2}{D5}%
  \ncline{D1}{D6}%
  %
  %\uput[ 158](D6){$\frac{1}{6}$}
  %\uput[ 150](D5){$\frac{1}{6}$}
  %\uput[ 210](D4){$\frac{1}{6}$}
  %\uput[  22](D3){$\frac{1}{6}$}
  %\uput[ -45](D2){$\frac{1}{6}$}
  %\uput[-158](D1){$\frac{1}{6}$}
\end{pspicture}%
%\end{minipage}

%\begin{minipage}{\tw-48mm}%
%%---------------------------------------
%\begin{example}%[Integer divides order relation]%
%\label{ex:ocs_532}
%%\footnotemark
%%---------------------------------------
%The triple $\otriple{\setn{1,2,3,5,6,10,15,30}}{\metricn}{|}$ is an \structe{ordered metric space} \xref{def:oms}
%where ``$|$" is the ``divides" order relation % \xref{ex:poset_532}, 
%and $\metricn$ is defined as
%\\\indentx$\metric{x}{y}\eqd\height(x\join y)-\height(x\meet y)\qquad{\scy\forall x,y\in\setn{1,2,3,5,6,10,15,30}}$\\
%where $\height$ is the \fncte{height} function. %\xxxref{def:height}{def:latmetric}{ex:l2e3_abc_h}.
%This structure is illustrated in \prefp{fig:oms} (F).
%\end{example}%
%\end{minipage}%
%%\footnotetext{
%%  \citerpg{maclane1999}{484}{0821816462},
%%  %\citerpg{menini2004}{60}{0824709853}\\
%%  %\citerp{huntington1933}{278}\\
%%  \citePpc{sheffer1920}{310}{footnote 1}
%%  }%
%\begin{minipage}{43mm}%
%\gsize%
%\centering%
%\psset{unit=7.5mm}%
%\input{../common/math/graphics/lat2235.tex}%
%\end{minipage}%

%---------------------------------------
\begin{example}
\label{ex:ocs_prng}
%---------------------------------------
\prefp{fig:oms} (F) illustrates a \structe{linear congruential pseudo-random number generator}
induced by the equation $y_{n+1}=(y_n+2)\mod5$ with $y_0=1$.
The structure is an \structe{unordered quasi-metric space}.
See \prefpp{ex:lcg7x1m9_xyz}--\prefpp{ex:lcg7x1m9_dgraph} for further demonstration.
\end{example}
