%============================================================================
% LaTeX File
% Daniel J. Greenhoe
%============================================================================
%======================================
\chapter{Quasi-symmetry Constraint}
\label{chp:symlets}
%======================================
%=======================================
%\section{Symmetry constraint}
\index{Symlets}
%=======================================
%=======================================
\section{Design details}
%=======================================
The Daubechies-$p$ wavelets \xref{chp:compactp} are in general \prope{asymmetric} because they are constructed
under a \prope{minimum phase} constraint---all zeros for polynomial $Q(z)$
are selected to be inside the unit circle.
Suppose we want to design filters such that the number of
vanishing moments is the same, but we are willing
to lift the minimum phase condition to gain more symmetry.
That is, instead of choosing the zeros from $Q(z)Q(z^{-1})$ that are
inside the unit circle only, we choose some that are inside and some that
are outside to give better balance.
However, for zeros that are complex, we still select them
in conjugate pairs so that the coefficients of $Q(z)$ are real.


%---------------------------------------
\begin{definition}
\label{def:Sp}
\index{Fej\'er-Riesz spectral factorization}
%---------------------------------------
Let $Q(z)$ be a polynomial with real coefficients and
\begin{align*}
  P(y)
  &\eqd \sum\limits_{k=0}^{p-1} {{p-1+k}\choose{k}} y^k
  &&    \text{}
  \\
  \dot{Q}
  &\eqd \set{Q(z)}{ Q(z)Q(z^{-1})=P\left( \frac{2-z-z^{-1}}{4}} \right)
  &&    \text{(Fej\'er-Riesz spectral factorizations of $P$)}
  \\
  \opR Q
  &\eqd \seq{r_n}{\text{$r_n$ is a root of $Q(z)$}}
  &&    \text{(roots of $Q(z)$)}
  \\
  [\opA Q](\omega)
  &\eqd \atan\left(\frac{\Imb{Q(e^{i\omega})}}{\Reb{Q(e^{i\omega})}}\right)
  &&    \text{(phase of $Q(e^{i\omega})$)}
  \\
  \opE Q
  &\eqd \inf_{m,c\in\R}\norm{[\opA Q](\omega)-[m\omega+c]}^2
  &&    \text{(linear phase estimation error of $Q(z)$)}
  \\
  \opM Q
  &\eqd \sum_{r_n\in\opR Q} |r_n|^2
  &&    \text{(magnitude of roots of $Q(z)$)}
  \\
  \dot{E}
  &\eqd \set{\varepsilon=\opE Q}{Q\in\dot{Q}}
  &&    \text{(set of all phase errors)}
  \\
  \dot{M}
  &\eqd \set{m=\opM Q}{Q\in\dot{Q}}
  &&    \text{(set of all root magnitudes)}
\end{align*}

Then the {\bf Symlet-$p$ scaling function} is the
sequence $\seq{h_n}{n\in\Z}$ with $z$-transform
$\Zh(z) = \sum_n h_n z^{-n}$ that satisfies
\defbox{\begin{array}{lrcl}
  1. & \Zh(z) &=& \ds \left(\frac{1+z^{-1}}{2}\right)^p Q(z)
  \\
  2. & Q(z)   &=& \ds\arg\min_{Q(z)\in\dot{Q}}
                  \set{\opM Q\in\dot{M}}
                      {\opE Q=\min\dot{E}}
\end{array}}
\end{definition}


\setlength{\unitlength}{0.10mm}
The Symlets of \prefp{def:Sp} can be implemented
\footnote{
  For an actual implementation using \hie{Octave},
  see \prefpp{sec:src_Ry}.
  }
by the following steps: \cittrpp{dau}{254}{257}
\begin{enumerate}
  \item Compute the polynomial $P(y)$. This polynomial has $p-1$ roots in $y$.
        \[
          P(y) \eqd \sum\limits_{k=0}^{p-1} {{p-1+k}\choose{k}} y^k
         \hspace{3em}\mbox{where }
         {{n}\choose{k}}\eqd\frac{n!}{k!(n-k)!}
        \]

  \item Compute $P\left(\frac{2-z-z^{-1}}{4}\right)$.
        This polynomial has $2p-2$ roots in $z$.
        {\center %============================================================================
% LaTeX File
% Daniel J. Greenhoe
%
% Daubechies-8 Pole Zero plot
%============================================================================
\begin{picture}(500,470)(-130,-200)
  \thicklines%
  \color{axis}%
    \put(-130,   0){\line(1,0){500} }%
    \put(   0,-130){\line(0,1){260} }%
    \put( 380,   0){\makebox(0,0)[l]{$\Reb{z}$}}%
    \put(   0, 140){\makebox(0,0)[b]{$\Imb{z}$}}%
  \color{circle}%
    \qbezier( 100,   0)( 100, 41.421356)(+70.710678,+70.710678)% % 0   -->1pi/4
    \qbezier(   0, 100)( 41.421356, 100)(+70.710678,+70.710678)% % pi/4-->2pi/4
    \qbezier(   0, 100)(-41.421356, 100)(-70.710678,+70.710678)% %2pi/4-->3pi/4
    \qbezier(-100,   0)(-100, 41.421356)(-70.710678,+70.710678)% %3pi/4--> pi
    \qbezier(-100,   0)(-100,-41.421356)(-70.710678,-70.710678)% % pi  -->5pi/4
    \qbezier(   0,-100)(-41.421356,-100)(-70.710678,-70.710678)% %5pi/4-->6pi/4
    \qbezier(   0,-100)( 41.421356,-100)( 70.710678,-70.710678)% %6pi/4-->7pi/4
    \qbezier( 100,   0)( 100,-41.421356)( 70.710678,-70.710678)% %7pi/4-->2pi
    %\put( 110, 110){\makebox(0,0)[lb]{$z=e^{i\omega}$}}%
    %\put( 105, 105){\vector(-1,-1){33}}%
  \color{pzasym}%
    \qbezier(41.4214,0)(41.4214, 58.5786)(0, 100)%      % inner upper arc
    \qbezier(41.4214,0)(41.4214,-58.5786)(0,-100)%      % inner lower arc
    \qbezier(0, 100)(41.421, 141.421)(100, 141.421)%    %outer upper arc
    \qbezier(100, 141.421)(158.579,141.421)(200,100)%
    \qbezier(200,100)(241.421,58.5786)(241.421,0)%
    \qbezier(0,-100)(41.421,-141.421)(100,-141.421)%outer lower arc
    \qbezier(100,-141.421)(158.579,-141.421)(200,-100)%
    \qbezier(200,-100)(241.421,-58.5786)(241.421,0)%
  \color{pole}%
   %\put(-100,-100){\makebox(0,0)[tr]{15 poles}}%
   %\put(-100,-100){\vector(1,1){91}}%
   %\put(   0,    0){\makebox(0,0)[c]{$\times$}}%
   %\put(   0,    0){\makebox(0,0)[c]{\hspace{1em}$^{15}$}}%
  \color{zero}%
   %\put(-150, -50){\makebox(0,0)[tr]{p=8 zeros}}%
   %\put(-150, -50){\vector( 1, 1){43}}%
   %\put(-100,    0){\circle{15}$^8$}%
    \qbezier[30](0,0)(51.90, 86.52)(+103.79714430713773 ,  +173.04352168829922 )%
    \qbezier[30](0,0)(51.90,-86.52)(+103.79714430713773 ,  -173.04352168829922 )%
    \qbezier[30](0,0)(96.94, 72.79)(+193.88494686323963 ,  +145.58242201373777 )%
    \qbezier[30](0,0)(96.94,-72.79)(+193.88494686323963 ,  -145.58242201373777 )%
    \qbezier[30](0,0)(126.48, 40.99)(+252.96496127413031 , + 81.97646490830808 )%
    \qbezier[30](0,0)(126.48,-40.99)(+252.96496127413031 , - 81.97646490830808 )%
    \put( 0,- 80){\dashbox{5}(60,160){}}%
    \put(-10, 30){\makebox(0,0)[rb]{$p-1$ zeros $\rightarrow$ }}%
    \put(+36.54035130742002 ,  0                 ) {\circle{15} }%
    \put(+35.77427639711839 , +11.59310245530727 ) {\circle{15} }%
    \put(+35.77427639711839 , -11.59310245530727 ) {\circle{15} }%
    \put(+32.98169959381100 , +24.76497421117944 ) {\circle{15} }%
    \put(+32.98169959381100 , -24.76497421117944 ) {\circle{15} }%
    \put(+25.49176775413241 , +42.49813706999120 ) {\circle{15} }%
    \put(+25.49176775413241 , -42.49813706999120 ) {\circle{15} }%
  \color{red}%
    \put(80,-200){\dashbox{5}(230,400){}}%
    \put(340, 30){\makebox(0,0)[lb]{$\leftarrow$ $p-1$ zeros}}%
    \put(+273.67005631344137 ,  0                  ){ \circle{15} }%
    \put(+103.79714430713773 , +173.04352168829922 ){ \circle{15} }%
    \put(+103.79714430713773 , -173.04352168829922 ){ \circle{15} }%
    \put(+193.88494686323963 , +145.58242201373777 ){ \circle{15} }%
    \put(+193.88494686323963 , -145.58242201373777 ){ \circle{15} }%
    \put(+252.96496127413031 , +081.97646490830808 ){ \circle{15} }%
    \put(+252.96496127413031 , -081.97646490830808 ){ \circle{15} }%
\end{picture}

 }

  \begin{enumerate}
    \item Because the coefficients of
          $P\left(\frac{2-z-z^{-1}}{4}\right)$
          are real, all of its roots occur in \hie{complex conjugate pairs}:
          \[ \text{root at } z_1=re^{i\theta}
             \iff
             \text{root at }
             z_1^\ast = \left(re^{i\theta}\right)^\ast
                      = re^{-i\theta}
          \]
          \begin{center}
          \scriptsize
          \setlength{\unitlength}{0.15mm}
          \begin{picture}(300,300)(-130,-130)
            \thicklines
            \color{axis}%  
              \put(-130,   0){\line(1,0){260} }%
              \put(   0,-130){\line(0,1){260} }%
              \put( 140,   0){\makebox(0,0)[l]{$\Reb{z}$}}%
              \put(   0, 140){\makebox(0,0)[b]{$\Imb{z}$}}%
              \qbezier[30](0,0)(62.5, 62.5)(125, 125)%
              \qbezier[30](0,0)(62.5,-62.5)(125,-125)%
            \color{circle}%
              %============================================================================
% NCTU - Hsinchu, Taiwan
% LaTeX File
% Daniel Greenhoe
%
% Unit circle with radius 100
%============================================================================

\qbezier( 100,   0)( 100, 41.421356)(+70.710678,+70.710678) % 0   -->1pi/4
\qbezier(   0, 100)( 41.421356, 100)(+70.710678,+70.710678) % pi/4-->2pi/4
\qbezier(   0, 100)(-41.421356, 100)(-70.710678,+70.710678) %2pi/4-->3pi/4
\qbezier(-100,   0)(-100, 41.421356)(-70.710678,+70.710678) %3pi/4--> pi 
\qbezier(-100,   0)(-100,-41.421356)(-70.710678,-70.710678) % pi  -->5pi/4
\qbezier(   0,-100)(-41.421356,-100)(-70.710678,-70.710678) %5pi/4-->6pi/4
\qbezier(   0,-100)( 41.421356,-100)( 70.710678,-70.710678) %6pi/4-->7pi/4
\qbezier( 100,   0)( 100,-41.421356)( 70.710678,-70.710678) %7pi/4-->2pi


%
              \put( 120, 120){\makebox(0,0)[lb]{$z=e^{i\theta}$}}%
              \put( 115, 115){\vector(-1,-1){43}}%
            \color{zero}%
              \put(  56, -56){\circle{10}}%
              \put(  56,  56){\circle{10}}%
            \normalcolor
              \put(  60,  60){\makebox(0,0)[bl]{$re^{i\phi}$}}%
              \put(  60, -60){\makebox(0,0)[tl]{$re^{-i\phi}$}}%
              \put(  25, - 2){\makebox(0,0)[tl]{$\phi$}}%
          \end{picture}%
          \end{center}

    \item All of the roots of $P\left(\frac{2-z-z^{-1}}{4}\right)$
          occur in \hie{conjugate reciprocal pairs}:
          \begin{multline*}
             P\left(\frac{2-z-z^{-1}}{4}\right)
             \text{ has a root at } z=re^{i\theta}
             \qquad \iff \\
             P\left(\frac{2-z-z^{-1}}{4}\right)
             \text{ has a root at } z=\left(re^{i\theta}\right)^{-1}
                      = \frac{1}{r}e^{-i\theta}
          \end{multline*}
      \begin{center}
      \scriptsize
      \setlength{\unitlength}{0.15mm}
      \begin{picture}(350,350)(-175,-175)
        \thicklines
        \color{axis}%  
          \put(-130,   0){\line(1,0){260} }%
          \put(   0,-130){\line(0,1){260} }%
          \put( 140,   0){\makebox(0,0)[l]{$\Reb{z}$}}%
          \put(   0, 140){\makebox(0,0)[b]{$\Imb{z}$}}%
          \qbezier[30](0,0)(62.5, 62.5)(125, 125)%
          \qbezier[30](0,0)(62.5,-62.5)(125,-125)%
        \color{circle}%
          %============================================================================
% NCTU - Hsinchu, Taiwan
% LaTeX File
% Daniel Greenhoe
%
% Unit circle with radius 100
%============================================================================

\qbezier( 100,   0)( 100, 41.421356)(+70.710678,+70.710678) % 0   -->1pi/4
\qbezier(   0, 100)( 41.421356, 100)(+70.710678,+70.710678) % pi/4-->2pi/4
\qbezier(   0, 100)(-41.421356, 100)(-70.710678,+70.710678) %2pi/4-->3pi/4
\qbezier(-100,   0)(-100, 41.421356)(-70.710678,+70.710678) %3pi/4--> pi 
\qbezier(-100,   0)(-100,-41.421356)(-70.710678,-70.710678) % pi  -->5pi/4
\qbezier(   0,-100)(-41.421356,-100)(-70.710678,-70.710678) %5pi/4-->6pi/4
\qbezier(   0,-100)( 41.421356,-100)( 70.710678,-70.710678) %6pi/4-->7pi/4
\qbezier( 100,   0)( 100,-41.421356)( 70.710678,-70.710678) %7pi/4-->2pi


%
          \put( 120, 120){\makebox(0,0)[lb]{$z=e^{i\theta}$}}%
          \put( 115, 115){\vector(-1,-1){43}}%
        \color{zero}%
          \put(  56,  56){\circle{10}}%
          \put( 125,-125){\circle{10}}%
        \normalcolor
          \put(  60,  60){\makebox(0,0)[bl]{$re^{i\phi}$}}%
          \put( 130,-125){\makebox(0,0)[tl]{$\frac{1}{r}e^{-i\phi}$}}%
          \put(  25, - 2){\makebox(0,0)[tl]{$\phi$}}%
      \end{picture}%
      \end{center}

  \end{enumerate}

  \item By the \thme{Fej\'er-Riesz spectral factorization} theorem \xref{thm:Fejer-Riesz}
        $P\brp{\frac{2-z-z^{-1}}{4}}$ can be factored in the form
        \[ P\left(\frac{2-z-z^{-1}}{4}\right) = Q(z)Q(z^{-1}). \]
        Find all such factors of $P\left(\frac{2-z-z^{-1}}{4}\right)$
        under the following constraints:
        \begin{enumerate}
          \item The selected $p-1$ roots for $Q(z)$ occur in complex conjugate pairs
              (so $Q(z)$ will have real coefficients).
          \item $Q(z)$ contains exactly one root from each conjugate reciprocal pair
                of $P\left(\frac{2-z-z^{-1}}{4}\right)$.
        \end{enumerate}
        The above two constraints imply that there are $\ds 2^\floor{p/2}$ choices of roots for $Q(z)$.

  \item Find the two choices that result in a phase that is the
        closest (in the least square sense) to a straight line.
        There are two minimum error choices because for each choice,
        selecting the complementary choice
        (selecting the zeros that were discarded and discarding
        the zeros that were selected)
        will result in the exact same amount of phase error.\\
        %Why this is, I cannot prove at this time.\cittrp{dau}{255} \problem

  \item Of these two choices, select the one where the sum of the magnitude
        of the $p-1$ roots is least.
        \footnote{This step is arbitrary and not required by
        Daubechies' definition of Symlets.}

  \item The scaling coefficients $\seq{h_n}{}$ are the coefficients of
        the polynomial
        \[ \Zh(z) = \mcom{\mcom{\brp{\frac{1+z^{-1}}{2}}^p}{order $p$} \mcom{Q(z)}{order $p-1$}}
                         {order $2p-1$ ($2p$ coefficients)} 
        \]
\end{enumerate}



%---------------------------------------
\begin{theorem}
%---------------------------------------
\thmbox{\parbox{\textwidth-8ex}{
  Symlet-$p$ wavelets have the following properties:
  \begin{enumerate}
     \item Symlet-$p$ wavelets have optimum linear phase
           (in the least squares sense)
           among all the Bezout polynomial factorization solutions.
     \item $\seq{h_n}{}$ has support size $2p-1$
           ($\seq{h_n}{}$) has $2p$ non-zero elements).
     \item The support size of $\seq{h_n}{}$ is the smallest possible for wavelets with $p$ vanishing moments.
  \end{enumerate}
}}
\end{theorem}
\begin{proof}
\begin{enumerate}
\item Proof that $\Fh(\omega)$ has optimum {\bf linear phase}:
      The optimization is performed exhaustively
      (all the phases are computed and linear error is measured).
      This is not a rigorous proof.  \problem

\item Proof that $\seq{h_n}{}$ has support size $2p-1$:
      See \pref{thm:cs_bezout_prop} \prefpo{thm:cs_bezout_prop}.

\item Proof that the support size of $\seq{h_n}{}$ is the smallest possible for wavelets with $p$ vanishing moments.
      See \pref{thm:cs_bezout_prop} \prefpo{thm:cs_bezout_prop}.
\end{enumerate}
\end{proof}




\begin{table}
\scriptsize
\exbox{\renewcommand{\arraystretch}{1}
\begin{array}[t]{rrr}
   p &    n &  \mc{1}{c}{h_n}                               \\
   \hline
   1 &    0 &  \frac{1}{\sqrt{2}}                           \\
     &    1 &  \frac{1}{\sqrt{2}}                           \\
                                                            \\
   2 &    0 &  \frac{\sqrt{2}}{8}(1+\sqrt{3})               \\
     &    1 &  \frac{\sqrt{2}}{8}(3+\sqrt{3})               \\
     &    2 &  \frac{\sqrt{2}}{8}(3-\sqrt{3})               \\
     &    3 &  \frac{\sqrt{2}}{8}(1-\sqrt{3})               \\
                                                            \\
   3 &    0 &  0.3326705530                                 \\
     &    1 &  0.8068915093                                 \\
     &    2 &  0.4598775021                                 \\
     &    3 & -0.1350110200                                 \\
     &    4 & -0.0854412739                                 \\
     &    5 &  0.0352262919                                 \\
                                                            \\
   4 &    0 & -0.0757657148                                 \\
     &    1 & -0.0296355276                                 \\
     &    2 &  0.4976186676                                 \\
     &    3 &  0.8037387518                                 \\
     &    4 &  0.2978577956                                 \\
     &    5 & -0.0992195436                                 \\
     &    6 & -0.0126039673                                 \\
     &    7 &  0.0322231006                                 \\
                                                            \\
   5 &    0 &  0.0273330683                                 \\
     &    1 &  0.0295194909                                 \\
     &    2 & -0.0391342493                                 \\
     &    3 &  0.1993975340                                 \\
     &    4 &  0.7234076904                                 \\
     &    5 &  0.6339789635                                 \\
     &    6 &  0.0166021058                                 \\
     &    7 & -0.1753280899                                 \\
     &    8 & -0.0211018340                                 \\
     &    9 &  0.0195388827                                 \\
                                                            \\
   6 &    0 &  0.0154041093                                 \\
     &    1 &  0.0034907121                                 \\
     &    2 & -0.1179901111                                 \\
     &    3 & -0.0483117426                                 \\
     &    4 &  0.4910559419                                 \\
     &    5 &  0.7876411410                                 \\
     &    6 &  0.3379294217                                 \\
     &    7 & -0.0726375228                                 \\
     &    8 & -0.0210602925                                 \\
     &    9 &  0.0447249018                                 \\
     &   10 &  0.0017677119                                 \\
     &   11 & -0.0078007083                                 \\
\end{array}
\hspace{2cm}
\begin{array}[t]{rrr}
   p &    n &  \mc{1}{c}{h_n}                               \\
   \hline
   7 &    0 &  0.0120154193                                 \\
     &    1 &  0.0172133763                                 \\
     &    2 & -0.0649080035                                 \\
     &    3 & -0.0641312898                                 \\
     &    4 &  0.3602184609                                 \\
     &    5 &  0.7819215933                                 \\
     &    6 &  0.4836109157                                 \\
     &    7 & -0.0568044769                                 \\
     &    8 & -0.1010109209                                 \\
     &    9 &  0.0447423495                                 \\
     &   10 &  0.0204642076                                 \\
     &   11 & -0.0181266051                                 \\
     &   12 & -0.0032832978                                 \\
     &   13 &  0.0022918340                                 \\
                                                            \\
   8 &    0 & -0.0033824160                                 \\
     &    1 & -0.0005421323                                 \\
     &    2 &  0.0316950878                                 \\
     &    3 &  0.0076074873                                 \\
     &    4 & -0.1432942384                                 \\
     &    5 & -0.0612733591                                 \\
     &    6 &  0.4813596513                                 \\
     &    7 &  0.7771857517                                 \\
     &    8 &  0.3644418948                                 \\
     &    9 & -0.0519458381                                 \\
     &   10 & -0.0272190299                                 \\
     &   11 &  0.0491371797                                 \\
     &   12 &  0.0038087520                                 \\
     &   13 & -0.0149522583                                 \\
     &   14 & -0.0003029205                                 \\
     &   15 &  0.0018899503                                 \\
                                                            \\
   9 &    0 &  0.0014009155                                 \\
     &    1 &  0.0006197809                                 \\
     &    2 & -0.0132719678                                 \\
     &    3 & -0.0115282102                                 \\
     &    4 &  0.0302248789                                 \\
     &    5 &  0.0005834627                                 \\
     &    6 & -0.0545689584                                 \\
     &    7 &  0.2387609146                                 \\
     &    8 &  0.7178970828                                 \\
     &    9 &  0.6173384491                                 \\
     &   10 &  0.0352724880                                 \\
     &   11 & -0.1915508313                                 \\
     &   12 & -0.0182337708                                 \\
     &   13 &  0.0620777893                                 \\
     &   14 &  0.0088592675                                 \\
     &   15 & -0.0102640640                                 \\
     &   16 & -0.0004731545                                 \\
     &   17 &  0.0010694900                                 \\
\end{array}
\hspace{2cm}
\begin{array}[t]{rrr}
   p &    n &  \mc{1}{c}{h_n}                               \\
   \hline
  10 &    0 &  0.0008625782                                 \\
     &    1 &  0.0007154205                                 \\
     &    2 & -0.0070567641                                 \\
     &    3 &  0.0005956828                                 \\
     &    4 &  0.0496861266                                 \\
     &    5 &  0.0262403651                                 \\
     &    6 & -0.1215521056                                 \\
     &    7 & -0.0150192388                                 \\
     &    8 &  0.5137098734                                 \\
     &    9 &  0.7669548366                                 \\
     &   10 &  0.3402160130                                 \\
     &   11 & -0.0878787115                                 \\
     &   12 & -0.0670899078                                 \\
     &   13 &  0.0338423547                                 \\
     &   14 & -0.0008687521                                 \\
     &   15 & -0.0230054614                                 \\
     &   16 & -0.0011404298                                 \\
     &   17 &  0.0050716492                                 \\
     &   18 &  0.0003401493                                 \\
     &   19 & -0.0004101159                                 \\
                                                            \\
  11 &    0 &  0.0006871194                                 \\
     &    1 &  0.0013826742                                 \\
     &    2 & -0.0039185532                                 \\
     &    3 & -0.0027931771                                 \\
     &    4 &  0.0372023572                                 \\
     &    5 &  0.0509417072                                 \\
     &    6 & -0.0540827111                                 \\
     &    7 & -0.0286938383                                 \\
     &    8 &  0.4078687490                                 \\
     &    9 &  0.7685266798                                 \\
     &   10 &  0.4520007834                                 \\
     &   11 & -0.0815151575                                 \\
     &   12 & -0.1499464788                                 \\
     &   13 &  0.0182541524                                 \\
     &   14 &  0.0237215478                                 \\
     &   15 & -0.0273470351                                 \\
     &   16 & -0.0085852863                                 \\
     &   17 &  0.0098741222                                 \\
     &   18 &  0.0024053043                                 \\
     &   19 & -0.0016456213                                 \\
     &   20 & -0.0002460505                                 \\
     &   21 &  0.0001222747                                 \\
 %                                                          \\
 %12 &    0 & -0.0002052660                                 \\
 %   &    1 & -0.0001769095                                 \\
 %   &    2 &  0.0021044473                                 \\
 %   &    3 &  0.0006915975                                 \\
 %   &    4 & -0.0130538410                                 \\
 %   &    5 & -0.0012870333                                 \\
 %   &    6 &  0.0600585962                                 \\
 %   &    7 &  0.0312568599                                 \\
 %   &    8 & -0.1235912129                                 \\
 %   &    9 & -0.0075179926                                 \\
 %   &   10 &  0.5166743898                                 \\
 %   &   11 &  0.7608721851                                 \\
 %   &   12 &  0.3434515018                                 \\
 %   &   13 & -0.0892710009                                 \\
 %   &   14 & -0.0801757818                                 \\
 %   &   15 &  0.0306867435                                 \\
 %   &   16 &  0.0018619255                                 \\
 %   &   17 & -0.0254930251                                 \\
 %   &   18 & -0.0005948328                                 \\
 %   &   19 &  0.0086342308                                 \\
 %   &   20 &  0.0006610377                                 \\
 %   &   21 & -0.0013865503                                 \\
 %   &   22 & -0.0000841826                                 \\
 %   &   23 &  0.0000976761                                 \\
\end{array}
}
\caption{
  Symlet-$p$ scaling coefficients $\seq{h_n}{}$
  \label{tbl:Sp_h}
  }
\end{table}


%=======================================
\section{Examples}
\label{sec:symlet_examples}
%=======================================
\begin{figure}
  \centering%
  \exboxt{\begin{tabular}{c|c}
    Daubechies-4 & Symlet-4
    \\\hline
    \includegraphics{../common/math/graphics/pdfs/D4_pz.pdf}&\includegraphics{../common/math/graphics/pdfs/S4_pz.pdf}\\
    \includegraphics{../common/math/graphics/pdfs/d4_phi_h.pdf}&\includegraphics{../common/math/graphics/pdfs/s4_phi_h.pdf}\\
    \includegraphics{../common/math/graphics/pdfs/d4_psi_g.pdf}&\includegraphics{../common/math/graphics/pdfs/s4_psi_g.pdf}
  \end{tabular}}
  \caption{Daubechies-4 and Symlet-4 wavelet systems \xref{ex:symlet_p4} \label{fig:symlet_p4}}
\end{figure}
%---------------------------------------
\begin{example}
\label{ex:symlet_p4}
%---------------------------------------
\prefpp{fig:symlet_p4} compares the Daubechies-4 and Symlet-4 wavelet structures:
\end{example}

\begin{figure}
  \centering%
  \exboxt{\begin{tabular}{c|c}
    Daubechies-8 & Symlet-8
    \\\hline
    \includegraphics{../common/math/graphics/pdfs/D8_pz.pdf}&\includegraphics{../common/math/graphics/pdfs/S8_pz.pdf}\\
    \includegraphics{../common/math/graphics/pdfs/d8_phi_h.pdf}&\includegraphics{../common/math/graphics/pdfs/s8_phi_h.pdf}\\
    \includegraphics{../common/math/graphics/pdfs/d8_psi_g.pdf}&\includegraphics{../common/math/graphics/pdfs/s8_psi_g.pdf}
  \end{tabular}}
  \caption{Daubechies-8 and Symlet-8 wavelet systems \xref{ex:symlet_p8} \label{fig:symlet_p8}}
\end{figure}
%---------------------------------------
\begin{example}
\label{ex:symlet_p8}
%---------------------------------------
\prefpp{fig:symlet_p8} compares the Daubechies-8 and Symlet-8 wavelet structures.
\end{example}

\begin{figure}
  \centering%
  \exboxt{\begin{tabular}{c|c}
    Daubechies-12 & Symlet-12
    \\\hline
    \includegraphics{../common/math/graphics/pdfs/D12_pz.pdf}&\includegraphics{../common/math/graphics/pdfs/S12_pz.pdf}\\
    \includegraphics{../common/math/graphics/pdfs/d12_phi_h.pdf}&\includegraphics{../common/math/graphics/pdfs/s12_phi_h.pdf}\\
    \includegraphics{../common/math/graphics/pdfs/d12_psi_g.pdf}&\includegraphics{../common/math/graphics/pdfs/s12_psi_g.pdf}
  \end{tabular}}
  \caption{Daubechies-12 and Symlet-12 wavelet systems \xref{ex:symlet_p12} \label{fig:symlet_p12}}
\end{figure}
%---------------------------------------
\begin{example}
\label{ex:symlet_p12}
%---------------------------------------
\prefpp{fig:symlet_p12} compares the Daubechies-12 and Symlet-12 wavelet structures.
\end{example}

\begin{figure}
  \centering%
  \exboxt{\begin{tabular}{c|c}
    Daubechies-16 & Symlet-16
    \\\hline
    \includegraphics{../common/math/graphics/pdfs/D16_pz.pdf}&\includegraphics{../common/math/graphics/pdfs/S16_pz.pdf}\\
    \includegraphics{../common/math/graphics/pdfs/d16_phi_h.pdf}&\includegraphics{../common/math/graphics/pdfs/s16_phi_h.pdf}\\
    \includegraphics{../common/math/graphics/pdfs/d16_psi_g.pdf}&\includegraphics{../common/math/graphics/pdfs/s16_psi_g.pdf}
  \end{tabular}}
  \caption{Daubechies-16 and Symlet-16 wavelet systems \xref{ex:symlet_p16} \label{fig:symlet_p16}}
\end{figure}
%---------------------------------------
\begin{example}
\label{ex:symlet_p16}
%---------------------------------------
\prefpp{fig:symlet_p16} compares the Daubechies-16 and Symlet-16 wavelet structures.
\end{example}

%---------------------------------------
\begin{example}
\label{ex:symlet_p4-p17}
%---------------------------------------
\prefpp{fig:symlet_p4-p7} and \prefpp{fig:symlet_p8-p17}
show the pole-zero plots for Symlet-4, Symlet-5,\ldots, Symlet-17 structures.
\end{example}

\begin{figure}
  \centering%
  \exbox{\begin{tabular}{c@{\qquad\qquad}c}
    \includegraphics{../common/math/graphics/pdfs/S4_pz.pdf}&\includegraphics{../common/math/graphics/pdfs/S5_pz.pdf}\\%
    \includegraphics{../common/math/graphics/pdfs/S6_pz.pdf}&\includegraphics{../common/math/graphics/pdfs/S7_pz.pdf}%
  \end{tabular}}%
  \caption{%
    Pole-zero plots for Symlet-4, Symlet-5, Symlet-6, and Symlet-7 structures 
    \xref{ex:symlet_p4-p17} 
    \label{fig:symlet_p4-p7}
    }
\end{figure}

\begin{figure}
  \centering%
  \exbox{\begin{tabular}{c@{\qquad\qquad}c}
      \includegraphics{../common/math/graphics/pdfs/S8_pz.pdf}&\includegraphics{../common/math/graphics/pdfs/S9_pz.pdf}%
    \\\includegraphics{../common/math/graphics/pdfs/S10_pz.pdf}&\includegraphics{../common/math/graphics/pdfs/S11_pz.pdf}%
    \\\includegraphics{../common/math/graphics/pdfs/S12_pz.pdf}&\includegraphics{../common/math/graphics/pdfs/S13_pz.pdf}%
    \\\includegraphics{../common/math/graphics/pdfs/S14_pz.pdf}&\includegraphics{../common/math/graphics/pdfs/S15_pz.pdf}%
    \\\includegraphics{../common/math/graphics/pdfs/S16_pz.pdf}&\includegraphics{../common/math/graphics/pdfs/S17_pz.pdf}%
  \end{tabular}}%
  \caption{%
    Pole-zero plots for Symlet-8, Symlet-9,\ldots, Symlet-17 structures 
    \xref{ex:symlet_p4-p17} 
    \label{fig:symlet_p8-p17}
    }
\end{figure}



