%============================================================================
% Daniel J. Greenhoe
% LaTeX File
%============================================================================


%======================================
\chapter*{Introduction}
%======================================


%======================================
\section{Description}
%======================================
This text describes two independent but related operations: \emph{Detection} of objects 
and \emph{Classification} of those objects.
The scope of these objects is that they are \emph{buried}. 
This scope qualitatively suggests that the objects are hidden from human vision 
and are in intimate contact with materials typically found in the materials constituting the earth such as soil, sand, clay,
rock, and/or moisture.

%======================================
\section{Method}
%======================================
This text follows the approach of \emph{Model-Based Signal Processing}\footnote{
  \citer{candy1985}, \citer{candy1992}, \citer{candy2005}, \citer{marston2006} }
which implies three components of the detection and classification operator pair:
\begin{enume}
  \item A \fncte{cost function}---an operation by which the effectiveness of the operations can be measured.
  \item A \structe{model}---a mathematical structure representing the entire system including the objects and their surrounding channel.
  \item An \fncte{algorithm}---an operator on the model which seeks to optimize (minimize) the cost function.
\end{enume}

%======================================
\section{System}
%======================================
The system consists of a channel and an excitation source.
The excitation source is a direct-coupled "shaker" or acoustically-coupled speaker
that excites a channel that may or may not contain an object.
The excitation signal propagates through the channel in the form of four "waves":
\begin{enume}
  \item P-waves
  \item S-waves
  \item Rayleigh waves
  \item Love waves
\end{enume}
The resulting response at any given location on the surface of the channel (e.g. surface of the ground)
can be measured using an accelerometer, an ultrasonic vibrometer, and/or a laser vibrometer.

%======================================
\section{Model}
%======================================
The interaction between earth materials and a \prope{compliant} foreign object shell 
has been demonstrated to be \prope{nonlinear}; this property in turn can be and has been exploited 
for the purposes of detection and classification.\footnote{
  \cite{donskoy1998}}



