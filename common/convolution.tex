%============================================================================
% Daniel J. Greenhoe
% LaTeX File
%============================================================================
%======================================
\chapter{Convolution}
%======================================
%======================================
\section{Convolution over continuous domains}
%======================================
%======================================
\subsection{Definition}
%======================================
%--------------------------------------
\begin{definition}
\footnote{
  \citerpgc{bachman2002}{268}{9780387988993}{Definition 5.2.1, but with $1/2\pi$ scaling factor},
  \citerpg{bachman1964}{6}{9781483267562},
  \citerpgc{bracewell1978}{224}{007007013X}{Table 11.1 Theorems for the Laplace Transform}
  }
\label{def:conv}
%--------------------------------------
Let $\spLLRBu$ be the space of all \structe{Lebesgue square-integrable functions}.
\defbox{\indxs{\conv}\begin{array}{M}
  The \opd{convolution} operation $\conv$ is defined as
  \\\indentx$\ds
    \brs{\ff\conv\fg}(x)
    \eqd \ff(x)\conv \fg(x)
    \eqd \int_{u\in\R} \ff(u)\fg(x-u) \du
    \qquad\scy\forall\ff,\fg\in\spLLRBu
  $
\end{array}}
\end{definition}

%======================================
\subsection{Properties}
%======================================
%--------------------------------------
\begin{theorem}
\footnote{
  \citerppg{bachman2002}{268}{270}{9780387988993},
  \citerpgc{schatzman2002}{147}{9780198508526}{7.2.1 Convolution}
  }
\label{thm:conv_algebra}
%--------------------------------------
Let $\conv$ be the \ope{convolution} operatation \xref{def:conv}.
Let $\spLLRBu$ be the space of all \structe{Lebesgue square-integrable functions}.
\thmbox{\begin{array}{>{\ds}rc>{\ds}l CD}
    \ff\conv\fg               &=& \fg\conv\ff                                       & \forall \ff,\fg\in\spLLRBu               & (\prope{commutative})
  \\\ff\conv\brp{\fg\conv\fh} &=& \brp{\ff\conv\fg}\conv\fh                         & \forall \ff,\fg\fh\in\spLLRBu            & (\prope{associative})
  \\\brp{\alpha\ff}\conv\fg   &=& \alpha\brp{\ff\conv\fg} = \ff\conv\brp{\alpha\fg} & \forall \ff,\fg\in\spLLRBu,\,\alpha\in\C & (\prope{homogeneous})
  \\\ff\conv\brp{\fg+\fh}     &=& \brp{\ff\conv\fg} + \brp{\ff\conv\fh}             & \forall \ff,\fg,\fh\in\spLLRBu           & (\prope{distributive})
\end{array}}
\end{theorem}
\begin{proof}
\begin{align*}
  \ff\conv\fg
    &\eqd \int_{u=-\infty}^{u=\infty} \ff(u)\fg(x-u) \du
    && \text{by definition of $\conv$}
    && \text{\xref{def:conv}}
  \\&\eqd \int_{x-v=-\infty}^{x-v=\infty} \ff(x-v)\fg(v) (-1)\dv
    && \text{where $v\eqd x-u$}
    && \text{$\implies$ $u=x-v$, $\du=-\dv$}
  \\&= -\int_{-v=-\infty}^{-v=\infty} \ff(x-v)\fg(v) \dv
  \\&= -\int_{v=\infty}^{v=-\infty} \ff(x-v)\fg(v) \dv
  \\&= \int^{v=\infty}_{v=-\infty} \fg(v)\ff(x-v) \dv
  \\&\eqd \fg\conv\ff
    && \text{by definition of $\conv$}
    && \text{\xref{def:conv}}
  \\
  \\
  \ff\conv\brp{\fg\conv\fh}
    &\eqd \ff\conv\int_{u\in\R} \fg(u)\fh(x-u) \du
    && \text{by definition of $\conv$}
    && \text{\xref{def:conv}}
  \\&\eqd \int_{v\in\R} \ff(v)\int_{u\in\R} \fg(u)\fh(x-v-u) \du\dv
    && \text{by definition of $\conv$}
    && \text{\xref{def:conv}}
  \\&= \int_{v\in\R} \int_{(w-v)\in\R} \ff(v)\fg(w-v)\fh(x-w) \dw\dv
    && \text{where $w\eqd u+v$}
    && \text{$\implies$ $u=w-v$}
  \\&= \int_{w\in\R} 
       \mcom{\brs{\int_{v\in\R} \ff(v)\fg(w-v)\dv}}{function of $w$} 
       \fh(x-w)\dw
  \\&\eqd \brs{\int_{v\in\R} \ff(v)\fg(w-v)\dv} \conv\fh
    && \text{by definition of $\conv$}
    && \text{\xref{def:conv}}
  \\&\eqd \brs{\ff\conv\fg} \conv\fh
    && \text{by definition of $\conv$}
    && \text{\xref{def:conv}}
  \\
  \\
  \brs{\alpha\ff}\conv\fg
    &\eqd \int_\R \brs{\alpha\ff(u)}\fg(x-u) \du
    && \text{by definition of $\conv$}
    && \text{\xref{def:conv}}
  \\&= \alpha\int_\R \brs{\alpha\ff(u)}\fg(x-u) \du
    && \mathrlap{\text{by \prope{homogeneous} property of $\int\dt$ operator}}
  \\&= \boxed{\alpha\brs{\ff\conv\fg}}
    && \text{by definition of $\conv$}
    && \text{\xref{def:conv}}
  \\&\eqd \alpha\brs{\int_\R \ff(u)\fg(x-u) \du}
    && \text{by definition of $\conv$}
    && \text{\xref{def:conv}}
  \\&\eqd \int_\R \ff(u) \brs{\alpha\fg(x-u)} \du
    && \mathrlap{\text{by \prope{homogeneous} property of $\int\dt$ operator}}
  \\&\eqd \boxed{\ff\conv\brs{\alpha\fg}}
    && \text{by definition of $\conv$}
    && \text{\xref{def:conv}}
  \\
  \\
  \ff\conv\brp{\fg+\fh}
    &\eqd \int_{u\in\R} \ff(u)\brs{\fg(x-u)+\fh(x-u)} \du
    && \text{by definition of $\conv$}
    && \text{\xref{def:conv}}
  \\&= \int_{u\in\R} \ff(u)\fg(x-u)\du
     + \int_{u\in\R} \fh(u)\fg(x-u)\du
    && \mathrlap{\text{by \prope{additive} property of $\int\dt$ operator}}
  \\&= \brp{\ff\conv\fg} + \brp{\ff\conv\fh}
    && \text{by definition of $\conv$}
    && \text{\xref{def:conv}}
\end{align*}
\end{proof}

%======================================
\section{Convolution over discrete domains}
%======================================
\ifdochasnot{seq}{
%--------------------------------------
\begin{definition}
\label{def:sequence}
\label{def:seq}
\footnote{
  \citerp{bromwich1908}{1},
  \citerpgc{thomson2008}{23}{143484367X}{Definition 2.1},
  \citerpg{joshi1997}{31}{8122408265}
  }
\label{def:tuple}
%--------------------------------------
Let $\clFyx$ be the set of all functions from a set $\setY$ to a set $\setX$.
Let $\Z$ be the set of integers.
\defbox{\begin{array}{l}
  \text{A function $\ff$ in $\clFyx$ is a \fnctd{sequence} over $\setX$ if $\setY=\Z$.}\\
  \text{A sequence may be denoted in the form $\ds\seqxZ{x_n}$ or simply as $\ds\seqn{x_n}$.}\\
  %\text{A function $\ff$ in $\clFyx$ is an \fnctd{n-tuple} over $\setX$ if $\setY=\setn{1,2,\ldots,\xN}$.}\\
  %\text{An n-tuple may be denoted in the form $\ds\tuplexn{x_n}$ or simply as $\ds\tuplen{x_n}$.}
\end{array}}
\end{definition}
  }

%--------------------------------------
\begin{definition}
\footnote{
  \citerpgc{kubrusly2011}{347}{0817649972}{Example 5.K}
  }
\label{def:spllR}
\label{def:spllF}
%--------------------------------------
Let $\fieldF$ be a \structe{field} \xref{def:field}.
\defboxt{
  The \structd{space of all absolutely square summable sequences} $\spllF$ over $\F$ is defined as
  \\\indentx$\ds\spllF\eqd\set{\seqxZ{x_n}}{\sum_{n\in\Z}\abs{x_n}^2 < \infty}$
  }
\end{definition}

The space $\spllR$ is an example of a \structe{separable Hilbert space}.
In fact, $\spllR$ is the \emph{only} separable Hilbert space in the sense that all separable Hilbert spaces
are isomorphically equivalent.
For example, $\spllR$ is isomorphic to $\spLLR$, the \structe{space of all absolutely square Lebesgue integrable functions}.
%That is, their topological structure is the same.
%Differences occur in the nature of operators on the spaces.

%--------------------------------------
\begin{definition}
\label{def:dsp_conv}
\label{def:convd}
\index{convolution}
%--------------------------------------
%Let $\seq{x_n}{n\in\Z}$ and $\seq{y_n}{n\in\Z}$ be sequences \xref{def:seq} in the space $\spllR$ \xref{def:spllR}.
\defbox{\begin{array}{M}
  The \opd{convolution} operation $\hxs{\convd}$ is defined as
  \\\indentx
  $\ds{\seqn{x_n}\convd\seqn{y_n}} \eqd \seq{\sum_{m\in\Z} x_{m} y_{n-m}}{n\in\Z}\qquad\scy\forall\seqxZ{x_n},\seqxZ{y_n}\in\spllR$
\end{array}}
\end{definition}

%--------------------------------------
\begin{proposition}
%--------------------------------------
Let $\convd$ be the \structe{convolution operator} \xref{def:dsp_conv}.
\propbox{
  \seqn{x_n}\convd\seqn{y_n} = \seqn{y_n}\convd\seqn{x_n}
  \qquad{\scy\forall\seqxZ{x_n},\seqxZ{y_n}\in\spllR}
  \qquad\text{($\convd$ is \prope{commutative})}
  }
\end{proposition}
\begin{proof}
\begin{align*}
  [x\convd y](n)
    &\eqd \sum_{m\in\Z} x_m y_{n-m}
    &&    \text{by \prefp{def:dsp_conv}}
  \\&=    \sum_{k\in\Z} x_{n-k} y(k)
    &&    \text{where $k\eqd n-m$ $\implies$ $m=n-k$}
  \\&=    \sum_{k\in\Z} x_{n-k} y(k)
    &&    \text{by \prope{commutativity} of addition}
  \\&=    \sum_{m\in\Z} x_{n-m} y_m
    &&    \text{by change of variables}
  \\&=    \sum_{m\in\Z} y_m x_{n-m}
    &&    \text{by \prop{commutative} property of the field over $\C$}
  \\&\eqd \brp{y\convd x}_n
    &&    \text{by \prefp{def:dsp_conv}}
\end{align*}
\end{proof}

%--------------------------------------
\begin{proposition}
\label{prop:conv_knk}
%--------------------------------------
Let $\convd$ be the \structe{convolution operator} \xref{def:dsp_conv}.
Let $\spllR$ be the set of \prope{absolutely summable} sequences \xref{def:spllR}.
\propbox{
  \brb{\begin{array}{F>{\ds}rc>{\ds}lD}
      (A).&\fx(n) &\in& \spllR & and
    \\(B).&\fy(n) &\in& \spllR
  \end{array}}
  \implies
  \brb{\begin{array}{>{\ds}rc>{\ds}l}
     \sum_{k\in\Z} \fx[k]\fy[n+k] &=& \fx[-n]\convd\fy(n)
  \end{array}}
  }
\end{proposition}
\begin{proof}
\begin{align*}
  \sum_{k\in\Z} \fx[k]\fy[n+k]
    &= \sum_{-p\in\Z} \fx[-p]\fy[n-p]
    && \text{where $p\eqd -k$}
    && \text{$\implies$ $k=-p$}
  \\&= \sum_{p\in\Z} \fx[-p]\fy[n-p]
    && \text{by \prope{absolutely summable} hypothesis}
    && \text{\xref{def:spllR}}
  \\&= \sum_{p\in\Z} \fx'[p]\fy[n-p]
    && \text{where $\fx'[n] \eqd\fx[-n]$}
    && \text{$\implies$ $\fx[-n]=\fx'[n]$}
  \\&\eqd \fx'[n]\convd\fy[n]
    && \text{by definition of \ope{convolution} $\convd$}
    && \text{\xref{def:convd}}
  \\&\eqd \fx[-n]\convd\fy[n]
    && \text{by definition of $\fx'[n]$}
\end{align*}
\end{proof}
