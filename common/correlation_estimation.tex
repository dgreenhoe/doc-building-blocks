%============================================================================
% LaTeX File
% Daniel J. Greenhoe
%============================================================================
%=======================================
\chapter{Correlation Estimation}
%=======================================
%=======================================
\section{Time series method}
%=======================================

%---------------------------------------
\begin{definition}
\footnote{
  \citerppc{jenkins1968}{180}{183}{\textsection ``5.3.4 Discrete time autocovariance estimates"},
  \citerpgc{vaseghi2000}{271}{0471626929}{\textsection ``9.3.3 Energy-Spectral Density and Power-Spectral Density"}
  }
%---------------------------------------
\defboxt{
  The \fnctd{windowed auto-correlation estimate} $\estRxx(m)$ is defined as
  \\\indentx$\ds\estRxx(m) \eqd \frac{1}{\xN} \sum_{n=0}^{\xN-\abs{m}} \rvx(n)\rvx(n+m)$
  }
\end{definition}

%---------------------------------------
\begin{theorem}
\footnote{
  \citer{jenkins1968},
  \citerppgc{clarkson1993}{54}{56}{0849386098}{\textsection ``2.6.2 Estimation of Moments --- Time Averages"},
  \citerpgc{vaseghi2000}{272}{0471626929}{\textsection ``9.3.3 Energy-Spectral Density and Power-Spectral Density"}
  }
%---------------------------------------
\thmbox{\begin{array}{rc>{\ds}lD}
  \pE \brs{\estRxx(m)} &=& \brp{1-\frac{\abs{m}}{\xN}}\Rxx(m) & (\prope{asymptotically unbiased})\\
  \var\brs{\estRxx(m)} &=& \frac{1}{\xN} \sum_{n\in\Z} \brs{\Rxx^2(n) + \Rxx(n-m)\Rxx(n+m)} & (\prope{consistent})
\end{array}}
\end{theorem}

%=======================================
\section{Spectral methods}
%=======================================
Here are two methods for estimating correlation using spectral methods:
\begin{enumerate}
  \item Calculate the estimate $\estH(\omega)$ \xref{chp:systemid} of the true system model $\fH(\omega)$
        and then calculate the \ope{Inverse Fourier Transform} of $\estH(\omega)$.
  \item Calculate an \ope{AR}, \ope{MA}, or \ope{ARMA} estimate of the true system model $\fH(z)$ and then 
        compute the \ope{Inverse Z-Transform} of the estimate.\footnotemark
\end{enumerate}
\footnotetext{
  \citerppgc{clarkson1993}{56}{59}{0849386098}{\textsection ``Appendix 2A --- Calculating the Correlation via Contour Integration"}
  }

