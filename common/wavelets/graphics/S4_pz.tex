%============================================================================
% Daniel J. Greenhoe
% LaTeX File
% Symlet-4 Pole Zero plot
% nominal font size = \gsize
% nominal unit = 10mm
%============================================================================
\begin{pspicture}(-1.5,-2)(4.1,2)%
  %-------------------------------------
  % settings
  %-------------------------------------
  \psset{radius=1mm,linewidth=0.75pt}%
  %-------------------------------------
  % nodes
  %-------------------------------------
  \pnode(0,0){origin}%          z-plane origin
  \pnode(-1,0){pzeroes}%        p zeroes location
  \pnode(0,0){poles}%           2p-1 poles location
  %-------------------------------------
  % asymptotic regions
  %   |z-1| = sqrt(2) (arc with radius sqrt(2) centered at z=1)
  %   |z+1| = sqrt(2) (arc with radius sqrt(2) centered at z=-1)
  % sqrt(2) ~= 1.4142135623730950488016887242097
  %-------------------------------------
  \pscustom[fillstyle=solid,fillcolor=pzasymshade,linestyle=solid,linecolor=pzasym]{%
    \psarc(-1,0){1.414214}{-45}{45}%
    \psarcn( 1,0){1.414214}{135}{-135}% 
    }%
  %\pnode(0.2,0.73){zm1P}%
  %\pnode(2,1){zp1P}%
  %\rput[bl](-1.5,0.433){\rnode[r]{zm1L}{$\scy\abs{z-1}=\sqrt{2}$}}%
  %\rput[br](2.9,1.6){\rnode[b]{zp1L}{$\scy\abs{z+1}=\sqrt{2}$}}%
  %\ncline[linecolor=pzasym]{->}{zm1L}{zm1P}%
  %\ncline[linecolor=pzasym,nodesepA=2pt]{->}{zp1L}{zp1P}%
  %-------------------------------------
  % axes
  %-------------------------------------
  \psaxes[linecolor=axis,labels=none]{<->}(0,0)(-1.5,-1.5)(3.5,1.5)%
  \uput{1pt}[0](3.5,0){\color{axis}$\scy\Reb{z}$}%
  \uput{2pt}[210](0,1.5){\color{axis}$\scy\Imb{z}$}%
  %-------------------------------------
  % unit circle
  %-------------------------------------
  \pscircle[linecolor=unitcircle](origin){1}%            unit circle
  \rput[tr](-1.5mm,-1.5mm){\rnode[b]{circleL}{$\scy\abs{z}=1$}}% circle label
  \pnode(-0.707,-0.707){circleP}% circle point
  \ncline[linecolor=unitcircle,linestyle=dotted,nodesepA=0pt]{->}{circleL}{circleP}% circle pointer
  %-------------------------------------
  % B-spline zeroes (at z=-1)
  %-------------------------------------
  \Cnode[linewidth=1pt](-1,0){zp}%
  \rput[bl](-1.45,-1.9){\rnode[t]{zpL}{$p$}$=4$ zeroes}%
  \ncline[linestyle=dotted,linecolor=zero,nodesepA=1pt]{->}{zpL}{zp}%
  \uput{1mm}[30](-1,0){\color{zero}$\scy4$}%
  %-------------------------------------
  % orthogonality zeroes
  %-------------------------------------
  \Cnode [linecolor=zero,linewidth=1pt]( 3.0406604616,  0.0000000000 ){z1}% 
  \Cnode [linecolor=zero,linewidth=1pt]( 0.2840962982, -0.2432282259 ){z2}% 
  \Cnode [linecolor=zero,linewidth=1pt]( 0.2840962982,  0.2432282259 ){z3}% 
  \Cnode*[linecolor=zero,linewidth=1pt]( 0.3288759178,  0.0000000000 ){z4}% 
  \Cnode*[linecolor=zero,linewidth=1pt]( 2.0311355121, -1.7389508076 ){z5}% 
  \Cnode*[linecolor=zero,linewidth=1pt]( 2.0311355121,  1.7389508076 ){z6}% 
  %-------------------------------------
  % p-1=3 radial lines
  %-------------------------------------
  \ncline[linestyle=dashed]{origin}{z4}%
  \ncline[linestyle=dashed]{origin}{z5}%
  \ncline[linestyle=dashed]{origin}{z6}%
  %-------------------------------------
  % poles
  %-------------------------------------
  %\rput(poles){\color{pole}$\mathbf\times$}%
  \rput(poles){%
    \psline[linecolor=pole,linewidth=1pt](-0.1,+0.1)(+0.1,-0.1)%
    \psline[linecolor=pole,linewidth=1pt](-0.1,-0.1)(+0.1,+0.1)%
    }%
  \rput[tl](-1.45,1.9){$2$\rnode[br]{polesL}{$p$}$-1=7$ poles}%
  \ncline[linestyle=dotted,linecolor=pole,nodesepA=0pt,nodesepB=5pt]{->}{polesL}{poles}%
  \uput{1mm}[30](0,0){\color{pole}$\scy7$}%
  %%-------------------------------------
  %% discarded zeroes region
  %%-------------------------------------
  %\psframe[linestyle=dotted](1.7,-1.9)(3.2,1.9)%
  %\rput[b]{-90}(3.2,0){\color{blue}\footnotesize$p-1=3$ discarded zeroes}%
\end{pspicture}%
%
%\begin{picture}(800,360)(-300,-180)
%  %Axis
%  %---------------------------
%  %\graphpaper[10](0,0)(200,200)                  
%  \thicklines
%  \color{axis}%
%  \put(-130,   0){\line(1,0){500} }
%  \put(   0,-130){\line(0,1){260} }
%  \put( 380,   0){\makebox(0,0)[l]{$\Reb{z}$}}
%  \put(   0, 140){\makebox(0,0)[b]{$\Imb{z}$}}
%
%  %Unit Circle
%  %---------------------------
%  %\input{circ512.inc}
%  \color{circle}%
%  \qbezier( 100,   0)( 100, 41.421356)(+70.710678,+70.710678) % 0   -->1pi/4
%  \qbezier(   0, 100)( 41.421356, 100)(+70.710678,+70.710678) % pi/4-->2pi/4
%  \qbezier(   0, 100)(-41.421356, 100)(-70.710678,+70.710678) %2pi/4-->3pi/4
%  \qbezier(-100,   0)(-100, 41.421356)(-70.710678,+70.710678) %3pi/4--> pi 
%  \qbezier(-100,   0)(-100,-41.421356)(-70.710678,-70.710678) % pi  -->5pi/4
%  \qbezier(   0,-100)(-41.421356,-100)(-70.710678,-70.710678) %5pi/4-->6pi/4
%  \qbezier(   0,-100)( 41.421356,-100)( 70.710678,-70.710678) %6pi/4-->7pi/4
%  \qbezier( 100,   0)( 100,-41.421356)( 70.710678,-70.710678) %7pi/4-->2pi
%  %\put( 110, 110){\makebox(0,0)[lb]{$z=e^{i\omega}$}}
%  %\put( 105, 105){\vector(-1,-1){33}}
%
%  %Asymptotic zone
%  %---------------------------
%  \color{pzasym}%
%    \qbezier(41.4214,0)(41.4214, 58.5786)(0, 100)      % inner upper arc
%    \qbezier(41.4214,0)(41.4214,-58.5786)(0,-100)      % inner lower arc
%
%    \qbezier(0, 100)(41.421, 141.421)(100, 141.421)    %outer upper arc
%    \qbezier(100, 141.421)(158.579,141.421)(200,100)
%    \qbezier(200,100)(241.421,58.5786)(241.421,0)
%
%    \qbezier(0,-100)(41.421,-141.421)(100,-141.421)    %outer lower arc
%    \qbezier(100,-141.421)(158.579,-141.421)(200,-100)
%    \qbezier(200,-100)(241.421,-58.5786)(241.421,0)
%
%  %Poles
%  %---------------------------
%  \color{pole}%
%  \put(-100,-100){\makebox(0,0)[tr]{7 poles}}
%  \put(-100,-100){\vector(1,1){91}}
%  \put(   0,    0){\makebox(0,0)[c]{$\times$}}
%  \put(   0,    0){\makebox(0,0)[c]{\hspace{1em}$^7$}}
%
%  %Zeros
%  %---------------------------
%  \color{zero}%
%  \put(-150, -50){\makebox(0,0)[tr]{p=4 zeros}}
%  \put(-150, -50){\vector( 1, 1){43}}
%  \put(-100,    0){\circle{15}$^4$}
%
%  \qbezier[30](0,0)(101.5,  87)(203,174)
%  \qbezier[30](0,0)(101.5, -87)(203,-174)
%
%  \put( 304.06604616,  000.00000000 ){\circle {15}} 
%  \put( 028.40962982, -024.32282259 ){\circle {15}} 
%  \put( 028.40962982,  024.32282259 ){\circle {15}} 
%  \put( 032.88759178,  000.00000000 ){\circle*{15}} 
%  \put( 203.11355121, -173.89508076 ){\circle*{15}} 
%  \put( 203.11355121,  173.89508076 ){\circle*{15}} 
%\end{picture}                                   

