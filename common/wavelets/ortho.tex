%============================================================================
% LaTeX File
% Daniel J. Greenhoe
%============================================================================

%======================================
\chapter{Orthonormality Constraint}
\label{chp:ortho}
%======================================


%=======================================
\section{Definition}
%=======================================
In any Hilbert space $\spLLR$ with a subspace $\spY$, $\spY\adds\spYo=\spLLR$
(\thme{Projection Theorem}\ifdochas{subspace}{, \xrefp{thm:projection}}).
The subspace $\spYo$ is the \structe{orthogonal complement} of $\spY$ in $\spLLR$.
In wavelet theory, the orthogonal complement of a scaling subspace $\spV_n$ in the subspace $\spV_{n+1}$
is the \structe{wavelet subspace} $\spW_n$ (next definition).
The wavelet subspace $\spW_n$ is \prope{unique} with respect to $\spV_n$ because
in a Hilbert space, $\spW_n^\orthog = \spV_n^{\orthog\orthog} = \spV_n$\ifsxref{subspace}{thm:inprod_orthog}.

\prefpp{def:wavsys} defines a \hie{general} wavelet system.
An \hie{orthonormal wavelet system} is simply a wavelet system
with additional constraints of orthonormality \xref{def:ows}.

%\ifdochasnot{relation}{
%%---------------------------------------
%\begin{definition}
%\label{def:kdelta}
%%---------------------------------------
%\defbox{\begin{array}{M}
%The \hid{Kronecker delta function} $\hxs{\kdelta_n}$ is defined as
%  \qquad
%  $\ds\kdelta_n\eqd
%    \brbl{\begin{array}{cMD}
%      1 & for $n=0$ & and \\
%      0 & for $n\ne 0$\..
%    \end{array}}
%  \qquad\scy
%  \forall n\in\Z$
%\end{array}}
%\end{definition}
%}

%--------------------------------------
\begin{definition}
\label{def:oms}
\label{def:ows}
%--------------------------------------
Let $\spO\eqd\wavsys$ be a \structe{wavelet system} \xref{def:wavsys}
and $\spGm\eqd\mrasys$ its component \structe{MRA system} \xref{def:mrasys}.
Let $\kdelta_n$ be the \structe{Kronecker delta function} \xref{def:kdelta}.
\defbox{\begin{array}{FMMlC}
  1. & The \structe{MRA system} $\spGm$    & is \hid{orthonormal} if & \inprod{\fphi}{\opTrn^n\fphi}=\kdelta_n & \forall n\in\Z. \\
  2. & The \structe{wavelet system} $\spO$ & is \hid{orthonormal} if & \inprod{\fpsi}{\opTrn^n\fpsi}=\kdelta_n & \forall n\in\Z  \\
  \mc{3}{r}{\text{and}}                                            & \inprod{\fphi}{\opTrn^n\fpsi}=0         & \forall n\in\Z. \\
  \mc{5}{M}{In the case of (1), $\spGm$ is an \hid{orthonormal MRA system}.}\\
  \mc{5}{M}{In the case of (2), $\spO$  is an \hid{orthonormal wavelet system}.}
\end{array}}
%
%\defbox{\begin{array}{M}
%  A wavelet system $\wavsys$ is \hid{orthonormal} if
%  \\\indentx
%  $\begin{array}{F>{\ds}l@{\qquad}D}
%    1. & \inprod{\fphi}{\opTrn^n\fphi}=\kdelta_n
%       & ($\setxZ{\opTrn^n\fphi}$ is \prope{orthonormal} in $\spLLR$)
%       \\
%    2. & \spV_0 \orthog \spW_0
%       & ($\spV_0$ and $\spW_0$ are \structe{orthonormal complementary subspaces})
%  \end{array}$
%\end{array}}
\end{definition}


%=======================================
\section{Orthonormal MRA systems}
%=======================================
%=======================================
\subsection{Scaling coefficients from scaling function}
%=======================================
The \thme{dilation equation} \xref{thm:dilation_eq} demonstrates that if we know
the \fncte{scaling coefficients} $\seqn{\fh_n}$ \xref{def:hn}, we can
(recursively) compute the scaling function $\fphi$ \xref{def:mra}.
The converse is in general not so easy to compute.
However, in an \structe{orthonormal MRA system},
we can compute the scaling coefficients $\seqn{\fh_n}$ from the scaling function $\fphi$ (next).
%--------------------------------------
\begin{proposition}
\footnote{
  \citerpgc{mallat}{228}{012466606X}{(7.29)},
  \citerpg{wojtaszczyk1997}{38}{0521578949}
  }
\label{prop:phi->h}
%--------------------------------------
Let $\mrasys$ be an \structe{MRA system} \xref{def:mrasys}.
\propbox{
  \mcom{{\inprod{\fphi}{\opTrn^n\fphi}=\kdelta_n}}{$\setn{\opTrn^n\fphi}$ is \prope{orthonormal}}  %{$\setn{\fphi_n}$ is \prope{orthonormal} in $\spLLR$}
  \qquad\implies\qquad
 \fh_n  = \ds \inprod{\fphi}{\opDil\opTrn^n\fphi}
        =\ds \cwt \int_\R \fphi(x)\fphi^\ast(2x-n)\dt
  \qquad\scy\forall n\in\Z
  }
\end{proposition}
\begin{proof}
\begin{align*}
   \inprod{\fphi}{\opDil\opTrn^n\fphi}
     &= \inprod{\sum_{m\in\Z}h_m\opDil\opTrn^m \fphi}{\opDil\opTrn^n\fphi}
     && \text{by the \thme{dilation equation}}
     && \text{\xref{thm:dilation_eq}}
   \\&= \sum_{m\in\Z}h_m\inprod{\opDil\opTrn^m \fphi}{\opDil\opTrn^n\fphi}
     && \text{by \prope{additive} property of $\inprodn$} 
     && \text{\xref{def:inprod}}
   \\&= \sum_{m\in\Z}h_m \inprod{\sqrt{2}\fphi(2x-m)}{\sqrt{2}\fphi(2x-n)}
     && \text{by definition of $\opTrn$ and $\opDil$}
     && \text{\xref{def:opTD}}
   \\&= \sum_{m\in\Z}h_m \int_{x\in\R} \sqrt{2}\fphi(2x-m)\sqrt{2}\fphi^\ast(2x-n) \dx
     && \text{by definition of $\inprodn$ in $\spLLR$}
     && \text{\xref{def:spLLR}}
   \\&= \sum_{m\in\Z}h_m \int_{u\in\R} \sqrt{2}\fphi(u-m)\sqrt{2}\fphi^\ast(u-n) \half \du
     && \text{where $u\eqd2x$ $\implies$}
     && \text{$\du=2\dx$}
   \\&= \sum_{m\in\Z}h_m \int_{u\in\R} \fphi(u-m)\fphi^\ast(u-n) \du
   \\&= \sum_{m\in\Z}h_m \int_{u\in\R} \fphi(u-m)\fphi^\ast(u-n) \du
   \\&= \sum_{m\in\Z}h_m \int_{v\in\R} \fphi(v)\fphi^\ast(v-(n-m)) \dv
     && \text{where $v\eqd u-m$ $\implies$}
     && \text{$u=v+m$}
   \\&= \sum_{m\in\Z}h_m \inprod{\fphi}{\opTrn^{n-m}\fphi}
     && \text{by definition of $\inprodn$ in $\spLLR$}
     && \text{\xref{def:spLLR}}
   \\&= \sum_{m\in\Z}h_m \kdelta_{m-n}
     && \text{by left hypothesis}
   \\&=\fh_n
     && \text{by definition of $\kdelta$}
     && \text{\xref{def:kdelta}}
   \\
   \\
   \inprod{\fphi}{\opDil\opTrn^n\fphi}
     &= \inprod{\fphi}{\opDil\fphi(x-n)}
     && \text{by definition of $\opTrn$}
     && \text{\xref{def:opT}}
   \\&= \inprod{\fphi}{\cwt\fphi(2x-n)}
     && \text{by definition of $\opDil$}
     && \text{\xref{def:opD}}
   \\&= \cwt \int_\R \fphi(x)\fphi^\ast(2x-n)\dt
     && \text{by definition of $\inprodn$ in $\spLLR$}
     && \text{\xref{def:spLLR}}
\end{align*}
\end{proof}


%-------------------------------------
\begin{example}[\exm{Haar scaling function}]
\index{Haar scaling function}
\label{ex:haarphi}
%-------------------------------------
%Let a function $\fphi$ be defined as follows:
\exbox{\begin{array}{Mclm{37mm}}
    Let $\fphi(x)$ &\eqd&
      \brbl{\begin{array}{lM}
         1  & for $x\in\intco{0}{1}$  \\
         0  & otherwise.
      \end{array}}
    &\includegraphics{../common/math/graphics/pdfs/d1_phi.pdf}
  \\
    Then, $\fh_n$ &=&
      \brbl{\begin{array}{MM}
        $\cwt$ & for $n\in\setn{0,1}$\\
        $0$                  & otherwise.
      \end{array}}
    &\includegraphics{../common/math/graphics/pdfs/d1_h.pdf}%
\end{array}}
\end{example}
\begin{proof}
\begin{align*}
 \fh_{n}
    &= \inprod{\fphi}{\opDil\opTrn^n\fphi}
    && \text{by \prefp{prop:phi->h}}
  \\&= \inprod{\setind_{[0,1)}(x)}{\opDil\opTrn^{-n}\setind_{[0,1)}(x)}
    && \text{by definition of $\fphi$}
  \\&= \inprod{\setind_{[0,1)}(x)}{\sqrt{2}\setind_{[0,1)}(2x-n)}
     && \text{by definition of $\opTrn$ and $\opDil$}
     && \text{\xref{def:opTD}}
  \\&= \sqrt{2}\inprod{\setind_{[0,1)}(x)}{\setind_{[0,1)}(2x-n)}
    && \text{by properties of $\inprodn$}
    && \text{\xref{def:inprod}}
  \\&= \sqrt{2}\int_\R \setind_{[0,1)}(x)\,\setind_{[0,1)}(2x-n)\dx
    && \text{by definition of $\inprodn$ in $\spLLR$}
    && \text{\xref{def:spLLR}}
  \\&= \sqrt{2}\int_0^1 \setind_{[0,1)}(2x-n)\dx
    && \text{by definition of $\setind$}
    && \text{\xref{def:setind}}
  \\&= \brbl{\begin{array}{lclM}
         \sqrt{2}\int_0^\half  1\dx &=& \cwt & for $n=0$ \\
         \sqrt{2}\int_\half ^1 1\dx &=& \cwt & for $n=1$ \\
         0                               &=& 0                  & otherwise
       \end{array}}
    && \text{by definition of $\setind$}
    && \text{\xref{def:setind}}
\end{align*}
\end{proof}

%=======================================
\subsection{Necessary conditions}
%=======================================
%---------------------------------------
\begin{proposition}
\footnote{
  \citerpgc{wojtaszczyk1997}{31}{0521578949}{Proposition 2.16},
  \citerppgc{pinsky2002}{315}{316}{0534376606}{6.4.3.1 Additional remarks}
  }
\label{prop:oms_phi1}
%--------------------------------------
Let $\spO\eqd\mrasys$ be an \structe{MRA system} \xref{def:mrasys}.
Let $\Fphi(\omega)$ be the \ope{Fourier transform} \xref{def:opFT} of $\fphi(x)$.
\propbox{
  \brb{\begin{array}{FMD}
    (A).  & $\seqn{\opTrn^n\fphi}$ is \prope{orthonormal} & and \\
    (B).  & $\Fphi(\omega)$ is \prope{continuous} at $0$  &
  \end{array}}
  \quad\implies\quad
  \brb{\begin{array}{F>{\ds}rc>{\ds}lD}
    (1). & \abs{\Fphi(0)}            &=& \frac{1}{\sqrt{2\pi}} & and \\
    (2). & \abs{\int_\R \fphi(x)\dx} &=& 1
  \end{array}}
  }
\end{proposition}
\begin{proof}
\begin{enumerate}
  \item Let $\opP_j$ be the \ope{projection operator} that generates the scaling subspace $\spV_j$ such that \label{item:oms_phi1_Pj}
    \\\indentx$\ds\opV_j=\set{\opP_j\ff}{\ff\in\spH}$

  \item Let $\fg(x)$ a function in $\spLLR$ such that its \ope{Fourier transform} \xref{def:opFT} $\Fg(\omega)$
        has support $\support\Fg\subseteq\intcc{-1}{1}$ \label{item:oms_phi1_g}

  \item lemma: The set $\ds\setnZ{\frac{1}{2^{j/2}{\sqrt{2\pi}}}e^{-i\frac{\omega}{2^j}n}}$ \label{ilem:oms_phi1_fs}
        is an \structe{orthonormal basis} for functions in $\spLLR$ that have support in $\intcc{-2^j\pi}{2^j\pi}$.
        Proof: This follows from \prefp{thm:opFSc}.
        {\begin{align*}
          \int_{-2^j\pi}^{2^j\pi} \brs{\frac{1}{2^{j/2}\sqrt{2\pi}}e^{-i\frac{\omega}{2^j}n}}
                                  \brs{\frac{1}{2^{j/2}\sqrt{2\pi}}e^{-i\frac{\omega}{2^j}m}}^\ast \dw
            &= \int_{-2^j\pi}^{2^j\pi} \frac{1}{2^{j}(2\pi)}e^{-i\frac{\omega}{2^j}(n-m)} \dw
          \\&= \frac{1}{2^{j}(2\pi)}\brp{2^j\pi+2^j\pi} \kdelta_{nm}
          \\&= \kdelta_{nm}
        \end{align*}}

  \item  Proof for (1): Let $\setA\eqd\intcc{-1}{1}$. \label{item:oms_phi_Fphi}
    \begin{align*}
      \norm{\fg}^2
        &=   \norm{\lim_{j\to\infty}\opP_j\fg}^2
      \\&=   \lim_{j\to\infty}\sum_{n\in\Z}\abs{\inprod{\opP_j\fg}{\opD^j\opTrn^n\fphi}}^2
        &&   \text{by (A) and \thme{Parseval's Identity}}
        &&   \text{\xref{thm:fst}}
      \\&=   \lim_{j\to\infty}\sum_{n\in\Z}\abs{\inprod{\fg}{\opD^j\opTrn^n\fphi}}^2
        &&   \text{by definition of $\opP_j$}
        &&   \text{\xref{item:oms_phi1_Pj}}
      \\&=   \lim_{j\to\infty}\sum_{n\in\Z}\abs{\inprod{\opFT\fg}{\opFT\opD^j\opTrn^n\fphi}}^2
        &&   \text{by \prope{unitary} property of $\opFT$}
        &&   \text{\xref{thm:ft_unitary}}
      \\&=   \lim_{j\to\infty}\sum_{n\in\Z}\abs{\inprodr{\Fg(\omega)}{\frac{1}{2^{j/2}}e^{-i\frac{\omega}{2^j} n}\Fphi\brp{\frac{\omega}{2^j}}}}^2
        &&   \text{by \prefp{prop:FTDf}}
      \\&=   \lim_{j\to\infty}\sum_{n\in\Z}\abs{\int_\R{\Fg(\omega)\Fphi^\ast\brp{\frac{\omega}{2^j}}} \, \frac{1}{2^{j/2}}e^{i\frac{\omega}{2^j} n}}^2 \dw
        &&   \text{by definition of $\inprodn$ in $\spLLR$}
        &&   \text{\xref{def:spLLR}}
      \\&=   \mathrlap{%
             \lim_{j\to\infty}\sum_{n\in\Z}\abs{\inprod{\sqrt{2\pi}\Fg(\omega)
             \Fphi^\ast\brp{\frac{\omega}{2^j}}}{\frac{1}{2^{j/2}\sqrt{2\pi}}e^{-i\frac{\omega}{2^j} n}}}^2
             \quad\text{by definition of $\inprodn$ in $\spLLR$ \xref{def:spLLR}}
             }
      \\&=   \lim_{j\to\infty}\norm{\sqrt{2\pi}\Fg(\omega)\Fphi^\ast\brp{\frac{\omega}{2^{j}}}}^2
        &&   \text{by \pref{ilem:oms_phi1_fs} and \thme{Parseval's Identity}}
        &&   \text{\xref{thm:fst}}
      \\&=   \norm{\sqrt{2\pi}\Fg(\omega)\Fphi^\ast(0)}^2
      \\&=   2\pi\abs{\Fphi^\ast(0)}^2\,\norm{\Fg(\omega)}^2
        &&   \text{by \prope{homogeneous} property of $\normn$}
        &&   \text{\xref{def:norm}}
      \\&=   2\pi\abs{\Fphi(0)}^2\,\norm{\fg}^2
        &&   \text{by \prope{unitary} property of $\opFT$}
        &&   \text{\xref{thm:ft_unitary}}
      \\&\implies \abs{\Fphi(0)}=\frac{1}{\sqrt{2\pi}}
    \end{align*}

  \item  Proof for (2): \label{item:oms_fphi}
    \begin{align*}
      \abs{\int_\R\fphi(x)\dx}
        &= {\abs{\sqrt{2\pi}\frac{1}{\sqrt{2\pi}}\int_\R\fphi(x)e^{-i\omega x}\dx}}_{\omega=0}
      \\&= {\abs{\sqrt{2\pi}\Fphi(\omega)}}_{\omega=0}
        && \text{by definition of $\opFT$}
        && \text{\xref{def:opFT}}
      \\&= \abs{\sqrt{2\pi}\frac{1}{\sqrt{2\pi}}}
        && \text{by (1)}
        && \text{\xref{item:oms_phi_Fphi}}
      \\&= 1
    \end{align*}
\end{enumerate}
\end{proof}


Every \structe{MRA system} $\mrasys$, regardless of whether it is \prope{orthonormal} or not,
has the following necessary conditions on $\seqn{\fh_n}$:
\\$\begin{array}{F>{\ds}lDD}
  1. & \sum_{n\in\Z}h_n = \sqrt{2}
     & \prope{admissibility condition}
     & \xref{thm:admiss}
     \\
  2. & \inprod{\fphi}{\opTrn^n \fphi} = \sum_{m\in\Z}\fh_m \sum_{k\in\Z}\fh_k^\ast \inprod{\fphi}{\opTrn^{2n-m+k} \fphi}
     & \prope{quadrature condition} in ``time"
     & \xref{thm:wav_quadcon}
     \\
  3. & \Fphi\brp{\omega} = \cwt\: \Dh\brp{\frac{\omega}{2}}\: \Fphi\brp{\frac{\omega}{2}}
     & \thme{frequency dilation equation}
     & \xref{lem:Fphi}
     \\
  4. & \abs{\Dh\left(\omega     \right)}^2 \Sphi(\omega) + \abs{\Dh\brp{\omega+\pi }}^2 \Sphi(\omega+\pi) = 2\Sphi(2\omega)
     & \prope{quadrature condition} in ``frequency"
     & \xref{thm:Sphi}
\end{array}$

In an \structe{orthonormal MRA system}, the \prope{orthonormal} property is equivalent to
 $\Swphi(\omega)=1$ (where $\Swphi$ is the \structe{auto-power spectrum}),
and so the quadrature conditions simplify considerably (next theorem).
%But although the \prope{orthonormal} and \prope{unitary auto-power spectrum} properties are equivalent,
The \prope{orthonormal quadrature condition}s in ``time" and ``frequency" are also equivalent to each other
and are implied by the $\Swphi(\omega)=1$ condition.
However, the \prope{orthonormal quadrature condition}s are \emph{not} equivalent to the $\Swphi(\omega)=1$ condition \xref{cnt:Swphi_quadcon}.
%--------------------------------------
\begin{lemma}[\lemd{orthonormal quadrature condition}s]
\footnote{
  \citerp{chui}{135},
  \citerppg{mallat}{229}{238}{012466606X},
  \citerp{goswami}{110},
  \citerp{vaidyanathan1990}{65}
  }
\label{lem:oms_quadcon}
%--------------------------------------
Let $\mrasys$ be an \structe{MRA system} \xref{def:mrasys}.
Let $\Dh(\omega)$ be the \structe{discrete time Fourier transform} of $\seqn{\fh_n}$ \xref{def:dtft}.
Let $\Swphi(\omega)$ be the \structe{auto-power spectrum} of $\fphi(x)$ \xref{def:Swfg}.
\lembox{\begin{array}{clcl}
  &
  \mcom{\inprod{\fphi}{\opTrn^n\fphi} = \kdelta_n \quad\scy\forall n\in\Z}
       {(1) $\setn{\opTrn^n\fphi}$ is \prope{orthonormal}}
  &\iff&
  \mcom{\Swphi(\omega)=1}{(2) auto-power spectrum is 1}
  \\\impnotimpby&
  \mcom{\brb{\begin{array}{>{\ds}rclC}
    \sum_{m\in\Z}\fh_m\fh_{m-2n}^\ast &=& \kdelta_n &\forall n\in\Z
  \end{array}}}{(3) \prope{orthogonal quadrature condition} in ``time"}
  &\iff&
  \mcom{\brb{\begin{array}{>{\ds}rc>{\ds}lcl}
   \abs{\Dh(\omega)}^2  &+& \abs{\Dh(\omega+\pi)}^2        &=& 2
  \end{array}}}{(4) \prope{orthogonal quadrature condition} in ``frequency"}
\end{array}}
\end{lemma}
\begin{proof}
\begin{enumerate}
  \item Proof that (1)$\iff$(2): by \prefp{thm:Swfgo}.
  \item Proof that [(1)/(2)]$\implies$(3):
    \begin{align*}
      \sum_{m\in\Z}h_m\fh_{m-2n}^\ast
        &= \sum_{m\in\Z}h_m \sum_{k\in\Z}\fh_k^\ast \kdelta_{k-m+2n}
        && \text{by definition of $\kdelta$}
        && \text{\xref{def:kdelta}}
      \\&= \sum_{m\in\Z}h_m \sum_{k\in\Z}\fh_k^\ast\inprod{\fphi }{\opTrn^{2n-m+k} \fphi}
        && \text{by (1)}
      \\&= \inprod{\fphi}{\opTrn^n \fphi}
        && \text{by \thme{quadrature condition in ``time"}}
        && \text{\xref{thm:wav_hh}}
      \\&= \kdelta_n
        && \text{by (1)}
    \end{align*}

  \item Proof that (3)$\notimplies$[(1)/(2)]: by \prefp{cnt:Swphi_quadcon}.
  \item Proof that (3)$\iff$(4): by \prefp{lem:ows_quadcon}.
%    \begin{align*}
%      \mcom{\inprod{\fphi}{\opTrn^n\fphi} = \kdelta_n}
%           {$\mrasys$ is \prope{orthonormal}}
%      &\implies
%          \mcom{\brb{\begin{array}{>{\ds}rcl}
%            \sum_{m\in\Z}\fh_m\fh_{m-2n}^\ast &=& \kdelta_n
%          \end{array}}}{orthogonal quadrature condition in ``time"}
%        && \text{by \prefp{lem:oms_quadcon}}
%      \\&\iff
%          \mcom{\brb{\begin{array}{>{\ds}rc>{\ds}lcl}
%           \abs{\Dh(\omega)}^2  &+& \abs{\Dh(\omega+\pi)}^2        &=& 2
%          \end{array}}}{orthogonal quadrature condition in ``frequency"}
%        && \text{by \prefp{lem:ows_quadcon}}
%    \end{align*}
\end{enumerate}
%\begin{align*}
%  \abs{\Dh(\omega)}^2  + \abs{\Dh(\omega+\pi)}^2
%  \\&= \abs{\Dh(\omega)}^2 \cancelto{1}{\Sphi(\omega)} + \abs{\Dh(\omega+\pi)}^2 \cancelto{1}{\Sphi(\omega+\pi)}
%    && \text{by left hypothesis and \prefp{thm:Swfgo}}
%  \\&= 2\cancelto{1}{\Sphi(2\omega)}
%    && \text{by \prefp{thm:gen_quadcon}}
%  \\&= 2
%    && \text{by left hypothesis and \prefp{thm:Swfgo}}
%\end{align*}
\end{proof}
%--------------------------------------
\begin{counterex}
\footnote{
  \citerpgc{vidakovic}{57}{0471293652}{Remark 3.3.3}
  }
\label{cnt:Swphi_quadcon}
%--------------------------------------
Let $\mrasys$ be an \structe{MRA system} \xref{def:mrasys}.
\cntbox{\begin{array}{Mm{50mm}}
    Let $\fphi(x) \eqd
      \brbl{\begin{array}{lM}
         \frac{1}{3}  & for $x\in\intco{0}{3}$  \\
         0            & otherwise.
      \end{array}}$
    &\includegraphics{../common/math/graphics/pdfs/pulse03.pdf}
  \\\mc{2}{M}{
    Then,
      $\brb{\begin{array}{rclD}
        \Dh(\omega) + \Dh(\omega+\pi) &=& 2 & but\\
        %\Swphi(\omega) &=& e^{-i\frac{3}{2}\omega}\frac{\sin(3\omega/2)}{3\omega/2} & and so\\
        \Swphi(\omega)  &=& \frac{1}{9}\brs{3 + 4\cos(\omega) + 2\cos(2\omega)} & and so \\
        \Dh(\omega) + \Dh(\omega+\pi) = 2 &\notimplies& \Swphi(\omega)=1 .
      \end{array}}$
    }
\end{array}}
\end{counterex}
\begin{proof}
\begin{align*}
  \Fphi(\omega)
    &\eqd \opFT\fphi(x)
  \\&\eqd \cft\int_\R \fphi(x) e^{-i\omega x} \dx
    &&    \text{by definition of $\opFT$}
    &&    \text{\xref{def:opFT}}
  \\&=    \cft\int_\R \setind_\intco{0}{3} e^{-i\omega x} \dx
    &&    \text{by definition of $\fphi(x)$}
  \\&=    \cft\int_0^3  e^{-i\omega x} \dx
    &&    \text{by definition of $\setind$}
    &&    \text{\xref{def:setind}}
  \\&=    \left.\cft\brp{\frac{e^{-i\omega x}}{-i\omega}}\right|_0^3
  \\&=    \brp{\frac{3i}{\sqrt{2\pi}}}\brp{\frac{e^{-i3\omega}-1}{3\omega}}
  \\
  \\
  \Dh(\omega)
    &= \sqrt{2}\frac{\Fphi(2\omega)}{\Fphi(\omega)}
    && \text{by \prefp{lem:Fphi}}
  \\&= \sqrt{2}\frac{\brp{\frac{3i}{\sqrt{2}\pi}}\brp{\frac{e^{-i4\omega}-1}{6\omega}}}
                    {\brp{\frac{3i}{\sqrt{2}\pi}}\brp{\frac{e^{-i2\omega}-1}{3\omega}}}
    && \text{by \prefp{lem:FTNn}}
   %&& \text{by previous $\Fphi(\omega)$ result}
  \\&= \cwt
       \brp{\frac{e^{-i6\omega}-1}{e^{-i3\omega}-1}}
  \\&= \cwt
       \brp{\frac{e^{-i6\omega}-1}{e^{-i3\omega}-1}}
       \brp{\frac{e^{-i3\omega}+1}{e^{-i3\omega}+1}}
  \\&= \cwt
       \brp{\frac{ \brp{e^{-i6\omega}-1}  \brp{e^{-i3\omega}+1}}{e^{-i6\omega}-1}}
  \\&= \cwt\brp{1+e^{-i3\omega}}
  \\&= \brp{\cwt} e^{-i\frac{3}{2}\omega}\brp{e^{i\frac{3}{2}\omega}+e^{-i\frac{3}{2}\omega}}
  \\&= \brp{\cwt} e^{-i\frac{3}{2}\omega}\brs{2\cos\brp{\frac{3}{2}\omega}}
  \\&= \sqrt{2} e^{-i\frac{3}{2}\omega}\cos\brp{\frac{3}{2}\omega}
  \\
  \\
  \Dh(\omega+\pi)
  \\&= \cwt\brp{1+e^{-i3(\omega+\pi)}}
    && \text{by previous $\Dh(\omega)$ result}
  \\&= \cwt\brp{1+e^{-i3\omega}e^{-i3\pi}}
  \\&= \cwt\brp{1-e^{-i3\omega}}
  \\&= \cwt e^{-i\frac{3}{2}\omega}\brp{e^{i\frac{3}{2}\omega}-e^{-i\frac{3}{2}\omega}}
  \\&= \cwt e^{-i\frac{3}{2}\omega}\brs{2i\sin\brp{\frac{3}{2}\omega}}
  \\&= i\sqrt{2} e^{-i\frac{3}{2}\omega}\sin\brp{\frac{3}{2}\omega}
  \\
  \\
  &\abs{\Dh(\omega)}^2+\abs{\Dh(\omega+\pi)}^2
  \\&= \abs{\sqrt{2} e^{-i\frac{3}{2}\omega}\cos\brp{\frac{3}{2}\omega}}^2
     + \abs{i\sqrt{2} e^{-i\frac{3}{2}\omega}\sin\brp{\frac{3}{2}\omega}}^2
  \\&= 2\cos^2\brp{\frac{3}{2}\omega} + 2\sin^2\brp{\frac{3}{2}\omega}
  \\&= 2\brs{\cos^2\brp{\frac{3}{2}\omega} + \sin^2\brp{\frac{3}{2}\omega}}
  \\&= 2
    && \text{by \thme{half-angle formulas}}
    && \text{\xref{thm:trig_sq}}
  \\
  \\
  \Swphi(\omega)
    &\eqd \left.\Szphi(z)\right|_{z=e^{in\omega}}
    &&    \text{by definition of $\Swphi$}
    &&    \text{\xref{def:Swff}}
  \\&\eqd \left.\sum_{n\in\Z} \inprod{\fphi}{\opTrn^n\fphi} z^{-n}\right|_{z=e^{i\omega}}
    &&    \text{by definition of $\Sphi$}
    &&    \text{\xref{def:Szff}}
  \\&\eqd \left.\sum_{n\in\Z}\brs{ \int_\R {\frac{1}{3}\setind_\intco{0}{3}(x)} {\frac{1}{3}\setind_\intco{-n}{3-n}(x)} \dx} z^{-n}\right|_{z=e^{i\omega}}
    &&    \text{by definition of $\fphi$}
  \\&=    \left.\frac{1}{9}\sum_{n\in\Z}\brs{ \int_0^3 \setind_\intco{-n}{3-n}(x) \dx} z^{-n}\right|_{z=e^{i\omega}}
    &&    \text{by definition of $\setind$}
    &&    \text{\xref{def:setind}}
  \\&=    \left.\frac{1}{9}\brs{z^{-2}+2z^{-1}+3+2z+z^2}\right|_{z=e^{i\omega}}
  \\&=    \frac{1}{9}\brs{e^{-i2\omega}+2e^{-i\omega}+3+2e^{-i\omega}+e^{-i2\omega}}
  \\&=    \frac{1}{9}\brs{3 + 4\cos(\omega) + 2\cos(2\omega)}
\end{align*}
\end{proof}

%Besides the quadrature condition,
%other necessary conditions of an orthonormal wavelet system include
%a zero at $z=-1$ and partition of unity:
%--------------------------------------
\begin{theorem}
\label{thm:quad=>zero_unity}
%--------------------------------------
Let $\spO\eqd\mrasys$ be a wavelet system \xref{def:wavsys}.
\\Let $\Zh(z)$ be the \fncte{z-transform} \xref{def:opZ} of $ h$.
\thmbox{
  \mcom{\inprod{\fphi}{\opTrn^n\fphi} = \kdelta_n\qquad\scy\forall n\in\Z}
       {$\setn{\opTrn^n\fphi}$ is \prope{orthonormal}}
  \quad\implies\quad
  \brb{\begin{array}{F>{\ds}lclDD}
    1. & \sum_{m\in\Z} (-1)^m \fh_m    &=& 0  &  (\prope{low-pass filter})    & and \\
    2. & \sum_{m\in\Z} \opTrn^m \fphi &=& 1  &  (\prope{partition of unity}) & and \\
    3. & \Zh(z)\big|_{z=-1}           &=& 0  &  (\prope{zero at $z=-1$})
  \end{array}}
  }
\end{theorem}
\begin{proof}
\begin{enumerate}
  \item Proof for (1):
    \begin{align*}
       0 &= \brs{-2 + \abs{\Dh(\omega)}^2 + \abs{\Dh(\omega+\pi)}^2}_{\omega=0}
         && \text{by \prope{orthonormal quadrature condition}}
         && \text{\xref{lem:ows_quadcon}}
       \\&= -2 + \abs{\Dh(0)}^2 + \abs{\Dh(\pi)}^2
       \\&= -2 + \abs{\sqrt{2}}^2 + \abs{\Dh(\pi)}^2
         && \text{by \prope{admissibility condition}}
         && \text{\xref{thm:admiss}}
       \\&= 0 + \abs{\Dh(\pi)}^2
       \\&= \abs{\sum_{n\in\Z}h_n e^{-i\omega n}}^2_{\omega=\pi}
         && \text{by definition of $\Dh$}
         && \text{\xref{def:dtft}}
       \\&= \abs{\sum_{n\in\Z} (-1)^n\fh_n }^2
         && \text{because $e^{-i\pi n}=\brp{e^{-i\pi}}^n=(-1)^n$}
       \\\implies&\sum_{n\in\Z} (-1)^n\fh_n = \pm 0 = 0
    \end{align*}

  \item Proof for (2): by (1) and \prefp{thm:pun_zero}.

  \item Proof for (3): by (1) and \prefp{prop:dsp_zminone}.
\end{enumerate}
\end{proof}




%=======================================
\subsection{Sufficient conditions}
%=======================================
%The \prope{admissibility condition} \xref{thm:admiss} and
%\prope{quadrature condition} \xref{lem:ows_quadcon} give
%\hib{necessary} conditions for $\seqn{\fh_n}$ to generate an
%orthogonal scaling function.
%\pref{thm:admiss_quad==>tight}
%shows that these two conditions are
%not only \hib{necessary}, but also \hib{sufficient}
%for a partition of unity of the scaling function and
%a tight frame of the wavelets.
%%--------------------------------------
%\begin{theorem}
%\footnote{
%  \citerp{mallat89}{76}
%  }
%\index{quadrature condition}
%\index{admissibility condition}
%\index{suport}
%\index{scaling coefficients!sufficient conditions}
%%--------------------------------------
%Let $\mrasys$ be an \structe{MRA system} \xref{def:mrasys}.
%\thmbox{
%  \brbr{\begin{array}{F>{\ds}lDD}
%  1. & {\abs{\fh_n} = \mathcal{O}\frac{1}{1+n^2}}
%     & (\prope{exponential decay})
%     & and
%     \\
%  2. & {\sum_{n\in\Z} \fh_n = \sqrt{2}}
%     & (\prope{admissibility condition})
%     & and
%     \\
%  3. & {\sum_{m\in\Z}\fh_m h^\ast_{m-2n} = \kdelta_n}
%     & (\prope{quadrature condition})
%     & and
%     \\
%  4. & {\Dh(\omega)\ne 0 \quad\scy \forall \omega\in\intcc{-\frac{\pi}{2}}{\frac{\pi}{2}}}
%     & (\prope{low-pass response})
%  \end{array}}
%  \implies
%  %\qquad\implies\qquad
%  \brbl{\begin{array}{FlD}
%  1. & \mcom{\inprod{\fphi}{\opTrn^n\fphi}=\kdelta_n}
%            {$\setxZ{\opTrn^n\fphi}$ is \prope{orthonormal}}
%     & and
%     \\
%  2. & \mcom{\linspan\setxZ{\opTrn^n\fphi}=\spV_0}
%            {$\setxZ{\opTrn^n\fphi}$ is a \structe{basis} for $\spV_0$}
%  \end{array}}
%}
%\end{theorem}
%\begin{proof}
%\begin{enumerate}
%  \item The proof of this theorem uses the \hie{Lebesgue dominated convergence theorem}.\\
%    Reference: \cite{mallat89}.
%  \item Note: A generalization to this theorem is available in \\
%    \citerp{lawton91}{58}.
%\end{enumerate}
%\end{proof}


The \prope{admissibility condition} \xref{thm:admiss} and
\prope{quadrature condition} \xref{lem:ows_quadcon} give
\hib{necessary} conditions for $\seqn{\fh_n}$ to generate an orthogonal scaling function.
\pref{thm:ortho_sc} (next) shows that these two conditions are
not only \hib{necessary}, but also \hib{sufficient} conditions.
%--------------------------------------
\begin{theorem}
\footnote{
  \citerppg{mallat}{229}{234}{012466606X},
  \citerp{mallat89}{76},
  \citerppgc{mallat2009}{271}{276}{0123743702}{Theorem 7.2},
  \citer{meyer1992}
  }
\label{thm:ortho_sc}
\label{thm:h_ns}
\index{scaling coefficients!necessary and sufficient conditions}
%--------------------------------------
%Let $\mrasys$ be an \structe{MRA system} \xref{def:mrasys}.
Let $\opTrn$ be the \ope{translation operator} and $\opDil$ the \ope{dilation operator} \xref{def:opT}.
Let $\setxZ{\fh_n}$ be a sequence in $\spllF$ over some field $\F$ and with \ope{DTFT} \xref{def:dtft} $\Dh(\omega)$.
Let $\fphi(x)$ be a function in $\spLLR$ with \ope{Fourier Transform} \xref{def:opFT} $\Fphi(\omega)$
and defined in terms of $\seqn{\fh_n}$ as
%\\\indentx          $\ds\fphi(x)\eqd\sum_{n\in\Z}h_n\opDil\opTrn^n\fphi(x)$
\\\indentx          $\ds\Fphi(\omega)\eqd \cft\prod_{k=1}^{\infty}\cwt\Dh\brp{\frac{\omega}{2^k}}$
  \qquad and \qquad $\spV_j\eqd\linspan\setnZ{\opDil^j\opTrn^n\fphi(x)}$ ${\scy\forall j\in\Z}$.
\\
  Let an \structe{MRA} be defined as in \prefp{def:mra}.
\\
\thmbox{
  \brb{\begin{array}{F>{\ds}lD}
    (A). & \sum_{n\in\Z} \fh_{n}  =  \sqrt{2}                         & and
  \\(B). & \sum_{m\in\Z}\fh_m h^\ast_{m-2n} = \kdelta_n               & and
  %\\(C). & \Dh(\omega) = \Dh(\omega+2\pi)
  %       & and
  %\\(C). & $$\Dh(\omega)$ is continuously$
  %\\     & $differentiable about $0$$
  \\(C). & $$\Fphi(\omega)$ is \prope{continuous} at $0$$             & and
  \\(D). & \inf_{\omega\in\intcc{-\pi/2}{\pi/2}}\abs{\Dh(\omega)}>0  &
  \end{array}}
  \implies
  \brb{\begin{array}{M}
    $\mrasys$ is an\\
    \structe{orthonormal MRA system}\\
    \xref{def:mrasys}
  \end{array}}
  %\brb{\begin{array}{FMD}
  %  %(2). & $\fphi(x)$ is a \fncte{scaling function}
  %  (1). & $\seqjZ{\spV_j}$ is an \structe{MRA}\qquad\scs and \\
  %  (2). & $\setnZ{\opTrn^n\fphi(x)}$ is \prope{orthonormal} &
  %\end{array}}
}
\end{theorem}
\begin{proof}
\begin{enumerate}
  \item Proof that (A)$\impliedby$(right hypothesis): by \prefp{thm:admiss}.
  \item Proof that (B)$\impliedby$(right hypothesis): by \prefp{lem:ows_quadcon}.
  %\item Proof for  (C): by \prefp{prop:dtft_2pi}.
  %\item Proof for  (D):
  %\item Proof for  (E):
  \item Proof that $\seqn{\opTrn^n\fphi}$ is \prope{orthonormal}: \label{item:h_ns_ortho}
    \begin{enumerate}
      \item lemma: $\seqn{\opTrn^n\fphi}$ is \prope{orthonormal} $\iff$
            $\int_\R\abs{\Fphi(\omega)}^2 e^{in\omega}\dw = \kdelta_n$. Proof: \label{ilem:h_ns_intkdelta}
        \begin{align*}
          \kdelta_n
            &= \inprod{\fphi}{\opTrn^n\fphi}
            && \text{by left hypothesis}
            && \text{(for $\implies$ case)}
          \\&= \inprod{\opFT\fphi}{\opFT\opTrn^n\fphi}
            && \text{by \prope{unitary} property of $\opFT$}
            && \text{\xref{thm:ft_unitary}}
          \\&= \inprod{\Fphi(\omega)}{e^{-i\omega n}\Fphi(\omega)}
            && \text{by \prefp{cor:FTD}}
          \\&= \int_\R \Fphi(\omega) \brs{\Fphi(\omega) e^{-i\omega n}}^\ast \dw
            && \text{by definition of $\inprodn$ in $\spLLR$}
            && \text{\xref{def:inprod}}
          \\&= \int_\R \abs{\Fphi(\omega)}^2 e^{in\omega} \dw
          \\&= \kdelta_n
            && \text{by right hypothesis}
            && \text{(for $\impliedby$ case)}
        \end{align*}

      \item definition: \label{idef:h_ns_FphiN}
        $\ds\Fphi_\xN(\omega)\eqd \cft\prod_{k=1}^\xN\cwt\Dh\brp{\frac{\omega}{2^k}}\setind_\setA(\omega)$
        \qquad and where\qquad $\setA\eqd\intcc{-2^\xN\pi}{2^\xN\pi}$.

      %\item lemma: $\ds\lim_{\xN\to\infty}\Fphi_\xN(\omega)=\Fphi(\omega)$ \label{ilem:h_ns_FphiN2}
      %  \begin{align*}
      %    \lim_{\xN\to\infty}\Fphi_\xN(\omega)
      %      &\eqd \lim_{\xN\to\infty} \cft\prod_{k=1}^\xN\cwt\Dh\brp{\frac{\omega}{2^k}}\setind_\setA(\omega)
      %      &&    \text{by definition of $\Fphi_\xN(\omega)$ \xref{idef:h_ns_FphiN}}
      %    \\&=    \Fphi(\omega)
      %      && \text{by definition of $\Fphi(\omega)$}
      %  \end{align*}

      \item definition: \label{idef:h_ns_In}
        Let $\qquad\ds\fI_\xN(n) \eqd \int_0^{2^\xN\pi}e^{in\omega}\prod_{k=1}^{\xN-1}\half \abs{\Dh\brp{\frac{\omega}{2^k}}}^2 \dw\qquad$

      \item lemma: \label{ilem:h_ns_In}
        $\ds\fI_\xN(n)=2\pi\kdelta_n{\quad\scy\forall\xN\in\Zp}$.
        Proof:
        \begin{enumerate}
          \item $\xN=1$ case: \label{ilem:h_ns_In_N=1}
            \begin{align*}
              \fI_1(n)
                &\eqd \brlr{\int_0^{2^\xN\pi}e^{in\omega}\prod_{k=1}^{\xN-1}\half \abs{\Dh\brp{\frac{\omega}{2^k}}}^2 \dw}_{\xN=1}
                &&    \text{by definition of $\fI_\xN$}
                &&    \text{\xref{idef:h_ns_In}}
              \\&=    \int_0^{2\pi}e^{in\omega}\prod_{k=1}^0\half \abs{\Dh\brp{\frac{\omega}{2^k}}}^2 \dw
              \\&=    \int_0^{2\pi}e^{in\omega}\dw
              \\&=    2\pi\kdelta_n
            \end{align*}

          \item Proof that $\fI_{\xN+1}(n)=\fI_{\xN}(n)$:\label{ilem:h_ns_In_N}
            {\begin{align*}
              &\fI_{\xN+1}(n)
              \\&=  \int_{0}^{2^{\xN+1}\pi} e^{in\omega}\prod_{k=1}^\xN\half \abs{\Dh\brp{\frac{\omega}{2^k}}}^2\dw
                    \qquad\text{by definition of $\fI_\xN$ \xref{idef:h_ns_In}}
              \\&=  \int_{0}^{2^{\xN}\pi}            e^{in\omega}\prod_{k=1}^\xN\half \abs{\Dh\brp{\frac{\omega}{2^k}}}^2\dw
                   +\int_{\omega=2^{\xN}\pi}^{\omega=2^{\xN+1}\pi} e^{in\omega}\prod_{k=1}^\xN\half \abs{\Dh\brp{\frac{\omega}{2^k}}}^2\dw
              \\&=  \int_{0}^{2^{\xN}\pi}     e^{in\omega}      \prod_{k=1}^\xN\half \abs{\Dh\brp{\frac{\omega}{2^k}}}^2\dw
                 +  \int_{v=0}^{v=2^{\xN}\pi} e^{i(v-2^\xN\pi)n} \prod_{k=1}^\xN\half \abs{\Dh\brp{\frac{v+2^\xN\pi}{2^k}}}^2\dv
                    \\&\qquad\text{where $v\eqd\omega-2^\xN\pi\implies\omega=v+2^\xN\pi$}
              \\&=  \int_{0}^{2^{\xN}\pi} e^{in\omega}\half \abs{\Dh\brp{\frac{\omega}{2^\xN}}}^2
                    \prod_{k=1}^{\xN-1}\half \abs{\Dh\brp{\frac{\omega}{2^k}}}^2\dw
                \\&+\quad  \int_{v=0}^{v=2^{\xN}\pi} e^{in(v}\half \abs{\Dh\brp{\frac{v}{2^\xN}+\pi}}
                    \prod_{k=1}^{\xN-1}\half \abs{\Dh\brp{\frac{v}{2^k}+2^{\xN-k}\pi}}^2\dv
              \\&= \int_{0}^{2^\xN\pi} e^{in\omega}\half
                   \brs{\abs{\Dh\brp{\frac{\omega}{2^\xN}}}^2+\abs{\Dh\brp{\frac{\omega}{2^\xN}+\pi}}^2}
                   \prod_{k=1}^{\xN-1}\half \abs{\Dh\brp{\frac{\omega}{2^k}}}^2 \dw
                   \qquad\text{by \prefp{prop:dtft_2pi}}
              \\&= \int_{0}^{2^\xN\pi} e^{in\omega}\half
                   \brs{2}
                   \prod_{k=1}^{\xN-1}\half \abs{\Dh\brp{\frac{\omega}{2^k}}}^2  \dw
                   \qquad\text{by (B) and \thme{orthonormal quad. cond.} \xref{lem:oms_quadcon}}
              \\&= \fI_\xN(n)
                   \qquad\text{by definition of $\fI_\xN(n)$ \xref{idef:h_ns_In}}
            \end{align*}}

          \item Proof that $\fI_{\xN}(n)=2\pi\kdelta_n$:
            \begin{align*}
              \fI_{\xN}(n)
                &= \fI_1(n)
                && \text{by induction and \prefp{ilem:h_ns_In_N}}
              \\&= 2\pi\kdelta_n
                && \text{by \prefp{ilem:h_ns_In_N=1}}
            \end{align*}
        \end{enumerate}

      \item lemma: \label{ilem:h_ns_FphiN}
            $\ds\int_\R\abs{\Fphi_\xN(\omega)}^2e^{in\omega} \dw=\kdelta_n\quad{\scy\forall\xN\in\Zp}$.
            Proof:
        \begin{align*}
          &\int_\R\abs{\Fphi_\xN(\omega)}^2e^{in\omega} \dw
          \\&=  \int_\R e^{in\omega}  \abs{\cft\prod_{k=1}^\xN\cwt\Dh\brp{\frac{\omega}{2^k}}\setind_\intcc{-2^\xN\pi}{2^\xN\pi}}^2 \dw
            \qquad\text{by definition of $\Fphi_\xN(\omega)$ \xref{idef:h_ns_FphiN}}
          \\&=  \ootpi \int_{-2^\xN\pi}^{2^\xN\pi} e^{in\omega}  \prod_{k=1}^\xN\half \abs{\Dh\brp{\frac{\omega}{2^k}}}^2 \dw
          \\&=  \ootpi \int_{-2^\xN\pi}^{0} e^{in\omega}  \prod_{k=1}^\xN\half \abs{\Dh\brp{\frac{\omega}{2^k}}}^2 \dw
             +  \ootpi \int_{0}^{2^\xN\pi}  e^{in\omega}  \prod_{k=1}^\xN\half \abs{\Dh\brp{\frac{\omega}{2^k}}}^2 \dw
          \\&=  \ootpi \int_{v=0}^{v=2^\xN\pi} e^{i(v-2^\xN\pi) n}  \prod_{k=1}^\xN\half \abs{\Dh\brp{\frac{v}{2^k}-2^{\xN-k}\pi}}^2  \dv
            \\&\qquad+  \ootpi \int_{0}^{2^\xN\pi}  e^{in\omega}  \prod_{k=1}^\xN\half \abs{\Dh\brp{\frac{\omega}{2^k}}}^2 \dw
               \qquad\text{where $v\eqd\omega+2^\xN\pi\implies\omega=v-2^\xN\pi$}
          \\&=  \ootpi \int_{v=0}^{v=2^\xN\pi} e^{ivn} \half \abs{\Dh\brp{\frac{v}{2^k}-\pi}}^2\prod_{k=1}^{\xN-1}\half \abs{\Dh\brp{\frac{v}{2^k}}}^2 \dv
            \\&\qquad+  \ootpi \int_{0}^{2^\xN\pi}  e^{in\omega}  \half \abs{\Dh\brp{\frac{\omega}{2^\xN}}}^2\prod_{k=1}^{\xN-1}\half \abs{\Dh\brp{\frac{\omega}{2^k}}}^2  \dw
               \qquad\text{by \prefp{prop:dtft_2pi}}
          \\&=  \ootpi \int_{0}^{2^\xN\pi} e^{in\omega}  \half \abs{\Dh\brp{\frac{\omega}{2^\xN}-\pi+2\pi}}^2\prod_{k=1}^{\xN-1}\half \abs{\Dh\brp{\frac{\omega}{2^k}}}^2 \dw
            \\&\qquad+  \ootpi \int_{0}^{2^\xN\pi}  e^{in\omega}  \half \abs{\Dh\brp{\frac{\omega}{2^\xN}}}^2\prod_{k=1}^{\xN-1}\half \abs{\Dh\brp{\frac{\omega}{2^k}}}^2 \dw
               \qquad\text{by \prefp{prop:dtft_2pi}}
          \\&=  \ootpi \int_{0}^{2^\xN\pi} e^{in\omega}  \half
               \brs{\abs{\Dh\brp{\frac{\omega}{2^\xN}}}^2+\abs{\Dh\brp{\frac{\omega}{2^\xN}+\pi}}^2}
               \prod_{k=1}^{\xN-1}\half \abs{\Dh\brp{\frac{\omega}{2^k}}}^2  \dw
          \\&=  \ootpi \int_{0}^{2^\xN\pi} e^{in\omega}  \half
               \brs{2}
               \prod_{k=1}^{\xN-1}\half \abs{\Dh\brp{\frac{\omega}{2^k}}}^2 \dw
               \qquad\text{by (B) and \prefp{lem:ows_quadcon}}
          \\&=    \ootpi \fI_\xN(n)
               \qquad\text{by definition of $\fI_\xN(n)$ \xref{idef:h_ns_In}}
          \\&=   \kdelta_n
               \qquad\text{by \prefp{ilem:h_ns_In}}
          %\\&\implies\quad\text{$\seqn{\opTrn^n\fphi}$ is \prope{orthonormal}}
          %  \qquad\text{by \prefp{ilem:h_ns_intkdelta}}
        \end{align*}

      \item lemma: \label{ilem:h_ns_spLLR}
            $\Fphi(\omega)$ is in $\spLLR$ (alternatively, proof that $\abs{\Fphi(\omega)}^2$ is \prope{integrable}).
            Proof:
            \begin{align*}
              \int_\R \abs{\Fphi(\omega)}^2 \dw
                &=   \int_\R \lim_{\xN\to\infty}\abs{\Fphi_\xN(\omega)}^2\dw
                &&   \text{by definition of $\Fphi_\xN(\omega)$}
                &&   \text{\xref{idef:h_ns_FphiN}}
              \\&\le \lim_{\xN\to\infty}\int_\R \abs{\Fphi_\xN(\omega)}^2\dw
                &&   \text{by \thme{Fatou's Lemma}}
              \\&=   1
                &&   \text{by \prefp{ilem:h_ns_FphiN}}
              \\&\implies\quad\text{$\Fphi(\omega)$ is in $\spLLR$}
               %&&   \text{by definition of $\spLLR$ \xref{def:spLLR}}
                && \text{by definition of $\inprodn$ in $\spLLR$}
                && \text{\xref{def:spLLR}}
            \end{align*}

      \item lemma (\prope{upper-bound condition}): \label{ilem:h_ns_ubc}
            $\ds\exists C\ge1 \st \abs{\abs{\Fphi_\xN(\omega)}^2 e^{in\omega}}\le C \abs{\Fphi(\omega)}^2$.
            Proof:
        \begin{enumerate}
          %\item Proof that \prope{upper-bound condition} is equivalent to
          %  $\ds\abs{\Fphi_\xN(\omega)}^2\le C\abs{\Fphi(\omega)}^2$:
          %  \begin{align*}
          %    \abs{\abs{\Fphi_\xN(\omega)}^2 e^{in\omega}}
          %      &= \abs{\Fphi_\xN(\omega)}^2
          %    \\&\le C\abs{\Fphi(\omega)}^2.
          %  \end{align*}

          \item Proof for $\abs{\omega} >  2^\xN\pi$:  \label{item:h_ns_wge}
            \begin{align*}
              \abs{\abs{\Fphi_\xN(\omega)}^2 e^{in\omega}}
                &= \abs{\Fphi_\xN(\omega)}^2
                && \text{by definition of $\absn$}
              \\&= \abs{\cft\prod_{k=1}^\xN\cwt\Dh\brp{\frac{\omega}{2^k}}\setind_\intcc{-2^\xN\pi}{2^\xN\pi}(\omega)}^2
               %&& \text{by definition of $\Fphi_\xN(\omega)$ \xref{idef:h_ns_FphiN}}
                && \text{by def. of $\Fphi_\xN(\omega)$}
                && \text{\xref{idef:h_ns_FphiN}}
              \\&= 0 \qquad\forall\abs{\omega}>2^\xN\pi
                && \text{by definition of $\setind$}
                && \text{\xref{def:setind}}
              \\&\le\abs{\Fphi(\omega)}^2
                && \text{by definition of $\absn$}
            \end{align*}
          \item lemma: \label{ilem:h_ns_wequ}
            $\ds\abs{\Fphi(\omega)}^2\ge \frac{1}{2\pi C}\,{\scy\forall\abs{\omega}\le\pi}
            \quad\implies\quad
            \abs{\abs{\Fphi_\xN(\omega)}^2 e^{in\omega}}\le C \abs{\Fphi(\omega)}^2\,{\scy\forall\abs{\omega}\le2^\xN\pi}$
              \begin{align*}
                \abs{\Fphi(\omega)}^2
                  &= \abs{\cft\prod_{k=1}^\infty\cwt\Dh\brp{\frac{\omega}{2^k}}}^2
                  && \text{by definition of $\Fphi(\omega)$}
                \\&= \abs{\Fphi\brp{\frac{\omega}{2^\xN}}\prod_{k=1}^\xN\cwt\Dh\brp{\frac{\omega}{2^k}}}^2
                  && \text{by \prefp{lem:Fphi_infty}}
                \\&= 2\pi \abs{\Fphi\brp{\frac{\omega}{2^\xN}}}^2\abs{\cft\prod_{k=1}^\xN\cwt\Dh\brp{\frac{\omega}{2^k}}}^2
                \\&= 2\pi \abs{\Fphi\brp{\frac{\omega}{2^\xN}}}^2\abs{\Fphi_\xN(\omega)}^2
                  && \text{by def. of $\Fphi_\xN(\omega)$}
                  && \text{\xref{idef:h_ns_FphiN}}
                \\&\ge (2\pi)\frac{1}{2\pi C}\abs{\Fphi_\xN(\omega)}^2 %\qquad\forall\abs{\omega}<\pi
                  && \text{by left hypothesis}
                \\&= \frac{1}{C}\abs{\abs{\Fphi_\xN(\omega)}^2e^{in\omega}} %\qquad\forall\abs{\omega}<\pi
                \\\implies& \text{right hypothesis}
              \end{align*}
            %\begin{enumerate}
            %  \item Proof for ($\implies$) case:
            %    \begin{align*}
            %      \abs{\Fphi_\xN(\omega)}^2
            %        &= \abs{\abs{\Fphi_\xN(\omega)}^2 e^{in\omega}}
            %        && \text{by property of $\absn$}
            %      \\&\le C\abs{\Fphi(\omega)}^2
            %        && \text{by left hypothesis}
            %      \\&= C\abs{\cft\prod_{k=1}^\infty\cwt\Dh\brp{\frac{\omega}{2^k}}}^2
            %        && \text{by definition of $\Fphi(\omega)$}
            %      \\&= C\abs{\Fphi\brp{\frac{\omega}{2^\xN}}\prod_{k=1}^\xN\cwt\Dh\brp{\frac{\omega}{2^k}}}^2
            %        && \text{by \prefp{lem:Fphi_infty}}
            %      \\&= 2\pi C\abs{\Fphi\brp{\frac{\omega}{2^\xN}}}^2\abs{\cft\prod_{k=1}^\xN\cwt\Dh\brp{\frac{\omega}{2^k}}}^2
            %      \\&= 2\pi C\abs{\Fphi\brp{\frac{\omega}{2^\xN}}}^2\abs{\Fphi_\xN(\omega)}^2
            %        && \text{by def. of $\Fphi_\xN(\omega)$ \xref{idef:h_ns_FphiN}}
            %      \\\implies& \abs{\Fphi\brp{\frac{\omega}{2^\xN}}}^2\ge\frac{1}{2\pi C}
            %      \\\implies& \abs{\Fphi(\omega)}^2\ge\frac{1}{2\pi C}
            %        && \text{letting $\xN=0$}
            %      \\\implies& \abs{\Fphi(\omega)}^2>0
            %        && \text{because $C\ge1$}
            %    \end{align*}
            %
            %  \item Proof for ($\impliedby$) case:
            %    \begin{align*}
            %      1
            %        &\le C\abs{\Fphi(\omega)}^2
            %        && \text{by right hypothesis}
            %      \\&= C\abs{\cft\prod_{k=1}^\infty\cwt\Dh\brp{\frac{\omega}{2^k}}}^2
            %        && \text{by definition of $\Fphi(\omega)$}
            %      \\&= C\abs{\Fphi\brp{\frac{\omega}{2^\xN}}\prod_{k=1}^\xN\cwt\Dh\brp{\frac{\omega}{2^k}}}^2
            %        && \text{by \prefp{lem:Fphi_infty}}
            %      \\&= 2\pi C\abs{\Fphi\brp{\frac{\omega}{2^\xN}}}^2\abs{\cft\prod_{k=1}^\xN\cwt\Dh\brp{\frac{\omega}{2^k}}}^2
            %      \\&= 2\pi C\abs{\Fphi\brp{\frac{\omega}{2^\xN}}}^2\abs{\Fphi_\xN(\omega)}^2
            %        && \text{by def. of $\Fphi_\xN(\omega)$ \xref{idef:h_ns_FphiN}}
            %      \\&= 2\pi C\abs{\Fphi\brp{\frac{\omega}{2^\xN}}}^2\abs{\abs{\Fphi_\xN(\omega)}^2 e^{in\omega}}
            %        && \text{by def. of $\Fphi_\xN(\omega)$ \xref{idef:h_ns_FphiN}}
            %      \\\implies& \abs{\abs{\Fphi_\xN\brp{\frac{\omega}{2^\xN}}}^2 e^{in\omega}}\ge 2\pi C\abs{\Fphi\brp{\frac{\omega}{2^\xN}}}^2
            %    \end{align*}
            %\end{enumerate}

          \item Proof for $0\le\abs{\omega}<\varepsilon$:
            \begin{enumerate}
              \item lemma: \label{ilem:h_ns_0ew}
                    There exists $\varepsilon>0$ such that $\abs{\Dh(\omega)}^2 \le 2 = \abs{\Dh(0)}^2$.
                    Proof:
                \begin{align*}
                  \abs{\Dh(\omega)}^2
                    &\le \abs{\Dh(\omega)}^2 + \abs{\Dh(\omega+\pi)}^2
                  \\&= 2
                    && \text{by (B) and \xref{lem:oms_quadcon}}
                  \\&= \abs{\Dh(0)}^2
                    && \text{by (A) and \prefp{prop:tzf}}
                \end{align*}

               \item It follows from \pref{ilem:h_ns_0ew} that there exists $\varepsilon>0$ such that $\forall\abs{\omega}\le\varepsilon$
                 \begin{align*}
                   0 & \ge \log_e\brp{\half\abs{\Dh(\omega)}^2}
                     && \text{by \pref{ilem:h_ns_0ew}}
                   \\
                   -\abs{\omega} & \le \log_e\brp{\half\abs{\Dh(\omega)}^2} \le 0
                     %&& \text{because $\abs{\Dh(\omega)}^2\le 2$ and decreasing with $\abs{\omega}$ \xref{ilem:h_ns_0ew}}
                 \end{align*}

               %\item lemma: For any $\varepsilon>0$ $\ds\Fphi(\omega)\ge \ootpi e^{-\varepsilon}$:
               \item last step in proof for $0<\varepsilon<\abs{\omega}\le 2^\xN\pi$:\label{ilem:h_ns_0ew_last}
                 \begin{align*}
                   \abs{\Fphi(\omega}^2
                     &= \abs{\cft\prod_{k=1}^\infty\cwt\Dh\brp{\frac{\omega}{2^k}}}^2
                     && \text{by definition of $\Fphi(\omega)$}
                   \\&= \ootpi\prod_{k=1}^\infty\half\abs{\Dh\brp{\frac{\omega}{2^k}}}^2
                   \\&= \ootpi\exp\log_e\brs{\prod_{k=1}^\infty\half\abs{\Dh\brp{\frac{\omega}{2^k}}}^2}
                   \\&= \ootpi\exp\brb{\log_e\brs{\prod_{k=1}^\infty\half\abs{\Dh\brp{\frac{\omega}{2^k}}}^2}}
                   \\&= \ootpi\exp\brb{\sum_{k=1}^\infty\log_e\brs{\half\abs{\Dh\brp{\frac{\omega}{2^k}}}^2}}
                   \\&\ge \ootpi\exp\brb{\log_e\brs{\half\abs{\Dh\brp{\frac{\omega}{2^0}}}^2}}
                   \\&\ge \ootpi e^{-\abs{\omega}}
                   \\&\ge \ootpi e^{-\varepsilon}
                   \\&\implies \quad \text{upper bound condition}
                     && \text{by \prefp{ilem:h_ns_wequ}}
                 \end{align*}
            \end{enumerate}

          \item Proof for $\varepsilon\le\abs{\omega}\le \pi$: \label{item:h_ns_ew}
            \begin{enumerate}
              \item Choose $\xN$ such that $2^{-\xN}\pi < \varepsilon$ \label{item:h_ns_ew_N}
              \item Let $K=\ds\inf_{\omega\in\intcc{-\omega/2}{\omega/2}}\abs{\Dh(\omega)}>0$  \label{idef:h_ns_ew_K}
              \item Then
              \begin{align*}
                \abs{\Fphi(\omega)}^2
                  &= \abs{\cft\prod_{k=1}^\infty\cwt\Dh\brp{\frac{\omega}{2^k}}}^2
                  && \text{by definition of $\Fphi(\omega)$}
                \\&= \abs{\Fphi\brp{\frac{\omega}{2^\xN}}\prod_{k=1}^\xN\cwt\Dh\brp{\frac{\omega}{2^k}}}^2
                  && \text{by \prefp{lem:Fphi_infty}}
                \\&= 2\pi \abs{\Fphi\brp{\frac{\omega}{2^\xN}}}^2\abs{\cft\prod_{k=1}^\xN\cwt\Dh\brp{\frac{\omega}{2^k}}}^2
                \\&\ge 2\pi \ootpi e^{-\varepsilon} \ootpi\prod_{k=1}^\xN\half\abs{\Dh\brp{\frac{\omega}{2^k}}}^2
                  && \text{by \pref{item:h_ns_ew_N} and \prefp{ilem:h_ns_0ew_last}}
                \\&\ge \ootpi e^{-\varepsilon} \prod_{k=1}^\xN\half\abs{K}^2
                \\&=   \ootpi e^{-\varepsilon} \frac{K^{2\xN}}{2^\xN}
              \end{align*}
            \end{enumerate}
        \end{enumerate}

      \item Final step in proof for (2) (proof that $\seqn{\opTrn^n\fphi}$ is \prope{orthonormal}):
        \begin{align*}
          \int_\R\abs{\Fphi(\omega)e^{in\omega}}^2\dw
            &= \int_\R \lim_{\xN\to\infty}\abs{\Fphi_\xN(\omega)e^{in\omega}}^2\dw
          \\&= \lim_{\xN\to\infty}\int_\R \abs{\Fphi_\xN(\omega)e^{in\omega}}^2\dw
            && \text{by \pref{ilem:h_ns_ubc} and \thme{Dominated Convergence Theorem}}
          \\&= \kdelta_n
            && \text{by \prefp{ilem:h_ns_FphiN}}
          \\\implies& \text{(2)}
            && \text{by \prefp{ilem:h_ns_intkdelta}}
        \end{align*}
    \end{enumerate}


  \item Proof that $\seqn{\spV_j}$ is an \structe{MRA} \xref{def:mra}:
    \begin{enumerate}
      \item $\spV_j\eqd\linspan\seqn{\opTrn^n\opD^j\fphi}$, so we just need to prove that $\fphi$ is a \structe{scaling function} \xref{def:mra},
            and then the claim that $\seqn{\spV_j}$ is an \structe{MRA} follows from \prefp{thm:mra_rdc}.
      \item Proof that $\setn{\opTrn^n\fphi}$ is an \structe{MRA}:
        \begin{enumerate}
          \item Proof that $\seqn{\opTrn^n\fphi}$ is a \structe{Riesz basis}: by (1) (\prope{orthonormal} property, \prefp{item:h_ns_ortho})
                and by \prefp{thm:rieszAB}.
          \item Proof that  $\ds\exists \seqn{\fh_n} \st \fphi(x)=\sum_{n\in\Z}h_n\opDil\opTrn^n\fphi(x)$:
                by definition of $\Fphi(\omega)$, hypothesis (C), and \prefp{lem:Fphi_infty}.
          \item Proof that $\Fphi(\omega)$ is \prope{continuous} at $0$: by hypothesis (C).
          \item Proof that $\Fphi(0)\neq0$: by hypothesis (D).
        \end{enumerate}
    \end{enumerate}

  \item Proof that $\seqn{\fh_n}$ is a \structe{scaling coefficient sequence} \xref{def:hn}:
    \begin{align*}
      &\Fphi(\omega)\eqd \cft\prod_{k=1}^{\infty}\cwt\Dh\brp{\frac{\omega}{2^k}}
      && \text{by definition of $\Fphi(\omega)$}
    \\&\implies \fphi(x) = \sum_{n\in\Z} \fh_n \opDil \opTrn^n \fphi(x)
      && \text{by \prefp{lem:Fphi_infty}}
    \\&\implies \quad\text{$\seqn{\fh_n}$ is a \structe{scaling coefficient sequence}}
      && \text{by \xref{def:hn}}
    \end{align*}
\end{enumerate}
\end{proof}



%=======================================
\subsection{Orthonormal basis from Riesz basis}
%=======================================
The definition of a general wavelet system \xrefP{def:wavsys} requires an MRA.
The last condition of the definition of an MRA \xrefP{def:mra}
only requires that a \hie{Riesz basis} exist for
the scaling subspace $\spV_0$.
The MRA definition does \hie{not} require this basis to be \hie{orthonormal}.
However, the definition of an \hie{orthonormal wavelet system}
\xrefP{def:ows}
does require an orthonormal basis for $\spV_0$.
The good news is that if you have a Riesz basis for $\spV_0$
(which of course you do since you have an MRA),
then from this Riesz basis you can generate an
orthonormal basis.
One way of doing this is given by \pref{thm:riesz->ortho} (next). That is,
\pref{thm:riesz->ortho}
gives a sufficient condition for generating
an orthonormal basis $\setn{\opTrn^n\fphi}$ from a
Riesz basis $\setn{\opTrn^n\theta}$.

%--------------------------------------
\begin{theorem}[\thmd{Battle-Lemari{/'e} orthogonalization}]
\label{thm:riesz->ortho}
\index{Riesz basis}
\footnote{
  \citerp{vidakovic}{71},
  \citerp{mallat89}{72},
  \citerpg{mallat}{225}{012466606X}
  }
%--------------------------------------
\ragr
Let $\mrasys$ be an \structe{MRA system} \xref{def:mrasys}.
Let $\Fphi(\omega)$ be the Fourier Transform \xref{def:ft} of an arbitrary function $\fphi\in\spLLR$.
\thmbox{
  \brb{\begin{array}{F>{\ds}lD}
    1. & \text{$\setxZ{\opTrn^n\ftheta}$ is a \structe{Riesz basis} for $\spV_0$} & and \\ \\
    2. & \Fphi(\omega) \eqd \frac{\Ftheta(\omega)}{\sqrt{2\pi\sum_n\abs{\Ftheta(\omega+2\pi n)}^2}}
  \end{array}}
  \implies
  \brb{\begin{array}{M}%
    $\ds\setxZ{\opDil^m\opTrn^n\fphi}$\\
    is an \structe{orthonormal basis} for $\spV_m$%
    \end{array}
    }%
  }
\end{theorem}
\begin{proof}
\begin{enumerate}
\item Proof that $\setxZ{\opTrn^n\fphi}$ is orthonormal: by \prefp{thm:blo}  %\cittrp{mallat}{225}
%\begin{align*}
%  \Sphi(\omega)
%    &= 2\pi\sum_{n\in\Z} \abs{\Fphi(\omega+2\pi n)}^2
%    && \text{by \pref{thm:Swfg}}
%  \\&= 2\pi\sum_{n\in\Z} \abs{\frac{\Ftheta(\omega+2\pi n)}{\sqrt{\ds2\pi\sum_{m\in\Z}\abs{\Ftheta(\omega+2\pi(n-m))}^2}}}^2
%    && \text{by left hypothesis}
%  \\&= \sum_{n\in\Z} \abs{\frac{\Ftheta(\omega+2\pi n)}{\sqrt{\ds\sum_{m\in\Z}\abs{\Ftheta(\omega+2\pi m)}^2}}}^2
%  \\&= \sum_{n\in\Z} \abs{ \frac{1}{\sqrt{\ds\sum_{m\in\Z}\abs{\Ftheta(\omega+2\pi m)}^2}} }^2
%               \abs{\Ftheta(\omega+2\pi n)}^2
%  \\&= \frac{1}{\ds\sum_{m\in\Z}\abs{\Ftheta(\omega+2\pi m)}^2}
%        \sum_{n\in\Z} \abs{\Ftheta(\omega+2\pi n)}^2
%  \\&= 1
%  \\ \implies & \text{$\set{\opTrn^n\fphi}{n\in\Z}$ is orthonormal}
%    && \text{by \prefp{thm:Swfgo}}
%\end{align*}

\item Proof that $\setxZ{\opTrn^n\fphi}$ is a basis for $\spV_0$: by \prefp{lem:bases_opFT}.

\item Proof that $\setxZ{\opDil^m\opTrn^n\fphi}$ is a basis for $\spV_m$: by \prefp{thm:V0Vn}.
\end{enumerate}
\end{proof}



%=======================================
\section{Orthonormal wavelet systems}
%=======================================
%=======================================
\subsection{Subspace properties}
%=======================================
%If an \structe{MRA system} is \prope{orthonormal}, then its associated \structe{wavelet system} is \prope{orthonormal} as well (next).
%%---------------------------------------
%\begin{proposition}
%\label{prop:ortho_mrawav}
%\footnote{
%  \citerpgc{hernandez1996}{58}{0849382742}{proof of \scshape Proposition 2.13}
%  }
%%---------------------------------------
%Let $\wavsys$ be a \structe{wavelet system}.
%\propbox{
%  \mcom{\inprod{\fphi}{\opTrn^n\fphi}=\kdelta_n}{$\setn{\opTrn^n\fphi}$ is \prope{orthonormal}}
%  \qquad\implies\qquad
%  \mcom{\inprod{\fpsi}{\opTrn^n\fpsi}=\kdelta_n}{$\setn{\opTrn^n\fpsi}$ is \prope{orthonormal}}
%  }
%\end{proposition}
%\begin{proof}
%\begin{align*}
%  \Spsi(\omega)
%    &= 2\pi\sum_{n\in\Z} \abs{\fpsi\brp{\omega + 2\pi n}}^2
%    && \text{by \prefp{thm:Swff}}
%  \\&= 2\pi\sum_{n\in\Z}        \abs{\Fg\brp{\frac{\omega + 2\pi n}{2}}\fphi\brp{\frac{\omega + 2\pi n}{2}}}^2
%  \\&= 2\pi\sum_{n\in\Z}        \abs{\Fg\brp{\frac{\omega}{2} + \pi n}}^2
%                                \abs{\fphi\brp{\frac{\omega}{2} + \pi n}}^2
%  \\&= \mcom{2\pi\sum_{n\in\Ze} \abs{\Fg\brp{\frac{\omega}{2} + \pi n}}^2
%                                \abs{\fphi\brp{\frac{\omega}{2} + \pi n}}^2}{even terms}
%     + \mcom{2\pi\sum_{n\in\Zo} \abs{\Fg\brp{\frac{\omega}{2} + \pi n}}^2
%                                \abs{\fphi\brp{\frac{\omega}{2} + \pi n}}^2}{odd terms}
%  \\&= 2\pi\sum_{n\in\Z}        \abs{\Fg  \brp{\frac{\omega}{2} + \pi (2n)}}^2
%                                \abs{\fphi\brp{\frac{\omega}{2} + \pi (2n)}}^2
%     + 2\pi\sum_{n\in\Z}        \abs{\Fg  \brp{\frac{\omega}{2} + \pi (2n+1)}}^2
%                                \abs{\fphi\brp{\frac{\omega}{2} + \pi (2n+1)}}^2
%  \\&= 2\pi\sum_{n\in\Z}        \abs{\Fg  \brp{\frac{\omega}{2} + 2\pi n}}^2
%                                \abs{\fphi\brp{\frac{\omega}{2} + 2\pi n}}^2
%    +  2\pi\sum_{n\in\Z}        \abs{\Fg  \brp{\frac{\omega}{2} + 2\pi n + \pi}}^2
%                                \abs{\fphi\brp{\frac{\omega}{2} + 2\pi n + \pi}}^2
%  \\&= 2\pi\sum_{n\in\Z}        \abs{\Fg  \brp{\frac{\omega}{2}}}^2
%                                \abs{\fphi\brp{\frac{\omega}{2}}}^2
%     + 2\pi\sum_{n\in\Z}        \abs{\Fg  \brp{\frac{\omega}{2} + \pi}}^2
%                                \abs{\fphi\brp{\frac{\omega}{2} + \pi}}^2
%    && \text{by \prope{periodic} property of \ope{DTFT} \xref{prop:dtft_2pi}}
%  \\&=  2
%    && \text{by \prefp{lem:ows_quadcon}}
%  \\\iff& \inprod{\fpsi}{\opTrn^n\fpsi} = \kdelta_n
%    && \text{by \prefp{thm:Swfgo}}
%\end{align*}
%\end{proof}
%
\prefpp{def:ows} requires that the scaling subspace $\spV_0$ and
the wavelet subspace $\spW_0$ be orthogonal.
However, \pref{prop:VperpW} (next) shows that
this constraint implies that for all scales $n$, $\spV_n$ is orthogonal to $\spW_n$:
%---------------------------------------
\begin{proposition}
\label{prop:VperpW}
\footnote{
  \citerpp{jawerth}{10}{11}
  }
%---------------------------------------
Let $\wavsys$ be a wavelet system \xref{def:wavsys}.
\propbox{
  \mcom{\spV_0 \perp \spW_0}{orthogonal subspaces}
  \quad\implies\quad
  \mcom{\spV_{n} \perp \spW_{n} \quad \forall n\in\Z}
       {orthogonal subspaces}
  }
\end{proposition}
\begin{proof}
  \begin{align*}
    0
      &= \inprod{\fphi(x)}{\fpsi(x)}
      && \text{because $\spV_0 \perp \spW_0$}
    \\&= \inprod{\fphi(2^ju)}{\fpsi(2^ju)}
      && \text{where } u=2^{-j}x \iff x=2^ju
    \\ \implies \hs{4ex} \spV_j &\perp \spW_j
  \end{align*}
\end{proof}

%=======================================
\subsection{Wavelet coefficients from wavelet and scaling functions}
%=======================================
In an \structe{orthonormal MRA system} \xref{def:oms}, the scaling coefficients $\seqn{\fh_n}$
can be computed from $\fphi$ \xref{prop:phi->h}.
In an \structe{orthonormal wavelet system} \xref{def:ows}, the wavelet coefficients $\seqn{\fg_n}$
can be computed from $\fphi$ and $\fpsi$ (next proposition).
%--------------------------------------
\begin{proposition}
\label{prop:psiphi->g}
%--------------------------------------
Let $\wavsys$ be a \structe{wavelet system} \xref{def:wavsys}.
\propbox{
  \mcom{{\inprod{\fphi}{\opTrn^n\fphi}=\kdelta_n}}{$\setn{\opTrn^n\fphi}$ is \prope{orthonormal}}  %{$\setn{\fphi_n}$ is \prope{orthonormal} in $\spLLR$}
  \qquad\implies\qquad
  {\fg_n  = \ds \inprod{\fpsi}{\opDil\opTrn^n\fphi}}
  \qquad\scy\forall n\in\Z
  }
\end{proposition}
\begin{proof}
\begin{align*}
   \inprod{\fpsi}{\opDil\opTrn^n\fphi}
     &= \inprod{\sum_{m\in\Z}g_m\opDil\opTrn^m \fphi}{\opDil\opTrn^n\fphi}
     && \text{by \prefp{thm:g->psi}}
   \\&= \sum_{m\in\Z}g_m\inprod{\opDil\opTrn^m \fphi}{\opDil\opTrn^n\fphi}
   \\&= \sum_{m\in\Z}g_m \inprod{\sqrt{2}\fphi(2x-m)}{\sqrt{2}\fphi(2x-n)}
     && \text{by definitions of $\opT$ and $\opD$}
     && \text{\xref{def:opTD}}
   \\&= \sum_{m\in\Z}g_m \int_{x\in\R} \sqrt{2}\fphi(2x-m)\sqrt{2}\fphi^\ast(2x-n) \dx
     && \text{by definition of $\inprodn$ in $\spLLR$}
     && \text{\xref{def:spLLR}}
   \\&= \sum_{m\in\Z}g_m \int_{u\in\R} \sqrt{2}\fphi(u-m)\sqrt{2}\fphi^\ast(u-n) \half \du
     && u=2x\qquad \du=2\dx
   \\&= \sum_{m\in\Z}g_m \int_{u\in\R} \fphi(u-m)\fphi^\ast(u-n) \du
   \\&= \sum_{m\in\Z}g_m \int_{u\in\R} \fphi(u-m)\fphi^\ast(u-n) \du
   \\&= \sum_{m\in\Z}g_m \int_{v\in\R} \fphi(v)\fphi^\ast(v-(n-m)) \dv
     && \text{where $v\eqd u-m$ $\implies$ $u=v+m$}
   \\&= \sum_{m\in\Z}g_m \inprod{\fphi}{\opTrn^{n-m}\fphi}
     && \text{by definition of $\inprodn$ in $\spLLR$}
     && \text{\xref{def:spLLR}}
   \\&= \sum_{m\in\Z}g_m \kdelta_{m-n}
     && \text{by left hypothesis}
   \\&=\fg_n
\end{align*}
\end{proof}



%-------------------------------------
\begin{example}[\exm{Haar wavelet}]
\index{Haar scaling function}
%-------------------------------------
%Let a function $\fphi$ be defined as follows:
\exbox{
  \begin{array}{Mclm{37mm}}
    Let $\fphi(x)$ &\eqd&
      \brbl{\begin{array}{lM}
         1  & for $x\in\intco{0}{1}$  \\
         0  & otherwise
      \end{array}}
    &\includegraphics{../common/math/graphics/pdfs/d1_phi.pdf}
  \\
    and $\fpsi(x)$ &\eqd&
      \brbl{\begin{array}{rM}
          1  & for $x\in\intco{0}{\half }$  \\
         -1  & for $x\in\intco{\half }{1}$  \\
          0  & otherwise.
      \end{array}}
    &\includegraphics{../common/math/graphics/pdfs/d1_psi.pdf}
  \\
    Then, $\fg_n$ &=&
      \brbl{\begin{array}{rM}
         \cwt  & for $n=0$\\
        -\cwt  & for $n=1$\\
         0                   & otherwise.
      \end{array}}
    &\includegraphics{../common/math/graphics/pdfs/d1_g.pdf}
\end{array}}
\end{example}
\begin{proof}
\begin{align*}
 \fg_{n}
    &= \inprod{\fpsi}{\opDil\opTrn^n\fphi}
    && \text{by \prefp{prop:psiphi->g}}
  \\&= \inprod{\setind_\intco{0}{1}(x)-2\setind_\intco{\half }{1}(x)}{\sqrt{2}\setind_{[0,1)}(2x-n)}
    && \text{by definition of $\fphi$ and $\fpsi$}
  \\&= \mathrlap{%
          \sqrt{2} \inprod{\setind_\intco{0}{1}(x)}{\setind_{[0,1)}(2x-n)} 
       - 2\sqrt{2}\inprod{\setind_\intco{\half }{1}(x)}{\setind_{[0,1)}(2x-n)}
       \quad\text{by property of $\inprodn$}
       }
  \\&=\fh_n - 2\sqrt{2}\int_\R \setind_\intco{\half }{1}(x) \setind_{[0,1)}(2x-n) \dx
    && \text{by \prefp{ex:haarphi}}
  \\&=\fh_n - 2\sqrt{2}\int_\intco{\half }{1}(x) \setind_{[0,1)}(2x-n) \dx
    && \text{by definition of $\setind$}
    && \text{\xref{def:setind}}
  \\&=\fh_n - 2\sqrt{2}\half \kdelta_{n-1}
    && \text{by definition of $\setind$}
    && \text{\xref{def:setind}}
  \\&= \brbl{\begin{array}{lcrM}
         \cwt - 0                   &=&  \cwt    & for $n=0$ \\
         \cwt - \frac{2\sqrt{2}}{2} &=& -\cwt    & for $n=1$ \\
         0                                        &=&  0                     & otherwise
       \end{array}}
    && \text{by definition of $\kdelta$}
    && \text{\xref{def:kdelta}}
\end{align*}
\end{proof}


%=======================================
\subsection{Necessary conditions}
%=======================================
%--------------------------------------
\begin{lemma}[\lemd{orthonormal quadrature condition}s]
\footnote{
  \citerp{chui}{135},
  \citerppg{mallat}{229}{238}{012466606X},
  \citerp{goswami}{110},
  \citerp{vaidyanathan1990}{65}
  }
\label{lem:ows_quadcon}
\index{quadrature condition}
%--------------------------------------
Let $\spO\eqd\wavsys$ be a wavelet system \xref{def:wavsys}.
Let $\Dh(\omega)$ and $\Fg(\omega)$ be the \structe{discrete time Fourier transform}s \xref{def:dtft}
of $\seqn{\fh_n}$ and $\seqn{\fg_n}$, respectively.
\lembox{\begin{array}{M}
  $\mcom{\brb{\begin{array}{FrclD}
    1. & \inprod{\fphi}{\opTrn^n\fphi} &=& \kdelta_n & and \\
    2. & \inprod{\fpsi}{\opTrn^n\fpsi} &=& \kdelta_n & and \\
    3. & \inprod{\fphi}{\opTrn^n\fpsi} &=& 0         &
  \end{array}}}
  {$\spO$ is \prope{orthonormal} \xref{def:ows}}
  \qquad\implies\qquad
  \mcom{\brb{\begin{array}{F>{\ds}rclD}
    A. & \sum_{m\in\Z}\fh_m\fh_{m-2n}^\ast &=& \kdelta_n & and \\
    B. & \sum_{m\in\Z}\fg_m\fg_{m-2n}^\ast &=& \kdelta_n & and \\
    C. & \sum_{m\in\Z}\fh_m\fg_{m-2n}^\ast &=& 0         &
  \end{array}}}{\fncte{orthonormal quadrature condition}s in ``time"}
  \qquad\scy\forall n\in\Z$
  \\\indentx$\iff\qquad
  \mcom{\brb{\begin{array}{F>{\ds}rc>{\ds}lclD}
   a. & \abs{\Dh(\omega)}^2  &+& \abs{\Dh(\omega+\pi)}^2        &=& 2   &and\\
   b. & \abs{\Fg(\omega)}^2  &+& \abs{\Fg(\omega+\pi)}^2        &=& 2   &and\\
   c. & \Dh(\omega)\Fg^\ast(\omega) &+& \Dh(\omega +\pi) \Fg^\ast(\omega +\pi)&=& 0 &
  \end{array}}}{\fncte{orthonormal quadrature condition}s in ``frequency"}$
\end{array}}
\end{lemma}
\begin{proof}
\begin{enumerate}
  \item Proof that (1,2,3)$\implies$(A): by \prefp{lem:oms_quadcon}.
  \item Proof that (1,2,3)$\implies$(B):
    \begin{align*}
      \sum_{m\in\Z}g_m\fg_{m-2n}^\ast
        &= \sum_{m\in\Z}g_m \sum_{k\in\Z}\fg_k^\ast \kdelta_{k-m+2n}
        && \text{by definition of $\kdelta_n$}
        && \text{\xref{def:kdelta}}
      \\&= \sum_{m\in\Z}g_m \sum_{k\in\Z}\fg_k^\ast\inprod{\fphi }{\opTrn^{2n-m+k} \fphi}
        && \text{by hypothesis (1)}
      \\&= \inprod{\fpsi}{\opTrn^n \fpsi}
        && \text{by \thme{quadrature conditions in ``time"}}
        && \text{\xref{thm:wavsys_quadcon}}
      \\&= \kdelta_n
        && \text{by hypothesis (2)}
    \end{align*}

  \item Proof that (1,2,3)$\implies$(C):
    \begin{align*}
      \sum_{m\in\Z}h_m\fg_{m-2n}^\ast
        &= \sum_{m\in\Z}h_m \sum_{k\in\Z}\fg_k^\ast \kdelta_{k-m+2n}
        && \text{by definition of $\kdelta_n$}
        && \text{\xref{def:kdelta}}
      \\&= \sum_{m\in\Z}h_m \sum_{k\in\Z}\fg_k^\ast\inprod{\fphi }{\opTrn^{2n-m+k} \fphi}
        && \text{by hypothesis (1)}
      \\&= \inprod{\fphi}{\opTrn^n\fpsi}
        && \text{by \thme{quadrature conditions in ``time"}}
        && \text{\xref{thm:wavsys_quadcon}}
      \\&= 0
        && \text{by hypothesis (3)}
    \end{align*}

  \item Proof that (A,B,C)$\iff$(a,b,c): by \prefp{lem:ows_quadcon}.
\end{enumerate}
%\begin{enumerate}
%  \item Proof for (1): by \prefp{lem:oms_quadcon}.
%
%  \item Proof for (2):
%    \begin{align*}
%      2
%        &= 2\cancelto{1}{\Spsi(2\omega)}
%        && \text{by \prefp{thm:Swfgo}}
%      \\&= \left|\Fg(\omega    )\right|^2  \cancelto{1}{\Sphi(\omega    )}
%          + \left|\Fg(\omega+\pi)\right|^2 \cancelto{1}{\Sphi(\omega+\pi)}
%        && \text{by \prefp{thm:gen_quadcon}}
%      \\&= \left|\Fg(\omega    )\right|^2
%          + \left|\Fg(\omega+\pi)\right|^2
%        && \text{by \prefp{thm:Swfgo}}
%    \end{align*}
%
%  \item Proof for (3):
%    \begin{align*}
%      0
%        &= \Shs(2\omega)
%        && \text{by \prefp{thm:Swfgo}}
%      \\&=  \Dh(\omega     ) \Fg^\ast(\omega     )\cancelto{1}{\Sphi(\omega)}
%          + \Dh(\omega +\pi) \Fg^\ast(\omega +\pi)\cancelto{1}{\Sphi(\omega+\pi)}
%        && \text{by \prefp{thm:gen_quadcon}}
%      \\&= \Dh(\omega)\Fg^\ast(\omega) + \Dh(\omega+\pi) \Fg^\ast(\omega +\pi)
%        && \text{by \prefp{thm:Swfgo}}
%    \end{align*}
%\end{enumerate}
\end{proof}

%--------------------------------------
\begin{proposition}
\footnote{
  \citerppgc{wojtaszczyk1997}{32}{36}{0521578949}{Proposition 2.18},
  \citerppgc{hernandez1996}{55}{57}{0849382742}{Lemma 2.11}
  }
\label{prop:ortho_fW0}
%--------------------------------------
Let $\spO\eqd\wavsys$ be a \structe{wavelet system} \xref{def:wavsys}.
Let $\Fv(\omega)$ be a \prope{$2\pi$ periodic} and \prope{measurable} function in $\spLLR$ \xref{def:spLLR}.
\propboxt{
  Let $\spV_0\orthog\spW_0$ and $\setn{\opTrn^n\fphi}$ be \prope{orthonormal}. Then
  \\\indentx
  $\mcom{\ff\in\spW_0}{(A)}
  \iff
  \brb{\begin{array}{FMD}
    (1). & $\ff \in \spV_1$ & and \\
    (2). & There exists a \prope{measurable} function $\ds\Fv(\omega)\in\spLLR$ such that
       \\& $\ds\Fv(\omega)= \Fv(\omega+2\pi)$ (\prope{$2\pi$ periodic}) and
       \\& $\Ff(\omega)= \cwt\ds e^{i\frac{\omega}{2}}\Fv(\omega)\Dh^\ast\brp{\frac{\omega}{2}+\pi}\Fphi\brp{\frac{\omega}{2}}$
  \end{array}}$
  }
\end{proposition}
\begin{proof}
\begin{enumerate}
  \item Proof that (A)$\implies$(1):
    \begin{align*}
      \ff\in\spW_0
        &\implies \ff\in\spW_0\adds\spV_0
        && \text{by definition of \ope{Minkowski addition} $\adds$}
        && \text{\xref{def:mph_add}}
      \\&\iff     \ff\in\spV_1
        && \text{because $\spV_1=\spV_0\adds\spW_0$}
        && \text{\xref{def:seqWn}}
    \end{align*}

  \item Proof that (A)$\implies$(2):
    \begin{enumerate}
      \item There exists a \prope{$2\pi$ periodic} function $\Fa(\omega)$ such that \label{item:ortho_fW0_faphi}
            $\Ff(\omega)= \Fa\brp{\frac{\omega}{2}}\Fphi\brp{\frac{\omega}{2}}$. Proof:
            \begin{align*}
              \Ff(\omega)
                &= \opFT \ff(x)
                && \text{by definition of $\opFT$}
                && \text{\xref{def:opFT}}
              \\&= \opFT \sum_{n\in\Z} a_n \opDil\opTrn^n\fphi
                && \text{by (A),\, $\ff\in\spW_0\subseteq\spV_1$}
              \\&= \sum_{n\in\Z} a_n \opFT \opDil\opTrn^n\fphi
              \\&= \sum_{n\in\Z} a_n e^{i\frac{\omega}{2}}\Fphi\brp{\frac{\omega}{2}}
                && \text{by \prefp{prop:FTDf}}
              \\&= \Fa\brp{\frac{\omega}{2}}\Fphi\brp{\frac{\omega}{2}}
                && \text{by definition of \ope{DTFT}}
                && \text{\xref{def:dtft}}
            \end{align*}

      \item \label{item:ortho_fW0_ahah}
            $\Fa\brp{\omega}\Dh^\ast\brp{\omega} + \Fa\brp{\omega+\pi}\Dh^\ast\brp{\omega+\pi} = 0$. Proof:
        \begin{align*}
          0
            &= \Sfphi(\omega)
            \qquad\text{by $\spV_0\orthog\spW_0$ hypothesis and \prefp{thm:Swfgo}}
          \\&= \sum_{n\in\Z} \Ff\brp{\omega+2\pi n} \Fphi^\ast\brp{\omega+2\pi n}
            \qquad\text{by \prefp{thm:Swfg}}
          \\&= \sum_{n\in\Z} \mcom{\Fa     \brp{\frac{\omega+2\pi n}{2}}\Fphi     \brp{\frac{\omega+2\pi n}{2}}}{$\Ff       (\omega+2\pi n)$}\,
                             \mcom{\Dh^\ast\brp{\frac{\omega+2\pi n}{2}}\Fphi^\ast\brp{\frac{\omega+2\pi n}{2}}}{$\Fphi^\ast(\omega+2\pi n)$}
            \qquad\text{by \prefp{item:ortho_fW0_faphi}}
          \\&= \mcom{
               \sum_{n\in\Z} \Fa\brp{\frac{\omega}{2}+2\pi n}\Dh^\ast\brp{\frac{\omega}{2}+2\pi n}\abs{\Fphi^\ast\brp{\frac{\omega}{2}+2\pi n}}^2
               }{even terms}
            \\&\qquad+ \mcom{
               \sum_{n\in\Z} \Fa\brp{\frac{\omega}{2}+2\pi n+\pi}\Dh^\ast\brp{\frac{\omega}{2}+2\pi n+\pi}\abs{\Fphi^\ast\brp{\frac{\omega}{2}+2\pi n+\pi}}^2
               }{odd terms}
          \\&= \sum_{n\in\Z} \Fa\brp{\frac{\omega}{2}}    \Dh^\ast\brp{\frac{\omega}{2}}    \abs{\Fphi^\ast\brp{\frac{\omega}{2}+2\pi n}}^2
             + \sum_{n\in\Z} \Fa\brp{\frac{\omega}{2}+\pi}\Dh^\ast\brp{\frac{\omega}{2}+\pi}\abs{\Fphi^\ast\brp{\frac{\omega}{2}+2\pi n+\pi}}^2
          \\&= \Fa\brp{\frac{\omega}{2}}    \Dh^\ast\brp{\frac{\omega}{2}}    \Sphi\brp{\frac{\omega}{2}}
             + \Fa\brp{\frac{\omega}{2}+\pi}\Dh^\ast\brp{\frac{\omega}{2}+\pi}\Sphi\brp{\frac{\omega}{2}+\pi}
          \\&= \Fa\brp{\frac{\omega}{2}}    \Dh^\ast\brp{\frac{\omega}{2}}
             + \Fa\brp{\frac{\omega}{2}+\pi}\Dh^\ast\brp{\frac{\omega}{2}+\pi}
            \qquad\text{by \prope{orthonormality} and \prefp{thm:Swfgo}}
          \\&\implies \Fa\brp{\omega}\Dh^\ast\brp{\omega} + \Fa\brp{\omega+\pi}\Dh^\ast\brp{\omega+\pi} = 0
        \end{align*}

      \item The vectors
          $\opairs{\Fa\brp{\omega}}{\Fa\brp{\omega+\pi}}$ and $\opairs{\Dh\brp{\omega}}{\Dh\brp{\omega+\pi}}$
          are \prope{orthogonal}. Proof: \label{item:ortho_fW0_ortho}
        \begin{align*}
          \brs{\begin{array}{c}
            \Dh\brp{\omega}\\
            \Dh\brp{\omega+\pi}
          \end{array}}^H
          \brs{\begin{array}{c}
            \Fa\brp{\omega}\\
            \Fa\brp{\omega+\pi}
          \end{array}}
          &=
          \brs{\begin{array}{cc}
            \Dh^\ast\brp{\omega} &
            \Dh^\ast\brp{\omega+\pi}
          \end{array}}
          \brs{\begin{array}{c}
            \Fa\brp{\omega}\\
            \Fa\brp{\omega+\pi}
          \end{array}}
        \\&= \Fa\brp{\omega}\Dh^\ast\brp{\omega} + \Fa\brp{\omega+\pi}\Dh^\ast\brp{\omega+\pi}
        \\&= 0
          && \text{by \prefp{item:ortho_fW0_ahah}}
        \end{align*}

      \item There exists a \prope{$2\pi$ periodic} function $\Fb(\omega)$ such that \label{item:ortho_fW0_aabhh}
          $\opairs{\Fa\brp{\omega}}{\Fa\brp{\omega+\pi}} = \Fb(\omega)\opairs{\Dh^\ast\brp{\omega+\pi}}{-\Dh^\ast\brp{\omega}}$. Proof:
        Because the two vectors are \prope{orthogonal} \xref{item:ortho_fW0_ortho},
        if one vector is rotated by $90^\circ$, then for fixed $\omega$ it is a scalar multiple of the other\ldots
        \begin{align*}
          \opairs{\Fa\brp{\omega}}{\Fa\brp{\omega+\pi}}
            &= \Fb(\omega)\opairs{\Dh^\ast\brp{\omega}}{\Dh^\ast\brp{\omega+\pi}}
               \mcom{\brs{\begin{array}{rr}
                 \cos90^\circ & -\sin90^\circ\\
                 \sin90^\circ &  \cos90^\circ
               \end{array}}}{\ope{rotation matrix}}
            && \text{by \prefp{item:ortho_fW0_ortho}}
          \\&= \Fb(\omega)\opairs{\Dh^\ast\brp{\omega}}{\Dh^\ast\brp{\omega+\pi}}
               \brs{\begin{array}{rr}
                 0      & -1     \\
                 1      &  0
               \end{array}}
          \\&= \Fb(\omega)\opairs{\Dh^\ast\brp{\omega+\pi}}{-\Dh^\ast\brp{\omega}}
        \end{align*}

      \item By \pref{item:ortho_fW0_aabhh},
            $\opairs{\Fa\brp{\omega+\pi}}{\Fa\brp{\omega}} = \Fb(\omega+\pi)\opairs{\Dh^\ast\brp{\omega}}{-\Dh^\ast\brp{\omega+\pi}}$.
            \label{item:ortho_fW0_aabhh_pi}
            Proof:
        \begin{align*}
          \opairs{\Fa\brp{\omega+\pi}}{\Fa\brp{\omega}}
            &= \opairs{\Fa\brp{\omega+\pi}}{\Fa\brp{\omega+\pi+\pi}}
          \\&= \Fb(\omega+\pi)\opairs{\Dh^\ast\brp{\omega+\pi+\pi}}{-\Dh^\ast\brp{\omega+\pi}}
            && \text{by \prefp{item:ortho_fW0_aabhh}}
          \\&= \Fb(\omega+\pi)\opairs{\Dh^\ast\brp{\omega}}{-\Dh^\ast\brp{\omega+\pi}}
        \end{align*}

      \item By \pref{item:ortho_fW0_aabhh} and \pref{item:ortho_fW0_aabhh_pi},\quad $\Fb(\omega)=-\Fb(\omega+\pi)$. Proof:
            \label{item:ortho_fW0_b-b}
        \\$\ds\begin{array}{c}
          \underbrace{\begin{array}{rclD}
            \Fa(\omega) &=&  \Fb(\omega)\Dh^\ast(\omega+\pi)     & by \pref{item:ortho_fW0_aabhh}\\
            \Fa(\omega) &=& -\Fb(\omega+\pi)\Dh^\ast(\omega+\pi) & by \pref{item:ortho_fW0_aabhh_pi}
          \end{array}}
          \\\Downarrow
          \\\Fb(\omega)=-\Fb(\omega+\pi)
        \end{array}$
        \hspace{10mm}
        $\ds\begin{array}{c}
          \underbrace{\begin{array}{rclD}
            \Fa(\omega+\pi) &=& -\Fb(\omega)\Dh^\ast(\omega)     & by \pref{item:ortho_fW0_aabhh}\\
            \Fa(\omega+\pi) &=&  \Fb(\omega+\pi)\Dh^\ast(\omega) & by \pref{item:ortho_fW0_aabhh_pi}
          \end{array}}
          \\\Downarrow
          \\\Fb(\omega)=-\Fb(\omega+\pi)
        \end{array}$

      \item Note that $\Fb(\omega)=-\Fb(\omega+\pi)$ $\iff$ $e^{-i(\omega+\pi)}\Fb(\omega+\pi)=e^{-i(\omega)}\Fb(\omega)$. Proof:
        \begin{align*}
          -\Fb(\omega+\pi)
            &= -e^{i(\omega+\pi)}e^{-i(\omega+\pi)}\Fb(\omega+\pi)
          \\&= -e^{i(\omega+\pi)}e^{-i\omega}\Fb(\omega)
            && \text{by right hypothesis}
          \\&= e^{-i\pi}e^{i(\omega+\pi)}e^{-i\omega}\Fb(\omega)
          \\&= e^{i\omega}e^{-i\omega}\Fb(\omega)
          \\&= \Fb(\omega)
          \\
          \\
          e^{-i(\omega+\pi)}\Fb(\omega+\pi)
            &= -e^{-i\omega}\Fb(\omega+\pi)
          \\&= e^{-i\omega}\Fb(\omega)
            && \text{by left hypothesis}
        \end{align*}

      \item Let $\Fv(\omega)\eqd \sqrt{2}e^{-i\frac{\omega}{2}}\Fb\brp{\frac{\omega}{2}}$. Then (2) is true \ldots
        \begin{enumerate}
          \item $\Fv(\omega)$ is \prope{$2\pi$ periodic}. Proof:
            \begin{align*}
              \Fv(\omega+2\pi)
                &\eqd \sqrt{2}e^{-i\frac{\omega}{2}}\Fb\brp{\frac{\omega}{2}}
                && \text{by definition of $\Fv(\omega)$}
              \\&= \sqrt{2}e^{-i\frac{\omega+2\pi}{2}}\Fb\brp{\frac{\omega+2\pi}{2}}
              \\&= \sqrt{2}e^{-i\frac{\omega}{2}+\pi}\Fb\brp{\frac{\omega}{2}+\pi}
              \\&= -\sqrt{2}e^{-i\frac{\omega}{2}}\Fb\brp{\frac{\omega}{2}+\pi}
              \\&= -\sqrt{2}e^{-i\frac{\omega}{2}}\brs{-\Fb\brp{\frac{\omega}{2}}}
                && \text{by \prefp{item:ortho_fW0_b-b}}
              \\&= \sqrt{2}e^{-i\frac{\omega}{2}}\Fb\brp{\frac{\omega}{2}}
                && \text{by \prefp{item:ortho_fW0_b-b}}
              \\&\eqd \Fv(\omega)
                && \text{by definition of $\Fv(\omega)$}
            \end{align*}
          \item $\Ff(\omega)= \cwt e^{i\frac{\omega}{2}}\Fv(\omega)\Dh^\ast\brp{\frac{\omega}{2}+\pi}\Fphi\brp{\frac{\omega}{2}}$. Proof:
            \begin{align*}
              \Ff(\omega)
                &= \Fa\brp{\frac{\omega}{2}}\Fphi\brp{\frac{\omega}{2}}
                && \text{by \prefp{item:ortho_fW0_faphi}}
              \\&= \Fb\brp{\frac{\omega}{2}}\Dh\brp{\frac{\omega}{2}+\pi}\Fphi\brp{\frac{\omega}{2}}
                && \text{by \prefp{item:ortho_fW0_aabhh}}
              \\&= \cwt e^{i\frac{\omega}{2}}\Fv(\omega)\Dh^\ast\brp{\frac{\omega}{2}+\pi}\Fphi\brp{\frac{\omega}{2}}
                && \text{by definition of $\Fv(\omega)$}
            \end{align*}
        \end{enumerate}
    \end{enumerate}

  \item lemma: proof that (2)$\implies\ff\orthog\spV_0$: \label{item:ortho_fW0_V0}
    \begin{align*}
      &\Sfphi(\omega)
      \\&= 2\pi\sum_{n\in\Z} \Fphi\brp{\omega + 2\pi n}\Ff^\ast\brp{\omega + 2\pi n}
        \qquad\text{by \prefp{thm:Swfg}}
      \\&= 2\pi\sum_{n\in\Z} \Fphi\brp{\omega + 2\pi n}
                             \brs{\cwt e^{i\brp{\frac{\omega + 2\pi n}{2}}}\Fv\brp{\omega + 2\pi n}\Dh^\ast\brp{\frac{\omega+2\pi n}{2}+\pi}\Fphi\brp{\frac{\omega+2\pi n}{2}}}^\ast
        \qquad\text{by (A)}
      \\&= \sqrt{2}\pi\Fv^\ast\brp{\omega}
           \sum_{n\in\Z} \Fphi\brp{\omega + 2\pi n}
                              \brs{e^{-i\brp{\frac{\omega}{2} + \pi n}}\Dh\brp{\frac{\omega}{2}+\pi n+\pi}\Fphi^\ast\brp{\frac{\omega}{2}+\pi n}}
        \qquad\text{by (C)}
      \\&= \sqrt{2}\pi\Fv^\ast\brp{\omega}
           \mcom{\sum_{n\in\Ze} \Fphi\brp{\omega + 2\pi n}
                         \brs{e^{-i\brp{\frac{\omega}{2} + \pi n}}\Dh\brp{\frac{\omega}{2}+\pi n+\pi}\Fphi^\ast\brp{\frac{\omega}{2}+\pi n}}}{even terms}
        \\&\qquad+ \sqrt{2}\pi\Fv^\ast\brp{\omega}
           \mcom{\sum_{n\in\Zo} \Fphi\brp{\omega + 2\pi n}
                         \brs{e^{-i\brp{\frac{\omega}{2} + \pi n}}\Dh\brp{\frac{\omega}{2}+\pi n+\pi}\Fphi^\ast\brp{\frac{\omega}{2}+\pi n}}}{odd terms}
      \\&= \sqrt{2}\pi\Fv^\ast\brp{\omega}
           \mcom{\sum_{n\in\Z} \Fphi\brp{\omega + 4\pi n}
                         \brs{e^{-i\brp{\frac{\omega}{2} + 2\pi n}}\Dh\brp{\frac{\omega}{2}+2\pi n+\pi}\Fphi^\ast\brp{\frac{\omega}{2}+2\pi n}}}{even terms}
        \\&\qquad+ \sqrt{2}\pi\Fv^\ast\brp{\omega}
           \mcom{\sum_{n\in\Z} \Fphi\brp{\omega + 4\pi n+2\pi}
                         \brs{e^{-i\brp{\frac{\omega}{2} + 2\pi n+\pi}}\Dh\brp{\frac{\omega}{2}+2\pi n+2\pi}\Fphi^\ast\brp{\frac{\omega}{2}+2\pi n+\pi}}}{odd terms}
      \\&= \sqrt{2}\pi\Fv^\ast\brp{\omega}\Dh\brp{\frac{\omega}{2}+\pi}e^{-i\brp{\frac{\omega}{2}}}
           \sum_{n\in\Z} \Fphi\brp{\omega + 4\pi n}
                         \Fphi^\ast\brp{\frac{\omega}{2}+2\pi n}
        \\&\qquad- \sqrt{2}\pi\Fv^\ast\brp{\omega}\Dh\brp{\frac{\omega}{2}}e^{-i\brp{\frac{\omega}{2}}}
           \sum_{n\in\Z} \Fphi\brp{\omega + 4\pi n+2\pi}
                         \Fphi^\ast\brp{\frac{\omega}{2}+2\pi n+\pi}
          \qquad\text{by \prefp{prop:dtft_2pi}}
      \\&= \sqrt{2}\pi\Fv^\ast\brp{\omega}\Dh\brp{\frac{\omega}{2}+\pi}e^{-i\brp{\frac{\omega}{2}}}
           \sum_{n\in\Z} \cwt\Dh\brp{\frac{\omega}{2} + 2\pi n}\Fphi\brp{\frac{\omega}{2} + 2\pi n}
                         \Fphi^\ast\brp{\frac{\omega}{2}+2\pi n}
        \\&\qquad- \sqrt{2}\pi\Fv^\ast\brp{\omega}\Dh\brp{\frac{\omega}{2}}e^{-i\brp{\frac{\omega}{2}}}
           \sum_{n\in\Z} \cwt\Dh\brp{\frac{\omega}{2} + 2\pi n+\pi}\Fphi\brp{\frac{\omega}{2} + 2\pi n+\pi}
                         \Fphi^\ast\brp{\frac{\omega}{2}+2\pi n+\pi}
      \\&= \frac{1}{4}\Fv^\ast\brp{\omega}e^{-i\brp{\frac{\omega}{2}}}\Dh\brp{\frac{\omega}{2}+\pi}\Dh\brp{\frac{\omega}{2}}
           \brs{2\pi\sum_{n\in\Z} \abs{\Fphi\brp{\frac{\omega}{2} + 2\pi n}}^2}
        \\&\qquad- \frac{1}{4}\Fv^\ast\brp{\omega}e^{-i\brp{\frac{\omega}{2}}}\Dh\brp{\frac{\omega}{2}}\Dh\brp{\frac{\omega}{2}+\pi}
           \brs{2\pi\sum_{n\in\Z} \abs{\Fphi\brp{\frac{\omega}{2}+\pi+2\pi n}}^2}
          \qquad\text{by \prefp{prop:dtft_2pi}}
      \\&= \frac{1}{4}\Fv^\ast\brp{\omega}e^{-i\brp{\frac{\omega}{2}}}\Dh\brp{\frac{\omega}{2}+\pi}\Dh\brp{\frac{\omega}{2}}
           \Sphi\brp{\frac{\omega}{2}}
        - \frac{1}{4}\Fv^\ast\brp{\omega}e^{-i\brp{\frac{\omega}{2}}}\Dh\brp{\frac{\omega}{2}}\Dh\brp{\frac{\omega}{2}+\pi}
           \Sphi\brp{\frac{\omega}{2}+\pi}
      \\&= \frac{1}{4}\Fv^\ast\brp{\omega}e^{-i\brp{\frac{\omega}{2}}}\Dh\brp{\frac{\omega}{2}+\pi}\Dh\brp{\frac{\omega}{2}}
        - \frac{1}{4}\Fv^\ast\brp{\omega}e^{-i\brp{\frac{\omega}{2}}}\Dh\brp{\frac{\omega}{2}}\Dh\brp{\frac{\omega}{2}+\pi}
        \qquad\text{by \prefp{thm:Swfgo}}
      \\&=  0
        \qquad \text{by \prefp{lem:ows_quadcon}}
      \\\iff& \inprod{\ff}{\opTrn^n\fphi}=\kdelta_n
      %\\\iff& \inprod{\Fphi}{\opTrn^n\Fpsi} = 0
        \qquad \text{by \prefp{thm:Swfgo}}
      \\\iff& \ff\orthog\linspan\setn{\opTrn^n\fphi}
      \\\iff& \ff\orthog\spV_0
    \end{align*}


  \item Proof that (A)$\impliedby$(1,2):
    \begin{enumerate}
      \item by (1), $\ff\in\spV_1=\spV_0\adds\spW_0$
      \item by hypothesis, $\spV_0\orthog\spW_0$
      \item by \prefp{item:ortho_fW0_V0}, $\ff\orthog\spV_0$
      \item so $\ff\in\spW_0$
    \end{enumerate}


\end{enumerate}

\end{proof}

%--------------------------------------
\begin{theorem}
\footnote{
  \citerppgc{wojtaszczyk1997}{32}{37}{0521578949}{Section 2.4, Theorem 2.20},
  \citerpgc{hernandez1996}{57}{0849382742}{\scshape Proposition 2.13},
  \citerppgc{mallat}{236}{238}{012466606X}{Theorem 7.3},
  %\citerppg{vidakovic}{58}{59}{0471293652},
  \citerpg{dau}{133}{0898712742}
  }
\label{thm:ows_g}
%--------------------------------------
Let $\wavsys$ be a \structe{wavelet system} \xref{def:wavsys}.
%with an orthonormal MRA system such that $\linspan\setn{\opTrn^n\fphi}=\spV_0$.
Let $\Fv(\omega)$ be a measurable function in $\spLLR$ \xref{def:spLLR}.
\thmboxt{
  If $\set{\fphi}{\opTrn^n\fphi}=\kdelta_n$ then\\
  $
    \brb{\begin{array}{FMD}
      (A). & $\Fpsi(\omega) = \cwt \mcom{e^{i\frac{\omega}{2}}\Fv(\omega)\Dh^\ast\brp{\frac{\omega}{2}+\pi}}{$\Fg\brp{\frac{\omega}{2}}$ \xref{def:gn}}\Fphi\brp{\frac{\omega}{2}}$ & and \\
      (B). & $\Fv(\omega)=\Fv(\omega+2\pi)$ & and \\
      (C). & $\abs{\Fv(\omega)}=1$ &
    \end{array}}
    \iff
    \mcom{\brb{\begin{array}{FrclD}
      (1). & \inprod{\fpsi}{\opTrn^n\fpsi} &=& \kdelta_n & and \\
      (2). & \inprod{\fphi}{\opTrn^n\fpsi} &=& 0         & and \\
      (3). & \linspan\setn{\opTrn^n\fpsi}  &=& \spW_0
    \end{array}}}{$\begin{array}{D}$\spV_0\orthog\spW_0$ and\\$\fpsi$ is an \structe{orthonormal wavelet} for $\spLLR$\end{array}$}
  $
  }
\end{theorem}
\begin{proof}
\begin{enumerate}
  \item lemma: proof that (A) and $\setn{\opTrn^n\fphi}=\kdelta_n$ $\implies$ $\Spsi(\omega)=\abs{\Fv(\omega)}^2$:
        \label{item:ows_g_Spsi}
    \begin{align*}
      &\Spsi(\omega)
      \\&= 2\pi\sum_{n\in\Z} \abs{\Fpsi\brp{\omega + 2\pi n}}^2
        \qquad\text{by \prefp{thm:Swff}}
      \\&= 2\pi\sum_{n\in\Z} \abs{\cwt e^{i\brp{\frac{\omega + 2\pi n}{2}}}\Fv\brp{\omega + 2\pi n}\Dh^\ast\brp{\frac{\omega+2\pi n}{2}+\pi}\Fphi\brp{\frac{\omega+2\pi n}{2}}}^2
        \qquad\text{by (A)}
      \\&= \pi\sum_{n\in\Z} \mcom{\abs{e^{i\brp{\frac{\omega + 2\pi n}{2}}}}^2}{$1$} \abs{\Fv\brp{\omega + 2\pi n}}^2 \abs{\Dh^\ast\brp{\frac{\omega+2\pi n}{2}+\pi}\Fphi\brp{\frac{\omega+2\pi n}{2}}}^2
        \qquad\text{by property of $\abs{\cdot}$}
      \\&= \pi\abs{\Fv\brp{\omega}}^2\sum_{n\in\Z}  \abs{\Dh^\ast\brp{\frac{\omega+2\pi n}{2}+\pi}\Fphi\brp{\frac{\omega+2\pi n}{2}}}^2
        \qquad\text{by (B) and (C)}
      \\&= \pi\abs{\Fv\brp{\omega}}^2\sum_{n\in\Ze}  \mcom{\abs{\Dh^\ast\brp{\frac{\omega}{2}+\pi n+\pi}  \Fphi\brp{\frac{\omega}{2}+\pi n}}^2}{even terms}
         + \pi\abs{\Fv\brp{\omega}}^2\sum_{n\in\Zo}  \mcom{\abs{\Dh^\ast\brp{\frac{\omega}{2}+\pi n+\pi}  \Fphi\brp{\frac{\omega}{2}+\pi n}}^2}{odd terms}
      \\&= \pi\abs{\Fv\brp{\omega}}^2\sum_{n\in\Z}   \mcom{\abs{\Dh^\ast\brp{\frac{\omega}{2}+2\pi n}            \Fphi\brp{\frac{\omega}{2}+2\pi n}}^2      }{even terms}
         + \pi\abs{\Fv\brp{\omega}}^2\sum_{n\in\Z}   \mcom{\abs{\Dh^\ast\brp{\frac{\omega}{2}+2\pi n+\pi}        \Fphi\brp{\frac{\omega}{2}+2\pi n + \pi}}^2}{odd terms}
      \\&= \frac{\abs{\Fv\brp{\omega}}^2}{2}\brb{
           \abs{\Dh^\ast\brp{\frac{\omega}{2}}    }^2 \mcom{\brs{2\pi\sum_{n\in\Z} \abs{\Fphi\brp{\frac{\omega}{2}+2\pi n}}^2      }}{$\Sphi\brp{\frac{\omega}{2}}      $}
         + \abs{\Dh^\ast\brp{\frac{\omega}{2}+\pi}}^2 \mcom{\brs{2\pi\sum_{n\in\Z} \abs{\Fphi\brp{\frac{\omega}{2}+\pi+2\pi n}}^2}}{$\Sphi\brp{\frac{\omega}{2} + \pi}$}
           }
        \\&\qquad\text{because \ope{DTFT} is $2\pi$ \prope{periodic} \xref{prop:dtft_2pi}}
      \\&= \frac{\abs{\Fv\brp{\omega}}^2}{2}\brb{
           \abs{\Dh^\ast\brp{\frac{\omega}{2}}    }^2 \Sphi\brp{\frac{\omega}{2}}
         + \abs{\Dh^\ast\brp{\frac{\omega}{2}+\pi}}^2 \Sphi\brp{\frac{\omega}{2} + \pi}
           }
      \\&= \frac{\abs{\Fv\brp{\omega}}^2}{2}\brb{
           \abs{\Dh^\ast\brp{\frac{\omega}{2}}    }^2 \cancelto{1}{\Sphi\brp{\frac{\omega}{2}}}
         + \abs{\Dh^\ast\brp{\frac{\omega}{2}+\pi}}^2 \cancelto{1}{\Sphi\brp{\frac{\omega}{2} + \pi}}
           }
        \qquad\text{by \prefp{thm:Swfgo}}
     %\\&= \half \brs{\abs{\Dh^\ast\brp{\frac{\omega}{2}}    }^2 + \abs{\Dh^\ast\brp{\frac{\omega}{2}+\pi}}^2}
      \\&=  \abs{\Fv\brp{\omega}}^2
        \qquad\text{by \prefp{lem:ows_quadcon}}
    \end{align*}

  \item Proof that (A,B,C)$\implies$(1):
    \begin{align*}
      \Spsi(\omega)
        &= \abs{\Fv(\omega)}^2
        && \text{by \prefp{item:ows_g_Spsi}}
      \\&=  1
        && \text{by (B)}
      \\\iff& \inprod{\Fpsi}{\opTrn^n\Fpsi} = \kdelta_n
        \qquad\text{by \prefp{thm:Swfgo}}
    \end{align*}

  \item Proof that (A,B)$\implies$(2): by \prefp{prop:ortho_fW0} with $\Ff(\omega)\eqd\Fpsi(\omega)$.
    %\begin{align*}
    %  &\Shs(\omega)
    %  \\&= 2\pi\sum_{n\in\Z} \Fphi\brp{\omega + 2\pi n}\Fpsi^\ast\brp{\omega + 2\pi n}
    %    \qquad\text{by \prefp{thm:Swfg}}
    %  \\&= 2\pi\sum_{n\in\Z} \Fphi\brp{\omega + 2\pi n}
    %                         \brs{\cwt e^{i\brp{\frac{\omega + 2\pi n}{2}}}\Fv\brp{\omega + 2\pi n}\Dh^\ast\brp{\frac{\omega+2\pi n}{2}+\pi}\Fphi\brp{\frac{\omega+2\pi n}{2}}}^\ast
    %    \qquad\text{by (A)}
    %  \\&= \sqrt{2}\pi\Fv^\ast\brp{\omega}
    %       \sum_{n\in\Z} \Fphi\brp{\omega + 2\pi n}
    %                          \brs{e^{-i\brp{\frac{\omega}{2} + \pi n}}\Dh\brp{\frac{\omega}{2}+\pi n+\pi}\Fphi^\ast\brp{\frac{\omega}{2}+\pi n}}
    %    \qquad\text{by (C)}
    %  \\&= \sqrt{2}\pi\Fv^\ast\brp{\omega}
    %       \mcom{\sum_{n\in\Ze} \Fphi\brp{\omega + 2\pi n}
    %                     \brs{e^{-i\brp{\frac{\omega}{2} + \pi n}}\Dh\brp{\frac{\omega}{2}+\pi n+\pi}\Fphi^\ast\brp{\frac{\omega}{2}+\pi n}}}{even terms}
    %    \\&\qquad+ \sqrt{2}\pi\Fv^\ast\brp{\omega}
    %       \mcom{\sum_{n\in\Zo} \Fphi\brp{\omega + 2\pi n}
    %                     \brs{e^{-i\brp{\frac{\omega}{2} + \pi n}}\Dh\brp{\frac{\omega}{2}+\pi n+\pi}\Fphi^\ast\brp{\frac{\omega}{2}+\pi n}}}{odd terms}
    %  \\&= \sqrt{2}\pi\Fv^\ast\brp{\omega}
    %       \mcom{\sum_{n\in\Z} \Fphi\brp{\omega + 4\pi n}
    %                     \brs{e^{-i\brp{\frac{\omega}{2} + 2\pi n}}\Dh\brp{\frac{\omega}{2}+2\pi n+\pi}\Fphi^\ast\brp{\frac{\omega}{2}+2\pi n}}}{even terms}
    %    \\&\qquad+ \sqrt{2}\pi\Fv^\ast\brp{\omega}
    %       \mcom{\sum_{n\in\Z} \Fphi\brp{\omega + 4\pi n+2\pi}
    %                     \brs{e^{-i\brp{\frac{\omega}{2} + 2\pi n+\pi}}\Dh\brp{\frac{\omega}{2}+2\pi n+2\pi}\Fphi^\ast\brp{\frac{\omega}{2}+2\pi n+\pi}}}{odd terms}
    %  \\&= \sqrt{2}\pi\Fv^\ast\brp{\omega}\Dh\brp{\frac{\omega}{2}+\pi}e^{-i\brp{\frac{\omega}{2}}}
    %       \sum_{n\in\Z} \Fphi\brp{\omega + 4\pi n}
    %                     \Fphi^\ast\brp{\frac{\omega}{2}+2\pi n}
    %    \\&\qquad- \sqrt{2}\pi\Fv^\ast\brp{\omega}\Dh\brp{\frac{\omega}{2}}e^{-i\brp{\frac{\omega}{2}}}
    %       \sum_{n\in\Z} \Fphi\brp{\omega + 4\pi n+2\pi}
    %                     \Fphi^\ast\brp{\frac{\omega}{2}+2\pi n+\pi}
    %      \qquad\text{by \prefp{prop:dtft_2pi}}
    %  \\&= \sqrt{2}\pi\Fv^\ast\brp{\omega}\Dh\brp{\frac{\omega}{2}+\pi}e^{-i\brp{\frac{\omega}{2}}}
    %       \sum_{n\in\Z} \cwt\Dh\brp{\frac{\omega}{2} + 2\pi n}\Fphi\brp{\frac{\omega}{2} + 2\pi n}
    %                     \Fphi^\ast\brp{\frac{\omega}{2}+2\pi n}
    %    \\&\qquad- \sqrt{2}\pi\Fv^\ast\brp{\omega}\Dh\brp{\frac{\omega}{2}}e^{-i\brp{\frac{\omega}{2}}}
    %       \sum_{n\in\Z} \cwt\Dh\brp{\frac{\omega}{2} + 2\pi n+\pi}\Fphi\brp{\frac{\omega}{2} + 2\pi n+\pi}
    %                     \Fphi^\ast\brp{\frac{\omega}{2}+2\pi n+\pi}
    %  \\&= \frac{1}{4}\Fv^\ast\brp{\omega}e^{-i\brp{\frac{\omega}{2}}}\Dh\brp{\frac{\omega}{2}+\pi}\Dh\brp{\frac{\omega}{2}}
    %       \brs{2\pi\sum_{n\in\Z} \abs{\Fphi\brp{\frac{\omega}{2} + 2\pi n}}^2}
    %    \\&\qquad- \frac{1}{4}\Fv^\ast\brp{\omega}e^{-i\brp{\frac{\omega}{2}}}\Dh\brp{\frac{\omega}{2}}\Dh\brp{\frac{\omega}{2}+\pi}
    %       \brs{2\pi\sum_{n\in\Z} \abs{\Fphi\brp{\frac{\omega}{2}+\pi+2\pi n}}^2}
    %      \qquad\text{by \prefp{prop:dtft_2pi}}
    %  \\&= \frac{1}{4}\Fv^\ast\brp{\omega}e^{-i\brp{\frac{\omega}{2}}}\Dh\brp{\frac{\omega}{2}+\pi}\Dh\brp{\frac{\omega}{2}}
    %       \Sphi\brp{\frac{\omega}{2}}
    %    - \frac{1}{4}\Fv^\ast\brp{\omega}e^{-i\brp{\frac{\omega}{2}}}\Dh\brp{\frac{\omega}{2}}\Dh\brp{\frac{\omega}{2}+\pi}
    %       \Sphi\brp{\frac{\omega}{2}+\pi}
    %  \\&= \frac{1}{4}\Fv^\ast\brp{\omega}e^{-i\brp{\frac{\omega}{2}}}\Dh\brp{\frac{\omega}{2}+\pi}\Dh\brp{\frac{\omega}{2}}
    %    - \frac{1}{4}\Fv^\ast\brp{\omega}e^{-i\brp{\frac{\omega}{2}}}\Dh\brp{\frac{\omega}{2}}\Dh\brp{\frac{\omega}{2}+\pi}
    %    \qquad\text{by \prefp{thm:Swfgo}}
    %  \\&=  0
    %    \qquad \text{by \prefp{lem:ows_quadcon}}
    %  \\\iff& \inprod{\Fphi}{\opTrn^n\Fpsi} = 0
    %    \qquad \text{by \prefp{thm:Swfgo}}
    %\end{align*}

  \item lemma: Proof that $\setn{\opTrn^n\fphi}=\kdelta_n$ and (A,B,C) $\implies$ \label{item:ows_g_abc}
        for all $\seqn{a_n}$, there exists $\seqn{b_n}$ and $\seqn{c_n}$ such that
          \\\qquad$\Fa\brp{\frac{\omega}{2}} = \Fb\brp{\omega}\Dh  \brp{\frac{\omega}{2}} + \Fc\brp{\omega}\Fg\brp{\frac{\omega}{2}}$\\
    \begin{enumerate}
      \item First note the following relations hold in this case:
        \\\indentx$\begin{array}{rclccD}
          \abs{\Dh(\omega)}^2  &+& \abs{\Dh(\omega+\pi)}^2        &=& 2   & by $\setn{\opTrn^n\fphi}=\kdelta_n$ and by \prefp{lem:ows_quadcon}\\
          \abs{\Fg(\omega)}^2  &+& \abs{\Fg(\omega+\pi)}^2        &=& 2   & by (1) and by \prefp{lem:ows_quadcon}\\
          \Dh(\omega)\Fg^\ast(\omega) &+& \Dh(\omega +\pi) \Fg^\ast(\omega +\pi)&=& 0  & by (2) and by \prefp{lem:ows_quadcon} \\
          \mc{5}{l}{\Fg\brp{\frac{\omega}{2}}=e^{i\frac{\omega}{2}}\Fv(\omega)\Dh^\ast\brp{\frac{\omega}{2}+\pi}}  & by (A)
        \end{array}$

      \item Then, the following values of $\seqn{b_n}$ and $\seqn{c_n}$ are solutions for all $\seqn{a_n}$ \label{item:ows_g_abcsol}
        \\\indentx$\begin{array}{rclcl}
          \fb\brp{\omega} &=& \half \Fa\brp{\frac{\omega}{2}}\Dh^\ast\brp{\frac{\omega}{2}} &+& \half \Fa\brp{\frac{\omega}{2}+\pi}\Dh^\ast\brp{\frac{\omega}{2}+\pi}\\
          \fc\brp{\omega} &=& \half \Fa\brp{\frac{\omega}{2}}\Fg^\ast\brp{\frac{\omega}{2}} &+& \half \Fa\brp{\frac{\omega}{2}+\pi}\Fg^\ast\brp{\frac{\omega}{2}+\pi}
        \end{array}$
    \end{enumerate}
    Proof:
    \begin{align*}
      &\Fb\brp{\omega}\Dh  \brp{\frac{\omega}{2}} + \Fc\brp{\omega}\Fg\brp{\frac{\omega}{2}}
      \\&=         \half \Fa\brp{\frac{\omega}{2}}\abs{\Dh\brp{\frac{\omega}{2}}}^2 + \half \Fa\brp{\frac{\omega}{2}+\pi}\Dh  \brp{\frac{\omega}{2}}\Dh^\ast\brp{\frac{\omega}{2}+\pi}
        \\&\quad + \half \Fa\brp{\frac{\omega}{2}}\abs{\Fg\brp{\frac{\omega}{2}}}^2 + \half \Fa\brp{\frac{\omega}{2}+\pi}\Fg\brp{\frac{\omega}{2}}\Fg^\ast\brp{\frac{\omega}{2}+\pi}
        \qquad\text{by \prefp{item:ows_g_abcsol}}
      \\&=         \half \Fa\brp{\frac{\omega}{2}}\abs{\Dh\brp{\frac{\omega}{2}}}^2
                 + \half \Fa\brp{\frac{\omega}{2}+\pi}\Dh  \brp{\frac{\omega}{2}}\Dh^\ast\brp{\frac{\omega}{2}+\pi}
        \\&\quad + \half \Fa\brp{\frac{\omega}{2}}\abs{e^{i\frac{\omega}{2}}\Fv(\omega)\Dh^\ast\brp{\frac{\omega}{2}+\pi}}^2
                 + \half \Fa\brp{\frac{\omega}{2}+\pi}e^{i\frac{\omega}{2}}\Fv(\omega)\Dh^\ast\brp{\frac{\omega}{2}+\pi}e^{-i\brp{\frac{\omega}{2}+\pi}}\Fv^\ast(\omega+2\pi)\Dh\brp{\frac{\omega}{2}+2\pi}
      \\&=         \half \Fa\brp{\frac{\omega}{2}}\abs{\Dh\brp{\frac{\omega}{2}}}^2
                 + \half \Fa\brp{\frac{\omega}{2}+\pi}\Dh  \brp{\frac{\omega}{2}}\Dh^\ast\brp{\frac{\omega}{2}+\pi}
        \\&\quad + \half \Fa\brp{\frac{\omega}{2}}\abs{\Dh^\ast\brp{\frac{\omega}{2}+\pi}}^2
                 - \half \Fa\brp{\frac{\omega}{2}+\pi}\Fv(\omega)\Dh^\ast\brp{\frac{\omega}{2}+\pi}\abs{\Fv(\omega)}^2\Dh\brp{\frac{\omega}{2}}
      \\&=         \half \Fa\brp{\frac{\omega}{2}}\brs{
                     \abs{\Dh\brp{\frac{\omega}{2}}}^2 + \abs{\Dh^\ast\brp{\frac{\omega}{2}+\pi}}^2
                     }
        \\&\quad + \half \Fa\brp{\frac{\omega}{2}+\pi}\brs{
                       \Dh     \brp{\frac{\omega}{2}}\Dh^\ast\brp{\frac{\omega}{2}+\pi}
                     - \Dh^\ast\brp{\frac{\omega}{2}+\pi}\Dh\brp{\frac{\omega}{2}}
                     }
      \\&=         \half \Fa\brp{\frac{\omega}{2}}\brs{2}
                 + \half \Fa\brp{\frac{\omega}{2}+\pi}\brs{0}
        \qquad\text{by \prefp{lem:ows_quadcon}}
      \\&= \Fa\brp{\frac{\omega}{2}}
    \end{align*}

  \item Proof that (A,B,C)$\implies$(3):
    {\begin{align*}
      &\linspan\setn{\opTrn^n\fpsi}  = \spW_0
      \\&\iff  \spV_1 = \spV_0 \adds \mcom{\linspan\setn{\opTrn^n\fpsi}}{$\spW_0$}
        \qquad\text{by \prefp{def:seqWn}}
      \\&\iff  \mcom{\linspan\setn{\opDil\opTrn^n\fphi}}{$\spV_1$}
                 = \mcom{\linspan\setn{\opTrn^n\fphi}}{$\spV_0$}
             \adds \mcom{\linspan\setn{\opTrn^n\fpsi}}{$\spW_0$}
        \qquad\text{by \prefp{def:mra}}
      \\&\iff  \begin{array}{C}\forall \seqn{a_n}\exists\seqn{b_n},\seqn{c_n}\\\st\end{array}\quad
                 \sum_{n\in\Z}a_n \opDil\opTrn^n\fphi
               = \sum_{n\in\Z}b_n \opTrn^n\fphi
               + \sum_{n\in\Z}c_n \opTrn^n\fpsi
        \qquad\text{by \prefp{def:span}}
       %&&     \text{by definition of $\linspan$ \xref{def:span}}
      \\&\iff  \begin{array}{C}\forall \seqn{a_n}\exists\seqn{b_n},\seqn{c_n}\\\st\end{array}\quad
                 \opFT\brs{\sum_{n\in\Z}a_n \opDil\opTrn^n\fphi}
               = \opFT\brs{\sum_{n\in\Z}b_n \opTrn^n\fphi    }
               + \opFT\brs{\sum_{n\in\Z}c_n \opTrn^n\fpsi    }
        \qquad\text{by \prefp{def:ft}}
       %&&     \text{by definition of $\opFT$ \xref{def:ft}}
      \\&\iff  \begin{array}{C}\forall \seqn{a_n}\exists\seqn{b_n},\seqn{c_n}\\\st\end{array}\quad
                 \sum_{n\in\Z}a_n \opFT\opDil\opTrn^n\fphi
               = \sum_{n\in\Z}b_n \opFT\opTrn^n\fphi
               + \sum_{n\in\Z}c_n \opFT\opTrn^n\fpsi
      %\\&\iff  \begin{array}{C}\forall \seqn{a_n}\exists\seqn{b_n},\seqn{c_n}\\\st\end{array}\quad
      %           \sum_{n\in\Z}a_n \opDili\opFT\opTrn^n\fphi
      %         = \sum_{n\in\Z}b_n e^{-i\omega n}\Fphi(\omega)
      %         + \sum_{n\in\Z}c_n e^{-i\omega n}\Fpsi(\omega)
      %  \qquad\text{by \prefp{cor:FTD}}
      %\\&\iff  \begin{array}{C}\forall \seqn{a_n}\exists\seqn{b_n},\seqn{c_n}\\\st\end{array}\quad
      %           \sum_{n\in\Z}a_n \opDili\brs{e^{-i\omega n} \Fphi(\omega)}
      %         = \sum_{n\in\Z}b_n e^{-i\omega n}\Fphi(\omega)
      %         + \sum_{n\in\Z}c_n e^{-i\omega n}\Fpsi(\omega)
      %  \qquad\text{by \pref{cor:FTD}}
      \\&\iff  \begin{array}{C}\forall \seqn{a_n}\exists\seqn{b_n},\seqn{c_n}\\\st\end{array}\quad
                 \sum_{n\in\Z}a_n \brs{\cwt e^{-i\frac{\omega}{2} n} \Fphi\brp{\frac{\omega}{2}}}
               = \sum_{n\in\Z}b_n e^{-i\omega n}\Fphi(\omega)
               + \sum_{n\in\Z}c_n e^{-i\omega n}\Fpsi(\omega)
        \\&\qquad\qquad\text{by \prefp{cor:FTD}}
      \\&\iff  \begin{array}{C}\forall \seqn{a_n}\exists\seqn{b_n},\seqn{c_n}\\\st\end{array}\quad
                 \cwt \Fphi\brp{\frac{\omega}{2}} \sum_{n\in\Z}a_n e^{-i\frac{\omega}{2} n}
               = \Fphi(\omega)\sum_{n\in\Z}b_n e^{-i\omega n}
               + \Fpsi(\omega)\sum_{n\in\Z}c_n e^{-i\omega n}
      \\&\iff  \begin{array}{C}\forall \seqn{a_n}\exists\seqn{b_n},\seqn{c_n}\\\st\end{array}\quad
                 \cwt \Fphi\brp{\frac{\omega}{2}} \Fa\brp{\frac{\omega}{2}}
               =      \Fphi\brp{\omega}           \Fb\brp{\omega}
               +      \Fpsi\brp{\omega}           \Fc\brp{\omega}
               \qquad\text{by def. of \ope{DTFT} \xref{def:dtft}}
      \\&\iff  \begin{array}{C}\forall \seqn{a_n}\exists\seqn{b_n},\seqn{c_n}\\\st\end{array}\quad
               \text{(by \prefp{lem:Fphi})}
         \\&\qquad\qquad \cwt \Fa\brp{\frac{\omega}{2}} \Fphi\brp{\frac{\omega}{2}}
                       = \cwt \Fb\brp{\omega}           \Dh  \brp{\frac{\omega}{2}} \Fphi\brp{\frac{\omega}{2}}
                       + \cwt \Fc\brp{\omega}           \Fg  \brp{\frac{\omega}{2}} \Fphi\brp{\frac{\omega}{2}}
      \\&\iff  \begin{array}{C}\forall \seqn{a_n}\exists\seqn{b_n},\seqn{c_n}\\\st\end{array}\quad
                 \Fa\brp{\frac{\omega}{2}}
               = \Fb\brp{\omega}           \Dh  \brp{\frac{\omega}{2}}
               + \Fc\brp{\omega}           \Fg  \brp{\frac{\omega}{2}}
      \\&\implies (3) \qquad\text{by \prefp{item:ows_g_abc}}
    \end{align*}}

  \item Proof that (A,B)$\impliedby$(3): \label{item:ows_g_AB3}
    \begin{align*}
      \linspan\setn{\opTrn^n\fpsi}=\spW_0
        &\implies \fpsi\in\spW_0
      \\&\implies \text{(A) and (C)}
        && \text{by \prefp{prop:ortho_fW0}}
    \end{align*}

  \item Proof that (C)$\impliedby$(3) and $\setn{\opTrn^n\fphi}=\kdelta_n$:
    \begin{align*}
      1
        &= \Spsi(\omega)
        && \text{by right hypothesis and \prefp{thm:Swfgo}}
      \\&= \abs{\Fv(\omega)}^2
        && \text{by \prefp{item:ows_g_Spsi}}
     \end{align*}
\end{enumerate}
\end{proof}



%=======================================
\subsection{Sufficient conditions}
%=======================================
If we know the \structe{scaling coefficient sequence} $\seqn{\fh_n}$ \xref{def:hn}, then
\prefp{thm:ows_g} gives a necessary and sufficient condition for the
\structe{wavelet coefficient sequence} $\seqn{\fg_n}$ \xref{def:gn} to generate a \structe{wavelet system} \xref{def:wavsys}.
The actual value of $\seqn{\fg_n}$ depends on $\seqn{\fh_n}$ and the choice of $\Fv(\omega)$ \xref{thm:ows_g}.
Here are some possible choices:\footnote{
  \citerpgc{strang1996}{109}{0961408871}{``A Brief History of $\fh_1$"},
  \citerpg{vidakovic}{59}{0471293652}
  }
\\$\begin{array}{|Frclcrclp{55mm}|}
  \hline
  1. & \Fv(\omega) &\eqd& 1
     & \implies
     & \fg_n       &=&    (-1)^n \fh_{-1-n}^\ast
     &
     \\
  2. & \Fv(\omega) &\eqd& e^{i\omega}
     & \implies
     & \fg_n       &=&    (-1)^n \fh_{-3-n}^\ast
     &
     \\
  3. & \Fv(\omega)&\eqd&  e^{-i\omega}
     &\implies
     & \fg_n       &=&    (-1)^n \fh_{1-n}^\ast
     & \scs\ragr\prope{conjugate mirror filter} (CMF)
     \tabularnewline
  4. & \Fv(\omega) &\eqd& \pm(-1)^{\xN+1}e^{-i\frac{\omega}{2}(\xN+1)}
     & \implies
     & \fg_n       &=&    \pm (-1)^n \fh_{\xN-n}^\ast
     & \scs\ragr\prope{conjugate quadrature filter} (CQF), $\xN\in\Z$,\\\xref{def:cqf}
     \tabularnewline
  \hline
\end{array}$

%, many choices
%a sufficient condition for $\seqn{\fg_n}$ \xref{def:gn} is the \prope{conjugate quadrature filter condition}
%\xref{def:cqf} $\fg_n=\pm(-1)^n h^\ast_{N-n}$.
%This condition is used repeatedly in this text.
%\prefpp{thm:ortho_qmr} supplies a sufficient condition for computing a sequence of
%\structe{wavelet coefficients} $\seqn{\fg_n}$ \xref{def:gn}.


%%--------------------------------------
% weaker version of {thm:ows_g}
%\begin{theorem}
%\footnote{
%  \citerpgc{hernandez1996}{57}{0849382742}{\scshape Proposition 2.13},
%  \citerppg{vidakovic}{58}{59}{0471293652},
%  \citerpg{dau}{133}{0898712742}
%  }
%
%%--------------------------------------
%Let $\wavsys$ be a \structe{wavelet system} \xref{def:wavsys}.
%%with an orthonormal MRA system such that $\linspan\setn{\opTrn^n\fphi}=\spV_0$.
%Let $\Fv$ be a measurable function in $\spLLR$ \xref{def:spLLR}.
%\thmbox{
%  \brb{\begin{array}{FMD}
%    1. & $\Fpsi(2\omega) = e^{i\omega}\Fv(2\omega)\Fg^\ast(\omega+\pi)\Fphi(\omega)$ & and \\
%    2. & $\Fv(\omega)=1$ & and \\
%    3. & $\Fv(\omega)=\Fv(\omega+2\pi)$
%  \end{array}}
%  \quad\implies\quad
%  \mcom{\brb{\begin{array}{FrclD}
%    %1. & \inprod{\fpsi}{\opTrn^n\fpsi} &=& \kdelta_n & and \\
%    %2. & \inprod{\fphi}{\opTrn^n\fpsi} &=& 0         & and \\
%     & \linspan\setn{\opTrn^n\fpsi}  &=& \spW_0
%  \end{array}}}{$\fpsi$ is a \structe{wavelet} for $\spLLR$}
%  }
%\end{theorem}


%--------------------------------------
\begin{theorem} %[\thmd{quadrature mirror relation}]
\label{thm:ortho_qmr}
\footnote{
  %\citerpg{vidakovic}{59}{0471293652},
  \citerppg{mallat}{236}{238}{012466606X}
  }
%--------------------------------------
Let $\wavsys$ be a \structe{wavelet system} \xref{def:ows}.
\thmbox{
  \brb{\begin{array}{FlD}
    (A). & \mcom{\inprod{\fphi}{\opTrn^n\fphi}=\kdelta_n}{$\setn{\opTrn^n\fphi}$ is \prope{orthonormal}} & and \\
    (B). & \fg_n  = \mcom{\pm (-1)^n \fh_{\xN-n}^\ast}{\prope{CQF} condition} & $\forall n\in\Z$
  \end{array}}
  \quad\implies\quad
  \brb{\begin{array}{FlclD}
    (1). & \inprod{\fphi}{\opTrn^n\fpsi}   &=&  0            & and \\
    (2). & \inprod{\fpsi}{\opTrn^n\fpsi}   &=&  \kdelta_n    & and \\
    (3). & \linspan\setxZ{\opTrn^n\fpsi}   &=&  \spW_0       &
  \end{array}}
  \quad\scy \forall n\in\Z
  }
%\thmboxt{
%  $\brb{\begin{array}{FlDD}
%    (1). & \inprod{\fphi}{\opTrn^n\fphi}=\kdelta_n  & ($\setn{\opTrn^n\fphi}$ is \prope{orthonormal}) & and \\
%    (2). & \fg_n  = \pm (-1)^n \fh_{\xN-n}^\ast      & (\prope{CQF} \xref{def:cqf})                    & $\forall n\in\Z$
%  \end{array}}$
%  \\\indentx$\implies$\qquad
%  $\brb{\begin{array}{FlclDD}
%    (A). & \inprod{\fphi}{\opTrn^n\fpsi}   &=&  0            & ($\setn{\opTrn^n\fphi}$ is \prope{orthonormal})         & and \\
%    (B). & \inprod{\fpsi}{\opTrn^n\fpsi}   &=&  \kdelta_n    & ($\spV_0\orthog\spW_0$)                                 & and \\
%    (C). & \linspan\setxZ{\opTrn^n\fpsi}   &=&  \spW_0       & ($\setn{\opTrn^n\fpsi}$ is a\prope{basis} for $\spW_0$) &
%  \end{array}}$
%  \\$\scy \forall n\in\Z$
%  }
\end{theorem}
\begin{proof}

 %\item Proof that (2)$\implies$(1,2,3) of \prefp{thm:ows_g}:
    \begin{align*}
     \fg_n = \pm(-1)^n h^\ast_{\xN-n}
        &\iff \Fg(\omega)=\pm(-1)^\xN e^{-i\omega\xN}\Dh^\ast(\omega+\pi)
        &&    \text{by \thme{CQF theorem} \xref{thm:cqf}}
      \\&\implies \Fpsi(2\omega)=\cwt \mcom{\pm(-1)^\xN e^{-i\omega\xN}\Dh^\ast(\omega+\pi)}{$\Fg(\omega)$} \Fphi(\omega)
        &&    \text{by \prefp{prop:psi_g_phi}}
      \\&\implies \Fpsi(2\omega)=\cwt \mcom{\brs{\pm(-1)^\xN e^{-i\omega(\xN+1)}}}{$v(2\omega)$} e^{i\omega}
                                 \Dh^\ast(\omega+\pi)\,\Fphi(\omega)
      \\&\implies
           \brb{\begin{array}{FlclD}
             (1). & \inprod{\fphi}{\opTrn^n\fpsi}   &=&  0         & and \\
             (2). & \inprod{\fpsi}{\opTrn^n\fpsi}   &=&  \kdelta_n & and \\
             (3). & \linspan\setxZ{\opTrn^n\fpsi}   &=&  \spW_0    &
           \end{array}}
        && \text{by \prefp{thm:ows_g}}
    \end{align*}

%\begin{enumerate}
%Let $s\eqd\pm1$.
%\begin{enumerate}
%  %\item Proof that $\fg_{n}  = \pm (-1)^n  h(\xN-n) \iff \Fg(\omega)=\pm (-1)^\xN e^{-i\xN\omega} \Dh^\ast(\omega+\pi)$: See \prefp{thm:ortho_qmr}
%%
%  \item Proof that $\inprod{\fphi}{\opTrn^n\fpsi}=0$:
%    \begin{align*}
%      \Shs(2\omega)
%        &=  \Dh(\omega)\Fg^\ast(\omega)\cancelto{1}{\Swphi(\omega)}
%          + \Dh(\omega+\pi)\Fg^\ast(\omega+\pi)\cancelto{1}{\Swphi(\omega+\pi)}
%        &&  \text{by \prefp{thm:gen_quadcon}}
%      \\&=  \Dh(\omega)\Fg^\ast(\omega) + \Dh(\omega+\pi)\Fg^\ast(\omega+\pi)
%        &&  \text{by hypothesis (1) and \prefp{thm:Swfgo}}
%      \\&=  \Dh(\omega)
%            \left[s(-1)^\xN e^{-i\xN\omega} \Dh^\ast(\omega+\pi) \right]^\ast
%          \\&\indentx+ \Dh(\omega+\pi)
%            \left[s(-1)^\xN e^{-i\xN(\omega+\pi)} \Dh^\ast(\omega+2\pi) \right]^\ast
%        &&  \text{by hypothesis (2) and \prefp{thm:ortho_qmr}}
%      \\&=  \Dh(\omega)
%            s(-1)^\xN e^{-i\xN\omega} \Dh(\omega+\pi)
%          \\&\indentx+ s(-1)^\xN \Dh(\omega+\pi)
%            e^{-i\xN\omega}e^{-i\xN\pi} \Dh(\omega)
%      \\&=  s(-1)^\xN \Dh(\omega)\Dh(\omega+\pi) e^{-i\xN\omega}
%            \cancelto{0}{[1+e^{-i\pi\xN}]}
%      \\&=  0
%      \\\iff &\inprod{\fphi(x)}{\opTrn^n\fpsi(x)}=0
%        &&  \text{by \prefp{thm:Swfgo}}
%    \end{align*}
%
%\item Proof that $\inprod{\fpsi}{\opTrn^n\fpsi}=\kdelta_n$:
%\begin{align*}
%  2\Spsi(2\omega)
%    &=  \abs{\Fg(\omega)}^2 \cancelto{1}{\Sphi(\omega)} + \abs{\Fg(\omega+\pi)}^2 \cancelto{1}{\Sphi(\omega+\pi)}
%    && \text{by \pref{thm:gen_quadcon}}
%  \\&=  \abs{\Fg(\omega)}^2 + \abs{\Fg(\omega+\pi)}^2
%        &&  \text{by \prefp{thm:Swfgo}}
%  \\&=  \abs{s(-1)^\xN  e^{-i\xN\omega} \Dh^\ast(\omega+\pi)}^2
%    \\&\indentx  + \abs{s(-1)^\xN  e^{-i\xN(\omega+\pi)} \Dh^\ast(\omega+\pi+\pi)}^2
%    &&  \text{by hypothesis (2) and \prefp{thm:ortho_qmr}}
%  \\&=  \abs{\Dh^\ast(\omega+\pi)}^2 + \abs{\Dh^\ast(\omega+\pi+\pi)}^2
%  \\&=  \abs{\Dh^\ast(\omega+\pi)}^2 + \abs{\Dh^\ast(\omega)}^2
%    &&  \text{by \prefp{prop:dtft_2pi}}
%  \\&=  2
%    && \text{by \prefp{lem:ows_quadcon}}
%  \\\iff& \inprod{\fpsi}{\opTrn^n\fpsi} = \kdelta_n
%    && \text{by \prefp{thm:Swfgo}}
%\end{align*}

%\item Proof that $\linspan\setn{\opTrn^n\fpsi}=\spW_0$: by \prefp{thm:ows_g}.
%  This follows from \pref{lem:ows_quadcon}.
%  \begin{enumerate}
%    \item Proof that
%      $\Fg(\omega)=s(-1)^\xN e^{i\xN\omega} \Dh^\ast(\omega+\pi)
%       \implies
%       \Dh(\omega)\Fg^\ast(\omega)+\Dh(\omega+\pi)\Fg^\ast(\omega+\pi)=0
%      $:
%    \begin{align*}
%      &\Dh(\omega)\Fg^\ast(\omega)+\Dh(\omega+\pi)\Fg^\ast(\omega+\pi)
%      \\&=  \Dh(\omega)
%            \left[ s(-1)^\xN e^{i\xN\omega} \Dh^\ast(\omega+\pi) \right]^\ast
%          + \Dh(\omega+\pi)
%            \left[ s(-1)^\xN e^{i\xN(\omega+\pi)} \Dh^\ast(\omega+2\pi) \right]^\ast
%            \quad\text{by left hypothesis}
%      \\&=  \Dh(\omega)
%            s(-1)^\xN e^{-i\xN\omega} \Dh(\omega+\pi)
%          + \Dh(\omega+\pi)
%            s(-1)^\xN e^{-i\xN\omega}e^{-i\xN\pi} \Dh(\omega)
%      \\&=  s(-1)^\xN \Dh(\omega)\Dh(\omega+\pi) e^{-i\xN\omega}
%            \left[1+e^{-i\xN\pi}\right]
%      \\&=  0
%    \end{align*}
%
%
%
%    \item Proof that $\left|\Fg(\omega)\right|^2 + \left|\Fg(\omega+\pi)\right|^2 = 2$:
%    \begin{align*}
%      &\left|\Fg(\omega)\right|^2 + \left|\Fg(\omega+\pi)\right|^2
%      \\&=  \left| s(-1)^\xN e^{i\xN\omega} \Dh^\ast(\omega+\pi)  \right|^2
%          + \left| s(-1)^\xN e^{i\xN(\omega+\pi)} \Dh^\ast(\omega+\pi+\pi) \right|^2
%            \quad\text{by left hypothesis}
%      \\&=  \left|\Dh^\ast(\omega+\pi)  \right|^2
%          + \left|\Dh^\ast(\omega) \right|^2
%      \\&=  \left|\Dh(\omega    ) \right|^2
%          + \left|\Dh(\omega+\pi) \right|^2
%      \\&=  2
%            \qquad\text{by \prefpp{thm:gen_quadcon} and \prefpp{thm:Swfgo}}
%    \end{align*}
%  \end{enumerate}
%
%\item
%\begin{align*}
%  0
%    &= \inprod{\fphi(x)}{\fpsi(x-k)}
%%   && \text{by \pref{def:psi}}
%  \\&= \inprod{\sqrt{2}\sum_{n\in\Z} \fh_{n} \fphi(2x-n)}
%              {\sqrt{2}\sum_{m\in\Z}\fg(m)\fphi(2x-2k-m)}
%    && \text{by \pref{thm:dilation_eq}}
%  \\&= \sqrt{2}\sqrt{2}\sum_{n\in\Z} \sum_{m\in\Z}\fh_{n}  \fg^\ast(m)
%       \inprod{\fphi(2x-n)}{\fphi(2x-2k-m)}
%  \\&= 2\sum_{n\in\Z} \sum_{m\in\Z}\fh_{n}  \fg^\ast(m) \kdelta(n-2k-m)
%  \\&= 2\sum_{n\in\Z} \fh_{n}  \fg^\ast(n-2k)
%  \\&= 2\sum_{n\mathrm{\;even}} \fh_{n}  \fg^\ast(n-2k)
%       +   2\sum_{n\mathrm{\;odd}} \fh_{n}  \fg^\ast(n-2k)
%  \\&= 2\sum_{n\in\Z}  h(2n) \fg^\ast(2n-2k)
%        +   2\sum_{n\in\Z}  h(2n+1) \fg^\ast(2n-2k+1)
%\end{align*}
%
%Proof not yet complete. \problem
%References: \\
%\begin{liste}
%  \item \citerp{burrus}{58}
%  \item \citerpp{burrus}{250}{251}
%  \item \citerpp{mallat}{236}{238}
%\end{liste}
%\end{enumerate}
\end{proof}

%---------------------------------------
\begin{theorem}
\label{thm:ows_sc}
%---------------------------------------
Let $\spO\eqd\wavsys$ be a tuple. %\structe{wavelet system} \xref{def:wavsys}.
\thmboxt{
  $\brb{\begin{array}{F>{\ds}lc>{\ds}lDD}
    (A). & \sum_n\fh_n                      &=& \sqrt{2}                             & [\prope{admissibility}                    \xref{thm:admiss}]     & and\\
    (B). & \sum_{m\in\Z}\fh_m h^\ast_{m-2n} &=& \kdelta_n                            & [\prope{ortho. quadrature condition}      \xref{lem:oms_quadcon}]& and\\
    (C). & \sum_{k\in\Z}(-1)^k k^n\fh_k     &=& 0\quad{\scy\forall n=0,1,\ldots,p-1} & [\prope{$p$ vanishing moments}            \xref{thm:gh_cqf}]     & and\\
    (D). & \inf_{\omega\in\intcc{-\pi/2}{\pi/2}}\abs{\Fh(\omega)}&>&0               & [\prope{low-pass response}]                                     & and\\
    (E). &\fg_n                             &=& (-1)^n h^\ast_{N-n}                  & [\prope{CQF condition}                    \xref{def:cqf}]
  \end{array}}$
  \\\indentx$\implies$\qquad
  $\brb{\begin{array}{FMD}
    (1). & $\spO$ is an \structe{orthonormal wavelet system} & and \\
    (2). & $\fpsi$ has \prope{$p$ vanishing moments}
  \end{array}}$
  }
\end{theorem}
\begin{proof}
\begin{enumerate}
  \item lemma: $\mrasys$ is an \structe{orthonormal MRA system}. Proof: \label{ilem:ows_sc_oms}
    \begin{enumerate}
      \item If $p=1$, then this is the \structe{Haar MRA system}.
      \item If $p\ge2$, then proof follows from \prefp{thm:ortho_sc}:
        \begin{enumerate}
          \item Proof that $\sum_{n\in\Z}h_{n}=\sqrt{2}$: by left hypothesis (A)
          \item Proof that $\sum_{m\in\Z}h_m h^\ast_{m-2n} = \kdelta_n$: by left hypothesis (B)
          \item Proof that $\Fphi(\omega)$ is \prope{continuous} at $0$: 
            \begin{align*}
              (C) &\implies \sum_{k\in\Z}(-1)^k k^n\fh_k = 0\quad{\scy\forall n=0,1,\ldots,p-1,\,p\ge2}
                \\&\implies \text{$\fpsi$ has \prope{$p$ vanishing moments}}
                  && \text{by \prefp{thm:gh_cqf}}
                \\&\implies \text{$\Fphi(\omega)$ is \prope{continuous} at $\omega=0$}
                  && \text{by \prefp{thm:vanish_continuous}}
            \end{align*}
          \item Proof that $\ds\inf_{\omega\in\intcc{-\pi/2}{\pi/2}}\abs{\Dh(\omega)}>0$: by left hypothesis (D)
        \end{enumerate}
    \end{enumerate}
  \item Proof for (1): by \pref{ilem:ows_sc_oms}, (E), and \prefp{thm:ortho_qmr}.
  \item Proof for (2): by (C) and \prefp{thm:gh_cqf}.
\end{enumerate}
\end{proof}




%%--------------------------------------
%\begin{theorem}
%\footnote{
%  \citerp{lawton90}{1899}
%  }
%\label{thm:admiss_quad==>tight}
%\index{quadrature condition}
%\index{admissibility condition}
%\index{quadrature mirror relation}
%\index{partition of unity}
%\index{frame!tight}
%\index{tight frame}
%\index{scaling coefficients!necessary and sufficient conditions}
%%--------------------------------------
%Let $\fphi(x)$ be a scaling function
%with scaling coefficients $\seq{\fh_{n} }{n\in\Z}$
%and $\fpsi(x)=\sum_{n\in\Z}\fg_{n} \fphi(2x-n)$ be a wavelet function.
%\ifdochas{frames}{\footnote{
%  \structe{tight frame}: \prefp{def:frame}
%  }}
%\thmbox{
%  \brbr{\begin{array}{FllD}
%       1. & \mcom{\sum_{n\in\Z} \fh_{n}  =  \sqrt{2}}{admissibility condition} & and
%     \\2. & \mcom{\sum_{m\in\Z} h(m) h^\ast(m-2n) = \kdelta_n}{quadrature condition} & and
%     \\3. & \mcom{\fg_n\eqd \pm(-1)^n \fh_{\xN-n},\quad\xN\in\Zo}{quadrature mirror condition} &
%  \end{array}}
%  \implies
%  \brbl{\begin{array}{FlC}
%       1. & \mcom{\sum_{n\in\Z} \fphi(x-n) = 1}{partition of unity} & \text{and}
%     \\2. & \mcom{\ff(x) = \sum_j\sum_{n\in\Z} \inprod{\ff(x)}{\fpsi_{jn}(x)}\fpsi_{jn}(x)}{tight frame}&\scy\forall\ff\in\spLLR
%  \end{array}}
%  }
%\end{theorem}






%=======================================
\section{Examples}
%=======================================
The scaling coefficients can be directly computed using
simultaneous equations generated from the following
necessary conditions.\cittrpp{vidakovic}{92}{93}
They can be solved using a symbolic equation software package%
\footnote{\begin{tabular}[t]{lll}
  {\em Maxima}\texttrademark      & \url{http://maxima.sourceforge.net/}  & free and open source   \\
  {\em Macsyma}\texttrademark     & \url{http://www.scientek.com/macsyma/mxmain.htm} & a very non-free relative of {\em Maxima}\\
  {\em MuPAD}\texttrademark       & \url{http://www.mupad.de/}            & 30 day free trial   \\
  {\em Maple}\texttrademark       & \url{http://www.maplesoft.com/} & \\
  {\em Mathematica}\texttrademark & \url{http://www.wolfram.com/} &
  \end{tabular}}
or by hand.

%--------------------------------------
\begin{example}[\exmd{2 coefficient case}---\exmd{Haar wavelet system}]
\label{ex:ortho_n=2}
\index{Haar wavelet}
\index{wavelets!Haar}
%--------------------------------------
Let $\wavsys$ be an \emph{orthogonal} wavelet system with two non-zero scaling coefficients.
\exbox{
  \brb{\begin{array}{FMMD}
    1. & $\support\fphi(x)=\intcc{0}{1}$                             & \xref{thm:support}     & and\\
    2. & \prope{admissibility condition}                             & \xref{thm:admiss}      & and \\
    3. & $\setn{\opTrn^n\fphi}$ is \prope{orthonormal}                & \xref{lem:ows_quadcon} & and\\
    4. & $\fg_n  = \pm (-1)^n \fh_{\xN-n}^\ast$ $\scy\forall n\in\Z$  & \xref{thm:ortho_qmr}        &
  \end{array}}
  \implies
  \brb{\begin{array}{r|l|l}%
    n  & \fh_{n}  &\fg_n\\
    \hline
    0  & \ds \cwt  & \ds \cwt  \\
    1  & \ds \cwt  & \ds -\cwt \\
    \text{other} & 0 & 0
  \end{array}}
  }%
\\
  \begin{tabular}{cc}
    \includegraphics{../common/math/graphics/pdfs/d1_phi_h.pdf}&\includegraphics{../common/math/graphics/pdfs/d1_psi_g.pdf}
  \end{tabular}
\end{example}
\begin{proof}
\begin{enumerate}
  \item Proof that (1) $\implies$ that only $\fh_0$ and $\fh_1$ are non-zero: by \prefp{thm:support}.

  \item Proof for values of $\fh_0$ and $\fh_1$:
    \begin{align*}
     \fh_0 +\fh_1 &= \sqrt{2} && \text{by admissibility condition---\prefp{thm:admiss}}\\
     \fh_0^2 +\fh_1^2 &= 1    && \text{orthogonal quadrature condition---\prefp{lem:ows_quadcon}}
    \end{align*}

    \begin{align*}
     \fh_0^2 + \brp{\sqrt{2}-h_0}^2 &= 1 \\
      2h_0^2 - 2\sqrt{2}h_0 + 1 &= 0 \\
     \fh_0 &= \frac{2\sqrt{2} \pm \sqrt{\brp{2\sqrt{2}}^2-4\cdot2\cdot1}}
           {2\cdot2}
        \\&= \cwt
     \\
     \\
    \fh_1 &= \sqrt{2} -\fh_0
       \\&= \sqrt{2} - \cwt
        \\&= \cwt
    \end{align*}


  \item Note: $\fh_0$ and $\fh_1$ are the \hie{Haar scaling coefficients}.
    They can also be produced using other systems of equations including the following:
    \begin{dingautolist}{"AC}
      \item Admissibility condition and \hie{partition of unity}---\prefpp{ex:pun_n=2}
      \item \hie{Daubechies-$p1$} wavelets computed using spectral techniques---\prefpp{ex:dau-p1}
    \end{dingautolist}

  \item Proof for values of $\fg_0$ and $\fg_1$: by (4) and \prefp{thm:ortho_qmr}.
\end{enumerate}
\end{proof}


%---------------------------------------
\begin{example}
\footnote{
  \citerp{vidakovic}{92},
  \citerpg{soman2010}{83}{9788120340534}
  }
\label{ex:pun_n=4}
%---------------------------------------
In this example,
%we remove the orthonormality constraint and replace it with
%the more general \hie{partition of unity} constraint
%(all orthonormal wavelet systems possess the partition of unity property---\prefp{thm:quad=>zero_unity}).
we constrain the system to have $p=2$ vanishing moments.
%In this case, the equations are independent and the coefficients
%can be calculated.
This same wavelet system is identical to the \hie{Daubechies-$p2$} wavelet system
which is computed using a spectral factorization technique---\prefpp{ex:dau-p2}.
Here are the equations:

\[\begin{array}{*{9}{r}D}
 \fh_0      &+&\fh_1      &+&\fh_2   &+&\fh_3   &=& \sqrt{2}
  &{(admissibility---\prefp{thm:admiss})} \\
 \fh_0      &-&\fh_1      &+&\fh_2   &-&\fh_3   &=& 0
  &{(vanishing 0th moment/partition of unity---\prefpo{thm:pun_zero})} \\
           &-&\fh_1      &+& 2h_2  &-&3h_3   &=& 0
  &{(vanishing 1st moment---\prefp{thm:gh_cqf})} \\
 \fh_0^2    &+&\fh_1^2    &+&\fh_2^2 &+&\fh_3^2 &=& 1
  &{($m=0$ ortho. quad. con.---\prefp{lem:ows_quadcon})}
\end{array}\]
Then the scaling coefficients have the following values:\\
\begin{minipage}{5\tw/16}%
  \exbox{\begin{array}{r|l|l}%
    n & \sqrt{2}\fh_n  & \sqrt{2}\fg_n  \\
    \hline
      0   & \frac{1+\sqrt{3}}{4}  &  \frac{1-\sqrt{3}}{4}  \\
      1   & \frac{3+\sqrt{3}}{4}  & -\frac{3-\sqrt{3}}{4}  \\
      2   & \frac{3-\sqrt{3}}{4}  &  \frac{3+\sqrt{3}}{4}  \\
      3   & \frac{1-\sqrt{3}}{4}  & -\frac{1+\sqrt{3}}{4}
  \end{array}}%
\end{minipage}%
\begin{minipage}{11\tw/16}%
  \begin{tabular}{cc}
    \includegraphics{../common/math/graphics/pdfs/d2_phi_h.pdf}&\includegraphics{../common/math/graphics/pdfs/d2_psi_g.pdf}
  \end{tabular}
\end{minipage}
\end{example}
\begin{proof}
\begin{enumerate}
\item Solve the 3 linear equations in terms of $\fh_3$:
\begin{align*}
\left[\begin{array}{rrrr@{\hspace{3ex}}r}
    1 &  1 &  1 &  1 & \sqrt{2}   \\
    1 & -1 &  1 & -1 & 0          \\
    0 & -1 &  2 & -3 & 0
\end{array}\right]
&\rightarrow&
\left[\begin{array}{rrrr@{\hspace{3ex}}r}
    1 &  1 &  1 &  1 & \sqrt{2}   \\
    0 & -2 &  0 & -2 &-\sqrt{2}   \\
    0 & -1 &  2 & -3 & 0
\end{array}\right]
&\rightarrow&
\left[\begin{array}{rrrr@{\hspace{3ex}}r}
    1 &  0 &  3 & -2 & \sqrt{2}   \\
    0 &  0 & -4 &  4 &-\sqrt{2}   \\
    0 &  1 & -2 &  3 & 0
\end{array}\right]
&\rightarrow&
\\
\left[\begin{array}{rrrr@{\hspace{3ex}}r}
    1 &  0 &  3 & -2 & \sqrt{2}   \\
    0 &  1 & -2 &  3 & 0          \\
    0 &  0 &  1 & -1 & \frac{\sqrt{2}}{4}
\end{array}\right]
&\rightarrow&
\left[\begin{array}{rrrr@{\hspace{3ex}}r}
    1 &  0 &  0 &  1 & \frac{\sqrt{2}}{4}   \\
    0 &  1 &  0 &  1 & \cwt   \\
    0 &  0 &  1 & -1 & \frac{\sqrt{2}}{4}
\end{array}\right]
\end{align*}
\begin{align*}
 \fh_0 &= \frac{\sqrt{2}}{4} -\fh_3  \\
 \fh_1 &= \cwt -\fh_3  \\
 \fh_2 &= \frac{\sqrt{2}}{4} +\fh_3  \\
\end{align*}

\item Solve the forth quadratic equation in terms of $\fh_3$:
\begin{align*}
  1
    &=  \left(\frac{\sqrt{2}}{4}-h_3\right)^2
    &&+ \left(\cwt-h_3\right)^2
    &&+ \left(\frac{\sqrt{2}}{4}+h_3\right)^2
    &&+\fh_3^2 \hs{1cm}
  \\&=  \left(\frac{1}{8}-\sqrt{2}{2}h_3+h_3^2\right)
    &&+ \left(\half -\sqrt{2}h_3+h_3\right)
    &&+ \left(\frac{1}{8}+\cwt+h_3^2\right)
    &&+\fh_3^2 \hs{1cm}
  \\&=  4h_3^2 -\sqrt{2}h_3 + \frac{3}{4}
\\
  4h_3^2 -\sqrt{2}h_3 - \frac{1}{4} &=0
\\
 \fh_3
    &= \frac{\sqrt{2}\pm\sqrt{2+4}}{8}
  \\&= \frac{\sqrt{2}\pm\sqrt{6}}{8}
  \\&= \frac{\sqrt{2}}{8}(1\pm\sqrt{3})
\end{align*}

\item Simplify the results:
\[\begin{array}{r*{4}{cl}}
 \fh_3 &=& \frac{\sqrt{2}}{8}(1-\sqrt{3})
\\
 \fh_2 &=& \frac{\sqrt{2}}{4} +\fh_3
      &=& \frac{\sqrt{2}}{4} + \frac{\sqrt{2}}{8}(1-\sqrt{3})
      &=& \frac{\sqrt{2}}{8}(2+1-\sqrt{3})
      &=& \frac{\sqrt{2}}{8}(3-\sqrt{3})
\\
 \fh_1 &=& \cwt -\fh_3
      &=& \cwt - \frac{\sqrt{2}}{8}(1-\sqrt{3})
      &=& \frac{\sqrt{2}}{8}(4-1+\sqrt{3})
      &=& \frac{\sqrt{2}}{8}(3+\sqrt{3})
\\
 \fh_0 &=& \frac{\sqrt{2}}{4} -\fh_3
      &=& \frac{\sqrt{2}}{4} - \frac{\sqrt{2}}{8}(1-\sqrt{3})
      &=& \frac{\sqrt{2}}{8}(2-1+\sqrt{3})
      &=& \frac{\sqrt{2}}{8}(1+\sqrt{3})
\end{array}\]

\item Solution using {\em Maxima}\texttrademark \hspace{1ex}(2 sets of solutions): \\
%\begin{minipage}{120mm}
\begin{lstlisting}[language=MuPAD,frame=single]
  solve([h0+h1+h2+h3=sqrt(2),h0-h1+h2-h3=0,-h1+2*h2-3*h3=0,h0^2+h1^2+h2^2+h3^2=1],
        [h0,h1,h2,h3]);
\end{lstlisting}
%\end{minipage}\\
\[\begin{array}{rcl}
   \fh_0 &=& -\frac{\sqrt{6}- \sqrt{2}}{8} \\
   \fh_1 &=& -\frac{\sqrt{6}-3\sqrt{2}}{8} \\
   \fh_2 &=&  \frac{\sqrt{6}+3\sqrt{2}}{8} \\
   \fh_3 &=&  \frac{\sqrt{6}+ \sqrt{2}}{8} \\
  \end{array}
  \hspace{2cm}
  \begin{array}{rcl}
   \fh_0 &=&  \frac{\sqrt{6}+ \sqrt{2}}{8} \\
   \fh_1 &=&  \frac{\sqrt{6}+3\sqrt{2}}{8} \\
   \fh_2 &=& -\frac{\sqrt{6}-3\sqrt{2}}{8} \\
   \fh_3 &=& -\frac{\sqrt{6}- \sqrt{2}}{8}
  \end{array}\]
%                SQRT(6) - SQRT(2)         SQRT(2) SQRT(3) - 3 SQRT(2)
%(%o36) [[h0 = ------------------, h1 = ----------------------------,
%                       8                             8
%
%     SQRT(2) SQRT(3) + 3 SQRT(2)       SQRT(2) SQRT(3) + SQRT(2)
%h2 =---------------------------, h3 =-------------------------],
%                 8                               8
%
%      SQRT(6) + SQRT(2)	      SQRT(2) SQRT(3) + 3 SQRT(2)
%[h0 =-----------------, h1 =---------------------------,
%              8                           8
%
%       SQRT(2) SQRT(3) - 3 SQRT(2)         SQRT(2) SQRT(3) - SQRT(2)
%h2 = ----------------------------, h3 = --------------------------]]
%                  8                                   8
%\end{verbatim}
\end{enumerate}
\end{proof}




%--------------------------------------
\begin{example}[4 scaling coefficient case]
\label{ex:ortho_n=4}
%--------------------------------------
This example attempts to use the admissibility condition (\prefp{thm:admiss})
and orthonormal quadrature condition (\prefp{lem:ows_quadcon})
to design a 4 coefficient wavelet system.
However, it fails due to the failure of the equations to be independent.

Let $\wavsys$ be an \emph{orthogonal} wavelet system with four non-zero scaling coefficients.
\begin{dingautolist}{"AC}
  \item We can use the following four equations to try to compute the values of the
        four scaling coefficients,
        and then compute the four wavelet coefficients from the four scaling coefficients.
        Note that the second, third, and fourth equations all assume orthonormality.
        \[\begin{array}{*{9}{r}l}
         \fh_0      &+&\fh_1      &+&\fh_2   &+&\fh_3   &=& \sqrt{2}
          &\text{(admissibility---\prefp{thm:admiss})}
          \\
         \fh_0      &-&\fh_1      &+&\fh_2   &-&\fh_3   &=& 0
          &\text{(zero at $-1$ from orthonormality---\prefpo{thm:quad=>zero_unity})}
          \\
         \fh_0^2    &+&\fh_1^2    &+&\fh_2^2 &+&\fh_3^2 &=& 1
          &\text{($m=0$ ortho. quad. con.---\prefpo{lem:ows_quadcon})}
          \\
          \fh_0\fh_2 &+& \fh_1\fh_3 & &         & &         &=& 0
          &\text{($m=1$ ortho. quad. con.---\prefpo{lem:ows_quadcon})}
        \end{array}\]

\item These equations are linearly independent, however they are
      still dependent
  \footnote{\begin{minipage}[t]{2\tw/16+1ex}%
      \ttfamily\fontsize{0.5mm}{0.5mm}%\fntmono% \usefont{T1}{put}{b}{n}%
      \begin{Verbatim}[frame=single, gobble=8, label={\footnotesize\rmfamily email}]
        Dear Dan,

        (1) a   + b   +  c   +  d   = sqrt(2)
        (2) a   - b   +  c   -  d   = 0
        (3) a^2 + b^2 +  c^2 +  d^2 = 1
        (4) ac  + bd                = 0
        from the eq(1) and eq(2), we have
        (5) a+c = 0.5 * sqrt(2)
        (6) b+d = 0.5 * sqrt(2)

        take square of (5) and (6), we have
        (7) a^2 + 2ac + c^2 = 1/2
        (8) b^2 + 2bd + d^2 = 1/2

        Set equation(9) equals eq (7) + eq (8) , we have
        (9) a^2 + b^2 + c^2 + d^2 + 2(ac + bd) = 1
        Then, you see
        eq (7), (8) , (3), (4) are dependent.


        Shin-Lin
      \end{Verbatim}
    \end{minipage}
    \hfill
    \begin{minipage}[t]{13\tw/16-3ex}
      Many many thanks to
      \zht{\zhtXie\zhtXin\zhtLin} (Xi/`e X/-in L/'in) for pointing this out to me in a 2005 October 13 email.
      (If you are viewing this text as a pdf file,
       zoom in on the figure to the left to see text from \zht{\zhtXie\zhtXin\zhtLin}'s 2005 October 13 email.)
    \end{minipage}
  }
  as shown next:
  \begin{align*}
    &  \brs{(h_0+h_1+h_2+h_3)+(h_0-h_1+h_2-h_3)}^2
  \\&\qquad+\brs{(h_0+h_1+h_2+h_3)-(h_0-h_1+h_2-h_3)}^2
  \\&= [2h_0+2h_2]^2+[2h_1+2h_3]^2
  \\&= 4[h_0+h_2]^2+4[h_1+h_3]^2
  \\&= 4[h_0^2+2h_0h_2+h_2^2]+4[h_1^2+2h_1h_3+h_3^2]
  \\&= 4[h_0^2 +\fh_1^2 +\fh_2^2 +\fh_3^2 + 2h_0h_2 + 2h_1h_3]
  \\&= 4(h_0^2 +\fh_1^2 +\fh_2^2 +\fh_3^2) + 8(h_0h_2 +\fh_1h_3)
  \\ \implies & \text{The last two equations are (non-linearly) dependent on the first two.}
  \end{align*}

\item Because of this dependence, the system of equations has one degree of freedom.

\item Solutions using {\em Maxima}\\ %\texttrademark \\
  {\footnotesize
  \begin{verbatim}
    solve([h0+h1+h2+h3=sqrt(2),h0-h1+h2-h3=0,h0^2+h1^2+h2^2+h3^2=1,h0*h2+h1*h3=0],
          [h0,h1,h2,h3]
         );
  \end{verbatim}
  }
  \begin{enumerate}
    \item {\em Maxima} first solution:
      \begin{align*}
       \fh_1 &= \frac{\sqrt{2}}{4}\left(\sqrt{-8h_0^2+4\sqrt{2}h_0+1}+1\right)
      \\h_2 &= \frac{2h_0-\sqrt{2}}{2}
      \\h_3 &= \frac{\sqrt{2}}{4}\left(\sqrt{-8h_0^2+4\sqrt{2}h_0+1}-1\right)
      \end{align*}

    \item {\em Maxima} second solution:
      \begin{align*}
       \fh_1 &= \frac{\sqrt{2}}{4}\left(\sqrt{-8h_0^2+4\sqrt{2}h_0+1}-1\right)
      \\h_2 &= -\frac{2h_0-\sqrt{2}}{2}
      \\h_3 &= \frac{\sqrt{2}}{4}\left(\sqrt{-8h_0^2+4\sqrt{2}h_0+1}+1\right)
      \end{align*}
  \end{enumerate}
\end{dingautolist}
\end{example}


%---------------------------------------
\begin{example}
\footnote{
  \citerpg{soman2010}{84}{9788120340534}
  }
\label{ex:pun_n=6}
%---------------------------------------
This example gives the equations for a 6 coefficient wavelet system.
%However, they are not solved at this time.
See also the six coefficient \hie{Daubechies-$p3$} wavelet system
which is computed using a spectral factorization technique \xref{ex:dau-p3}.
%\begin{enumerate}
%\item Here are the equations:

{\scs
\[\begin{array}{*{13}{r}D}
 \fh_0 &+&\fh_1 &+&\fh_2 &+&\fh_3 &+&\fh_4 &+&\fh_5   &=& \sqrt{2}
  &{(admissibility)}
  \\
 \fh_0 &-&\fh_1 &+&\fh_2 &-&\fh_3 &+&\fh_4 &-&\fh_5 &=& 0
  &{(vanishing 0th moment)}%, partition of unity, zero at $-1$)}
  \\
      &-&\fh_1 &+& 2h_2 &-& 3h_3 &+& 4h_4 &-& 5h_5   &=& 0
  &{(vanishing 1st moment)}
  \\
      &-&\fh_1 &+& 4h_2 &-& 9h_3 &+& 16h_4 &-& 25h_5  &=& 0
  &{(vanishing 2nd moment)}
  \\
 \fh_0^2 &+&\fh_1^2 &+&\fh_2^2 &+&\fh_3^2 &+&\fh_4^2 &+&\fh_5^2 &=& 1
  &{($m=0$ orthonormal quadrature condition)}
  \\
 \fh_0h_2 &+&\fh_1h_3 &+&\fh_2h_4 &+&\fh_3h_5  &&  &&    &=& 0
  &{($m=1$ orthonormal quadrature condition)}
%  \\
% \fh_0h_4 &+&\fh_1h_5 &&  &&  &&    &=& 0
%  &\text{($m=2$ orthonormal quadrature condition)}
\end{array}\]
}

%\item Solutions using {\em Maxima}\texttrademark \\
%  \begin{lstlisting}[language=MuPAD]
%    solve([h0+h1+h2+h3+h4+h5=sqrt(2),h0-h1+h2-h3+h4-h5=0,
%           -h1+2*h2-3*h3+4*h4-5*h5=0, -h1+4*h2-9*h3+16*h4-25*h5=0,
%           h0^2+h1^2+h2^2+h3^2+h4^2+h5^2=1,h0*h2+h1*h3+h2*h4+h3*h5=0],
%          [h0,h1,h2,h3,h4,h5]
%         );
%  \end{lstlisting}
%
%\item  No solutions at this time. %\problem
%\end{enumerate}
\end{example}






%%We end this chapter with some simple wavelet design examples.
%\begin{enumerate}
%\item Let $\spV_0$ be the space of all piecewise constant functions
%      on intervals $\set{[n,nb)}{n\in\Z}$.
%
%\item The sequence $\{\spV_j\}$ is an MRA:
%  \begin{enumerate}
%  \item $\spV_j \subset \spV_{j+1}$
%  \item $\ff(x)\in\spV_j \iff    \ff(t-2^{-j}n)\in\spV_j $
%  \item $\ff(x)\in\spV_j \iff    \ff(2x)\in\spV_{j+1}$
%  \item $\ds \bigcap_{j\in\Z} \spV_j = \{0\}$
%  \item $\ds \cls{\bigcup_{j\in\Z} \spV_j} = \spLL $
%  \item Riesz basis
%
%   {\color{figcolor} \begin{fsL} \begin{center}
%   \setlength{\unitlength}{0.10mm}
%   \begin{picture}(350,200)(-100,-50)
%     %\graphpaper[10](0,0)(200,200)
%     \thinlines
%     \put(-100,   0){\line(1,0){300} }
%     \put(   0,   5){\line(0,1){ 95} }
%     \put(   0, 100){\line(1,0){ 95} }
%     \put( 100,   0){\line(0,1){ 95} }
%
%     \put(  0,   0){\circle {10}}
%     \put(  0, 100){\circle*{10}}
%     \put(100,   0){\circle*{10}}
%     \put(100, 100){\circle {10}}
%
%     \put( 210,   0){\makebox(0,0)[l]{$t$}}
%     \put(-10,100){\makebox(0,0)[r]{$a$}}
%     \put(0,-10){\makebox(0,0)[t]{$0$}}
%     \put(100,-10){\makebox(0,0)[t]{$b$}}
%     \put(120,50){\makebox(0,0)[l]{$\ftheta(x)$}}
%   \end{picture}
%   \end{center} \end{fsL} }
%  \end{enumerate}
%
%\item Find orthonormal basis $\{\fphi(x-n)\}$ for $\spV_0$:
%
%The set $\{\ftheta(x-n)\}$ is already orthogonal.
%To make it orthonormal, let $a=1/\sqrt{b}$. Then
%  \begin{eqnarray*}
%    \norm{\theta(x)}^2
%      &=& \norm{a\setind_{[0,b)}}^2
%       =  a^2 \norm{\setind_{[0,b)}}^2
%       =  a^2 \int_{x\in\R} \setind_{[0,b)} \setind_{[0,b)} \dx
%       =  a^2 \int_0^b \dx
%       =  a^2 b
%       =  1
%  \end{eqnarray*}
%   {\color{figcolor} \begin{fsL} \begin{center}
%   \setlength{\unitlength}{0.10mm}
%   \begin{picture}(350,200)(-100,-50)
%     %\graphpaper[10](0,0)(200,200)
%     \thinlines
%     \put(-100,   0){\line(1,0){300} }
%     \put(   0,   5){\line(0,1){ 95} }
%     \put(   0, 100){\line(1,0){ 95} }
%     \put( 100,   0){\line(0,1){ 95} }
%
%     \put(  0,   0){\circle {10}}
%     \put(  0, 100){\circle*{10}}
%     \put(100,   0){\circle*{10}}
%     \put(100, 100){\circle {10}}
%
%     \put( 210,   0){\makebox(0,0)[l]{$t$}}
%     \put(-10,100){\makebox(0,0)[r]{$\ds a=\frac{1}{\sqrt{b}}$}}
%     \put(0,-10){\makebox(0,0)[t]{$0$}}
%     \put(100,-10){\makebox(0,0)[t]{$b$}}
%     \put(120,50){\makebox(0,0)[l]{$\fphi(x)$}}
%   \end{picture}
%   \end{center} \end{fsL} }
%
%\item Compute $\seqn{\fh_n}$:
%From Proposition~\ref{prop:phi->h}
%\begin{eqnarray*}
%  \fh_{n}
%    &=& \sqrt{2}            \inprod{\fphi(x)}{\fphi(2x-n)}
%  \\&=& \sqrt{2}            \inprod{\frac{1}{\sqrt{b}}\setind_{[0,b)}(x)}{\frac{1}{\sqrt{b}}\setind_{[0,b)}(2x-n)}
%  \\&=& \sqrt{2}\frac{1}{b} \inprod{\setind_{[0,b)}(x)}{\setind_{[0,b)}(2x-n)}
%  \\&=& \sqrt{2}\frac{1}{b} \inprod{\setind_{[0,b)}(x)}{\setind_{[0,b)}(2x-n)}
%  \\&=& \left\{\begin{array}{ll}
%          \sqrt{2}\frac{1}{b} \frac{b}{2} &:n=0,1 \\
%          0                               &:\text{otherwise.}
%        \end{array}\right.
%  \\&=& \left\{\begin{array}{ll}
%          \cwt &:n=0,1 \\
%          0                  &:\text{otherwise.}
%        \end{array}\right.
%\end{eqnarray*}
%
%   {\color{figcolor} \begin{center}\begin{fsL}
%   \setlength{\unitlength}{0.10mm}
%   \begin{picture}(400,440)(-180,-220)
%     %\graphpaper[10](0,0)(200,200)
%     \thinlines
%     \put(-100,   0){\line(1,0){300} }
%     \put( 210,   0){\makebox(0,0)[l]{$n$}}
%
%     \put(   0,   0){\line(0,1){71} }
%     \put(   0,  71){\circle*{10} }
%     \put( 100,   0){\line(0,1){71} }
%     \put( 100,  71){\circle*{10} }
%
%     \multiput(-100,-100)(100,0){4}{\line(0,1){10} }
%     \put(-100, -110){\makebox(0,0)[t]{$-1$}}
%     \put(   0, -110){\makebox(0,0)[t]{$0$}}
%     \put( 100, -110){\makebox(0,0)[t]{$1$}}
%     \put( 200, -110){\makebox(0,0)[t]{$2$}}
%
%     \multiput(-140,-100)(0,25){9}{\line(1,0){10} }
%     \put(-150, 100){\makebox(0,0)[r]{$1.00$}}
%     \put(-150,  50){\makebox(0,0)[r]{$0.50$}}
%     \put(-150,   0){\makebox(0,0)[r]{$0$}}
%     \put(-150, -50){\makebox(0,0)[r]{$-0.50$}}
%     \put(-150,-100){\makebox(0,0)[r]{$-1.00$}}
%   \end{picture}
%   \end{fsL}\end{center}}
%
%
%
%\item Compute $\seqn{\fg_n}$:
%From \pref{thm:ortho_qmr}
%\begin{eqnarray*}
%  \fg_{n}
%    &=& \pm (-1)^n  h(\xN-n)
%  \\&=& (-1)^n  h(1-n)
%  \\&=& \left\{\begin{array}{ll}
%          +\cwt &:n=0 \\
%          -\cwt &:n=1 \\
%          0                  &:\text{otherwise.}
%        \end{array}\right.
%\end{eqnarray*}
%
%   {\color{figcolor} \begin{center}\begin{fsL}
%   \setlength{\unitlength}{0.10mm}
%   \begin{picture}(400,440)(-180,-220)
%     %\graphpaper[10](0,0)(200,200)
%     \thinlines
%     \put(-100,   0){\line(1,0){300} }
%     \put( 210,   0){\makebox(0,0)[l]{$n$}}
%
%     \put(   0,   0){\line(0,1){71} }
%     \put(   0,  71){\circle*{10} }
%     \put( 100,   0){\line(0,-1){71} }
%     \put( 100, -71){\circle*{10} }
%
%     \multiput(-100,-100)(100,0){4}{\line(0,1){10} }
%     \put(-100, -110){\makebox(0,0)[t]{$-1$}}
%     \put(   0, -110){\makebox(0,0)[t]{$0$}}
%     \put( 100, -110){\makebox(0,0)[t]{$1$}}
%     \put( 200, -110){\makebox(0,0)[t]{$2$}}
%
%     \multiput(-140,-100)(0,25){9}{\line(1,0){10} }
%     \put(-150, 100){\makebox(0,0)[r]{$1.00$}}
%     \put(-150,  50){\makebox(0,0)[r]{$0.50$}}
%     \put(-150,   0){\makebox(0,0)[r]{$0$}}
%     \put(-150, -50){\makebox(0,0)[r]{$-0.50$}}
%     \put(-150,-100){\makebox(0,0)[r]{$-1.00$}}
%   \end{picture}
%   \end{fsL}\end{center}}
%
%\item Compute $\fpsi(x)$:
%\begin{eqnarray*}
%  \fpsi(x)
%    &=& \sqrt{2} \sum_{n\in\Z} \fg_{n}  \fphi(2x-n)
%  \\&=& \sqrt{2} \left[\cwt \fphi(2x-0)-\cwt \fphi(2x-1)\right]
%  \\&=& \fphi(2x)-\fphi(2x-1)
%\end{eqnarray*}
%
%   {\color{figcolor} \begin{fsL} \begin{center}
%   \setlength{\unitlength}{0.10mm}
%   \begin{picture}(350,300)(-100,-150)
%     %\graphpaper[10](0,0)(200,200)
%     \thinlines
%     \put(-100,   0){\line(1,0){300} }
%     \put(   0,   5){\line(0,1){ 95} }
%     \put(   0, 100){\line(1,0){ 45} }
%     \put(  50,-100){\line(0,1){195} }
%     \put(  50,-100){\line(1,0){ 45} }
%     \put( 100,- 95){\line(0,1){100} }
%
%     \put(  0,   0){\circle {10}}
%     \put(  0, 100){\circle*{10}}
%     \put( 50, 100){\circle {10}}
%     \put( 50,-100){\circle*{10}}
%     \put(100,   0){\circle*{10}}
%     \put(100,-100){\circle {10}}
%
%     \put( 210,   0){\makebox(0,0)[l]{$t$}}
%     \put(-20, 100){\makebox(0,0)[r]{$\ds  \frac{1}{\sqrt{b}}$}}
%     \put(-20,-100){\makebox(0,0)[r]{$\ds -\frac{1}{\sqrt{b}}$}}
%     \put(0,-10){\makebox(0,0)[t]{$0$}}
%     \put(100,10){\makebox(0,0)[b]{$b$}}
%     \put( 70,70){\makebox(0,0)[l]{$\fpsi(x)$}}
%   \end{picture}
%   \end{center} \end{fsL} }
%\end{enumerate}







