%============================================================================
% LaTeX File
% Daniel Greenhoe
%============================================================================

%======================================
\chapter{Wavelets with Compact Support}
%======================================



%=======================================
\section {Using simultaneous equations to compute scaling coefficients}
%=======================================
The scaling coefficients can be directly computed using
simultaneous equations generated from the following 
necessary conditions. \cittrpp{vidakovic}{92}{93}
They can be solved using a symbolic equation software package 
\footnote{
\begin{tabular}[t]{lll}
  {\em Maxima}\texttrademark      & \url{http://maxima.sourceforge.net/}  & free and open source   \\
  {\em Macsyma}\texttrademark     & \url{http://www.scientek.com/macsyma/mxmain.htm} & a very non-free relative of {\em Maxima}\\
  {\em MuPAD}\texttrademark       & \url{http://www.mupad.de/}            & 30 day free trial   \\
  {\em Maple}\texttrademark       & \url{http://www.maplesoft.com/} & \\
  {\em Mathematica}\texttrademark & \url{http://www.wolfram.com/} & 
  \end{tabular}
  }
or by hand.
\[\begin{array}{llcllll}
  1. & \ds\sum_n  h_n   &=& \sqrt{2} 
     & \text{(admissiblility condition)}
     & \text{\pref{thm:admiss}}
     & \text{page~\pageref{thm:admiss}}
\\
\\
  2. & \ds\sum_m h_m \fh^\ast_{m-2n} &=& \kdelta_n
     & \text{(quadrature condition)}
     & \text{\pref{thm:quadcon}}
     & \text{page~\pageref{thm:quadcon}}
\\
\\
  3. & \ds\sum_m (-1)^m m^p  h_m   &=& 0
     & \text{(vanishing $p$th moment)}
     & \text{\pref{thm:vanish}} 
     & \text{page~\pageref{thm:vanish}}
\end{array}\]

Note that the ``zero at $z=-1$" constraint 
(\pref{thm:zero_unity_evenodd} page~\pageref{thm:zero_unity_evenodd})
is just the vanishing 0th moment (and hence included in 3. for $p=0$).
Also, the ``partition of unity" and ``even-odd" constraints 
are, by the same theorem, equivalent  to the 
``zero at $z=-1$" constraint.


%=======================================
\subsection {Simultaneous system of 2 coefficients}
\label{sec:sim_two}
%=======================================
Several methods can be used to compute the values of the 
two scaling coefficient case.
All of the methods use the admissibility equation plus one other equation.
All of these methods yield the same result.
That is, there is no degrees of freedom for the two scaling coefficient
case.
\exbox{\begin{array}{r|l}
  n  &  h_n  \\
  \hline
  0  & \ds \frac{1}{\sqrt{2}}  \\
  1  & \ds \frac{1}{\sqrt{2}}
\end{array}}

\begin{enumerate}
\item Admissibility and 1 vanishing moment constraints:\\
\begin{align*}
  h_0 + h_1 &= \sqrt{2}    &&\text{(admissibility equation)} \\
  h_0 - h_1 &= 0           &&\text{(zero at $-1$ equation)}  \\
  \\
  2h_0 &= \sqrt{2}         &&\text{(add two equations together)}\\
  2h_1 &= \sqrt{2}         &&\text{(subtract second from first)} \\
  \\
  h_0 &= \frac{1}{\sqrt{2}} \\
  h_1 &= \frac{1}{\sqrt{2}} 
\end{align*}


\item Admissibility and partition of unity with $\int_t\phi(t)\dt=1$ constraints:\\
By \pref{thm:zero_unity_evenodd} (page~\pageref{thm:zero_unity_evenodd}),
this is equivalent to ``1." above.

\item Admissibility and orthogonal quadrature constraints:
\begin{align*}
  h_0 + h_1 &= \sqrt{2}    &&\text{(admissibility constraint)} \\
  h_0^2 + h_1^2 &= 1       &&\text{(quadrature constraint)}  \\
  \\
  h_0^2 + (\sqrt{2}-h_0)^2        &= 1  \\
  h_0^2 + (h_0^2 -2\sqrt{2}h_0+2) &= 1  \\
  2h_0^2 + -2\sqrt{2}h_0+1        &= 0  
\end{align*}
\begin{align*}
  h_0 & = \frac{2\sqrt{2} \pm \sqrt{8-8}}{4}  
      &&= \frac{2\sqrt{2}}{4}  
      &&= \frac{1}{\sqrt{2}}  
      \\
  h_1 &= \sqrt{2} - \frac{1}{\sqrt{2}} 
      &&= \frac{1}{\sqrt{2}}  
\end{align*}

\end{enumerate}



%=======================================
\subsection {Simultaneous system of 4 coefficients}
\label{sec:sim_four}
%=======================================
\begin{proof}
\begin{enumerate}

\item Solution using admissibility constraint, one quadrature constraint, and two vanishing moment constraints:

\begin{enumerate}
\item Set of simultaneous equations:
\[\begin{array}{*{9}{r}l}
  h_0      &+& h_1      &+& h_2   &+& h_3   &=& \sqrt{2}      
  &\text{(admissibility)} \\
  h_0      &-& h_1      &+& h_2   &-& h_3   &=& 0          
  &\text{(vanishing 0th moment, partition of unity, zero at $-1$)} \\
           &-& h_1      &+& 2h_2  &-&3h_3   &=& 0
  &\text{(vanishing 1st moment)} \\
  h_0^2    &+& h_1^2    &+& h_2^2 &+& h_3^2 &=& 1                  
  &\text{($m=0$ orthonormal quadrature condition)} \\
\end{array}\]

\item Solve the 3 linear equations in terms of $h_3$:
\begin{align*}
\left[\begin{array}{rrrr@{\hspace{3ex}}r}
    1 &  1 &  1 &  1 & \sqrt{2}   \\
    1 & -1 &  1 & -1 & 0          \\
    0 & -1 &  2 & -3 & 0
\end{array}\right]
&\rightarrow&
\left[\begin{array}{rrrr@{\hspace{3ex}}r}
    1 &  1 &  1 &  1 & \sqrt{2}   \\
    0 & -2 &  0 & -2 &-\sqrt{2}   \\
    0 & -1 &  2 & -3 & 0          
\end{array}\right]
&\rightarrow&
\left[\begin{array}{rrrr@{\hspace{3ex}}r}
    1 &  0 &  3 & -2 & \sqrt{2}   \\
    0 &  0 & -4 &  4 &-\sqrt{2}   \\
    0 &  1 & -2 &  3 & 0          
\end{array}\right]
&\rightarrow&
\\ 
\left[\begin{array}{rrrr@{\hspace{3ex}}r}
    1 &  0 &  3 & -2 & \sqrt{2}   \\
    0 &  1 & -2 &  3 & 0          \\
    0 &  0 &  1 & -1 & \frac{\sqrt{2}}{4}
\end{array}\right]
&\rightarrow&
\left[\begin{array}{rrrr@{\hspace{3ex}}r}
    1 &  0 &  0 &  1 & \frac{\sqrt{2}}{4}   \\
    0 &  1 &  0 &  1 & \cwt    \\
    0 &  0 &  1 & -1 & \frac{\sqrt{2}}{4}
\end{array}\right]
\end{align*}
\begin{align*}
  h_0 &= \frac{\sqrt{2}}{4} - h_3  \\
  h_1 &= \cwt  - h_3  \\
  h_2 &= \frac{\sqrt{2}}{4} + h_3  \\
\end{align*}

\item Solve the forth quadratic equation in terms of $h_3$:
\begin{align*}
  1 
    &=  \left(\frac{\sqrt{2}}{4}-h_3\right)^2
    &&+ \left(\cwt -h_3\right)^2
    &&+ \left(\frac{\sqrt{2}}{4}+h_3\right)^2
    &&+ h_3^2 \hs{1cm}
  \\&=  \left(\frac{1}{8}-\sqrt{2}{2}h_3+h_3^2\right)
    &&+ \left(\frac{1}{2}-\sqrt{2}h_3+h_3\right)
    &&+ \left(\frac{1}{8}+\cwt +h_3^2\right)
    &&+ h_3^2 \hs{1cm}
  \\&=  4h_3^2 -\sqrt{2}h_3 + \frac{3}{4}
\\
  4h_3^2 -\sqrt{2}h_3 - \frac{1}{4} &=0
\\
  h_3 
    &= \frac{\sqrt{2}\pm\sqrt{2+4}}{8}
  \\&= \frac{\sqrt{2}\pm\sqrt{6}}{8}
  \\&= \frac{\sqrt{2}}{8}(1\pm\sqrt{3})
\end{align*}

\item The results:
\exbox{\begin{array}{r*{4}{cl}}
  h_3 &=& \frac{\sqrt{2}}{8}(1-\sqrt{3})  
\\
  h_2 &=& \frac{\sqrt{2}}{4} + h_3  
      &=& \frac{\sqrt{2}}{4} + \frac{\sqrt{2}}{8}(1-\sqrt{3})
      &=& \frac{\sqrt{2}}{8}(2+1-\sqrt{3})
      &=& \frac{\sqrt{2}}{8}(3-\sqrt{3})
\\      
  h_1 &=& \cwt  - h_3
      &=& \cwt  - \frac{\sqrt{2}}{8}(1-\sqrt{3})
      &=& \frac{\sqrt{2}}{8}(4-1+\sqrt{3})
      &=& \frac{\sqrt{2}}{8}(3+\sqrt{3})
\\      
  h_0 &=& \frac{\sqrt{2}}{4} - h_3  
      &=& \frac{\sqrt{2}}{4} - \frac{\sqrt{2}}{8}(1-\sqrt{3})
      &=& \frac{\sqrt{2}}{8}(2-1+\sqrt{3})
      &=& \frac{\sqrt{2}}{8}(1+\sqrt{3})
\end{array}}

\item Solution using {\em Maxima}\texttrademark \hspace{1ex}(2 sets of solutions): \\
{\scriptsize
\begin{verbatim}
(%i36) solve([h0+h1+h2+h3=sqrt(2),h0-h1+h2-h3=0,-h1+2*h2-3*h3=0,h0^2+h1^2+h2^2+h3^2=1],[h0,h1,h2,h3]);
                SQRT(6) - SQRT(2)         SQRT(2) SQRT(3) - 3 SQRT(2)
(%o36) [[h0 = ------------------, h1 = ----------------------------, 
                       8                             8

     SQRT(2) SQRT(3) + 3 SQRT(2)       SQRT(2) SQRT(3) + SQRT(2)
h2 =---------------------------, h3 =-------------------------], 
                 8                               8

      SQRT(6) + SQRT(2)	      SQRT(2) SQRT(3) + 3 SQRT(2)
[h0 =-----------------, h1 =---------------------------, 
              8                           8

       SQRT(2) SQRT(3) - 3 SQRT(2)         SQRT(2) SQRT(3) - SQRT(2)
h2 = ----------------------------, h3 = --------------------------]]
                  8                                   8
\end{verbatim}
}
\end{enumerate}




\item Solution using admissibility constraint, two quadrature constraints, and one vanishing moment constraint:
\begin{enumerate}
\item Set of simultaneous equations:
\[\begin{array}{*{9}{r}l}
  h_0      &+& h_1      &+& h_2   &+& h_3   &=& \sqrt{2}      
  &\text{(admissibility)} 
  \\
  h_0      &-& h_1      &+& h_2   &-& h_3   &=& 0          
  &\text{(vanishing 0th moment, partition of unity, zero at $-1$)} 
  \\
  h_0^2    &+& h_1^2    &+& h_2^2 &+& h_3^2 &=& 1                  
  &\text{($m=0$ orthonormal quadrature condition)} 
  \\
  \fh_0\fh_2 &+& \fh_1\fh_3 & &         & &         &=& 0
  &\text{($m=1$ orthonormal quadrature condition)} 
\end{array}\]

\item These equations are linearly independent, however they are 
      still dependent\footnote{
        Many thanks to X\=in L\'in X\`ie for pointing this out to me!!!
%        \begin{tabular}[t]{ccc}
%          {\MiQ\cH225} & \z{\MgQ\cH106} & \z{\MjQ\cH204}  \\
%          X\`ie        & X\=in          & L\'in           \\
%          to thank     & glad           & continuous heavy rain
%        \end{tabular}
        }
      as shown next:
  \begin{align*}
    &  [(h_0+h_1+h_2+h_3)+(h_0-h_1+h_2-h_3)]^2+[(h_0+h_1+h_2+h_3)-(h_0-h_1+h_2-h_3)]^2
  \\&= [2h_0+2h_2]^2+[2h_1+2h_3]^2
  \\&= 4[h_0+h_2]^2+4[h_1+h_3]^2
  \\&= 4[h_0^2+2h_0h_2+h_2^2]+4[h_1^2+2h_1h_3+h_3^2]
  \\&= 4[h_0^2 + h_1^2 + h_2^2 + h_3^2 + 2h_0h_2 + 2h_1h_3]
  \\&= 4(h_0^2 + h_1^2 + h_2^2 + h_3^2) + 8(h_0h_2 + h_1h_3)
  \\ \implies & \text{The last two equations are (non-linearly) dependent on the first two.}
  \end{align*}

\item Because of this dependence, the system of equations has one degree of freedom.

\item Solutions using {\em Maxima}\texttrademark \\
  \begin{verbatim}
    solve([h0+h1+h2+h3=sqrt(2),h0-h1+h2-h3=0,h0^2+h1^2+h2^2+h3^2=1,h0*h2+h1*h3=0],
          [h0,h1,h2,h3]
         );
  \end{verbatim}
\begin{enumerate}
\item {\em Maxima} first solution:
  \begin{align*}
    h_1 &= \frac{\sqrt{2}}{4}\left(\sqrt{-8h_0^2+4\sqrt{2}h_0+1}+1\right)
  \\h_2 &= \frac{2h_0-\sqrt{2}}{2}
  \\h_3 &= \frac{\sqrt{2}}{4}\left(\sqrt{-8h_0^2+4\sqrt{2}h_0+1}-1\right)
  \end{align*}

\item {\em Maxima} second solution:
  \begin{align*}
    h_1 &= \frac{\sqrt{2}}{4}\left(\sqrt{-8h_0^2+4\sqrt{2}h_0+1}-1\right)
  \\h_2 &= -\frac{2h_0-\sqrt{2}}{2}
  \\h_3 &= \frac{\sqrt{2}}{4}\left(\sqrt{-8h_0^2+4\sqrt{2}h_0+1}+1\right)
  \end{align*}
\end{enumerate}
\end{enumerate}
\end{enumerate}
\end{proof}

%=======================================
\subsection {Simultaneous system of 6 coefficients}
\label{sec:sim_six}
%=======================================
\begin{proof}
\begin{enumerate}

\item Solution using admissibility constraint, two quadrature constraints, 
      and three vanishing moment constraints:

\begin{enumerate}
\item Set of simultaneous equations:
\[\begin{array}{*{13}{r}l}
  h_0 &+& h_1 &+& h_2 &+& h_3 &+& h_4 &+& h_5   &=& \sqrt{2}      
  &\text{(admissibility)} 
  \\
  h_0 &-& h_1 &+& h_2 &-& h_3 &+& h_4 &-& h_5 &=& 0          
  &\text{(vanishing 0th moment, partition of unity, zero at $-1$)} 
  \\
      &-& h_1 &+& 2h_2 &-& 3h_3 &+& 4h_4 &-& 5h_5   &=& 0
  &\text{(vanishing 1st moment)} 
  \\
      &-& h_1 &+& 4h_2 &-& 9h_3 &+& 16h_4 &-& 25h_5  &=& 0
  &\text{(vanishing 2nd moment)} 
  \\
  h_0^2 &+& h_1^2 &+& h_2^2 &+& h_3^2 &+& h_4^2 &+& h_5^2 &=& 1                  
  &\text{($m=0$ orthonormal quadrature condition)} 
  \\
  h_0h_2 &+& h_1h_3 &+& h_2h_4 &+& h_3h_5  &&  &&    &=& 0
  &\text{($m=1$ orthonormal quadrature condition)}
%  \\
%  h_0h_4 &+& h_1h_5 &&  &&  &&    &=& 0  
%  &\text{($m=2$ orthonormal quadrature condition)}
\end{array}\]

\item Solutions using {\em Maxima}\texttrademark \\
  \begin{verbatim}
    solve([h0+h1+h2+h3+h4+h5=sqrt(2),h0-h1+h2-h3+h4-h5=0,
           -h1+2*h2-3*h3+4*h4-5*h5=0, -h1+4*h2-9*h3+16*h4-25*h5=0,
           h0^2+h1^2+h2^2+h3^2+h4^2+h5^2=1,h0*h2+h1*h3+h2*h4+h3*h5=0],
          [h0,h1,h2,h3,h4,h5]
         );
  \end{verbatim}
  
  No solutions at this time. \problem
\end{enumerate}
\end{enumerate}
\end{proof}

%=======================================
\section {Pollen Parameterized Wavelets}
%=======================================
David Pollen showed that there are an uncountably infinite number of 
compactly supported scaling functions (and wavelets) available for 
coefficient length 4 and greater.
This is stated in \pref{thm:pollen},
demonstrated in Example~\ref{ex:pollen4} and 
illustrated in Figure~\ref{fig:pollen4}.
%---------------------------------------
\begin{theorem}[Pollen parameterization theorem]
\label{thm:pollen}
\citetbl{
  \citerp{ak}{3}
  }
\index{Pollen parameterization theorem}
\index{theorems!Pollen parameterization theorem}
%---------------------------------------
\formbox{
\begin{aligned}
  \Zh(z) 
    &= \cwt 
       \left[-1 \quad 1 \right]
       E(z^2)
       \left[\begin{array}{c}-z^{-1}\\ 1 \end{array}\right]
  \\[1ex]
    & \text{where}
  \\[2ex]
  E(z)
    &= \begin{cases}
         U_1 U_2 \dotsm U_{2k}    & \text{for $ h_n $ of length $4k+2$, $k\in\Znn$} \\
         U_1 U_2 \dotsm U_{2k+1}  & \text{for $ h_n $ of length $4k+4$, $k\in\Znn$}
       \end{cases}
  \\
  U(z)
    &= \left[\begin{array}{rr}
         \fu(z)      & \fv(z) \\
        -\fv(z^{-1}) & \fu(z^{-1})
       \end{array}\right]
  \\
  \fu(z) &= \frac{1}{2}\left[(1-\cos\theta)z + (1+\cos\theta) \right]
  \\
  \fv(z) &= \frac{1}{2}\left[(-\sin\theta) + (\sin\theta)z^{-1} \right]
\end{aligned}
}
\end{theorem}
\begin{proof}
No proof at this time.
\end{proof}

%---------------------------------------
\begin{example}[Pollen length-4 scaling coefficients]
\label{ex:pollen4}
\citetbl{
  \citerp{ak}{3}\\
  \citerp{burrus}{66}
  }
%---------------------------------------
\exbox{\begin{array}{r|l}
  n  &  h_n  \\
  \hline
  0  & \ds \frac{\sqrt{2}}{4}(1-\cos\theta+\sin\theta)  \\
  1  & \ds \frac{\sqrt{2}}{4}(1+\cos\theta+\sin\theta)  \\ 
  2  & \ds \frac{\sqrt{2}}{4}(1+\cos\theta-\sin\theta)  \\ 
  3  & \ds \frac{\sqrt{2}}{4}(1-\cos\theta-\sin\theta)
\end{array}}
\end{example}
\begin{proof}
\begin{align*}
  \Zh(z) 
    &= \cwt 
       \left[-1 \quad 1 \right]
       E(z^2)
       \left[\begin{array}{c}-z^{-1}\\ 1 \end{array}\right]
  \\&= \cwt 
       \left[-1 \quad 1 \right]
       U(z^2)
       \left[\begin{array}{c}-z^{-1}\\ 1 \end{array}\right]
  \\&= \cwt 
       \left[-1 \quad 1 \right]
       \left[\begin{array}{rr}
         \fu(z^2)      & \fv(z^2) \\
        -\fv(z^{-2})   & \fu(z^{-2})
       \end{array}\right]
       \left[\begin{array}{c}-z^{-1}\\ 1 \end{array}\right]
  \\&= \cwt 
       \left[-1 \quad 1 \right]
       \left[\begin{array}{cc}
           \frac{1}{2}\left[(1-\cos\theta)z^2 + (1+\cos\theta) \right]
       &   \frac{1}{2}\left[(-\sin\theta) + (\sin\theta)z^{-2} \right]
       \\ -\frac{1}{2}\left[(-\sin\theta) + (\sin\theta)z^{2} \right]
       &   \frac{1}{2}\left[(1-\cos\theta)z^{-2} + (1+\cos\theta) \right]
       \end{array}\right]
       \left[\begin{array}{c}-z^{-1}\\ 1 \end{array}\right]
  \\&= \frac{\sqrt{2}}{4}
       \left[\begin{array}{ll}
          -\left[(1-\cos\theta)z^2 + (1+\cos\theta) \right]
          +\left[(-\sin\theta) + (\sin\theta)z^{2} \right]
       \\ -\left[(-\sin\theta) + (\sin\theta)z^{-2} \right]
          +\left[(1-\cos\theta)z^{-2} + (1+\cos\theta) \right]
       \end{array}\right]^T
       \left[\begin{array}{c}-z^{-1}\\ 1 \end{array}\right]
  \\&= \frac{\sqrt{2}}{4}
       \Bigg[
          (1-\cos\theta)z +(1+\cos\theta)z^{-1}
          -(\sin\theta)z^{-1} +(\sin\theta)z
          + \sin\theta -(\sin\theta)z^{-2}
          +(1-\cos\theta)z^{-2} + (1+\cos\theta)
       \Bigg]
  \\&= \frac{\sqrt{2}}{4}
       \Bigg[
           (1-\cos\theta+\sin\theta)z
          +(1+\cos\theta+\sin\theta)
  \\&\qquad          +(1+\cos\theta-\sin\theta)z^{-1}
          +(1-\cos\theta-\sin\theta)z^{-2}
       \Bigg]
\end{align*}
\end{proof}


\begin{figure}[ht] \color{figcolor}
\begin{center}
\begin{tabular*}{\textwidth}{@{\extracolsep{\fill}}cc}
   \epsfig{file=../common/wavelets/pollen4h.eps, height=7cm, width=7cm}  &
   \epsfig{file=../common/wavelets/pollen4g.eps, height=7cm, width=7cm}
   \\
   $h_n(\theta)$ & $g_n(\theta)$
   \\
   \epsfig{file=../common/wavelets/pollen4s.eps, height=7cm, width=7cm}  &
   \epsfig{file=../common/wavelets/pollen4w.eps, height=7cm, width=7cm}
   \\
   $\fphi(t)$ & $\fpsi(t)$
\end{tabular*}
\caption{Pollen scaling and wavelet functions with varying parameter $\theta$
  \label{fig:pollen4}
  }
\end{center}
\end{figure}


\clearpage
%---------------------------------------
\begin{example}[Pollen length-6 scaling coefficients]
\label{ex:pollen6}
\citetbl{
  \citerp{burrus}{66}\\
  \citerp{vidakovic}{95}
  }
%---------------------------------------
Length-6 scaling coefficients have two parameters, 
here designated $\alpha$ and $\beta$.
The scaling coefficients are as follows.
\exbox{
 \begin{array}{c|l}
   n & h_n
   \\ \hline
      0 & \frac{1}{4\sqrt{2}} [(1+\cos(\alpha)+\sin(\alpha))(1-\cos(\beta)-\sin(\beta)) + 2\sin(\beta)\cos(\alpha)] 
   \\ 1 & \frac{1}{4\sqrt{2}}[(1-\cos(\alpha)+\sin(\alpha))(1+\cos(\beta)-\sin(\beta)) - 2\sin(\beta)\cos(\alpha)] 
   \\ 2 & \frac{1}{2\sqrt{2}}[(1+\cos(\alpha-\beta)+\sin(\alpha-\beta)] 
   \\ 3 & \frac{1}{2\sqrt{2}}[(1+\cos(\alpha-\beta)-\sin(\alpha-\beta)] 
   \\ 4 & \frac{1}{ \sqrt{2}} - h(0) - h(2) 
   \\ 5 & \frac{1}{ \sqrt{2}} - h(1) - h(3) 
  \end{array}
}
\end{example}
\begin{proof}
\begin{align*}
  \Zh(z) 
    &= \cwt 
       \left[-1 \quad 1 \right]
       E(z^2)
       \left[\begin{array}{c}-z^{-1}\\ 1 \end{array}\right]
  \\&= \cwt 
       \left[-1 \quad 1 \right]
       U_\alpha(z^2)
       U_\beta(z^2)
       \left[\begin{array}{c}-z^{-1}\\ 1 \end{array}\right]
  \\&= \cwt 
       \left[-1 \quad 1 \right]
       \left[\begin{array}{rr}
         \fu_\alpha(z^2)      & \fv_\alpha(z^2) \\
        -\fv_\alpha(z^{-2})   & \fu_\alpha(z^{-2})
       \end{array}\right]
       \left[\begin{array}{rr}
         \fu_\beta(z^2)      & \fv_\beta(z^2) \\
        -\fv_\beta(z^{-2})   & \fu_\beta(z^{-2})
       \end{array}\right]
       \left[\begin{array}{c}-z^{-1}\\ 1 \end{array}\right]
  \\&= \cwt 
       \left[\begin{array}{c@{\qquad}c}
         -\fu_\alpha(z^2)-\fv_\alpha(z^{-2}) &
         -\fv_\alpha(z^2)+\fu_\alpha(z^{-2})
       \end{array}\right]
       \left[\begin{array}{r}
         -z^{-1}\fu_\beta(z^2)+ \fv_\beta(z^2) \\
        z^{-1}\fv_\beta(z^{-2}) +\fu_\beta(z^{-2})
       \end{array}\right]
  \\&= \cwt 
       \left[
         \left(-\fu_\alpha(z^2)        - \fv_\alpha(z^{-2}) \right)
         \left(-z^{-1}\fu_\beta(z^2)   + \fv_\beta(z^2)     \right)+
         \left(-\fv_\alpha(z^2)        + \fu_\alpha(z^{-2}) \right)
         \left(z^{-1}\fv_\beta(z^{-2}) + \fu_\beta(z^{-2})  \right)
       \right]
  \\&= \cwt 
       \Bigg[
         \left(
            z^{-1}\fu_\alpha(z^2)   \fu_\beta(z^2) 
           -      \fu_\alpha(z^2)   \fv_\beta(z^2)
           +z^{-1}\fv_\alpha(z^{-2})\fu_\beta(z^2)
           -      \fv_\alpha(z^{-2})\fv_\beta(z^2)
         \right)
       \\&\qquad+
         \left(
           -z^{-1}\fv_\alpha(z^2)   \fv_\beta(z^{-2})
           -      \fv_\alpha(z^2)   \fu_\beta(z^{-2})
           +z^{-1}\fu_\alpha(z^{-2})\fv_\beta(z^{-2})
           +      \fu_\alpha(z^{-2})\fu_\beta(z^{-2})
         \right)
       \Bigg]
  \\&= \cwt  \frac{1}{4}
       \Bigg[
            z^{-1}\left[(1-\cos\alpha)z^2       + (1+\cos\alpha)     \right]   
                  \left[(1-\cos\beta)z^2        + (1+\cos\beta)      \right] 
       \\&\qquad\qquad
           -      \left[(1-\cos\alpha)z^2       + (1+\cos\alpha)     \right]
                  \left[(-\sin\beta)            + (\sin\beta)z^{-2}  \right]
       \\&\qquad\qquad
           +z^{-1}\left[(-\sin\alpha)           + (\sin\alpha)z^{2}  \right]
                  \left[(1-\cos\beta)z^2        + (1+\cos\beta)      \right]
       \\&\qquad\qquad
           -      \left[(-\sin\alpha)           + (\sin\alpha)z^{2}  \right]
                  \left[(-\sin\beta)            + (\sin\beta)z^{-2}  \right]
       \\&\qquad\qquad
           -z^{-1}\left[(-\sin\alpha)           + (\sin\alpha)z^{-2} \right]
                  \left[(-\sin\beta)            + (\sin\beta)z^{2}   \right]
       \\&\qquad\qquad
           -      \left[(-\sin\alpha)           + (\sin\alpha)z^{-2} \right] 
                  \left[(1-\cos\beta)z^{-2}     + (1+\cos\beta)      \right]
       \\&\qquad\qquad
           +z^{-1}\left[(1-\cos\alpha)z^{-2}    + (1+\cos\alpha)     \right]
                  \frac{1}{2}\left[(-\sin\beta) + (\sin\beta)z^{2}   \right]
       \\&\qquad\qquad
           +      \left[(1-\cos\alpha)z^{-2}    + (1+\cos\alpha)     \right]
                  \left[(1-\cos\beta)z^{-2}     + (1+\cos\beta)      \right]
       \Bigg]
  \\ \vdots
\end{align*}
Proof not yet complete. Current result taken from  \problem \\
\cite[page 66]{burrus}
\end{proof}




