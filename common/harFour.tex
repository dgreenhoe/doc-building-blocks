%============================================================================
% LaTeX File
% Daniel J. Greenhoe
%============================================================================

%======================================
\chapter{Fourier Transform}
\index{harmonic analysis}
\index{fourier analysis}
\label{app:fourier}
%======================================
%20191118 can't sufficiently substantiate
%20191118\qboxnpq{
%20191118  Joseph Fourier (1768--1830)
%20191118  \index{Fourier, Joseph}
%20191118  \index{quotes!Fourier, Joseph}
%20191118  \footnotemark
%20191118  }
%20191118  {../common/people/fourier_bunzil_wkp_pdomain_gray.jpg}
%20191118  {%
%20191118  An arbitrary function, continuous or with discontinuities, defined in a finite interval by an arbitrarily
%20191118  capricious graph can always be expressed as a sum of sinusoids%
%20191118  }
%20191118  \footnotetext{\begin{tabular}[t]{ll}
%20191118    quote: & \citer{fourier1822}, \citer{fourier1822e}, \citerpg{gao2011h}{19}{9781441915450}\\  %https://a2i2.deakin.edu.au/2016/08/24/signal-analysis-fourier-transform-or-wavelet-transform/
%20191118    image: & \url{http://en.wikipedia.org/wiki/File:Fourier2.jpg}, public domain
%20191118  \end{tabular}}

\qboxnpq{
  Joseph Fourier (1768--1830)
  \index{Fourier, Joseph}
  \index{quotes!Fourier, Joseph}
  \footnotemark
  }
  {../common/people/fourier_bunzil_wkp_pdomain_gray.jpg}
  {%
  Up to this point we have supposed that the function
  whose development is required in a series of sines of multiple
  arcs can be developed in a series arranged according to powers
  of the variable x,
  %and that only odd powers enter into that series.
  \ldots
  We can extend the same results to any functions, even
  to those which are discontinuous and entirely arbitrary.
  \ldots
  even entirely arbitrary functions may be developed in series of sines
  of multiple arcs.
  }
  \footnotetext{\begin{tabular}[t]{ll}
    quote: & \citerpc{fourier1822e}{184,186}{\textsection 219,220}\\
    image: & \url{http://en.wikipedia.org/wiki/File:Fourier2.jpg}, public domain
  \end{tabular}}

%======================================
\section{Introduction}
%======================================
Historically, before the Fourier Transform was the Taylor Expansion (transform).
The Taylor Expansion demonstrates that for \propb{analytic} functions
knowledge of the derivatives of a function at a location $x=a$
allows you to determine (predict) arbitrarily closely all the points $\ff(x)$ in the vicinity of $x=a$ \xref{chp:taylor}.
But analytic functions are by definition functions for which all their derivatives exist.
Thus, if a function is \prope{discontinuous}, it is simply not a candidate for a Taylor Expansion.
And some 300 years ago, mathematician giants of the day were fairly content with this.

But then in came an engineer named Joseph Fourier whose day job was working as a governor of lower Egypt under Napolean.
He claimed that, rather than expansion based on derivatives, one could expand based on integrals over sinusoids,
and that this would work not just for analytic functions, but for \propb{discontinuous} ones as well!\footnote{\citePp{robinson1982}{886}}

\begin{minipage}{\tw-80mm}
Needless to say, this did not go over too well initially in the mathematical community.
But over time (on the order of 200 or so years), the Fourier Transform has in many ways won the day.
\end{minipage}\hfill%
\tboxc{\includegraphics[width=75mm]{../common/graphics/portraits/legendre_fourier.jpg}\footnotemark}
\footnotetext{
  Caricature of Legendre (left) and Fourier (right), 1820,
  by Julien-L/'eopold Boilly (1796--1874).
  ``\emph{Album de 73 Portraits-Charge Aquarelle's des Membres de I'Institute} (watercolor portrait \#29).
  Biliotheque de l'Institut de France."
  Public domain.
  \url{https://en.wikipedia.org/wiki/File:Legendre_and_Fourier_(1820).jpg}
  }

%======================================
\section{Definitions}
%======================================
%%======================================
%\section{Domain}
%%======================================
This chapter deals with the \ope{Fourier Transform} in the space of
\structe{Lebesgue square-integrable functions} $\spLLRBu$,
where $\R$ is the set of real numbers, $\borel$ is the set of \structe{Borel sets} on $\R$,
$\msm$ is the standard \fncte{Borel measure} on $\B$, and
\\\indentx$\ds\spLLRBu\eqd\set{\ff\in\clFrr}{\int_\R \abs{\ff}^2 \dmu < \infty}$.\\
Furthermore, $\inprodn$ is the \fncte{inner product} induced by the operator $\int_\R\dmu$ such that
\\\indentx$\ds\inprod{\ff}{\fg} \eqd \int_{\R}\ff(x)\fg^\ast(x)\dx$,\\
and $\opair{\spLLRBu}{\inprodn}$ is a \structe{Hilbert space}.

%--------------------------------------
\begin{definition}
\label{def:fkern}
%--------------------------------------
Let $\fkernn$ be a \structe{function} in $\clF{\R^2}{\C}$.
  \defboxt{
    The function $\fkernn$ is the \fnctd{Fourier kernel} if
    \qquad$\ds\fkernn(x,\omega) \eqd \fkerne{x}{\omega}$
    \qquad$\scy \forall x,\omega\in\R$
    }
\end{definition}

%--------------------------------------
\begin{definition}
\footnote{
  \citerpg{bachman2002}{363}{9780387988993},
  \citerpg{chorin2009}{13}{9781441910011},
  \citerp{loomis1965}{144},
  \citerppg{knappb2005}{374}{375}{0817632506},
  %\citerp{mathews}{102},
  %\citerp{hassani}{219},
  \citor{fourier1822},
  \citorp{fourier1822e}{336?}
  %\citorp{carothers}{2}
  }
\label{def:ft}
\label{def:opFT}
%--------------------------------------
%Let $\fkern{x}{\omega}$ be the \hie{kernel} $\fkern{x}{\omega}\eqd\fkerne{x}{\omega}$.
Let $\spLLRBu$ be the space of all
\structe{Lebesgue square-integrable functions}.\\
\defboxt{
  The \opd{Fourier Transform} operator $\opFT$ is defined as
  \\\indentx$\ds
    \brs{\hxs{\opFT} \ff}(\omega)
    %\eqd \fscale \inprod   {\ff(x)}{\fkern  {x}{\omega}}
    \eqd \fscale \inprodinx{\ff(x)}{\fkernea{x}{\omega}}
    \qquad\scy\forall \ff\in\spLLRBu
  $
  \\This definition of the Fourier Transform is also called the \opd{unitary Fourier Transform}.
  }
\end{definition}

%--------------------------------------
\begin{remark}[\hib{Fourier transform scaling factor}]
\footnote{
  \citerpg{chorin2009}{13}{9781441910011},
  \citerppg{jeffrey2008}{xxxi}{xxxii}{0080556841},
  \citerppg{knappb2005}{374}{375}{0817632506}
  }
%--------------------------------------
  If the Fourier transform operator $\opFT$ and inverse Fourier transform operator $\opFTi$ are defined as
    \\\indentx
      $\ds\opFT \ff(x)\eqd\fF(\omega)\eqd A \int_\R \ff(x)      e^{-i\omega x} \dx$
      \qquad{and}\qquad
      $\ds\opFTi\Ff(\omega) \eqd B \int_\R \fF(\omega) e^{ i\omega x} \dw$
    \\
  then $A$ and $B$ can be any constants as long as $AB=\frac{1}{2\pi}$.
  The Fourier transform is often defined with the scaling factor $A$ set equal to 1
  such that $\left[\opFT \ff(x)\right](\omega) \eqd \inprodinx{\ff(x)}{\fkernea{x}{\omega}}$.
  In this case, the inverse Fourier transform operator $\opFTi$ is either defined as
  \\\indentx\begin{tabular}{cMl}
    \imark &\left[\opFTi \ff(x)\right](f) \eqd \inprodinx{\ff(x)}{\fkerne{x}{2\pi f}}
           &(using oscillatory frequency free variable $f$) or
    \\\imark & \left[\opFTi \ff(x)\right](\omega) \eqd \frac{1}{2\pi}\inprodinx{\ff(x)}{\fkerne{x}{\omega}}
             &(using angular frequency free variable $\omega$).
  \end{tabular}\\
  In short, the $2\pi$ has to show up somewhere, either in the argument of the exponential
  ($e^{-i2\pi f t}$) or in front of the integral ($\frac{1}{2\pi}\int\cdots$).
  One could argue that it is unnecessary to burden the exponential
  argument with the $2\pi$ factor ($e^{-i2\pi f t}$),
  and thus could further argue in favor of using the angular frequency variable $\omega$
  thus giving the inverse operator definition
  $\left[\opFTi \ff(x)\right](\omega) \eqd \frac{1}{2\pi}\inprodinx{\ff(x)}{\fkernea{x}{\omega}}$.
  But this causes a new problem. In this case, the Fourier operator $\opFT$ is not
  \hie{unitary} (see \prefp{thm:ft_unitary})---in particular,
  $\opFT\opFTa\ne \opI$, where $\opFTa$ is the \hie{adjoint} of $\opFT$;
  but rather,
  $\opFT\left(\frac{1}{2\pi}\opFTa\right) = \left(\frac{1}{2\pi}\opFTa\right)\opFT = \opI$.
  But if we define the operators $\opFT$ and $\opFTi$ to both have the scaling factor
  $\fscale$, then $\opFT$ and $\opFTi$ are inverses {\em and}
  $\opFT$ is \hie{unitary}---that is,
  $\opFT\opFTa = \opFTa\opFT = \opI$.
\end{remark}


%======================================
\section{Operator properties}
%======================================
%---------------------------------------
\begin{theorem}[\thmd{Inverse Fourier transform}]
\label{thm:opFTi}
\footnote{
  \citerpg{chorin2009}{13}{9781441910011}
  %\citerp{mathews}{102}
  }
\index{Inverse Fourier transform|textbf}
\index{Fourier transform!inverse|textbf}
\index{transform!inverse Fourier|textbf}
%---------------------------------------
Let $\opFT$ be the Fourier Transform operator \xref{def:opFT}.
%let $\fkern{x}{\omega}$ be the \hie{kernel} $\fkern{x}{\omega}\eqd\fkerne{x}{\omega}$.
The inverse $\opFTi$ of $\opFT$ is
\thmbox{
  \brs{\opFTi \Ff}(x)
    %\quad\eqd\quad \fscalei \inprod           {\Ff(\omega)}{\fkerna{x}{\omega}}
    = \fscalei \int_\R \Ff(\omega) \fkerne{x}{\omega} \dw
    \qquad\scy\forall \Ff\in\spLLRBu
  }
\end{theorem}
%\begin{proof}
%\begin{align*}
%  \opFTi\opF \ff
%    &= \fscalei
%       \inprod{\opF\ff}
%              {\fkernea{x}{\omega}}
%  \\&= \fscalei
%       \inprod{ \fscale \inprod{\ff}{\fkerne{x}{\omega}}}
%              {\fkernea{x}{\omega}}
%  \\&= \fscalei \fscale
%       \inprodinx[\omega]{\inprodinx{\ff(x)}{\fkernea{x}{\omega}}}
%              {\fkerne{x}{\omega}}
%  \\&= \frac{1}{2\pi}
%       \inprodinx[\omega]{\inprodinx{\ff(x)}{\fkernea{x}{\omega}}}
%              {\fkerne{x}{\omega}}
%  \\&= \frac{1}{2\pi} \int_\omega \int_{u\in\R} \ff(u) \fkernea{u}{\omega} \fkerne{x}{\omega} \du \dw
%  \\&= \frac{1}{2\pi} \int_{u\in\R} \int_\omega \ff(u) \fkernea{u}{\omega} \fkerne{x}{\omega} \du \dw
%  \\&= \frac{1}{2\pi} \int_{u\in\R} \ff(u) \int_\omega \fkerne{(x-u)}{\omega} \du \dw
%  \\&\eqq \frac{1}{2\pi} \int_{u\in\R} \ff(u) 2\pi \delta(x-u) \dw
%  \\&= \ff(x)
%  \\&= \opI \ff
%  \\
%  \\
%  \opF\opFTi \ff
%    &= \fscale \inprod{\opFTi\ff}{\fkerne{x}{\omega}}
%  \\&= \fscale \inprod{\fscalei  \inprod{\ff}{\fkernea{x}{\omega}}}
%              {\fkerne{x}{\omega}}
%  \\&= \fscale \fscalei
%       \inprodinx{\inprodinx[\omega]{\ff(\omega)}{\fkerne{x}{\omega}}}
%              {\fkernea{x}{\omega}}
%  \\&= \frac{1}{2\pi}
%       \inprodinx{\inprodinx[\omega]{\ff(\omega)}{\fkerne{x}{\omega}}}
%              {\fkernea{x}{\omega}}
%  \\&= \frac{1}{2\pi} \int_\R \int_{u\in\R} \ff(u) \fkerne{x}{u} \fkernea{x}{\omega} \du \dw
%  \\&= \frac{1}{2\pi} \int_{u\in\R} \int_\R \ff(u) \fkerne{x}{(u-\omega)} \fkerne{x}{\omega} \du \dw
%  \\&= \frac{1}{2\pi} \int_{u\in\R} \ff(u) \int_\R \fkerne{x}{(u-\omega)} \du \dw
%  \\&\eqq \frac{1}{2\pi} \int_{u\in\R} \ff(u) 2\pi \delta(u-\omega) \dw
%  \\&= \ff(\omega)
%  \\&= \opI \ff
%\end{align*}
%\end{proof}



%---------------------------------------
\begin{theorem}
\label{thm:opFTa}
\label{thm:ft_unitary}
\index{Fourier Transform!adjoint|textbf}
%---------------------------------------
Let $\opFT$ be the Fourier Transform operator with
inverse $\opFTi$ and adjoint $\opFTa$.
\thmbox{
  \opFTa = \opFTi
 }
\end{theorem}
\begin{proof}
  \begin{align*}
    \inprod{\opFT \ff}{\fg}
      &= \inprod{ \fscale \inprodinx{\ff(x)}{ \fkernea{x}{\omega} } }{\fg(\omega)}
      && \text{by definition of $\opFT$ \prefpo{def:ft}}
    \\&= \fscale \inprodinx{\ff(x)}{ \inprod{\fkernea{x}{\omega}}{\fg(\omega)}  }
      && \text{by \hie{additive property} of $\inprodn$ \ifdochas{vsinprod}{\prefpo{def:inprod}}}
    \\&= \inprodinx{\ff(x)}{\fscale  \inprod{\fg(\omega)}{\fkernea{x}{\omega}}^\ast  }
      && \text{by \hie{conjugate symmetric property} of $\inprodn$ \ifdochas{vsinprod}{\prefpo{def:inprod}}}
    \\&= \inprod{\ff(x)}{ \fscale \inprod{\fg(\omega)}{\fkernea{x}{\omega}}  }
      && \text{by definition of $\inprodn$}
    \\&= \inprod{\ff}{\mcom{\opFTi}{$\opFTa$} \fg}
      && \text{by \prefp{thm:opFTi}}
  \end{align*}
\end{proof}


The Fourier Transform operator has several nice properties:
\begin{liste}
  \item $\opFT$ is \hie{unitary}
    \ifdochas{operator}{\footnote{{\em \prop{unitary} operators}: \prefp{def:op_unitary} }}
    (\pref{cor:ft_unitary}---next corollary).
  \item Because $\opFT$ is unitary, it automatically has several other nice
        properties %such as being \hie{isometric}, and satisfying   \hie{Parseval's equation}, satisfying \hie{Plancheral's formula}, and more
        \xref{thm:ft_prop}.
\end{liste}

%---------------------------------------
\begin{corollary}
\label{cor:ft_unitary}
\index{operator!unitary}
%---------------------------------------
Let $\opI$ be the identity operator
and let $\opFT$ be the Fourier Transform operator with
adjoint $\opFTa$ and inverse $\opFTi$.
\corbox{
 \mcom{\opFT\opFTa = \opFTa\opFT = \opI}{$\opFTa=\opFTi$}
 \qquad \text{($\opFT$ is unitary)}
 }
\end{corollary}
\begin{proof}
  This follows directly from the fact that $\opFTa=\opFTi$ (\prefp{thm:opFTa}).
\end{proof}

%---------------------------------------
\begin{theorem}
\label{thm:ft_prop}
\label{thm:planform}
%---------------------------------------
Let $\opFT$ be the Fourier transform operator with adjoint $\opFTa$ and inverse $\opFT$.
Let $\normopn$ be the operator norm with respect to the vector norm $\normn$ with respect to the
Hilbert space $\opair{\clFrc}{\inprodn}$.
Let $\oppR(\opA)$ be the range of an operator $\opA$.
\thmbox{\begin{array}{rcl cl D}
  \oppR(\opF\tau)               &=& \oppR(\opFTi)                   &=& \spLLR            & \\
 %\oppN(\opF\tau)               &=& \oppN(\opFTi)                   &\eqq& \setn{0}       & \\
  \normop{\opFT   }             &=& \normop{\opFTi}                 &=& 1                 & (\prope{unitary})              \\
  \inprod{\opFT \ff}{\opFT \fg} &=& \inprod{\opFTi \ff}{\opFTi \fg} &=& \inprod{\ff}{\fg} & (\structe{Parseval's equation})  \\
  \norm{\opFT  \ff}             &=& \norm{\opFTi \ff}               &=& \norm{\ff}        & (\structe{Plancherel's formula}) \\
  \norm{\opFT \ff-\opFT \fg}    &=& \norm{\opFTi \ff-\opFTi \fg}    &=& \norm{\ff-\fg}    & (\prope{isometric})
\end{array}}
\end{theorem}
\begin{proof}
These results follow directly from the fact that $\opFT$
is \prop{unitary} (\prefp{cor:ft_unitary}) and from
the properties of \prop{unitary} operators\ifdochas{operator}{ (\prefp{thm:unitary_prop})}.
\end{proof}

%======================================
\section{Shift relations}
%======================================
%--------------------------------------
\begin{theorem}[Shift relations]
\label{thm:ft_shift}
%--------------------------------------
Let $\opFT$ be the Fourier Transform operator \xref{def:opFT}.
\thmbox{\begin{array}{>{\ds}rc>{\ds}l}
  \opFT \brs{\ff(x-y)}(\omega)            &=& e^{-i\omega y} \;\brs{\opFT \ff(x)}(\omega) \\
  \brs{\opFT\brp{e^{irx} \fg(x)}}(\omega) &=& \brs{\opFT\fg(x)}(\omega-r)
\end{array}}
\end{theorem}
\begin{proof}
Let $\opLT$ be the \ope{Laplace Transform} operator\ifsxref{laplace}{def:opLT}.
\begin{align*}
  \opFT \brs{\ff(x-y)}(\omega)
    &= \brlr{\opLT \brs{\ff(x-y)}(s)}_{s=i\omega}
    && \text{by definition of $\opLT$}         && \text{\ifxref{laplace}{def:opLT}}
  \\&= \brlr{e^{-sy} \;\brs{\opLT \ff(x)}(s)}_{s=i\omega}
    && \text{by \thme{Laplace shift relation}} && \text{\ifxref{laplace}{thm:opLT_shift}}
  \\&= e^{-i\omega y} \;\brs{\opFT \ff(x)}(\omega)
    && \text{by definition of $\opFT$}         && \text{\xref{def:opFT}}
  \\
  \brs{\opFT\brp{e^{irx} \fg(x)}}(\omega)
    &= \brlr{\brs{\opLT\brp{e^{irx} \fg(x)}}(s)}_{s=i\omega}
    && \text{by definition of $\opLT$}         && \text{\ifxref{laplace}{def:opLT}}
  \\&= \brlr{\brs{\brs{\opLT\fg(x)}(s-r)}}_{s=i\omega}
    && \text{by \thme{Laplace shift relation}} && \text{\ifxref{laplace}{thm:opLT_shift}}
  \\&= \brs{\opFT\fg(x)}(\omega-r)
    && \text{by definition of $\opFT$}         && \text{\xref{def:opFT}}
\end{align*}
\end{proof}

%--------------------------------------
\begin{theorem}[Complex conjugate]
\label{thm:ft_conj}
\label{thm:ft_neg}
%--------------------------------------
Let $\opFT$ be the Fourier Transform operator and $\ast$ represent the complex conjugate operation on the set of
complex numbers.
\thmbox{\begin{array}{rclCD}
  \opFT\ff^\ast(-x)                        &=& -\brs{\opFT\ff(x)}^\ast  & \forall\ff\in\spLLRBu\\
  \text{$\ff$ is real}\implies\Ff(-\omega) &=& \brs{\Ff(\omega)}^\ast  & \forall \omega\in\R   & \prope{reality condition}
\end{array}}
\end{theorem}
\begin{proof}
\begin{align*}
  \brs{\opFT\ff^\ast(-x)}(\omega)
     &\eqd \frac{1}{\sqrt{2\pi}}\:\int \ff^\ast(-x) \fkernea{x}{\omega}\dx
     &&    \text{by definition of $\opFT$}
     &&    \text{\xref{def:opFT}}
   \\&=    \frac{1}{\sqrt{2\pi}}\:\int \ff^\ast(u) e^{i\omega u}(-1)\du
     &&    \text{where $u\eqd-x\implies\dx=-\du$}
   \\&=    -\brs{\frac{1}{\sqrt{2\pi}}\:\int \ff(u) e^{-i\omega u}\du}^\ast
   \\&\eqd -\brs{\opFT\ff(x)}^\ast
     &&    \text{by definition of $\opFT$}
     &&    \text{\xref{def:opFT}}
   \\
   \Ff(-\omega)
     &\eqd \frac{1}{\sqrt{2\pi}}\:\int \ff(x) \fkernea{x}{(-\omega)}\dx
     &&    \text{by definition of $\opFT$}
     &&    \text{\xref{def:opFT}}
   \\&=    \brs{\frac{1}{\sqrt{2\pi}}\:\int \ff^\ast(x) \fkernea{x}{\omega}\dx}^\ast
   \\&=    \brs{\frac{1}{\sqrt{2\pi}}\:\int \ff     (x) \fkernea{x}{\omega}\dx}^\ast
     &&    \text{by $\ff$ is real hypothesis}
   \\&\eqd \Ff^\ast(\omega)
     &&    \text{by definition of $\opFT$}
     &&    \text{\xref{def:opFT}}
\end{align*}
\end{proof}

%======================================
%\subsection{Relationships between time and frequency}
%======================================
%======================================
\section{Convolution relations}
\label{sec:conv}
\index{convolution}
%======================================
\pref{thm:conv} (next) demonstrates that multiplication in the ``time domain"
is equivalent to convolution in the ``frequency domain" and
vice-versa.
%--------------------------------------
\begin{theorem}[\thmd{convolution theorem}]
\footnote{
  \citerppgc{bachman2002}{269}{270}{9780387988993}{5.2.3 Convolutions to Products},
  \citerpg{bachman1964}{8}{9781483267562},
  \citerpg{bracewell1978}{110}{007007013X}
  }
\label{thm:conv}
%--------------------------------------
Let $\opFT$ be the Fourier Transform operator \xref{def:opFT}
and $\conv$ the convolution operator \xref{def:conv}.
\thmbox{
\begin{array}{rcl@{\qquad}C}
  \mcom{\opFT\brs{\ff(x)\conv\fg(x)}(\omega)}{convolution    in ``time domain"} &=& \mcom{\sqrt{2\pi}\brs{\opFT\ff}(\omega)\, \brs{\opFT\fg}(\omega)}    {multiplication in ``frequency domain"} & \forall\ff,\fg\in\spLLRBu\\
  \mcom{\opFT\brs{\ff(x) \fg(x)}(\omega)}    {multiplication in ``time domain"} &=& \mcom{\frac{1}{\sqrt{2\pi}}\brs{\opFT\ff}(\omega) \conv \brs{\opFT\fg}(\omega)}{convolution    in ``frequency domain"} & \forall\ff,\fg\in\spLLRBu.
\end{array}
}
\end{theorem}
\begin{proof}
Let $\opLT$ be the \ope{Laplace Transform} operator\ifsxref{laplace}{def:opLT}.
\begin{align*}
   \opFT \brs{\ff(x)\conv\fg(x)}(\omega)
     &= \brlr{\opLT \brs{\ff(x)\conv\fg(x)}(s)}_{s=i\omega}
     && \text{by definition of $\opLT$} && \text{\ifxref{laplace}{def:opLT}}
   \\&= \brlr{\sqrt{2\pi}\brs{\opLT\ff}(s)\, \brs{\opLT\fg}(s)}_{s=i\omega}
     && \text{by \thme{Laplace convolution} result} && \text{\xref{thm:opLT_conv}}
   \\&= \sqrt{2\pi}\brs{\opFT\ff}(\omega)\, \brs{\opFT\fg}(\omega)
   \\
   \opFT[\ff(x)\fg(x)](\omega)
     &= \brlr{\opLT[\ff(x)\fg(x)](s)}_{s=i\omega}
   \\&= \brlr{\fscale\brs{\opLT\ff}(s) \conv \brs{\opLT\fg}(s)}_{s=i\omega}
   \\&= \fscale\brs{\opFT\ff}(\omega) \conv \brs{\opFT\fg}(\omega)
\end{align*}
\end{proof}

%======================================
\section{Calculus relations}
%======================================
%---------------------------------------
\begin{theorem}
%---------------------------------------
Let $\opFT$ be the \ope{Fourier Transform} operator \xref{def:opFT}.
\thmbox{
    \brb{\lim_{t\to-\infty}\fx(t)=0}
    \qquad\implies\qquad
    \brb{\opFT \brs{\ddt \fx(t)} = i\omega \brs{\opFT\fx}(\omega)}
  }
\end{theorem}
\begin{proof}
Let $\opLT$ be the \ope{Laplace Transform} operator\ifsxref{laplace}{def:opLT}.
\begin{align*}
  \opFT \brs{\ddt \fx(t)}
    &\eqd \brlr{\opLT\brs{\ddt \fx(t)}(s)}_{s=i\omega}
    && \text{by definitions of $\opLT$ and $\opFT$}    && \text{\ifxref{laplace}{def:opLT}}
  \\&= \brlr{s \brs{\opLT\fx(t)}(s)}_{s=i\omega}
    && \text{by \prefp{thm:opLT_diff}}
  \\&= i\omega \brs{\opFT\fx}(\omega)
\end{align*}
\end{proof}

%---------------------------------------
\begin{theorem}
%---------------------------------------
Let $\opFT$ be the \ope{Fourier Transform} operator \xref{def:opFT}.
\thmbox{
    \opFT \int_{u=-\infty}^{u=t} \fx(u) \du = \frac{1}{i\omega} \brs{\opFT\fx}(\omega)
  }
\end{theorem}
Let $\opLT$ be the \ope{Laplace Transform} operator\ifsxref{laplace}{def:opLT}.
\begin{proof}
\begin{align*}
  \opFT \int_{u=-\infty}^{u=t} \fx(u) \du
    &\eqd \brlr{\opLT \int_{u=-\infty}^{u=t} \fx(u) \du}_{s=i\omega}
  \\&=    \brlr{\frac{1}{s} \brs{\opLT\fx(t)}(s)}_{s=i\omega}
    && \text{by \prefp{thm:opLT_int}}
  \\&=    \frac{1}{i\omega} \brs{\opFT\fx(t)}(\omega)
\end{align*}
\end{proof}

%======================================
\section{Real valued functions}
%======================================
\begin{figure}[ht]
\begin{center}
\begin{fsL}
\setlength{\unitlength}{0.08mm}
\begin{tabular}{c@{\hspace{1cm}}c@{\hspace{1cm}}c@{\hspace{1cm}}c}
\begin{picture}(340,300)(-150,-150)
  %\graphpaper[10](0,0)(600,200)
  \thicklines
  \put(-150,   0){\line(1,0){300} }
  \put(   0,-150){\line(0,1){300} }
  \put( 160,   0){\makebox(0,0)[l]{$f$} }
  \put(-100,   0){\line( 1,1){100} }
  \put( 100,   0){\line(-1,1){100} }
  \put(  60,  60 ){\makebox(0,0)[bl]{$\Reb{\Fx(\omega)}$}}
\end{picture}
&
\begin{picture}(340,300)(-150,-150)
  %\graphpaper[10](0,0)(600,200)
  \thicklines
  \put(-150,   0){\line(1,0){300} }
  \put(   0,-150){\line(0,1){300} }
  \put( 160,   0){\makebox(0,0)[l]{$f$} }
  \qbezier(0,0)( 20, 80)( 100, 100)
  \qbezier(0,0)(-20,-80)(-100,-100)
  \put( 100,   0){\line(0, 1){100} }
  \put(-100,   0){\line(0,-1){100} }
  \put(- 10,  60 ){\makebox(0,0)[br]{$\Imb{\Fx(\omega)}$}}
\end{picture}
&
\begin{picture}(340,300)(-150,-150)
  %\graphpaper[10](0,0)(600,200)
  \thicklines
  \put(-150,   0){\line(1,0){300} }
  \put(   0,-150){\line(0,1){300} }
  \put( 160,   0){\makebox(0,0)[l]{$f$} }
  \qbezier(0,100)( 20,20)( 100, 0)
  \qbezier(0,100)(-20,20)(-100, 0)
  \put( 60,  60 ){\makebox(0,0)[bl]{$|\Fx(\omega)|$}}
\end{picture}
&
\begin{picture}(340,300)(-150,-150)
  %\graphpaper[10](0,0)(600,200)
  \thicklines
  \put(-150,   0){\line(1,0){300} }
  \put(   0,-150){\line(0,1){300} }
  \put( 160,   0){\makebox(0,0)[l]{$f$} }
  \put( 100,   0){\line(0, 1){100} }
  \put(-100,   0){\line(0,-1){100} }
  \put(-100,-100){\line(1, 1){200} }
  \put(- 10,  60 ){\makebox(0,0)[br]{$\angle\Fx(\omega)$}}
\end{picture}
\\
(symmetric) & (anti-symmetric) & (symmetric) & (anti-symmetric)
\end{tabular}
\end{fsL}
\end{center}
\caption{
   Fourier transform components of real-valued signal
   \label{fig:FTreal}
   }
\end{figure}

%---------------------------------------
\begin{theorem}
\label{thm:FTreal}
%---------------------------------------
Let $\ff(x)$ be a function in $\spLLR$ and $\Ff(\omega)$
the \ope{Fourier Transform} of $\ff(x)$.
%real-valued such that $\ff\in\clFrr$. Then
\thmbox{
  \brb{\begin{array}{M}
    $\ff(x)$ is \prope{real-valued}\\
    ($\ff\in\clFrr$)
  \end{array}}
  \implies
  \brb{\begin{array}{rclM}
    \Ff(\omega)       &=& \Ff^\ast(-\omega)    & (\prope{Hermitian symmetric}) \\
    \Reb{\Ff(\omega)} &=& \Reb{\Ff(-\omega)}   & (\prope{symmetric}) \\
    \Imb{\Ff(\omega)} &=& -\Imb{\Ff(-\omega)}  & (\prope{anti-symmetric}) \\
    |\Ff(\omega)|     &=& |\Ff(-\omega)|       & (\prope{symmetric}) \\
    \angle\Ff(\omega) &=& \angle\Ff(-\omega)   & (\prope{anti-symmetric}).
  \end{array}}
  }
\end{theorem}
\begin{proof}
\[\begin{array}{*{10}{l}}
   \Ff(\omega)
     &\eqd& [\opFT{\ff(x)}](\omega)
     &\eqd& \inprod{\ff(x)}{\fkerne{x}{\omega}}
     &=&     \inprod{\ff(x)}{e^{i(-\omega)x}}^\ast
     &\eqd& \Ff^\ast(-\omega)
\\
   \Reb{\Ff(\omega)}
     &=& \Reb{\Ff^\ast(-\omega)}
     &=& \Reb{\Ff(-\omega)}
\\
   \Imb{\Ff(\omega)}
     &=& \Imb{\Ff^\ast(-\omega)}
     &=& -\Imb{\Ff(-\omega)}
\\
   |\Ff(\omega)|
     &=& |\Ff^\ast(-\omega)|
     &=& |\Ff(-\omega)|
\\
   \angle\Ff(\omega)
     &=& \angle\Ff^\ast(-\omega)
     &=& -\angle\Ff(-\omega)
\end{array}\]
\end{proof}

%======================================
\section{Moment properties}
%======================================
%--------------------------------------
\begin{definition}
\footnote{
  \citePpp{jawerth}{16}{17},
  \citePp{sweldens93}{2},
  \citerpg{vidakovic}{83}{9780471293651}
  %\citerp{goswami}{102}    \\
  %\citerpp{mallat}{241}{243}
  }
\label{def:Mn_ft}
%--------------------------------------
\defboxt{
  The quantity $\fM_n$ is the \vald{$n$th moment} of a function $\ff(x)\in\spLLR$ if
  \\\indentx$\ds \fM_n \eqd \int_{\R} x^n \ff(x) \dx \qquad \text{for }n\in\Znn$.
  }
\end{definition}

%--------------------------------------
\begin{lemma}
\footnote{
  \citerpp{goswami}{38}{39}
  }
\label{lem:har_moment}
%--------------------------------------
Let $\fM_n$ be the \fncte{$n$th moment} \xref{def:Mn_ft} and
$\Ff(\omega)\eqd\brs{\opFT\ff}(\omega)$ the \fncte{Fourier transform} \xref{def:opFT} of a function $\ff(x)$ in $\spLLR$ \xref{def:spLLR}.
\lembox{\begin{array}{>{\ds}rc>{\ds}l@{\qquad}C}
  \fM_n  &=&  \brlr{\sqrt{2\pi}(i)^{n}\opddwn  \Ff(\omega)}_{\omega=0} & \forall n\in\Znn,\,\ff\in\spLLR\\
  \brlr{\opddwn  \Ff(\omega)}_{\omega=0}  &=& \cft(-i)^n \fM_n         & \forall n\in\Znn,\,\ff\in\spLLR
\end{array}}
\end{lemma}
\begin{proof}
  \begin{align*}
    \sqrt{2\pi}(i)^n\brs{\opddwn \Ff(\omega)}_{\omega=0}
      &= \sqrt{2\pi}(i)^n\brs{\opddwn  \fscale\int_\R \ff(x) \fkernea{x}{\omega}\dx}_{\omega=0}
      && \text{by definition of $\opFT$}
      && \text{\xref{def:opFT}}
    \\&= \left.(i)^n\int_\R \ff(x) \brs{\opddwn    \fkernea{x}{\omega}}\dx\right|_{\omega=0}
    \\&= \left.(i)^n\int_\R \ff(x) \brs{(-i)^n x^n \fkernea{x}{\omega}}\dx\right|_{\omega=0}
    \\&= (-i^2)^n \int_\R \ff(x) x^n \dx
    \\&= \int_\R \ff(x) x^n \dx
    \\&\eqd \fM_n
      && \text{by definition of $\fM_n$}
      && \text{\xref{def:Mn_ft}}
  \end{align*}
\end{proof}

%--------------------------------------
\begin{lemma}
\footnote{
  \citerppg{vidakovic}{82}{83}{9780471293651},
  \citerpp{mallat}{241}{242}
  }
\label{lem:vanish_deriv}
\index{vanishing moments}
%--------------------------------------
Let $\fM_n$ be the \fncte{$n$th moment} \xref{def:Mn_ft} and
$\Ff(\omega)\eqd\brs{\opFT\ff}(\omega)$ the \fncte{Fourier transform} \xref{def:opFT} of a function $\ff(x)$ in $\spLLR$ \xref{def:spLLR}.
\lembox{
  %\mcom{\inprod{\ff(x)}{x^n}=0}{$\ff(x)$ has a vanishing $n$th moment}
  \fM_n=0
  \qquad\iff\qquad
  \left.\opddwn \Ff(\omega)\right|_{\omega=0} =0
  \qquad\scy\forall n\in\Znn
  }
\end{lemma}
\begin{proof}
\begin{enumerate}
  \item Proof for $(\implies)$ case:
    \begin{align*}
      0 &= \inprod{\ff(x)}{x^n}
        && \text{by left hypothesis}
      \\&= \left.\sqrt{2\pi}(-i)^{-n}\opddwn  \Ff(\omega)\right|_{\omega=0}
        && \text{by \prefp{lem:har_moment}}
      \\&\implies \left.\opddwn \Ff(\omega)\right|_{\omega=0} =0
    \end{align*}

  \item Proof for $(\impliedby)$ case:
    \begin{align*}
      0
        &= \left. \opddwn  \Ff(\omega)\right|_{\omega=0}
        && \text{by right hypothesis}
      \\&= \left.\opddwn  \fscale\int_\R \ff(x) \fkernea{x}{\omega}\dx \right|_{\omega=0}
        && \text{by definition of $\Ff(\omega)$}
      \\&= \left.\fscale\int_\R \ff(x) \opddwn  \fkernea{x}{\omega}\dx \right|_{\omega=0}
      \\&= \left.\fscale\int_\R \ff(x) \left[(-i)^n x^n \fkernea{x}{\omega}\right]\dx \right|_{\omega=0}
      \\&= (-i)^n \fscale\int_\R \ff(x) x^n \dx
      \\&= (-i)^n \fscale\inprod{\ff(x)}{x^n}
        && \text{by definition of $\inprod{\cdot}{\cdot}$ in $\spLLR$ \xref{def:spLLR}}
    \end{align*}
\end{enumerate}
\end{proof}

%--------------------------------------
\begin{lemma}[\lemd{Strang-Fix condition}]
\label{lem:sfixceq}
\footnote{
  \citePpp{jawerth}{16}{17},
  \citePp{sweldens93}{2},
  \citerpg{vidakovic}{83}{9780471293651},
  %\citerp{goswami}{102},
  \citerpp{mallat}{241}{243},
  \citor{fix1969}
  }
%--------------------------------------
Let $\ff(x)$ be a function in $\spLLR$ and $\fM_n$ the \hie{$n$th moment} \xref{def:Mn_ft} of $\ff(x)$.
Let $\opTrn$ be the \fncte{translation operator} \xref{def:opT}.
\lembox{
  \mcom{{\sum_{k\in\Z} \opTrn^k x^n \ff(x) = \fM_n}}{\prope{Strang-Fix condition} in ``time"}
  %\mcom{{\sum_{k\in\Z} (x-k)^n \ff(x-k) = \fM_n}}{\prope{Strang-Fix condition} in ``time"}
  \qquad\iff\qquad
  \mcom{{\left.\opddwn  \Ff(\omega) \right|_{\omega=2\pi k} = \cft (-i)^n \kdelta_k \fM_n}}{\prope{Strang-Fix condition} in ``frequency"}
  %\qquad\scy\text{for $n=1,2,\ldots,p-1$}
  }
\end{lemma}
\begin{proof}
\begin{enumerate}
  \item Proof for ($\implies$) case:
    \begin{align*}
      \brs{\opddwn  \Ff(\omega)}_{\omega=2\pi k}
        &= \sum_{k\in\Z} \brs{\opddwn  \Ff(\omega)}_{\omega=2\pi k}  e^{i2\pi kx} \kdelta_k
      \\&= \cft \sum_{k\in\Z} \brs{\opddwn \int_\R  \ff(x) e^{-i\omega x} \dx}_{\omega=2\pi k} e^{i2\pi kx}  \kdelta_k
        && \text{by definition of $\Ff(\omega)$}
        && \text{\xref{def:opFT}}
      \\&= \cft \sum_{k\in\Z} \brs{\int_\R  \ff(x)(-ix)^n e^{-i\omega x} \dx}_{\omega=2\pi k} e^{i2\pi kx}  \kdelta_k
      \\&= (-i)^n \cft \sum_{k\in\Z} \brs{\int_\R  x^n \ff(x)  e^{-i\omega x} \dx}_{\omega=2\pi k} e^{i2\pi kx}  \kdelta_k
      \\&= (-i)^n \cft \sum_{k\in\Z} (x-k)^n \ff(x-k)  \kdelta_k
        && \text{by \thme{PSF}}
        && \text{\xref{thm:psf}}
      \\&= \cft (-i)^n \kdelta_k \fM_n
        && \text{by left hypothesis}
    \end{align*}

  %The quantity $(-i)^n \fM_n$ is not a function of $x$ which implies
  %that
  %$\sum_{k\in\Z} \left.\opddwn  \Ff(\omega)\right|_{\omega=2\pi k}=0$
  %for $k\ne 0$.
  %This implies
  %\[ \left.\opddwn  \Ff(\omega)\right|_{\omega=2\pi k}
  %   = (-i)^n \kdelta_k \fM_n.
  %\]

  \item Proof for ($\impliedby$) case:
    \begin{align*}
      \cft (-i)^n \fM_n
        &= \cft \sum_{k\in\Z} \left[(-i)^n \kdelta_k \fM_n \right] e^{-i2\pi kx}
        && \text{by definition of $\kdelta$}
        && \text{\ifxref{frames}{def:kdelta}}
      \\&= \sum_{k\in\Z} \left.
           \left[\opddwn  \Ff(\omega)\right]\right|_{\omega=2\pi k}
           e^{-i2\pi kx}
        && \text{by right hypothesis}
      \\&= \sum_{k\in\Z} \left.
           \left[
           \opddwn
           \int_\R  \ff(x) e^{-i\omega x} \dx
           \right]
           \right|_{\omega=2\pi k} e^{-i2\pi kx}
      \\&= \sum_{k\in\Z} \left.
           \left[
           \int_\R  \ff(x)(-ix)^n e^{-i\omega x} \dx
           \right]
           \right|_{\omega=2\pi k} e^{-i2\pi kx}
      \\&= (-i)^n \sum_{k\in\Z} \left.
           \left[
           \int_\R  x^n \ff(x)  e^{-i\omega x} \dx
           \right]
           \right|_{\omega=2\pi k} e^{-i2\pi kx}
      \\&= (-i)^n \sum_{k\in\Z} (x-k)^n \ff(x-k)
        && \text{by \thme{PSF}}
        && \text{\xref{thm:psf}}
    \end{align*}

\end{enumerate}
\end{proof}

%======================================
\section{Examples}
%======================================
%\begin{figure}[ht]
%\color{figcolor}
%\index{Nyquist sampling rate}
%\setlength{\unitlength}{0.1mm}
%\begin{center}
%\begin{fsL}
%\begin{tabular}{ccc}
%\begin{picture}(350,300)(-150,-150)
%  %\graphpaper[10](0,0)(300,300)
%  \thicklines
%  \put        (-150,   0){\line( 1, 0){ 300}}
%  \put        (   0, -50){\line( 0, 1){ 180}}
%  {\color{red}
%    \put        (-100,   0){\line( 1, 1){ 100}}
%    \put        ( 100,   0){\line(-1, 1){ 100}}
%  }
%  \put        ( -20, 100 ){\makebox (  0, 0)[r]{$a$}     }
%  \put        (-100, -10 ){\makebox (0, 0)[t]{$-b$}     }
%  \put        ( 100, -10 ){\makebox (0, 0)[t]{$-b$}     }
%  \put        ( 160,  0 ){\makebox (  0, 0)[l]{$\tau$}     }
%\end{picture}
%&
%$\iff$
%&
%\begin{picture}(750,300)(-350,-150)
%  %\graphpaper[10](0,0)(300,300)
%  \thicklines
%  \put        (-350,   0){\line( 1, 0){ 700}}
%  \put        (   0, -50){\line( 0, 1){ 180}}
%  {\color{red}
%    \qbezier    (-100,   0)(   0, 200)( 100,   0)
%    \qbezier    ( 100,   0)( 150,-100)( 200,   0)
%    \qbezier    ( 200,   0)( 250,  70)( 300,   0)
%    \qbezier    (-100,   0)(-150,-100)(-200,   0)
%    \qbezier    (-200,   0)(-250,  70)(-300,   0)
%    }
%  \put        ( -20, 100 ){\makebox (  0, 0)[r]{$ab$}     }
%  \multiput(-300,0)(100,  0){7}{\qbezier[20](  0,-50)(  0,-25)(  0, 0)}
%  \put        (-300, -60 ){\makebox (0, 0)[t]{$-\frac{3}{b}$}     }
%  \put        (-200, -60 ){\makebox (0, 0)[t]{$-\frac{2}{b}$}     }
%  \put        (-100, -60 ){\makebox (0, 0)[t]{$-\frac{1}{b}$}     }
%  \put        (   0, -60 ){\makebox (0, 0)[t]{$0$}     }
%  \put        ( 100, -60 ){\makebox (0, 0)[t]{$ \frac{1}{b}$}     }
%  \put        ( 200, -60 ){\makebox (0, 0)[t]{$ \frac{2}{b}$}     }
%  \put        ( 300, -60 ){\makebox (0, 0)[t]{$ \frac{3}{b}$}     }
%  \put        ( 360,  0 ){\makebox (  0, 0)[l]{$f$}     }
%\end{picture}
%\\
%\begin{picture}(350,300)(-150,-150)
%  %\graphpaper[10](0,0)(300,300)
%  \thicklines
%  \put        (-150,   0){\line( 1, 0){ 300}}
%  \put        (   0, -50){\line( 0, 1){ 180}}
%  {\color{red}
%    \put        (-100,   0){\line( 0, 1){ 100}}
%    \put        (-100, 100){\line( 1, 0){ 200}}
%    \put        ( 100,   0){\line( 0, 1){ 100}}
%    }
%  \put        ( -20, 110 ){\makebox (  0, 0)[br]{$a$}     }
%  \put        (-100, -10 ){\makebox (0, 0)[t]{$-b$}     }
%  \put        ( 100, -10 ){\makebox (0, 0)[t]{$-b$}     }
%  \put        ( 160,  0 ){\makebox (  0, 0)[l]{$\tau$}     }
%\end{picture}
%&
%%$\iff$
%&
%\begin{picture}(750,300)(-350,-150)
%  %\graphpaper[10](0,0)(300,300)
%  \thicklines
%  \put        (-350,   0){\line( 1, 0){ 700}}
%  \put        (   0, -50){\line( 0, 1){ 180}}
%  {\color{red}
%    \qbezier    (-100,   0)(   0, 200)( 100,   0)
%    \qbezier    ( 100,   0)( 150,-100)( 200,   0)
%    \qbezier    ( 200,   0)( 250,  70)( 300,   0)
%    \qbezier    (-100,   0)(-150,-100)(-200,   0)
%    \qbezier    (-200,   0)(-250,  70)(-300,   0)
%    }
%  \put        ( -20, 100 ){\makebox (  0, 0)[r]{$2ab$}     }
%  \multiput(-300,0)(100,  0){7}{\qbezier[20](  0,-50)(  0,-25)(  0, 0)}
%  \put        (-300, -60 ){\makebox (0, 0)[t]{$-\frac{3}{2b}$}     }
%  \put        (-200, -60 ){\makebox (0, 0)[t]{$-\frac{2}{2b}$}     }
%  \put        (-100, -60 ){\makebox (0, 0)[t]{$-\frac{1}{2b}$}     }
%  \put        (   0, -60 ){\makebox (0, 0)[t]{$0$}     }
%  \put        ( 100, -60 ){\makebox (0, 0)[t]{$ \frac{1}{2b}$}     }
%  \put        ( 200, -60 ){\makebox (0, 0)[t]{$ \frac{2}{2b}$}     }
%  \put        ( 300, -60 ){\makebox (0, 0)[t]{$ \frac{3}{2b}$}     }
%  \put        ( 360,  0 ){\makebox (  0, 0)[l]{$f$}     }
%\end{picture}
%\end{tabular}
%\end{fsL}
%\caption{
%  Fourier transform pairs
%  \label{fig:ft_pairs}
%  }
%\end{center}
%\end{figure}


   % \begin{figure}[ht]
   % \begin{center}
   % \input{"sinc_ab.lpl"}
   % \end{center}
   % \caption{
   %   Fourier transform of rectangular function:
   %   $2ab \frac{\sin(\omega b)}{\omega b}$
   %   \label{fig:sinc_ab}
   % }
   % \end{figure}
   %
   % \begin{figure}[ht]
   % \begin{center}
   % \input{"sinc_sq.lpl"}
   % \end{center}
   % \caption{
   %   Fourier transform of triangular function:
   %   $ab \frac{\sin(\pi fb)}{\pi fb}$
   %   \label{fig:sinc_sq}
   % }
   % \end{figure}

%---------------------------------------
\begin{example}[\exmd{rectangular pulse}]
\label{ex:ft_pulse}
\index{sinc}
%---------------------------------------
Let $\Ff(\omega)$ be the \fncte{Fourier transform} of a function $\ff(x)\in\spLLR$.
\exbox{\begin{array}{*{2}{>{\ds}c}}
  \ff(x)  =  \brbl{\begin{array}{lM}
                     c & for $x\in\intco{a}{b}$ \\
                     0 & otherwise
                   \end{array}}
  &
  \Ff(\omega)
      = \frac{c(b-a)}{\sqrt{2\pi}}
        e^{-i\brp{\frac{a+b}{2}\omega}}
        \brs{\frac{\sin\brp{\frac{b-a}{2}\omega}}{\brp{\frac{b-a}{2}\omega}}}
  \\
  \includegraphics{../common/math/graphics/pdfs/pulseabc.pdf}
  &
  \includegraphics{../common/math/graphics/pdfs/pulseabcft.pdf}
\end{array}}
\end{example}
\begin{proof}
{\begin{align*}
   \Ff(\omega)
     &= \opFT\brs{\ff(x)}(\omega)
     && \text{by definition of $\Ff(\omega)$}
   \\&= e^{-i\brp{\frac{a+b}{2}}\omega}\opFT\brs{\ff\brp{x-\frac{a+b}{2}}}(\omega)
     && \text{by \thme{shift relation}}&&\text{\xref{thm:ft_shift}}
   \\&= e^{-i\brp{\frac{a+b}{2}}\omega}\opFT\brs{c\setind_\intco{a}{b}\brp{x-\frac{a+b}{2}}}(\omega)
     && \text{by definition of $\ff(x)$}
   \\&= e^{-i\brp{\frac{a+b}{2}}\omega}\opFT\brs{c\setind_\intco{-\frac{b-a}{2}}{\frac{b-a}{2}}(x)}(\omega)
     && \text{by definition of $\setind$}&&\text{\xref{def:setind}}
    %&& \text{by definition of $\intcon$ \xref{def:intervals}}
   \\&= \cft e^{-i\brp{\frac{a+b}{2}}\omega}\int_\R c \setind_\intco{-\frac{b-a}{2}}{\frac{b-a}{2}}(x) e^{-i\omega x}\dx
     && \text{by definition of $\opFT$}&&\text{ \xref{def:opFT}}
   \\&= \cft e^{-i\brp{\frac{a+b}{2}}\omega}\int_{-\frac{b-a}{2}}^{\frac{b-a}{2}}c e^{-i\omega x}\dx
     && \text{by definition of $\setind$}&&\text{\xref{def:setind}}
   \\&= \left.\frac{c}{\sqrt{2\pi}}  e^{-i\brp{\frac{a+b}{2}}\omega}\frac{1}{-i\omega } e^{-i\omega x}  \right|_{-\frac{b-a}{2}}^\frac{b-a}{2}
   \\&= \frac{2c}{\sqrt{2\pi}\omega}
        e^{-i\brp{\frac{a+b}{2}}\omega}
        \brs{\frac{\ds e^{i\brp{\frac{b-a}{2}\omega}} - e^{-i\brp{\frac{b-a}{2}\omega}}}{2i}}
   \\&= \frac{c(b-a)}{\sqrt{2\pi}}
        e^{-i\brp{\frac{a+b}{2}\omega}}
        \brs{\frac{\sin\brp{\frac{b-a}{2}\omega}}{\brp{\frac{b-a}{2}\omega}}}
     && \text{by \thme{Euler formulas}}&&\text{\xref{cor:trig_ceesee}}
\end{align*}}
\end{proof}

%---------------------------------------
\begin{example}[\exmd{triangle}]
\label{ex:ft_triangle}
\index{sinc}
%---------------------------------------
Let $\Ff(\omega)$ be the \fncte{Fourier transform} of a function $\ff(x)\in\spLLR$.
\exbox{\begin{array}{*{2}{>{\ds}c}}
  \ff(x) = \brb{\begin{array}{lM} c\brs{1-\frac{\abs{2x-b-a}}{b-a}} & for $x\in\intco{a}{b}$ \\
                                  0 & otherwise
                  \end{array}}
  &
  \Ff(\omega)
      = \frac{c(b-a)^2}{4\sqrt{2\pi}}
        e^{-i\brp{\frac{a+b}{2}\omega}}
        \brs{\frac{\sin\brp{\frac{b-a}{4}\omega}}{\brp{\frac{b-a}{4}\omega}}}^2
  \\
  \mc{2}{c}{
    \includegraphics{../common/math/graphics/pdfs/triangleabc.pdf} \hspace{10mm}
    \includegraphics{../common/math/graphics/pdfs/triangleabcft.pdf}
    }
\end{array}}
\end{example}
\begin{proof}
\begin{align*}
   \Ff(\omega)
     &= \opFT\brs{\ff(x)}(\omega)
     && \text{by definition of $\Ff(\omega)$}
   \\&= e^{-i\brp{\frac{a+b}{2}}\omega}\opFT\brs{\ff\brp{x-\frac{a+b}{2}}}(\omega)
     && \text{by \thme{shift relation}}&&\text{\xref{thm:ft_shift}}
   \\&= \opFT\brs{c\brp{1-\frac{\abs{2x-b-a}}{b-a}}\setind_\intco{a}{b}(x)}(\omega)
     && \text{by definition of $\ff(x)$}
   \\&= c\opFT\brs{\setind_\intco{\frac{a}{2}}{\frac{b}{2}}(x)\conv\setind_\intco{\frac{a}{2}}{\frac{b}{2}}(x)}(\omega)
   \\&= c\sqrt{2\pi}\opFT\brs{\setind_\intco{\frac{a}{2}}{\frac{b}{2}}}\,\opFT\brs{\setind_\intco{\frac{a}{2}}{\frac{b}{2}}}
     && \text{by \thme{convolution theorem}}&&\text{\xref{thm:conv}}
   \\&= c\sqrt{2\pi}\brp{\opFT\brs{\setind_\intco{\frac{a}{2}}{\frac{b}{2}}}}^2
   \\&= c\sqrt{2\pi}\brp{
        \frac{\brp{\frac{b}{2}-\frac{a}{2}}}{\sqrt{2\pi}}
        e^{-i\brp{\frac{a+b}{4}\omega}}
        \brs{\frac{\sin\brp{\frac{b-a}{4}\omega}}{\brp{\frac{b-a}{4}\omega}}}
        }^2
     && \text{by \exme{Rectangular pulse} ex.}&&\text{\prefp{ex:ft_pulse}}
   \\&= \frac{c(b-a)^2}{4\sqrt{2\pi}}
        e^{-i\brp{\frac{a+b}{2}\omega}}
        \brs{\frac{\sin\brp{\frac{b-a}{4}\omega}}{\brp{\frac{b-a}{4}\omega}}}^2
\end{align*}
\end{proof}

%Fourier transforms of the sine and cosine functions result in a delta function.
%Delta functions use the extended real number system and therefore extreme caution should be used
%when using them in further calculations.
%However, delta functions are \structe{distributions}, and therefore  it is fairly ``safe" to use
%the Fourier transforms of sine and cosine within an integral.
%%---------------------------------------
%\begin{example}[Cosine]
%\label{prop:ft_cos}
%\index{cosine}
%%---------------------------------------
%Let $\ff\in\spLLR$ with Fourier Transform $\Fx\in\spLLR$.
%Then
%\exbox{
%  \ff(x) = \cos(\omega_cx)
%  \qquad\iff\qquad
%  \Fx(\omega) = \frac{1}{2} \delta( f+f_c ) + \frac{1}{2} \delta( f-f_c ).
%  }
%\end{example}
%\begin{proof}
%\begin{align*}
%   \Ff(\omega)
%      &= \cft\int_\R \ff(x)e^{-i\omega x} \dx
%      && \text{by definition of \fncte{Fourier transform} \xref{def:opFT}}
%    \\&= \cft\int_\R \cos\brp{\omega_cx}e^{-i\omega x} \dx
%      && \text{by definition of $\ff(x)$}
%    \\&= \cft\int_\R \brp{\frac{e^{i\omega_cx}+e^{-i\omega_cx}}{2}}e^{-i\omega x} \dx
%      && \text{by \prefp{cor:trig_ceesee}}
%    \\&= \frac{1}{2\sqrt{2\pi}}\int_\R e^{-i(\omega-\omega_c)x}+e^{-i(\omega+\omega_c)x} \dx
%    \\
%    \\&= \inprod{\frac{1}{2}e^{i\omega _ct} + \frac{1}{2}e^{-i\omega _ct}}{\fkerne{x}{\omega}}
%    \\&= \frac{1}{2}\inprod{e^{ i\omega _ct}}{\fkerne{x}{\omega}}+
%         \frac{1}{2}\inprod{e^{-i\omega _ct}}{\fkerne{x}{\omega}}
%    \\&= \frac{1}{2}\int_\R e^{ i\omega _ct} e^{-i\omega x} dt +
%         \frac{1}{2}\int_\R e^{-i\omega _ct} e^{-i\omega x} dt
%    \\&= \frac{1}{2}\int_\R e^{-i2\pi (f-f_c)t} dt +
%         \frac{1}{2}\int_\R e^{-i2\pi (f+f_c)t} dt
%    \\&= \frac{1}{2}\lim_{h\to0}\sum_{n\in\Z} e^{-i2\pi (f-f_c)nh} h +
%         \frac{1}{2}\lim_{h\to0}\sum_{n\in\Z} e^{-i2\pi (f+f_c)nh} h
%    \\&= \frac{1}{2}\lim_{h\to0}h \sum_{n\in\Z} e^{i2\pi (f-f_c)nh}  +
%         \frac{1}{2}\lim_{h\to0}h \sum_{n\in\Z} e^{i2\pi (f+f_c)nh}
%    \\&= \frac{1}{2}\lim_{h\to0} \sum_{n\in\Z} \delta\left( \frac{2\pi (f-f_c)}{2\pi} - n\frac{1}{h}\right)  +
%         \frac{1}{2}\lim_{h\to0} \sum_{n\in\Z} \delta\left( \frac{2\pi (f+f_c)}{2\pi} - n\frac{1}{h}\right)
%    \\&= \frac{1}{2} \delta( f-f_c )  +  \frac{1}{2} \delta( f+f_c )
%\end{align*}
%\end{proof}
%
%
%%---------------------------------------
%\begin{proposition}[Sine]
%\label{prop:ft_sin}
%\index{sine}
%%---------------------------------------
%Let $\ff\in\clFrr$ with Fourier Transform $\Fx\in\spLLR$.
%Then
%\propbox{
%  \ff(x) = \sin(\omega _ct)
%  \qquad\iff\qquad
%  \Fx(\omega) = \frac{i}{2} \delta( f+f_c )   -\frac{i}{2} \delta( f-f_c ).
%  }
%\end{proposition}
%\begin{proof}
%\begin{eqnarray*}
%   \Fx(\omega)
%      &=& \inprod{\sin(\omega _ct)}{\fkerne{x}{\omega}}
%    \\&=& \inprod{\frac{1}{2i}e^{i\omega _ct} i \frac{1}{2i}e^{-i\omega _ct}}{\fkerne{x}{\omega}}
%    \\&=& \frac{1}{2i}\inprod{e^{ i\omega _ct}}{\fkerne{x}{\omega}}-
%          \frac{1}{2i}\inprod{e^{-i\omega _ct}}{\fkerne{x}{\omega}}
%    \\&=& \frac{1}{2i}\int_\R e^{ i\omega _ct} e^{-i\omega x} dt -
%          \frac{1}{2i}\int_\R e^{-i\omega _ct} e^{-i\omega x} dt
%    \\&=& \frac{1}{2i}\int_\R e^{-i2\pi (f-f_c)t} dt -
%          \frac{1}{2i}\int_\R e^{-i2\pi (f+f_c)t} dt
%    \\&=& \frac{1}{2i}\lim_{h\to0}\sum_{n\in\Z} e^{-i2\pi (f-f_c)nh} h -
%          \frac{1}{2i}\lim_{h\to0}\sum_{n\in\Z} e^{-i2\pi (f+f_c)nh} h
%    \\&=& \frac{1}{2i}\lim_{h\to0}h \sum_{n\in\Z} e^{i2\pi (f-f_c)nh}  -
%          \frac{1}{2i}\lim_{h\to0}h \sum_{n\in\Z} e^{i2\pi (f+f_c)nh}
%    \\&=& \frac{1}{2i}\lim_{h\to0} \sum_{n\in\Z} \delta\left( \frac{2\pi (f-f_c)}{2\pi} - n\frac{1}{h}\right)  -
%          \frac{1}{2i}\lim_{h\to0} \sum_{n\in\Z} \delta\left( \frac{2\pi (f+f_c)}{2\pi} - n\frac{1}{h}\right)
%    \\&=& \frac{1}{2i} \delta( f-f_c )  -  \frac{1}{2i} \delta( f+f_c )
%    \\&=& \frac{i}{2} \delta( f+f_c )   -\frac{i}{2} \delta( f-f_c )
%\end{eqnarray*}
%\end{proof}

%-------------------------------------
\begin{example}
\label{ex:ft_cos2}
%-------------------------------------
Let a function $\ff$ be defined in terms of the cosine function \xref{def:cos} as follows:

%\exbox{\begin{array}{m{58mm}m{\tw-76mm}}
\exbox{\begin{array}{*{2}{>{\ds}c}}
  \ff(x) \eqd \brbl{%
  \begin{array}{>{\ds}lM}
     \cos^2\brp{\frac{\pi}{2}x}  & for $\abs{x}\le 1$  \\
     0    & otherwise
  \end{array}}
  &
  \Ff(\omega) = \frac{1}{2\sqrt{2\pi}}\Big[{\mcom{\frac{2\sin\omega}{\omega}}{$2\sinc(\omega)$}+\mcom{\frac{\sin(\omega+\pi)}{(\omega+\pi)}}{$\sinc(\omega+\pi)$} + \mcom{\frac{\sin(\omega-\pi)}{(\omega-\pi)}}{$\sinc(\omega-\pi)$}}\Big]
  \\
  \includegraphics{../common/math/graphics/pdfs/coscos11.pdf}&\includegraphics{../common/math/graphics/pdfs/coscos11ft.pdf}
\end{array}}
\end{example}
\begin{proof}
Let $\setind_\setA(x)$ be the \fncte{set indicator function} \xref{def:setind} on a set $\setA$.
\begin{align*}
  \Ff(\omega)
    &\eqd \cft\int_\R \ff(x) e^{-i\omega x} \dx
    && \text{by definition of $\Ff(\omega)$ \xrefn{def:opFT}}
  \\&= \cft\int_\R \cos^2\brp{\frac{\pi}{2}x}\setind_\intcc{-1}{1}(x) e^{-i\omega x} \dx
    && \text{by definition of $\ff(x)$}
  \\&= \cft\int_{-1}^{1} \cos^2\brp{\frac{\pi}{2}x} e^{-i\omega x} \dx
    && \text{by definition of $\setind$ \xrefn{def:setind}}
  \\&= \cft\int_{-1}^{1} \brs{\frac{e^{i{\frac{\pi}{2}x}} + e^{-i{\frac{\pi}{2}x}}}{2}}^2 e^{-i\omega x} \dx
    && \text{by \prefp{cor:trig_ceesee}}
  \\&= \frac{1}{4\sqrt{2\pi}} \int_{-1}^{1} \brs{2 + e^{i\pi x} + e^{-i\pi x}} e^{-i\omega x} \dx
  \\&= \frac{1}{4\sqrt{2\pi}} \int_{-1}^{1} 2e^{-i\omega x} + e^{-i(\omega+\pi)x} + e^{-i(\omega-\pi)x}  \dx
  \\&= \frac{1}{4\sqrt{2\pi}}\brs{
           2\frac{e^{-i\omega x}}{-i\omega}
         +  \frac{e^{-i(\omega+\pi)x}}{-i(\omega+\pi)}
         +  \frac{e^{-i(\omega-\pi)x}}{-i(\omega-\pi)}
       }_{-1}^{1}
  \\&= \frac{1}{2\sqrt{2\pi}}\brs{
           2\frac{e^{-i\omega}-e^{+i\omega}}{-2i\omega}
         +  \frac{e^{-i(\omega+\pi)}-e^{+i(\omega+\pi)}}{-2i(\omega+\pi)}
         +  \frac{e^{-i(\omega-\pi)}-e^{+i(\omega-\pi)}}{-2i(\omega-\pi)}
       }_{-1}^{1}
  \\&= \frac{1}{2\sqrt{2\pi}}\Big[{
            \mcom{\frac{2\sin\omega}{\omega}}{$2\sinc(\omega)$}
         +  \mcom{\frac{\sin(\omega+\pi)}{(\omega+\pi)}}{$\sinc(\omega+\pi)$}
         +  \mcom{\frac{\sin(\omega-\pi)}{(\omega-\pi)}}{$\sinc(\omega-\pi)$}
       }\Big]
\end{align*}

\end{proof}

%-------------------------------------
\begin{example}
\label{ex:eatau}
\footnote{
  \url{https://math.stackexchange.com/questions/4015842/}
  }
%-------------------------------------
\exbox{\begin{array}{>{\ds}rc>{\ds}l}
  \opFT\brs{e^{-\alpha\abs{x}}}
    &=& \frac{1}{\sqrt{2\pi}}\brs{\frac{2\alpha}{\alpha^2+\omega^2}}
\end{array}}
\end{example}
\begin{proof}
\begin{align*}
  \sqrt{2\pi}\opFT\brs{e^{-\alpha\abs{x}}}
    &\eqd \brs{\sqrt{2\pi}}\frac{1}{\sqrt{2\pi}}\int_{-\infty}^{\infty} e^{-\alpha\abs{\tau}} e^{-j\omega\tau}\dtau
    && \text{by definition Fourier Transform}
  \\&= \int_{-\infty}^{0} e^{-\alpha(-\tau)} e^{-j\omega\tau}\dtau
     + \int_{0}^{\infty} e^{-\alpha(\tau)} e^{-j\omega\tau}\dtau
  \\&= \int_{-\infty}^{0} e^{\tau(\alpha-j\omega)}\dtau
     + \int_{0}^{\infty} e^{\tau(-\alpha-j\omega)}\dtau
  \\&= \brlr{\frac{e^{\tau(\alpha-j\omega)}}{\alpha-j\omega}}_{-\infty}^{0}
     + \brlr{\frac{e^{\tau(-\alpha-j\omega)}}{-\alpha-j\omega}}_{0}^{\infty}
    && \text{by Fundamental Theorem of Calculus}
  \\&= \brs{\frac{1}{\alpha-j\omega} - 0}
     + \brs{0- \frac{1}{-\alpha-j\omega}}
  \\&= \brs{\frac{1}{\alpha-j\omega}}\brs{\frac{\alpha-j\omega}{\alpha-j\omega}}
     + \brs{\frac{1}{\alpha+j\omega}}\brs{\frac{\alpha+j\omega}{\alpha+j\omega}}
  \\&= \frac{\alpha-j\omega}{\alpha^2+\omega^2}
     + \frac{\alpha+j\omega}{\alpha^2+\omega^2}
  \\&= \brs{\frac{2\alpha}{\alpha^2+\omega^2}}
\end{align*}
\end{proof}

%-------------------------------------
\begin{example}
\label{ex:coseatau}
%-------------------------------------
\exbox{
  \opFT\brs{\cos(\beta x)e^{-\alpha\abs{x}}}
    = \frac{1}{\sqrt{2\pi}}\brs{\frac{2\alpha}{(\omega-\beta)^2 + \alpha^2}}
    = \frac{1}{\sqrt{2\pi}}
       \brs{\frac{2\alpha}
           {\omega^2 -2\beta\omega + (\alpha^2 + \beta^2)}}
 }
\end{example}
\begin{proof}
\begin{align*}
  \opFT\brs{\cos(\beta x)e^{-\alpha\abs{x}}}
    &\eqd \frac{1}{\sqrt{2\pi}}\brlr{\opLT\brs{\cos(\beta x)e^{-\alpha\abs{x}}}}_{s=i\omega}
    && \text{by definition Fourier Transform}
    && \text{\xref{def:opFT}}
  \\&= \frac{1}{\sqrt{2\pi}}\brs{\frac{-2\alpha}{ (s -i\beta)^2 -\alpha^2}}_{s=i\omega}
    && \text{by \prefp{ex:opLT_cosexp}}
  \\&= \frac{1}{\sqrt{2\pi}}\brs{\frac{-2\alpha}{ (i\omega -i\beta)^2 -\alpha^2}}
    && \text{by $s=i\omega\in\intoo{-\alpha}{\alpha}$}
  \\&= \frac{1}{\sqrt{2\pi}}
       \brs{\frac{2\alpha}
                 {(\omega - \beta)^2 + \alpha^2}}
  \\&= \frac{1}{\sqrt{2\pi}}
       \brs{\frac{2\alpha}
                 {\omega^2 -2\beta\omega + (\alpha^2 + \beta^2)}}
\end{align*}
\end{proof}

%-------------------------------------
\begin{example}
\label{ex:sineatau}
%-------------------------------------
\exbox{
  \opFT\brs{\sin(\beta x)e^{-\alpha\abs{x}}}
    = 0
 }
\end{example}
\begin{proof}
\begin{align*}
  \opFT\brs{\sin(\beta x)e^{-\alpha\abs{x}}}
    &\eqd \frac{1}{\sqrt{2\pi}}\brlr{\opLT\brs{\sin(\beta x)e^{-\alpha\abs{x}}}}_{s=i\omega}
    && \text{by definition Fourier Transform}
    && \text{\xref{def:opFT}}
  \\&= 0
    && \text{by \prefp{ex:opLT_cosexp}}
\end{align*}
\end{proof}

%%-------------------------------------
%\begin{example}
%\label{ex:cosexp}
%\footnote{
%  \url{https://math.stackexchange.com/questions/4015842/}
%  }
%%-------------------------------------
%\exbox{
%  \opFT\brs{\cos(\beta x)e^{-\alpha\abs{x}}}
%  = \frac{1}{\sqrt{2\pi}}\brs{\frac{\alpha}{\alpha^2+(\beta-\omega)^2}}
%  }
%\end{example}
%\begin{proof}
%\begin{align*}
%  \sqrt{2\pi}\opFT\brs{\cos(\beta\tau)e^{-\alpha\abs{\tau}}}
%    &\eqd \brs{\sqrt{2\pi}}\frac{1}{\sqrt{2\pi}}\int_{-\infty}^{\infty} \cos(\beta x)e^{-\alpha\abs{\tau}} e^{-j\omega\tau}\dtau
%    && \text{by definition Fourier Transform}
%  \\&= \int_{-\infty}^{\infty} \brs{\frac{e^{i\beta\tau}+e^{-i\beta\tau}}{2}} e^{-\alpha\abs{\tau}} e^{-j\omega\tau}\dtau
%    &&  \text{by \thme{Euler's identity}}&&\text{\xref{thm:eid}}
%  \\&= \frac{1}{2}\int_{-\infty}^{\infty} \brs{e^{-\alpha\abs{\tau}i\beta\tau-j\omega\tau}+e^{-\alpha\abs{\tau}-i\beta\tau-j\omega\tau}} \dtau
%  \\
%  \\
%  \\&= \frac{1}{\sqrt{2\pi}}\brs{\frac{\alpha}{\alpha^2+(\beta-\omega)^2}}
%\end{align*}
%\end{proof}


