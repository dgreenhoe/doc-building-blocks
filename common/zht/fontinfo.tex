%============================================================================
% Daniel J. Greenhoe
% XeLaTeX file
%============================================================================
%=======================================
\chapter{Font Information}
%=======================================
\lstset{language=[LaTeX]TeX}
%=======================================
\section{Typesetting engine}
%=======================================
This text was typeset using \hi{\XeLaTeX},
which is part of the \TeX family of typesetting engines, 
which is arguably the greatest development since the \hi{Gutenberg Press}.
Graphics were rendered using the \hie{pstricks} and related packages, and \hi{\LaTeX} graphics support.
\begin{lstlisting}
 \usepackage{etex}               % deal with counter problem
 \usepackage{xifthen}            % if then else support, includes \cnttest
 \usepackage{calc}               % calculation
 \usepackage{array}              % new tabular and array support, \newcolumntype
 \usepackage{longtable}          % tabular with page breaks
 \usepackage{xcolor}             % color support
 \usepackage{graphicx}           % 
 \usepackage{fancyhdr}           % headers
 \usepackage[Glenn]{fncychap}    % fancy chapter headings: Sonny, Lenny, Glenn, Conny, Rejne, Bjarne
 \usepackage{url}                % url support: \url{}  \path{}
 \usepackage{index}              % multiple index support
 \usepackage{fancybox}           % shadow, oval, and double boxes
 \usepackage[multiple]{footmisc} % footnote support
 \usepackage{fancyvrb}           % fancy verbatim -- supports verbatim in footnotes plus more
 \usepackage{nextpage}           % \cleartooddpage, \cleartoevenpage
 \usepackage{multicol}           % multiple column support
 \usepackage{colortbl}           % color in tables
%\usepackage{xetex-pstricks}     % pstricks help for XeLaTeX and XeTeX
 \usepackage{pst-node}           % psmatrix
 \usepackage[tiling]{pst-fill}   % pst fill package
 \usepackage{pst-plot}           % ps plots
 \usepackage{multido}            % loop support for ps diagrams
 \usepackage{nccrules}           % dashed patterns
 \usepackage{pstricks}           % graphics support
 \usepackage{pstricks-add}       % fixe and addons for pstricks
 \usepackage{pst-grad}           % gradients
 \usepackage{geometry}           %
 \usepackage{hyperref}           % hyperlinks in text
 \usepackage{prettyref}          % references with prefixes
 \usepackage{listings}           % source code listing support
\end{lstlisting}

%=======================================
\section{Latin fonts}
\label{sec:type_latin}
%=======================================
The main font for this document is \hie{Heros}---a \prope{sans-serif} font similar to Helvetica.
\hie{Heros} is from the \hie{\TeX-Gyre Project} and included in the {\TeX}Live distribution.
\begin{lstlisting}
 \setmainfont[
   Extension      = {.otf},
   UprightFont    = {*-regular},
   BoldFont       = {*-bold},
   ItalicFont     = {*-italic},
   BoldItalicFont = {*-bolditalic},
   Ligatures      = {NoCommon},
   ]{texgyreheros}   % sans-serif font
\end{lstlisting}
%{\rmfamily roman}, {\itshape italic}, and {\bfseries bold} font typefaces used 
%are all from the \hie{Heuristica} family of typefaces
%(based on the \hie{Utopia} typeface, released by \hie{Adobe Systems Incorporated}).
%The math font is {{\fntXits XITS}} from the {{\fntXits XITS font project}}.
The font used in quotation boxes on page~\pageref{quotepair} is adapted from {\fntZapf Zapf Chancery Medium Italic},
originally from URW++ Design and Development Incorporated.
The font used for the text in the title is {\fntAdventor Adventor} (similar to \hie{Avant-Garde}) from the \hie{\TeX-Gyre Project}.
The font used for the version identifier in the footer of individual pages is 
\mbox{\fntDigital\footnotesize Liquid} \mbox{\fntDigital\footnotesize Crystal} 
%{\fntDigital Liq}-{\fntDigital uid Crystal} 
(\hie{Liquid Crystal}) from \hie{FontLab Studio}.
%The Latin handwriting font is {\fntLavi Lavi} from the \hie{Free Software Foundation}.
%The traditional Chinese font used for the author's Chinese name is {{\fntzhthw 王漢宗中仿宋繁}}.\footnote{\sffamily%
%pinyin: \hi{W{/'a}ng H{/`an} Z{/=o}ng Zh/=ong F{/va}ng S{/`o}ng F{/'an}}; 
%translation: H/`an Z/=ong W/'ang's Medium-weight S/`ong-style Traditional Characters;
%literal:
%{\fntzhthw 王漢宗}$\sim$font designer's name;
%{\fntzhthw 中}$\sim$medium;
%{\fntzhthw 仿}$\sim$to imitate;
%{\fntzhthw 宋}$\sim$S/`ong (a dynasty);
%{\fntzhthw 繁}$\sim$traditional
%}

%=======================================
\section{IPA typesetting}
%=======================================
The IPA information is typeset using the \hie{GnuFree Sans-Serif font family}:
\\\indentx\url{http://www.gnu.org/s/freefont/}\\
The glyphs were entered using a \fncte{TECkit mapping}
in an {\XeLaTeX} environment with font feature specification as follows:
\\\indentx\lstinline*\newcommand{\fntipa}{\fntFreeSans\addfontfeatures{Mapping=/dan/r/common/TECkit/ascii-uipa}}*\\
\ldots where \verb|ascii-uipa| is a TECkit mapping compiled from a TECkit 
\verb|ascii-uipa.map| file which is about 350 lines long and looks in part something like this:\\
\begin{lstlisting}
 latin_capital_letter_S        <> latin_small_letter_esh         ;tipa| U+0053 <> U+0283
 latin_capital_letter_Z        <> latin_small_letter_ezh         ;tipa| U+005A <> U+0292
 BSL COL latin_small_letter_s  <> latin_small_letter_s_with_hook ;tipa| \:s    <> U+0282
 FSL COL latin_small_letter_s   > latin_small_letter_s_with_hook ;    | /:s    <> U+0282
\end{lstlisting}
As a result, ``\verb|{\fntipa San}|" typesets to ``{\fntipa San}".

%=======================================
\section{PinYin typesetting}
%=======================================
The PinYin information is typeset using the documents main Latin font (see \pref{sec:type_latin}) 
but with assistance from a \fncte{TECkit mapping}
in an {\XeLaTeX} environment with font feature specification as follows:\\
\begin{lstlisting}
\defaultfontfeatures{%
  Ligatures         = {NoCommon},% e.g. "fi" be represented as a single character?
  Mapping           = /dan/r/common/TECkit/punctuation,
  SmallCapsFeatures  = {Letters=SmallCaps},
  }
\end{lstlisting}
\ldots where \verb|punctuation| is a TECkit mapping compiled from a TECkit 
\verb|punctuation.map| file which is about 230 lines long and contains statements like these:\\
\begin{lstlisting}
 FSL grave_accent    latin_small_letter_a <> latin_small_letter_a_with_grave       ;/`a <> U+00E0
 FSL apostrophe      latin_small_letter_a <> latin_small_letter_a_with_acute       ;/'a <> U+00E1
 FSL colon           latin_small_letter_a <> latin_small_letter_a_with_diaeresis   ;/:a <> U+00E4
 FSL less_than_sign  latin_small_letter_a <> latin_small_letter_a_with_circumflex  ;/<a <> U+00E2
 FSL HYP             latin_small_letter_a <> latin_small_letter_a_with_macron      ;/-a <> U+0101
\end{lstlisting}
As a result, ``\verb|sh/-an|" typesets to ``sh/-an".

%=======================================
\section{Traditional Chinese typesetting}
%=======================================
\ftbox{\zhtssH{筆}{b/>i}{pen}}\hfill%
\begin{minipage}{\tw-50mm}
Traditional Chinese glyphs appearing in the text are from Google's 
\hie{Noto Sans CJK} font family: \url{ https://www.google.com/get/noto/help/cjk/}.
The Noto project has both \prope{sans-serif} (illustrated to the left) and \prope{serif} (illustrated to the right) 
fonts available.
The serif font is perhaps closer to the traditional brush stroke
(as is the Roman serif fonts closer to what was made by Romans using a chisel to 
carve letters in stones).
But it is likely that most people writing Chinese characters by hand will first pick up a pen,
not a brush, when learning to write Chinese. 
And pen strokes are by nature basically sans-serif.
\end{minipage}\hfill%
\ftbox{\zhtssHserif{筆}{b/>i}{pen}}
\\
\begin{lstlisting}
 \newfontfamily{\fntNotoSansCJK}[
   ExternalLocation,
   Path           = {/xfonts/noto/},
   Extension      = {.otf},
   UprightFont    = {*-Regular},
   BoldFont       = {*-Bold},
   ItalicFont     = {*-DemiLight},
   BoldItalicFont = {*-Black},
   Ligatures      = {NoCommon},% e.g. "fi" be represented as a single character?
   ]{NotoSansCJKtc}
\end{lstlisting}
\begin{tabular}{|c|c|c|c|}
  \hline
   \zhtssH{書}{sh/-u}{book}
  &\zhtssH{\bfseries 書}{sh/-u}{book}
  &\zhtssH{\itshape 書}{sh/-u}{book}
  &\zhtssH{\bfseries\itshape 書}{sh/-u}{book}
  \\\hline
  upright & bold & italic & bold italic
  \\\hline
\end{tabular}

%\begin{minipage}{\tw-62mm}
%Traditional Chinese glyphs and ZhuYin appearing in the text are from the font {\fntzht 王漢宗}
%(font designer's name?)
%family of font typefaces.
%\end{minipage}%
%\hfill%
%\ftbox{\zhtss{王}{W/'ang} {King}\zhtss{漢}{H/`an}  {the Han people}\zhtss{宗}{Z/-ong} {ancestor}}

%\begin{lstlisting}
%%----------------------------------------------------------------------------
%% Default font features
%%----------------------------------------------------------------------------
%\defaultfontfeatures{%
%  Ligatures         = {NoCommon},% e.g. "fi" be represented as a single character?
%  Mapping           = /dan/r/common/TECkit/punctuation,
%  SmallCapsFeatures = {Letters=SmallCaps},
%  }
%%----------------------------------------------------------------------------
%% Normal traditional Chinese typefaces
%% http://apt.nc.hcc.edu.tw/pub/FreeSoftware/free_fonts/wangttf/
%%----------------------------------------------------------------------------
%\newfontfamily{\fntWangClear}[
%  ExternalLocation,
%  Path           = {/xfonts/zht/},
%  Extension      = {.ttf},
%  UprightFont    = {wt002},
%  BoldFont       = {wt004},
%  ItalicFont     = {wt024},
%  BoldItalicFont = {wt034},
%  ]{wt002}
%\end{lstlisting}
%
%\begin{longtable}{|ll|}
%  \hline
%  upright font: & 
%    \zhtss{王}{W/'ang} {King}
%    \zhtss{漢}{H/`an}  {the Han people}
%    \zhtss{宗}{Z/-ong} {ancestor}
%    \zhtss{中}{zh/-ong}{medium}
%    \zhtss{明}{M/'ing} {M/'ing Dynasty}
%    \zhtss{體}{t/>i}   {style}
%    \zhtss{繁}{f/'an}  {traditional}
%    \\
%  Bold font: & 
%    \zhtss{\bfseries 王}{W/'ang} {King}
%    \zhtss{\bfseries 漢}{H/`an}  {the Han people}
%    \zhtss{\bfseries 宗}{Z/-ong} {ancestor}
%    \zhtss{\bfseries 特}{t/`e}   {special}
%    \zhtss{\bfseries 明}{M/'ing} {M/'ing Dynasty}
%    \zhtss{\bfseries 體}{t/>i}   {style}
%    \zhtss{\bfseries 繁}{f/'an}  {traditional}
%    \\
%  Italic font: & 
%    \zhtss{\itshape 王}{W/'ang} {King}
%    \zhtss{\itshape 漢}{H/`an}  {the Han people}
%    \zhtss{\itshape 宗}{Z/-ong} {ancestor}
%    \zhtss{\itshape 中}{zh/-ong}{medium}
%    \zhtss{\itshape 仿}{f/>ang} {to imitate}
%    \zhtss{\itshape 宋}{S/`ong} {Sung Dynasty}
%    \zhtss{\itshape 繁}{f/'an}  {traditional}
%    \\
%  Bold Italic font: & 
%    \zhtss{\bfseries\itshape 王}{W/'ang} {King}
%    \zhtss{\bfseries\itshape 漢}{H/`an}  {the Han people}
%    \zhtss{\bfseries\itshape 宗}{Z/-ong} {ancestor}
%    \tcom{%
%      \zhtss{\bfseries\itshape 勘}{k/-an}  {to collate}
%      \zhtss{\bfseries\itshape 亭}{t/'ing} {pavillion}
%      \zhtss{\bfseries\itshape 流}{li/'u}  {style}
%      }{K/-anT/'ing Calligraphic Style}
%    \zhtss{\bfseries\itshape 繁}{f/'an}  {traditional}
%  \\\hline
%\end{longtable}
%%   UprightFont    = {wt002},                          % 王漢宗中明體繁  HanWangMingMedium
%%   BoldFont       = {wt004},                          % 王漢宗特明體繁  HanWangMingHeavy
%%   ItalicFont     = {wt024},                          % 王漢宗中仿宋繁  HanWangFangSongMedium
%%   BoldItalicFont = {wt034},                          % 王漢宗勘亭流繁  HanWangKanTan
%
%
%%(H/`an Z/-ong W/'ang Medium-weight Clear Style Traditional):

%=======================================
\section{Ruby characters}
%=======================================
\begin{minipage}{\tw-70mm}
Ruby characters are Asian characters with pronunciation symbols adjacent to that character.
For traditional Chinese characters, the symbols are ZhuYin and are located to the right of the character.
\end{minipage}%
\hfill%
\ftbox{\Huge\fntzhtzy 我看到你}

In this document, the font used to achieve this is
\\\indentx
\tcom{%
\zhtss{王}{W/'ang} {King}
\zhtss{漢}{H/`an}  {the Han people}
\zhtss{宗}{Z/-ong} {ancestor}
}{Font designer's name?}
\zhtss{中}{zh/-ong}{medium}
\zhtss{明}{M/'ing} {M/'ing Dynasty}
\zhtss{體}{t/>i}   {style}
\zhtsd{注}{音}{zh/`u}{y/-in}{to annotate}{sound}{ZhuYin}

\begin{lstlisting}
%----------------------------------------------------------------------------
% Chinese typefaces with Zhuyin
% http://apt.nc.hcc.edu.tw/pub/FreeSoftware/free_fonts/wangttf/
%----------------------------------------------------------------------------
 \newfontfamily{\fntWangClearZY}[%
   ExternalLocation,
   Path           = {/xfonts/zht/},
   Extension      = {.ttf},
   UprightFont    = {wp010-05},
   BoldFont       = {wp010-05},
   ItalicFont     = {wp010-05},
   BoldItalicFont = {wp010-05},
   ]{wp010-05}
\end{lstlisting}

%=======================================
\section{Simplified Chinese typesetting}
%=======================================
Just as the Traditional Chinese glyphs, Simplified Chinese glyphs are from 
Google's Noto Sans CJK (both traditional and simplified are in the same OpenType files).
%\begin{lstlisting}
%%----------------------------------------------------------------------------
%% Firefly
%% Arphic public license
%% http://www.study-area.org/apt/firefly-font/
%% includes traditional and simplified characters
%% includes Kangxi radicals (U+2F00 - U+2FDF)
%%----------------------------------------------------------------------------
% \newfontfamily{\fntFirefly}[
%   ExternalLocation,
%   Path           = {/xfonts/zh/firefly/},
%   Extension      = {.ttf},
%   Ligatures      = {ResetAll},
%   ]{fireflysung}
%\end{lstlisting}
But in this document, the simplified characters were not typed in directly, 
rather traditional characters were
typed, and then mapped to simplified characters using a \fncte{TECkit mapping}
in an {\XeLaTeX} environment with font feature specification as follows:
\\\indentx\lstinline*\newcommand{\fntzhs}{\fntNotoSansCJK\addfontfeatures{Mapping=../common/TECkit/zht2zhs}}*\\
\ldots where \verb|zht2zhs| is a TECkit mapping compiled from a TECkit 
\verb|zht2zhs.map| file which is about 3000 lines long and in part looks something like this:
\begin{lstlisting}
;==============================================================================
; Variants to TECkit mapping file
; Daniel J. Greenhoe 
;==============================================================================
 ...
U+346F	>  U+3454 ;     37
U+3473	>  U+3447 ;     38
U+3493	>  U+20242 ;     41
U+34E8	>  U+523E ;     52
U+35F2	>  U+20D7E ;     68
U+361A	>  U+360E ;     72
U+3704	>  U+36AF ;     87
U+370F	>  U+36E3 ;     88
 ...
\end{lstlisting}

{\XeLaTeX} commands have also been defined such as the following:
\begin{lstlisting}
\newcommand{\zhtss}[3]{% 
  \begin{tabular}[t]{@{}c@{}}
    \color{red}\footnotesize#2\\
    \color{black}\large{\fntzht #1}\\
    \color{black}\large{\fntzhs #1}\\
    \color{blue}\footnotesize#3%
  \end{tabular}%
  \index[xchar]{{\fntzht #1}/{\fntzhs #1} (#2) #3}
  \index[xpinyin]{#2 ({\fntzht #1/\fntzhs #1}) #3}
  \index[xeng]{#3 {\fntzht #1/\fntzhs #1} (#2)}
  }
\end{lstlisting}

\begin{minipage}{\tw-21mm}
In the actual body of the document, the information for a character can be input as follows:
%\begin{lstlisting}
\quad\verb|\zhtss{|\zht{見}\verb|}{ji/`an}{to see}|
%\end{lstlisting}
\\which writes index information to three different index files
(a character index file, a PinYin index file, and an English word index file)
and typesets glyphs as appearing to the right:
\end{minipage}\hfill%
\ftbox{\zhtss{見}{ji/`an}{to see}}

\begin{minipage}{\tw-69mm}
Note that it is arguably safer to map from traditional to simplified rather than the converse because
the traditional to simplified mapping tends to be \prope{many-to-one}. 
An example is the set of traditional characters 
%$\setn{\text{\zht{干}, \zht{幹}, \zht{乾}, \zht{榦}}}$,
$\setn{\text{\zht{干}, \zht{幹}, \zht{乾}}}$,
which all map to the simplified character \zhs{干}.\footnotemark
\end{minipage}%
\hfill% 干幹乾榦
\begin{tabular}{|*{3}{c|}}
  \hline
  \footnotesize U+4E7E
 &\footnotesize U+5E72
 &\footnotesize U+5E79
%&\footnotesize U+69A6
  \\
   \zhtss{乾}{g/-an}{dry}
  &\zhtss{干}{g/-an}{to offend}
  &\zhtss{幹}{g/`an}{tree trunk}
  %&\zhtss{榦}{g/`an}{railings}}\citetblt{\url{https://people.w3.org/rishida/scripts/chinese/}}
  \\
  \footnotesize U+5E72
 &\footnotesize U+5E72
 &\footnotesize U+5E72%
%&\footnotesize U+5E72
  \\\hline
\end{tabular}

However, this is not always the case.\\ 
Sometimes the traditional to simplified mapping is \prope{one-to-many}.\citetbl{
  \url{http://hanzidb.org/character-list/multiple-simplified-variants}\\
  %\url{http://hanzidb.org/character/%E5%BE%81}\\
  %\url{http://hanzidb.org/character/%E5%BE%B5}
  }
\\\begin{tabular}{|>{\fntzht}cc>{\fntzhs}cllc|}
  \hline
  traditional & & simplified & PinYin & meaning & Unicode\\
  徵 &$\leftarrow$&        征 & zh/-eng & to invade &\ttfamily U+5FB5 $\leftarrow$ U+5F81 \\
  徵 &$\leftarrow$&\fntzht 徵 & zh/-eng & to summon &\ttfamily U+5FB5 $\leftarrow$ U+5FB5 
  \\\hline
\end{tabular}

%=======================================
\section{Generating the zht2zhs TECkit map file}
\label{sec:fontinfo_var2map}
%=======================================
As indicated previously, this document supports both traditional and simplified characters
where the simplified is in most all cases mapped, when necessary, from the traditional using
the TECkit map 
%``\href{https://github.com/dgreenhoe/unihan}{\verb|zht2zhs.map|}".
``\href{https://github.com/dgreenhoe/unihan}{zht2zhs.map}".
A mapping is ``necessary" when there is ``variation" between the two for a 
given character.

The traditional-simplified variants are identified in a document from \url{www.unicode.org}
called 
%``\href{https://www.unicode.org/Public/10.0.0/ucd/}{\verb|Unihan_Variants.txt|}". 
``\href{https://www.unicode.org/Public/10.0.0/ucd/}{Unihan\_Variants.txt}". 
Version 10.0.0, for example, of this file can be found in 
%``\href{https://www.unicode.org/Public/10.0.0/ucd/}{\verb|Unihan.zip|}"
``\href{https://www.unicode.org/Public/10.0.0/ucd/}{Unihan.zip}"
at \url{https://www.unicode.org/Public/10.0.0/ucd/}.

For this document, 
%``\href{https://github.com/dgreenhoe/unihan}{\verb|zht2zhs.map|}"
``\href{https://github.com/dgreenhoe/unihan}{zht2zhs.map}"
was generated from 
%``\href{https://www.unicode.org/Public/10.0.0/ucd/}{\verb|Unihan_Variants.txt|}"
``\href{https://www.unicode.org/Public/10.0.0/ucd/}{Unihan\_Variants.txt}"
using a simple command line utility compiled from the C source file called 
%``\href{https://github.com/dgreenhoe/unihan}{\verb|var2map.c|}" 
``\href{https://github.com/dgreenhoe/unihan}{var2map.c}" 
resulting in a program which
can be used like this:
\\\indentx\verb|var2map.exe Unihan_Variants_10-0-0.txt zht2zhs_10-0-0.map|
%For a source listing, see \prefpp{sec:src_var2map}.
To download the C source, go to \url{https://github.com/dgreenhoe/unihan}.

%=======================================
\section{Default bullet}
%=======================================
\textimgr[11mm]{../common/graphics/watercraft/ghind_blue.pdf}{%
  The ship 
  %$\ds\brp{\text{\raisebox{-2mm}{\includegraphics*[height=8mm]{../common/graphics/watercraft/ghindgray_djg.eps}}}}$
  appearing throughout this text is loosely based on the \hie{Golden Hind}, a sixteenth century English galleon 
  famous for circumnavigating the globe.\footnotemark
  The ship image as appears in this document was rendered using PStricks packages
  using code which includes something like that listed below:
  }
\footnotetext{\citerpgc{paine2000}{63}{0395984149}{Golden Hind}}
\begin{lstlisting}
%-------------------------------------
% forward and middle mast and sails
%-------------------------------------
{\psset{fillstyle=gradient}%
\multirput(21,31)(38,-6){2}{% forward and middle masts and sails
  \pscustom[linewidth=2pt,arrows=-]{% upper sail
    \pscurve(-18,8)(-20,22)(-8,36)%
    \psline(-8,36)(8,52)%
    \pscurve(8,52)(9,42)(14,40)%
    \pscurve(14,40)(2,30)(-18,8)%
    }%
  \pscustom[linewidth=2pt,arrows=-]{% lower sail
    \psline(-16,8)(16,40)%
    \pscurve(16,40)(12,28)(16,20)%
    \psline(16,20)(-16,-8)%
    \pscurve(-16,-8)(-20,2)(-16,8)%
    }%
  \psline[linewidth=10pt](-17,28)(8,52)% upper crossbeam
  \psline[linewidth=10pt](-16,8)(16,40)% lower crossbeam
  \psline[linewidth=10pt](0,0)(0,50)% vertical mast
  }%
\end{lstlisting}

