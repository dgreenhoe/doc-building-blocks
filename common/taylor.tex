%============================================================================
% XeLaTeX File
% Daniel J. Greenhoe
%============================================================================
%=======================================
\chapter{Taylor Expansions (Transforms)}
\label{chp:taylor}
%=======================================
%=======================================
\section{Introduction}
%=======================================
For modeling real-world processes above the quantum level, measurements are \prope{continuous} in time---that is,
the first derivative of a function over time representing the measurement \prope{exists}.

But even for ``simple" physical systems, it is not just the first derivative that matters.
For example, the classical ``vibrating string" vertical displacement $\fu(x,t)$ wave equation can be described as
        \\\indentx$\ds\ppderiv{u}{x^2} - \frac{1}{c^2}\ppderiv{u}{t^2} = 0$

\begin{figure}
  \centering
  \begin{tabular}{c}
    \includegraphics[width=\tw-10mm]{../common/math/graphics/pdfs/sunspots.pdf}\\
    \includegraphics[width=\tw-10mm,height=50mm]{../common/math/graphics/pdfs/earthquake_kapi_20180928.pdf}
  \end{tabular}
  \caption{Sunspot and earthquake measurements\label{fig:sunspot}}
\end{figure}
Not only do physical systems demonstrate heavy dependence on the derivatives of their measurement functions,
but also commonly exhibit \prope{oscillation}, as demonstrated by sunspot activity over the last 300 years or
earthquake activity \xref{fig:sunspot}.

In fact, derivatives and oscillations are fundamentally linked
as demonstrated by the fact that 
all solutions of homogeneous second order differential equations
        are linear combinations of sine and cosine functions\ifsxref{harTrig}{thm:D2f_cos_sin}:
        \\\indentx$\ds  \brb{\opDiff\ff + \ff=0}
  %     {\fncte{2nd order homogeneous differential equation}}
  \quad\iff\quad
  \brb{\ff(x) = \ff(0)\,\cos(x) + \ff^\prime(0)\,\sin(x)}
  \qquad\msizes\forall\ff\in\spC,\,\forall x\in\R$

Derivatives are calculated \prope{locally} about a point.
Oscillations are observed \prope{globally} over a range,
and analyzed (decomposed) by projecting the function onto a sequence of basis functions---sinusoids 
in the case of Fourier Transform family.
Projection is accomplished using inner products, and often these are calculated using \ope{integration}.
Note that derivatives and integrals are also fundamentally linked as demonstrated by the
\thme{Fundamental Theorem of Calculus}\ldots which shows that integration 
can be calculated using anti-differentiation:
\\\indentx$\ds\int_a^b \ff(x)\dx = \fF(b) - \fF(a)$\qquad where $\fF(x)$ is the \fncte{antiderivative} of $\ff(x)$.

%In nature, take displacement, velocity, and acceleration for example:
%\\\indentx$\begin{array}{rc>{\ds}l}
%  \fx(t) &=& \int_{u=0}^t \fv(u) \du + \mcom{\fx(0)}{initial condition}
%  \\
%  \fv(t) &=& \int_{u=0}^t \fa(u) \du + \mcom{\fv(0)}{initial condition}
%\end{array}$

Brook Taylor showed that for \prope{analytic} functions,\footnote{
  \prope{analytic} functions: Functions for which all their derivatives exist.}
knowledge of the derivatives of a function at a location $x=a$
%\\\indentx$\seqn{\ff(a),\, \frac{1}{1!}\ff'(a),\,  \frac{1}{2!}\ff''(a),\,  \frac{1}{3!}\ff'''(a),\cdots}$\\
%of the Taylor polynomial at the point $x=a$ 
allows you to determine (predict) arbitrarily closely all the points $\ff(x)$ in the vicinity of $x=a$:\footnote{
\citePp{robinson1982}{886}
}
\\\indentx$\ds\ff(x) = \ff(a) + \frac{1}{1!}\ff'  (a)\brs{x-a}
                              + \frac{1}{2!}\ff'' (x)\brs{x-a}^2
                              + \frac{1}{3!}\ff'''(x)\brs{x-a}^3
                              + \cdots
                  $

On the other hand, the \ope{Fourier Transform} is a kind of counter-part of the Taylor expansion:\footnote{
        \citePp{robinson1982}{886}
        }
        %\begin{tabular}{|@{2}{p{\tw/2-5mm}|}}
        \\\begin{tabular}{|c|l|l|}
            \hline
              & \mc{1}{|c|}{Taylor coefficients} & \mc{1}{c|}{Fourier coefficients}
            \\\hline
              \imark&Depend on derivatives $\ds\dndxn\ff(x)$        &Depend on integrals   $\ds\int_{x\in\R} \ff(x)e^{-i\omega x} \dx$
            \\\imark&Behavior in the vicinity of a point.           &Behavior over the entire function.
            \\\imark&Demonstrate trends locally.                    &Demonstrate trends globally, such as oscillations.
            \\\imark&Admits \prope{analytic} functions only.        &Admits \prope{non-analytic} functions as well.
            \\\imark&Function must be \prope{continuous}.           &Function can be \prope{discontinuous}.
            \\\hline
        \end{tabular}

%=======================================
\section{Taylor Expansion}
%=======================================
%--------------------------------------
\begin{theorem}[\thmd{Taylor Series}]
\footnote{
  \citerpgc{flanigan1983}{221}{0486613887}{Theorem 15},
  \citerpg{strichartz1995}{281}{0867204710},
  \citerpgc{sohrab2003}{317}{0817642110}{Theorem 8.4.9},
  \citor{taylor},
  \citor{maclaurin}
  %\citerc{as}{pages 14, 880}
  %\citerp{mallat}{164}
  }
\label{thm:taylor}
%--------------------------------------
Let $\spC$ be the space of all \prope{analytic} functions
and $\hxs{\opDif}$ in $\clF{\spC}{\spC}$ the \ope{differentiation operator}. 
\thmboxt{
  A \opd{Taylor Series} about the point $x=a$ of a function $\ff(x)\in\clF{\spC}{\spC}$ is
  \\\indentx$\ds\begin{array}{rc>{\ds}lC}
    \ff(x) &=& \sum_{n=0}^\infty \mcom{\frac{\brs{\opDif^n\ff}(a)}{n!}}{coefficient}\:\mcoml{(x-a)^n}{basis function}
           &   \forall a\in\R,\,\ff\in\spC
  \end{array}$
  \\A \opd{Maclaurin Series} is a \ope{Taylor Series} about the point $a=0$.
  }
\end{theorem}

