%============================================================================
% Daniel J. Greenhoe
% LaTeX File
%============================================================================

%======================================
\chapter{Foundational Mathematical Structures}
%======================================

\qboxnpqt
  {Ren/'e Descartes, philosopher and mathematician (1596--1650)
   \index{Descartes, Ren/'e}
   \index{quotes!Descartes, Ren/'e}
   \footnotemark}
  %{../common/people/small/descart.jpg}
  %{../common/people/descartes_fransHals_wkp.jpg}
  {../common/people/descartes_fransHals_wkp_bw.jpg}
  {Je me plaisois surtout aux math/'ematiques,
    /`a cause de la certitude et de l'/'evidence de leurs raisons:
    mais je ne remarquois point encore leur vrai usage;
    et, pensant qu'elles ne servoient qu'aux arts m/'ecaniques,
    je m'/'etonnois de ce que leurs fondements /'etant si fermes et si solides,
    on n'avoit rien b/<ati dessus de plus relev/'e:}
  {I was especially delighted with the mathematics,
    on account of the certitude and evidence of their reasonings;
    but I had not as yet a precise knowledge of their true use;
    and thinking that they but contributed to the advancement of the mechanical arts,
    I was astonished that foundations, so strong and solid,
    should have had no loftier superstructure reared on them.}
  \citetblt{
    quote: & \citer{descartes_method} \\
    translation: & \citerc{descartes_method_eng}{part I, paragraph 10} \\
    image: & \scs\url{http://en.wikipedia.org/wiki/File:Frans_Hals_-_Portret_van_Ren\%C3\%A9_Descartes.jpg}, public domain
    }

The definition of some basic mathematical structures makes the explanation of wavelet structures more efficient and precise.
%=======================================
\section{Sets, Relations, and Order}
%=======================================
%\begin{liste}
\begin{itemize}
  \item A structure $\hxs{\setX}$ is a \hib{set} if $\setX$ is any collection of objects.
        For example, the set of all intergers $\Z$ is the collection of the numbers
        $\setn{\cdots,\, -2,\, -1,\, 0,\, 1,\, 2,\, \cdots}$.%
        \ifdochas{found}{\footnote{\hie{sets}: see \prefpp{def:set}.}}
        The notation $x\in\setX$ denotes that $x$ is an element of the set $\setX$.

  \item A set $\setY$ is a \hib{subset} of a set $\setX$ if every element of $\setY$ is also an element in $\setX$;
        and $\setY\hxs{\subseteq}\setX$ denotes that $\setY$ is a subset of $\setX$.
        Note that $\setX$ is a subset of itself ($\setX\subseteq\setX$).
        A structure $\hxs{\psetx}$ is the \hib{power set} of a set $\setX$ if 
        $\psetx$ is the set  of all subsets of $\setX$.
        A set $\setX\hxs{\setu}\setY$ is the \hib{union} of a set $\setX$ and a set $\setY$ if 
        $\setX\setu\setY=\set{z}{z\in\setX \text{ or } z\in\setY \text{ or both}}$.
        A set $\setYc$ is the \hib{complement} of a set $\setY$ in a set $\setX$ if
        $\setYc\setu\setY=\setX$.
        \ifdochas{found}{\footnote{\hie{sets}: see \prefpp{def:set}.}}

  \item A structure $\hxs{\opair{x}{y}}$ is an \hib{ordered pair} from the set $\setX$ to the set $\setY$
        if $x$ is an element of $\setX$ (denoted $x\in\setX$) and $y$ is an element of $\setY$.
        For example, if $\Z$ is the set of integers and $\setA\eqd\setn{a,\,b,\,c,\,\cdots,\,z}$, then
        the following are \emph{ordered pairs} from $\Z$ to $\setA$:
        $\opair{1}{a}$, $\opair{-5}{r}$, and $\opair{6}{e}$.%
        \ifdochas{found}{\footnote{\hie{ordered pair}: see \prefpp{def:(a,b)}.}}

  \item A set $\hxs{\setX\times\setY}$ is the \hib{Cartesian product} of the sets $\setX$ and $\setY$
        if $\setX\times\setY$ is the set of all ordered pairs from $\setX$ to $\setY$ (\prefp{def:AxB}).
          \\\indentx$\setX\times\setY \eqd \set{{\opair{x}{y}}}{x\in\setX\text{ and }y\in\setY}.$\\
        For example, if $\setX\eqd\setn{1,\,2,\,3}$ and $\setY\eqd\setn{A,\,B}$ then
          \\\indentx$\setX\times\setY = \setn{\opair{1}{A},\,\opair{1}{B},\,\opair{2}{A},\,\opair{2}{B},\,\opair{3}{A},\,\opair{3}{B}}.$

  \item A set $\hxs{\rel}$ is a \hib{relation} with \hib{domain} $\setX$ and \hib{range} $\setY$ if
        $\rel$ is any subset of the Cartesian product $\setX\times\setY$.%
        \ifdochas{found}{\footnote{\hie{relation}: see \prefpp{def:relation}.}}
        The set of all relations with domain $\setX$ and range $\setY$ is denoted $\clRxy$.
        For example, if $\setX\eqd\setn{1,\,2,\,3}$ and $\setY\eqd\setn{A,\,B}$ then one relation in $\clRxy$ is
        \\\indentx$\relation = \setn{\opair{1}{A},\,\opair{1}{B},\,\opair{2}{A}}$.
        \\Expressed in an alternative form, we can say that the following statements are true for the relation $\relation$: 
        \\\indentx$1\relation A$, $1\relation B$, and $2\relation A$.
%        Common examples of relations include the \hie{order} relations in $\clRrr$ such as $\le$ and $\ge$.

        \begin{minipage}{\tw-50mm-10mm}
          In another example, if $\setX\eqd\setn{1,\,2,\,3}$ and $\setY\eqd\setn{1,\,2}$ then a relation in $\clRxy$ is
          the standard order relation
          \\\indentx$\le = \setn{\opair{1}{1},\,\opair{1}{2},\,\opair{2}{2}}$.
          \\Expressed in an alternative form, we can say that the following statements are true for the relation $\le$: 
        \\\indentx$1\le 1$, $1\le 2$, and $2\le 2$.
        \\This relation is illustrated to the right.
        \indxs{\le}
        \end{minipage}
        \hfill
        \begin{minipage}{50mm}
          %============================================================================
% LaTeX file
% Daniel J. Greenhoe
% X={1,2,3}  Y={1,2}  <= relation in 2^{XY}
% nominal unit = 7.5mm
%============================================================================
\begin{pspicture}(-1,-2)(5,3)
  \psset{fillstyle=none,linecolor=black,fillcolor=black}%
  \psellipse[linecolor=set](0,0)(1,2)%
  \psellipse[linecolor=set](4,0)(1,2)%
  \psdot(0, 1)% 1
  \psdot(0, 0)% 2
  \psdot(0,-1)% 3
  \psdot(4, 1)% 1
  \psdot(4, 0)% 2
  \psbezier[linecolor=blue]{->}(0, 1)(1, 2)(3, 2)(4, 1)%  (1,1)
  \psline  [linecolor=blue]{->}(0, 1)(4, 0)%              (1,2)
  \psbezier[linecolor=red] {->}(0, 0)(1,-1)(3,-1)(4, 0)%  (2,2)
  \uput[90]{0}(0, 2.1){$\setX$}
  \uput[90]{0}(4, 2.1){$\setY$}
  \uput[90]{0}(2, 2){$\leq$}
  \uput[180]{0}(-0, 1){$1$}
  \uput[180]{0}(-0, 0){$2$}
  \uput[180]{0}(-0,-1){$3$}
  \uput  [0]{0}( 4, 1){$1$}
  \uput  [0]{0}( 4, 0){$2$}
  {\scriptsize
  \rput[t](2,1.6){$\opair{1}{1}$}
  \rput[t](2,0.2){$\opair{1}{2}$}
  \rput[t](2,-0.9){$\opair{2}{2}$}
  }
  %\psgrid[gridcolor=green,subgridcolor=green](-1,-2)(5,3)
\end{pspicture}%

        \end{minipage}

  \item The concept of order is much more general than illustrated in the previous item.
        And not only is it very general, the \hie{order relation}
        is one of the most fundamental relations in mathematics.
         A relation $\orel$ is an \hib{order relation} on $\setX$ if%
        \ifdochas{found}{\footnote{\hie{order relation}: see \prefpp{def:orel}.}}
        \indxs{\orel}
         \\$\begin{array}{@{\qquad}Fl@{\qquad}C@{\qquad}DD@{}r@{}D}
           \cline{6-6}
           1. & x \orel x
              & \forall x\in\setX
              & (\prop{reflexive})
              & and \hspace{2ex}
              & \text{ }\vline
              & \text{\hspace{2ex}preorder}
              \\
           2. & x \orel y \text{ and } y \orel z \implies x \orel z
              & \forall x,y,z\in\setX
              & (\prop{transitive})
              & and
              & \vline
           \\\cline{6-6}
           3. & x \orel y \text{ and } y \orel x \implies x=y
              & \forall x,y\in\setX
              & (\prop{anti-symmetric})
              &
              &
         \end{array}$
         \\
         A set $\setX$ together with an order relation $\orel$ on $\setX$ is an
         \hib{ordered set} $\hxs{\opair{\setX}{\orel}}$.
         \\
         \begin{minipage}[c]{41mm}
           \begin{center}%
           %============================================================================
% Daniel J. Greenhoe
% LaTeX file
% lattice (2^{x,y,z}, subseteq)
% recommended unit = 10mm
%============================================================================
{\psset{unit=0.75\psunit}%
\begin{pspicture}(-2.4,-.3)(2.4,3.3)
  %---------------------------------
  % settings
  %---------------------------------
  \psset{%
    labelsep=1.5mm,
    }%
  %---------------------------------
  % nodes
  %---------------------------------
  \Cnode(0,3){t}
  \Cnode(-1,2){xy} \Cnode(0,2){xz} \Cnode(1,2){yz}
  \Cnode(-1,1){x}  \Cnode(0,1){y}  \Cnode(1,1){z}
  \Cnode(0,0){b}
  %---------------------------------
  % node connections
  %---------------------------------
  \ncline{t}{xy}\ncline{t}{xz}\ncline{t}{yz}
  \ncline{x}{xy}\ncline{x}{xz}
  \ncline{y}{xy}\ncline{y}{yz}
  \ncline{z}{xz}\ncline{z}{yz}
  \ncline{b}{x} \ncline{b}{y} \ncline{b}{z}
  %---------------------------------
  % node labels
  %---------------------------------
  \uput[180](t) {$\setn{x,y,z}$}%
  \uput[180](xy){$\setn{x,y}$}%   
 %\uput{1pt}[ 70](xz){$\setn{x,z}$} 
  \uput[0](yz){$\setn{y,z}$}%
  \uput[180](x) {$\setn{x}$}%     
 %\uput{1pt}[-45](y) {$\setn{y}$}   
  \uput[0](z) {$\setn{z}$}%
  \uput[180](b) {$\szero$}%
  \uput[0](1,3){\rnode{xzlabel}{$\setn{x,z}$}}% 
  \uput[0](1,  0){\rnode{ylabel}{$\setn{y}$}}%
  \ncline[linestyle=dotted,linecolor=red,nodesep=1pt]{->}{xzlabel}{xz}%
  \ncline[linestyle=dotted,linecolor=red,nodesep=1pt]{->}{ylabel}{y}%
\end{pspicture}
}%
           \end{center}%
         \end{minipage}
         \hfill
         \begin{minipage}{\tw-41mm-15mm}
           An example of an ordered set is $\opair{\pset{\setn{x,y,z}}}{\subseteq}$,
           where $\hxs{\pset{\setn{x,y,z}}}$ is the \hib{power set} (the set of all subsets)
           of the set $\setn{x,y,z}$ and $\subseteq$ is the standard subset order relation on $\pset{\setn{x,y,z}}$.
           The ordered set $\opair{\pset{\setn{x,y,z}}}{\subseteq}$ is illustrated to the left using a
           \hie{Hasse diagram}. In a Hasse diagram, elements of an ordered set $\opair{\setX}{\rel}$ are represented by dots,
           and an element $x$ drawn somewhere below an element $y$ with a line connecting them 
           implies that $x\rel y$.\footnotemark
         \end{minipage}%
         \footnotetext{\structe{Hasse diagram}: \prefp{def:hasse}}

         \begin{minipage}{\tw-41mm-15mm}
           Another example of an ordered set is
           $\opair{\setn{1,2,3,5,6,10,15,30}}{|}$
           where $\hxs{|}$ is the ``divides" relation on the set $\Zp$ of positive integers such that
           $n|m$ represents $m$ divides $n$.
           The ordered set
           is illustrated by a \hie{Hasse diagram} to the right.
           It is easy to see the that the architecture of this and the previous lattice are essentially identical,
           only the labels are different.
           \indxs{|}
         \end{minipage}%
         \citetblt{
           \citerpg{maclane1999}{484}{0821816462}\\
           %\citerpg{menini2004}{60}{0824709853}\\
           %\citerp{huntington1933}{278}\\
           \citorpc{sheffer1920}{310}{footnote 1}
           }%
         \begin{minipage}{41mm}%
         \begin{center}
         \input{../common/math/graphics/lat2235.tex}%
         \end{center}
         \end{minipage}%

  \item A relation $\hxs{\ff}$ is a \hib{function} if each element in the domain of $\ff$
        is paired with no more than one element in the range of $\ff$. That is,
        \ifdochas{found}{\footnote{\hie{function}: see \prefpp{def:fnd_f(x)}}}
        \indxs{\ff}
          \\\indentx$\ds\mcom{\opair{x}{y_1},\opair{x}{y_2}\in\ff}{$\ff(x)=y_1$ and $\ff(x)=y_2$} \quad\implies\quad y_1=y_2.$
          \\
        For example, if $\setX\eqd\setn{1,\,2,\,3}$ and $\setY\eqd\setn{A,\,B,\,C}$ then
        \\$\begin{array}{cclM}
          \rel  &\eqd& \setn{\opair{1}{A},\,\opair{1}{B},\,\opair{2}{C}} & \emph{is} a \hi{relation} but is \emph{not} a function.\\
          \rela &\eqd& \setn{\opair{1}{A},\,\opair{2}{A}}                & \emph{is} a \hi{relation} \emph{and} is also a \emph{function}.
        \end{array}$
        \\
        \prefpp{fig:wavfound_relfnctd}
        illustrates two discrete examples of relations that \emph{are} functions
        and two that are \emph{not}.
        \prefpp{fig:wavfound_relfnctc}
        illustrates one example of a continuous relation that is \emph{not} a function
        and one that \emph{is}.
        \begin{figure}[t]%
          \footnotesize%
          %\psset{unit=7mm}%
          %============================================================================
% LaTeX file
% Daniel J. Greenhoe
% X={1,2,3}  Y={1,2}  <= relation in 2^{XY}
%============================================================================
%-------------------------------------
% settings
%-------------------------------------
{\psset{%
  labelsep=2.8mm
  }
\begin{tabular*}{\tw}{c@{\extracolsep\fill}ccc}
%=======================================
% Diagram 1
%=======================================
\begin{pspicture}(-1,0.5)(3,5)%
  %-------------------------------------
  % Venn shapes
  %-------------------------------------
  \psellipse[linecolor=red,linewidth=1pt](0,2.5)(0.5,2)%
  \psellipse[linecolor=red,linewidth=1pt](2,2.5)(0.5,2)%
  %-------------------------------------
  % nodes
  %-------------------------------------
  \Cnode*(0,4){x4}% 
  \Cnode*(0,3){x3}%
  \Cnode*(0,2){x2}%
  \Cnode*(0,1){x1}%
  \Cnode*(2,4){y4}%
  \Cnode*(2,3){y3}%
  \Cnode*(2,2){y2}%
  \Cnode*(2,1){y1}%
  %-------------------------------------
  % node connections
  %-------------------------------------
  \ncline[linecolor=blue]{->}{x4}{y3}%
  \ncline[linecolor=blue]{->}{x3}{y4}%
  \ncline[linecolor=blue]{->}{x2}{y3}%
  \ncline[linecolor=blue]{->}{x1}{y2}%
  %-------------------------------------
  % node labels
  %-------------------------------------
  \uput [180]{0}(x4){$x_4$}%  
  \uput [180]{0}(x3){$x_3$}%  
  \uput [180]{0}(x2){$x_2$}%  
  \uput [180]{0}(x1){$x_1$}%  
  \uput [0]{0}(y4){$y_4$}%  
  \uput [0]{0}(y3){$y_3$}%  
  \uput [0]{0}(y2){$y_2$}%  
  \uput [0]{0}(y1){$y_1$}%  
  \rput[t](0,5){$\setX$}%
  \rput[t](2,5){$\setY$}%
\end{pspicture}
%=======================================
&% Diagram 2
%=======================================
\begin{pspicture}(-1,0.5)(3,5)%
  %-------------------------------------
  % Venn shapes
  %-------------------------------------
  \psellipse[linecolor=red,linewidth=1pt](0,2.5)(0.5,2)%
  \psellipse[linecolor=red,linewidth=1pt](2,2.5)(0.5,2)%
  \rput[t](0,5){$\setX$}%
  \rput[t](2,5){$\setY$}%
  %-------------------------------------
  % nodes
  %-------------------------------------
  \Cnode*(0,4){x4}% 
  \Cnode*(0,3){x3}%
  \Cnode*(0,2){x2}%
  \Cnode*(0,1){x1}%
  \Cnode*(2,4){y4}%
  \Cnode*(2,3){y3}%
  \Cnode*(2,2){y2}%
  \Cnode*(2,1){y1}%
  %-------------------------------------
  % node connections
  %-------------------------------------
  \ncline[linecolor=blue]{->}{x4}{y3}%
  \ncline[linecolor=blue]{->}{x3}{y4}%
  \ncline[linecolor=blue]{->}{x2}{y2}%
  \ncline[linecolor=blue]{->}{x1}{y1}%
  %-------------------------------------
  % node labels
  %-------------------------------------
  \uput [180]{0}(x4){$x_4$}\uput [0]{0}(y4){$y_4$}%
  \uput [180]{0}(x3){$x_3$}\uput [0]{0}(y3){$y_3$}%
  \uput [180]{0}(x2){$x_2$}\uput [0]{0}(y2){$y_2$}%
  \uput [180]{0}(x1){$x_1$}\uput [0]{0}(y1){$y_1$}%
\end{pspicture}
%=======================================
&% Diagram 3
%=======================================
\begin{pspicture}(-1,0.5)(3,5)%
  %-------------------------------------
  % Venn shapes
  %-------------------------------------
  \psellipse[linecolor=red,linewidth=1pt](0,2.5)(0.5,2)%
  \psellipse[linecolor=red,linewidth=1pt](2,2.5)(0.5,2)%
  \rput[t](0,5){$\setX$}%
  \rput[t](2,5){$\setY$}%
  %-------------------------------------
  % nodes
  %-------------------------------------
  \Cnode*(0,4){x4}% 
  \Cnode*(0,3){x3}%
  \Cnode*(0,2){x2}%
  \Cnode*(0,1){x1}%
  \Cnode*(2,4){y4}%
  \Cnode*(2,3){y3}%
  \Cnode*(2,2){y2}%
  \Cnode*(2,1){y1}%
  %-------------------------------------
  % node connections
  %-------------------------------------
  \ncline[linecolor=blue]{->}{x4}{y3}%
  \ncline[linecolor=blue]{->}{x3}{y4}%
  \ncline[linecolor=blue]{->}{x2}{y2}%
  \ncline[linecolor=blue]{->}{x2}{y1}%
  \ncline[linecolor=blue]{->}{x1}{y1}%
  %-------------------------------------
  % node labels
  %-------------------------------------
  \uput [180]{0}(x4){$x_4$}\uput [0]{0}(y4){$y_4$}%
  \uput [180]{0}(x3){$x_3$}\uput [0]{0}(y3){$y_3$}%
  \uput [180]{0}(x2){$x_2$}\uput [0]{0}(y2){$y_2$}%
  \uput [180]{0}(x1){$x_1$}\uput [0]{0}(y1){$y_1$}%
\end{pspicture}
%=======================================
&% Diagram 4
%=======================================
\begin{pspicture}(-1,0.5)(3,5)%
  %-------------------------------------
  % Venn shapes
  %-------------------------------------
  \psellipse[linecolor=red,linewidth=1pt](0,2.5)(0.5,2)%
  \psellipse[linecolor=red,linewidth=1pt](2,2.5)(0.5,2)%
  \rput[t](0,5){$\setX$}%
  \rput[t](2,5){$\setY$}%
  %-------------------------------------
  % nodes
  %-------------------------------------
  \Cnode*(0,4){x4}% 
  \Cnode*(0,3){x3}%
  \Cnode*(0,2){x2}%
  \Cnode*(0,1){x1}%
  \Cnode*(2,4){y4}%
  \Cnode*(2,3){y3}%
  \Cnode*(2,2){y2}%
  \Cnode*(2,1){y1}%
  %-------------------------------------
  % node connections
  %-------------------------------------
  \ncline[linecolor=blue]{->}{x4}{y3}%
  \ncline[linecolor=blue]{->}{x3}{y3}%
  \ncline[linecolor=blue]{->}{x2}{y2}%
  \ncline[linecolor=blue]{->}{x2}{y1}%
  \ncline[linecolor=blue]{->}{x1}{y1}%
  %-------------------------------------
  % node labels
  %-------------------------------------
  \uput [180]{0}(x4){$x_4$}\uput [0]{0}(y4){$y_4$}%
  \uput [180]{0}(x3){$x_3$}\uput [0]{0}(y3){$y_3$}%
  \uput [180]{0}(x2){$x_2$}\uput [0]{0}(y2){$y_2$}%
  \uput [180]{0}(x1){$x_1$}\uput [0]{0}(y1){$y_1$}%
\end{pspicture}
\\\mc{2}{c}{two \structe{relation}s in $\clRxy$ that \emph{are} \structe{function}s}
 &\mc{2}{c}{two \structe{relation}s in $\clRxy$ that are \emph{not} \structe{function}s}
%\\  a relation     & a relation     & a relation         & a relation
%\\  that \emph{is} & that \emph{is} & that is \emph{not} & that is \emph{not}
%\\  a function     & a function     & a function         & a function
\end{tabular*}}%%
          \caption{\label{fig:wavfound_relfnctd}%
            Four discrete relations---two \emph{are} functions and two are \emph{not}.
            }
        \end{figure}%
        \begin{figure}[t]
          \footnotesize%
          \psset{unit=12mm}
          %============================================================================
% LaTeX file
% Daniel J. Greenhoe
% X={1,2,3}  Y={1,2}  <= relation in 2^{XY}
%============================================================================
% \psset{unit=12mm}
  \begin{tabular}{c|c}
    \begin{pspicture}(-3,-1.5)(3,1.5)
      \pscircle[linecolor=blue](0,0){1}
      \psaxes[linecolor=axis]{<->}(0,0)(-2,-1.25)(2,1.25)
      \uput[0]{0}(2,0){$x$}
      \uput[0]{0}(0,1.25){$y$}
    \end{pspicture}
    &
    \begin{pspicture}(-2,-1.5)(2,1.5)
      \psaxes[linecolor=green]{<->}(0,0)(-2,-1.25)(2,1.25)
      \psarc [linecolor=blue](0,0){1}{0}{180}
      \psline[linecolor=blue](1,0)(1.5,0)
      \psline[linecolor=blue](-1,0)(-1.5,0)
      \uput[0]{0}(2,0){$x$}
      \uput[0]{0}(0,1.25){$y$}
    \end{pspicture}
    \\
    $\set{\opair{x}{y}\in\cprodXY}{x^2+y^2=1}$
    &
    $\set{\opair{x}{y}\in\cprodXY}{\begin{array}{ll}
       y=\sqrt{1-x^2} & \text{for } -1<x<1 \\
       y=0            & \text{otherwise}
       \end{array}}$
    \\
    (a relation that is \emph{not} a function) & (a relation that \emph{is} a function)
  \end{tabular}


          \caption{\label{fig:wavfound_relfnctc}%
            Two continuous relations---one is \emph{not} a function and one \emph{is}.
            }
        \end{figure}
        %\\
        Common examples of functions include $\ff(x)\eqd x^2$ and $\fg(x)\eqd\cos(x)$.
        The set of all functions with domain $\setX$ and range $\setY$ is denoted $\hxs{\clFxy}$.

  \item Every relation has an \hib{inverse}. The inverse of a relation 
        is simply the same set of ordered pairs as the relation, but with each pair of elements in each order pair swapped.
        That is each pair $\opair{x}{y}$ in the relation becomes $\opair{y}{x}$ in it's inverse.
        Every function is a relation, and therefore each function also has an inverse.
        That is, the inverse of a function always exists as a relation, but that inverse may not itself also be a function.

  %\item A \hib{functional} is a function $\ff:\setX\to\setY$ where
  %      $\setX$ is a \hie{linear space} over a field $\setY$.

  %\item A set $\set{\setP_n\in\psetx}{n\in\Z}$ is a \hib{partition} of the set $\sid$ if
  %      \citetbl{
  %        \citerp{munkres2000}{23}\\
  %        \citerp{rota1964}{498} \\
  %        \citerpg{halmos1950}{31}{0387900888}\\
  %        }
  %      \[\begin{array}{@{\qquad}l >{\ds}rcl@{\qquad}C@{\qquad}D@{\qquad}D}
  %        1. & \setP_i               &\ne & \szero & \forall i\in\setn{i=1,2,\ldots,n} & (non-empty) & and\\
  %        2. & \setP_i \seti \setP_j &=&     \szero & \forall i\ne j                   & (mutually exclusive) & and \\
  %        3. & \setopu_i \setP_i     &=&     \sid    &
  %      \end{array}\]
  %      For example, there are exactly five partitions on the set $\setn{x,y,z}$:
  %      \[\sssP{\setn{x,y,z}} =
  %      \setn{\begin{array}{lcl *{8}{l} l}
  %        \ssetP_{ 1} &=& \{ & &           &           &           &             &             &             & \setn{x,y,z} & \}\\
  %        \ssetP_{ 2} &=& \{ & & \setn{x}, &           &           &             &             & \setn{y,z}, &              & \}\\
  %        \ssetP_{ 3} &=& \{ & &           & \setn{y}, &           &             & \setn{x,z}, &             &              & \}\\
  %        \ssetP_{ 4} &=& \{ & &           &           & \setn{z}, & \setn{x,y}  &             &             &              & \}\\
  %        \ssetP_{ 5} &=& \{ & & \setn{x}, & \setn{y}, & \setn{z}  &             &             &             &              & \}
  %      \end{array}}\]
%\end{liste}
\end{itemize}
%=======================================
\section{Mathematical Spaces}
%=======================================
\begin{figure}[th]
  \begin{center}
  %============================================================================
% Daniel J. Greenhoe
% LaTeX file
%============================================================================
\begin{pspicture}(-7,0.5)(5,9)%
  %-------------------------------------
  % settings
  %-------------------------------------
   \psset{%
    arrowsize=4mm,
    arrowlength=0.6,
    arrowinset=0.1,
    %linecolor=blue,
    %linewidth=1pt,
     cornersize=relative,
     framearc=0.25,
    % gridcolor=graph,
    % subgriddiv=1,
    % gridlabels=4pt,
    % gridwidth=0.2pt,
    xunit=1\latunit,
    yunit=1.25\latunit,
     }%
  %-------------------------------------
  % nodes
  %-------------------------------------
   \begin{tabstr}{0.75}%
     \rput(-1, 8){\rnode{spaces}    {\psframebox{\begin{tabular}{c}abstract spaces\end{tabular}}}}%
     \rput(-4, 7){\rnode{lin}       {\psframebox{\begin{tabular}{c}linear spaces\ifnxref{vector}{def:vspace}\end{tabular}}}}%
     \rput( 2, 7){\rnode{top}       {\psframebox{\begin{tabular}{c}topological spaces\ifnxref{vstopo}{def:toplinspace}\end{tabular}}}}%
     \rput( 2, 6){\rnode{metric}    {\psframebox{\begin{tabular}{c}metric spaces\ifnxref{metric}{def:metric}\end{tabular}}}}%
     \rput( 2, 4.5){\rnode{commetric} {\psframebox{\begin{tabular}{c}complete metric spaces\ifnxref{seq}{def:ms_complete}\end{tabular}}}}%
     \rput(-4, 5){\rnode{metriclin} {\psframebox{\begin{tabular}{c}metric linear spaces\end{tabular}}}}%
     \rput(-4, 4){\rnode{normlin}   {\psframebox{\begin{tabular}{c}normed linear spaces\ifnxref{vsnorm}{def:norm}\end{tabular}}}}%
     \rput( 2, 3){\rnode{banach}    {\psframebox{\begin{tabular}{c}Banach spaces\ifnxref{seq}{def:banach}\end{tabular}}}}%
     \rput(-4, 3){\rnode{inprod}    {\psframebox{\begin{tabular}{c}inner-product spaces\ifnxref{vsinprod}{def:inprod}\end{tabular}}}}%
     \rput( 2, 2){\rnode{hilbert}   {\psframebox{\begin{tabular}{c}Hilbert spaces\ifnxref{seq}{def:hilbert}\end{tabular}}}}%
     \rput( 2, 1){\rnode{zero}      {\psframebox{\begin{tabular}{c}$\spZero$\ifnxref{subspace}{prop:subspace_0X}\end{tabular}}}}%
   \end{tabstr}%
  %-------------------------------------
  % connecting lines/arrows
  %-------------------------------------
   %\ncline[doubleline=true]{->}{lin}{spaces} double arrow seems to cause trouble for xdvipdfmx
   \ncline{lin}{spaces}
   \ncline{top}{spaces}%
   \ncline{com}{spaces}%
   \ncline{metriclin}{lin}%
   \ncline{normlin}{metriclin}%
   \ncline{inprod}{normlin}
   \ncline{banach}{normlin}%
   \ncline{hilbert}{inprod}%
   \ncline{hilbert}{banach}%
   \ncline{zero}{hilbert}%
   \ncline{metric}{top}%
   \ncline{metriclin}{metric}
   \ncline{commetric}{metric}%
   \ncline{banach}{commetric}%
  %-------------------------------------
  % labeling
  %-------------------------------------
   %\psccurve[linestyle=dashed,linecolor=red,fillstyle=none]%
   %  (0,-5)(20,4)(70,60)(20,60)(15,55)(-5,38)(-20,35)(-10,20)(-25,5)%
   %\psline[linecolor=red]{->}(60,75)(60,68)%
   %\uput[135](60,75){complete spaces}%
   %\psline[linecolor=red]{->}(26,80)(15,74)%
   %\psline[linecolor=red]{->}(30,80)(30,62.5)%
   %\uput[90](28,80){analytic spaces}%
   %
   %\psgrid[unit=10mm](-8,-1)(8,9)%
\end{pspicture}%

  \caption{Lattice of mathematical spaces\label{fig:wavfound_spaces}}
  \end{center}
\end{figure}%
The concept of the abstract \hie{space} was introduced by
Maurice Fr\'echet in his 1906 Ph.D. thesis.%
\citetbl{
  \citor{frechet1906} \\
  \citor{frechet1928} \\
  %\citerp{frechet1950}{147}
  }
An abstract \hib{space} in mathematics does not really have a rigorous definition.
But in general it is a set together with some other unifying structure.
The lattice of some common spaces, ordered by the set inclusion relation $\subseteq$, 
is illustrated in \prefpp{fig:wavfound_spaces} by a Hasse diagram.

\qboxnps
  {\href{http://en.wikipedia.org/wiki/Frechet}{Marice Ren\'e Fr\'echet}
   (\href{http://www-history.mcs.st-andrews.ac.uk/Timelines/TimelineF.html}{1878--1973}),
   \href{http://www-history.mcs.st-andrews.ac.uk/BirthplaceMaps/Places/France.html}{French mathematician}
   who in his 1906 Ph.D. dissertation introduced the concept of the \hie{metric space}
   \index{Fr\'echet, Marice Ren\'e}
   \index{quotes!Fr\'echet, Marice Ren\'e}
   \footnotemark
  }
  {../common/people/frechet_mthma.jpg}
  {\ldots
   A collection of these abstract elements will be called an \emph{abstract set}.
   If to this set there is added some rule of association of these elements,
   or some relation between them,
   the set will be called an \emph{abstract space}.}
  \citetblt{
    quote: & \citerp{frechet1950}{147} \\
            %\citerp{carothers2000}{36}
    image: & \url{http://www-history.mcs.st-and.ac.uk/PictDisplay/Frechet.html}
    }

%=======================================
\subsection{Linear Spaces}
%=======================================
\begin{liste}
  \item %Suppose $\otriple{\setS}{+}{\cdot}$ is a field, $\setX$ is a set, and
        %$\oplus$ and $\otimes$ are binary operators.
        The 4-tuple $\hxs{\oquad{\setX}{F}{\oplus}{\otimes}}$ is a \hib{linear space} over
        the field $F\eqd\hxs{\otriple{\setS}{+}{\cdot}}$ if (\prefp{def:vspace})
        %\citetbl{
        %  \citerp{haaser1991}{41} \\
        %  \citerpp{halmos1948}{1}{2} \\
        %  \citorc{peano1888}{Chapter IX}  \\
        %  \citorpp{peano1888e}{119}{120} \\
        %  \citorpp{banach1922}{134}{135}
        %  }
          %\[\begin{array}{l rcl @{\quad}C @{\quad}D@{}r@{}}
          \\\indentx$\ds\begin{array}{F rcl @{\quad}C @{\quad}D}
            %\\\cline{7-7}
            1.& \exists \vzero\in\setX \st \vx \oplus \vzero &=& \vx
              & \forall \vx\in\setX
              & ($\oplus$ identity)
              %& \ast\vline
              \\
            2.& \exists \vy\in\setX \st \vx\oplus\vy &=& \vzero
              & \forall \vx \in\setX
              & ($\oplus$ inverse)
              %& \vline
              \\
            3.& (\vx\oplus\vy)\oplus\vz &=& \vx\oplus(\vy\oplus\vz)
              & \forall \vx,\vy,\vz\in\setX
              & ($\oplus$ is associative)
            %  & \text{ }\vline
              \\
            4.& \vx\oplus\vy &=& \vy\oplus\vx
              & \forall \vx,\vy\in\setX
              & ($\oplus$ is commutative)
            %  & \vline
            %  \\\cline{7-7}
              \\
            5.& 1\otimes \vx &=& \vx
              & \forall \vx\in\setX
              & ($\otimes$ identity)
              \\
            6.& \alpha\otimes(\beta\otimes\vx) &=& (\alpha\cdot\beta)\otimes\vx
              & \forall \alpha,\beta\in\setS \text{ and } \vx\in\setX
              & ($\cdot$ associates with $\otimes$)
              \\
            7.& \alpha\otimes(\vx\oplus\vy) &=& (\alpha \otimes\vx)\oplus(\alpha\otimes\vy)
              & \forall \alpha\in\setS \text{ and } \vx,\vy\in\setX
              & ($\otimes$ distributes over $\oplus$)
              \\
            8.& (\alpha+\beta)\otimes\vx &=& (\alpha\otimes \vx)\oplus(\beta\otimes \vx)
              & \forall \alpha,\beta\in\setS \text{ and } \vx\in\setX
              & %($\otimes$ pseudo-distributes over $+$)
          \end{array}$
          \\
          The elements of $\setX$ are called \hib{vectors}.
          The elements of $\setS$ are called \hib{scalars}.
          A linear space is also called a \hib{vector space}.

  \item A function $\ff$ in $\hxs{\clFxy}$ is an \hib{operator} if
        the set $\setX$ and the set $\setY$ are \emph{both} linear spaces.\footnote{\structe{linear operator: \prefp{def:operator}}}
        One example is the \hib{integral operator} $\opP$
        with \hib{kernel} $\hxs{\kernel(t,s)}$ such that
        \[\brs{\opP\fx(t)}(s) \eqd \int_{t\in\R} \fx(t) \mcom{\kernelb{t,s}}{kernel} \dt \]

  \item An operator $\opL$ is a \hib{linear operator} if\footnote{\structe{linear operator: \prefp{def:linop}}}
    \\\indentx$\begin{array}{FrclCDD}
        1. & \opL(\vx + \vy)    &=&  \opL \vx + \opL \vy & \forall \vx,\vy\in\spX  & (\prope{additive}) & and\\
        2. & \opL(\alpha \vx)   &=&  \alpha\opL \vx    & \forall \vx\in\spX,\quad \forall\alpha\in\F & (\prope{homogeneous}).
     \end{array}$.\\
    Linear operators are of special interest in mathematics because they preserve the ``structure" of a vector (such as a function)
    when mapping it from one linear space to another.

%  \item An operator $\opP$ in $\clFxy$ is a \hib{projection operator} if it satisfies the \hie{idempotent} property
%        $\opP^2=\opP$ for all $x\in\setX$.
%        An example of a projection operator is the ``real part" operator $\Re$ in $\clF{\C}{\R}$, where $\C$ is the set
%        of complex numbers and $\R$ is the set of real numbers, such that
%        \[ \Re\mcom{(x + iy)}{element in $\C$} \eqd \mcom{x}{element in $\R$}, \qquad\forall x,y\in\R.\]
%        Note that $\Re\Re=\Re$ because $\Re\Re(x+iy)=\Re x = x = \Re(x+iy)$.

%  \item The structure $\seq{x_1,\,x_2,\,x_3,\,\cdots}{x\in\setX}$ is a \hib{sequence} on the set $\setX$ if
%          \[\seqn{x_1,\,x_2,\,x_3,\,\cdots} \eqd \opair{x_1}{\opair{x_2}{\opair{x_3}{\cdots}}}.\]
%        An example of a sequence is the sequence of integers
%        $\seqn{\cdots\,\-3,\,-2,\,-1,\,0,\,1,\,2,\,3,\,\cdots}$.

  \item A set $\spV\eqd\oquad{\setY}{F}{\oplus}{\otimes}$ is a \hib{linear subspace} of a
        linear space
        $\hxs{\oquad{\setX}{F}{\oplus}{\otimes}}$ if $\setY\subseteq\setX$ and if $\spV$ is itself a
        linear space.

  \item The quantity $\spV\hxs{\adds}\spW$ is the \hib{Minkowski sum} of linear subspaces $\spV$ and $\spW$ in
        a linear space $\spX$ if
          \\\indentx$\spV\adds\spW = \set{\vx+\vy}{\vx\in\spV\text{ and }\vy\in\spW}.$\\
        Here are some examples of set union, set intersection, and subspace addition
        (Minkowski addition) in $\R^3$:
        \\%============================================================================
% Daniel J. Greenhoe
% LaTeX file
%============================================================================


%Some examples of Minkowski addition in $\R^3$ are illustrated next:


  \psset{
    unit=0.5mm,
    linewidth=1pt,
    %cornersize=relative,
    %framearc=0.1,
    %linearc=0.1,
    linecolor=purple,
    fillstyle=solid,
    fillcolor=white,
    %fillstyle=hlines,
    hatchangle=45,
    hatchcolor=purple,
    hatchsep=2pt,
    }%
\begin{align*}
  \farraybox{\begin{pspicture}(-10,-10)(10,10)%
    %\psframe[linecolor=gray](-12,-12)(12,12)%
    \psline[linecolor=red]{<->}(-10,0)(10,0)% horizontal double vector
    %\psline[linecolor=blue] {<->}(-10,0)(10,0)% vertical   double vector
  \end{pspicture}}
  \spu
  \farraybox{\begin{pspicture}(-10,-10)(10,10)%
    %\psline[linecolor=red]{<->}(-10,0)(10,0)% horizontal double vector
    \psline[linecolor=blue] {<->}(0,-10)(0,10)% vertical   double vector
  \end{pspicture}}
  &=
  \farraybox{\begin{pspicture}(-10,-10)(10,10)%
    \psline[linecolor=red]{<->}(-10,0)(10,0)% horizontal double vector
    \psline[linecolor=blue] {<->}(0,-10)(0,10)% vertical   double vector
  \end{pspicture}}
  \\
  \farraybox{\begin{pspicture}(-10,-10)(10,10)%
    \psline[linecolor=red]{<->}(-10,0)(10,0)% horizontal double vector
  \end{pspicture}}
  \adds
  \farraybox{\begin{pspicture}(-10,-10)(10,10)%
    \psline[linecolor=blue] {<->}(0,-10)(0,10)% vertical   double vector
  \end{pspicture}}
  &=
  \farraybox{\begin{pspicture}(-10,-10)(10,10)%
    \psframe(-10,-10)(10,10)%
  \end{pspicture}}
  \\
  \farraybox{\begin{pspicture}(-10,-10)(10,10)%
    \rput{45}{\psline[linecolor=red]{<->}(-10,0)(10,0)}% horizontal double vector
  \end{pspicture}}
  \adds
  \farraybox{\begin{pspicture}(-10,-10)(10,10)%
    \psline[linecolor=blue] {<->}(-10,0)(10,0)% vertical   double vector
  \end{pspicture}}
  &=
  \farraybox{\begin{pspicture}(-10,-10)(24.14,10)%
    \rput(0,2.9289){\pspolygon[hatchcolor=red](-10,-10)(4.14,4.14)(24.14,4.14)(10,-10)}%
  \end{pspicture}}
  \\
  \farraybox{\begin{pspicture}(-10,-10)(10,10)%
    \rput{45}{\psline[linecolor=red]{<->}(-10,0)(10,0)}% horizontal double vector
  \end{pspicture}}
  \adds
  \farraybox{\begin{pspicture}(-10,-10)(10,10)%
    \psframe[linecolor=blue,hatchcolor=blue](-10,-10)(10,10)%
  \end{pspicture}}
  &=
  \farraybox{\begin{pspicture}(-10,-10)(15,15)%
    \pspolygon[hatchcolor=red](-10,10)(-5,15)(15,15)(10,10)% top
    \pspolygon[hatchcolor=red](10,-10)(10,10)(15,15)(15,-5)% side
    \psframe[hatchcolor=blue](-10,-10)(10,10)% front
  \end{pspicture}}
  \\
  \farraybox{\begin{pspicture}(-10,-10)(24.14,10)%
    \rput(0,2.9289){\pspolygon[hatchcolor=red](-10,-10)(4.14,4.14)(24.14,4.14)(10,-10)}%
  \end{pspicture}}
  \spi
  \farraybox{\begin{pspicture}(-10,-10)(10,10)%
    \psframe(-10,-10)(10,10)%
  \end{pspicture}}
  &=
  \farraybox{\begin{pspicture}(-10,-10)(10,10)%
    \psline[linecolor=blue] {<->}(-10,0)(10,0)% vertical   double vector
  \end{pspicture}}
  \\
  \farraybox{\begin{pspicture}(-10,-10)(10,10)%
    \rput{45}{\psline[linecolor=red]{<->}(-10,0)(10,0)}% horizontal double vector
  \end{pspicture}}
  \adds
  \brp{%
    \farraybox{\begin{pspicture}(-10,-10)(10,10)%
      \psline[linecolor=green] {<->}(-10,0)(10,0)% vertical   double vector
    \end{pspicture}}
    \spu
    \farraybox{\begin{pspicture}(-10,-10)(10,10)%
      \psline[linecolor=blue] {<->}(0,-10)(0,10)% vertical   double vector
    \end{pspicture}}%
    }%
  &=
  \farraybox{\begin{pspicture}(-10,-10)(15,15)%
    %\rput(0,2.9289){\pspolygon[hatchcolor=red](-10,-10)(4.14,4.14)(24.14,4.14)(10,-10)}%
    \pspolygon[linecolor=red] (0,0)(-10,  0)(-5, 5)(5,5)% left wing
    \pspolygon[linecolor=blue](0,0)(  0, 10)( 5,15)(5,5)% top fin
    \pspolygon[linecolor=blue](0,0)(  0,-10)( 5,-5)(5,5)% lower fin
    \pspolygon[linecolor=red] (0,0)( 10,  0)(15, 5)(5,5)% right wing
  \end{pspicture}}%
%  &=
%  \picboxab{\begin{pspicture}(-10,-10)(24.14,10)%
%    \rput(0,2.9289){\pspolygon[hatchcolor=red](-10,-10)(4.14,4.14)(24.14,4.14)(10,-10)}%
%  \end{pspicture}}
\end{align*}




%\setlength{\unitlength}{1\tw/(600*3)}%
%\begin{align*}
%  \picboxab{%
%    \color{green}%
%      \put(0,0){\vector( 1, 0){100} }%
%      \put(0,0){\vector(-1, 0){100} }%
%    }
%  \spu
%  \picboxab{%
%    \color{blue}%
%      \put(0,0){\vector( 0, 1){100} }%
%      \put(0,0){\vector( 0,-1){100} }%
%    }%
%  &=
%  \picboxab{%
%    \color{green}%
%      \put(0,0){\vector( 1, 0){100} }%
%      \put(0,0){\vector(-1, 0){100} }%
%    \color{blue}%
%      \put(0,0){\vector( 0, 1){100} }%
%      \put(0,0){\vector( 0,-1){100} }%
%    }%
%  \\
%  \picboxab{%
%    \color{green}%
%      \put(0,0){\vector( 1, 0){100} }%
%      \put(0,0){\vector(-1, 0){100} }%
%    }%
%  \adds
%  \picboxab{%
%    \color{blue}%
%      \put(0,0){\vector( 0, 1){100} }%
%      \put(0,0){\vector( 0,-1){100} }%
%    }%
%  &=
%  \picboxab{%
%    \color{green}%
%     \put(-100, 100){\line( 1, 0){200} }%
%     \put(-100,-100){\line( 1, 0){200} }%
%    \color{blue}%
%     \put(-100,-100){\line( 0, 1){200} }%
%     \put( 100,-100){\line( 0, 1){200} }%
%    }%
%  \\
%  \picboxab{%
%    \color{red}%
%      \put(0,0){\vector( 1, 1){100} }%
%      \put(0,0){\vector(-1,-1){100} }%
%    }%
%  \adds
%  \picboxab{%
%    \color{green}%
%      \put(0,0){\vector( 1, 0){100} }%
%      \put(0,0){\vector(-1, 0){100} }%
%    }%
%  &=
%  \picboxab{%
%    \color{red}%
%      \put(-150, -50){\line( 1, 1){100} }%
%      \put(  50, -50){\line( 1, 1){100} }%
%    \color{green}%
%      \put(-150, -50){\line( 1, 0){200} }%
%      \put( 150,  50){\line(-1, 0){200} }%
%    }%
%  %-------------------------------------
%  \\% x + yz = xyz
%  %-------------------------------------
%  \picboxab{%
%    \color{red}%
%      \put(0,0){\vector( 1, 1){100} }%
%      \put(0,0){\vector(-1,-1){100} }%
%    }%
%  \adds
%  \picboxab{%
%    \color{green}%
%     \put(-100, 100){\line( 1, 0){200} }%
%     \put(-100,-100){\line( 1, 0){200} }%
%    \color{blue}%
%     \put(-100,-100){\line( 0, 1){200} }%
%     \put( 100,-100){\line( 0, 1){200} }%
%    }%
%  &=
%  \picboxab{%
%    \thicklines
%    \color{red}%
%      \put(  50,  50){\line( 1, 1){50} }%
%      \put(  50,-100){\line( 1, 1){50} }%
%      \put(-100,  50){\line( 1, 1){50} }%
%    \color{green}%
%      \put(- 50, 100){\line( 1, 0){150} }%
%      \put(-100,  50){\line( 1, 0){150} }%
%      \put(-100,-100){\line( 1, 0){150} }%
%    \color{blue}%
%      \put( 100,- 50){\line( 0, 1){150} }%
%      \put(  50,-100){\line( 0, 1){150} }%
%      \put(-100,-100){\line( 0, 1){150} }%
%    }%
%  %-------------------------------------
%  \\% x+(y U z) = ...
%  %-------------------------------------
%  \picboxab{%
%    \color{red}%
%      \put(0,0){\vector( 1, 1){100} }%
%      \put(0,0){\vector(-1,-1){100} }%
%    }%
%  \adds
%  \brp{
%    \picboxab{%
%      \color{green}%
%        \put(0,0){\vector( 1, 0){100} }%
%        \put(0,0){\vector(-1, 0){100} }%
%      }%
%    \spu
%    \picboxab{%
%      \color{blue}%
%        \put(0,0){\vector( 0, 1){100} }%
%        \put(0,0){\vector( 0,-1){100} }%
%      }%
%  }
%  &=
%    \picboxab{%
%    \color{green}%
%      \put(-100,   0){\line( 1, 0){200} }%
%      \put( -50,  50){\line( 1, 0){ 50} }%
%      \put(  50,  50){\line( 1, 0){100} }%
%    \color{blue}%
%      \put(   0,-100){\line( 0, 1){200} }%
%      \put(  50,  50){\line( 0, 1){100} }%
%      \put(  50, -50){\line( 0, 1){ 50} }%
%    \color{red}%
%      \put(   0,   0){\line( 1, 1){ 50} }%
%      \put(-100,   0){\line( 1, 1){ 50} }%
%      \put( 100,   0){\line( 1, 1){ 50} }%
%      \put(   0, 100){\line( 1, 1){ 50} }%
%      \put(   0,-100){\line( 1, 1){ 50} }%
%    }%
%\end{align*}


  \item A linear subspace $\hxs{\spVc}$ is a \hib{complement} of a linear subspace $\spV$ in a linear space $\spX$ if
          $\spVc \adds \spV = \spX$.
        In general, $\spVc$ is not the only complement of $\spV$ in 
        $\spX$---there may be an infinite number of complements of $\spV$.
        To uniquely define a complement with respect to $\spV$, one or more additional constraints may be necessary.
        The most common additional constraint is \hie{orthogonality}, such that
        $\spVc$ and $\spV$ are orthogonal to each other.
        But the property orthogonality requires additional 
        structure---it requires an \hie{inner-product} (page \pageref{def:wavfound_inprod}).
        In such an inner-product space, we may denote the orthogonal complement
        of a subspace $\spV$ as $\hxs{\spVo}$.

{\psset{unit=1mm}
\begin{pspicture}(0,-10)(160,10)
  \psset{
    linecolor=blue,
    linewidth=1pt,
    fillstyle=hlines,
    hatchangle=45,
    hatchcolor=purple,
    }%
  \rput(10,0){%
    \psline{<->}(-10,0)(10,0)%
    \rput{90}(0,0){\psline[linecolor=red]{<->}(-10,0)(10,0)}%
    %\psline{<->}(0,-10)(0,10)%
    \uput{4}[45](0,0){$\adds$}
    \rput(15,0){$=$}%
    }%
  \rput(40,0){%
    \psline{<->}(-10,0)(10,0)%
    \rput{45}(0,0){\psline[linecolor=red]{<->}(-10,0)(10,0)}%
    %\psline{<->}(0,-10)(0,10)%
    \uput{4}[22.5](0,0){$\adds$}
    \rput(15,0){$=$}%
    }%
  \rput(70,0){%
    \psline{<->}(-10,0)(10,0)%
    \rput{135}(0,0){\psline[linecolor=red]{<->}(-10,0)(10,0)}%
    %\psline{<->}(0,-10)(0,10)%
    \uput{4}[67.5](0,0){$\adds$}
    \rput(15,0){$=$}%
    }%
  \rput(100,0){%
    \psframe[linewidth=0](-10,-10)(10,10)%
    \rput(0,0){$\R^2$}
    }%
\end{pspicture}}

\end{liste}

%=======================================
\subsection{Topological Spaces}
%=======================================
For the purpose of analysis, the most useful spaces are those with a \hie{topology}.
There are several spaces with topologies including
\hie{topological spaces}, \hie{metric spaces}, \hie{normed linear spaces},
and \hie{inner-product spaces}.
A topology allows the evaluation of ``nearness", which is important for
concepts such as \hie{convergence}.
All metric spaces are topological spaces (the metric generates a topology),
all normed linear spaces are metric spaces (the norm generates a metric),
and all inner-product spaces are normed linear spaces (the inner-product
generates a norm).

\begin{liste}
  \item A family of sets $\hxs{\ssT}$ is a \hib{topology} on a set $\setX$ if (\prefp{def:topology})
        \citetbl{
          %\citerppg{davis2005}{41}{42}{0071243399} \\
          \citerpg{munkres2000}{76}{0131816292} \\
          %\citerpg{ab}{57}{0120502577} \\
          %\citerpg{vel1993}{3}{0444815058}  \\
          \citor{riesz1909} \\
          \citor{hausdorff1914} \\
          \citorpg{hausdorff1937e}{258}{0828401195}
          }
      \\
      $\begin{array}{@{\qquad}F >{\ds}lc>{\ds}l @{\qquad}C @{\qquad}D@{\qquad}D}
          1. & \emptyset &\in& \ssT
             &
             & (\prop{union identity element})
             & and
        \\2. & \setX &\in& \ssT
             &
             & (\prop{intersection identity element})
             & and
        \\3. & \setA\seti \setB &\in& \ssT
             & \forall \setA,\setB\in\ssT
             & (\prop{closed for finite intersections})
             & and
        \\4. & \setopu_i \setA_i &\in& \ssT
             & \forall \seq{\setA_n\in\ssT}{n\in\Z}
             & (\prop{closed for arbitrary unions}).
      \end{array}$
      \\
      A \hib{topological space} is the pair $\hxs{\opair{\setX}{\ssT}}$.
      An \hib{open set} is any member of $\ssT$.
      A \hib{closed set} is any set $\setD$ such that $\cmpD$ is open.
      %\begin{figure}[t]
      %\begin{center}
      %%============================================================================
% Daniel J. Greenhoe
% LaTeX File
% lattice of topologies over the set {x,y,z}
%============================================================================
%{\psset{xunit=0.18mm,yunit=0.20mm}%
\begin{pspicture}(-450,-40)(450,540)%
  %---------------------------------
  % settings
  %---------------------------------
  \fns%
  %\psset{labelsep=1.5mm,radius=75\psunit}
  \psset{
    labelsep=8mm,
    radius=7.5mm,
    linearc=100\psxunit,
    }
  %---------------------------------
  % developement tools
  %---------------------------------
  %\psgrid[xunit=100\psxunit,yunit=100\psyunit](-5,-1)(5,5)%
  %---------------------------------
  % nodes
  %---------------------------------
  \Cnode[fillstyle=solid,fillcolor=latlatshade](   0,500){T77}%
  %
  \Cnode( 200,420){T66}%
  \Cnode( 350,400){T56}%
  \Cnode( 100,420){T65}%
  \Cnode(-100,420){T35}%
  \Cnode(-350,400){T53}%
  \Cnode(-200,420){T33}%
  %
  \Cnode( 300,300){T46}%
  \Cnode(   0,365){T25}%\Cnode( 200,300){T25}%
  \Cnode(-300,300){T13}%\Cnode( 100,300){T13}%
  \Cnode( 200,300){T64}%\Cnode(-100,300){T64}%
  \Cnode(   0,280){T52}%\Cnode(-200,300){T52}%
  \Cnode(-200,300){T31}%
  %
  \Cnode( 300,200){T44}%
  \Cnode( 200,200){T24}%
  \Cnode[fillstyle=solid,fillcolor=latlatshade]( 400,250){T14}%
  \Cnode( 100,200){T42}%
  \Cnode[fillstyle=solid,fillcolor=latlatshade](   0,200){T22}%
  \Cnode(-100,200){T12}%
  \Cnode[fillstyle=solid,fillcolor=latlatshade](-400,250){T41}%
  \Cnode(-200,200){T21}%
  \Cnode(-300,200){T11}%
  %
  \Cnode( 200,100){T40}%%\Cnode( 300,100){T40}%
  \Cnode(   0,100){T20}%%\Cnode( 200,100){T20}%
  \Cnode(-200,100){T10}%%\Cnode( 100,100){T10}%
  \Cnode( 350, 75){T04}%  %\Cnode(-100,100){T04}%
  \Cnode(-100, 75){T02}%  %\Cnode(-200,100){T02}%
  \Cnode(-350, 75){T01}%  %\Cnode(-300,100){T01}%
  \Cnode[fillstyle=solid,fillcolor=latlatshade](   0,0)  {T00}%
  %---------------------------------
  % node connections
  %---------------------------------
  \ncline{T77}{T33}%
  \ncline{T77}{T53}%
  \ncline{T77}{T35}%
  \ncline{T77}{T65}%
  \ncline{T77}{T56}%
  \ncline{T77}{T66}%
  %
  \ncline{T33}{T31}%
  \ncline{T33}{T22}%
  \ncline{T33}{T13}%
  \ncline{T53}{T41}%
  \ncline{T53}{T52}%
  \ncline{T53}{T13}%
  \ncline{T35}{T31}%
  \ncline{T35}{T14}%
  \ncline{T35}{T25}%
  \ncline{T65}{T64}%
  \ncline{T65}{T41}%
  \ncline{T65}{T25}%
  \ncline{T56}{T52}%
  \ncline{T56}{T14}%
  \ncline{T56}{T46}%
  \ncline{T66}{T64}%
  \ncline{T66}{T22}%
  \ncline{T66}{T46}%
  %
  \ncline{T31}{T11}%
  \ncline{T31}{T21}%
  \ncline{T52}{T12}%
  \ncline{T52}{T42}%
  \ncline{T64}{T24}%
  \ncline{T64}{T44}%
  \ncline{T13}{T11}%
  \ncline{T13}{T12}%
  \ncline{T25}{T21}%
  \ncline{T25}{T24}%
  \ncline{T46}{T42}%
  \ncline{T46}{T44}%
  %
  \ncline{T01}{T11}%
  \ncline{T01}{T21}%
  \ncline{T01}{T41}%
  \ncline{T02}{T12}%
  \ncline{T02}{T22}%
  \ncline{T02}{T42}%
  \ncline{T04}{T14}%
  \ncline{T04}{T24}%
  \ncline{T04}{T44}%
  \ncline{T10}{T11}%
  \ncline{T10}{T12}%
  \ncline{T10}{T14}%
  \ncline{T20}{T21}%
  \ncline{T20}{T22}%
  \ncline{T20}{T24}%
  \ncline{T40}{T41}%
  \ncline{T40}{T42}%
  \ncline{T40}{T44}%
  %
  \ncline{T00}{T01}%
  \ncline{T00}{T02}%
  \ncline{T00}{T04}%
  \ncline{T00}{T10}%
  \ncline{T00}{T20}%
  \ncline{T00}{T40}%
  %---------------------------------
  % node labels
  %---------------------------------
  \uput[  0](T77){$\topT_{77}$}%
  \uput[ 45](T66){$\topT_{66}$}%
  \uput[ 45](T56){$\topT_{56}$}%
  \uput[ 45](T65){$\topT_{65}$}%
  \uput[135](T35){$\topT_{35}$}%
  \uput[135](T53){$\topT_{53}$}%
  \uput[135](T33){$\topT_{33}$}%
  \uput[  0](T46){$\topT_{46}$}%
  \uput[ 90](T25){$\topT_{25}$}%
  \uput[180](T13){$\topT_{13}$}%
  \uput[180](T64){$\topT_{64}$}%
  \uput[180](T52){$\topT_{52}$}%
  \uput[  0](T31){$\topT_{31}$}%
  \uput[-45](T44){$\topT_{44}$}%
  \uput[-90](T24){$\topT_{24}$}%
  \uput[-60](T14){$\topT_{14}$}%
  \uput[ 90](T42){$\topT_{42}$}%
  \uput[ 45](T22){$\topT_{22}$}%
  \uput[ 90](T12){$\topT_{12}$}%
  \uput[240](T41){$\topT_{41}$}%
  \uput[-90](T21){$\topT_{21}$}%
  \uput[-80](T11){$\topT_{11}$}%
  \uput[  0](T40){$\topT_{40}$}%
  \uput[-45](T20){$\topT_{20}$}%
  \uput[-90](T10){$\topT_{10}$}%
  \uput[-90](T04){$\topT_{04}$}%
  \uput[225](T02){$\topT_{02}$}%
  \uput[-90](T01){$\topT_{01}$}%
  \uput[-20](T00){$\topT_{00}$}%
  %---------------------------------
  % discriptions
  %---------------------------------
  %\rput[bl](-450,0){%left N5 lattice
  %  \psframe[linestyle=dashed,linecolor=red](0,0)(200,450)%
  %  \uput[-45](200,0){$N5$ lattice}
  %  }%
  %\rput[br](450,0){%right N5 lattice
  %  \psframe[linestyle=dashed,linecolor=red](0,0)(-200,450)%
  %  \uput[-45](-200,0){$N5$ lattice}
  %  }%
% \ncbox[nodesep=20\psyunit,boxsize=100\psxunit,linestyle=dashed,linecolor=red]{T01}{T53}%
% \ncbox[nodesep=20\psyunit,boxsize=100\psxunit,linestyle=dashed,linecolor=red]{T04}{T56}%
%  \rput[t](-350,10){N5 lattice}
%  \rput[t]( 350,10){N5 lattice}
  %---------------------------------
  % node inner lattices
  %---------------------------------
  \psset{
    unit=0.04mm,
    radius=1mm,
    dotsep=0.5pt,
    linecolor=blue,
    }%
  \rput(T77){\begin{pspicture}(-100,0)(100,300)
                           \Cnode(0,300){t}
      \Cnode(-100,200){xy} \Cnode(0,200){xz} \Cnode(100,200){yz}
      \Cnode(-100,100){x}  \Cnode(0,100){y}  \Cnode(100,100){z}
                           \Cnode(0,  0){b}
      \psset{linestyle=dotted}%
      \ncline{t}{xy}\ncline{t}{xz}\ncline{t}{yz}
      \ncline{x}{xy}\ncline{x}{xz}
      \ncline{y}{xy}\ncline{y}{yz}
      \ncline{z}{xz}\ncline{z}{yz}
      \ncline{b}{x} \ncline{b}{y} \ncline{b}{z}
    \end{pspicture}}%
  \rput(T66){\begin{pspicture}(-100,0)(100,300)
                           \Cnode(0,300){t}%
      \pnode(-100,200){xy} \Cnode(0,200){xz} \Cnode(100,200){yz}%
      \pnode(-100,100){x}  \Cnode(0,100){y}  \Cnode(100,100){z}%
                           \Cnode(0,  0){b}%
      \psset{linestyle=dotted}%
      \ncline{t}{xy}\ncline{t}{xz}\ncline{t}{yz}
      \ncline{x}{xy}\ncline{x}{xz}
      \ncline{y}{xy}\ncline{y}{yz}
      \ncline{z}{xz}\ncline{z}{yz}
      \ncline{b}{x} \ncline{b}{y} \ncline{b}{z}
    \end{pspicture}}%
  \rput(T56){\begin{pspicture}(-100,0)(100,300)
                           \Cnode(0,300){t}%
      \Cnode(-100,200){xy} \pnode(0,200){xz} \Cnode(100,200){yz}%
      \pnode(-100,100){x}  \Cnode(0,100){y}  \Cnode(100,100){z}%
                           \Cnode(0,  0){b}%
      \psset{linestyle=dotted}%
      \ncline{t}{xy}\ncline{t}{xz}\ncline{t}{yz}
      \ncline{x}{xy}\ncline{x}{xz}
      \ncline{y}{xy}\ncline{y}{yz}
      \ncline{z}{xz}\ncline{z}{yz}
      \ncline{b}{x} \ncline{b}{y} \ncline{b}{z}
    \end{pspicture}}%
  \rput(T65){\begin{pspicture}(-100,0)(100,300)
                           \Cnode(0,300){t}%
      \pnode(-100,200){xy} \Cnode(0,200){xz} \Cnode(100,200){yz}%
      \Cnode(-100,100){x}  \pnode(0,100){y}  \Cnode(100,100){z}%
                           \Cnode(0,  0){b}%
      \psset{linestyle=dotted}%
      \ncline{t}{xy}\ncline{t}{xz}\ncline{t}{yz}
      \ncline{x}{xy}\ncline{x}{xz}
      \ncline{y}{xy}\ncline{y}{yz}
      \ncline{z}{xz}\ncline{z}{yz}
      \ncline{b}{x} \ncline{b}{y} \ncline{b}{z}
    \end{pspicture}}%
  \rput(T35){\begin{pspicture}(-100,0)(100,300)
                           \Cnode(0,300){t}%
      \Cnode(-100,200){xy} \Cnode(0,200){xz} \pnode(100,200){yz}%
      \Cnode(-100,100){x}  \pnode(0,100){y}  \Cnode(100,100){z}%
                           \Cnode(0,  0){b}%
      \psset{linestyle=dotted}%
      \ncline{t}{xy}\ncline{t}{xz}\ncline{t}{yz}
      \ncline{x}{xy}\ncline{x}{xz}
      \ncline{y}{xy}\ncline{y}{yz}
      \ncline{z}{xz}\ncline{z}{yz}
      \ncline{b}{x} \ncline{b}{y} \ncline{b}{z}
    \end{pspicture}}%
  \rput(T53){\begin{pspicture}(-100,0)(100,300)
                           \Cnode(0,300){t}%
      \Cnode(-100,200){xy} \pnode(0,200){xz} \Cnode(100,200){yz}%
      \Cnode(-100,100){x}  \Cnode(0,100){y}  \pnode(100,100){z}%
                           \Cnode(0,  0){b}%
      \psset{linestyle=dotted}%
      \ncline{t}{xy}\ncline{t}{xz}\ncline{t}{yz}
      \ncline{x}{xy}\ncline{x}{xz}
      \ncline{y}{xy}\ncline{y}{yz}
      \ncline{z}{xz}\ncline{z}{yz}
      \ncline{b}{x} \ncline{b}{y} \ncline{b}{z}
    \end{pspicture}}%
  \rput(T33){\begin{pspicture}(-100,0)(100,300)
                           \Cnode(0,300){t}%
      \Cnode(-100,200){xy} \Cnode(0,200){xz} \pnode(100,200){yz}%
      \Cnode(-100,100){x}  \Cnode(0,100){y}  \pnode(100,100){z}%
                           \Cnode(0,  0){b}%
      \psset{linestyle=dotted}%
      \ncline{t}{xy}\ncline{t}{xz}\ncline{t}{yz}
      \ncline{x}{xy}\ncline{x}{xz}
      \ncline{y}{xy}\ncline{y}{yz}
      \ncline{z}{xz}\ncline{z}{yz}
      \ncline{b}{x} \ncline{b}{y} \ncline{b}{z}
    \end{pspicture}}%
  \rput(T46){\begin{pspicture}(-100,0)(100,300)
                           \Cnode(0,300){t}%
      \pnode(-100,200){xy} \pnode(0,200){xz} \Cnode(100,200){yz}%
      \pnode(-100,100){x}  \Cnode(0,100){y}  \Cnode(100,100){z}%
                           \Cnode(0,  0){b}%
      \psset{linestyle=dotted}%
      \ncline{t}{xy}\ncline{t}{xz}\ncline{t}{yz}
      \ncline{x}{xy}\ncline{x}{xz}
      \ncline{y}{xy}\ncline{y}{yz}
      \ncline{z}{xz}\ncline{z}{yz}
      \ncline{b}{x} \ncline{b}{y} \ncline{b}{z}
    \end{pspicture}}%
  \rput(T25){\begin{pspicture}(-100,0)(100,300)
                           \Cnode(0,300){t}%
      \pnode(-100,200){xy} \Cnode(0,200){xz} \pnode(100,200){yz}%
      \Cnode(-100,100){x}  \pnode(0,100){y}  \Cnode(100,100){z}%
                           \Cnode(0,  0){b}%
      \psset{linestyle=dotted}%
      \ncline{t}{xy}\ncline{t}{xz}\ncline{t}{yz}
      \ncline{x}{xy}\ncline{x}{xz}
      \ncline{y}{xy}\ncline{y}{yz}
      \ncline{z}{xz}\ncline{z}{yz}
      \ncline{b}{x} \ncline{b}{y} \ncline{b}{z}
    \end{pspicture}}%
  \rput(T13){\begin{pspicture}(-100,0)(100,300)
                           \Cnode(0,300){t}%
      \Cnode(-100,200){xy} \pnode(0,200){xz} \pnode(100,200){yz}%
      \Cnode(-100,100){x}  \Cnode(0,100){y}  \pnode(100,100){z}%
                           \Cnode(0,  0){b}%
      \psset{linestyle=dotted}%
      \ncline{t}{xy}\ncline{t}{xz}\ncline{t}{yz}
      \ncline{x}{xy}\ncline{x}{xz}
      \ncline{y}{xy}\ncline{y}{yz}
      \ncline{z}{xz}\ncline{z}{yz}
      \ncline{b}{x} \ncline{b}{y} \ncline{b}{z}
    \end{pspicture}}%
  \rput(T64){\begin{pspicture}(-100,0)(100,300)
                           \Cnode(0,300){t}%
      \pnode(-100,200){xy} \Cnode(0,200){xz} \Cnode(100,200){yz}%
      \pnode(-100,100){x}  \pnode(0,100){y}  \Cnode(100,100){z}%
                           \Cnode(0,  0){b}%
      \psset{linestyle=dotted}%
      \ncline{t}{xy}\ncline{t}{xz}\ncline{t}{yz}
      \ncline{x}{xy}\ncline{x}{xz}
      \ncline{y}{xy}\ncline{y}{yz}
      \ncline{z}{xz}\ncline{z}{yz}
      \ncline{b}{x} \ncline{b}{y} \ncline{b}{z}
    \end{pspicture}}%
  \rput(T52){\begin{pspicture}(-100,0)(100,300)
                           \Cnode(0,300){t}%
      \Cnode(-100,200){xy} \pnode(0,200){xz} \Cnode(100,200){yz}%
      \pnode(-100,100){x}  \Cnode(0,100){y}  \pnode(100,100){z}%
                           \Cnode(0,  0){b}%
      \psset{linestyle=dotted}%
      \ncline{t}{xy}\ncline{t}{xz}\ncline{t}{yz}
      \ncline{x}{xy}\ncline{x}{xz}
      \ncline{y}{xy}\ncline{y}{yz}
      \ncline{z}{xz}\ncline{z}{yz}
      \ncline{b}{x} \ncline{b}{y} \ncline{b}{z}
    \end{pspicture}}%
  \rput(T31){\begin{pspicture}(-100,0)(100,300)
                           \Cnode(0,300){t}%
      \Cnode(-100,200){xy} \Cnode(0,200){xz} \pnode(100,200){yz}%
      \Cnode(-100,100){x}  \pnode(0,100){y}  \pnode(100,100){z}%
                           \Cnode(0,  0){b}%
      \psset{linestyle=dotted}%
      \ncline{t}{xy}\ncline{t}{xz}\ncline{t}{yz}
      \ncline{x}{xy}\ncline{x}{xz}
      \ncline{y}{xy}\ncline{y}{yz}
      \ncline{z}{xz}\ncline{z}{yz}
      \ncline{b}{x} \ncline{b}{y} \ncline{b}{z}
    \end{pspicture}}%
  \rput(T44){\begin{pspicture}(-100,0)(100,300)
                           \Cnode(0,300){t}%
      \pnode(-100,200){xy} \pnode(0,200){xz} \Cnode(100,200){yz}%
      \pnode(-100,100){x}  \pnode(0,100){y}  \Cnode(100,100){z}%
                           \Cnode(0,  0){b}%
      \psset{linestyle=dotted}%
      \ncline{t}{xy}\ncline{t}{xz}\ncline{t}{yz}
      \ncline{x}{xy}\ncline{x}{xz}
      \ncline{y}{xy}\ncline{y}{yz}
      \ncline{z}{xz}\ncline{z}{yz}
      \ncline{b}{x} \ncline{b}{y} \ncline{b}{z}
    \end{pspicture}}%
  \rput(T24){\begin{pspicture}(-100,0)(100,300)
                           \Cnode(0,300){t}%
      \pnode(-100,200){xy} \Cnode(0,200){xz} \pnode(100,200){yz}%
      \pnode(-100,100){x}  \pnode(0,100){y}  \Cnode(100,100){z}%
                           \Cnode(0,  0){b}%
      \psset{linestyle=dotted}%
      \ncline{t}{xy}\ncline{t}{xz}\ncline{t}{yz}
      \ncline{x}{xy}\ncline{x}{xz}
      \ncline{y}{xy}\ncline{y}{yz}
      \ncline{z}{xz}\ncline{z}{yz}
      \ncline{b}{x} \ncline{b}{y} \ncline{b}{z}
    \end{pspicture}}%
  \rput(T14){\begin{pspicture}(-100,0)(100,300)
                           \Cnode(0,300){t}%
      \Cnode(-100,200){xy} \pnode(0,200){xz} \pnode(100,200){yz}%
      \pnode(-100,100){x}  \pnode(0,100){y}  \Cnode(100,100){z}%
                           \Cnode(0,  0){b}%
      \psset{linestyle=dotted}%
      \ncline{t}{xy}\ncline{t}{xz}\ncline{t}{yz}
      \ncline{x}{xy}\ncline{x}{xz}
      \ncline{y}{xy}\ncline{y}{yz}
      \ncline{z}{xz}\ncline{z}{yz}
      \ncline{b}{x} \ncline{b}{y} \ncline{b}{z}
    \end{pspicture}}%
  \rput(T42){\begin{pspicture}(-100,0)(100,300)
                           \Cnode(0,300){t}%
      \pnode(-100,200){xy} \pnode(0,200){xz} \Cnode(100,200){yz}%
      \pnode(-100,100){x}  \Cnode(0,100){y}  \pnode(100,100){z}%
                           \Cnode(0,  0){b}%
      \psset{linestyle=dotted}%
      \ncline{t}{xy}\ncline{t}{xz}\ncline{t}{yz}
      \ncline{x}{xy}\ncline{x}{xz}
      \ncline{y}{xy}\ncline{y}{yz}
      \ncline{z}{xz}\ncline{z}{yz}
      \ncline{b}{x} \ncline{b}{y} \ncline{b}{z}
    \end{pspicture}}%
  \rput(T22){\begin{pspicture}(-100,0)(100,300)
                           \Cnode(0,300){t}%
      \pnode(-100,200){xy} \Cnode(0,200){xz} \pnode(100,200){yz}%
      \pnode(-100,100){x}  \Cnode(0,100){y}  \pnode(100,100){z}%
                           \Cnode(0,  0){b}%
      \psset{linestyle=dotted}%
      \ncline{t}{xy}\ncline{t}{xz}\ncline{t}{yz}
      \ncline{x}{xy}\ncline{x}{xz}
      \ncline{y}{xy}\ncline{y}{yz}
      \ncline{z}{xz}\ncline{z}{yz}
      \ncline{b}{x} \ncline{b}{y} \ncline{b}{z}
    \end{pspicture}}%
  \rput(T12){\begin{pspicture}(-100,0)(100,300)
                           \Cnode(0,300){t}%
      \Cnode(-100,200){xy} \pnode(0,200){xz} \pnode(100,200){yz}%
      \pnode(-100,100){x}  \Cnode(0,100){y}  \pnode(100,100){z}%
                           \Cnode(0,  0){b}%
      \psset{linestyle=dotted}%
      \ncline{t}{xy}\ncline{t}{xz}\ncline{t}{yz}
      \ncline{x}{xy}\ncline{x}{xz}
      \ncline{y}{xy}\ncline{y}{yz}
      \ncline{z}{xz}\ncline{z}{yz}
      \ncline{b}{x} \ncline{b}{y} \ncline{b}{z}
    \end{pspicture}}%
  \rput(T41){\begin{pspicture}(-100,0)(100,300)
                           \Cnode(0,300){t}%
      \pnode(-100,200){xy} \pnode(0,200){xz} \Cnode(100,200){yz}%
      \Cnode(-100,100){x}  \pnode(0,100){y}  \pnode(100,100){z}%
                           \Cnode(0,  0){b}%
      \psset{linestyle=dotted}%
      \ncline{t}{xy}\ncline{t}{xz}\ncline{t}{yz}
      \ncline{x}{xy}\ncline{x}{xz}
      \ncline{y}{xy}\ncline{y}{yz}
      \ncline{z}{xz}\ncline{z}{yz}
      \ncline{b}{x} \ncline{b}{y} \ncline{b}{z}
    \end{pspicture}}%
  \rput(T21){\begin{pspicture}(-100,0)(100,300)
                           \Cnode(0,300){t}%
      \pnode(-100,200){xy} \Cnode(0,200){xz} \pnode(100,200){yz}%
      \Cnode(-100,100){x}  \pnode(0,100){y}  \pnode(100,100){z}%
                           \Cnode(0,  0){b}%
      \psset{linestyle=dotted}%
      \ncline{t}{xy}\ncline{t}{xz}\ncline{t}{yz}
      \ncline{x}{xy}\ncline{x}{xz}
      \ncline{y}{xy}\ncline{y}{yz}
      \ncline{z}{xz}\ncline{z}{yz}
      \ncline{b}{x} \ncline{b}{y} \ncline{b}{z}
    \end{pspicture}}%
  \rput(T11){\begin{pspicture}(-100,0)(100,300)
                           \Cnode(0,300){t}%
      \Cnode(-100,200){xy} \pnode(0,200){xz} \pnode(100,200){yz}%
      \Cnode(-100,100){x}  \pnode(0,100){y}  \pnode(100,100){z}%
                           \Cnode(0,  0){b}%
      \psset{linestyle=dotted}%
      \ncline{t}{xy}\ncline{t}{xz}\ncline{t}{yz}
      \ncline{x}{xy}\ncline{x}{xz}
      \ncline{y}{xy}\ncline{y}{yz}
      \ncline{z}{xz}\ncline{z}{yz}
      \ncline{b}{x} \ncline{b}{y} \ncline{b}{z}
    \end{pspicture}}%
  \rput(T40){\begin{pspicture}(-100,0)(100,300)
                           \Cnode(0,300){t}%
      \pnode(-100,200){xy} \pnode(0,200){xz} \Cnode(100,200){yz}%
      \pnode(-100,100){x}  \pnode(0,100){y}  \pnode(100,100){z}%
                           \Cnode(0,  0){b}%
      \psset{linestyle=dotted}%
      \ncline{t}{xy}\ncline{t}{xz}\ncline{t}{yz}
      \ncline{x}{xy}\ncline{x}{xz}
      \ncline{y}{xy}\ncline{y}{yz}
      \ncline{z}{xz}\ncline{z}{yz}
      \ncline{b}{x} \ncline{b}{y} \ncline{b}{z}
    \end{pspicture}}%
  \rput(T20){\begin{pspicture}(-100,0)(100,300)
                           \Cnode(0,300){t}%
      \pnode(-100,200){xy} \Cnode(0,200){xz} \pnode(100,200){yz}%
      \pnode(-100,100){x}  \pnode(0,100){y}  \pnode(100,100){z}%
                           \Cnode(0,  0){b}%
      \psset{linestyle=dotted}%
      \ncline{t}{xy}\ncline{t}{xz}\ncline{t}{yz}
      \ncline{x}{xy}\ncline{x}{xz}
      \ncline{y}{xy}\ncline{y}{yz}
      \ncline{z}{xz}\ncline{z}{yz}
      \ncline{b}{x} \ncline{b}{y} \ncline{b}{z}
    \end{pspicture}}%
  \rput(T10){\begin{pspicture}(-100,0)(100,300)
                           \Cnode(0,300){t}%
      \Cnode(-100,200){xy} \pnode(0,200){xz} \pnode(100,200){yz}%
      \pnode(-100,100){x}  \pnode(0,100){y}  \pnode(100,100){z}%
                           \Cnode(0,  0){b}%
      \psset{linestyle=dotted}%
      \ncline{t}{xy}\ncline{t}{xz}\ncline{t}{yz}
      \ncline{x}{xy}\ncline{x}{xz}
      \ncline{y}{xy}\ncline{y}{yz}
      \ncline{z}{xz}\ncline{z}{yz}
      \ncline{b}{x} \ncline{b}{y} \ncline{b}{z}
    \end{pspicture}}%
  \rput(T04){\begin{pspicture}(-100,0)(100,300)
                           \Cnode(0,300){t}%
      \pnode(-100,200){xy} \pnode(0,200){xz} \pnode(100,200){yz}%
      \pnode(-100,100){x}  \pnode(0,100){y}  \Cnode(100,100){z}%
                           \Cnode(0,  0){b}%
      \psset{linestyle=dotted}%
      \ncline{t}{xy}\ncline{t}{xz}\ncline{t}{yz}
      \ncline{x}{xy}\ncline{x}{xz}
      \ncline{y}{xy}\ncline{y}{yz}
      \ncline{z}{xz}\ncline{z}{yz}
      \ncline{b}{x} \ncline{b}{y} \ncline{b}{z}
    \end{pspicture}}%
  \rput(T02){\begin{pspicture}(-100,0)(100,300)
                           \Cnode(0,300){t}%
      \pnode(-100,200){xy} \pnode(0,200){xz} \pnode(100,200){yz}%
      \pnode(-100,100){x}  \Cnode(0,100){y}  \pnode(100,100){z}%
                           \Cnode(0,  0){b}%
      \psset{linestyle=dotted}%
      \ncline{t}{xy}\ncline{t}{xz}\ncline{t}{yz}
      \ncline{x}{xy}\ncline{x}{xz}
      \ncline{y}{xy}\ncline{y}{yz}
      \ncline{z}{xz}\ncline{z}{yz}
      \ncline{b}{x} \ncline{b}{y} \ncline{b}{z}
    \end{pspicture}}%
  \rput(T01){\begin{pspicture}(-100,0)(100,300)
                           \Cnode(0,300){t}%
      \pnode(-100,200){xy} \pnode(0,200){xz} \pnode(100,200){yz}%
      \Cnode(-100,100){x}  \pnode(0,100){y}  \pnode(100,100){z}%
                           \Cnode(0,  0){b}%
      \psset{linestyle=dotted}%
      \ncline{t}{xy}\ncline{t}{xz}\ncline{t}{yz}
      \ncline{x}{xy}\ncline{x}{xz}
      \ncline{y}{xy}\ncline{y}{yz}
      \ncline{z}{xz}\ncline{z}{yz}
      \ncline{b}{x} \ncline{b}{y} \ncline{b}{z}
    \end{pspicture}}%
  \rput(T00){\begin{pspicture}(-100,0)(100,300)
                           \Cnode(0,300){t}%
      \pnode(-100,200){xy} \pnode(0,200){xz} \pnode(100,200){yz}%
      \pnode(-100,100){x}  \pnode(0,100){y}  \pnode(100,100){z}%
                           \Cnode(0,  0){b}%
      \psset{linestyle=dotted}%
      \ncline{t}{xy}\ncline{t}{xz}\ncline{t}{yz}
      \ncline{x}{xy}\ncline{x}{xz}
      \ncline{y}{xy}\ncline{y}{yz}
      \ncline{z}{xz}\ncline{z}{yz}
      \ncline{b}{x} \ncline{b}{y} \ncline{b}{z}
    \end{pspicture}}%
\end{pspicture}%
%}%%
      %%\footnotesize\setlength{\unitlength}{\textwidth/2000}\input{../common/lattop3.inp}
      %\end{center}
      %\caption{
      %  Lattice of topologies of $\sssT{\setn{x,y,z}}$
      %  \label{fig:wavfound_lat_top_xyz}
      %  }
      %\end{figure}
      %For example, there are a total of 29 topologies on the set $\setn{x,y,z}$.
      %With the exception of
      %$\ssT_{00}$ (the \hie{indiscrete topology}) and $\ssT_{77}$ (the \hie{discrete topology}),
      %each of those topologies has exactly two \hie{complements}.
      %%Listed below are the 29 topologies on $\setn{x,y,z}$ along with their respective complements.
      %These topologies and and their complements are listed in the following table:
      %%They are illustrated by a Hasse diagram in \prefpp{fig:wavfound_lat_top_xyz}.
      %
      %\arrayrulecolor{ex}
      %\begin{longtable}{|>{$}l<{=$}
      %                  @{$\{$}*{8}{>{$}l<{$}@{\;}}
      %                  @{$\}$\quad} |>{$}l<{$\quad}| >{\quad$}l<{$}|}
      %  \hline
      %  \rowcolor{ex}
      %  \mc{9}{|G|}{topologies on $\setn{x,y,z}$}
      %   & \mc{1}{G|}{1st complement}
      %   & \mc{1}{G|}{2nd compl.}
      %  \\\hline
      %  \ssT_{00} & \szero, &&&&&&& \sid
      %              & \ssT_{77}
      %              &
      %              \\
      %  \ssT_{01} & \szero,&\setn{x},&&&&&&\sid
      %              & \ssT_{56}
      %              & \ssT_{66}
      %              \\
      %  \ssT_{02} & \szero,&&\setn{y},&&&&&\sid
      %              & \ssT_{65}
      %              & \ssT_{35}
      %              \\
      %  \ssT_{04} & \szero,&&&\setn{z},&&&&\sid
      %              & \ssT_{53}
      %              & \ssT_{33}
      %              \\
      %  \ssT_{10} & \szero,&&&&\setn{x,y},&&&\sid
      %              & \ssT_{65}
      %              & \ssT_{66}
      %              \\
      %  \ssT_{20} & \szero,&&&&&\setn{x,z},&&\sid
      %              & \ssT_{53}
      %              & \ssT_{56}
      %              \\
      %  \ssT_{40} & \szero,&&&&&&\setn{y,z},&\sid
      %              & \ssT_{33}
      %              & \ssT_{35}
      %              \\
      %  \ssT_{11} & \szero,&\setn{x},&&&\setn{x,y},&&&\sid
      %              & \ssT_{64}
      %              & \ssT_{46}
      %              \\
      %  \ssT_{21} & \szero,&\setn{x},&&&&\setn{x,z},&&\sid
      %              & \ssT_{52}
      %              & \ssT_{46}
      %              \\
      %  \ssT_{41} & \szero,&\setn{x},&&&&&\,\setn{y,z},&\sid
      %              & \ssT_{22}
      %              & \ssT_{14}
      %              \\
      %  \ssT_{12} & \szero,&&\setn{y},&&\setn{x,y},&&&\sid
      %              & \ssT_{64}
      %              & \ssT_{25}
      %              \\
      %  \ssT_{22} & \szero,&&\setn{y},&&&\setn{x,z},&&\sid
      %              & \ssT_{41}
      %              & \ssT_{14}
      %              \\
      %  \ssT_{42} & \szero,&&\setn{y},&&&&\setn{y,z},&\sid
      %              & \ssT_{31}
      %              & \ssT_{25}
      %              \\
      %  \ssT_{14} & \szero,&&&\setn{z},&\setn{x,y},&&&\sid
      %              & \ssT_{41}
      %              & \ssT_{22}
      %              \\
      %  \ssT_{24} & \szero,&&&\setn{z},&&\setn{x,z},&&\sid
      %              & \ssT_{52}
      %              & \ssT_{13}
      %              \\
      %  \ssT_{44} & \szero,&&&\setn{z},&&&\setn{y,z},&\sid
      %              & \ssT_{31}
      %              & \ssT_{13}
      %              \\
      %  \ssT_{31} & \szero,&\setn{x},&&&\setn{x,y},&\setn{x,z},&&\sid
      %              & \ssT_{42}
      %              & \ssT_{44}
      %              \\
      %  \ssT_{52} & \szero,&&\setn{y},&&\setn{x,y},&\setn{x,z},&&\sid
      %              & \ssT_{21}
      %              & \ssT_{24}
      %              \\
      %  \ssT_{64} & \szero,&&&\setn{z},&&\setn{x,z},&\setn{y,z},&\sid
      %              & \ssT_{11}
      %              & \ssT_{12}
      %              \\
      %  \ssT_{13} & \szero,&\setn{x},&\setn{y},&&\setn{x,y},&&&\sid
      %              & \ssT_{24}
      %              & \ssT_{44}
      %              \\
      %  \ssT_{25} & \szero,&\setn{x},&&\setn{z},&&\setn{x,z},&&\sid
      %              & \ssT_{12}
      %              & \ssT_{42}
      %              \\
      %  \ssT_{46} & \szero,&&\setn{y},&\setn{z},&&&\setn{y,z},&\sid
      %              & \ssT_{11}
      %              & \ssT_{21}
      %              \\
      %  \ssT_{33} & \szero,&\setn{x},&\setn{y},&&\setn{x,y},&\setn{x,z},&&\sid
      %              & \ssT_{04}
      %              & \ssT_{40}
      %              \\
      %  \ssT_{53} & \szero,&\setn{x},&\setn{y},&&\setn{x,y},&&\setn{y,z},&\sid
      %              & \ssT_{04}
      %              & \ssT_{20}
      %              \\
      %  \ssT_{35} & \szero,&\setn{x},&&\setn{z},&\setn{x,y},&\setn{x,z},&&\sid
      %              & \ssT_{02}
      %              & \ssT_{40}
      %              \\
      %  \ssT_{65} & \szero,&\setn{x},&&\setn{z},&&\setn{x,z},&\setn{y,z},&\sid
      %              & \ssT_{02}
      %              & \ssT_{10}
      %              \\
      %  \ssT_{56} & \szero,&&\setn{y},&\setn{z},&\setn{x,y},&&\setn{y,z},&\sid
      %              & \ssT_{01}
      %              & \ssT_{20}
      %              \\
      %  \ssT_{66} & \szero,&&\setn{y},&\setn{z},&&\setn{x,z},&\setn{y,z},&\sid
      %              & \ssT_{01}
      %              & \ssT_{10}
      %              \\
      %  \ssT_{77} & \szero,&\setn{x},&\setn{y},&\setn{z},&\setn{x,y},&\setn{x,z},&\setn{y,z},&\sid
      %              & \ssT_{00}
      %              &
      %              \\
      %  \hline
      %\end{longtable}

      %   \begin{minipage}{\tw-40mm-10mm}
      %     Let each of the eight elements of the power set $\pset{\setn{x,y,z}}$
      %     be represented by the eight vertices of the diagram to the right.
      %     Let such a diagram with circles located at some of its vertices represent
      %     a subset of $\pset{\setn{x,y,z}}$ with elements corresponding to those circles.
      %     Then a Hasse diagram for the 29 topologies listed in the preceding table and under the
      %     order relation $\hxs{\subseteq}$ is illustrated in \prefpp{fig:wavfound_lat_top_xyz}.
      %   \end{minipage}%
      %   \hfill%
      %   \begin{minipage}[c]{40mm}%
      %     \begin{center}%
      %     %\psset{unit=1mm}\footnotesize%============================================================================
% Daniel J. Greenhoe
% LaTeX file
% vertices of lattice (2^{x,y,z}, subseteq)
%============================================================================
\begin{pspicture}(-20,-5)(20,35)
  \psset{%
    linecolor=latline,
    linewidth=1pt,
    gridcolor=graph,
    gridwidth=0.6pt,
    gridlabels=14pt,
    subgriddiv=5,
    }%
  \pnode(  0,30){xyz}\uput{1pt}[ 90](  0,30){$\setn{x,y,z}$}
  \pnode( 10,20){yz} \uput{1pt}[22.5]( 10,20){$\setn{y,z}$}
  \pnode(  0,20){xz} \uput{1pt}[90](  0,20){$\setn{x,z}$}
  \pnode(-10,20){xy} \uput{1pt}[157.5](-10,20){$\setn{x,y}$}
  \pnode( 10,10){z}  \uput{1pt}[-22.5]( 10,10){$\setn{z}$}
  \pnode(  0,10){y}  \uput{1pt}[-22.5](  0,10){$\setn{y}$}
  \pnode(-10,10){x}  \uput{1pt}[202.5](-10,10){$\setn{x}$}
  \pnode(  0, 0){e}  \uput{1pt}[-90](  0, 0){$\emptyset$}
  \ncline{xyz}{xy}\ncline{xyz}{xz}\ncline{xyz}{yz}%
  \ncline{x}{xy}\ncline{x}{xz} \ncline{y}{xy}\ncline{y}{yz} \ncline{z}{yz}\ncline{z}{xz}%
  \ncline{e}{x}\ncline{e}{y}\ncline{e}{z}%
  %\psgrid[unit=10\psunit](-2,-1)(2,4)
\end{pspicture}%
      %     %============================================================================
% Daniel J. Greenhoe
% LaTeX file
% lattice (2^{x,y,z}, subseteq)
%============================================================================
\begin{pspicture}(-2.7,-\latbot)(2.7,3.4)
  %---------------------------------
  % settings
  %---------------------------------
  %\psset{%
  %  labelsep=1.5mm,
  %  dotsep=0.5pt,
  %  }%
  %---------------------------------
  % nodes
  %---------------------------------
  \Cnode(0,3){t}%
  \Cnode(-1,2){xy} \Cnode(0,2){xz} \Cnode(1,2){yz}%
  \Cnode(-1,1){x}  \Cnode(0,1){y}  \Cnode(1,1){z}%
  \Cnode(0,0){b}%
  %---------------------------------
  % node connections
  %---------------------------------
  \psset{linestyle=dotted}%
  \ncline{t}{xy}\ncline{t}{xz}\ncline{t}{yz}%
  \ncline{x}{xy}\ncline{x}{xz}%
  \ncline{y}{xy}\ncline{y}{yz}%
  \ncline{z}{xz}\ncline{z}{yz}%
  \ncline{b}{x} \ncline{b}{y} \ncline{b}{z}%
  %---------------------------------
  % node labels
  %---------------------------------
  \uput[180](t) {$\setn{x,y,z}$}%
  \uput[180](xy){$\setn{x,y}$}%   
 %\uput{1pt}[ 70](xz){$\setn{x,z}$} 
  \uput[0](yz){$\setn{y,z}$}%
  \uput[180](x) {$\setn{x}$}%     
 %\uput{1pt}[-45](y) {$\setn{y}$}   
  \uput[0](z) {$\setn{z}$}%
  \uput[180](b) {$\szero$}%
  \uput[0](1,3){\rnode{xzlabel}{$\setn{x,z}$}}% 
  \uput[0](1,0.25){\rnode{ylabel}{$\setn{y}$}}%
  \ncline[linestyle=solid,nodesep=1pt,linecolor=red]{->}{xzlabel}{xz}%
  \ncline[linestyle=solid,nodesep=1pt,linecolor=red]{->}{ylabel}{y}%
\end{pspicture}%

      %     \end{center}%
      %   \end{minipage}
      %%They are illustrated by a Hasse diagram in \prefpp{fig:wavfound_lat_top_xyz}.

  \item The very general topological space is a landmark achievement in mathematics. \label{item:top_converge}
        However, such a space is a little too general for many of applications of interest in analysis. 
        In particular, in a general topological space, a sequence may converge to two or more distinct points (\prefp{ex:seq_xt31}).
        %For example, define a topology
        %  \\\indentx$\topT\eqd\setn{\emptyset,\, \setn{x}, \, \setn{x,y},\, \setn{x,z},\, \setn{x,y,z}}.$\\
        %Suppose we have a sequence $\seqn{x,x,x,\ldots}$.
        %Then this sequence of course converges to $x$.
        %But the sequence also converges to both $y$ and $z$ because $x$ is in every open set containing $y$ and 
        %also in every open set containing $z$.\citetbl{
        %  \citerpgc{munkres2000}{98}{0131816292}{``Hausdorff Spaces"}
        %  %def: & \prefp{def:top_converge}
        %  }
        Of course we can ``tighten up" the constraints on a topological space to make sure this doesn't happen (\prefp{ex:seq_xt56}).
        One such topological space is the \hib{Hausdorff Space} (\prefp{def:hausdorff}). 
        Another much more common one is the \hib{metric space} (next).

  \item A \hib{metric space} is simply a set together with a ``\hie{distance}" function,
        which is called the \hib{metric} of the \prop{metric space} (\prefp{def:metric}).
        With a metric on a set, we can measure the ``distance" between points in the set.
        If the points represent a sequence
        (for example, the sequence of values $1,\frac{1}{2},\frac{1}{3},\ldots\frac{1}{n}$)
        we can measure whether or not the sequence \hie{converges} to some value.
        %by measuring whether or not the distance between points gets closer and closer.

        Let $\setX$ be a set.
        A function $\hxs{\metricn}\in\clF{\setX^2}{\brp{\Rnn}}$ is a \hib{metric} on $\setX$ if
        \citetbl{
        %  \citerp{dieudonne1969}{28}\\
        %  \citerp{copson1968}{21} \\
        %  \citorp{hausdorff1937e}{109} \\
          \citor{frechet1928} \\
          \citorp{frechet1906}{30}
          %\citor{hausdorff1914}\\
          %\cithrp{ab}{34}
          }
        \\\indentx
        $\begin{array}{F rcl @{\qquad}C @{\qquad}D @{\qquad}D}
            1. & \metric{x}{y} &\ge& 0                                 & \forall x,y   \in\setX & (\prope{non-negative})   \index{property!non-negative}            & and 
          \\2. & \metric{x}{y} &=  & 0  \iff x=y                       & \forall x,y   \in\setX & (\prope{nondegenerate})  \index{property!nondegenerate}           & and 
          \\3. & \metric{x}{y} &=  & \metric{y}{x}                     & \forall x,y   \in\setX & (\prope{symmetric})      \index{property!symmetric}               & and 
          \\4. & \metric{x}{y} &\le& \metric{x}{z}+\metric{z}{y}       & \forall x,y,z \in\setX & (\prope{triangle inequality})\index{property!triangle inequality}.
        \end{array}$
        \\
        A \hib{metric space} is the pair $\hxs{\opair{\setX}{\metricn}}$.
        A metric is also called a \hib{distance function}.
        The \hie{unit ball} in $\opair{\setX}{\metricn}$ is the set of all points
        with distance less than or equal
        to $1$ from the point $0$.

        The balls of a metric space form a \hib{basis} for a topology.
        In particular, every ball, every intersection of balls, and every union of balls is an open set in the generated topology.\cittrpg{munkres2000}{78}{0131816292}

        There are several examples of metrics on the Euclidean plane. Here are some of them
        with illustrations of their unit balls.
        \setlength{\unitlength}{\tw/1200}
        \begin{enumerate}

%---------------------------------------
\item Taxi-cab metric:
      \citetbl{
        \citerp{deza2006}{240} \\
        \citerp{dieudonne1969}{29}
        }
      $\ds\metric{\vx}{\vy} \eqd \sum_{i=1}^n \abs{x_i-y_i}$
\label{item:metric_taxicab}
%---------------------------------------
  \begin{center}
  \begin{fsL}
  %\setlength{\unitlength}{\tw/300}
  \begin{picture}(300,300)(-130,-130)
    \thicklines
    \color{axis}%
      \put(-130,   0){\line(1,0){260} }%
      \put(   0,-130){\line(0,1){260} }%
      \put(-100, -10){\line(0,1){20} }%
      \put( 100, -10){\line(0,1){20} }%
      \put( -10,-100){\line(1,0){20} }%
      \put( -10, 100){\line(1,0){20} }%
      \put( -10, 110){\makebox(0,0)[br]{$1$} }%
      \put( -10,-110){\makebox(0,0)[tr]{$-1$} }%
      \put(-110,  10){\makebox(0,0)[br]{$-1$} }%
      \put( 110,  10){\makebox(0,0)[bl]{$1$} }%
      %\put( 140,   0){\makebox(0,0)[l]{$x$}}%
      %\put(   0, 140){\makebox(0,0)[b]{$y$}}%
    \color{blue}%
      \put(-100,   0){\line( 1, 1){100} }%
      \put(-100,   0){\line( 1,-1){100} }%
      \put( 100,   0){\line(-1, 1){100} }%
      \put( 100,   0){\line(-1,-1){100} }%
  \end{picture}
  \end{fsL}
  \end{center}
%\footnotetext{\cite[page 2]{norfolk}}

%---------------------------------------
\item Euclidean metric:
 \citetbl{
    %\citerp{norfolk}{2} \\
    \citerp{dieudonne1969}{29}
    }
 $\metric{\vx}{\vy} \eqd \sqrt{\sum_{i=1}^n |x_i-y_i|^2}$
\label{item:metric_euclidean}
%---------------------------------------
  \begin{center}
  \begin{fsL}
  %\setlength{\unitlength}{\tw/300}
  \begin{picture}(300,300)(-130,-130)
    \thicklines
    \color{axis}%
      \put(-130,   0){\line(1,0){260} }%
      \put(   0,-130){\line(0,1){260} }%
      \put(-100, -10){\line(0,1){20} }%
      \put( 100, -10){\line(0,1){20} }%
      \put( -10,-100){\line(1,0){20} }%
      \put( -10, 100){\line(1,0){20} }%
      \put( -10, 110){\makebox(0,0)[br]{$1$} }%
      \put( -10,-110){\makebox(0,0)[tr]{$-1$} }%
      \put(-110,  10){\makebox(0,0)[br]{$-1$} }%
      \put( 110,  10){\makebox(0,0)[bl]{$1$} }%
      %\put( 140,   0){\makebox(0,0)[l]{$x$}}%
      %\put(   0, 140){\makebox(0,0)[b]{$y$}}%
    \color{blue}%============================================================================
% NCTU - Hsinchu, Taiwan
% LaTeX File
% Daniel Greenhoe
%
% Unit circle with radius 100
%============================================================================

\qbezier( 100,   0)( 100, 41.421356)(+70.710678,+70.710678) % 0   -->1pi/4
\qbezier(   0, 100)( 41.421356, 100)(+70.710678,+70.710678) % pi/4-->2pi/4
\qbezier(   0, 100)(-41.421356, 100)(-70.710678,+70.710678) %2pi/4-->3pi/4
\qbezier(-100,   0)(-100, 41.421356)(-70.710678,+70.710678) %3pi/4--> pi 
\qbezier(-100,   0)(-100,-41.421356)(-70.710678,-70.710678) % pi  -->5pi/4
\qbezier(   0,-100)(-41.421356,-100)(-70.710678,-70.710678) %5pi/4-->6pi/4
\qbezier(   0,-100)( 41.421356,-100)( 70.710678,-70.710678) %6pi/4-->7pi/4
\qbezier( 100,   0)( 100,-41.421356)( 70.710678,-70.710678) %7pi/4-->2pi


%
  \end{picture}
  \end{fsL}
  \end{center}

%---------------------------------------
\item Sup metric: $\ds\metric{\vx}{\vy} \eqd \max\set{|x_i-y_i|}{i=1,2,\dots,n}$
\label{item:metric_sup}
%---------------------------------------
  \begin{center}
  \begin{fsL}
  %\setlength{\unitlength}{\tw/300}
  \begin{picture}(300,300)(-130,-130)
    \thicklines
    \color{axis}%
      \put(-130,   0){\line(1,0){260} }%
      \put(   0,-130){\line(0,1){260} }%
      \put(-100, -10){\line(0,1){20} }%
      \put( 100, -10){\line(0,1){20} }%
      \put( -10,-100){\line(1,0){20} }%
      \put( -10, 100){\line(1,0){20} }%
      \put( -10, 110){\makebox(0,0)[br]{$1$} }%
      \put( -10,-110){\makebox(0,0)[tr]{$-1$} }%
      \put(-110,  10){\makebox(0,0)[br]{$-1$} }%
      \put( 110,  10){\makebox(0,0)[bl]{$1$} }%
      %\put( 140,   0){\makebox(0,0)[l]{$x$}}%
      %\put(   0, 140){\makebox(0,0)[b]{$y$}}%
    \color{blue}%
      \put(-100,-100){\line( 1, 0){200} }%
      \put(-100,-100){\line( 0, 1){200} }%
      \put( 100, 100){\line(-1, 0){200} }%
      \put( 100, 100){\line( 0,-1){200} }%
  \end{picture}
  \end{fsL}
  \end{center}

%---------------------------------------
\item Parabolic metric:\label{item:metric_parabolic}
  \citetbl{
    \citerp{norfolk}{2} \\
    \url{http://groups.google.com/group/sci.math/msg/c0eb7e19631c31ea}
    }
 $\ds\metric{\vx}{\vy}\eqd \sum_{i=1}^n \sqrt{\abs{x_i-y_i}}$
%---------------------------------------
  \begin{center}
  \begin{fsL}
  %\setlength{\unitlength}{\tw/300}
  \begin{picture}(300,300)(-130,-130)%
    %{\color{graphpaper}\graphpaper[10](-150,-150)(300,300)}%
    \thicklines%
    \color{axis}%
      \put(-130,   0){\line(1,0){260} }%
      \put(   0,-130){\line(0,1){260} }%
      \put( 140,   0){\makebox(0,0)[l]{$x$}}%
      \put(   0, 140){\makebox(0,0)[b]{$y$}}%
      \put(-100, -10){\line(0,1){20} }%
      \put( 100, -10){\line(0,1){20} }%
      \put( -10,-100){\line(1,0){20} }%
      \put( -10, 100){\line(1,0){20} }%
      \put( -15, 100){\makebox(0,0)[r]{$+1$} }%
      \put( -15,-100){\makebox(0,0)[r]{$-1$} }%
      \put(-100, -15){\makebox(0,0)[t]{$-1$} }%
      \put( 100, -15){\makebox(0,0)[t]{$+1$} }%
    \color{blue}%
      \qbezier( 100,0)(0,0)(0, 100)%
      \qbezier( 100,0)(0,0)(0,-100)%
      \qbezier(-100,0)(0,0)(0,-100)%
      \qbezier(-100,0)(0,0)(0, 100)%
    \color{red}%
      \qbezier[20]( 75,0)(37.5,37.5)(0, 75)%
      \put( 10,90){\makebox(0,0)[bl]{$\scriptstyle \lambda\vv_\circ+(1-\lambda)\vw_\circ$} }%
      \put( 90,90){\vector(-1,-1){50} }%
    \color{black}%
      \put(-100,-100){\vector(1,1){125} }%
      \put(-120, -110){\makebox(0,0)[tl]{$\scriptstyle x^2+y^2-2xy-2x-2y+1=0$} }%
      \put(-120, -140){\makebox(0,0)[tl]{\scriptsize(parabolic equation)} }%
  \end{picture}
  \end{fsL}
  \end{center}

%%---------------------------------------
%\item Inverse tangent:
%  \cittrp{copson1968}{25}
%  $\ds\metric{\vx}{\vy}\eqd \sum_{i=1}^n \abs{\arctan x_i - \arctan y_i}$
%  where $\setX$ be a set and
%  $\vx\eqd\seq{x_i\in\setX}{i=1,2,\ldots,n}$ and
%  $\vy\eqd\seq{y_i\in\setX}{i=1,2,\ldots,n}$
%  are sequences on $\setX$.
%%---------------------------------------
%  \begin{center}
%  \begin{fsL}
%  %\setlength{\unitlength}{\tw/300}
%  \begin{picture}(300,300)(-130,-130)%
%    %{\color{graphpaper}\graphpaper[10](-150,-150)(300,300)}%
%    \thicklines%
%    \color{axis}%
%      \put(-130,   0){\line(1,0){260} }%
%      \put(   0,-130){\line(0,1){260} }%
%      \put( 140,   0){\makebox(0,0)[l]{$x$}}%
%      \put(   0, 140){\makebox(0,0)[b]{$y$}}%
%      \put(-100, -10){\line(0,1){20} }%
%      \put( 100, -10){\line(0,1){20} }%
%      \put( -10,-100){\line(1,0){20} }%
%      \put( -10, 100){\line(1,0){20} }%
%      \put( -15, 100){\makebox(0,0)[r]{$+1$} }%
%      \put( -15,-100){\makebox(0,0)[r]{$-1$} }%
%      \put(-100, -15){\makebox(0,0)[t]{$-1$} }%
%      \put( 100, -15){\makebox(0,0)[t]{$+1$} }%
%    \color{blue}%
%      \qbezier( 100,0)(0,0)(0, 100)%
%      \qbezier( 100,0)(0,0)(0,-100)%
%      \qbezier(-100,0)(0,0)(0,-100)%
%      \qbezier(-100,0)(0,0)(0, 100)%
%    \color{red}%
%      \qbezier[20]( 75,0)(37.5,37.5)(0, 75)%
%      \put( 10,90){\makebox(0,0)[bl]{$\scriptstyle \lambda\vv_\circ+(1-\lambda)\vw_\circ$} }%
%      \put( 90,90){\vector(-1,-1){50} }%
%    %\color{black}%
%      %\put(-100,-100){\vector(1,1){125} }%
%      %\put(-120, -110){\makebox(0,0)[tl]{$\scriptstyle x^2+y^2-2xy-2x-2y+1=0$} }%
%      %\put(-120, -140){\makebox(0,0)[tl]{\scriptsize(parabolic equation)} }%
%  \end{picture}
%  \end{fsL}
%  \end{center}
%
%%---------------------------------------
%\item Exponential:
%      $\ds\metric{\vx}{\vy}\eqd 2\sum_{i=1}^n  \abs{\brp{\frac{3}{2}}^{x_i} - \brp{\frac{3}{2}}^{y_i}}$
%      where $\setX$ is a set,
%      $\vx\eqd\seq{x_i\in\setX}{i=1,2,\ldots,n}$ and
%      $\vy\eqd\seq{y_i\in\setX}{i=1,2,\ldots,n}$ are sequences on $\setX$.
%\label{item:metric_exp}
%%---------------------------------------
%  \begin{center}
%  \begin{fsL}
%  \setlength{\unitlength}{\tw/1600}
%  \begin{picture}(400,400)(-200,-200)%
%    %{\color{graphpaper}\graphpaper[10](-150,-150)(300,300)}%
%    \thicklines%
%    \color{axis}%
%      \put(-200,   0){\line(1,0){400} }%
%      \put(   0,-200){\line(0,1){400} }%
%      %\put( 140,   0){\makebox(0,0)[l]{$x$}}%
%      %\put(   0, 140){\makebox(0,0)[b]{$y$}}%
%      \put( 100, -10){\line(0,1){20} }%
%      \put(-100, -10){\line(0,1){20} }%
%      \put( -10,-100){\line(1,0){20} }%
%      \put(-171, -10){\line(0,1){20} }%
%      \put( -10,-171){\line(1,0){20} }%
%      \put( -10, 100){\line(1,0){20} }%
%      \put( -15, 100){\makebox(0,0)[r]{$+1$} }%
%      \put( -15,-100){\makebox(0,0)[r]{$-1$} }%
%      \put(-181, -15){\makebox(0,0)[tr]{$\frac{-\ln2}{\ln3-\ln2}$} }%
%      \put( -15,-181){\makebox(0,0)[tr]{$\frac{-\ln2}{\ln3-\ln2}$} }%
%      \put(-100, -15){\makebox(0,0)[t]{$-1$} }%
%      \put( 100, -15){\makebox(0,0)[t]{$+1$} }%
%    \color{blue}%
%      \qbezier( 100,0)(100,100)(0, 100)%
%      \qbezier(-171,0)(-50,50)(0, 100)%
%      \qbezier(-171,0)(-50,-50)(0,-171)%
%      \qbezier( 100,0)(50,-50)(0,-171)%
%    \color{red}%
%      \put(180,200){\makebox(0,0)[rt]{$\frac{1}{\ln3-\ln2}\,\ln\brs{\frac{5}{2}-\brp{\frac{3}{2}}^x}$} }%
%      \put(140, 140){\vector(-1,-1){60} }%
%  \end{picture}
%  \end{fsL}
%  \end{center}
%
%%---------------------------------------
%\item Tangential:
%      $\ds\metric{\vx}{\vy}\eqd \sum_{i=1}^n  \abs{\tan\brp{\frac{\pi}{2}x_i} - \tan\brp{\frac{\pi}{2} y_i}}$
%      where $\setX=\set{x\in\R}{x\in(-1,1)}$ is a set,
%      $\vx\eqd\seq{x_i\in\setX}{i=1,2,\ldots,n}$ and
%      $\vy\eqd\seq{y_i\in\setX}{i=1,2,\ldots,n}$
%      are sequences on $\setX$.
%\label{item:metric_tan}
%%---------------------------------------
%  \begin{center}
%  \begin{fsL}
%  %\setlength{\unitlength}{\tw/300}
%  \begin{picture}(300,300)(-150,-150)%
%    %{\color{graphpaper}\graphpaper[10](-150,-150)(300,300)}%
%    \thicklines%
%    \color{axis}%
%      \put(-130,   0){\line(1,0){260} }%
%      \put(   0,-130){\line(0,1){260} }%
%      \put( 140,   0){\makebox(0,0)[l]{$x$}}%
%      \put(   0, 140){\makebox(0,0)[b]{$y$}}%
%      \put( 100, -10){\line(0,1){20} }%
%      \put(-100, -10){\line(0,1){20} }%
%      \put( -10,-100){\line(1,0){20} }%
%      \put( -10, 100){\line(1,0){20} }%
%      \put( -15, 100){\makebox(0,0)[r]{$\frac{+1}{2}$} }%
%      \put( -15,-100){\makebox(0,0)[r]{$\frac{-1}{2}$} }%
%      \put(-100, -15){\makebox(0,0)[t]{$\frac{-1}{2}$} }%
%      \put( 100, -15){\makebox(0,0)[t]{$\frac{+1}{2}$} }%
%    \color{blue}%
%      \qbezier( 100,0)(70,70)(0, 100)%
%      \qbezier(-100,0)(-70,70)(0, 100)%
%      \qbezier(-100,0)(-70,-70)(0,-100)%
%      \qbezier( 100,0)(70,-70)(0,-100)%
%    \color{red}%
%      \put(-60,50){\makebox(0,0)[rb]{$y=\frac{2}{\pi}\atan\brp{\pm1\mp\abs{\tan\brp{\frac{\pi}{2}x}}}$} }%
%      %\put(140, 140){\vector(-1,-1){74} }%
%  \end{picture}
%  \end{fsL}
%  \end{center}


%
%%---------------------------------------
%\begin{example}
%\citep{giles1987}{34}
%\index{metrics!Fr\'echet}
%%---------------------------------------
%The \hib{Fr\'echet metric} is defined as
%\exbox{\begin{array}{>{\ds}l}
%  \fd(\vx,\vy) \eqd \sum_{n=1}^\infty \frac{1}{2^n} \: \frac{\abs{y_n-x_n}}{1+\abs{y_n-x_n}}
%  \\
%  \text{where}\qquad
%  \vx \eqd \seq{x_n}{n\in\Zp}
%  \qquad\text{and}\qquad
%  \vy \eqd \seq{y_n}{n\in\Zp}
%\end{array}}
%The ball generated by this metric is {\bf not convex}.
%\end{example}

\end{enumerate}



  \item In a linear space, the norm of a vector can be described as the ``\hie{length}"
        or the ``\hie{magnitude}" of the vector.
        %A norm is similar to the concept of a \hie{measure}.
        %But a measure operates on a set, whereas a norm operates on a vector.

        Let $\spV=\oquad{\setX}{F}{\oplus}{\cdot}$ be a linear space and
        $F$ be a field with absolute value function $\abs{\cdot}:F\to F$.
        \ifdochas{algebra}{\footnote{
          \hie{absolute value}: \prefp{def:abs}
          }}
          A function $\normn$ in $\clF{\setX}{\R}$ is a \hib{norm} if
          \\$\begin{array}{@{\qquad}F rcl @{\qquad}C @{\qquad}D @{\qquad}D}
            1. & \norm{ \vx}      &\ge& 0                     & \forall \vx \in X            & (\prop{strictly positive})  \nocite{michel1993} & and  %page 115
          \\2. & \norm{ \vx}      &=  & 0 \iff \vx=\vzero     & \forall \vx \in X            & (\prop{nondegenerate})                                    & and
          \\3. & \norm{a\vx}      &=  & |a|\norm{\vx}         & \forall \vx \in X, \; a\in\C & (\prop{homogeneous})                                      & and
          \\4. & \norm{\vx+\vy}   &\le& \norm{\vx}+\norm{\vy} & \forall \vx,\vy \in X        & (\prop{subadditive}/\prop{triangle inquality}).
          \end{array}$
          \\A \hib{normed linear space} is the pair $\hxs{\opair{\spV}{\normn}}$.
        \citetbl{
          %\citerpp{ab}{217}{218} \\
          \citorp{banach1932}{53}  \\
          \citorp{banach1932e}{33} \\
          \citorp{banach1922}{135}
         %\citerp{michel1993}{344} \\
         %\citerp{horn}{259}  \\
          }
        Every normed linear space is also a metric space (and hence also a topological space) 
        because $\metric{\vx}{\vy}\eqd\norm{\vx-\vy}$ is a metric.
        Constraint to a normed space as opposed to the more general metric space has some
        advantages:\citetbl{
          \citerpgc{giles2000}{2}{0521653754}{1.2 Remarks}\\
          \citerppgc{giles1987}{22}{26}{0521359287}{2.4 Theorem, 2.11 Theorem}
          %\citerp{rudinp}{31} \\
          %David C. Ullrich (2007), \url{http://groups.google.com/group/sci.math/msg/4a217391a5607f83}
          }
        \\\indentx$\ds
        \brb{\begin{array}{>{\ds}l}
          \exists \normn\in\clFxr \st \\
          \mcom{\metric{\vx}{\vy}=\norm{\vy-\vx}}
               {$\metricn$ is generated by a norm}
        \end{array}}
        \implies
        \brb{\begin{array}{>{\scy}r>{\ds}lD}
        %1. & \metric{\vx+\vz}{\vy+vz} = \metric{\vx}{\vy} & (\prope{translation invariant}) \\ % included in <norms generated by metrics> section
        1. & \ball{\vx}{r} = \vx + B(0,r) \\
        2. & \ball{\vzero}{r} = r\,\ball{\vzero}{1} \\
        3. & \ball{\vx}{r} \text{ is \prope{convex}} \\
        4. & \vx\in \ball{\vzero}{r} \iff -\vx\in \ball{\vzero}{r} & (\prope{symmetric}) 
        \end{array}}$


  The following are all norms in the linear space $\C^n$:
  \\\indentx$\begin{array}{rc>{\ds}l@{\qquad}>{(}D<{)}}
    \norm{\vx}_1      &\eqd& \sum_{i=1}^n \abs{x_i}             & \prop{$l_1$-norm} or \prop{taxi cab norm} \\
    \norm{\vx}_2      &\eqd& \sqrt{\sum_{i=1}^n \abs{x_i}^2}    & \prop{$l_2$-norm} or \prop{Euclidean norm}  \\
    \norm{\vx}_\infty &\eqd& \max\set{\abs{x_i}}{i=1,2,\dots,n} & \prop{$l_\infty$-norm} or \prop{sup norm}
  \end{array}$
  \cittrp{giles2000}{3}

  %\item In a normed linear space $\opair{\setX}{\normn}$
  %      with basis $\Phi\eqd\seq{\fphi_n}{n\in\Z}$,
  %      $\Phi$ is a \hib{Riesz basis} if for some finite values $a$ and $b$
  %      \[  \]
  \item Let $\spV\eqd\oquad{\setX}{\C}{+}{\cdot}$ be a linear space.
  A function $\inprodn:X\times X\to\C$ is an \hib{inner product} on $\spV$ if
  \label{def:wavfound_inprod}
  \\
  $\ds\begin{array}{@{\qquad}F rcl @{\qquad}C @{\qquad}D @{\qquad}D}
   1. & \inprod{\vx    }{\vx} &\ge& 0
      & \forall \vx\in X
      & (\prop{non-negative})
      & and
      \\
   2. & \inprod{\alpha\vx}{\vy}    &=& \alpha\inprod{\vx}{{\vy}}
      & \forall \vx,\vy\in X,\;\forall\alpha\in\C
      & (\prop{homogeneous})
      & and
      \\
   3. & \inprod{\vx+\vy}{\vu} &=& \inprod{\vx}{{\vu}} + \inprod{\vy}{{\vu}}
      & \forall \vx,\vy,\vu\in X
      & (\prop{additive})
      & and
      \\
   4. & \inprod{\vx    }{\vy} &=& \inprod{\vy}{\vx}^\ast
      & \forall \vx,\vy\in X
      & (\prop{conjugate symmetric}).
      & and
      \\
   5. & \inprod{\vx    }{\vx} &=& 0 \iff \vx=\vzero
      & \forall \vx\in X
      & (\prop{nondegenerate})
  \end{array}$
  \\
  An inner product is also called a \hib{scalar product}.
  An \hib{inner product space} is the ordered pair $\hxs{\opair{\spV}{\inprodn}}$.~\citetbl{%
  \citerpg{istratescu1987}{111}{9027721823}\\
  \citerp{haaser1991}{277} \\
  \citerp{ab}{276} \\
  \citorp{peano1888e}{72}
  }
\\
An example of an inner product space involving real functions is $\opair{\clFrr}{\inprodn}$ where
  \[ \inprod{\fx(t)}{\fy(t)} \eqd \int_{t\in\R} \fx(t)\fy(t) \dt \]
An example of an inner product space involving \hie{linear algebra}
(\hie{matrix algebra}) is $\opair{\clF{\brp{\Z^n}}{\brp{\R^n}}}{\inprodn}$ where
  \[
    \inprod
      {\brs{\begin{array}{c} x_1 \\ x_2 \\ \vdots \\ x_n \end{array}}}
      {\brs{\begin{array}{c} y_1 \\ y_2 \\ \vdots \\ y_n \end{array}}}
    \eqd
    \brs{\begin{array}{c} x_1 \\ x_2 \\ \vdots \\ x_n \end{array}}^T
    \brs{\begin{array}{c} y_1 \\ y_2 \\ \vdots \\ y_n \end{array}}
    \eqd
    x_1y_1 + x_2y_2 +\cdots+ x_ny_n
  \]

\end{liste}

%=======================================
\subsection{Complete Metric Spaces}
%=======================================
\begin{liste}
  \item A topological space is sufficient for describing the concept of \prope{continuity},
        but it is inadequate for the property of concept of \prope{convergence}\footnote{\prope{convergence}: \prefp{item:top_converge}}.
        Convergence is normally evaluated in metric space.
        Convergent sequences, and the more relaxed Cauchy sequences, are as follows:
  \\
  \begin{tabular}{@{\qquad}cp{\tw-20mm}}
    \circOne & A sequence $\seqn{x_n}$ in a metric space $\opair{\setX}{\metricn}$ is \hib{convergent} (\prefp{def:converge}) if for some value $x$ and
               for any $\varepsilon$ there exists an $N$ such that 
               \\&\indentx$\ds\metric{x_n}{x}<\varepsilon \quad\forall n>N.$\\

    \circTwo & A sequence $\seqn{x_n}$ in a metric space $\opair{\setX}{\metricn}$ is \hib{Cauchy} (\prefp{def:cauchy}) if
               for every $\varepsilon$ there exists an $N$ such that 
               \\&\indentx$\ds\metric{x_n}{x_m}<\varepsilon \quad\forall n,m>N$
  \end{tabular}
  %\begin{tabular}{@{\quad}llp{5\tw/8}}
  %  \circOne & \hie{convergent sequence}: & sequence $\seqn{x_n}$ converges to a fixed value $x$ \ifdochas{sequence}{(\prefp{def:ms_converge})}\\
  %  \circTwo & \hie{Cauchy sequence}:     & sequence $\seqn{x_n}$ converges to other elements in the sequence \ifdochas{sequence}{(\prefp{def:ms_cauchy})}
  %\end{tabular}
  \\
  %\item
  The \hie{convergent} condition is ``stronger" than the \hie{Cauchy} condition
  in the sense that all convergent sequences are Cauchy
  but not all Cauchy sequences are convergent\ifdochas{seq}{ (\prefp{thm:convergent==>cauchy})}.

  \item
  A metric space is \hib{complete} if
  each Cauchy sequence in the metric space is also convergent;
  that is, each Cauchy sequence converges to a specific point in the metric space.
  %(\pref{def:ms_complete}).
  A normed linear space $\hxs{\opair{\spX}{\normn}}$ is a \hib{Banach space} if
  it is \hie{complete}.
  An inner-product space $\hxs{\opair{\spX}{\inprodn}}$ is a \hib{Hilbert space} if
  it is \hie{complete} (\prefp{def:hilbert}).
  Completeness is particularly important in basis theory where we want linear combinations of basis vectors
  to converge to some element in the space.
\end{liste}


%%=======================================
%\section{Convexity}
%%=======================================
%\index{standard convexity}
%\index{convexity!standard}
%%--------------------------------------
%A set $\setD$ is \hib{convex} in the linear space $\oquad{\setX}{F}{+}{\cdot}$ if
%\citetbl{
%  \citerppg{vel1993}{5}{6}{0444815058}\\
%  \citerpg{bollobas1999}{2}{0521655773}
%  }
%  \\\indentx$\ds
%    \brb{\vx,\vy\in\setD \text{ and } \lambda\in(0,1) }
%    \quad\implies\quad
%    \lambda \vx + (1-\lambda)\vy \in\setD$
%  \\
%A set that is \emph{not} convex is \hib{concave}.
%
%Take for example the following two regions in $\R^2$. 
%The one on the left is convex, the one on the right is not.
%\\
%\begin{minipage}{\tw/3}
%  \begin{center}
%  \color{figcolor}
%  \begin{fsL}
%  \setlength{\unitlength}{\tw/500}
%  \begin{picture}(300,300)(-130,-130)
%    %\graphpaper[10](0,0)(600,600)
%    \thicklines
%    \put(-130,   0){\line(1,0){260} }
%    \put(   0,-130){\line(0,1){260} }
%    \qbezier(-100,0)(0, 160)(100,0)
%    \qbezier(-100,0)(0,-160)(100,0)
%    {\color{red}
%      \put(0,0){\vector(1, 1){55}}
%      \put(0,0){\vector(1,-1){55}}
%      \put(  60, 60){\makebox(0,0)[bl]{$\vx$} }
%      \put(  60,-60){\makebox(0,0)[tl]{$\vy$} }
%      \put(55,-55){\line(0,1){110}}
%      \put(-100,100){\makebox(0,0)[bl]{$\lambda \vx + (1-\lambda)\vy$} }
%      \put(-50,90){\vector(3,-2){105}}
%    }
%  \end{picture}
%  \end{fsL}
%  \end{center}
%\end{minipage}%
%%
%\begin{minipage}{\tw/3}
%  \begin{example}
%    \[
%      \Leftarrow
%      \left\{
%        \parbox[c][][c]{\tw-6ex}
%        {The set of points enclosed by the figure to the left is a \hie{convex} set in $\R^2$.}
%      \right.
%    \]
%    \[
%      \left.
%        \parbox[c][][c]{\tw-6ex}
%        {The set of points enclosed by the figure to the right is a \hie{concave} set in $\R^2$.}
%      \right\}
%      \Rightarrow
%    \]
%  \end{example}
%\end{minipage}%
%%
%\begin{minipage}{\tw/3}
%  \begin{center}
%  \color{figcolor}
%  \begin{fsL}
%  \setlength{\unitlength}{\tw/500}
%  \begin{picture}(300,300)(-130,-130)
%    %\graphpaper[10](-130,-130)(300,300)
%    \thicklines
%    \put(-130,   0){\line(1,0){260} }
%    \put(   0,-130){\line(0,1){260} }
%    \qbezier(-100,0)(0, 100)(100, 100)
%    \qbezier(-100,0)(0,-100)(100,-100)
%    \qbezier(100,100)(-50,0)(100,-100)
%    {\color{red}
%      %\put(0,0){\vector(1, 1){55}}
%      %\put(0,0){\vector(1,-1){55}}
%      \put(  75, 60){\makebox(0,0)[tl]{$\vx$} }
%      \put(  75,-60){\makebox(0,0)[bl]{$\vy$} }
%      \put(60,-68){\line(0,1){136}}
%      \put(-100,100){\makebox(0,0)[bl]{$\lambda \vx + (1-\lambda)\vy$} }
%      \put(-50,90){\vector(3,-2){105}}
%    }
%  \end{picture}
%  \end{fsL}
%  \end{center}
%\end{minipage}
%
%Further examples are provided by metric spaces.
%Some balls in metric spaces are convex, and some are not.%
%\citetblt{
%  \citerp{norfolk}{2} \\
%  \url{http://groups.google.com/group/sci.math/msg/c0eb7e19631c31ea}
%  }
%The balls in the metric spaces with these metrics \textbf{are} convex:
%\\\indentx
%  \begin{tabular}{lll}
%    \hie{Taxi-cab metric}:   & \pref{item:metric_taxicab}   & page \pageref{item:metric_taxicab}\\
%    \hie{Euclidean metric}:  & \pref{item:metric_euclidean} & page \pageref{item:metric_euclidean}\\
%    \hie{Sup metric}:        & \pref{item:metric_sup}       & page \pageref{item:metric_sup}\\
%    \hie{Tangential metric}: & \pref{item:metric_tan}       & page \pageref{item:metric_tan}
%  \end{tabular}
%
%And the balls in the metric spaces with these metrics are \textbf{not} convex:
%\\\indentx
%  \begin{tabular}{lll}
%    \hie{Parabolic metric}:  & \pref{item:metric_parabolic} & page \pageref{item:metric_parabolic}\\
%    \hie{Exponential metric}:& \pref{item:metric_exp}       & page \pageref{item:metric_exp}
%  \end{tabular}
%
%%Examples of convex balls in metric spaces include those generated by the following metrics:\\
%%\begin{tabular}{>{$\imark$}llll}
%%  & Taxi-cab metric   \ifdochas{metric}{& \pref{ex:ms_taxi}       & \prefpo{ex:ms_taxi}} \\
%%  & Euclidean metric  \ifdochas{metric}{& \pref{ex:ms_euclidean} & \prefpo{ex:ms_euclidean}} \\
%%  & Sup metric        \ifdochas{metric}{& \pref{ex:ms_sup}       & \prefpo{ex:ms_sup}} \\
%%  & Tangential metric \ifdochas{metric}{& \pref{ex:ms_tan}       & \prefpo{ex:ms_tan}}
%%\end{tabular}
%%
%%Examples of metrics generating spaces in which balls are \emph{not} convex include the following:\\
%%\begin{tabular}{>{$\imark$}llll}
%%  & Parabolic metric    \ifdochas{metric}{& \pref{ex:ms_parabolic} & \prefpo{ex:ms_parabolic}} \\
%%  & Exponential metric  \ifdochas{metric}{& \pref{ex:ms_32x}       & \prefpo{ex:ms_32x}} \\
%%\end{tabular}
%
%%%--------------------------------------
%%\begin{definition}
%%\label{def:convex_f}
%%\citetbl{
%%  \citerp{bollobas1999}{3} \\
%%  \citerp{jensen1906}{176}
%%  }
%%\index{convex!functional}
%%\index{convex!strictly}
%%\index{strictly convex}
%%%--------------------------------------
%%Let $\oquad{\setX}{F}{+}{\cdot}$ be a linear space and $D$ a convex set in $X$.
%%\defbox{\begin{array}{rcl@{\qquad}l}
%%  \mc{4}{l}{\text{A functional $\ff\in\clF{\setD}{F}$ is \hib{convex} if }}\\
%%  \ff\big(\lambda \vx + [1-\lambda]\vy\big) 
%%    &\le& \lambda \ff(\vx) + (1-\lambda)\:\ff(\vy)  
%%    &     \forall \vx,\vy\in D \text{ and } \forall\lambda\in(0,1)
%%  \\ 
%%  \\
%%  \mc{4}{l}{\text{A functional $\fg\in\clF{\setD}{F}$ is \hib{strictly convex} if }}\\
%%  \fg\big(\lambda \vx + [1-\lambda]\vy\big) 
%%    &=&   \lambda \fg(\vx) + (1-\lambda)\:\fg(\vy)  
%%    &     \forall \vx,\vy\in D,\;\vx\ne \vy,\; \text{and } \forall\lambda\in(0,1)
%%  \end{array}}
%%\end{definition}
%
%One of the more spectacular results of convexity theory is \hib{Jensen's Inequality},
%which says this:
%\citetbl{
%  \citerpg{bollobas1999}{3}{0521655773} \\
%  \citerpg{lay1982}{7}{0471095842}\\
%  \citorpp{jensen1906}{179}{180}
%  }
%\index{inequalities!Jensen's}
%Let $\oquad{\setX}{F}{+}{\cdot}$ be a linear space,
%$\setD$ a convex subset of $\setX$, and 
%$\ff\in\clF{\setD}{F}$ a functional. Then
%\[
%  \brb{
%    \begin{array}{lD}
%      \text{$\ff$ is convex} & and\\
%      \sum_{n=1}^N \lambda_n=1
%    \end{array}
%    }
%  \qquad\implies\qquad
%  \ff\brp{\sum_{n=1}^N \lambda_n \: \vx_n} 
%  \le \sum_{n=1}^N \lambda_n\:\ff\brp{\vx_n}
%  \qquad
%  \forall \vx_n \in \setD,\;N\in\Zp
%  \]



%=======================================
\section{Bases}
%=======================================
\begin{liste}
  \item A set of vectors $\setxn{\vphi_n\in\setX}$ in a linear space $\spO\eqd\linearspaceX$
        is \hib{linearly independent} in $\spO$ if\footnote{\prope{linear independence}: \prefp{def:linin}}
        \\\indentx$\ds\sum_{n=1}^N \alpha_n \vphi_n = 0 \qquad\implies\qquad \alpha_1=\alpha_2=\cdots=\alpha_N=0$.
        \\If the vectors in $\setn{\vphi_n}$ are not linearly independent, then they are \hib{linearly dependent}.
  %\\A sequence $\seqxZ{\vphi_n\in\setX}$ is \prope{linearly independent} if
  %\\\qquad every finite subset $\setxn{\vphi_{\ff(n)}}$ is \prope{linearly independent}.

  \item The \hib{span} $\hxs{\Span}\setxn{\vx_n}$
        of a set of vectors $\setxn{\vx_n}$
        is the linear subspace generated by all the linear combinations of those vectors.\footnote{\structe{span}: \prefp{def:span}}

  \item The set of vectors $\setxn{\fphi_n}$ in a linear space $\spX$ is a
        \hib{Hamel basis} for $\spX$ if (\prefp{def:basis_hamel})
        \\$\begin{array}{@{\qquad}FMD}
           1. & $\setxn{\vphi_n}$ \prope{spans} $\spO$           & and\\
           2. & $\setxn{\vphi_n}$ is \prope{linearly independent}.
        \end{array}$\\
        The Hamel basis is typically only used for \emph{finite} linear spaces.
        In an infinite linear space, we end up having to evaluate the infinite sum $\ds\sum_{n=1}^\infty \alpha_n \fphi_n$.
        This sum is defined in terms of a limit of a sequence of partial sums: 
        \\\indentx$\ds\sum_{n=1}^\infty \alpha_n \fphi_n \eqd \lim_{n\to\infty}\sum_{k=1}^n \alpha_k \fphi_k$.
        \\The limit $\lim_{n\to\infty}$ is defined in terms of a topology. 
        We have already seen that a general topological space is too general for the job (\prefp{item:top_converge}).
        A metric linear space provides sufficient structure. 
        But very often the topological structure that is actually used is the Banach space, as is the case for the 
        \structe{Schauder basis} (next).

  \item The sequence of vectors $\seqxZ{\fphi_n}$ is a
        \hib{Schauder basis}
        for the Banach space $\opair{\spX}{\normn}$ if
    for every $x\in\spX$ there exists an \prope{unique} sequence $\seq{\alpha_n}{n\in\Z}$
    such that\footnote{\structe{Schauder basis}: \prefp{def:basis_schauder}}
      \\\indentx $\ds\lim_{n\to\infty}\norm{x-\sum_{k=1}^n \alpha_k\fphi_k}=0$.

  \item A Schauder basis $\seq{\fphi_n}{n\in\Z}$ for a Hilbert space $\opair{\spX}{\inprodn}$ is \hib{orthogonal} if\footnote{\prope{orthogonality}: \prefp{def:orthogonal_basis}}
    \\\indentx$
    \begin{array}{lccMll}
      \inprod{\fphi_n}{\fphi_m} &=&    0 & for & n\neq m & and\\
      \inprod{\fphi_n}{\fphi_m} &\neq& 0 & for & n =   m.
    \end{array}$
    \\
       A Schauder basis $\seq{\fphi_n}{n\in\Z}$ for a Hilbert space $\opair{\spX}{\inprodn}$ is \hib{orthonormal} if\footnote{\structe{orthonormal basis}: \prefp{def:orthonormal_basis}}
    \\\indentx$\inprod{\fphi_n}{\fphi_m} =
       \brbl{\begin{array}{cMlD}
         0 & for & n\neq m & and\\
         1 & for & n=m
       \end{array}}$

  \item Let $\seq{\fpsi_n}{n\in\Z}$ be an orthonormal basis for a Hilbert space $\opair{\spX}{\inprodn}$.
        The sequence of vectors $\seq{\fphi_n}{n\in\Z}$ is a \hib{Riesz basis}
        for $\opair{\spX}{\inprodn}$
        if there exists a linear bounded invertible operator $\opR$
        such that $\seq{\fphi_n}{n\in\Z}=\opR\seq{\fpsi_n}{n\in\Z}$.
        \citetbl{
          \citerpgc{christensen2003}{63}{0817642951}{Definition 3.6.1}\\
          \citerpg{heil2011}{196}{9780817646868}
          %\citerpg{loiseau2009}{89}{3642028969}
          %\citerpg{bsu1996}{233}{3764353449}\\
          %\citerpg{nikolskij1992}{134}{3540505849}\\
          }

\end{liste}

%=======================================
\section{Analyses}
%=======================================
  A sequence $\seq{\spV_n}{n\in\Z}$ of linear subspaces of a linear space $\spX$
        is an \hib{analysis} of $\spX$.
        %if  $\seq{\spV_n}{n\in\Z}$ is a partition of $\spX$.
        The partial or complete reconstruction of $\spX$ from $\seq{\spV_n}{n\in\Z}$ is a \hib{synthesis}.%
        \footnote{%
          The word \hie{analysis} comes from the Greek word
          {\fntagreek{>av'alusis}},
          meaning ``dissolution" (\citerpc{perschbacher1990}{23}{entry 359}),
          which in turn means
          ``the resolution or separation into component parts"
          (\citer{collins2009}, \scs\url{http://dictionary.reference.com/browse/dissolution})
          }
  An analysis is usually completely \hie{characterized} by a \hie{transform}.
  For example, a Fourier analysis is a sequence of subspaces with sinusoidal bases.
  Examples of subspaces in a Fourier analysis include $\spV_1=\Span\setn{e^{it}}$, 
  $\spV_{2.3}=\Span\setn{e^{i2.3t}}$, $\spV_{\sqrt{2}}=\Span\setn{e^{i\sqrt{2}t}}$, etc.
  A \hib{transform} is loosely defined as a function that maps a family of functions
  into an analysis.
  A very useful transform (a ``\hie{Fourier transform}") for Fourier Analysis is
  \[\brs{\opFT \fx}(\omega) \eqd \frac{1}{\sqrt{2\pi}} \int_{t\in\R} \fx(t) e^{-i\omega t} \dt\]


%  \item A sequence $\opT$ in $\clFxy$ is a \hib{transform} \label{item:wavstrct_T}
%        if each element in the sequence is a projection operator in $\clFxy$.
%        An example of a transform is the \hib{cosine transform} $\opT$ in $\clFrr$ such that
%        \begin{align*}
%          \opT\fx(t) &\eqd \seq{\opP_n}{n\in\Z}
%             \\&\eqd \seq{\int_{t\in\R} \fx(t)\,\mcom{\cos(nt)}{kernel} \dt}{n\in\Z}
%             \\&\eqd \seqn{\cdots,\,
%                           %\int_{t\in\R} \cos\brs{(-2)t}\,\fx(t) \dt,\,
%                           \int_{t\in\R} \fx(t)\,\cos\brs{(-1)t} \dt,\,
%                           \int_{t\in\R} \fx(t)\,                \dt,\,
%                           \int_{t\in\R} \fx(t)\,\cos\brs{( 1)t} \dt,\,
%                           %\int_{t\in\R} \cos\brs{(-2)t}\,\fx(t) \dt,\,
%                           \cdots
%                          }
%        \end{align*}
%        Further examples of transforms include the \hie{Fourier Transform} and various \hie{Wavelet Transforms}.
% it is a \hib{sequence} of projection operators on $\A function $\opT$ in $\clFxy$ is a \hib{transform} if with domain $\clFxy$ and range $\clF{\setA}{\setB}$ if

    \begin{minipage}{\tw-65mm}%
      An analysis can also be partially characterized by its order structure with respect
      to an order relation such as the set inclusion relation $\subseteq$.
      Most transforms have a very simple M-$n$ order structure,
      as illustrated to the right.
      The M-$n$ lattices for $n\ge3$ are \prope{modular} but not \prope{distributive}.
      Analyses typically have one subspace that is a \hie{scaling} subspace;
      and this subspace is often simply a family of constants
      (as is the case with \hi{Fourier Analysis}).
    \end{minipage}%
    \hfill%
    \begin{minipage}{60mm}%
      \mbox{}\\% force (just above?) top of graphic to be the top of the minipage
      %============================================================================
% Daniel J. Greenhoe
% LaTeX File
%============================================================================
\begin{pspicture}(-3.1,-\latbot)(3.1,2.65)%
  %---------------------------------
  % settings
  %---------------------------------
  \psset{labelsep=5pt}
  %---------------------------------
  % nodes
  %---------------------------------
                        \Cnode( 0,2){X}%    1
  \Cnode(-2,1){V0}\Cnode(-1,1){V1}\Cnode( 0,1){V2}\Cnode( 2,1){Vn1}%
                        \Cnode( 0,0){Z}%    0
  \rput{ 0}(1,1){{\color{blue}\Large$\cdots$}}%
  %---------------------------------
  % node connections
  %---------------------------------
  \ncline{X}{V0}\ncline{X}{V1}\ncline{X}{V2}\ncline{X}{Vn1}%
  \ncline{Z}{V0}\ncline{Z}{V1}\ncline{Z}{V2}\ncline{Z}{Vn1}%
  %---------------------------------
  % node labels
  %---------------------------------
  \uput[ 15](X)  {$\spX$}%
  \uput[180](V0) {$\spV_{0}$}%
  \uput[180](V1) {$\spV_{1}$}%
  \uput[  0](V2) {$\spV_{2}$}%
  \uput[  0](Vn1){$\spV_{n-1}$}%
  \uput[-10](Z)  {$\spZero$}%
  %---------------------------------
  % other labels
  %---------------------------------
  \pnode(2,1.36){analysisP}%
  \rput[ 0] (V2){\psellipse[fillstyle=none,linestyle=dashed,linecolor=red](0,0)(3,0.5)}%
  \rput[tl](-3,2.6){\rnode[b]{scalingL}{scaling subspace}}%
  \rput[tr]( 3,2.6){\rnode[b]{analysisL}{analysis of $\spX$}}%
  \ncline[linecolor=red,linestyle=dotted]{->}{analysisL}{analysisP}%
  \ncline[linecolor=blue,linestyle=dotted]{->}{scalingL}{V0}%
\end{pspicture}%%
    \end{minipage}%

    \begin{minipage}{\tw-55mm}%
      A special characteristic of wavelet analysis is that there is not just one
      scaling subspace,
      %(as is with the case of Fourier and other analyses),
      but an entire sequence of scaling subspaces.
      These scaling subspaces are \prope{linearly ordered} with respect to the
      ordering relation $\subseteq$.
    \end{minipage}%
    \hfill%
    \begin{minipage}{50mm}%
      \mbox{}\\% force (just above?) top of graphic to be the top of the minipage
      \psset{unit=0.5mm}%
      \fns%
      %============================================================================
% Daniel J. Greenhoe
% LaTeX File
%============================================================================
\begin{pspicture}(-0.75,-0.5)(4,7.5)%
  \psset{%
    labelsep=7pt,
    }
  %---------------------------------
  % nodes
  %---------------------------------
  \rput(0,6){{\large$\vdots$}}% 
  \Cnode*(  0,7){X}%    1
  \Cnode (  0,5){V2}%    V_2
  \Cnode (  0,4){V1}%    V_1
  \Cnode (  0,3){V0}%    V_0
  \Cnode (  0,2){Vn1}%    V_{n-1}
  \Cnode (  0,0) {Z}%    0
  \rput(0,1){{\color{blue}\large$\vdots$}}% 
  %---------------------------------
  % node connections
  %---------------------------------
  \ncline{Vn1}{V0}%
  \ncline{V0}{V1}%
  \ncline{V1}{V2}%
  %---------------------------------
  % node labels
  %---------------------------------
  \uput[180](X){$\spLLR$}%
  \uput[180](V2){$\spV_{2}$}%
  \uput[180](V1){$\spV_{1}$}%
  \uput[180](V0){$\spV_{0}$}%
  \uput[180](Vn1){$\spV_{-1}$}%
  \uput[180](Z){$\spZero$}%
  %---------------------------------
  % other labels
  %---------------------------------
  \rput[l](1.10,7){\rnode{labelentire}{entire linear space}}%
  \rput[c](2,4){\rnode{labellarger}{larger subspaces}}%
  \rput[c](2,3){\rnode{labelsmaller}{smaller subspaces}}%
  \rput[l](1.10,0) {\rnode{labelsmallest}{smallest subspace}}
  \ncline[labelsep=2pt,linecolor=red]{->}{labelentire}  {X}%  
  \psline[linecolor=red]{->}(2,4.20)(2,6.20)%
  \psline[linecolor=red]{->}(2,2.80)(2,0.80)%  
  \ncline[labelsep=2pt,linecolor=red]{->}{labelsmallest}{Z}%  
  %---------------------------------
  % design support
  %---------------------------------
  %\psgrid[unit=100\psunit](-1,-1)(5,8)%
\end{pspicture}%
%
    \end{minipage}%

    \begin{minipage}{\tw-65mm}%
      A second special characteristic of wavelet analysis is that it's order structure
      with respect to the $\subseteq$ relation is not a simple M-$n$ lattice 
     (as is with the case of Fourier and other analyses).
      Rather, it is a lattice of the form illustrated to the right.
      This lattice is \prope{non-complemented}, \prope{non-distributive},
      \prope{non-modular}, and \prope{non-Boolean}.
    \end{minipage}%
    \hfill%
    \begin{minipage}{60mm}%
      \mbox{}\\% force (just above?) top of graphic to be the top of the minipage
      %============================================================================
% Daniel J. Greenhoe
% LaTeX file
% wavelet subspace lattice
%============================================================================
\begin{pspicture}(-2.7,-0.4)(2.7,3.6)%
  \fns%
  \psset{
    boxsize=0.40\psunit,
    linearc=0.40\psunit,
    %unit=0.1mm,
    %fillstyle=none,
    % cornersize=relative,
    %framearc=0.5,
    %gridcolor=graph,
    %linewidth=1pt,
    %radius=1.25mm,
    %dotsep=1pt,
    %labelsep=1pt,
    %linecolor=latline,
    }%
  %---------------------------------
  % nodes
  %---------------------------------
  \Cnode( 0,0){Z}%    0
  \Cnode( 0,3){X}%    1
  \Cnode(-1,1){W0}% 
  \Cnode(-2,1){V0}%
  \Cnode( 0,1){W1}%
  \Cnode( 2,1){Wn1}%
  \Cnode(-1.50,1.5){V1}% V1
  \Cnode(-1,2){V2}% V2
  %---------------------------------
  % node labels
  %---------------------------------
  \uput{1.5mm}[ 90](X)  {$\spLLR$}%
  \uput{1.5mm}[135](V2) {$\spV_{2}$}%
  \uput{1.5mm}[135](V1) {$\spV_{1}$}%
  \uput{1.5mm}[180](V0) {$\spV_{0}$}%
  \uput{1.5mm}[180](W0) {$\spW_{0}$}%
  \uput{1.5mm}[  0](W1) {$\spW_{1}$}%
  %\uput{1.5mm}[  0](Wn1){$\spW_{n-1}$}%
  \uput{1.5mm}[-90](Z)  {$\spZero$}%
  %---------------------------------
  % node connections
  %---------------------------------
  \ncline{Z}  {V0}%  0    --> V0
  \ncline{Z}  {W0}%  0    --> W0
  \ncline{Z}  {W1}%  0    --> W1
  \ncline{Z}  {Wn1}% 0    --> W_{n-1}
  \ncline{Wn1}{X}%   Wn-1 --> 1
  \ncline{V0} {V1}%  V0   --> V1
  \ncline{W0} {V1}%  W0   --> V1
  \ncline{V1} {V2}%  V1   --> V2
  \rput{45}(-0.5,2.50){{\color{blue}\Large$\cdots$}}%
  \rput[c]{ 0}(1,1){{\color{blue}\Large$\cdots$}}%
  \ncline{W1}{V2}%   W1 --> V2
  %---------------------------------
  % discriptions
  %---------------------------------
  \ncbox[nodesep=0.25\psunit,linestyle=dotted,linecolor=red]{W0}{Wn1}%
  \ncbox[nodesep=0.25\psunit,linestyle=dotted,linecolor=red]{V0}{X}%
  %\ncbox[nodesep=7pt,linestyle=dotted,linecolor=red]{W0}{Wn1}%
  %\ncbox[nodesep=7pt,linestyle=dotted,linecolor=red]{V0}{X}%
  %\rnode{wavsubbox}  {\ncbox[nodesep=50\psunit,linestyle=dotted,linecolor=red]{W0}{Wn1}}%
  %\rnode{scalesubbox}{\ncbox[nodesep=50\psunit,linestyle=dotted,linecolor=red]{V0}{X}}%
  \pnode[0,-0.40](Wn1){wavsubbox}%  
  \pnode[0,0.60](V2){scalesubbox}%
  %\rput[ 0](  5,10){\psellipse[fillstyle=none,linestyle=dashed,linecolor=red](0,0)(20,5)}%
  %\rput{45}(-12,22){\psellipse[fillstyle=none,linestyle=dashed,linecolor=red](0,0)(25,6)}%
  \rput[br](2.5,1.5){\rnode{wavsublabel}{wavelet subspaces}}%
  %\psline[linecolor=red]{->}(24,24)(20,14)%
  \rput[bl]{45}(-2.4,1.6){\rnode{scalesublabel}{scaling subspaces}}%
 %\psline[linecolor=red]{->}(-15,36)(-15,26)%
  %\ncline[linecolor=red]{->}{wavsublabel}{wavsubbox}%
  %\ncline[linecolor=red]{->}{scalesublabel}{scalesubbox}%
  %---------------------------------
  % debug support
  %---------------------------------
%  \psgrid[unit=\psunit](-30,-10)(30,40)%
  %\psgrid[unit=10\psunit](-3,-1)(3,4)%
\end{pspicture}%%
    \end{minipage}%

    \begin{minipage}{\tw-65mm}%
      In the world of mathematical structures,
      there is circumstantial evidence that the order structure of wavelet analyses is quite rare,
      if not outright unique.
      For example, suppose we replace the wavelet subspaces with prime numbers
      and the scaling subspaces with their products as illustrated to the right.
      The resulting sequence $\seqn{1,\,2,\,6,\,30,\,210}$ as of 2011 July 30
      has no matches in Neil J.A. Sloane's  \emph{Online Encyclopedia of Integer Sequences}
      (hosted by \emph{AT\&T Research}).\footnotemark
    \end{minipage}%
    \footnotetext{%
      \citerc{oeis}{enter $1,2,6,30,210$}%
      }%
    \hfill%
    \begin{minipage}{60mm}%
      \mbox{}\\% force (just above?) top of graphic to be the top of the minipage
      %============================================================================
% Daniel J. Greenhoe
% LaTeX file
% wavelet subspace lattice with prime number illustration
%============================================================================
  \begin{pspicture}(-3,-0.40)(3,3.60)%
  \psset{
    boxsize=0.40\psunit,
    linearc=0.40\psunit,
    }%
  \fns%
      %---------------------------------
      % nodes
      %---------------------------------
      \Cnode( 0,0){Z}%    0
      \Cnode( 0,3){X}%    1
      \Cnode(-1,1){W0}% 
      \Cnode( 0,1){W1}%
      \Cnode( 1,1){W2}%
      \Cnode( 2,1){Wn1}%
      \Cnode(-2,1){V0}%
      \Cnode(-1.5,1.5){V1}% V1
      \Cnode(-1,2){V2}% V2
      \Cnode(-0.5,2.5){V3}% V2
      %---------------------------------
      % node labels
      %---------------------------------
      \uput{1.5mm}[ 90](X)  {$2310$}%
      \uput{1.5mm}[135](V3) {$210$}%
      \uput{1.5mm}[135](V2) {$30$}%
      \uput{1.5mm}[135](V1) {$6$}%
      \uput{1.5mm}[180](V0) {$2$}%
      \uput{1.5mm}[180](W0) {$3$}%
      \uput{1.5mm}[  0](W1) {$5$}%
      \uput{1.5mm}[  0](W2) {$7$}%
      \uput{1.5mm}[  0](Wn1){$11$}%
      \uput{1.5mm}[-90](Z)  {$1$}%
      %---------------------------------
      % node connections
      %---------------------------------
      \ncline{Z}  {V0}%  0    --> V0
      \ncline{Z}  {W0}%  0    --> W0
      \ncline{Z}  {W1}%  0    --> W1
      \ncline{Z}  {W2}%  0    --> W2
      \ncline{Z}  {Wn1}% 0    --> W_{n-1}
      \ncline{Wn1}{X}%   Wn-1 --> 1
      \ncline{V0} {V1}%  V0   --> V1
      \ncline{W0} {V1}%  W0   --> V1
      \ncline{V1} {V2}%  V1   --> V2
      \ncline{V2} {V3}%  
      \ncline{V3} {X}%  
      %\rput{45}(-50,250){{\color{blue}\Large$\cdots$}}%
      %\rput[c]{ 0}(100,100){{\color{blue}\Large$\cdots$}}%
      \ncline{W1}{V2}%   W1 --> V2
      \ncline{W2}{V3}%   W1 --> V2
      %---------------------------------
      % discriptions
      %---------------------------------
      \ncbox[nodesep=.25\psunit,linestyle=dotted,linecolor=red]{V0}{Wn1}%
      \ncbox[nodesep=.25\psunit,linestyle=dotted,linecolor=red]{V0}{X}%
      \pnode[0,-.40](Wn1){wavsubbox}%  
      \pnode[0,.60](V2){scalesubbox}%
      %\ncline[linecolor=red]{->}{wavsublabel}{wavsubbox}%
      %\ncline[linecolor=red]{->}{scalesublabel}{scalesubbox}%
      \rput[br](2.5,1.5){\rnode{wavsublabel}{prime numbers}}%
      \rput[bl]{45}(-2.4,1.6){\rnode{scalesublabel}{primorial numbers}}%
      %---------------------------------
      % debug support
      %---------------------------------
%      \psgrid[unit=10\psunit](-30,-10)(30,40)%
      %\psgrid[unit=100\psunit](-3,-1)(3,4)%
  \end{pspicture}%
%
    \end{minipage}%

  An analysis can be represented using three different structures:
    %\paragraph{Equivalence of lattice representations.}
    %So far we have discussed representing a wavelet analysis using three different structures:
\\\begin{tabular}{@{\qquad}ll}
  \circOne    & sequence of subspaces             \\
  \circTwo    & sequence of bases vectors         \\
  \circThree  & sequence of bases coefficients
\end{tabular}\\
These structures are isomorphic to each other---and can therefore be used interchangeably.
%(see \prefp{thm:VPb_isomorphic}).
%(see \prefp{fig:wav_VPb_isomorphic}).
%That is, a ``\hie{wavelet analysis}" can be described using any of these structures.
%However, sometimes when introducing theorems about wavelets,
%it is convenient to use elements from not just one, but from multiple lattices;
%and so it is convenient to have a ``collection" of wavelet analysis elements
%all assembled together into one formally defined tuple.
%\pref{def:wavsys} (next) does just that---it defines a \hie{wavelet analysis} in terms of a tuple with elements
%extracted from the four wavelet structures.


%---------------------------------------
% isomorphic lattices
%---------------------------------------
%\begin{figure}[th]
  \begin{center}%
  \begin{fsL}%
  \begin{minipage}[c]{12\tw/16}
  \begin{minipage}[c]{4\tw/16}%
  \center
  \latmatlw{4}{0.5}
    {
           &       & \null                 \\
           & \null                         \\
     \null &       & \null &       & \null \\
           &       & \null
    }
    {\ncline{1,3}{2,2}\ncline{2,2}{3,1}
     \ncline{1,3}{3,5}
     \ncline{2,2}{3,3}
     \ncline{4,3}{3,1}\ncline{4,3}{3,3}\ncline{4,3}{3,5}
    }
    {\nput{ 90}{1,3}{$\spV_3$}
     \nput{135}{2,2}{$\spV_2$}
     \nput{180}{3,1}{$\spV_1$}
     \nput{ 67}{3,3}{$\spW_1$}
     \nput{  0}{3,5}{$\spW_2$}
     \nput{-90}{4,3}{$\spZero$}
    }
  \end{minipage}
  \hfill{\Large$\thapprox$}\hfill
  \begin{minipage}[c]{4\tw/16}%
  \center
  \latmatlw{4}{0.5}
    {
           &       & \null                 \\
           & \null                         \\
     \null &       & \null &       & \null \\
           &       & \null
    }
    {\ncline{1,3}{2,2}\ncline{2,2}{3,1}
     \ncline{1,3}{3,5}
     \ncline{2,2}{3,3}
     \ncline{4,3}{3,1}\ncline{4,3}{3,3}\ncline{4,3}{3,5}
    }
    {\nput{ 90}{1,3}{$\seqn{h_n}_3$}
     \nput{135}{2,2}{$\seqn{h_n}_2$}
     \nput{180}{3,1}{$\seqn{h_n}_1$}
     \nput{ 67}{3,3}{$\seqn{g_n}_1$}
     \nput{  0}{3,5}{$\seqn{g_n}_2$}
     \nput{-90}{4,3}{$\opZero$}
    }
  \end{minipage}
  \hfill{\Large$\thapprox$}\hfill
  \begin{minipage}[c]{4\tw/16}%
  \center
  \latmatlw{4}{0.5}
    {
           &       & \null                 \\
           & \null                         \\
     \null &       & \null &       & \null \\
           &       & \null
    }
    {\ncline{1,3}{2,2}\ncline{2,2}{3,1}
     \ncline{1,3}{3,5}
     \ncline{2,2}{3,3}
     \ncline{4,3}{3,1}\ncline{4,3}{3,3}\ncline{4,3}{3,5}
    }
    {\nput{ 90}{1,3}{$\seqn{\phi_{3,m}}$}
     \nput{135}{2,2}{$\seqn{\phi_{2,m}}$}
     \nput{180}{3,1}{$\seqn{\phi_{1,m}}$}
     \nput{ 67}{3,3}{$\seqn{\psi_{1,m}}$}
     \nput{  0}{3,5}{$\seqn{\psi_{2,m}}$}
     \nput{-90}{4,3}{$\opZero$}
    }
  \end{minipage}
\end{minipage}
\end{fsL}
\end{center}
%\caption{
%  Subspace, coefficient, and basis lattice isomorphisms
%  \label{fig:wav_VPb_isomorphic}
%  }
%\end{figure}

  Here are some examples of the order structures of some analyses,
        including two wavelet analyses:

{\begin{center}%
  \begin{fsL}%
\psset{unit=9mm}%
\begin{longtable}{|c|c|}%
\hline%
\mc{1}{B}{Cosine analysis  (even Fourier series)} & \mc{1}{B}{Cosine polynomial analysis}%
\\%
  %============================================================================
% Daniel J. Greenhoe
% XeLaTeX file
% nominal unit = 10mm
%============================================================================
\begin{pspicture}(-3.8,-0.5)(3.8,3.5)
  \psset{linewidth=1pt,linecolor=latline,radius=0.75\psunit}%
  %-------------------------------------
  % nodes
  %-------------------------------------
  \rput(0,3){\ovalnode{lub}{$\spX=\oppS\set{\cos(2\pi nx)}{n=0,1,2,3}$}}%
  %\cnodeput(0,2){lub}{$\spX$}
  %\rput[l](1,2){\rnode{Xlabel}{\footnotesize$\spX=\oppS\set{\cos(2\pi n x)}{n=0,1,2,3}$}}%
  \rput[bl](-3.5,-0.3){\rnode[tl]{slabel}{scaling subspace}}%
  \Cnode(-3,1.5){cos0hz}%
  \Cnode(-1,1.5){cos1hz}%
  \Cnode( 1,1.5){cos2hz}%
  \Cnode( 3,1.5){cos3hz}%
  \cnodeput( 0,0){glb}{$\spZero$}%
  %-------------------------------------
  % connecting lines
  %-------------------------------------
  \ncline{cos0hz}{lub}%
  \ncline{cos1hz}{lub}%
  \ncline{cos2hz}{lub}%
  \ncline{cos3hz}{lub}%
  \ncline{cos0hz}{glb}%
  \ncline{cos1hz}{glb}%
  \ncline{cos2hz}{glb}%
  \ncline{cos3hz}{glb}%
  \ncline{->}{Xlabel}{lub}%
  \ncline{->}{slabel}{cos0hz}%
  %\ncarc[arcangle=30]{->}{slabel}{cos0hz}
  %-------------------------------------
  % plots
  %-------------------------------------
  \rput(cos0hz){%
    \psline[linecolor=axis]{<->}(-0.7,0)(0.7,0)%
    \psline[linecolor=axis]{<->}(0,-0.7)(0,0.7)%
    \psplot[linecolor=blue,plotpoints=64]{-0.4}{0.4}{x 360 mul 2 mul 0 mul cos 0.5 mul}%
    \psplot[linecolor=blue,plotpoints=32,linestyle=dotted,dotsep=0.05]{0.4}{0.7}{x 360 mul 2 mul 0 mul cos 0.5 mul}%
    \psplot[linecolor=blue,plotpoints=32,linestyle=dotted,dotsep=0.05]{-0.4}{-0.7}{x 360 mul 2 mul 0 mul cos 0.5 mul}%
    }%
  \rput(cos1hz){%
    \psline[linecolor=axis]{<->}(-0.7,0)(0.7,0)%
    \psline[linecolor=axis]{<->}(0,-0.7)(0,0.7)%
    \psplot[linecolor=red,plotpoints=64]{-0.6}{0.6}{x 360 mul 2 mul 1 mul cos 0.5 mul}%
    \psplot[linecolor=red,plotpoints=32,linestyle=dotted,dotsep=0.075]{0.6}{0.7}{x 360 mul 2 mul 1 mul cos 0.5 mul}%
    \psplot[linecolor=red,plotpoints=32,linestyle=dotted,dotsep=0.075]{-0.6}{-0.7}{x 360 mul 2 mul 1 mul cos 0.5 mul}%
    }%
  \rput(cos2hz){%
    \psline[linecolor=axis]{<->}(-0.7,0)(0.7,0)%
    \psline[linecolor=axis]{<->}(0,-0.7)(0,0.7)%
    \psplot[linecolor=red,plotpoints=256]{-0.6}{0.6}{x 360 mul 2 mul 2 mul cos 0.5 mul}%
    \psplot[linecolor=red,plotpoints=32,linestyle=dotted,dotsep=0.05]{0.6}{0.7}{x 360 mul 2 mul 2 mul cos 0.5 mul}%
    \psplot[linecolor=red,plotpoints=32,linestyle=dotted,dotsep=0.05]{-0.6}{-0.7}{x 360 mul 2 mul 2 mul cos 0.5 mul}%
    }%
  \rput(cos3hz){%
    \psline[linecolor=axis]{<->}(-0.7,0)(0.7,0)%
    \psline[linecolor=axis]{<->}(0,-0.7)(0,0.7)%
    \psplot[linecolor=red,plotpoints=512]{-0.6}{0.6}{x 360 mul 2 mul 3 mul cos 0.5 mul}%
    \psplot[linecolor=red,plotpoints=32,linestyle=dotted,dotsep=0.05]{0.6}{0.65}{x 360 mul 2 mul 3 mul cos 0.5 mul}%
    \psplot[linecolor=red,plotpoints=32,linestyle=dotted,dotsep=0.05]{-0.6}{-0.65}{x 360 mul 2 mul 3 mul cos 0.5 mul}%
    }%
  %-------------------------------------
  % development support
  %-------------------------------------
  %\psgrid(-4,-1.8)(4,1.8)%
\end{pspicture}%%
&%
  %============================================================================
% Daniel J. Greenhoe
% XeLaTeX file
% nominal unit = 10mm
%============================================================================
\begin{pspicture}(-3.8,-0.5)(3.8,3.5)
  \psset{linewidth=1pt,linecolor=latline,radius=0.75\psunit}%
  %-------------------------------------
  % nodes
  %-------------------------------------
  \rput(0,3){\ovalnode{lub}{$\spX=\oppS\set{\cos^n(2\pi x)}{n=0,1,2,3}$}}%
  %\cnodeput(0,2){lub}{$\spX$}
  %\rput[l](1,2){\rnode{Xlabel}{$\spX=\oppS\set{\cos^n(2\pi x)}{n=0,1,2,3}$}}%
  \rput[bl](-3.5,-0.3){\rnode[tl]{slabel}{scaling subspace}}%
  \Cnode(-3,1.5){cos0hz}%
  \Cnode(-1,1.5){cos1hz}%
  \Cnode( 1,1.5){cos2hz}%
  \Cnode( 3,1.5){cos3hz}%
  \cnodeput( 0,0){glb}{$\spZero$}%
  %-------------------------------------
  % connecting lines
  %-------------------------------------
  \ncline{cos0hz}{lub}%
  \ncline{cos1hz}{lub}%
  \ncline{cos2hz}{lub}%
  \ncline{cos3hz}{lub}%
  \ncline{cos0hz}{glb}%
  \ncline{cos1hz}{glb}%
  \ncline{cos2hz}{glb}%
  \ncline{cos3hz}{glb}%
  \ncline{->}{Xlabel}{lub}%
  \ncline{->}{slabel}{cos0hz}%
  %\ncarc[arcangle=30]{->}{slabel}{cos0hz}
  %\ncdiag[angleA=120,angleB=-120]{->}{slabel}{cos0hz}
  %-------------------------------------
  % plots
  %-------------------------------------
  \rput(cos0hz){%
    \psline[linecolor=axis]{<->}(-0.7,0)(0.7,0)%
    \psline[linecolor=axis]{<->}(0,-0.7)(0,0.7)%
    \psplot[linecolor=blue,plotpoints=64]{-0.4}{0.4}{x 360 mul 2 mul cos 0 exp 0.5 mul}%
    \psplot[linecolor=blue,plotpoints=32,linestyle=dotted,dotsep=0.05]{0.4}{0.7}{x 360 mul 2 mul 0 mul cos 0.5 mul}%
    \psplot[linecolor=blue,plotpoints=32,linestyle=dotted,dotsep=0.05]{-0.4}{-0.7}{x 360 mul 2 mul 0 mul cos 0.5 mul}%
    }%
  \rput(cos1hz){%
    \psline[linecolor=axis]{<->}(-0.7,0)(0.7,0)%
    \psline[linecolor=axis]{<->}(0,-0.7)(0,0.7)%
    \psplot[linecolor=red,plotpoints=64]{-0.6}{0.6}{x 360 mul 2 mul cos 1 exp 0.5 mul}%
    \psplot[linecolor=red,plotpoints=32,linestyle=dotted,dotsep=0.075]{0.6}{0.7}{x 360 mul 2 mul cos 1 exp 0.5 mul}%
    \psplot[linecolor=red,plotpoints=32,linestyle=dotted,dotsep=0.075]{-0.6}{-0.7}{x 360 mul 2 mul cos 1 exp 0.5 mul}%
    }%
  \rput(cos2hz){%
    \psline[linecolor=axis]{<->}(-0.7,0)(0.7,0)%
    \psline[linecolor=axis]{<->}(0,-0.7)(0,0.7)%
    \psplot[linecolor=red,plotpoints=256]{-0.6}{0.6}{x 360 mul 2 mul cos 2 exp 0.5 mul}%
    \psplot[linecolor=red,plotpoints=32,linestyle=dotted,dotsep=0.05]{0.6}{0.7}{x 360 mul 2 mul cos 2 exp 0.5 mul}%
    \psplot[linecolor=red,plotpoints=32,linestyle=dotted,dotsep=0.05]{-0.6}{-0.7}{x 360 mul 2 mul cos 2 exp 0.5 mul}%
    }%
  \rput(cos3hz){%
    \psline[linecolor=axis]{<->}(-0.7,0)(0.7,0)%
    \psline[linecolor=axis]{<->}(0,-0.7)(0,0.7)%
    \psplot[linecolor=red,plotpoints=512]{-0.6}{0.6}{x 360 mul 2 mul cos 3 exp 0.5 mul}%
    \psplot[linecolor=red,plotpoints=32,linestyle=dotted,dotsep=0.05]{0.6}{0.7}{x 360 mul 2 mul cos 3 exp 0.5 mul}%
    \psplot[linecolor=red,plotpoints=32,linestyle=dotted,dotsep=0.05]{-0.6}{-0.7}{x 360 mul 2 mul cos 3 exp 0.5 mul}%
    }%
  %-------------------------------------
  % development support
  %-------------------------------------
  %\psgrid(-4,-1.8)(4,1.8)%
\end{pspicture}%%
\\\hline%
\mc{1}{|B|}{Chebyshev polynomial analysis\cittrp{rivlin1974}{4}}&\mc{1}{|B|}{Hadamard-3 analysis}%
\\%
  %============================================================================
% Daniel J. Greenhoe
% XeLaTeX file
% nominal unit = 10mm
%============================================================================
\begin{pspicture}(-3.8,-0.5)(3.8,3.5)
  %-------------------------------------
  % settings
  %-------------------------------------
  \psset{linewidth=1pt,linecolor=latline,radius=0.75\psunit}%
  %-------------------------------------
  % nodes
  %-------------------------------------
  \rput(0,3){\ovalnode{lub}{$\spX=\oppS\set{T^n(x)}{n=0,1,2,3}$}}%
  %\cnodeput(0,2){lub}{$\spX$}
  %\rput[l](1,2){\rnode{Xlabel}{$\spX=\oppS\set{\cos^n(2\pi x)}{n=0,1,2,3}$}}%
  \rput[bl](-3.5,-0.3){\rnode[tl]{slabel}{scaling subspace}}%
  \Cnode(-3,1.5){cos0hz}%
  \Cnode(-1,1.5){cos1hz}%
  \Cnode( 1,1.5){cos2hz}%
  \Cnode( 3,1.5){cos3hz}%
  \cnodeput( 0,0){glb}{$\spZero$}%
  %-------------------------------------
  % connecting lines
  %-------------------------------------
  \ncline{cos0hz}{lub}%
  \ncline{cos1hz}{lub}%
  \ncline{cos2hz}{lub}%
  \ncline{cos3hz}{lub}%
  \ncline{cos0hz}{glb}%
  \ncline{cos1hz}{glb}%
  \ncline{cos2hz}{glb}%
  \ncline{cos3hz}{glb}%
  \ncline{->}{Xlabel}{lub}
  \ncline{->}{slabel}{cos0hz}
  %\ncarc[arcangle=30]{->}{slabel}{cos0hz}
  %\ncdiag[angleA=120,angleB=-120]{->}{slabel}{cos0hz}
  %-------------------------------------
  % plots
  %-------------------------------------
  \rput(cos0hz){% %(-3, 0){% cos(0x)
    \psline[linecolor=axis]{<->}(-0.7,0)(0.7,0)%
    \psline[linecolor=axis]{<->}(0,-0.7)(0,0.7)%
    \psplot[linecolor=blue,plotpoints=64]{-0.4}{0.4}{1 0.5 mul}
    \psplot[linecolor=blue,plotpoints=32,linestyle=dotted,dotsep=0.05]{0.4}{0.6}{1 0.5 mul}
    \psplot[linecolor=blue,plotpoints=32,linestyle=dotted,dotsep=0.05]{-0.4}{-0.6}{1 0.5 mul}
    }
  \rput(cos1hz){% (-1, 0){% cos(x)
    \psline[linecolor=axis]{<->}(-0.7,0)(0.7,0)%
    \psline[linecolor=axis]{<->}(0,-0.7)(0,0.7)%
    \psplot[linecolor=red,plotpoints=64]{-0.4}{0.4}{x}
    \psplot[linecolor=red,plotpoints=32,linestyle=dotted,dotsep=0.05]{0.4}{0.6}{x}
    \psplot[linecolor=red,plotpoints=32,linestyle=dotted,dotsep=0.05]{-0.4}{-0.6}{x}
    }
  \rput(cos2hz){% ( 1, 0){% cos(2x)
    \psline[linecolor=axis]{<->}(-0.7,0)(0.7,0)%
    \psline[linecolor=axis]{<->}(0,-0.7)(0,0.7)%
    \psplot[linecolor=red,plotpoints=64]{-0.45}{0.45}{2 x 2 mul 2 exp mul 1 sub 0.5 mul}
    \psplot[linecolor=red,plotpoints=32,linestyle=dotted,dotsep=0.05]{0.45}{0.55}{2 x 2 mul 2 exp mul 1 sub 0.5 mul}
    \psplot[linecolor=red,plotpoints=32,linestyle=dotted,dotsep=0.05]{-0.45}{-0.55}{2 x 2 mul 2 exp mul 1 sub 0.5 mul}
    }
  \rput(cos3hz){% ( 3, 0){% cos(3x)
    \psline[linecolor=axis]{<->}(-0.7,0)(0.7,0)%
    \psline[linecolor=axis]{<->}(0,-0.7)(0,0.7)%
    \psplot[linecolor=red,plotpoints=64]{-0.45}{0.45}{4 x 2 mul 3 exp mul 3 x 2 mul mul sub 0.5 mul}
    \psplot[linecolor=red,plotpoints=32,linestyle=dotted,dotsep=0.05]{0.45}{0.5}{4 x 2 mul 3 exp mul 3 x 2 mul mul sub 0.5 mul}
    \psplot[linecolor=red,plotpoints=32,linestyle=dotted,dotsep=0.05]{-0.45}{-0.5}{4 x 2 mul 3 exp mul 3 x 2 mul mul sub 0.5 mul}
    }
\end{pspicture}%
%
&%
  %============================================================================
% Daniel J. Greenhoe
% XeLaTeX file
% Hadamard matrix
% 
% H1 = [1]
% 
% H2 = |H1 H1 | = |1  1|
%      |H1 H1^|   |1 -1|
%
%                 |1  1   1   1|
% H3 = |H2 H2 | = |1 -1   1  -1|
%      |H2 H2^|   |1  1  -1  -1|
%                 |1 -1  -1   1|
%============================================================================
\begin{pspicture}(-4,-2)(4,2)%
  \psset{%
    linewidth=1pt,%
    linecolor=latline,%
    radius=0.75\psunit,%
    dotsize=5pt,%
    }%
  %-------------------------------------
  % nodes
  %-------------------------------------
  \rput(0,1.5){\ovalnode{lub}{$\spX=\oppS H_3$}}%
  %\cnodeput(0,2){lub}{$\spX$}
  %\rput[l](1,2){\rnode{Xlabel}{\footnotesize$\spX=\oppS\set{\cos(2\pi n x)}{n=0,1,2,3}$}}%
  \cnodeput( 0,-1.5){glb}{$\spZero$}%
  \rput[bl](-4,-1.8){\rnode[t]{slabel}{scaling subspace}}%
  \Cnode(-3, 0){h1111}%
  \Cnode(-1, 0){h11nn}%
  \Cnode( 1, 0){h1nn1}%
  \Cnode( 3, 0){h1n1n}%
  %-------------------------------------
  % connecting lines
  %-------------------------------------
  \ncline{h1111}{lub}%
  \ncline{h1n1n}{lub}%
  \ncline{h11nn}{lub}%
  \ncline{h1nn1}{lub}%
  \ncline{h1111}{glb}%
  \ncline{h1n1n}{glb}%
  \ncline{h11nn}{glb}%
  \ncline{h1nn1}{glb}%
  \ncline{->}{Xlabel}{lub}%
  \ncline{->}{slabel}{h1111}%
  %\ncarc[arcangle=30]{->}{slabel}{h1111}
  %-------------------------------------
  % plots
  %-------------------------------------
  \psset{yunit=3.5mm,xunit=2.75mm}%
  \rput(h1111){\begin{pspicture}(-1,-2)(4,2)%
      \psset{linecolor=blue}%
      \psline[linecolor=axis]{->}(0,0)(4,0)%
      \psline[linecolor=axis]{<->}(0,-1.75)(0,1.75)%
      \psline{-o}(0,0)(0, 1)%  1
      \psline{-o}(1,0)(1, 1)%  1
      \psline{-o}(2,0)(2, 1)%  1
      \psline{-o}(3,0)(3, 1)%  1
    \end{pspicture}}%
  \rput(h1n1n){\begin{pspicture}(-1,-2)(4,2)%
      \psset{linecolor=red}%
      \psline[linecolor=axis]{->}(0,0)(4,0)%
      \psline[linecolor=axis]{<->}(0,-1.75)(0,1.75)%
      \psline{-o}(0,0)(0, 1)%  1
      \psline{-o}(1,0)(1,-1)% -1
      \psline{-o}(2,0)(2, 1)%  1
      \psline{-o}(3,0)(3,-1)% -1
    \end{pspicture}}%
  \rput(h11nn){\begin{pspicture}(-1,-2)(4,2)%
      \psset{linecolor=red}%
      \psline[linecolor=axis]{->}(0,0)(4,0)%
      \psline[linecolor=axis]{<->}(0,-1.75)(0,1.75)%
      \psline{-o}(0,0)(0, 1)%  1
      \psline{-o}(1,0)(1, 1)%  1
      \psline{-o}(2,0)(2,-1)% -1
      \psline{-o}(3,0)(3,-1)% -1
    \end{pspicture}}%
  \rput(h1nn1){\begin{pspicture}(-1,-2)(4,2)%
      \psset{linecolor=red}%
      \psline[linecolor=axis]{->}(0,0)(4,0)%
      \psline[linecolor=axis]{<->}(0,-1.75)(0,1.75)%
      \psline{-o}(0,0)(0, 1)%  1
      \psline{-o}(1,0)(1,-1)% -1
      \psline{-o}(2,0)(2,-1)% -1
      \psline{-o}(3,0)(3, 1)%  1
      %\psgrid(-4,-1.8)(4,1.8)%
    \end{pspicture}}%
\end{pspicture}%%
\\\hline%
\mc{1}{|B|}{Haar/Daubechies-$p1$ wavelet analysis} & \mc{1}{B|}{Daubechies-$p2$ wavelet analysis}%
\\%
  %============================================================================
% Daniel J. Greenhoe
% LaTeX file
% nominal unit = 8mm
%============================================================================
\begin{pspicture}(-4.2,-2)(4.2,7.2)
  %-------------------------------------
  % settings
  %-------------------------------------
  \psset{radius=0.75\psunit}%
  %-------------------------------------
  % nodes
  %-------------------------------------
  \rput[l](2,5){\parbox{2\psunit}{scaling\\subspaces}}%
  \psline[linecolor=red]{->}(2,5)(1,5)%
  \cnodeput( 0,-1.5){glb}{$\spZero$}%
  \rput{63}(-1.5,3){\psellipse[linestyle=dashed,linecolor=red](0,0)(4.5,1.5)}%
  \Cnode( 0, 6){V3}%
  \Cnode(-1, 4){V2}%
  \Cnode(-2, 2){V1}%
  \Cnode(-3, 0){V0}%
  \Cnode(-1, 0){W0}%
  \Cnode( 1, 0){W1}%
  \Cnode( 3, 0){W2}%
  %-------------------------------------
  % connecting lines
  %-------------------------------------
  \ncline{V3}{V2}\ncline{V3}{W2}%
  \ncline{V2}{V1}\ncline{V2}{W1}%
  \ncline{V1}{V0}\ncline{V1}{W0}%
  \ncline{V0}{glb}\ncline{W0}{glb}\ncline{W1}{glb}\ncline{W2}{glb}%
  %\ncline{->}{slabel}{scalingSubspaces}%
  %-------------------------------------
  % plots
  %-------------------------------------
  \rput(-0.5, 6){% V3
    \psline[linecolor=axis]{->}(0,0)(1.4,0)%
    \psline[linecolor=axis]{<->}(0,-0.7)(0,0.7)%
    \psline[linecolor=blue](0,0)(0,0.424)(0.15,0.424)(0.15,0)%
    }
  \rput(-1.5, 4){% V2
    \psline[linecolor=axis]{->}(0,0)(1.4,0)%
    \psline[linecolor=axis]{<->}(0,-0.7)(0,0.7)%
    \psline[linecolor=blue](0,0)(0,0.300)(0.3,0.300)(0.3,0)%
    }%
  \rput(-2.5, 2){% V1
    \psline[linecolor=axis]{->}(0,0)(1.4,0)%
    \psline[linecolor=axis]{<->}(0,-0.7)(0,0.7)%
    \psline[linecolor=blue](0,0)(0,0.300)(0.6,0.300)(0.6,0)%
    }%
  \rput(-3.5, 0){% V0
    \psline[linecolor=axis]{->}(0,0)(1.4,0)%
    \psline[linecolor=axis]{<->}(0,-0.7)(0,0.7)%
    \psline[linecolor=blue](0,0)(0,0.212)(1.2,0.212)(1.2,0)%
    }%
  \rput(-1.5, 0){% W1
    \psline[linecolor=axis]{->}(0,0)(1.4,0)%
    \psline[linecolor=axis]{<->}(0,-0.7)(0,0.7)%
    \psline[linecolor=red](0,0)(0,0.30)(0.6,0.30)(0.6,-0.30)(1.2,-0.30)(1.2,0)%
    }%
  \rput(0.5, 0){% W2
    \psline[linecolor=axis]{->}(0,0)(1.4,0)%
    \psline[linecolor=axis]{<->}(0,-0.7)(0,0.7)%
    \psline[linecolor=red](0,0)(0,0.42)(0.3,0.42)(0.3,-0.42)(0.6,-0.42)(0.6,0)%
    }%
  \rput(2.5, 0){% W3
    \psline[linecolor=axis]{->}(0,0)(1.4,0)%
    \psline[linecolor=axis]{<->}(0,-0.7)(0,0.7)%
    \psline[linecolor=red](0,0)(0,0.6)(0.15,0.6)(0.15,-0.6)(0.30,-0.6)(0.30,0)%
    }%
  %\psgrid(-4,-1.8)(4,7.2)%
\end{pspicture}%%
&%
  %============================================================================
% Daniel J. Greenhoe
% LaTeX file
% nominal unit = 8mm
%============================================================================
\begin{pspicture}(-4.2,-2)(4.2,7.2)
  %-------------------------------------
  % settings
  %-------------------------------------
  \psset{radius=0.75\psunit}%
  %-------------------------------------
  % nodes
  %-------------------------------------
  %\rput[l](2,5){\parbox{2\psunit}{scaling\\subspaces}}%
  %\psline[linecolor=red]{->}(2,5)(1,5)%
  \cnodeput( 0,-1.5){glb}{$\spZero$}%
  \rput{63}(-1.5,3){\psellipse[linestyle=dashed,linecolor=red](0,0)(4.5,1.5)}%
  \Cnode( 0, 6){V3}%
  \Cnode(-1, 4){V2}%
  \Cnode(-2, 2){V1}%
  \Cnode(-3, 0){V0}%
  \Cnode(-1, 0){W0}%
  \Cnode( 1, 0){W1}%
  \Cnode( 3, 0){W2}%
  %-------------------------------------
  % connecting lines
  %-------------------------------------
  \ncline{V3}{V2}\ncline{V3}{W2}%
  \ncline{V2}{V1}\ncline{V2}{W1}%
  \ncline{V1}{V0}\ncline{V1}{W0}%
  \ncline{V0}{glb}\ncline{W0}{glb}\ncline{W1}{glb}\ncline{W2}{glb}%
  %\ncline{->}{slabel}{scalingSubspaces}%
  %-------------------------------------
  % plots
  %-------------------------------------
  \rput(-0.50, 6){% V3
    \psline[linecolor=axis]{->}(0,0)(1.40,0)%
    \psline[linecolor=axis]{<->}(0,-0.70)(0,0.70)%
    \fileplot[linecolor=blue,xunit=0.0625\psunit,yunit=0.707\psunit]{../wavelets/d2_phi.dat}%
    }
  \rput(-1.5, 4){% V2
    \psline[linecolor=axis]{->}(0,0)(1.4,0)%
    \psline[linecolor=axis]{<->}(0,-0.70)(0,0.70)%
    \fileplot[linecolor=blue,xunit=0.125\psunit,yunit=0.50\psunit]{../wavelets/d2_phi.dat}%
    }%
  \rput(-2.5, 2){% V1
    \psline[linecolor=axis]{->}(0,0)(1.40,0)%
    \psline[linecolor=axis]{<->}(0,-0.70)(0,0.70)%
    \fileplot[linecolor=blue,xunit=0.25\psunit,yunit=0.3535\psunit]{../wavelets/d2_phi.dat}%
    }%
  \rput(-3.5, 0){% V0
    \psline[linecolor=axis]{->}(0,0)(1.40,0)%
    \psline[linecolor=axis]{<->}(0,-0.70)(0,0.70)%
    \fileplot[linecolor=blue,xunit=0.50\psunit,yunit=0.25\psunit]{../wavelets/d2_phi.dat}%
    }%
  \rput(-1.50, 0){% W0
    \psline[linecolor=axis]{->}(0,0)(1.40,0)%
    \psline[linecolor=axis]{<->}(0,-0.70)(0,0.70)%
    \fileplot[linecolor=red,xunit=0.50\psunit,yunit=0.25\psunit]{../wavelets/d2_psi.dat}%
    }%
  \rput(0.50, 0){% W1
    \psline[linecolor=axis]{->}(0,0)(1.40,0)%
    \psline[linecolor=axis]{<->}(0,-0.70)(0,0.70)%
    \fileplot[linecolor=red,xunit=0.25\psunit,yunit=0.3535\psunit]{../wavelets/d2_psi.dat}%
    }%
  \rput(2.50, 0){% W2
    \psline[linecolor=axis]{->}(0,0)(1.40,0)%
    \psline[linecolor=axis]{<->}(0,-0.70)(0,0.70)%
    \fileplot[linecolor=red,xunit=0.125\psunit,yunit=0.50\psunit]{../wavelets/d2_psi.dat}%
    }%
\end{pspicture}%%
\\\hline%
\end{longtable}%
  \end{fsL}%
\end{center}}%

%=======================================
\section{Approximation}
%=======================================
The approximation of an element in a subspace means finding the ``closet match"
of the element in the subspace.
This presumably means that the subspace has some kind of topology.
But in general, it is difficult to perform approximation in a topological space
with no additional structure.
For example, contained in the topological space
  \\\indentx$\psetx\eqd\setn{\szero,\,\setn{x},\,\setn{y},\,\setn{z},\,\setn{x,y},\,\setn{x,z},\,\setn{y,z},\,\sid}$\\
is the topological subspace
  \\\indentx$\setn{\szero,\,\setn{x},\,\setn{y},\,\setn{x,y},\,\sid}$.\\
In this subspace, we know that the element $\setn{x}$ is ``near" the element $\setn{x,y}$
because $\setn{x}\subseteq\setn{x,y}$.
So because topological spaces are useful for determining ``nearness", they are
useful also for determining properties such as \hie{convergence}.
But what about approximation?
For example, what is the closest match in the subspace to the element $\setn{z}$?
Element $\setn{z}$ is not in the subspace.
Without additional structure, approximation is not possible because it is not well defined
in this very general space.

Arguably the minimum structure required to perform the operation $\opA$ of approximation
in a space $\spX$
is a metric space $\opair{\spX}{\metricn}$.
In a metric space $\opair{\spX}{\metricn}$ we can define approximation of an element $\ff$ as
the element $\fy$ in $\spX$ that minimizes the ``distance" between $\fx$ and $\fy$; that is,
  \[ \mcom{\opA\fx \eqd \arg\min_{\fy} \metric{\fx}{\fy}}
          {($\opA\fx$ is the element $\fy$ that minimizes $\metric{\fx}{\fy}$ for some element $\fx$)} \]
While this is well defined, there is no way to simplify the expression, and thus
performing approximation in a metric space with no additional structure
may require an exhaustive search over the space $\spX$.

In a normed linear space $\opair{\spX}{\normn}$, the norm $\normn$ generates a metric
$\metricn$ such that $\metric{\fx}{\fy}\eqd\norm{\fx-\fy}$.
If the linear space also has a basis $\set{\fphi_n}{n\in\Z}$,
we can represent $\fy$ in terms of the basis functions as
$\fy=\sum_{n\in\Z}\alpha_n\fphi_n$.
The approximation problem then becomes
  \begin{align*}
    \opA\fx
        &\eqd \arg\min_{\fy} \metric{\fx}{\fy}
      \\&\eqd \arg\min_{\fy} \norm{\fx-\fy}
      \\&=    \arg\min_{\fy} \norm{\fx-\sum_{n\in\Z}\alpha_n\fphi_n}
  \end{align*}
But again it is difficult to proceed further without additional structure.

In an inner-product space $\opair{\spX}{\inprodn}$,
the inner-product $\inprodn$ generates a norm $\normn$ which in turn
generates a metric $\metricn$ such that
  \\\indentx$\metricsq{\fx}{\fy}\eqd\norm{\fx-\fy}^2\eqd\inprod{\fx-\fy}{\fx-\fy}$.
\\
If the linear space also has an \prope{orthogonal} basis $\set{\fphi_n}{n\in\Z}$,
then the approximation problem becomes
  \begin{align*}
    \brp{\opA\fx}^2
        &\eqd \arg\min_{\fy} \metricsq{\fx}{\fy}
      \\&\eqd \arg\min_{\fy} \norm{\fx-\fy}^2
      \\&=    \arg\min_{\fy} \norm{\fx-\sum_{n\in\Z}\alpha_n\fphi_n}^2
      \\&=    \arg\min_{\fy} \inprod{\fx-\sum_{n\in\Z}\alpha_n\fphi_n}{\fx-\sum_{m\in\Z}\alpha_m\fphi_m}
      \\&=    \arg\min_{\fy} \brp{
              \inprod{\fx}{\fx-\sum_{n\in\Z}\alpha_n\fphi_n}
              -\inprod{\sum_{n\in\Z}\alpha_n\fphi_n}{\fx-\sum_{m\in\Z}\alpha_m\fphi_m}
              }
      \\&=    \arg\min_{\fy} \brp{
              \inprod{\fx}{\fx}
              -2\inprod{\fx}{\sum_{n\in\Z}\alpha_n\fphi_n}
              +\inprod{\sum_{n\in\Z}\alpha_n\fphi_n}{\sum_{m\in\Z}\alpha_m\fphi_m}
              }
      \\&=    \arg\min_{\fy} \brp{
              \norm{\fx}^2
              -2\sum_{n\in\Z}\alpha_n\inprod{\fx}{\fphi_n}
              +\sum_{n\in\Z}\sum_{m\in\Z}\alpha_n\alpha_m\cancelto{\text{$0$ for $n\neq m$}}{\inprod{\fphi_n}{\fphi_m}}
              }
      \\&=    \arg\min_{\fy} \brp{
              \norm{\fx}^2
              -2\sum_{n\in\Z}\alpha_n\inprod{\fx}{\fphi_n}
              +\sum_{n\in\Z}\alpha^2_n\norm{\fphi_n}^2
              }
  \end{align*}
If in addition the space $\spX$ is \prope{convex}, then we can use partial derivatives
to find the optimal sequence $\seq{\alpha_n}{n\in\Z}$ that represents the optimal
approximate element $\fy$:
  \begin{align*}
    \alpha_k
      &=    \arg\min_{\alpha_k} \brp{
              \norm{\fx}^2
              -2\sum_{n\in\Z}\alpha_n\inprod{\fx}{\fphi_n}
              +\sum_{n\in\Z}\alpha^2_n\norm{\fphi_n}^2
              }
    \\&=    \arg_{\alpha_k} \brs{\pderiv{}{\alpha_k}\brp{
              \norm{\fx}^2
              -2\sum_{n\in\Z}\alpha_n\inprod{\fx}{\fphi_n}
              +\sum_{n\in\Z}\alpha^2_n\norm{\fphi_n}^2
              }=0}
    \\&=    \arg_{\alpha_k} \brp{
              \cancelto{0}{\pderiv{}{\alpha_k}\norm{\fx}^2}
              -\pderiv{}{\alpha_k}2\sum_{n\in\Z}\alpha_n\inprod{\fx}{\fphi_n}
              +\cancelto{\text{$0$ for $n\neq k$}}{\pderiv{}{\alpha_k}\sum_{n\in\Z}\alpha^2_n\norm{\fphi_n}^2}
              =0}
    \\&=    \arg_{\alpha_k} \brp{
              0
              -2\inprod{\fx}{\fphi_k}
              +\pderiv{}{\alpha_k}\alpha^2_k\norm{\fphi_k}^2
              =0}
    \\&=    \arg_{\alpha_k} \brp{
              -2\inprod{\fx}{\fphi_k}
              +2\alpha_k\norm{\fphi_k}^2
              =0}
    \\&=    \frac{1}{\norm{\fphi_k}^2}\inprod{\fx}{\fphi_k}
  \end{align*}

%\fi

\begin{minipage}{\tw-65mm}
%\psset{unit=1mm}%
For example, let $\fp(t)$ be the \hie{pulse function} such that\footnotemark
  \[ \fp(t) = \brbl{\begin{array}{lMl}
                      1  & for & 0\le t < 1 \\
                      0  & otherwise
                    \end{array}}
  \]
\end{minipage}%
\footnotetext{For a more complete example, see \prefpp{ex:wavstrct_haar_sin}.}%
\hfill%
\begin{minipage}{60mm}%
  \mbox{}\\%
  \psset{unit=1mm}%
  %============================================================================
% Daniel J. Greenhoe
% LaTeX file
%============================================================================
{\begin{pspicture}(-2,-0.5)(3,1.2)%
  \scs
  %\psset{
    %linecolor=blue,
    %linewidth=1pt,
    %dotsize=5pt,
    %dotsep=1pt,
    %}%
  \psaxes[linecolor=axis,yAxis=false,linewidth=0.75pt]{<->}(0,0)(-2,0)(3,1.2)%
  \psline[linestyle=dotted](0,0)(0,1)%
  \psline[linestyle=dotted](1,1)(1,0)%
  \psline{-o}(-1.2, 0)( 0, 0)% left horizontal
  \psline{*-o}(0,1)(1,1)% middle horizontal
  \psline{*-}(1,0)(2.2,0)% right horizontal
  \psline[linestyle=dotted](2.2,0)(2.75,0)%
  \psline[linestyle=dotted](-1.2,0)(-1.75,0)%
  \uput{3.5pt}[180](0,1){$1$}%
\end{pspicture}}
%
\end{minipage}

Let the subspace $\spV$ be the span $\spV\eqd\Span\set{\fp(t-n)}{n\in\Z}$.

\begin{minipage}{\tw-95mm}
Let the function $\ff(t)$ that we wish to approximate be $\ff(t)\eqd\sin(\pi t)$.
\end{minipage}%
\hfill%
\begin{minipage}{90mm}%
  \mbox{}\\%
  \psset{unit=1mm}%
  %============================================================================
% Daniel J. Greenhoe
% LaTeX file
% sin(t)
% nominal unit = 8mm
%============================================================================
\begin{pspicture}(-3.5,-1.5)(3.5,1.5)%
  \psaxes[linecolor=axis]{<->}(0,0)(-3.5,-1.5)(3.5,1.5)%
  \psplot[plotpoints=100]{-3}{3}{x 180 mul sin}%
  \psplot[plotpoints=10,linestyle=dotted,linewidth=2pt]{3}{3.5}{x 180 mul sin}%
  \psplot[plotpoints=10,linestyle=dotted,linewidth=2pt]{-3.5}{-3}{x 180 mul sin}%
  %\rput[b](17.5,11){$\ff(t)\eqd\sin(\pi t)$}%
  %\rput[0](35,0){$t$}%
\end{pspicture}%
\end{minipage}

Then the transform $\opR$ of $\ff(t)$ onto the subspace $\spV$ is
illustrated below.
The values of the coefficients are given by
\\
\begin{minipage}{\tw-95mm}
  \begin{align*}
    \brs{\opR\ff(t)}(n)
      &=    \frac{1}{\norm{\fp(t)}^2}\inprod{\ff(t)}{\fp(t-n)}
    \\&\eqd \frac{1}{\int_{\R} \fp^2(t)\dt}\; \int_{\R} \ff(t)\fp(t-n) \dt
    \\&=    \frac{1}{\int_0^1 1\dt}\; \int_0^1 \sin(\pi t-\pi n) \dt
    \\&=    \left. \frac{1}{\left. t\right|_0^1} \; \brp{\frac{-1}{\pi}}\cos(\pi t-\pi n) \right|_0^1
    \\&=    \cancelto{1}{\frac{1}{1-0}}\brp{\frac{-1}{\pi}}\brs{\cos(\pi -\pi n)-\cos(-\pi n)}%
    \\&=    \frac{1}{\pi}\brs{\cos(n\pi )-\cos((n-1)\pi)}%
    \\&=    \frac{2}{\pi}(-1)^n
  \end{align*}
\end{minipage}%
\hfill%
\begin{minipage}{90mm}%
  \mbox{}\\%
  \psset{unit=9mm}%
  %%============================================================================
% Daniel J. Greenhoe
% LaTeX file
% sin(t)
%============================================================================
%  \psset{unit=1mm}
\begin{pspicture}(-40,-15)(40,15)%
  \footnotesize
  \psset{linecolor=blue}%
  %\rput(0,0){% axis
  %  \psset{linecolor=axis}
  %  \multirput(-30,0)(10,0){7}{\psline(0,-1)(0,1)}% markers on x axis
  %  \psline{<->}(-35,0)(35,0)% x axis
  %  \psline{<->}(0,-15)(0,15)%    y axis
  %  \psline(-1,10)(1,10)%
  %  \psline(-1,-10)(1,-10)%
  %  \uput[180](0,10){$\frac{1}{\pi}$}% y=1
  %  \uput[0](0,-10){$\frac{-1}{\pi}$}% y=1
  %  \multido{\ival=-3+1,\ipos=-30+10}{7}{%
  %    \uput[-90](\ipos,0){$\ival$}% x=
  %    }%
  %  \uput[0](40,0){$t$}%
  %  }%
  \psaxes[linecolor=axis,unit=10,labels=x]{<->}(0,0)(-3.5,-1.5)(3.5,1.5)%
  \multirput(-20,0)(20,0){3}{\psline{-o}(0,0)(0,10)}%
  \multirput(-30,0)(20,0){4}{\psline{-o}(0,0)(0,-10)}%
  \uput[180](0,10){$\frac{2}{\pi}$}% y=1
  \uput[0](0,-10){$\frac{-2}{\pi}$}% y=1
  \uput[0](35,0){$t$}%
  \rput[b](17.5,10){$\inprod{\ff(t-n)}{\sin(\pi t)}$}%
\end{pspicture}
%
  %============================================================================
% Daniel J. Greenhoe
% LaTeX file
%
% transform of a sin(pi t) for the Haar k=0 subspace 
%
%     2
% --------- = 0.63661977236758134307553505349006
%    pi
% nominal unit = 8mm
%============================================================================
\begin{pspicture}(-3.5,-1.5)(4,1.5)%
  \psaxes[linecolor=axis,labels=none,ticks=y]{<->}(0,0)(-3.5,-1.5)(3.5,1.5)%
  \multirput(-2,0)(2,0){3}{\psline{-o}(0,0)(0,0.6366)}%
  \multirput(-3,0)(2,0){4}{\psline{-o}(0,0)(0,-0.6366)}%
  \uput[90](-3,0){$-3$}%
  \uput[-90](-2,0){$-2$}%
  \uput[90](-1,0){$-1$}%
  \uput[90](1,0){$1$}%
  \uput[-90](2,0){$2$}%
  \uput[90](3,0){$3$}%
  \uput[180](0,0.6366){$\frac{2}{\pi}$}% y=1
  \uput[0](0,-0.6366){$\frac{-2}{\pi}$}% y=1
  \uput[0](3.5,0){$n$}%
  \psplot[plotpoints=100,linestyle=dashed,linecolor=red,linewidth=1pt]{-3}{3}{x 180 mul sin}%
  %\rput[b](17.5,10){$\inprod{\ff(t-n)}{\sin(\pi t)}$}%
\end{pspicture}

\end{minipage}


And the approximation $\opA\ff(t)$ in the space $\spV$ is \\
\begin{minipage}{\tw-77mm}
  \begin{align*}
    \opA\ff(t)
      &= \sum_{n\in\Z} \brs{\opR\ff(t)}(n) \fp(t-n)
    \\&= \frac{2}{\pi}(-1)^n \fp(t-n)
  \end{align*}
\end{minipage}%
\hfill%
\begin{minipage}{72mm}%
  \mbox{}\\%
  \psset{unit=9mm}%
  %%============================================================================
% Daniel J. Greenhoe
% LaTeX file
% sin(t)
%============================================================================
%  \psset{unit=1mm}
  \begin{pspicture}(-40,-15)(40,15)%
    \footnotesize
    \psset{linecolor=blue}%
    \psaxes[linecolor=axis,unit=10,labels=x]{<->}(0,0)(-3.5,-1.5)(3.5,1.5)%
    \uput[180](0,10){$\frac{2}{\pi}$}% y=1
    \uput[0](0,-10){$\frac{-2}{\pi}$}% y=1
    \rput[r](-32,5){$\cdots$}% ... (left)
    \rput[l]( 32,5){$\cdots$}% ... (right)
    \multiput(-30,-10)(10,0){7}{\psline[linestyle=dotted,dotsep=0.5](0,0)(0,20)}% vertical dotted lines
    \multiput(-20, 10)(20,0){3}{\psline{*-o}(0,0)(10,0)}% upper horizontal lines
    \multiput(-30,-10)(20,0){3}{\psline{*-o}(0,0)(10,0)}% lower horizontal lines
    \rput(5,5){$\opA_0\ff(t)$}%
    \uput[0](35,0){$t$}%
  \end{pspicture}

  %============================================================================
% Daniel J. Greenhoe
% LaTeX file
%
% approximation of a sin(pi t) in the Haar k=0 subspace 
%
%     2
% --------- = 0.63661977236758134307553505349006
%    pi
% nominal unit = 8mm
%============================================================================
\begin{pspicture}(-3.5,-1.5)(4,1.5)%
  \psaxes[linecolor=axis,labels=x]{<->}(0,0)(-3.5,-1.5)(3.5,1.5)%
  \multirput(-2,0)(1,0){6}{\psline[linestyle=dotted](0, 0.6366)(0,-0.6366)}% dotted vertical segments
  \multirput(-2,0)(2,0){3}{\psline{*-o}(0, 0.6366)(1, 0.6366)}%
  \multirput(-3,0)(2,0){3}{\psline{*-o}(0,-0.6366)(1,-0.6366)}%
  \uput[180](0,0.6366){$\frac{2}{\pi}$}% y=1
  \uput[0](0,-0.6366){$\frac{-2}{\pi}$}% y=1
  \uput[0](3.5,0){$x$}%
  \psplot[plotpoints=100,linestyle=dashed,linecolor=red,linewidth=1pt]{-3}{3}{x 180 mul sin}%
  %\rput[b](17.5,10){$\inprod{\ff(t-n)}{\sin(\pi t)}$}%
\end{pspicture}
%
\end{minipage}

%\fi

%%=======================================
%\section{Representations of Transforms}
%%=======================================
%On \prefpo{item:wavstrct_T} a transform is defined as a sequence of projection operators.
%Besides this representation, there are two additional representations.
%Here then are three useful representations for transforms.
%\\\begin{tabular}{@{\qquad}llll}
%    \circOne   & a sequence of projection operators  & $\ds\seq{\opP_n}{n\in\Z}$
%  \\\circTwo   & a sequence of subspaces             & $\ds\seq{\spV_n}{n\in\Z}$  & where $\spV_n\eqd\opP_n\spX$
%  \\\circThree & a sequence of bases vectors         & $\ds\seq{\fphi_{nm}}{n,m\in\Z}$ & where $\seq{\fphi_{nm}}{m\in\Z}$ spans the subspace $\spV_n$
%\end{tabular}
%These transform representations are isomorphic to each other
%(\prefpp{thm:VPb_isomorphic}), and hence can be used interchangeably.

%\paragraph{Purpose of transforms.}
%The purpose of a transform is to analyze the objects contained in a space.
%This space of objects is the \hie{domain} of the transform.
%The advantages of the analysis might include applications such as
%approximation, signal processing, compression, etc.


%\fi