%============================================================================
% Daniel J. Greenhoe
% XeLaTeX file
% stochastic systems
%============================================================================

%=======================================
\section{Operations on outcome subspaces}
%=======================================
\begin{tabstr}{0.75}
%=======================================
\subsection{Summation}
%=======================================
\begin{figure}[h]%
  \gsize%
  \centering%
  %$\mcom{\begin{array}{N}{\input{../common/math/graphics/ocs/ocs_rdie.tex}}\end{array}}{\structe{real die} \xref{ex:realdie}}$
  $\mcom{\begin{array}{N}{\includegraphics{../common/math/graphics/pdfs/ocs_rdie.pdf}}\end{array}}{\structe{real die} \xref{ex:realdie}}$
  \begin{tabular}{c} \Huge$+$                                         \end{tabular}
  %$\mcom{\begin{array}{N}{\input{../common/math/graphics/ocs/ocs_rdie.tex}}\end{array}}{\structe{real die} \xref{ex:realdie}}$
  $\mcom{\begin{array}{N}{\includegraphics{../common/math/graphics/pdfs/ocs_rdie.pdf}}\end{array}}{\structe{real die} \xref{ex:realdie}}$
  %\begin{tabular}{c}{\input{../common/math/graphics/ocs/ocs_rdie.tex}} \end{tabular}
  \begin{tabular}{c} \Huge$=$                                         \end{tabular}
  %\begin{tabular}{c}{%============================================================================
% Daniel J. Greenhoe
% LaTeX file
% pair of dice
%============================================================================
{\psset{unit=1.5\psunit}%
\begin{pspicture}(-2.2,-1.4)(2.2,1.4)%
  %---------------------------------
  % options
  %---------------------------------
  \psset{%
    radius=1.25ex,
    linecolor=blue,%
    labelsep=2.5mm,
    }%
  %---------------------------------
  % node locations
  %---------------------------------
  \Cnode[fillstyle=solid,fillcolor=snode](0,0){D7}%
  \Cnode(-0.9239,0.3827){D4}\Cnode(-0.9239,-0.3827){D5}\Cnode(0.9239,-0.3827){D9}\Cnode(0.9239,0.3827){D10}%
  \Cnode(-0.3827,0.9239){D3}\Cnode(-0.3827,-0.9239){D6}\Cnode(0.3827,-0.9239){D8}\Cnode(0.3827,0.9239){D11}%
  \Cnode(-2,0){D2}\Cnode(2,0){D12}%
  %---------------------------------
  % edges
  %---------------------------------
  \ncline{D11}{D12}%
  \ncline{D10}{D11}\ncline{D10}{D12}%
  \ncline{D9}{D10}\ncline{D9}{D11}\ncline{D9}{D12}%
  \ncline{D8}{D9}\ncline{D8}{D10}\ncline{D8}{D11}\ncline{D8}{D12}%
  \ncline{D7}{D8}\ncline{D7}{D9}\ncline{D7}{D10}\ncline{D7}{D11}%
  \ncline{D6}{D7}\ncline{D6}{D8}\ncline{D6}{D9}\ncline{D6}{D10}%
  \ncline{D5}{D6}\ncline{D5}{D7}\ncline{D5}{D8}\ncline{D5}{D9}%
  \ncline{D4}{D5}\ncline{D4}{D6}\ncline{D4}{D7}\ncline{D4}{D8}%
  \ncline{D3}{D4}\ncline{D3}{D5}\ncline{D3}{D6}\ncline{D3}{D7}%
  \ncline{D2}{D3}\ncline{D2}{D4}\ncline{D2}{D5}\ncline{D2}{D6}%
  %---------------------------------
  % node labels
  %---------------------------------
  \rput(D12){$12$}%
  \rput(D11){$11$}%
  \rput(D10){$10$}%
  \rput(D9){$9$}%
  \rput(D8){$8$}%
  \rput(D7){$7$}%
  \rput(D6){$6$}%
  \rput(D5){$5$}%
  \rput(D4){$4$}%
  \rput(D3){$3$}%
  \rput(D2){$2$}%
  %---------------------------------
  % probability labels
  %---------------------------------
  \uput[-90](D2){$\frac{1}{36}$}\uput[-90](D12){$\frac{1}{36}$}%
  \uput[135](D3){$\frac{2}{36}$}\uput[45](D11){$\frac{2}{36}$}%
  \uput[  22](D4){$\frac{3}{36}$}\uput[168](D10){$\frac{3}{36}$}%
  \uput[ 135](D5){$\frac{4}{36}$}\uput[45](D9){$\frac{4}{36}$}%
  \uput[-135](D6){$\frac{5}{36}$}\uput[-45](D8){$\frac{5}{36}$}%
  \uput[90](D7){$\frac{6}{36}$}%
  %
  %---------------------------------
  % labels
  %---------------------------------
  %\rput(6.5,1.25){$\rvX(\cdot)$}%
  %\ncline[linestyle=dotted,nodesep=1pt]{->}{xzlabel}{xz}%
  %\ncline[linestyle=dotted,nodesep=1pt]{->}{ylabel}{y}%
\end{pspicture}
}%}\end{tabular}
  \begin{tabular}{c}{\includegraphics{../common/math/graphics/pdfs/pairdice.pdf}}\end{tabular}
    \begin{tabstr}{0.75}%
    $\begin{array}{|r|*{11}{c}|}%
       \hline
         & 2 & 3 & 4 & 5 & 6 & 7 & 8 & 9 & 10 & 11 & 12\\
       \hline
       2 & 0 & 1 & 1 & 1 & 1 & 2 & 2 & 2 & 2  & 3  & 4 \\
       3 & 1 & 0 & 1 & 1 & 1 & 1 & 2 & 2 & 2  & 2  & 3 \\
       4 & 1 & 1 & 0 & 1 & 1 & 1 & 1 & 2 & 2  & 2  & 2 \\
       5 & 1 & 1 & 1 & 0 & 1 & 1 & 1 & 1 & 2  & 2  & 2 \\
       6 & 1 & 1 & 1 & 1 & 0 & 1 & 1 & 1 & 1  & 2  & 2 \\
       7 & 2 & 1 & 1 & 1 & 1 & 0 & 1 & 1 & 1  & 1  & 2 \\
       8 & 2 & 2 & 1 & 1 & 1 & 1 & 0 & 1 & 1  & 1  & 1 \\
       9 & 2 & 2 & 2 & 1 & 1 & 1 & 1 & 0 & 1  & 1  & 1 \\
      10 & 2 & 2 & 2 & 2 & 1 & 1 & 1 & 1 & 0  & 1  & 1 \\
      11 & 3 & 2 & 2 & 2 & 2 & 1 & 1 & 1 & 1  & 0  & 1 \\
      12 & 4 & 3 & 2 & 2 & 2 & 2 & 1 & 1 & 1  & 1  & 0 \\
      \hline
    \end{array}$%
    \end{tabstr}%
  \caption{metrics based on number of die edges for a \structe{pair of real dice} \xref{ex:pairdice}\label{fig:pairdice}}
\end{figure}
\footnotetext{Many many thanks to Katie L. Greenhoe and Jonathan J. Greenhoe for help computing these values.}
%---------------------------------------
\begin{example}[\exmd{pair of dice outcome subspace}]
\label{ex:pairdice}
%---------------------------------------
A \structe{pair of real dice} has a structure as illustrated in \prefpp{fig:pairdice}.
The values represent the standard sum of die faces and thus range from $2$ to $12$.
The table in the figure provides the metric distances between summed values based on the number of edges
that must be transversed to move from the first value to the second value.
Alternatively, the distance is the number of times the dice must be rotated 90 degrees to move from the first 
value being face up to the second value being face up.
This structure is also illustrated in the undirected graph on the right in \pref{fig:pairdice},
along with each value's standard probability.
\end{example}
\begin{proof}
\begin{align*}
  \ocscen(\ocsG)
    &\eqd \argmin_{x\in\ocsG}\max_{y\in\ocsG}\metric{x}{y}\psp(y)
    &&\text{by definition of $\ocscen$ \xref{def:ocscen}}
   \\&=\mathrlap{\argmin_{x\in\ocsG}\max_{y\in\ocsG}
         \setn{\begin{array}{cccccc}
           \metricn(2,2)\psp(2) &\metricn(2,3)\psp(3)  &\metricn(2,4)\psp(4)  &\cdots & \metricn(2,12)\psp(12)\\
           \metricn(3,2)\psp(2) &\metricn(3,3)\psp(3)  &\metricn(3,4)\psp(4)  &\cdots & \metricn(3,12)\psp(12)\\
           \vdots               &\vdots                &\vdots                &\ddots & \vdots                \\
           \metricn(12,2)\psp(2)&\metricn(12,3)\psp(3) &\metricn(12,4)\psp(4) &\cdots & \metricn(12,12)\psp(12)
         \end{array}}}
      \\&=\mathrlap{\argmin_{x\in\ocsG}\max_{y\in\ocsG}\frac{1}{36}
         \setn{\begin{array}{*{12}{c}}
            %  2          3           4           5           6           7           8           9           10         11          12
           0\times1 & 1\times2 & 1\times3 & 1\times4 & 1\times5 & 2\times6 & 2\times5 & 2\times4 & 2\times3 & 3\times2 & 4\times1 \\
           1\times1 & 0\times2 & 1\times3 & 1\times4 & 1\times5 & 1\times6 & 2\times5 & 2\times4 & 2\times3 & 2\times2 & 3\times1 \\
           1\times1 & 1\times2 & 0\times3 & 1\times4 & 1\times5 & 1\times6 & 1\times5 & 2\times4 & 2\times3 & 2\times2 & 2\times1 \\
           1\times1 & 1\times2 & 1\times3 & 0\times4 & 1\times5 & 1\times6 & 1\times5 & 1\times4 & 2\times3 & 2\times2 & 2\times1 \\
           1\times1 & 1\times2 & 1\times3 & 1\times4 & 0\times5 & 1\times6 & 1\times5 & 1\times4 & 1\times3 & 2\times2 & 2\times1 \\
           2\times1 & 1\times2 & 1\times3 & 1\times4 & 1\times5 & 0\times6 & 1\times5 & 1\times4 & 1\times3 & 1\times2 & 2\times1 \\
           2\times1 & 2\times2 & 1\times3 & 1\times4 & 1\times5 & 1\times6 & 0\times5 & 1\times4 & 1\times3 & 1\times2 & 1\times1 \\
           2\times1 & 2\times2 & 2\times3 & 1\times4 & 1\times5 & 1\times6 & 1\times5 & 0\times4 & 1\times3 & 1\times2 & 1\times1 \\
           2\times1 & 2\times2 & 2\times3 & 2\times4 & 1\times5 & 1\times6 & 1\times5 & 1\times4 & 0\times3 & 1\times2 & 1\times1 \\
           3\times1 & 2\times2 & 2\times3 & 2\times4 & 2\times5 & 1\times6 & 1\times5 & 1\times4 & 1\times3 & 0\times2 & 1\times1 \\
           4\times1 & 3\times2 & 2\times3 & 2\times4 & 2\times5 & 2\times6 & 1\times5 & 1\times4 & 1\times3 & 1\times2 & 0\times1 \\
         \end{array}}}
     \\&=\mathrlap{\argmin_{x\in\ocsG}\max_{y\in\ocsG}\frac{1}{36}
         \setn{\begin{array}{*{12}{c}}
           0 & 2 & 3 & 4 & 5 &12 &10 & 8 & 6 & 6 & 4 \\
           1 & 0 & 3 & 4 & 5 & 6 &10 & 8 & 6 & 4 & 3 \\
           1 & 2 & 0 & 4 & 5 & 6 & 5 & 8 & 6 & 4 & 2 \\
           1 & 2 & 3 & 0 & 5 & 6 & 5 & 4 & 6 & 4 & 2 \\
           1 & 2 & 3 & 4 & 0 & 6 & 5 & 4 & 3 & 4 & 2 \\
           2 & 2 & 3 & 4 & 5 & 0 & 5 & 4 & 3 & 2 & 2 \\
           2 & 4 & 3 & 4 & 5 & 6 & 0 & 4 & 3 & 2 & 1 \\
           2 & 4 & 6 & 4 & 5 & 6 & 5 & 0 & 3 & 2 & 1 \\
           2 & 4 & 6 & 8 & 5 & 6 & 5 & 4 & 0 & 2 & 1 \\
           3 & 4 & 6 & 8 &10 & 6 & 5 & 4 & 3 & 0 & 1 \\
           4 & 6 & 6 & 8 &10 &12 & 5 & 4 & 3 & 2 & 0 \\
         \end{array}}
       = \argmin_{x\in\ocsG}\frac{1}{36}
         \setn{\begin{array}{c}
           12\\% 2
           10\\% 3
            8\\% 4
            6\\% 5
            6\\% 6
            5\\% 7
            6\\% 8
            6\\% 9
            8\\%10
           10\\%11
           12\\%12
         \end{array}}
  \quad= \setn{7}}
  \\
  \ocscena(\ocsG)
    &\eqd \argmin_{x\in\ocsG}\sum_{y\in\ocsG}\metric{x}{y}\psp(y)
    &&\text{by definition of $\ocscena$ \xref{def:ocscenx}}
     \\&=\mathrlap{\argmin_{x\in\ocsG}\frac{1}{36}
         \setn{\begin{array}{*{23}{@{\,}c}}
           0 &+& 2 &+& 3 &+& 4 &+& 5 &+&12 &+&10 &+& 8 &+& 6 &+& 6 &+& 4 \\%  2
           1 &+& 0 &+& 3 &+& 4 &+& 5 &+& 6 &+&10 &+& 8 &+& 6 &+& 4 &+& 3 \\%  3
           1 &+& 2 &+& 0 &+& 4 &+& 5 &+& 6 &+& 5 &+& 8 &+& 6 &+& 4 &+& 2 \\%  4
           1 &+& 2 &+& 3 &+& 0 &+& 5 &+& 6 &+& 5 &+& 4 &+& 6 &+& 4 &+& 2 \\%  5
           1 &+& 2 &+& 3 &+& 4 &+& 0 &+& 6 &+& 5 &+& 4 &+& 3 &+& 4 &+& 2 \\%  6
           2 &+& 2 &+& 3 &+& 4 &+& 5 &+& 0 &+& 5 &+& 4 &+& 3 &+& 2 &+& 2 \\%  7
           2 &+& 4 &+& 3 &+& 4 &+& 5 &+& 6 &+& 0 &+& 4 &+& 3 &+& 2 &+& 1 \\%  8
           2 &+& 4 &+& 6 &+& 4 &+& 5 &+& 6 &+& 5 &+& 0 &+& 3 &+& 2 &+& 1 \\%  9
           2 &+& 4 &+& 6 &+& 8 &+& 5 &+& 6 &+& 5 &+& 4 &+& 0 &+& 2 &+& 1 \\% 10
           3 &+& 4 &+& 6 &+& 8 &+&10 &+& 6 &+& 5 &+& 4 &+& 3 &+& 0 &+& 1 \\% 11
           4 &+& 6 &+& 6 &+& 8 &+&10 &+&12 &+& 5 &+& 4 &+& 3 &+& 2 &+& 0 \\% 12
         \end{array}}
       = \argmin_{x\in\ocsG}\frac{1}{36}
         \setn{\begin{array}{c}
            60\\%  2
            50\\%  3
            43\\%  4
            38\\%  5
            34\\%  6
            32\\%  7
            34\\%  8
            38\\%  9
            43\\% 10
            50\\% 11
            60\\% 12
         \end{array}}
       \quad= \setn{7}}
  \\
  \ocsVar(\ocsG)
    &\eqd \sum_{x\in\ocsG}\brs{\metric{\ocsE(\ocsG)}{x}}^2\psp(x)
    &&\text{by definition of $\ocsVar$ \xref{def:ocsVar}}
  \\&= \sum_{x\in\ocsG}\brs{\metric{7}{x}}^2\psp(x)
    &&\text{by definition of $\ocsVar$ \xref{def:ocsVar}}
  \\&=\mathrlap{\frac{1}{36}\brp{2^2\times1 + 1^2\times2 + 1^2\times3 + 1^2\times4 + 1^2\times5 + 0^2\times6 +
                        1^2\times5 + 1^2\times4 + 1^2\times3 + 1^2\times2 + 2^2\times1}}
  \\&= \frac{1}{36}\brp{4+2+3+4+5+0+5+4+3+2+4}
  \\&= \frac{36}{36}
  \\&= 1
\end{align*}
\end{proof}




The next two examples are examples of sums of \structe{outcome subspace}s \xref{def:ocs}:
\prefp{ex:dicepair_moments} (sum of dice pair) and \prefp{ex:pspinner_xy} (sum of spinner pair).

\begin{figure}[h]
  \gsize%
  \centering%
  %{%============================================================================
% Daniel J. Greenhoe
% LaTeX file
% pair of dice with isomorphic and linear mappings
%============================================================================
\begin{pspicture}(-5.2,-2.5)(11.3,2.7)%
  %---------------------------------
  % options
  %---------------------------------
  \psset{%
    radius=1.25ex,
    linecolor=blue,%
    labelsep=2.5mm,
    }%
  %---------------------------------
  % pair of dice outcome space
  %---------------------------------
  \rput(0,0){\psset{unit=1.5\psunit}%
    \Cnode[fillstyle=solid,fillcolor=snode](0,0){D7}%
    \Cnode(-0.9239,0.3827){D4}\Cnode(-0.9239,-0.3827){D5}\Cnode(0.9239,-0.3827){D9}\Cnode(0.9239,0.3827){D10}%
    \Cnode(-0.3827,0.9239){D3}\Cnode(-0.3827,-0.9239){D6}\Cnode(0.3827,-0.9239){D8}\Cnode(0.3827,0.9239){D11}%
    \Cnode(-2,0){D2}\Cnode(2,0){D12}%
    }%
  \ncline{D11}{D12}%
  \ncline{D10}{D11}\ncline{D10}{D12}%
  \ncline{D9}{D10}\ncline{D9}{D11}\ncline{D9}{D12}%
  \ncline{D8}{D9}\ncline{D8}{D10}\ncline{D8}{D11}\ncline{D8}{D12}%
  \ncline{D7}{D8}\ncline{D7}{D9}\ncline{D7}{D10}\ncline{D7}{D11}%
  \ncline{D6}{D7}\ncline{D6}{D8}\ncline{D6}{D9}\ncline{D6}{D10}%
  \ncline{D5}{D6}\ncline{D5}{D7}\ncline{D5}{D8}\ncline{D5}{D9}%
  \ncline{D4}{D5}\ncline{D4}{D6}\ncline{D4}{D7}\ncline{D4}{D8}%
  \ncline{D3}{D4}\ncline{D3}{D5}\ncline{D3}{D6}\ncline{D3}{D7}%
  \ncline{D2}{D3}\ncline{D2}{D4}\ncline{D2}{D5}\ncline{D2}{D6}%
  %
  \rput(D12){$12$}%
  \rput(D11){$11$}%
  \rput(D10){$10$}%
  \rput(D9){$9$}%
  \rput(D8){$8$}%
  \rput(D7){$7$}%
  \rput(D6){$6$}%
  \rput(D5){$5$}%
  \rput(D4){$4$}%
  \rput(D3){$3$}%
  \rput(D2){$2$}%
  %
  \uput[-90](D2){$\frac{1}{36}$}\uput[-90](D12){$\frac{1}{36}$}%
  \uput[135](D3){$\frac{2}{36}$}\uput[45](D11){$\frac{2}{36}$}%
  \uput[  22](D4){$\frac{3}{36}$}\uput[168](D10){$\frac{3}{36}$}%
  \uput[ 135](D5){$\frac{4}{36}$}\uput[45](D9){$\frac{4}{36}$}%
  \uput[-135](D6){$\frac{5}{36}$}\uput[-45](D8){$\frac{5}{36}$}%
  \uput[90](D7){$\frac{6}{36}$}%
  %
  %---------------------------------
  % random variable mapping X to real line
  %---------------------------------
  \rput(-4.5,-2.8){\psset{unit=0.4\psunit}%
    \pnode(0,13){XB}%
    \pnode(0,12){X12}%
    \pnode(0,11){X11}%
    \pnode(0,10){X10}%
    \pnode(0, 9){X9}%
    \pnode(0, 8){X8}%
    \pnode(0, 7){X7}%
    \pnode(0, 6){X6}%
    \pnode(0, 5){X5}%
    \pnode(0, 4){X4}%
    \pnode(0, 3){X3}%
    \pnode(0, 2){X2}%
    \pnode(0, 1){XA}%
    }
  %\rput(4,-3){\psset{unit=0.5\psunit}%
  %  \pnode( 6,0){XB}%
  %  \pnode( 5,0){X12}%
  %  \pnode( 4,0){X11}%
  %  \pnode( 3,0){X10}%
  %  \pnode( 2,0){X9}%
  %  \pnode( 1,0){X8}%
  %  \pnode( 0,0){X7}%
  %  \pnode(-1,0){X6}%
  %  \pnode(-2,0){X5}%
  %  \pnode(-3,0){X4}%
  %  \pnode(-4,0){X3}%
  %  \pnode(-5,0){X2}%
  %  \pnode(-6,0){XA}%
  %  }
  \ncline{->}{X12}{XB}
  \ncline{X2}{X12}
  \ncline{->}{X2}{XA}
  %
  \pscircle[fillstyle=solid,linecolor=snode,fillcolor=snode](X7){1ex}%
  \pscircle[fillstyle=none,linecolor=red,fillcolor=red](X7){1ex}%
  \rput(X12){\psline(-0.15,0)(0.15,0)}%
  \rput(X11){\psline(-0.15,0)(0.15,0)}%
  \rput(X10){\psline(-0.15,0)(0.15,0)}%
  \rput(X9) {\psline(-0.15,0)(0.15,0)}%
  \rput(X8) {\psline(-0.15,0)(0.15,0)}%
  \rput(X7) {\psline(-0.15,0)(0.15,0)}%
  \rput(X6) {\psline(-0.15,0)(0.15,0)}%
  \rput(X5) {\psline(-0.15,0)(0.15,0)}%
  \rput(X4) {\psline(-0.15,0)(0.15,0)}%
  \rput(X3) {\psline(-0.15,0)(0.15,0)}%
  \rput(X2) {\psline(-0.15,0)(0.15,0)}%
  %
  \uput[180](X12){$12$}%
  \uput[180](X11){$11$}%
  \uput[180](X10){$10$}%
  \uput[180](X9){$9$}%
  \uput[180](X8){$8$}%
  \uput[180](X7){$7$}%
  \uput[180](X6){$6$}%
  \uput[180](X5){$5$}%
  \uput[180](X4){$4$}%
  \uput[180](X3){$3$}%
  \uput[180](X2){$2$}%
  \uput[-22](XB){$\omsR$}%
  %
  %\uput[-90](X2){$\frac{1}{36}$}\uput[-90](X12){$\frac{1}{36}$}%
  %\uput[135](X3){$\frac{2}{36}$}\uput[45](X11){$\frac{2}{36}$}%
  %\uput[  22](X4){$\frac{3}{36}$}\uput[168](X10){$\frac{3}{36}$}%
  %\uput[ 135](X5){$\frac{4}{36}$}\uput[45](X9){$\frac{4}{36}$}%
  %\uput[-135](X6){$\frac{5}{36}$}\uput[-45](X8){$\frac{5}{36}$}%
  %\uput[90](X7){$\frac{6}{36}$}%
  %
  \ncarc[arcangle=22,linewidth=0.75pt,linecolor=red]{->}{D2}{X2}%
  \ncarc[arcangle=22,linewidth=0.75pt,linecolor=red]{->}{D3}{X3}%
  \ncarc[arcangle=22,linewidth=0.75pt,linecolor=red]{->}{D4}{X4}%
  \ncarc[arcangle=22,linewidth=0.75pt,linecolor=red]{->}{D5}{X5}%
  \ncarc[arcangle=22,linewidth=0.75pt,linecolor=red]{->}{D6}{X6}%
  \ncarc[arcangle=22,linewidth=0.75pt,linecolor=red]{->}{D7}{X7}%
  \ncarc[arcangle=45,linewidth=0.75pt,linecolor=red]{->}{D8}{X8}%
  \ncarc[arcangle=-45,linewidth=0.75pt,linecolor=red]{->}{D9}{X9}%
  \ncarc[arcangle=-11,linewidth=0.75pt,linecolor=red]{->}{D10}{X10}%
  \ncarc[arcangle=-22,linewidth=0.75pt,linecolor=red]{->}{D11}{X11}%
  \ncarc[arcangle=-45,linewidth=0.75pt,linecolor=red]{->}{D12}{X12}%
  %---------------------------------
  % random variable mapping Y to isomorphic outcome space
  %---------------------------------
  \rput(8,0){\psset{unit=1.5\psunit}%
    \Cnode[fillstyle=solid,fillcolor=snode](0,0){Y7}%
    \Cnode(-0.9239,0.3827){Y4}\Cnode(-0.9239,-0.3827){Y5}\Cnode(0.9239,-0.3827){Y9}\Cnode(0.9239,0.3827){Y10}%
    \Cnode(-0.3827,0.9239){Y3}\Cnode(-0.3827,-0.9239){Y6}\Cnode(0.3827,-0.9239){Y8}\Cnode(0.3827,0.9239){Y11}%
    \Cnode(-2,0){Y2}\Cnode(2,0){Y12}%
    }
  \ncline{Y11}{Y12}%
  \ncline{Y10}{Y11}\ncline{Y10}{Y12}%
  \ncline{Y9}{Y10}\ncline{Y9}{Y11}\ncline{Y9}{Y12}%
  \ncline{Y8}{Y9}\ncline{Y8}{Y10}\ncline{Y8}{Y11}\ncline{Y8}{Y12}%
  \ncline{Y7}{Y8}\ncline{Y7}{Y9}\ncline{Y7}{Y10}\ncline{Y7}{Y11}%
  \ncline{Y6}{Y7}\ncline{Y6}{Y8}\ncline{Y6}{Y9}\ncline{Y6}{Y10}%
  \ncline{Y5}{Y6}\ncline{Y5}{Y7}\ncline{Y5}{Y8}\ncline{Y5}{Y9}%
  \ncline{Y4}{Y5}\ncline{Y4}{Y6}\ncline{Y4}{Y7}\ncline{Y4}{Y8}%
  \ncline{Y3}{Y4}\ncline{Y3}{Y5}\ncline{Y3}{Y6}\ncline{Y3}{Y7}%
  \ncline{Y2}{Y3}\ncline{Y2}{Y4}\ncline{Y2}{Y5}\ncline{Y2}{Y6}%
  %
  \rput(Y12){$12$}%
  \rput(Y11){$11$}%
  \rput(Y10){$10$}%
  \rput(Y9){$9$}%
  \rput(Y8){$8$}%
  \rput(Y7){$7$}%
  \rput(Y6){$6$}%
  \rput(Y5){$5$}%
  \rput(Y4){$4$}%
  \rput(Y3){$3$}%
  \rput(Y2){$2$}%
  %
  %\uput[-90](Y2){$\frac{1}{36}$}\uput[-90](Y12){$\frac{1}{36}$}%
  %\uput[135](Y3){$\frac{2}{36}$}\uput[45](Y11){$\frac{2}{36}$}%
  %\uput[  22](Y4){$\frac{3}{36}$}\uput[168](Y10){$\frac{3}{36}$}%
  %\uput[ 135](Y5){$\frac{4}{36}$}\uput[45](Y9){$\frac{4}{36}$}%
  %\uput[-135](Y6){$\frac{5}{36}$}\uput[-45](Y8){$\frac{5}{36}$}%
  %\uput[90](Y7){$\frac{6}{36}$}%
  %
  \ncarc[arcangle=67,linewidth=0.75pt,linecolor=green]{->}{D2}{Y2}%
  \ncarc[arcangle=22,linewidth=0.75pt,linecolor=green]{->}{D3}{Y3}%
  \ncarc[arcangle=22,linewidth=0.75pt,linecolor=green]{->}{D4}{Y4}%
  \ncarc[arcangle=-22,linewidth=0.75pt,linecolor=green]{->}{D5}{Y5}%
  \ncarc[arcangle=-22,linewidth=0.75pt,linecolor=green]{->}{D6}{Y6}%
  \ncarc[arcangle=11,linewidth=0.75pt,linecolor=green]{->}{D7}{Y7}%
  \ncarc[arcangle=-22,linewidth=0.75pt,linecolor=green]{->}{D8}{Y8}%
  \ncarc[arcangle=-22,linewidth=0.75pt,linecolor=green]{->}{D9}{Y9}%
  \ncarc[arcangle=22,linewidth=0.75pt,linecolor=green]{->}{D10}{Y10}%
  \ncarc[arcangle=22,linewidth=0.75pt,linecolor=green]{->}{D11}{Y11}%
  \ncarc[arcangle=-67,linewidth=0.75pt,linecolor=green]{->}{D12}{Y12}%
  %---------------------------------
  % labels
  %---------------------------------
  \rput[t](0,2.3){$\ocsG$}%
  \rput[t](8,2.3){$\ocsH$}%
  \rput(-3.8,0){$\rvX(\cdot)$}%
  \rput( 4,0){$\rvY(\cdot)$}%
  %\ncline[linestyle=dotted,nodesep=1pt]{->}{xzlabel}{xz}%
  %\ncline[linestyle=dotted,nodesep=1pt]{->}{ylabel}{y}%
\end{pspicture}%}%
  {\includegraphics{../common/math/graphics/pdfs/pdice_xy.pdf}}%
  \caption{\structe{pair of dice} mappings \xref{ex:wdie_xy}\label{fig:pdice_xy}}
\end{figure}
%---------------------------------------
\begin{example}[\exmd{pair of dice and hypothesis testing}]
\label{ex:dicepair_moments}
%---------------------------------------
Let $\ocsG$ be the \structe{pair of dice outcome subspace} \xref{ex:pairdice},
    $\rvX\in\clOCSgr$ an \fncte{outcome random variable} mapping from $\ocsG$ to the \structe{real line} \xref{def:Rline},
and $\rvX\in\clOCSgh$ a mapping to a structure $\omsH$ that is \prope{isomorphic} to $\ocsG$,
as illustrated in \prefpp{fig:pdice_xy}.
This yields the following statistics:
\\$\begin{array}{>{\footnotesize}Mlcc lcccl}
  geometry of $\ocsG$:                                          & \ocscen(\ocsG)=\ocscena(\ocsG) &=& \setn{7}     & \ocsVaro(\ocsG) &=& 1\\
  traditional statistics on real line $\omsR$:                  & \pE  (\rvX)     &=& 7            & \ocsVar(\rvW;\pE)   &=& \frac{35}{6}&\approx& 5.833\\
  outcome subspace statistics on real line $\omsR$:             & \ocsE(\rvX)     &=& \setn{7}     & \ocsVar(\rvW;\ocsE) &=& \frac{35}{6}&\approx& 5.833\\
  outcome subspace statistics on  isomorphic structure $\ocsH$: & \ocsE(\rvY)     &=& \setn{7}     & \ocsVar(\rvY;\ocsE) &=& 1           &       &
\end{array}$  
\\
Although the expected values of both \structe{outcome subspace}s are the essentially the same ($7$ and $\setn{7}$), 
the isomorphic structure $\omsH$ yields a much smaller variance (a much smaller expected error).
This is significant in statistical applications such as hypothesis testing. 
Suppose for example we have two pair of \structe{real dice} \xref{ex:realdie}, 
one pair being made of two uniformly distributed die and one pair of weighted die. 
We want to know which pair is the uniform die.
So we roll each pair one time. Suppose the outcome of the first pair is $11$ and the 
the outcome of the second pair is $6$. Which pair is more likely to be the uniform pair?
Using traditional statistical analysis, the answer is the second pair, because it is closer to the 
expected value ($0.414$ standard deviations as opposed to $1.656$ standard deviations).
However, this result is deceptive, because as can be seen in \prefpp{fig:pairdice},
the distance from the expected value to the values $11$ and $6$ are the same ($\metric{7}{11}=\metric{7}{6}=1 = 1$ standard deviation).
So arguably the outcome of the single roll test would contribute nothing to a good decision algorithm.
\end{example}
\begin{proof}
    \begin{align*}
      \ocscen (\ocsG) &= \setn{7}  && \text{by \prefpp{ex:pairdice}}
      \\
      \ocsVar (\ocsG) &= 1         && \text{by \prefpp{ex:pairdice}}
      \\
      \pE(\rvX)
        &\eqd\int_{x\in\R} x\psp(x)\dx
        && \text{by definition of $\pE$ \xref{def:pE}}
      \\&\eqd\frac{1}{36}\sum_{x\in\Z} x36\psp(x)
        && \text{by definition of $\psp$}
      %\\&= \frac{1}{36}\sum_{x\in\R} x 36\psp(x)
      \\&= \mathrlap{\frac{1}{36}\brp{2\times1 + 3\times2 + 4\times3 +5\times4 + 6\times5 + 7\times6 + 
                            8\times5 + 9\times4 + 10\times3 + 11\times2 + 12\times1}}
      \\&= \frac{252}{36}
         = 7
      \\
      \ocsVar(\rvX;\pE)
        &= \pVar(\rvX)
        && \text{by \prefp{thm:ocsVar}}
      \\&= \int_{x\in\R} \brs{x-\pE(\rvX)}^2 \psp(x)\dx
        &&\text{by definition of $\pVar$ \xref{def:pVar}}
      \\&= \sum_{x\in\Z} \brs{x-\pE(\rvX)}^2 \psp(x)
        &&\text{by definition of $\psp$}
      \\&= \sum_{x\in\Z} \brp{x-7}^2 \frac{1}{36}
        &&\text{by $\pE(\rvX)$ result}
      \\&= \mathrlap{2\frac{25\times1 + 16\times2 + 9\times3 + 4\times4 + 1\times 5}{36}
         = \frac{35}{6} \approx 5.833}
      \\
      \ocsE(\rvX)
        &= \pE(\rvX)
        && \text{by \prefpp{thm:pEocsE}}
      \\&= \setn{7}
        && \text{by $\pE(\rvX)$ result}
      \\
      \ocsVar(\rvX;\ocsE)
        &= \ocsVar(\rvX;\pE)
        && \text{because $\ocsE(\rvX)\equiv\pE(\rvX)$}
      \\&= \frac{35}{6} \approx 5.833
        && \text{by $\ocsVar(\rvX;\pE)$ result}
      \\
      \ocsE(\rvY)
        &= \rvY\brs{\ocscen(\ocsG)}
        && \text{because $\ocsG$ and $\omsH$ are \prope{isometric} under $\rvY$}
      \\&= \rvY\brs{\setn{7}}
        && \text{by $\ocscen(\ocsG)$ result}
      \\&= \setn{7}
        && \text{by definition of $\rvY$}
      \\
      \ocsVar(\rvY;\ocsE)
        &= \ocsVaro(\ocsG)
        && \text{because $\ocsG$ and $\omsH$ are \prope{isometric} under $\rvY$}
      \\&= 1
        && \text{by $\ocsVaro(\ocsG)$ result}
  \end{align*}
\end{proof}


\begin{figure}[h]
  \gsize%
  \centering%
  %$\mcom{\begin{array}{N}{\input{../common/math/graphics/ocs/ocs_spinner.tex}}\end{array}}{\structe{spinner} \xref{ex:spinner}}$
  $\mcom{\begin{array}{N}{\includegraphics{../common/math/graphics/pdfs/ocs_spinner.pdf}}\end{array}}{\structe{spinner} \xref{ex:spinner}}$
  \begin{tabular}{c} \Huge$+$                                         \end{tabular}
  %$\mcom{\begin{array}{N}{\input{../common/math/graphics/ocs/ocs_spinner.tex}}\end{array}}{\structe{spinner} \xref{ex:spinner}}$
  $\mcom{\begin{array}{N}{\includegraphics{../common/math/graphics/pdfs/ocs_spinner.pdf}}\end{array}}{\structe{spinner} \xref{ex:spinner}}$
  %\begin{tabular}{c}{\input{../common/math/graphics/ocs/ocs_rdie.tex}} \end{tabular}
  \begin{tabular}{c} \Huge$=$                                         \end{tabular}
  %\begin{tabular}{c}{%============================================================================
% Daniel J. Greenhoe
% LaTeX file
% pair of dice
%============================================================================
{\psset{unit=1.5\psunit}%
\begin{pspicture}(-1.3,-1.4)(1.3,1.4)%
  %---------------------------------
  % options
  %---------------------------------
  \psset{%
    radius=1.25ex,
    linecolor=blue,%
    labelsep=2.5mm,
    }%
  %---------------------------------
  % node locations
  %---------------------------------
  \rput( 0,0){%\psset{unit=2\psunit}%
    \rput{342}(0,0){\rput(1,0){\Cnode(0,0){S8}}}%
    \rput{306}(0,0){\rput(1,0){\Cnode(0,0){S9}}}%
    \rput{270}(0,0){\rput(1,0){\Cnode(0,0){S10}}}%
    \rput{234}(0,0){\rput(1,0){\Cnode(0,0){S11}}}%
    \rput{198}(0,0){\rput(1,0){\Cnode(0,0){S12}}}%
    \Cnode[fillstyle=solid,fillcolor=snode](0,0){S7}%
    \rput{162}(0,0){\rput(1,0){\Cnode(0,0){S6}}}%
    \rput{126}(0,0){\rput(1,0){\Cnode(0,0){S5}}}%
    \rput{ 90}(0,0){\rput(1,0){\Cnode(0,0){S4}}}%
    \rput{ 54}(0,0){\rput(1,0){\Cnode(0,0){S3}}}%
    \rput{ 18}(0,0){\rput(1,0){\Cnode(0,0){S2}}}%
    }
  %---------------------------------
  % edges
  %---------------------------------
  \ncline{S2}{S3}\ncline{S3}{S4}\ncline{S4}{S5}\ncline{S5}{S6}%
  \ncline{S8}{S9}\ncline{S9}{S10}\ncline{S10}{S11}\ncline{S11}{S12}%
  \ncline{S3}{S8}\ncline{S4}{S9}\ncline{S5}{S10}\ncline{S6}{S11}%
  \ncline{S7}{S2}\ncline{S7}{S6}\ncline{S7}{S8}\ncline{S7}{S12}%
  %---------------------------------
  % node labels
  %---------------------------------
  \rput(S12){$12$}%
  \rput(S11){$11$}%
  \rput(S10){$10$}%
  \rput(S9){$9$}%
  \rput(S8){$8$}%
  \rput(S7){$7$}%
  \rput(S6){$6$}%
  \rput(S5){$5$}%
  \rput(S4){$4$}%
  \rput(S3){$3$}%
  \rput(S2){$2$}%
  %---------------------------------
  % probability labels
  %---------------------------------
  \uput[198](S12){$\frac{1}{36}$}%
  \uput[234](S11){$\frac{2}{36}$}%
  \uput[270](S10){$\frac{3}{36}$}%
  \uput[306](S9) {$\frac{4}{36}$}%
  \uput[342](S8) {$\frac{5}{36}$}%
  \uput[-90](S7) {$\frac{6}{36}$}%
  \uput[162](S6) {$\frac{5}{36}$}%
  \uput[126](S5) {$\frac{4}{36}$}%
  \uput[ 90](S4) {$\frac{3}{36}$}%
  \uput[ 54](S3) {$\frac{2}{36}$}%
  \uput[ 18](S2) {$\frac{1}{36}$}%
  %
  %---------------------------------
  % labels
  %---------------------------------
  %\rput(6.5,1.25){$\rvX(\cdot)$}%
  %\ncline[linestyle=dotted,nodesep=1pt]{->}{xzlabel}{xz}%
  %\ncline[linestyle=dotted,nodesep=1pt]{->}{ylabel}{y}%
\end{pspicture}
}%}\end{tabular}
  \begin{tabular}{c}{\includegraphics{../common/math/graphics/pdfs/pspinner.pdf}}\end{tabular}
%  \\
  %\begin{tabular}{cc}
    \begin{tabstr}{0.75}%
    $\begin{array}{|r|*{11}{c}|}%
       \hline
         & 2 & 3 & 4 & 5 & 6 & 7 & 8 & 9 & 10 & 11 & 12\\
       \hline
       2 & 0 & 1 & 2 & 3 & 2 & 1 & 2 & 3 & 4  & 3  & 2 \\
       3 & 1 & 0 & 1 & 2 & 3 & 2 & 1 & 2 & 3  & 4  & 3 \\
       4 & 2 & 1 & 0 & 1 & 2 & 3 & 2 & 1 & 2  & 3  & 4 \\
       5 & 3 & 2 & 1 & 0 & 1 & 2 & 3 & 2 & 1  & 2  & 3 \\
       6 & 2 & 3 & 2 & 1 & 0 & 1 & 2 & 3 & 2  & 1  & 2 \\
       7 & 1 & 2 & 3 & 2 & 1 & 0 & 1 & 2 & 3  & 2  & 1 \\
       8 & 2 & 1 & 2 & 3 & 2 & 1 & 0 & 1 & 2  & 3  & 2 \\
       9 & 3 & 2 & 1 & 2 & 3 & 2 & 1 & 0 & 1  & 2  & 3 \\
      10 & 4 & 3 & 2 & 1 & 2 & 3 & 2 & 1 & 0  & 1  & 2 \\
      11 & 3 & 4 & 3 & 2 & 1 & 2 & 3 & 2 & 1  & 0  & 1 \\
      12 & 2 & 3 & 4 & 3 & 2 & 1 & 2 & 3 & 2  & 1  & 0 \\
      \hline
    \end{array}$%
    \end{tabstr}%
  %  &%
  %\\
  %  \begin{tabular}{c}%
  %    \gsize%
  %    %\psset{unit=10mm}%
  %  \end{tabular}
  %  \\%
  %  table form & graph form with standard probabilities and shaded center
  %\end{tabular}
  \caption{metrics based on a pair of spinners \xref{ex:pspinner}\label{fig:pspinner}}
\end{figure}

%---------------------------------------
\begin{example}[\exmd{pair of spinners}]
\label{ex:pspinner}
%---------------------------------------
A pair of \structe{spinner}s \xref{ex:spinner} has a structure as illustrated in \prefpp{fig:pspinner}.
The values represent the standard sum of spinner positions ($1, 2, \ldots, 6$) and thus range from $2$ to $12$.
The table in the figure provides the metric distances between summed values based on how many positions
one must traverse to get from one value to the next (in which ever direction is shortest).
This structure is also illustrated in the undirected graph in the upper right of \pref{fig:pspinner},
along with each value's standard probability.
\end{example}
\begin{proof}
\begin{align*}
  \ocscen(\ocsG)
    &\eqd \argmin_{x\in\ocsG}\max_{y\in\ocsG}\metric{x}{y}\psp(y)
    &&\text{by definition of $\ocscen$ \xref{def:ocscen}}
   \\&=\mathrlap{\argmin_{x\in\ocsG}\max_{y\in\ocsG}
         \setn{\begin{array}{cccccc}
           \metricn(2,2)\psp(2) &\metricn(2,3)\psp(3)  &\metricn(2,4)\psp(4)  &\cdots & \metricn(2,12)\psp(12)\\
           \metricn(3,2)\psp(2) &\metricn(3,3)\psp(3)  &\metricn(3,4)\psp(4)  &\cdots & \metricn(3,12)\psp(12)\\
           \vdots               &\vdots                &\vdots                &\ddots & \vdots                \\
           \metricn(12,2)\psp(2)&\metricn(12,3)\psp(3) &\metricn(12,4)\psp(4) &\cdots & \metricn(12,12)\psp(12)
         \end{array}}}
      \\&=\mathrlap{\argmin_{x\in\ocsG}\max_{y\in\ocsG}\frac{1}{36}
         \setn{\begin{array}{*{12}{c}}
            %  2          3           4           5           6           7           8           9           10         11          12
           0\times1 & 1\times2 & 2\times3 & 3\times4 & 2\times5 & 1\times6 & 2\times5 & 3\times4 & 4\times3  & 3\times2  & 2\times1 \\
           1\times1 & 0\times2 & 1\times3 & 2\times4 & 3\times5 & 2\times6 & 1\times5 & 2\times4 & 3\times3  & 4\times2  & 3\times1 \\
           2\times1 & 1\times2 & 0\times3 & 1\times4 & 2\times5 & 3\times6 & 2\times5 & 1\times4 & 2\times3  & 3\times2  & 4\times1 \\
           3\times1 & 2\times2 & 1\times3 & 0\times4 & 1\times5 & 2\times6 & 3\times5 & 2\times4 & 1\times3  & 2\times2  & 3\times1 \\
           2\times1 & 3\times2 & 2\times3 & 1\times4 & 0\times5 & 1\times6 & 2\times5 & 3\times4 & 2\times3  & 1\times2  & 2\times1 \\
           1\times1 & 2\times2 & 3\times3 & 2\times4 & 1\times5 & 0\times6 & 1\times5 & 2\times4 & 3\times3  & 2\times2  & 1\times1 \\
           2\times1 & 1\times2 & 2\times3 & 3\times4 & 2\times5 & 1\times6 & 0\times5 & 1\times4 & 2\times3  & 3\times2  & 2\times1 \\
           3\times1 & 2\times2 & 1\times3 & 2\times4 & 3\times5 & 2\times6 & 1\times5 & 0\times4 & 1\times3  & 2\times2  & 3\times1 \\
           4\times1 & 3\times2 & 2\times3 & 1\times4 & 2\times5 & 3\times6 & 2\times5 & 1\times4 & 0\times3  & 1\times2  & 2\times1 \\
           3\times1 & 4\times2 & 3\times3 & 2\times4 & 1\times5 & 2\times6 & 3\times5 & 2\times4 & 1\times3  & 0\times2  & 1\times1 \\
           2\times1 & 3\times2 & 4\times3 & 3\times4 & 2\times5 & 1\times6 & 2\times5 & 3\times4 & 2\times3  & 1\times2  & 0\times1   
         \end{array}}}
     \\&=\mathrlap{\argmin_{x\in\ocsG}\max_{y\in\ocsG}\frac{1}{36}
         \setn{\begin{array}{*{12}{c}}
           0 & 2 & 6 &12 &10 & 6 &10 &12 &12  & 6  & 2 \\
           1 & 0 & 3 & 8 &15 &12 & 5 & 8 & 9  & 8  & 3 \\
           2 & 2 & 0 & 4 &10 &18 &10 & 4 & 6  & 6  & 4 \\
           3 & 4 & 3 & 0 & 5 &12 &15 & 8 & 3  & 4  & 3 \\
           2 & 6 & 6 & 4 & 0 & 6 &10 &12 & 6  & 2  & 2 \\
           1 & 4 & 9 & 8 & 5 & 0 & 5 & 8 & 9  & 4  & 1 \\
           2 & 2 & 6 &12 &10 & 6 & 0 & 4 & 6  & 6  & 2 \\
           3 & 4 & 3 & 8 &15 &12 & 5 & 0 & 3  & 4  & 3 \\
           4 & 6 & 6 & 4 &10 &18 &10 & 4 & 0  & 2  & 2 \\
           3 & 8 & 9 & 8 & 5 &12 &15 & 8 & 3  & 0  & 1 \\
           2 & 6 &12 &12 &10 & 6 &10 &12 & 6  & 2  & 0 \\
         \end{array}}
       = \argmin_{x\in\ocsG}\frac{1}{36}
         \setn{\begin{array}{c}
           12\\% 2
           15\\% 3
           18\\% 4
           15\\% 5
           12\\% 6
            9\\% 7
           12\\% 8
           15\\% 9
           18\\%10
           15\\%11
           12%12
         \end{array}}
  \quad= \setn{7}}
  \\
  \ocscena(\ocsG)
    &\eqd \argmin_{x\in\ocsG}\sum_{y\in\ocsG}\metric{x}{y}\psp(y)
    &&\text{by definition of $\ocscena$ \xref{def:ocscenx}}
     \\&=\mathrlap{\argmin_{x\in\ocsG}\frac{1}{36}
         \setn{\begin{array}{*{23}{@{\,}c}}
           0 &+&  2 &+&  6 &+& 12 &+& 10 &+&  6 &+& 10 &+& 12 &+& 12  &+&  6  &+&  2 \\
           1 &+&  0 &+&  3 &+&  8 &+& 15 &+& 12 &+&  5 &+&  8 &+&  9  &+&  8  &+&  3 \\
           2 &+&  2 &+&  0 &+&  4 &+& 10 &+& 18 &+& 10 &+&  4 &+&  6  &+&  6  &+&  4 \\
           3 &+&  4 &+&  3 &+&  0 &+&  5 &+& 12 &+& 15 &+&  8 &+&  3  &+&  4  &+&  3 \\
           2 &+&  6 &+&  6 &+&  4 &+&  0 &+&  6 &+& 10 &+& 12 &+&  6  &+&  2  &+&  2 \\
           1 &+&  4 &+&  9 &+&  8 &+&  5 &+&  0 &+&  5 &+&  8 &+&  9  &+&  4  &+&  1 \\
           2 &+&  2 &+&  6 &+& 12 &+& 10 &+&  6 &+&  0 &+&  4 &+&  6  &+&  6  &+&  2 \\
           3 &+&  4 &+&  3 &+&  8 &+& 15 &+& 12 &+&  5 &+&  0 &+&  3  &+&  4  &+&  3 \\
           4 &+&  6 &+&  6 &+&  4 &+& 10 &+& 18 &+& 10 &+&  4 &+&  0  &+&  2  &+&  2 \\
           3 &+&  8 &+&  9 &+&  8 &+&  5 &+& 12 &+& 15 &+&  8 &+&  3  &+&  0  &+&  1 \\
           2 &+&  6 &+& 12 &+& 12 &+& 10 &+&  6 &+& 10 &+& 12 &+&  6  &+&  2  &+&  0 \\
         \end{array}}
       =\argmin_{x\in\ocsG}\frac{1}{36}
         \setn{\begin{array}{c}
           78\\%  2
           72\\%  3
           66\\%  4
           60\\%  5
           56\\%  6
           54\\%  7
           56\\%  8
           60\\%  9
           66\\% 10
           72\\% 11
           78%   12
         \end{array}}
       \quad= \setn{7}}
  \\
  \ocsVar(\ocsG)
    &\eqd \sum_{x\in\ocsG}\brs{\metric{\ocsE(\ocsG)}{x}}^2\psp(x)
    &&\text{by definition of $\ocsVar$ \xref{def:ocsVar}}
  \\&= \sum_{x\in\ocsG}\brs{\metric{7}{x}}^2\psp(x)
    &&\text{by definition of $\ocsVar$ \xref{def:ocsVar}}
  \\&=\mathrlap{\frac{1}{36}\brp{1^2\times1 + 2^2\times2 + 3^2\times3 + 2^2\times4 + 1^2\times5 + 0^2\times6 +
                        1^2\times5 + 2^2\times4 + 2^2\times3 + 2^2\times2 + 1^2\times1}}
  \\&= \frac{1}{36}\brp{2+8+27+16+5+0+5+16+12+8+1}
  \\&= \frac{100}{36} = \frac{25}{9} = 2\frac{7}{9} \approx 2.778
\end{align*}
\end{proof}




\begin{figure}[h]
  \gsize%
  \centering%
  %{%============================================================================
% Daniel J. Greenhoe
% LaTeX file
% pair of dice with isomorphic and linear mappings
%============================================================================
\begin{pspicture}(-6.2,-2.5)(9.8,2.5)%
  %---------------------------------
  % options
  %---------------------------------
  \psset{%
    radius=1.25ex,
    linecolor=blue,%
    labelsep=2.5mm,
    }%
  %---------------------------------
  % pair of spinners outcome space
  %---------------------------------
  \rput( 0,0){\psset{unit=1.5\psunit}%
    \rput{342}(0,0){\rput(1,0){\Cnode(0,0){S8}}}%
    \rput{306}(0,0){\rput(1,0){\Cnode(0,0){S9}}}%
    \rput{270}(0,0){\rput(1,0){\Cnode(0,0){S10}}}%
    \rput{234}(0,0){\rput(1,0){\Cnode(0,0){S11}}}%
    \rput{198}(0,0){\rput(1,0){\Cnode(0,0){S12}}}%
    \Cnode[fillstyle=solid,fillcolor=snode](0,0){S7}%
    \rput{162}(0,0){\rput(1,0){\Cnode(0,0){S6}}}%
    \rput{126}(0,0){\rput(1,0){\Cnode(0,0){S5}}}%
    \rput{ 90}(0,0){\rput(1,0){\Cnode(0,0){S4}}}%
    \rput{ 54}(0,0){\rput(1,0){\Cnode(0,0){S3}}}%
    \rput{ 18}(0,0){\rput(1,0){\Cnode(0,0){S2}}}%
    }
  \ncline{S2}{S3}\ncline{S3}{S4}\ncline{S4}{S5}\ncline{S5}{S6}%
  \ncline{S8}{S9}\ncline{S9}{S10}\ncline{S10}{S11}\ncline{S11}{S12}%
  \ncline{S3}{S8}\ncline{S4}{S9}\ncline{S5}{S10}\ncline{S6}{S11}%
  \ncline{S7}{S2}\ncline{S7}{S6}\ncline{S7}{S8}\ncline{S7}{S12}%
  %
  \rput(S12){$12$}%
  \rput(S11){$11$}%
  \rput(S10){$10$}%
  \rput(S9){$9$}%
  \rput(S8){$8$}%
  \rput(S7){$7$}%
  \rput(S6){$6$}%
  \rput(S5){$5$}%
  \rput(S4){$4$}%
  \rput(S3){$3$}%
  \rput(S2){$2$}%
  %
  \uput[198](S12){$\frac{1}{36}$}%
  \uput[234](S11){$\frac{2}{36}$}%
  \uput[270](S10){$\frac{3}{36}$}%
  \uput[306](S9) {$\frac{4}{36}$}%
  \uput[342](S8) {$\frac{5}{36}$}%
  \uput[-90](S7) {$\frac{6}{36}$}%
  \uput[162](S6) {$\frac{5}{36}$}%
  \uput[126](S5) {$\frac{4}{36}$}%
  \uput[ 90](S4) {$\frac{3}{36}$}%
  \uput[ 54](S3) {$\frac{2}{36}$}%
  \uput[ 18](S2) {$\frac{1}{36}$}%
  %
  %---------------------------------
  % random variable mapping X to real line
  %---------------------------------
  \rput(-5.5,0){\psset{unit=0.4\psunit}%
    \pnode(0, 6){XB}%
    \pnode(0, 5){X12}%
    \pnode(0, 4){X11}%
    \pnode(0, 3){X10}%
    \pnode(0, 2){X9}%
    \pnode(0, 1){X8}%
    \pnode(0, 0){X7}%
    \pnode(0,-1){X6}%
    \pnode(0,-2){X5}%
    \pnode(0,-3){X4}%
    \pnode(0,-4){X3}%
    \pnode(0,-5){X2}%
    \pnode(0,-6){XA}%
    }
  %\rput(4,-3){\psset{unit=0.5\psunit}%
  %  \pnode( 6,0){XB}%
  %  \pnode( 5,0){X12}%
  %  \pnode( 4,0){X11}%
  %  \pnode( 3,0){X10}%
  %  \pnode( 2,0){X9}%
  %  \pnode( 1,0){X8}%
  %  \pnode( 0,0){X7}%
  %  \pnode(-1,0){X6}%
  %  \pnode(-2,0){X5}%
  %  \pnode(-3,0){X4}%
  %  \pnode(-4,0){X3}%
  %  \pnode(-5,0){X2}%
  %  \pnode(-6,0){XA}%
  %  }
  \ncline{->}{X12}{XB}
  \ncline{X2}{X12}
  \ncline{->}{X2}{XA}
  %
  \pscircle[fillstyle=solid,linecolor=snode,fillcolor=snode](X7){1ex}%
  \pscircle[fillstyle=none,linecolor=red,fillcolor=red](X7){1ex}%
  %
  \rput(X12){\psline(-0.15,0)(0.15,0)}%
  \rput(X11){\psline(-0.15,0)(0.15,0)}%
  \rput(X10){\psline(-0.15,0)(0.15,0)}%
  \rput(X9) {\psline(-0.15,0)(0.15,0)}%
  \rput(X8) {\psline(-0.15,0)(0.15,0)}%
  \rput(X7) {\psline(-0.15,0)(0.15,0)}%
  \rput(X6) {\psline(-0.15,0)(0.15,0)}%
  \rput(X5) {\psline(-0.15,0)(0.15,0)}%
  \rput(X4) {\psline(-0.15,0)(0.15,0)}%
  \rput(X3) {\psline(-0.15,0)(0.15,0)}%
  \rput(X2) {\psline(-0.15,0)(0.15,0)}%
  %
  \uput[180](X12){$12$}%
  \uput[180](X11){$11$}%
  \uput[180](X10){$10$}%
  \uput[180](X9){$9$}%
  \uput[180](X8){$8$}%
  \uput[180](X7){$7$}%
  \uput[180](X6){$6$}%
  \uput[180](X5){$5$}%
  \uput[180](X4){$4$}%
  \uput[180](X3){$3$}%
  \uput[180](X2){$2$}%
  \uput[-22](XB){$\omsR$}%
  %
  %\uput[-90](X2){$\frac{1}{36}$}\uput[-90](X12){$\frac{1}{36}$}%
  %\uput[135](X3){$\frac{2}{36}$}\uput[45](X11){$\frac{2}{36}$}%
  %\uput[  22](X4){$\frac{3}{36}$}\uput[168](X10){$\frac{3}{36}$}%
  %\uput[ 135](X5){$\frac{4}{36}$}\uput[45](X9){$\frac{4}{36}$}%
  %\uput[-135](X6){$\frac{5}{36}$}\uput[-45](X8){$\frac{5}{36}$}%
  %\uput[90](X7){$\frac{6}{36}$}%
  %
  \ncarc[arcangle=22,linewidth=0.75pt,linecolor=red]{->}{S2}{X2}%
  \ncarc[arcangle=22,linewidth=0.75pt,linecolor=red]{->}{S3}{X3}%
  \ncarc[arcangle=22,linewidth=0.75pt,linecolor=red]{->}{S4}{X4}%
  \ncarc[arcangle=22,linewidth=0.75pt,linecolor=red]{->}{S5}{X5}%
  \ncarc[arcangle=22,linewidth=0.75pt,linecolor=red]{->}{S6}{X6}%
  \ncarc[arcangle=22,linewidth=0.75pt,linecolor=red]{->}{S7}{X7}%
  \ncarc[arcangle=45,linewidth=0.75pt,linecolor=red]{->}{S8}{X8}%
  \ncarc[arcangle=-45,linewidth=0.75pt,linecolor=red]{->}{S9}{X9}%
  \ncarc[arcangle=-11,linewidth=0.75pt,linecolor=red]{->}{S10}{X10}%
  \ncarc[arcangle=-22,linewidth=0.75pt,linecolor=red]{->}{S11}{X11}%
  \ncarc[arcangle=-45,linewidth=0.75pt,linecolor=red]{->}{S12}{X12}%
  %---------------------------------
  % random variable mapping Y to isomorphic outcome space
  %---------------------------------
  \rput( 8,0){\psset{unit=1.5\psunit}%
    \rput{342}(0,0){\rput(1,0){\Cnode(0,0){Y8}}}%
    \rput{306}(0,0){\rput(1,0){\Cnode(0,0){Y9}}}%
    \rput{270}(0,0){\rput(1,0){\Cnode(0,0){Y10}}}%
    \rput{234}(0,0){\rput(1,0){\Cnode(0,0){Y11}}}%
    \rput{198}(0,0){\rput(1,0){\Cnode(0,0){Y12}}}%
    \Cnode[fillstyle=solid,fillcolor=snode](0,0){Y7}%
    \rput{162}(0,0){\rput(1,0){\Cnode(0,0){Y6}}}%
    \rput{126}(0,0){\rput(1,0){\Cnode(0,0){Y5}}}%
    \rput{ 90}(0,0){\rput(1,0){\Cnode(0,0){Y4}}}%
    \rput{ 54}(0,0){\rput(1,0){\Cnode(0,0){Y3}}}%
    \rput{ 18}(0,0){\rput(1,0){\Cnode(0,0){Y2}}}%
    }
  \ncline{Y2}{Y3}\ncline{Y3}{Y4}\ncline{Y4}{Y5}\ncline{Y5}{Y6}%
  \ncline{Y8}{Y9}\ncline{Y9}{Y10}\ncline{Y10}{Y11}\ncline{Y11}{Y12}%
  \ncline{Y3}{Y8}\ncline{Y4}{Y9}\ncline{Y5}{Y10}\ncline{Y6}{Y11}%
  \ncline{Y7}{Y2}\ncline{Y7}{Y6}\ncline{Y7}{Y8}\ncline{Y7}{Y12}%
  %
  \rput(Y12){$12$}%
  \rput(Y11){$11$}%
  \rput(Y10){$10$}%
  \rput(Y9){$9$}%
  \rput(Y8){$8$}%
  \rput(Y7){$7$}%
  \rput(Y6){$6$}%
  \rput(Y5){$5$}%
  \rput(Y4){$4$}%
  \rput(Y3){$3$}%
  \rput(Y2){$2$}%
  %
  %\uput[-90](Y2){$\frac{1}{36}$}\uput[-90](Y12){$\frac{1}{36}$}%
  %\uput[135](Y3){$\frac{2}{36}$}\uput[45](Y11){$\frac{2}{36}$}%
  %\uput[  22](Y4){$\frac{3}{36}$}\uput[168](Y10){$\frac{3}{36}$}%
  %\uput[ 135](Y5){$\frac{4}{36}$}\uput[45](Y9){$\frac{4}{36}$}%
  %\uput[-135](Y6){$\frac{5}{36}$}\uput[-45](Y8){$\frac{5}{36}$}%
  %\uput[90](Y7){$\frac{6}{36}$}%
  %
  \ncarc[arcangle=67,linewidth=0.75pt,linecolor=green]{->}{S2}{Y2}%
  \ncarc[arcangle=22,linewidth=0.75pt,linecolor=green]{->}{S3}{Y3}%
  \ncarc[arcangle=22,linewidth=0.75pt,linecolor=green]{->}{S4}{Y4}%
  \ncarc[arcangle=-22,linewidth=0.75pt,linecolor=green]{->}{S5}{Y5}%
  \ncarc[arcangle=-22,linewidth=0.75pt,linecolor=green]{->}{S6}{Y6}%
  \ncarc[arcangle=11,linewidth=0.75pt,linecolor=green]{->}{S7}{Y7}%
  \ncarc[arcangle=-22,linewidth=0.75pt,linecolor=green]{->}{S8}{Y8}%
  \ncarc[arcangle=-22,linewidth=0.75pt,linecolor=green]{->}{S9}{Y9}%
  \ncarc[arcangle=22,linewidth=0.75pt,linecolor=green]{->}{S10}{Y10}%
  \ncarc[arcangle=22,linewidth=0.75pt,linecolor=green]{->}{S11}{Y11}%
  \ncarc[arcangle=-67,linewidth=0.75pt,linecolor=green]{->}{S12}{Y12}%
  %---------------------------------
  % labels
  %---------------------------------
  \rput[t](0.55,2.3){$\ocsG$}%
  \rput[t](8,2.3){$\omsH$}%
  \rput(-3.8,0){$\rvX(\cdot)$}%
  \rput( 4,0){$\rvY(\cdot)$}%
  %\ncline[linestyle=dotted,nodesep=1pt]{->}{xzlabel}{xz}%
  %\ncline[linestyle=dotted,nodesep=1pt]{->}{ylabel}{y}%
\end{pspicture}%}%
  {\includegraphics{../common/math/graphics/pdfs/pspinner_xy.pdf}}%
  \caption{\structe{pair of spinner} mappings \xref{ex:pspinner_xy}\label{fig:pspinner_xy}}
\end{figure}
%---------------------------------------
\begin{example}[\exmd{pair of spinner and hypothesis testing}]
\label{ex:pspinner_xy}
%---------------------------------------
Let $\ocsG$ be a \structe{pair of spinners} \xref{ex:pspinner},
    $\rvX$ a \fncte{random variable} mapping to the \structe{real line} \xref{def:Rline},
and $\rvY$ a \fncte{random variable} mapping to an \structe{ordered metric space} \xref{def:oms}
that is \prope{isomorphic} to $\ocsG$ under $\rvY$,
as illustrated in \prefpp{fig:pspinner_xy}.
This yields the following statistics:
\\$\begin{array}{>{\footnotesize}Mlcc lcccl}
  geometry of $\ocsG$:                                          & \ocscen (\ocsG)=\ocscena(\ocsG) &=& \setn{7}     & \ocsVaro(\ocsG) &=& \frac{25}{9} &\approx& 2.778\\
  traditional statistics on real line $\omsR$:                  & \pE  (\rvX)     &=& 7            & \ocsVar(\rvW;\pE)   &=& \frac{35}{6}&\approx& 5.833\\
  outcome subspace statistics on real line $\omsR$:             & \ocsE(\rvX)     &=& \setn{7}     & \ocsVar(\rvW;\ocsE) &=& \frac{35}{6}&\approx& 5.833\\
  outcome subspace statistics on  isomorphic structure $\ocsH$: & \ocsE(\rvY)     &=& \setn{7}     & \ocsVar(\rvY;\ocsE) &=& 1           &       &
\end{array}$  
\\
Although the expected value of both \structe{outcome subspace}s are the same ($\pE(\rvX)=\ocsE(\rvY)=7$), 
the isomorphic outcome subspace $\ocsH$ yields a much smaller variance (a much smaller expected error).
This is significant in statistical applications such as hypothesis testing. 
Suppose for example we have two pair of \structe{spinners} \xref{ex:spinner}, 
one pair being made of two uniformly distributed spinners, 
and one pair of weighted spinners. We want to estimate which is which.
So we spin each pair one time. Suppose the outcome of the first pair is $12$ and the 
the outcome of the second pair is $10$. Which pair is more likely to be the uniform pair?
Using traditional statistical analysis, the answer is the second pair, because it is closer to the 
expected value ($\metric{7}{10}=\abs{7-10}=3=1.8$ standard deviations as opposed to 
$\metric{7}{12}=\abs{7-12}=5=3$ standard deviations).
However, this result is deceptive, because as can be seen in the table in \prefpp{fig:pspinner},
$12$ is acually closer to the expected value in $\ocsG$ than is $10$
($\metric{7}{12}=1 < 3=\metric{7}{10}$).
So arguably the better choice, based on this one trial, is the first pair.
\end{example}
\begin{proof}
\begin{align*}
  \ocscen (\ocsG) &= \setn{7}                  && \text{by \prefpp{ex:pspinner}}\\
  \ocscena(\ocsG) &= \setn{7}                  && \text{by \prefpp{ex:pspinner}}\\
  \ocsVaro(\ocsG) &= \frac{25}{9}\approx 2.778 && \text{by \prefpp{ex:pspinner}}
  \\
  \pE  (\rvX) &= \frac{252}{36} = 7            && \text{by \prefpp{ex:dicepair_moments}}  \\
  \pVar(\rvX) &= \frac{35}{6}   \approx 5.833  && \text{by \prefpp{ex:dicepair_moments}}
  \\
  \ocsE(\rvX)
    &= \pE(\rvX)
    && \text{because on \structe{real line}, $\psp$ is \prope{symmetric}, and by \prefp{thm:pEocsE}}
  \\&= \setn{7}
    && \text{by $\pE(\rvX)$ result}
  %  &\eqd \argmin_{x\in\R}\max_{y\in\R} \ocsmom(x,y)
  %  &&\text{by definition of $\ocsE$ \xref{def:ocsE}}
  %\\&\eqd \argmin_{x\in\R}\max_{y\in\R} \metric{x}{y}\psp(y)
  %  &&\text{by definition of $\ocsmom$ \xref{def:ocsmom}}
  %\\&= \argmin_{x\in\R}\setn{30,25,20,16,12,9,12,16,20,25,30}
  %  &&\text{by definition of \structe{real line} $\omsR$ \xref{def:Rline}}
  %\\&= \setn{7}
  \\
  \ocsVar(\rvX;\ocsE)
    &= \ocsVar(\rvX;\pE)
    && \text{because $\ocsE(\rvX)=\pE(\rvX)$}
  \\&= \pVar(\rvX)
    && \text{by \prefp{thm:ocsVar}}
  \\&= \frac{35}{6} \approx 5.833
  \\
  \ocsE(\rvY)
    &= \rvY\brs{\ocscen(\ocsG)}
    &&\text{because $\ocsG$ and $\omsH$ are \prope{isomorphic} under $\rvY$}
  \\&= \rvY\brs{\setn{7}}
    &&\text{by \prefpp{ex:pspinner}}
  \\&= \setn{7}
    &&\text{by definition of $\rvX$}
  %\\
  %\ocsEa(\rvX)
  %  &\eqd \argmin_{x\in\ocsH}\sum_{y\in\ocsH}\ocsmom(x,y)
  %  &&\text{by definition of $\ocsEa$ \xref{def:ocsEa}}
  %\\&= \rvX\brs{\argmin_{x\in\ocsG}\sum_{y\in\ocsG}\ocsmom(x,y)}
  %  &&\text{because $\ocsG$ and $\ocsH$ are \prope{isomorphic}}
  %\\&= \rvX\brs{\ocscena(\ocsG)}
  %  &&\text{by definition of $\ocscena$ \xref{def:ocscena}}
  %\\&= \rvX\brs{\setn{7}}
  %  &&\text{by \prefpp{ex:pspinner}}
  %\\&= \setn{7}
  %  &&\text{by definition of $\rvX$}
  \\
  \ocsVar(\rvY;\ocsE)
  %  &\eqd \sum_{x,y\in\ocsH}\brs{\metric{x}{y}}^2\psp(y)
  %  &&\text{by definition of $\ocsVar$ \xref{def:ocsVar}}
  %\\&=    \sum_{y\in\ocsH}\brs{\metric{7}{y}}^2\psp(y)
  %\\&= \frac{1}{36}\brp{2^2\times1 + 1^2\times2 + 1^2\times3 + 1^2\times4 + 1^2\times5 + 0^2\times6 +
  %                      1^2\times5 + 1^2\times4 + 1^2\times3 + 1^2\times2 + 2^2\times1}
  %\\&= \frac{1}{36}\brp{4+2+3+4+5+0+5+4+3+2+4}
  %\\&= \frac{36}{36}
    &= \ocsVar(\ocsG)
    &&\text{because $\ocsG$ and $\omsH$ are \prope{isomorphic} under $\rvY$}
  \\&= 1
    && \text{by \prefpp{ex:pspinner}}
\end{align*}
\end{proof}




%\fi

\begin{figure}[h]
  \gsize%
  \centering%
  %{\input{../common/math/graphics/sto/rline_argminmax_11312.tex}}%
  {\includegraphics{../common/math/graphics/pdfs/rline_argminmax_11312.pdf}}%
  \caption{real line addition $\argmin_x\max_y$ calculation graph \xref{ex:rline_11312}\label{fig:rline_11312}}
\end{figure}
%---------------------------------------
\begin{example}[\exmd{linear addition}]
\label{ex:rline_11312}
%---------------------------------------
Let $\rvX$ be a \structe{random variable} \xref{def:ocsrv} mapping to 
a \structe{real line} \structe{ordered metric space} \xref{def:Rline} resulting in probability values of 
\\\indentx$\psp(0)=\psp(2)=\frac{1}{2}$, and $\psp(x)=0$ otherwise.\\
Let $\rvY$ be a \structe{random variable} mapping to 
a \structe{real line} \structe{ordered metric space} resulting in probability values of 
\\\indentx$\psp(0)=\psp(1)=\frac{1}{4}$, $\psp(2)=\frac{1}{2}$, and $\psp(x)=0$ otherwise.\\
Let $\rvZ\eqd\rvX+\rvY$ be the random variable mapping to the \structe{outcome subspace} \xref{def:ocs} induced by 
adding $\rvX$ and $\rvY$ resulting in probabilities 
\\\indentx$\begin{array}{r|ccccc}
  z      & 0 & 1 & 2 & 3 & 4 
  \\\hline
 \psp(z) & \frac{1}{8} & \frac{1}{8} & \frac{3}{8} & \frac{1}{8} & \frac{2}{8}  
\end{array}$, and $\psp(z)=0$ otherwise.
\\
Note that although the traditional expectation $\pE$ \xref{def:pE} \prope{distributes} over addition such that
\\\indentx$\pE(\rvX+\rvY) = \frac{18}{8} = 1 + \frac{5}{4} = \pE(\rvX) + \pE(\rvY)$,\\
the alternative expecation $\ocsE$ \xref{def:ocsE} does \emph{not}:
\\\indentx$\ocsE(\rvX+\rvY) = \frac{8}{3} \neq \frac{7}{3} =  1 + \frac{4}{3} = \ocsE(\rvX) + \ocsE(\rvY)$.
\end{example}
\begin{proof}
    \begin{align*}
      \pE(\rvX)
        &= \cramped{\int_{x\in\R} x\psp(x) \dx}
        && \text{by definition of $\pE$ \xref{def:pE}}
      \\&= \sum_{x\in\Z} x\psp(x) \dx
        && \text{by definition of $\psp$ and \prefp{prop:pE}}
      \\&= 0\times\frac{1}{2} + 2\times\frac{1}{2}
        && \text{by definition of $\psp$}
      \\&= 1
      \\
      \pE(\rvY)
        &= \cramped{\int_{y\in\R} y\psp(y) \dy}
        && \text{by definition of $\pE$ \xref{def:pE}}
      \\&= \sum_{y\in\Z} y\psp(y) \dy
        && \text{by definition of $\psp$ and \prefp{prop:pE}}
      \\&= 0\times\frac{1}{4} + 1\times\frac{1}{4} + 2\times\frac{1}{2}
        && \text{by definition of $\psp$}
      \\&= \frac{5}{4}
      \\
      \pE(\rvZ)
        &= \cramped{\int_{z\in\R} z\psp(z) \dz}
        && \text{by definition of $\pE$ \xref{def:pE}}
      \\&= \sum_{z\in\Z} z\psp(z) \dz
        && \text{by definition of $\psp$ and \prefp{prop:pE}}
      \\&= 0\times\frac{1}{8} + 1\times\frac{1}{8} + 2\times\frac{3}{8} + 3\times\frac{1}{8} + 4\times\frac{2}{8}
        && \text{by definition of $\psp$}
      \\&= \frac{9}{4}
      \\
      \ocsE(\rvX)
        &= \argmin_{x\in\R}\max_{y\in\R} \metric{x}{y}\psp(y)
        && \text{by definition of $\ocsE$ \xref{def:ocsE}}
      \\&= \argmin_{x\in\Z}\max_{y\in\Z} \metric{x}{y}\psp(y)
        && \text{by definition of $\psp(x)$}
      \\&= \argmin_{x\in\Z}\brb{\begin{array}{lM}
             \abs{2-x}\frac{1}{2} & for $x\orel 1$\\
             \abs{x}\frac{1}{2}   & otherwise
           \end{array}}
      \\&= \setn{1}
        && \text{because $\max(x)$ is minimized at $x=1$}
      \\
      \ocsE(\rvY)
        &= \argmin_{x\in\R}\max_{y\in\R} \metric{x}{y}\psp(y)
        && \text{by definition of $\ocsE$ \xref{def:ocsE}}
      \\&= \argmin_{x\in\Z}\max_{y\in\Z} \metric{x}{y}\psp(y)
        && \text{by definition of $\psp(x)$}
      \\&= \argmin_{x\in\Z}\brb{\begin{array}{lM}
             \abs{2-x}\frac{1}{2} & for $\frac{4}{3}\ge x \ge 4$\\
             \abs{x}\frac{1}{4} & otherwise
           \end{array}}
      \\&= \setn{\frac{4}{3}}
        && \text{because $\max(y)$ is minimized at $y=\frac{4}{3}$}
      \\
      \ocsE(\rvZ)
        &= \argmin_{x\in\R}\max_{y\in\R} \metric{x}{y}\psp(y)
        && \text{by definition of $\ocsE$ \xref{def:ocsE}}
      \\&= \argmin_{x\in\Z}\max_{y\in\Z} \metric{x}{y}\psp(y)
        && \text{by definition of $\psp(x)$}
      \\&= \argmin_{x\in\Z}\brb{\begin{array}{lM}
             \abs{x-2}\frac{3}{8} & for $-2 \ge x \ge 3$\\
             \abs{x-4}\frac{2}{8} & for $-2 \le x \le \frac{8}{3}$\\
             \abs{x}\frac{1}{8} &   for $\frac{8}{3} \le x \le 3$
            %\abs{x}\frac{1}{8} &   for $\frac{8}{3} \le x \le 3$ (otherwise)
           \end{array}}
      \\&= \setn{\frac{8}{3}}
        && \text{because $\max(z)$ is minimized at $z=\frac{8}{3}$ (see \prefp{fig:rline_11312})}
    \end{align*}
\end{proof}



%=======================================
\subsection{Multiplication}
%=======================================
\begin{figure}[h]
  \gsize%
  \centering%
  %{%============================================================================
% Daniel J. Greenhoe
% LaTeX file
% spinner non-linear mappings
%============================================================================
\begin{pspicture}(-4.75,-1.6)(4.75,1.6)%
  %---------------------------------
  % options
  %---------------------------------
  \psset{%
    radius=1.25ex,
    labelsep=2.5mm,
    linecolor=blue,%
    }%
  %---------------------------------
  % spinner graph
  %---------------------------------
  \rput(0,0){%\psset{unit=2\psunit}%
    \rput{ 210}(0,0){\rput(1,0){\Cnode[fillstyle=solid,fillcolor=snode](0,0){S5}}}%
    \rput{ 150}(0,0){\rput(1,0){\Cnode(0,0){S4}}}%
    \rput{  90}(0,0){\rput(1,0){\Cnode(0,0){S3}}}%
    \rput{  30}(0,0){\rput(1,0){\Cnode(0,0){S2}}}%
    \rput{ -30}(0,0){\rput(1,0){\Cnode(0,0){S1}}}%
    \rput{ -90}(0,0){\rput(1,0){\Cnode(0,0){S0}}}%
    \rput(0,0){$\ocsG$}%
    }
  \rput(S5){$5$}%
  \rput(S4){$4$}%
  \rput(S3){$3$}%
  \rput(S2){$2$}%
  \rput(S1){$1$}%
  \rput(S0){$0$}%
  %
  \ncline{S5}{S0}%
  \ncline{S4}{S5}%
  \ncline{S3}{S4}%
  \ncline{S2}{S3}%
  \ncline{S1}{S2}%
  \ncline{S0}{S1}%
  %
  \uput[210](S5){$\frac{5}{10}$}
  \uput[150](S4){$\frac{1}{10}$}
  \uput[ 90](S3){$\frac{1}{10}$}
  \uput[ 30](S2){$\frac{1}{10}$}
  \uput[-30](S1){$\frac{1}{10}$}
  \uput[-90](S0){$\frac{1}{10}$}
  %---------------------------------
  % X range graph
  %---------------------------------
  \rput(-3.5,0){%\psset{unit=2\psunit}%
    \rput{ 210}(0,0){\rput(1,0){\Cnode[fillstyle=solid,fillcolor=snode](0,0){X5}}}%
    \rput{ 150}(0,0){\rput(1,0){\Cnode(0,0){X4}}}%
    \rput{  90}(0,0){\rput(1,0){\Cnode(0,0){X3}}}%
    \rput{  30}(0,0){\rput(1,0){\Cnode(0,0){X2}}}%
    \rput{ -30}(0,0){\rput(1,0){\Cnode(0,0){X1}}}%
    \rput{ -90}(0,0){\rput(1,0){\Cnode(0,0){X0}}}%
    \rput(0,0){$\ocsH_1$}%
    }
  \rput(X5){$5$}%
  \rput(X4){$4$}%
  \rput(X3){$3$}%
  \rput(X2){$2$}%
  \rput(X1){$1$}%
  \rput(X0){$0$}%
  %
  \ncline{X5}{X0}%
  \ncline{X4}{X5}%
  \ncline{X3}{X4}%
  \ncline{X2}{X3}%
  \ncline{X1}{X2}%
  \ncline{X0}{X1}%
  %
  \uput[210](X5){$\frac{5}{10}$}
  \uput[150](X4){$\frac{1}{10}$}
  \uput[ 90](X3){$\frac{1}{10}$}
  \uput[ 30](X2){$\frac{1}{10}$}
  \uput[-30](X1){$\frac{1}{10}$}
  \uput[-90](X0){$\frac{1}{10}$}
  %
  \ncarc[arcangle=-22,linewidth=0.75pt,linecolor=red]{->}{S5}{X5}%
  \ncarc[arcangle= 22,linewidth=0.75pt,linecolor=red]{->}{S4}{X4}%
  \ncarc[arcangle=-22,linewidth=0.75pt,linecolor=red]{->}{S3}{X3}%
  \ncarc[arcangle=-22,linewidth=0.75pt,linecolor=red]{->}{S2}{X2}%
  \ncarc[arcangle=-22,linewidth=0.75pt,linecolor=red]{->}{S1}{X1}%
  \ncarc[arcangle= 22,linewidth=0.75pt,linecolor=red]{->}{S0}{X0}%
  %---------------------------------
  % Y=2X range graph
  %---------------------------------
  \rput(3.5,0){%\psset{unit=2\psunit}%
    \rput{ 150}(0,0){\rput(1,0){\Cnode[fillstyle=solid,fillcolor=snode](0,0){Y4}}}%
    \rput{  30}(0,0){\rput(1,0){\Cnode(0,0){Y2}}}%
    \rput{ -90}(0,0){\rput(1,0){\Cnode(0,0){Y0}}}%
    \rput(0,0){$\ocsH_2$}%
    }
  \rput(Y4){$4$}%
  \rput(Y2){$2$}%
  \rput(Y0){$0$}%
  %
  \ncline{Y4}{Y0}%
  \ncline{Y2}{Y4}%
  \ncline{Y0}{Y2}%
  %
  \uput[150](Y4){$\frac{6}{10}$}
  \uput[ 30](Y2){$\frac{2}{10}$}
  \uput[-90](Y0){$\frac{2}{10}$}
  %
  \ncarc[arcangle= 22,linewidth=0.75pt,linecolor=green]{->}{S5}{Y4}%
  \ncarc[arcangle= 22,linewidth=0.75pt,linecolor=green]{->}{S4}{Y2}%
  \ncarc[arcangle= 22,linewidth=0.75pt,linecolor=green]{->}{S3}{Y0}%
  \ncarc[arcangle= 22,linewidth=0.75pt,linecolor=green]{->}{S2}{Y4}%
  \ncarc[arcangle=-22,linewidth=0.75pt,linecolor=green]{->}{S1}{Y2}%
  \ncarc[arcangle=-22,linewidth=0.75pt,linecolor=green]{->}{S0}{Y0}%
  %---------------------------------
  % labels
  %---------------------------------
  \rput(-1.75,0){$\rvX(\cdot)$}%
  \rput(1.75,0){$2\rvX(\cdot)$}%
  %\ncline[linestyle=dotted,nodesep=1pt]{->}{xzlabel}{xz}%
  %\ncline[linestyle=dotted,nodesep=1pt]{->}{ylabel}{y}%
\end{pspicture}%}%
  {\includegraphics{../common/math/graphics/pdfs/spinner_x2x.pdf}}%
  \caption{\structe{pair of spinner} mappings \xref{ex:spinner_x2x}\label{fig:spinner_x2x}}
\end{figure}
%---------------------------------------
\begin{example}[\exmd{ring multiplication}]
\label{ex:spinner_x2x}
%---------------------------------------
Let $\rvX\in\clOCSgh$ be a random variable where $\ocsG$ is the \structe{weighted spinner}s 
illustrated in \prefpp{fig:spinner_x2x}.
Note that, in agreement with \prefpp{cor:ocsrv_Eax_R}, %{thm:EfX},
\\\indentx$\ocsE(2\rvX) = \setn{4} = \setn{2\times5\mod6} = 2\setn{5}\mod6 = 2\ocsE(\rvX)\mod6$ .
\end{example}
\begin{proof}
    \begin{align*}
      \ocscen(\ocsG)
        &\eqd \argmin_{x\in\ocsH}\max_{y\in\ocsH}\metric{x}{y}\psp(y)
        &&\text{by definition of $\ocscen$ \xref{def:ocscen}}
      \\&=\mathrlap{\argmin_{x\in\ocsH}\max_{y\in\ocsH}\frac{1}{10}
             \setn{\begin{array}{cccccc}
               {0}\times1 & {1}\times1 & {2}\times1 &{3}\times1 &{2}\times1 &{1}\times5\\
               {1}\times1 & {0}\times1 & {1}\times1 &{2}\times1 &{3}\times1 &{2}\times5\\
               {2}\times1 & {1}\times1 & {0}\times1 &{1}\times1 &{2}\times1 &{3}\times5\\
               {3}\times1 & {2}\times1 & {1}\times1 &{0}\times1 &{1}\times1 &{2}\times5\\
               {2}\times1 & {3}\times1 & {2}\times1 &{1}\times1 &{0}\times1 &{1}\times5\\
               {1}\times1 & {2}\times1 & {3}\times1 &{2}\times1 &{1}\times1 &{0}\times5
             \end{array}}
      \quad= \argmin_{x\in\ocsH}\frac{1}{10}
             \setn{\begin{array}{ccccccc}
                5\\
               10\\
               15\\
               10\\
                5\\
                3
             \end{array}}}
      \\&= \setn{5}
      \\
      \ocscena(\ocsG)
        &\eqd \argmin_{x\in\ocsH}\sum_{y\in\ocsH}\metric{x}{y}\psp(y)
        &&\text{by definition of $\ocscena$ \xref{def:ocscena}}
      \\&=\mathrlap{\argmin_{x\in\ocsH}\frac{1}{10}
             \setn{\begin{array}{*{11}{@{\,}c}}
               {0}\times1 &+& {1}\times1  &+&  {2}\times1  &+& {3}\times1  &+& {2}\times1  &+& {1}\times5\\
               {1}\times1 &+& {0}\times1  &+&  {1}\times1  &+& {2}\times1  &+& {3}\times1  &+& {2}\times5\\
               {2}\times1 &+& {1}\times1  &+&  {0}\times1  &+& {1}\times1  &+& {2}\times1  &+& {3}\times5\\
               {3}\times1 &+& {2}\times1  &+&  {1}\times1  &+& {0}\times1  &+& {1}\times1  &+& {2}\times5\\
               {2}\times1 &+& {3}\times1  &+&  {2}\times1  &+& {1}\times1  &+& {0}\times1  &+& {1}\times5\\
               {1}\times1 &+& {2}\times1  &+&  {3}\times1  &+& {2}\times1  &+& {1}\times1  &+& {0}\times5
             \end{array}}
         = \argmin_{x\in\ocsH}\frac{1}{10}
             \setn{\begin{array}{c}
               13\\
               17\\
               21\\
               17\\
               13\\
                9
             \end{array}}}
      \\&= \setn{5}
      \\
      %\\
      %\ocsVar(\rvY)
      %  &\eqd \sum_{x\in\ocsH}\ocsmom_2(\ocscen(\ocsG),x)
      %  &&\text{by definition of $\ocsVar$ \xref{def:ocsVar}}
      %\\&= \mathrlap{\ocsmom_2(0,0)+\ocsmom_2(0,1)+\ocsmom_2(0,2)+\ocsmom_2(0,3)+\ocsmom_2(6,4)+\ocsmom_2(6,5)+\ocsmom_2(6,6)}
      %\\&= \brs{\metric{0}{0}}^2\psp(0)+\brs{\metric{0}{1}}^2\psp(1)+\brs{\metric{0}{2}}^2\psp(2)+\brs{\metric{0}{3}}^2\psp(3)
      %  \\&\qquad+\brs{\metric{6}{4}}^2\psp(4)+\brs{\metric{6}{5}}^2\psp(5)+\brs{\metric{6}{6}}^2\psp(6)
      %\\&= \mathrlap{\brs{0}^2\frac{3}{10}+\brs{1}^2\frac{1}{10}+\brs{2}^2\frac{1}{10}+\brs{3}^2\frac{1}{10}+\brs{2}^2\frac{1}{10}+\brs{1}^2\frac{1}{10}+\brs{0}^2\frac{3}{10}}
      %\\&= \frac{1}{10}\brs{19}
      %\\&= 1.9
  %  \begin{align*}
  %    \pE  (\rvX) &= \frac{252}{36} &= 7            && \text{by \prefpp{ex:dicepair_moments}}  \\
  %    \pVar(\rvX) &= \frac{35}{6}   &\approx 5.833  && \text{by \prefpp{ex:dicepair_moments}}
  %  \end{align*}
  %
  %\item alternative statistics of random variable mapping $\rvX$:
  %  \begin{align*}
  %    \ocsE(\rvX)
  %      &\eqd \argmin_{x\in\R}\max_{y\in\R} \ocsmom(x,y)
  %      &&\text{by definition of $\ocsE$ \xref{def:ocsE}}
  %    \\&\eqd \argmin_{x\in\R}\max_{y\in\R} \metric{x}{y}\psp(y)
  %      &&\text{by definition of $\ocsmom$ \xref{def:ocsmom}}
  %    \\&= \argmin_{x\in\R}\setn{30,25,20,16,12,9,12,16,20,25,30}
  %      &&\text{by definition of \structe{real line} $\omsR$ \xref{def:Rline}}
  %    \\&= \setn{7}
  %  \end{align*}
      \ocsE(\rvX)
        &= \rvX\brs{\ocscen(\ocsG)}
        &&\text{because $\ocsG$ and $\ocsH_1$ are \prope{isomorphic}}
      \\&= \rvX\brs{\setn{5}}
        &&\text{by $\ocscen(\ocsG)$ result}
      \\&= \setn{5}
        &&\text{by definition of $\rvX$}
      %\\
      %\ocsEa(\rvX)
      %  &\eqd \argmin_{x\in\ocsH_1}\sum_{y\in\ocsH_1}\ocsmom(x,y)
      %  &&\text{by definition of $\ocsEa$ \xref{def:ocsEa}}
      %\\&= \rvX\brs{\argmin_{x\in\ocsG}\sum_{y\in\ocsG}\ocsmom(x,y)}
      %  &&\text{because $\ocsG$ and $\ocsH_1$ are \prope{isomorphic}}
      %\\&= \rvX\brs{\ocscena(\ocsG)}
      %  &&\text{by definition of $\ocscena$ \xref{def:ocscena}}
      %\\&= \rvX\brs{\setn{5}}
      %  &&\text{by \prefpp{item:pspinner_x2x_geo}}
      %\\&= \setn{5}
      %  &&\text{by definition of $\rvX$}
    %  \\
    %\ocsVar(\rvX)
    %  &\eqd \sum_{x\in\ocsH}\ocsmom_2(\ocsE(\ocsG),x)
    %  &&\text{by definition of $\ocsVar$ \xref{def:ocsVar}}
    %\\&\eqd \sum_{x,y\in\ocsH}\brs{\metric{x}{y}}^2\psp(y)
    %  &&\text{by definition of $\pVar$ \xref{def:ocsmom}}
    %\\&= \ocsVar(\ocsG)
    %\\&= 1
    %  && \text{by \prefpp{ex:pspinner}}
    \\
      \ocsE(2\rvX)
        &\eqd \argmin_{x\in\ocsH_2}\max_{y\in\ocsH_2}\metric{x}{y}\psp(y)
        &&\text{by definition of $\ocsE$ \xref{def:ocsE}}
      \\&=\mathrlap{\argmin_{x\in\ocsH_2}\max_{y\in\ocsH_2}\frac{1}{10}
             \setn{\begin{array}{ccc}
               {0}\times2 & {1}\times2 & {1}\times6\\
               {1}\times2 & {0}\times2 & {1}\times6\\
               {1}\times2 & {1}\times2 & {0}\times6
             \end{array}}
      \quad= \argmin_{x\in\ocsH_2}\frac{1}{10}
             \setn{\begin{array}{c}
                6\\
                6\\
                2\\
             \end{array}}}
      \\&= \setn{4}
      %\\
      %\ocsEa(2\rvX)
      %  &\eqd \argmin_{x\in\ocsH_2}\sum_{y\in\ocsH_2}\ocsmom(x,y)
      %  &&\text{by definition of $\ocscena$ \xref{def:ocscena}}
      %\\&\eqd \argmin_{x\in\ocsH_2}\sum_{y\in\ocsH_2}\metric{x}{y}\psp(y)
      %  &&\text{by definition of $\ocsmom$ \xref{def:ocsmom}}
      %\\&=\mathrlap{\argmin_{x\in\ocsH_2}\frac{1}{10}
      %       \setn{\begin{array}{*{5}{@{\,}c}}
      %         {0}\times2 &+& {1}\times2  &+&  {1}\times6\\
      %         {1}\times2 &+& {0}\times2  &+&  {1}\times6\\
      %         {1}\times2 &+& {1}\times2  &+&  {0}\times6\\
      %       \end{array}}
      %   = \argmin_{x\in\ocsH_2}\frac{1}{10}
      %       \setn{\begin{array}{c}
      %          8\\
      %          8\\
      %          4
      %       \end{array}}}
      %\\&= \setn{4}
      %\\
      %\ocsVar(\rvY)
      %  &\eqd \sum_{x\in\ocsH}\ocsmom_2(\ocscen(\ocsG),x)
      %  &&\text{by definition of $\ocsVar$ \xref{def:ocsVar}}
      %\\&= \mathrlap{\ocsmom_2(0,0)+\ocsmom_2(0,1)+\ocsmom_2(0,2)+\ocsmom_2(0,3)+\ocsmom_2(6,4)+\ocsmom_2(6,5)+\ocsmom_2(6,6)}
      %\\&= \brs{\metric{0}{0}}^2\psp(0)+\brs{\metric{0}{1}}^2\psp(1)+\brs{\metric{0}{2}}^2\psp(2)+\brs{\metric{0}{3}}^2\psp(3)
      %  \\&\qquad+\brs{\metric{6}{4}}^2\psp(4)+\brs{\metric{6}{5}}^2\psp(5)+\brs{\metric{6}{6}}^2\psp(6)
      %\\&= \mathrlap{\brs{0}^2\frac{3}{10}+\brs{1}^2\frac{1}{10}+\brs{2}^2\frac{1}{10}+\brs{3}^2\frac{1}{10}+\brs{2}^2\frac{1}{10}+\brs{1}^2\frac{1}{10}+\brs{0}^2\frac{3}{10}}
      %\\&= \frac{1}{10}\brs{19}
      %\\&= 1.9
    \end{align*}
\end{proof}


%=======================================
\subsection{Metric transformation}
%=======================================
It is possible to use a \fncte{metric transform} \xref{def:mpf} to transform the structure of an 
\structe{outcome subspace} \xref{def:ocs} into a completely different \structe{outcome subspace}. 
This is demonstrated in \prefpp{ex:ocsop_discrete}--\prefpp{ex:ocsop_x1x2}.
Naturally, by doing so one can sometimes even change the geometric \structe{center}s \xxref{def:ocscen}{def:ocscenx}
of the outcome subspaces, and hence also the statistics of random variables that map to/from them.
This is demonstrated in \prefpp{ex:ocsop_x1x2}--\prefpp{ex:rline_11312a}.

%---------------------------------------
\begin{theorem}
\label{thm:ocsop_mpf}
%---------------------------------------
Let $\fphi$ be a \fncte{metric preserving function} \xref{def:mpf}.
Let $\ocsG\eqd\ocsD$ and $\ocsH$ be \structe{outcome subspace}s \xref{def:ocs}.
\thmbox{
  \brb{\begin{array}{FMD}
    (1). & $\fphi(\ocsH)=\ocsG$ & and\\
    (2). & $\fphi$ is \prope{strictly isotone} & and \\
    (3). & $\psp$ is uniform
  \end{array}}
  \quad\implies\quad
  \ocscen(\ocsH)=\ocscen(\ocsG)
  }
\end{theorem}
\begin{proof}
\begin{align*}
  \ocscen(\ocsH)
    &= \fphi\brs{\ocscen(\ocsH)}
    && \text{by hypothesis (1)}
  \\&\eqd \argmin_{x\in\ocsG}\max_{y\in\ocsG}\fphi\brs{\metric{x}{y}}\psp(y)
    && \text{by definition of $\ocscen$ \xref{def:ocscen}}
  \\&= \argmin_{x\in\ocsG}\max_{y\in\ocsG}\fphi\brs{\metric{x}{y}}
    && \text{by hypothesis (3) and \prefp{lem:argminmaxphi}}
  \\&= \argmin_{x\in\ocsG}\max_{y\in\ocsG}\metric{x}{y}
    && \text{by hypothesis (2) and \prefp{lem:argminmaxphi}}
  \\&= \ocscen(\ocsG)
    && \text{by definition of $\ocscen$ \xref{def:ocscen}}
\end{align*}
\end{proof}

\begin{figure}[h]
  \gsize%
  \centering%
  \psset{unit=7.5mm}%
  $\ds
  \text{\Large$\fphi$}
  %\brp{\mcom{\begin{array}{c}{\input{../common/math/graphics/ocs/ocs_rdie.tex}}\end{array}}{\structe{real die} \xref{ex:realdie}}}
  \brp{\mcom{\begin{array}{c}{\includegraphics{../common/math/graphics/pdfs/ocs_rdie.pdf}}\end{array}}{\structe{real die} \xref{ex:realdie}}}
  \qquad\text{\Large$=$}\qquad
  \text{\Large$\fphi\circ\fg$}
  %\brp{\mcom{\begin{array}{c}{\input{../common/math/graphics/ocs/ocs_spinner.tex}}\end{array}}{\structe{spinner} \xref{ex:spinner}}}
  \brp{\mcom{\begin{array}{c}{\includegraphics{../common/math/graphics/pdfs/ocs_spinner.pdf}}\end{array}}{\structe{spinner} \xref{ex:spinner}}}
  \qquad\text{\Large$=$}\qquad
  %\brp{\mcom{\begin{array}{c}{%============================================================================
% Daniel J. Greenhoe
% LaTeX file
% discrete metric real dice mapping to linearly ordered L6
%============================================================================
\begin{pspicture}(-1.4,-1.4)(1.4,1.4)%
  %---------------------------------
  % options
  %---------------------------------
  \psset{%
    linecolor=blue,%
    radius=1.25ex,
    labelsep=2.5mm,
    }%
  %---------------------------------
  % dice graph
  %---------------------------------
  \rput(0,0){%\psset{unit=2\psunit}%
    \uput{1}[210](0,0){\Cnode[fillstyle=solid,fillcolor=snode](0,0){D4}}%
    \uput{1}[150](0,0){\Cnode[fillstyle=solid,fillcolor=snode](0,0){D5}}%
    \uput{1}[ 90](0,0){\Cnode[fillstyle=solid,fillcolor=snode](0,0){D6}}%
    \uput{1}[ 30](0,0){\Cnode[fillstyle=solid,fillcolor=snode](0,0){D3}}%
    \uput{1}[-30](0,0){\Cnode[fillstyle=solid,fillcolor=snode](0,0){D2}}%
    \uput{1}[-90](0,0){\Cnode[fillstyle=solid,fillcolor=snode](0,0){D1}}%
    }%
  \rput(D6){$\diceF$}%
  \rput(D5){$\diceE$}%
  \rput(D4){$\diceD$}%
  \rput(D3){$\diceC$}%
  \rput(D2){$\diceB$}%
  \rput(D1){$\diceA$}%
  %
  \ncline{D5}{D6}%
  \ncline{D4}{D5}\ncline{D4}{D6}%
  \ncline{D3}{D5}\ncline{D3}{D6}%
  \ncline{D2}{D3}\ncline{D2}{D4}\ncline{D2}{D6}%
  \ncline{D1}{D2}\ncline{D1}{D3}\ncline{D1}{D4}\ncline{D1}{D5}%
  \ncline{D3}{D4}%
  \ncline{D2}{D5}%
  \ncline{D1}{D6}%
  %
  \uput[ 158](D6){$\frac{1}{6}$}
  \uput[ 150](D5){$\frac{1}{6}$}
  \uput[ 210](D4){$\frac{1}{6}$}
  \uput[  22](D3){$\frac{1}{6}$}
  \uput[ -45](D2){$\frac{1}{6}$}
  \uput[-158](D1){$\frac{1}{6}$}
\end{pspicture}}\end{array}}{\structe{fair die} \xref{ex:fairdie}}}
  \brp{\mcom{\begin{array}{c}{\includegraphics{../common/math/graphics/pdfs/ocs_fdie.pdf}}\end{array}}{\structe{fair die} \xref{ex:fairdie}}}
  $
  \caption{\fncte{discrete metric preserving function} $\fphi$ on outcome subspaces \xref{ex:ocsop_discrete}\label{fig:ocsop_mpf_discrete}}
\end{figure}
%---------------------------------------
\begin{example}[\exmd{discrete metric transform on outcome subspaces}]
\label{ex:ocsop_discrete}
%---------------------------------------
Let $\fg$ be a function (a \fncte{pullback function} \xrefnp{thm:pullback}) such that
$\fg(\circOne)=\diceA$,
$\fg(\circTwo)=\diceB$,
$\fg(\circThree)=\diceC$,
$\fg(\circFour)=\diceF$,
$\fg(\circFive)=\diceE$, and
$\fg(\circSix)=\diceD$.
Then under the \fncte{dicrete metric preserving function} $\fphi$ \xref{ex:mpf_discrete}
the \structe{real die outcome subspace} \xref{ex:realdie} becomes 
the \structe{fair die outcome subspace} \xref{ex:fairdie},
and under $\fphi\circ\fg$ 
the \structe{spinner outcome subspace} \xref{ex:spinner} also becomes 
the \structe{fair die outcome subspace},
as illustrated in \prefpp{fig:ocsop_mpf_discrete}.
This yields the following geometric statistics:
\\\indentx$\ocscen(\ocsG)=\ocscen(\ocsH)=\setn{\diceA,\diceB,\diceC,\diceD,\diceE,\diceF}$ .\\
%with the equalities being just as predicted by \prefpp{thm:ocsop_mpf}.
\end{example}





\begin{figure}[h]
  \gsize%
  \centering%
  \psset{unit=6mm}%
  $\ds
  \text{\Large$\fphi_{1}$}
  %\brp{\mcom{\begin{array}{c}{\input{../common/math/graphics/ocs/ocs_spinner_121141.tex}}\end{array}}{\structe{spinner} \xref{ex:spinner}}}
  \brp{\mcom{\begin{array}{c}{\includegraphics{../common/math/graphics/pdfs/ocs_spinner_121141.pdf}}\end{array}}{\structe{spinner}}}
  \text{\Large$=$}
  %\brp{\mcom{\begin{array}{c}{\input{../common/math/graphics/ocs/ocs_wagon_121141.tex}}\end{array}}{\structe{wagon wheel}}}
  \brp{\mcom{\begin{array}{c}{\includegraphics{../common/math/graphics/pdfs/ocs_wagon_121141.pdf}}\end{array}}{\structe{wagon wheel}}}
  \text{\Large$.$}
  \quad%
  \text{\Large$\fphi_{2}\circ\fg$}
  %\brp{\mcom{\begin{array}{c}{\input{../common/math/graphics/ocs/ocs_spinner_121141.tex}}\end{array}}{\structe{spinner} \xref{ex:spinner}}}
  \brp{\mcom{\begin{array}{c}{\includegraphics{../common/math/graphics/pdfs/ocs_spinner_121141.pdf}}\end{array}}{\structe{spinner}}}
  \text{\Large$=$}
  %\brp{\mcom{\begin{array}{c}{\input{../common/math/graphics/ocs/ocs_wdie_121141.tex}}\end{array}}{\structe{weighted die} \xref{ex:wdie}}}
  \brp{\mcom{\begin{array}{c}{\includegraphics{../common/math/graphics/pdfs/ocs_wdie_121141.pdf}}\end{array}}{\structe{weighted die}}}
  \text{\Large$.$}
  $
  \caption{\prefpp{ex:mpf_0121} \fncte{metric preserving function} $\fphi_{1}$  and 
           \prefpp{ex:mpf_x1x2} \fncte{metric preserving function} $\fphi_{2}$
           on \structe{spinner outcome subspace} 
           \xxref{ex:ocsop_0121}{ex:ocsop_x1x2}\label{fig:ocsop_x1x2_0121}}
\end{figure}
%---------------------------------------
\begin{example}
\label{ex:ocsop_0121}
%---------------------------------------
Let $\fphi_{1}$ be the \fncte{metric preserving function} defined in \prefpp{ex:mpf_0121}.
Then under $\fphi_{1}$, the \structe{spinner outcome subspace} \xref{ex:spinner} becomes what is here called the 
\structe{wagon wheel output subspace},
as illustrated on the left in \prefpp{fig:ocsop_x1x2_0121}.
Let $\ocsG$ be the \structe{spinner outcome subspace} and $\ocsH$    the \structe{wagon wheel outcome subspace}.
This yields the following geometric statistics:
\\\indentx
 $\ocscen(\ocsG)=\setn{\circFour,\circSix} \qquad \ocscen(\ocsH)=\setn{\circFive}$ .
\\
Note that the metric transform $\fphi_{1}$ also moves the \structe{outcome center} from one that is 
\emph{not} maximally likely, to one that \emph{is}.
\end{example}
\begin{proof}
\begin{align*}
  \ocscen(\ocsG)
    &\eqd \mathrlap{\argmin_{x\in\ocsG}\max_{y\in\ocsG}\metric{x}{y}\psp(y)
    \qquad\text{by definition of $\ocscen$ \xref{def:ocscen}}}
  \\&=\argmin_{x\in\ocsG}\max_{y\in\ocsG}\frac{1}{10}
         \setn{\begin{array}{cccccc}
           {0}\times1&{1}\times2 & {2}\times1 &{3}\times1 &{2}\times4 &{1}\times1\\
           {1}\times1&{0}\times2 & {1}\times1 &{2}\times1 &{3}\times4 &{2}\times1\\
           {2}\times1&{1}\times2 & {0}\times1 &{1}\times1 &{2}\times4 &{3}\times1\\
           {3}\times1&{2}\times2 & {1}\times1 &{0}\times1 &{1}\times4 &{2}\times1\\
           {2}\times1&{3}\times2 & {2}\times1 &{1}\times1 &{0}\times4 &{1}\times1\\
           {1}\times1&{2}\times2 & {3}\times1 &{2}\times1 &{1}\times4 &{0}\times1
         \end{array}}
     &&= \argmin_{x\in\ocsG}\frac{1}{10}
         \setn{\begin{array}{c}
            8\\
           12\\
            8\\
            4\\
            6\\
            4
         \end{array}}
     &&= \setn{\begin{array}{c}
           \mbox{ }\\
           \mbox{ }\\
           \mbox{ }\\
           \circFour\\
           \mbox{ }\\
           \circSix
         \end{array}}
  \\
  \ocscen(\ocsH)
    &\eqd \mathrlap{\argmin_{x\in\ocsG}\max_{y\in\ocsG}\metric{x}{y}\psp(y)
    \qquad\text{by definition of $\ocscen$ \xref{def:ocscen}}}
  \\&=\argmin_{x\in\ocsH}\max_{y\in\ocsH}\frac{1}{10}
         \setn{\begin{array}{cccccc}
           {0}\times1&{1}\times2 & {2}\times1 &{1}\times1 &{2}\times4 &{1}\times1\\
           {1}\times1&{0}\times2 & {1}\times1 &{2}\times1 &{1}\times4 &{2}\times1\\
           {2}\times1&{1}\times2 & {0}\times1 &{1}\times1 &{2}\times4 &{1}\times1\\
           {1}\times1&{2}\times2 & {1}\times1 &{0}\times1 &{1}\times4 &{2}\times1\\
           {2}\times1&{1}\times2 & {2}\times1 &{1}\times1 &{0}\times4 &{1}\times1\\
           {1}\times1&{2}\times2 & {1}\times1 &{2}\times1 &{1}\times4 &{0}\times1
         \end{array}}
    &&= \argmin_{x\in\ocsH}\frac{1}{10}
         \setn{\begin{array}{c}
            8\\
            4\\
            8\\
            4\\
            2\\
            4
         \end{array}}
    &&= \setn{\begin{array}{c}
           \mbox{ }\\
           \mbox{ }\\
           \mbox{ }\\
           \mbox{ }\\
           \circFive\\
           \mbox{ }
         \end{array}}
\end{align*}
\end{proof}

%---------------------------------------
\begin{example}
\label{ex:ocsop_x1x2}
%---------------------------------------
Let $\fphi_{2}$ be the \fncte{metric preserving function} defined in \prefpp{ex:mpf_x1x2}.
Let $\fg$ be the function defined in \prefpp{ex:ocsop_discrete}.
Then under under $\fphi_{2}\circ\fg$, 
the \structe{spinner outcome subspace} \xref{ex:spinner} becomes 
the \structe{weighted die outcome subspace} \xref{ex:wdie},
as illustrated on the right in \prefpp{fig:ocsop_x1x2_0121}.
Let $\ocsG$ be the \structe{spinner outcome subspace} and $\ocsH$ the \structe{weighted die outcome subspace}.
These structures have the following geometric statistics:
\\\indentx
 $\ocscen(\ocsG)=\setn{\circFour,\circSix} \qquad \ocscen(\ocsH)=\setn{\circOne,\circThree,\circFour,\circFive,\circSix}$ .
\\
Note that in \prefpp{ex:ocsop_0121}, the metric transform $\fphi_{1}$ results in a smaller 
(smaller \fncte{cardinality} \xrefnp{def:seto}) \structe{center} 
($\seto{\ocscen(\ocsG)}=2>1=\seto{\ocscen(\ocsH)}$).
But here, the metric transform $\fphi_{2}\circ\fg$ results in a larger \structe{center}.
($\seto{\ocscen(\ocsG)}=2<5=\seto{\ocscen(\ocsH)}$).
\end{example}
\begin{proof}
\begin{align*}
  \ocscen(\ocsG)
    &= \setn{\circFour,\circSix}
    && \text{by \prefpp{ex:ocsop_0121}}
    \\
  \ocscen(\ocsH)
    &\eqd \argmin_{x\in\ocsG}\max_{y\in\ocsG}\metric{x}{y}\psp(y)
    && \text{by definition of $\ocscen$ \xref{def:ocscen}}
  \\&= \mathrlap{\argmin_{x\in\ocsH}\max_{y\in\ocsH}\frac{1}{10}
         \setn{\begin{array}{cccccc}
           {0}\times1&{1}\times2 & {1}\times1 &{1}\times1 &{1}\times4 &{2}\times1\\
           {1}\times1&{0}\times2 & {1}\times1 &{1}\times1 &{2}\times4 &{1}\times1\\
           {1}\times1&{1}\times2 & {0}\times1 &{2}\times1 &{1}\times4 &{1}\times1\\
           {1}\times1&{1}\times2 & {2}\times1 &{0}\times1 &{1}\times4 &{1}\times1\\
           {1}\times1&{2}\times2 & {1}\times1 &{1}\times1 &{0}\times4 &{1}\times1\\
           {2}\times1&{1}\times2 & {1}\times1 &{1}\times1 &{1}\times4 &{0}\times1
         \end{array}}
      = \argmin_{x\in\ocsH}\frac{1}{10}
         \setn{\begin{array}{c}
            4\\
            8\\
            4\\
            4\\
            4\\
            4
         \end{array}}
      = \setn{\begin{array}{c}
           \circOne\\
           \mbox{ }\\
           \circThree\\
           \circFour\\
           \circFive\\
           \circSix\\
         \end{array}}}
\end{align*}
\end{proof}





\begin{figure}[h]
  \gsize%
  \centering%
  {\includegraphics{../common/math/graphics/pdfs/rline_argminmax_11312a.pdf}}%
  %{%============================================================================
% Daniel J. Greenhoe
% XeLaTeX file
%============================================================================
%\psset{unit=8mm}
\begin{pspicture}(-4.5,-0.5)(10.5,2.5)%
  \psset{%
    labelsep=1pt,
    linewidth=1pt,
    }%
  \psaxes[linecolor=axis,yAxis=false]{<->}(0,0)(-4,0)(10,2.5)% x axis
  \psaxes[linecolor=axis,xAxis=false]{ ->}(0,0)(-4,0)(2,2.5)% y axis
  %\psline[linewidth=3pt,linecolor=yellow](-3,1.875)(-2,1.5)(2.667,0.333)(3,0.375)(9,2.625)%
  %\psline(-3,0.375)(0,0)(9,1.125)% 1/8 |x|
  %\psline(-3,0.5)(1,0)(9,1)%       1/8 |x-1|
  %\psline(-3,1.875)(2,0)(9,2.625)% 3/8 |x-2|
  %\psline(-3,0.75)(3,0)(9,0.75)%   1/8 |x-3|
  %\psline(-3,1.75)(4,0)(9,1.25)%   2/8 |x-4|
  %
  \psplot[plotpoints=64,linewidth=3pt,linecolor=yellow]{ 1}{2.333}{x 4 sub abs 2.060 exp 2 mul 8 div}
  \psplot[plotpoints=64,linewidth=3pt,linecolor=yellow]{2.33}{4}{x abs 2.060 exp 8 div}
  %
  \psplot[plotpoints=64]{-3}{4}{x abs 2.060 exp 8 div}
  \psplot[plotpoints=64]{-3}{5}{x 1 sub abs 2.060 exp 8 div}
  \psplot[plotpoints=64]{-0.5}{4}{x 2 sub abs 2.060 exp 3 mul 8 div}
  \psplot[plotpoints=64]{-1}{6}{x 3 sub abs 2.060 exp 8 div}
  \psplot[plotpoints=64]{ 1}{6}{x 4 sub abs 2.060 exp 2 mul 8 div}
  \psline[linestyle=dotted,linecolor=red](2.333,0.716)(2.333,0)%
  \psline[linestyle=dotted,linecolor=red](2.333,0.716)(0,0.716)%
  %\psline[linestyle=dotted,linecolor=red](-2,1.5)(-2,0)%
  %\psline[linestyle=dotted,linecolor=red](-2,1.5)(0,1.5)%
  \psline[linestyle=dotted,linecolor=red](2.667,0.333)(2.667,0)%
  \psline[linestyle=dotted,linecolor=red](2.667,0.333)(0,0.333)%
  %\psline[linestyle=dotted,linecolor=red](3,0.375)(3,0)%
  %\psline[linestyle=dotted,linecolor=red](-2,1.5)(-2,0)%
  \uput[0]{0}(10,0){$x$}%
  \uput[-90]{0}(2.333,0){$\frac{7}{3}$}%
  \uput[-90]{0}(2.667,0){$\frac{8}{3}$}%
  \uput[180]{0}(0,0.333){$\frac{1}{3}$}%
  \uput[180]{0}(0,0.716){$\approx0.716$}%
  \uput[0]{0}(0,1.5){$1.5$}%
\end{pspicture}
}%
  \caption{real line addition $\argmin_x\max_y$ calculation graph \xref{ex:rline_11312a}\label{fig:rline_11312a}}
\end{figure}
%---------------------------------------
\begin{example}[\exmd{linear addition with metric transform}]
\label{ex:rline_11312a}
%---------------------------------------
\prefpp{ex:rline_11312} gave the result $\ocsE(\rvX+\rvY) = \frac{8}{3}$ rather than the perhaps more desirable result of 
$\frac{7}{3}$ (which equals $1 + \frac{4}{3} = \ocsE(\rvX) + \ocsE(\rvY)$).
Again, we adjust the geometric statistics of an outcome subspace by use of a \structe{metric preserving function} \xref{def:mpf}.
In particular, we use the \exme{power transform}/\exme{snowflake transform} \xref{ex:mpf_snowflake}
$\ff(x)= x^a$. If we let $a=\frac{\ln2}{\ln7-\ln5}\approx2.0600427$, then
$\ocsE(\rvX+\rvY) = \frac{7}{3}$, as illustrated in \prefpp{fig:rline_11312a}.
\end{example}



\end{tabstr}




