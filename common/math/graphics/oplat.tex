%============================================================================
% Daniel J. Greenhoe
% LaTeX file
% lattice ({factors of 30}, |)
%============================================================================


  \psset{xunit=1.2mm,yunit=1.5mm}
  \begin{pspicture}(-60,-24)(60,32)%
     \footnotesize
     \psset{%
       linecolor=blue,
       cornersize=relative,
       framearc=0.25,
       gridcolor=graph,
       subgriddiv=1,
       gridlabels=4pt,
       gridwidth=0.2pt,
       }%
     \begin{tabstr}{0.75}
     \rput(  0, 30){\rnode{relations} {\psframebox{\begin{tabular}{c}relations\end{tabular}}}}%
     \rput(  0, 20){\rnode{functions} {\psframebox{\begin{tabular}{c}functions\end{tabular}}}}%
     \rput(  0, 10){\rnode{op}        {\psframebox{\begin{tabular}{c}operators\\\ifxref{operator}{def:operator}\end{tabular}}}}%
     \rput( 40,  0){\rnode{nonlinop}  {\psframebox{\begin{tabular}{c}non-linear   operators\end{tabular}}}}%
     \rput(  0,  0){\rnode{linop}     {\psframebox{\begin{tabular}{c}linear       operators\\\ifxref{operator}{def:linop}\end{tabular}}}}%
     \rput(-40,-10){\rnode{opN}       {\psframebox{\begin{tabular}{c}normal       operators\\$\opNa\opN=\opN\opNa$\\\ifxref{operator}{def:op_normal}\end{tabular}}}}%
     \rput(  0,-10){\rnode{opP}       {\psframebox{\begin{tabular}{c}projection   operators\\$\opP^2=\opP$\\\ifxref{operator}{def:opP}\end{tabular}}}}%
     \rput( 40,-10){\rnode{opIso}     {\psframebox{\begin{tabular}{c}isometric    operators\\$\opAa\opA=\opI$\\\ifxref{operator}{def:op_isometric}\end{tabular}}}}%
     \rput(-40,-20){\rnode{opSA}      {\psframebox{\begin{tabular}{c}self adjoint operators\\$\opA=\opAa$\\\ifxref{operator}{def:op_selfadj}\end{tabular}}}}%
     \rput(  0,-20){\rnode{opU}       {\psframebox{\begin{tabular}{c}unitary      operators\\$\opU\opUa=\opUa\opU=\opI$\\\ifxref{operator}{def:op_unitary}\end{tabular}}}}%
     \end{tabstr}
     %
     \ncline{relations}{functions}%
     \ncline{functions}{op}%
     \ncline{op}{linop}%
     \ncline{op}{nonlinop}%
     \ncline{linop}{opN}%
     \ncline{linop}{opP}%
     \ncline{linop}{opIso}%
     \ncline{opN}  {opSA}%
     \ncline{opU}  {opN}%
     \ncline{opU}  {opIso}%
     %
     %\psgrid[unit=10mm](-8,-1)(8,9)%
  \end{pspicture}
