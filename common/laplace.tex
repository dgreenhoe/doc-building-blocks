%============================================================================
% XeLaTeX File
% Daniel J. Greenhoe
%============================================================================
%=======================================
\chapter{Laplace Transform}
%=======================================

%=======================================
\section{Definition}
%=======================================
%--------------------------------------
\begin{definition}
\label{def:opLT}
%--------------------------------------
Let $\spLLRBu$ be the space of all 
\structe{Lebesgue square-integrable functions}.\\
\defboxt{
  The \opd{Laplace Transform} operator $\opLT$ is here defined as
  \\\indentx$\ds
    \brs{\hxs{\opLT} \ff}(\omega)
    \eqd \fscale 
    \int_{x\in\R} \ff(x) e^{-sx} \dx
    \qquad\scy\forall \ff\in\spLLRBu
  $
  %\\This definition of the Fourier Transform is also called the \opd{unitary Laplace Transform}.
  }
\end{definition}

In \pref{def:opLT}, the scaling factor $\fscale$ is not normally found 
in most definitions of the Laplace Transform.
However it is included here to make the operator $\opLT$ more directly compatible 
with the \ope{Unitary Fourier Transform} operator $\opFT$\ifsxref{harFour}{def:opFT}.

%=======================================
\section{Shift relations}
%=======================================
%--------------------------------------
\begin{theorem}[Shift relations]
\label{thm:opLT_shift}
%--------------------------------------
Let $\opLT$ be the \ope{Laplace Transform} operator \xref{def:opLT}.
\thmbox{\begin{array}{>{\ds}rc>{\ds}l}
  \opLT \brs{\ff(x-y)}(s)           &=& e^{-sy} \;\brs{\opLT \ff(x)}(s) \\
  \brs{\opLT\brp{e^{rx} \fg(x)}}(s) &=& \brs{\opLT\fg(x)}(s-r)
\end{array}}
\end{theorem}
\begin{proof}
\begin{align*}
  \opLT \brs{\ff(x-y)}(s)
    &= \fscale \int_{x\in\R} \ff(x-y) e^{-sx} \dx         && \text{by definition of $\opLT$} && \text{\xref{def:opLT}}
  \\&= \fscale \int_{u\in\R} \ff(u) e^{-s(y+u)} \du       && \text{where $u\eqd x-y$}        && \text{$\implies$ $x=y+u$}
  \\&= e^{-sy} \;\fscale \int_{u\in\R} \ff(u) e^{-su} \du && \text{by property of exponents} && \text{$a^{x+y}=a^xa^y$}
  \\&= e^{-sy} \;\fscale \int_{x\in\R} \ff(x) e^{-sx} \du && \text{by change of variable}    && \text{$u\to x$}
  \\&= e^{-sy} \;\brs{\opLT \ff(x)}(s)                    && \text{by definition of $\opLT$} && \text{\xref{def:opLT}}
  \\
  \brs{\opLT\brp{e^{rx} \fg(x)}}(s)
    &= \fscale \int_{x\in\R} e^{rx} \fg(x) e^{-sx} \dx    && \text{by definition of $\opLT$} && \text{\xref{def:opLT}}
  \\&= \fscale \int_{x\in\R} \fg(x) e^{-(s-r)x} \dx       && \text{by property of exponents} && \text{$a^{x+y} = a^x a^y$}
  \\&= \brs{\opLT\fg(x)}(s-r)                             && \text{by definition of $\opLT$} && \text{\xref{def:opLT}}
\end{align*}
\end{proof}

%======================================
\section{Convolution relations}
%======================================
%--------------------------------------
\begin{definition}
\label{def:conv}
\footnote{
  \citerp{bachman1964}{6},
  \citerpgc{bracewell1978}{108}{007007013X}{Convolution theorem}
  }
%--------------------------------------
\defbox{\indxs{\conv}\begin{array}{M}
  The \opd{convolution operation} is defined as % $\conv\in\clF{\R^2}{\R}$ is defined as
  \\\indentx$\ds
    \brs{\ff\conv\fg}(x) 
    \eqd \ff(x)\conv \fg(x)
    \eqd \int_{u\in\R} \ff(u)\fg(x-u) \du
    \qquad\scy\forall\ff,\fg\in\spLLRBu
  $
\end{array}}
\end{definition}

\pref{thm:opLT_conv} (next) demonstrates that multiplication in the ``time domain"
is equivalent to convolution in the ``s domain" and
vice-versa.
%--------------------------------------
\begin{theorem}[\thmd{convolution theorem}]
\label{thm:opLT_conv}
%--------------------------------------
Let $\opLT$ be the \ope{Laplace Transform} operator \xref{def:opLT}
and $\conv$ the convolution operator \xref{def:conv}.
\thmbox{
\begin{array}{rcl@{\qquad}C}
  \mcom{\opLT\brs{\ff(x)\conv\fg(x)}(\omega)}{convolution    in ``time domain"} 
       &=& \mcom{\sqrt{2\pi}\brs{\opLT\ff}(s)\, \brs{\opLT\fg}(s)}  {multiplication in ``s domain"} 
       & \forall\ff,\fg\in\spLLRBu\\
  \mcom{\opLT\brs{\ff(x) \fg(x)}(\omega)} {multiplication in ``time domain"} 
       &=& \mcom{\fscale\brs{\opLT\ff}(s) \conv \brs{\opLT\fg}(s)}{convolution    in ``s domain"} 
       & \forall\ff,\fg\in\spLLRBu.
\end{array}
}
\end{theorem}
\begin{proof}
\begin{align*}
   \opLT \brs{\ff(x)\conv\fg(x)}(s)
     &= \opLT \brs{ \int_{u\in\R} \ff(u)\fg(x-u) \du }(s)
     && \text{by definition of $\conv$} &&\text{\xref{def:conv}}
   \\&=  \int_{u\in\R} \ff(u)\brs{\opLT \fg(x-u)}(s) \du 
   \\&=  \int_{u\in\R} \ff(u) e^{-su} \; \brs{\opLT\fg(x)}(s) \du 
     && \text{by \thme{Fourier shift theorem}} &&\text{\xref{thm:ft_shift}}
   \\&= \sqrt{2\pi}\mcom{\brp{\fscale\int_{u\in\R} \ff(u) e^{-su} \du}}
             {$\brs{\opLT\ff}(s)$} \; 
        \brs{\opLT\fg}(s)
   \\&= \sqrt{2\pi}\brs{\opLT\ff}(s)\;  \brs{\opLT\fg}(s)
     && \text{by definition of $\opLT$} &&\text{\xref{def:opFT}}
   \\
   \opLT[\ff(x)\fg(x)](s)
     &= \opLT\brs{\brp{\opLT^{-1}\opLT\ff(x)}\;\fg(x)}(s)
     && \text{by def. of operator inverse} &&\text{\ifxref{operator}{def:op_inv}}
   \\&= \opLT\brs{\brp{ \fscalei \int_{v\in\R} \brs{\opLT\ff(x)}(v) e^{sxv} \dv }\;\fg(x)}(s)
     && \text{by \prefp{thm:opFTi}}
   \\&= \fscalei \int_{v\in\R} \brs{\opLT\ff(x)}(v) \brs{\opLT\brp{e^{sxv} \;\fg(x)}}(s,v) \dv 
   \\&= \fscalei \int_{v\in\R} \brs{\opLT\ff(x)}(v) \brs{\opLT\fg(x)}(s-v) \dv 
     && \text{by \prefp{thm:ft_shift}}
   \\&= \fscalei\brs{\opLT\ff}(s)\conv \brs{\opLT\fg}(s)
     && \text{by definition of $\conv$} &&\text{\xref{def:conv}}
\end{align*}
\end{proof}

%=======================================
\section{Calculus relations}
%=======================================
%---------------------------------------
\begin{theorem}
\label{thm:opLT_diff}
%---------------------------------------
Let $\opLT$ be the \ope{Laplace Transform} operator \xref{def:opLT}.
\mbox{}\\
\thmbox{
    \brb{\lim_{t\to-\infty}\fx(t)=0}
    \qquad\implies\qquad
    \brb{\opLT \brs{\ddt \fx(t)} = s \brs{\opLT\fx}(s)}
  }
\end{theorem}
\begin{proof}
\begin{align*}
  \boxed{\opLT \brs{\ddt \fx(t)}} 
    &\eqd \int_{t\in\R} \mcom{\brs{\ddt \fx(t)}}{$\dv$} \mcom{e^{-st}}{$u$} \dt
    && \text{by definition of $\opLT$}
  \\&= \brlr{\mcom{e^{-st}}{$u$} \mcom{\fx(t)}{$v$}}_{t=-\infty}^{t=+\infty}
      -\int_{t\in\R} \mcom{\fx(t)}{$v$} \mcom{(-s)e^{-s t}}{$\du$} \dt
    && \text{by \thme{Integration by Parts}}
  \\&= \cancelto{0}{e^{-s \infty}}\fx(\infty) - e^{s \infty}\cancelto{0}{\fx(-\infty)} 
      -(-s)\mcom{\int_{t\in\R} \fx(t) e^{-st} \dt}{\ope{Laplace Transform} of $\fx(t)$}
    && \text{by left hypothesis}
  \\&= \boxed{s \brs{\opLT\fx}(s)}
\end{align*}
\end{proof}

%---------------------------------------
\begin{theorem}
\label{thm:opLT_int}
%---------------------------------------
Let $\opLT$ be the \ope{Laplace Transform} operator \xref{def:opLT}.
\thmbox{
    %\brb{\lim_{t\to-\infty}\fx(t)=0}
    %\qquad\implies\qquad
    \opLT \int_{u=-\infty}^{u=t} \fx(u) \du = \frac{1}{s} \brs{\opLT\fx}(s)
  }
\end{theorem}
\begin{proof}
\begin{enumerate}
  \item Define the \fncte{Heaviside function} $\fh(t)$ as\qquad \label{idef:heaviside}
    $\ds \fh(t)\eqd\brbl{\begin{array}{lM}
                           1 & for $t\ge0$
                         \\0 & otherwise
                         \end{array}}$

  \item Remainder of proof\ldots
    \begin{align*}
      \boxed{\opLT \int_{u=-\infty}^{u=t} \fx(u) \du}
        &\eqd \int_{t=-\infty}^{t=+\infty} \brs{\int_{u=-\infty}^{u=t} \fx(u) \du} e^{-s t} \dt
        && \text{by definition of $\opLT$}
      \\&= \int_{t=-\infty}^{t=+\infty} \brs{\int_{u=-\infty}^{u=+\infty} \fx(u) \fh(t-u) \du} e^{-s t} \dt
        && \brp{\begin{array}{M}
             by definition of \fncte{Heaviside function}\\
             \pref{idef:heaviside}
           \end{array}}
      \\&= \int_{v=-\infty}^{v=+\infty} \int_{u=-\infty}^{u=+\infty} \fx(u) \fh(v)  e^{-s (u+v)} \du \dv
        && \brp{\begin{array}{Mr@{\hspace{2pt}}c@{\hspace{2pt}}l}
             where      & v&\eqd&t-u\\ 
             $\implies$ & t&=&u+v
           \end{array}}
      \\&= \brs{\int_{v=-\infty}^{v=+\infty} \fh(v) e^{-s v} \dv} 
            \mcom{\brs{\int_{u=-\infty}^{u=+\infty} \fx(u)   e^{-s u} \du }}
                 {\ope{Laplace Transform} of $\fx(t)$}
      \\&= \brs{\int_{v=0}^{v=\infty} e^{-s v} \dv} \brs{\opLT\fx}(s)
        && \brp{\begin{array}{M}
             by definition of \fncte{Heaviside function}\\
             \pref{idef:heaviside}
           \end{array}}
      \\&= \brlr{\frac{1}{-s}e^{-s v}}_{v=0}^{v=\infty} \brs{\opLT\fx}(s)
        && \text{by \thme{Fundamental Theorem of Calculus}}
      \\&= \boxed{\frac{1}{s} \brs{\opLT\fx}(s)}
    \end{align*}
\end{enumerate}
\end{proof}

