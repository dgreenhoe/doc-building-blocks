%============================================================================
% LaTeX File
% Daniel J. Greenhoe
%============================================================================

%======================================
\chapter{Continuous Random Processes}
\label{app:random_processes}
%======================================
\qboxnps
  {Aristotle (384 BC -- 322 BC)
    \index{Aristotle}
    \index{quotes!Aristotle}
    \footnotemark
  }
  {../common/people/aristot.jpg}
  {A likely impossibility is always preferable to an
  unconvincing possibility.}
  \footnotetext{\begin{tabular}[t]{ll}
    quote: & \url{http://en.wikiquote.org/wiki/Aristotle} \\
    image: & \url{http://en.wikipedia.org/wiki/Aristotle}
  \end{tabular}}



%=======================================
%\section{Continuous-time random processes}
%=======================================
%=======================================
\section{Definitions}
%=======================================
%---------------------------------------
\begin{definition}
\index{random variable}
\index{random process}
\citetbl{
  \citerp{papoulis}{63},
  \citerp{papoulis}{285}
  }
%---------------------------------------
Let $\ps$ be a \structe{probability space}.\\
\defboxt{
  The function $\rvx:\pso\to\R$ is a \fnctd{random variable}.\\
  The function $\rvy:\R\times\pso\to\R$ is a \fnctd{random process}.
  }
\end{definition}

The random process $\rvx(t,\omega)$, where $t$ commonly represents time
and $\omega\in\pso$ is an outcome of an experiment,
can take on more specialized forms depending on whether
$t$ and $\omega$ are fixed or allowed to vary.
These forms are illustrated in \prefp{fig:X(t,w)}\footnote{\citerpp{papoulis}{285}{286}}
and \prefp{fig:X(t,w)graph}.

\begin{figure}[ht]\color{figcolor}
\begin{center}
   \begin{tabular}{|c||c|c|}
      \hline
         $\rvx(t,\omega)$ &  fixed $t$      & variable $t$   \\
      \hline
      \hline
         fixed    $\omega$ & number          & time function  \\
      \hline
         variable $\omega$ & random variable & random process \\
      \hline
   \end{tabular}
\caption{
   Specialized forms of a random process $\rvx(t,\omega)$
   \label{fig:X(t,w)}
   }
\end{center}
\end{figure}

\begin{figure}[ht]\color{figcolor}
\begin{center}
\includegraphics[height=8cm,width=12cm]{../common/x_tw.eps}
\end{center}
\caption{
  Example of a random process $\rvx(t,\omega)$
  \label{fig:X(t,w)graph}
}
\end{figure}


%---------------------------------------
\begin{definition}
\footnote{
  \citerpgc{papoulis1984}{216}{0070484686}{$R_{xy}(t_1,t_2)=E\brb{\rvx(t_1)\rvy^\ast(t_2)}$ (9-35)},
  }
\label{def:Rxx}
\label{def:opR}
\label{def:Rxy}
%---------------------------------------
Let $\rvx(t)$ and $\rvy(t)$ be random processes.\\
\defbox{
  \begin{array}{MlMlc>{\ds}l}
       The \fnctd{mean}                     & \pmeanx(t) & of $\rvx(t)$              is & \pmeanx(t)&\eqd& \pE\brs{\rvx(t)}                    %& \text{(\prope{mean functional})} \\
     \\The \fnctd{cross-correlation}        & \Rxy(t)    & of $\rvx(t)$ and $\rvy(t)$is & \Rxy(t,u) &\eqd& \pE\brs{\rvx(t)\rvy^\ast(u)}        %& \text{(\prope{cross-correlation bilinear functional})} \\
     \\The \fnctd{autocorrelation function} & \Rxx(t)    & of $\rvx(t)$              is & \Rxx(t,u) &\eqd& \pE\brs{\rvx(t)\rvx^\ast(u)}        %& \text{(\prope{auto-correlation bilinear functional})} \\
     \\The \fnctd{autocorrelation operator} & \opR\ff    & of $\ff(t)$               is & \opR f    &\eqd& \int_{u\in\R}\Rxx(t,u)\ff(u) \du     %& \text{(\prope{auto-correlation operator})}
  \end{array}
  }
\end{definition}

%---------------------------------------
\begin{remark}
\footnote{
  \citer{fredholm1900},
  \citerp{fredholm1903}{365},
  \citerp{michel1993}{97},
  \citerp{keener}{101}
  }
%---------------------------------------
The equation $\int_{u\in\R}\Rxx(t,u)\ff(u) \du$ is a
   \ope{Fredholm integral equation of the first kind} and
   $\Rxx(t,u)$ is the \ope{kernel} of the equation.
\end{remark}

%---------------------------------------
\section{Properties}
%---------------------------------------

%---------------------------------------
\begin{theorem}
\label{thm:Rxx_prop}
\index{cross-correlation}
\index{symmetric!conjugate}
\index{conjuage symmetric}
%---------------------------------------
Let $\fx(t)$ and $\fy(t)$ be random processes with
cross-correlation $\Rxy(t,u)$ and
let $\Rxx(t,u)$ be the auto-correlation of $\fx(t)$.
\thmbox{
\begin{array}{rcll}
   \Rxx(t,u) &=& \Rxx^\ast(u,t) & \text{(\prope{conjugate symmetric})}\\
   \Rxy(t,u) &=& \Ryx^\ast(u,t) & 
\end{array}
}
\end{theorem}
\begin{proof}
\begin{align*}
   \Rxx(t,u)
      &\eqd \pE\brs{\rvx(t) \rvx^\ast(u)}
      &=    \pE\brs{\rvx^\ast(u) \rvx(t)}
      &=    \brp{ \pE\brs{\rvx(u) \rvx^\ast(t)}}^\ast
      &\eqd \Rxx^\ast(u,t)
\\
   \Rxy(t,u)
      &\eqd \pE\brs{\rvx(t) \rvy^\ast(u)}
      &=    \pE\brs{\rvy^\ast(u) \rvx(t) }
      &=    \brp{\pE\brs{\rvy(u) \rvx^\ast(t)}}^\ast
      &\eqd \Ryx^\ast(u,t)
\end{align*}
\end{proof}


%---------------------------------------
\begin{theorem}
\index{non-negative}
\index{positive definite}
%---------------------------------------
Let $\opR:\spX\to\spX$ be an auto-correlation operator.
\thmbox{\begin{array}{lll}
  \inprod{\opR \fx}{\fx} \ge 0
    & \forall \fx\in\spX
    & \text{(\prope{non-negative})}  \\
  \inprod{\opR \fx}{\fy} = \inprod{\fx}{\opR \fy}
    & \forall \fx,\fy\in\spX
    & \text{(\prope{self-adjoint})}
\end{array}}
\end{theorem}
\begin{proof}
\begin{align*}
\intertext{1. Proof that $\opR$ is non-negative:}
   \inprod{\opR \fy}{\fy}
     &= \inprod{\int_{u\in\R}\Rxx(t,u) \fy(u) \du}{\fy(t)}
     && \text{by definition of $\opR$}
     && \text{\xref{def:opR}}
   \\&= \inprod{\int_{u\in\R}\pE\brs{\fx(t)\fx^\ast(u)} \fy(u) \du}{\fy(t)}
     && \text{by definition of $\Rxx(t,u)$}
     && \text{\xref{def:Rxx}}
   \\&= \pE\brs{\inprod{\int_{u\in\R}\fx(t)\fx^\ast(u) \fy(u) \du}{\fy(t)}}
     && \text{by \prope{linearity} of $\inprodn$ and $\int$}
     && \text{\ifsxref{vsinprod}{def:inprod}}
   \\&= \pE\brs{\int_{u\in\R}\fx^\ast(u) \fy(u) \du \inprod{\fx(t)}{\fy(t)}}
     && \text{by \prope{additivity} property of $\inprodn$}
     && \text{\ifsxref{vsinprod}{def:inprod}}
   \\&= \pE\brs{\inprod{\fy(u)}{\fx(u)} \inprod{\fx(t) }{\fy(t)}}
     && \text{by local definition of $\inprodn$}
     && \text{\ifsxref{vsinprod}{def:inprod}}
   \\&= \pE\brs{\inprod{\fx(u)}{\fy(u)}^\ast \inprod{\fx(t) }{\fy(t)}}
     && \text{by \prope{conjugate symmetry} prop.}
     && \text{\ifsxref{vsinprod}{def:inprod}}
   \\&= \pE{\abs{\inprod{\fx(t) }{\fy(t)}}^2}
     && \text{by definition of $\absn$} 
     && \text{\xref{def:abs}}
   \\&\ge 0
\intertext{2. Proof that $\opR$ is self-adjoint:}
   \inprod{\brs{\opR \fx}(t)}{\fy}
     &= \inprod{\int_{u\in\R}\Rxx(t,u) \fx(u) \du}{\fy(t)}
     && \text{by definition of $\opR$}
     && \text{\xref{def:opR}}
   \\&= \int_{u\in\R}\fx(u) \inprod{\Rxx(t,u)  }{\fy(t)} \du
     && \text{by \prope{additive} property of $\inprodn$}
     && \text{\ifsxref{vsinprod}{def:inprod}}
   \\&= \int_{u\in\R}\fx(u) \inprod{\fy(t)}{\Rxx(t,u)}^\ast \du
     && \text{by \prope{conjugate symmetry} prop.}
     && \text{\ifsxref{vsinprod}{def:inprod}}
   \\&= \inprod{ \fx(u) }{\inprod{\fy(t)}{\Rxx(t,u)} }
     && \text{by local definition of $\inprodn$}
     && \text{\ifsxref{vsinprod}{def:inprod}}
   \\&= \inprod{ \fx(u) }{\int_{t\in\R}\fy(t) \Rxx^\ast(t,u)\dt }
     && \text{by local definition of $\inprodn$}
     && \text{\ifsxref{vsinprod}{def:inprod}}
   \\&= \inprod{ \fx(u) }{\int_{t\in\R}\fy(t) \Rxx(u,t)\dt }
     && \text{by property of $\Rxx$}
     && \text{\xref{thm:Rxx_prop}}
   \\&= \inprod{ \fx(u) }{\mcom{\opR}{$\opRa$} \fy }
     && \text{by definition of $\opR$}
     && \text{\xref{def:opR}}
   \\\implies&\qquad \opR=\opRa \qquad\implies \text{$\opR$ is \prope{self adjoint}}
\end{align*}
\end{proof}

%---------------------------------------
\begin{theorem}
\footnote{
  \citerpp{keener}{114}{119}
  }
%---------------------------------------
Let $\seq{\lambda_n}{n\in\Z}$ be the eigenvalues and
    $\seq{\fpsi_n}{n\in\Z}$ be the eigenfunctions of
    operator $\opR$ such that
    $\opR \psi_n = \lambda_n \psi_n$.
\thmbox{\begin{array}{rlp{7cm}}
  1. & \lambda_n \in \R
     & (eigenvalues of $\opR$ are \prope{real})
     \\
  2. & \lambda_n\ne \lambda_m \implies \inprod{\psi_n}{\psi_m}=0
     & (eigenfunctions associated with distinct eigenvalues are \prope{orthogonal})
     \\
  3. & \norm{\psi_n(t)}^2>0 \implies \lambda_n\ge0
     & (eigenvalues are \prope{non-negative})
     \\
  4. & \norm{\psi_n(t)}^2>0, \inprod{\opR f}{f} > 0 \implies \lambda_n>0
     & (if $\opR$ is \prope{positive definite}, then eigenvalues are \prope{positive})
\end{array}}
\end{theorem}
\begin{proof}
\begin{enumerate}
\item Proof that eigenvalues are \prope{real-valued}:
Because $\opR$ is self-adjoint, its eigenvalues are real\ifsxref{operator}{thm:self_adjoint}.

\item eigenfunctions associated with distinct eigenvalues are orthogonal:
Because $\opR$ is self-adjoint, this property follows\ifsxref{operator}{thm:self_adjoint}.

\item Proof that eigenvalues are \prope{non-negative}:
\begin{align*}
   0 &\ge \inprod{\opR \psi_n}{\psi_n}
     &&   \text{by definition of non-negative definite}
   \\&=   \inprod{\lambda_n \psi_n}{\psi_n}
     &&   \text{by hypothesis}
   \\&=   \lambda_n \inprod{\psi_n}{\psi_n}
     &&   \text{by definition of inner-products}
   \\&=   \lambda_n \norm{\psi_n}^2
     &&   \text{by definition of norm induced by inner-product}
\end{align*}

\item Eigenvalues are \prope{positive} if $\opR$ is \prope{positive definite}:
\begin{align*}
   0 &> \inprod{\opR \psi_n}{\psi_n}
     && \text{by definition of \prope{positive definite}}
   \\&= \inprod{\lambda_n \psi_n}{\psi_n}
     && \text{by hypothesis}
   \\&= \lambda_n \inprod{\psi_n}{\psi_n}
     && \text{by definition of inner-products}
   \\&= \lambda_n \norm{\psi_n}^2
     && \text{by definition of norm induced by inner-product}
\end{align*}

\end{enumerate}
\end{proof}


%=======================================
\section{Basis for random processes}
\label{sec:KL}
%=======================================
If a random process $\rvx(t)$ is white
\footnote{{\em white noise process}: random process $\rvx(t)$ with autocorrelation $\Rxx(\tau)=\delta(\tau)$}
and $\Psi=\{\psi_1(t),\psi_2(t),\ldots,\psi_N(t)\}$ is \textbf{any} set of orthonormal basis functions,
then the innerproducts
$\inprod{n(t)}{\psi_n(t)}$ and $\inprod{n(t)}{\psi_m(t)}$ are \prope{uncorrelated}
for $m\ne  n$.
However, if $\rvx(t)$ is colored (not white), then the innerproducts are not
in general uncorrelated.
But if the elements of $\Psi$ are chosen to be the eigenfunctions of $\opR$ such
that
\[ \opR \psi_n = \lambda_n \psi_n,\]
then by \prefp{thm:Rxx_prop}, $\{\psi_n(t)\}$ are orthogonal and
the innerproducts \textbf{are} uncorrelated eventhough $\rvx(t)$ is
not white.
This criterion is called the  Karhunen-Lo\`{e}ve criterion for $\rvx(t)$.

%---------------------------------------
\begin{theorem}[\thmd{Karhunen-Lo/`eve Expansion}]
\footnote{
  \citerpp{keener}{114}{119}
  }
%---------------------------------------
Let $\opR$ be the \ope{autocorrelation operator} of a \fncte{random process} $\rvx(t)$.
\\
\thmboxt{
  $\brb{\begin{array}{FMD}
      (A).&$\seq{\lambda_n}{n\in\Z}$ are the eigenvalues of $\opR$ & and
    \\(B).&$\seq{\fpsi_n}{n\in\Z}$ are the eigenfunctions of $\opR$ & such that
    \\(C).&$\opR \psi_n = \lambda_n \psi_n$
  \end{array}}$
  \\\indentx$\implies\quad
  \brb{\begin{array}{M}
       $\ds\pE\brb{\abs{ x(t)-\sum_{n\in\Z}\inprod{x(t)}{\psi_n(t)} \psi_n(t) }^2} = 0$
     \\($\setn{\psi_n(t)}$ is a \structe{basis} for $\rvx(t)$)
  \end{array}}$
  }
\end{theorem}
\begin{proof}
\begin{enumerate}
\item $\{\psi_n(t)\}$ is a basis for $\rvx(t)$
      \[ \pE\brb{\left| x(t)-\sum_{n\in\Z}\dot{x}_n \psi_n(t) \right|^2} = 0
         \hspace{1cm}\mbox{where } \dot{x}_n\eqd \inprod{x(t)}{\psi_n(t)}
      \]

\begin{align*}
   \pE\brs{x(t) \brp{\sum_{n\in\Z}\dot{x}_n \psi_n(t)}^\ast}
     &= \pE\brs{x(t) \brp{\sum_{n\in\Z}\int_{u\in\R}x(u)\psi_n^\ast(u)\du \psi_n(t)}^\ast}
   \\&= \sum_{n\in\Z}\brp{\int_{u\in\R}\pE\brs{x(t)x^\ast(u)}\psi_n(u)\du} \psi_n^\ast(t)
   \\&= \sum_{n\in\Z}\brp{\int_{u\in\R}\Rxx(t,u)\psi_n(u)\du} \psi_n^\ast(t)
   \\&= \sum_{n\in\Z}\lambda_n\psi_n(t) \psi_n^\ast(t)
   \\&= \sum_{n\in\Z}\lambda_n \left|\psi_n(t) \right|^2
\\ \\
   \pE\brs{\sum_{n\in\Z}\dot{x}_n \psi_n(t)\brp{\sum_{m\in\Z}\dot{x}_m \psi_m(t)}^\ast}
     &= \pE\brs{\sum_{n\in\Z}\int_{u\in\R}x(u)\psi_n^\ast(u)\du   \psi_n(t)\brp{\sum_{m\in\Z}\int_v x(v)\psi_m^\ast(v)\dv \psi_m(t)}^\ast}
   \\&= \sum_{n\in\Z}\sum_{m\in\Z}\int_u\brp{\int_v \pE\brs{x(u)x^\ast(v)}\psi_m(v)\dv} \psi_n^\ast(u)\du   \psi_n(t)   \psi_m^\ast(t)
   \\&= \sum_{n\in\Z}\sum_{m\in\Z}\int_u\brp{\int_v \Rxx(u,v)\psi_m(v)\dv} \psi_n^\ast(u)\du   \psi_n(t)   \psi_m^\ast(t)
   \\&= \sum_{n\in\Z}\sum_{m\in\Z}\int_u\brp{\lambda_m\psi_m(u)} \psi_n^\ast(u)\du   \psi_n(t)   \psi_m^\ast(t)
   \\&= \sum_{n\in\Z}\sum_{m\in\Z}\lambda_m \brp{\int_{u\in\R}\psi_m(u) \psi_n^\ast(u)\du }   \psi_n(t)   \psi_m^\ast(t)
   \\&= \sum_{n\in\Z}\sum_{m\in\Z}\lambda_m \kdelta_{mn}   \psi_n(t)   \psi_m^\ast(t)
   \\&= \sum_{n\in\Z}\lambda_n   \psi_n(t)   \psi_n^\ast(t)
   \\&= \sum_{n\in\Z}\lambda_n  \left| \psi_n(t) \right|^2
\end{align*}


\item Using the previous two results, we can prove the following:

\begin{align*}
  &\pE\brb{\left| x(t)-\sum_{n\in\Z}\dot{x}_n \psi_n(t) \right|^2}
  \\&= \pE\brs{\brs{ x(t)-\sum_{n\in\Z}\dot{x}_n \psi_n(t) }\brs{ x(t)-\sum_{m\in\Z}\dot{x}_m \psi_m(t) }^\ast}
  \\&= \pE\brs{x(t)x^\ast(t) -x(t)\brp{\sum_{n\in\Z}\dot{x}_n \psi_n(t)}^\ast -x^\ast(t)\sum_{n\in\Z}\dot{x}_n \psi_n(t) + \sum_{n\in\Z}\dot{x}_n \psi_n(t) \brp{\sum_{m\in\Z}\dot{x}_m \psi_m(t) }^\ast }
  \\&= \pE\brs{x(t)x^\ast(t)} -\pE\brs{x(t)\brp{\sum_{n\in\Z}\dot{x}_n \psi_n(t)}^\ast} -\pE\brs{x^\ast(t)\sum_{n\in\Z}\dot{x}_n \psi_n(t)} + \pE\brs{\sum_{n\in\Z}\dot{x}_n \psi_n(t) \brp{\sum_{m\in\Z}\dot{x}_m \psi_m(t) }^\ast }
  \\&\text{by \thme{Mercer's Theorem}\xref{thm:mercer}}
  \\&= \sum_{n\in\Z}\lambda_n |\psi_n(t)|^2 -\sum_{n\in\Z}\lambda_n \left|\psi_n(t) \right|^2  -\brp{\sum_{n\in\Z}\lambda_n \left|\psi_n(t) \right|^2}^\ast + \sum_{n\in\Z}\lambda_n \left|\psi_n(t) \right|^2
  \\&= \sum_{n\in\Z}\lambda_n |\psi_n(t)|^2 -\sum_{n\in\Z}\lambda_n \left|\psi_n(t) \right|^2  -\sum_{n\in\Z}\lambda_n \left|\psi_n(t) \right|^2 + \sum_{n\in\Z}\lambda_n \left|\psi_n(t) \right|^2
  \\&= 0
\end{align*}
\end{enumerate}
\end{proof}

