%============================================================================
% Daniel J. Greenhoe
% GNU Ocatave file
% order 1 low pass filter design
%============================================================================
%=======================================
\chapter{Coefficient Calculation}
%=======================================
%--------------------------------------
\section{IIR order 1 filter}
%--------------------------------------
%--------------------------------------
\begin{lemma}[\thmd{order 1 filter}]
\label{lem:iir1}
%--------------------------------------
Let $\Zh(z)\eqd\frac{a + bz^{-1}}{1 + cz^{-1}}$
be the \ope{Z-transform}             $\brs{\opZ\fh(n)}(z)$         \xref{def:opZ} 
and $\Fh(\omega)$ be the \ope{DTFT}  $\brs{\opDTFT\fh(n)}(\omega)$ \xref{def:dtft} 
of a sequence $\fh(n)$.
\\
\lembox{
  \brb{\begin{array}{FrclD}
      (1). & \setn{a,b,c}       &\in&    \R                  & and
    \\(2). & c                  &\notin& \setn{-1,+1}        & and
    \\(3). & b &\neq& ac
  \end{array}}
  \implies
  \brb{\begin{array}{FrclD}
      (A). & \abs{\Fh(\omega)}^2 &=& \ds
             \frac{a^2 + 2ab\cos\brp{\omega} + b^2}
                  {1   + 2 c\cos\brp{\omega} + c^2}                       & and
    \\(B). & \mc{3}{M}{$\Zh(z)$ has a \structe{zero} at $z=-\frac{b}{a}$} & and
    \\(C). & \mc{3}{M}{$\Zh(z)$ has a \structe{pole} at $z=-c$}           & 
  \end{array}}
  }
\end{lemma}
\begin{proof}
\begin{enumerate}
\item Proof for (A):
  \begin{align*}
    \boxed{\abs{\Zh(z)}^2_{z=e^{i\omega}}}
      &= \brlr{\Zh(z)\Zh^\ast(z)}_{z=e^{i\omega}}
    \\&= \brlr{\Zh(z)\Zh(z^{-1})}_{z=e^{i\omega}}
     &&  \text{by left hypothesis (1)}
    \\&= \brp{\frac{a + be^{-i\omega}}{1 + ce^{-i\omega}}}
         \brp{\frac{a + be^{i\omega}}{1 + ce^{i\omega}}}
     &&= \frac{a^2 + abe^{-i\omega} + abe^{i\omega} + b^2}
              {1   +  ce^{i\omega} +  ce^{-i\omega} + c^2}
    \\&= \frac{a^2 + 2ab\cos\brp{\omega} + b^2}
              {1   + 2 c\cos\brp{\omega} + c^2}
      && \text{by \thme{Euler formulas} \xref{cor:eform}}
  \end{align*}

  \item Proof for (B): 
    \begin{align*}
      \brlr{\Zh(z)}_{z=-\sfrac{b}{a}}
        &\eqd \brlr{\frac{a + bz^{-1}}{1 + cz^{-1}}}_{z=-\sfrac{b}{a}}=0
      \\&=    \frac{a + b\brp{-\frac{b}{a}}^{-1}}
                   {1 + c\brp{-\frac{b}{a}}^{-1}}
      \\&=    \frac{a - a}
                   {1 - \frac{ac}{b}}
      \\&=    \frac{0}{b - ac}
    \end{align*}
\end{enumerate}
\end{proof}

%--------------------------------------
\section{1st Order Low-Pass calculation}
%--------------------------------------
\begin{figure}[h]
  \centering
  \begin{tabular}{|@{\hspace{2pt}}c@{\hspace{2pt}}|@{\hspace{2pt}}c@{\hspace{2pt}}|}
     \hline
     \includegraphics{../common/math/graphics/pdfs/lpfwc.pdf}
    &\includegraphics{../common/math/graphics/pdfs/iir1pole.pdf}
  \\magnitude $\abs{\Fh(\omega)}$ for $\omega_c\in\setn{\frac{\pi}{3}, \frac{\pi}{2},\frac{2\pi}{3}}$
    &\structe{pole} locations of $\Zh(z)$ for $\omega_c\in\intcc{0}{\pi}$
  \\\hline
  \end{tabular}
  \caption{order 1 low pass filter of \prefpp{thm:lpf1} characteristics \label{fig:lpf1}}
\end{figure}
%--------------------------------------
\begin{theorem}[\thmd{order 1 low-pass filter}]
\label{thm:lpf1}
%--------------------------------------
Let $\Zh(z)\eqd\frac{a + bz^{-1}}{1 + cz^{-1}}$
be the \ope{Z-transform}             $\brs{\opZ\fh(n)}(z)$         \xref{def:opZ} 
and $\Fh(\omega)$ be the \ope{DTFT}  $\brs{\opDTFT\fh(n)}(\omega)$ \xref{def:dtft} 
of a sequence $\fh(n)$.
\\
\thmbox{
  \brb{\begin{array}{FrclD}
      (1). & \Fh(0)               &=&      1            & and
    \\(2). & \Fh(\pi)             &=&      0            & and
    \\(3). & \abs{\Fh(\omega_c)}^2&=&      \frac{1}{2}  & and
    \\(4). & \setn{a,b,c}         &\in&    \R           & and
    \\(5). & c                    &\notin& \mc{2}{l}{\setn{-1,+1}}
  \end{array}}
  \implies
  \brb{\begin{array}{FrclD}
      (A). & c&=&\ds\frac{-1 + \sin\brp{\omega_c}}{\cos\brp{\omega_c}} & and
    \\(B). & b&=&(c+1)/2                                               & and
    \\(C). & a&=&b                                                     & and 
    \\(D). & \mc{3}{M}{$\Zh(z)$ has a \structe{zero} at $z=-1$}        & and
    \\(E). & \mc{3}{M}{$\Zh(z)$ has a \structe{pole} at $z=-c$}        & and
    \\(F). & \abs{\Fh(\omega)}^2 &=& 
             \mc{2}{l}{\ds
             \frac{1}{2}\brp{
             \frac{\brp{c+1}^2\brs{1 +  \cos\brp{\omega}}}
                  {c^2 + 2 c\cos\brp{\omega} + 1}}                       
             }
  \end{array}}
  }
\end{theorem}
\begin{proof}
\begin{enumerate}
\item Proof that $a=b$: \label{item:lpf1_ab}
  \begin{align*}
    0 
      &= \Fh(\pi)                             && \text{by left hypothesis (2)}
    \\&= \brb{\Zh(z)}_{z=e^{i\pi}}
       = \brb{\Zh(z)}_{z=-1}
       = \brb{\frac{a + bz^{-1}}{1 + cz^{-1}}}_{z=-1}
       = \frac{a - b}{1 - c}
    \\&\implies \boxed{a=b}                   && \text{by left hypothesis (5)}
  \end{align*}

\item Proof that $a=(c+1)/2$: \label{item:lpf1_ac2}
  \begin{align*}
    1 
      &= \Fh(0)                               && \text{by left hypothesis (1)}
       = \brb{\Zh(z)}_{z=e^{i0}}
       = \brb{\Zh(z)}_{z=1}
       = \brb{\frac{a + bz^{-1}}{1 + cz^{-1}}}_{z=1}
       = \frac{a + b}{1 + c}
    \\&= \frac{2a}{1 + c}                     && \text{by \prefp{item:lpf1_ab}}
    \\&\implies \boxed{a=\frac{c+1}{2}}       && \text{by left hypothesis (5)}
  \end{align*}

  \item Proof for (F): \label{item:lpf1_hw}
    \begin{align*}
      \Fh(\omega)
        &= \frac{a^2 + 2ab\cos\brp{\omega} + b^2}
                {1   + 2 c\cos\brp{\omega} + c^2}
        && \text{by \prefp{lem:iir1}}
      \\&= \frac{2a^2\brs{1 +  \cos\brp{\omega}}}
                {c^2 + 2 c\cos\brp{\omega} + 1}
        && \text{by \prefp{item:lpf1_ab}}
      \\&= \frac{2\brp{\frac{c+1}{2}}^2\brs{1 +  \cos\brp{\omega}}}
                {c^2 + 2 c\cos\brp{\omega} + 1}
        && \text{by \prefp{item:lpf1_ac2}}
      \\&= \brp{\frac{1}{2}}
                \frac{\brp{c+1}^2\brs{1 +  \cos\brp{\omega}}}
                     {c^2 + 2 c\cos\brp{\omega} + 1}
    \end{align*}

\item lemma. $c^2{\cos\brp{\omega_c}} + 2c + \cos\brp{\omega_c} = 0$. Proof: \label{ilem:lpf1_c2}
  \begin{align*}
    \frac{1}{2}
      &= \abs{\Fh(\omega_c)}^2
      && \text{by left hypothesis (3)}
    \\&= \frac{1}{2}\brp{
         \frac{\brp{c+1}^2\brs{1 +  \cos\brp{\omega_c}}}
              {c^2 + 2 c\cos\brp{\omega_c} + 1}}
      && \text{by \prefp{item:lpf1_hw}}
    \\\\\implies &c^2 + 2 c\cos\brp{\omega_c} + 1 
               = \brp{c+1}^2\brs{1 +  \cos\brp{\omega_c}}
    \\\\\implies & c^2\brs{1-1-\cos\brp{\omega_c}} 
               + c  \brs{2\cos\brp{\omega_c}-2-2\cos\brp{\omega_c}}
               +    \brs{1-1-\cos\brp{\omega_c}}
               \\&= 0
    \\\\\implies & c^2{\cos\brp{\omega_c}} 
               + 2c
               +    {\cos\brp{\omega_c}}
               = 0
  \end{align*}

\item Proof for (A):
  \begin{align*}
    c &= \frac{-2 \pm \sqrt{(2)^2 -4\cos^2\brp{\omega_c}}}
              {2\cos\brp{\omega_c}}
      && \text{by \prefp{ilem:lpf1_c2} and \thme{Quadratic Equation}}
    \\&= \frac{-1 \pm \sqrt{1 -\cos^2\brp{\omega_c}}}
              {\cos\brp{\omega_c}}
    \\&= \frac{-1 \pm \sin\brp{\omega_c}}
              {\cos\brp{\omega_c}}
      && \text{by \prefp{thm:ccss1}}
    \\
    \\\implies &\ds\boxed{c = \frac{-1 + \sin\brp{\omega_c}}
                      {\cos\brp{\omega_c}}}
  \end{align*}

  \item Proof that the zero is at $z=-1$:
    \begin{align*}
      z &= -\frac{b}{a}
        && \text{by \prefp{lem:iir1}}
      \\&= -\frac{a}{a}
        && \text{by \prefp{item:lpf1_ab}}
      \\&= -1
    \end{align*}

  \item Proof that the pole is at $-c$: by \prefp{lem:iir1}

\end{enumerate}
\end{proof}

\begin{figure}
  \centering%
  \includegraphics{../common/math/graphics/pdfs/dfI_order1_abc.pdf}
  \includegraphics{../common/math/graphics/pdfs/dfII_order1_abc.pdf}
\end{figure}
%---------------------------------------
\begin{example}[order 1 low-pass filter with corner frequency $\omega_c=\frac{2}{3}\pi$]
%---------------------------------------
{ \begin{align*}
  c &= \frac{-1 + \sin\brp{\omega_c}}{\cos\brp{\omega_c}}
     = \frac{-1 + \sin\brp{\sfrac{1}{3}\pi}}{\cos\brp{\sfrac{1}{3}\pi}}
     = \frac{-1 + \frac{\sqrt{3}}{2}}{\frac{1}{2}}
   &&= \sqrt{3} - 2
  \\
  a &= \frac{c+1}{2}
     = \frac{\brp{\sqrt{3} - 2}+1}{2}
   &&= \frac{\sqrt{3}+1}{2}
  \\
  b &= a
   &&= \frac{\sqrt{3} + 1}{2}
  \\
  \\
  \abs{H(\omega)}^2
    &= \abs{\Zh(z)}^2_{z=e^{i\omega}}
     = \brlr{\Zh(z)\Zh^\ast(z)}_{z=e^{i\omega}}
   &&= \brp{\frac{1}{2}}\frac{\brp{c+1}^2\brs{1 +  \cos\brp{\omega}}}
            {c^2 + 2 c\cos\brp{\omega} + 1}
\end{align*}}
%%%???where is this one???\includegraphics{../common/math/graphics/pdfs/lpfwc_sq.pdf}

$\ds\abs{H(\omega)}^2$
\end{example}


For a C++ implementation, see \prefpp{sec:src_df2_order1_cpp}.

%--------------------------------------
\section{1st Order High-Pass calculation}
%--------------------------------------
$\ds\boxed{ \Zh(z) = \frac{a + bz^{-1}}
                        {1 + cz^{-1}}
          }$

\begin{align*}
  0 &= {\abs{\Zh(z)}}_{z=e^{i0}=1}    
    &&=\brlr{\frac{a + bz^{-1}}{1 + cz^{-1}}}_{z=1} 
    &&=\frac{a + b}{1 + c}                   
    && \implies \boxed{a=-b}
  \\\\
  1 &= \abs{\Zh(z)}_{z=e^{i\pi}=-1} 
   &&= \brlr{\frac{a + bz^{-1}}{1 + cz^{-1}}}_{z=-1} 
   &&= \frac{a - b}{1 - c} = \frac{2a}{1-c}                   
   &&  \implies \boxed{a=\frac{1-c}{2}}
\end{align*}


%
\begin{align*}
  \boxed{\abs{\Zh(z)}^2_{z=e^{i\omega}}}
    &= \brs{\Zh(z)\Zh^\ast(z)}_{z=e^{i\omega}}
  \\&= \brp{\frac{a + be^{-i\omega}}{1 + ce^{-i\omega}}}
       \brp{\frac{a + be^{i\omega}}{1 + ce^{i\omega}}}
   &&= \frac{a^2 + abe^{-i\omega} + abe^{i\omega} + b^2}
            {1   +  ce^{i\omega} +  ce^{-i\omega} + c^2}
  \\&= \frac{a^2 + 2ab\cos\brp{\omega} + b^2}
            {1   + 2 c\cos\brp{\omega} + c^2}
  \\&= \frac{2a^2\brs{1 -  \cos\brp{\omega}}}
            {c^2 + 2 c\cos\brp{\omega} + 1}
    && \text{because $a=-b$}
  \\&= \frac{2\brp{\frac{1-c}{2}}^2\brs{1 -  \cos\brp{\omega}}}
            {c^2 + 2 c\cos\brp{\omega} + 1}
    && \text{because $a=\frac{1-c}{2}$}
  \\&= \boxed{\brp{\frac{1}{2}}
              \frac{\brp{1-c}^2\brs{1 -  \cos\brp{\omega}}}
                   {c^2 + 2 c\cos\brp{\omega} + 1}
             }
\end{align*}

{\begin{align*}
  \frac{1}{2}
    &= \abs{\Zh(z)}^2_{z=e^{i\omega_c}}
     = \brp{\frac{1}{2}}
       \frac{\brp{1-c}^2\brs{1 -  \cos\brp{\omega}}}
            {c^2 + 2 c\cos\brp{\omega} + 1}
  \\\\\implies &c^2 + 2 c\cos\brp{\omega_c} + 1 
             = \brp{1-c}^2\brs{1 -  \cos\brp{\omega_c}}
  \\\\\implies & c^2\brs{1-1+\cos\brp{\omega_c}} + 
             \\& c  \brs{2\cos\brp{\omega_c}+2-2\cos\brp{\omega_c}} +
             \\&    \brs{1-1+\cos\brp{\omega_c}}
             \\&= 0
  \\\\\implies & c^2{\cos\brp{\omega_c}} 
             + 2c 
             +    \cos\brp{\omega_c}
             = 0
\end{align*}}



{\begin{align*}
  \\\implies c &= \frac{-2 \pm \sqrt{(2)^2 -4\cos^2\brp{\omega_c}}}
                       {2\cos\brp{\omega_c}}
               && \text{by \thme{Quadratic Equation}}
             \\&= \frac{-1 \pm \sqrt{1 -\cos^2\brp{\omega_c}}}
                       {\cos\brp{\omega_c}}
             \\&= \frac{-1 \pm \sin\brp{\omega_c}}
                       {\cos\brp{\omega_c}}
               && \text{because $\sin^2x + \cos^2x = 1$ for all $x\in\R$}
  \\\implies c &=
      \boxed{\frac{-1 + \sin\brp{\omega_c}}
                    {\cos\brp{\omega_c}}}
               && \text{because want pole inside unit circle}
\end{align*}}


%---------------------------------------
Where is the zero? Where is the pole?
%---------------------------------------

The zero is at z=+1. \qquad The pole is at 
$\ds\boxed{z=-c = \frac{1 - \sin\brp{\omega_c}}
                    {\cos\brp{\omega_c}}}$

\includegraphics{../common/math/graphics/pdfs/iir1pole.pdf}


%---------------------------------------
\begin{example}order 1 \propb{high-pass} with corner frequency $\omega_c=\frac{1}{3}\pi$
%---------------------------------------
\begin{align*}
  c &= \frac{-1 + \sin\brp{\omega_c}}{\cos\brp{\omega_c}}
     = \frac{-1 + \sin\brp{\sfrac{1}{3}\pi}}{\cos\brp{\sfrac{1}{3}\pi}}
     = \frac{-1 + \frac{\sqrt{3}}{2}}{\frac{1}{2}}
   &&= \sqrt{3} - 2
  \\
  a &= \frac{1-c}{2}
     = \frac{1-\brp{\sqrt{3} - 2}}{2}
   &&= \frac{1-\sqrt{3}}{2}
  \\
  b &= -a
   &&= \frac{\sqrt{3} - 1}{2}
  \\
  \\
  \abs{H(\omega)}^2
    &= \abs{\Zh(z)}^2_{z=e^{i\omega}}
     = \brlr{\Zh(z)\Zh^\ast(z)}_{z=e^{i\omega}}
   &&= \brp{\frac{1}{2}}\frac{2\brp{\frac{1-c}{2}}^2\brs{1 -  \cos\brp{\omega}}}
            {c^2 + 2 c\cos\brp{\omega} + 1}
\end{align*}


\includegraphics{../common/math/graphics/pdfs/hpfwc.pdf}

\includegraphics{../common/math/graphics/pdfs/dfI_order1_abc.pdf}
\includegraphics{../common/math/graphics/pdfs/dfII_order1_abc.pdf}
\end{example}

%---------------------------------------
%---------------------------------------

So $c=\sqrt{3}-2$,\qquad $a=\frac{1-c}{2}=\frac{3-\sqrt{3}}{2}$, \qquad $b=-a=\frac{\sqrt{3}-3}{2}$
\begin{align*}
  H_{hp}(z)
    &= \frac{a + bz^{-1}}{1+cz^{-1}}
  \\&= \frac{\brp{\frac{3-\sqrt{3}}{2}} + \brp{\frac{\sqrt{3}-3}{2}}z^{-1}}
            {1+\brp{\sqrt{3}-2}z^{-1}}
  \\&= \frac{\brp{\frac{[2-\sqrt{3}]+1}{2}} + \brp{\frac{[2-\sqrt{3}]+1}{2}}(-z)^{-1}}
            {1+\brp{2-\sqrt{3}}(-z)^{-1}}
  \\&= H_{lp}(-z)
\end{align*}



