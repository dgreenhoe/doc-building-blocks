%============================================================================
% LaTeX File
% Daniel J. Greenhoe
%============================================================================

%======================================
\chapter{Foundational Construction}
\label{chp:found}
%======================================
\paragraph{Fundamental mathematical structures.}
This chapter presents six of the most fundamental structures in mathematics:
\\\indentx\begin{tabular}{cl>{\footnotesize(}l>{\footnotesize}l<{)}}
    \circOne   & \hie{sets}                  & \pref{def:naive}    & \prefpo{def:naive}
  \\\circTwo   & \hie{relations}             & \pref{def:fnd_relation} & \prefpo{def:fnd_relation}
  \\\circThree & \hie{equivalence relations} & \pref{def:eq_rel}    & \prefpo{def:eq_rel}
  \\\circFour  & \hie{functions}             & \pref{def:fnd_f(x)}     & \prefpo{def:fnd_f(x)}
  \\\circFive  & \hie{logic operators}       & \pref{def:flogic}     & \prefpo{def:flogic}
  \\\circSix   & \hie{set operations}        & \pref{def:set_ops}   & \prefpo{def:set_ops}
\end{tabular}\\
These structures are ``fundamental" in the sense that much of the remaining concepts of mathematics
is built on them.
%\paragraph{Construction blueprint.}
We construct these structures as outlined next:
\begin{liste}
  \item We start with the concept of a \hie{set}: a collection of objects
        that is uniquely defined by the objects that constitute the collection
        and that can be specified by any intelligible statement \xref{def:set}.        
  \item Using sets, we can construct an \hie{ordered pair} \xref{def:(a,b)}.
  \item The collection of all ordered pairs on a pair of sets
        is a \hie{Cartesian product} \xref{def:AxB}.
  \item Any subset of a Cartesian product is a \hie{relation} \xref{def:fnd_relation}.
  \item An \hie{equivalence relation} is a relation that is 
        \prope{reflexive}, \prope{symmetric}, and \prope{transitive} \xref{def:eq_rel}.
  \item A relation is a \hie{function} if 
        no value from the primary set is mapped to 
        two different values in the secondary set \xref{def:fnd_f(x)}.
  \item Using the concept of the function, we define the the logical operators 
        \hie{not} $\lnot$, \ope{logical and} $\land$, and \ope{logical or} $\lor$,\ldots
        which are also functions \xref{def:flogic}.
  \item Using logic operators, we can easily define the set operators
        \ope{complement} $\hxs{\setopc}$, \ope{union} $\hxs{\setu}$, \ope{intersection} $\hxs{\seti}$,
        \ope{set difference} $\hxs{\setd}$, and \ope{symmetric difference} $\hxs{\sets}$ \xref{def:set_ops}.
\end{liste}

%=======================================
\section{Sets}
\index{sets}
\index{Naive set theory}
%=======================================
We accept without definition the concepts of \hie{true} and \prope{false}
and that any given statement is either true or false, but never both,
and never anything inbetween (next axiom).
%---------------------------------------
\begin{axiom}[axiom of the excluded middle]
\label{ax:plogic}
\citetbl{
  \citePpp{cantor_manifolds_3}{114}{115}  
  }
\index{axiom of the excluded middle}
\index{axioms!excluded middle}
%---------------------------------------
\axbox{\begin{array}{M}
  There exists a pair of properties $\ltrue$ and $\lfalse$,\\
  such that any given statement has exactly one of these properties.\\
  The symbol $\ltrue$ is called \hid{true} and $\lfalse$ is called \hid{false}.
\end{array}}
\end{axiom}

%\qboxnpqt
\qboxnpq
  {
    \href{http://en.wikipedia.org/wiki/Georg_Cantor}{Georg Cantor} 
    (\href{http://www-history.mcs.st-andrews.ac.uk/Timelines/TimelineF.html}{1845--1918}), 
    \href{http://www-history.mcs.st-andrews.ac.uk/BirthplaceMaps/Places/Germany.html}{German}
    set theory pioneer
    \index{Cantor, Georg}
    \index{quotes!Cantor, Georg}
    \footnotemark
  }
  {../common/people/cantor.jpg}
  %{Eine Mannichfaltigkeit (ein Inbegriff, eine Menge) von Elementen,
  %  die irgend welcher Begriffssph\"are angeh\"oren, nenne ich {\em wohldefinirt},
  %  wenn auf Grund ihrer Definition und in Folge des logischen
  %  Princips vom ausgeshchlossenen Dritten es als {\em intern bestimmt} angesehen
  %  werden muss, {\em sowohl} ob irgend ein derselben Begriffssph\"are angeh\"origes
  %  Object zu der gedachten Mannichfaltigkeit als Element geh\"ort oder
  %  nicht, {\em wie auch} ob zwei zur Menge geh\"orige Objecte, trotz formaler
  %  Unterschiede in der Art des Gegebenseins einander gleich sind oder nicht.}
  {I call a manifold (an aggregate, a set) of elements,
    which belong to any conceptual sphere, well-defined, if on the
    basis of its definition and in consequence of the logical principle
    of excluded middle, it must be recognized that it is internally
    determined whether an arbitrary object of this conceptual sphere
    belongs to the manifold or not, and also, whether two objects in
    the set , in spite of formal differences in the manner in which they
    are given, are equal or not.}
    % In general the relevant distinctions
    % cannot in practice be made with certainty and exactness by the
    % capabilities or methods presently availabe. But that is not of any
    % concern. The only concern is the internal determination from
    % which in concrete cases, where it is required, an actual (external)
    % determination is to be developed by means of a perfection of
    % resources. [1932, p. 150]
  \citetblt{
    quote:       & \citePpp{cantor_manifolds_3}{114}{115}  \\
    translation: & \citerp{tait}{271}  \\
    image:       & \url{http://en.wikipedia.org/wiki/Image:Georg_Cantor.jpg}
    }

There is more than one definition of a \hie{set}.
The most basic form of set theory is called \hie{naive set theory} (\pref{def:naive}---next).
However, in general, Naive set theory is flawed as shown by
\hie{Russell's Paradox} and illustrated by the \hie{Barber of Seville} anecdote.
The most commonly used replacement for the flawed Naive set theory
is \hie{Zermelo-Fraenkel (ZF) set theory}.%
\footnote{
  \citePpp{zermelo1908}{263}{267} (7 axioms),
  \citer{zermelo1908e} (English translation),
  \citeP{fraenkel1922},
  \citerp{wolf}{139}
  }

%---------------------------------------
\begin{definition}[Naive set theory]
\label{def:set}
\label{def:naive}
\citetbl{
  \citerppg{halmos1960}{1}{6}{0387900926}
  }
\index{Naive set}
\index{axiom of extension}
\index{axiom of specification}
%---------------------------------------
\defbox{\begin{tabular}{lp{10\tw/16}@{\qquad}>{\scriptsize}l}
  \mc{3}{l}{A \hid{set} is any collection of objects that satisfies the following axioms:}
  \\
  1. & Any two sets are equal if and only if their elements are the same.
     & (\hie{axiom of extension})
     \\
  2. & For every set $\setX$ and statement $s(x)$,
       there exists a set $\setA$ consisting exactly of elements $x$ from set $\setX$
       such that $s(x)$ is true.
     & (\hie{axiom of specification})
  \end{tabular}}
\end{definition}

The \hie{axiom of extension} states that
the collection completely defines the set,
and there is no other characteristic (no {\em extension})
which further defines the set.
So if two sets have exactly the same elements,
then the two sets are equal.
The \hie{axiom of specification} states that,
you can form any subset $\setA$ of a set $\setX$ as long as 
you can \emph{specify} which elements of $\setX$ you want included in $\setA$ 
by any possible intelligible statement $\fs(x)$:
  \[ \setA = \set{x\in\setX}{\fs(x) \text{ is true}} \]

An element $x$ that is included in a set $\setX$ is denoted $x\in\setX$ 
(next definition). 
A set that has no elements is called the \hie{empty set} 
and is denoted by the symbol $\emptyset$ \xref{def:set_emptyset}.
\xref{prop:set_emptyset} shows that, under the axioms of \prope{naive set theory},
the emptyset exists and that there is 
only one set that is the emptyset.
%---------------------------------------
\begin{definition}
\label{def:set_in}
\citetbl{
  \citerpg{halmos1960}{2}{0387900926},
  \citerpg{hausdorff1937e}{12}{0828401195},
  \citePp{kuratowski1921}{161}
  }
%---------------------------------------
\defbox{\indxs{\in}\indxs{\notin}
  \begin{array}{ll}
    \text{An element $x$ is a member of a set $\setX$ is denoted by} & x\hxsd{\in}\setX. \\
    \text{An element $x$ is \emph{not} a member of a set $\setX$ is denoted by} & x\hxsd{\notin}\setX.
  \end{array}}
\end{definition}


%---------------------------------------
\begin{definition}
\label{def:set_emptyset}
\citetbl{
  \citerpg{halmos1960}{8}{0387900926},
  \citerpg{kelley1955}{3}{0387901256},
  \citerp{kuratowski_stt}{26}
  }
%---------------------------------------
\defbox{\indxs{\emptyset}
  \begin{array}{>{\ds}c}
    \text{The \structd{empty set} $\hxsd{\emptyset}$ is defined as} \\
    \emptyset \eqd \set{x\in\setX}{x\ne x}
  \end{array}}
\end{definition}

%---------------------------------------
\begin{proposition}
\label{prop:set_emptyset}
\citetbl{
  \citerpg{halmos1960}{8}{0387900926},
  \citerpg{hausdorff1937e}{13}{0828401195}
  }
%---------------------------------------
Let $\emptyset$ be the empty set.
\propbox{\begin{tabular}{ll}
  1. & $\emptyset\;$ exists. \\
  2. & $\emptyset\;$ is unique.
\end{tabular}}
\end{proposition}
\begin{proof}
  \begin{enumerate}
    \item $\emptyset$ exists by the \hie{axiom of specification} \xref{def:naive}
            $\emptyset \eqd \set{x\in\setX}{x\ne x}$.
    \item $\emptyset$ is unique by the \hie{axiom of extension} \xref{def:naive}.
  \end{enumerate}
\end{proof}



\begin{minipage}[c]{2\tw/3}
  \pref{def:fnd_powerset} (next) defines the \hie{power set} $\psetx$ of a set $\setX$.
  The power set is simply the set of all subsets of $\setX$.
  The notation $\psetx$ is meaningful because the number of elements (subsets of $\setX$)
  in a finite set $\setX$ is $2^\seto{\setX}$ \xref{prop:pset_order}.
  The tuple $\lattice{\psetx}{\subseteq}{\seti}{\setu}$ forms a \structe{lattice} \xref{def:lattice}.
  The lattice $\lattice{\pset{\setn{x,y,z}}}{\subseteq}{\seti}{\setu}$
  is illustrated to the right.
\end{minipage}%
\hfill%
\begin{tabular}{c}%
  \includegraphics{graphics/lat8_2e3_setxyz.pdf}%
\end{tabular}
%\begin{minipage}[c]{\tw/3}%
%  \begin{center}%
%  %============================================================================
% Daniel J. Greenhoe
% LaTeX file
% lattice (2^{x,y,z}, subseteq)
% recommended unit = 10mm
%============================================================================
{\psset{unit=0.75\psunit}%
\begin{pspicture}(-2.4,-.3)(2.4,3.3)
  %---------------------------------
  % settings
  %---------------------------------
  \psset{%
    labelsep=1.5mm,
    }%
  %---------------------------------
  % nodes
  %---------------------------------
  \Cnode(0,3){t}
  \Cnode(-1,2){xy} \Cnode(0,2){xz} \Cnode(1,2){yz}
  \Cnode(-1,1){x}  \Cnode(0,1){y}  \Cnode(1,1){z}
  \Cnode(0,0){b}
  %---------------------------------
  % node connections
  %---------------------------------
  \ncline{t}{xy}\ncline{t}{xz}\ncline{t}{yz}
  \ncline{x}{xy}\ncline{x}{xz}
  \ncline{y}{xy}\ncline{y}{yz}
  \ncline{z}{xz}\ncline{z}{yz}
  \ncline{b}{x} \ncline{b}{y} \ncline{b}{z}
  %---------------------------------
  % node labels
  %---------------------------------
  \uput[180](t) {$\setn{x,y,z}$}%
  \uput[180](xy){$\setn{x,y}$}%   
 %\uput{1pt}[ 70](xz){$\setn{x,z}$} 
  \uput[0](yz){$\setn{y,z}$}%
  \uput[180](x) {$\setn{x}$}%     
 %\uput{1pt}[-45](y) {$\setn{y}$}   
  \uput[0](z) {$\setn{z}$}%
  \uput[180](b) {$\szero$}%
  \uput[0](1,3){\rnode{xzlabel}{$\setn{x,z}$}}% 
  \uput[0](1,  0){\rnode{ylabel}{$\setn{y}$}}%
  \ncline[linestyle=dotted,linecolor=red,nodesep=1pt]{->}{xzlabel}{xz}%
  \ncline[linestyle=dotted,linecolor=red,nodesep=1pt]{->}{ylabel}{y}%
\end{pspicture}
}%%
%  \end{center}%
%\end{minipage}

%---------------------------------------
\begin{definition}
\label{def:fnd_powerset}
\index{set!power}
%---------------------------------------
\defbox{\begin{array}{M}\indxs{\psetx}
  The \structd{power set} $\hxsd{\psetx}$ on a set $\setX$ is defined as
    \\\indentx$\ds\psetx \eqd \set{\setA}{\setA\subseteq\setX}$
      \qquad\scriptsize(the set of all subsets of $\setX$)
\end{array}}
\end{definition}

%---------------------------------------
\begin{proposition}
\label{prop:pset_order}
%---------------------------------------
Let $\seto{\setX}$ represent the number of elements in a finite set $\setX$.
\propbox{
  \seto{\psetx} = 2^{\seto{\setX}}
  } 
\end{proposition}
\begin{proof}
\begin{enumerate}
  \item Let $n=\seto{\setX}$ be the number of elements in a set.
  \item Let $\setX\eqd\setn{x_1,\,x_2,\,x_3,\,\ldots,\,x_n}$.
  \item Each subset of $\setX$ (each element in $\psetx$) has 2 possibilities for each of
        the $n$ elements of $\setX$: Either the subset \emph{has} a given element of $\setX$
       (represent by $1$), or does \emph{not} have it (represented by a $0$).
  \item Therefore, the number of possible subsets (the number of elements in $\psetx$) is
    \begin{align*}
      \seto{\psetx}
        &= \mcom{\seto{\setn{\text{has $x_1$},\,\text{not have $x_1$}}} \cdot
                 \seto{\setn{\text{has $x_2$},\,\text{not have $x_2$}}} \cdots
                 \seto{\setn{\text{has $x_n$},\,\text{not have $x_n$}}} 
                }{$n$ times}
      \\&= \mcom{2\cdot2\cdots2}{$n$ times}
      \\&= 2^n
      \\&= 2^\seto{\setX}
    \end{align*}
\end{enumerate}
\end{proof}





%=======================================
\section{Relations}
\label{sec:relations}
%=======================================

\paragraph{Ordered pair.}
In the set $\setn{a,b}$, the order in which the elements are listed
does \emph{not} matter. That is, $\setn{a,b}$ is equivalent to $\setn{b,a}$.
However, in some applications the order does (very much) matter.
To help with this, we have the concept of the \structe{ordered pair}.
In an ordered pair $\opair{a}{b}$, the order in which the elements are listed \emph{is} significant.
That is, $\opair{a}{b}$ is equivalent to $(b,a)$ if and only if $a=b$.
The ordered pair can be defined in different ways.
One of the most common definitions is due to
mathematician Kazimierz Kuratowski \index{Kuratowski, Kazimierz}
in 1921 and is presented next:%
\footnote{\emph{Alternative ordered pair definition}:
  As an alternative to the Kuratowski definition, 
  the ordered pair can also be taken as an \emph{axiom}. References: 
  \\\begin{tabular}{l}
      \citerp{bourbaki_tos}{72},
      \citerp{munkres2000}{13}
    \end{tabular}
  }
%---------------------------------------
\begin{definition}
\label{def:(a,b)}
\footnote{
  \citerpg{suppes1972}{32}{0486616304},
  \citerpg{halmos1960}{23}{0387900926},
  %\citerp{menini2004}{39} \\
  %\citerp{wolf}{164}  \\
  \citerp{kuratowski_stt}{39},
  \citePc{kuratowski1921}{Def. V, page 171},
  \citeP{wiener1914}
  %\citer{veblen1904} -- I did not find any reference to ordered pairs in this paper (42 pages long) ,
  }
%---------------------------------------
\defbox{\indxs{\opairn}
  \text{The \structd{ordered pair} $\hxsd{\opair{a}{b}}$ is defined as }
  \qquad
  \opair{a}{b} \eqd \setn{ \setn{a},\setn{a,b} } }
\end{definition}

This definition may seem a little strange---it even has the strange consequence
that $\setn{a,b}\in\opair{a}{b}$.
But the crucial test of validity is if 
  \[\opair{a}{b}=\opair{c}{d} \qquad\iff\qquad a=c \text{ and } b=d.\]
This statement is true under \pref{def:(a,b)} but is most easily proved once we have defined
the \ope{set intersection operator} $\seti$ and 
\hie{set symmetric difference operator} $\sets$.
These are not defined until \prefpp{def:set_ops}.
And subsequently the above ordered pair equality is proved in 
\prefpp{cor:(a,b)=(c,d)}.



\paragraph{Cartesian product.}
%---------------------------------------
\begin{definition}
\label{def:rel_and}
%---------------------------------------
Let $\setX$ and $\setY$ be sets, $\hxs{\ltrue}$ denote the logical property of ``true", 
and $\hxs{\lfals}$ the logical property of ``false" \xref{ax:plogic}.
\defbox{\begin{array}{M}\indxs{\owedge}
  $\hxsd{\owedge}$ is the ``\opd{relational and}" defined as
  \\\indentx$\ds 
  \owedge \eqd 
    \setn{
      \opair{\opair{\lfals}{\lfals}}{\lfals},\;
      \opair{\opair{\lfals}{\ltrue}}{\lfals},\;
      \opair{\opair{\ltrue}{\lfals}}{\lfals},\;
      \opair{\opair{\ltrue}{\ltrue}}{\ltrue}
      }$
\end{array}}
\end{definition}


\begin{minipage}{3\tw/4}
  The ``relational and" of \pref{def:rel_and} can be illustrated
  using the \structe{truth table} to the right $\rightarrow$.
  Later, the relational and will be replaced by the \ope{logical and};
  but the \ope{logical and} is a \ope{function}, and functions are not defined
  until \prefpp{def:fnd_f(x)}.
\end{minipage}%
\begin{minipage}{\tw/4}
  \[\begin{array}[t]{cc|c}
      x & y & x\owedge y  \\
      \hline
      \lfals & \lfals & \lfals \\
      \lfals & \ltrue & \lfals \\
      \ltrue & \lfals & \lfals \\
      \ltrue & \ltrue & \ltrue
    \end{array}\]
\end{minipage}

%---------------------------------------
\begin{definition}
\label{def:AxB}
\footnote{
  \citerpg{halmos1960}{24}{0387900926},
  G. Frege, 2007 August 25, \url{http://groups.google.com/group/sci.logic/msg/3b3294f5ac3a76f0}
  }
%---------------------------------------
Let $\setX$ and $\setY$ be sets, and
let $\hxs{\owedge}$ be the ``relational and" relation of \prefpp{def:rel_and}.
\defbox{\begin{array}{M}
  The \opd{Cartesian product} $\hxsd{\cprodXY}$ is defined as
  \\\indentx$\ds \cprodXY \eqd \set{\opair{x}{y}}{(x\in\setX)\; \owedge \; (y\in\setY)}$
\end{array}}
\end{definition}




\paragraph{Relations.}
%In general, a relation from a set $\setX$ to a set $\setY$
%is just \emph{some} of the possible combinations of elements from $\setX$ as the 
%first element and elements from $\setY$ as the second element. 
%But suppose we have a relation $\relation$ which includes \emph{all} of 
%the possible ordered pairs of $\setX$ and $\setY$ such that
% \[ \relation \eqd \set{\opair{x}{y}}{x\in\setX \text{ and } y\in\setY}. \]
%Then this relation is called the \hie{Cartesian product} $\cprodXY$
%(next definition).
%In particular, relations from a set $\setX$ to a set $\setY$ are a special 
%case of the Cartesian product of $\setX$ to $\setY$. 
%In fact, such a relation is a \emph{subset} of the Cartesian product $\cprodXY$
%(\pref{prop:relation}---next proposition).
A set of ordered pairs represents a \emph{relationship} between
the set formed by the first elements of the ordered pairs 
and the set formed by the second elements of the ordered pairs.
Any such set is called a \structe{relation} (next definition).
The set of all relations in $\cprodXY$ is denoted $\clRxy$.
This notation is meaningful because the number of relations in $\clRxy$ is
$2^{\seto{\setX}\seto{\setY}}$ \xref{prop:rel_order}.

%---------------------------------------
\begin{definition}
\label{def:fnd_relation}
\citetbl{
  \citerpg{halmos1960}{26}{0387900926}
  }
%---------------------------------------
Let $\setX$ and $\setY$ be sets.
\defbox{\begin{array}{M}\indxs{\relation}\indxs{\clRxy}
  A \reld{relation} $\hxsd{\relation}:\setX\to\setY$ 
  is any subset of $\hxs{\cprodXY}$. That is,
    \\\indentx$\ds\relation \subseteq \cprodXY $
    \\
  The set of all relations that are subsets of $\cprodXY$ is denoted $\clRxy$. That is,
    \\\indentx$\ds\clRxy \eqd \set{\relation}{\relation\subseteq\brp{\cprodXY}} $
\end{array}}
\end{definition}



%%---------------------------------------
%\begin{proposition}
%\label{prop:relation}
%\citetbl{
%  \citerp{kelley1955}{7} ,
%  \citerp{halmos1960}{27} ,
%  %\citerp{kbr}{107}
%  }
%%---------------------------------------
%Let $\relation:\setX\to\setY$ be a \hie{relation} from a set $\setX$ to a set $\setY$
%and $\cprodXY$ be the \hie{Cartesian product} of the sets $\setX$ and $\setY$.
%\propbox{
%  \mcom{\relation:\setX\to\setY \text{ is a relation}}
%       {$\relation$ is a relation from $\setX$ to $\setY$}
%  \qquad\implies\qquad
%  \mcoml{\relation \subseteq \cprodXY}
%        {$\relation$ is a \hie{subset} of the Cartesian product $\cprodXY$}
%  }
%\end{proposition}
%


%%---------------------------------------
%\begin{definition}
%\label{def:rel_comp}
%%---------------------------------------
%The compostion operator $\circ$ on a pair of functions $\ff\in\clFxy$ and $\fg\in\clF{\setY}{\setZ}$ is defined as 
%\defbox{\indxs{\circ}
%  \brbr{\begin{array}{rcl}
%    \ff &\eqd& \set{\opair{x_i}{y_i}}{i\in\setI_f} \\
%    \fg &\eqd& \set{\opair{y_i}{z_i}}{i\in\setI_g} 
%  \end{array}}
%  \qquad\implies\qquad
%  \fg\circ\ff \eqd\set{\opair{x_i}{z_i}}{i\in\setI_f}
%  }
%\end{definition}



%=======================================
\section{Equivalence relations}
%=======================================
%\qboxnpq
%  {\href{http://en.wikipedia.org/wiki/Aristotle}{Aristotle}
%   (\href{http://www-history.mcs.st-andrews.ac.uk/Timelines/TimelineA.html}{384BC--322BC}),
%   \href{http://www-history.mcs.st-andrews.ac.uk/BirthplaceMaps/Places/Greece.html}{Greek} philosopher
%   \index{Aristotle}
%   \index{quotes!Aristotle}
%   \footnotemark}
%  {../common/people/aristot}
%  {Things are said to be named 'equivocally' when,
%   though they have a common name,
%   the definition corresponding with the name differs for each.
%   \ldots
%   On the other hand, things are said to be named 'univocally'
%   which have both the name and the definition answering to the name in common.
%   \ldots
%   Things are said to be named 'derivatively',
%   which derive their name from some other name, but differ from it in termination.}
%  \citetblt{
%    quote: & \citorp{aristotle_categoriae}{7} \\
%    image: & \url{http://en.wikipedia.org/wiki/Aristotle}
%    }
%
\qboxnps
  {\href{http://en.wikipedia.org/wiki/Robert_Recorde}{Robert Recorde}
   \href{http://www-history.mcs.st-andrews.ac.uk/Timelines/TimelineC.html}{(1510--1558)},
   \href{http://www-history.mcs.st-andrews.ac.uk/BirthplaceMaps/Places/UK.html}{Welsh} 
   physician and mathematician
   \footnotemark
  }
  %{../common/people/recorde.jpg}
  {../common/people/wow_eq.jpg}
  {To avoide the tedious repetition of these woordes: is equal to:
   I will sette as I doe often in woorke use, a paire of parralles,
   or Gemowe lines of one lengthe, thus: $=\!=\!=\!=\!=$,
   bicause noe 2 thynges, can be moare equalle.}
  \citetblt{
    quote: & \citerpu{potts1860}{109}{http://books.google.com/books?id=eck2AAAAMAAJ\&pg=PA109} \\
           & \citer{recorde1557} \\
    image: & \url{http://nsm1.nsm.iup.edu/gsstoudt/history/images/witte.jpg}  \\
           & \url{http://www-gap.dcs.st-and.ac.uk/~history/PictDisplay/Recorde.html} \\
           & \url{http://members.aol.com/jeff94100/witte.jpg}
    }



%Examples of special types of relations include the following:
%\\\begin{tabular}{>{$\imark$ }lp{5\tw/16}ll}
%  \hie{equivalence relation}:
%    & a relation that is reflexive, symmetric, and transitive.
%    & \pref{def:eq_rel}
%    & \prefpo{def:eq_rel}
%  \\
%  \hie{ordering relation}:
%    & a relation that is reflexive, anti-symmetric, and transitive.
%    & \pref{def:order_rel}
%    & \prefpo{def:order_rel}
%\end{tabular}

%---------------------------------------
\begin{definition}
\label{def:eq_rel}
\citetbl{
  \citerp{ab}{7},
  \citerp{peano1889e}{91}
  }
\index{relation!equivalence}
%---------------------------------------
\defbox{\begin{array}{M}\indxs{\eqcirc}
  A relation $\hxsd{\eqcirc}\in\clRxx$ is an \reld{equivalence relation} on a set $\setX$ if
  \\\indentx$\ds\begin{array}{l l@{\qquad}C@{\qquad}D@{\qquad}D}
    1. & x\eqcirc x
       & \forall x\in\setX
       & (\prope{reflexive})
       & and
       \\
    2. & x\eqcirc y \implies y\eqcirc x
       & \forall x,y\in\setX
       & (\prope{symmetric})
       & and
       \\
    3. & x\eqcirc y \mbox{ and }  y\eqcirc z \implies x\eqcirc z
       & \forall x,y,z\in\setX
       & (\prope{transitive})
  \end{array}$
\end{array}}
\end{definition}

%---------------------------------------
\begin{example}
%---------------------------------------
Examples of equivalence relations include
\exbox{\begin{array}{ll}
  1. & \text{The equality relation $\hxs{=}$ on the set of integers $\Z$.} \\
  2. & \text{The set equality relation $=$ ($\setA=\setB\implies \setA\subsetneq \setB$ and $\setB\subsetneq \setA$) on the set of all sets.} \\
  3. & \text{The similarity relation $\hxs{\sim}$ (all angles equal) on the set of triangles.} \\
  4. & \text{The modulo relation $\sim$ of all whole numbers that divide $60$ ($60|\cdot$).}
\end{array}}
\end{example}


%---------------------------------------
\begin{definition}
\label{def:eqclass}
%---------------------------------------
\defbox{\begin{array}{M}
  A set $\setE$ is an \structd{equivalence class} with respect to the equivalence relation $=$
  if
  \\\indentx
  $x=y \qquad \forall x,y\in\setE$
\end{array}}
\end{definition}

Equivalence relations occur, of course, in the context of relations.
In the context of sets, a structure that is ``equivalent" to the equivalence relation
is the \structe{partition}\ifsxref{setstrct}{def:ss_partition}.
In particular, an equivalence relation on a set generates a partition,
and a partition on a set defines an equivalence relation.

%=======================================
\section{Functions}
%=======================================
\qboxnpqt
  { \href{http://www-history.mcs.st-andrews.ac.uk/Biographies/Poincare.html}{Jules Henri Poincar\'e} 
    %\href{http://en.wikipedia.org/wiki/Henri_Poincar%C3%A9}{Jules Henri Poincar\'e} 
    \href{http://www-history.mcs.st-andrews.ac.uk/Timelines/TimelineF.html}{(1854--1912)}, 
    \href{http://www-history.mcs.st-andrews.ac.uk/BirthplaceMaps/Places/France.html}{French physicist and mathematician}
    \index{Poincar\'e, Jules Henri}
    \index{quotes!Poincar\'e, Jules Henri}
    \footnotemark
  }
  {../common/people/poincare.jpg}
  {La logique parfois engendre des monstres. 
   Depuis un demi-sie`cle on a vu surgir une foule de fonctions bizarres qui 
   semblent s'efforcer de ressembler aussi peu que possible aux honn\^etes 
   fonctions qui servent \`a quelque chose.}
  %{Logic sometimes makes monsters. 
  % For half a century we have seen a mass of bizarre functions which appear 
  % to be forced to resemble as little as possible honest functions which serve some purpose.}
  {Logic sometimes breeds monsters. 
   %For half a century there has been springing up a host of weird functions, 
   For half a century there has been springing up a host of bizarre functions, 
   which seem to strive to have as little resemblance as possible to honest functions
   that are of some use.}
  \citetblt{
    quote:      & \citerc{poincare_sam}{book 2, chap. 2, sec. 5, par. 3},
                  \citerpg{poincare_sam_eng}{125}{0486432696}\\
   %translation & \citerpg{poincare_sam_eng}{125}{0486432696}\\
    image:  & \url{http://www-groups.dcs.st-and.ac.uk/~history/PictDisplay/Poincare.html}
    }

A \structe{relation} is a kind of mapping between two sets.
A relation $\relation\subseteq \cprodXY$ permits
a single point in $\setX$ to map to two or more points in $\setY$.
That is, it is possible that $\opair{x}{y_1}$ and $\opair{x}{y_2}$ with $y_1\ne y_2$
are both members of some
relation $\relation\subseteq \cprodXY$.
%($x\in\setX$ is related to both $y_1$ and $y_2$).
An example of this is the relation $\le$ in the set $\Z\cprod\Z$ where
$1\le2$ {\bf and} $1\le3$; so in this example the element $1\in\setX$ has a
relationship with both $2\in\setY$ and $3\in\setY$.
Alternatively, we can say that both $\opair{1}{2}$ and $\opair{1}{3}$ are 
elements of the relation $\le$.
However, there is a special case of relations where this is not allowed;
\structe{functions} (next definition) are a kind of relation where
the only way that both $\opair{x}{y_1}$ and $\opair{x}{y_2}$ can be elements 
of the \prope{relation} is if $y_1=y_2$.
The set of all functions in $\cprodXY$ is denoted $\clFxy$.
This notation is meaningful because the number of functions in $\clFxy$ is
$\seto{\setY}^{\seto{\setX}}$ \xref{prop:f_order}.
%---------------------------------------
\begin{definition}
\label{def:fnd_f(x)}
\citetbl{
  %\citerpg{halmos1960}{30}{0387900926}
  \citerpg{suppes1972}{86}{0486616304},
  \citerpg{kelley1955}{10}{0387901256},
  %\citerp{kbr}{161},
  \citer{bourbaki1939}
  %\cithrp{bottazzini}{7}
  }
%---------------------------------------
Let $\setX$ and $\setY$ be sets.
Let $\hxs{\owedge}$ be the ``relational and" \xref{def:rel_and}.
\defbox{\begin{array}{M}\indxs{\ff}\indxs{\clF{\setX}{\setY}}
  A relation $\hxsd{\ff}\in\clRxy$ is a \reld{function} if
    \\\indentx$\ds
       \opair{x}{y_1}\in\ff \; \owedge \; \opair{x}{y_2}\in\ff \implies y_1=y_2
       \qquad\scriptstyle
       \text{(for each $x$, there is only one $\ff(x)$)}
    $\\
  The set of all functions in $\hxs{\clRxy}$ is denoted 
    \\\indentx
      $\ds\hxsd{\clFxy} \eqd \set{\ff\in\clRxy}{\text{$\ff$ is a function}}.$
\end{array}}
\end{definition}


%=======================================
\section{Logic}
%=======================================
\qboxnpq
  {
    \href{http://en.wikipedia.org/wiki/Paul_halmos}{Paul R. Halmos}
    \href{http://www-history.mcs.st-andrews.ac.uk/Timelines/TimelineG.html}{(1916--2006)},
    \href{http://www-history.mcs.st-andrews.ac.uk/BirthplaceMapsOld/Places/Germany.html}{Hungarian}-born
    Jewish-\href{http://www-history.mcs.st-andrews.ac.uk/BirthplaceMapsOld/Places/USA.html}{American} mathematician
    \index{Halmos, Paul R.}
    \index{quotes!Halmos, Paul R.}
    \footnotemark
  }
  {../common/people/halmos.jpg}
  {My most nearly immortal contributions are an abbreviation and a typographical symbol.
      I invented ``iff", for ``if and only if"---but I could never believe that I was really
      its first inventor.%
      \ldots
      %I am quite prepared to beieve that it existed before me,
      %but I don't know that it did, and my invention (re-invention?)
      %of it is what spread it thorugh the mathematical world.
      The symbol is definitely not my invention---it appeared in popular magazines
      (not mathematical ones) before I adopted it,%
      \ldots
      %but, once again,
      %I seem to have introduced it into mathematics.
      It is the symbol that sometimes looks like $\hxsd{\mdlgwhtsquare}$, %$\square$,
      and is used to indicate an end, usually the end of a proof.}
      %It is most frequently called the ``tombstone", but at least one generous author
      %referred to it as the ``halmos".}
  \citetblt{
    quote: & \citerpg{halmos1985}{403}{3540960783}  \\
    image: & \url{http://en.wikipedia.org/wiki/Image:Paul_Halmos.jpeg}
    }



%---------------------------------------
\begin{definition}[logic order relation]
%\citep{wolf}{20}
\label{def:==>}
\label{def:implies}
\label{def:onlyif}
%---------------------------------------
Let $\ltrue$ and $\lfalse$ represent the logical properties of ``\prope{true}" and
``\prope{false}", respectively \xref{ax:plogic}.
\defbox{\begin{array}{M}\indxs{\implies}\index{logical operations!implies}
  The \opd{implies}, or \opd{only if}, relation $\implies$ is defined as
  \\\indentx$\hxsd{\implies}\quad\eqd\quad\setn{\opair{\opair{\lfals}{\lfals}}{\ltrue},\; \opair{\opair{\lfals}{\ltrue}}{\ltrue},\; \opair{\opair{\ltrue}{\lfals}}{\lfals},\; \opair{\opair{\ltrue}{\ltrue}}{\ltrue}}$
\end{array}}
\end{definition}

%---------------------------------------
\begin{definition}
\label{def:<==}
\label{def:<==>}
\label{def:impliedby}
\label{def:if}
%\label{def:iff}
%---------------------------------------
Let $\ltrue$ and $\lfalse$ represent the logical properties of ``\prope{true}" and
``\prope{false}", respectively.
\defbox{
  %\indxs{\impliedby}\indxs{\iff}
  \index{logical operations!implied by}\index{logical operations!if and only if}
  \begin{array}{lMlDl@{\qquad}C}
    x \hxsd{\impliedby} y & if & y \implies x &     &                & \forall x,y\in\setft \\
    x \hxsd{\iff}       y & if & x \implies y & and & y \implies x   & \forall x,y\in\setft
  \end{array}}
\end{definition}

Alternatively, the relations presented in \xref{def:==>} and \xref{def:<==}
can be represented in the form of truth tables:
\[\begin{array}[t]{l@{\qquad}l@{\qquad}l}
   \begin{array}[t]{|cc|c|}
      \hline
      \mc{3}{|Z|}{``implies" or ``only if"}\\
      \mc{3}{|c|}{\implies:\setft^2\to\setft}\\
      \hline
      x & y & x\implies y  \\
      \hline
      \lfals & \lfals & \ltrue \\
      \lfals & \ltrue & \ltrue \\
      \ltrue & \lfals & \lfals \\
      \ltrue & \ltrue & \ltrue \\
      \hline
   \end{array}
&
   \begin{array}[t]{|cc|c|}
      \hline
      \mc{3}{|Z|}{``implied by" or ``if"}\\
      \mc{3}{|c|}{\impliedby:\setft^2\to\setft}\\
      \hline
      x & y & x\impliedby y  \\
      \hline
      \lfals & \lfals & \ltrue \\
      \lfals & \ltrue & \lfals \\
      \ltrue & \lfals & \ltrue \\
      \ltrue & \ltrue & \ltrue \\
      \hline
   \end{array}
&
   \begin{array}[t]{|cc|c|}
      \hline
      \mc{3}{|Z|}{``if and only if"}\\
      \mc{3}{|c|}{\iff:\setft^2\to\setft}\\
      \hline
      x & y & x\iff y  \\
      \hline
      \lfals & \lfals & \ltrue \\
      \lfals & \ltrue & \lfals \\
      \ltrue & \lfals & \lfals \\
      \ltrue & \ltrue & \ltrue \\
      \hline
   \end{array}
\end{array}\]







%---------------------------------------
\begin{definition}[Propositional logic]
\label{def:flogic}
%\label{def:lnot}
%\label{def:land}
%\label{def:lor}
%\label{def:lxor}
\index{propositional logic}
\index{logic!propositional}
\index{logical operation}
\index{logical operation!AND}
\index{logical operation!OR}
\index{logical operation!NOT}
\index{logical operation!XOR}
%---------------------------------------
Let $\ltrue$ and $\lfalse$ represent the logical properties of ``\prope{true}" and 
``\prope{false}", respectively \xref{ax:plogic}.
The following binary operators are defined according to the expressions below:
\defbox{%\indxs{\land}\indxs{\lor}\indxs{\lnot}\indxs{\lxor}
  \begin{array}{rclD}
    \hxsd{\lnot} &\eqd& \setn{\opair{\lfals}{\ltrue},\; \opair{\ltrue}{\lfals}}                                                                                                                     & (\prope{logical not})
  \\\hxsd{\lor}  &\eqd& \setn{\opair{\opair{\lfals}{\lfals}}{\lfals},\; \opair{\opair{\lfals}{\ltrue}}{\ltrue},\; \opair{\opair{\ltrue}{\lfals}}{\ltrue},\; \opair{\opair{\ltrue}{\ltrue}}{\ltrue}} & (\prope{logical or})
  \\\hxsd{\land} &\eqd& \setn{\opair{\opair{\lfals}{\lfals}}{\lfals},\; \opair{\opair{\lfals}{\ltrue}}{\lfals},\; \opair{\opair{\ltrue}{\lfals}}{\lfals},\; \opair{\opair{\ltrue}{\ltrue}}{\ltrue}} & (\prope{logical and})
  \\\hxsd{\lxor} &\eqd& \setn{\opair{\opair{\lfals}{\lfals}}{\lfals},\; \opair{\opair{\lfals}{\ltrue}}{\ltrue},\; \opair{\opair{\ltrue}{\lfals}}{\ltrue},\; \opair{\opair{\ltrue}{\ltrue}}{\lfals}} & (\prope{logical exclusive or})
\end{array}}
\end{definition}

Alternatively, the relations presented in \pref{def:flogic}
can be represented in the form of truth tables:
\[\begin{array}{l*{3}{@{\qquad}l}}
   \begin{array}[t]{|c|c|}
      \hline
      \mc{2}{|c|}{\cellcolor{blue}\text{\bfseries\color{white}\ope{logical not}}}\\
      %\mc{2}{|c|}{\lnot:\setft\to\setft}\\
      \hline
      x & \lnot x  \\
      \hline
       \lfals & \ltrue \\
       \ltrue & \lfals \\
      \hline
   \end{array}
&
   \begin{array}[t]{|cc|c|}
      \hline
      \mc{3}{|c|}{\cellcolor{blue}\text{\bfseries\color{white}\ope{logical or}}}\\
      %\mc{3}{|c|}{\lor:\setft^2\to\setft}\\
      \hline
      x & y & x\lor y  \\
      \hline
      \lfals & \lfals & \lfals \\
      \lfals & \ltrue & \ltrue \\
      \ltrue & \lfals & \ltrue \\
      \ltrue & \ltrue & \ltrue \\
      \hline
   \end{array}
&
   \begin{array}[t]{|cc|c|}
      \hline
      \mc{3}{|c|}{\cellcolor{blue}\text{\bfseries\color{white}\ope{logical and}}}\\
      %\mc{3}{|c|}{\land:\setft^2\to\setft}\\
      \hline
      x & y & x\land y  \\
      \hline
      \lfals & \lfals & \lfals \\
      \lfals & \ltrue & \lfals \\
      \ltrue & \lfals & \lfals \\
      \ltrue & \ltrue & \ltrue \\
      \hline
   \end{array}
&
   \begin{array}[t]{|cc|c|}
      \hline
      \mc{3}{|c|}{\cellcolor{blue}\text{\bfseries\color{white}\ope{logical exclusive-or}}}\\
      %\mc{3}{|c|}{\lxor:\setft^2\to\setft}\\
      \hline
      x & y & x\lxor y  \\
      \hline
      \lfals & \lfals & \lfals \\
      \lfals & \ltrue & \ltrue \\
      \ltrue & \lfals & \ltrue \\
      \ltrue & \ltrue & \lfals \\
      \hline
   \end{array}
\end{array}\]


Note that the ``logical and" \emph{function} $\land$ as defined in \pref{def:flogic} (previous)
and the ``\prope{relational and} "\emph{relation} $\owedge$ as defined in \prefpp{def:AxB} 
are equivalent relations.

\ifdochas{logic}{%
  More information about logic structure is presented in \prefpp{chp:logic}.
  Key concepts concerning the structure of logic is more conveniently demonstrated 
  once the concepts of
  \structe{order}            \ifsxref{order}{chp:order},
  \structe{lattices}         \ifsxref{lattice}{chp:lattice}, and
  \structe{Boolean algebras} \ifsxref{boolean}{chp:boolean}
  have been introduced.
  }


%---------------------------------------
\begin{definition}[predicate logic]
\label{def:predicate_logic}
\index{existensial quantifier}
\index{universal quantifier}
%---------------------------------------
\defbox{\begin{array}{rc>{\ds}l @{\qquad}D}
  \exists i\in I \st p(x_i) \text{ is true}
    &\iff& \hxsd{\bigvee_{i\in I}} [ p(x_i) \text{ is true}]
    & (\propd{existensial quantifier})
  \\
  \forall i\in I,\; p(x_i) \text{ is true}
  &\iff& \hxsd{\bigwedge_{i\in I}} [ p(x_i) \text{ is true}]
  & (\propd{universal quantifier})
\end{array}}
\end{definition}


%---------------------------------------
\begin{theorem}
\citep{wolf}{62}
\label{thm:notQuant}
%---------------------------------------
Let $x$ be a condition and $P(x)$ be a logical statement dependent on $x$.
\thmbox{\begin{array}{l@{\hs{8ex}}c@{\hs{8ex}}l}
  \lnot \left[\forall x,\; P(x)\right]
  &\text{is equivalent to}&
  \exists x \st \lnot P(x)
\\
  \lnot \left[\exists x,\st P(x)\right]
  &\text{is equivalent to}&
  \forall x,\; \lnot P(x)
\end{array}}
\end{theorem}

%---------------------------------------
\begin{theorem}
\citetbl{
  \citeP{godel1930},
  \citeP{godel1930e},
  \citeP{henkin1949},
  \citerpgc{takeuti1975}{43}{0444104925}{Theorem 8.2}
  }
\label{thm:found_firstpred_complete}
%---------------------------------------
\thmbox{\begin{array}{M}
  Any statement constructed using first order predicate logic is \prope{complete}.
\end{array}}
\end{theorem}

%=======================================
\section{Set operations}
%=======================================
Equipped with the concept of the \structe{relation},
we are now in a position to construct the following fundamental relationships
between sets:
  \\\begin{tabular}{@{\quad}>{$\imark$ }llll}
    The \ope{set inclusion} relation  & $\setA\subseteq\setB$      &means& $x\in\setA \implies x\in\setB$. \\
    The \ope{equality}      relation  & $\setA=\setB$              &means& $\setA\subseteq\setB$ and $\setB\subseteq\setA$. \\
    The \ope{non-equality}  relation  & $\setA\ne \setB$           &means& $\setA=\setB$ is false ($\lfalse$). \\
    The \ope{proper subset} relation  & $\setA\subsetneq\setB$     &means& $\setA\subseteq\setB$ and $\setA\ne \setB$. \\
  \end{tabular}\\
These informal set relation descriptions are reinforced with their formal
definitions next.
%---------------------------------------
\begin{definition}
%\label{def:subset}
\citetbl{
  \citerpg{kelley1955}{2}{0387901256}
  }
%---------------------------------------
Let $\setA$ and $\setB$ be sets.
\defbox{\indxs{\subseteq}\indxs{=}\indxs{\ne }\indxs{\subset}
  %\begin{array}{p{6\tw/16}@{\qquad} >{\ds}l}
  \begin{array}{MD>{\ds}l}
  \text{The relation ``\reld{subset}" $\hxsd{\subseteq}$ } 
    & is defined as the set
    & \set{\opair{\setA}{\setB}}{x\in\setA \implies x\in\setB}
    \\
  \text{The relation ``\reld{set equality}" $\hxsd{=}$ } 
    & is defined as the set
    & \set{\opair{\setA}{\setB}}
          {\brp{\setA \subseteq \setB} \;\land\; \brp{\setB \subseteq \setA}}
    \\
  \text{The relation ``\reld{set non-equality}" $\hxsd{\ne}$ } 
    & is defined as the set
    & \set{\opair{\setA}{\setB}}{\setA=\setB \text{ is false ($\lfalse$)}}
    \\
  \text{The relation ``\reld{proper subset}" $\hxsd{\subsetneq}$ } 
    & is defined as the set
    & \set{\opair{\setA}{\setB}}
          {\brp{\setA\subseteq\setB} \;\land\; \brp{\setA\ne \setB}}
\end{array}}
\end{definition}


%---------------------------------------
\begin{definition}
\label{def:set_ops}
\citetbl{
  \citerppg{verescagin2002}{1}{2}{0821827316},
  \citerpp{ab}{2}{4}
  }
\footnote{Origin of $\setu$ and $\seti$: \citer{peano1888}, \citer{peano1888e}}
%  \\\begin{tabular}{l}
%    \citer{peano1888}  \\
%    \citer{peano1888e}
%  \end{tabular}}
\index{sets!operations}
%---------------------------------------
Let $\ssP{\setX}$ be the \structe{power set} \xref{def:fnd_powerset} on a set $\setX$.
\defbox{%\indxs{\setu}\indxs{\seti}\indxs{\setd}\indxs{\sets}
  \begin{array}{l>{\eqd}cl@{\qquad}C@{\qquad}D}
   \hxsd{\cmpA}
     && \set{x\in\setX}{\lnot(x\in\setA)  }
     & \forall \setA\in\ssP{\setX}
     &(\opd{complement})
     \\
   \setA\hxsd{\setu}\setB
     && \set{x\in\setX}{ (x\in\setA) \lor  (x\in\setB) }
     & \forall \setA,\setB\in\ssP{\setX}
     &(\opd{union})
     \\
   \setA\hxsd{\seti}\setB
     && \set{x\in\setX}{ (x\in\setA) \land (x\in\setB) }
     & \forall \setA,\setB\in\ssP{\setX}
     &(\opd{intersection})
   \\
   \setA\hxsd{\setd}\setB
     && \set{x\in\setX}{ (x\in\setA) \land \lnot(x\in\setB) }
     & \forall \setA,\setB\in\ssP{\setX}
     &(\opd{difference})
     \\
   \setA\hxsd{\sets}\setB
     && \set{x\in\setX}{(x\in\setA) \lxor (x\in\setB)}
     & \forall \setA,\setB\in\ssP{\setX}
     &(\opd{symmetric difference})
\end{array}}
\end{definition}


\prefpp{def:(a,b)} defined the \structe{ordered pair} using Kuratowski's
somewhat cryptic expression
 $\opair{a}{b} \eqd \setn{\setn{a},\, \setn{a,b}}$.
\pref{thm:(a,b)_extract} (next) helps show that this expression is reasonable
in that $a$ and $b$ can be extracted
from the set $\setn{\; \setn{a},\; \setn{a,b}\; }$
by simple set operations on its elements.
Also, a key test to the usefulness of the definition of ordered pairs is whether
or not $\opair{a}{b}=\opair{c}{d}$ if and only if $a=c \text{ and } b=d$.
In fact, this statement is true and demonstrated by \prefpp{cor:(a,b)=(c,d)}.

%---------------------------------------
\begin{theorem}
\label{thm:(a,b)_extract}
%---------------------------------------
\thmbox{\begin{array}{rcl clcl}
   \setn{a} &=&  \setopi\opair{a}{b} &=& \setopi\setn{ \setn{a},\setn{a,b} } &=& \setn{a}\seti\setn{a,b} \\
   \setn{b} &=&  \setops\opair{a}{b} &=& \setops\setn{ \setn{a},\setn{a,b} } &=& \setn{a}\sets\setn{a,b}
\end{array}}
\end{theorem}


%---------------------------------------
\begin{corollary}
\label{cor:(a,b)=(c,d)}
\citetbl{
  \citerpg{apostol1975}{33}{0201002884},
  \citerpg{hausdorff1937e}{15}{0828401195}
  }
%---------------------------------------
\corbox{
  \opair{a}{b}=(c,d)
  \qquad\iff\qquad
  a=c \text{ and } b=d
  }
\end{corollary}
\begin{proof}
\begin{align*}
  \intertext{1. Proof that $\opair{a}{b}=(c,d) \implies a=c \text{ and } b=d$:}
    \setn{a}
      &= \Seti\opair{a}{b}
      && \text{by \prefp{thm:(a,b)_extract}}
    \\&= \Seti(c,d)
      && \text{by left hypothesis}
    \\&= \setn{c}
      && \text{by \prefp{thm:(a,b)_extract}}
    \\
    \setn{b}
      &= \Sets\opair{a}{b}
      && \text{by \prefp{thm:(a,b)_extract}}
    \\&= \Sets(c,d)
      && \text{by left hypothesis}
    \\&= \setn{d}
      && \text{by \prefp{thm:(a,b)_extract}}
    \\
  \intertext{2. Proof that $\opair{a}{b}=(c,d) \impliedby a=c \text{ and } b=d$:}
    \opair{a}{b}
      &= (c,d)
      && \text{by right hypothesis}
\end{align*}
\end{proof}



%=======================================
\section{Abstract Mathematical Spaces}
\label{sec:space}
%=======================================
\begin{figure}[th]
  \centering
  \includegraphics{graphics/spaces.pdf}%
  \caption{Lattice of mathematical spaces\label{fig:found_spaces}}
\end{figure}%
The abstract space was introduced by Maurice Fr{/'e}chet in his 1906 Ph.D. thesis.%
\citetbl{
  \citeP{frechet1906},
  \citer{frechet1928}
  %\citePp{frechet1950}{147}
  }
An abstract \hid{space} in mathematics does not really have a rigorous definition;
but in general it is a set together with some other unifying structure.

\qboxnps
  {\href{http://en.wikipedia.org/wiki/Frechet}{Marice Ren{/'e} Fr{/'e}chet}
   (\href{http://www-history.mcs.st-andrews.ac.uk/Timelines/TimelineF.html}{1878--1973}),
   \href{http://www-history.mcs.st-andrews.ac.uk/BirthplaceMaps/Places/France.html}{French mathematician}
   who in his 1906 Ph.D. dissertation introduced the concept of the \structe{metric space}
   \index{Fr/'echet, Marice Ren/'e}
   \index{quotes!Fr/'echet, Marice Ren/'e}
   \footnotemark
  }
  {../common/people/frechet.jpg}
  {\ldots
   A collection of these abstract elements will be called an \emph{abstract set}.
   If to this set there is added some rule of association of these elements,
   or some relation between them,
   the set will be called an \emph{abstract space}.}
  \citetblt{
    quote: & \citePp{frechet1950}{147} \\
            %\citerp{carothers2000}{36}
    image: & \url{http://en.wikipedia.org/wiki/Frechet}
    }

Examples of common abstract spaces include the following:
\begin{enumerate}
  \item \structe{linear space}:
        a set of ``vectors" over a field
        with a vector-vector addition operator $+$ such vectors can be added together
        and a scalar-vector multiplication operator $\cprod$
        such that field elements can be multiplied with vectors.
        \\\citor{peano1888}

  \item \structe{metric space}:
        A set of elements together with a metric $\metricn$ such giving the ``distance"
        between any two elements.
        \\\citeP{frechet1906}

  \item \structe{measure space}:
        A set of sets and a measure $\msm$ that measures the ``size" of a set.

  \item \structe{normed vector space}:
        A vector space together with a norm $\normn$ that measures the ``length" of a vector.
        \\\citor{banach1922}

  \item \structe{inner-product space}:
        A vector space together with an inner-product $\inprodn$.

  \item \structe{Banach space}:
        a normed vector space that is ``complete" with respect to the norm $\normn$.
        \\\citor{banach1922}

  \item \structe{Hilbert space}:
        an inner-product space that is ``complete" with respect to the norm induced by the
        inner-product $\inprodn$.
        \\\citor{vonNeumann1929}
\end{enumerate}

