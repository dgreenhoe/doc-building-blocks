%============================================================================
% LaTeX File
% Daniel J. Greenhoe
%============================================================================

%=======================================
\chapter*{Preface}
\addcontentsline{toc}{section}{Preface}
\markboth{Preface}{Preface}
%=======================================

%\begin{personal}

%=======================================
\section*{Purpose}
%=======================================
This book is intended for people who are fluent in English and who want to learn Mandarin Chinese.
It is somewhat unique among Chinese language learning books
in that most every Chinese character that appears in the book is accompanied by
Romanized pronunciation above the character and the character's English meaning below.

Many examples are included in this text.
And many of these examples are extracted from children's story books.
There are several effective methods for language acquisition;
and one of the most effective is reading.
One big advantage of reading children's Chinese story books is that
most every character includes phonetic pronunciation for easy pronunciation
and reference in a Chinese-English dictionary.

Some examples of children's Chinese story books include the following:
\begin{enumerate}
  \item The \zht{寶寶心靈成長雙語繪本}
        {\normalfont(b\v{a}o b\v{a}o xi\=n li\'ng ch\v{e}ng zh\v{a}ng sh\=uang y\v{u} h\`ui y\v{u})}
        Baby intelligent growth bilingual illustrated book series:
    \\\citer{baobao_cow}
    \\\citer{baobao_elephant}
    \\\citer{baobao_rooster}
    \\\citer{baobao_comb}
    \\\citer{baobao_rabbit}
    \\\citer{baobao_tiger}
    \\\citer{baobao_lion}
    \\\citer{baobao_pig}
    \\\citer{baobao_bear}
    \\\citer{baobao_monkey}
    \\\citer{baobao_gorilla}

  \item The \zht{愛與心靈成長學習繪本}
        {\normalfont(\`ai y\v{u} xi\=n li\'ng ch\'eng zh\v{a}ng x\'ue \'xi h\`ui b\v{e}n)}
        Love and intelligence growth and learning illustrated book series:
    \\\citer{lovemind_dog}
    \\\citer{lovemind_beth}
    \\\citer{lovemind_cat}
    \\\citer{lovemind_rain}
    \\\citer{lovemind_octopus}
    \\\citer{lovemind_bear}
    \\\citer{lovemind_pig}
    \\\citer{lovemind_water}

  \item Here is an online reference with many children's stories with audio:
        \footnote{Many thanks to Dr. Po-Ning Chen \zht{(陳伯寧)} for referring me to this web site.}
        \\\url{http://children.cca.gov.tw/search/animation_list.php?module=painting&class=&categories=&date=&x=21&y=11}

\end{enumerate}



%=======================================
\section*{Typesetting Tools}
%=======================================
  This text was typeset using Xe\LaTeX.



%=======================================
\section*{Writing with hope}
%=======================================
\parbox{32mm}{%
  \includegraphics*[width=30mm]{../common/people/dan.jpg}
  }%
\hfill%
\parbox{\tw-33mm}{%
  It is my hope that others will find this text useful in their struggle to learn Mandarin Chinese;
  and that these can in turn use their knowledge for the benefit of others.

  \color{signature}{Daniel J. Greenhoe \zhthw{(柯晨光)}}
  }



%\end{personal}