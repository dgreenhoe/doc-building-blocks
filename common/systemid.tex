%============================================================================
% Daniel J. Greenhoe
% LaTeX file
%============================================================================
%============================================================================
\chapter{System Identification}
%============================================================================
\qboxnpq{
  Karl R. Popper (1902--1994)
  \index{Popper, Karl}
  \index{quotes!Popper, Karl}
  \footnotemark
  }
  {../common/graphics/portraits/popper_karl_wkp_pdomain.jpg}
  %{Mathematics compares the most diverse phenomena and discovers the secret analogies that unite them.}
  {%
  I can therefore gladly admit that falsificationists like myself much 
  prefer an attempt to solve an interesting problem by a bold conjecture, 
  even (and especially) if it so turns out to be false, 
  to any recital of a sequence of irrelevant truisms. 
  We prefer this because we believe that this is the way in which we can 
  learn from our mistakes and that in finding that our conjecture was false 
  we shall have learned much about the truth, 
  and shall have got nearer to the truth.
  }
  \footnotetext{\begin{tabular}[t]{ll}
    quote: & \citerp{popper1962}{231}, \citerpg{popper1963}{313}{1135971307}\\
    image: & \url{https://en.wikipedia.org/wiki/File:Karl_Popper.jpg}, ``no known copyright restrictions"
  \end{tabular}}

Let $\opS$ be a \structe{system} with \fncte{impulse response} $\fh[n]$ with
with \ope{DTFT} $\fH(\omega)$,
input $\rvx[n]$, and output $\rvy[n]$. 
Often in the field of ``digital signal processing" (DSP), $\opS$ is a ``filter" 
with known $\fh[n]$ and $\fH(\omega)$ because the filter $\opS$ was 
designed by a designer who had direct control over $\fh[n]$.

However in many other practical situations, $\opS$ is some other system 
for which $\fh[n]$ and $\fH(\omega)$ is \emph{not} known\ldots but which we may
want to \ope{estimate}. Examples of such $\opS$ is a 
device on an industrial shaker table, a communication channel, or the entire earth.

Determining $\fh[n]$ and/or $\fH(\omega)$ is part of an operation called ``\ope{system identication}".
Determining $\fH(\omega)$ in particular is referred to as
``\ope{Frequency Response Identification}"\footnote{\citerpg{shin2008}{292}{0470725648}}
or as 
``\ope{Frequency Response Function}" (``\ope{FRF}") estimation.\footnote{\citePpc{cobb1988}{1}{FRF ``measurement"}}
\ope{FRF} estimation is a challenging problem and one that
many people have devoted much effort to. 
This chapter describes some of that effort.

In the early days, people used a rather obvious technique for determining $\fH(\omega)$---the simple
\fncte{sine sweep}. That is, they made the input a sine wave with slowly increasing (or decreasing) 
frequency while measuring the resulting output.
This technique, although effective, was ``very slow".\footnote{
  \citeP{leuridan1986}{911}{``Stepped Sine Testing"},
  \citePpc{cobb1988}{1}{Chapter 1---Introduction},
  \citerppgc{ewins1986}{125}{140}{0863800173}{3.7 \scshape Use of different excitation types}
  }
\emph{And} there is another problem---we don't always have control over the input signal.
Examples of this include earthquake and volcanic activity analysis.

An alternative to the sine-sweep input is \fncte{random sequence} input. 
All the techniques that follow in this chapter are of this type.
A problem with using random sequences directly for estimating $\fH(\omega)$ is that the 
estimate $\estH(\omega)$ is itself also random.
This is not what we want. We want an estimate that we can actually write down 
on paper or at least plot on paper.

A solution to this is to not use the random sequences directly to estimate $\fH(\omega)$, 
but instead to first use the \ope{expectation} operator $\pE$ \xref{def:pE}.
The expectation operator takes a quantity $\rvX$ that is inherently ``random" 
(with some probability distribution $\pdfp(x)$) and 
turns it into a deterministic ``constant" $\pE\rvX$.

The operator $\pE$ is also used by the spectral density functions 
$\Swxx(\omega)$ and $\Swxy(\omega)$ \xref{def:Swxy}.
And $\Swxx(\omega)$ and $\Swxy(\omega)$ are what are typically used to calculate 
an estimate $\estH(\omega)$.

%=======================================
\section{Definitions}
%=======================================
%\begin{figure}
%  \centering%
%  \includegraphics{graphics/sysH_xyuvpq.pdf}
%  \caption{System model with additive input and output noise\label{fig:sysH_xyuvpq}}
%\end{figure}
%Consider the additive noise system $\opS$ illustrated here: % in \prefpp{fig:sysH_xyuvpq}.
\begin{minipage}{\tw-70mm}
Consider the additive noise system $\opS$ illustrated to the right:
\end{minipage}
\hfill\tbox{\includegraphics{graphics/sysH_xyuv--.pdf}}
\\
In the very special (a.k.a unrealistic) case of $\opS$ having \prope{zero measurement noise}
($\rvu[n]=\rvv[n]=0$), 
$\fh[n]$ being \prope{linear time invariant} (\prope{LTI}), 
and input $\rvx[n]$ being \prope{wide sense stationary}\ldots
then we can determine (a.k.a ``identify") $\fh[n]$ or $\fH(\omega)$
exactly by $\fH(\omega)=\ffrac{\Swyx(\omega)}{\Swxx(\omega)}$ \xref{cor:Swxy}.

However, in practical situations, there is measurement error \xref{fig:xhyplusv}. 
An example may include 
``road noise" from a test being performed in a moving vehicle.

\begin{minipage}{\tw-70mm}
If the measurement error is at the output only (and under the assumptions of \prope{LTI} and \prope{WSS})
then $\estH_1$ (next definition) is the ideal estimator in the sense that $\estH_1=\fH$ \xref{prop:estH1}.
\end{minipage}
\hfill\tbox{\includegraphics{graphics/sysH_xy0v--.pdf}}
\\
%---------------------------------------
\begin{definition}[\fnctd{least squares estimate} of $\fH(\omega)$]
\footnote{
  \citerppgc{bendat1993}{106}{109}{0471570559}{5.1.1 Optimality of Calculations},
  \citerpgc{bendat2011}{185}{1118210824}{$H_1(f) = \ffrac{G_{xy}(f)}{G_{xx}(f)}$ (6.37)},
  \citerpgc{shin2008}{293}{0470725648}{
    $H_1(f)=\ffrac{\Swxy(f)}{\Swxx(f)}$ (9.63);
    which differs from \pref{def:H1}, but see \prefp{rem:Rxym}
    },
  \citeP{bendat1978}{cited by Cobb(1988)---variance estimate for $\estH_1$},
  \citePc{allemang1979}{cited by Shin(2008)},
  \citePpc{leuridan1986}{910}{\fncte{Least Squares Technique}; (8) $[G_{xx}](H)=[G_{xy}]$},
  \citeP{abom1986}{cited by Cobb(1988)---variance estimate for $\estH_1$},
  \citePppc{allemang1987}{54}{55}{5.3.1 $H_1$ Technique; $[H]=[G_{XF}][G_{FF}]^{-1}$ (11)},
  \citePpc{cobb1988}{2}{$^1\hat{H}(f)=\ffrac{\hat{G}_{yx}(f)}{\hat{G}_{xx}(f)}$ (1)},
  \citePpc{goyder1984}{438}{$H(i\omega)=\ffrac{S_{qp}}{S_{pp}}$ (3)},
  \citerpgc{pintelon2012}{233}{1118287398}{$\hat{G}(\Omega_k)=S_{yu}(j\omega_k)S_{uu}^{-1}(j\omega_k)$ (7-30)}
  }
\label{def:H1}
%---------------------------------------
Let $\opS$ be a \structe{system} with input $\rvx[n]$ and output $\rvy[n]$.
\defbox{\begin{array}{Mrc>{\ds}l}
  The transfer function $\estH_1(\omega)$ is defined as & \estH_1(\omega) &\eqd& \frac{\Swyx(\omega)}{\Swxx(\omega)}
\end{array}}
\end{definition}

The estimator $\estH_1$ is a good start.
However in the early 1980s, L. D. Mitchell pointed out that in the presence of input noise,
$\estH_1$ is far from ideal in that it is \prope{biased} with respect to $\fH$;
in fact, $\estH_1$ \prope{under estimates} $\fH$ \xref{prop:estH2}.
Mitchell proposed a new estimator $\estH_2$ (next definition).


\begin{minipage}{\tw-70mm}
This estimator has the special property that when there is input noise but no 
output noise (and under \prope{LTI}, \prope{WSS}, and \prope{uncorrelated} assumptions), 
then it is ideal in the sense that $\estH_2(\omega)=\fH(\omega)$ \xref{prop:estH2}.
\end{minipage}
\hfill\tbox{\includegraphics{graphics/sysH_xyu0--.pdf}}
\\
Note also that in the case of both no input and no output noise, then $\estH_1=\estH_2$ \xref{cor:Swxy}.
%---------------------------------------
\begin{definition}[\fnctd{Inverse Method}]
\footnote{
  \citerpgc{shin2008}{293}{0470725648}{$H_2(f)=\ffrac{\Swyy(f)}{\Swyx(f)}$ (9.65);
    which differs from \pref{def:H2}, but see \prefp{rem:Rxym}
    },
  \citerpgc{bendat2011}{186}{1118210824}{$H_2(f) = \ffrac{G_{yy}(f)}{G_{yx}(f)}$ (6.42)},
  \citePc{mitchell1980}{cited by Cobb(1988)},
  \citePpc{mitchell1982}{278}{``Define what will be called an inverse method for calculation of a FRF as\ldots"; 
    $\~H_2(f)=\ffrac{G_{yy}}{G_{yx}}$ (6); 
    Note this differs with \pref{def:H2} by a conjugate, but note that Mitchell seems to follow Bendat (see his [3] and [4]), 
    which would explain this difference \xref{rem:Rxym}},
  \citePpc{cobb1988}{3}{$^2\hat{H}(f)=\ffrac{\hat{G}_{yy}(f)}{\hat{G}_{xy}(f)}$ (1)}
  }
\label{def:H2}
%---------------------------------------
Let $\opS$ be a \structe{system} with input $\rvx[n]$ and output $\rvy[n]$.
\defbox{\begin{array}{Mrc>{\ds}l}
  The transfer function $\estH_2(\omega)$ is defined as & \estH_2(\omega) &\eqd& \frac{\Swyy(\omega)}{\Swxy(\omega)}
\end{array}}
\end{definition}

Mitchell's $\estH_2$ contribution ``generated a flurry of activity"\footnote{\citePp{cobb1988}{3}}
and soon more $\fH$ estimators appeared. 
%As it turns out, in the presence of \emph{both} input and output measurement noise,
%the traditional estimator $\estH_1$, although ideal when input noise = 0, tends to \prope{under estimate} $\fH$ \xref{prop:estH1}
%and Mitchell's $\estH_2$, although ideal when output noise = 0, 
%tends to \prope{over estimate} $\fH$ \xref{prop:estH2}.
%But what about the case when there is noise on both input and output?
%The next estimator is ideal when noise is present on both and the noise power of the two is equal.
So far we have 
\begin{listi}
  \item $\estH_1$ which is ideal when there is no input noise but 
        \prope{under estimate}s $\fH$ when there is \xref{prop:estH1}
  \item $\estH_2$ which is ideal when there is no output noise but
        \prope{over estimate}s $\fH$ when there is \xref{prop:estH2}.
\end{listi}
But what about estimators for when there is noise on both input and output?
Armed with two estimators that between them account for both input and output noise,
an ``ad hoc" solution might be to somehow take mean values of $\estH_1$ and $\estH_2$ 
to induce new estimators---this approach summarizes the next three definitions.
An arguably more mature approach is to find estimators that are optimal with respect to least squares measures---and
this approach summarizes \pref{def:Hv} -- \prefpp{def:Hkappa}.

%---------------------------------------
\begin{definition}[\fnctd{Arithmetic mean estimator}]
\footnote{
  \citePpc{mitchell1982}{279}{``Frequency Response Calculation: The Average Method"},
  \citePpc{zheng2002}{918}{``1.3 Arithmetic Mean Estimator $H_3$"}
  }
\label{def:Havg}
%---------------------------------------
Let $\opS$ be a \structe{system} with input $\rvx[n]$ and output $\rvy[n]$.
\defbox{\begin{array}{Mrc>{\ds}>{\ds}l}
  The transfer function $\estHavg(\omega)$ is defined as & \estHavg(\omega) &\eqd& \frac{\estH_1(\omega) + \estH_2(\omega)}{2}
\end{array}}
\end{definition}

%---------------------------------------
\begin{definition}[\fnctd{Geometric mean estimator}]
\footnote{
  \citePpc{zheng2002}{918}{``1.4 Geometric Mean Estimator $H_4$"}
  }
\label{def:Hgm}
%---------------------------------------
Let $\opS$ be a \structe{system} with input $\rvx[n]$ and output $\rvy[n]$.
\defbox{\begin{array}{Mrc>{\ds}>{\ds}l}
  The transfer function $\estHgm(\omega)$ is defined as & \estHgm(\omega) &\eqd& \sqrt{\estH_1(\omega)\estH_2(\omega)}
\end{array}}
\end{definition}

%---------------------------------------
\begin{definition}[\fnctd{Harmonic mean estimator}]
\footnote{
  \citePc{carne2006}{$H_C = [H_A^{-1} + H_B^{-1}]^{-1}$}
  %\citergc{ewins1985}{086380036X}{cited by Carne (2006)}
  %\citerpgc{ewins1986}{???}{0863800173}{...}
  }
\label{def:Hharm}
%---------------------------------------
Let $\opS$ be a \structe{system} with input $\rvx[n]$ and output $\rvy[n]$.
\defbox{\begin{array}{Mrc>{\ds}>{\ds}l}
  The transfer function $\estHharm(\omega)$ is defined as & \estHharm(\omega) &\eqd& \frac{1}{\frac{1}{\estH_1(\omega)} + \frac{1}{\estH_2(\omega)}}
\end{array}}
\end{definition}

%---------------------------------------
\begin{definition}[\fnctd{Total Least Squares estimator}]
\footnote{
  \citePpc{white2006}{679}{(6)},
  \citerpgc{shin2008}{294}{0470725648}{(9.69)}
  }
\label{def:Hv}
%---------------------------------------
Let $\opS$ be a \structe{system} with input $\rvx[n]$ and output $\rvy[n]$.
\defboxt{
  The estimate $\estHv(\omega)$ is defined as 
  \\\indentx$\ds
  \estHv(\omega) \eqd 
  \frac{2\Swyx(\omega)}
       {\Swxx(\omega)-\Swyy(\omega) + \sqrt{\brs{\Swxx(\omega)-\Swyy(\omega)}^2 + 4\abs{\Swxy(\omega)}^2}}
  $
  }
\end{definition}

The estimate $\estHv$ and it's refusal to assume zero input or output noise 
was definitely a step in the right direction.
But it still assumes that the input and output noise power to be equal.
And what if it's not? 
Enter Wicks and Vold who in 1986 introduced such an estimator (next definition)
that makes no such assumption.
It features a parameter $s$ in the range $\intco{0}{\infty}$
that can be adjusted depending on the ratio of output and input noise.
And as it turns out, $\estHs = \estH_1$ when $s=0$, $\estHs=\estHv$ when $s=1$, and 
$\estHs\to\estH_2$ as $s\to\infty$. % \xref{thm:Hs}.
%---------------------------------------
\begin{definition}[\fnctd{scaling estimate}]
\footnote{
  \citePc{leclere2012}{(10) with $x$ and $y$ swapped}
  }
\label{def:Hs}
%---------------------------------------
Let $\opS$ be a \structe{system} with input $\rvx[n]$ and output $\rvy[n]$.
\defboxt{
  The estimate $\estHs(\omega)$ with \vald{scaling parameter} $s\in\intco{0}{\infty}$ is defined as 
  \\\indentx$\ds
  \estHs(\omega; s) \eqd 
  \frac{2\Swyx(\omega)}
       {\Swxx(\omega)-s^2\Swyy(\omega) + \sqrt{\brs{\Swxx(\omega)-s^2\Swyy(\omega)}^2 + 4s^2\abs{\Swxy(\omega)}^2}}
  $
  }
\end{definition}

Wicks and Vold's $\estHs$ estimates $\fH$ using a single parameter $s$
that is \prope{constant} with respect to frequency.
This is suitable for additive white noise $\rvw[n]$, which has constant spectral density 
$\Swww(\omega)=\pvar_w$.
However, not all noise is white noise.
And later in 2006, White, Tan and Hammond proposed a generalization (next definition) of $\estHs$ with a scaling 
function $\kappa(\omega)$---that is, one that is not in general constant with respect to frequency
and is thus suitable in the presence of \prope{colored} noise.
%---------------------------------------
\begin{definition}
\footnote{
  \citerpgc{shin2008}{293}{0470725648}{(9.67)},
  \citeP{white2006},
  \citePc{leclere2012}{(10) $\kappa(f)=1/s^2$ and $x$ and $y$ swapped}
  }
\label{def:Hkappa}
%---------------------------------------
Let $\opS$ be a \structe{system} with input $\rvx[n]$ and output $\rvy[n]$.
\defboxt{
  The transfer function $\estHkappa(\omega; \kappa)$ is defined as 
  \\\indentx$\ds
  \estHkappa(\omega) \eqd 
  \frac{2\kappa(\omega)\Swyx(\omega)}
       {\kappa(\omega)\Swxx(\omega)-\Swyy(\omega) + \sqrt{\brs{\kappa(\omega)\Swxx(\omega)-\Swyy(\omega)}^2 + 4\kappa(\omega)\abs{\Swxy(\omega)}^2}}
  $
  }
\end{definition}

The previous estimators all assumed two signals: an input $\rvx[n]$ and an output $\rvy[n]$.
However, in many practical systems, there is a third signal that is ``driving" the system.
In 1984 Goyder proposed an estimator (next definition) that is based on three signals.
%---------------------------------------
\begin{definition}[Three channel estimate]
\footnote{
  \citerpgc{shin2008}{297}{0470725648}{$H_3(f)=\ffrac{S_{ry}(f)}{S_{rx}(f)}$ (9.78)},
  \citePpc{cobb1988}{4}{$^c\hat{H}(f)=\ffrac{\hat{G}_{ys}(f)}{\hat{G}_{xs}(f)}$ (1.4)},
  \citePpc{goyder1984}{440}{$H(i\omega)=\ffrac{S_{qz}}{S_{pz}}$ (5)},
  \citePpc{cobb1990}{450}{(1)},
  \citerpgc{pintelon2012}{241}{1118287398}{$\hat{G}(\Omega_k)=\hat{G}_{ry}(\Omega_k)\hat{G}_{ru}^{-1}(\Omega_k)$ (7-49)}
  }
\label{def:Hc}
%---------------------------------------
Let $\opS$ be a system with input $\rvx[n]$, output $\rvy[n]$, and a driver $\rvp[n]$.
\defboxt{
  The transfer function $\estHc(\omega)$ is defined as 
  \\\indentx$\ds
  \estHc(\omega) \eqd \frac{\Swxx[py](\omega)}{\Swxx[px](\omega)}
  $
  }
\end{definition}

%=======================================
\section{Properties}
%=======================================
%---------------------------------------
\begin{proposition}
\footnote{
  \citerpgc{shin2008}{293}{0470725648}{(9.67)},
  \citeP{white2006},
  \citePc{leclere2012}{(10) $\kappa(f)=1/s^2$ and $x$ and $y$ swapped}
  }
\label{prop:Hkappa}
%---------------------------------------
Let $\estHkappa(\omega)$ be defined as in \prefpp{def:Hkappa}.
\propbox{
  \estHkappa(\omega) = \frac{\Swyy(\omega)-\kappa(\omega)\Swxx(\omega) + \sqrt{\brs{\Swyy(\omega) - \kappa(\omega)\Swxx(\omega)}^2 + 4\kappa(\omega)\abs{\Swxy(\omega)}^2}}
                            {2\Swxy(\omega)}
  }
\end{proposition}

\begin{figure}
  \centering%
  \includegraphics{graphics/sysH_estH.pdf}
  \caption{Ideal least squares estimator for output noise only system\label{fig:sysH_estH}}
\end{figure}
%---------------------------------------
\begin{minipage}{\tw-65mm}
\begin{theorem}
\footnotemark
%---------------------------------------
Let $\opS$ be the \structe{system} illustrated to the right.
Let $\norm{\rvx}\eqd\pE\brs{\rvx\rvx^\ast}$.
\end{theorem}
\end{minipage}
\hfill\tbox{\includegraphics{graphics/sysH_estH.pdf}}
\footnotetext{
  \citerppgc{bendat1980}{98}{100}{0471058874}{5.1.1 Optimal Character of Calculations},
  \citerppgc{bendat1993}{106}{109}{0471570559}{5.1.1 Optimality of Calculations}
  }
\thmbox{
  \mcom{\argmin_{\estH}\norm{\FW(\omega)-\FQ(\omega)}^2 = \estH_1(\omega)}
       {$\estH_1$ is the \prope{least-squares estimator} of $\opS$}
  }
\begin{proof}
\begin{align*}
  \fCost\brp{\estH}
    &\eqd \norm{\FW(\omega)-\FQ(\omega)}^2
  \\&\eqd \inprod{\FW(\omega)-\FQ(\omega)}{\FW(\omega)-\FQ(\omega)}
  \\&\eqd \pE\brp{\brs{\FW(\omega)-\FQ(\omega)}\brs{\FW(\omega)-\FQ(\omega)}^\ast}
  \\&=    \pE\brs{\FW(\omega)\FW^\ast(\omega)} 
        - \pE\brs{\FW(\omega)\FQ^\ast(\omega)} 
        - \pE\brs{\FQ(\omega)\FW^\ast(\omega)} 
        + \pE\brs{\FQ(\omega)\FQ^\ast(\omega)} 
  \\&= \Swww(\omega) - \Swwq(\omega) - \Swwq^\ast(\omega) + \Swqq(\omega)
    \qquad\text{by \thme{Wiener-Kinchin relationship}}
  \\&= \Swww(\omega) - \Swwy(\omega) - \Swwy^\ast(\omega) + \Swyy(\omega)-\Swvv(\omega)
  \\&= \Swxx(\omega)\abs{\estH(\omega)}^2 - \estH(\omega)\Swyx(\omega) - \estH^\ast(\omega)\Swyx^\ast(\omega) + \Swyy(\omega)-\Swvv(\omega)
  \\
  0
    &= \pderiv{}{\estH}\fCost\brp{\estH} 
  \\&= \pderiv{}{\estH}
       \brp{\Swxx\abs{\estH}^2 - \estH\Swyx - \estH^\ast\Swyx^\ast + \Swyy-\Swvv}
  \\&= \pderiv{}{\estH}
       \brp{\Swxx\abs{\estH}^2 - \estH\Swyx - \estH^\ast\Swyx^\ast + \cancelto{0}{\Swyy}-\cancelto{0}{\Swvv}}
  \\&= \brs{\begin{array}{c}\pderiv{}{\estH_R}\\\pderiv{}{\estH_I}\end{array}}
       \brp{\Swxx\abs{\estH}^2 - \estH\Swyx - \estH^\ast\Swyx^\ast}
  \\&= \brs{\begin{array}{c}\pderiv{}{\estH_R}\\\pderiv{}{\estH_I}\end{array}}
       \brp{\Swxx\abs{\estH}^2 - \estH\Swyx - \estH^\ast\Swyx^\ast}
  \\&= \brs{\begin{array}{c}\pderiv{}{\estH_R}\\\pderiv{}{\estH_I}\end{array}}
       \Swxx\brp{\estH_R^2+\estH_I^2} 
     - \brs{\begin{array}{c}\pderiv{}{\estH_R}\estH_R\Re\Swyx\\\pderiv{}{\estH_I}  \estH_I   \Im\Swyx\end{array}}
     - \brs{\begin{array}{c}\pderiv{}{\estH_R}\estH_R\Re\Swyx\\\pderiv{}{\estH_I}(-\estH_I)(-\Im\Swyx)\end{array}}
  \\&= \brs{\begin{array}{c}2\Swxx\estH_R\\2\Swxx\estH_I\end{array}}
     - \brs{\begin{array}{c}\Re\Swyx\\\Im\Swyx\end{array}}
     - \brs{\begin{array}{c}\Re\Swyx\\\Im\Swyx\end{array}}
  \\&= 2\Swxx\brs{\begin{array}{c}\estH_R\\\estH_I\end{array}}
     - 2\brs{\begin{array}{c}\Re\Swyx\\\Im\Swyx\end{array}}
  \\\implies 0
    &= 2\Swxx(\omega)\estH(\omega) - 2\Swyx(\omega)
  \\\implies \estH(\omega)
    &= \frac{\Swyx(\omega)}{\Swxx(\omega)}
\end{align*}
\end{proof}




\begin{figure}
  \centering%
  \includegraphics{graphics/xhyplusv.pdf}
  \caption{System model with additive input and output noise\label{fig:xhyplusv}}
\end{figure}
%---------------------------------------
\begin{proposition}[Output noise only case]
\footnote{
  \citerpgc{shin2008}{294}{0470725648}{$H_1(f)=H(f)$ (9.70); $H_2(f)=H(f)\brp{1+\ffrac{S_{n_yn_y}(f)}{Syy(f)}}$ (9.71)},
  \citePpc{mitchell1982}{278}{$\~H_2(f)=H_0(f)\brp{1+\ffrac{G_{mm}}{G_{vv}}}$}
  }
\label{prop:estH1}
%---------------------------------------
Let $\opS$ be a \structe{system} with \fncte{impulse response} $\fh[n]$,
input $\fx[n]$, and output $\fy[n]$.
Let $\rvu[n]$ and $\rvv[n]$ be \fncte{noise sequence}.
Let $\rvx'[n]\eqd\rvx[n]+\rvu[n]$ and $\rvy'[n]\eqd\rvy[n]+\rvv[n]$.
\propbox{
  \brbr{\begin{array}{FMMD}
      (A).& The system $\fh$ is         &\prope{LTI}          & and
    \\(B).& $\rvx[n]$ and $\rvy[n]$ are &\prope{WSS}          & and
    \\(C).& $\rvx[n]$ and $\rvv[n]$ are &                     &     
        \\& \mc{2}{M}{\quad \prope{uncorrelated}}             & and
    \\(D).& $\rvu[n]=0$                 
  \end{array}}
  \implies
  \brbl{\begin{array}{F>{\ds}rc>{\ds}lDD}
      (1).&\estH_1(\omega) &=& \fH(\omega)                                              & (\prope{unbiased}) & and
    \\(2).&\estH_2(\omega) &=& \fH(\omega)\brs{1 + \frac{\Swvv(\omega)}{\Swyy(\omega)}} & \mc{2}{D}{\begin{tabular}{l}(\prope{biased})\\(\prope{over estimated})\end{tabular}}
  \end{array}}
  }
\end{proposition}
\begin{proof}
\begin{align*}
  \estH_1(\omega)
    &\eqd \frac{\Swxx[y'x](\omega)}{\Swxx(\omega)}
    && \text{by definition of $\estH_1$}
    && \text{\xref{def:H1}}
  \\&=    \frac{\Swyx(\omega)}{\Swxx(\omega)}
    && \text{by \prefp{prop:Swxyv}}
  \\&= \fH^\ast(\omega)
    && \text{by \prope{LTI} hypothesis (A)}
    && \text{and \prefp{cor:Swxy}}
  \\
  \estH_2(\omega)
    &\eqd \frac{\Swxx[y'y'](\omega)}{\Swxy(\omega)}
    && \text{by definition of $\estH_2$}
    && \text{\xref{def:H2}}
  \\&=    \frac{\Swyy(\omega) + \Swvv(\omega)}{\Swxy(\omega)}
    && \text{by \prefp{prop:Swxyv}}
  \\&= \frac{\Swyy(\omega)}{\Swxy(\omega)}
     + \frac{\Swvv(\omega)}{\Swxy(\omega)}
  \\&= \fH(\omega)
     + \frac{\Swvv(\omega)}{\frac{\Swyy(\omega)}{\fH(\omega)}}
    && \text{by \prefp{cor:Swxy}}
  \\&= \fH(\omega)\brs{1 + \frac{\Swvv(\omega)}{\Swyy(\omega)}}
\end{align*}
\end{proof}

%---------------------------------------
\begin{proposition}[Input noise only case]
\footnote{
  \citerpgc{shin2008}{294}{0470725648}{$H_1(f)=\ffrac{H(f)}{\brp{1+\ffrac{S_{n_xn_x}}{S_{xx}(f)}}}$ (9.72); $H_2(f)=H(f)$ (9.73)},
  \citePpc{mitchell1982}{277}{$\~H_1(f)=\ffrac{H_0(f)}{\brp{1+\ffrac{G_{nn}}{G_{uu}}}}$}
  }
\label{prop:estH2}
%---------------------------------------
Let $\opS$ be a \structe{system} with \fncte{impulse response} $\fh[n]$,
input $\fx[n]$, and output $\fy[n]$.
Let $\rvu[n]$ and $\rvv[n]$ be \fncte{noise sequence}.
Let $\rvx'[n]\eqd\rvx[n]+\rvu[n]$ and $\rvy'[n]\eqd\rvy[n]+\rvv[n]$.
\propbox{
  \brbr{\begin{array}{FMMD}
      (A).& The system $\fh$ i s        &\prope{LTI}          & and
    \\(B).& $\rvx[n]$ and $\rvy[n]$ are &\prope{WSS}          & and
    \\(C).& $\rvy[n]$ and $\rvu[n]$ are &                     &     
        \\& \mc{2}{M}{\quad              \prope{uncorrelated}}& and
    \\(D).& $\rvv[n]=0$                 
  \end{array}}
  \implies
  \brbl{\begin{array}{F>{\ds}rc>{\ds}lDD}
      (1).&\estH_1(\omega) &=& \frac{\fH(\omega)}{1 + \frac{\ds\Swuu(\omega)}{\ds\Swxx(\omega)}} & \begin{tabular}{l}(\prope{biased})\\(\prope{under estimated})\end{tabular} &and
    \\(2).&\estH_2(\omega) &=& \fH(\omega)                                                 & (\prope{unbiased}) &
  \end{array}}
  }
\end{proposition}
\begin{proof}
\begin{align*}
  \\
  \estH_1(\omega)
    &\eqd \frac{\Swxx[yx'](\omega)}{\Swxx[x'x'](\omega)}
    && \text{by definition of $\estH_1$}
    && \text{\xref{def:H1}}
  \\&= \frac{\Swyx(\omega) + \Swyu(\omega)}
            {\Swxx(\omega) + \Swuu(\omega)}
    && \text{by \prefp{prop:Swxyv}}
  \\&= \frac{\Swyx(\omega) + \cancelto{0}{\Swyu(\omega)}}
            {\Swxx(\omega) + \Swuu(\omega)}
    && \text{because $\rvy[n]$ and $\rvu[n]$ are \prope{uncorrelated}}
    && \text{(hypothesis (C))}
  \\&= \frac{\frac{\Swyx(\omega)}{\Swxx(\omega)}}
            {\frac{\Swxx(\omega)}{\Swxx(\omega)} + \frac{\Swuu(\omega)}{\Swxx(\omega)}}
   &&= \frac{\fH(\omega)}
            {1 + \frac{\Swuu(\omega)}{\Swxx(\omega)}}
  \\
  \estH_2(\omega)
    &\eqd \frac{\Swyy(\omega)}{\Swxx[x'y](\omega)}
    && \text{by definition of $\estH_2$}
    && \text{\xref{def:H2}}
  \\&= \frac{\Swyy(\omega)}{\Swxy(\omega) + \Swuy(\omega)}
    && \text{by \prefp{prop:Swxyv}}
  \\&= \frac{\Swyy(\omega)}{\Swxy(\omega) + \cancelto{0}{\Swuy(\omega)}}
    && \text{because $\rvy[n]$ and $\rvu[n]$ are \prope{uncorrelated}}
    && \text{(hypothesis (C))}
  \\&= \fH(\omega)
    && \text{by \prefp{cor:Swxy}}
    && \text{and \prope{LTI} hypothesis (A)}
\end{align*}
\end{proof}

%=======================================
\section{Beware of estimators}
%=======================================
Estimators yield, as the name implies, estimates.
These estimates in general contain some error.

%---------------------------------------
\begin{example}[The K=1 Welch estimate of coherence]
%---------------------------------------
Suppose we have two \prope{uncorrelated} stationary sequences $\rvx[n]$ and $\rvy[n]$. Then, there
CSD $\Sxy(\omega)$ should be $0$ because
\begin{align*}
  \Sxy(\omega)
    &\eqd \opDTFT\pE\Rxy(m)
  \\&=    \opDTFT\pE\brs{x[n]y[n+m]}
  \\&=    \opDTFT\brs{\pEx[n]} \brs{\pEy[n+m]}
  \\&=    \opDTFT\brs{0} \brs{0}
  \\&=    0
\end{align*}

This will give a coherence of $0$ also:
\[ C(\omega) = \frac{\Sxy}{\sqrt{\Sxx\Syy}} = 0\]

However, the Welch estimate with $K=1$ will yield
\begin{align*}
  \abs{C(\omega)}
    &= \abs{\frac{\ds\Sxy}{\sqrt{\ds\Sxx\Syy}}}
  \\&= \abs{\frac{\ds (\opFT x)(\opFT y)^\ast}{\sqrt{\ds\abs{\opFT x}^2\abs{\opFT y}^2}}}
  \\&= 1
\end{align*}

\end{example}



%=======================================
\section{Coherence}
%=======================================
%---------------------------------------
\begin{definition}
\footnote{
  \citerppgc{liang2015}{363}{365}{1498702341}{7.4.2 Coherence function},
  \citerpgc{ewins1986}{131}{0863800173}{$\gamma^2=\ffrac{H_1(\omega)}{H_2(\omega)}$ (3.8)}
  }
%---------------------------------------
Let $\opS$ be a \structe{system} with input $\rvx[n]$ and output $\rvy[n]$.
\defboxt{
  The \fnctd{coherence} function $\ds\Cxy \eqd \frac{\Swyx(\omega)}{\Swxx(\omega)}$
  }
\end{definition}
