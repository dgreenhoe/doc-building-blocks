%============================================================================
% Daniel J. Greenhoe
%============================================================================
%\chapter{Order and metric geometry preserving stochastic processing}\label{chp:sto}
%\chapter{Stochastic processing on weighted graphs}\label{chp:sto}
%\chapter[SP on weighted graphics]{Stochastic processing on weighted graphs}\label{chp:sto}
\chapter{Stochastic processing on weighted graphs}\label{chp:sto}
\qboxnpqt
  { Jules Henri Poincar\'e (1854-1912), physicist and mathematician
    \index{Poincar\'e, Jules Henri}
    \index{quotes!Poincar\'e, Jules Henri}
    \footnotemark
  }
  {../common/people/small/poincare.jpg}
  {Les math\'ematiciens n'\'etudient pas des objets, 
   mais des relations entre les objets; 
   il leur est donc indiff\'erent de remplacer ces objets par d'autres, 
   pourvu que les relations ne changent pas. 
   La mati\`ere ne leur importe pas, la forme seule les int\'eresse.}
  {Mathematicians do not study objects, but the relations between objects;
   to them it is a matter of indifference if these objects are replaced by others,
   provided that the relations do not change.
   Matter does not engage their attention,
   they are interested in form alone.}
  \citetblt{
    quote:       & \citerc{poincare_sah}{Chapter 2} \\
    translation: & \citerp{poincare_sah_eng}{20} \\
    %image:       & \url{http://en.wikipedia.org/wiki/Image:Poincare_jh.jpg}
    image:       & \url{http://www-groups.dcs.st-and.ac.uk/~history/PictDisplay/Poincare.html}
    }
%\qboxnps
%  {\href{http://en.wikipedia.org/wiki/Aristotle}{Aristotle}
%   \href{http://www-history.mcs.st-andrews.ac.uk/Timelines/TimelineA.html}{384 BC -- 322 BC},
%   \href{http://www-history.mcs.st-andrews.ac.uk/BirthplaceMaps/Places/Greece.html}{Greek philosopher}
%   \index{Aristotle}
%   \index{quotes!Aristotle}
%   \footnotemark
%  }
%  {../common/people/small/aristotle.jpg}
%  {\ldots those who assert that the mathematical sciences
%     say nothing of the beautiful or the good are in error.
%     For these sciences say and prove a great deal about them;
%     if they do not expressly mention them, but prove attributes
%     which are their results or definitions, it is not true that they tell
%     us nothing about them.
%     The chief forms of beauty are order and symmetry and definiteness,
%     which the mathematical sciences demonstrate in a special degree.}
%  \citetblt{
%    %quote:  \citerc{aristotle}{paragraphs 34--35?} \\
%    quote:  \citerc{aristotle_metaphysics}{Book XIII Part 3} \\
%    %        \url{http://en.wikiquote.org/wiki/Aristotle} \\
%    %image:  \url{http://en.wikipedia.org/wiki/Aristotle}
%    image:  \url{http://upload.wikimedia.org/wikipedia/commons/9/98/Sanzio_01_Plato_Aristotle.jpg}
%    }
 %\input{sto/sto_intro.tex}
 %%============================================================================
% Daniel J. Greenhoe
% XeLaTeX file
% stochastic systems
%============================================================================

%=======================================
\section{Ordered metric spaces}
%=======================================
%=======================================
\subsection{Definitions}
%=======================================
%---------------------------------------
\begin{definition}
\label{def:oms}
\label{def:oqms}
%---------------------------------------
\defboxp{
A triple $\omsG\eqd\omsD$ is an \structd{ordered quasi-metric space}
if $\opair{\setX}{\metricn}$ is a \structe{quasi-metric space} \xref{def:qmetric}
and $\opair{\setX}{\orel}$ is an \structe{ordered set} \xref{def:oset}.
\\\indentx\begin{tabular}{ll}
      $\omsG$ is an \structd{ordered metric space}         &if $\metricn$ is a \structe{metric} \xref{def:metric}.
    \\$\omsG$ is an \structd{unordered quasi-metric space} &if $\orel=\emptyset$.
    \\$\omsG$ is an \structd{unordered metric space}       &if $\metricn$ is a \structe{metric} and $\orel=\emptyset$.
\end{tabular}
}
\end{definition}

%---------------------------------------
\begin{remark}
%---------------------------------------
Note that the four structures defined in \pref{def:oms} are not mutually exclusive.\\
For example, by \pref{def:oms},
\\{$\begin{array}{lclcl}
  \{\text{\structe{unordered metric space}}\}       &\subseteq&
  \{\text{\structe{unordered quasi-metric space}}\} &\subseteq&
  \{\text{\structe{ordered quasi-metric space}}\}  
  \\
  \{\text{\structe{unordered metric space}}\}       &\subseteq&
  \{\text{\structe{ordered   metric space}}\}       &\subseteq& 
  \{\text{\structe{ordered quasi-metric space}}\}.  
\end{array}$}
\end{remark}

%---------------------------------------
\begin{remark}
\label{rem:qmetric}
%---------------------------------------
The use of the \fncte{quasi-metric} rather than exclusive use of the more restrictive \fncte{metric} in \pref{def:oms} is motivated
by state machines, where metrics measuring distances between states are in some cases by nature \prope{non-symmetric}.
One such example is the \structe{linear congruential pseudo-random number generator} \xref{ex:lcg7x1m9_dgraph}.
%See \prefpp{ex:lcg7x1m9_xyz}--\prefpp{ex:lcg7x1m9_dgraph} for demonstration.
\end{remark}

\begin{figure}
\centering%
\gsize%
{\footnotesize\begin{tabular}{ccc}
   \includegraphics{sto/graphics/Zline.pdf}%   \psset{unit=3mm}%============================================================================
% Daniel J. Greenhoe
% LaTeX file
% ordered metric space
% real line (R, |.|, <=)
%============================================================================
{%\psset{unit=0.5\psunit}%
\begin{pspicture}(-0.5,-3)(0.5,3)%
  %---------------------------------
  % options
  %---------------------------------
  \psset{%
    %radius=1.25ex,
    %labelsep=2.5mm,
    linecolor=blue,%
    }%
    %\psline{<->}(0,-4)(0,4)%
    %\pnode(0,4){L8}%
    \pnode(0,3){L7}%
    \Cnode(0,2){L6}%
    \Cnode(0,1){L5}%
    %\Cnode[fillstyle=solid,fillcolor=snode](0,0){L4}%
    \Cnode(0,0){L4}%
    \Cnode(0,-1){L3}%
    \Cnode(0,-2){L2}%
    \pnode(0,-3){L1}%
    %\pnode(0,-4){L0}%
   %\Cnode[fillstyle=solid,linecolor=snode,fillcolor=snode,radius=0.5ex](0,0){LC}%
  %\ncline[linestyle=dotted]{L7}{L8}%
  \ncline[linestyle=dotted]{L6}{L7}%
  \ncline{L5}{L6}%
  \ncline{L4}{L5}%
  \ncline{L3}{L4}%
  \ncline{L2}{L3}%
  \ncline[linestyle=dotted]{L1}{L2}%
  %\ncline[linestyle=dotted]{L1}{L0}%
  %\rput(L7){\psline[linewidth=1pt](-0.1,0)(0.1,0)}%
  %\rput(L6){\psline[linewidth=1pt](-0.1,0)(0.1,0)}%
  %\rput(L5){\psline[linewidth=1pt](-0.1,0)(0.1,0)}%
  %\rput(L4){\psline[linewidth=1pt](-0.1,0)(0.1,0)}%
  %\rput(L3){\psline[linewidth=1pt](-0.1,0)(0.1,0)}%
  %\rput(L2){\psline[linewidth=1pt](-0.1,0)(0.1,0)}%
  %\rput(L1){\psline[linewidth=1pt](-0.1,0)(0.1,0)}%
  %
  %\rput(L7){$3$}%
  \uput[180](L6){$2$}%
  \uput[180](L5){$1$}%
  \uput[180](L4){$0$}%
  \uput[180](L3){$-1$}%
  \uput[180](L2){$-2$}%
  %\rput(L1){$-3$}%
\end{pspicture}
}%%   
  &\includegraphics{sto/graphics/realline.pdf}%\psset{unit=3mm}%============================================================================
% Daniel J. Greenhoe
% LaTeX file
% ordered metric space
% real line (R, |.|, <=)
%============================================================================
{%\psset{unit=0.5\psunit}%
\begin{pspicture}(-0.5,-3)(0.5,3)%
  %---------------------------------
  % options
  %---------------------------------
  \psset{%
    %radius=1.25ex,
    %labelsep=2.5mm,
    linecolor=blue,%
    }%
    \psline{<->}(0,-3)(0,3)%
    %\pnode(0,3){L7}%
    \pnode(0,2){L6}%
    \pnode(0,1){L5}%
    \pnode(0,0){L4}%
    \pnode(0,-1){L3}%
    \pnode(0,-2){L2}%
    %\pnode(0,-3){L1}%
   %\Cnode[fillstyle=solid,linecolor=snode,fillcolor=snode,radius=0.5ex](0,0){L4}%
  %\ncline{L5}{L6}%
  %\ncline{L4}{L5}%
  %\ncline{L3}{L4}%
  %\ncline{L2}{L3}%
  %\ncline{L1}{L2}%
  %\rput(L7){\psline[linewidth=1pt](-0.1,0)(0.1,0)}%
  \rput(L6){\psline[linewidth=1pt](-0.2,0)(0.2,0)}%
  \rput(L5){\psline[linewidth=1pt](-0.2,0)(0.2,0)}%
  \rput(L4){\psline[linewidth=1pt](-0.2,0)(0.2,0)}%
  \rput(L3){\psline[linewidth=1pt](-0.2,0)(0.2,0)}%
  \rput(L2){\psline[linewidth=1pt](-0.2,0)(0.2,0)}%
  %\rput(L1){\psline[linewidth=1pt](-0.1,0)(0.1,0)}%
  %
  %\uput[0](L7){$3$}%
  \uput[180](L6){$2$}%
  \uput[180](L5){$1$}%
  \uput[180](L4){$0$}%
  \uput[180](L3){$-1$}%
  \uput[180](L2){$-2$}%
  %\uput[0](L1){$-3$}%
\end{pspicture}
}%%
  &\includegraphics{sto/graphics/Cplane.pdf}%  \psset{unit=3mm}%============================================================================
% Daniel J. Greenhoe
% LaTeX file
% ordered metric space
% real line (R, |.|, <=)
%============================================================================
{%\psset{unit=0.5\psunit}%
\begin{pspicture}(-4,-3)(4.5,3)%
  %---------------------------------
  % options
  %---------------------------------
  \psset{%
    %radius=1.25ex,
    %labelsep=2.5mm,
    linecolor=blue,%
    }%
    \psline{<->}(0,-3)(0,3)%
    \pnode(0, 3){I7}%
    \pnode(0, 2){I6}%
    \pnode(0, 1){I5}%
    \pnode(0, 0){I4}%
    \pnode(0,-1){I3}%
    \pnode(0,-2){I2}%
    \pnode(0,-3){I1}%
    %
    \psline{<->}(-3,0)(3,0)%
    \pnode( 3,0){R7}%
    \pnode( 2,0){R6}%
    \pnode( 1,0){R5}%
    \pnode( 0,0){R4}%
    \pnode(-1,0){R3}%
    \pnode(-2,0){R2}%
    \pnode(-3,0){R1}%
   %\Cnode[fillstyle=solid,linecolor=snode,fillcolor=snode,radius=0.5ex](0,0){LC}%
  %\ncline{L5}{L6}%
  %\ncline{L4}{L5}%
  %\ncline{L3}{L4}%
  %\ncline{L2}{L3}%
  %\ncline{L1}{L2}%
  %\rput(R7){\psline[linewidth=1pt](0,-0.1)(0,0.1)}%
  \rput(R6){\psline[linewidth=1pt](0,-0.1)(0,0.1)}%
  \rput(R5){\psline[linewidth=1pt](0,-0.1)(0,0.1)}%
  %\rput(R4){\psline[linewidth=1pt](0,-0.1)(0,0.1)}%
  \rput(R3){\psline[linewidth=1pt](0,-0.1)(0,0.1)}%
  \rput(R2){\psline[linewidth=1pt](0,-0.1)(0,0.1)}%
  %\rput(R1){\psline[linewidth=1pt](0,-0.1)(0,0.1)}%
  %
  %\rput(I7){\psline[linewidth=1pt](-0.1,0)(0.1,0)}%
  \rput(I6){\psline[linewidth=1pt](-0.1,0)(0.1,0)}%
  \rput(I5){\psline[linewidth=1pt](-0.1,0)(0.1,0)}%
  %\rput(I4){\psline[linewidth=1pt](-0.1,0)(0.1,0)}%
  \rput(I3){\psline[linewidth=1pt](-0.1,0)(0.1,0)}%
  \rput(I2){\psline[linewidth=1pt](-0.1,0)(0.1,0)}%
  %\rput(I1){\psline[linewidth=1pt](-0.1,0)(0.1,0)}%
  %
  %\uput[-90](R7){$3$}%
  %\uput[-90](R6){$2$}%
  %\uput[-90](R5){$1$}%
  %\uput[-90](R4){$0$}%
  %\uput[-90](R3){$-1$}%
  %\uput[-90](R2){$-2$}%
  %\uput[-90](R1){$-3$}%
  %
  %\uput[180](I7){$3$}%
  %\uput[180](I6){$2$}%
  %\uput[180](I5){$1$}%
  %\uput[180](I4){$0$}%
  %\uput[180](I3){$-1$}%
  %\uput[180](I2){$-2$}%
  %\uput[180](I1){$-3$}%
  %
  \uput[0](0,2.5){$\Im$}%
  \uput[-90](2.5,0){$\Re$}%
\end{pspicture}
}%%  
  \\{\scs(A)} \structe{integer line} \xrefr{ex:Zline}
   &{\scs(B)} \structe{real line} \xrefr{ex:Rline}
   &{\scs(C)} \structe{complex plane} \xrefr{ex:Cplane}
 \\ linear ordered metric space
   &linear ordered metric space
   &ordered/unordered metric space
  \\
    %%============================================================================
% Daniel J. Greenhoe
% LaTeX file
% spinner 6 mapping to linearly ordered L6
%============================================================================
{%\psset{unit=0.5\psunit}%
\begin{pspicture}(-1.5,-1.5)(1.5,1.5)%
  %---------------------------------
  % options
  %---------------------------------
  \psset{%
    linecolor=blue,%
    radius=1.25ex,
    labelsep=2.5mm,
    }%
  %---------------------------------
  % spinner graph
  %---------------------------------
  \rput(0,0){%\psset{unit=2\psunit}%
    \Cnode(-0.8660,-0.5){D6}%
    \Cnode(-0.8660,0.5){D5}%
    \Cnode(0,1){D4}%
    \Cnode(0.8660,0.5){D3}%
    \Cnode(0.8660,-0.5){D2}%
    \Cnode(0,-1){D1}%
    }
  \rput[-150](D6){$5$}%
  \rput[ 150](D5){$4$}%
  \rput[  90](D4){$3$}%
  \rput[  30](D3){$2$}%
  \rput[   0](D2){$1$}%
  \rput[ -90](D1){$0$}%
  %
  \ncline{D6}{D1}%
  \ncline{D5}{D6}%
  \ncline{D4}{D5}%
  \ncline{D3}{D4}%
  \ncline{D2}{D3}%
  \ncline{D1}{D2}%
  %
  %\uput[ 210](D6){$\frac{1}{6}$}
  %\uput[ 150](D5){$\frac{1}{6}$}
  %\uput[  22](D4){$\frac{1}{6}$}
  %\uput[  30](D3){$\frac{1}{6}$}
  %\uput[ -30](D2){$\frac{1}{6}$}
  %\uput[ -22](D1){$\frac{1}{6}$}
  %
  %\uput[ 210](D6){${\scy\psp(\circSix)=}\frac{1}{6}$}
  %\uput[ 150](D5){${\scy\psp(\circFive)=}\frac{1}{6}$}
  %\uput[  22](D4){${\scy\psp(\circFour)=}\frac{1}{6}$}
  %\uput[  30](D3){${\scy\psp(\circThree)=}\frac{1}{6}$}
  %\uput[ -30](D2){${\scy\psp(\circTwo)=}\frac{1}{6}$}
  %\uput[-22](D1){${\scy\psp(\circOne)=}\frac{1}{6}$}
\end{pspicture}
}%%
    \includegraphics{sto/graphics/ring6.pdf}%
   %&%============================================================================
% Daniel J. Greenhoe
% LaTeX file
% discrete metric real dice mapping to linearly ordered L6
%============================================================================
\begin{pspicture}(-1.4,-1.4)(1.4,1.4)%
  %---------------------------------
  % options
  %---------------------------------
  \psset{%
    linecolor=blue,%
    radius=1.25ex,
    labelsep=2.5mm,
    }%
  %---------------------------------
  % dice graph
  %---------------------------------
  \rput(0,0){%\psset{unit=2\psunit}%
    \Cnode(-0.8660,-0.5){D4}%
    \Cnode(-0.8660,0.5){D5}%
    \Cnode(0,1){D6}%
    \Cnode(0.8660,0.5){D3}%
    \Cnode(0.8660,-0.5){D2}%
    \Cnode(0,-1){D1}%
    }
  \rput(D6){$5$}%
  \rput(D5){$4$}%
  \rput(D4){$3$}%
  \rput(D3){$2$}%
  \rput(D2){$1$}%
  \rput(D1){$0$}%
  %
  \ncline{D5}{D6}%
  \ncline{D4}{D5}\ncline{D4}{D6}%
  \ncline{D3}{D5}\ncline{D3}{D6}%
  \ncline{D2}{D3}\ncline{D2}{D4}\ncline{D2}{D6}%
  \ncline{D1}{D2}\ncline{D1}{D3}\ncline{D1}{D4}\ncline{D1}{D5}%
  \ncline{D3}{D4}%
  \ncline{D2}{D5}%
  \ncline{D1}{D6}%
  %
  %\uput[ 158](D6){$\frac{1}{6}$}
  %\uput[ 150](D5){$\frac{1}{6}$}
  %\uput[ 210](D4){$\frac{1}{6}$}
  %\uput[  22](D3){$\frac{1}{6}$}
  %\uput[ -45](D2){$\frac{1}{6}$}
  %\uput[-158](D1){$\frac{1}{6}$}
\end{pspicture}%
   &\includegraphics{sto/graphics/discretemetric6.pdf}%
   %&%============================================================================
% Daniel J. Greenhoe
% LaTeX file
% Boolean L_2^3 for integer divides relation
%============================================================================
\begin{pspicture}(-1.5,-1.5)(1.5,1.5)%
  %---------------------------------
  % options
  %---------------------------------
  \psset{%
    linecolor=blue,%
    radius=1.25ex,
    labelsep=2.5mm,
    }%
  %---------------------------------
  % spinner graph
  %---------------------------------
  \rput(0,0){%\psset{unit=2\psunit}%
    \Cnode(0,1){D30}%
    \Cnode(0.8660,0.5){D15}%
    \Cnode(0,0.333){D10}%
    \Cnode(-0.8660,0.5){D6}%
    \Cnode(0.8660,-0.5){D5}%
    \Cnode(0,-0.333){D3}%
    \Cnode(-0.8660,-0.5){D2}%
    \Cnode(0,-1){D1}%
    }
  \rput(D30){$30$}%
  \rput(D15){$15$}%
  \rput(D10){$10$}%
  \rput(D6){$6$}%
  \rput(D5){$5$}%
  \rput(D3){$3$}%
  \rput(D2){$2$}%
  \rput(D1){$1$}%
  %
  \ncline{D30}{D6}\ncline{D30}{D10}\ncline{D30}{D15}%
  \ncline{D5}{D10}\ncline{D5}{D15}%
  \ncline{D3}{D6}\ncline{D3}{D15}%
  \ncline{D2}{D6}\ncline{D2}{D10}%
  \ncline{D1}{D2}\ncline{D1}{D3}\ncline{D1}{D5}%
  %
  %\uput[ 210](D6){$\frac{1}{6}$}
  %\uput[ 150](D5){$\frac{1}{6}$}
  %\uput[  22](D4){$\frac{1}{6}$}
  %\uput[  30](D3){$\frac{1}{6}$}
  %\uput[ -30](D2){$\frac{1}{6}$}
  %\uput[ -22](D1){$\frac{1}{6}$}
\end{pspicture}%%
  %&%============================================================================
% Daniel J. Greenhoe
% LaTeX file
% linear congruential (LCG) pseudo-random number generator (PRNG) mappings
% x_{n+1} = (7x_n+5)mod 9
% y_{n+1} = (y_n+2)mod 5
%============================================================================
\begin{pspicture}(-1.2,-1.2)(1.2,1.2)%
  %---------------------------------
  % options
  %---------------------------------
  \psset{%
    radius=1.25ex,
    labelsep=2.5mm,
    linecolor=blue,%
    }%
  \rput{288}(0,0){\rput(1,0){\Cnode(0,0){S4}}}%
  \rput{216}(0,0){\rput(1,0){\Cnode(0,0){S2}}}%
  \rput{144}(0,0){\rput(1,0){\Cnode(0,0){S0}}}%
  \rput{ 72}(0,0){\rput(1,0){\Cnode[fillstyle=solid,fillcolor=snode](0,0){S3}}}%
  \rput{  0}(0,0){\rput(1,0){\Cnode(0,0){S1}}}%
  %
  \rput(S4){$4$}%
  \rput(S3){$3$}%
  \rput(S2){$2$}%
  \rput(S1){$1$}%
  \rput(S0){$0$}%
  %
  \ncline{->}{S4}{S1}\ncline{->}{S2}{S4}\ncline{->}{S3}{S4}%
  \ncline{->}{S2}{S4}%
  \ncline{->}{S0}{S2}%
  \ncline{->}{S3}{S0}%
  \ncline{->}{S1}{S3}%
  %
  %\uput[288](S4){$\frac{3}{9}$}
  %\uput[ 72](S3){$\frac{2}{9}$}
  %\uput[216](S2){$\frac{1}{9}$}
  %\uput[  0](S1){$\frac{1}{9}$}
  %\uput[144](S0){$\frac{2}{9}$}
  %\rput(0,0){$\ocsG$}%
\end{pspicture}%%
  &\includegraphics{sto/graphics/oms_wring5shortd.pdf}%
  \\{\scs(D)} \structe{6 element ring} \xrefr{ex:ring6}
   &{\scs(E)} \structe{6 element discrete metric} \xrefr{ex:discretemetric6}
  %&{\scs(F)} \structe{integer divides order relation} \xref{ex:ocs_532}
   &{\scs(F)} \structe{directed PRNG state machine} \xrefr{ex:ocs_prng}
 \\ unordered metric space
   &unordered metric space
   &unordered quasi-metric space
\end{tabular}}
\caption{Examples of \structe{ordered quasi-metric space}s \xref{def:oqms}
         %with \colorbox{snode}{shaded} \structe{center}s \xref{def:gcen} 
         \label{fig:oms}}   %{def:ocscen}
\end{figure}

%---------------------------------------
\begin{remark}
\label{rem:realtop}
%---------------------------------------
This text makes extensive reference to the \structe{real line} (next definition).
% and the \structe{integer line} (next two definitions).
There are several ways to define the real line.
In particular, there are many possible ordering relations on $\R$
and several possible topologies on $\R$.\footnote{
  \citerpgc{adams2008}{31}{0131848690}{"six topologies on the real line"},
  \citerppgc{salzmann2007}{64}{70}{0521865166}{Weird topologies on the real line},
  \citerpgc{murdeshwar1990}{53}{8122402461}{``often used topologies on the real line"},
  \citerppgc{joshi1983}{85}{91}{0852264445}{\textsection4.2 Examples of Topological Spaces}
  }
In fact, order and topology are closely related in that 
an order relation $\orel$ \xref{def:orel} on a set always induces a topology 
(called the \structe{order topology} / \structe{interval topology}\footnote{
  \citerpgc{salzmann2007}{23}{0521865166}{3.2 Topology induced by an ordering}
  \citerpgc{willard1970}{43}{0486434796}{6D. Ordered Spaces},
  \citerpgc{steen1978}{66}{0486319296}{39. Order Topology}
  }); 
and in the case of the real line, a topology induces an order structure up to the order relation's 
\rele{dual} \xref{def:oreld}.\footnote{
  \citerpcu{hocking1961}{52}{2--5 The interval and the circle}{https://archive.org/details/Topology_972},
  \citerppgc{salzmann2007}{69}{70}{0521865166}{5.75 Note: Ordering and topology on $\R$, see also 5.10 Theorem page 36}
  }
This text uses a fairly standard structure, as defined next.
% using standard linear ordering and the \fncte{usual metric}.
%The \structe{real line}, as used in this text, is defined next and illustrated in \prefp{fig:oms} (B).
\end{remark}

%---------------------------------------
%\begin{minipage}{\tw-13mm}
\begin{definition}
\label{def:Rline}
\label{def:rline}
%---------------------------------------
The triple $\omsR$ is the \structd{real line} 
%if $\omsR$ is an \structe{ordered metric space} \xref{def:oms},
if $\R$ is the \structe{set of real numbers} \xref{def:R}, 
$\metric{x}{y}\eqd\abs{x-y}$ is the \fncte{usual metric} on $\R$ \xref{def:d_usual},
and $\orel$ is the standard \structe{linear order relation} \xref{def:chain} on $\R$.
\end{definition}

%---------------------------------------
%\begin{minipage}{\tw-13mm}
\begin{definition}
\label{def:Zline}
\label{def:zline}
%---------------------------------------
The triple $\omsZ$ is the \structd{integer line} 
if $\Z$ is the \structe{set of integers} \xref{def:Z},
$\metric{m}{n}\eqd\abs{m-n}$ is the \fncte{usual metric} \xref{def:d_usual} on $\R$ restricted to $\Z$,
and $\orel$ is the standard \structe{linear order relation} on $\Z$
as induced by \structe{Peano's Axioms}.\footnote{
  \citerpg{landau1966}{2}{082182693X},
  \citerpg{halmos1960}{46}{0387900926},
  \citerpg{thurston1956}{51}{0486458067},
  \citer{peano1889},
  \citerpg{peano1889e}{94}{158348597X},
  \citeP{dedekind1888},
  \citePp{dedekind1888e}{67},
  \citerppgc{cori2001}{8}{15}{0198500513}{recursion theory}
  }
\end{definition}
%\end{minipage}\hfill%
%\begin{minipage}{10mm}
%  \gsize%
%  \centering%
%  \psset{unit=5mm}%
%  %============================================================================
% Daniel J. Greenhoe
% LaTeX file
% ordered metric space
% real line (R, |.|, <=)
%============================================================================
{%\psset{unit=0.5\psunit}%
\begin{pspicture}(-0.5,-3)(0.5,3)%
  %---------------------------------
  % options
  %---------------------------------
  \psset{%
    %radius=1.25ex,
    %labelsep=2.5mm,
    linecolor=blue,%
    }%
    %\psline{<->}(0,-4)(0,4)%
    %\pnode(0,4){L8}%
    \pnode(0,3){L7}%
    \Cnode(0,2){L6}%
    \Cnode(0,1){L5}%
    %\Cnode[fillstyle=solid,fillcolor=snode](0,0){L4}%
    \Cnode(0,0){L4}%
    \Cnode(0,-1){L3}%
    \Cnode(0,-2){L2}%
    \pnode(0,-3){L1}%
    %\pnode(0,-4){L0}%
   %\Cnode[fillstyle=solid,linecolor=snode,fillcolor=snode,radius=0.5ex](0,0){LC}%
  %\ncline[linestyle=dotted]{L7}{L8}%
  \ncline[linestyle=dotted]{L6}{L7}%
  \ncline{L5}{L6}%
  \ncline{L4}{L5}%
  \ncline{L3}{L4}%
  \ncline{L2}{L3}%
  \ncline[linestyle=dotted]{L1}{L2}%
  %\ncline[linestyle=dotted]{L1}{L0}%
  %\rput(L7){\psline[linewidth=1pt](-0.1,0)(0.1,0)}%
  %\rput(L6){\psline[linewidth=1pt](-0.1,0)(0.1,0)}%
  %\rput(L5){\psline[linewidth=1pt](-0.1,0)(0.1,0)}%
  %\rput(L4){\psline[linewidth=1pt](-0.1,0)(0.1,0)}%
  %\rput(L3){\psline[linewidth=1pt](-0.1,0)(0.1,0)}%
  %\rput(L2){\psline[linewidth=1pt](-0.1,0)(0.1,0)}%
  %\rput(L1){\psline[linewidth=1pt](-0.1,0)(0.1,0)}%
  %
  %\rput(L7){$3$}%
  \uput[180](L6){$2$}%
  \uput[180](L5){$1$}%
  \uput[180](L4){$0$}%
  \uput[180](L3){$-1$}%
  \uput[180](L2){$-2$}%
  %\rput(L1){$-3$}%
\end{pspicture}
}%%
%\end{minipage}


%=======================================
\subsection{Examples}
%=======================================
%---------------------------------------
\begin{example}
\label{ex:Zline}
%---------------------------------------
The \structe{integer line} \xref{def:Zline} is an \structe{ordered metric space} \xref{def:oms},
and is illustrated in \prefp{fig:oms} (A).
\end{example}

%---------------------------------------
\begin{example}
\label{ex:Rline}
%---------------------------------------
The \structe{real line} \xref{def:Rline} is an \structe{ordered metric space} \xref{def:oms},
and is illustrated in \prefp{fig:oms} (B).
\end{example}

%---------------------------------------
%\begin{minipage}{\tw-48mm}
\begin{example}
\label{ex:Cplane}
%---------------------------------------
The \structd{complex plane} $\otriple{\C}{\absn}{\orel}$ is an \structe{ordered metric space} \xref{def:oms}
where $\C\eqd\R^2$ is the \structe{set of complex numbers}, 
$\metric{x}{y}\eqd\abs{x-y}\eqd\sqrt{\Re x-\Re y)^2+(\Im x-\Im y)^2}$,
$\Re x\eqd\Re\opair{a}{b}\eqd a$ ${\scy\forall\opair{a}{b}\in\C}$ ($\Re x$ is the \vald{real part} of $x$),
$\Im x\eqd\Im\opair{a}{b}\eqd b$ ${\scy\forall\opair{a}{b}\in\C}$ ($\Im x$ is the \vald{imaginary part} of $x$),
and $\orel$ is any \rele{order relation} defined on $\C$.
Possible order relations include the \rele{coordinatewise order relation} \xref{ex:order_coordinatewise},
the \rele{lexicographical order relation} \xref{ex:order_lex},
and $\orel=\emptyset$ (in which case the \structe{complex plane} is \prope{unordered}).
The \structe{complex plane} is illustrated in \prefp{fig:oms} (C).
\end{example}
%\end{minipage}\hfill%
%\begin{minipage}{43mm}
%  \gsize%
%  \centering%
%  \psset{unit=3mm}%
%  %============================================================================
% Daniel J. Greenhoe
% LaTeX file
% ordered metric space
% real line (R, |.|, <=)
%============================================================================
{%\psset{unit=0.5\psunit}%
\begin{pspicture}(-4,-3)(4.5,3)%
  %---------------------------------
  % options
  %---------------------------------
  \psset{%
    %radius=1.25ex,
    %labelsep=2.5mm,
    linecolor=blue,%
    }%
    \psline{<->}(0,-3)(0,3)%
    \pnode(0, 3){I7}%
    \pnode(0, 2){I6}%
    \pnode(0, 1){I5}%
    \pnode(0, 0){I4}%
    \pnode(0,-1){I3}%
    \pnode(0,-2){I2}%
    \pnode(0,-3){I1}%
    %
    \psline{<->}(-3,0)(3,0)%
    \pnode( 3,0){R7}%
    \pnode( 2,0){R6}%
    \pnode( 1,0){R5}%
    \pnode( 0,0){R4}%
    \pnode(-1,0){R3}%
    \pnode(-2,0){R2}%
    \pnode(-3,0){R1}%
   %\Cnode[fillstyle=solid,linecolor=snode,fillcolor=snode,radius=0.5ex](0,0){LC}%
  %\ncline{L5}{L6}%
  %\ncline{L4}{L5}%
  %\ncline{L3}{L4}%
  %\ncline{L2}{L3}%
  %\ncline{L1}{L2}%
  %\rput(R7){\psline[linewidth=1pt](0,-0.1)(0,0.1)}%
  \rput(R6){\psline[linewidth=1pt](0,-0.1)(0,0.1)}%
  \rput(R5){\psline[linewidth=1pt](0,-0.1)(0,0.1)}%
  %\rput(R4){\psline[linewidth=1pt](0,-0.1)(0,0.1)}%
  \rput(R3){\psline[linewidth=1pt](0,-0.1)(0,0.1)}%
  \rput(R2){\psline[linewidth=1pt](0,-0.1)(0,0.1)}%
  %\rput(R1){\psline[linewidth=1pt](0,-0.1)(0,0.1)}%
  %
  %\rput(I7){\psline[linewidth=1pt](-0.1,0)(0.1,0)}%
  \rput(I6){\psline[linewidth=1pt](-0.1,0)(0.1,0)}%
  \rput(I5){\psline[linewidth=1pt](-0.1,0)(0.1,0)}%
  %\rput(I4){\psline[linewidth=1pt](-0.1,0)(0.1,0)}%
  \rput(I3){\psline[linewidth=1pt](-0.1,0)(0.1,0)}%
  \rput(I2){\psline[linewidth=1pt](-0.1,0)(0.1,0)}%
  %\rput(I1){\psline[linewidth=1pt](-0.1,0)(0.1,0)}%
  %
  %\uput[-90](R7){$3$}%
  %\uput[-90](R6){$2$}%
  %\uput[-90](R5){$1$}%
  %\uput[-90](R4){$0$}%
  %\uput[-90](R3){$-1$}%
  %\uput[-90](R2){$-2$}%
  %\uput[-90](R1){$-3$}%
  %
  %\uput[180](I7){$3$}%
  %\uput[180](I6){$2$}%
  %\uput[180](I5){$1$}%
  %\uput[180](I4){$0$}%
  %\uput[180](I3){$-1$}%
  %\uput[180](I2){$-2$}%
  %\uput[180](I1){$-3$}%
  %
  \uput[0](0,2.5){$\Im$}%
  \uput[-90](2.5,0){$\Re$}%
\end{pspicture}
}%%
%\end{minipage}

%---------------------------------------
%\begin{minipage}{\tw-48mm}
\begin{example}
\label{ex:ring6}
%---------------------------------------
A \structd{6 element ring} $\otriple{\setn{0,1,2,3,4,5}}{\metricn}{\emptyset}$ is an \structe{unordered metric space} \xref{def:oms}
where the metric $\metricn$ is defined on a ring as illustrated in \prefp{fig:oms} (D), with each line segment representing a 
distance of 1.
\end{example}
%\end{minipage}\hfill%
%\begin{minipage}{43mm}
%  \gsize%
%  \centering%
%  \psset{unit=7.5mm}%
%  %============================================================================
% Daniel J. Greenhoe
% LaTeX file
% spinner 6 mapping to linearly ordered L6
%============================================================================
{%\psset{unit=0.5\psunit}%
\begin{pspicture}(-1.5,-1.5)(1.5,1.5)%
  %---------------------------------
  % options
  %---------------------------------
  \psset{%
    linecolor=blue,%
    radius=1.25ex,
    labelsep=2.5mm,
    }%
  %---------------------------------
  % spinner graph
  %---------------------------------
  \rput(0,0){%\psset{unit=2\psunit}%
    \Cnode(-0.8660,-0.5){D6}%
    \Cnode(-0.8660,0.5){D5}%
    \Cnode(0,1){D4}%
    \Cnode(0.8660,0.5){D3}%
    \Cnode(0.8660,-0.5){D2}%
    \Cnode(0,-1){D1}%
    }
  \rput[-150](D6){$5$}%
  \rput[ 150](D5){$4$}%
  \rput[  90](D4){$3$}%
  \rput[  30](D3){$2$}%
  \rput[   0](D2){$1$}%
  \rput[ -90](D1){$0$}%
  %
  \ncline{D6}{D1}%
  \ncline{D5}{D6}%
  \ncline{D4}{D5}%
  \ncline{D3}{D4}%
  \ncline{D2}{D3}%
  \ncline{D1}{D2}%
  %
  %\uput[ 210](D6){$\frac{1}{6}$}
  %\uput[ 150](D5){$\frac{1}{6}$}
  %\uput[  22](D4){$\frac{1}{6}$}
  %\uput[  30](D3){$\frac{1}{6}$}
  %\uput[ -30](D2){$\frac{1}{6}$}
  %\uput[ -22](D1){$\frac{1}{6}$}
  %
  %\uput[ 210](D6){${\scy\psp(\circSix)=}\frac{1}{6}$}
  %\uput[ 150](D5){${\scy\psp(\circFive)=}\frac{1}{6}$}
  %\uput[  22](D4){${\scy\psp(\circFour)=}\frac{1}{6}$}
  %\uput[  30](D3){${\scy\psp(\circThree)=}\frac{1}{6}$}
  %\uput[ -30](D2){${\scy\psp(\circTwo)=}\frac{1}{6}$}
  %\uput[-22](D1){${\scy\psp(\circOne)=}\frac{1}{6}$}
\end{pspicture}
}%%
%\end{minipage}

%---------------------------------------
%\begin{minipage}{\tw-48mm}
\begin{example}
\label{ex:discretemetric6}
%---------------------------------------
A \structd{6 element discrete metric} $\otriple{\setn{0,1,2,3,4,5}}{\metricn}{\emptyset}$ is an \structe{unordered metric space} \xref{def:oms}
where the metric $\metricn$ is the \fncte{discrete metric} \xref{def:dmetric}.
This structure is illustrated in \prefp{fig:oms} (E).
\end{example}
%\end{minipage}\hfill%
%\begin{minipage}{43mm}
%  \gsize%
%  \centering%
%  \psset{unit=7.5mm}%
%  %============================================================================
% Daniel J. Greenhoe
% LaTeX file
% discrete metric real dice mapping to linearly ordered L6
%============================================================================
\begin{pspicture}(-1.4,-1.4)(1.4,1.4)%
  %---------------------------------
  % options
  %---------------------------------
  \psset{%
    linecolor=blue,%
    radius=1.25ex,
    labelsep=2.5mm,
    }%
  %---------------------------------
  % dice graph
  %---------------------------------
  \rput(0,0){%\psset{unit=2\psunit}%
    \Cnode(-0.8660,-0.5){D4}%
    \Cnode(-0.8660,0.5){D5}%
    \Cnode(0,1){D6}%
    \Cnode(0.8660,0.5){D3}%
    \Cnode(0.8660,-0.5){D2}%
    \Cnode(0,-1){D1}%
    }
  \rput(D6){$5$}%
  \rput(D5){$4$}%
  \rput(D4){$3$}%
  \rput(D3){$2$}%
  \rput(D2){$1$}%
  \rput(D1){$0$}%
  %
  \ncline{D5}{D6}%
  \ncline{D4}{D5}\ncline{D4}{D6}%
  \ncline{D3}{D5}\ncline{D3}{D6}%
  \ncline{D2}{D3}\ncline{D2}{D4}\ncline{D2}{D6}%
  \ncline{D1}{D2}\ncline{D1}{D3}\ncline{D1}{D4}\ncline{D1}{D5}%
  \ncline{D3}{D4}%
  \ncline{D2}{D5}%
  \ncline{D1}{D6}%
  %
  %\uput[ 158](D6){$\frac{1}{6}$}
  %\uput[ 150](D5){$\frac{1}{6}$}
  %\uput[ 210](D4){$\frac{1}{6}$}
  %\uput[  22](D3){$\frac{1}{6}$}
  %\uput[ -45](D2){$\frac{1}{6}$}
  %\uput[-158](D1){$\frac{1}{6}$}
\end{pspicture}%
%\end{minipage}

%\begin{minipage}{\tw-48mm}%
%%---------------------------------------
%\begin{example}%[Integer divides order relation]%
%\label{ex:ocs_532}
%%\footnotemark
%%---------------------------------------
%The triple $\otriple{\setn{1,2,3,5,6,10,15,30}}{\metricn}{|}$ is an \structe{ordered metric space} \xref{def:oms}
%where ``$|$" is the ``divides" order relation % \xref{ex:poset_532}, 
%and $\metricn$ is defined as
%\\\indentx$\metric{x}{y}\eqd\height(x\join y)-\height(x\meet y)\qquad{\scy\forall x,y\in\setn{1,2,3,5,6,10,15,30}}$\\
%where $\height$ is the \fncte{height} function. %\xxxref{def:height}{def:latmetric}{ex:l2e3_abc_h}.
%This structure is illustrated in \prefp{fig:oms} (F).
%\end{example}%
%\end{minipage}%
%%\footnotetext{
%%  \citerpg{maclane1999}{484}{0821816462},
%%  %\citerpg{menini2004}{60}{0824709853}\\
%%  %\citerp{huntington1933}{278}\\
%%  \citePpc{sheffer1920}{310}{footnote 1}
%%  }%
%\begin{minipage}{43mm}%
%\gsize%
%\centering%
%\psset{unit=7.5mm}%
%\input{../common/math/graphics/lat2235.tex}%
%\end{minipage}%

%---------------------------------------
\begin{example}
\label{ex:ocs_prng}
%---------------------------------------
\prefp{fig:oms} (F) illustrates a \structe{linear congruential pseudo-random number generator}
induced by the equation $y_{n+1}=(y_n+2)\mod5$ with $y_0=1$.
The structure is an \structe{unordered quasi-metric space}.
See \prefpp{ex:lcg7x1m9_xyz}--\prefpp{ex:lcg7x1m9_dgraph} for further demonstration.
\end{example}
 % moved to chp_intro
 %%============================================================================
% Daniel J. Greenhoe
% LaTeX file
%============================================================================
%=======================================
\subsection{Monotone functions on ordered sets}
%=======================================
%=======================================
%\subsection{Definitions}
%=======================================
%---------------------------------------
\begin{definition}
\footnote{
  \citerpgc{rudeanu2001}{182}{1852332662}{Definition 2.1},
  \citerp{burris2000}{10},
  \citePp{topkis1978}{308}
  }
\label{def:isotone}
\label{def:antitone}
\label{def:orderpre}
\label{def:smono}
%---------------------------------------
Let $\osetX$ and $\osetsqY$ be \structe{ordered set}s \xref{def:orel}. %{def:qrel}.
%Let the relation $x\qrel  y$ imply $x\orel  y$ but $x\neq y$.
%Let the relation $x\qrelsq y$ imply $x\orelb y$ but $x\neq y$.
Let $\fphi$ be a function in $\clFxy$ \xref{def:clFxy}.
\\\defboxp{
\indentx$\begin{array}{MMC}
  $\fphi$ is \propd{isotone}          &in $\clF{\osetX}{\osetsqY}$ \quad if \quad $x\orel y \implies \fphi(x)\orelsq\fphi(y)$ & \forall x,y\in\setX.\\
  $\fphi$ is \propd{strictly isotone} &in $\clF{\osetX}{\osetsqY}$ \quad if \quad $x\qrel y \implies \fphi(x)\qrelsq\fphi(y)$ & \forall x,y\in\setX.\\
  $\fpsi$ is \propd{antitone}         &in $\clF{\osetX}{\osetsqY}$ \quad if \quad $x\orel y \implies \fpsi(y)\orelsq\fpsi(x)$ & \forall x,y\in\setX.\\
  $\fpsi$ is \propd{strictly antitone}&in $\clF{\osetX}{\osetsqY}$ \quad if \quad $x\qrel y \implies \fpsi(y)\qrelsq\fpsi(x)$ & \forall x,y\in\setX.
\end{array}$
\\
A \structe{function} is \propd{monotone}          if it is \prope{isotone} or \prope{antitone} 
and        \propd{strictly monotone} if it is \prope{strictly isotone} or \prope{strictly antitone}.
An \prope{isotone} function in $\clF{\osetX}{\osetsqY}$ 
is also said to be \propd{order preserving} in $\clF{\osetX}{\osetsqY}$.
}
\end{definition}

%=======================================
%\subsection{Properties}
%=======================================

%---------------------------------------
\begin{lemma}
\label{lem:argminphi}
%---------------------------------------
Let $\latticeX$ and $\latYsq$ be \structe{lattice}s \xref{def:lattice}.
\\Let $\ff$ be a \structe{function} in $\clFxx$.
Let $\fphi$ be a \structe{function} in $\clFxy$ \xref{def:clFxy}.
\lembox{
  \brb{\begin{array}{M}
    $\fphi$ is \prope{isotone}\\
    \xref{def:isotone}
  \end{array}}
  \quad\implies\quad
  \brb{\begin{array}{F>{\ds}rc>{\ds}lCD}
    1. & \arg\joinop_{x\in\setX}\ff(x) &\subseteq& \arg\joinopsq_{x\in\setX}\fphi\brs{\ff(x)} & and\\
    2. & \arg\meetop_{x\in\setX}\ff(x) &\subseteq& \arg\meetopsq_{x\in\setX}\fphi\brs{\ff(x)} & .
  \end{array}}
  }
\end{lemma}
\begin{proof}
\begin{align*}
  \arg\joinop_{x\in\setX}\ff(x) 
    %&\eqd \arg\joinop\set{\ff(x)}{x\in\setX}
    %&& \text{by definition of $\joinop_{x\in\setX}$}
    &= \arg_x\set{\ff(x)}{\ff(y)\orel\ff(x)\quad\forall x,y\in\setX}
    && \text{because $\ff\in\clFxx$ and by \prefp{lem:lubX}}
  \\&\subseteq \arg_x\set{\ff(x)}{\fphi\brs{\ff(y)}\orel\fphi\brs{\ff(x)}\quad\forall x,y\in\setX}
    && \text{by \prope{isotone} hypothesis \xref{def:isotone}}
  \\&= \arg_x\set{\fphi\brs{\ff(x)}}{\fphi\brs{\ff(y)}\orel\fphi\brs{\ff(x)}\quad\forall x,y\in\setX}
    && \text{because $\arg_x\set{\ff(x)}{P(x)}=\arg_x\set{\fg\brs{\ff(x)}}{P(x)}$}
  \\&\eqd \arg\joinopsq_{x\in\setX} \fphi\brs{\ff(x)}
    && \text{because $\ff\in\clFxx$ and by \prefp{lem:lubX}}
  \\
  \arg\meetop_{x\in\setX}\ff(x) 
    %&\eqd \arg\meetop\set{\ff(x)}{x\in\setX}
    %&& \text{by definition of $\meetop_{x\in\setX}$}
    &= \arg_x\set{\ff(x)}{\ff(x)\orel\ff(y)\quad\forall x,y\in\setX}
    && \text{because $\ff\in\clFxx$ and by \prefp{lem:lubX}}
  \\&\subseteq \arg_x\set{\ff(x)}{\fphi\brs{\ff(x)}\orel\fphi\brs{\ff(y)}\quad\forall x,y\in\setX}
    && \text{by \prope{isotone} hypothesis \xref{def:isotone}}
  \\&= \arg_x\set{\ff(x)}{\fphi\brs{\ff(x)}\orel\fphi\brs{\ff(y)}\quad\forall x,y\in\setX}
    && \text{by \prope{isotone} hypothesis \xref{def:isotone}}
  \\&= \arg_x\set{\fphi\brs{\ff(x)}}{\ff(x)\orel\ff(y)\quad\forall x,y\in\setX}
    && \text{because $\arg_x\set{\ff(x)}{P(x)}=\arg_x\set{\fg\brs{\ff(x)}}{P(x)}$}
  \\&\eqd \arg\meetopsq_{x\in\setX} \fphi\brs{\ff(x)}
    && \text{because $\ff\in\clFxx$ and by \prefp{lem:lubX}}
\end{align*}
%$\begin{array}[t]{Mrc>{\ds}lDrc>{\ds}lD}
%  Let & \setA &\eqd& \arg\joinop_{x\in\setX}\ff(x)  &and& \setB &\eqd& \arg\joinopsq_{x\in\setX}\fphi\brs{\ff(x)} & and \\
%      & \setC &\eqd& \arg\meetop_{x\in\setX}\ff(x)  &and& \setD &\eqd& \arg\meetopsq_{x\in\setX}\fphi\brs{\ff(x)} & .
%\end{array}$
%\begin{align*}
%  a\in\setA
%    &\iff     \ff(x)\orel\ff(a)                   && \forall x\in\setX   && \text{by definition of set $\setA$}
%  \\&\implies \fphi\brs{\ff(x)}\orelsq\fphi\brs{\ff(a)} && \forall x\in\setX   && \text{by \prope{isotone} hypothesis}
%  \\&         \iff a\in\setB                      &&                     && \text{by definition of set $\setB$}
%  \\&\implies \setA\subseteq\setB                 &&                     && \text{by definition of relation $\subseteq$}
%  \\
%  c\in\setC
%    &\iff     \ff(c)\orel\ff(x)                  && \forall x\in\setX   && \text{by definition of set $\setC$}
%  \\&\implies \fphi\brs{\ff(c)}\orelsq\fphi\brs{\ff(x)}&& \forall x\in\setX   && \text{by \prope{isotone} hypothesis}
%  \\&\iff     c\in\setD                           &&                     && \text{by definition of set $\setD$}
%  \\&\implies \setC\subseteq\setD                 &&                     && \text{by definition of relation $\subseteq$}
%\end{align*}
\end{proof}


\begin{minipage}{\tw-45mm}%
%---------------------------------------
\begin{remark}
\footnotemark
\label{rem:order_M2_L4}
%---------------------------------------
Let $\osetX$ and $\osetsqY$ be \structe{ordered set}s \xref{def:oset}.
%Let $\qrelsq$ be the \rele{quasi-order relation} of $\orelsq$ \xref{def:qrel}. 
Let $\fphi$ be a \structe{function} in $\clFxy$ \xref{def:clFxy}.
Note that even if $\fphi$ is \prope{bijective} \xref{def:ftypes} and \prope{strictly isotone} \xref{def:isotone},
\\\indentx$\begin{array}{rclC}
  x \qrel y &\notimpliedby& \fphi(x) \qrelsq \fphi(y) & \forall x,y\in\setX
\end{array}$.\\
An example is illustrated to the right where $\fphi(l)\qrelsq\fphi(r)$, but $l\orelnot r$.
%However, the relation ``$\notimpliedby$" does become ``$\iff$" if $\osetX$ and $\osetsqY$ are \prope{linear} (next lemma).
\end{remark}
\end{minipage}%
\footnotetext{%
  \citerp{burris2000}{10}%
  }%
\hfill%
\begin{tabular}{c}
  \gsize%
  \psset{unit=7.5mm}%
  %%============================================================================
% Daniel J. Greenhoe
% LaTeX file
% lattice M2 on M2
%============================================================================
{%\psset{unit=0.5\psunit}%
\begin{pspicture}(-2.8,-0.2)(1.9,3.2)%
  %---------------------------------
  % options
  %---------------------------------
  \psset{%
    linecolor=blue,%
    }%
  %---------------------------------
  % M2 lattice
  %---------------------------------
  \rput(-1.5,0){%
    \Cnode(0,3){t}%
    \Cnode(-1,1.5){x}%
    \Cnode(1,1.5){y}%
    \Cnode(0,0){b}%
    \ncline{t}{x}\ncline{t}{y}%
    \ncline{b}{x}\ncline{b}{y}%
    \uput[180](t){$t$}%
    \uput[-45](y){$r$}%
    \uput[135](x){$l$}%
    \uput[180](b){$b$}%
    \rput(0,1.5){$\osetX$}%  left lattice
    }
  %---------------------------------
  % L4 lattice
  %---------------------------------
  \rput(1.5,0){%
    \Cnode(0,3){lt}%
    \Cnode(0,2){d}%
    \Cnode(0,1){c}%
    \Cnode(0,0){lb}%
    \ncline{d}{lt}%
    \ncline{c}{d}%
    \ncline{lb}{c}%
    \uput[0](lt){$3$}%
    \uput[0](d){$2$}%
    \uput[0](c){$1$}%
    \uput[0](lb){$0$}%
    \uput[180](0,2.5){$\osetsqY$}% right lattice
    }
  %---------------------------------
  % mapping from M2 to L4
  %---------------------------------
  \ncline[linewidth=0.75pt,linecolor=red]{->}{t}{lt}%
  \ncline[linewidth=0.75pt,linecolor=red]{->}{y}{d}%
  %\ncline[linewidth=0.75pt,linecolor=red]{->}{x}{c}%
  \ncarc[arcangle=-22.5,linewidth=0.75pt,linecolor=red]{->}{x}{c}%
  \ncline[linewidth=0.75pt,linecolor=red]{->}{b}{lb}%
 %\ncarc[arcangle=-45]{->}{x}{c}%
  %---------------------------------
  % labels
  %---------------------------------
  %\uput{0.1\psunit}[90](t){$\opair{\setX}{\orel}$}%  left lattice
  %\uput{0.1\psunit}[90](lt){$\opair{\setY}{\orelb}$}% right lattice
  \rput(0.5,0.5){$\fphi\in\clFxy$}% function
  %\uput[ 90](t) {$\setn{x,y,z}$}%
  %\uput[180](xy){$\setn{x,y}$}%   
  %\uput[0](yz){$\setn{y,z}$}%
  %\uput[180](x) {$\setn{x}$}%     
  %\uput[0](z) {$\setn{z}$}%
  %\uput[-90](b) {$\szero$}%
  %\uput[0](100,300){\rnode{xzlabel}{$\setn{x,z}$}}% 
  %\uput[0](100,  0){\rnode{ylabel}{$\setn{y}$}}%
  %\ncline[linestyle=dotted,nodesep=1pt]{->}{xzlabel}{xz}%
  %\ncline[linestyle=dotted,nodesep=1pt]{->}{ylabel}{y}%
\end{pspicture}
}%%
  \includegraphics{sto/graphics/m2tol4phi.pdf}%
\end{tabular}


%---------------------------------------
\begin{lemma}
\label{lem:linisoiff}
%---------------------------------------
Let $\spX\eqd\osetX$ and $\spY\eqd\osetsqY$ be \structe{ordered set}s.
Let $\fphi$ be a \structe{function} in $\clFxy$.
\\\lemboxt{$
  \brbr{\begin{array}{FM}
    A. & $\fphi$ is \prope{strictly isotone}  \scs and\\
    B. & $\spX$ and $\spY$ are \prope{linearly ordered}       
  \end{array}}
  \implies
  \brbl{\begin{array}{FrclCD}
    1. & x\orel y &\iff& \fphi(x)\orelsq\fphi(y) & \forall x,y\in\setX & and\\
    2. & x\qrel y &\iff& \fphi(x)\qrelsq\fphi(y) & \forall x,y\in\setX & .
  \end{array}}
  $}
\end{lemma}
\begin{proof}
\begin{align*}
  \boxed{\fphi(x)\orelsq\fphi(y)} 
    &\implies y \qrelnot x           && \text{by contrapositive of \prope{strictly isotone} hypothesis (A)}
  \\&\implies \boxed{x \orel y}      && \text{by \prope{linear} hypothesis (B)}
  \\&\implies \boxed{\fphi(x)\orelsq\fphi(y)} && \text{by \prope{strictly isotone} hypothesis (A)}
  \\
  \boxed{\fphi(x)\qrelsq\fphi(y)} 
    &\implies y \orelnot x           && \text{by contrapositive of \prope{strictly isotone} hypothesis (A)}
  \\&\implies \boxed{x \qrel y}      && \text{by \prope{linear} hypothesis (B)}
  \\&\implies \boxed{\fphi(x)\qrelsq\fphi(y)} && \text{by \prope{strictly isotone} hypothesis (A)}
\end{align*}
\end{proof}

%---------------------------------------
\begin{lemma}
\label{lem:minphi}
%---------------------------------------
Let $\spX\eqd\latticeX$ and $\spY\eqd\latYsq$ be \structe{lattice}s \xref{def:lattice}.\footnote{
\begin{tabular}[t]{ll}
  \prope{strictly isotone}   & \prefp{def:isotone}  \\
  \prope{linearly ordered}   & \prefp{def:chain}    \\
  \prope{linearly ordered}   & \prefp{def:chain}    
\end{tabular}}
%Let $\joinop\setA$ be the \structe{least upper bound}   \xref{def:lub}
%and $\meetop\setA$    the \structe{greatest lower bound} \xref{def:glb} of a set $\setA\in\psetx$ \xref{def:pset}.
%Let $\joinopsq\setB$ be the \structe{least upper bound}   
%and $\meetopsq\setB$    the \structe{greatest lower bound} of a set $\setB\in\psety$.
\\Let $\ff$   be a \structe{function} in $\clFxx$.
Let $\fphi$ be a \structe{function} in $\clFxy$ \xref{def:clFxy}.
\\\lemboxt{$
  \brbr{\begin{array}{FMD}
    A. & $\fphi$ is \prope{strictly isotone}  & and \\
    B. & $\spX$ is \prope{linearly ordered}   & and \\
    C. & $\spY$ is \prope{linearly ordered}   & 
  \end{array}}
  \implies
  \brbl{\begin{array}{F>{\ds}rc>{\ds}lD}
    1. & \joinopsq_{x\in\setX} \fphi\brs{\ff(x)} &=& \fphi\brsBig{\joinop_{x\in\setX} \ff(x)}     &and \\
    2. & \meetopsq_{x\in\setX} \fphi\brs{\ff(x)} &=& \fphi\brsBig{\meetop_{x\in\setX} \ff(x)}.    
  \end{array}}
  $}
\end{lemma}
\begin{proof}
%$\begin{array}[t]{Mrc>{\ds}lDrc>{\ds}lD}
%  Let & \setA &\eqd& \joinopsq_{x\in\setX}\brs{\ff(x)}  &and& \setB &\eqd& \arg\fphi\brs{\joinop_{x\in\setX}{\ff(x)}} & and \\
%      & \setC &\eqd& \meetopsq_{x\in\setX}\brs{\ff(x)}  &and& \setD &\eqd& \arg\fphi\brs{\meetop_{x\in\setX}{\ff(x)}} & .
%\end{array}$
%\begin{align*}
%  a\in\setB
%    &\iff 
%\end{align*}
\begin{align*}
  \fphi\brsBig{\joinop_{x\in\setX}\ff(x)}
    &= \fphi\brs{\set{\ff(a)}{\ff(x)\orel\ff(a)\quad\forall x,a\in\setX}}
    && \text{because $\ff\in\clFxx$ and by \prefp{lem:lubX}}
  \\&= \set{\fphi\brs{\ff(a)}}{\ff(x)\orel\ff(a)\quad\forall x\in\setX}
  \\&= \set{\fphi\brs{\ff(a)}}{\fphi\brs{\ff(x)}\orelsq\fphi\brs{\ff(a)}\quad\forall x\in\setX}
    && \text{by \prefpp{lem:linisoiff}}
  \\&\eqd \joinopsq_{x\in\setX} \fphi\brs{\ff(x)}
    && \text{by definition of $\latYsq$}
  \\
  \fphi\brsBig{\meetop_{x\in\setX}\ff(x)}
    &= \fphi\brs{\set{\ff(a)}{\ff(a)\orel\ff(x)\quad\forall x\in\setX}}
    && \text{because $\ff\in\clFxx$ and by \prefp{lem:lubX}}
  \\&= \set{\fphi\brs{\ff(a)}}{\ff(a)\orel\ff(x)\quad\forall x\in\setX}
  \\&= \set{\fphi\brs{\ff(a)}}{\fphi\brs{\ff(a)}\orelsq\fphi\brs{\ff(x)}\quad\forall x\in\setX}
    && \text{by \prefpp{lem:linisoiff}}
  \\&\eqd \meetopsq_{x\in\setX} \fphi\brs{\ff(x)}
    && \text{by definition of $\latYsq$}
  \\
\end{align*}
\end{proof}


%---------------------------------------
\begin{lemma}
\label{lem:minphia}
%---------------------------------------
Let $\spX\eqd\latticeX$ and $\spY\eqd\latYsq$ be \structe{lattice}s.\footnote{
%Let $\joinop\setA$ be the \structe{least upper bound}   \xref{def:lub}
%and $\meetop\setA$    the \structe{greatest lower bound} \xref{def:glb} of a set $\setA\in\psetx$ \xref{def:pset}.
%Let $\joinopsq\setB$ be the \structe{least upper bound}   
%and $\meetopsq\setB$    the \structe{greatest lower bound} of a set $\setB\in\psety$.
Let $\ff$   be a \structe{function} in $\clFxx$.
\\Let $\fpsi$ be a \structe{function} in $\clFxy$. % \xref{def:clFxy}.
\begin{tabular}[t]{ll}
  \prope{strictly antitone}  & \prefp{def:antitone}  \\
  \prope{linearly ordered}   & \prefp{def:chain}     \\
  \prope{linearly ordered}   & \prefp{def:chain}    
\end{tabular}}
\\\lemboxt{$
  \brbr{\begin{array}{FMD}
    A. & $\fpsi$ is \prope{strictly antitone} & and \\
    B. & $\spX$ is \prope{linearly ordered}   & and \\
    C. & $\spY$ is \prope{linearly ordered}   & 
  \end{array}}
  \implies
  \brbl{\begin{array}{F>{\ds}rc>{\ds}lD}
    1. & \joinopsq_{x\in\setX} \fpsi\brs{\ff(x)} &=& \fpsi\brsBig{\meetop_{x\in\setX} \ff(x)}     &and \\
    2. & \meetopsq_{x\in\setX} \fpsi\brs{\ff(x)} &=& \fpsi\brsBig{\joinop_{x\in\setX} \ff(x)}.
  \end{array}}
  $}
\end{lemma}
\begin{proof}
\begin{align*}
  \fpsi\brsBig{\joinop_{x\in\setX}\ff(x)}
    &= \fpsi\brs{\set{\ff(a)}{\ff(x)\orel\ff(a)\quad\forall x\in\setX}}
    && \text{by definition of $\latticeX$}
  \\&= \set{\fpsi\brs{\ff(a)}}{\ff(x)\orel\ff(a)\quad\forall x\in\setX}
  \\&= \set{\fpsi\brs{\ff(a)}}{\fpsi\brs{\ff(a)}\orelsq\fpsi\brs{\ff(x)}\quad\forall x\in\setX}
    && \text{by definition of \prope{strictly antitone} \xref{def:isotone}}
  \\&\eqd \meetopsq_{x\in\setX} \fpsi\brs{\ff(x)}
    && \text{by definition of $\latYsq$}
  \\
  \fpsi\brsBig{\meetop_{x\in\setX}\ff(x)}
    &= \fpsi\brs{\set{\ff(a)}{\ff(a)\orel\ff(x)\quad\forall x\in\setX}}
    && \text{by definition of $\latticeX$}
  \\&= \set{\fpsi\brs{\ff(a)}}{\ff(a)\orel\ff(x)\quad\forall x\in\setX}
  \\&= \set{\fpsi\brs{\ff(a)}}{\fpsi\brs{\ff(x)}\orelsq\fpsi\brs{\ff(a)}\quad\forall x\in\setX}
    && \text{by definition of \prope{strictly antitone} \xref{def:isotone}}
  \\&\eqd \joinopsq_{x\in\setX} \fpsi\brs{\ff(x)}
    && \text{by definition of $\latYsq$}
\end{align*}
\end{proof}



%---------------------------------------
\begin{lemma}
\label{lem:linargminphi}
%---------------------------------------
Let $\spX\eqd\latticeX$ and $\spY\eqd\latYsq$ be \structe{lattice}s \xref{def:lattice}.
Let $\ff$ be a function in $\clFxx$.
Let $\fphi$ and $\fpsi$ be functions in $\clFxy$.\footnote{
  \begin{tabular}{ll}
    \prope{strictly isotone}      & \prefp{def:isotone}\\
    \prope{linearly ordered}      & \prefp{def:chain}
  \end{tabular}}
\\\lemboxt{$
  \brb{\begin{array}{FMD}
    A. & $\fphi$ is \prope{strictly isotone} & and \\
    B. & $\spX$ is \prope{linearly ordered}  
  \end{array}}
  \implies
  \brb{\begin{array}{F>{\ds}rc>{\ds}lD}
    1. & \arg\joinop_{x\in\setX}\ff(x) &=& \arg\joinopsq_{x\in\setX}\fphi\brs{\ff(x)} & and\\
    2. & \arg\meetop_{x\in\setX}\ff(x) &=& \arg\meetopsq_{x\in\setX}\fphi\brs{\ff(x)} & .
  \end{array}}
  $}
\end{lemma}
\begin{proof}
\begin{align*}
  \arg\joinop_{x\in\setX}\ff(x) 
    &\eqd \arg\joinop\set{\ff(x)}{x\in\setX}
  \\&= \arg\set{\ff(a)}{\ff(x)\orel\ff(a)\quad\forall x,a\in\setX}
    && \text{because $\ff\in\clFxx$ and by \prefp{lem:lubX}}
  \\&= \arg\set{\ff(a)}{\fphi\brs{\ff(x)}\orel\fphi\brs{\ff(a)}\quad\forall x,a\in\setX}
    %&& \text{by \prope{strictly isotone} hypothesis \xref{def:isotone} and \prefp{lem:minphia}}
    && \text{by hypothesis (A) and \prefp{lem:minphia}}
  \\&= \arg\set{\fphi\brs{\ff(a)}}{\ff(x)\orel\ff(a)\quad\forall x,a\in\setX}
    && \text{because $\arg_x\set{\ff(x)}{P(x)}=\arg_x\set{\fg\brs{\ff(x)}}{P(x)}$}
  \\&\eqd \joinopsq_{x\in\setX} \fphi\brs{\ff(x)}
    && \text{because $\ff\in\clFxx$ and by \prefp{lem:lubX}}
  \\
  \arg\meetop_{x\in\setX}\ff(x) 
    &\eqd \arg\meetop\set{\ff(x)}{x\in\setX}
  \\&= \arg\set{\ff(a)}{\ff(a)\orel\ff(x)\quad\forall x,a\in\setX}
    && \text{because $\ff\in\clFxx$ and by \prefp{lem:lubX}}
  \\&= \arg\set{\ff(a)}{\fphi\brs{\ff(a)}\orel\fphi\brs{\ff(x)}\quad\forall x,a\in\setX}
    %&& \text{by \prope{strictly isotone} hypothesis \xref{def:isotone} and \prefp{lem:minphia}}
    && \text{by hypothesis (A) and \prefp{lem:minphia}}
  \\&= \arg\set{\fphi\brs{\ff(a)}}{\ff(a)\orel\ff(x)\quad\forall x,a\in\setX}
    && \text{because $\arg_x\set{\ff(x)}{P(x)}=\arg_x\set{\fg\brs{\ff(x)}}{P(x)}$}
  \\&\eqd \meetopsq_{x\in\setX} \fphi\brs{\ff(x)}
    && \text{because $\ff\in\clFxx$ and by \prefp{lem:lubX}}
\end{align*}
%Let $\setA$, $\setB$, $\setC$, and $\setD$ be defined as in \prefpp{lem:argminphi}.
%\begin{align*}
%  a\in\setA
%    &\iff \ff(x)\orel\ff(a)                   && \forall x\in\setX   && \text{by definition of set $\setA$}
%  \\&\iff \fphi\brs{\ff(x)}\orelsq\fphi\brs{\ff(a)} && \forall x\in\setX   && \text{by (A) and (B) and \prefpp{lem:linisoiff}}
%  \\&\iff a\in\setB                           &&                     && \text{by definition of set $\setB$}
%  \\&\implies \setA=\setB
%  \\
%  c\in\setC
%    &\iff \ff(c)\orel\ff(x)                  && \forall x\in\setX   && \text{by definition of set $\setC$}
%  \\&\iff \fphi\brs{\ff(c)}\orelsq\fphi\brs{\ff(x)}&& \forall x\in\setX   && \text{by (A) and (B) and \prefpp{lem:linisoiff}}
%  \\&\iff c\in\setD                           &&                     && \text{by definition of set $\setD$}
%  \\&\implies \setC=\setD
%\end{align*}
\end{proof}

%---------------------------------------
\begin{remark}
%2015 February 01 Sunday 6:35am
%---------------------------------------
Using the definitions of \prefpp{lem:linargminphi}, and letting $\fg$ be a function in $\clFxx$,
and despite the results of \pref{lem:linargminphi}, 
note that\\ 
\remboxt{$
  \brbr{\begin{array}{FMD}
    A. & $\fphi$ is \prope{strictly isotone}      & and\\
    B. & \mc{2}{M}{$\spX$ is \prope{linearly ordered}}
  \end{array}}
  \notimplies
  \brbl{\begin{array}{>{\ds}rc>{\ds}l}
    \arg\meetop_{x\in\setX}\brs{\ff(x)\fg(x)} &=& \arg\meetopsq_{x\in\setX}\fphi\brs{\ff(x)}\fg(x)
  \end{array}}
  $}\\
\tbox{
  For example, let $\ff(x)\eqd x$, $\fg(x)\eqd-x+2$, and $\fphi(x)\eqd x^2$. Then
  \\\indentx$\begin{array}{>{\ds}rc>{\ds}l}
    \arg\meetop_{x\in\setX}\brs{\ff(x)\fg(x)}
       &\eqd& \arg\meetop_{x\in\setX}\brs{-x^2+2x}
     \\&=&    1 \neq \frac{4}{3}  
     \\&=&    \arg\meetopsq_{x\in\setX}\brs{-x^3+2x^2}
     \\&\eqd& \arg\meetopsq_{x\in\setX}\fphi\brs{\ff(x)}\fg(x)
    \end{array}$}
\hfill\tbox{\includegraphics{sto/graphics/argmin_xx-2x2.pdf}}
\end{remark}


%---------------------------------------
\begin{lemma}
\label{lem:argminmaxphi}
%---------------------------------------
Let $\spX\eqd\latticeX$ and $\spY\eqd\latYsq$ be \structe{lattice}s \xref{def:lattice}.
\\Let $\ff$ be a function in $\clF{\setX}{(\setX\times\setX)}$.
Let $\fphi$ and $\fpsi$ be functions in $\clFxy$.
\\\lemboxt{
  $\brbr{\begin{array}{FMD}
    A. & $\fphi$ is \prope{strictly isotone}      & and\\
    B. & \mc{2}{M}{$\spX$ is \prope{linearly ordered}}
  \end{array}}$
  \\\qquad$\implies$
  $\brbl{\begin{array}{F>{\ds}rc>{\ds}lCD}
    1. & \arg\meetop_{x\in\setX}\joinop_{y\in\setX} \ff(x,y) &=& \arg\meetopsq_{x\in\setX}\joinopsq_{y\in\setX} \fphi\brs{\ff(x,y)} & \forall x,y\in\setX & and \\ 
    2. & \arg\meetop_{x\in\setX}\joinop_{y\in\setX} \ff(x,y) &=& \arg\meetopsq_{x\in\setX}\meetopsq_{y\in\setX} \fphi\brs{\ff(x,y)} & \forall x,y\in\setX & and \\ 
    3. & \arg\joinop_{x\in\setX}\meetop_{y\in\setX} \ff(x,y) &=& \arg\joinopsq_{x\in\setX}\meetopsq_{y\in\setX} \fphi\brs{\ff(x,y)} & \forall x,y\in\setX & 
  \end{array}}$
  }
\end{lemma}
\begin{proof}
\begin{align*}
  \arg\meetopsq_{x\in\setX}\joinopsq_{y\in\setX} \fphi\brs{\ff(x,y)} 
    &= \arg\meetopsq_{x\in\setX}\fphi\brs{\joinop_{y\in\setX} \ff(x,y)} 
    && \text{by \prefpp{lem:minphi}}
  \\&= \arg\meetop_{x\in\setX}\joinop_{y\in\setX} \ff(x,y)
    && \text{by \prefpp{lem:argminphi}}
  \\
  \arg\meetopsq_{x\in\setX}\meetopsq_{y\in\setX} \fphi\brs{\ff(x,y)} 
    &= \arg\meetopsq_{x\in\setX}\fphi\brs{\meetop_{y\in\setX} \ff(x,y)} 
    && \text{by \prefpp{lem:minphi}}
  \\&= \arg\meetop_{x\in\setX}\meetop_{y\in\setX} \ff(x,y)
    && \text{by \prefpp{lem:argminphi}}
  \\
  \arg\joinopsq_{x\in\setX}\meetopsq_{y\in\setX} \fphi\brs{\ff(x,y)} 
    &= \arg\meetopsq_{x\in\setX}\fphi\brs{\meetop_{y\in\setX} \ff(x,y)} 
    && \text{by \prefpp{lem:minphi}}
  \\&= \arg\joinop_{x\in\setX}\meetop_{y\in\setX} \ff(x,y)
    && \text{by \prefpp{lem:argminphi}}
\end{align*}
\end{proof}


%---------------------------------------
\begin{remark}
\label{rem:sumphi}
%---------------------------------------
Let $\osetX$ be an \structe{ordered set} \xref{def:oset}.
Let $\fphi$ be a function in $\clFxx$.
\\\remboxt{
If $\fphi$ is \prope{strictly isotone} then
\\\indentx$\begin{array}{>{\ds}ll>{\ds}l c >{\ds}ll>{\ds}l M}
        \sum_{x\in\setX} \ff(x) &<& \sum_{x\in\setX} \fg(x) 
        &\implies&
        \fphi\brs{\sum_{x\in\setX} \ff(x)} &<& \fphi\brs{\sum_{x\in\setX} \fg(x)} 
      & \emph{but}
    \\
        \sum_{x\in\setX}\fphi\brs{\ff(x)} &<& \sum_{x\in\setX}\fphi\brs{\fg(x)} 
        &\notimplies&
        \fphi\brs{\sum_{x\in\setX} \ff(x)} &<& \fphi\brs{\sum_{x\in\setX} \fg(x)} 
      & .
  \end{array}$
  }
\end{remark}
\begin{proof}
\begin{enumerate}
  \item Proof for (1): this follows directly from the definition of \prope{isotone} \xref{def:isotone}.
  \item Proof for (2): Let 
    $\ff(x)=\seqn{\frac{1}{2},\,\frac{1}{2},\,\frac{1}{2}}$,
    $\fg(x)=\seqn{0,\,0,\,1}$, and 
    $\fphi(x)=x^2$. 
    \\$\begin{array}{M>{\ds}lclclclclc>{\ds}l}
      Then &
      \sum_{x\in\setX} \fphi\brs{\ff(x)} &=& \brp{\frac{1}{2}}^2 + \brp{\frac{1}{2}}^2 + \brp{\frac{1}{2}}^2 &=& \frac{3}{4}
      &<&
      1 &=& 0^2 + 0^2 + 1^2 &=& \sum_{x\in\setX} \fphi\brs{\fg(x)}
      \\
      but &
      \fphi\brs{\sum_{x\in\setX} \ff(x)} &=& \brp{\frac{1}{2}+ \frac{1}{2} + \frac{1}{2}}^2 &=& \frac{9}{4}
      &>&
      1 &=& \brp{0 + 0 + 1}^2 &=& \fphi\brs{\sum_{x\in\setX} \fg(x)}
    \end{array}$

\end{enumerate}
\end{proof}






 % moved to chp_intro
  %============================================================================
% Daniel J. Greenhoe
% XeLaTeX file
% stochastic systems
%============================================================================

%=======================================
\section{Outcome subspaces}
\label{sec:ocs}
%=======================================
%=======================================
\subsection{Definitions}
%=======================================
Traditional probability theory is performed in a \structe{probability space} $\ps$. % \xref{def:ps}.
This section extends\footnote{
  \citerpgc{feldman2010}{4}{3642051588}{``The name ``random variable" is actually a misnomer, since it is not random and not a variable.\ldots the random variable simply maps each point (outcome) in the sample space to a number on the real line\ldots Technically, the space into which the random variable maps the sample space may be more general than the real line\ldots"}
  }
 the probability space structure to include what herein is called an
\structe{outcome subspace} (next definition).
%---------------------------------------
\begin{definition}
\label{def:ocs}
\label{def:ocsm}
%---------------------------------------
\defboxp{
  An \structd{extended probability space} is the tuple $\epsX$ where 
  $\ps$ is a \structe{probability space} \xref{def:ps}
  and $\otriple{\epso}{\metricn}{\orel}$ is an \structe{ordered quasi-metric space} \xref{def:oms}.
  The 4-tuple $\ocsD$ is an \structd{outcome subspace}
  of the \structe{extended probability space} $\epsX$.
  }
\end{definition}

%---------------------------------------
\begin{definition}
\label{def:ocsmom}
%---------------------------------------
%Let $\Zp\eqd\setn{1,2,3,\ldots}$ be the set of \structe{natural numbers}.
Let $\ocsG\eqd\ocsD$ be an \structe{outcome subspace} \xref{def:ocs}.
\\\defboxt{$\begin{array}{MrclC}
    The \structd{$n$th-moment} $\ocsmom_n(x,y)$ from $x$ to $y$ in $\ocsG$ is defined as & \ocsmom(x,y) &\eqd& \brs{\metric{x}{y}}^n\psp(y) & \forall x,y\in\ocso,\,n\in\Zp.\\
    The \structd{moment}       $\ocsmom(x,y)$   from $x$ to $y$ in $\ocsG$ is defined as & \ocsmom(x,y) &\eqd& \ocsmom_1(x,y)               & \forall x,y\in\ocso.
  \end{array}$}
\end{definition}

This paper introduces a quantity called the \structe{outcome center} of an \structe{outcome subspace} (next definition)
which is in essence the same as the \structe{center} of a \structe{graph} \xref{def:gcen}.
%---------------------------------------
\begin{definition}
\label{def:ocscen}
%---------------------------------------
Let $\ocsG\eqd\ocsD$ be an \structe{outcome subspace} \xref{def:ocsm}.
\\\defbox{
  \begin{array}{>{\ds}rc>{\ds}lMM}
      \ocscen (\ocsG) &\eqd& \argmin_{x\in\ocso}\max_{y\in\ocso}\mcom{\metric{x}{y}\psp(y)}{$\ocsmom(x,y)$} & is the \structd{outcome center}    &of $\ocsG$. 
  \end{array}
  }
\end{definition}

The following additional definitions are of interest due in part to 
\prefpp{cor:means} and the \ineqe{minimax inequality} \xref{thm:minimax}.
They are illustrated in several examples in this section.
However, %their usefulness is somewhat limited in the context of this paper and 
most of them are not used outside this section.
%---------------------------------------
\begin{definition}
\label{def:ocscena}
\label{def:ocsceng}
\label{def:ocscenh}
\label{def:ocscenm}
\label{def:ocscenM}
\label{def:ocscenx}
%---------------------------------------
Let $\ocsG\eqd\ocsD$ be an \structe{outcome subspace} \xref{def:ocsm}.
\\\defbox{
  \begin{array}{>{\ds}rc>{\ds}lMM}
      \ocscena(\ocsG) &\eqd& \argmin_{x\in\ocso}\sum_{y\in\ocso}\metric{x}{y}\psp(y)                                   & is the \structd{arithmetic center} &of $\ocsG$.
    \\\ocsceng(\ocsG) &\eqd& \argmin_{x\in\ocso}\prod_{y\in\ocso\setd\setn{x}}\brs{\metric{x}{y}^{\psp(y)}}            & is the \structd{geometric center}  &of $\ocsG$.
    \\\ocscenh(\ocsG) &\eqd& \argmin_{x\in\ocso}\brp{\sum_{y\in\ocso\setd\setn{x}}\frac{1}{\metric{x}{y}}\psp(y)}^{-1} & is the \structd{harmonic center}   &of $\ocsG$.
    \\\ocscenm(\ocsG) &\eqd& \argmin_{x\in\ocso}\min_{y\in\ocso\setd\setn{x}}\metric{x}{y}\psp(y)                      & is the \structd{minimal center}    &of $\ocsG$.
    \\\ocscenM(\ocsG) &\eqd& \argmax_{x\in\ocso}\min_{y\in\ocso\setd\setn{x}}\metric{x}{y}\psp(y)                      & is the \structd{maxmin center}    &of $\ocsG$.
  \end{array}
  }
\end{definition}

In a manner similar to the traditional \fncte{variance} function \xref{def:pVar},
the \fncte{outcome variance} (next) is a kind of measure of the quality of the outcome center as a representative estimate 
of all the values of the \structe{outcome subspace}.
Said another way, it is in essence the expected error of the center measure.
%---------------------------------------
\begin{definition}
\label{def:ocsVarG}
%---------------------------------------
Let $\ocsG\eqd\ocsD$ be an \structe{outcome subspace} \xref{def:ocsm}.
\\\defboxp{
  The \structd{outcome variance} $\ocsVaro(\ocsG;\ocscen_x)$ of $\ocsG$ with respect to $\ocscen_x$ is
  \indentx$\ds\ocsVaro(\ocsG;\ocscen_x)\eqd\sum_{x\in\ocso} \mcom{\metricsq{\ocscen_x(\ocsG)}{x}\psp(x)}{$\ocsmom_2(\ocscen(\ocsG),x)$}$.\\
  where $\ocscen_x$ is any of the operators defined in \pref{def:ocscen} or \prefpp{def:ocsceng}.\\
  Moreover, $\ocsVaro(\ocsG)\eqd\ocsVaro(\ocsG;\ocscen)$, where $\ocscen$ is the \structe{outcome center} \xref{def:ocscen}.
  }
\end{definition}

%=======================================
%\subsection{Properties}
%=======================================
%%---------------------------------------
%\begin{theorem}
%\label{thm:ocs_isotone}
%%---------------------------------------
%Let $\ocsG\eqd\ocs{\ocso}{\metricn} {\orel}{\psp}$
%and $\ocsH\eqd\ocs{\ocso}{\metrican}{\orel}{\psp}$ be \structe{outcome subspace}s \xref{def:ocs}.
%Let $\ff\in\clFrr$ be a function from $\R$ onto $\R$.
%\thmbox{
%  \brb{\begin{array}{FMD}
%    A. & $\ff$ is \prope{strictly isotone} \xref{def:smono}   & and\\
%    B. & $\metric{x}{y} = \ff\brs{\metrica{x}{y}}$ & $\forall x,y\in\ocso$
%  \end{array}}
%  \quad\implies\quad
%  \brb{\begin{array}{FlclD}
%    1. & \ocscen (\ocsG) &=& \ocscen (\ocsH) & and \\
%    2. & \ocscenm(\ocsG) &=& \ocscenm(\ocsH) & and \\
%    3. & \ocscenM(\ocsG) &=& \ocscenM(\ocsH) & 
%  \end{array}}
%  }
%\end{theorem}
%\begin{proof}
%\begin{align*}
%  \ocscen(\ocsG)
%    &\eqd \argmin_{x\in\ocso}\max_{y\in\ocso} \metric{x}{y}\psp(y)
%    && \text{by definition of $\ocscen$ \xref{def:ocscen} and $\ocsG$}
%  \\&= \argmin_{x\in\ocso}\max_{y\in\ocso} \metrica{x}{y} \psp(y)
%    && \text{by $\ff$ is \prope{strictly isotone} hypothesis and \prefpp{lem:argminmaxphi}}
%  \\&\eqd \ocscen(\ocsK)
%    && \text{by definition of $\ocscen$ \xref{def:ocscen} and $\ocsK$}
%  \\
%  \ocscenm(\ocsG)
%    &\eqd \argmin_{x\in\ocso}\min_{y\in\ocso} \metric{x}{y}\psp(y)
%    && \text{by definition of $\ocscen$ \xref{def:ocscen} and $\ocsG$}
%  \\&= \argmin_{x\in\ocso}\min_{y\in\ocso} \metrica{x}{y} \psp(y)
%    && \text{by $\ff$ is \prope{strictly isotone} hypothesis and \prefpp{lem:argminmaxphi}}
%  \\&\eqd \ocscenm(\ocsK)
%    && \text{by definition of $\ocscen$ \xref{def:ocscen} and $\ocsK$}
%  \\
%  \ocscenM(\ocsG)
%    &\eqd \argmax_{x\in\ocso}\min_{y\in\ocso} \metric{x}{y}\psp(y)
%    && \text{by definition of $\ocscen$ \xref{def:ocscen} and $\ocsG$}
%  \\&= \argmax_{x\in\ocso}\min_{y\in\ocso} \metrica{x}{y} \psp(y)
%    && \text{by $\ff$ is \prope{strictly isotone} hypothesis and \prefpp{lem:argminmaxphi}}
%  \\&\eqd \ocscenM(\ocsK)
%    && \text{by definition of $\ocscen$ \xref{def:ocscen} and $\ocsK$}
%\end{align*}
%\end{proof}
%---------------------------------------
\begin{remark}
%---------------------------------------
The quantity $\ds\metrican\brp{x,\ocso}\eqd \sum_{y\in\ocso}\metric{x}{y}\psp(y)$
in the \structe{arithmetic center} $\ocscena(\ocsG)$ \xref{def:ocscena}
is itself a \structe{metric} \xref{def:metric}.
Thus, $\ocscena(\ocsG)$ is the $x$ that produces the minimum of all the metrics with center $x$.
\end{remark}
\begin{proof}
This follows directly from \structe{power mean metrics} theorem with $r=1$ \xref{thm:met_power}.
\end{proof}

%%---------------------------------------
%\begin{proposition}
%%---------------------------------------
%Let $\ocsmom_n(x,y)$ be the \structe{$n$th-moment} \xref{def:ocsmom} 
%on the \structe{outcome subspace} $\ocsD$ \xref{def:ocs}.
%\propbox{
%  \ocsmom_n(x,x)=0 \qquad n=1,2,3,\ldots
%  }
%\end{proposition}
%\begin{proof}
%\begin{align*}
%  \ocsmom_n(x,x)
%    &= \brs{\metric{x}{x}}^n\psp(x)
%    && \text{by \prefp{def:ocsmom}}
%  \\&= \brs{0}^n\psp(x)
%    && \text{by \prope{nondegenerate} property of \fncte{metric}s \xref{def:metric}}
%  \\&= 0 
%    && \text{by field property of $0$}
%\end{align*}
%\end{proof}
%
%%---------------------------------------
%\begin{proposition}
%%---------------------------------------
%Let $\ocsG\eqd\ocsD$ be an \structe{outcome subspace} \xref{def:ocs}.
%Let $\ocsmom(x,y)$ be the \structe{moment} \xref{def:ocsmom} on $\ocsG$
%and $\ocscen(\ocsG)$ be the \structe{center} \xref{def:ocscen} of $\ocsG$.
%\propbox{
%  \ocscen(\ocsG) = \argmin_{x\in\ocso}\max_{y\in\ocso}\ocsmom(x,y)
%  }
%\end{proposition}
%\begin{proof}
%\begin{align*}
%  \ocscen(\ocsG)
%    &\eqd \argmin_{x\in\ocso}\max_{y\in\ocso}\metric{x}{y}\psp(y)
%    && \text{by definition of $\ocscen$ \xref{def:ocscen}}
%  \\&= \argmin_{x\in\ocso}\max_{y\in\ocso}\ocsmom(x,y)
%    && \text{by definition of $\ocsmom$ \xref{def:ocsmom}}
%\end{align*}
%\end{proof}

%\if 0
%=======================================
\subsection{Specific outcome subspaces}
%=======================================
\begin{figure}[h]
  \centering%
  \footnotesize%
  \begin{tabular}{|*{4}{c|}}
    \hline
     \includegraphics{../common/math/graphics/pdfs/ocs_fdie.pdf}%
    &\includegraphics{../common/math/graphics/pdfs/ocs_rdie.pdf}%
    &\includegraphics{../common/math/graphics/pdfs/ocs_wrdie.pdf}%
    &\includegraphics{../common/math/graphics/pdfs/ocs_wdie.pdf}%
    \\
      \structe{fair die} \xrefr{def:fdie}%
     &\structe{real die} \xrefr{def:rdie}%
     &\structe{weighted real die} \xrefr{def:wrdie}%
     &\structe{weighted die} \xrefr{def:wdie}%
    \\\hline
     \includegraphics{../common/math/graphics/pdfs/ocs_spinner.pdf}%
    &\includegraphics{../common/math/graphics/pdfs/ocs_dna.pdf}%
    &\includegraphics{../common/math/graphics/pdfs/ocs_dnan.pdf}%
    &
    \\
     \structe{spinner} \xrefr{def:spinner}%
    &\structe{DNA} \xrefr{def:dna}%
    &\structe{scaffold DNA} \xrefr{def:dnan}%
    &
    \\\hline
  \end{tabular}%
  \caption{example \structe{outcome subspace}s \xref{def:ocs} 
     illustrated by \structe{weighted graph}s\label{fig:ocs}
     with shaded \ope{expected value}s.
     %For more details, see \citeP{greenhoe2015sto}%
     }
\end{figure}

%---------------------------------------
\begin{definition}
%\footnote{\citeP{greenhoe2015sto}}
\label{def:wdie}
%---------------------------------------
\defboxp{
  The structure
  $\ocsG\eqd\ocs{\setn{\diceA,\diceB,\diceC,\diceD,\diceE,\diceF}}{\ocsd}{\ocsr}{\ocsp}$
  is the \structd{weighted die outcome subspace} if $\ocsG$ is an \structe{outcome subspace}, %\xref{def:ocsm},
  $\ocsr=\emptyset$ (\prope{unordered} \xrefnp{def:unordered}),
  and $\ocsd$ is the \fncte{discrete metric} \xref{def:discretemetric}.
  }
\end{definition}

%---------------------------------------
\begin{definition}
%\footnote{\citeP{greenhoe2015sto}}
\label{def:fdie}
%---------------------------------------
\defboxp{
  The structure
  $\ocsG\eqd\ocs{\setn{\diceA,\diceB,\diceC,\diceD,\diceE,\diceF}}{\ocsd}{\ocsr}{\ocsp}$
  is the \structd{fair die outcome subspace} if $\ocsG$ is a \structe{weighted die outcome subspace} \xrefr{def:wdie},
  and 
  \\\indentx$\ocsp(\dieA)=\ocsp(\dieB)=\ocsp(\dieC)=\ocsp(\dieD)=\ocsp(\dieE)=\ocsp(\dieF)=\sfrac{1}{6}$.
  }
\end{definition}

%---------------------------------------
\begin{minipage}{\tw-53mm}%
\begin{definition}
%\footnotemark
\label{def:wrdie}
%---------------------------------------
The structure
$\ocsG\eqd\ocs{\setn{\diceA,\diceB,\diceC,\diceD,\diceE,\diceF}}{\ocsd}{\emptyset}{\ocsp}$
is the \structd{weighted real die outcome subspace} if $\ocsG$ is an \structe{outcome subspace}, % \xref{def:ocsm},
%$\ocsr=\emptyset$, % (\prope{unordered} \xrefnp{def:unordered}),
%\\\indentx$\ocsp(\dieA)=\ocsp(\dieB)=\ocsp(\dieC)=\ocsp(\dieD)=\ocsp(\dieE)=\ocsp(\dieF)=\sfrac{1}{6}$,\\
and \fncte{metric} $\ocsd$ is defined as in the table to the right.
\end{definition}
\end{minipage}%
%\citetblt{\citeP{greenhoe2015sto}}%
\hfill%
  $\begin{tabstr}{0.75}\begin{array}{|c|*{6}{@{\hspace{2pt}}c}|}
    \hline
    \ocsd(x,y) & \dieA &\dieB &\dieC &\dieD &\dieE &\dieF
    \\\hline
      \dieA &    0   &   1   &   1   &   1   &   1   &   2
    \\\dieB &    1   &   0   &   1   &   1   &   2   &   1
    \\\dieC &    1   &   1   &   0   &   2   &   1   &   1
    \\\dieD &    1   &   1   &   2   &   0   &   1   &   1
    \\\dieE &    1   &   2   &   1   &   1   &   0   &   1
    \\\dieF &    2   &   1   &   1   &   1   &   1   &   0
    \\\hline
  \end{array}\end{tabstr}$

%---------------------------------------
\begin{definition}
%\footnote{\citeP{greenhoe2015sto}}
\label{def:rdie}
%---------------------------------------
The structure
$\ocsG\eqd\ocs{\setn{\diceA,\diceB,\diceC,\diceD,\diceE,\diceF}}{\ocsd}{\emptyset}{\ocsp}$
is the \structd{real die outcome subspace} if $\ocsG$ is a
\structe{weighted real die outcome subspace} \xref{def:wrdie} with
\\\indentx$\ocsp(\dieA)=\ocsp(\dieB)=\ocsp(\dieC)=\ocsp(\dieD)=\ocsp(\dieE)=\ocsp(\dieF)=\sfrac{1}{6}$.
\end{definition}

%---------------------------------------
\begin{minipage}{\tw-53mm}%
\begin{definition} %[\structd{real die outcome subspace}]
%\footnote{\citeP{greenhoe2015sto}}
\label{def:spinner}%\mbox{}\\
%---------------------------------------
The structure
$\ocsG\eqd\ocs{\setn{\circOne,\circTwo,\circThree,\circFour,\circFive,\circSix}}{\ocsd}{\emptyset}{\ocsp}$
is the 
\\\structd{spinner outcome subspace} if $\ocsG$ is an \structe{outcome subspace}, % \xref{def:ocsm},
%\\$\ocsr=\emptyset$, % (\prope{unordered} \xrefnp{def:unordered}),
\\\indentx$\ocsp(\circOne)=\ocsp(\circTwo)=\ocsp(\circThree)=\ocsp(\circFour)=\ocsp(\circFive)=\ocsp(\circSix)=\sfrac{1}{6}$,
\\and \fncte{metric} $\ocsd$ is defined as in the table to the right.
\end{definition}
\end{minipage}%
\hfill%
  $\begin{tabstr}{0.75}\begin{array}{|c|*{6}{@{\hspace{5pt}}c}|}
    \hline
    \ocsd(x,y)    & \circOne & \circTwo & \circThree & \circFour & \circFive & \circSix
    \\\hline
      \circOne    &    0     &   1      &   2        &   3       &   2       &   1
    \\\circTwo    &    1     &   0      &   1        &   2       &   3       &   2
    \\\circThree  &    2     &   1      &   0        &   1       &   2       &   3
    \\\circFour   &    3     &   2      &   1        &   0       &   1       &   2
    \\\circFive   &    2     &   3      &   2        &   1       &   0       &   1
    \\\circSix    &    1     &   2      &   3        &   2       &   1       &   0
    \\\hline
  \end{array}\end{tabstr}$

%---------------------------------------
\begin{definition}
%\footnote{\citeP{greenhoe2015sto}}
\label{def:dna}
%---------------------------------------
The structure
$\ocsH\eqd\ocs{\setn{\symA,\symT,\symC,\symG}}{\ocsd}{\emptyset}{\ocsp}$
is the \structd{DNA outcome subspace}, or \structd{genome outcome subspace}, 
if $\ocsH$ is an \structe{outcome subspace}, % \xref{def:ocsm},
and $\ocsd$ is the \fncte{discrete metric} \xref{def:dmetric}.
\end{definition}

%---------------------------------------
\begin{minipage}{\tw-58mm}%
\begin{definition}
%\footnotemark
\label{def:dnan}
%---------------------------------------
The structure
$\ocsH\eqd\ocs{\setn{\symA,\symT,\symC,\symG,\symN}}{\ocsd}{\ocsr}{\ocsp}$\\
is the \structd{DNA scaffold outcome subspace}, or \structd{genome scaffold outcome subspace}, 
if $\ocsH$ is an \structe{outcome subspace}, % \xref{def:ocsm},
\\\indentx
  $\ocsr=\setn{\opair{\symN}{\symA},\,\opair{\symN}{\symT},\,\opair{\symN}{\symC},\,\opair{\symN}{\symG},\,}$ 
\\
($\symN<\symA$, $\symN<\symT$, $\symN<\symC$, and $\symN<\symG$, but otherwise \prope{unordered}),\\
$\ocsp$ is a \fncte{probability function},
and \fncte{metric} $\ocsd$ is defined as in the table to the right.
\end{definition}
\end{minipage}%
%\citetblt{\citeP{greenhoe2015sto}}%
\hfill%
  $\begin{tabstr}{0.75}\begin{array}{|c|*{5}{@{\hspace{5pt}}c}|}
    \hline
    \ocsd(x,y)    & \symN    & \symA    & \symT      & \symC     & \symG 
    \\\hline
      \symN       &    \ds  0                 & \ds\sfrac{\sqrt{2}}{2} & \ds\sfrac{\sqrt{2}}{2} & \ds\sfrac{\sqrt{2}}{2}  & \ds\sfrac{\sqrt{2}}{2} 
    \\\symA       &    \ds\sfrac{\sqrt{2}}{2} &   0                 &   1                 &   1                  &   1   
    \\\symT       &    \ds\sfrac{\sqrt{2}}{2} &   1                 &   0                 &   1                  &   1   
    \\\symC       &    \ds\sfrac{\sqrt{2}}{2} &   1                 &   1                 &   0                  &   1   
    \\\symG       &    \ds\sfrac{\sqrt{2}}{2} &   1                 &   1                 &   1                  &   0   
    \\\hline
  \end{array}\end{tabstr}$


%=======================================
\subsection{Example calculations}
%=======================================
\begin{tabstr}{0.75}
%---------------------------------------
\begin{example}%[\exmd{fair die outcome subspace} / \exmd{discrete outcome subspace}]%\mbox{}\\
\label{ex:fairdie}
%---------------------------------------
%\begin{minipage}{\tw-35mm}%
Let $\ocsG\eqd\ocs{\setn{\diceA,\diceB,\diceC,\diceD,\diceE,\diceF}}{\ocsd}{\ocsr}{\ocsp}$ 
be the \structe{fair die outcome subspace} \xref{def:fdie}.
This structure is illustrated by the \structe{weighted graph} \xref{def:wgraph} in \prefpp{fig:ocs} (A), %to the right, 
where each line segment represents a distance of $1$.
%
%be the \structe{outcome subspace} \xref{def:ocsm} generated by a \structe{fair die}
%where $\metricn$ is the \fncte{discrete metric} \xref{def:dmetric},
%$\orel\eqd\emptyset$ (completely unordered set), and 
%$\psp(\diceA)=\psp(\diceB)=\cdots=\psp(\diceF)=\frac{1}{6}$.
%This is illustrated by the \structe{weighted graph} \xref{def:wgraph} to the right, 
%where each line segment represents a distance of $1$, and elements in the \structe{outcome center} $\ocscen(\ocsG)$
%are \colorbox{snode}{shaded}.
This structure has the following geometric values:
  \\\indentx$\begin{array}{l}
      \ocscen(\ocsG)=\ocscena(\ocsG)=\ocsceng(\ocsG)=\ocscenh(\ocsG)=\ocscenm(\ocsG)=\ocscenM(\ocsG)  
         = \setn{\diceA,\diceB,\diceC,\diceD,\diceE,\diceF} %& \xxref{def:ocscen}{def:ocscenx}  & (shaded in illustration)
    \\\ocsVaro(\ocsG) = \ocsVaro(\ocsG;\ocscena) = \ocsVaro(\ocsG;\ocsceng) = \ocsVaro(\ocsG;\ocscenh) = \ocsVaro(\ocsG;\ocscenm) = \ocsVaro(\ocsG;\ocscenM)  =  0     
  \end{array}$
%\end{minipage}\hfill%
%\begin{tabular}{c}
%  \gsize%
%  \centering%
%  {\includegraphics{../common/math/graphics/pdfs/ocs_fdie.pdf}}%
%\end{tabular}
\end{example}
\begin{proof}
\begin{align*}
  \ocscen(\ocsG)
    &\eqd \argmin_{x\in\ocsG}\max_{y\in\ocsG}\metric{x}{y}\psp(y)
    && \text{by definition of $\ocscen$ \xref{def:ocscen}}
  \\&= \argmin_{x\in\ocsG}\max_{y\in\ocsG}\metric{x}{y}\frac{1}{6}
    && \text{by definition of $\ocsG$}
  %\\&= \argmin_{x\in\ocsG}\max_{y\in\ocsG}\metric{x}{y}
  %  && \text{because $\ff(x)=\frac{1}{6}x$ is \prope{strictly isotone} and by \prefpp{lem:argminmaxphi}}
  \\&= \argmin_{x\in\ocsG}\frac{1}{6}\setn{1,\,1,\,1,\,1,\,1,\,1}
    && \text{by definition of \structe{discrete metric} \xref{def:dmetric}}
  \\&= \setn{\diceA,\diceB,\diceC,\diceD,\diceE,\diceF}
    && \text{by definition of $\ocsG$}
  \\
  \ocscena(\ocsG)
    &\eqd \argmin_{x\in\ocsG}\sum_{y\in\ocsG}\metric{x}{y}\psp(y)
    && \text{by definition of $\ocscena$ \xref{def:ocscenx}}
  \\&= \argmin_{x\in\ocsG}\sum_{y\in\ocsG}\metric{x}{y}\frac{1}{6}
    && \text{by definition of $\ocsG$}
  %\\&= \argmin_{x\in\ocsG}\sum_{y\in\ocsG}\metric{x}{y}
  %  && \text{because $\ff(x)=\frac{1}{6}x$ is \prope{strictly isotone} and by \prefpp{lem:argminmaxphi}}
  \\&= \argmin_{x\in\ocsG}\frac{1}{6}\setn{5,\,5,\,5,\,5,\,5,\,5}
    && \text{by definition of \structe{discrete metric} \xref{def:dmetric}}
  \\&= \setn{\diceA,\diceB,\diceC,\diceD,\diceE,\diceF}
    && \text{by definition of $\ocsG$}
  \\
  \ocsceng(\ocsG)
    &\eqd \argmin_{x\in\ocsG}\prod_{y\in\ocsG\setd\setn{x}}\metric{x}{y}^{\psp(y)}
    &&\text{by definition of $\ocsceng$ \xref{def:ocsceng}}
  \\&= \argmin_{x\in\ocsG}\prod_{y\in\ocsG\setd\setn{x}}1^{\frac{1}{6}}
    && \text{by definition of $\ocsG$ and \structe{discrete metric} \xref{def:dmetric}}
  \\&= \argmin_{x\in\ocsG}\setn{1,1,1,1,1,1}
  \\&= \setn{\diceA,\diceB,\diceC,\diceD,\diceE,\diceF}
    && \text{by definition of $\ocsG$}
  \\
  \ocscenh(\rvX)
    &\eqd \argmin_{x\in\ocsG}\brp{\sum_{y\in\ocsG\setd\setn{x}}\frac{1}{\metric{x}{y}}\psp(y)}^{-1}
    &&\text{by definition of $\ocscenh$ \xref{def:ocscenh}}
  \\&= \argmin_{x\in\ocsG}\brp{\sum_{y\in\ocsG\setd\setn{x}}1\times\frac{1}{6}}^{-1}
    &&\text{by definition of $\ocsG$}
  %\\&= \argmin_{x\in\ocsG}\brp{\sum_{y\in\ocsG\setd\setn{x}}1\times\frac{1}{6}}^{-1}
  \\&= \argmin_{x\in\ocsG}\setn{\frac{6}{5},\frac{6}{5},\frac{6}{5},\frac{6}{5},\frac{6}{5},\frac{6}{5}}
  \\&= \setn{\diceA,\diceB,\diceC,\diceD,\diceE,\diceF}
    && \text{by definition of $\ocsG$}
  \\
  \ocscenm(\ocsG)
    &\eqd \argmin_{x\in\ocsG}\min_{y\in\ocsG\setd\setn{x}}\metric{x}{y}\psp(y)
    &&\text{by definition of $\ocscenm$ \xref{def:ocscenm}}
  \\&= \argmin_{x\in\ocsG}\min_{y\in\ocsG\setd\setn{x}}1\times\frac{1}{6}
    &&\text{by definition of $\ocsG$ and \structe{discrete metric} \xref{def:dmetric}}
  \\&= \argmin_{x\in\ocsG}\frac{1}{6}\setn{1,1,1,1,1,1}
  \\&= \setn{\diceA,\diceB,\diceC,\diceD,\diceE,\diceF}
    && \text{by definition of $\ocsG$}
  \\
  \ocscenM(\ocsG)
    &\eqd \argmax_{x\in\ocsG}\min_{y\in\ocsG\setd\setn{x}}\metric{x}{y}\psp(y)
    &&\text{by definition of $\ocscenM$ \xref{def:ocscenM}}
  \\&= \argmax_{x\in\ocsG}\frac{1}{6}\setn{1,1,1,1,1,1}
    && \text{by $\ocscenm(\ocsG)$ result}
  \\&= \setn{\diceA,\diceB,\diceC,\diceD,\diceE,\diceF}
    && \text{by definition of $\ocsG$}
  \\
  \ocsVaro(\ocsG;\ocscena)
    &= \mathrlap{\ocsVaro(\ocsG;\ocsceng)= \ocsVaro(\ocsG;\ocscenh)= \ocsVaro(\ocsG;\ocscenm)= \ocsVaro(\ocsG;\ocscenM) = \ocsVaro(\ocsG)}
  \\&\eqd \sum_{x\in\ocsG}\brs{\metric{\ocscen(\ocsG)}{x}}^2\psp(x)
    && \text{by definition of $\ocsVaro$ \xref{def:ocsVarG}}
  \\&= \sum_{x\in\ocsG}\brp{0^2}\frac{1}{6}
    && \text{because $\ocscen(\ocsG)=\ocsG$}
  \\&= 0
    && \text{by field property of \vale{additive identity element} $0$}
\end{align*}
\end{proof}


%---------------------------------------
\begin{example} %[\exmd{real die outcome subspace}]
\label{ex:realdie}%\mbox{}\\
%---------------------------------------
%\begin{minipage}{\tw-35mm}%
Let $\ocsG\eqd\ocs{\setn{\diceA,\diceB,\diceC,\diceD,\diceE,\diceF}}{\emptyset}{\orel}{\psp}$ 
be the \structe{real die outcome subspace} \xref{def:rdie}.
%be the \structe{outcome subspace} \xref{def:ocsm} generated by a \structe{real die}.
%On a real die (as opposed to a \structe{fair die} \xrefnp{ex:fairdie}),
%some symbols (die faces) are physically closer (in real 3 dimensional space) than others.
%Instead of using the discrete metric, we define the distance $\metric{x}{y}$ from face $x$ to face $y$
%to be the number of physical die edges that must be crossed to go from $x$ to $y$.
%In this case, it is still true that $\metric{\diceA}{\diceA}=0$, and
%\\\indentx$\begin{array}{rclclclclM}
%  \metric{\diceA}{\diceB}&=&\metric{\diceA}{\diceC}&=&\metric{\diceA}{\diceD}&=&\metric{\diceA}{\diceE}&=&1,&but\\
%  \metric{\diceA}{\diceF}&=&\metric{\diceB}{\diceE}&=&\metric{\diceC}{\diceD}&=&2&\neq& 1.
%\end{array}$
%\end{minipage}\hfill%
%\begin{tabular}{c}
%  \gsize%
%  %\psset{unit=5mm}%
%  %{%============================================================================
% Daniel J. Greenhoe
% LaTeX file
% discrete metric real dice mapping to linearly ordered O6c
%============================================================================
{%\psset{unit=0.5\psunit}%
\begin{pspicture}(-1.4,-1.4)(1.4,1.4)%
  %---------------------------------
  % options
  %---------------------------------
  \psset{%
    radius=1.25ex,
    labelsep=2.5mm,
    linecolor=blue,%
    }%
  %---------------------------------
  % dice graph
  %---------------------------------
  \rput(0,0){%\psset{unit=2\psunit}%
    \uput{1}[210](0,0){\Cnode[fillstyle=solid,fillcolor=snode](0,0){D4}}%
    \uput{1}[150](0,0){\Cnode[fillstyle=solid,fillcolor=snode](0,0){D5}}%
    \uput{1}[ 90](0,0){\Cnode[fillstyle=solid,fillcolor=snode](0,0){D6}}%
    \uput{1}[ 30](0,0){\Cnode[fillstyle=solid,fillcolor=snode](0,0){D3}}%
    \uput{1}[-30](0,0){\Cnode[fillstyle=solid,fillcolor=snode](0,0){D2}}%
    \uput{1}[-90](0,0){\Cnode[fillstyle=solid,fillcolor=snode](0,0){D1}}%
    }
  \rput(D6){$\diceF$}%
  \rput(D5){$\diceE$}%
  \rput(D4){$\diceD$}%
  \rput(D3){$\diceC$}%
  \rput(D2){$\diceB$}%
  \rput(D1){$\diceA$}%
  %
  \ncline{D5}{D6}%
  \ncline{D4}{D5}\ncline{D4}{D6}%
  \ncline{D3}{D5}\ncline{D3}{D6}%
  \ncline{D2}{D3}\ncline{D2}{D4}\ncline{D2}{D6}%
  \ncline{D1}{D2}\ncline{D1}{D3}\ncline{D1}{D4}\ncline{D1}{D5}%
  %
  \uput[158](D6){$\frac{1}{6}$}
  \uput[150](D5){$\frac{1}{6}$}
  \uput[210](D4){$\frac{1}{6}$}
  \uput[ 22](D3){$\frac{1}{6}$}
  \uput[-45](D2){$\frac{1}{6}$}
  \uput[-158](D1){$\frac{1}{6}$}
\end{pspicture}
}%}%
%  {\includegraphics{../common/math/graphics/pdfs/ocs_rdie.pdf}}%
%\end{tabular}
%\\
This structure is illustrated by the \structe{weighted graph} \xref{def:wgraph} in \prefpp{fig:ocs} (B), %to the right, 
where each line segment represents a distance of $1$.
The structure has the following geometric values:
\\\indentx$\begin{array}{rclDD}
  \ocscen(\ocsG)=\ocscena(\ocsG)=\ocsceng(\ocsG)=\ocscenh(\ocsG)=\ocscenm(\ocsG)=\ocscenM(\ocsG)  
    &=& \setn{\diceA,\diceB,\diceC,\diceD,\diceE,\diceF} &%\xxref{def:ocscen}{def:ocscenx}  & (shaded in illustration)
    \\
  \ocsVaro(\ocsG) = \ocsVaro(\ocsG;\ocscena) = \ocsVaro(\ocsG;\ocsceng) = \ocsVaro(\ocsG;\ocscenh) = \ocsVaro(\ocsG;\ocscenm) = \ocsVaro(\ocsG;\ocscenM) &=& 0
\end{array}$
\end{example}
\begin{proof}
\begin{align*}
  \ocscen(\ocsG)
    &\eqd \argmin_{x\in\ocsG}\max_{y\in\ocsG}\metric{x}{y}\psp(y)
    && \text{by definition of $\ocscen$ \xref{def:ocscen}}
  \\&= \argmin_{x\in\ocsG}\max_{y\in\ocsG}\metric{x}{y}\frac{1}{6}
    && \text{by definition of $\ocsG$}
  %\\&= \argmin_{x\in\ocsG}\max_{y\in\ocsG}\metric{x}{y}
  %  && \text{because $\ff(x)=\frac{1}{6}x$ is \prope{strictly isotone} and by \prefpp{lem:argminmaxphi}}
  \\&= \argmin_{x\in\ocsG}\frac{1}{6}\setn{2,\,2,\,2,\,2,\,2,\,2}
    && \text{because for each $x$, there is a $y$ such that $\metric{x}{y}=2$}
  \\&= \setn{\diceA,\diceB,\diceC,\diceD,\diceE,\diceF}
    && \text{by definition of $\ocsG$}
  \\
  \ocscena(\ocsG)
    &\eqd \argmin_{x\in\ocsG}\sum_{y\in\ocsG}\metric{x}{y}\psp(y)
    &&\text{by definition of $\ocscena$ \xref{def:ocscenx}}
  \\&= \argmin_{x\in\ocsG}\sum_{y\in\ocsG}\metric{x}{y}\frac{1}{6}
    &&\text{by definition of $\ocsG$}
  %\\&= \argmin_{x\in\ocsG}\sum_{y\in\ocsG}\metric{x}{y}
  %  &&\text{because $\ff(x)=\frac{1}{6}x$ is \prope{strictly isotone} and by \prefpp{lem:argminmaxphi}}
  \\&= \mathrlap{\argmin_{x\in\ocsG}\frac{1}{6}
         \setn{\begin{array}{*{11}{@{\hspace{2pt}}c}}
           0 &+& 1 &+& 1 &+& 1 &+& 1 &+& 2\\
           1 &+& 0 &+& 1 &+& 1 &+& 2 &+& 1\\
           1 &+& 1 &+& 0 &+& 2 &+& 1 &+& 1\\
           1 &+& 1 &+& 2 &+& 0 &+& 1 &+& 1\\
           1 &+& 2 &+& 1 &+& 1 &+& 0 &+& 1\\
           2 &+& 1 &+& 1 &+& 1 &+& 1 &+& 0
         \end{array}}
       = \argmin_{x\in\ocsG}\frac{1}{6}
         \setn{\begin{array}{c}
           6\\
           6\\
           6\\
           6\\
           6\\
           6
         \end{array}}
       = \setn{\begin{array}{c}
           \diceA\\
           \diceB\\
           \diceC\\
           \diceD\\
           \diceE\\
           \diceF
         \end{array}}}
    \\
  \ocsceng(\ocsG)
    &\eqd \argmin_{x\in\ocsG}\prod_{y\in\ocsG\setd\setn{x}}\metric{x}{y}^{\psp(y)}
    &&\text{by definition of $\ocsceng$ \xref{def:ocsceng}}
  \\&\eqd \argmin_{x\in\ocsG}\prod_{y\in\ocsG\setd\setn{x}}\metric{x}{y}^{\frac{1}{6}}
    &&\text{by definition of $\ocsG$}
  \\&= \argmin_{x\in\ocsG}\brp{\prod_{y\in\ocsG\setd\setn{x}}\metric{x}{y}}^{\frac{1}{6}}
  \\&= \argmin_{x\in\ocsG}\prod_{y\in\ocsG\setd\setn{x}}\metric{x}{y}
    &&\text{because $\ff(x)=x^\frac{1}{6}$ is \prope{strictly isotone} and by \prefpp{lem:argminphi}}
  \\&= \argmin_{x\in\ocsG}\setn{2,2,2,2,2,2}
  \\&= \setn{\diceA,\diceB,\diceC,\diceD,\diceE,\diceF}
  \\
  \ocscenh(\rvX)
    &\eqd \argmin_{x\in\ocsG}\brp{\sum_{y\in\ocsG\setd\setn{x}}\frac{1}{\metric{x}{y}}\psp(y)}^{-1}
    &&\text{by definition of $\ocscenh$ \xref{def:ocscenh}}
  \\&= \argmin_{x\in\ocsG}\brp{\sum_{y\in\ocsG\setd\setn{x}}\frac{1}{\metric{x}{y}}\frac{1}{6}}^{-1}
    && \text{by definition of $\ocsG$}
  \\&= \argmin_{x\in\ocsG}6\brp{\sum_{y\in\ocsG\setd\setn{x}}\frac{1}{\metric{x}{y}}}^{-1}
  \\&= \argmax_{x\in\ocsG}\frac{1}{2}6\sum_{y\in\ocsG\setd\setn{x}}\frac{2}{\metric{x}{y}}
    && \text{because $\fphi(x)\eqd x^{-1}$ is \prope{strictly antitone} and by \prefp{lem:minphia}}
  \\&= \mathrlap{\argmin_{x\in\ocsG}3
         \setn{\begin{array}{*{11}{c}}
           0 &+& 2 &+& 2 &+& 2 &+& 2 &+& 1\\
           2 &+& 0 &+& 2 &+& 2 &+& 1 &+& 2\\
           2 &+& 2 &+& 0 &+& 1 &+& 2 &+& 2\\
           2 &+& 2 &+& 1 &+& 0 &+& 2 &+& 2\\
           2 &+& 1 &+& 2 &+& 2 &+& 0 &+& 2\\
           1 &+& 2 &+& 2 &+& 2 &+& 2 &+& 0
         \end{array}}
       = \argmin_{x\in\ocsG}\frac{1}{6}
         \setn{\begin{array}{c}
           9\\
           9\\
           9\\
           9\\
           9\\
           9
         \end{array}}
       = \setn{\begin{array}{c}
           \diceA\\
           \diceB\\
           \diceC\\
           \diceD\\
           \diceE\\
           \diceF
         \end{array}}}
  \\
  \ocscenm(\ocsG)
    &\eqd \argmin_{x\in\ocsG}\min_{y\in\ocsG\setd\setn{x}}\metric{x}{y}\psp(y)
    &&\text{by definition of $\ocscenm$ \xref{def:ocscenm}}
  \\&= \argmin_{x\in\ocsG}\min_{y\in\ocsG\setd\setn{x}}1\times\frac{1}{6}
    &&\text{by definition of $\ocscenm$ \xref{def:ocscen}}
  \\&= \frac{1}{6}\setn{2,2,2,2,2,2}
  \\&= \setn{\diceA,\diceB,\diceC,\diceD,\diceE,\diceF}
    && \text{by definition of $\ocsG$}
  \\
  \ocscenM(\ocsG)
    &\eqd \argmax_{x\in\ocsG}\min_{y\in\ocsG\setd\setn{x}}\metric{x}{y}\psp(y)
    &&\text{by definition of $\ocscenM$ \xref{def:ocscenM}}
  \\&= \frac{1}{6}\setn{2,2,2,2,2,2}
    && \text{by $\ocscenm(\ocsG)$ result}
  \\&= \setn{\diceA,\diceB,\diceC,\diceD,\diceE,\diceF}
    && \text{by definition of $\ocsG$}
  \\
  \ocsVaro(\ocsG;\ocscena)
    &= \mathrlap{\ocsVaro(\ocsG;\ocsceng)= \ocsVaro(\ocsG;\ocscenh)= \ocsVaro(\ocsG;\ocscenm)= \ocsVaro(\ocsG;\ocscenM) = \ocsVaro(\ocsG)}
  \\&\eqd \sum_{x\in\ocsG}\brs{\metric{\ocscen(\ocsG)}{x}}^2\psp(x)
    && \text{by definition of $\ocsVaro$ \xref{def:ocsVarG}}
  \\&= \sum_{x\in\ocsG}(0)\frac{1}{6}
    && \text{because $\ocscen(\ocsG)=\ocsG$}
  \\&= 0
    && \text{by field property of \vale{additive identity element} $0$}
\end{align*}
\end{proof}


%---------------------------------------
\begin{remark}
\label{rem:dietop}
\label{rem:mgeo}
%---------------------------------------
Let $\ocsG\eqd\ocs{\ocso}{\metricn} {\orel}{\psp}$ be the \structe{fair die outcome subspace} \xref{ex:fairdie}.
Let $\ocsH\eqd\ocs{\ocso}{\metrican}{\orel}{\psp}$ be the \structe{real die outcome subspace} \xref{ex:realdie}.
These two subspaces are identical except for their metrics $\metricn$ and $\metrican$.
So we can say that $\ocsG$ and $\ocsH$ \emph{are} distinguished by their metrics.
However, note that they are \emph{indistinguishable} by the topologies induced by their metrics,
because they both induce the same topology---the \structe{discrete topology} $\pset{\ocso}$ \xref{def:pset}.
That is, the geometric distinction provided in metric spaces is in general lost in topological spaces.
Thus, topological spaces are arguably too general for the type of stochastic processing presented in this paper;
rather, the stochastic processing discussed in this paper calls for metric space structure.
And in this paper, this type of metric space structure is referred to as \structd{metric geometry}.
\end{remark}
\begin{proof}
\begin{enumerate}
  \item Every \structe{metric space} $\opair{\ocso}{\metricn}$ \xref{def:metric} 
        induces a \structd{topological space} $\opair{\ocso}{\topT}$.
  \item In particular, a \structe{metric} $\metricn$ induces an \structd{open ball} 
        $\ball{x}{r}\in\clF{\brp{\ocso\times\Rp}}{\brp{\pset{\ocso}}}$ centered at $x$ with radius $r$ such that 
        $\ball{x}{r}\eqd\set{y\in\ocso}{\metric{x}{y}<r}$.
  \item At each outcome $x$ in $\ocsG$, only two \structe{open ball}s are possible:
        $\ball{x}{r}=\brb{\begin{array}{lM}
                            \setn{x}  & for $0<r\le1$\\
                            \ocso     & for $r>1$
                          \end{array}}$.
  \item Let $x'$ represent the die face which, 
        when its numeric value is summed ``in the usual way" with the numeric value of the die face $x$,
        equals $7$.
        Then at each point $x$ in $\ocsH$, three \structe{open ball}s are possible:\\
        $\ball{x}{r}=\brb{\begin{array}{lM}
                             \setn{x}     & for $0<r\le1$\\
                             \setn{x,x'}  & for $1<r\le2$\\
                             \ocso        & for $r>2$
                           \end{array}}$.

  \item The \structe{open ball}s of $\opair{\ocso}{\metricn}$ or $\opair{\ocso}{\metrican}$ 
        in turn induce a \structd{base} for a \structd{topology} $\topT$, such that
        \\$\topT=\set{\setU\in\pset{\ocso}}{\text{$\setU$ is a union of \structe{open ball}s}}$.
        The topology induced by $\ocsG$ is the \structe{discrete topology} $\pset{\ocso}$ \xref{def:pset}.
        The topology induced by $\ocsH$ is also the \structe{discrete topology} $\pset{\ocso}$.

  \item So the metrics of $\ocsG$ and $\ocsH$ are different.
        And the balls induced by $\ocsG$ and those induced by $\ocsH$ are different.
        However, the topologies induced by $\ocsG$ and $\ocsH$ are the same.
        %they both induce the \structe{discrete topology} $\pset{\ocso}$.
\end{enumerate}
\end{proof}


%---------------------------------------
\begin{example} %[\exmd{weighted real die outcome subspace}]
\label{ex:wdie}
%---------------------------------------
%A \structe{weighted die} generates an \structe{outcome subspace} $\ocsG$ \xref{def:ocs}.
The \structe{weighted real die outcome subspace} \xref{def:wdie} illustrated in \prefp{fig:ocs} (C) 
has the following geometric characteristics:
\\ %\begin{minipage}{\tw-35mm}%
\indentx$\begin{array}{rcl @{\qquad} lclcl}
    \ocscen(\ocsG)=\ocscena(\ocsG) &=&\setn{\diceD}                      & \ocsVaro(\ocsG)          &=&\frac{101}{300} &\approx& 0.33\\
    \ocsceng(\ocsG)                &=&\setn{\diceB}                      & \ocsVaro(\ocsG;\ocsceng) &=&\frac{127}{150} &\approx& 0.847 \\ 
    \ocscenh(\ocsG)                &=&\setn{\diceE}                      & \ocsVaro(\ocsG;\ocscenh) &=&\frac{145}{150} &\approx& 0.967 \\
    \ocscenm(\ocsG)                &=&\setn{\diceA,\diceC,\diceD,\diceF} & \ocsVaro(\ocsG;\ocscenm) &=&\frac{ 11}{ 50} &=&       0.22 \\
    \ocscenM(\ocsG)                &=&\setn{\diceB,\diceE}               & \ocsVaro(\ocsG;\ocscenM) &=&\frac{ 11}{ 50} &\approx& 0.767
  \end{array}$.
%\end{minipage}\hfill%
%\begin{minipage}{30mm}
%  \gsize%
%  %\psset{unit=8mm}%
%  \centering%
%  %{%============================================================================
% Daniel J. Greenhoe
% LaTeX file
% discrete metric real dice mapping to linearly ordered O6c
%============================================================================
\begin{pspicture}(-1.5,-1.5)(1.5,1.5)%
  %---------------------------------
  % settings
  %---------------------------------
  \psset{%
    radius=1.25ex,
    labelsep=2.5mm,
    %linecolor=blue,%
    }%
  %---------------------------------
  % dice graph nodes
  %---------------------------------
  \rput(0,0){%
    \uput{1}[210](0,0){\Cnode[fillstyle=solid,fillcolor=snode](0,0){D4}}%
    \uput{1}[150](0,0){\Cnode(0,0){D5}}%
    \uput{1}[ 90](0,0){\Cnode(0,0){D6}}%
    \uput{1}[ 30](0,0){\Cnode(0,0){D3}}%
    \uput{1}[-30](0,0){\Cnode(0,0){D2}}%
    \uput{1}[-90](0,0){\Cnode(0,0){D1}}%
    }%
  %-------------------------------------
  % graph node labels
  %-------------------------------------
  \rput(D6){$\diceF$}%
  \rput(D5){$\diceE$}%
  \rput(D4){$\diceD$}%
  \rput(D3){$\diceC$}%
  \rput(D2){$\diceB$}%
  \rput(D1){$\diceA$}%
  %-------------------------------------
  % graph edges
  %-------------------------------------
  \ncline{D5}{D6}%
  \ncline{D4}{D5}\ncline{D4}{D6}%
  \ncline{D3}{D5}\ncline{D3}{D6}%
  \ncline{D2}{D3}\ncline{D2}{D4}\ncline{D2}{D6}%
  \ncline{D1}{D2}\ncline{D1}{D3}\ncline{D1}{D4}\ncline{D1}{D5}%
  %\ncline{D3}{D4}%
  %\ncline{D2}{D5}%
  %\ncline{D1}{D6}%
  %-------------------------------------
  % labels
  %-------------------------------------
  \uput[ 158](D6){$\frac{1}{30}$}
  \uput[ 150](D5){$\frac{1}{50}$}
  \uput[ 210](D4){$\frac{3}{5}$}
  \uput[  22](D3){$\frac{1}{30}$}
  \uput[ -45](D2){$\frac{1}{20}$}
  \uput[-158](D1){$\frac{1}{10}$}
\end{pspicture}%}%
%  {\includegraphics{../common/math/graphics/pdfs/ocs_wdie.pdf}}%
%\end{minipage}
\\
Note that the \fncte{outcome center} $\ocscen(\ocsG)$ and \fncte{arithmetic center} $\ocscena(\ocsG)$ 
again yield identical results.
Also note that of the four center measures of cardinality $1$
($\seto{\ocscen(\ocsG)}=\seto{\ocscena(\ocsG)}=\seto{\ocsceng(\ocsG)}=\seto{\ocscenh(\ocsG)}=1$ \xrefnp{def:seto}),
$\ocscen$ and $\ocscena$ yield by far the lowest variance measures.
\end{example}
\begin{proof}
\begin{align*}
  \ocscen(\ocsG)
    &\eqd \argmin_{x\in\ocsG}\max_{y\in\ocsG}\metric{x}{y}\psp(y)
    &&\text{by definition of $\ocscen$ \xref{def:ocscen}}
  \\&\eqd \argmin_{x\in\ocsG}\max_{y\in\ocsG}\frac{1}{300}\metric{x}{y}\psp(y)300
  \\&\eqd \argmin_{x\in\ocsG}\max_{y\in\ocsG}\metric{x}{y}300\psp(y)
   %&&\text{because $\ff(x)=\frac{1}{300}x$ is \prope{strictly isotone}  and by \prefpp{lem:argminmaxphi}}
   %&&\text{because $\ff(x)=\frac{1}{300}x$ is \prope{strictly isotone}}
    &&\text{by \prefpp{lem:argminmaxphi}}
  \\&=\mathrlap{\argmin_{x\in\ocsG}\max_{y\in\ocsG}300
    \setn{\begin{array}{ccccc}
      \metricn(\diceA,\diceA)\psp(\diceA) &\metricn(\diceA,\diceB)\psp(\diceB) &\metricn(\diceA,\diceC)\psp(\diceC) & \cdots & \metricn(\diceA,\diceF)\psp(\diceF) \\
      \metricn(\diceB,\diceA)\psp(\diceA) &\metricn(\diceB,\diceB)\psp(\diceB) &\metricn(\diceB,\diceC)\psp(\diceC) & \cdots & \metricn(\diceB,\diceF)\psp(\diceF) \\
      \vdots                              &\vdots                              &\vdots                              & \ddots & \vdots                              \\
      \metricn(\diceF,\diceA)\psp(\diceA) &\metricn(\diceF,\diceB)\psp(\diceB) &\metricn(\diceF,\diceC)\psp(\diceC) & \cdots & \metricn(\diceF,\diceF)\psp(\diceF) 
    \end{array}}}
   \\&=\mathrlap{\argmin_{x\in\ocsG}\max_{y\in\ocsG}
         \setn{\begin{array}{cccccc}
           0\times30 & 1\times15 & 1\times10 & 1\times180 & 1\times6 & 2\times10 \\
           1\times30 & 0\times15 & 1\times10 & 1\times180 & 2\times6 & 1\times10 \\
           1\times30 & 1\times15 & 0\times10 & 2\times180 & 1\times6 & 1\times10 \\
           1\times30 & 1\times15 & 2\times10 & 0\times180 & 1\times6 & 1\times10 \\
           1\times30 & 2\times15 & 1\times10 & 1\times180 & 0\times6 & 1\times10 \\
           2\times30 & 1\times15 & 1\times10 & 1\times180 & 1\times6 & 0\times10 \\
         \end{array}}
     %\\&=\argmin_{x\in\ocsG}\max_{y\in\ocsG}
     %    \setn{\begin{array}{cccccc}
     %       0 & 15 & 10 & 180 &  6 & 20\\
     %      30 &  0 & 10 & 180 & 12 & 10\\
     %      30 & 15 &  0 & 360 &  6 & 10\\
     %      30 & 15 & 20 &   0 &  6 & 10\\
     %      30 & 30 & 10 & 180 &  0 & 10\\
     %      60 & 15 & 10 & 180 &  6 & 0
     %    \end{array}}
       = \argmin_{x\in\ocsG}
         \setn{\begin{array}{c}
           180 \\
           180 \\
           360 \\
            30 \\
           180 \\
           180 \\
         \end{array}}
       = \setn{\begin{array}{c}
           \mbox{ } \\
           \mbox{ } \\
           \mbox{ } \\
           \diceD \\
           \mbox{ } \\
           \mbox{ } \\
         \end{array}}}
  \\
  \ocscena(\ocsG)
    &\eqd \argmin_{x\in\ocsG}\sum_{y\in\ocsG}\metric{x}{y}\psp(y)
    &&\text{by definition of $\ocscena$ \xref{def:ocscena}}
  \\&= \argmin_{x\in\ocsG}\frac{1}{300}\sum_{y\in\ocsG}\metric{x}{y}\psp(y)300
    &&\text{by definition of $\ocsG$}
  \\&= \argmin_{x\in\ocsG}\sum_{y\in\ocsG}\metric{x}{y}\psp(y)300
    %&&\text{because $\ff(x)=\frac{1}{300}x$ is \prope{strictly isotone} and by \prefpp{lem:argminmaxphi}}
    %&&\text{because $\ff(x)=\frac{1}{300}x$ is \prope{strictly isotone}}
    &&\text{by \prefpp{lem:argminphi}}
  \\&=\mathrlap{\argmin_{x\in\ocsG}
         \setn{\begin{array}{*{11}{r}}
            0 &+& 15 &+& 10 &+& 180 &+&  6 &+& 20\\
           30 &+&  0 &+& 10 &+& 180 &+& 12 &+& 10\\
           30 &+& 15 &+&  0 &+& 360 &+&  6 &+& 10\\
           30 &+& 15 &+& 20 &+&   0 &+&  6 &+& 10\\
           30 &+& 30 &+& 10 &+& 180 &+&  0 &+& 10\\
           60 &+& 15 &+& 10 &+& 180 &+&  6 &+& 0     
         \end{array}}
       = \argmin_{x\in\ocsG}
         \setn{\begin{array}{r}
           231\\
           242\\
           421\\
            81\\
           260\\
           271
         \end{array}}
       = \setn{\begin{array}{c}
           \mbox{ } \\
           \mbox{ } \\
           \mbox{ } \\
           \diceD \\
           \mbox{ } \\
           \mbox{ } \\
         \end{array}}}
  \\
  \ocsceng(\ocsG)
    &\eqd \argmin_{x\in\ocsG}\prod_{y\in\ocsG\setd\setn{x}}\brs{\metric{x}{y}^{\psp(y)}}
    &&\text{by definition of $\ocsceng$ \xref{def:ocsceng}}
  \\&= \argmin_{x\in\ocsG}\prod_{y\in\ocsG\setd\setn{x}}\brs{\metric{x}{y}^{300\psp(y)\frac{1}{300}}}
  \\&= \argmin_{x\in\ocsG}\brp{\prod_{y\in\ocsG\setd\setn{x}}\brs{\metric{x}{y}^{300\psp(y)}}}^\frac{1}{300}
  \\&= \argmin_{x\in\ocsG}\prod_{y\in\ocsG\setd\setn{x}}\brs{\metric{x}{y}^{300\psp(y)}}
    %&&\text{because $\ff(x)=\frac{1}{300}x$ is \prope{strictly isotone} and by \prefpp{lem:argminmaxphi}}
    %&&\text{because $\ff(x)=x^\frac{1}{300}$ is \prope{strictly isotone}}
    &&\text{by \prefpp{lem:argminphi}}
  \\&=\mathrlap{\argmin_{x\in\ocsG}
         \setn{\begin{array}{*{11}{c}}
                  &\times& 1^{15} &\times& 1^{10} &\times& 1^{180} &\times& 1^{6} &\times& 2^{10} \\
           1^{30} &\times&        &\times& 1^{10} &\times& 1^{180} &\times& 2^{6} &\times& 1^{10} \\
           1^{30} &\times& 1^{15} &\times&        &\times& 2^{180} &\times& 1^{6} &\times& 1^{10} \\
           1^{30} &\times& 1^{15} &\times& 2^{10} &\times&         &\times& 1^{6} &\times& 1^{10} \\
           1^{30} &\times& 2^{15} &\times& 1^{10} &\times& 1^{180} &\times&       &\times& 1^{10} \\
           2^{30} &\times& 1^{15} &\times& 1^{10} &\times& 1^{180} &\times& 1^{6} &      &           
         \end{array}}
       = \argmin_{x\in\ocsG}
         \setn{\begin{array}{l}
           2^{10}\\
           2^{6 }\\
           2^{180}\\
           2^{10}\\
           2^{15}\\
           2^{30}
         \end{array}}
       = \setn{\begin{array}{c}
           \mbox{ } \\
           \diceB{ } \\
           \mbox{ } \\
           \mbox{ } \\
           \mbox{ } \\
           \mbox{ } \\
         \end{array}}}
  \\
  \ocscenh(\ocsG)
    &\eqd \argmin_{x\in\ocsG}\brp{\sum_{y\in\ocsG\setd\setn{x}}\frac{1}{\metric{x}{y}}\psp(y)}^{-1}
    &&\text{by definition of $\ocscenh$ \xref{def:ocscenh}}
  \\&= \argmax_{x\in\ocsG}\sum_{y\in\ocsG\setd\setn{x}}\frac{1}{\metric{x}{y}}\psp(y)
    && \text{by \prefp{lem:minphi}}
   %&& \text{because $\fphi(x)\eqd x^{-1}$ is \prope{strictly antitone} and by \prefp{lem:minphia}}
  \\&= \argmax_{x\in\ocsG}\frac{1}{300}\sum_{y\in\ocsG\setd\setn{x}}\frac{1}{\metric{x}{y}}300\psp(y)
  \\&= \argmax_{x\in\ocsG}\sum_{y\in\ocsG\setd\setn{x}}\frac{300\psp(y)}{\metric{x}{y}}
    %&&\text{because $\ff(x)=\frac{1}{300}x$ is \prope{strictly isotone} and by \prefpp{lem:argminmaxphi}}
    %&&\text{because $\ff(x)=\frac{1}{300}x$ is \prope{strictly isotone}}
    &&\text{by \prefpp{lem:argminphi}}
  \\&=\mathrlap{\argmax_{x\in\ocsG}
         %\setn\begin{array}{*{11}{@{\,}c}}
         \setn{\begin{array}{*{11}{c}}
                        &+& \frac{15}{1} &+& \frac{10}{1} &+& \frac{180}{1} &+& \frac{6}{1} &+& \frac{10}{2} \\
           \frac{30}{1} &+&              &+& \frac{10}{1} &+& \frac{180}{1} &+& \frac{6}{2} &+& \frac{10}{1} \\
           \frac{30}{1} &+& \frac{15}{1} &+&              &+& \frac{180}{2} &+& \frac{6}{1} &+& \frac{10}{1} \\
           \frac{30}{1} &+& \frac{15}{1} &+& \frac{10}{2} &+&               &+& \frac{6}{1} &+& \frac{10}{1} \\
           \frac{30}{1} &+& \frac{15}{2} &+& \frac{10}{1} &+& \frac{180}{1} &+&             &+& \frac{10}{1} \\
           \frac{30}{2} &+& \frac{15}{1} &+& \frac{10}{1} &+& \frac{180}{1} &+& \frac{6}{1} &+&                     
         \end{array}}
       = \argmax_{x\in\ocsG}
         \setn{\begin{array}{r}
           216.0\\
           233.0\\
           151.0\\
            66.0\\
           237.5\\
           226.0
         \end{array}}
       = \setn{\begin{array}{c}
           \mbox{ } \\
           \mbox{ } \\
           \mbox{ } \\
           \mbox{ } \\
           \diceE \\
           \mbox{ }
         \end{array}}}
  \\
  \ocscenm(\ocsG)
    &\eqd \argmin_{x\in\ocsG}\min_{y\in\ocsG}\metric{x}{y}\psp(y)
    &&\text{by definition of $\ocscenm$ \xref{def:ocscenm}}
     \\&=\mathrlap{\argmin_{x\in\ocsG}\min_{y\in\ocsG\setd\setn{x}}
         \setn{\begin{array}{rrrrrr}
            0 & 15 & 10 & 180 &  6 & 20\\
           30 &  0 & 10 & 180 & 12 & 10\\
           30 & 15 &  0 & 360 &  6 & 10\\
           30 & 15 & 20 &   0 &  6 & 10\\
           30 & 30 & 10 & 180 &  0 & 10\\
           60 & 15 & 10 & 180 &  6 & 0
         \end{array}}
       = \argmin_{x\in\ocsG}
         \setn{\begin{array}{r}
            6\\
           10\\
            6\\
            6\\
           10\\
            6\\
         \end{array}}
       = \setn{\begin{array}{r}
           \diceA\\
                 \\
           \diceC\\
           \diceD\\
                 \\
           \diceF  
         \end{array}}}
  \\
  \ocscenM(\ocsG)
    &\eqd \argmax_{x\in\ocsG}\min_{y\in\ocsG}\metric{x}{y}\psp(y)
    &&\text{by definition of $\ocscenM$ \xref{def:ocscenM}}
  \\&= \argmax_{x\in\ocsG}\setn{6,10,6,6,10,6}
    && \text{by $\ocscenm(\ocsG)$ result}
  \\&= \setn{\diceB,\diceE}
    && \text{by definition of $\ocsG$}
  \\
  \ocsVaro(\ocsG)
    &\eqd \sum_{x\in\ocsG}\metricsq{\ocscen(\ocsG)}{x} \psp(x)
    &&\text{by definition of $\ocsVaro$ \xref{def:ocsVarG}}
  \\&= \sum_{x\in\ocsG}\metricsq{\diceD}{x} \psp(x)
    &&\text{by $\ocscen(\ocsG)$ result}
  \\&= 1^2\times\frac{1}{10} + 1^2\times\frac{1}{20} + 2^2\times\frac{1}{30} + 
       0^2\times\frac{3}{5}  + 1^2\times\frac{1}{50} + 1^2\times\frac{1}{30}  
  \\&= \frac{1}{300}(30+15+40+0+6+10)
     = \frac{101}{300}
     \approx 0.337
  \\
  \ocsVaro(\ocsG;\ocsceng)
    &\eqd \sum_{x\in\ocsG}\metricsq{\ocsceng(\ocsG)}{x} \psp(x)
    &&\text{by definition of $\ocsVaro$ \xref{def:ocsVarG}}
  \\&= \sum_{x\in\ocsG}\metricsq{\diceB}{x} \psp(x)
    &&\text{by $\ocsceng(\ocsG)$ result}
  \\&= 1^2\times\frac{1}{10} + 0^2\times\frac{1}{20} + 1^2\times\frac{1}{30} + 
       1^2\times\frac{3}{5}  + 2^2\times\frac{1}{50} + 1^2\times\frac{1}{30}  
  \\&= \frac{1}{300}(30+0+10 + 180+24+10)
     = \frac{254}{300}
     = \frac{127}{150}
     \approx 0.847
  \\
  \ocsVaro(\ocsG;\ocscenh)
    &\eqd \sum_{x\in\ocsG}\metricsq{\ocscenh(\ocsG)}{x} \psp(x)
    &&\text{by definition of $\ocsVaro$ \xref{def:ocsVarG}}
  \\&= \sum_{x\in\ocsG}\metricsq{\diceE}{x} \psp(x)
    &&\text{by $\ocscenh(\ocsG)$ result}
  \\&= 1^2\times\frac{1}{10} + 2^2\times\frac{1}{20} + 1^2\times\frac{1}{30} + 
       1^2\times\frac{3}{5}  + 0^2\times\frac{1}{50} + 1^2\times\frac{1}{30}  
  \\&= \frac{1}{300}(30+60+10 + 180+0+10)
     = \frac{290}{300}
     = \frac{145}{150}
     \approx 0.967
  \\
  \ocsVaro(\ocsG;\ocscenm)
    &\eqd \sum_{x\in\ocsG}\metricsq{\ocscenm(\ocsG)}{x} \psp(x)
    &&\text{by definition of $\ocsVaro$ \xref{def:ocsVarG}}
  \\&= \sum_{x\in\ocsG}\metricsq{\diceA,\diceC,\diceD,\diceF}{x} \psp(x)
    &&\text{by $\ocscenm(\ocsG)$ result}
  \\&= 0^2\times\frac{1}{10} + 1^2\times\frac{1}{20} + 0^2\times\frac{1}{30} + 
       0^2\times\frac{3}{5}  + 1^2\times\frac{1}{50} + 0^2\times\frac{1}{30}  
  \\&= \frac{1}{300}( 0+60+ 0 +   0+6+ 0)
     = \frac{ 66}{300}
     = \frac{ 11}{ 50}
     =       0.22 
  \\
  \ocsVaro(\ocsG;\ocscenM)
    &\eqd \sum_{x\in\ocsG}\metricsq{\ocscenM(\ocsG)}{x} \psp(x)
    &&\text{by definition of $\ocsVaro$ \xref{def:ocsVarG}}
  \\&= \sum_{x\in\ocsG}\metricsq{\diceB,\diceE}{x} \psp(x)
    &&\text{by $\ocscenM(\ocsG)$ result}
  \\&= 1^2\times\frac{1}{10} + 0^2\times\frac{1}{20} + 1^2\times\frac{1}{30} + 
       1^2\times\frac{3}{5}  + 0^2\times\frac{1}{50} + 1^2\times\frac{1}{30}  
  \\&= \frac{1}{300}(30+ 0+10 + 180+0+10)
     = \frac{230}{300}
     = \frac{ 11}{ 50}
     \approx 0.767
\end{align*}
\end{proof}


%\if 0
%---------------------------------------
%\begin{minipage}{\tw-35mm}%
\begin{example}[\exmd{board game spinner outcome subspace}]
\label{ex:spinner} %\mbox{}\\
%---------------------------------------
The six value \structe{spinner outcome subspace} \xref{def:spinner} %illustrated to the right 
has the following geometric values:
\\$\begin{array}{rcl}
  \ocscen(\ocsG)=\ocscena(\ocsG)=\ocsceng(\ocsG)=\ocscenh(\ocsG)=\ocscenm(\ocsG)=\ocscenM(\ocsG) &=& \setn{1,2,3,4,5,6} \\
  \ocsVaro(\ocsG) = \ocsVaro(\ocsG;\ocscena) = \ocsVaro(\ocsG;\ocsceng) = \ocsVaro(\ocsG;\ocscenh) = \ocsVaro(\ocsG;\ocscenm) = \ocsVaro(\ocsG;\ocscenM)  &=&  0     
\end{array}$
\end{example}
%\end{minipage}\hfill%
%\begin{tabular}{c}
%  \gsize%
%  \psset{unit=8mm}%
%  \centering%
%  %{%============================================================================
% Daniel J. Greenhoe
% LaTeX file
% spinner 6 mapping to linearly ordered L6
%============================================================================
{%\psset{unit=0.5\psunit}%
\begin{pspicture}(-1.5,-1.5)(1.5,1.5)%
  %---------------------------------
  % options
  %---------------------------------
  \psset{%
    linecolor=blue,%
    radius=1.25ex,
    labelsep=2.5mm,
    }%
  %---------------------------------
  % spinner graph
  %---------------------------------
  \rput(0,0){%\psset{unit=2\psunit}%
    \uput{1}[210](0,0){\Cnode[fillstyle=solid,fillcolor=snode](0,0){D6}}%
    \uput{1}[150](0,0){\Cnode[fillstyle=solid,fillcolor=snode](0,0){D5}}%
    \uput{1}[ 90](0,0){\Cnode[fillstyle=solid,fillcolor=snode](0,0){D4}}%
    \uput{1}[ 30](0,0){\Cnode[fillstyle=solid,fillcolor=snode](0,0){D3}}%
    \uput{1}[-30](0,0){\Cnode[fillstyle=solid,fillcolor=snode](0,0){D2}}%
    \uput{1}[-90](0,0){\Cnode[fillstyle=solid,fillcolor=snode](0,0){D1}}%
    }
  \rput[-150](D6){$\circSix$}%
  \rput[ 150](D5){$\circFive$}%
  \rput[  90](D4){$\circFour$}%
  \rput[  30](D3){$\circThree$}%
  \rput[   0](D2){$\circTwo$}%
  \rput[ -90](D1){$\circOne$}%
  %
  \ncline{D6}{D1}%
  \ncline{D5}{D6}%
  \ncline{D4}{D5}%
  \ncline{D3}{D4}%
  \ncline{D2}{D3}%
  \ncline{D1}{D2}%
  %
  \uput[ 210](D6){$\frac{1}{6}$}
  \uput[ 150](D5){$\frac{1}{6}$}
  \uput[  22](D4){$\frac{1}{6}$}
  \uput[  30](D3){$\frac{1}{6}$}
  \uput[ -30](D2){$\frac{1}{6}$}
  \uput[ -22](D1){$\frac{1}{6}$}
  %
  %\uput[ 210](D6){${\scy\psp(\circSix)=}\frac{1}{6}$}
  %\uput[ 150](D5){${\scy\psp(\circFive)=}\frac{1}{6}$}
  %\uput[  22](D4){${\scy\psp(\circFour)=}\frac{1}{6}$}
  %\uput[  30](D3){${\scy\psp(\circThree)=}\frac{1}{6}$}
  %\uput[ -30](D2){${\scy\psp(\circTwo)=}\frac{1}{6}$}
  %\uput[-22](D1){${\scy\psp(\circOne)=}\frac{1}{6}$}
\end{pspicture}
}%}%
%  {\includegraphics{../common/math/graphics/pdfs/ocs_spinner.pdf}}%
%\end{tabular}
\begin{proof}
\begin{align*}
  \ocscen(\ocsG)
    &\eqd \argmin_{x\in\ocsG}\max_{y\in\ocsG}\metric{x}{y}\psp(y)
    && \text{by definition of $\ocscen$ \xref{def:ocscen}}
  \\&= \argmin_{x\in\ocsG}\max_{y\in\ocsG}\metric{x}{y}\frac{1}{6}
    && \text{by definition of $\ocsG$}
  \\&= \argmin_{x\in\ocsG}\max_{y\in\ocsG}\metric{x}{y}
    && \text{because $\ff(x)=\frac{1}{6}x$ is \prope{strictly isotone} and by \prefpp{lem:argminmaxphi}}
   \\&=\mathrlap{\argmin_{x\in\ocsG}\max_{y\in\ocsG}
        %\setn{\begin{array}{cccc}
        %  \metricn(1,1)&\metricn(1,2)&\cdots & \metricn(1,6) \\
        %  \metricn(2,1)&\metricn(2,2)&\ddots & \metricn(2,6) \\
        %  \vdots       &\ddots       &\ddots & \vdots        \\
        %  \metricn(6,1)&\metricn(6,2)&\cdots & \metricn(6,6)
        %\end{array}}
        \setn{\begin{array}{cccc}
          \metricn(1,1)&\cdots & \metricn(1,6) \\
          \metricn(2,1)&\cdots & \metricn(2,6) \\
          \metricn(3,1)&\cdots & \metricn(3,6) \\
          \metricn(4,1)&\cdots & \metricn(4,6) \\
          \metricn(5,1)&\cdots & \metricn(5,6) \\
          \metricn(6,1)&\cdots & \metricn(6,6)
        \end{array}}
   \quad=\argmin_{x\in\ocsG}\max_{y\in\ocsG}
        \setn{\begin{array}{cccccc}
           0 & 1 & 2 & 3 & 2 & 1 \\
           1 & 0 & 1 & 2 & 3 & 2 \\
           2 & 1 & 0 & 1 & 2 & 3 \\
           3 & 2 & 1 & 0 & 1 & 2 \\
           2 & 3 & 2 & 1 & 0 & 1 \\
           1 & 2 & 3 & 2 & 1 & 0 
        \end{array}}
     \quad= \argmin_{x\in\ocsG}
        \setn{\begin{array}{c}
          {3}\\
          {3}\\
          {3}\\
          {3}\\
          {3}\\
          {3}
        \end{array}}}
  %\\&= \argmin_{x\in\ocsG}\setn{3,\,3,\,3,\,3,\,3,\,3}
  %  && \text{by definition of $\metricn$ \xref{def:dmetric}}
  \\&= \setn{\diceA,\diceB,\diceC,\diceD,\diceE,\diceF}
    && \text{by definition of $\ocsG$}
  \\
  \ocscena(\ocsG)
    &\eqd \argmin_{x\in\ocsG}\sum_{y\in\ocsG}\metric{x}{y}\psp(y)
    && \text{by definition of $\ocscena$ \xref{def:ocscenx}}
  \\&= \argmin_{x\in\ocsG}\sum_{y\in\ocsG}\metric{x}{y}\frac{1}{6}
    && \text{by definition of $\ocsG$}
  \\&= \argmin_{x\in\ocsG}\sum_{y\in\ocsG}\metric{x}{y}
    && \text{because $\ff(x)=\frac{1}{6}x$ is \prope{strictly isotone} and by \prefpp{lem:argminphi}}
  \\&=\mathrlap{\argmin_{x\in\ocsG}
         \setn{\begin{array}{*{11}{c}}
           0 &+& 1 &+& 2 &+& 3 &+& 2 &+& 1 \\
           1 &+& 0 &+& 1 &+& 2 &+& 3 &+& 2 \\
           2 &+& 1 &+& 0 &+& 1 &+& 2 &+& 3 \\
           3 &+& 2 &+& 1 &+& 0 &+& 1 &+& 2 \\
           2 &+& 3 &+& 2 &+& 1 &+& 0 &+& 1 \\
           1 &+& 2 &+& 3 &+& 2 &+& 1 &+& 0 
         \end{array}}
       = \argmin_{x\in\ocsG}
         \setn{\begin{array}{c}
           9\\
           9\\
           9\\
           9\\
           9\\
           9
         \end{array}}
       = \setn{\begin{array}{c}
           \diceA\\
           \diceB\\
           \diceC\\
           \diceD\\
           \diceE\\
           \diceF
         \end{array}}}
  \\
  \ocsceng(\ocsG)
    &\eqd \argmin_{x\in\ocsG}\prod_{y\in\ocsG\setd\setn{x}}\brs{\metric{x}{y}^{\psp(y)}}
    &&\text{by definition of $\ocsceng$ \xref{def:ocsceng}}
  \\&= \argmin_{x\in\ocsG}\prod_{y\in\ocsG\setd\setn{x}}\brs{\metric{x}{y}^\frac{1}{6}}
    &&\text{by definition of $\ocsG$}
  \\&= \argmin_{x\in\ocsG}\brp{\prod_{y\in\ocsG\setd\setn{x}}\metric{x}{y}}^\frac{1}{6}
  \\&= \argmin_{x\in\ocsG}\prod_{y\in\ocsG\setd\setn{x}}\metric{x}{y}
    %&&\text{because $\ff(x)=\frac{1}{300}x$ is \prope{strictly isotone} and by \prefpp{lem:argminmaxphi}}
    %&&\text{because $\ff(x)=x^\frac{1}{300}$ is \prope{strictly isotone}}
    &&\text{by \prefpp{lem:argminphi}}
  \\&=\mathrlap{\argmin_{x\in\ocsG}
         \setn{\begin{array}{*{11}{c}}
             &\times& 1 &\times& 2 &\times& 3 &\times& 2 &\times& 1 \\
           1 &\times&   &\times& 1 &\times& 2 &\times& 3 &\times& 2 \\
           2 &\times& 1 &\times&   &\times& 1 &\times& 2 &\times& 3 \\
           3 &\times& 2 &\times& 1 &\times&   &\times& 1 &\times& 2 \\
           2 &\times& 3 &\times& 2 &\times& 1 &\times&   &\times& 1 \\
           1 &\times& 2 &\times& 3 &\times& 2 &\times& 1 &\times&   
         \end{array}}
       = \argmin_{x\in\ocsG}
         \setn{\begin{array}{c}
           12\\
           12\\
           12\\
           12\\
           12\\
           12
         \end{array}}
       = \setn{\begin{array}{c}
           \diceA\\
           \diceB\\
           \diceC\\
           \diceD\\
           \diceE\\
           \diceF
         \end{array}}}
  \\
  \ocscenh(\ocsG)
    &\eqd \argmin_{x\in\ocsG}\brp{\sum_{y\in\ocsG\setd\setn{x}}\frac{1}{\metric{x}{y}}\psp(y)}^{-1}
    &&\text{by definition of $\ocscenh$ \xref{def:ocscenh}}
  \\&= \argmax_{x\in\ocsG}\sum_{y\in\ocsG\setd\setn{x}}\frac{1}{\metric{x}{y}}\psp(y)
    && \text{because $\fphi(x)\eqd x^{-1}$ is \prope{strictly antitone} and by \prefp{lem:minphia}}
  \\&= \argmax_{x\in\ocsG}\sum_{y\in\ocsG\setd\setn{x}}\frac{1}{\metric{x}{y}}\frac{1}{6}
  \\&= \argmax_{x\in\ocsG}\sum_{y\in\ocsG\setd\setn{x}}\frac{1}{\metric{x}{y}}
    %&&\text{because $\ff(x)=\frac{1}{300}x$ is \prope{strictly isotone} and by \prefpp{lem:argminmaxphi}}
    %&&\text{because $\ff(x)=\frac{1}{6}x$ is \prope{strictly isotone}}
    &&\text{by \prefpp{lem:argminphi}}
  \\&=\mathrlap{\argmax_{x\in\ocsG}
         %\setn\begin{array}{*{11}{@{\,}c}}
         \setn{\begin{array}{*{11}{c}}
                       &+& \frac{1}{1} &+& \frac{1}{2} &+& \frac{1}{3} &+& \frac{1}{2} &+& \frac{1}{1} \\
           \frac{1}{1} &+&             &+& \frac{1}{1} &+& \frac{1}{2} &+& \frac{1}{3} &+& \frac{1}{2} \\
           \frac{1}{2} &+& \frac{1}{1} &+&             &+& \frac{1}{1} &+& \frac{1}{2} &+& \frac{1}{3} \\
           \frac{1}{3} &+& \frac{1}{2} &+& \frac{1}{1} &+&             &+& \frac{1}{1} &+& \frac{1}{2} \\
           \frac{1}{2} &+& \frac{1}{3} &+& \frac{1}{2} &+& \frac{1}{1} &+&             &+& \frac{1}{1} \\
           \frac{1}{1} &+& \frac{1}{2} &+& \frac{1}{3} &+& \frac{1}{2} &+& \frac{1}{1} &+&            
         \end{array}}
       = \argmax_{x\in\ocsG}\frac{1}{6}
         \setn{\begin{array}{r}
           20\\
           20\\
           20\\
           20\\
           20\\
           20
         \end{array}}
       = \setn{\begin{array}{c}
           \diceA\\
           \diceB\\
           \diceC\\
           \diceD\\
           \diceE\\
           \diceF
         \end{array}}}
  \\
  \ocscenm(\ocsG)
    &\eqd \argmin_{x\in\ocsG}\min_{y\in\ocsG\setd\setn{x}}\metric{x}{y}\psp(y)
    &&\text{by definition of $\ocscenm$ \xref{def:ocscenm}}
  \\&= \argmin_{x\in\ocsG}\min_{y\in\ocsG\setd\setn{x}}\metric{x}{y}\frac{1}{6}
  \\&= \argmin_{x\in\ocsG}\min_{y\in\ocsG\setd\setn{x}}\metric{x}{y}
    %&& \text{because $\ff(x)=\frac{1}{6}x$ is \prope{strictly isotone}}
    && \text{by \prefpp{lem:argminmaxphi}}
   %&& \text{because $\ff(x)=\frac{1}{6}x$ is \prope{strictly isotone} and by \prefpp{lem:argminmaxphi}}
  \\&= \argmin_{x\in\ocsG}\setn{1,1,1,1,1,1}
  \\&= \setn{\diceA,\diceB,\diceC,\diceD,\diceE,\diceF}
    && \text{by definition of $\ocsG$}
  \\
  \\
  \ocscenM(\ocsG)
    &\eqd \argmax_{x\in\ocsG}\min_{y\in\ocsG\setd\setn{x}}\metric{x}{y}\psp(y)
    &&\text{by definition of $\ocscenM$ \xref{def:ocscenM}}
  \\&= \argmax_{x\in\ocsG}\setn{1,1,1,1,1,1}
    && \text{by $\ocscenm(\ocsG)$ result}
  \\&= \setn{\diceA,\diceB,\diceC,\diceD,\diceE,\diceF}
    && \text{by definition of $\ocsG$}
  \\
  \\
  \ocsVaro(\ocsG;\ocscena)
    &= \mathrlap{\ocsVaro(\ocsG;\ocsceng)= \ocsVaro(\ocsG;\ocscenh)= \ocsVaro(\ocsG;\ocscenm)= \ocsVaro(\ocsG;\ocscenM) = \ocsVaro(\ocsG)}
  \\&\eqd \sum_{x\in\ocsG}\brs{\metric{\ocscen(\ocsG)}{x}}^2\psp(x)
    && \text{by definition of $\ocsVaro$ \xref{def:ocsVarG}}
  \\&= \sum_{x\in\ocsG}\brp{0^2}\frac{1}{6}
    && \text{because $\ocscen(\ocsG)=\ocsG$}
  \\&= 0
    && \text{by field property of \vale{additive identity element} $0$}
\end{align*}
\end{proof}

%---------------------------------------
\begin{example}[\exmd{weighted spinner outcome subspace}]
\label{ex:wspinner}\mbox{}\\
%---------------------------------------
\begin{minipage}{\tw-35mm}%
The six value \structe{weighted spinner outcome subspace} $\ocsG$ \xref{def:ocs} 
illustrated to the right has the following geometric values:
\\\indentx$\begin{array}{rcl@{\qquad}lclcl}
  \ocscen(\ocsG)=\ocscena(\ocsG)=\ocsceng(\ocsG) &=& \setn{1,6}         &\ocsVaro(\ocsG) &=& \frac{5}{3} &\approx& 1.667\\
  \ocscenh(\ocsG)                                &=& \setn{2,5}         &\ocsVaro(\ocsG;\ocscenh) &=& \frac{4}{3} &\approx& 1.333\\
  \ocscenm(\ocsG)=\ocscenM(\ocsG)                &=& \setn{1,2,3,4,5,6} &\ocsVaro(\ocsG;\ocscenm) &=& 0           &=&       0
\end{array}$
\end{minipage}\hfill%
\begin{tabular}{c}
  \gsize%
  \psset{unit=8mm}%
  \centering%
  %{%============================================================================
% Daniel J. Greenhoe
% LaTeX file
% spinner 6 mapping to linearly ordered L6
%============================================================================
{%\psset{unit=0.5\psunit}%
\begin{pspicture}(-1.5,-1.5)(1.5,1.5)%
  %---------------------------------
  % options
  %---------------------------------
  \psset{%
    linecolor=blue,%
    radius=1.25ex,
    labelsep=2.5mm,
    }%
  %---------------------------------
  % spinner graph
  %---------------------------------
  \rput(0,0){%\psset{unit=2\psunit}%
    \Cnode[fillstyle=solid,fillcolor=snode](-0.8660,-0.5){D6}%
    \Cnode(-0.8660,0.5){D5}%
    \Cnode(0,1){D4}%
    \Cnode(0.8660,0.5){D3}%
    \Cnode(0.8660,-0.5){D2}%
    \Cnode[fillstyle=solid,fillcolor=snode](0,-1){D1}%
    }
  \rput[-150](D6){$\circSix$}%
  \rput[ 150](D5){$\circFive$}%
  \rput[  90](D4){$\circFour$}%
  \rput[  30](D3){$\circThree$}%
  \rput[   0](D2){$\circTwo$}%
  \rput[ -90](D1){$\circOne$}%
  %
  \ncline{D6}{D1}%
  \ncline{D5}{D6}%
  \ncline{D4}{D5}%
  \ncline{D3}{D4}%
  \ncline{D2}{D3}%
  \ncline{D1}{D2}%
  %
  \uput[ 210](D6){$\frac{3}{6}$}
  \uput[ 150](D5){$\frac{1}{6}$}
  \uput[  22](D4){$\frac{1}{6}$}
  \uput[  30](D3){$\frac{1}{6}$}
  \uput[ -30](D2){$\frac{1}{6}$}
  \uput[ -22](D1){$\frac{3}{6}$}
  %
  %\uput[ 210](D6){${\scy\psp(\circSix)=}\frac{1}{6}$}
  %\uput[ 150](D5){${\scy\psp(\circFive)=}\frac{1}{6}$}
  %\uput[  22](D4){${\scy\psp(\circFour)=}\frac{1}{6}$}
  %\uput[  30](D3){${\scy\psp(\circThree)=}\frac{1}{6}$}
  %\uput[ -30](D2){${\scy\psp(\circTwo)=}\frac{1}{6}$}
  %\uput[-22](D1){${\scy\psp(\circOne)=}\frac{1}{6}$}
\end{pspicture}
}%}%
  {\includegraphics{../common/math/graphics/pdfs/wspinner.pdf}}%
\end{tabular}\\
The \structe{outcome center} result is used later in \prefpp{ex:wspinner_xyz}.
Note that, unlike the \structe{weighted real die outcome subspace} \xref{ex:wdie},
of the center measures of cardinality 2 or less, the \structe{harmonic center} $\ocscenh(\ocsG)$
yields the lowest \fncte{outcome variance} \xref{def:ocsVarG}.
This is surprising since it suggests that $\ocscenh(\ocsG)$ is superior to 
all the other \fncte{center measures} \xxref{def:ocscen}{def:ocscenx},
but yet unlike the other center measures, it yields center values that are not maximally likely.
\end{example}
\begin{proof}
  \begin{align*}
    \ocscen(\ocsG)
      &\eqd \argmin_{x\in\ocsG}\max_{y\in\ocsG}\metric{x}{y}\psp(y)
      &&\text{by definition of $\ocscen$ \xref{def:ocscen}}
    %\\&=\mathrlap{\argmin_{x\in\ocsG}\max_{y\in\ocsG}
    %       \setn{\begin{array}{cccccc}
    %         \metricn(1,1)\frac{3}{10}&\metricn(1,2)\frac{1}{10}&\metricn(1,3)\frac{1}{10}&\metricn(1,4)\frac{1}{10}&\metricn(1,5)\frac{1}{10}&\metricn(1,6)\frac{3}{10}\\
    %         \metricn(2,1)\frac{3}{10}&\metricn(2,2)\frac{1}{10}&\metricn(2,3)\frac{1}{10}&\metricn(2,4)\frac{1}{10}&\metricn(2,5)\frac{1}{10}&\metricn(2,6)\frac{3}{10}\\
    %         \metricn(3,1)\frac{3}{10}&\metricn(3,2)\frac{1}{10}&\metricn(3,3)\frac{1}{10}&\metricn(3,4)\frac{1}{10}&\metricn(3,5)\frac{1}{10}&\metricn(3,6)\frac{3}{10}\\
    %         \metricn(4,1)\frac{3}{10}&\metricn(4,2)\frac{1}{10}&\metricn(4,3)\frac{1}{10}&\metricn(4,4)\frac{1}{10}&\metricn(4,5)\frac{1}{10}&\metricn(4,6)\frac{3}{10}\\
    %         \metricn(5,1)\frac{3}{10}&\metricn(5,2)\frac{1}{10}&\metricn(5,3)\frac{1}{10}&\metricn(5,4)\frac{1}{10}&\metricn(5,5)\frac{1}{10}&\metricn(5,6)\frac{3}{10}\\
    %         \metricn(6,1)\frac{3}{10}&\metricn(6,2)\frac{1}{10}&\metricn(6,3)\frac{1}{10}&\metricn(6,4)\frac{1}{10}&\metricn(6,5)\frac{1}{10}&\metricn(6,6)\frac{3}{10}\\
    %       \end{array}}}
    \\&=\mathrlap{\argmin_{x\in\ocsG}\max_{y\in\ocsG}\frac{1}{10}
           \setn{\begin{array}{cccccc}
             {0}\times3&{1}\times1 & {2}\times1 &{3}\times1 &{2}\times1 &{1}\times3\\
             {1}\times3&{0}\times1 & {1}\times1 &{2}\times1 &{3}\times1 &{2}\times3\\
             {2}\times3&{1}\times1 & {0}\times1 &{1}\times1 &{2}\times1 &{3}\times3\\
             {3}\times3&{2}\times1 & {1}\times1 &{0}\times1 &{1}\times1 &{2}\times3\\
             {2}\times3&{3}\times1 & {2}\times1 &{1}\times1 &{0}\times1 &{1}\times3\\
             {1}\times3&{2}\times1 & {3}\times1 &{2}\times1 &{1}\times1 &{0}\times3
           \end{array}}
    \quad= \argmin_{x\in\ocsG}\frac{1}{10}
           \setn{\begin{array}{ccccccc}
             {3}\\
             {6}\\
             {9}\\
             {9}\\
             {6}\\
             {3}
           \end{array}}
    \quad= \setn{\begin{array}{c}
             1\\
             \mbox{ }\\
             \mbox{ }\\
             \mbox{ }\\
             \mbox{ }\\
             6
           \end{array}}}
    \\
    \ocscena(\ocsG)
      &\eqd \argmin_{x\in\ocsG}\sum_{y\in\ocsG}\metric{x}{y}\psp(y)
      &&\text{by definition of $\ocscena$ \xref{def:ocscena}}
    \\&=\mathrlap{\argmin_{x\in\ocsG}\frac{1}{10}
           \setn{\begin{array}{*{11}{c@{\,}}}
             {0}\times3 &+& {1}\times1  &+&  {2}\times1  &+& {3}\times1  &+& {2}\times1  &+& {1}\times3\\
             {1}\times3 &+& {0}\times1  &+&  {1}\times1  &+& {2}\times1  &+& {3}\times1  &+& {2}\times3\\
             {2}\times3 &+& {1}\times1  &+&  {0}\times1  &+& {1}\times1  &+& {2}\times1  &+& {3}\times3\\
             {3}\times3 &+& {2}\times1  &+&  {1}\times1  &+& {0}\times1  &+& {1}\times1  &+& {2}\times3\\
             {2}\times3 &+& {3}\times1  &+&  {2}\times1  &+& {1}\times1  &+& {0}\times1  &+& {1}\times3\\
             {1}\times3 &+& {2}\times1  &+&  {3}\times1  &+& {2}\times1  &+& {1}\times1  &+& {0}\times3
           \end{array}}
       = \argmin_{x\in\ocsG}\frac{1}{10}
           \setn{\begin{array}{c}
             11\\
             15\\
             19\\
             19\\
             15\\
             11
           \end{array}}
    = \setn{\begin{array}{c}
             1\\
             \mbox{ }\\
             \mbox{ }\\
             \mbox{ }\\
             \mbox{ }\\
             6
           \end{array}}}
  \\
  \ocsceng(\ocsG)
    &\eqd \argmin_{x\in\ocsG}\prod_{y\in\ocsG\setd\setn{x}}\brs{\metric{x}{y}^{\psp(y)}}
    &&\text{by definition of $\ocsceng$ \xref{def:ocsceng}}
  \\&= \argmin_{x\in\ocsG}\prod_{y\in\ocsG\setd\setn{x}}\brs{\metric{x}{y}^{6\psp(y)\frac{1}{6}}}
  \\&= \argmin_{x\in\ocsG}\brp{\prod_{y\in\ocsG\setd\setn{x}}\brs{\metric{x}{y}^{6\psp(y)}}}^\frac{1}{6}
  \\&= \argmin_{x\in\ocsG}\prod_{y\in\ocsG\setd\setn{x}}\brs{\metric{x}{y}^{6\psp(y)}}
    %&&\text{because $\ff(x)=\frac{1}{300}x$ is \prope{strictly isotone} and by \prefpp{lem:argminmaxphi}}
    %&&\text{because $\ff(x)=x^\frac{1}{6}$ is \prope{strictly isotone}}
    &&\text{by \prefpp{lem:argminphi}}
  \\&=\mathrlap{\argmin_{x\in\ocsG}
         %\setn\begin{array}{*{11}{@{\,}c}}
         \setn{\begin{array}{*{11}{c}}
                 &\times& 1^{1} &\times& 2^{1} &\times& 3^{1} &\times& 2^{1} &\times& 1^{3} \\
           1^{3} &\times&       &\times& 1^{1} &\times& 2^{1} &\times& 3^{1} &\times& 2^{3} \\
           2^{3} &\times& 1^{1} &\times&       &\times& 1^{1} &\times& 2^{1} &\times& 3^{3} \\
           3^{3} &\times& 2^{1} &\times& 1^{1} &\times&       &\times& 1^{1} &\times& 2^{3} \\
           2^{3} &\times& 3^{1} &\times& 2^{1} &\times& 1^{1} &\times&       &\times& 1^{3} \\
           1^{3} &\times& 2^{1} &\times& 3^{1} &\times& 2^{1} &\times& 1^{1} &      &           
         \end{array}}
       = \argmin_{x\in\ocsG}
         \setn{\begin{array}{r}
            2^2 \times 3^1\\
            2^4 \times 3^1\\
            2^4 \times 3^3\\
            2^4 \times 3^3\\
            2^4 \times 3^1\\
            2^2 \times 3^1
         \end{array}}
       %= \argmin_{x\in\ocsG}
       %  \setn{\begin{array}{r}
       %     12\\
       %     64\\
       %    432\\
       %    432\\
       %     48\\
       %     12
       %  \end{array}}
    = \setn{\begin{array}{c}
             1\\
             \mbox{ }\\
             \mbox{ }\\
             \mbox{ }\\
             \mbox{ }\\
             6
           \end{array}}}
  \\
  \ocscenh(\ocsG)
    &\eqd \argmin_{x\in\ocsG}\brp{\sum_{y\in\ocsG\setd\setn{x}}\frac{1}{\metric{x}{y}}\psp(y)}^{-1}
    &&\text{by definition of $\ocscenh$ \xref{def:ocscenh}}
  \\&= \argmax_{x\in\ocsG}\sum_{y\in\ocsG\setd\setn{x}}\frac{1}{\metric{x}{y}}\psp(y)
    && \text{because $\fphi(x)\eqd x^{-1}$ is \prope{strictly antitone} and by \prefp{lem:minphia}}
  \\&= \argmax_{x\in\ocsG}\sum_{y\in\ocsG\setd\setn{x}}\frac{1}{\metric{x}{y}}6\psp(y)\frac{1}{6}
  \\&= \argmax_{x\in\ocsG}\sum_{y\in\ocsG\setd\setn{x}}\frac{6\psp(y)}{\metric{x}{y}}
    %&&\text{because $\ff(x)=\frac{1}{300}x$ is \prope{strictly isotone} and by \prefpp{lem:argminmaxphi}}
    %&&\text{because $\ff(x)=\frac{1}{6}x$ is \prope{strictly isotone}}
    &&\text{by \prefpp{lem:argminphi}}
  \\&=\mathrlap{\argmax_{x\in\ocsG}
         %\setn\begin{array}{*{11}{@{\,}c}}
         \setn{\begin{array}{*{11}{c}}
                       &+& \frac{1}{1} &+& \frac{1}{2} &+& \frac{1}{3} &+& \frac{1}{2} &+& \frac{3}{1} \\
           \frac{3}{1} &+&             &+& \frac{1}{1} &+& \frac{1}{2} &+& \frac{1}{3} &+& \frac{3}{2} \\
           \frac{3}{2} &+& \frac{1}{1} &+&             &+& \frac{1}{1} &+& \frac{1}{2} &+& \frac{3}{3} \\
           \frac{3}{3} &+& \frac{1}{2} &+& \frac{1}{1} &+&             &+& \frac{1}{1} &+& \frac{3}{2} \\
           \frac{3}{2} &+& \frac{1}{3} &+& \frac{1}{2} &+& \frac{1}{1} &+&             &+& \frac{3}{1} \\
           \frac{3}{1} &+& \frac{1}{2} &+& \frac{1}{3} &+& \frac{1}{2} &+& \frac{1}{1} &+&               
         \end{array}}
       = \argmax_{x\in\ocsG}\frac{1}{6}
         \setn{\begin{array}{r}
           32\\
           38\\
           30\\
           30\\
           38\\
           32
         \end{array}}
       = \setn{\begin{array}{r}
             \\
            2\\
             \\
             \\
            5\\
           \mbox{}
         \end{array}}}
  \\
  \ocscenm(\ocsG)
    &\eqd \argmin_{x\in\ocsG}\min_{y\in\ocsG}\metric{x}{y}\psp(y)
    &&\text{by definition of $\ocscenm$ \xref{def:ocscenm}}
     \\&=\mathrlap{\argmin_{x\in\ocsG}\min_{y\in\ocsG\setd\setn{x}}
         \setn{\begin{array}{cccccc}
             & 1 & 2 & 3 & 2 & 1\\
           3 &   & 1 & 2 & 3 & 2\\
           6 & 1 &   & 1 & 2 & 3\\
           9 & 2 & 1 &   & 1 & 2\\
           6 & 3 & 2 & 1 &   & 1\\
           3 & 2 & 3 & 2 & 1 &  
         \end{array}}
       = \argmin_{x\in\ocsG}
         \setn{\begin{array}{c}
           1\\
           1\\
           1\\
           1\\
           1\\
           1\\
         \end{array}}
       = \setn{\begin{array}{c}
           1\\
           2\\
           3\\
           4\\
           5\\
           6\\
         \end{array}}}
  \\
  \ocscenM(\ocsG)
    &\eqd \argmax_{x\in\ocsG}\min_{y\in\ocsG}\metric{x}{y}\psp(y)
    &&\text{by definition of $\ocscenM$ \xref{def:ocscenM}}
  \\&= \argmax_{x\in\ocsG}\setn{1,1,1,1,1,1}
    && \text{by $\ocscenm(\ocsG)$ result}
  \\&= \setn{1,2,3,4,5,6}
  \\
    \ocsVaro(\ocsG;\ocscena) 
      &= \ocsVaro(\ocsG;\ocsceng) = \ocsVaro(\ocsG)
    \\&\eqd \sum_{x\in\ocsG}\metricsq{\ocscen(\ocsG)}{x}\psp(x)
      &&\text{by definition of $\ocsVaro$ \xref{def:ocsVarG}}
    \\&= \sum_{x\in\ocsG}\metricsq{\setn{1,6}}{x}\psp(x)
      &&\text{by $\ocscen(\ocsG)$ result}
    \\&= \mathrlap{(0)^2\frac{3}{6}+(1)^2\frac{1}{6}+(2)^2\frac{1}{6}+(2)^2\frac{1}{6}+(1)^2\frac{1}{6}+(0)^2\frac{3}{6}}
    \\&= \frac{10}{6}= \frac{5}{3}= 1\frac{2}{3}\approx 1.667
  \\
    \ocsVaro(\ocsG;\ocscenh)
      &\eqd \sum_{x\in\ocsG}\metricsq{\ocscenh(\ocsG)}{x}\psp(x)
      &&\text{by definition of $\ocsVaro$ \xref{def:ocsVarG}}
    \\&= \sum_{x\in\ocsG}\metricsq{\setn{2,5}}{x}\psp(x)
      &&\text{by $\ocscenh(\ocsG)$ result}
    \\&= \mathrlap{(1)^2\frac{3}{6}+(0)^2\frac{1}{6}+(1)^2\frac{1}{6}+(1)^2\frac{1}{6}+(0)^2\frac{1}{6}+(1)^2\frac{3}{6}}
    \\&= \frac{8}{6}= \frac{4}{3} \approx 1.333
  \\
    \ocsVaro(\ocsG;\ocscenM)
      &= \ocsVaro(\ocsG;\ocscenm)
    \\&\eqd \sum_{x\in\ocsG}\metricsq{\ocscenm(\ocsG)}{x}\psp(x)
      &&\text{by definition of $\ocsVaro$ \xref{def:ocsVarG}}
    \\&= \sum_{x\in\ocsG}\metricsq{\setn{1,2,3,4,5,6}}{x}\psp(x)
      &&\text{by $\ocscenh(\ocsG)$ result}
    \\&= \sum_{x\in\ocsG}0^2\psp(x)
       = 0
  \end{align*}
\end{proof}



%\begin{figure}
%  \gsize%
%  \centering%
%  %============================================================================
% Daniel J. Greenhoe
% LaTeX file
% linear congruential (LCG) pseudo-random number generator (PRNG) mappings
% x_{n+1} = (7x_n+5)mod 9
% y_{n+1} = (y_n+2)mod 5
%============================================================================
\begin{pspicture}(-1.3,-1.5)(1.5,1.5)%
  %---------------------------------
  % options
  %---------------------------------
  \psset{%
    radius=1.25ex,
    labelsep=2.5mm,
    linecolor=blue,%
    }%
  \rput{288}(0,0){\rput(1,0){\Cnode[fillstyle=solid,fillcolor=snode](0,0){S4}}}%
  \rput{216}(0,0){\rput(1,0){\Cnode(0,0){S3}}}%
  \rput{144}(0,0){\rput(1,0){\Cnode(0,0){S2}}}%
  \rput{ 72}(0,0){\rput(1,0){\Cnode(0,0){S1}}}%
  \rput{  0}(0,0){\rput(1,0){\Cnode(0,0){S0}}}%
  %
  \rput(S4){$4$}%
  \rput(S3){$3$}%
  \rput(S2){$2$}%
  \rput(S1){$1$}%
  \rput(S0){$0$}%
  %
  \ncline{S4}{S0}%
  \ncline{S3}{S4}%
  \ncline{S2}{S3}%
  \ncline{S1}{S2}%
  \ncline{S0}{S1}%
  %
  \uput[-30](S4){$\frac{3}{9}$}
  \uput[210](S3){$\frac{2}{9}$}
  \uput[144](S2){$\frac{1}{9}$}
  \uput[ 30](S1){$\frac{1}{9}$}
  \uput[  0](S0){$\frac{2}{9}$}
  \rput(0,0){$\ocsG$}%
\end{pspicture}%%
%  \caption{weighted ring \xref{ex:wring5}\label{fig:wring5}}
%\end{figure}
%---------------------------------------
\begin{example}[\exmd{weighted ring}]
\label{ex:wring5}
%---------------------------------------
The weighted five element ring illustrated to the right has the geometric values below.
The \structe{outcome center} result is used later in \prefpp{ex:lcg7x1m9_xyz}.
\\\begin{minipage}{\tw-65mm}%
\indentx$\begin{array}{rcl @{\qquad} lcccl}
    \ocscen (\ocsG) &=&\setn{4}      & \ocsVaro(\ocsG)          &=& \frac{11}{9} &\approx& 1.222 \\
    \ocscena(\ocsG) &=&\setn{3,4}    & \ocsVaro(\ocsG;\ocscena) &=& \frac{ 7}{9} &\approx& 0.778\\
    \ocsceng(\ocsG) &=&\setn{3}      & \ocsVaro(\ocsG;\ocsceng) &=& \frac{16}{9} &\approx& 1.778\\
    \ocscenh(\ocsG) &=&\setn{1,2,3}  & \ocsVaro(\ocsG;\ocscenh) &=& \frac{5}{9}  &\approx& 0.556\\
    \ocscenm(\ocsG) &=&\setn{0,3,4}  & \ocsVaro(\ocsG;\ocscenm) &=& \frac{2}{9}  &\approx& 0.222\\
    \ocscenM(\ocsG) &=&\setn{1,2}    & \ocsVaro(\ocsG;\ocscenM) &=& \frac{16}{9} &\approx& 1.778
  \end{array}$.
\end{minipage}\hfill%
\begin{tabular}{c}
  \gsize%
  %\psset{unit=5mm}%
  %{%============================================================================
% Daniel J. Greenhoe
% LaTeX file
% linear congruential (LCG) pseudo-random number generator (PRNG) mappings
% x_{n+1} = (7x_n+5)mod 9
% y_{n+1} = (y_n+2)mod 5
%============================================================================
\begin{pspicture}(-1.3,-1.5)(1.5,1.5)%
  %---------------------------------
  % options
  %---------------------------------
  \psset{%
    radius=1.25ex,
    labelsep=2.5mm,
    linecolor=blue,%
    }%
  \rput{288}(0,0){\rput(1,0){\Cnode[fillstyle=solid,fillcolor=snode](0,0){S4}}}%
  \rput{216}(0,0){\rput(1,0){\Cnode(0,0){S3}}}%
  \rput{144}(0,0){\rput(1,0){\Cnode(0,0){S2}}}%
  \rput{ 72}(0,0){\rput(1,0){\Cnode(0,0){S1}}}%
  \rput{  0}(0,0){\rput(1,0){\Cnode(0,0){S0}}}%
  %
  \rput(S4){$4$}%
  \rput(S3){$3$}%
  \rput(S2){$2$}%
  \rput(S1){$1$}%
  \rput(S0){$0$}%
  %
  \ncline{S4}{S0}%
  \ncline{S3}{S4}%
  \ncline{S2}{S3}%
  \ncline{S1}{S2}%
  \ncline{S0}{S1}%
  %
  \uput[-30](S4){$\frac{3}{9}$}
  \uput[210](S3){$\frac{2}{9}$}
  \uput[144](S2){$\frac{1}{9}$}
  \uput[ 30](S1){$\frac{1}{9}$}
  \uput[  0](S0){$\frac{2}{9}$}
  \rput(0,0){$\ocsG$}%
\end{pspicture}%}%
  {\includegraphics{../common/math/graphics/pdfs/wring5.pdf}}%
\end{tabular}
\end{example}
\begin{proof}
    \begin{align*}
      \ocscen(\ocsG)
        &\eqd \argmin_{x\in\ocsG}\max_{y\in\ocsG}\metric{x}{y}\psp(y)
        &&\text{by definition of $\ocscen$ \xref{def:ocscen}}
      \\&= \argmin_{x\in\ocsG}\max_{y\in\ocsG}\frac{1}{9}\metric{x}{y}\psp(y)9
      \\&= \argmin_{x\in\ocsG}\max_{y\in\ocsG}\metric{x}{y}\psp(y)9
        &&\text{because $\ff(x)=\frac{1}{9}x$ is \prope{strictly isotone} and by \prefpp{lem:argminmaxphi}}
      \\&=\mathrlap{\argmin_{x\in\ocsG}\max_{y\in\ocsG}
             %\setn{\begin{array}{*{5}{@{\,\,}c@{\,\,}}}
             \setn{\begin{array}{*{5}{c}}
               \metricn(0,0)\psp(0)9&\metricn(0,1)\psp(1)9&\metricn(0,2)\psp(2)9&\metricn(0,3)\psp(3)9&\metricn(0,4)\psp(4)9\\
               \metricn(1,0)\psp(0)9&\metricn(1,1)\psp(1)9&\metricn(1,2)\psp(2)9&\metricn(1,3)\psp(3)9&\metricn(1,4)\psp(4)9\\
               \vdots              &\vdots              &\vdots              &\vdots              &\vdots              \\
              %\metricn(2,0)\psp(0)9&\metricn(2,1)\psp(1)9&\metricn(2,2)\psp(2)9&\metricn(2,3)\psp(3)9&\metricn(2,4)\psp(4)9\\
              %\metricn(3,0)\psp(0)9&\metricn(3,1)\psp(1)9&\metricn(3,2)\psp(2)9&\metricn(3,3)\psp(3)9&\metricn(3,4)\psp(4)9\\
               \metricn(4,0)\psp(0)9&\metricn(4,1)\psp(1)9&\metricn(4,2)\psp(2)9&\metricn(4,3)\psp(3)9&\metricn(4,4)\psp(4)9\\
             \end{array}}}
      \\&=\mathrlap{\argmin_{x\in\ocsG}\max_{y\in\ocsG}
             \setn{\begin{array}{*{5}{c}}
               {0}\times2 & {1}\times1 & {2}\times1 & {2}\times2 & {1}\times3\\
               {1}\times2 & {0}\times1 & {1}\times1 & {2}\times2 & {2}\times3\\
               {2}\times2 & {1}\times1 & {0}\times1 & {1}\times2 & {2}\times3\\
               {2}\times2 & {2}\times1 & {1}\times1 & {0}\times2 & {1}\times3\\
               {1}\times2 & {2}\times1 & {2}\times1 & {1}\times2 & {0}\times3\\
             \end{array}}
      %\quad= \argmin_{x\in\ocsG}\max_{y\in\ocsG}
      %       \setn{\begin{array}{*{5}{c}}
      %         0 & 2 & 1 & 2 & 6\\
      %         4 & 0 & 2 & 2 & 3\\
      %         2 & 2 & 0 & 4 & 3\\
      %         2 & 1 & 2 & 0 & 3\\
      %         4 & 1 & 1 & 2 & 0\\
      %       \end{array}}
      \quad= \argmin_{x\in\ocsG}
             \setn{\begin{array}{c}
                4\\
                6\\
                6\\
                4\\
                2
             \end{array}}
      \quad= \setn{\begin{array}{c}
                \mbox{ }\\
                \mbox{ }\\
                \mbox{ }\\
                \mbox{ }\\
                4
             \end{array}}}
      \\
      \ocscena(\ocsG)
        &\eqd \argmin_{x\in\ocsG}\sum_{y\in\ocsG}\metric{x}{y}\psp(y)
        &&\text{by definition of $\ocscena$ \xref{def:ocscena}}
      \\&= \argmin_{x\in\ocsG}\sum_{y\in\ocsG}\frac{1}{9}\metric{x}{y}\psp(y)9
      \\&= \argmin_{x\in\ocsG}\sum_{y\in\ocsG}\metric{x}{y}\psp(y)9
        &&\text{because $\ff(x)=\frac{1}{9}x$ is \prope{strictly isotone} and by \prefpp{lem:argminphi}}
      \\&=\mathrlap{\argmin_{x\in\ocsG}%\sum_{y\in\ocsG}%\frac{1}{9}
             \setn{\begin{array}{ccccccccc}
               {0}\times2&+&{2}\times1&+&{1}\times1&+&{1}\times2&+&{2}\times3\\
               {2}\times2&+&{0}\times1&+&{2}\times1&+&{1}\times2&+&{1}\times3\\
               {1}\times2&+&{2}\times1&+&{0}\times1&+&{2}\times2&+&{1}\times3\\
               {1}\times2&+&{1}\times1&+&{2}\times1&+&{0}\times2&+&{1}\times3\\
               {2}\times2&+&{1}\times1&+&{1}\times1&+&{1}\times2&+&{0}\times3\\
             \end{array}}
      \quad= \argmin_{x\in\ocsG}%\frac{1}{9}
             \setn{\begin{array}{ccccc}
                11\\
                11\\
                11\\
                 8\\
                 8
             \end{array}}
      \quad= \argmin_{x\in\ocsG}%\frac{1}{9}
             \setn{\begin{array}{c}
                \mbox{ }\\
                \mbox{ }\\
                \mbox{ }\\
                3\\
                4
             \end{array}}}
      \\
      \ocsceng(\ocsG)
        &\eqd \argmin_{x\in\ocsG}\prod_{y\in\ocsG\setd\setn{x}}{\metric{x}{y}^{\psp(y)}}
        &&\text{by definition of $\ocsceng$ \xref{def:ocsceng}}
      \\&= \argmin_{x\in\ocsG}\prod_{y\in\ocsG\setd\setn{x}}{\metric{x}{y}^{9\psp(y)\frac{1}{9}}}
      \\&= \argmin_{x\in\ocsG}\brs{\prod_{y\in\ocsG\setd\setn{x}}{\metric{x}{y}^{9\psp(y)}}}^\frac{1}{9}
      \\&= \argmin_{x\in\ocsG}\prod_{y\in\ocsG\setd\setn{x}}{\metric{x}{y}^{9\psp(y)}}
        &&\text{because $\ff(x)\eqd x^\frac{1}{9}$ is \prope{strictly isotone} and by \prefpp{lem:argminphi}}
      \\&=\mathrlap{\argmin_{x\in\ocsG}
             \setn{\begin{array}{ccccccccc}
                     &      & {2}^1 &\times& {1}^1 &\times& {1}^2 &\times& {2}^3\\
               {2}^2 &\times&       &      & {2}^1 &\times& {1}^2 &\times& {1}^3\\
               {1}^2 &\times& {2}^1 &\times&       &      & {2}^2 &\times& {1}^3\\
               {1}^2 &\times& {1}^1 &\times& {2}^1 &\times&       &      & {1}^3\\
               {2}^2 &\times& {1}^1 &\times& {1}^1 &\times& {1}^2 &      &
             \end{array}}
      \quad= \argmin_{x\in\ocsG}
             \setn{\begin{array}{c}
                16\\
                 8\\
                 8\\
                 2\\
                 4
             \end{array}}
      \quad= \argmin_{x\in\ocsG}%\frac{1}{9}
             \setn{\begin{array}{c}
                \mbox{ }\\
                \mbox{ }\\
                \mbox{ }\\
                3\\
                \mbox{}
             \end{array}}}
      \\
      \ocscenh(\ocsG)
        &\eqd \argmin_{x\in\ocso}\brp{\sum_{y\in\ocso}\frac{1}{\metric{x}{y}}\psp(y)}^{-1} 
        &&\text{by definition of $\ocscenh$ \xref{def:ocscenh}}
      \\&=\argmax_{x\in\ocso}\brp{\sum_{y\in\ocso}\frac{1}{\metric{x}{y}}\psp(y)}
        && \text{because $\fphi(x)\eqd x^{-1}$ is \prope{strictly antitone} and by \prefp{lem:minphia}}
      \\&= \argmax_{x\in\ocso}\brp{\frac{1}{9}\sum_{y\in\ocso}\frac{1}{\metric{x}{y}}\psp(y)9}
      \\&= \argmax_{x\in\ocso}\sum_{y\in\ocso}\frac{9\psp(y)}{\metric{x}{y}}
        &&\text{because $\ff(x)=\frac{1}{9}x$ is \prope{strictly isotone} and by \prefpp{lem:argminphi}}
      \\&=\mathrlap{\argmax_{x\in\ocsG}
             \setn{\begin{array}{*{4}{cc}c}
               0           &+& \frac{1}{2} &+& \frac{1}{1} &+& \frac{2}{1} &+& \frac{3}{2}\\
               \frac{2}{2} &+& 0           &+& \frac{1}{2} &+& \frac{2}{1} &+& \frac{3}{1}\\
               \frac{2}{1} &+& \frac{1}{2} &+& 0           &+& \frac{2}{2} &+& \frac{3}{1}\\
               \frac{2}{1} &+& \frac{1}{1} &+& \frac{1}{2} &+& 0           &+& \frac{3}{1}\\
               \frac{2}{2} &+& \frac{1}{1} &+& \frac{1}{1} &+& \frac{2}{1} &+& 0          
             \end{array}}
      \quad= \argmax_{x\in\ocsG}\frac{1}{2}
             \setn{\begin{array}{c}
               {10}\\
               {13}\\
               {13}\\
               {13}\\
               {11}
             \end{array}}
      \quad= \argmin_{x\in\ocsG}%\frac{1}{9}
             \setn{\begin{array}{c}
                \mbox{ }\\
                1\\
                2\\
                3\\
                \mbox{ }
             \end{array}}}
      \\
      \ocscenm(\ocsG)
        &\eqd \argmin_{x\in\ocsG}\min_{y\in\ocsG\setd\setn{x}}\metric{x}{y}\psp(y)
        &&\text{by definition of $\ocscenm$ \xref{def:ocscenm}}
      \\&= \argmin_{x\in\ocsG}\min_{y\in\ocsG\setd\setn{x}}\frac{1}{9}\metric{x}{y}\psp(y)9
      \\&= \argmin_{x\in\ocsG}\min_{y\in\ocsG\setd\setn{x}}\metric{x}{y}\psp(y)9
        &&\text{because $\ff(x)=\frac{1}{9}x$ is \prope{strictly isotone} and by \prefpp{lem:argminmaxphi}}
      \\&=\mathrlap{\argmin_{x\in\ocsG}\min_{y\in\ocsG\setd\setn{x}}%\frac{1}{9}
             \setn{\begin{array}{ccccccccc}
                         &{2}\times1&{1}\times1&{1}\times2&{2}\times3\\
               {2}\times2&          &{2}\times1&{1}\times2&{1}\times3\\
               {1}\times2&{2}\times1&          &{2}\times2&{1}\times3\\
               {1}\times2&{1}\times1&{2}\times1&          &{1}\times3\\
               {2}\times2&{1}\times1&{1}\times1&{1}\times2& 
             \end{array}}
      \quad= \argmin_{x\in\ocsG}
             \setn{\begin{array}{c}
                1\\
                2\\
                2\\
                1\\
                1
             \end{array}}
      \quad= \argmin_{x\in\ocsG}%\frac{1}{9}
             \setn{\begin{array}{c}
                0\\
                \mbox{ }\\
                \mbox{ }\\
                3\\
                4
             \end{array}}}
      \\
      \ocscenM(\ocsG)
        &\eqd \argmax_{x\in\ocsG}\min_{y\in\ocsG\setd\setn{x}}\metric{x}{y}\psp(y)
        &&\text{by definition of $\ocscenM$ \xref{def:ocscenM}}
      \\&= \argmax_{x\in\ocsG}\setn{1,2,2,1,1}
        && \text{by $\ocscenm(\ocsG)$ result}
      \\&= \setn{1,2}
      \\
      \ocsVaro(\ocsG)
        &\eqd \sum_{x\in\ocsG}\metricsq{\ocscen(\ocsG)}{x}\psp(x)
        && \text{by definition of $\ocsVaro$ \xref{def:ocsVarG}}
      \\&= \sum_{x\in\ocsG}\metricsq{\setn{4}}{x}\psp(x)
        && \text{by $\ocscen(\ocsG)$ result}
      \\&= \mathrlap{
           (1)^2\frac{2}{9}+  % x = 0
           (2)^2\frac{1}{9}+  % x = 1
           (2)^2\frac{1}{9}+  % x = 2
           (1)^2\frac{2}{9}+  % x = 3
           (0)^2\frac{3}{9}   % x = 4
         = \frac{11}{9}  \approx 1.222}
      \\
      \ocsVaro(\ocsG;\ocscena)
        &\eqd \sum_{x\in\ocsG}\metricsq{\ocscena(\ocsG)}{x}\psp(x)
        && \text{by definition of $\ocsVaro$ \xref{def:ocsVarG}}
      \\&= \sum_{x\in\ocsG}\metricsq{\setn{3,4}}{x}\psp(x)
        && \text{by $\ocscena(\ocsG)$ result}
      \\&= \mathrlap{
           (1)^2\frac{2}{9}+  % x = 0
           (2)^2\frac{1}{9}+  % x = 1
           (1)^2\frac{1}{9}+  % x = 2
           (0)^2\frac{2}{9}+  % x = 3
           (0)^2\frac{3}{9}   % x = 4
         = \frac{7}{9}  \approx 0.778}
      \\
      \ocsVaro(\ocsG;\ocsceng)
        &\eqd \sum_{x\in\ocsG}\metricsq{\ocsceng(\ocsG)}{x}\psp(x)
        && \text{by definition of $\ocsVaro$ \xref{def:ocsVarG}}
      \\&= \sum_{x\in\ocsG}\metricsq{\setn{3}}{x}\psp(x)
        && \text{by $\ocsceng(\ocsG)$ result}
      \\&= \mathrlap{
           (2)^2\frac{2}{9}+  % x = 0
           (2)^2\frac{1}{9}+  % x = 1
           (1)^2\frac{1}{9}+  % x = 2
           (0)^2\frac{2}{9}+  % x = 3
           (1)^2\frac{3}{9}   % x = 4
         = \frac{16}{9}  \approx 1.778}
      \\
      \ocsVaro(\ocsG;\ocscenh)
        &\eqd \sum_{x\in\ocsG}\metricsq{\ocscenh(\ocsG)}{x}\psp(x)
        && \text{by definition of $\ocsVaro$ \xref{def:ocsVarG}}
      \\&= \sum_{x\in\ocsG}\metricsq{\setn{1,2,3}}{x}\psp(x)
        && \text{by $\ocscenh(\ocsG)$ result}
      \\&= \mathrlap{
           (1)^2\frac{2}{9}+  % x = 0
           (0)^2\frac{1}{9}+  % x = 1
           (0)^2\frac{1}{9}+  % x = 2
           (0)^2\frac{2}{9}+  % x = 3
           (1)^2\frac{3}{9}   % x = 4
         = \frac{5}{9} \approx 0.556}
      \\
      \ocsVaro(\ocsG;\ocscenm)
        &\eqd \sum_{x\in\ocsG}\metricsq{\ocscenm(\ocsG)}{x}\psp(x)
        && \text{by definition of $\ocsVaro$ \xref{def:ocsVarG}}
      \\&= \sum_{x\in\ocsG}\metricsq{\setn{0,3,4}}{x}\psp(x)
        && \text{by $\ocscenm(\ocsG)$ result}
      \\&= \mathrlap{
           (0)^2\frac{2}{9}+  % x = 0
           (1)^2\frac{1}{9}+  % x = 1
           (1)^2\frac{1}{9}+  % x = 2
           (0)^2\frac{2}{9}+  % x = 3
           (0)^2\frac{3}{9}   % x = 4
         = \frac{2}{9} \approx 0.222}
      \\
      \ocsVaro(\ocsG;\ocscenM)
        &\eqd \sum_{x\in\ocsG}\metricsq{\ocscenM(\ocsG)}{x}\psp(x)
        && \text{by definition of $\ocsVaro$ \xref{def:ocsVarG}}
      \\&= \sum_{x\in\ocsG}\metricsq{\setn{1,2}}{x}\psp(x)
        && \text{by $\ocscenm(\ocsG)$ result}
      \\&= \mathrlap{
           (1)^2\frac{2}{9}+  % x = 0
           (0)^2\frac{1}{9}+  % x = 1
           (0)^2\frac{1}{9}+  % x = 2
           (1)^2\frac{2}{9}+  % x = 3
           (2)^2\frac{3}{9}   % x = 4
         = \frac{16}{9} \approx 1.778}
    \end{align*}
\end{proof}


%---------------------------------------
\begin{minipage}{\tw-65mm}%
\begin{example}%[\exmd{weigted ring}]
\label{ex:wring5short}
%---------------------------------------
The weighted five element structure illustrated to the right has the following geometric values:
\\\indentx$\begin{array}{rcl@{\qquad} lcccl}
    \ocscen(\ocsG)=\ocsceng(\ocsG) &=&\setn{3}      & \ocsVaro(\ocsG)          &=& \frac{10}{9} &\approx& 1.111 \\
    \ocscena(\ocsG)                &=&\setn{3,4}    & \ocsVaro(\ocsG;\ocscena) &=& \frac{ 4}{9} &\approx& 0.444  \\
    \ocscenh(\ocsG)                &=&\setn{1,2,3}  & \ocsVaro(\ocsG;\ocscenh) &=& \frac{ 5}{9} &\approx& 0.555  \\
    \ocscenm(\ocsG)                &=&\setn{0,3,4}  & \ocsVaro(\ocsG;\ocscenm) &=& \frac{ 2}{9} &\approx& 0.222  \\
    \ocscenM(\ocsG)                &=&\setn{1,2}    & \ocsVaro(\ocsG;\ocscenM) &=& \frac{7}{9}  &\approx& 0.778
  \end{array}$.
\end{example}
\end{minipage}\hfill%
\begin{tabular}{c}
  \gsize%
  %\psset{unit=5mm}%
  %{%============================================================================
% Daniel J. Greenhoe
% LaTeX file
% linear congruential (LCG) pseudo-random number generator (PRNG) mappings
% x_{n+1} = (7x_n+5)mod 9
% y_{n+1} = (y_n+2)mod 5
%============================================================================
\begin{pspicture}(-1.3,-1.5)(1.5,1.5)%
  %---------------------------------
  % options
  %---------------------------------
  \psset{%
    radius=1.25ex,
    labelsep=2.5mm,
    linecolor=blue,%
    }%
  \rput{288}(0,0){\rput(1,0){\Cnode(0,0){S4}}}%
  \rput{216}(0,0){\rput(1,0){\Cnode(0,0){S2}}}%
  \rput{144}(0,0){\rput(1,0){\Cnode(0,0){S0}}}%
  \rput{ 72}(0,0){\rput(1,0){\Cnode[fillstyle=solid,fillcolor=snode](0,0){S3}}}%
  \rput{  0}(0,0){\rput(1,0){\Cnode(0,0){S1}}}%
  %
  \rput(S4){$4$}%
  \rput(S3){$3$}%
  \rput(S2){$2$}%
  \rput(S1){$1$}%
  \rput(S0){$0$}%
  %
  \ncline{S3}{S0}\ncline{S3}{S4}%
  \ncline{S1}{S3}%
  \ncline{S4}{S1}%
  \ncline{S2}{S4}%
  \ncline{S0}{S2}%
  %
  \uput[-30](S4){$\frac{3}{9}$}
  \uput[ 30](S3){$\frac{2}{9}$}
  \uput[216](S2){$\frac{1}{9}$}
  \uput[  0](S1){$\frac{1}{9}$}
  \uput[144](S0){$\frac{2}{9}$}
  \rput(0,0){$\ocsG$}%
\end{pspicture}%}%
  {\includegraphics{../common/math/graphics/pdfs/wring5short.pdf}}%
\end{tabular}
\\
The \structe{outcome center} result is used later in \prefpp{ex:lcg7x1m9_seqorder}.
Note that only the operators $\ocscen$ and $\ocsceng$ were able to successfully isolate a single center point
($\seto{\ocscen(\ocsG)}=\seto{\ocsceng(\ocsG)}=\seto{\setn{3}}=1$).
\\
\begin{proof}
  \begin{align*}
      \ocscen(\ocsG)
        &\eqd \argmin_{x\in\ocsG}\max_{y\in\ocsG}\metric{x}{y}\psp(y)
        &&\text{by definition of $\ocscen$ \xref{def:ocscen}}
      \\&= \argmin_{x\in\ocsG}\max_{y\in\ocsG}\frac{1}{9}\metric{x}{y}\psp(y)9
      \\&= \argmin_{x\in\ocsG}\max_{y\in\ocsG}\metric{x}{y}\psp(y)9
        &&\text{because $\ff(x)=\frac{1}{9}x$ is \prope{strictly isotone} and by \prefpp{lem:argminmaxphi}}
      \\&=\mathrlap{\argmin_{x\in\ocsG}\max_{y\in\ocsG}
             \setn{\begin{array}{*{5}{c}}
               \metricn(0,0)\psp(0)9&\metricn(0,1)\psp(1)9&\metricn(0,2)\psp(2)9&\metricn(0,3)\psp(3)9&\metricn(0,4)\psp(4)9\\
               \metricn(1,0)\psp(0)9&\metricn(1,1)\psp(1)9&\metricn(1,2)\psp(2)9&\metricn(1,3)\psp(3)9&\metricn(1,4)\psp(4)9\\
               \vdots              &\vdots              &\vdots              &\vdots              &\vdots              \\
              %\metricn(2,0)\psp(0)9&\metricn(2,1)\psp(1)9&\metricn(2,2)\psp(2)9&\metricn(2,3)\psp(3)9&\metricn(2,4)\psp(4)9\\
              %\metricn(3,0)\psp(0)9&\metricn(3,1)\psp(1)9&\metricn(3,2)\psp(2)9&\metricn(3,3)\psp(3)9&\metricn(3,4)\psp(4)9\\
               \metricn(4,0)\psp(0)9&\metricn(4,1)\psp(1)9&\metricn(4,2)\psp(2)9&\metricn(4,3)\psp(3)9&\metricn(4,4)\psp(4)9\\
             \end{array}}}
      %\\&= \argmin_{x\in\ocsG}\max_{y\in\ocsG}
      %       \setn{\begin{array}{*{5}{c}}
      %         \metricn(0,0)\frac{2}{9}&\metricn(0,1)\frac{1}{9}&\cdots&\metricn(0,4)\frac{3}{9}\\
      %         \metricn(1,0)\frac{2}{9}&\metricn(1,1)\frac{1}{9}&\cdots&\metricn(1,4)\frac{3}{9}\\
      %         \vdots                  &\ddots                  &\ddots&\vdots                  \\
      %         \metricn(4,0)\frac{2}{9}&\metricn(4,1)\frac{1}{9}&\cdots&\metricn(4,4)\frac{3}{9}\\
      %       \end{array}}
      \\&=\mathrlap{\argmin_{x\in\ocsG}\max_{y\in\ocsG}
             \setn{\begin{array}{*{5}{c}}
               {0}\times2 & {2}\times1 & {1}\times1 & {1}\times2 & {2}\times3\\
               {2}\times2 & {0}\times1 & {2}\times1 & {1}\times2 & {1}\times3\\
               {1}\times2 & {2}\times1 & {0}\times1 & {2}\times2 & {1}\times3\\
               {1}\times2 & {1}\times1 & {2}\times1 & {0}\times2 & {1}\times3\\
               {2}\times2 & {1}\times1 & {1}\times1 & {1}\times2 & {0}\times3\\
             \end{array}}
      %\quad= \argmin_{x\in\ocsG}\max_{y\in\ocsG}
      %       \setn{\begin{array}{*{5}{c}}
      %         0 & 2 & 1 & 2 & 6\\
      %         4 & 0 & 2 & 2 & 3\\
      %         2 & 2 & 0 & 4 & 3\\
      %         2 & 1 & 2 & 0 & 3\\
      %         4 & 1 & 1 & 2 & 0\\
      %       \end{array}}
      \quad= \argmin_{x\in\ocsG}
             \setn{\begin{array}{c}
                6\\
                4\\
                4\\
                3\\
                4
             \end{array}}
      \quad= \argmin_{x\in\ocsG}%\frac{1}{9}
             \setn{\begin{array}{c}
                \mbox{ }\\
                \mbox{ }\\
                \mbox{ }\\
                3\\
                \mbox{}
             \end{array}}}
      \\
      \ocscena(\ocsG)
        &\eqd \argmin_{x\in\ocsG}\sum_{y\in\ocsG}\metric{x}{y}\psp(y)
        &&\text{by definition of $\ocscena$ \xref{def:ocscena}}
      \\&=\mathrlap{\argmin_{x\in\ocsG}%\sum_{y\in\ocsG}%\frac{1}{9}
             \setn{\begin{array}{*{9}{c}}
               {0}\times2 &+& {2}\times1 &+& {1}\times1 &+& {1}\times2 &+& {2}\times3\\
               {2}\times2 &+& {0}\times1 &+& {2}\times1 &+& {1}\times2 &+& {1}\times3\\
               {1}\times2 &+& {2}\times1 &+& {0}\times1 &+& {2}\times2 &+& {1}\times3\\
               {1}\times2 &+& {1}\times1 &+& {2}\times1 &+& {0}\times2 &+& {1}\times3\\
               {2}\times2 &+& {1}\times1 &+& {1}\times1 &+& {1}\times2 &+& {0}\times3
             \end{array}}
      \quad= \argmin_{x\in\ocsG}%\frac{1}{9}
             \setn{\begin{array}{c}
                11\\
                11\\
                11\\
                 8\\
                 8
             \end{array}}
      \quad= \argmin_{x\in\ocsG}%\frac{1}{9}
             \setn{\begin{array}{c}
                \mbox{ }\\
                \mbox{ }\\
                \mbox{ }\\
                3\\
                4
             \end{array}}}
      \\
      \ocsceng(\ocsG)
        &\eqd \argmin_{x\in\ocsG}\prod_{y\in\ocsG\setd\setn{x}}{\metric{x}{y}^{\psp(y)}}
        &&\text{by definition of $\ocsceng$ \xref{def:ocsceng}}
      \\&= \argmin_{x\in\ocsG}\prod_{y\in\ocsG\setd\setn{x}}{\metric{x}{y}^{9\psp(y)\frac{1}{9}}}
      \\&= \argmin_{x\in\ocsG}\brs{\prod_{y\in\ocsG\setd\setn{x}}{\metric{x}{y}^{9\psp(y)}}}^\frac{1}{9}
      \\&= \argmin_{x\in\ocsG}\prod_{y\in\ocsG\setd\setn{x}}{\metric{x}{y}^{9\psp(y)}}
        &&\text{because $\ff(x)\eqd x^\frac{1}{9}$ is \prope{strictly isotone} and by \prefpp{lem:argminphi}}
      %\\&= \argmin_{x\in\ocsG}
      %       \setn{\begin{array}{ccccccccc}
      %                                 &      & \metric{0}{1}^{\psp(1)} &\times& \metric{0}{2}^{\psp(2)} &\times& \metric{0}{3}^{\psp(3)} &\times& \metric{0}{4}^{\psp(4)}\\
      %         \metric{1}{0}^{\psp(0)} &      &                         &\times& \metric{1}{2}^{\psp(2)} &\times& \metric{1}{3}^{\psp(3)} &\times& \metric{1}{4}^{\psp(4)}\\
      %         \metric{2}{0}^{\psp(0)} &\times& \metric{2}{1}^{\psp(1)} &      &                         &\times& \metric{2}{3}^{\psp(3)} &\times& \metric{2}{4}^{\psp(4)}\\
      %         \metric{3}{0}^{\psp(0)} &\times& \metric{3}{1}^{\psp(1)} &\times& \metric{3}{2}^{\psp(2)} &      &                         &\times& \metric{3}{4}^{\psp(4)}\\
      %         \metric{4}{0}^{\psp(0)} &\times& \metric{4}{1}^{\psp(1)} &\times& \metric{4}{2}^{\psp(2)} &\times& \metric{4}{3}^{\psp(3)} &      &
      %       \end{array}}
      %\\&= \argmin_{x\in\ocsG}
      %       \setn{\begin{array}{ccccccccc}
      %                                   &      & \metricn(0,1)^\frac{1}{9} &\times& \metricn(0,2)^\frac{1}{9} &\times& \metricn(0,3)^\frac{2}{9} &\times& \metricn(0,4)^\frac{3}{9}\\
      %         \metricn(1,0)^\frac{2}{9} &      &                           &\times& \metricn(1,2)^\frac{1}{9} &\times& \metricn(1,3)^\frac{2}{9} &\times& \metricn(1,4)^\frac{3}{9}\\
      %         \metricn(2,0)^\frac{2}{9} &\times& \metricn(2,1)^\frac{1}{9} &      &                           &\times& \metricn(2,3)^\frac{2}{9} &\times& \metricn(2,4)^\frac{3}{9}\\
      %         \metricn(3,0)^\frac{2}{9} &\times& \metricn(3,1)^\frac{1}{9} &\times& \metricn(3,2)^\frac{1}{9} &      &                           &\times& \metricn(3,4)^\frac{3}{9}\\
      %         \metricn(4,0)^\frac{2}{9} &\times& \metricn(4,1)^\frac{1}{9} &\times& \metricn(4,2)^\frac{1}{9} &\times& \metricn(4,3)^\frac{2}{9} &      &                          
      %       \end{array}}
      \\&=\mathrlap{\argmin_{x\in\ocsG}
             \setn{\begin{array}{ccccccccc}
                     &      & {2}^1 &\times& {1}^1 &\times& {1}^2 &\times& {2}^3\\
               {2}^2 &\times&       &      & {2}^1 &\times& {1}^2 &\times& {1}^3\\
               {1}^2 &\times& {2}^1 &\times&       &      & {2}^2 &\times& {1}^3\\
               {1}^2 &\times& {1}^1 &\times& {2}^1 &\times&       &      & {1}^3\\
               {2}^2 &\times& {1}^1 &\times& {1}^1 &\times& {1}^2 &      &
             \end{array}}
      \quad= \argmin_{x\in\ocsG}
             \setn{\begin{array}{c}
                 2^4\\
                 2^3\\
                 2^3\\
                 2^1\\
                 2^2
             \end{array}}
      \quad= \argmin_{x\in\ocsG}%\frac{1}{9}
             \setn{\begin{array}{c}
                \mbox{ }\\
                \mbox{ }\\
                \mbox{ }\\
                3\\
                \mbox{ }
             \end{array}}}
      \\
      \ocscenh(\ocsG)
        &\eqd \argmin_{x\in\ocso}\brp{\sum_{y\in\ocso}\frac{1}{\metric{x}{y}}\psp(y)}^{-1} 
        &&\text{by definition of $\ocscenh$ \xref{def:ocscenh}}
      \\&= \argmax_{x\in\ocso}\brp{\sum_{y\in\ocso}\frac{1}{\metric{x}{y}}\psp(y)}
        && \text{because $\fphi(x)\eqd x^{-1}$ is \prope{strictly antitone} and by \prefp{lem:minphia}}
      \\&= \argmax_{x\in\ocso}\brp{\frac{1}{9}\sum_{y\in\ocso}\frac{1}{\metric{x}{y}}\psp(y)9}
      \\&= \argmax_{x\in\ocso}\sum_{y\in\ocso}\frac{9\psp(y)}{\metric{x}{y}}
        &&\text{because $\ff(x)=\frac{1}{9}x$ is \prope{strictly isotone} and by \prefpp{lem:argminphi}}
      %\\&= \argmin_{x\in\ocsG}
      %       \setn{\begin{array}{ccccc}
      %         0                                + \frac{1}{\metric{0}{1}}{\psp(1)} + \frac{1}{\metric{0}{2}}{\psp(2)} + \frac{1}{\metric{0}{3}}{\psp(3)} + \frac{1}{\metric{0}{4}}{\psp(4)}\\
      %         \frac{1}{\metric{1}{0}}{\psp(0)} + 0                                + \frac{1}{\metric{1}{2}}{\psp(2)} + \frac{1}{\metric{1}{3}}{\psp(3)} + \frac{1}{\metric{1}{4}}{\psp(4)}\\
      %         \frac{1}{\metric{2}{0}}{\psp(0)} + \frac{1}{\metric{2}{1}}{\psp(1)} + 0                                + \frac{1}{\metric{2}{3}}{\psp(3)} + \frac{1}{\metric{2}{4}}{\psp(4)}\\
      %         \frac{1}{\metric{3}{0}}{\psp(0)} + \frac{1}{\metric{3}{1}}{\psp(1)} + \frac{1}{\metric{3}{2}}{\psp(2)} + 0                                + \frac{1}{\metric{3}{4}}{\psp(4)}\\
      %         \frac{1}{\metric{4}{0}}{\psp(0)} + \frac{1}{\metric{4}{1}}{\psp(1)} + \frac{1}{\metric{4}{2}}{\psp(2)} + \frac{1}{\metric{4}{3}}{\psp(3)} + 0                         
      %       \end{array}}
      %\\&= \argmin_{x\in\ocsG}
      %       \setn{\begin{array}{ccccc}
      %         \brs{                            + \frac{1}{2}\cdot\frac{1}{9} + \frac{1}{1}\cdot\frac{1}{9} + \frac{1}{1}\cdot\frac{2}{9} + \frac{1}{2}\cdot\frac{3}{9}}^{-1}\\
      %         \brs{\frac{1}{2}\cdot\frac{2}{9} + 0                           + \frac{1}{2}\cdot\frac{1}{9} + \frac{1}{1}\cdot\frac{2}{9} + \frac{1}{1}\cdot\frac{3}{9}}^{-1}\\
      %         \brs{\frac{1}{1}\cdot\frac{2}{9} + \frac{1}{2}\cdot\frac{1}{9} + 0                           + \frac{1}{2}\cdot\frac{2}{9} + \frac{1}{1}\cdot\frac{3}{9}}^{-1}\\
      %         \brs{\frac{1}{1}\cdot\frac{2}{9} + \frac{1}{1}\cdot\frac{1}{9} + \frac{1}{2}\cdot\frac{1}{9} + 0                           + \frac{1}{1}\cdot\frac{3}{9}}^{-1}\\
      %         \brs{\frac{1}{2}\cdot\frac{2}{9} + \frac{1}{1}\cdot\frac{1}{9} + \frac{1}{1}\cdot\frac{1}{9} + \frac{1}{1}\cdot\frac{2}{9} + 0                          }^{-1}
      %       \end{array}}
      %\\&= \argmin_{x\in\ocsG}9
      %       \setn{\begin{array}{ccccc}
      %         \brs{0           + \frac{1}{2} + \frac{1}{1} + \frac{2}{1} + \frac{3}{2}}^{-1}\\
      %         \brs{\frac{2}{2} + 0           + \frac{1}{2} + \frac{2}{1} + \frac{3}{1}}^{-1}\\
      %         \brs{\frac{2}{1} + \frac{1}{2} + 0           + \frac{2}{2} + \frac{3}{1}}^{-1}\\
      %         \brs{\frac{2}{1} + \frac{1}{1} + \frac{1}{2} + 0           + \frac{3}{1}}^{-1}\\
      %         \brs{\frac{2}{2} + \frac{1}{1} + \frac{1}{1} + \frac{2}{1} + 0          }^{-1}
      %       \end{array}}
      %\\&= \argmin_{x\in\ocsG}9
      %       \setn{\begin{array}{c}
      %         \brs{\frac{10}{2}}^{-1}\\
      %         \brs{\frac{13}{2}}^{-1}\\
      %         \brs{\frac{13}{2}}^{-1}\\
      %         \brs{\frac{13}{2}}^{-1}\\
      %         \brs{\frac{11}{2}}^{-1}
      %       \end{array}}
      \\&=\mathrlap{\argmax_{x\in\ocsG}
             \setn{\begin{array}{*{4}{cc}c}
               0           &+& \frac{1}{2} &+& \frac{1}{1} &+& \frac{2}{1} &+& \frac{3}{2}\\
               \frac{2}{2} &+& 0           &+& \frac{1}{2} &+& \frac{2}{1} &+& \frac{3}{1}\\
               \frac{2}{1} &+& \frac{1}{2} &+& 0           &+& \frac{2}{2} &+& \frac{3}{1}\\
               \frac{2}{1} &+& \frac{1}{1} &+& \frac{1}{2} &+& 0           &+& \frac{3}{1}\\
               \frac{2}{2} &+& \frac{1}{1} &+& \frac{1}{1} &+& \frac{2}{1} &+& 0          
             \end{array}}
      \quad= \argmax_{x\in\ocsG}\frac{1}{2}
             \setn{\begin{array}{c}
               {10}\\
               {13}\\
               {13}\\
               {13}\\
               {10}
             \end{array}}
      \quad= \setn{\begin{array}{c}
                \\
               1\\
               2\\
               3\\
               \mbox{}
             \end{array}}}
      \\
      \ocscenm(\ocsG)
        &\eqd \argmin_{x\in\ocsG}\min_{y\in\ocsG\setd\setn{x}}\metric{x}{y}\psp(y)
        &&\text{by definition of $\ocscenm$ \xref{def:ocscenm}}
      \\&= \argmin_{x\in\ocsG}\min_{y\in\ocsG\setd\setn{x}}\frac{1}{9}\metric{x}{y}\psp(y)9
      \\&= \argmin_{x\in\ocsG}\min_{y\in\ocsG\setd\setn{x}}\metric{x}{y}\psp(y)9
        &&\text{because $\ff(x)=\frac{1}{9}x$ is \prope{strictly isotone} and by \prefpp{lem:argminmaxphi}}
      \\&=\mathrlap{\argmin_{x\in\ocsG}\min_{y\in\ocsG\setd\setn{x}}%\frac{1}{9}
             \setn{\begin{array}{ccccccccc}
                         &{2}\times1&{1}\times1&{1}\times2&{2}\times3\\
               {2}\times2&          &{2}\times1&{1}\times2&{1}\times3\\
               {1}\times2&{2}\times1&          &{2}\times2&{1}\times3\\
               {1}\times2&{1}\times1&{2}\times1&          &{1}\times3\\
               {2}\times2&{1}\times1&{1}\times1&{1}\times2& 
             \end{array}}
      \quad= \argmin_{x\in\ocsG}
             \setn{\begin{array}{c}
                1\\
                2\\
                2\\
                1\\
                1
             \end{array}}
      \quad= \setn{\begin{array}{c}
                0\\
                 \\
                 \\
                3\\
                4
             \end{array}}}
      \\
      \ocscenM(\ocsG)
        &\eqd \argmax_{x\in\ocsG}\min_{y\in\ocsG\setd\setn{x}}\metric{x}{y}\psp(y)
        &&\text{by definition of $\ocscenM$ \xref{def:ocscenM}}
      \\&= \argmax_{x\in\ocsG}\setn{1,2,2,1,1}
        &&\text{by $\ocscenm(\ocsG)$ result}
      \\&= \setn{1,2}
      \\
      \ocsVaro(\ocsG)
        &= \ocsVaro(\ocsG;\ocsceng)
        && \text{by $\ocscen(\ocsG)$ and $\ocsceng(\ocsG)$ results}
      \\&\eqd \sum_{x\in\ocsG}\metricsq{\ocsceng(\ocsG)}{x}\psp(x)
        && \text{by definition of $\ocsVaro$ \xref{def:ocsVarG}}
      \\&= \sum_{x\in\ocsG}\metricsq{\setn{3}}{x}\psp(x)
        && \text{by $\ocsceng(\ocsG)$ result}
      \\&= \mathrlap{
           (1)^2\frac{2}{9}+ % x=0
           (1)^2\frac{1}{9}+ % x=1
           (2)^2\frac{1}{9}+ % x=2
           (0)^2\frac{2}{9}+ % x=3
           (1)^2\frac{3}{9}  % x=4
         = \frac{10}{9} \approx 1.111}
      \\
      \ocsVaro(\ocsG;\ocscena)
        &\eqd \sum_{x\in\ocsG}\metricsq{\ocscena(\ocsG)}{x}\psp(x)
        && \text{by definition of $\ocsVaro$ \xref{def:ocsVarG}}
      \\&= \sum_{x\in\ocsG}\metricsq{\setn{3,4}}{x}\psp(x)
        && \text{by $\ocscena(\ocsG)$ result}
      \\&= \mathrlap{
           (1)^2\frac{2}{9}+ % x=0
           (1)^2\frac{1}{9}+ % x=1
           (1)^2\frac{1}{9}+ % x=2
           (0)^2\frac{2}{9}+ % x=3
           (0)^2\frac{3}{9}  % x=4
         = \frac{4}{9} \approx 0.444}
      \\
      \ocsVaro(\ocsG;\ocscenh)
        &\eqd \sum_{x\in\ocsG}\metricsq{\ocscenh(\ocsG)}{x}\psp(x)
        && \text{by definition of $\ocsVaro$ \xref{def:ocsVarG}}
      \\&= \sum_{x\in\ocsG}\metricsq{\setn{1,2,3}}{x}\psp(x)
        && \text{by $\ocscenh(\ocsG)$ result}
      \\&= \mathrlap{
           (1)^2\frac{2}{9}+ % x=0
           (0)^2\frac{1}{9}+ % x=1
           (0)^2\frac{1}{9}+ % x=2
           (0)^2\frac{2}{9}+ % x=3
           (1)^2\frac{3}{9}  % x=4
         = \frac{5}{9} \approx 0.555}
      \\
      \ocsVaro(\ocsG;\ocscenm)
        &\eqd \sum_{x\in\ocsG}\metricsq{\ocscenm(\ocsG)}{x}\psp(x)
        && \text{by definition of $\ocsVaro$ \xref{def:ocsVarG}}
      \\&= \sum_{x\in\ocsG}\metricsq{\setn{0,3,4}}{x}\psp(x)
        && \text{by $\ocscenm(\ocsG)$ result}
      \\&= \mathrlap{
           (0)^2\frac{2}{9}+ % x=0
           (1)^2\frac{1}{9}+ % x=1
           (1)^2\frac{1}{9}+ % x=2
           (0)^2\frac{2}{9}+ % x=3
           (0)^2\frac{3}{9}  % x=4
         = \frac{2}{9} \approx 0.222}
      \\
      \ocsVaro(\ocsG;\ocscenM)
        &\eqd \sum_{x\in\ocsG}\metricsq{\ocscenM(\ocsG)}{x}\psp(x)
        && \text{by definition of $\ocsVaro$ \xref{def:ocsVarG}}
      \\&= \sum_{x\in\ocsG}\metricsq{\setn{1,2}}{x}\psp(x)
        && \text{by $\ocscenM(\ocsG)$ result}
      \\&= \mathrlap{
           (1)^2\frac{2}{9}+ % x=0
           (0)^2\frac{1}{9}+ % x=1
           (0)^2\frac{1}{9}+ % x=2
           (1)^2\frac{2}{9}+ % x=3
           (1)^2\frac{3}{9}  % x=4
         = \frac{7}{9} \approx 0.778}
    \end{align*}
\end{proof}


%---------------------------------------
\begin{minipage}{\tw-65mm}%
\begin{example}%[\exmd{weigted ring}]
\label{ex:wring5shortd}
%---------------------------------------
The \structe{outcome subspace} \xref{def:ocs} illustrated to the right, with a \fncte{quasi-metric} \xref{def:qmetric}
has the following geometric values:
\\\indentx$\begin{array}{rcl@{\qquad} lcccl}
  \ocscen(\ocsG)=\ocscena(\ocsG)=\ocsceng(\ocsG)=\ocscenh(\ocsG) &=& \setn{3}    & \ocsVaro(\ocsG)          &=& \frac{12}{9} &\approx& 1.333\\
  \ocscenm(\ocsG)                                                &=& \setn{0,4}  & \ocsVaro(\ocsG;\ocscenm) &=& \frac{10}{9} &\approx& 1.111\\
  \ocscenM(\ocsG)                                                &=& \setn{1,2,3}& \ocsVaro(\ocsG;\ocscenM) &=& \frac{ 5}{9} &\approx& 0.555
\end{array}$
\end{example}
\end{minipage}\hfill%
\begin{tabular}{c}
  \gsize%
  %\psset{unit=5mm}%
  %{%============================================================================
% Daniel J. Greenhoe
% LaTeX file
% linear congruential (LCG) pseudo-random number generator (PRNG) mappings
% x_{n+1} = (7x_n+5)mod 9
% y_{n+1} = (y_n+2)mod 5
%============================================================================
\begin{pspicture}(-1.3,-1.5)(1.5,1.5)%
  %---------------------------------
  % options
  %---------------------------------
  \psset{%
    radius=1.25ex,
    labelsep=2.5mm,
    linecolor=blue,%
    }%
  \rput{288}(0,0){\rput(1,0){\Cnode(0,0){S4}}}%
  \rput{216}(0,0){\rput(1,0){\Cnode(0,0){S2}}}%
  \rput{144}(0,0){\rput(1,0){\Cnode(0,0){S0}}}%
  \rput{ 72}(0,0){\rput(1,0){\Cnode[fillstyle=solid,fillcolor=snode](0,0){S3}}}%
  \rput{  0}(0,0){\rput(1,0){\Cnode(0,0){S1}}}%
  %
  \rput(S4){$4$}%
  \rput(S3){$3$}%
  \rput(S2){$2$}%
  \rput(S1){$1$}%
  \rput(S0){$0$}%
  %
  \ncline{->}{S4}{S1}\ncline{->}{S2}{S4}\ncline{->}{S3}{S4}%
  \ncline{->}{S2}{S4}%
  \ncline{->}{S0}{S2}%
  \ncline{->}{S3}{S0}%
  \ncline{->}{S1}{S3}%
  %
  \uput[-30](S4){$\frac{3}{9}$}
  \uput[ 30](S3){$\frac{2}{9}$}
  \uput[216](S2){$\frac{1}{9}$}
  \uput[  0](S1){$\frac{1}{9}$}
  \uput[144](S0){$\frac{2}{9}$}
  \rput(0,0){$\ocsG$}%
\end{pspicture}%}%
  {\includegraphics{../common/math/graphics/pdfs/wring5shortd.pdf}}%
\end{tabular}
\\
This is the first example in this section to use a \structe{directed graph} 
(rather than an \structe{undirected graph} \xrefnp{def:dgraph})
and to require the use of a \fncte{quasi-metric} \xref{def:qmetric} that is not a \fncte{metric}.
Unlike \prefpp{ex:wring5short}, which had neither of these restrictions,
twice as many center operators (4 rather than 2) were able to successfully isolate a single center point.
The \structe{outcome center} result is used later in \prefpp{ex:lcg7x1m9_dgraph}.
\\
\begin{proof}
    \begin{align*}
      \ocscen(\ocsG)
        &\eqd \argmin_{x\in\ocsG}\max_{y\in\ocsG}\metric{x}{y}\psp(y)
        &&\text{by definition of $\ocscen$ \xref{def:ocscen}}
      \\&\eqd \argmin_{x\in\ocsG}\max_{y\in\ocsG}\frac{1}{9}\metric{x}{y}\psp(y)9
      \\&\eqd \argmin_{x\in\ocsG}\max_{y\in\ocsG}\metric{x}{y}\psp(y)9
        &&\text{because $\ff(x)=\frac{1}{9}x$ is \prope{strictly isotone} and by \prefpp{lem:argminmaxphi}}
      \\&=\mathrlap{\argmin_{x\in\ocsG}\max_{y\in\ocsG}
             \setn{\begin{array}{*{5}{c}}
               \metricn(0,0)\psp(0)9&\metricn(0,1)\psp(1)9&\metricn(0,2)\psp(2)9&\metricn(0,3)\psp(3)9&\metricn(0,4)\psp(4)9\\
               \metricn(1,0)\psp(0)9&\metricn(1,1)\psp(1)9&\metricn(1,2)\psp(2)9&\metricn(1,3)\psp(3)9&\metricn(1,4)\psp(4)9\\
               \vdots              &\vdots              &\vdots              &\vdots              &\vdots              \\
              %\metricn(2,0)\psp(0)9&\metricn(2,1)\psp(1)9&\metricn(2,2)\psp(2)9&\metricn(2,3)\psp(3)9&\metricn(2,4)\psp(4)9\\
              %\metricn(3,0)\psp(0)9&\metricn(3,1)\psp(1)9&\metricn(3,2)\psp(2)9&\metricn(3,3)\psp(3)9&\metricn(3,4)\psp(4)9\\
               \metricn(4,0)\psp(0)9&\metricn(4,1)\psp(1)9&\metricn(4,2)\psp(2)9&\metricn(4,3)\psp(3)9&\metricn(4,4)\psp(4)9\\
             \end{array}}}
      %\\&= \argmin_{x\in\ocsG}\max_{y\in\ocsG}
      %       \setn{\begin{array}{*{5}{c}}
      %         \metricn(0,0)\frac{2}{9}&\metricn(0,1)\frac{1}{9}&\cdots&\metricn(0,4)\frac{3}{9}\\
      %         \metricn(1,0)\frac{2}{9}&\metricn(1,1)\frac{1}{9}&\cdots&\metricn(1,4)\frac{3}{9}\\
      %         \vdots                  &\ddots                  &\ddots&\vdots                  \\
      %         \metricn(4,0)\frac{2}{9}&\metricn(4,1)\frac{1}{9}&\cdots&\metricn(4,4)\frac{3}{9}\\
      %       \end{array}}
      \\&=\mathrlap{\argmin_{x\in\ocsG}\max_{y\in\ocsG}
             %\setn{\begin{array}{*{5}{@{\hspace{1pt}}c@{\hspace{1pt}}}}
             \setn{\begin{array}{*{5}{c}}
               {0}\times2 & {3}\times1 & {1}\times1 & {4}\times2 & {2}\times3\\
               {2}\times2 & {0}\times1 & {3}\times1 & {1}\times2 & {2}\times3\\
               {4}\times2 & {2}\times1 & {0}\times1 & {3}\times2 & {1}\times3\\
               {1}\times2 & {2}\times1 & {2}\times1 & {0}\times2 & {1}\times3\\
               {3}\times2 & {1}\times1 & {4}\times1 & {2}\times2 & {0}\times3\\
             \end{array}}
      \quad= \argmin_{x\in\ocsG}
             \setn{\begin{array}{c}
                8\\
                6\\
                8\\
                3\\
                6
             \end{array}}
      \quad= \setn{\begin{array}{c}
                \mbox{ }\\
                \mbox{ }\\
                \mbox{ }\\
                3\\
                \mbox{ }
             \end{array}}}
      \\\\
      \ocscena(\ocsG)
        &\eqd \argmin_{x\in\ocsG}\sum_{y\in\ocsG}\metric{x}{y}\psp(y)
        &&\text{by definition of $\ocscena$ \xref{def:ocscena}}
      \\&= \argmin_{x\in\ocsG}\sum_{y\in\ocsG}\frac{1}{9}\metric{x}{y}\psp(y)9
      \\&= \argmin_{x\in\ocsG}\sum_{y\in\ocsG}\metric{x}{y}\psp(y)9
        &&\text{because $\ff(x)=\frac{1}{9}x$ is \prope{strictly isotone} and by \prefpp{lem:argminphi}}
      %\\&= \argmin_{x\in\ocsG}\max_{y\in\ocsG}
      %       \setn{\begin{array}{ccccccccc}
      %         \metricn(0,0)\frac{2}{9}&+&\metricn(0,1)\frac{1}{9}&+&\metricn(0,2)\frac{1}{9}&+&\metricn(0,3)\frac{2}{9}&+&\metricn(0,4)\frac{3}{9}\\
      %         \metricn(1,0)\frac{2}{9}&+&\metricn(1,1)\frac{1}{9}&+&\metricn(1,2)\frac{1}{9}&+&\metricn(1,3)\frac{2}{9}&+&\metricn(1,4)\frac{3}{9}\\
      %         \metricn(2,0)\frac{2}{9}&+&\metricn(2,1)\frac{1}{9}&+&\metricn(2,2)\frac{1}{9}&+&\metricn(2,3)\frac{2}{9}&+&\metricn(2,4)\frac{3}{9}\\
      %         \metricn(3,0)\frac{2}{9}&+&\metricn(3,1)\frac{1}{9}&+&\metricn(3,2)\frac{1}{9}&+&\metricn(3,3)\frac{2}{9}&+&\metricn(3,4)\frac{3}{9}\\
      %         \metricn(4,0)\frac{2}{9}&+&\metricn(4,1)\frac{1}{9}&+&\metricn(4,2)\frac{1}{9}&+&\metricn(4,3)\frac{2}{9}&+&\metricn(4,4)\frac{3}{9}\\
      %       \end{array}}
      \\&=\mathrlap{\argmin_{x\in\ocsG}\max_{y\in\ocsG}%\frac{1}{9}
             \setn{\begin{array}{ccccccccc}
               {0}\times2 &+& {3}\times1 &+& {1}\times1 &+& {4}\times2 &+& {2}\times3\\
               {2}\times2 &+& {0}\times1 &+& {3}\times1 &+& {1}\times2 &+& {2}\times3\\
               {4}\times2 &+& {2}\times1 &+& {0}\times1 &+& {3}\times2 &+& {1}\times3\\
               {1}\times2 &+& {2}\times1 &+& {2}\times1 &+& {0}\times2 &+& {1}\times3\\
               {3}\times2 &+& {1}\times1 &+& {4}\times1 &+& {2}\times2 &+& {0}\times3\\
             \end{array}}
      \quad= \argmin_{x\in\ocsG}%\frac{1}{9}
             \setn{\begin{array}{ccccc}
                18\\
                15\\
                19\\
                 9\\
                15
             \end{array}}
      \quad= \setn{\begin{array}{c}
                \mbox{ }\\
                \mbox{ }\\
                \mbox{ }\\
                3\\
                \mbox{ }
             \end{array}}}
      \\\\
      \ocsceng(\ocsG)
        &\eqd \argmin_{x\in\ocsG}\prod_{y\in\ocsG\setd\setn{x}}{\metric{x}{y}^{\psp(y)}}
        &&\text{by definition of $\ocsceng$ \xref{def:ocsceng}}
      \\&= \argmin_{x\in\ocsG}\prod_{y\in\ocsG\setd\setn{x}}{\metric{x}{y}^{9\psp(y)\frac{1}{9}}}
      \\&= \argmin_{x\in\ocsG}\brs{\prod_{y\in\ocsG\setd\setn{x}}{\metric{x}{y}^{9\psp(y)}}}^\frac{1}{9}
      \\&= \argmin_{x\in\ocsG}\prod_{y\in\ocsG\setd\setn{x}}{\metric{x}{y}^{9\psp(y)}}
        &&\text{because $\ff(x)\eqd x^\frac{1}{9}$ is \prope{strictly isotone} and by \prefpp{lem:argminphi}}
      \\&=\mathrlap{\argmin_{x\in\ocsG}
             \setn{\begin{array}{ccccccccc}
                     &      & {3}^1 &\times& {1}^1 &\times& {4}^2 &\times& {2}^3\\
               {2}^2 &\times&       &      & {3}^1 &\times& {1}^2 &\times& {2}^3\\
               {4}^2 &\times& {2}^1 &\times&       &      & {3}^2 &\times& {1}^3\\
               {1}^2 &\times& {2}^1 &\times& {2}^1 &\times&       &      & {1}^3\\
               {3}^2 &\times& {1}^1 &\times& {4}^1 &\times& {2}^2 &      &
             \end{array}}
      \quad= \argmin_{x\in\ocsG}
             \setn{\begin{array}{c}
               384\\
               192\\
               432\\
                24\\
               144
             \end{array}}
      \quad= \setn{\begin{array}{c}
                \mbox{ }\\
                \mbox{ }\\
                \mbox{ }\\
                3\\
                \mbox{ }
             \end{array}}}
      \\\\
      \ocscenh(\ocsG)
        &\eqd \argmin_{x\in\ocso}\brp{\sum_{y\in\ocso}\frac{1}{\metric{x}{y}}\psp(y)}^{-1} 
        &&\text{by definition of $\ocscenh$ \xref{def:ocscenh}}
      \\&= \argmax_{x\in\ocso}\brp{\sum_{y\in\ocso}\frac{1}{\metric{x}{y}}\psp(y)}
        && \text{because $\fphi(x)\eqd x^{-1}$ is \prope{strictly antitone} and by \prefp{lem:minphia}}
      \\&= \argmax_{x\in\ocso}\brp{\frac{1}{9}\sum_{y\in\ocso}\frac{1}{\metric{x}{y}}\psp(y)9}
      \\&= \argmax_{x\in\ocso}\sum_{y\in\ocso}\frac{9\psp(y)}{\metric{x}{y}}
        &&\text{because $\ff(x)=\frac{1}{9}x$ is \prope{strictly isotone} and by \prefpp{lem:argminphi}}
      \\&=\mathrlap{\argmax_{x\in\ocsG}
             \setn{\begin{array}{*{4}{cc}c}
               0           &+& \frac{1}{3} &+& \frac{1}{1} &+& \frac{2}{4} &+& \frac{3}{2}\\
               \frac{2}{2} &+& 0           &+& \frac{1}{3} &+& \frac{2}{1} &+& \frac{3}{2}\\
               \frac{2}{4} &+& \frac{1}{2} &+& 0           &+& \frac{2}{3} &+& \frac{3}{1}\\
               \frac{2}{1} &+& \frac{1}{2} &+& \frac{1}{2} &+& 0           &+& \frac{3}{1}\\
               \frac{2}{3} &+& \frac{1}{1} &+& \frac{1}{4} &+& \frac{2}{2} &+& 0          
             \end{array}}
      \quad= \argmax_{x\in\ocsG}\frac{1}{2}
             \setn{\begin{array}{c}
               {20}\\
               {27}\\
               {28}\\
               {36}\\
               {35}
             \end{array}}
      \quad= \setn{\begin{array}{c}
                \mbox{ }\\
                \mbox{ }\\
                \mbox{ }\\
                3\\
                \mbox{ }
             \end{array}}}
      \\
      \ocscenm(\ocsG)
      \\&\eqd \argmin_{x\in\ocsG}\min_{y\in\ocsG\setd\setn{x}}\metric{x}{y}\psp(y)
        &&\text{by definition of $\ocscenm$ \xref{def:ocscenm}}
      \\&= \argmin_{x\in\ocsG}\min_{y\in\ocsG\setd\setn{x}}\frac{1}{9}\metric{x}{y}\psp(y)9
      \\&= \argmin_{x\in\ocsG}\min_{y\in\ocsG\setd\setn{x}}\metric{x}{y}\psp(y)9
        &&\text{because $\ff(x)=\frac{1}{9}x$ is \prope{strictly isotone} and by \prefpp{lem:argminmaxphi}}
      \\&=\mathrlap{\argmin_{x\in\ocsG}\min_{y\in\ocsG\setd\setn{x}}%\frac{1}{9}
             \setn{\begin{array}{ccccccccc}
                          & {3}\times1 & {1}\times1 & {4}\times2 & {2}\times3\\
               {2}\times2 &            & {3}\times1 & {1}\times2 & {2}\times3\\
               {4}\times2 & {2}\times1 &            & {3}\times2 & {1}\times3\\
               {1}\times2 & {2}\times1 & {2}\times1 &            & {1}\times3\\
               {3}\times2 & {1}\times1 & {4}\times1 & {2}\times2 &             
             \end{array}}
      \quad= \argmin_{x\in\ocsG}
             \setn{\begin{array}{c}
                1\\
                2\\
                2\\
                2\\
                1
             \end{array}}
      \quad= \setn{\begin{array}{c}
                0\\
                \mbox{ }\\
                \mbox{ }\\
                \mbox{ }\\
                4
             \end{array}}}
    \\
      \ocscenM(\ocsG)
        &\eqd \argmax_{x\in\ocsG}\min_{y\in\ocsG\setd\setn{x}}\metric{x}{y}\psp(y)
        &&\text{by definition of $\ocscenM$ \xref{def:ocscenM}}
      \\&\eqd \argmax_{x\in\ocsG}\setn{1,2,2,2,1}
        &&\text{by $\ocscenm(\ocsG)$ result}
      \\&= \setn{1,2,3}
      \\
      \ocsVaro(\ocsG)
        &= \mathrlap{\ocsVaro(\ocsG;\ocscena)= \ocsVaro(\ocsG;\ocsceng)= \ocsVaro(\ocsG;\ocscenh)}
        && \text{by $\ocscen(\ocsG)$, $\ocscena(\ocsG)$, $\ocsceng(\ocsG)$, and  $\ocscenh(\ocsG)$ results}
      \\&\eqd \sum_{x\in\ocsG}\metricsq{\ocscenh(\ocsG)}{x}\psp(x)
        && \text{by definition of $\ocsVaro$ \xref{def:ocsVarG}}
      \\&= \sum_{x\in\ocsG}\metricsq{\setn{3}}{x}\psp(x)
        && \text{by $\ocscenh(\ocsG)$ result}
      \\&= \mathrlap{
           (1)^2\frac{2}{9}+  % x=0
           (2)^2\frac{1}{9}+  % x=1
           (2)^2\frac{1}{9}+  % x=2
           (0)^2\frac{2}{9}+  % x=3
           (1)^2\frac{3}{9}   % x=4
         = \frac{12}{9} = \frac{4}{3} \approx 1.333}
      \\
      \ocsVaro(\ocsG;\ocscenm)
        &\eqd \sum_{x\in\ocsG}\metricsq{\ocscenm(\ocsG)}{x}\psp(x)
        && \text{by definition of $\ocsVaro$ \xref{def:ocsVarG}}
      \\&= \sum_{x\in\ocsG}\metricsq{\setn{0,4}}{x}\psp(x)
        && \text{by $\ocscenm(\ocsG)$ result}
      \\&= \mathrlap{
           (0)^2\frac{2}{9}+  % x=0
           (1)^2\frac{1}{9}+  % x=1
           (1)^2\frac{1}{9}+  % x=2
           (2)^2\frac{2}{9}+  % x=3
           (0)^2\frac{3}{9}   % x=4
         = \frac{10}{9} \approx 1.111}
      \\
      \ocsVaro(\ocsG;\ocscenM)
        &\eqd \sum_{x\in\ocsG}\metricsq{\ocscenM(\ocsG)}{x}\psp(x)
        && \text{by definition of $\ocsVaro$ \xref{def:ocsVarG}}
      \\&= \sum_{x\in\ocsG}\metricsq{\setn{1,2,3}}{x}\psp(x)
        && \text{by $\ocscenm(\ocsG)$ result}
      \\&= \mathrlap{
           (1)^2\frac{2}{9}+  % x=0
           (0)^2\frac{1}{9}+  % x=1
           (0)^2\frac{1}{9}+  % x=2
           (0)^2\frac{2}{9}+  % x=3
           (1)^2\frac{3}{9}   % x=4
         = \frac{5}{9} \approx 0.555}
    \end{align*}
\end{proof}



%\begin{table}%
%  \gsize%
%  \centering%
%  \begin{tabular}{|r|*{11}{c}|c||c|}%
%     \hline
%         & \mc{11}{c|}{moments $\ocsmom(x,y)/36$ {\scs($\max$ values shaded)}}         &  sum   &             \\
%     x/y & 2 & 3 & 4 & 5 & 6 & 7 & 8 & 9 & 10 & 11 & 12 &\scs ($\min$ value shaded) & $\ocsmom_2\brs{\ocsE(\rvX),x}/36$\\
%     \hline
%     2 & 0 & 2 & 3 & 3 & 5 &\scell{12} &10 & 6 & 6  & 6  & 4 & 57 & 4  \\
%     3 & 1 & 0 & 3 & 3 & 5 & 6 &\scell{10} & 6 & 6  & 4  & 3 & 47 & 2  \\
%     4 & 1 & 2 & 0 & 3 & 5 & \scell{6} & 5 & \scell{6} & \scell{6}  & 4  & 2 & 40 & 3  \\
%     5 & 1 & 2 & 3 & 0 & 5 & \scell{6} & 5 & 3 & \scell{6}  & 4  & 2 & 37 & 3  \\
%     6 & 1 & 2 & 3 & 3 & 0 & \scell{6} & 5 & 3 & 3  & 4  & 2 & 32 & 5  \\
%    {7} & 2 & 2 & 3 & 3 & \scell{5} & 0 & \scell{5} & 3 & 3  & 2  & 2 & \scell{30} & 0  \\
%     8 & 2 & 4 & 3 & 3 & 5 & \scell{6} & 0 & 3 & 3  & 2  & 1 & 32 & 5 \\
%     9 & 2 & 4 & \scell{6} & 3 & 5 & \scell{6} & 5 & 0 & 3  & 2  & 1 & 37 & 3 \\
%    10 & 2 & 4 & \scell{6} & \scell{6} & 5 & \scell{6} & 5 & 3 & 0  & 2  & 1 & 40 & 3 \\
%    11 & 3 & 4 & 6 & 6 & \scell{10}& 6 & 5 & 3 & 3  & 0  & 1 & 47 & 2 \\
%    12 & 4 & 6 & 6 & 6 & 10&\scell{12} & 5 & 3 & 3  & 2  & 0 & 57 & 4 \\
%    \hline
%       &   &   &   &   &   &   &   &   &    &    &   &    &$\pVar(\rvX)=\sfrac{34}{36}$\\
%    \hline
%  \end{tabular}%
%  \caption{moments of a pair of fair dice \xref{ex:dicepair_moments}\label{tbl:pairdice_moments}}
%\end{table}


%The physical geometry inducing a stochastic process and the \structe{metric geometry} \xref{rem:mgeo} 
%of the stochastic process itself 
%may be very different, as illustrated in the next two examples.\\
%{\begin{tabular}{c}%
%  \gsize%
%  \centering%
%  %\psset{unit=3mm}%
%  %{%============================================================================
% Daniel J. Greenhoe
% LaTeX file
% ocs archery
%============================================================================
{%\psset{unit=0.5\psunit}%
\begin{pspicture}(-6,-6)(6,6)%
  %---------------------------------
  % options
  %---------------------------------
  \psset{%
    linecolor=blue,%
    %radius=1.25ex,
    %labelsep=2.5mm,
    }%
  \pscircle(0,0){1}%
  \pscircle(0,0){2}%
  \pscircle(0,0){3}%
  \pscircle(0,0){4}%
  \pscircle(0,0){5}%
  \psframe(-6,-6)(6,6)%
  %
  %
  \rput(0,0){$5$}%
  \rput(0,-1.5){$4$}%
  \rput(0,-2.5){$3$}%
  \rput(0,-3.5){$2$}%
  \rput(0,-4.5){$1$}%
  \rput(0,-5.5){$0$}%
\end{pspicture}
}%}%
%  {\includegraphics{../common/math/graphics/pdfs/archerytarget.pdf}}%
%\end{tabular}}\hfill%
%%---------------------------------------
%\begin{minipage}{\tw-62mm}
%\begin{example}[\exmd{archery}]
%\label{ex:archery}
%%---------------------------------------
%Consider the archery target illustrated to the left.
%It consists of several concentric circles, each with a different point value.
%However, its' \structe{outcome subspace} structure, has a very different geometry, as illustrated to the right.
%Assuming uniform distribution, the graph center is shaded in the illustration to right.
%\end{example}
%\end{minipage}\hfill%
%{\begin{tabular}{c}%
%  \gsize%
%  \centering%
%  %\psset{unit=5mm}%
%  %{%============================================================================
% Daniel J. Greenhoe
% LaTeX file
% ocs archery
%============================================================================
\begin{pspicture}(-0.5,-0.5)(0.5,5.5)%
  %---------------------------------
  % options
  %---------------------------------
  \psset{%
    linecolor=blue,%
    radius=1.25ex,
    labelsep=2.5mm,
    }%
  \Cnode(0,5){L5}%
  \Cnode(0,4){L4}%
  \Cnode[fillstyle=solid,fillcolor=snode](0,3){L3}%
  \Cnode[fillstyle=solid,fillcolor=snode](0,2){L2}%
  \Cnode(0,1){L1}%
  \Cnode(0,0){L0}%
  %
  \ncline{L4}{L5}%
  \ncline{L3}{L4}%
  \ncline{L2}{L3}%
  \ncline{L1}{L2}%
  \ncline{L0}{L1}%
  %
  \rput(L5){$5$}%
  \rput(L4){$4$}%
  \rput(L3){$3$}%
  \rput(L2){$2$}%
  \rput(L1){$1$}%
  \rput(L0){$0$}%
\end{pspicture}%}%
%  {\includegraphics{../common/math/graphics/pdfs/ocsarchery.pdf}}%
%\end{tabular}}
%
%{\begin{tabular}{c}%
%  \gsize%
%  \centering%
%  %\psset{unit=3mm}%
%  %{%============================================================================
% Daniel J. Greenhoe
% LaTeX file
% simplified darts target
%============================================================================
{%\psset{unit=0.5\psunit}%
\begin{pspicture}(-6,-6)(6,6)%
  %---------------------------------
  % options
  %---------------------------------
  \psset{%
    linecolor=blue,%
    %radius=1.25ex,
    %labelsep=2.5mm,
    }%
  \rput[b]{0}{\psline(0,1)(0,5)}%
  \rput[b]{-120}{\psline(0,1)(0,5)}%
  \rput[b]{120}{\psline(0,1)(0,5)}%
  %
  \pscircle(0,0){1}%
  \pscircle(0,0){3}%
  \pscircle(0,0){5}%
  \psframe(-6,-6)(6,6)%
  %
  %
  \rput(-3.464,2){$3$}%
  \rput(3.464,2){$1$}%
  \rput(-1.732,1){$6$}%
  \rput(1.732,1){$4$}%
  \rput(0,0){$7$}%
  \rput(0,-2){$5$}%
  \rput(0,-4){$2$}%
  \rput(5.196,-3){$0$}%
\end{pspicture}
}%}%
%  {\includegraphics{../common/math/graphics/pdfs/dartstarget.pdf}}%
%\end{tabular}}\hfill%
%%---------------------------------------
%\begin{minipage}{\tw-70mm}
%\begin{example}[\exmd{darts}]
%\label{ex:darts}
%%---------------------------------------
%Consider the simplified dart board illustrated to the left
%and \structe{outcome subspace} illustrated to the right.
%%It consists of several concentric circles, each with a different point value.
%Unlike the archery \structe{outcome subspace} \xref{ex:archery}, the outcome subspace is \prope{non-linear}.
%%Its' \structe{outcome subspace} strurture, is has a very different geometry, as illustrated to the right.
%Assuming uniform distribution, the graph center is shaded in the illustration.
%\end{example}
%\end{minipage}\hfill%
%{\begin{tabular}{c}%
%  \gsize%
%  \psset{unit=7mm}%
%  \centering%
%  %{%============================================================================
% Daniel J. Greenhoe
% LaTeX file
% ocs darts
%============================================================================
{%\psset{unit=0.5\psunit}%
\begin{pspicture}(-1.3,-0.3)(1.3,5.3)%
  %---------------------------------
  % options
  %---------------------------------
  \psset{%
    linecolor=blue,%
    radius=1.25ex,
    labelsep=2.5mm,
    }%
  \Cnode(0,5){D7}%
  \Cnode[fillstyle=solid,fillcolor=snode](-1,4){D6}%
  \Cnode[fillstyle=solid,fillcolor=snode]( 1,4){D4}%
  \Cnode[fillstyle=solid,fillcolor=snode]( 0,3){D5}%
  \Cnode[fillstyle=solid,fillcolor=snode](-1,2){D3}%
  \Cnode[fillstyle=solid,fillcolor=snode]( 0,1){D2}%
  \Cnode[fillstyle=solid,fillcolor=snode]( 1,2){D1}%
  \Cnode(0,0){D0}%
  %
  \ncline{D4}{D5}\ncline{D5}{D6}\ncline{D6}{D4}%
  \ncline{D1}{D2}\ncline{D2}{D3}\ncline{D3}{D1}%
  \ncline{D7}{D4}\ncline{D7}{D5}\ncline{D7}{D6}%
  \ncline{D3}{D6}\ncline{D2}{D5}\ncline{D1}{D4}%
  \ncline{D0}{D1}\ncline{D0}{D2}\ncline{D0}{D3}%
  %
  \rput(D7){$7$}%
  \rput(D6){$6$}%
  \rput(D5){$5$}%
  \rput(D4){$4$}%
  \rput(D3){$3$}%
  \rput(D2){$2$}%
  \rput(D1){$1$}%
  \rput(D0){$0$}%
\end{pspicture}
}%}%
%  {\includegraphics{../common/math/graphics/pdfs/ocsdarts.pdf}}%
%\end{tabular}}
%
%
%---------------------------------------
\begin{example}[\exmd{DNA}]
\label{ex:dna}
%---------------------------------------
\structe{Genomic Signal Processing} (\structe{GSP}) analyzes biological sequences called \structe{genome}s.
These sequences are constructed over a set of 4 symbols that are commonly referred to as 
$\symA$, $\symT$, $\symC$, and $\symG$,
each of which corresponds to a nucleobase (adenine,  thymine, cytosine, and guanine, 
respectively).\footnote{
  \citePc{mendel1853e}{Mendel (1853): gene coding uses discrete symbols},
  \citePpc{watson1953}{737}{Watson and Crick (1953): gene coding symbols are adenine,  thymine, cytosine, and guanine},
  \citePp{watson1953may}{965},
  \citerpg{pommerville2013}{52}{1449647960}
  }
A typical genome sequence contains a large number of symbols 
(about 3 billion for humans, 29751 for the SARS virus).%
\footnote{%
  \citeWuc{genbank}{http://www.ncbi.nlm.nih.gov/genome/guide/human/}{Homo sapiens, NC\_000001--NC\_000022 (22 chromosome pairs), NC\_000023 (X chromosome), NC\_000024 (Y chromosome), NC\_012920 (mitochondria)},
  \citeWuc{genbank}{http://www.ncbi.nlm.nih.gov/nuccore/30271926}{SARS coronavirus, NC\_004718.3}
  \citePc{gregory2006}{homo sapien chromosome 1},
  \citePc{he2004}{SARS coronavirus}
  }
\\[0.3ex]\begin{minipage}{\tw-37mm}%
Let $\ocsG\eqd\ocs{\setn{\symA,\symT,\symC,\symG}}{\metricn}{\orel}{\psp}$ 
be the \structe{outcome subspace} \xref{def:ocsm} generated by a \structe{genome}
where $\metricn$ is the \fncte{discrete metric} \xref{def:dmetric},
$\orel\eqd\emptyset$ (completely unordered set), and 
$\psp(\symA)=\psp(\symT)=\psp(\symC)=\psp(\symG)=\frac{1}{4}$.
This space is illustrated by the \structe{graph} \xref{def:graph} to the right
with shaded \structe{center} \xref{def:ocscen}.
\end{minipage}%
\hfill%
{\begin{tabular}{c}%
  \gsize%
  \psset{unit=8mm}%
  \centering%
  %{%============================================================================
% Daniel J. Greenhoe
% LaTeX file
%============================================================================
{%\psset{unit=0.5\psunit}%
\begin{pspicture}(-1.4,-1.4)(1.4,1.4)%
  %---------------------------------
  % options
  %---------------------------------
  \psset{%
    linecolor=blue,%
    radius=1.25ex,
    labelsep=2.5mm,
    }%
  %---------------------------------
  % DNA graph
  %---------------------------------
  \rput(0,0){%
    \uput{1}[135](0,0){\Cnode[fillstyle=solid,fillcolor=snode](0,0){Da}}%
    \uput{1}[ 45](0,0){\Cnode[fillstyle=solid,fillcolor=snode](0,0){Dt}}%
    \uput{1}[225](0,0){\Cnode[fillstyle=solid,fillcolor=snode](0,0){Dc}}%
    \uput{1}[-45](0,0){\Cnode[fillstyle=solid,fillcolor=snode](0,0){Dg}}%
    %\Cnode[fillstyle=solid,fillcolor=snode](-1, 1){Da}%
    %\Cnode[fillstyle=solid,fillcolor=snode]( 1, 1){Dt}%
    %\Cnode[fillstyle=solid,fillcolor=snode](-1,-1){Dc}%
    %\Cnode[fillstyle=solid,fillcolor=snode]( 1,-1){Dg}%
    %\Cnode[fillstyle=solid,fillcolor=snode](-0.707, 0.707){Da}%
    %\Cnode[fillstyle=solid,fillcolor=snode]( 0.707, 0.707){Dt}%
    %\Cnode[fillstyle=solid,fillcolor=snode](-0.707,-0.707){Dc}%
    %\Cnode[fillstyle=solid,fillcolor=snode]( 0.707,-0.707){Dg}%
    }%
  \ncline{Dc}{Dg}%
  \ncline{Dt}{Dc}\ncline{Dt}{Dg}%
  \ncline{Da}{Dt}\ncline{Da}{Dc}\ncline{Da}{Dg}%
  \rput(Dg){$\symG$}%
  \rput(Dc){$\symC$}%
  \rput(Dt){$\symT$}%
  \rput(Da){$\symA$}%
  %
  \uput[-45](Dg){$\frac{1}{4}$}
  \uput[-135](Dc){$\frac{1}{4}$}
  \uput[45](Dt){$\frac{1}{4}$}
  \uput[135](Da){$\frac{1}{4}$}
\end{pspicture}
}%}%
  {\includegraphics{../common/math/graphics/pdfs/dna.pdf}}%
\end{tabular}}
\\
The graph has the following geometric values:
  \\\indentx$\begin{array}{rclDD}
    \ocscen(\ocsG)  &=& \setn{\symA,\symT,\symC,\symG} & \xref{def:ocscen}  & (shaded in illustration)\\
    \ocscena(\ocsG) &=& \setn{\symA,\symT,\symC,\symG} & \xref{def:ocscenx} & (shaded in illustration)\\
    \ocsVaro(\ocsG)  &=& 0                              & \xref{def:ocsVarG}     
  \end{array}$
\end{example}
\begin{proof}
\begin{align*}
  \ocscen(\ocsG)
    &\eqd \argmin_{x\in\ocsG}\max_{y\in\ocsG}\metric{x}{y}\psp(y)
    && \text{by definition of $\ocscen$ \xref{def:ocscen}}
  \\&= \argmin_{x\in\ocsG}\max_{y\in\ocsG}\metric{x}{y}\frac{1}{4}
    && \text{by definition of $\ocsG$}
  \\&= \argmin_{x\in\ocsG}\max_{y\in\ocsG}\metric{x}{y}
    && \text{because $\ff(x)=\frac{1}{4}x$ is \prope{strictly isotone} and by \prefpp{lem:argminmaxphi}}
  \\&= \argmin_{x\in\ocsG}\setn{1,\,1,\,1,\,1}
    && \text{because for the \fncte{discrete metric} \xref{def:dmetric}, $\max\metricn=1$}
  \\&= \setn{\symA,\symT,\symC,\symG}
    && \text{by definition of $\ocsG$}
  \\
  \ocscena(\ocsG)
    &\eqd \argmin_{x\in\ocsG}\sum_{y\in\ocsG}\metric{x}{y}\psp(y)
    &&\text{by definition of $\ocscena$ \xref{def:ocscenx}}
  \\&= \argmin_{x\in\ocsG}\sum_{y\in\ocsG}\metric{x}{y}\frac{1}{6}
    &&\text{by definition of $\ocsG$}
  \\&= \argmin_{x\in\ocsG}\sum_{y\in\ocsG}\metric{x}{y}
    &&\text{because $\ff(x)=\frac{1}{4}x$ is \prope{strictly isotone} and by \prefpp{lem:argminmaxphi}}
  \\&= \mathrlap{\argmin_{x\in\ocsG}
         \setn{\begin{array}{*{11}{@{\hspace{2pt}}c}}
           0 &+& 1 &+& 1 &+& 1\\
           1 &+& 0 &+& 1 &+& 1\\
           1 &+& 1 &+& 0 &+& 1\\
           1 &+& 1 &+& 1 &+& 0
         \end{array}}
       = \argmin_{x\in\ocsG}
         \setn{\begin{array}{c}
           3\\
           3\\
           3\\
           3  
         \end{array}}
     = \setn{\symA,\symT,\symC,\symG}}
    \\
  \ocsVaro(\ocsG)
    &\eqd \sum_{x\in\ocsG}\brs{\metric{\ocscen(\ocsG)}{x}}^2\psp(x)
    && \text{by definition of $\ocsVaro$ \xref{def:ocsVarG}}
  \\&= \sum_{x\in\ocsG}(0)\frac{1}{6}
    && \text{because $\ocscen(\ocsG)=\ocsG$}
  \\&= 0
    && \text{by field property of \vale{additive identity element} $0$}
\end{align*}
\end{proof}


%
%%---------------------------------------
%\begin{example}[\exmd{DQPSK}]
%\footnote{
%  \citer{wesolowski2009}
%  }
%\label{ex:dqpsk}
%%---------------------------------------
%In digital communications, there are several modulation techniques available.
%Most of these manipulate (``modulate") the parameters a sinusoidal signal (called the \structe{carrier})
%at the transmitter to ``carry" information (such as a person's voice) to a receiver
%where under certain reasonable conditions the information can be recovered 
%with an acceptably low error rate (e.g. $\le0.0001$\%).
%Parameters of the sinusoid that may be manipulated include the sinusoid's amplitude
%(\hie{Amplitude Shift Keying} or \hie{ASK}), frequency (\hie{Frequency Shift Keying} or \hie{FSK}),
%or phase (\hie{Phase Shift Keying} or \hie{PSK}).
%The information to be carried is first encoded into a sequence of ``symbols", 
%with each symbol carrying $\xN$ bits (typically $\xN=$ 1, 2, or 3).
%All of modulation techniques generate a code space with $2^\xN$ code points.
%In each modulation techique, the receiver must somehow have a reference by which it can recover the 
%information from the carrier.
%A receiver may easily generate an amplitude reference for ASK modulation by using a 
%simple low pass filter to find the 0 Hertz component of the received signal.
%A receiver may generate a frequency reference for FSK modulation by using an oscillator circuit
%that oscillates at the same frequency as the unmodulated carrier.
%For PSK, generating a reference cannot be done without assistance from the transmitter. 
%The transmitter may provide this assistance by also transmitting a reference sinusoid (one that does not 
%change it's phase), or by encoding the reference signal into the information sequence itself.
%\\[0.3ex]\begin{minipage}{\tw-37mm}%
%One way to do the latter using $\xN=2$ bit encoding is a modulation technique called 
%\hie{Differential Quadrature Phase Shift Keying} or \hie{DQPSK}.
%In DQPSK, consecutive code points at the transmitter cannot change by more than 1 bit.
%In the illustration, this means that they cannot ``jump" across the square to the opposite corner. 
%That is, each symbol is partly a function of the previous symbol.
%By doing so, the phase information can be encoded into the sequence.
%\end{minipage}%
%\hfill%
%{\begin{tabular}{c}%
%  \gsize%
%  \psset{unit=10mm}%
%  \centering%
%  %{%============================================================================
% D11niel J. Greenhoe
% LaTeX file
% lattice M2 on M2
%============================================================================
{%\psset{unit=0.5\psunit}%
\begin{pspicture}(-1.5,-1.5)(1.5,1.5)%
  %---------------------------------
  % options
  %---------------------------------
  \psset{%
    linecolor=blue,%
    radius=1.35ex,
    labelsep=2.5mm,
    }%
  %---------------------------------
  % DNA graph
  %---------------------------------
  \rput(0,0){%
    %\psaxes[linecolor=axis,linewidth=0.5pt]{<->}(0,0)(-1.5,1.5)(-1.5,1.5)% x axis
    \Cnode(-1, 1){D11}%
    \Cnode( 1, 1){D01}%
    \Cnode(-1,-1){D10}%
    \Cnode( 1,-1){D00}%

    }%
  \ncline{D10}{D00}\nbput[labelsep=0pt]{${\scy\metric{00}{10}=}1$}%%
  \ncline{D10}{D01}\naput[labelsep=0pt,nrot=:U,npos=0.25]{${\scy\metric{01}{10}=}2$}%
  \ncline{D00}{D01}\nbput[labelsep=0pt,nrot=:U]{${\scy\metric{00}{01}=}1$}%
  \ncline{D11}{D01}\naput[labelsep=0pt]{${\scy\metric{01}{11}=}1$}%
  \ncline{D10}{D11}\naput[labelsep=0pt,nrot=:U]{${\scy\metric{10}{11}=}1$}%
  \ncline{D11}{D00}\naput[labelsep=0pt,nrot=:U]{${\scy\metric{00}{11}=}2$}%
  %
  \rput(D00){$00$}%
  \rput(D10){$10$}%
  \rput(D01){$01$}%
  \rput(D11){$11$}%
  %
  %\uput[-45](D00){$\frac{1}{4}$}
  %\uput[-135](D10){$\frac{1}{4}$}
  %\uput[45](D01){$\frac{1}{4}$}
  %\uput[135](D11){$\frac{1}{4}$}
\end{pspicture}
}%}%
%  {\includegraphics{../common/math/graphics/pdfs/dqpsk.pdf}}%
%\end{tabular}}
%\end{example}
%






%\begin{figure}
%  \gsize%
%  \centering%
%  \psset{unit=10mm}%
%  {%============================================================================
% Daniel J. Greenhoe
% LaTeX file
% spinner 6 mapping to linearly ordered L6
%============================================================================
{\psset{yunit=1.25\psunit}%
\begin{pspicture}(-6.5,-0.5)(6.5,3.5)%
  %---------------------------------
  % options
  %---------------------------------
  \psset{%
    linecolor=blue,%
    cornersize=relative,
    framearc=0.25,
    subgriddiv=1,
    gridlabels=4pt,
    gridwidth=0.2pt,
    }%
  %---------------------------------
  % states
  %---------------------------------
     \begin{tabstr}{0.75}
     \rput( 5,1.5){\rnode{Sdeposit} {\psframebox{\begin{tabular}{c}deposit coins\\in vault\end{tabular}}}}%
     \rput(-5,1.5){\rnode{Sreturn}  {\psframebox{\begin{tabular}{c}return coins          \end{tabular}}}}%
     \rput( 0,3)  {\rnode{Seject}   {\psframebox{\begin{tabular}{c}deliver product       \end{tabular}}}}%
     \rput( 0,2)  {\rnode{Scheck}   {\psframebox{\begin{tabular}{c}check availability    \end{tabular}}}}%
     \rput( 0,1)  {\rnode{Swait}    {\psframebox{\begin{tabular}{c}wait for selection    \end{tabular}}}}%
     \rput( 0,0)  {\rnode{Sidle}    {\psframebox{\begin{tabular}{c}idle                  \end{tabular}}}}%
     \end{tabstr}
  %\Cnode(3,1.5) {Sdeposit}%
  %\Cnode(-3,1.5){Sreturn}%
  %\Cnode(0,3)   {Seject}%
  %\Cnode(0,2)   {Scheck}%
  %\Cnode(0,1)   {Swait}%
  %\Cnode(0,0)   {Sidle}%
  %
  %\uput[  90](Seject)  {deliver product}%
  %\uput[   0](Sdeposit){deposit coins in vault}%
  %\uput[ -90](Sreturn) {return coins}%
  %\uput[ 180](Scheck)  {check availability}%
  %\uput[ 180](Swait)   {wait for selection}%
  %\uput[ -90](Sidle)   {idle}%
  %---------------------------------
  % edges
  %---------------------------------
  {\psset{labelsep=1pt}%
  \ncline{->}{Sreturn}{Sidle}\nbput[nrot=:U]{coins returned}%
  \ncline{<-}{Sreturn}{Seject}\naput[nrot=:U]{jammed}%
  \ncline{->}{Seject}{Sdeposit}\naput[nrot=:U]{delivery successful}%
  \ncline{<-}{Sidle}{Sdeposit}\nbput[nrot=:U]{deposit successful}%
  \ncline{<-}{Sreturn}{Scheck}\nbput[nrot=:U]{not available}%
  \ncline{->}{Scheck}{Seject}\ncput{available}%
  \ncline{->}{Swait}{Scheck}\ncput{selection made}%
  \ncline{->}{Sidle}{Swait}\ncput{coins inserted}%
  }%
\end{pspicture}
}%}%
%  \caption{state machine for vending machine \xref{ex:vending}\label{fig:vending}}
%\end{figure}
%%---------------------------------------
%\begin{example}[\exmd{state machine}]
%\label{ex:vending}
%%---------------------------------------
%A simple state machine for a vending machine is illustrated with a directed graph in \prefpp{fig:vending}.
%At any given time, the state machine is in exactly one of the 6 states. 
%The sequence of states it is in at sampled time intervals is a random process.
%Again, the order and topology of this structure is very dissimilar to that of the real line mapped to by 
%the traditional random variable.
%\end{example}
%
%%\begin{figure}
%%  \gsize%
%%  \centering%
%%  \psset{unit=5mm}%
%%  {%============================================================================
% Daniel J. Greenhoe
% LaTeX file
% "Fishing in Sierpinski's Sea" illustration
%============================================================================
{\psset{xunit=1.5\psunit}%
\begin{pspicture}(-1.1,-0.5)(1.1,4.5)%
  %---------------------------------
  % options
  %---------------------------------
  \psset{%
    linecolor=blue,%
    %cornersize=relative,
    %framearc=0.25,
    %subgriddiv=1,
    %gridlabels=4pt,
    %gridwidth=0.2pt,
    }%
  %---------------------------------
  % states
  %---------------------------------
  \Cnode(1,2.66){H}
  \Cnode(1,1.33){G}
  \Cnode(-1,2)  {F}
  \Cnode(0,4)   {B}
  \Cnode(0,3)   {E}
  \Cnode(0,2)   {D}
  \Cnode(0,1)   {C}
  \Cnode(0,0)   {A}
  %
  \uput[ 0](B){$B$}
  \uput[ 0](A){$A$}
  %---------------------------------
  % edges
  %---------------------------------
  \ncline{G}{D}%
  \ncline{H}{D}\ncline{H}{C}\ncline{H}{G}%
  \ncline{E}{B}\ncline{B}{F}\ncline{B}{H}%
  \ncline{D}{E}%
  \ncline{C}{D}%
  \ncline{A}{C}\ncline{A}{F}\ncline{A}{G}%
\end{pspicture}
}%}%
%%  \caption{state machine for vending machine \xref{ex:network}\label{fig:network}}
%%\end{figure}
%\begin{minipage}{\tw-25mm}%
%%---------------------------------------
%\begin{example}[\exmd{network}]
%\label{ex:vending}
%%---------------------------------------
%A simple network is illustrated with an undirected graph to the right. %in \prefpp{fig:network}.
%Suppose a packet is sent from $A$ to $B$. 
%%At any given time, the packet is in exactly one of the 8 servers. 
%The sequence of servers the packet transverses through is a random process.
%Again, the order and topology of this structure is very dissimilar to that of the real line mapped to by 
%the traditional random variable.
%\end{example}
%\end{minipage}\hfill%
%\begin{tabular}{c}
%  \gsize%
%  \centering%
%  \psset{unit=5mm}%
%  {%============================================================================
% Daniel J. Greenhoe
% LaTeX file
% "Fishing in Sierpinski's Sea" illustration
%============================================================================
{\psset{xunit=1.5\psunit}%
\begin{pspicture}(-1.1,-0.5)(1.1,4.5)%
  %---------------------------------
  % options
  %---------------------------------
  \psset{%
    linecolor=blue,%
    %cornersize=relative,
    %framearc=0.25,
    %subgriddiv=1,
    %gridlabels=4pt,
    %gridwidth=0.2pt,
    }%
  %---------------------------------
  % states
  %---------------------------------
  \Cnode(1,2.66){H}
  \Cnode(1,1.33){G}
  \Cnode(-1,2)  {F}
  \Cnode(0,4)   {B}
  \Cnode(0,3)   {E}
  \Cnode(0,2)   {D}
  \Cnode(0,1)   {C}
  \Cnode(0,0)   {A}
  %
  \uput[ 0](B){$B$}
  \uput[ 0](A){$A$}
  %---------------------------------
  % edges
  %---------------------------------
  \ncline{G}{D}%
  \ncline{H}{D}\ncline{H}{C}\ncline{H}{G}%
  \ncline{E}{B}\ncline{B}{F}\ncline{B}{H}%
  \ncline{D}{E}%
  \ncline{C}{D}%
  \ncline{A}{C}\ncline{A}{F}\ncline{A}{G}%
\end{pspicture}
}%}%
%\end{tabular}
%
%%---------------------------------------
%\begin{example}[\exmd{Fishing in Sierpi/'nski's Sea}]
%\label{ex:ssea}
%%---------------------------------------
%Suppose there is a sea, called ``Sierpi/'nski's Sea", with only two fish---one blue fish and one red fish.
%This is a very special sea with very special fish.
%You throw a net into Sierpi/'nski's Sea.
%In this sea, there are only three possibilities of what you may find when you pull the net back in:
%\\\indentx$\begin{array}{clM}
%  \circOne   & \emptyset:  & the net is empty\\
%  \circTwo   & \setn{b}:   & the blue fish is in the net\\
%  \circThree & \setn{b,r}: & both the blue and red fish are in the net
%\end{array}$\\
%However, in this very special sea of very special fish, the blue fish is a very loyal friend to the 
%red fish and so he will always rescue the red fish from your net, or be caught together with the red fish attempting 
%to do so. Thus, it is not possible that only the red fish is in the net.
%The set of three possible sets $\topT\eqd\setn{\emptyset,\,\setn{b},\,\setn{b,r}}$ form a topology on $\setX\eqd\setn{b,r}$
%that is \prope{non-metrizable}. 
%The \structe{topological space} $\opair{\setX}{\topT}$ is called \structe{Sierpi/'nski's space}.
%\\\begin{minipage}{\tw-25mm}%
%Suppose the probabilities of the three outcomes are 
%$\psp(\emptyset)=\frac{5}{8}$, $\psp(\setn{b})=\frac{1}{8}$, and $\psp(\setn{b,r})=\frac{1}{4}$.
%The structure $\ocs{\setX}{\topT}{\subseteq}{\psp}$ is an \structe{outcome subspace} \xref{def:ocs}.
%The set $\setX$ is ordered using the subset relation $\subseteq$, producing a 
%\structe{directed graph} \xref{def:graph} that is also a \structe{linearly ordered lattice} \xref{def:chain},
%as illustrated to the right.
%\end{minipage}\hfill%
%\begin{tabular}{c}
%  \gsize%
%  \centering%
%  \psset{unit=5mm}%
%  {%============================================================================
% Daniel J. Greenhoe
% LaTeX file
% "Fishing in Sierpinski's Sea" illustration
%============================================================================
\begin{pspicture}(-2,-0.5)(2,2.5)%
  %---------------------------------
  % options
  %---------------------------------
  \psset{%
    linecolor=blue,%
    %cornersize=relative,
    %framearc=0.25,
    %subgriddiv=1,
    %gridlabels=4pt,
    %gridwidth=0.2pt,
    }%
  %---------------------------------
  % states
  %---------------------------------
  \Cnode(0,2)   {BR}
  \Cnode(0,1)   {B}
  \Cnode(0,0)   {E}
  %
  \uput[180](BR){$\setn{b,r}$}
  \uput[180](B){$\setn{b}$}
  \uput[180](E){$\emptyset$}
  %
  \uput[0](BR){${\scy\psp(\setn{b,r})=\frac{1}{4}}$}
  \uput[0](B) {${\scy\psp(\setn{b})=\frac{1}{8}}$}
  \uput[0](E) {${\scy\psp(\emptyset)=\frac{5}{8}}$}
  %---------------------------------
  % edges
  %---------------------------------
  \ncline{B}{BR}%
  \ncline{E}{B}%
\end{pspicture}%}%
%\end{tabular}
%\end{example}
%
%
%%---------------------------------------
%\begin{example}[\exmd{program termination}]
%\label{ex:csob}
%%---------------------------------------
%For a certain computer program, define the following:\footnote{
%  \citeI{smyth1992},
%  \citePp{schroder2006}{605}
%  }
%\\\indentx$\begin{array}{cM}
%  \setn{\downtack}:         & the program terminates after a finite time\\
%  \setn{\uptack}:           & the program does \emph{not} terminate in a finite time\\
%  \setn{\downtack,\uptack}: & the program has been executed\\
%  \emptyset:                & the program has not been executed
%\end{array}$\\
%That is, $\setn{\uptack}$ indicates a program is ``stuck" in an ``infinite loop".
%The topology $\topT\eqd\setn{\emptyset,\setn{\downtack},\setn{\downtack,\uptack}}$ 
%on the set $\setX\eqd\setn{\downtack,\uptack}$ is a \structe{Sierpi/'nsk Space} \xref{ex:ssea}.
%The \structe{open sets} \xref{def:openset} of $\topT$ are all \prope{observable}.
%However, $\setn{\uptack}$ is \emph{not} observable.
%\end{example}

%\fi

\end{tabstr}

  %============================================================================
% Daniel J. Greenhoe
% XeLaTeX file
% stochastic systems
%============================================================================

%=======================================
\section{Random variables on outcome subspaces}
%=======================================

%=======================================
\subsection{Definitions}
%=======================================
The traditional \fncte{random variable} \xref{def:rvt} is a mapping from a \structe{probability space} \xref{def:ps}
to the \structe{real line} \xref{def:Rline}.
This paper extends this definition to include functions with additional structure in the domain and 
expanded structure in the range (next definition).
%---------------------------------------
\begin{definition}
\label{def:ocsrv}
%---------------------------------------
%Let $\clFxy$ be the set of all functions with domain $\setX$ and range $\setY$.\\
\defboxp{
  A function $\rvX\in\clOCSgh$ \xref{def:clFxy} is an \fnctd{outcome random variable} if
  $\ocsG$ is an \structe{outcome subspace} \xref{def:ocsm} and
  $\omsH$ is an \structe{ordered quasi-metric space} \xref{def:oms}.
  }
\end{definition}

The definitions of \fncte{outcome expected value} and \fncte{outcome variance} (next definition) of an 
\fncte{outcome random variable}
are, in essence, identical to 
the \fncte{outcome center} \xref{def:ocsE} and \fncte{outcome variance} \xref{def:ocsVar} of 
\structe{outcome subspace}s \xref{def:ocs}
that \fncte{outcome random variable}s map from and by induction, to.
%---------------------------------------
\begin{definition}
\label{def:ocsE}
\label{def:ocsEa}
\label{def:ocsVar}
%---------------------------------------
Let $\ocsG$ be an \structe{outcome subspace} \xref{def:ocs},
    $\omsH$ an \structe{ordered quasi-metric space} \xref{def:oms},
and $\rvX$ be an \structe{outcome random variable} \xref{def:ocsrv} in $\in\clOCSgh$.
Let $\ocsH\eqd\ocsD$ be the \structe{outcome subspace} induced by $\omsH$, $\ocsG$, and $\rvX$.
Let $\ocsE_x$ be a function from $\ocso$ to the power set $\pset{\ocso}$.
%Let $\ocscen(\ocsH)$, $\ocscena(\ocsH)$, and $\ocsVaro(\ocsH)$ be the 
%\structe{outcome center}, \structd{arithmetic center} \xref{def:ocscen}, and 
%\structe{outcome variance} \xref{def:ocsVarG}, 
%respectively, of $\ocsH$.
\defbox{\begin{array}{MlM>{\ds}l}
  The \fnctd{outcome expected value}   &\ocsE (\rvX)       & of $\rvX$ is &\ocsE (\rvX) \eqd \argmin_{x\in\ocso}\max_{y\in\ocso}\metric{x}{y}\psp(y).\\
  The \fnctd{outcome variance}         &\ocsVar(\rvX;\pE_x)& of $\rvX$ is &\ocsVar(\rvX)\eqd \sum_{x\in\ocso}\metricsq{\pE_x(\rvX)}{x}\psp(x).\\
  \mc{4}{M}{Moreover, $\ocsVar(\rvX)\eqd\ocsVar(\rvX;\ocsE)$, where $\ocsE$ is the \fncte{outcome expected value} function.}      
  %The \fnctd{outcome expected value} &\ocsE (\rvX)  & of $\rvX$ is &\ocsE (\rvX) \eqd \ocscen (\ocsH). \\
  %The \fnctd{outcome variance}       &\ocsVar(\rvX) & of $\rvX$ is &\ocsVar(\rvX)\eqd \ocsVaro(\ocsH).
%  The \fnctd{arithmetic expected value} &\ocsEa(\rvX)       & of $\rvX$ is &\ocsEa(\rvX) \eqd \ocscena(\ocsH).\\
%  The \fnctd{arithmetic expected value} &\ocsEa(\rvX)       & of $\rvX$ is &\ocsEa(\rvX) \eqd \argmin_{x\in\ocso}\sum_{y\in\ocso}\metric{x}{y}\psp(y).\\
%  \mc{4}{M}{Also, $\ocsVar(\rvX)\eqd\ocsVar(\rvX;\ocsE)$.}      
\end{array}}
\end{definition}

%%---------------------------------------
%\begin{definition}
%\label{def:ocsE}
%\label{def:ocsEa}
%\label{def:ocsVar}
%%---------------------------------------
%Let $\rvX\in\clOCSgh$ be a \structe{random variable} \xref{def:ocsrv}.
%Let $\omsH'$ be the \structe{outcome subspace} \xref{def:ocs} generated by 
%the \structe{ordered quasi-metric space} $\omsH$ \xref{def:oms}, 
%the \structe{outcome subspace} $\ocsG$,
%and the \fncte{random variable} $\rvX$.
%Let $\ocscen(\omsH')$ be the \structe{center} \xref{def:ocscen},
%    $\ocscena(\omsH')$ the \structe{arithmetic center} \xref{def:ocscena}, 
%and $\ocsVaro(\omsH')$ the \fncte{outcome variance} \xref{def:ocsVarG} 
%of $\omsH'$.
%\\\defboxt{
%  The \fnctd{expected value} $\ocsE(\rvX)$ of $\rvX$ and 
%  the \fnctd{arithmetic expected value} $\ocsEa(\rvX)$ of $\rvX$ is
%  \\\indentx$\ds\ocsE (\rvX)\eqd \ocscen(\omsH')  \qquad\qquad   \ds\ocsEa(\rvX)\eqd \ocscena(\omsH')$.\\
%  The \fnctd{variance} $(\rvX)$ of $\rvX$ is 
%  \\\indentx$\ds\ocsVar(\rvX)\eqd\ocsVaro(\omsH)$.
%  }
%\end{definition}

%=======================================
\subsection{Properties}
%=======================================
%%---------------------------------------
%\begin{theorem}
%\label{thm:E}
%%---------------------------------------
%\mbox{}\\\begin{tabular}{lMll}
%    Let & \ocsG\eqd\ocs{\ocso_\ocsG}{\metricn_\ocsG}{\orel_\ocsG}{\psp_\ocsG} & be an \structe{outcome subspace}                         & \xref{def:ocs}.
%  \\Let & \omsH\eqd\oms{\ocso_\omsH}{\metricn_\omsH}{\orel_\omsH}             & be an \structe{ordered quasi-metric space}               & \xref{def:oqms}.
%  \\Let & \omsK\eqd\oms{\ocso_\omsK}{\metricn_\omsK}{\orel_\omsK}             & be an \structe{ordered quasi-metric space}              & \xref{def:oqms}.
%  \\Let & \rvX\in\clOCSgh                                                     & be a \structe{random variable} from $\ocsG$ onto $\omsH$ & \xref{def:ocsrv}.
%  \\Let & \ff\in\clF{\ocso_\omsH}{\ocso_\omsK}                                & be a function from $\ocso_\omsH$ onto $\ocso_\omsK$.      &
%  \\Let & \fg\in\clF{\R}{\R}                                                  & be a function from $\R$ into $\R$.                        &
%  \\Let & \ocsH\eqd\ocs{\ocso_\omsH}{\metricn_\omsH}{\orel_\omsH}{\psp_\omsH} & \mc{2}{l}{be an \structe{outcome subspace} induced by $\ocsG$, $\omsH$, and $\rvX$.}
%  \\Let & \ocsK\eqd\ocs{\ocso_\omsK}{\metricn_\omsK}{\orel_\omsK}{\psp_\omsK} & \mc{2}{l}{be an \structe{outcome subspace} induced by $\omsK$, $\ocsH$ and $\ff$.}
%\end{tabular}
%\thmbox{
%  \brb{\begin{array}{FMDD}
%    1. & $\ff$ is \prope{bijective}                           &  & and\\
%    2. & $\fg$ is \prope{isotone}                            &  & and\\
%    3. & $\metricn_\omsH\brp{\ff(x),\ff(y)} = \metricn_\omsH\brp{x,y}$ &
%    %1. & $\omsH$ and $\omsK$ are \prope{isometric}            & \xref{def:isometry} & and\\
%    %2. & $\ff$ is an \prope{isometry} in $\clF{\omsH}{\omsK}$ & \xref{def:isometry} &
%  \end{array}}
%  \quad\implies\quad
%  \brb{\ocsE\brs{\ff(\rvX)} = \ff\brs{\ocsE(\rvX)}}
%  }
%\end{theorem}
%\begin{proof}
%\begin{align*}
%  \ocsE\brs{\ff(\rvX)}
%    &= \argmin_{x\in\ocso_\omsK}\max_{y\in\ocso_\omsK} \metricn_\omsK(x,y)\psp_\omsK(y)
%    && \text{by definition of $\ocsE$ \xref{def:ocsE} and $\omsK$}
%  \\&= \ff\brs{\argmin_{x\in\ocso_\omsH}\max_{y\in\ocso_\omsH} \metricn_\omsK\brp{\ff(x),\ff(y)}\psp_\omsK(\ff(y))}
%    && \text{by $\ff$ \prope{bijection} hypothesis}
%  \\&= \ff\brs{\argmin_{x\in\ocso_\omsH}\max_{y\in\ocso_\omsH} \metricn_\omsH\brp{\ff(x),\ff(y)}\psp_\omsH(y)}
%    && \text{by $\ff$ \prope{bijection} hypothesis}
%  \\&= \ff\brs{\argmin_{x\in\ocso_\omsH}\max_{y\in\ocso_\omsH} \fg\brs{\metricn_\omsH(x,y)}\psp_\omsH(y)}
%    && \text{by $\metricn_\omsH$ hypothesis}
%  \\&= \ff\brs{\argmin_{x\in\ocso_\omsH}\fg\brs{\max_{y\in\ocso_\omsH} \metricn_\omsH(x,y)}\psp_\omsH(y)}
%    && \text{by $\fg$ is \prope{isotone} hypothesis}
%  \\&= \ff\brs{\argmin_{x\in\ocso_\omsH}\max_{y\in\ocso_\omsH} \metricn_\omsH(x,y)\psp_\omsH(y)}
%    && \text{by $\fg$ is \prope{isotone} hypothesis}
%  \\&= \ff\ocsE(\rvX)
%    && \text{by definition of $\ocsE$ \xref{def:ocsE} and $\rvX$}
%\end{align*}
%\end{proof}

%---------------------------------------
\begin{theorem}
\label{thm:ocsVar}
% 2015 February 03 Tuesday ~6:00 PM Taiwan
%---------------------------------------
Let $\rvX\in\clOCSgh$ be a \fncte{random variable} \xref{def:ocsrv}
on an \structe{ordered quasi-metric space} \xref{def:oqms} $\omsH$.
Let $\ocsH\eqd\ocsD$ be the \structe{outcome subspace} \xref{def:ocs} induced by $\omsH$, $\ocsG$, and $\rvX$.
Let $\pVar(\rvX)$ be the \fncte{traditional variance} \xref{def:pVar} of $\rvX$.
Let $\ocsVar(\rvX)$ be the \fncte{outcome subspace variance} of $\rvX$ \xref{def:ocsVar}.
\thmbox{
  \brb{\begin{array}{M}
    $\omsH\eqd\omsR$ is the \structe{real line} \xref{def:Rline}
  \end{array}}
  \quad\implies\quad
  \brb{\begin{array}{c}
    \ocsVar(\rvX;\pE) = \pVar(\rvX)
  \end{array}}
  }
\end{theorem}
\begin{proof}
\begin{align*}
  \ocsVar(\rvX;\pE)
    &\eqd \sum_{x\in\omsH} \metricsq{\pE(\rvX)}{x}\psp(x)
    && \text{by definition of $\ocsVar$ \xref{def:ocsVar}}
  \\&= \sum_{x\in\R} \abs{\pE(\rvX)-x}^2\psp(x)
    && \text{by definition of real line $\omsH$ \xref{def:Rline}}
  \\&= \int_{\R} \brp{x-\pE(\rvX)}^2\psp(x) \dx
    && \text{by definition of Lebesgue integration on $\R$}
  \\&= \pVar(\rvX)
    && \text{by definition of $\pVar$ \xref{def:pVar}}
\end{align*}
\end{proof}

%---------------------------------------
\begin{remark}
\label{rem:pEocsE}
%---------------------------------------
Despite the correspondence of traditional variance and outcome variance on the \structe{real line} 
as demonstrated in \prefpp{thm:ocsVar},
the situation is different for expected values.
Even when both are calculated on the same \structe{real line},
the \vale{traditional expected value} $\pE(\rvX)$ \xref{def:pE} 
and the \vale{outcome expected value} $\ocsE(\rvX)$ \xref{def:ocsE} don't always yield the same value.
Demonstrations of this include \prefpp{ex:wdie_xy} and \prefpp{ex:lcg7x1m9_xyz}.
However, there is one common situation in which the two statistics do correspond (next theorem).
\end{remark}

%---------------------------------------
\begin{theorem}
\label{thm:pEocsE}
%---------------------------------------
Let $\rvX$, $\omsH$, $\ocsG$ be defined as in \prefpp{thm:ocsVar}.
Let $\pE(\rvX)$ be the \fncte{traditional expected value} \xref{def:pE}
and $\ocsE(\rvX)$ the \fncte{outcome expected value} of $\rvX$ \xref{def:ocsE}.
\thmbox{
  \brb{\begin{array}{FlDD}
    1. & \omsH\eqd\omsR                             & (\structe{real line} \xrefnp{def:Rline}) & and \\
    2. & \psp(a-x)=\psp(a+x)\quad\scy\forall x\in\R & (\prope{symmetric} about $a$)
  \end{array}}
  \quad\implies\quad
  \brb{\begin{array}{rclcl}
    \ocsE(\rvX) &=& a &=& \pE(\rvX)
  \end{array}}
  }
\end{theorem}
\begin{proof}
\begin{align*}
  \boxed{\ocsE(\rvX)}
    &\eqd \argmin_{x\in\R}\max_{y\in\R}\metric{x}{y}\psp(y)
    && \text{by definition of $\ocsE$ \xref{def:ocsE}}
  \\&= \argmin_{x\in\R}\max_{y\in\R}\abs{x-y}\psp(y)
    && \text{by definition of \structe{real line} \xref{def:Rline}}
  \\&= \boxed{a}
    %&& \text{\gsize\centering\psset{unit=6mm}{%============================================================================
% Daniel J. Greenhoe
% XeLaTeX file
%============================================================================
{%\psset{yunit=2\psunit}%
\begin{pspicture}(-2.5,-0.3)(7,2.5)%
  \psset{%
    labelsep=1pt,
    linewidth=1pt,
    }%
  \psaxes[linecolor=axis,yAxis=false,labels=none,ticks=none]{<->}(0,0)(-2.5,0)(6.5,2.5)% x axis
  \psaxes[linecolor=axis,xAxis=false,labels=none,ticks=none]{->}(0,0)(-2.5,0)(6.5,2.5)% y axis
  \rput(1,0){% g(y)=|x-y|
    \psline(-2,2)(0,0)(2,2)%
    \uput[-90]{0}(0,0){$x$}%
    \uput[-22]{0}(2,2){$\ff(y)\eqd\abs{x-y}$}
    }%
  \rput(1.75,0){% Gaussian like distribution
    %\psplot[plotpoints=64]{-2}{6}{2.718 x 2 sub 2 exp neg 2 div exp 2.507 div}% Gaussian distribution with sigma=1 and mean=2
    \psplot[plotpoints=64]{-4}{4}{2.718 x 2 exp neg 2 div exp }% 
    \uput[-90]{0}(0,0){$a$}%
    \psline[linestyle=dotted,linecolor=red](0,1)(0,0)%
    \uput[45]{0}(1.2,0.5){$\psp(y)$}
    }%
  %\psline[linestyle=dotted,linecolor=red](0,0.75)(3.5,0.75)%
  %
  \uput[0]{0}(6.5,0){$y$}%
  %\rput[t](3.5,2){$\ds\ff(x)\eqd\max_{y\in\R}\abs{x-y}\psp(y)$ minimized when $x=a$}%
\end{pspicture}}%
}}
    && \text{\gsize\centering\psset{unit=6mm}{\includegraphics{../common/math/graphics/pdfs/pEocsE_max.pdf}}}
  \\&&&\text{because $\fh(x)\eqd\max_{y\in\R}\abs{x-y}\psp(y)$ is minimized when $x=a$}%
  \\&= \boxed{\pE(\rvX)}
    && \text{by \prefpp{prop:pspsym}}
%{prop:pspsym}
\end{align*}
\end{proof}

%---------------------------------------
\begin{theorem}
\label{thm:EfX}
%---------------------------------------
\mbox{}\\\begin{tabular}{lMll}
    Let & \ocsG\eqd\ocs{\ocso_\ocsG}{\metricn_\ocsG}{\orel_\ocsG}{\psp_\ocsG} & be an \structe{outcome subspace}                         & \xref{def:ocs}.
  \\Let & \omsH\eqd\oms{\ocso_\omsH}{\metricn_\omsH}{\orel_\omsH}             & be an \structe{ordered quasi-metric space}               & \xref{def:oqms}.
  \\Let & \omsK\eqd\oms{\ocso_\omsK}{\metricn_\omsK}{\orel_\omsK}             & be an \structe{ordered quasi-metric space}              & \xref{def:oqms}.
  \\Let & \rvX\in\clOCSgh                                                     & be a \structe{random variable} from $\ocsG$ onto $\omsH$ & \xref{def:ocsrv}.
  \\Let & \ff\in\clF{\ocso_\omsH}{\ocso_\omsK}                                & be a function from $\ocso_\omsH$ onto $\ocso_\omsK$ (\fncte{pullback}) & \xref{thm:met_sumpf}.
  \\Let & \fphi\in\clF{\R}{\R}                                                & be a function from $\R$ into $\R$ (\fncte{pushforward})  & \xref{def:mpf}.
  \\Let & \ocsH\eqd\ocs{\ocso_\omsH}{\metricn_\omsH}{\orel_\omsH}{\psp_\omsH} & \mc{2}{l}{be an \structe{outcome subspace} induced by $\ocsG$, $\omsH$, and $\rvX$.}
  \\Let & \ocsK\eqd\ocs{\ocso_\omsK}{\metricn_\omsK}{\orel_\omsK}{\psp_\omsK} & \mc{2}{l}{be an \structe{outcome subspace} induced by $\omsK$, $\ocsH$ and $\ff$.}
\end{tabular}
\thmbox{
  \brb{\begin{array}{FMDD}
    1. & $\ff$ is \prope{injective}                           &  & and\\
    2. & $\fphi$ is \prope{strictly isotone}                  &  & and\\
    3. & $\metricn_\omsH\brp{\ff(x),\ff(y)}\psp(y) = \fphi\brs{\metricn_\omsH\brp{x,y}\psp(y)}$ &
    %1. & $\omsH$ and $\omsK$ are \prope{isometric}            & \xref{def:isometry} & and\\
    %2. & $\ff$ is an \prope{isometry} in $\clF{\omsH}{\omsK}$ & \xref{def:isometry} &
  \end{array}}
  \quad\implies\quad
  \brb{\ocsE\brs{\ff(\rvX)} = \ff\brs{\ocsE(\rvX)}}
  }
\end{theorem}
\begin{proof}
\begin{align*}
  \ocsE\brs{\ff(\rvX)}
    &= \argmin_{x\in\ocso_\omsK}\max_{y\in\ocso_\omsK} \metricn_\omsK(x,y)\psp_\omsK(y)
    && \text{by definition of $\ocsE$ \xref{def:ocsE} and $\omsK$}
  \\&= \ff\brs{\argmin_{x\in\ocso_\omsH}\max_{y\in\ocso_\omsH} \metricn_\omsK\brp{\ff(x),\ff(y)}\psp_\omsK(\ff(y))}
    && \text{by $\ff$ \prope{bijection} hypothesis}
  \\&= \ff\brs{\argmin_{x\in\ocso_\omsH}\max_{y\in\ocso_\omsH} \metricn_\omsH\brp{\ff(x),\ff(y)}\psp_\omsH(y)}
    && \text{by $\ff$ \prope{bijection} hypothesis}
  \\&= \ff\brs{\argmin_{x\in\ocso_\omsH}\max_{y\in\ocso_\omsH} \fphi\brs{\metricn_\omsH(x,y)\psp_\omsH(y)}}
    && \text{by $\metricn_\omsH$ hypothesis}
  \\&= \ff\brs{\argmin_{x\in\ocso_\omsH}\max_{y\in\ocso_\omsH} \metricn_\omsH(x,y)\psp_\omsH(y)}
    && \text{by $\fphi$ is \prope{strictly isotone} hypothesis and \prefp{lem:argminmaxphi}}
  \\&= \ff\ocsE(\rvX)
    && \text{by definition of $\ocsE$ \xref{def:ocsE} and $\rvX$}
\end{align*}
\end{proof}


%%---------------------------------------
%\begin{corollary}
%\label{cor:ocsrv_Eax_group}
%%---------------------------------------
%Let $\omsH\eqd\omsD$ be an \structe{ordered metric space} \xref{def:oms}
%and $\rvX\in\clOCSgh$ a \structe{random variable} \xref{def:ocsrv} onto $\omsH$.
%Let $\addmult$ be a binary operator on $\ocso$ (a function in the set $\clF{\ocso\times\ocso}{\ocso}$).
%\thmbox{
%  \brb{\begin{array}{FMD}
%    1. & $\opair{\ocso}{\addmult}$ is a \structe{group} & and\\
%    2. & for some $a\in\ocso$, &\\
%       & $\ff(x)\eqd a\addmult x$ generates a \fncte{permutation} of $\opair{\ocso}{\addmult}$
%    %(either \prope{multiplicative} or \prope{additive group)
%  \end{array}}
%  \quad\implies\quad
%  \brb{\begin{array}{l}
%    \ocsE(a\addmult\rvX) = a\addmult\ocsE(\rvX)
%  \end{array}}
%  }
%\end{corollary}
%\begin{proof}
%\begin{align*}
%  \ocsE(a\addmult\rvX)
%    &= a\addmult\ocsE(\rvX)
%    && \text{because $\ff(x)=a\addmult x$ is an \prope{isometry} on $\ocsH$ and by \prefpp{thm:ocsrv_isometry}}
%\end{align*}
%\end{proof}


%---------------------------------------
\begin{corollary}
\label{cor:ocsrv_Eax_R}
%---------------------------------------
Let $\omsH$ be an \structe{ordered metric space} \xref{def:oms}
and $\rvX\in\clOCSgh$ a \structe{random variable} \xref{def:ocsrv} onto $\omsH$.
Let $\oms{\R}{\absn}{\le}$ be the \structe{real line ordered metric space} \xref{def:Rline}.
\thmbox{
    \mcom{\omsH=\oms{\R}{\absn}{\le}}{(real line)}
  \quad\implies\quad
  \brb{\begin{array}{lC}
    \ocsE(a\rvX) = a\ocsE(\rvX) & \forall a\in\Rnn
  \end{array}}
  }
\end{corollary}
\begin{proof}
\begin{enumerate}
  \item Proof for $a=0$ case: 
    \begin{align*}
      \ocsE(0\cdot\rvX)
        &= \argmin_{x\in0\cdot\omsH}\max_{y\in0\cdot\omsH} \metricn(x,y)\psp(y)
        && \text{by definition of $\ocsE$ \xref{def:ocsE}}
      \\&= \argmin_{x\in\setn{0}}\max_{y\in\setn{0}} \metricn(x,y)\psp(y)
      \\&= \argmin_{x\in\setn{0}}\max_{y\in\setn{0}} \metricn(0,0)\psp(y)
      \\&= \argmin_{x\in\setn{0}}\max_{y\in\setn{0}} 0\psp(y)
        && \text{by \prope{nondegenerate} property of $\metricn$ \xref{def:metric}}
      \\&= 0
      \\&= 0\cdot\ocsE(\rvX)
    \end{align*}

  \item Proof for $a>0$ case:
    $\metricn(\ff(x),\ff(y))\psp(y) \eqd \abs{ax-ay}\psp(y) = \abs{a}\abs{x-y}\psp(y) \eqd \abs{a}\abs{\metricn(x,y)\psp(y)}$
    \begin{align*}
      \ocsE(a\rvX)
        &= a\ocsE(\rvX)
        && \text{because $\ff(x)=ax$ is \prope{strictly isotone} on the real line and by \prefpp{thm:EfX}}
    \end{align*}
\end{enumerate}
\end{proof}

%In a ring or even a multiplicative group, $\ocsE(a\rvX)=a\ocsE(\rvX)$,
%as demonstrated in the next theorem and illusrated in \prefpp{ex:pspinner_x2x}.
%%---------------------------------------
%\begin{corollary}
%\label{cor:ocsrv_2x}
%%---------------------------------------
%Let $\omsH$ be a \structe{multiplicative group} such that $a\omsH=\setn{ah_1,ah_2,\ldots ah_\xN}$ for all $x,h_n\in\omsH$.
%Let $\rvX\in\clOCSgh$ be a \structe{random variable} \xref{def:ocsrv}.
%Let $\R$ be the set of real numbers and let $\ff$ be a function in $\clF{\omsH}{\R}$.
%\corbox{
%  \brb{\begin{array}{rclC}
%    \metric{ax}{ay} &=& \ff(a)\metric{x}{y}
%  \end{array}}
%  \quad\implies\quad
%  \ocsE(a\rvX) = \abs{\ff(a)}\ocsE(\rvX)
%  }
%\end{corollary}
%\begin{proof}
%\begin{align*}
%  \ocsE(a\rvX) 
%    &= a\argmin_{x\in a\omsH}\max_{y\in a\omsH} \metric{x}{y}\psp_{a\omsH}(y)
%    && \text{by definition of $\ocsE$ \xref{def:ocsE}}
%  \\&= a\argmin_{x\in\omsH}\max_{y\in\omsH} \metric{ax}{ay}\psp_{\omsH}(ay)
%    && \text{by definition of $\ocsE$ \xref{def:ocsE}}
%  \\&= a\argmin_{x\in\omsH}\max_{y\in\omsH} \metric{ax}{ay}\psp_{\rvX}(y)
%  \\&= a\argmin_{x\in\omsH}\max_{y\in\omsH} \abs{\metric{ax}{ay}}\psp(y)
%    && \text{by \prope{non-negative} property of $\metricn$ \xref{def:metric}}
%  \\&= a\argmin_{x\in\omsH}\max_{y\in\omsH} \abs{\ff(a)\metric{x}{y}}\psp(y)
%    && \text{by left hypothesis}
%  \\&= a\argmin_{x\in\omsH}\max_{y\in\omsH} \abs{\ff(a)}\abs{\metric{x}{y}}\psp(y)
%    && \text{by property of $\absn$}
%  \\&= a\abs{\ff(a)}\argmin_{x\in\omsH}\max_{y\in\omsH} {\metric{x}{y}}\psp(y)
%  \\&\eqd a\ocsE(\rvX)
%\end{align*}
%\end{proof}

%\if 0
%=======================================
\subsection{Problem statement}
%=======================================
The \fncte{traditional random variable} $\rvX$ \xref{def:rvt} is a function that maps from a 
\structe{stochastic process} to the \structe{real line} \xref{def:Rline}.
The traditional expectation value $\pE(\rvX)$ of $\rvX$ is then often a poor choice of a statistic
when the stochastic process that $\rvX$ maps from is a structure other than the real line or 
some substructure of the real line.
There are two fundamental problems:
\begin{enumerate}
  \item A traditional random variable $\rvX$ maps \textbf{to} 
        the \prope{linearly ordered} real line.
        However, $\rvX$ often maps \textbf{from} a random process that is \prope{non-linearly ordered}
        (or even \prope{unordered} \xxrefnp{def:order}{def:chain}).

  \item A traditional random variable $\rvX$ maps \textbf{to} 
        the real line with a \structe{metric geometry} \xref{rem:mgeo} 
        induced by the \fncte{usual metric} \xref{def:d_usual}.
        But many random processes have a fundamentally different \structe{metric geometry},
        a common one being that induced by the \fncte{discrete metric} \xref{def:dmetric}.
\end{enumerate}
Thus, the order structure of the domain and range of $\rvX$ are often fundamentally dissimilar,
leading to statistics, such as $\pE(\rvX)$, that are
of poor quality with regards to qualitative intuition and quantitative variance (expected error) measurements,
and of dubious suitability for tasks such as decision making, prediction, and hypothesis testing.

%---------------------------------------
\begin{remark}
%---------------------------------------
Unlike in traditional statistical processing,
it in general \textbf{not true} that $\ocsE(\rvX+\rvY)=\ocsE(\rvX)+\ocsE(\rvY$).
See \prefpp{ex:rline_11312a} for a counter example.
\end{remark}

%=======================================
%\subsection{A poor solution}
%\paragraph{A poor solution.}
%=======================================
%---------------------------------------
\begin{remark}
%---------------------------------------
A possible solution to the traditional random variable order and metric geometry problem is to allow the 
random variable to map into the \structe{complex plane} \xref{ex:Cplane} with the usual metric, 
rather than into the real line only. 
However, this is a poor solution, as demonstrated in \prefpp{ex:gsp_C}.
\end{remark}

%=======================================
\subsection{Examples}
%=======================================
\begin{tabstr}{0.75}
%\if 0
%=======================================
\subsubsection{Fair die examples}
%=======================================
\begin{figure}[h]
  \centering%
  \gsize%
  %{%============================================================================
% Daniel J. Greenhoe
% LaTeX file
% real dice mappings to real line and integer line
%============================================================================
\begin{pspicture}(-4.3,-1.5)(4.3,1.8)%
  %---------------------------------
  % options
  %---------------------------------
  \psset{%
    %radius=1.25ex,
    labelsep=2.5mm,
    linecolor=blue,%
    }%
  %---------------------------------
  % dice graph
  %---------------------------------
  \rput(0,0){%\psset{unit=2\psunit}%
    \Cnode[radius=1.25ex,fillstyle=solid,fillcolor=snode](-0.8660,-0.5){D4}%
    \Cnode[radius=1.25ex,fillstyle=solid,fillcolor=snode](-0.8660,0.5){D5}%
    \Cnode[radius=1.25ex,fillstyle=solid,fillcolor=snode](0,1){D6}%
    \Cnode[radius=1.25ex,fillstyle=solid,fillcolor=snode](0.8660,0.5){D3}%
    \Cnode[radius=1.25ex,fillstyle=solid,fillcolor=snode](0.8660,-0.5){D2}%
    \Cnode[radius=1.25ex,fillstyle=solid,fillcolor=snode](0,-1){D1}%
    }
  \uput{5mm}[90](D6){$\ocsG$}%
  \rput(D6){$\diceF$}%
  \rput(D5){$\diceE$}%
  \rput(D4){$\diceD$}%
  \rput(D3){$\diceC$}%
  \rput(D2){$\diceB$}%
  \rput(D1){$\diceA$}%
  %
  \ncline{D5}{D6}%
  \ncline{D4}{D5}\ncline{D4}{D6}%
  \ncline{D3}{D4}\ncline{D3}{D5}\ncline{D3}{D6}%
  \ncline{D2}{D3}\ncline{D2}{D4}\ncline{D2}{D5}\ncline{D2}{D6}%
  \ncline{D1}{D2}\ncline{D1}{D3}\ncline{D1}{D4}\ncline{D1}{D5}\ncline{D1}{D6}%
  %
  \uput[158](D6){$\frac{1}{6}$}
  \uput[150](D5){$\frac{1}{6}$}
  \uput[210](D4){$\frac{1}{6}$}
  \uput[ 22](D3){$\frac{1}{6}$}
  \uput[-45](D2){$\frac{1}{6}$}
  \uput[-158](D1){$\frac{1}{6}$}
  %---------------------------------
  % Real line
  %---------------------------------
  \rput(-2.75,0){\psset{unit=0.35\psunit}%
    \psline{<->}(0,-3.5)(0,3.5)%
    \pnode(0,2.5){R6}%
    \pnode(0,1.5){R5}%
    \pnode(0,0.5){R4}%
    \pnode(0,0){R34}%
    \pnode(0,-0.5){R3}%
    \pnode(0,-1.5){R2}%
    \pnode(0,-2.5){R1}%
    \Cnode[fillstyle=solid,linecolor=snode,fillcolor=snode,radius=0.5ex](0,0){RC}%
    \uput[135](0,3.5){$\omsR$}%
    }%
  \pscircle[fillstyle=none,linecolor=red,fillcolor=red](RC){0.6ex}%
  %\ncline{R5}{R6}%
  %\ncline{R4}{R5}%
  %\ncline{R3}{R4}%
  %\ncline{R2}{R3}%
  %\ncline{R1}{R2}%
  \rput(R6){\psline[linewidth=1pt](-0.1,0)(0.1,0)}%
  \rput(R5){\psline[linewidth=1pt](-0.1,0)(0.1,0)}%
  \rput(R4){\psline[linewidth=1pt](-0.1,0)(0.1,0)}%
  \rput(R3){\psline[linewidth=1pt](-0.1,0)(0.1,0)}%
  \rput(R2){\psline[linewidth=1pt](-0.1,0)(0.1,0)}%
  \rput(R1){\psline[linewidth=1pt](-0.1,0)(0.1,0)}%
  %
  \uput[180](R6){$6$}%
  \uput[180](R5){$5$}%
  \uput[180](R4){$4$}%
  \uput[180](R3){$3$}%
  \uput[180](R2){$2$}%
  \uput[180](R1){$1$}%
  \uput[0](R34){$3.5$}%
  %
  \ncarc[arcangle=-22,linewidth=0.75pt,linecolor=red]{->}{D6}{R6}%
  \ncarc[arcangle=-22,linewidth=0.75pt,linecolor=red]{->}{D5}{R5}%
  \ncarc[arcangle=-22,linewidth=0.75pt,linecolor=red]{->}{D4}{R4}%
  \ncarc[arcangle= 22,linewidth=0.75pt,linecolor=red]{->}{D3}{R3}%
  \ncarc[arcangle= 22,linewidth=0.75pt,linecolor=red]{->}{D2}{R2}%
  \ncarc[arcangle= 22,linewidth=0.75pt,linecolor=red]{->}{D1}{R1}%
  %
  \rput(-1.8,1){$\rvX(\cdot)$}%
  %---------------------------------
  % integer line
  %---------------------------------
  \rput(2.75,0){\psset{unit=0.35\psunit}%
    \pnode(0,3.5){I7}%
    \Cnode(0,2.5){I6}%
    \Cnode(0,1.5){I5}%
    \Cnode[fillstyle=solid,fillcolor=snode](0,0.5){I4}%
    \Cnode[fillstyle=solid,fillcolor=snode](0,-0.5){I3}%
    \Cnode(0,-1.5){I2}%
    \Cnode(0,-2.5){I1}%
    \pnode(0,-3.5){I0}%
    }%
  \uput[45](I7){$\omsZ$}%
  \ncline[linestyle=dotted]{I6}{I7}%
  \ncline{I5}{I6}%
  \ncline{I4}{I5}%
  \ncline{I3}{I4}%
  \ncline{I2}{I3}%
  \ncline{I1}{I2}%
  \ncline[linestyle=dotted]{I1}{I0}%
  %
  \uput[0](I6){$6$}%
  \uput[0](I5){$5$}%
  \uput[0](I4){$4$}%
  \uput[0](I3){$3$}%
  \uput[0](I2){$2$}%
  \uput[0](I1){$1$}%
  %
  \ncarc[arcangle= 22,linewidth=0.75pt,linecolor=green]{->}{D6}{I6}%
  \ncarc[arcangle= 22,linewidth=0.75pt,linecolor=green]{->}{D5}{I5}%
  \ncarc[arcangle= 22,linewidth=0.75pt,linecolor=green]{->}{D4}{I4}%
  \ncarc[arcangle=-22,linewidth=0.75pt,linecolor=green]{->}{D3}{I3}%
  \ncarc[arcangle=-22,linewidth=0.75pt,linecolor=green]{->}{D2}{I2}%
  \ncarc[arcangle=-22,linewidth=0.75pt,linecolor=green]{->}{D1}{I1}%
  %
  \rput(1.8,1){$\rvY(\cdot)$}%
\end{pspicture}%}%
  {\includegraphics{../common/math/graphics/pdfs/fairdieXRYZ.pdf}}%
  \caption{random variable mappings from the fair die to the real line and integer line \label{fig:fairdieXRYZ}}
\end{figure}
%---------------------------------------
\begin{example}[\exmd{fair die mappings to real line and integer line}]
\label{ex:fairdieXRYZ} %\mbox{}\\
%---------------------------------------
Let $\ocsG$ be the \structe{fair die outcome subspace} \xref{ex:fairdie}.
Let $\rvX\in\clOCSgr$ be a \fncte{random variable} \xref{def:ocsrv} mapping from $\ocsG$ to the \structe{real line} \xref{def:Rline},
and $\rvY\in\clOCSgz$ be a \fncte{random variable} \xref{def:ocsrv} mapping from $\ocsG$ to the \structe{integer line} \xref{def:Zline},
as illustrated in \prefpp{fig:fairdieXRYZ}.
%
Let $\pE$   be the \fncte{traditional expected value} function \xref{def:pE},
    $\pVar$ the \fncte{traditional variance} function \xref{def:pVar},
    $\ocsE$ the \fncte{outcome expected value} function \xref{def:ocsE}, 
and $\ocsVar$ the \fncte{outcome variance} function \xref{def:ocsVar}.
%
This yields the following statistics:
\\\indentx$\begin{array}{>{\gsize}Mrcc rcccl}
  geometry of $\ocsG$:                                    & \ocscen (\ocsG) &=& \mc{4}{l}{\setn{\diceA,\diceB,\diceC,\diceD,\diceE,\diceF}}  \\
  traditional      statistics on \structe{real line}:     & \pE  (\rvX)     &=& 3.5           & \ocsVar(\rvX;\pE)   &=& \frac{35}{12}&\approx& 2.917 \\
  outcome subspace statistics on \structe{real line}:     & \ocsE (\rvX)    &=& \setn{3.5}    & \ocsVar(\rvX;\ocsE) &=& \frac{35}{12}&\approx& 2.917 \\
  outcome subspace statistics on \structe{integer line}:  & \ocsE (\rvY)    &=& \setn{3,\,4}  & \ocsVar(\rvY;\ocsE) &=& \frac{20}{12}&\approx& 1.667 
\end{array}$
\end{example}
\begin{proof}
    \begin{align*}
      \ocscen(\ocsG)
        %&=\ocscena(\ocsG)
        %&&\text{by \prefp{ex:fairdie}} 
        &=\setn{\diceA,\diceB,\diceC,\diceD,\diceE,\diceF}
        &&\text{by \prefp{ex:fairdie}} 
      \\
      \pE(\rvX)
        &\eqd\sum_{x\in\R} x\psp(x)
        && \text{by definition of $\pE$ \xref{def:pE}}
      \\&= \mathrlap{
           \sum_{x\in\R} x \frac{1}{6}
         = \frac{1+2+3+4+5+6}{6}
         = \frac{21}{6}
         = \frac{7}{2}
         = 3.5}
      \\
      \ocsVar(\rvX;\pE)
        &= \pVar(\rvX)
        && \text{by \prefp{thm:ocsVar}}
      \\&\eqd \sum_{x\in\R} \brs{x-\pE(\rvX)}^2 \psp(x)
        &&\text{by definition of $\pVar$ \xref{def:pVar}}
      \\&= \sum_{x\in\R} \brp{x-\frac{7}{2}}^2 \frac{1}{6}
        &&\text{by $\pE(\rvX)$ result}
      \\&= \mathrlap{\brs{\brp{1-\frac{7}{2}}^2 + \brp{2-\frac{7}{2}}^2 + \brp{3-\frac{7}{2}}^2 + \brp{4-\frac{7}{2}}^2 + \brp{5-\frac{7}{2}}^2 + \brp{6-\frac{7}{2}}^2 }\frac{1}{6}}
      \\&= \mathrlap{\brs{(2-7)^2 + (4-7)^2 + (6-7)^2 + (8-7)^2 + (10-7)^2 + (12-7)^2}\frac{1}{2^2\times6}}
      \\&= \mathrlap{\frac{25+9+1+1+9+25}{24}
         = \frac{70}{24}
         = \frac{35}{12}
        \approx 2.917}
  %  \end{align*}
  %
  %\item \structe{outcome subspace} statistics of random variable mapping $\rvX$ to \structe{real line} \xref{def:Rline}:
  %  \begin{align*}
      \\
      \ocsE(\rvX)
        &= \pE(\rvX)
        && \text{by \prefpp{thm:pEocsE}}
      \\&= \setn{\frac{7}{2}}=\setn{3.5}
        && \text{by $\pE(\rvX)$ result}
      %  && \text{by $\ocsE(\rvX)$ result}
      %  &\eqd \argmin_{x\in\R}\max_{y\in\R}\metric{x}{y}\psp(y)
      %  && \text{by definition of $\ocsE$ \xref{def:ocsE}}
      %\\&\eqd \argmin_{x\in\R}\max_{y\in\R}\abs{x-y}\frac{1}{6}
      %  && \text{by definition of \structe{real line} \xref{def:Rline} and $\ocsG$}
      %\\&= \argmin_{x\in\R}\max_{y\in\R}\abs{x-y}
      %  && \text{because $\fphi(x)=\frac{1}{6}x$ is \prope{strictly isotone} and by \prefp{lem:argminmaxphi}}
      %\\&= \argmin_{x\in\R}\brbl{\begin{array}{lM}
      %       \abs{x-1} & for $x\ge3.5$\\
      %       \abs{x-6} & for $x<3.5$
      %     \end{array}}
      %\\&= \setn{3.5}
      %  && \text{because expression is minimized at $x=3.5$}
      \\
      \ocsVar(\rvX;\ocsE)
        &\eqd \sum_{x\in\R}\metricsq{\ocsE(\rvX)}{x}\psp(x)
        && \text{by definition of $\ocsVar$ \xref{def:ocsVar}}
      \\&= \sum_{x\in\R}\metricsq{\pE(\rvX)}{x}\psp(x)
        && \text{by $\pE(\rvX)$ and $\ocsE(\rvX)$ results}
      \\&\ocsVar(\rvX;\pE)
        && \text{by definition of $\ocsVar$ \xref{def:ocsVar}}
      \\&= \frac{35}{12} \approx 2.917
        && \text{by $\pVar(\rvX)$ result}
%    \end{align*}
%
%  \item \structe{outcome subspace} statistics of random variable mapping $\rvY$ to \structe{integer line} \xref{def:Zline}:
%    \begin{align*}
      \\
      \ocsE(\rvY)
        &\eqd \argmin_{x\in\Z}\max_{y\in\omsH}\metric{x}{y}\psp(y)
        &&\text{by definition of $\ocsE$ \xref{def:ocsE}}
      \\&=\argmin_{x\in\Z}\max_{y\in\omsH}\abs{x-y}\frac{1}{6}
        && \text{by definition of \structe{integer line} \xref{def:Zline} and $\ocsG$}
      \\&=\argmin_{x\in\Z}\max_{y\in\omsH}\abs{x-y}
        && \text{because $\fphi(x)=\frac{1}{6}x$ is \prope{strictly isotone} and by \prefp{lem:argminmaxphi}}
      \\&=\mathrlap{\argmin_{x\in\Z}\max_{y\in\omsH}
             \setn{\begin{array}{cccccc}
                0 & 1 & 2 & 3 & 4 & 5 \\
                1 & 0 & 1 & 2 & 3 & 4 \\
                2 & 1 & 0 & 1 & 2 & 3 \\
                3 & 2 & 1 & 0 & 1 & 2 \\
                4 & 3 & 2 & 1 & 0 & 1 \\
                5 & 4 & 3 & 2 & 1 & 0 
             \end{array}}
      \quad= \argmin_{x\in\Z}
             \setn{\begin{array}{c}
                5\\
                4\\
                3\\
                3\\
                4\\
                5
             \end{array}}
      \quad= \argmin_{x\in\Z}
             \setn{\begin{array}{c}
                 \\
                 \\
                3\\
                4\\
                 \\
                \mbox{}
             \end{array}}}
      %\\
      %\ocsEa(\rvY)
      %  &\eqd \argmin_{x\in\omsH}\sum_{y\in\omsH}\metric{x}{y}\psp(y)
      %  &&\text{by definition of $\ocsE$ \xref{def:ocsE}}
      %\\&=\argmin_{x\in\omsH}\sum_{y\in\omsH}\abs{x-y}\frac{1}{6}
      %  && \text{by definition of \structe{integer line} \xref{def:Zline} and $\ocsG$}
      %\\&=\argmin_{x\in\omsH}\sum_{y\in\omsH}\abs{x-y}
      %  && \text{because $\fphi(x)=\frac{1}{6}x$ is \prope{strictly isotone} and by \prefp{lem:argminmaxphi}}
      %\\&=\mathrlap{\argmin_{x\in\omsH}\max_{y\in\omsH}
      %       \setn{\begin{array}{*{11}{@{\,}c}}
      %          0 &+& 1 &+& 2 &+& 3 &+& 4 &+& 5 \\
      %          1 &+& 0 &+& 1 &+& 2 &+& 3 &+& 4 \\
      %          2 &+& 1 &+& 0 &+& 1 &+& 2 &+& 3 \\
      %          3 &+& 2 &+& 1 &+& 0 &+& 1 &+& 2 \\
      %          4 &+& 3 &+& 2 &+& 1 &+& 0 &+& 1 \\
      %          5 &+& 4 &+& 3 &+& 2 &+& 1 &+& 0 
      %       \end{array}}
      %\quad= \argmin_{x\in\omsH}
      %       \setn{\begin{array}{c}
      %         15\\
      %         11\\
      %          9\\
      %          9\\
      %         11\\
      %         15
      %       \end{array}}
      %\quad= \argmin_{x\in\omsH}
      %       \setn{\begin{array}{c}
      %           \\
      %           \\
      %          3\\
      %          4\\
      %           \\
      %         \mbox{}
      %       \end{array}}}
      \\
      \ocsVar(\rvY;\ocsE)
        &\eqd \sum_{x\in\Z}\metricsq{\ocsE(\rvY)}{x}\psp(x)
        && \text{by definition of $\ocsVar$ \xref{def:ocsVar}}
      \\&= \sum_{x\in\Z}\metricsq{\setn{3,4}}{x}\frac{1}{6}
        && \text{by $\ocsE(\rvY)$ result and definition of $\ocsG$}
      \\&= \mathrlap{\frac{1}{6}\brp{\abs{3-1}^2+\abs{3-2}^2+\abs{3-3}^2+\abs{4-4}^2+\abs{4-5}^2+\abs{4-6}^2}}
      \\&= \frac{1}{6}\brp{4+1+0+0+1+4} = \frac{10}{6} = \frac{5}{3}
  \end{align*}
%\end{enumerate}
\end{proof}


The random variable mappings in \prefpp{ex:fairdieXRYZ} have two fundamental problems:
%(see figure to right where the die structure has a discrete metric topology illustrated with an undirected graph):
\begin{enumerate}
  \item The order structure of the \structe{fair die}
and the order structure of \structe{real line}
are inherently dissimilar in that 
while the \prope{bijective} \xref{def:ftypes} mapping $\rvX$ is trivially \prope{order preserving} \xref{def:orderpre}, 
its inverse is \prope{not order preserving}.
And this is a problem.
In the \prope{linearly ordered} \xref{def:chain} range of $\rvX$, it is true that $\rvX(\diceA)=1<2=\rvX(\diceB)$.
But in the unordered domain of $\rvX$ $\setn{\diceA,\diceB,\ldots,\diceF}$, it is \emph{not} true that $\diceA<\diceB$;
rather $\diceA$ and $\diceB$ are simply symbols without order.
This causes problems when we attempt to use the random variable to make statistical inferences involving moments \xref{def:ocsmom}.
The \fncte{traditional expected value} \xref{def:pE} of a \structe{fair die} \xref{ex:fairdie} 
is $\pE(X)=\frac{1}{6}(1+2+\cdots+6)=3.5$.
This implies that we expect the outcome of $\diceC$ or $\diceD$ more than we expect the outcome of say 
$\diceA$ or $\diceB$.
But these results have no relationship with reality or with intuition because the values of a fair die are merely symbols.
For a fair die, we would expect any pair of values equally.
We would not expect the outcome [$\diceC$ or $\diceD$] more than we would expect the outcome [$\diceA$ or $\diceB$],
or more than we would expect any other outcome pair.

  \item The metric geometry \xref{rem:dietop} of the \structe{fair die outcome subspace} is very dissimilar to the 
metric geometry of the \structe{real line} \xref{def:Rline} that it is mapped to by the random variable $\rvX$.
And this is a problem.
In the metric geometry of the fair die induced by the \fncte{discrete metric} \xref{def:dmetric}, 
$\diceA$ is no closer to $\diceB$ than it is to $\diceC$
($\metric{\diceA}{\diceB}=1=\metric{\diceA}{\diceC}$).
However in the metric geometry of the real line induced by the \fncte{usual metric} $\metric{x}{y}\eqd\abs{x-y}$ \xref{def:d_usual},
$\rvX(\diceA)=1$ is closer to $\rvX(\diceB)=2$ than it is to $\rvX(\diceC)=3$
($\abs{1-2}=1\neq2=\abs{1-3}$).
\end{enumerate}

%with no topology (and hence no metric) and 
%with no order, whereas the set of real numbers 
%\emph{does} have a topology 
%(induced by the metric $\metric{x}{y}\eqd\abs{x-y}$) and 
%\emph{does} have an order structure 
%(a linear order structure induced by the standard ordering relation $\orel$ on $\R$).
%\\\begin{minipage}{\tw-53mm}%
%\end{minipage}%
%\hfill%


\begin{figure}[h][h]
  \centering%
  \gsize%
  \begin{tabular}{c@{\qquad\qquad}c}
    %{%============================================================================
% Daniel J. Greenhoe
% LaTeX file
% discrete metric real dice mapping to linearly ordered O6c
%============================================================================
{%\psset{unit=0.5\psunit}%
\begin{pspicture}(-3.2,-1.5)(3.2,1.5)%
  %---------------------------------
  % options
  %---------------------------------
  \psset{%
    radius=1.25ex,
    labelsep=2.5mm,
    linecolor=blue,%
    }%
  %---------------------------------
  % dice graph
  %---------------------------------
  \rput(-1.75,0){%\psset{unit=2\psunit}%
    %\rput{ 210}(0,0){\rput(1,0){\Cnode[fillstyle=solid,fillcolor=snode](0,0){D4}}}%
    %\rput{ 150}(0,0){\rput(1,0){\Cnode[fillstyle=solid,fillcolor=snode](0,0){D5}}}%
    %\rput{  90}(0,0){\rput(1,0){\Cnode[fillstyle=solid,fillcolor=snode](0,0){D6}}}%
    %\rput{  30}(0,0){\rput(1,0){\Cnode[fillstyle=solid,fillcolor=snode](0,0){D3}}}%
    %\rput{ -30}(0,0){\rput(1,0){\Cnode[fillstyle=solid,fillcolor=snode](0,0){D2}}}%
    %\rput{ -90}(0,0){\rput(1,0){\Cnode[fillstyle=solid,fillcolor=snode](0,0){D1}}}%
    \uput{1}[210](0,0){\Cnode[fillstyle=solid,fillcolor=snode](0,0){D4}}%
    \uput{1}[150](0,0){\Cnode[fillstyle=solid,fillcolor=snode](0,0){D5}}%
    \uput{1}[ 90](0,0){\Cnode[fillstyle=solid,fillcolor=snode](0,0){D6}}%
    \uput{1}[ 30](0,0){\Cnode[fillstyle=solid,fillcolor=snode](0,0){D3}}%
    \uput{1}[-30](0,0){\Cnode[fillstyle=solid,fillcolor=snode](0,0){D2}}%
    \uput{1}[-90](0,0){\Cnode[fillstyle=solid,fillcolor=snode](0,0){D1}}%
    \uput{1.25}[120](0,0){$\ocsG$}%
    }
  \rput(D6){$\diceF$}%
  \rput(D5){$\diceE$}%
  \rput(D4){$\diceD$}%
  \rput(D3){$\diceC$}%
  \rput(D2){$\diceB$}%
  \rput(D1){$\diceA$}%
  %
  \ncline{D5}{D6}%
  \ncline{D4}{D5}\ncline{D4}{D6}%
  \ncline{D3}{D4}\ncline{D3}{D5}\ncline{D3}{D6}%
  \ncline{D2}{D3}\ncline{D2}{D4}\ncline{D2}{D5}\ncline{D2}{D6}%
  \ncline{D1}{D2}\ncline{D1}{D3}\ncline{D1}{D4}\ncline{D1}{D5}\ncline{D1}{D6}%
  %
  \uput[158](D6){$\frac{1}{6}$}
  \uput[150](D5){$\frac{1}{6}$}
  \uput[210](D4){$\frac{1}{6}$}
  \uput[ 22](D3){$\frac{1}{6}$}
  \uput[-45](D2){$\frac{1}{6}$}
  \uput[-158](D1){$\frac{1}{6}$}
  %---------------------------------
  % range graph
  %---------------------------------
  \rput(1.75,0){%\psset{unit=2\psunit}%
    %\Cnode[fillstyle=solid,fillcolor=snode](-0.8660,-0.5){E4}%
    %\Cnode[fillstyle=solid,fillcolor=snode](-0.8660,0.5){E5}%
    %\Cnode[fillstyle=solid,fillcolor=snode](0,1){E6}%
    %\Cnode[fillstyle=solid,fillcolor=snode](0.8660,0.5){E3}%
    %\Cnode[fillstyle=solid,fillcolor=snode](0.8660,-0.5){E2}%
    %\Cnode[fillstyle=solid,fillcolor=snode](0,-1){E1}%
    \uput{1}[210](0,0){\Cnode[fillstyle=solid,fillcolor=snode](0,0){E4}}%
    \uput{1}[150](0,0){\Cnode[fillstyle=solid,fillcolor=snode](0,0){E5}}%
    \uput{1}[ 90](0,0){\Cnode[fillstyle=solid,fillcolor=snode](0,0){E6}}%
    \uput{1}[ 30](0,0){\Cnode[fillstyle=solid,fillcolor=snode](0,0){E3}}%
    \uput{1}[-30](0,0){\Cnode[fillstyle=solid,fillcolor=snode](0,0){E2}}%
    \uput{1}[-90](0,0){\Cnode[fillstyle=solid,fillcolor=snode](0,0){E1}}%
    \uput{1.25}[60](0,0){$\omsH$}%
    }
  \rput(E6){$6$}%
  \rput(E5){$5$}%
  \rput(E4){$4$}%
  \rput(E3){$3$}%
  \rput(E2){$2$}%
  \rput(E1){$1$}%
  %\rput(E0){$0$}%
  %
  \ncline{E5}{E6}%
  \ncline{E4}{E5}\ncline{E4}{E6}%
  \ncline{E3}{E4}\ncline{E3}{E5}\ncline{E3}{E6}%
  \ncline{E2}{E3}\ncline{E2}{E4}\ncline{E2}{E5}\ncline{E2}{E6}%
  \ncline{E1}{E2}\ncline{E1}{E3}\ncline{E1}{E4}\ncline{E1}{E5}\ncline{E1}{E6}%
  %\ncline{E0}{E1}\ncline{E0}{E2}\ncline{E0}{E3}\ncline{E0}{E4}\ncline{E0}{E5}\ncline{E0}{E6}%
  %
  \uput[ 22](E6){$\frac{1}{6}$}
  \uput[135](E5){$\frac{1}{6}$}
  \uput[200](E4){$\frac{1}{6}$}
  \uput[ 30](E3){$\frac{1}{6}$}
  \uput[-30](E2){$\frac{1}{6}$}
  \uput[-22](E1){$\frac{1}{6}$}
  %\uput[  0](E0){$\sfrac{0}{6}$}
  %---------------------------------
  % mapping from die to O6c
  %---------------------------------
  \ncarc[arcangle= 22,linewidth=0.75pt,linecolor=red]{->}{D6}{E6}%
  \ncarc[arcangle= 22,linewidth=0.75pt,linecolor=red]{->}{D5}{E5}%
  \ncarc[arcangle= 22,linewidth=0.75pt,linecolor=red]{->}{D4}{E4}%
  \ncarc[arcangle=-22,linewidth=0.75pt,linecolor=red]{->}{D3}{E3}%
  \ncarc[arcangle=-22,linewidth=0.75pt,linecolor=red]{->}{D2}{E2}%
  \ncarc[arcangle=-22,linewidth=0.75pt,linecolor=red]{->}{D1}{E1}%
  %---------------------------------
  % labels
  %---------------------------------
  \rput(0,0){$\rvX(\cdot)$}%
  %\ncline[linestyle=dotted,nodesep=1pt]{->}{xzlabel}{xz}%
  %\ncline[linestyle=dotted,nodesep=1pt]{->}{ylabel}{y}%
\end{pspicture}
}%}&{%============================================================================
% Daniel J. Greenhoe
% LaTeX file
% discrete metric real dice mapping to linearly ordered O6c
%============================================================================
{%\psset{unit=0.5\psunit}%
\begin{pspicture}(-3.2,-1.5)(3.2,1.5)%
  %---------------------------------
  % options
  %---------------------------------
  \psset{%
    radius=1.25ex,
    labelsep=2.5mm,
    linecolor=blue,%
    }%
  %---------------------------------
  % dice graph
  %---------------------------------
  \rput(-1.75,0){%\psset{unit=2\psunit}%
    \uput{1}[210](0,0){\Cnode[fillstyle=solid,fillcolor=snode](0,0){D4}}%
    \uput{1}[150](0,0){\Cnode[fillstyle=solid,fillcolor=snode](0,0){D5}}%
    \uput{1}[ 90](0,0){\Cnode[fillstyle=solid,fillcolor=snode](0,0){D6}}%
    \uput{1}[ 30](0,0){\Cnode[fillstyle=solid,fillcolor=snode](0,0){D3}}%
    \uput{1}[-30](0,0){\Cnode[fillstyle=solid,fillcolor=snode](0,0){D2}}%
    \uput{1}[-90](0,0){\Cnode[fillstyle=solid,fillcolor=snode](0,0){D1}}%
    \uput{1.25}[120](0,0){$\ocsG$}%
    }
  \rput(D6){$\diceF$}%
  \rput(D5){$\diceE$}%
  \rput(D4){$\diceD$}%
  \rput(D3){$\diceC$}%
  \rput(D2){$\diceB$}%
  \rput(D1){$\diceA$}%
  %
  \ncline{D5}{D6}%
  \ncline{D4}{D5}\ncline{D4}{D6}%
  \ncline{D3}{D4}\ncline{D3}{D5}\ncline{D3}{D6}%
  \ncline{D2}{D3}\ncline{D2}{D4}\ncline{D2}{D5}\ncline{D2}{D6}%
  \ncline{D1}{D2}\ncline{D1}{D3}\ncline{D1}{D4}\ncline{D1}{D5}\ncline{D1}{D6}%
  %
  \uput[158](D6){$\frac{1}{6}$}
  \uput[150](D5){$\frac{1}{6}$}
  \uput[210](D4){$\frac{1}{6}$}
  \uput[ 22](D3){$\frac{1}{6}$}
  \uput[-45](D2){$\frac{1}{6}$}
  \uput[-158](D1){$\frac{1}{6}$}
  %---------------------------------
  % range graph
  %---------------------------------
  \rput(1.75,0){%\psset{unit=2\psunit}%
    \uput{1}[210](0,0){\Cnode(0,0){E4}}%
    \uput{1}[150](0,0){\Cnode(0,0){E5}}%
    \uput{1}[ 90](0,0){\Cnode(0,0){E6}}%
    \uput{1}[ 30](0,0){\Cnode(0,0){E3}}%
    \uput{1}[-30](0,0){\Cnode(0,0){E2}}%
    \uput{1}[-90](0,0){\Cnode(0,0){E1}}%
    \Cnode[fillstyle=solid,fillcolor=snode](0,0){E0}%
    \uput{1.25}[60](0,0){$\omsH$}%
    }
  \rput(E6){$6$}%
  \rput(E5){$5$}%
  \rput(E4){$4$}%
  \rput(E3){$3$}%
  \rput(E2){$2$}%
  \rput(E1){$1$}%
  \rput(E0){$0$}%
  %
  %\ncline{E5}{E6}%
  %\ncline{E4}{E5}\ncline{E4}{E6}%
  %\ncline{E3}{E4}\ncline{E3}{E5}\ncline{E3}{E6}%
  %\ncline{E2}{E3}\ncline{E2}{E4}\ncline{E2}{E5}\ncline{E2}{E6}%
  %\ncline{E1}{E2}\ncline{E1}{E3}\ncline{E1}{E4}\ncline{E1}{E5}\ncline{E1}{E6}%
  \ncline{E0}{E1}\naput{$\frac{1}{2}$}
  \ncline{E0}{E2}\naput{$\frac{1}{2}$}
  \ncline{E0}{E3}\naput{$\frac{1}{2}$}
  \ncline{E0}{E4}\naput{$\frac{1}{2}$}
  \ncline{E0}{E5}\naput{$\frac{1}{2}$}
  \ncline{E0}{E6}\naput{$\frac{1}{2}$}%
  %
  \uput[ 22](E6){$\frac{1}{6}$}
  \uput[135](E5){$\frac{1}{6}$}
  \uput[200](E4){$\frac{1}{6}$}
  \uput[ 30](E3){$\frac{1}{6}$}
  \uput[-30](E2){$\frac{1}{6}$}
  \uput[-22](E1){$\frac{1}{6}$}
  \uput[  0](E0){$\sfrac{0}{6}$}
  %---------------------------------
  % mapping from die to O6c
  %---------------------------------
  \ncarc[arcangle= 22,linewidth=0.75pt,linecolor=red]{->}{D6}{E6}%
  \ncarc[arcangle= 22,linewidth=0.75pt,linecolor=red]{->}{D5}{E5}%
  \ncarc[arcangle= 22,linewidth=0.75pt,linecolor=red]{->}{D4}{E4}%
  \ncarc[arcangle=-11,linewidth=0.75pt,linecolor=red]{->}{D3}{E3}%
  \ncarc[arcangle=-11,linewidth=0.75pt,linecolor=red]{->}{D2}{E2}%
  \ncarc[arcangle=-22,linewidth=0.75pt,linecolor=red]{->}{D1}{E1}%
  %---------------------------------
  % labels
  %---------------------------------
  \rput(0,0){$\rvX(\cdot)$}%
  %\ncline[linestyle=dotted,nodesep=1pt]{->}{xzlabel}{xz}%
  %\ncline[linestyle=dotted,nodesep=1pt]{->}{ylabel}{y}%
\end{pspicture}
}%} \\
    {\includegraphics{../common/math/graphics/pdfs/fairdieXO6.pdf}}&{\includegraphics{../common/math/graphics/pdfs/fairdieXO6c.pdf}} \\
    (A) mapping to isomorphic structure  & (B) mapping to extended structure
  \end{tabular}
  \caption{order preserving random variable mappings from \structe{fair die}\label{fig:fairdieXO6}}
\end{figure}
%---------------------------------------
\begin{example}[\exmd{fair die mapping to isomorphic structure}]
%\mbox{}\\
\label{ex:fdieXO6}
%---------------------------------------
%\begin{minipage}{\tw-65mm}%
%Let $\ocsG\eqd\ocsD$ 
Let $\ocsG\eqd\ocs{\setn{\diceA,\diceB,\diceC,\diceD,\diceE,\diceF}}{\metricn}{\emptyset}{\psp}$ 
be a \structe{fair die outcome subspace} \xref{ex:fairdie},
and $\omsH\eqd\oms{\setn{1,2,3,4,5,6}}{\metricn}{\emptyset}$ 
be an \structe{unordered metric space} \xref{def:oms}.
\prefpp{ex:fairdieXRYZ} presented mappings from $\ocsG$ to structures with structures dissimilar to $\ocsG$.
%structure \prope{linearly ordered lattice}s \xref{def:lattice}.
\prefp{fig:fairdieXO6} (A) illustrates a mapping to the isomorphic structure
$\ocsH\eqd\ocs{\setn{1,2,3,4,5,6}}{\metricn}{\emptyset}{\rvX(\psp)}$,
yielding the following statistics:
\\\indentx$
  \ocsE(\rvX)  = \setn{1,2,3,4,5,6} \qquad \ocsVar(\rvX)  = 0
  %\ocsEa(\rvX) &=& \setn{1,2,3,4,5,6}\\
  $\\
%\end{minipage}\hfill%
%\begin{tabular}{c}
%  \gsize%
%  %\psset{unit=5mm}%
%  {%============================================================================
% Daniel J. Greenhoe
% LaTeX file
% discrete metric real dice mapping to linearly ordered O6c
%============================================================================
{%\psset{unit=0.5\psunit}%
\begin{pspicture}(-3.2,-1.5)(3.2,1.5)%
  %---------------------------------
  % options
  %---------------------------------
  \psset{%
    radius=1.25ex,
    labelsep=2.5mm,
    linecolor=blue,%
    }%
  %---------------------------------
  % dice graph
  %---------------------------------
  \rput(-1.75,0){%\psset{unit=2\psunit}%
    %\rput{ 210}(0,0){\rput(1,0){\Cnode[fillstyle=solid,fillcolor=snode](0,0){D4}}}%
    %\rput{ 150}(0,0){\rput(1,0){\Cnode[fillstyle=solid,fillcolor=snode](0,0){D5}}}%
    %\rput{  90}(0,0){\rput(1,0){\Cnode[fillstyle=solid,fillcolor=snode](0,0){D6}}}%
    %\rput{  30}(0,0){\rput(1,0){\Cnode[fillstyle=solid,fillcolor=snode](0,0){D3}}}%
    %\rput{ -30}(0,0){\rput(1,0){\Cnode[fillstyle=solid,fillcolor=snode](0,0){D2}}}%
    %\rput{ -90}(0,0){\rput(1,0){\Cnode[fillstyle=solid,fillcolor=snode](0,0){D1}}}%
    \uput{1}[210](0,0){\Cnode[fillstyle=solid,fillcolor=snode](0,0){D4}}%
    \uput{1}[150](0,0){\Cnode[fillstyle=solid,fillcolor=snode](0,0){D5}}%
    \uput{1}[ 90](0,0){\Cnode[fillstyle=solid,fillcolor=snode](0,0){D6}}%
    \uput{1}[ 30](0,0){\Cnode[fillstyle=solid,fillcolor=snode](0,0){D3}}%
    \uput{1}[-30](0,0){\Cnode[fillstyle=solid,fillcolor=snode](0,0){D2}}%
    \uput{1}[-90](0,0){\Cnode[fillstyle=solid,fillcolor=snode](0,0){D1}}%
    \uput{1.25}[120](0,0){$\ocsG$}%
    }
  \rput(D6){$\diceF$}%
  \rput(D5){$\diceE$}%
  \rput(D4){$\diceD$}%
  \rput(D3){$\diceC$}%
  \rput(D2){$\diceB$}%
  \rput(D1){$\diceA$}%
  %
  \ncline{D5}{D6}%
  \ncline{D4}{D5}\ncline{D4}{D6}%
  \ncline{D3}{D4}\ncline{D3}{D5}\ncline{D3}{D6}%
  \ncline{D2}{D3}\ncline{D2}{D4}\ncline{D2}{D5}\ncline{D2}{D6}%
  \ncline{D1}{D2}\ncline{D1}{D3}\ncline{D1}{D4}\ncline{D1}{D5}\ncline{D1}{D6}%
  %
  \uput[158](D6){$\frac{1}{6}$}
  \uput[150](D5){$\frac{1}{6}$}
  \uput[210](D4){$\frac{1}{6}$}
  \uput[ 22](D3){$\frac{1}{6}$}
  \uput[-45](D2){$\frac{1}{6}$}
  \uput[-158](D1){$\frac{1}{6}$}
  %---------------------------------
  % range graph
  %---------------------------------
  \rput(1.75,0){%\psset{unit=2\psunit}%
    %\Cnode[fillstyle=solid,fillcolor=snode](-0.8660,-0.5){E4}%
    %\Cnode[fillstyle=solid,fillcolor=snode](-0.8660,0.5){E5}%
    %\Cnode[fillstyle=solid,fillcolor=snode](0,1){E6}%
    %\Cnode[fillstyle=solid,fillcolor=snode](0.8660,0.5){E3}%
    %\Cnode[fillstyle=solid,fillcolor=snode](0.8660,-0.5){E2}%
    %\Cnode[fillstyle=solid,fillcolor=snode](0,-1){E1}%
    \uput{1}[210](0,0){\Cnode[fillstyle=solid,fillcolor=snode](0,0){E4}}%
    \uput{1}[150](0,0){\Cnode[fillstyle=solid,fillcolor=snode](0,0){E5}}%
    \uput{1}[ 90](0,0){\Cnode[fillstyle=solid,fillcolor=snode](0,0){E6}}%
    \uput{1}[ 30](0,0){\Cnode[fillstyle=solid,fillcolor=snode](0,0){E3}}%
    \uput{1}[-30](0,0){\Cnode[fillstyle=solid,fillcolor=snode](0,0){E2}}%
    \uput{1}[-90](0,0){\Cnode[fillstyle=solid,fillcolor=snode](0,0){E1}}%
    \uput{1.25}[60](0,0){$\omsH$}%
    }
  \rput(E6){$6$}%
  \rput(E5){$5$}%
  \rput(E4){$4$}%
  \rput(E3){$3$}%
  \rput(E2){$2$}%
  \rput(E1){$1$}%
  %\rput(E0){$0$}%
  %
  \ncline{E5}{E6}%
  \ncline{E4}{E5}\ncline{E4}{E6}%
  \ncline{E3}{E4}\ncline{E3}{E5}\ncline{E3}{E6}%
  \ncline{E2}{E3}\ncline{E2}{E4}\ncline{E2}{E5}\ncline{E2}{E6}%
  \ncline{E1}{E2}\ncline{E1}{E3}\ncline{E1}{E4}\ncline{E1}{E5}\ncline{E1}{E6}%
  %\ncline{E0}{E1}\ncline{E0}{E2}\ncline{E0}{E3}\ncline{E0}{E4}\ncline{E0}{E5}\ncline{E0}{E6}%
  %
  \uput[ 22](E6){$\frac{1}{6}$}
  \uput[135](E5){$\frac{1}{6}$}
  \uput[200](E4){$\frac{1}{6}$}
  \uput[ 30](E3){$\frac{1}{6}$}
  \uput[-30](E2){$\frac{1}{6}$}
  \uput[-22](E1){$\frac{1}{6}$}
  %\uput[  0](E0){$\sfrac{0}{6}$}
  %---------------------------------
  % mapping from die to O6c
  %---------------------------------
  \ncarc[arcangle= 22,linewidth=0.75pt,linecolor=red]{->}{D6}{E6}%
  \ncarc[arcangle= 22,linewidth=0.75pt,linecolor=red]{->}{D5}{E5}%
  \ncarc[arcangle= 22,linewidth=0.75pt,linecolor=red]{->}{D4}{E4}%
  \ncarc[arcangle=-22,linewidth=0.75pt,linecolor=red]{->}{D3}{E3}%
  \ncarc[arcangle=-22,linewidth=0.75pt,linecolor=red]{->}{D2}{E2}%
  \ncarc[arcangle=-22,linewidth=0.75pt,linecolor=red]{->}{D1}{E1}%
  %---------------------------------
  % labels
  %---------------------------------
  \rput(0,0){$\rvX(\cdot)$}%
  %\ncline[linestyle=dotted,nodesep=1pt]{->}{xzlabel}{xz}%
  %\ncline[linestyle=dotted,nodesep=1pt]{->}{ylabel}{y}%
\end{pspicture}
}%}%
%\end{tabular}
%\\
Here, $\ocsE(\rvX)$ equals the entire base set of $\omsH$, 
indicating a statistic carrying no information about an expected outcome.
That is, there is no best guess concerning outcome.
This is much different than the traditional probability of $3.5$ \xref{ex:fairdieXRYZ} which 
deceptively suggests a likely outcome of $\diceC$ or $\diceD$.
And one could easily argue that no information is much better than misleading information.
\end{example}
\begin{proof}
    \begin{align*}
      \ocsE(\rvX)
        &\eqd \argmin_{x\in\omsH}\max_{y\in\omsH}\metric{x}{y}\psp(y)
        &&\text{by definition of $\ocsE$ \xref{def:ocsE}}
      \\&= \argmin_{x\in\ocsG}\max_{y\in\ocsG}\metric{x}{y}\psp(y)
        &&\text{because $\ocsG$ and $\omsH$ are \prope{isomorphic}}
      \\&= \rvX\brs{\ocscen(\ocsG)}
        &&\text{by definition of $\ocscen$ \xref{def:ocscen}}
      \\&= \rvX\brs{\setn{\diceA,\diceB,\diceC,\diceD,\diceE,\diceF}}
        &&\text{by \prefpp{ex:fairdieXRYZ}}
      \\&= \setn{1,2,3,4,5,6}
        &&\text{by definition of $\rvX$}
      \\
      \ocsVar(\rvX)
        &\eqd \sum_{x\in\omsH}\metricsq{\ocsE(\rvX)}{x}\psp(x)
        &&\text{by definition of $\ocsVar$ \xref{def:ocsVar}}
      \\&=\sum_{x\in\ocsG}\metricsq{\rvX(\ocscen(\ocsG)}{x}\psp(x)
        &&\text{because $\ocsG$ and $\omsH$ are \prope{isomorphic}}
      \\&\eqd\ocsVar(\ocsG)
        && \text{by definition of $\ocsVar$ \xref{def:ocsVarG}}
      \\&= 0
        &&\text{by \prefpp{ex:fairdieXRYZ}}
      %\\
      %\ocsEa(\rvX)
      %  &\eqd \argmin_{x\in\omsH}\sum_{y}\ocsmom(x,y)
      %  &&\text{by definition of $\ocsEa$ \xref{def:ocsEa}}
      %\\&\eqd \argmin_{x\in\omsH}\max_{y\in\omsH}\metric{x}{y}\psp(y)
      %  &&\text{by definition of $\ocsmom$ \xref{def:ocsmom}}
      %\\&=\argmin_{x\in\omsH}\max_{y\in\omsH}\metric{x}{y}\frac{1}{6}
      %  &&\text{by definition of $\ocsG$}
      %\\&= \argmin_{x\in\omsH}\max_{y\in\omsH}\metric{x}{y}
%        && \text{because $\fphi(x)=\frac{1}{6}x$ is \prope{strictly isotone} and by \prefp{lem:argminmaxphi}}
      %\\&=\mathrlap{\argmin_{x\in\omsH}
      %       \setn{\begin{array}{*{11}{c}}
      %         {0} &+& {1} &+& {1} &+& {1} &+& {1} &+& {2}\\
      %         {1} &+& {0} &+& {1} &+& {1} &+& {2} &+& {1}\\
      %         {1} &+& {1} &+& {0} &+& {2} &+& {1} &+& {1}\\
      %         {1} &+& {1} &+& {2} &+& {0} &+& {1} &+& {1}\\
      %         {1} &+& {2} &+& {1} &+& {1} &+& {0} &+& {1}\\
      %         {2} &+& {1} &+& {1} &+& {1} &+& {1} &+& {0}
      %       \end{array}}
      %  %&&\text{by \prefpp{item:fdieXO6_gcen}}
      %\quad= \argmin_{x\in\omsH}
      %       \setn{\begin{array}{c}
      %         6\\
      %         6\\
      %         6\\
      %         6\\
      %         6\\
      %         6
      %       \end{array}}}
      %\\&= \setn{1,2,3,4,5,6}
    \end{align*}
\end{proof}


Although all the coefficients of the polynomial equation $x^2-2x+2=0$ are in the set of real numbers $\R$,
the solutions of the equation ($x=1+i$ and $x=1-i$) are not. 
Rather, the two solutions are in the \structe{complex plane} $\R^2$ \xref{ex:Cplane}, of which $\R$ is a substructure.
This is an example of extending a structure (from $\R$ to $\R^2$) to achieve more useful results.
The same idea can be applied to a random variable $\rvX\in\clOCSgh$.
The definition of an \fncte{outcome random variable} \xref{def:ocsrv} does not require a bijection
between $\ocsG$ and $\omsH$;
rather, it only requires that the mapping be ``into" the base set of $\omsH$ \xref{def:ftypes}.
In \prefpp{ex:fdieXO6} in which $\ocsG$ is isomorphic to $\ocsH$, 
the expected value of $\rvX$ is a set with six values.
However, we could extend $\omsH$, while still preserving the order and metric geometry of $\ocsG$,
to produce a random variable with a simpler expected value (next example). 
\\
%---------------------------------------
\begin{example}[\exmd{fair die mapping with extended range}]
\label{ex:fdieXO6c}
%---------------------------------------
Let $\ocsG\eqd\ocs{\setn{\diceA,\diceB,\diceC,\diceD,\diceE,\diceF}}{\metricn}{\emptyset}{\psp}$ 
be a \structe{fair die outcome subspace} \xref{ex:fairdie},
and $\omsH\eqd\oms{\setn{1,2,3,4,5,6,0}}{\metrican}{\emptyset}$ 
be an \structe{unordered metric space} \xref{def:oms}.
\prefp{fig:fairdieXO6} (B) illustrates a random variable mapping $\rvX$ from $\ocsG$ to the extended structure $\omsH$,
yielding the following statistics:
\\\indentx$\ocsE(\rvX) = \setn{0} \qquad \ocsVar(\rvX)  = \frac{1}{4}$\\
As in \pref{ex:fdieXO6}, order and metric geometry are still preserved.
Here, an expected value of $\setn{0}$ simply means that no real physical value is expected 
more or less than any other real physical value.
Note also that the variance (expected error) is more than 11 times smaller than that of 
the corresponding statistical estimates on the real line 
($\sfrac{3}{12}$ versus $\sfrac{35}{12}$ \xrefnp{ex:fairdieXRYZ}).
%\end{minipage}\hfill%
%\begin{tabular}{c}
%  \gsize%
%  %\psset{unit=5mm}%
%  {%============================================================================
% Daniel J. Greenhoe
% LaTeX file
% discrete metric real dice mapping to linearly ordered O6c
%============================================================================
{%\psset{unit=0.5\psunit}%
\begin{pspicture}(-3.2,-1.5)(3.2,1.5)%
  %---------------------------------
  % options
  %---------------------------------
  \psset{%
    radius=1.25ex,
    labelsep=2.5mm,
    linecolor=blue,%
    }%
  %---------------------------------
  % dice graph
  %---------------------------------
  \rput(-1.75,0){%\psset{unit=2\psunit}%
    \uput{1}[210](0,0){\Cnode[fillstyle=solid,fillcolor=snode](0,0){D4}}%
    \uput{1}[150](0,0){\Cnode[fillstyle=solid,fillcolor=snode](0,0){D5}}%
    \uput{1}[ 90](0,0){\Cnode[fillstyle=solid,fillcolor=snode](0,0){D6}}%
    \uput{1}[ 30](0,0){\Cnode[fillstyle=solid,fillcolor=snode](0,0){D3}}%
    \uput{1}[-30](0,0){\Cnode[fillstyle=solid,fillcolor=snode](0,0){D2}}%
    \uput{1}[-90](0,0){\Cnode[fillstyle=solid,fillcolor=snode](0,0){D1}}%
    \uput{1.25}[120](0,0){$\ocsG$}%
    }
  \rput(D6){$\diceF$}%
  \rput(D5){$\diceE$}%
  \rput(D4){$\diceD$}%
  \rput(D3){$\diceC$}%
  \rput(D2){$\diceB$}%
  \rput(D1){$\diceA$}%
  %
  \ncline{D5}{D6}%
  \ncline{D4}{D5}\ncline{D4}{D6}%
  \ncline{D3}{D4}\ncline{D3}{D5}\ncline{D3}{D6}%
  \ncline{D2}{D3}\ncline{D2}{D4}\ncline{D2}{D5}\ncline{D2}{D6}%
  \ncline{D1}{D2}\ncline{D1}{D3}\ncline{D1}{D4}\ncline{D1}{D5}\ncline{D1}{D6}%
  %
  \uput[158](D6){$\frac{1}{6}$}
  \uput[150](D5){$\frac{1}{6}$}
  \uput[210](D4){$\frac{1}{6}$}
  \uput[ 22](D3){$\frac{1}{6}$}
  \uput[-45](D2){$\frac{1}{6}$}
  \uput[-158](D1){$\frac{1}{6}$}
  %---------------------------------
  % range graph
  %---------------------------------
  \rput(1.75,0){%\psset{unit=2\psunit}%
    \uput{1}[210](0,0){\Cnode(0,0){E4}}%
    \uput{1}[150](0,0){\Cnode(0,0){E5}}%
    \uput{1}[ 90](0,0){\Cnode(0,0){E6}}%
    \uput{1}[ 30](0,0){\Cnode(0,0){E3}}%
    \uput{1}[-30](0,0){\Cnode(0,0){E2}}%
    \uput{1}[-90](0,0){\Cnode(0,0){E1}}%
    \Cnode[fillstyle=solid,fillcolor=snode](0,0){E0}%
    \uput{1.25}[60](0,0){$\omsH$}%
    }
  \rput(E6){$6$}%
  \rput(E5){$5$}%
  \rput(E4){$4$}%
  \rput(E3){$3$}%
  \rput(E2){$2$}%
  \rput(E1){$1$}%
  \rput(E0){$0$}%
  %
  %\ncline{E5}{E6}%
  %\ncline{E4}{E5}\ncline{E4}{E6}%
  %\ncline{E3}{E4}\ncline{E3}{E5}\ncline{E3}{E6}%
  %\ncline{E2}{E3}\ncline{E2}{E4}\ncline{E2}{E5}\ncline{E2}{E6}%
  %\ncline{E1}{E2}\ncline{E1}{E3}\ncline{E1}{E4}\ncline{E1}{E5}\ncline{E1}{E6}%
  \ncline{E0}{E1}\naput{$\frac{1}{2}$}
  \ncline{E0}{E2}\naput{$\frac{1}{2}$}
  \ncline{E0}{E3}\naput{$\frac{1}{2}$}
  \ncline{E0}{E4}\naput{$\frac{1}{2}$}
  \ncline{E0}{E5}\naput{$\frac{1}{2}$}
  \ncline{E0}{E6}\naput{$\frac{1}{2}$}%
  %
  \uput[ 22](E6){$\frac{1}{6}$}
  \uput[135](E5){$\frac{1}{6}$}
  \uput[200](E4){$\frac{1}{6}$}
  \uput[ 30](E3){$\frac{1}{6}$}
  \uput[-30](E2){$\frac{1}{6}$}
  \uput[-22](E1){$\frac{1}{6}$}
  \uput[  0](E0){$\sfrac{0}{6}$}
  %---------------------------------
  % mapping from die to O6c
  %---------------------------------
  \ncarc[arcangle= 22,linewidth=0.75pt,linecolor=red]{->}{D6}{E6}%
  \ncarc[arcangle= 22,linewidth=0.75pt,linecolor=red]{->}{D5}{E5}%
  \ncarc[arcangle= 22,linewidth=0.75pt,linecolor=red]{->}{D4}{E4}%
  \ncarc[arcangle=-11,linewidth=0.75pt,linecolor=red]{->}{D3}{E3}%
  \ncarc[arcangle=-11,linewidth=0.75pt,linecolor=red]{->}{D2}{E2}%
  \ncarc[arcangle=-22,linewidth=0.75pt,linecolor=red]{->}{D1}{E1}%
  %---------------------------------
  % labels
  %---------------------------------
  \rput(0,0){$\rvX(\cdot)$}%
  %\ncline[linestyle=dotted,nodesep=1pt]{->}{xzlabel}{xz}%
  %\ncline[linestyle=dotted,nodesep=1pt]{->}{ylabel}{y}%
\end{pspicture}
}%}%
%\end{tabular}
\end{example}
\begin{proof}
    \begin{align*}
      \ocsE(\rvX)
        &\eqd \argmin_{x\in\omsH}\max_{y\in\omsH}\metric{x}{y}\psp(y)
        &&\text{by definition of $\ocsE$ \xref{def:ocsE}}
      \\&= \argmin_{x\in\omsH}\max_{y\in\omsH\setd\setn{0}}\metric{x}{y}\psp(y)
        &&\text{because $\psp(0)=0$}
      \\&= \argmin_{x\in\omsH}\max_{y\in\omsH\setd\setn{0}}\metric{x}{y}\frac{1}{6}
      \\&= \argmin_{x\in\omsH}\max_{y\in\omsH\setd\setn{0}}\metric{x}{y}
        && \text{because $\ff(x)=\frac{1}{6}x$ is \prope{strictly isotone} and by \prefpp{lem:argminmaxphi}}
      %\\&= \argmin_{x\in\omsH}\max_{y\in\omsH}
      %       \setn{\begin{array}{cccccc}
      %         \metricn(1,1) &\metricn(1,2) &\metricn(1,3) &\metricn(1,4) &\metricn(1,5) &\metricn(1,6) \\
      %         \metricn(2,1) &\metricn(2,2) &\metricn(2,3) &\metricn(2,4) &\metricn(2,5) &\metricn(2,6) \\
      %         \metricn(3,1) &\metricn(3,2) &\metricn(3,3) &\metricn(3,4) &\metricn(3,5) &\metricn(3,6) \\
      %         \metricn(4,1) &\metricn(4,2) &\metricn(4,3) &\metricn(4,4) &\metricn(4,5) &\metricn(4,6) \\
      %         \metricn(5,1) &\metricn(5,2) &\metricn(5,3) &\metricn(5,4) &\metricn(5,5) &\metricn(5,6) \\
      %         \metricn(6,1) &\metricn(6,2) &\metricn(6,3) &\metricn(6,4) &\metricn(6,5) &\metricn(6,6) \\
      %         \metricn(0,1) &\metricn(0,2) &\metricn(0,3) &\metricn(0,4) &\metricn(0,5) &\metricn(0,6) 
      %       \end{array}}
      \\&=\mathrlap{\argmin_{x\in\omsH}\max_{y\in\omsH}
             \setn{\begin{array}{cccc}
               \metricn(1,1) &\metricn(1,2) &\cdots &\metricn(1,6) \\
               \metricn(2,1) &\metricn(2,2) &\cdots &\metricn(2,6) \\
               \metricn(3,1) &\metricn(3,2) &\cdots &\metricn(3,6) \\
               \metricn(4,1) &\metricn(4,2) &\cdots &\metricn(4,6) \\
               \metricn(5,1) &\metricn(5,2) &\cdots &\metricn(5,6) \\
               \metricn(6,1) &\metricn(6,2) &\cdots &\metricn(6,6) \\
               \metricn(0,1) &\metricn(0,2) &\cdots &\metricn(0,6) 
             \end{array}}
      = \argmin_{x\in\omsH}\max_{y\in\omsH}
             \setn{\begin{array}{*{6}{@{\,\,}c}}
               {0}&{1}&{1}&{1}&{1}&{2}\\
               {1}&{0}&{1}&{1}&{2}&{1}\\
               {1}&{1}&{0}&{2}&{1}&{1}\\
               {1}&{1}&{2}&{0}&{1}&{1}\\
               {1}&{2}&{1}&{1}&{0}&{1}\\
               {2}&{1}&{1}&{1}&{1}&{0}\\
               {1}&{1}&{1}&{1}&{1}&{1}
             \end{array}}
      = \argmin_{x\in\omsH}
             \setn{\begin{array}{c}
               {2}\\
               {2}\\
               {2}\\
               {2}\\
               {2}\\
               {2}\\
               {1}
             \end{array}}}
      \\&= \setn{0}
      %\\
      %\ocsEa(\rvX)
      %  &\eqd \argmin_{x\in\omsH}\sum_{y}\ocsmom(x,y)
      %  &&    \text{by definition of $\ocsEa$ \xref{def:ocsEa}}
      %\\&=\mathrlap{\argmin_{x\in\omsH}
      %       \setn{\begin{array}{*{13}{@{\,}c}}
      %         {0} &+& {1} &+& {1} &+& {1} &+& {1} &+& {2} &+& 0\\
      %         {1} &+& {0} &+& {1} &+& {1} &+& {2} &+& {1} &+& 0\\
      %         {1} &+& {1} &+& {0} &+& {2} &+& {1} &+& {1} &+& 0\\
      %         {1} &+& {1} &+& {2} &+& {0} &+& {1} &+& {1} &+& 0\\
      %         {1} &+& {2} &+& {1} &+& {1} &+& {0} &+& {1} &+& 0\\
      %         {2} &+& {1} &+& {1} &+& {1} &+& {1} &+& {0} &+& 0\\
      %         {1} &+& {1} &+& {1} &+& {1} &+& {1} &+& {1} &+& 0
      %       \end{array}}
      %= \argmin_{x\in\omsH}
      %       \setn{\begin{array}{c}
      %         6\\
      %         6\\
      %         6\\
      %         6\\
      %         6\\
      %         6\\
      %         6
      %       \end{array}}}
      %\\&= \setn{0,1,2,3,4,5,6}
  \\
  \ocsVar(\rvX)
    &\eqd \sum_{x\in\omsH}\metricsq{\ocsE(\rvX)}{x}\psp(x)
    && \text{by definition of $\ocsVar$ \xref{def:ocsVar}}
  \\&= \sum_{x\in\omsH}\metricsq{\setn{0}}{x}\psp(x)
    && \text{by $\ocsE(\rvX)$ result}
  \\&= \sum_{x\in\omsH\setd\setn{0}}\metricsq{\setn{0}}{x}\frac{1}{6}
    && \text{by definition of $\ocsG$}
  \\&= 6 \brp{\frac{1}{2}}^2 \frac{1}{6} = \frac{1}{4}
    && \text{by definition of $\omsH$}
\end{align*}
%\end{enumerate}
\end{proof}

%%---------------------------------------
%\begin{minipage}{\tw-65mm}%
%\begin{example}[\exmd{fair die mapping with extended range 2}]
%\label{ex:dieXO6c2}\mbox{}\\
%%---------------------------------------
%%\prefpp{ex:realdie} presented a mapping from a real die to a linearly ordered lattice.
%In this example, the range of the random variable $\rvX$ has been extended as compared to 
%\prefpp{ex:dieXO6}, yielding simpler statistical results (see illustration to the right).
%As in \pref{ex:dieXO6}, order and topology are still preserved.
%\\\indentx$\begin{array}{rclD}
%  \ocsE(\rvX)  &=& \setn{0} & \\%(shaded in \prefp{fig:pairdice})\\
%  \ocsEa(\rvX) &=& \setn{0}\\
%  \ocsVar(\rvX)  &=& 1
%\end{array}$
%\end{example}
%\end{minipage}\hfill%
%\begin{tabular}{c}
%  \gsize%
%  %\psset{unit=5mm}%
%  {%============================================================================
% Daniel J. Greenhoe
% LaTeX file
% discrete metric real dice mapping to linearly ordered O6c
%============================================================================
{%\psset{unit=0.5\psunit}%
\begin{pspicture}(-3.2,-1.5)(3.2,1.5)%
  %---------------------------------
  % options
  %---------------------------------
  \psset{%
    radius=1.25ex,
    labelsep=2.5mm,
    linecolor=blue,%
    }%
  %---------------------------------
  % dice graph
  %---------------------------------
  \rput(-1.75,0){%\psset{unit=2\psunit}%
    \Cnode[fillstyle=solid,fillcolor=snode](-0.8660,-0.5){D4}%
    \Cnode[fillstyle=solid,fillcolor=snode](-0.8660,0.5){D5}%
    \Cnode[fillstyle=solid,fillcolor=snode](0,1){D6}%
    \Cnode[fillstyle=solid,fillcolor=snode](0.8660,0.5){D3}%
    \Cnode[fillstyle=solid,fillcolor=snode](0.8660,-0.5){D2}%
    \Cnode[fillstyle=solid,fillcolor=snode](0,-1){D1}%
    }
  \rput(D6){$\diceF$}%
  \rput(D5){$\diceE$}%
  \rput(D4){$\diceD$}%
  \rput(D3){$\diceC$}%
  \rput(D2){$\diceB$}%
  \rput(D1){$\diceA$}%
  %
  \ncline{D5}{D6}%
  \ncline{D4}{D5}\ncline{D4}{D6}%
  \ncline{D3}{D5}\ncline{D3}{D6}%
  \ncline{D2}{D3}\ncline{D2}{D4}\ncline{D2}{D6}%
  \ncline{D1}{D2}\ncline{D1}{D3}\ncline{D1}{D4}\ncline{D1}{D5}%
  %
  \uput[158](D6){$\frac{1}{6}$}
  \uput[150](D5){$\frac{1}{6}$}
  \uput[210](D4){$\frac{1}{6}$}
  \uput[ 22](D3){$\frac{1}{6}$}
  \uput[-45](D2){$\frac{1}{6}$}
  \uput[-158](D1){$\frac{1}{6}$}
  %---------------------------------
  % range graph
  %---------------------------------
  \rput(1.75,0){%\psset{unit=2\psunit}%
    \Cnode(-0.8660,-0.5){E4}%
    \Cnode(-0.8660,0.5){E5}%
    \Cnode(0,1){E6}%
    \Cnode(0.8660,0.5){E3}%
    \Cnode(0.8660,-0.5){E2}%
    \Cnode(0,-1){E1}%
    \Cnode[fillstyle=solid,fillcolor=snode](0,0){E0}%
    }
  \rput(E6){$6$}%
  \rput(E5){$5$}%
  \rput(E4){$4$}%
  \rput(E3){$3$}%
  \rput(E2){$2$}%
  \rput(E1){$1$}%
  \rput(E0){$0$}%
  %
  \ncline{E5}{E6}%
  \ncline{E4}{E5}\ncline{E4}{E6}%
  \ncline{E3}{E4}\ncline{E3}{E5}\ncline{E3}{E6}%
  \ncline{E2}{E3}\ncline{E2}{E4}\ncline{E2}{E5}\ncline{E2}{E6}%
  \ncline{E1}{E2}\ncline{E1}{E3}\ncline{E1}{E4}\ncline{E1}{E5}\ncline{E1}{E6}%
  \ncline{E0}{E1}\ncline{E0}{E2}\ncline{E0}{E3}\ncline{E0}{E4}\ncline{E0}{E5}\ncline{E0}{E6}%
  %
  \uput[ 22](E6){$\frac{1}{6}$}
  \uput[135](E5){$\frac{1}{6}$}
  \uput[200](E4){$\frac{1}{6}$}
  \uput[ 30](E3){$\frac{1}{6}$}
  \uput[-30](E2){$\frac{1}{6}$}
  \uput[-22](E1){$\frac{1}{6}$}
  \uput[  0](E0){$\sfrac{0}{6}$}
  %---------------------------------
  % mapping from die to O6c
  %---------------------------------
  \ncarc[arcangle= 22,linewidth=0.75pt,linecolor=red]{->}{D6}{E6}%
  \ncarc[arcangle= 22,linewidth=0.75pt,linecolor=red]{->}{D5}{E5}%
  \ncarc[arcangle= 22,linewidth=0.75pt,linecolor=red]{->}{D4}{E4}%
  \ncarc[arcangle=-22,linewidth=0.75pt,linecolor=red]{->}{D3}{E3}%
  \ncarc[arcangle=-22,linewidth=0.75pt,linecolor=red]{->}{D2}{E2}%
  \ncarc[arcangle=-22,linewidth=0.75pt,linecolor=red]{->}{D1}{E1}%
  %---------------------------------
  % labels
  %---------------------------------
  \rput(0,0){$\rvX(\cdot)$}%
  %\ncline[linestyle=dotted,nodesep=1pt]{->}{xzlabel}{xz}%
  %\ncline[linestyle=dotted,nodesep=1pt]{->}{ylabel}{y}%
\end{pspicture}
}%}%
%\end{tabular}


%=======================================
\subsubsection{Real die examples}
%=======================================
\begin{figure}[h]
  \centering%
  \gsize%
  %{%============================================================================
% Daniel J. Greenhoe
% LaTeX file
% discrete metric real die mapping to 4 outcome spaces
%============================================================================
{%\psset{unit=0.5\psunit}%
\begin{pspicture}(-6,-3.6)(6,3.8)%
  %---------------------------------
  % options
  %---------------------------------
  \psset{%
    radius=1.25ex,
    labelsep=2.5mm,
    linecolor=blue,%
    fillstyle=none,
    }%
  %---------------------------------
  % die graph
  %---------------------------------
  \rput(0,0){%\psset{unit=2\psunit}%
    \uput{1}[210](0,0){\Cnode[fillstyle=solid,fillcolor=snode](0,0){D4}}%
    \uput{1}[150](0,0){\Cnode[fillstyle=solid,fillcolor=snode](0,0){D5}}%
    \uput{1}[ 90](0,0){\Cnode[fillstyle=solid,fillcolor=snode](0,0){D6}}%
    \uput{1}[ 30](0,0){\Cnode[fillstyle=solid,fillcolor=snode](0,0){D3}}%
    \uput{1}[-30](0,0){\Cnode[fillstyle=solid,fillcolor=snode](0,0){D2}}%
    \uput{1}[-90](0,0){\Cnode[fillstyle=solid,fillcolor=snode](0,0){D1}}%
    \rput(0,0){$\ocsG$}%
    }
  \rput(D6){$\diceF$}%
  \rput(D5){$\diceE$}%
  \rput(D4){$\diceD$}%
  \rput(D3){$\diceC$}%
  \rput(D2){$\diceB$}%
  \rput(D1){$\diceA$}%
  %
  \ncline{D5}{D6}%
  \ncline{D4}{D5}\ncline{D4}{D6}%
  \ncline{D3}{D5}\ncline{D3}{D6}%
  \ncline{D2}{D3}\ncline{D2}{D4}\ncline{D2}{D6}%
  \ncline{D1}{D2}\ncline{D1}{D3}\ncline{D1}{D4}\ncline{D1}{D5}%
  %
  \uput[ 210](D4){$\frac{1}{6}$}
  \uput[ 150](D5){$\frac{1}{6}$}
  \uput[  90](D6){$\frac{1}{6}$}
  \uput[  30](D3){$\frac{1}{6}$}
  \uput[ -30](D2){$\frac{1}{6}$}
  \uput[ -90](D1){$\frac{1}{6}$}
  %---------------------------------
  % isomorphic mapping Y
  %---------------------------------
  \rput(-4.5,0){%\psset{unit=2\psunit}%
    \uput{1}[210](0,0){\Cnode[fillstyle=solid,fillcolor=snode](0,0){Y4}}%
    \uput{1}[150](0,0){\Cnode[fillstyle=solid,fillcolor=snode](0,0){Y5}}%
    \uput{1}[ 90](0,0){\Cnode[fillstyle=solid,fillcolor=snode](0,0){Y6}}%
    \uput{1}[ 30](0,0){\Cnode[fillstyle=solid,fillcolor=snode](0,0){Y3}}%
    \uput{1}[-30](0,0){\Cnode[fillstyle=solid,fillcolor=snode](0,0){Y2}}%
    \uput{1}[-90](0,0){\Cnode[fillstyle=solid,fillcolor=snode](0,0){Y1}}%
    \rput(0,0){$\omsH$}%
    }
  \rput(Y6){$6$}%
  \rput(Y5){$5$}%
  \rput(Y4){$4$}%
  \rput(Y3){$3$}%
  \rput(Y2){$2$}%
  \rput(Y1){$1$}%
  %
  \ncline{Y5}{Y6}%
  \ncline{Y4}{Y5}\ncline{Y4}{Y6}%
  \ncline{Y3}{Y5}\ncline{Y3}{Y6}%
  \ncline{Y2}{Y3}\ncline{Y2}{Y4}\ncline{Y2}{Y6}%
  \ncline{Y1}{Y2}\ncline{Y1}{Y3}\ncline{Y1}{Y4}\ncline{Y1}{Y5}%
  %
  \uput[210](Y4){$\frac{1}{6}$}%
  \uput[150](Y5){$\frac{1}{6}$}%
  \uput[ 90](Y6){$\frac{1}{6}$}%
  \uput[ 30](Y3){$\frac{1}{6}$}%
  \uput[-30](Y2){$\frac{1}{6}$}%
  \uput[-90](Y1){$\frac{1}{6}$}%
  %
  \ncarc[arcangle=-22,linewidth=0.75pt,linecolor=blue]{->}{D6}{Y6}%
  \ncarc[arcangle=-67,linewidth=0.75pt,linecolor=blue]{->}{D5}{Y5}%
  \ncarc[arcangle= 67,linewidth=0.75pt,linecolor=blue]{->}{D4}{Y4}%
  \ncarc[arcangle=-67,linewidth=0.75pt,linecolor=blue]{->}{D3}{Y3}%
  \ncarc[arcangle= 67,linewidth=0.75pt,linecolor=blue]{->}{D2}{Y2}%
  \ncarc[arcangle= 22,linewidth=0.75pt,linecolor=blue]{->}{D1}{Y1}%
  %---------------------------------
  % random variable mapping Z to extended outcome space mapping Z
  %---------------------------------
  \rput(4.5,0){%\psset{unit=2\psunit}%
    \uput{1}[210](0,0){\Cnode(0,0){Z4}}%
    \uput{1}[150](0,0){\Cnode(0,0){Z5}}%
    \uput{1}[ 90](0,0){\Cnode(0,0){Z6}}%
    \uput{1}[ 30](0,0){\Cnode(0,0){Z3}}%
    \uput{1}[-30](0,0){\Cnode(0,0){Z2}}%
    \uput{1}[-90](0,0){\Cnode(0,0){Z1}}%
    \Cnode[fillstyle=solid,fillcolor=snode](0,0){Z0}%
    \uput{1.25}[60](0,0){$\omsK$}%
    }%
  \rput(Z6){$6$}%
  \rput(Z5){$5$}%
  \rput(Z4){$4$}%
  \rput(Z3){$3$}%
  \rput(Z2){$2$}%
  \rput(Z1){$1$}%
  \rput(Z0){$0$}%
  %
  \ncline{Z5}{Z6}%
  \ncline{Z4}{Z5}\ncline{Z4}{Z6}%
  \ncline{Z3}{Z5}\ncline{Z3}{Z6}%
  \ncline{Z2}{Z3}\ncline{Z2}{Z4}\ncline{Z2}{Z6}%
  \ncline{Z1}{Z2}\ncline{Z1}{Z3}\ncline{Z1}{Z4}\ncline{Z1}{Z5}%
  \ncline{Z0}{Z1}\ncline{Z0}{Z2}\ncline{Z0}{Z3}\ncline{Z0}{Z4}\ncline{Z0}{Z5}\ncline{Z0}{Z6}%
  %
  \uput[210](Z4){$\frac{1}{6}$}
  \uput[150](Z5){$\frac{1}{6}$}
  \uput[ 90](Z6){$\frac{1}{6}$}
  \uput[ 30](Z3){$\frac{1}{6}$}
  \uput[-30](Z2){$\frac{1}{6}$}
  \uput[-90](Z1){$\frac{1}{6}$}
  \uput[  0](Z0){$\sfrac{0}{6}$}
  %
  \ncarc[arcangle= 22,linewidth=0.75pt,linecolor=green]{->}{D6}{Z6}%
  \ncarc[arcangle= 67,linewidth=0.75pt,linecolor=green]{->}{D5}{Z5}%
  \ncarc[arcangle=-67,linewidth=0.75pt,linecolor=green]{->}{D4}{Z4}%
  \ncarc[arcangle= 67,linewidth=0.75pt,linecolor=green]{->}{D3}{Z3}%
  \ncarc[arcangle=-67,linewidth=0.75pt,linecolor=green]{->}{D2}{Z2}%
  \ncarc[arcangle=-22,linewidth=0.75pt,linecolor=green]{->}{D1}{Z1}%
  %---------------------------------
  % random variable mapping W from G to real line 
  %---------------------------------
  \rput(0,-3){%\psset{unit=0.75\psunit}%
    \multirput(-2.5,0)(1,0){6}{\psline(0,-0.1)(0,0.1)}%
    \pnode(3.5,0){WB}%
    \pnode(2.5,0){W6}%
    \pnode(1.5,0){W5}%
    \pnode(0.5,0){W4}%
    \pnode(0,0){W34}%
    \pnode(-0.5,0){W3}%
    \pnode(-1.5,0){W2}%
    \pnode(-2.5,0){W1}%
    \pnode(-3.5,0){WA}%
    }
  \uput[-90](W6){$6$}%
  \uput[-90](W5){$5$}%
  \uput[-90](W4){$4$}%
  \uput[-90](W3){$3$}%
  \uput[-90](W2){$2$}%
  \uput[-90](W1){$1$}%
  %
  \ncline{<->}{WA}{WB}%
  \pscircle[fillstyle=solid,linecolor=snode,fillcolor=snode](W34){1ex}%
  \pscircle[fillstyle=none,linecolor=red,fillcolor=red](W34){1ex}%
  %\rput(W233){\pscircle[fillstyle=none,linecolor=red,fillcolor=red](0,0){1ex}}%
  %
  \uput[ 0](WB){$\omsR$}
  \uput[90](W6){$\frac{1}{6}$}
  \uput[90](W5){$\frac{1}{6}$}
  \uput[90](W4){$\frac{1}{6}$}
  \uput[90](W3){$\frac{1}{6}$}
  \uput[90](W2){$\frac{1}{6}$}
  \uput[90](W1){$\frac{1}{6}$}
  %
  \ncarc[arcangle= 67,linewidth=0.75pt,linecolor=purple]{->}{D6}{W6}%
  \ncarc[arcangle=-67,linewidth=0.75pt,linecolor=purple]{->}{D5}{W5}%
  \ncarc[arcangle= 22,linewidth=0.75pt,linecolor=purple]{->}{D4}{W4}%
  \ncarc[arcangle= 45,linewidth=0.75pt,linecolor=purple]{->}{D3}{W3}%
  \ncarc[arcangle= 22,linewidth=0.75pt,linecolor=purple]{->}{D2}{W2}%
  \ncarc[arcangle=-10,linewidth=0.75pt,linecolor=purple]{->}{D1}{W1}%
  %
  %---------------------------------
  % random variable mapping X from G to integer line 
  %---------------------------------
  \rput(0,3){%\psset{unit=0.75\psunit}%
    \pnode(3.5,0){XB}%
    \Cnode(2.5,0){X6}%
    \Cnode(1.5,0){X5}%
    \Cnode[fillstyle=solid,fillcolor=snode](0.5,0){X4}%
    \Cnode[fillstyle=solid,fillcolor=snode](-0.5,0){X3}%
    \Cnode(-1.5,0){X2}%
    \Cnode(-2.5,0){X1}%
    \pnode(-3.5,0){XA}%
    }
  \rput(X6){$6$}%
  \rput(X5){$5$}%
  \rput(X4){$4$}%
  \rput(X3){$3$}%
  \rput(X2){$2$}%
  \rput(X1){$1$}%
  %
  \ncline[linestyle=dotted]{X6}{XB}%
  \ncline{X5}{X6}%
  \ncline{X4}{X5}%
  \ncline{X3}{X4}%
  \ncline{X2}{X3}%
  \ncline{X1}{X2}%
  \ncline[linestyle=dotted]{X1}{XA}%
  %
  \uput[ 0](XB){$\omsZ$}
  \uput[90](X6){$\frac{1}{6}$}
  \uput[90](X5){$\frac{1}{6}$}
  \uput[90](X4){$\frac{1}{6}$}
  \uput[90](X3){$\frac{1}{6}$}
  \uput[90](X2){$\frac{1}{6}$}
  \uput[90](X1){$\frac{1}{6}$}
  %
  \ncarc[arcangle= 10,linewidth=0.75pt,linecolor=red]{->}{D6}{X6}%
  \ncarc[arcangle= 22,linewidth=0.75pt,linecolor=red]{->}{D5}{X5}%
  \ncarc[arcangle= 67,linewidth=0.75pt,linecolor=red]{->}{D4}{X4}%
  \ncarc[arcangle=-45,linewidth=0.75pt,linecolor=red]{->}{D3}{X3}%
  \ncarc[arcangle=-67,linewidth=0.75pt,linecolor=red]{->}{D2}{X2}%
  \ncarc[arcangle= 67,linewidth=0.75pt,linecolor=red]{->}{D1}{X1}%
  %
  %---------------------------------
  % labels
  %---------------------------------
  \rput(2.3,0){$\rvZ(\cdot)$}%
  \rput(-2.3,0){$\rvY(\cdot)$}%
  \rput(0,2){$\rvX(\cdot)$}%
  \rput(0,-2.2){$\rvW(\cdot)$}%
\end{pspicture}
}%}%
  {\includegraphics{../common/math/graphics/pdfs/rdie_wxyz.pdf}}%
  \caption{random variable mappings from the \structe{real die outcome subspace} to several 
  \structe{ordered metric space}s \xref{ex:realdieXRYZ} \label{fig:realdieXRYZ}}
\end{figure}
%\begin{figure}[h]
%  \centering%
%  \gsize%
%  {%============================================================================
% Daniel J. Greenhoe
% LaTeX file
% real dice mappings to real line and integer line
%============================================================================
\begin{pspicture}(-3,-1.5)(3.2,1.5)%
  %---------------------------------
  % options
  %---------------------------------
  \psset{%
    %radius=1.25ex,
    labelsep=2.5mm,
    linecolor=blue,%
    }%
  %---------------------------------
  % dice graph
  %---------------------------------
  \rput(0,0){%\psset{unit=2\psunit}%
    \Cnode[radius=1.25ex,fillstyle=solid,fillcolor=snode](-0.8660,-0.5){D4}%
    \Cnode[radius=1.25ex,fillstyle=solid,fillcolor=snode](-0.8660,0.5){D5}%
    \Cnode[radius=1.25ex,fillstyle=solid,fillcolor=snode](0,1){D6}%
    \Cnode[radius=1.25ex,fillstyle=solid,fillcolor=snode](0.8660,0.5){D3}%
    \Cnode[radius=1.25ex,fillstyle=solid,fillcolor=snode](0.8660,-0.5){D2}%
    \Cnode[radius=1.25ex,fillstyle=solid,fillcolor=snode](0,-1){D1}%
    }
  \rput(D6){$\diceF$}%
  \rput(D5){$\diceE$}%
  \rput(D4){$\diceD$}%
  \rput(D3){$\diceC$}%
  \rput(D2){$\diceB$}%
  \rput(D1){$\diceA$}%
  %
  \ncline{D5}{D6}%
  \ncline{D4}{D5}\ncline{D4}{D6}%
  \ncline{D3}{D5}\ncline{D3}{D6}%
  \ncline{D2}{D3}\ncline{D2}{D4}\ncline{D2}{D6}%
  \ncline{D1}{D2}\ncline{D1}{D3}\ncline{D1}{D4}\ncline{D1}{D5}%
  %
  \uput[158](D6){$\frac{1}{6}$}
  \uput[150](D5){$\frac{1}{6}$}
  \uput[210](D4){$\frac{1}{6}$}
  \uput[ 22](D3){$\frac{1}{6}$}
  \uput[-45](D2){$\frac{1}{6}$}
  \uput[-158](D1){$\frac{1}{6}$}
  %---------------------------------
  % Real line
  %---------------------------------
  \rput(-2.75,0){\psset{unit=0.35\psunit}%
    \psline{<->}(0,-3.5)(0,3.5)%
    \pnode(0,2.5){R6}%
    \pnode(0,1.5){R5}%
    \pnode(0,0.5){R4}%
    \pnode(0,0){R34}%
    \pnode(0,-0.5){R3}%
    \pnode(0,-1.5){R2}%
    \pnode(0,-2.5){R1}%
    \Cnode[fillstyle=solid,linecolor=snode,fillcolor=snode,radius=0.5ex](0,0){RC}%
    }%
  %\ncline{R5}{R6}%
  %\ncline{R4}{R5}%
  %\ncline{R3}{R4}%
  %\ncline{R2}{R3}%
  %\ncline{R1}{R2}%
  \rput(R6){\psline[linewidth=1pt](-0.1,0)(0.1,0)}%
  \rput(R5){\psline[linewidth=1pt](-0.1,0)(0.1,0)}%
  \rput(R4){\psline[linewidth=1pt](-0.1,0)(0.1,0)}%
  \rput(R3){\psline[linewidth=1pt](-0.1,0)(0.1,0)}%
  \rput(R2){\psline[linewidth=1pt](-0.1,0)(0.1,0)}%
  \rput(R1){\psline[linewidth=1pt](-0.1,0)(0.1,0)}%
  %
  \uput[180](R6){$6$}%
  \uput[180](R5){$5$}%
  \uput[180](R4){$4$}%
  \uput[180](R3){$3$}%
  \uput[180](R2){$2$}%
  \uput[180](R1){$1$}%
  \uput[0](R34){$3.5$}%
  %
  \ncarc[arcangle=-22,linewidth=0.75pt,linecolor=red]{->}{D6}{R6}%
  \ncarc[arcangle=-22,linewidth=0.75pt,linecolor=red]{->}{D5}{R5}%
  \ncarc[arcangle=-22,linewidth=0.75pt,linecolor=red]{->}{D4}{R4}%
  \ncarc[arcangle= 22,linewidth=0.75pt,linecolor=red]{->}{D3}{R3}%
  \ncarc[arcangle= 22,linewidth=0.75pt,linecolor=red]{->}{D2}{R2}%
  \ncarc[arcangle= 22,linewidth=0.75pt,linecolor=red]{->}{D1}{R1}%
  %
  \rput(-1.8,1){$\rvX(\cdot)$}%
  %---------------------------------
  % integer line
  %---------------------------------
  \rput(2.75,0){\psset{unit=0.35\psunit}%
    \pnode(0,3.5){I7}%
    \Cnode(0,2.5){I6}%
    \Cnode(0,1.5){I5}%
    \Cnode[fillstyle=solid,fillcolor=snode](0,0.5){I4}%
    \Cnode[fillstyle=solid,fillcolor=snode](0,-0.5){I3}%
    \Cnode(0,-1.5){I2}%
    \Cnode(0,-2.5){I1}%
    \pnode(0,-3.5){I0}%
    }%
  \ncline[linestyle=dotted]{I6}{I7}%
  \ncline{I5}{I6}%
  \ncline{I4}{I5}%
  \ncline{I3}{I4}%
  \ncline{I2}{I3}%
  \ncline{I1}{I2}%
  \ncline[linestyle=dotted]{I1}{I0}%
  %
  \uput[0](I6){$6$}%
  \uput[0](I5){$5$}%
  \uput[0](I4){$4$}%
  \uput[0](I3){$3$}%
  \uput[0](I2){$2$}%
  \uput[0](I1){$1$}%
  %
  \ncarc[arcangle= 22,linewidth=0.75pt,linecolor=green]{->}{D6}{I6}%
  \ncarc[arcangle= 22,linewidth=0.75pt,linecolor=green]{->}{D5}{I5}%
  \ncarc[arcangle= 22,linewidth=0.75pt,linecolor=green]{->}{D4}{I4}%
  \ncarc[arcangle=-22,linewidth=0.75pt,linecolor=green]{->}{D3}{I3}%
  \ncarc[arcangle=-22,linewidth=0.75pt,linecolor=green]{->}{D2}{I2}%
  \ncarc[arcangle=-22,linewidth=0.75pt,linecolor=green]{->}{D1}{I1}%
  %
  \rput(1.8,1){$\rvY(\cdot)$}%
\end{pspicture}%}%
%  \caption{random variable mappings from the real die to the real line and integer line \label{fig:realdieXRYZ}}
%\end{figure}
%---------------------------------------
%\begin{minipage}{\tw-50mm}%
\begin{example}[\exmd{real die mappings}]
\label{ex:realdieXRYZ} %\mbox{}\\
%---------------------------------------
Let $\ocsG$ be the \structe{real die outcome subspace} \xref{ex:realdie}.
Let $\rvW$, $\rvX$, $\rvY$ and $\rvZ$ be \fncte{random variable} \xref{def:ocsrv} mappings 
as illustrated in \prefpp{fig:realdieXRYZ}.
%Let $\omsH$ be the \structe{ordered metric space} \xref{def:oms} induced by the \structe{real line} \xref{def:Rline}.
%Let $\omsK$ be the \structe{ordered metric space} induced by the \structe{integer line} \xref{def:Zline}.
%Let $\rvX$ be a random variable in $\clOCSgh$ (a mapping from $\ocsG$ to $\omsH$).
%Let $\rvY$ be a random variable in $\clOCSgk$.
%These structures are illustrated in \prefpp{fig:realdieXRYZ}.
Let $\pE$, $\pVar$, $\ocsE$, and $\ocsVar$ be defined as in \prefpp{ex:fairdieXRYZ}.
This yields the following statistics:
\\$\begin{array}{>{\gsizes}Mlcl@{\quad}lcccl}
  geometry of $\ocsG$:                                  & \ocscen (\ocsG) &=& \mc{4}{l}{\setn{\diceA,\diceB,\diceC,\diceD,\diceE,\diceF}}              \\
  traditional statistics on real line:                  & \pE   (\rvW) &=& 3.5                & \ocsVar(\rvW;\pE)   &=& \frac{35}{12} &\approx& 2.917  \\
  outcome subspace statistics on real line:             & \ocsE (\rvW) &=& \setn{3.5}         & \ocsVar(\rvW;\ocsE) &=& \frac{35}{12} &\approx& 2.917  \\
  outcome subspace statistics on integer line:          & \ocsE (\rvX) &=& \setn{3,\,4}       & \ocsVar(\rvX;\ocsE) &=& \frac{20}{12} &\approx& 1.667  \\
  outcome subspace statistics on isomorphic structure:  & \ocsE (\rvY) &=& \setn{1,2,\ldots,6} & \ocsVar(\rvY;\ocsE) &=& 0             &       &        \\
  outcome subspace statistics on extended   structure:  & \ocsE (\rvZ) &=& \setn{0}           & \ocsVar(\rvZ;\ocsE) &=& 1             &       &        
\end{array}$
\\
Similar to \prefpp{ex:fdieXO6}, the statistic $\ocsE(\rvZ)=\setn{0}$ 
indicates a statistic carrying no information about an expected outcome.
Again, one could easily argue that no information is much better than misleading information.
\end{example}
\begin{proof}
%\begin{enumerate}
%  \item geometry of \structe{real die} \structe{outcome subspace} $\ocsG$ \xref{def:ocs}:\label{item:rdieXO6_gcen}
    \begin{align*}
      \ocscen(\ocsG) 
        &=\setn{\diceA,\diceB,\diceC,\diceD,\diceE,\diceF}
        &&\text{by \prefpp{ex:realdie}}
      \\
      \pE(\rvW) 
      %  &\eqd\sum_{x\in\omsH} x\psp(x)
      %  && \text{by definition of $\pE$ \xref{def:pE}}
        &= \frac{7}{2} = 3.5
        && \text{by $\pE(\rvX)$ result of \prefp{ex:fairdieXRYZ}}
      \\
      \ocsVar(\rvW;\pE)
      %  &= \pVar(\rvW)
      %  && \text{by \prefp{thm:ocsVar}}
      %\\&\eqd \sum_{x\in\omsH} \brs{x-\pE(\rvX)}^2 \psp(x)
      %  &&\text{by definition of $\pVar$ \xref{def:pVar}}
      %\\&= \sum_{x\in\omsH} \brp{x-\frac{7}{2}}^2 \frac{1}{6}
      %  &&\text{by $\pE(\rvX)$ result}
      %\\&= \mathrlap{\brs{\brp{1-\frac{7}{2}}^2 + \brp{2-\frac{7}{2}}^2 + \brp{3-\frac{7}{2}}^2 + \brp{4-\frac{7}{2}}^2 + \brp{5-\frac{7}{2}}^2 + \brp{6-\frac{7}{2}}^2 }\frac{1}{6}}
      %\\&= \mathrlap{\brs{(2-7)^2 + (4-7)^2 + (6-7)^2 + (8-7)^2 + (10-7)^2 + (12-7)^2}\frac{1}{2^2\times6}}
      %\\&= \mathrlap{\frac{25+9+1+1+9+25}{24}
      %   = \frac{70}{24}
      %   = \frac{35}{12}
      %  \approx 2.917}
        &= \frac{35}{12}\approx 2.917
        && \text{by $\pVar(\rvX)$ result of \prefp{ex:fairdieXRYZ}}
      \\
      \ocsE(\rvW)
      %  &\eqd \argmin_{x\in\omsH}\max_{y\in\omsH}\metric{x}{y}\psp(y)
      %  &&\text{by definition of $\ocsE$ \xref{def:ocsE}}
        &= \setn{3.5}
        && \text{by $\ocsE(\rvX)$ result of \prefp{ex:fairdieXRYZ}}
      \\
      \ocsVar(\rvW;\ocsE)
      %  &\eqd \sum_{x\in\omsH}\metricsq{\ocsE(\rvX)}{x}\psp(x)
      %  && \text{by definition of $\ocsVar$ \xref{def:ocsVar}}
        &= \frac{35}{12} \approx 2.917
        && \text{by $\ocsVar(\rvX;\ocsE)$ result of \prefp{ex:fairdieXRYZ}}
      \\
      \ocsE(\rvX)
      %  &\eqd \argmin_{x\in\omsH}\max_{y\in\omsH}\metric{x}{y}\psp(y)
      %  &&\text{by definition of $\ocsE$ \xref{def:ocsE}}
        &= \setn{3,\,4}
        && \text{by $\ocsE(\rvY)$ result of \prefp{ex:fairdieXRYZ}}
      \\
      \ocsVar(\rvX;\ocsE)
      %  &\eqd \sum_{x\in\omsH}\metricsq{\ocsE(\rvX)}{x}\psp(x)
      %  && \text{by definition of $\ocsVar$ \xref{def:ocsVar}}
        &= \frac{5}{3} \approx 1.667
        && \text{by $\ocsVar(\rvY;\ocsE)$ result of \prefp{ex:fairdieXRYZ}}
      \\
      \ocsE(\rvY)
        &= \rvY\brp{\setn{\diceA,\diceB,\diceC,\diceD,\diceE,\diceF}}
        && \text{by $\ocscen(\ocsG)$ result of \prefp{ex:realdie}}
      \\&= \setn{1,2,3,4,5,6}
        && \text{by definition of $\rvY$}
      \\
      \ocsVar(\rvY;\ocsE)
        &\eqd \sum_{x\in\omsH}\metricsq{\ocsE(\rvY)}{x}\psp(x)
        && \text{by definition of $\ocsVar$ \xref{def:ocsVar}}
      \\&= \sum_{x\in\omsH}\metricsq{\omsH}{x}\psp(x)
        && \text{by $\ocsE(\rvY)$ result}
      \\&= \sum_{x\in\omsH}0^2{x}\psp(x)
        && \text{by \prope{nondegenerate} property of \fncte{quasi-metric}s \xref{def:qmetric}}
      \\&= 0
    \\
    \ocsE(\rvZ)
      &\eqd \argmin_{x\in\omsK}\max_{y\in\omsK}\metric{x}{y}\psp(y)
      &&\text{by definition of $\ocsE$ \xref{def:ocsE}}
    \\&= \argmin_{x\in\omsK}\max_{y\in\omsK\setd\setn{0}}\metric{x}{y}\psp(y)
      &&\text{because $\psp(0)=0$}
    \\&= \argmin_{x\in\omsK}\max_{y\in\omsK\setd\setn{0}}\metric{x}{y}\frac{1}{6}
      && \text{by definition of $\ocsG$ and $\rvZ$}
    \\&= \argmin_{x\in\omsK}\max_{y\in\omsK\setd\setn{0}}\metric{x}{y}
        && \text{because $\ff(x)=\frac{1}{6}x$ is \prope{strictly isotone} and by \prefpp{lem:argminmaxphi}}
    \\&=\mathrlap{\argmin_{x\in\omsK}\max_{y\in\omsK}
           \setn{\begin{array}{*{4}{@{\,}c}}
             \metricn(1,1) &\metricn(1,2) &\cdots&\metricn(1,6) \\
             \metricn(2,1) &\metricn(2,2) &\cdots&\metricn(2,6) \\
             \metricn(3,1) &\metricn(3,2) &\cdots&\metricn(3,6) \\
             \metricn(4,1) &\metricn(4,2) &\cdots&\metricn(4,6) \\
             \metricn(5,1) &\metricn(5,2) &\cdots&\metricn(5,6) \\
             \metricn(6,1) &\metricn(6,2) &\cdots&\metricn(6,6) \\
             \metricn(0,1) &\metricn(0,2) &\cdots&\metricn(0,6) 
           \end{array}}
    = \argmin_{x\in\omsK}\max_{y\in\omsK}
           \setn{\begin{array}{*{6}{@{\,\,}c}}
              0 & 1 & 1 & 1 & 1 & 2\\
              1 & 0 & 1 & 1 & 2 & 1\\
              1 & 1 & 0 & 2 & 1 & 1\\
              1 & 1 & 2 & 0 & 1 & 1\\
              1 & 2 & 1 & 1 & 0 & 1\\
              2 & 1 & 1 & 1 & 1 & 0\\
              1 & 1 & 1 & 1 & 1 & 1
           \end{array}}
    = \argmin_{x\in\omsK}
           \setn{\begin{array}{c}
             {2}\\
             {2}\\
             {2}\\
             {2}\\
             {2}\\
             {2}\\
             {1}
           \end{array}}
    = \setn{\begin{array}{c}
              \mbox{ }\\
              \mbox{ }\\
              \mbox{ }\\
              \mbox{ }\\
              \mbox{ }\\
              \mbox{ }\\
              0 
           \end{array}}}
    %\\
    %\ocsEa(\rvZ)
    %  &\eqd \argmin_{x\in\omsK}\sum_{y}\ocsmom(x,y)
    %  &&\text{by definition of $\ocsEa$ \xref{def:ocsEa}}
    %\\&= \argmin_{x\in\omsK}
    %       \setn{\begin{array}{*{11}{@{\hspace{2pt}}c}}
    %         0 &+& 1 &+& 1 &+& 1 &+& 1 &+& 2 \\
    %         1 &+& 0 &+& 1 &+& 1 &+& 2 &+& 1 \\
    %         1 &+& 1 &+& 0 &+& 2 &+& 1 &+& 1 \\
    %         1 &+& 1 &+& 2 &+& 0 &+& 1 &+& 1 \\
    %         1 &+& 2 &+& 1 &+& 1 &+& 0 &+& 1 \\
    %         2 &+& 1 &+& 1 &+& 1 &+& 1 &+& 0 \\
    %         1 &+& 1 &+& 1 &+& 1 &+& 1 &+& 1 
    %       \end{array}}
    %\quad= \argmin_{x\in\omsK}
    %       \setn{\begin{array}{c}
    %         6\\
    %         6\\
    %         6\\
    %         6\\
    %         6\\
    %         6\\
    %         6
    %       \end{array}}
    %\\&= \setn{0,1,2,3,4,5,6}
%  \end{align*}
   \\
%  \begin{align*}
    \ocsVar(\rvZ)
      &\eqd \sum_{x\in\omsK}\metricsq{\ocsE(\rvZ)}{x}\psp(x)
      && \text{by definition of $\ocsVar$ \xref{def:ocsVar}}
    \\&= \sum_{x\in\omsK\setd\setn{0}}\metricsq{\setn{0}}{x}\psp(x)
      && \text{by $\ocsE(\rvZ)$ result}
    \\&= 6\times1^2\times\frac{1}{6} = 1
      && \text{by $\ocsE(\rvZ)$ result}
  \end{align*}
%\end{enumerate}
\end{proof}



\begin{figure}[h]
  \gsize%
  \centering%
  %{%============================================================================
% Daniel J. Greenhoe
% LaTeX file
% discrete metric real dice mapping to linearly ordered L6
%============================================================================
{%\psset{unit=0.5\psunit}%
\begin{pspicture}(-6.2,-1.8)(6,2)%
  %---------------------------------
  % options
  %---------------------------------
  \psset{%
    linecolor=blue,%
    radius=1.25ex,
    labelsep=2.5mm,
    }%
  %---------------------------------
  % dice graph
  %---------------------------------
  \rput(0,0){%\psset{unit=2\psunit}%
    \uput{1}[210](0,0){\Cnode[fillstyle=solid,fillcolor=snode](0,0){D4}}%
    \uput{1}[150](0,0){\Cnode(0,0){D5}}%
    \uput{1}[ 90](0,0){\Cnode(0,0){D6}}%
    \uput{1}[ 30](0,0){\Cnode(0,0){D3}}%
    \uput{1}[-30](0,0){\Cnode(0,0){D2}}%
    \uput{1}[-90](0,0){\Cnode(0,0){D1}}%
    \rput(0,0){$\ocsG$}%
    }
  \rput(D6){$\diceF$}%
  \rput(D5){$\diceE$}%
  \rput(D4){$\diceD$}%
  \rput(D3){$\diceC$}%
  \rput(D2){$\diceB$}%
  \rput(D1){$\diceA$}%
  %
  \ncline{D5}{D6}%
  \ncline{D4}{D5}\ncline{D4}{D6}%
  \ncline{D3}{D5}\ncline{D3}{D6}%
  \ncline{D2}{D3}\ncline{D2}{D4}\ncline{D2}{D6}%
  \ncline{D1}{D2}\ncline{D1}{D3}\ncline{D1}{D4}\ncline{D1}{D5}%
  %
  \uput[  90](D6){$\frac{1}{30}$}
  \uput[ 150](D5){$\frac{1}{50}$}
  \uput[ 210](D4){$\frac{3}{5}$}
  \uput[  30](D3){$\frac{1}{30}$}
  \uput[ -60](D2){$\frac{1}{20}$}
  \uput[ -90](D1){$\frac{1}{10}$}
  %---------------------------------
  % mapping X to real line
  %---------------------------------
  \rput(-4.5,-1.75){\psset{unit=0.5\psunit}%
    \pnode(0,7){X7}%
    \pnode(0,6){X6}%
    \pnode(0,5){X5}%
    \pnode(0,4){X4}%
    \Cnode*[linecolor=snode,fillstyle=solid,fillcolor=snode](0,3){X3}%
    \pnode(0,2){X2}%
    \pnode(0,1){X1}%
    \pnode(0,0){X0}%
    \pscircle[fillstyle=none,linecolor=red,fillcolor=red](0,3.57){1ex}%
    }%
  \uput[180](X7){$\omsR$}%
  \ncline{->}{X6}{X7}%
  \ncline{X5}{X6}%
  \ncline{X4}{X5}%
  \ncline{X3}{X4}%
  \ncline{X2}{X3}%
  \ncline{X1}{X2}%
  \ncline{->}{X1}{X0}%
  %
  \rput(X6){\psline[linewidth=1pt](-0.1,0)(0.1,0)}%
  \rput(X5){\psline[linewidth=1pt](-0.1,0)(0.1,0)}%
  \rput(X4){\psline[linewidth=1pt](-0.1,0)(0.1,0)}%
  \rput(X3){\psline[linewidth=1pt](-0.1,0)(0.1,0)}%
  \rput(X2){\psline[linewidth=1pt](-0.1,0)(0.1,0)}%
  \rput(X1){\psline[linewidth=1pt](-0.1,0)(0.1,0)}%
  %
  \uput[180](X6){$6$}%
  \uput[180](X5){$5$}%
  \uput[180](X4){$4$}%
  \uput[180](X3){$3$}%
  \uput[180](X2){$2$}%
  \uput[180](X1){$1$}%
  %
  %\uput[ 158](X6){$\psp(6)=\frac{1}{30}$}
  %\uput[ 150](X5){$\psp(5)=\frac{1}{50}$}
  %\uput[ 210](X4){$\psp(4)=\frac{3}{5}$}
  %\uput[  22](X3){$\psp(3)=\frac{1}{30}$}
  %\uput[ -45](X2){$\psp(2)=\frac{1}{20}$}
  %\uput[-158](X1){$\psp(1)=\frac{1}{10}$}
  %
  \ncarc[arcangle=-22,linewidth=0.75pt,linecolor=red]{->}{D6}{X6}%
  \ncarc[arcangle=-22,linewidth=0.75pt,linecolor=red]{->}{D5}{X5}%
  \ncarc[arcangle=-22,linewidth=0.75pt,linecolor=red]{->}{D4}{X4}%
  \ncarc[arcangle= 22,linewidth=0.75pt,linecolor=red]{->}{D3}{X3}%
  \ncarc[arcangle= 45,linewidth=0.75pt,linecolor=red]{->}{D2}{X2}%
  \ncarc[arcangle= 22,linewidth=0.75pt,linecolor=red]{->}{D1}{X1}%
  %---------------------------------
  % isomorphic mapping Y
  %---------------------------------
  \rput(4.5,0){%\psset{unit=2\psunit}%
    \uput{1}[210](0,0){\Cnode[fillstyle=solid,fillcolor=snode](0,0){Y4}}%
    \uput{1}[150](0,0){\Cnode(0,0){Y5}}%
    \uput{1}[ 90](0,0){\Cnode(0,0){Y6}}%
    \uput{1}[ 30](0,0){\Cnode(0,0){Y3}}%
    \uput{1}[-30](0,0){\Cnode(0,0){Y2}}%
    \uput{1}[-90](0,0){\Cnode(0,0){Y1}}%
    \rput(0,0){$\omsK$}%
    }
  \rput(Y6){$6$}%
  \rput(Y5){$5$}%
  \rput(Y4){$4$}%
  \rput(Y3){$3$}%
  \rput(Y2){$2$}%
  \rput(Y1){$1$}%
  %
  \ncline{Y5}{Y6}%
  \ncline{Y4}{Y5}\ncline{Y4}{Y6}%
  \ncline{Y3}{Y5}\ncline{Y3}{Y6}%
  \ncline{Y2}{Y3}\ncline{Y2}{Y4}\ncline{Y2}{Y6}%
  \ncline{Y1}{Y2}\ncline{Y1}{Y3}\ncline{Y1}{Y4}\ncline{Y1}{Y5}%
  %
  \uput[  90](Y6){$\frac{1}{30}$}
  \uput[ 150](Y5){$\frac{1}{50}$}
  \uput[ 210](Y4){$\frac{3}{5}$}
  \uput[  30](Y3){$\frac{1}{30}$}
  \uput[ -60](Y2){$\frac{1}{20}$}
  \uput[ -90](Y1){$\frac{1}{10}$}
  %
  \ncarc[arcangle= 22,linewidth=0.75pt,linecolor=green]{->}{D6}{Y6}%
  \ncarc[arcangle= 67,linewidth=0.75pt,linecolor=green]{->}{D5}{Y5}%
  \ncarc[arcangle=-67,linewidth=0.75pt,linecolor=green]{->}{D4}{Y4}%
  \ncarc[arcangle= 67,linewidth=0.75pt,linecolor=green]{->}{D3}{Y3}%
  \ncarc[arcangle=-67,linewidth=0.75pt,linecolor=green]{->}{D2}{Y2}%
  \ncarc[arcangle=-22,linewidth=0.75pt,linecolor=green]{->}{D1}{Y1}%
  %---------------------------------
  % labels
  %---------------------------------
  \rput(-2.3,0){$\rvX(\cdot)$}%
  \rput(2.3,0){$\rvY(\cdot)$}%
  %\ncline[linestyle=dotted,nodesep=1pt]{->}{xzlabel}{xz}%
  %\ncline[linestyle=dotted,nodesep=1pt]{->}{ylabel}{y}%
\end{pspicture}
}%}%
  {\includegraphics{../common/math/graphics/pdfs/wdie_xy.pdf}}%
  \caption{\structe{weighted die} mappings \xref{ex:wdie_xy}\label{fig:wdie_xy}}
\end{figure}
%---------------------------------------
\begin{example}[\exmd{weighted die mappings}]
\label{ex:wdie_xy}
%---------------------------------------
Let $\ocsG$ be \structe{weighted die outcome subspace} \xref{ex:wdie},
and $\rvX$ and $\rvY$ be \fncte{random variable}s, as illustrated in \prefpp{fig:wdie_xy}.
Let $\pE$, $\pVar$, $\ocsE$, and $\ocsVar$ be defined as in \prefpp{ex:fairdieXRYZ}.
This yields the following statistics:
\\$\begin{array}{>{\gsizes}Mlcl @{\quad} rcccl}
  geometry of $\ocsG$:                                & \ocscen (\ocsG) &=& \mc{4}{l}{\setn{\diceD}}  \\
  traditional statistics on real line:                & \pE   (\rvX)    &=& 3                                       & \ocsVar(\rvX;\pE)   &=& \frac{143}{100} &=&       1.43  \\
  outcome subspace statistics on real line:           & \ocsE (\rvX)    &=& \setn{\frac{25}{7}} \approx \setn{3.57} & \ocsVar(\rvX;\ocsE) &=& \frac{3361}{2940} &\approx& 1.143 \\
  outcome subspace stats. on isomorphic structure & \ocsE (\rvY)    &=& \setn{4}                                & \ocsVar(\rvX;\ocsE) &=& \frac{101}{300} &\approx& 0.337   
\end{array}$
\\
The statistic $\pE(\rvX)=3$ evaluated on the \structe{real line} is arguably very poor because
it suggests that we ``expect" the event $\diceC$ rather than $\diceD$, even though 
$\psp(\diceD)$ is very large, $\psp(\diceC)$ is very small, 
and the physical distance $\metric{\diceD}{\diceC}=2$ on the die from $\diceD$ to $\diceC$ is twice as much as it is to any of the other 
four die faces.
%Intuitively and according to the variance (expected error) measures, even
If we retain use of the real line but replace the \fncte{traditional expected value} $\pE(\rvX)$
with the \fncte{outcome expected value} $\ocsE(\rvX)$,
a small but significant improvement is made ($\ocsVar(\rvX;\ocsE)\approx1.143<1.43=\ocsVar(\rvX;\pE)$).
%
Arguably a better choice still is to abandon the real line altogether
in favor of the isomorphic structure $\omsK$ and the statistic $\ocsE(\rvY)=\setn{4}$ evaluated on $\omsK$,
yielding not only an intuitively better result but also a variance $\ocsVar(\rvY;\ocsE)$ 
that is more than 4 times smaller than that of $\pE(\rvX)$
($\ocsVar(\rvY;\ocsE)\approx0.337<1.43=\ocsVar(\rvX;\pE)$).
%\begin{tabular}{c}
%  \gsize%
%  \psset{unit=5mm}%
%  {%============================================================================
% Daniel J. Greenhoe
% LaTeX file
% discrete metric real dice mapping to linearly ordered L6
%============================================================================
{%\psset{unit=0.5\psunit}%
\begin{pspicture}(0,-2.6)(8.5,2.6)%
  %---------------------------------
  % options
  %---------------------------------
  \psset{%
    linecolor=blue,%
    radius=1.25ex,
    labelsep=2.5mm,
    }%
  %---------------------------------
  % dice graph
  %---------------------------------
  \rput(2.4,0){\psset{unit=2\psunit}%
    \Cnode[fillstyle=solid,fillcolor=snode](-0.8660,-0.5){D4}%
    \Cnode(-0.8660,0.5){D5}%
    \Cnode(0,1){D6}%
    \Cnode(0.8660,0.5){D3}%
    \Cnode(0.8660,-0.5){D2}%
    \Cnode(0,-1){D1}%
    }
  \rput(D6){$\diceF$}%
  \rput(D5){$\diceE$}%
  \rput(D4){$\diceD$}%
  \rput(D3){$\diceC$}%
  \rput(D2){$\diceB$}%
  \rput(D1){$\diceA$}%
  %
  \ncline{D5}{D6}%
  \ncline{D4}{D5}\ncline{D4}{D6}%
  \ncline{D3}{D5}\ncline{D3}{D6}%
  \ncline{D2}{D3}\ncline{D2}{D4}\ncline{D2}{D6}%
  \ncline{D1}{D2}\ncline{D1}{D3}\ncline{D1}{D4}\ncline{D1}{D5}%
  %
  \uput[ 158](D6){$\frac{1}{30}$}
  \uput[ 150](D5){$\frac{1}{50}$}
  \uput[ 210](D4){$\frac{3}{5}$}
  \uput[  22](D3){$\frac{1}{30}$}
  \uput[ -45](D2){$\frac{1}{20}$}
  \uput[-158](D1){$\frac{1}{10}$}
  %---------------------------------
  % L6 lattice
  %---------------------------------
  \rput(8,0){\psset{unit=0.7\psunit}%
    \pnode(0,3.5){L7}%
    \pnode(0,2.5){L6}%
    \pnode(0,1.5){L5}%
    \pnode(0,0.5){L4}%
    \Cnode*[linecolor=snode,fillstyle=solid,fillcolor=snode](0,-0.5){L3}%
    \pnode(0,-1.5){L2}%
    \pnode(0,-2.5){L1}%
    \pnode(0,-3.5){L0}%
    }%
  \ncline{->}{L6}{L7}%
  \ncline{L5}{L6}%
  \ncline{L4}{L5}%
  \ncline{L3}{L4}%
  \ncline{L2}{L3}%
  \ncline{L1}{L2}%
  \ncline{->}{L1}{L0}%
  %
  \rput(L6){\psline[linewidth=1pt](-0.1,0)(0.1,0)}%
  \rput(L5){\psline[linewidth=1pt](-0.1,0)(0.1,0)}%
  \rput(L4){\psline[linewidth=1pt](-0.1,0)(0.1,0)}%
  \rput(L3){\psline[linewidth=1pt](-0.1,0)(0.1,0)}%
  \rput(L2){\psline[linewidth=1pt](-0.1,0)(0.1,0)}%
  \rput(L1){\psline[linewidth=1pt](-0.1,0)(0.1,0)}%
  %
  \uput[0](L6){$6$}%
  \uput[0](L5){$5$}%
  \uput[0](L4){$4$}%
  \uput[0](L3){$3$}%
  \uput[0](L2){$2$}%
  \uput[0](L1){$1$}%
  %
  %\uput[ 158](L6){$\psp(6)=\frac{1}{30}$}
  %\uput[ 150](L5){$\psp(5)=\frac{1}{50}$}
  %\uput[ 210](L4){$\psp(4)=\frac{3}{5}$}
  %\uput[  22](L3){$\psp(3)=\frac{1}{30}$}
  %\uput[ -45](L2){$\psp(2)=\frac{1}{20}$}
  %\uput[-158](L1){$\psp(1)=\frac{1}{10}$}
  %---------------------------------
  % mapping from die to L6
  %---------------------------------
  \ncarc[arcangle= 22,linewidth=0.75pt,linecolor=red]{->}{D6}{L6}%
  \ncarc[arcangle= 22,linewidth=0.75pt,linecolor=red]{->}{D5}{L5}%
  \ncarc[arcangle= 22,linewidth=0.75pt,linecolor=red]{->}{D4}{L4}%
  \ncarc[arcangle=-22,linewidth=0.75pt,linecolor=red]{->}{D3}{L3}%
  \ncarc[arcangle=-22,linewidth=0.75pt,linecolor=red]{->}{D2}{L2}%
  \ncarc[arcangle=-22,linewidth=0.75pt,linecolor=red]{->}{D1}{L1}%
  %\ncline[linewidth=0.75pt,linecolor=red]{->}{b}{lb}%
  %---------------------------------
  % labels
  %---------------------------------
  \rput(6.5,1.25){$\rvX(\cdot)$}%
  %\ncline[linestyle=dotted,nodesep=1pt]{->}{xzlabel}{xz}%
  %\ncline[linestyle=dotted,nodesep=1pt]{->}{ylabel}{y}%
\end{pspicture}
}%}%
%\end{tabular}
\end{example}
\begin{proof}
    \begin{align*}
      \ocscen(\ocsG)
      %  &\eqd \argmin_{x\in\ocsG}\max_{y\in\omsH}\ocsmom(x,y)
      %  &&\text{by definition of $\ocscen$ \xref{def:ocscen}}
      %\\&\eqd \argmin_{x\in\omsH}\max_{y\in\omsH}\metric{x}{y}\psp(y)
      %  &&\text{by definition of $\ocsmom$ \xref{def:ocsmom}}
        &= \setn{\diceD}
        &&\text{by \prefpp{ex:wdie}}
      %\\
      %\ocscena(\ocsG)
      %  &\eqd \argmin_{x\in\ocsG}\max_{y\in\omsH}\ocsmom(x,y)
      %  &&\text{by definition of $\ocscena$ \xref{def:ocscena}}
      %\\&\eqd \argmin_{x\in\omsH}\max_{y\in\omsH}\metric{x}{y}\psp(y)
      %  &&\text{by definition of $\ocsmom$ \xref{def:ocsmom}}
      %\\&= \setn{\diceD}
      %  &&\text{by \prefpp{ex:wdie}}
      %\\
      %\ocsVar(\ocsG)
      %  &\eqd \sum_{x\in\ocsG}\brs{\metric{\diceD}{x}}^2\psp(x)
      %  &&\text{by definition of $\ocsVar$ \xref{def:ocsVar}}
      %\\&= \frac{101}{300}
      %   \approx 0.33
      %  &&\text{by \prefpp{ex:wdie}}
      \\
      \pE(\rvX) 
        &\eqd \int_{\R} x\psp(x)\dx
        && \text{by definition of $\pE$ \xref{def:pE}}
      \\&= \sum_{x\in\Z} x\psp(x)
        && \text{by definition of $\psp$}
      \\&= \mathrlap{1\times\frac{1}{10}+2\times\frac{1}{20}+3\times\frac{1}{30}+4\times\frac{3}{5}+5\times\frac{1}{50}+6\times\frac{1}{30}}
      \\&=\mathrlap{\frac{1}{300}\brp{1\times30 + 2\times15 + 3\times10 + 4\times180 + 5\times6 + 6\times10}
         = \frac{900}{300} = 3}
      \\
      \ocsVar(\rvX;\pE)
        &= \pVar(\rvX)
        && \text{by \prefp{thm:ocsVar}}
      \\&\eqd \int_{\R} \brs{x-\pE(\rvX)}^2\psp(x)
        && \text{by definition of $\pVar$ \xref{def:pVar}}
      \\&= \sum_{x\in\Z} \brs{x-\pE(\rvX)}^2\psp(x)
        && \text{by definition of $\psp$}
      \\&= \mathrlap{\frac{1}{10}(1-3)^2 + \frac{1}{20}(2-3)^2 + \frac{1}{30}(3-3)^2 + \frac{3}{5}(4-3)^2 + \frac{1}{50}(5-3)^2 + \frac{1}{30}(6-3)^2} 
      \\&= \mathrlap{\frac{1}{300}\brp{120+15+0+180+24+90} = \frac{429}{300}=\frac{143}{100}=1.43}
      \\
      \ocsE(\rvX)
        &\eqd \argmin_{x\in\R}\max_{y\in\R}\metric{x}{y}\psp(y)
        &&\text{by definition of $\ocsE$ \xref{def:ocsE}}
      \\&\eqd \argmin_{x\in\R}\max_{y\in\R}\abs{x-y}\psp(y)
        &&\text{by definition standard metric on real line \xref{def:Rline}}
      \\&= \argmin_{x\in\R}\brbl{\begin{array}{lM}
             \abs{x-1}\frac{1}{10} & for $\frac{25}{7}\le x\le\frac{23}{5}$\\
             \abs{x-4}\frac{3}{5}  & otherwise
           \end{array}}
        %&& \text{\psset{unit=6mm}\gsize%============================================================================
% Daniel J. Greenhoe
% XeLaTeX file
%============================================================================
{%\psset{yunit=2\psunit}%
\begin{pspicture}(-1,0)(8,2.75)%
  \psset{%
    labelsep=1pt,
    linewidth=1pt,
    }%
  \psaxes[linecolor=axis,yAxis=false,labels=none]{->}(0,0)(0,0)(7.75,2.75)% x axis
  \psaxes[linecolor=axis,xAxis=false,labels=none]{->}(0,0)(0,0)(7.75,2.75)% y axis
  %
  \psline(0,2.4)(3.571,0.2571)(4.6,0.36)(7,1.8)%
  %
  \psline[linestyle=dotted,linecolor=red](3.571,0.2571)(3.571,0)%
  \psline[linestyle=dotted,linecolor=red](3.571,0.2571)(0,0.2571)%
  \psline[linestyle=dotted,linecolor=red](4.6,0.36)(4.6,0)%
  \psline[linestyle=dotted,linecolor=red](4.6,0.36)(0,0.36)%
  \psline[linestyle=dotted,linecolor=red](7,1.8)(7,0)%
  \psline[linestyle=dotted,linecolor=red](7,1.8)(0,1.8)%
  %
  \uput[0]{0}(7.75,0){$x$}%
  \uput[-90]{0}(3.3571,0){$\frac{25}{7}$}%
  \uput[-90]{0}(4.6,0){$\frac{23}{5}$}%
  \uput[180]{0}(0,2.4){$2.4$}%
  \uput[180]{0}(0,1.8){$1.8$}%
  \uput[150]{0}(0,0.36){$\sfrac{9}{25}$}%
  \uput[195]{0}(0,0.2571){$\sfrac{9}{35}$}%
  \rput[t](3.5,2.75){$\ds\ff(x)\eqd\max_{y\in\R}\abs{x-y}\psp(y)$}%
\end{pspicture}}%
}
        && \text{\psset{unit=6mm}\gsize\includegraphics{../common/math/graphics/pdfs/wdie_max.pdf}}
      \\&= \setn{\frac{25}{7}} \approx \setn{3.5714}
        && \text{because $\ff(x)$ is minimized at argument $x=\frac{25}{7}$}
      \\
      \ocsVar(\rvX;\ocsE)
        &\eqd \sum_{x\in\R}\metricsq{\ocsE(\rvX)}{x}\psp(x)
        && \text{by definition of $\ocsVar$ \xref{def:ocsVar}}
      \\&= \sum_{x\in\R}\metricsq{\frac{25}{7}}{x}\psp(x)
        && \text{by $\ocsE(\rvX)$ result}
      \\&= \mathrlap{
           \brp{\frac{25}{7}-1}^2\frac{1}{10} +
           \brp{\frac{25}{7}-2}^2\frac{1}{20} +
           \brp{\frac{25}{7}-3}^2\frac{1}{30} +
           \brp{\frac{25}{7}-4}^2\frac{3}{5}  +
           \brp{\frac{25}{7}-5}^2\frac{1}{50} +
           \brp{\frac{25}{7}-6}^2\frac{1}{30} 
           }
      %\\&= \mathrlap{\frac{1}{49\times300}
      %      30\brp{25- 7}^2 +
      %      15\brp{25-14}^2 +
      %      10\brp{25-21}^2 +
      %     180\brp{25-28}^2 +
      %       6\brp{25-35}^2 +
      %      10\brp{25-42}^2 
      %\\&= \mathrlap{\frac{1}{49\times300}
      %      30\brp{18}^2 +
      %      15\brp{11}^2 +
      %      10\brp{ 4}^2 +
      %     180\brp{ 3}^2 +
      %       6\brp{10}^2 +
      %      10\brp{17}^2 +
      %     }
      \\&= \mathrlap{\frac{16805}{49*300} = \frac{3361}{49*60} = \frac{3361}{2940} \approx 1.143}
      \\
      \ocsE(\rvY)
        &= \rvY\brp{\ocscen(\ocsG)}
        && \text{because $\ocsG$ and $\ocsH$ are \prope{isomorphic}}
      \\&= \rvY\brp{\setn{\diceD}}
        &&\text{by \prefpp{ex:wdie}}
      \\&= \setn{4}
        &&\text{by definition of $\rvY$}
      %\\
      %\ocsEa(\rvY)
      %  &\eqd \argmin_{x\in\omsH}\max_{y\in\omsH}\ocsmom(x,y)
      %  &&\text{by definition of $\ocsEa$ \xref{def:ocsEa}}
      %\\&= \rvY\brs{\argmin_{x\in\ocsG}\max_{y\in\ocsG}\ocsmom(x,y)}
      %  &&\text{because $\ocsG$ and $\omsH$ are \prope{isomorphic}}
      %\\&= \rvY\brs{\ocscena(\ocsG)}
      %  &&\text{by definition of $\ocscena$ \xref{def:ocscena}}
      %\\&= \rvY\brs{\setn{\diceD}}
      %  &&\text{by \prefpp{item:wdie_xy_geo}}
      %\\&= \setn{4}
      %  &&\text{by definition of $\rvY$}
      \\
      \ocsVar(\rvY;\ocsE)
        &= \ocsVaro(\ocsG)
         = \frac{101}{300} \approx 0.337
        &&\text{by \prefpp{ex:wdie}}
    \end{align*}
\end{proof}

%=======================================
\subsubsection{Spinner examples}
%=======================================
\begin{figure}[h]
  \centering%
  \gsize%
  %\psset{unit=15mm}%
%  \gsize\psset{unit=5mm}{%============================================================================
% Daniel J. Greenhoe
% LaTeX file
% spinner 6 mapping to linearly ordered L6
%============================================================================
{%\psset{unit=0.5\psunit}%
\begin{pspicture}(0,-0.4)(8.5,5.5)%
  %---------------------------------
  % options
  %---------------------------------
  \psset{%
    linecolor=blue,%
    radius=1.25ex,
    labelsep=2.5mm,
    }%
  %---------------------------------
  % spinner graph
  %---------------------------------
  \rput(2.4,2.5){\psset{unit=2\psunit}%
    \Cnode[fillstyle=solid,fillcolor=snode](-0.8660,-0.5){D6}%
    \Cnode[fillstyle=solid,fillcolor=snode](-0.8660,0.5){D5}%
    \Cnode[fillstyle=solid,fillcolor=snode](0,1){D4}%
    \Cnode[fillstyle=solid,fillcolor=snode](0.8660,0.5){D3}%
    \Cnode[fillstyle=solid,fillcolor=snode](0.8660,-0.5){D2}%
    \Cnode[fillstyle=solid,fillcolor=snode](0,-1){D1}%
    }
  \rput[-150](D6){$\circSix$}%
  \rput[ 150](D5){$\circFive$}%
  \rput[  90](D4){$\circFour$}%
  \rput[  30](D3){$\circThree$}%
  \rput[   0](D2){$\circTwo$}%
  \rput[ -90](D1){$\circOne$}%
  %
  \ncline{D6}{D1}%
  \ncline{D5}{D6}%
  \ncline{D4}{D5}%
  \ncline{D3}{D4}%
  \ncline{D2}{D3}%
  \ncline{D1}{D2}%
  %
  \uput[ 158](D6){$\frac{1}{6}$}
  \uput[ 150](D5){$\frac{1}{6}$}
  \uput[ 210](D4){$\frac{1}{6}$}
  \uput[  22](D3){$\frac{1}{6}$}
  \uput[ -45](D2){$\frac{1}{6}$}
  \uput[-158](D1){$\frac{1}{6}$}
  %---------------------------------
  % L6 lattice
  %---------------------------------
  \rput(8,0){%
    \Cnode(0,5){L6}%
    \Cnode(0,4){L5}%
    \Cnode[fillstyle=solid,fillcolor=snode](0,3){L4}%
    \Cnode[fillstyle=solid,fillcolor=snode](0,2){L3}%
    \Cnode(0,1){L2}%
    \Cnode(0,0){L1}%
    }%
  \ncline{L5}{L6}%
  \ncline{L4}{L5}%
  \ncline{L3}{L4}%
  \ncline{L2}{L3}%
  \ncline{L1}{L2}%
  %
  \rput(L6){$6$}%
  \rput(L5){$5$}%
  \rput(L4){$4$}%
  \rput(L3){$3$}%
  \rput(L2){$2$}%
  \rput(L1){$1$}%
  %---------------------------------
  % mapping from die to L6
  %---------------------------------
  \ncarc[arcangle= 22,linewidth=0.75pt,linecolor=red]{->}{D6}{L6}%
  \ncarc[arcangle= 22,linewidth=0.75pt,linecolor=red]{->}{D5}{L5}%
  \ncarc[arcangle= 22,linewidth=0.75pt,linecolor=red]{->}{D4}{L4}%
  \ncarc[arcangle=-22,linewidth=0.75pt,linecolor=red]{->}{D3}{L3}%
  \ncarc[arcangle=-22,linewidth=0.75pt,linecolor=red]{->}{D2}{L2}%
  \ncarc[arcangle=-22,linewidth=0.75pt,linecolor=red]{->}{D1}{L1}%
  %\ncline[linewidth=0.75pt,linecolor=red]{->}{b}{lb}%
  %---------------------------------
  % labels
  %---------------------------------
  \rput(6.5,1.25){$\rvX(\cdot)$}%
  %\ncline[linestyle=dotted,nodesep=1pt]{->}{xzlabel}{xz}%
  %\ncline[linestyle=dotted,nodesep=1pt]{->}{ylabel}{y}%
\end{pspicture}
}%}%
%  \gsize\psset{unit=15mm}{%============================================================================
% Daniel J. Greenhoe
% LaTeX file
% discrete metric real dice mapping to linearly ordered O6c
%============================================================================
{%\psset{unit=0.5\psunit}%
\begin{pspicture}(-3.3,-1.45)(3.3,1.45)%
  %---------------------------------
  % options
  %---------------------------------
  \psset{%
    radius=1.25ex,
    labelsep=2.5mm,
    linecolor=blue,%
    }%
  %---------------------------------
  % dice graph
  %---------------------------------
  \rput(-1.75,0){%\psset{unit=2\psunit}%
    \Cnode[fillstyle=solid,fillcolor=snode](-0.8660,-0.5){D6}%
    \Cnode[fillstyle=solid,fillcolor=snode](-0.8660,0.5){D5}%
    \Cnode[fillstyle=solid,fillcolor=snode](0,1){D4}%
    \Cnode[fillstyle=solid,fillcolor=snode](0.8660,0.5){D3}%
    \Cnode[fillstyle=solid,fillcolor=snode](0.8660,-0.5){D2}%
    \Cnode[fillstyle=solid,fillcolor=snode](0,-1){D1}%
    }
  \rput(D6){\circSix}% 
  \rput(D5){\circFive}%
  \rput(D4){\circFour}%
  \rput(D3){\circThree}%
  \rput(D2){\circTwo}% 
  \rput(D1){\circOne}% 
  %
  \ncline{D6}{D1}\nbput[labelsep=0pt,nrot=:U]{${\scy\metric{\circSix}  {\circOne}=}1$}%
  \ncline{D6}{D5}\naput[labelsep=0pt,nrot=:U]{${\scy\metric{\circFive} {\circSix}=}1$}%
  \ncline{D5}{D4}\naput[labelsep=0pt,nrot=:U]{${\scy\metric{\circFour} {\circFive}=}1$}%
  \ncline{D4}{D3}\naput[labelsep=0pt,nrot=:U]{${\scy\metric{\circThree}{\circFour}=}1$}%
  \ncline{D2}{D3}\nbput[labelsep=0pt,nrot=:U]{${\scy\metric{\circTwo}  {\circThree}=}1$}%
  \ncline{D1}{D2}\nbput[labelsep=0pt,nrot=:U]{${\scy\metric{\circOne}  {\circTwo}=}1$}%
  %
  \uput[ 180](D6){${\scy\psp(\circSix)=}\frac{1}{6}$}
  \uput[ 150](D5){${\scy\psp(\circFive)=}\frac{1}{6}$}
  \uput[  90](D4){${\scy\psp(\circFour)=}\frac{1}{6}$}
  \uput[  22](D3){${\scy\psp(\circThree)=}\frac{1}{6}$}
  \uput[ -45](D2){${\scy\psp(\circTwo)=}\frac{1}{6}$}
  \uput[ -90](D1){${\scy\psp(\circOne)=}\frac{1}{6}$}
  %---------------------------------
  % range graph
  %---------------------------------
  \rput(1.75,0){%\psset{unit=2\psunit}%
    \Cnode(-0.8660,-0.5){E6}%
    \Cnode(-0.8660,0.5){E5}%
    \Cnode(0,1){E4}%
    \Cnode(0.8660,0.5){E3}%
    \Cnode(0.8660,-0.5){E2}%
    \Cnode(0,-1){E1}%
    \Cnode[fillstyle=solid,fillcolor=snode](0,0){E0}%
    }
  \rput(E6){$6$}%
  \rput(E5){$5$}%
  \rput(E4){$4$}%
  \rput(E3){$3$}%
  \rput(E2){$2$}%
  \rput(E1){$1$}%
  \rput(E0){$0$}%
  %
  \ncline{E6}{E1}\nbput[labelsep=0pt,nrot=:U]{${\scy\metric{6}{1}=}1$}%
  \ncline{E6}{E5}\naput[labelsep=0pt,nrot=:U]{${\scy\metric{5}{6}=}1$}%
  \ncline{E5}{E4}\naput[labelsep=0pt,nrot=:U]{${\scy\metric{4}{5}=}1$}%
  \ncline{E4}{E3}\naput[labelsep=0pt,nrot=:U]{${\scy\metric{3}{4}=}1$}%
  \ncline{E2}{E3}\nbput[labelsep=0pt,nrot=:U]{${\scy\metric{2}{3}=}1$}%
  \ncline{E1}{E2}\nbput[labelsep=0pt,nrot=:U]{${\scy\metric{1}{2}=}1$}%
  %
  \ncline{E0}{E1}\naput[labelsep=0pt,nrot=:U]{${\scy\metric{0}{1}=}\frac{3}{2}$}%
  \ncline{E0}{E2}\naput[labelsep=0pt,nrot=:U]{${\scy\metric{0}{2}=}\frac{3}{2}$}%
  \ncline{E0}{E3}\naput[labelsep=0pt,nrot=:U]{${\scy\metric{0}{3}=}\frac{3}{2}$}%
  \ncline{E0}{E4}\naput[labelsep=0pt,nrot=:U]{${\scy\metric{0}{4}=}\frac{3}{2}$}%
  \ncline{E5}{E0}\nbput[labelsep=0pt,nrot=:U]{${\scy\metric{0}{5}=}\frac{3}{2}$}%
  \ncline{E6}{E0}\nbput[labelsep=0pt,nrot=:U]{${\scy\metric{0}{6}=}\frac{3}{2}$}%
  %
  \uput[180](E6){${\scy\psp(6)=}\frac{1}{6}$}
  \uput[135](E5){${\scy\psp(5)=}\frac{1}{6}$}
  \uput[ 90](E4){${\scy\psp(4)=}\frac{1}{6}$}
  \uput[ 30](E3){${\scy\psp(3)=}\frac{1}{6}$}
  \uput[-30](E2){${\scy\psp(2)=}\frac{1}{6}$}
  \uput[-90](E1){${\scy\psp(1)=}\frac{1}{6}$}
  \uput[ 15](E0){${\scy\psp(0)=}\sfrac{0}{6}$}
  %---------------------------------
  % mapping from die to O6c
  %---------------------------------
  \ncarc[arcangle= 22,linewidth=0.75pt,linecolor=red]{->}{D6}{E6}%
  \ncarc[arcangle= 22,linewidth=0.75pt,linecolor=red]{->}{D5}{E5}%
  \ncarc[arcangle= 22,linewidth=0.75pt,linecolor=red]{->}{D4}{E4}%
  \ncarc[arcangle=-22,linewidth=0.75pt,linecolor=red]{->}{D3}{E3}%
  \ncarc[arcangle=-22,linewidth=0.75pt,linecolor=red]{->}{D2}{E2}%
  \ncarc[arcangle=-22,linewidth=0.75pt,linecolor=red]{->}{D1}{E1}%
  %---------------------------------
  % labels
  %---------------------------------
  \rput(0,0){$\rvX(\cdot)$}%
  %\ncline[linestyle=dotted,nodesep=1pt]{->}{xzlabel}{xz}%
  %\ncline[linestyle=dotted,nodesep=1pt]{->}{ylabel}{y}%
\end{pspicture}
}%}%
 %%============================================================================
% Daniel J. Greenhoe
% LaTeX file
% discrete metric real dice mapping to linearly ordered O6c
%============================================================================
{\psset{unit=1.5\psunit}%
\begin{pspicture}(-5.3,-2.5)(5.3,2.25)%
  %---------------------------------
  % options
  %---------------------------------
  \psset{%
    radius=1.25ex,
    labelsep=2.5mm,
    linecolor=blue,%
    }%
  %---------------------------------
  % spinner structure H
  %---------------------------------
  \rput(0,0){%\psset{unit=2\psunit}%
    \uput{1}[210](0,0){\Cnode[fillstyle=solid,fillcolor=snode](0,0){D6}}%
    \uput{1}[150](0,0){\Cnode[fillstyle=solid,fillcolor=snode](0,0){D5}}%
    \uput{1}[ 90](0,0){\Cnode[fillstyle=solid,fillcolor=snode](0,0){D4}}%
    \uput{1}[ 30](0,0){\Cnode[fillstyle=solid,fillcolor=snode](0,0){D3}}%
    \uput{1}[-30](0,0){\Cnode[fillstyle=solid,fillcolor=snode](0,0){D2}}%
    \uput{1}[-90](0,0){\Cnode[fillstyle=solid,fillcolor=snode](0,0){D1}}%
    \rput(0,0){$\ocsG$}%
    }
  \rput(D6){\circSix}% 
  \rput(D5){\circFive}%
  \rput(D4){\circFour}%
  \rput(D3){\circThree}%
  \rput(D2){\circTwo}% 
  \rput(D1){\circOne}% 
  %
  \ncline{D6}{D1}\nbput[labelsep=0pt,nrot=:U]{${\scy\metric{\circSix}  {\circOne}=}1$}%
  \ncline{D6}{D5}\naput[labelsep=0pt,nrot=:U]{${\scy\metric{\circFive} {\circSix}=}1$}%
  \ncline{D5}{D4}\naput[labelsep=0pt,nrot=:U]{${\scy\metric{\circFour} {\circFive}=}1$}%
  \ncline{D4}{D3}\naput[labelsep=0pt,nrot=:U]{${\scy\metric{\circThree}{\circFour}=}1$}%
  \ncline{D2}{D3}\nbput[labelsep=0pt,nrot=:U]{${\scy\metric{\circTwo}  {\circThree}=}1$}%
  \ncline{D1}{D2}\nbput[labelsep=0pt,nrot=:U]{${\scy\metric{\circOne}  {\circTwo}=}1$}%
  %
  \uput[ 180](D6){${\scy\psp(\circSix)=}\frac{1}{6}$}
  \uput[ 150](D5){${\scy\psp(\circFive)=}\frac{1}{6}$}
  \uput[  90](D4){${\scy\psp(\circFour)=}\frac{1}{6}$}
  \uput[  22](D3){${\scy\psp(\circThree)=}\frac{1}{6}$}
  \uput[ -45](D2){${\scy\psp(\circTwo)=}\frac{1}{6}$}
  \uput[ -90](D1){${\scy\psp(\circOne)=}\frac{1}{6}$}
  %---------------------------------
  % W mapping to real line
  %---------------------------------
  \rput(0,-2){
    \psline{<->}(-3,0)(3,0)%
    \multirput(-2.5,0)(1,0){6}{\psline(0,-0.1)(0,0.1)}%
    \pnode( 3,0){RB}%
    \pnode( 2.5,0){R6}%
    \pnode( 1.5,0){R5}%
    \pnode( 0.5,0){R4}%
    \pnode(0,0){R34}%
    \pnode(-0.5,0){R3}%
    \pnode(-1.5,0){R2}%
    \pnode(-2.5,0){R1}%
    \pnode(-3,0){RA}%
    }
  %
  \pscircle[fillstyle=solid,linecolor=snode,fillcolor=snode](R34){1ex}%
  \pscircle[fillstyle=none,linecolor=red,fillcolor=red](R34){1ex}%
  %
  \uput[-90](R6){$6$}%
  \uput[-90](R5){$5$}%
  \uput[-90](R4){$4$}%
  \uput[-90](R3){$3$}%
  \uput[-90](R2){$2$}%
  \uput[-90](R1){$1$}%
  \uput[0](RB){$\omsR$}%
  %
  \ncarc[arcangle= 22,linewidth=0.75pt,linecolor=purple]{->}{D6}{R6}%
  \ncarc[arcangle= 22,linewidth=0.75pt,linecolor=purple]{->}{D5}{R5}%
  \ncarc[arcangle= 22,linewidth=0.75pt,linecolor=purple]{->}{D4}{R4}%
  \ncarc[arcangle= 22,linewidth=0.75pt,linecolor=purple]{->}{D3}{R3}%
  \ncarc[arcangle= 22,linewidth=0.75pt,linecolor=purple]{->}{D2}{R2}%
  \ncarc[arcangle=-22,linewidth=0.75pt,linecolor=purple]{->}{D1}{R1}%
  %---------------------------------
  % X mapping to integer line
  %---------------------------------
  \rput(0,2){%\psset{unit=2\psunit}%
    \pnode(3,0){XB}%
    \Cnode(2.5,0){X6}%
    \Cnode(1.5,0){X5}%
    \Cnode[fillstyle=solid,fillcolor=snode](0.5,0){X4}%
    \Cnode[fillstyle=solid,fillcolor=snode](-0.5,0){X3}%
    \Cnode(-1.5,0){X2}%
    \Cnode(-2.5,0){X1}%
    \pnode(-3,0){XA}%
    }%
  \ncline[linestyle=dotted]{X6}{XB}%
  \ncline{X5}{X6}%
  \ncline{X4}{X5}%
  \ncline{X3}{X4}%
  \ncline{X2}{X3}%
  \ncline{X1}{X2}%
  \ncline[linestyle=dotted]{X1}{XA}%
  %
  \rput(X6){$6$}%
  \rput(X5){$5$}%
  \rput(X4){$4$}%
  \rput(X3){$3$}%
  \rput(X2){$2$}%
  \rput(X1){$1$}%
  %
  \ncarc[arcangle=-22,linewidth=0.75pt,linecolor=red]{->}{D6}{X6}%
  \ncarc[arcangle= 22,linewidth=0.75pt,linecolor=red]{->}{D5}{X5}%
  \ncarc[arcangle=-22,linewidth=0.75pt,linecolor=red]{->}{D4}{X4}%
  \ncarc[arcangle=-22,linewidth=0.75pt,linecolor=red]{->}{D3}{X3}%
  \ncarc[arcangle= 22,linewidth=0.75pt,linecolor=red]{->}{D2}{X2}%
  \ncarc[arcangle= 22,linewidth=0.75pt,linecolor=red]{->}{D1}{X1}%
  %
  %\uput[90](X6){${\scy\psp(6)=}\frac{1}{6}$}
  %\uput[90](X5){${\scy\psp(5)=}\frac{1}{6}$}
  %\uput[90](X4){${\scy\psp(4)=}\frac{1}{6}$}
  %\uput[90](X3){${\scy\psp(3)=}\frac{1}{6}$}
  %\uput[90](X2){${\scy\psp(2)=}\frac{1}{6}$}
  %\uput[90](X1){${\scy\psp(1)=}\frac{1}{6}$}
  \uput[0](XB){$\omsZ$}%
  %---------------------------------
  % Y mapping to isomorphic structure H
  %---------------------------------
  \rput(-3.5,0){%\psset{unit=2\psunit}%
    \uput{1}[210](0,0){\Cnode[fillstyle=solid,fillcolor=snode](0,0){Y6}}%
    \uput{1}[150](0,0){\Cnode[fillstyle=solid,fillcolor=snode](0,0){Y5}}%
    \uput{1}[ 90](0,0){\Cnode[fillstyle=solid,fillcolor=snode](0,0){Y4}}%
    \uput{1}[ 30](0,0){\Cnode[fillstyle=solid,fillcolor=snode](0,0){Y3}}%
    \uput{1}[-30](0,0){\Cnode[fillstyle=solid,fillcolor=snode](0,0){Y2}}%
    \uput{1}[-90](0,0){\Cnode[fillstyle=solid,fillcolor=snode](0,0){Y1}}%
    \rput(0,0){$\ocsH$}%
    }%
  \rput(Y6){$6$}%
  \rput(Y5){$5$}%
  \rput(Y4){$4$}%
  \rput(Y3){$3$}%
  \rput(Y2){$2$}%
  \rput(Y1){$1$}%
  %
  \ncline{Y6}{Y1}\nbput[labelsep=0pt,nrot=:U]{${\scy\metric{6}{1}=}1$}%
  \ncline{Y6}{Y5}\naput[labelsep=0pt,nrot=:U]{${\scy\metric{5}{6}=}1$}%
  \ncline{Y5}{Y4}\naput[labelsep=0pt,nrot=:U]{${\scy\metric{4}{5}=}1$}%
  \ncline{Y4}{Y3}\naput[labelsep=0pt,nrot=:U]{${\scy\metric{3}{4}=}1$}%
  \ncline{Y2}{Y3}\nbput[labelsep=0pt,nrot=:U]{${\scy\metric{2}{3}=}1$}%
  \ncline{Y1}{Y2}\nbput[labelsep=0pt,nrot=:U]{${\scy\metric{1}{2}=}1$}%
  %
  \uput[180](Y6){${\scy\psp(6)=}\frac{1}{6}$}
  \uput[135](Y5){${\scy\psp(5)=}\frac{1}{6}$}
  \uput[ 90](Y4){${\scy\psp(4)=}\frac{1}{6}$}
  \uput[ 30](Y3){${\scy\psp(3)=}\frac{1}{6}$}
  \uput[-30](Y2){${\scy\psp(2)=}\frac{1}{6}$}
  \uput[-90](Y1){${\scy\psp(1)=}\frac{1}{6}$}
  %
  \ncarc[arcangle= 22,linewidth=0.75pt,linecolor=blue]{->}{D6}{Y6}%
  \ncarc[arcangle= 22,linewidth=0.75pt,linecolor=blue]{->}{D5}{Y5}%
  \ncarc[arcangle=-22,linewidth=0.75pt,linecolor=blue]{->}{D4}{Y4}%
  \ncarc[arcangle=-22,linewidth=0.75pt,linecolor=blue]{->}{D3}{Y3}%
  \ncarc[arcangle=-22,linewidth=0.75pt,linecolor=blue]{->}{D2}{Y2}%
  \ncarc[arcangle= 22,linewidth=0.75pt,linecolor=blue]{->}{D1}{Y1}%
  %---------------------------------
  % Z mapping to extended structure H
  %---------------------------------
  \rput(3.5,0){%\psset{unit=2\psunit}%
    \uput{1}[210](0,0){\Cnode(0,0){Z6}}%
    \uput{1}[150](0,0){\Cnode(0,0){Z5}}%
    \uput{1}[ 90](0,0){\Cnode(0,0){Z4}}%
    \uput{1}[ 30](0,0){\Cnode(0,0){Z3}}%
    \uput{1}[-30](0,0){\Cnode(0,0){Z2}}%
    \uput{1}[-90](0,0){\Cnode(0,0){Z1}}%
    \Cnode[fillstyle=solid,fillcolor=snode](0,0){Z0}%
    \uput{1.25}[60](0,0){$\ocsK$}%
    }
  \rput(Z6){$6$}%
  \rput(Z5){$5$}%
  \rput(Z4){$4$}%
  \rput(Z3){$3$}%
  \rput(Z2){$2$}%
  \rput(Z1){$1$}%
  \rput(Z0){$0$}%
  %
  \ncline{Z6}{Z1}\nbput[labelsep=0pt,nrot=:U]{${\scy\metric{6}{1}=}1$}%
  \ncline{Z6}{Z5}\naput[labelsep=0pt,nrot=:U]{${\scy\metric{5}{6}=}1$}%
  \ncline{Z5}{Z4}\naput[labelsep=0pt,nrot=:U]{${\scy\metric{4}{5}=}1$}%
  \ncline{Z4}{Z3}\naput[labelsep=0pt,nrot=:U]{${\scy\metric{3}{4}=}1$}%
  \ncline{Z2}{Z3}\nbput[labelsep=0pt,nrot=:U]{${\scy\metric{2}{3}=}1$}%
  \ncline{Z1}{Z2}\nbput[labelsep=0pt,nrot=:U]{${\scy\metric{1}{2}=}1$}%
  %
  \ncline{Z0}{Z1}\naput[labelsep=0pt,nrot=:U]{${\scy\metric{0}{1}=}\frac{3}{2}$}%
  \ncline{Z0}{Z2}\naput[labelsep=0pt,nrot=:U]{${\scy\metric{0}{2}=}\frac{3}{2}$}%
  \ncline{Z0}{Z3}\naput[labelsep=0pt,nrot=:U]{${\scy\metric{0}{3}=}\frac{3}{2}$}%
  \ncline{Z0}{Z4}\naput[labelsep=0pt,nrot=:U]{${\scy\metric{0}{4}=}\frac{3}{2}$}%
  \ncline{Z5}{Z0}\nbput[labelsep=0pt,nrot=:U]{${\scy\metric{0}{5}=}\frac{3}{2}$}%
  \ncline{Z6}{Z0}\nbput[labelsep=0pt,nrot=:U]{${\scy\metric{0}{6}=}\frac{3}{2}$}%
  %
  \uput[180](Z6){${\scy\psp(6)=}\frac{1}{6}$}
  \uput[135](Z5){${\scy\psp(5)=}\frac{1}{6}$}
  \uput[ 90](Z4){${\scy\psp(4)=}\frac{1}{6}$}
  \uput[ 30](Z3){${\scy\psp(3)=}\frac{1}{6}$}
  \uput[-30](Z2){${\scy\psp(2)=}\frac{1}{6}$}
  \uput[-90](Z1){${\scy\psp(1)=}\frac{1}{6}$}
  \uput[ 15](Z0){${\scy\psp(0)=}\sfrac{0}{6}$}
  %
  \ncarc[arcangle= 22,linewidth=0.75pt,linecolor=green]{->}{D6}{Z6}%
  \ncarc[arcangle= 22,linewidth=0.75pt,linecolor=green]{->}{D5}{Z5}%
  \ncarc[arcangle= 22,linewidth=0.75pt,linecolor=green]{->}{D4}{Z4}%
  \ncarc[arcangle=-22,linewidth=0.75pt,linecolor=green]{->}{D3}{Z3}%
  \ncarc[arcangle=-22,linewidth=0.75pt,linecolor=green]{->}{D2}{Z2}%
  \ncarc[arcangle=-22,linewidth=0.75pt,linecolor=green]{->}{D1}{Z1}%
  %---------------------------------
  % labels
  %---------------------------------
  \rput(0,-1.5){$\rvW(\cdot)$}%
  \rput(0, 1.5){$\rvX(\cdot)$}%
  \rput(-2.25,0){$\rvY(\cdot)$}%
  \rput(1.75,0){$\rvZ(\cdot)$}%
  %\ncline[linestyle=dotted,nodesep=1pt]{->}{xzlabel}{xz}%
  %\ncline[linestyle=dotted,nodesep=1pt]{->}{ylabel}{y}%
\end{pspicture}%
}%%
  \includegraphics{../common/math/graphics/pdfs/ocsrv_spinner_xy.pdf}%
  \caption{Six value fair spinner with assorted random variable mappings \xref{ex:spinner_xy} \label{fig:spinner_xy}}
\end{figure}
%---------------------------------------
%\begin{minipage}{\tw-48mm}%
\begin{example}[\exmd{spinner mappings}]
\label{ex:spinner_xy} %\mbox{}\\
%\label{ex:spinnerXO6c}%\mbox{}\\
%---------------------------------------
A six value board game spinner has a cyclic structure as illustrated in \prefpp{fig:spinner_xy}.
Again, the order and metric geometry of the real line mapped to by the random variable $\rvX$
is very dissimilar to that of the \structe{outcome subspace} that it is supposed to represent.
Therefore, statistical inferences based on $\rvX$ will likely result in values that are
arguably unacceptable.
Both random variables $\rvY$ and $\rvZ$ map to structures in which order and metric geometry are preserved.
The mappings yield the following statistics:
\\$\begin{array}{>{\gsizes}Mrcl rcccl}
  geometry of $\ocsG$:                                 & \ocscen (\ocsG) &=& \mc{4}{l}{\setn{\circOne,\circTwo,\circThree,\circFour,\circFive,\circSix}} \\
  traditional statistics on real line:                 & \pE   (\rvW)    &=& 3.5                & \ocsVar(\rvW;\pE)   &=& \frac{35}{12}&\approx& 2.917 \\
  outcome subspace statistics on real line:            & \ocsE (\rvW)    &=& \setn{3.5}         & \ocsVar(\rvW;\ocsE) &=& \frac{35}{12}&\approx& 2.917 \\
  outcome subspace statistics on integer line:         & \ocsE (\rvX)    &=& \setn{3,\,4}       & \ocsVar(\rvX;\ocsE) &=& \frac{20}{12}&\approx& 1.667 \\
  outcome subspace stats. on  isomorphic structure:& \ocsE (\rvY)    &=& \setn{1,2,3,4,5,6} & \ocsVar(\rvY;\ocsE) &=& 0            &       &       \\
  outcome subspace stats. on  extended structure:  & \ocsE (\rvZ)    &=& \setn{0}           & \ocsVar(\rvZ;\ocsE) &=& \frac{9}{4}  &=&       2.25  
\end{array}$
\end{example}
\begin{proof}
\begin{align*}
  \ocscen(\ocsG)
    &= \setn{\circOne,\circTwo,\circThree,\circFour,\circFive,\circSix}
    && \text{by \prefpp{ex:spinner}}
    \\
  \pE(\rvW)
      &\eqd \sum_{x\in\R} x\psp(x)
      && \text{by definition of $\pE$ \xref{def:pE}}
    \\&= \frac{7}{2} = 3.5
      && \text{by \exme{fair die} example \prefpp{ex:fairdieXRYZ}}
    \\
  \ocsVar(\rvW;\pE)
      &= \pVar(\rvX)
      && \text{by \prefp{thm:ocsVar}}
    \\&\eqd \sum_{x\in\R} \brs{x-\pE(\rvX)}^2\psp(x)\dx
      && \text{by definition of $\pVar$ \xref{def:pVar}}
    \\&= \frac{35}{12}\approx2.917
      && \text{by \exme{fair die} example \prefpp{ex:fairdieXRYZ}}
    \\
  \ocsE(\rvW)
      &= \pE(\rvW)
      && \text{because on \structe{real line}, $\psp$ is \prope{symmetric}, and by \prefp{thm:pEocsE}}
    \\&= \setn{3.5}
      && \text{by $\pE(\rvW)$ result}
    \\
  \ocsVar(\rvW;\ocsE)
      &= \ocsVar(\rvW;\pE)
      && \text{because $\ocsE(\rvW)=\pE(\rvW)$}
    \\&= \frac{35}{12} \approx 2.917
      && \text{by $\ocsVar(\rvW;\pE)$ result}
    \\
  \ocsE(\rvX)
      &\eqd \argmin_{x\in\Z}\max_{y\in\Z}\metric{x}{y}\psp(y)
      && \text{by definition of $\ocsE$ \xref{def:ocsE}}
    \\&\eqd \argmin_{x\in\Z}\max_{y\in\Z}\abs{x-y}\frac{1}{6}
      && \text{by definition of \structe{integer line} \xref{def:Zline} and $\ocsG$}
    \\&= \setn{3,\,4}
      && \text{by \exme{fair die} example \prefpp{ex:fairdieXRYZ}}
    \\
  \ocsVar(\rvX;\ocsE)
      &= \frac{5}{3} \approx 1.667
      && \text{by \exme{fair die} example \prefpp{ex:fairdieXRYZ}}
    \\
  \ocsE(\rvY)
      &= \rvY\brs{\ocscen(\ocsG)}
      && \text{because $\ocsG$ and $\omsH$ are \prope{isomorphic} under mapping $\rvY$}
    \\&= \rvY\brs{\setn{\circOne,\circTwo,\circThree,\circFour,\circFive,\circSix}}
      && \text{by $\ocscen(\ocsG)$ result}
    \\&= \setn{1,2,3,4,5,6}
      && \text{by definition of $\rvY$}
    \\
  \ocsVar(\rvY;\ocsE)
      &= \ocsVaro(\ocsG) = 0
      && \text{by \exme{spinner outcome subspace} example \xref{ex:spinner}}
    \\
  \ocsE(\rvZ)
    &\eqd \argmin_{x\in\omsH}\max_{y\in\omsH}\metric{x}{y}\psp(y)
    &&\text{by definition of $\ocsE$ \xref{def:ocsE}}
  \\&= \argmin_{x\in\omsH}\max_{y\in\omsH\setd\setn{0}}\metric{x}{y}\psp(y)
    &&\text{because $\psp(0)=0$}
  \\&= \argmin_{x\in\omsH}\max_{y\in\omsH\setd\setn{0}}\metric{x}{y}\frac{1}{6}
    &&\text{by definition of $\ocsG$}
  \\&= \argmin_{x\in\omsH}\max_{y\in\omsH\setd\setn{0}}\metric{x}{y}
    && \text{because $\ff(x)=\frac{1}{6}x$ is \prope{strictly isotone} and by \prefpp{lem:argminmaxphi}}
  \\&=\mathrlap{\argmin_{x\in\omsH}\max_{y\in\omsH}
         \setn{\begin{array}{c@{\,}c@{\,}c}
           \metricn(1,1) & \cdots &\metricn(1,6) \\
           \metricn(2,1) & \cdots &\metricn(2,6) \\
           \metricn(3,1) & \cdots &\metricn(3,6) \\
           \metricn(4,1) & \cdots &\metricn(4,6) \\
           \metricn(5,1) & \cdots &\metricn(5,6) \\
           \metricn(6,1) & \cdots &\metricn(6,6) \\
           \metricn(0,1) & \cdots &\metricn(0,6)
         \end{array}} 
     =\argmin_{x\in\omsH}\max_{y\in\omsH\setd\setn{0}}
         \setn{\begin{array}{*{6}{c@{\,\,}}}
           {0}&{1}&{2}&{3}&{2}&{1}\\
           {1}&{0}&{1}&{2}&{3}&{2}\\
           {2}&{1}&{0}&{1}&{2}&{3}\\
           {3}&{2}&{1}&{0}&{1}&{2}\\
           {2}&{3}&{2}&{1}&{0}&{1}\\
           {1}&{2}&{3}&{2}&{1}&{0}\\
           \frac{3}{2}&\frac{3}{2}&\frac{3}{2}&\frac{3}{2}&\frac{3}{2}&\frac{3}{2}
         \end{array}}
     = \argmin_{x\in\omsH}
         \setn{\begin{array}{c}
           {3}\\
           {3}\\
           {3}\\
           {3}\\
           {3}\\
           {3}\\
           {\frac{3}{2}}
         \end{array}}
     =   \setn{\begin{array}{c}
           \mbox{ }\\
           \mbox{ }\\
           \mbox{ }\\
           \mbox{ }\\
           \mbox{ }\\
           \mbox{ }\\
           0
         \end{array}}}
  %\\
  %\ocsEa(\rvX)
  %  &\eqd \argmin_{x\in\omsH}\sum_{y}\ocsmom(x,y)
  %  &&\text{by definition of $\pE$ \xref{def:ocsE}}
  %\\&=\mathrlap{\argmin_{x\in\omsH}
  %       \setn{\begin{array}{*{13}{@{\,}c}}
  %         {0} &+& {1} &+& {2} &+& {3} &+& {2} &+& {1} &+& 0\\
  %         {1} &+& {0} &+& {1} &+& {2} &+& {3} &+& {2} &+& 0\\
  %         {2} &+& {1} &+& {0} &+& {1} &+& {2} &+& {3} &+& 0\\
  %         {3} &+& {2} &+& {1} &+& {0} &+& {1} &+& {2} &+& 0\\
  %         {2} &+& {3} &+& {2} &+& {1} &+& {0} &+& {1} &+& 0\\
  %         {1} &+& {2} &+& {3} &+& {2} &+& {1} &+& {0} &+& 0\\
  %         \frac{3}{2} &+& \frac{3}{2} &+& \frac{3}{2} &+& \frac{3}{2} &+& \frac{3}{2} &+& \frac{3}{2} &+& 0
  %       \end{array}}
  %\quad= \argmin_{x\in\omsH}
  %       \setn{\begin{array}{c}
  %         9\\
  %         9\\
  %         9\\
  %         9\\
  %         9\\
  %         9\\
  %         3
  %       \end{array}}}
  %\\&= \setn{0}
  \\
  \ocsVar(\rvZ)
    &\eqd \sum_{x\in\omsH}\metricsq{\ocscen(\ocsG)}{x}\psp(x)
    && \text{by definition of $\ocsVar$ \xref{def:ocsVar}}
  \\&= \sum_{x\in\omsH}\metricsq{\setn{0}}{x}\psp(x)
    && \text{by $\ocsE(\rvX)$ result}
  \\&= \mathrlap{
       \sum_{x\in\omsH\setd\setn{0}}\brp{\frac{3}{2}}^2\frac{1}{6}
     = \seto{\omsH\setd\setn{0}}\brp{\frac{3}{2}}^2\frac{1}{6}
     = 6\brp{\frac{3}{2}}^2\frac{1}{6}
     = \frac{9}{4}
     }
\end{align*}
\end{proof}

\begin{figure}[h]%
  \gsize%
  \centering%
  %{%============================================================================
% Daniel J. Greenhoe
% LaTeX file
% spinner non-linear mappings
%============================================================================
\begin{pspicture}(-4.75,-2.35)(4.75,1.35)%
  %---------------------------------
  % options
  %---------------------------------
  \psset{%
    radius=1.25ex,
    %labelsep=2.5mm,
    linecolor=blue,%
    }%
  %---------------------------------
  % spinner graph
  %---------------------------------
  \rput(0,0){%\psset{unit=2\psunit}%
    \rput{ 210}(0,0){\rput(1,0){\Cnode[fillstyle=solid,fillcolor=snode](0,0){D6}}}%
    \rput{ 150}(0,0){\rput(1,0){\Cnode(0,0){D5}}}%
    \rput{  90}(0,0){\rput(1,0){\Cnode(0,0){D4}}}%
    \rput{  30}(0,0){\rput(1,0){\Cnode(0,0){D3}}}%
    \rput{ -30}(0,0){\rput(1,0){\Cnode(0,0){D2}}}%
    \rput{ -90}(0,0){\rput(1,0){\Cnode[fillstyle=solid,fillcolor=snode](0,0){D1}}}%
    \rput(0,0){$\ocsG$}%
    }
  \rput(D6){\circSix}% 
  \rput(D5){\circFive}%
  \rput(D4){\circFour}%
  \rput(D3){\circThree}%
  \rput(D2){\circTwo}% 
  \rput(D1){\circOne}% 
  %
  \ncline{D6}{D1}%
  \ncline{D5}{D6}%
  \ncline{D4}{D5}%
  \ncline{D3}{D4}%
  \ncline{D2}{D3}%
  \ncline{D1}{D2}%
  %
  \uput[-150](D6){$\frac{3}{10}$}
  \uput[ 150](D5){$\frac{1}{10}$}
  \uput[ 158](D4){$\frac{1}{10}$}
  \uput[  22](D3){$\frac{1}{10}$}
  \uput[ -45](D2){$\frac{1}{10}$}
  \uput[ -45](D1){$\frac{3}{10}$}
  %---------------------------------
  % Y mapping to isomorphic structure H
  %---------------------------------
  \rput(-3.5,0){%\psset{unit=2\psunit}%
    \rput{ 210}(0,0){\rput(1,0){\Cnode[fillstyle=solid,fillcolor=snode](0,0){X6}}}%
    \rput{ 150}(0,0){\rput(1,0){\Cnode(0,0){X5}}}%
    \rput{  90}(0,0){\rput(1,0){\Cnode(0,0){X4}}}%
    \rput{  30}(0,0){\rput(1,0){\Cnode(0,0){X3}}}%
    \rput{ -30}(0,0){\rput(1,0){\Cnode(0,0){X2}}}%
    \rput{ -90}(0,0){\rput(1,0){\Cnode[fillstyle=solid,fillcolor=snode](0,0){X1}}}%
    \rput(0,0){$\omsH$}%
    }
  \rput(X6){$6$}%
  \rput(X5){$5$}%
  \rput(X4){$4$}%
  \rput(X3){$3$}%
  \rput(X2){$2$}%
  \rput(X1){$1$}%
  %
  \ncline{X6}{X1}%
  \ncline{X5}{X6}%
  \ncline{X4}{X5}%
  \ncline{X3}{X4}%
  \ncline{X2}{X3}%
  \ncline{X1}{X2}%
  %
  %\uput[ 158](X6){$\frac{3}{10}$}
  %\uput[ 150](X5){$\frac{1}{10}$}
  %\uput[ 210](X4){$\frac{1}{10}$}
  %\uput[ -22](X3){$\frac{1}{10}$}
  %\uput[ -22](X2){$\frac{1}{10}$}
  %\uput[-158](X1){$\frac{3}{10}$}
  %
  \ncarc[arcangle=-22,linewidth=0.75pt,linecolor=red]{->}{D6}{X6}%
  \ncarc[arcangle= 22,linewidth=0.75pt,linecolor=red]{->}{D5}{X5}%
  \ncarc[arcangle= 22,linewidth=0.75pt,linecolor=red]{->}{D4}{X4}%
  \ncarc[arcangle=-22,linewidth=0.75pt,linecolor=red]{->}{D3}{X3}%
  \ncarc[arcangle=-22,linewidth=0.75pt,linecolor=red]{->}{D2}{X2}%
  \ncarc[arcangle= 22,linewidth=0.75pt,linecolor=red]{->}{D1}{X1}%
  %---------------------------------
  % Z mapping to continuous ring K
  %---------------------------------
  \rput(3.5,0){%\psset{unit=2\psunit}%
    \pscircle(0,0){1}%
    \rput{ 210}(0,0){\rput(1,0){\pnode(0,0){Y6}}}%
    \rput{ 150}(0,0){\rput(1,0){\pnode(0,0){Y5}}}%
    \rput{  90}(0,0){\rput(1,0){\pnode(0,0){Y4}}}%
    \rput{  30}(0,0){\rput(1,0){\pnode(0,0){Y3}}}%
    \rput{ -30}(0,0){\rput(1,0){\pnode(0,0){Y2}}}%
    \rput{ -90}(0,0){\rput(1,0){\pnode(0,0){Y1}}}%
    \rput{-120}(0,0){\rput(1,0){\pnode(0,0){Y16}}}%
    \rput(0,0){$\omsK$}%
    }
  \rput{ 210}(Y6){\psline(-0.1,0)(0.1,0)}%
  \rput{ 150}(Y5){\psline(-0.1,0)(0.1,0)}%
  \rput{  90}(Y4){\psline(-0.1,0)(0.1,0)}%
  \rput{  30}(Y3){\psline(-0.1,0)(0.1,0)}%
  \rput{ -30}(Y2){\psline(-0.1,0)(0.1,0)}%
  \rput{ -90}(Y1){\psline(-0.1,0)(0.1,0)}%
  %
  \uput[ 210](Y6){$6$}%
  \uput[ 150](Y5){$5$}%
  \uput[  90](Y4){$4$}%
  \uput[  30](Y3){$3$}%
  \uput[ -30](Y2){$2$}%
  \uput[ -90](Y1){$1$}%
  %
  \rput(Y16){\pscircle[fillstyle=solid,linecolor=snode,fillcolor=snode](0,0){1ex}}%
  %
  %\ncline{Y5}{Y6}%
  %\ncline{Y4}{Y5}
  %\ncline{Y3}{Y5}
  %\ncline{Y2}{Y3}
  %\ncline{Y1}{Y2}
  %\ncline{Y0}{Y1}
  %
  %\uput[  22](Y6){$\frac{3}{10}$}%
  %\uput[ 200](Y5){$\frac{1}{10}$}%
  %\uput[ 210](Y4){$\frac{1}{10}$}%
  %\uput[  22](Y3){$\frac{1}{10}$}%
  %\uput[ -45](Y2){$\frac{1}{10}$}%
  %\uput[ -22](Y1){$\frac{3}{10}$}%
  %
  \ncarc[arcangle= 22,linewidth=0.75pt,linecolor=green]{->}{D6}{Y6}%
  \ncarc[arcangle= 22,linewidth=0.75pt,linecolor=green]{->}{D5}{Y5}%
  \ncarc[arcangle= 22,linewidth=0.75pt,linecolor=green]{->}{D4}{Y4}%
  \ncarc[arcangle=-22,linewidth=0.75pt,linecolor=green]{->}{D3}{Y3}%
  \ncarc[arcangle=-22,linewidth=0.75pt,linecolor=green]{->}{D2}{Y2}%
  \ncarc[arcangle=-22,linewidth=0.75pt,linecolor=green]{->}{D1}{Y1}%
  %---------------------------------
  % real line
  %---------------------------------
  \rput(-3.5,-2){
    \psline{<->}(0.5,0)(6.5,0)%
    \multirput(1,0)(1,0){6}{\psline(0,-0.1)(0,0.1)}%
    \pnode(6,0){L6}%
    \pnode(5,0){L5}%
    \pnode(4,0){L4}%
    \pnode(3,0){L3}%
    \pnode(2,0){L2}%
    \pnode(1,0){L1}%
    \pscircle[fillstyle=solid,linecolor=snode,fillcolor=snode](3.5,0){1ex}%
    \pscircle[fillstyle=none,linecolor=red,fillcolor=red](3.5,0){1ex}%
    \uput[0](6.5,0){$\omsR$}%
    }
  %
  \uput[-90](L6){$6$}%
  \uput[-90](L5){$5$}%
  \uput[-90](L4){$4$}%
  \uput[-90](L3){$3$}%
  \uput[-90](L2){$2$}%
  \uput[-90](L1){$1$}%
  %
  \ncarc[arcangle= 22,linewidth=0.75pt,linecolor=purple]{->}{D6}{L6}%
  \ncarc[arcangle= 22,linewidth=0.75pt,linecolor=purple]{->}{D5}{L5}%
  \ncarc[arcangle= 22,linewidth=0.75pt,linecolor=purple]{->}{D4}{L4}%
  \ncarc[arcangle=-22,linewidth=0.75pt,linecolor=purple]{->}{D3}{L3}%
  \ncarc[arcangle=-22,linewidth=0.75pt,linecolor=purple]{->}{D2}{L2}%
  \ncarc[arcangle=-22,linewidth=0.75pt,linecolor=purple]{->}{D1}{L1}%
  %---------------------------------
  % labels
  %---------------------------------
  \rput(-0.8,-1.7){$\rvX(\cdot)$}%
  \rput(-1.75,0){$\rvY(\cdot)$}%
  \rput(1.75,0){$\rvZ(\cdot)$}%
  %\ncline[linestyle=dotted,nodesep=1pt]{->}{xzlabel}{xz}%
  %\ncline[linestyle=dotted,nodesep=1pt]{->}{ylabel}{y}%
\end{pspicture}%}%
  {\includegraphics{../common/math/graphics/pdfs/spinnerXO6Ycircle.pdf}}%
  %\hfill%
  %{%============================================================================
% Daniel J. Greenhoe
% XeLaTeX file
%============================================================================
{\psset{yunit=2\psunit}%
\begin{pspicture}(-0.7,-0.25)(7.25,1)%
  \psset{%
    labelsep=3pt,
    linewidth=1pt,
    }%
  \psaxes[linecolor=axis,yAxis=false]{->}(0,0)(0,0)(7,1)% x axis
  \psaxes[linecolor=axis,xAxis=false]{->}(0,0)(0,0)(7,1)% y axis
  \psline(0,0.3)(0.5,0.25)(1,0.3)(3,0.9)(3.5,0.75)(4,0.9)(6,0.3)(6.5,0.25)(7,0.3)%
  %
  \psline[linestyle=dotted,linecolor=red](0.5,0.25)(0.5,0)%
  \psline[linestyle=dotted,linecolor=red](3,0.9)(3,0)%
  \psline[linestyle=dotted,linecolor=red](3.5,0.75)(3.5,0)%
  \psline[linestyle=dotted,linecolor=red](4,0.9)(4,0)%
  \psline[linestyle=dotted,linecolor=red](6.5,0.25)(6.5,0)%
  %
  \psline[linestyle=dotted,linecolor=red](0,0.9)(4,0.9)%
  \psline[linestyle=dotted,linecolor=red](0,0.75)(3.5,0.75)%
  \psline[linestyle=dotted,linecolor=red](0,0.3)(7,0.3)%
  \psline[linestyle=dotted,linecolor=red](0,0.25)(6.5,0.25)%
  \uput[0]{0}(7,0){$x$}%
  \uput[-90]{0}(0.5,0){$0.5$}%
  \uput[-90]{0}(3.5,0){$3.5$}%
  \uput[-90]{0}(6.5,0){$0.5$}%
  %
  \uput[170]{0}(0,0.9){$0.9$}%
  \uput[190]{0}(0,0.75){$0.75$}%
  \uput[155]{0}(0,0.3){$0.3$}%
  \uput[205]{0}(0,0.25){$0.25$}%
  \rput[b](3.5,0.3){$\ds\ff(x)\eqd\max_{y\in\omsK}\metric{x}{y}\psp(y)$}%
\end{pspicture}}%
}%
  \caption{weighted spinner mappings \xref{ex:wspinner_xyz}\label{fig:spinnerXO6Ycircle}}%
\end{figure}
%---------------------------------------
%\begin{minipage}{\tw-65mm}%
\begin{example}[\exmd{weighted spinner mappings}]
\label{ex:wspinner_xyz}
%---------------------------------------
Let $\ocsG$ be \structe{weighted spinner outcome subspace} \xref{ex:wspinner}
with random variable mappings as illustrated in \prefpp{fig:spinnerXO6Ycircle}.
This yields the following statistics:
\\\indentx$\begin{array}{>{\gsizes}Mrcl lcccl}
  geometry of $\ocsG$:                                          & \ocscen (\ocsG) &=& \mc{4}{l}{\setn{\circOne,\circSix}} \\
  traditional statistics on real line $\omsR$:                  & \pE  (\rvX)     &=& 3.5          & \ocsVar(\rvW;\pE)   &=& \frac{17}{4}&\approx& 4.25\\
  outcome subspace statistics on real line $\omsR$:             & \ocsE(\rvX)     &=& \setn{3.5}   & \ocsVar(\rvW;\ocsE) &=& \frac{17}{4}&\approx& 4.25\\
  outcome subspace statistics on  isomorphic structure $\ocsH$: & \ocsE(\rvY)     &=& \setn{1,\,6} & \ocsVar(\rvY;\ocsE) &=& \frac{5}{3} &\approx& 1.67\\
  outcome subspace statistics on  continuous structure $\ocsK$: & \ocsE(\rvZ)     &=& \setn{0.5}   & \ocsVar(\rvZ;\ocsE) &=& \frac{37}{20}&=&      1.85
\end{array}$\\
Note that based on the variance values, the statistic $\ocsE(\rvZ)$ on the continuous ring $\ocsK$
is arguably a much better statistic than $\ocsE(\rvX)$ on the (continuous) real line $\omsR$.
\end{example}
\begin{proof}
    \begin{align*}
      \ocscen(\ocsG)
      %  &\eqd \argmin_{x\in\omsH}\max_{y\in\omsH}\ocsmom(x,y)
      %  &&\text{by definition of $\ocscen$ \xref{def:ocscen}}
      %\\&\eqd \argmin_{x\in\omsH}\max_{y\in\omsH}\metric{x}{y}\psp(y)
      %  &&\text{by definition of $\ocsmom$ \xref{def:ocsmom}}
        &= \setn{1,6}
        && \text{by \exme{weighted spinner outcome subspace} example \xref{ex:wspinner}}
      %\\
      %\ocscena(\ocsG)
      %%  &\eqd \argmin_{x\in\omsH}\max_{y\in\omsH}\ocsmom(x,y)
      %%  &&\text{by definition of $\ocscena$ \xref{def:ocscena}}
      %%\\&\eqd \argmin_{x\in\omsH}\max_{y\in\omsH}\metric{x}{y}\psp(y)
      %%  &&\text{by definition of $\ocsmom$ \xref{def:ocsmom}}
      %  &= \setn{1,6}
      %  &&\text{by \prefpp{ex:wspinner}}
      %\\
      %\ocsVar(\rvY)
      %%  &\eqd \sum_{x\in\omsH}\ocsmom_2(C,x)
      %%  &&\text{by definition of $\ocsVar$ \xref{def:ocsVar}}
      %  &= \frac{1}{10}\brs{19}
      %   = 1.9
      %  &&\text{by \prefpp{ex:wspinner}}
      \\
      \pE(\rvX) 
        &\eqd \sum_{x\in\Z} x\psp(x)
      \\&= \mathrlap{1\times\frac{3}{10}+2\times\frac{1}{10}+3\times\frac{1}{10}+4\times\frac{1}{10}+5\times\frac{1}{10}+6\times\frac{1}{10}}
      \\&= \mathrlap{\frac{1}{10}\brp{1\times3 + 2 + 3 + 4 + 5 + 6\times3}
         =    \frac{35}{10}
         =    \frac{7}{2}
         =    3.5}
      \\
      \ocsVar(\rvX;\pE)
        &= \pVar(\rvX)
        && \text{by \prefp{thm:ocsVar}}
      \\&\eqd \sum_{x\in\Z} \brs{x-\pE(\rvX)}^2\psp(x)
        && \text{by definition of $\pVar$ \xref{def:pVar}}
      \\&= \mathrlap{\brp{1-\frac{7}{2}}^2\frac{3}{10}+\brp{2-\frac{7}{2}}^2\frac{1}{10}+\brp{3-\frac{7}{2}}^2\frac{1}{10}+\brp{4-\frac{7}{2}}^2\frac{1}{10}+\brp{5-\frac{7}{2}}^2\frac{1}{10}+\brp{6-\frac{7}{2}}^2\frac{3}{10}}
      \\&= \mathrlap{\frac{1}{10}\brs{\brp{-\frac{5}{2}}^2\times3+\brp{-\frac{3}{2}}^2+\brp{-\frac{1}{2}}^2+\brp{\frac{1}{2}}^2+\brp{\frac{3}{2}}^2+\brp{\frac{5}{2}}^2\times3}}
      \\&= \mathrlap{\frac{1}{40}\brs{75+9+1+1+9+75}
         = \frac{170}{40}
         = \frac{17}{4}
         = 4.25}
      \\
      \ocsE(\rvX)
        &= \pE(\rvX)
        && \text{because on \structe{real line}, $\psp$ is \prope{symmetric}, and by \prefp{thm:pEocsE}}
      \\&= \setn{3.5}
        && \text{by $\pE(\rvX)$ result}
      %  &\eqd \argmin_{x\in\R}\max_{y\in\R}\ocsmom(x,y)
      %  && \text{by definition of $\pE$ \xref{def:ocsE}}
      %\\&\eqd \argmin_{x\in\R}\max_{y\in\R}\metric{x}{y}\psp(y)
      %  && \text{by definition of $\pE$ \xref{def:ocsmom}}
      %\\&\eqd \argmin_{x\in\R}\max_{y\in\R}\abs{x-y}\psp(y)
      %  && \text{by definition usual metric on real line}
      %\\&= \argmin_{x\in\R}\brbl{\begin{array}{lM}
      %       \abs{x-1}\psp(1) & for $x\ge3.5$\\
      %       \abs{x-6}\psp(6) & for $x<3.5$
      %     \end{array}}
      %  && \text{\gsize\centering\psset{unit=6mm}{%============================================================================
% Daniel J. Greenhoe
% XeLaTeX file
%============================================================================
{%\psset{yunit=2\psunit}%
\begin{pspicture}(-1,0)(8,2)%
  \psset{%
    labelsep=1pt,
    linewidth=1pt,
    }%
  \psaxes[linecolor=axis,yAxis=false,labels=none]{->}(0,0)(0,0)(7.5,2.5)% x axis
  \psaxes[linecolor=axis,xAxis=false,labels=none]{->}(0,0)(0,0)(7.5,2.5)% y axis
  \psline(0,1.8)(3.5,0.75)(7,1.8)%
  %
  \psline[linestyle=dotted,linecolor=red](3.5,0.75)(3.5,0)%
  \psline[linestyle=dotted,linecolor=red](0,0.75)(3.5,0.75)%
  %
  \uput[0]{0}(8,0){$x$}%
  \uput[-90]{0}(3.5,0){$3.5$}%
  \uput[180]{0}(0,0.75){$0.75$}%
  \uput[180]{0}(0,1.8){$1.8$}%
  \rput[t](3.5,2){$\ds\ff(x)\eqd\max_{y\in\omsR}\abs{x-y}\psp(y)$}%
\end{pspicture}}%
}}
      %%\\&= \argmin_{x\in\R}\brbl{\begin{array}{lM}
      %%       \abs{x-1}\frac{3}{10} & for $x\ge3.5$\\
      %%       \abs{x-6}\frac{3}{10} & for $x<3.5$
      %%     \end{array}}
      %\\&= \setn{3.5}
      %  && \text{because expression is minimized at argument $x=3.5$}
      \\
      \ocsVar(\rvX;\ocsE)
        &= \ocsVar(\rvX;\pE)
        && \text{by $\pE(\rvX)$ and $\ocsE(\rvX)$ results}
      \\&= \frac{17}{4}= 4.25
        && \text{by $\ocsVar(\rvX;\pE)$ result}
      \\
      \ocsE(\rvY)
        &= \rvY\brs{\ocscen(\ocsG)}
        && \text{because $\ocsG$ and $\omsH$ are \prope{isomorphic} under $\rvY$}
      \\&= \rvY\brs{\setn{\circOne,\circSix}}
        && \text{by $\ocscen(\ocsG)$ result}
      \\&= \setn{1,6}
        && \text{by definition of $\rvY$}
      \\
      \ocsVar(\rvY;\ocsE)
        &= \ocsVaro(\ocsG)
        && \text{because $\ocsG$ and $\omsH$ are \prope{isomorphic} under $\rvY$}
      \\&= \frac{5}{3} \approx 1.667
        && \text{by \exme{weighted spinner outcome subspace} example \xref{ex:wspinner}}
      \\
      \ocsE(\rvZ)
        &\eqd \argmin_{x\in\omsK}\max_{y\in\omsK}\metric{x}{y}\psp(y)
        && \text{by definition of $\ocsE$ \xref{def:ocsE}}
      \\&= 0.5
        %&& \text{\gsize\centering\psset{unit=7.5mm}{%============================================================================
% Daniel J. Greenhoe
% XeLaTeX file
%============================================================================
{\psset{yunit=2\psunit}%
\begin{pspicture}(-0.7,-0.25)(7.25,1)%
  \psset{%
    labelsep=3pt,
    linewidth=1pt,
    }%
  \psaxes[linecolor=axis,yAxis=false]{->}(0,0)(0,0)(7,1)% x axis
  \psaxes[linecolor=axis,xAxis=false]{->}(0,0)(0,0)(7,1)% y axis
  \psline(0,0.3)(0.5,0.25)(1,0.3)(3,0.9)(3.5,0.75)(4,0.9)(6,0.3)(6.5,0.25)(7,0.3)%
  %
  \psline[linestyle=dotted,linecolor=red](0.5,0.25)(0.5,0)%
  \psline[linestyle=dotted,linecolor=red](3,0.9)(3,0)%
  \psline[linestyle=dotted,linecolor=red](3.5,0.75)(3.5,0)%
  \psline[linestyle=dotted,linecolor=red](4,0.9)(4,0)%
  \psline[linestyle=dotted,linecolor=red](6.5,0.25)(6.5,0)%
  %
  \psline[linestyle=dotted,linecolor=red](0,0.9)(4,0.9)%
  \psline[linestyle=dotted,linecolor=red](0,0.75)(3.5,0.75)%
  \psline[linestyle=dotted,linecolor=red](0,0.3)(7,0.3)%
  \psline[linestyle=dotted,linecolor=red](0,0.25)(6.5,0.25)%
  \uput[0]{0}(7,0){$x$}%
  \uput[-90]{0}(0.5,0){$0.5$}%
  \uput[-90]{0}(3.5,0){$3.5$}%
  \uput[-90]{0}(6.5,0){$0.5$}%
  %
  \uput[170]{0}(0,0.9){$0.9$}%
  \uput[190]{0}(0,0.75){$0.75$}%
  \uput[155]{0}(0,0.3){$0.3$}%
  \uput[205]{0}(0,0.25){$0.25$}%
  \rput[b](3.5,0.3){$\ds\ff(x)\eqd\max_{y\in\omsK}\metric{x}{y}\psp(y)$}%
\end{pspicture}}%
}}
        && \text{\gsize\centering\psset{unit=7.5mm}{\includegraphics{../common/math/graphics/pdfs/wspinnermax.pdf}}}
       %&& \text{(see plot of $\ds\ff(x)\eqd\max_{y\in\omsK}\metric{x}{y}\psp(y)$ in \prefp{fig:spinnerXO6Ycircle}}
      \\
      \ocsVar(\rvZ;\ocsE)
        &\eqd \sum_{x\in\ocsK} \metricsq{\ocsE(\rvZ)}{x}\psp(x)
        && \text{by definition of $\ocsVar$ \xref{def:ocsVar}}
      \\&= \sum_{x\in\ocsK} \metricsq{\frac{1}{2}}{x}\psp(x)
        && \text{by $\ocsE(\rvZ)$ result}
      \\&= \mathrlap{
           \brp{\frac{1}{2}}^2\frac{3}{10}
          +\brp{\frac{3}{2}}^2\frac{1}{10}
          +\brp{\frac{5}{2}}^2\frac{1}{10}
          +\brp{\frac{5}{2}}^2\frac{1}{10}
          +\brp{\frac{3}{2}}^2\frac{1}{10}
          +\brp{\frac{1}{2}}^2\frac{3}{10}}
      \\&= \mathrlap{\frac{1}{40}\brp{3+9+25+25+9+3} = \frac{74}{40} = \frac{37}{20} = 1.85}
    \end{align*}
\end{proof}


%=======================================
\subsubsection{Pseudo-random number generator (PRNG) examples}
%=======================================
\begin{figure}[h]
  \gsize%
  \centering%
  %%============================================================================
% Daniel J. Greenhoe
% LaTeX file
% linear congruential (LCG) pseudo-random number generator (PRNG) mappings
% x_{n+1} = (7x_n+5)mod 9
% y_{n+1} = (y_n+2)mod 5
%============================================================================
\begin{pspicture}(-2.25,-3.8)(11.5,3.8)%
  %---------------------------------
  % options
  %---------------------------------
  \psset{%
    radius=1.25ex,
    labelsep=2.5mm,
    linecolor=blue,%
    }%
  %---------------------------------
  % LCG PRNG graph G
  % x_{n+1} = (7x_n+5)mod 9
  %   n  0   1   2   3   4   5   6   7   8  ;  9
  % x_n  1   3   8   7   0   5   4   6   2  ;  1
  %---------------------------------
  \rput(0,0){\psset{unit=1.5\psunit}%
    \rput{   0}(0,0){\rput(1,0){\Cnode[fillstyle=solid,fillcolor=snode](0,0){G0}}}%
    \rput{  40}(0,0){\rput(1,0){\Cnode[fillstyle=solid,fillcolor=snode](0,0){G1}}}%
    \rput{  80}(0,0){\rput(1,0){\Cnode[fillstyle=solid,fillcolor=snode](0,0){G2}}}%
    \rput{ 120}(0,0){\rput(1,0){\Cnode[fillstyle=solid,fillcolor=snode](0,0){G3}}}%
    \rput{ 160}(0,0){\rput(1,0){\Cnode[fillstyle=solid,fillcolor=snode](0,0){G4}}}%
    \rput{ 200}(0,0){\rput(1,0){\Cnode[fillstyle=solid,fillcolor=snode](0,0){G5}}}%
    \rput{ 240}(0,0){\rput(1,0){\Cnode[fillstyle=solid,fillcolor=snode](0,0){G6}}}%
    \rput{ 280}(0,0){\rput(1,0){\Cnode[fillstyle=solid,fillcolor=snode](0,0){G7}}}%
    \rput{ 320}(0,0){\rput(1,0){\Cnode[fillstyle=solid,fillcolor=snode](0,0){G8}}}%
    \rput(0,0){$\ocsG_9$}%
    }%
  \rput(G8){$8$}%
  \rput(G7){$7$}%
  \rput(G6){$6$}%
  \rput(G5){$5$}%
  \rput(G4){$4$}%
  \rput(G3){$3$}%
  \rput(G2){$2$}%
  \rput(G1){$1$}%
  \rput(G0){$0$}%
  %          
  \ncline{G8}{G0}%
  \ncline{G7}{G8}%
  \ncline{G6}{G7}%
  \ncline{G5}{G6}%
  \ncline{G4}{G5}%
  \ncline{G3}{G4}%
  \ncline{G2}{G3}%
  \ncline{G1}{G2}%
  \ncline{G0}{G1}%
  %
  \uput[  0](G0){$\frac{1}{9}$}
  \uput[ 40](G1){$\frac{1}{9}$}
  \uput[ 80](G2){$\frac{1}{9}$}
  \uput[120](G3){$\frac{1}{9}$}
  \uput[160](G4){$\frac{1}{9}$}
  \uput[200](G5){$\frac{1}{9}$}
  \uput[240](G6){$\frac{1}{9}$}
  \uput[280](G7){$\frac{1}{9}$}
  \uput[320](G8){$\frac{1}{9}$}
  %
  %
  %---------------------------------
  %---------------------------------
  %---------------------------------
  \rput(5,0){% shaped outcomes spaces
  %---------------------------------
  %---------------------------------
  %---------------------------------
  %
  %
  %---------------------------------
  % distribution shaping mapping y_n=s(x_n) to 5 element set
  %   n  0   1   2   3   4   5   6   7   8  ;  9
  % x_n  1   3   8   7   0   5   4   6   2  ;  1
  % y_n  1   3   0   2   4 ; 0   3   4   4  ;  1
  %---------------------------------
  \rput( 0,0){%\psset{unit=2\psunit}%
    \rput{288}(0,0){\rput(1,0){\Cnode[fillstyle=solid,fillcolor=snode](0,0){S4}}}%
    \rput{216}(0,0){\rput(1,0){\Cnode(0,0){S3}}}%
    \rput{144}(0,0){\rput(1,0){\Cnode(0,0){S2}}}%
    \rput{ 72}(0,0){\rput(1,0){\Cnode(0,0){S1}}}%
    \rput{  0}(0,0){\rput(1,0){\Cnode(0,0){S0}}}%
    \rput(0,0){$\ocsG$}
    }
  \rput(S4){$4$}%
  \rput(S3){$3$}%
  \rput(S2){$2$}%
  \rput(S1){$1$}%
  \rput(S0){$0$}%
  %
  \ncline{S4}{S0}%
  \ncline{S3}{S4}%
  \ncline{S2}{S3}%
  \ncline{S1}{S2}%
  \ncline{S0}{S1}%
  %
  \uput[288](S4){$\frac{3}{9}$}
  \uput[216](S3){$\frac{2}{9}$}
  \uput[144](S2){$\frac{1}{9}$}
  \uput[ 72](S1){$\frac{1}{9}$}
  \uput[  0](S0){$\frac{2}{9}$}
  %
  \ncarc[arcangle=-22,linewidth=0.75pt,linecolor=blue]{->}{G8}{S0}%
  \ncarc[arcangle=-45,linewidth=0.75pt,linecolor=blue]{->}{G7}{S2}%
  \ncarc[arcangle=-45,linewidth=0.75pt,linecolor=blue]{->}{G6}{S4}%
  \ncarc[arcangle=-22,linewidth=0.75pt,linecolor=blue]{->}{G5}{S0}%
  \ncarc[arcangle=-67,linewidth=0.75pt,linecolor=blue]{->}{G4}{S3}%
  \ncarc[arcangle= 45,linewidth=0.75pt,linecolor=blue]{->}{G3}{S3}%
  \ncarc[arcangle=-22,linewidth=0.75pt,linecolor=blue]{->}{G2}{S4}%
  \ncarc[arcangle= 22,linewidth=0.75pt,linecolor=blue]{->}{G1}{S1}%
  \ncarc[arcangle= 37,linewidth=0.75pt,linecolor=blue]{->}{G0}{S4}%
  %---------------------------------
  % random variable mapping Y from S to integer line 
  %---------------------------------
  \rput(0,3){%\psset{unit=0.75\psunit}%
    \pnode(3,0){YB}%
    \Cnode(2,0){Y4}%
    \Cnode[fillstyle=solid,fillcolor=snode](1,0){Y3}%
    \Cnode[fillstyle=solid,fillcolor=snode](0,0){Y2}%
    \Cnode(-1,0){Y1}%
    \Cnode(-2,0){Y0}%
    \pnode(-3,0){YA}%
    }
  \rput(Y4){$4$}%
  \rput(Y3){$3$}%
  \rput(Y2){$2$}%
  \rput(Y1){$1$}%
  \rput(Y0){$0$}%
  \uput[0](YB){$\omsZ$}%
  %
  \ncline[linestyle=dotted]{Y4}{YB}%
  \ncline{Y3}{Y4}%
  \ncline{Y2}{Y3}%
  \ncline{Y1}{Y2}%
  \ncline{Y0}{Y1}%
  \ncline[linestyle=dotted]{Y0}{YA}%
  %
  \uput[90](Y4){$\frac{3}{9}$}
  \uput[90](Y3){$\frac{2}{9}$}
  \uput[90](Y2){$\frac{1}{9}$}
  \uput[90](Y1){$\frac{1}{9}$}
  \uput[90](Y0){$\frac{2}{9}$}
  %
  \ncarc[arcangle=-45,linewidth=0.75pt,linecolor=red]{->}{S4}{Y4}%
  \ncarc[arcangle= 22,linewidth=0.75pt,linecolor=red]{->}{S3}{Y3}%
  \ncarc[arcangle= 45,linewidth=0.75pt,linecolor=red]{->}{S2}{Y2}%
  \ncarc[arcangle=-22,linewidth=0.75pt,linecolor=red]{->}{S1}{Y1}%
  \ncarc[arcangle= 22,linewidth=0.75pt,linecolor=red]{->}{S0}{Y0}%
  %---------------------------------
  % random variable mapping Z from S to ring-like structure 
  %---------------------------------
  \rput( 5,0){%\psset{unit=2\psunit}%
    \rput{288}(0,0){\rput(1,0){\Cnode[fillstyle=solid,fillcolor=snode](0,0){Z4}}}%
    \rput{216}(0,0){\rput(1,0){\Cnode(0,0){Z3}}}%
    \rput{144}(0,0){\rput(1,0){\Cnode(0,0){Z2}}}%
    \rput{ 72}(0,0){\rput(1,0){\Cnode(0,0){Z1}}}%
    \rput{  0}(0,0){\rput(1,0){\Cnode(0,0){Z0}}}%
    \rput(0,0){$\omsH$}%
    }
  \rput(Z4){$4$}%
  \rput(Z3){$3$}%
  \rput(Z2){$2$}%
  \rput(Z1){$1$}%
  \rput(Z0){$0$}%
  %
  \ncline{Z4}{Z0}%
  \ncline{Z3}{Z4}%
  \ncline{Z2}{Z3}%
  \ncline{Z1}{Z2}%
  \ncline{Z0}{Z1}%
  %
  \uput[288](Z4){$\frac{3}{9}$}
  \uput[216](Z3){$\frac{2}{9}$}
  \uput[144](Z2){$\frac{1}{9}$}
  \uput[ 72](Z1){$\frac{1}{9}$}
  \uput[  0](Z0){$\frac{2}{9}$}
  %
  \ncarc[arcangle=-22,linewidth=0.75pt,linecolor=green]{->}{S4}{Z4}%
  \ncarc[arcangle=-45,linewidth=0.75pt,linecolor=green]{->}{S3}{Z3}%
  \ncarc[arcangle=-10,linewidth=0.75pt,linecolor=green]{->}{S2}{Z2}%
  \ncarc[arcangle= 10,linewidth=0.75pt,linecolor=green]{->}{S1}{Z1}%
  \ncarc[arcangle=-10,linewidth=0.75pt,linecolor=green]{->}{S0}{Z0}%
  %---------------------------------
  % random variable mapping X from S to real line
  %---------------------------------
  \rput(0,-3){%\psset{unit=0.5\psunit}%
    \psline{<->}(-2.5,0)(2.5,0)%
    \multirput(-2,0)(1,0){5}{\psline(0,-0.1)(0,0.1)}%
    \pnode(2.5,0){XB}%
    \pnode( 2,0){X4}%
    \pnode( 1,0){X3}%
    \pnode(0.4,0){X240}%
    \pnode(0.33,0){X233}%
    \pnode( 0,0){X2}%
    \pnode(-1,0){X1}%
    \pnode(-2,0){X0}%
    \pnode(-2.5,0){XA}%
    }
  %
  \uput[0](XB){$\omsR$}%
  \rput(X240){\pscircle[fillstyle=solid,linecolor=snode,fillcolor=snode](0,0){1ex}}%
  \rput(X233){\pscircle[fillstyle=none,linecolor=red,fillcolor=red](0,0){1ex}}%
  %
  \uput{1.5mm}[-90](X4){$\frac{3}{9}$}%
  \uput{1.5mm}[-90](X3){$\frac{2}{9}$}%
  \uput{1.5mm}[-90](X2){$\frac{1}{9}$}%
  \uput{1.5mm}[-90](X1){$\frac{1}{9}$}%
  \uput{1.5mm}[-90](X0){$\frac{2}{9}$}%
  %
  \uput{1.5mm}[90](X4){$4$}%
  \uput{1.5mm}[90](X3){$3$}%
  \uput{1.5mm}[90](X2){$2$}%
  \uput{1.5mm}[90](X1){$1$}%
  \uput{1.5mm}[90](X0){$0$}%
  %
  \ncarc[arcangle= 22,linewidth=0.75pt,linecolor=purple]{->}{S4}{X4}%
  \ncarc[arcangle=-22,linewidth=0.75pt,linecolor=purple]{->}{S3}{X3}%
  \ncarc[arcangle=-22,linewidth=0.75pt,linecolor=purple]{->}{S2}{X2}%
  \ncarc[arcangle= 45,linewidth=0.75pt,linecolor=purple]{->}{S1}{X1}%
  \ncarc[arcangle=-22,linewidth=0.75pt,linecolor=purple]{->}{S0}{X0}%
  %---------------------------------
  % labels
  %---------------------------------
  \rput(0,1.9){$\rvY(\cdot)$}%
  \rput(0,-1.9){$\rvX(\cdot)$}%
  \rput(2.75,0){$\rvZ(\cdot)$}%
  }%
  \rput(2.75,0){$\fs(x_n)$}%
\end{pspicture}%%
  \includegraphics{../common/math/graphics/pdfs/lcg7x1m9_XYZ.pdf}%
  \caption{LCG mappings to \prope{linear} ($\rvX$), non-linear discrete ($\rvY$)
  and non-linear continuous ($\rvZ$) ordered metric spaces \xref{ex:lcg7x1m9_xyz}\label{fig:lcg7x1m9_xyz}}
\end{figure}
%---------------------------------------
%\begin{minipage}{\tw-65mm}%
\begin{example}[\exmd{LCG mappings, standard ordering}]
\label{ex:lcg7x1m9_xyz}\mbox{}\\
%---------------------------------------
The equation $x_{n+1}=(7x_n+5)\mod9$ with $x_0=1$ is a \structe{linear congruential} (LCG) 
\structe{pseudo-random number generator} (PRNG)
that has \prope{full period}\footnote{
  \citeP{hull1962},
  \citerpgc{kennedy1980}{137}{0824768981}{Theorem 6.1},
  \citerpgc{severance2001}{86}{0471496944}{Hull-Dobell Theorem}
  }
of 9 values.
These 9 values can be mapped, using a \prope{surjective} \xref{def:ftypes} function $\fs\in\clF{\ocsG_9}{\ocsG}$ 
to the 5 element set $\setn{0,1,2,3,4}$ to ``shape" the distribution from a \prope{uniform} distribution to 
\prope{non-uniform}:\footnote{The sequence $\seqn{1,3,0,2,4,\,\,1,3,0,2,4,\,\,1,3,\cdots}$
is generated by the equation $y_{n+1}=(y_n+2)\mod5$ with $y_0=1$}.
\\\indentx$\begin{array}{r|*{9}{c}|*{4}{c}}
    n             & 0 & 1 & 2 & 3 & 4 & 5 & 6 & 7 & 8    & 9 & 10 & 11 & \cdots\\\hline 
  x_n             & 1 & 3 & 8 & 7 & 0 & 5 & 4 & 6 & 2    & 1 &  3 &  8 & \cdots\\
  y_n\eqd\fs(x_n) & 1 & 3 & 0 & 2 & 4 & 1 & 3 & 4 & 4    & 1 &  3 &  0 & \cdots%\\\hline
\end{array}$\\
Let $\ocsG$ be the \structe{outcome subspace} and 
$\rvX$, $\rvY$, and $\rvZ$ be the \fncte{outcome random variables} 
illustrated in \prefpp{fig:lcg7x1m9_xyz}.
%Let $\rvX$ be a random variable from $\ocsG$ to the \structe{real line} \xref{def:Rline},
%    $\rvY$    a random variable from $\ocsG$ to the \structe{integer line} \xref{def:Zline},
%and $\rvZ$    a random variable from $\ocsG$ to the 5 element ring, as illustrated in \pref{fig:lcg7x1m9}.
This yields the following statistics:
\\$\quad\begin{array}{>{\gsizes}Mrcc rcccl}
  geometry of $\ocsG_9$:                                & \ocscen (\ocsG) &=& \mc{4}{l}{\setn{0,1,2,3,4,5,6,7,8}}  \\
  geometry of $\ocsG$:                                  & \ocscen (\ocsG) &=& \setn{4}  \\
  traditional statistics on real line:                  & \pE  (\rvX)     &=& \frac{7}{3}\approx2.333  & \ocsVar(\rvX;\pE)   &=& \frac{22}{9}    &\approx& 2.444\\
  outcome subspace statistics on real line:             & \ocsE (\rvX)    &=& \setn{\frac{12}{5}=2.4}  & \ocsVar(\rvX;\ocsE) &=& \frac{551}{225} &\approx& 2.449\\
  outcome subspace statistics on integer line:          & \ocsE (\rvY)    &=& \setn{2,\,3}             & \ocsVar(\rvY;\ocsE) &=& \frac{16}{9}     &\approx& 1.778 \\
  outcome subspace statistics on isomorphic structure:  & \ocsE (\rvZ)    &=& \setn{4}                 & \ocsVar(\rvZ;\ocsE) &=& \frac{4}{3}     &\approx& 1.333
\end{array}$\\
Note that unlike the statistics $\pE(\rvX)$ and $\ocsE(\rvX)$ on the \structe{real line},
the statistic $\ocsE(\rvZ)$ on the \prope{isomorphic} structure $\ocsK$ yields the \prope{maximally likely} result,
and a much smaller variance as well.
\end{example}
%\end{minipage}\hfill%
%\begin{tabular}{c}
%  \gsize%
%  %\psset{unit=5mm}%
%  {%============================================================================
% Daniel J. Greenhoe
% LaTeX file
% discrete metric real dice mapping to linearly ordered O6c
%============================================================================
{%\psset{unit=0.5\psunit}%
\begin{pspicture}(-3.25,-1.5)(3.25,1.5)%
  %---------------------------------
  % options
  %---------------------------------
  \psset{%
    radius=1.25ex,
    labelsep=2.5mm,
    linecolor=blue,%
    }%
  %---------------------------------
  % dice graph
  %---------------------------------
  \rput(-1.75,0){%\psset{unit=2\psunit}%
    \Cnode[fillstyle=solid,fillcolor=snode](-0.8660,-0.5){D4}%
    \Cnode(-0.8660,0.5){D5}%
    \Cnode(0,1){D6}%
    \Cnode(0.8660,0.5){D3}%
    \Cnode(0.8660,-0.5){D2}%
    \Cnode(0,-1){D1}%
    }
  \rput(D6){$\diceF$}%
  \rput(D5){$\diceE$}%
  \rput(D4){$\diceD$}%
  \rput(D3){$\diceC$}%
  \rput(D2){$\diceB$}%
  \rput(D1){$\diceA$}%
  %
  \ncline{D5}{D6}%
  \ncline{D4}{D5}\ncline{D4}{D6}%
  \ncline{D3}{D5}\ncline{D3}{D6}%
  \ncline{D2}{D3}\ncline{D2}{D4}\ncline{D2}{D6}%
  \ncline{D1}{D2}\ncline{D1}{D3}\ncline{D1}{D4}\ncline{D1}{D5}%
  %
  \uput[ 158](D6){$\frac{1}{30}$}
  \uput[ 150](D5){$\frac{1}{50}$}
  \uput[ 210](D4){$\frac{3}{5}$}
  \uput[  22](D3){$\frac{1}{30}$}
  \uput[ -45](D2){$\frac{1}{20}$}
  \uput[-158](D1){$\frac{1}{10}$}
  %---------------------------------
  % range graph
  %---------------------------------
  \rput(1.75,0){%\psset{unit=2\psunit}%
    \Cnode[fillstyle=solid,fillcolor=snode](-0.8660,-0.5){E4}%
    \Cnode(-0.8660,0.5){E5}%
    \Cnode(0,1){E6}%
    \Cnode(0.8660,0.5){E3}%
    \Cnode(0.8660,-0.5){E2}%
    \Cnode(0,-1){E1}%
    \Cnode(0,0){E0}%
    }
  \rput(E6){$6$}%
  \rput(E5){$5$}%
  \rput(E4){$4$}%
  \rput(E3){$3$}%
  \rput(E2){$2$}%
  \rput(E1){$1$}%
  \rput(E0){$0$}%
  %
  \ncline{E5}{E6}%
  \ncline{E4}{E5}\ncline{E4}{E6}%
  \ncline{E3}{E5}\ncline{E3}{E6}%
  \ncline{E2}{E3}\ncline{E2}{E4}\ncline{E2}{E6}%
  \ncline{E1}{E2}\ncline{E1}{E3}\ncline{E1}{E4}\ncline{E1}{E5}%
  \ncline{E0}{E1}\ncline{E0}{E2}\ncline{E0}{E3}\ncline{E0}{E4}\ncline{E0}{E5}\ncline{E0}{E6}%
  %
  \uput[ 158](E6){$\frac{1}{30}$}
  \uput[ 150](E5){$\frac{1}{50}$}
  \uput[ 210](E4){$\frac{3}{5}$}
  \uput[  22](E3){$\frac{1}{30}$}
  \uput[ -45](E2){$\frac{1}{20}$}
  \uput[-158](E1){$\frac{1}{10}$}
  \uput[  0](E0){$\sfrac{0}{6}$}
  %---------------------------------
  % mapping from die to O6c
  %---------------------------------
  \ncarc[arcangle= 22,linewidth=0.75pt,linecolor=red]{->}{D6}{E6}%
  \ncarc[arcangle= 22,linewidth=0.75pt,linecolor=red]{->}{D5}{E5}%
  \ncarc[arcangle= 22,linewidth=0.75pt,linecolor=red]{->}{D4}{E4}%
  \ncarc[arcangle=-22,linewidth=0.75pt,linecolor=red]{->}{D3}{E3}%
  \ncarc[arcangle=-22,linewidth=0.75pt,linecolor=red]{->}{D2}{E2}%
  \ncarc[arcangle=-22,linewidth=0.75pt,linecolor=red]{->}{D1}{E1}%
  %---------------------------------
  % labels
  %---------------------------------
  \rput(0,0){$\rvX(\cdot)$}%
  %\ncline[linestyle=dotted,nodesep=1pt]{->}{xzlabel}{xz}%
  %\ncline[linestyle=dotted,nodesep=1pt]{->}{ylabel}{y}%
\end{pspicture}
}%}%
%\end{tabular}
\begin{proof}
    \begin{align*}
      \ocscen(\ocsG_9)
        &\eqd \argmin_{x\in\ocsG_9}\max_{y\in\ocsG_9}\metric{x}{y}\psp(y)
        &&\text{by definition of $\ocscen$ \xref{def:ocscen}}
      \\&= \argmin_{x\in\ocsG_9}\max_{y\in\ocsG_9}\metric{x}{y}\frac{1}{9}
        &&\text{by definition of $\ocsG_9$}
      \\&= \argmin_{x\in\ocsG_9}\max_{y\in\ocsG_9}\metric{x}{y}
        && \text{because $\fphi(x)=\frac{1}{9}x$ is \prope{strictly isotone} and by \prefp{lem:argminmaxphi}}
      \\&= \argmin_{x\in\ocsG_9}\setn{4,\,4,\,4,\,4,\,4,\,4,\,4,\,4,\,4}
        && \text{because the maximum distance in $\ocsG_9$ from any $x$ is $4$}
      \\&= \setn{0,1,2,\cdots,8}
        && \text{because the distances for values of $x$ in $\ocsG_9$ are the same}
      \\
      \ocscen(\ocsG)
        &= \setn{4} 
        && \text{by \exme{weighted ring outcome subspace} example \xref{ex:wring5}}
      \\
      \pE(\rvX) 
        &\eqd \sum_{x\in\R} x\psp(x)
        &&\text{by definition of $\pE$ \xref{def:pE}}
      \\&=\mathrlap{0\times\frac{2}{9}+1\times\frac{1}{9}+2\times\frac{1}{9}+3\times\frac{2}{9}+4\times\frac{3}{9} 
         = \frac{21}{9} = \frac{7}{3} \approx 2.333}
      %\\&=\mathrlap{
      %        \frac{1}{9}\brp{0 + 1 + 2 + 6 + 12}
      %   =    \frac{21}{9}
      %   =    \frac{7}{3}
      %   =    2\!\frac{1}{3} \approx 2.33}
      \\
      \ocsVar(\rvX;\pE)
        &= \pVar(\rvX)
        && \text{by \prefp{thm:ocsVar}}
      \\&\eqd \sum_{x\in\R} \brs{x-\pE(\rvX)}^2\psp(x)
        &&\text{by definition of $\pVar$ \xref{def:pVar}}
      \\&= \mathrlap{\brp{0-\frac{7}{3}}^2\frac{2}{9}+\brp{1-\frac{7}{3}}^2\frac{1}{9}+\brp{2-\frac{7}{3}}^2\frac{1}{9}+\brp{3-\frac{7}{3}}^2\frac{2}{9}+\brp{4-\frac{7}{3}}^2\frac{3}{9} 
         = \frac{198}{81} = \frac{22}{9} \approx 2.457}
      %\\&= \mathrlap{\frac{1}{9}\brs{\brp{-\frac{7}{3}}^2\times2+\brp{-\frac{4}{3}}^2\times1+\brp{-\frac{1}{3}}^2\times1+\brp{\frac{2}{3}}^2\times2+\brp{\frac{5}{3}}^2\times3}}
      %\\&= \mathrlap{\frac{1}{81}\brs{98+16+1+8+75} = \frac{198}{81} = \frac{22}{9} \approx 2.457}
      \\
      \ocsE(\rvX)
        &\eqd \argmin_{x\in\R}\max_{y\in\R}\metric{x}{y}\psp(y)
        &&\text{by definition of $\ocsE$ \xref{def:ocsE}}
      \\&\eqd \argmin_{x\in\R}\max_{y\in\R}\abs{x-y}\psp(y)
        &&\text{by definition usual metric on real line \xref{def:Rline}}
      \\&= \mathrlap{\argmin_{x\in\R}\brbl{\begin{array}{lM}
             \abs{x-4}\psp(4) & for $x\le\frac{12}{5}$\\
             \abs{x-0}\psp(0) & otherwise
           \end{array}}
        %\qquad\text{\psset{unit=6mm}\gsize%============================================================================
% Daniel J. Greenhoe
% XeLaTeX file
%============================================================================
{%\psset{yunit=2\psunit}%
\begin{pspicture}(-1.5,-0.6)(9,2)%
  \psset{%
    labelsep=3pt,
    linewidth=1pt,
    }%
  \psaxes[linecolor=axis,yAxis=false,labels=none]{<->}(0,0)(-1.5,0)(8.5,2)% x axis
  \psaxes[linecolor=axis,xAxis=false,labels=none]{->}(0,0)(-1.5,0)(8.5,2)% y axis
  \psline(-1,1.6667)(2.4,0.5333)(8,1.3333)%
  %
  \psline[linestyle=dotted,linecolor=red](2.4,0.5333)(2.4,0)%
  \psline[linestyle=dotted,linecolor=red](0,0.5333)(2.4,0.5333)%
  %
  \uput[0]{0}(8.5,0){$x$}%
  \uput[-90]{0}(2.4,0){$\sfrac{12}{5}$}%
  \uput[180]{0}(0,0.533){$\sfrac{8}{15}$}%
  \rput[b](3,1){$\ds\ff(x)\eqd\max_{y\in\omsR}\abs{x-y}\psp(y)$}%
\end{pspicture}}%
}
        \qquad\text{\includegraphics{../common/math/graphics/pdfs/lcg7x1m9_max.pdf}}}
      \\&= \setn{\frac{12}{5}} = 2.4
        && \text{because expression is minimized at argument $x=\frac{12}{5}$}
      \\
      \ocsVar(\rvX;\ocsE)
        &\eqd \sum_{x\in\R}\metricsq{\ocsE(\rvX)}{x}\psp(x)
        && \text{by definition of $\ocsVar$ \xref{def:ocsVar}}
      \\&= \sum_{x\in\R}\metricsq{\frac{12}{5}}{x}\psp(x)
        && \text{by $\ocsE(\rvX)$ result}
      \\&= \mathrlap{
           \brp{\frac{12}{5}-0}^2\frac{2}{9} +
           \brp{\frac{12}{5}-1}^2\frac{1}{9} +
           \brp{\frac{12}{5}-2}^2\frac{1}{9} +
           \brp{\frac{12}{5}-3}^2\frac{2}{9} +
           \brp{\frac{12}{5}-4}^2\frac{3}{9} 
           }
      \\&= \mathrlap{\frac{1}{25\times9}
           2\brp{12- 0}^2 +
           1\brp{12- 5}^2 +
           1\brp{12-10}^2 +
           2\brp{12-15}^2 +
           3\brp{12-20}^2 
         = \frac{551}{225} \approx 2.449
           }
      \\
      \ocsE(\rvY)
      \\&\eqd \argmin_{x\in\omsH}\max_{y\in\omsH}\metric{x}{y}\psp(y)
        &&\text{by definition of $\ocsE$ \xref{def:ocsE}}
      \\&= \argmin_{x\in\omsH}\max_{y\in\omsH}\abs{x-y}\psp(y)
        &&\text{by definition of \structe{integer line} \xref{def:Zline}}
      \\&=\mathrlap{\argmin_{x\in\omsH}\max_{y\in\omsH}\frac{1}{9}
             \setn{\begin{array}{ccccc}
               {0}\times2&{1}\times1&{2}\times1&{3}\times2&{4}\times3\\
               {1}\times2&{0}\times1&{1}\times1&{2}\times2&{3}\times3\\
               {2}\times2&{1}\times1&{0}\times1&{1}\times2&{2}\times3\\
               {3}\times2&{2}\times1&{1}\times1&{0}\times2&{1}\times3\\
               {4}\times2&{3}\times1&{2}\times1&{1}\times2&{0}\times3
             \end{array}}
      \quad= \argmin_{x\in\omsH}\frac{1}{9}
             \setn{\begin{array}{c}
               12\\
                9\\
                6\\
                6\\
                8
             \end{array}}
      \quad= \argmin_{x\in\omsH}\frac{1}{9}
             \setn{\begin{array}{c}
               \mbox{ }\\
               \mbox{ }\\
               2\\
               3\\
               \mbox{ }
             \end{array}}}
      \\
      \ocsVar(\rvY;\ocsE)
        &\eqd \sum_{x\in\Z}\metricsq{\ocsE(\rvY)}{x}\psp(x)
        && \text{by definition of $\ocsVar$ \xref{def:ocsVar}}
      \\&= \sum_{x\in\Z}\metricsq{\setn{2,3}}{x}\psp(x)
        && \text{by $\ocsE(\rvY)$ result}
      \\&= \mathrlap{
           2^2\times\frac{2}{9} +
           1^2\times\frac{1}{9} +
           0^2\times\frac{1}{9} +
           0^2\times\frac{2}{9} +
           1^2\times\frac{3}{9} 
         = \frac{16}{9} \approx 1.778}
      \\
      \ocsE(\rvZ)
        &= \rvZ\brs{\ocscen(\ocsG)}
        && \text{because $\ocsG$ and $\omsH$ are \prope{isomorphic} under $\rvZ$}
      \\&= \setn{4}
        &&\text{by $\ocscen(\ocsG)$ result}
      \\
      \ocsVar(\rvZ;\ocsE)
        &\eqd \sum_{x\in\omsH} \metricsq{\ocsE(\rvZ)}{x}\psp(x)
        && \text{by definition of $\ocsVar$ \xref{def:ocsVar}}
      \\&= \sum_{x\in\omsH} \metricsq{\setn{4}}{x}\psp(x)
        && \text{by $\ocsE(\rvZ)$ result}
      \\&= \mathrlap{
           1^2\times\frac{2}{9} +
           2^2\times\frac{1}{9} +
           2^2\times\frac{1}{9} +
           1^2\times\frac{2}{9} +
           0^2\times\frac{3}{9} 
         = \frac{12}{9}= \frac{4}{3} \approx 1.333}
       \\
%      \ocsEM(\rvZ)
%        &\eqd \argmax_{x\in\omsH}\min_{y\in\omsH\setd\setn{x}}\ocsmom(x,y)
%        &&\text{by definition of $\ocsEM$ \xref{def:ocsEM}}
%      \\&\eqd \argmin_{x\in\omsH}\max_{y\in\omsH\setd\setn{x}}\metric{x}{y}\psp(y)
%        &&\text{by definition of $\ocsmom$ \xref{def:ocsmom}}
%      \\&= \argmin_{x\in\omsH}\max_{y\in\omsH\setd\setn{x}}\frac{1}{9}\metric{x}{y}\psp(y)9
%      \\&= \argmin_{x\in\omsH}\max_{y\in\omsH\setd\setn{x}}\metric{x}{y}\psp(y)9
%        && \text{because $\fphi(x)=\frac{1}{6}x$ is \prope{strictly isotone} and by \prefp{lem:argminmaxphi}}
%      %\\&= \argmax_{x\in\omsH}\min_{y\in\omsH\setd\setn{x}}
%      %       \setn{\begin{array}{ccccccccc}
%      %                                 &\metricn(0,1)\frac{1}{9}&\metricn(0,2)\frac{1}{9}&\metricn(0,3)\frac{2}{9}&\metricn(0,4)\frac{3}{9}\\
%      %         \metricn(1,0)\frac{2}{9}&                        &\metricn(1,2)\frac{1}{9}&\metricn(1,3)\frac{2}{9}&\metricn(1,4)\frac{3}{9}\\
%      %         \metricn(2,0)\frac{2}{9}&\metricn(2,1)\frac{1}{9}&                        &\metricn(2,3)\frac{2}{9}&\metricn(2,4)\frac{3}{9}\\
%      %         \metricn(3,0)\frac{2}{9}&\metricn(3,1)\frac{1}{9}&\metricn(3,2)\frac{1}{9}&                        &\metricn(3,4)\frac{3}{9}\\
%      %         \metricn(4,0)\frac{2}{9}&\metricn(4,1)\frac{1}{9}&\metricn(4,2)\frac{1}{9}&\metricn(4,3)\frac{2}{9}&                          
%      %       \end{array}}
%      \\&=\mathrlap{\argmax_{x\in\omsH}\min_{y\in\omsH\setd\setn{x}}%\frac{1}{9}
%             \setn{\begin{array}{ccccccccc}
%                         &{2}\times1&{1}\times1&{1}\times2&{2}\times3\\
%               {2}\times2&          &{2}\times1&{1}\times2&{1}\times3\\
%               {1}\times2&{2}\times1&          &{2}\times2&{1}\times3\\
%               {1}\times2&{1}\times1&{2}\times1&          &{1}\times3\\
%               {2}\times2&{1}\times1&{1}\times1&{1}\times2& 
%             \end{array}}
%      \quad= \argmax_{x\in\omsH}
%             \setn{\begin{array}{c}
%                1\\
%                2\\
%                2\\
%                1\\
%                1
%             \end{array}}
%      \quad= \setn{1,2}}
%      \\\\
%      \ocsEm(\rvZ)
%        &\eqd \argmin_{x\in\omsH}\min_{y\in\omsH\setd\setn{x}}\ocsmom(x,y)
%        &&\text{by definition of $\ocsEm$ \xref{def:ocsEm}}
%      \\&\eqd \argmin_{x\in\omsH}\min_{y\in\omsH\setd\setn{x}}\metric{x}{y}\psp(y)
%        &&\text{by definition of $\ocsmom$ \xref{def:ocsmom}}
%      \\&= \argmin_{x\in\omsH}\min_{y\in\omsH\setd\setn{x}}\frac{1}{9}\metric{x}{y}\psp(y)9
%      \\&= \argmin_{x\in\omsH}\min_{y\in\omsH\setd\setn{x}}\metric{x}{y}\psp(y)9
%        && \text{because $\fphi(x)=\frac{1}{9}x$ is \prope{strictly isotone} and by \prefp{lem:argminmaxphi}}
%      \\&=\mathrlap{\argmin_{x\in\omsH}\min_{y\in\omsH\setd\setn{x}}%\frac{1}{9}
%             \setn{\begin{array}{ccccccccc}
%                         &{2}\times1&{1}\times1&{1}\times2&{2}\times3\\
%               {2}\times2&          &{2}\times1&{1}\times2&{1}\times3\\
%               {1}\times2&{2}\times1&          &{2}\times2&{1}\times3\\
%               {1}\times2&{1}\times1&{2}\times1&          &{1}\times3\\
%               {2}\times2&{1}\times1&{1}\times1&{1}\times2& 
%             \end{array}}
%      \quad= \argmin_{x\in\omsH}
%             \setn{\begin{array}{c}
%                1\\
%                2\\
%                2\\
%                1\\
%                1
%             \end{array}}
%      \quad= \setn{0,3,4}}
%      \\
%      \ocsEa(\rvZ)
%        &\eqd \argmin_{x\in\omsH}\sum_{y\in\omsH}\ocsmom(x,y)
%        &&\text{by definition of $\ocsEa$ \xref{def:ocsEa}}
%      \\&\eqd \argmin_{x\in\omsH}\max_{y\in\omsH}\metric{x}{y}\psp(y)
%        &&\text{by definition of $\ocsmom$ \xref{def:ocsmom}}
%      \\&= \argmin_{x\in\omsH}\max_{y\in\omsH}\frac{1}{9}\metric{x}{y}\psp(y)9
%      \\&= \argmin_{x\in\omsH}\max_{y\in\omsH}\metric{x}{y}\psp(y)9
%        && \text{because $\fphi(x)=\frac{1}{9}x$ is \prope{strictly isotone} and by \prefp{lem:argminmaxphi}}
%      %\\&= \argmin_{x\in\omsH}\max_{y\in\omsH}
%      %       \setn{\begin{array}{ccccccccc}
%      %         \ocsmom(0,0)&+&\ocsmom(0,1)&+&\ocsmom(0,2)&+&\ocsmom(0,3)&+&\ocsmom(0,4)\\
%      %         \ocsmom(1,0)&+&\ocsmom(1,1)&+&\ocsmom(1,2)&+&\ocsmom(1,3)&+&\ocsmom(1,4)\\
%      %         \ocsmom(2,0)&+&\ocsmom(2,1)&+&\ocsmom(2,2)&+&\ocsmom(2,3)&+&\ocsmom(2,4)\\
%      %         \ocsmom(3,0)&+&\ocsmom(3,1)&+&\ocsmom(3,2)&+&\ocsmom(3,3)&+&\ocsmom(3,4)\\
%      %         \ocsmom(4,0)&+&\ocsmom(4,1)&+&\ocsmom(4,2)&+&\ocsmom(4,3)&+&\ocsmom(4,4)
%      %       \end{array}}
%      %\\&= \argmin_{x\in\omsH}\max_{y\in\omsH}
%      %       \setn{\begin{array}{ccccccccc}
%      %         \metricn(0,0)\frac{2}{9}&+&\metricn(0,1)\frac{1}{9}&+&\metricn(0,2)\frac{1}{9}&+&\metricn(0,3)\frac{2}{9}&+&\metricn(0,4)\frac{3}{9}\\
%      %         \metricn(1,0)\frac{2}{9}&+&\metricn(1,1)\frac{1}{9}&+&\metricn(1,2)\frac{1}{9}&+&\metricn(1,3)\frac{2}{9}&+&\metricn(1,4)\frac{3}{9}\\
%      %         \metricn(2,0)\frac{2}{9}&+&\metricn(2,1)\frac{1}{9}&+&\metricn(2,2)\frac{1}{9}&+&\metricn(2,3)\frac{2}{9}&+&\metricn(2,4)\frac{3}{9}\\
%      %         \metricn(3,0)\frac{2}{9}&+&\metricn(3,1)\frac{1}{9}&+&\metricn(3,2)\frac{1}{9}&+&\metricn(3,3)\frac{2}{9}&+&\metricn(3,4)\frac{3}{9}\\
%      %         \metricn(4,0)\frac{2}{9}&+&\metricn(4,1)\frac{1}{9}&+&\metricn(4,2)\frac{1}{9}&+&\metricn(4,3)\frac{2}{9}&+&\metricn(4,4)\frac{3}{9}\\
%      %       \end{array}}
%      \\&=\mathrlap{\argmin_{x\in\omsH}\max_{y\in\omsH}%\frac{1}{9}
%             \setn{\begin{array}{ccccccccc}
%               {0}\times2&+&{2}\times1&+&{1}\times1&+&{1}\times2&+&{2}\times3\\
%               {2}\times2&+&{0}\times1&+&{2}\times1&+&{1}\times2&+&{1}\times3\\
%               {1}\times2&+&{2}\times1&+&{0}\times1&+&{2}\times2&+&{1}\times3\\
%               {1}\times2&+&{1}\times1&+&{2}\times1&+&{0}\times2&+&{1}\times3\\
%               {2}\times2&+&{1}\times1&+&{1}\times1&+&{1}\times2&+&{0}\times3\\
%             \end{array}}
%      \quad= \argmin_{x\in\omsH}%\frac{1}{9}
%             \setn{\begin{array}{ccccc}
%                11\\
%                11\\
%                11\\
%                 8\\
%                 8
%             \end{array}}
%      \quad= \setn{3,4}}
%      \\
%      \ocsEg(\rvZ)
%        &\eqd \argmin_{x\in\omsH}\prod_{y\in\omsH\setd\setn{x}}{\metric{x}{y}^{\psp(y)}}
%        &&\text{by definition of $\ocsEg$ \xref{def:ocsEg}}
%      \\&= \argmin_{x\in\omsH}\prod_{y\in\omsH\setd\setn{x}}{\metric{x}{y}^{9\psp(y)\frac{1}{9}}}
%      \\&= \argmin_{x\in\omsH}\brs{\prod_{y\in\omsH\setd\setn{x}}{\metric{x}{y}^{9\psp(y)}}}^\frac{1}{9}
%      \\&= \argmin_{x\in\omsH}\prod_{y\in\omsH\setd\setn{x}}{\metric{x}{y}^{9\psp(y)}}
%        && \text{because $\fphi(x)=\frac{1}{9}x$ is \prope{strictly isotone} and by \prefp{lem:argminmaxphi}}
%      %\\&= \argmin_{x\in\omsH}
%      %       \setn{\begin{array}{ccccccccc}
%      %                                 &      & \metric{0}{1}^{\psp(1)} &\times& \metric{0}{2}^{\psp(2)} &\times& \metric{0}{3}^{\psp(3)} &\times& \metric{0}{4}^{\psp(4)}\\
%      %         \metric{1}{0}^{\psp(0)} &      &                         &\times& \metric{1}{2}^{\psp(2)} &\times& \metric{1}{3}^{\psp(3)} &\times& \metric{1}{4}^{\psp(4)}\\
%      %         \metric{2}{0}^{\psp(0)} &\times& \metric{2}{1}^{\psp(1)} &      &                         &\times& \metric{2}{3}^{\psp(3)} &\times& \metric{2}{4}^{\psp(4)}\\
%      %         \metric{3}{0}^{\psp(0)} &\times& \metric{3}{1}^{\psp(1)} &\times& \metric{3}{2}^{\psp(2)} &      &                         &\times& \metric{3}{4}^{\psp(4)}\\
%      %         \metric{4}{0}^{\psp(0)} &\times& \metric{4}{1}^{\psp(1)} &\times& \metric{4}{2}^{\psp(2)} &\times& \metric{4}{3}^{\psp(3)} &      &
%      %       \end{array}}
%      %\\&= \argmin_{x\in\omsH}
%      %       \setn{\begin{array}{ccccccccc}
%      %                                   &      & \metricn(0,1)^\frac{1}{9} &\times& \metricn(0,2)^\frac{1}{9} &\times& \metricn(0,3)^\frac{2}{9} &\times& \metricn(0,4)^\frac{3}{9}\\
%      %         \metricn(1,0)^\frac{2}{9} &      &                           &\times& \metricn(1,2)^\frac{1}{9} &\times& \metricn(1,3)^\frac{2}{9} &\times& \metricn(1,4)^\frac{3}{9}\\
%      %         \metricn(2,0)^\frac{2}{9} &\times& \metricn(2,1)^\frac{1}{9} &      &                           &\times& \metricn(2,3)^\frac{2}{9} &\times& \metricn(2,4)^\frac{3}{9}\\
%      %         \metricn(3,0)^\frac{2}{9} &\times& \metricn(3,1)^\frac{1}{9} &\times& \metricn(3,2)^\frac{1}{9} &      &                           &\times& \metricn(3,4)^\frac{3}{9}\\
%      %         \metricn(4,0)^\frac{2}{9} &\times& \metricn(4,1)^\frac{1}{9} &\times& \metricn(4,2)^\frac{1}{9} &\times& \metricn(4,3)^\frac{2}{9} &      &                          
%      %       \end{array}}
%      \\&=\mathrlap{\argmin_{x\in\omsH}
%             \setn{\begin{array}{ccccccccc}
%                     &      & {2}^1 &\times& {1}^1 &\times& {1}^2 &\times& {2}^3\\
%               {2}^2 &\times&       &      & {2}^1 &\times& {1}^2 &\times& {1}^3\\
%               {1}^2 &\times& {2}^1 &\times&       &      & {2}^2 &\times& {1}^3\\
%               {1}^2 &\times& {1}^1 &\times& {2}^1 &\times&       &      & {1}^3\\
%               {2}^2 &\times& {1}^1 &\times& {1}^1 &\times& {1}^2 &      &
%             \end{array}}
%      \quad= \argmin_{x\in\omsH}
%             \setn{\begin{array}{c}
%                16\\
%                 8\\
%                 8\\
%                 2\\
%                 4
%             \end{array}}
%      \quad= \setn{3}}
%      \\
%      \ocsEh(\rvZ)
%        &\eqd \argmin_{x\in\ocso}\brp{\sum_{y\in\ocso}\frac{1}{\metric{x}{y}}\psp(y)}^{-1} 
%        &&\text{by definition of $\ocsEh$ \xref{def:ocsEh}}
%      \\&=\argmax_{x\in\ocso}\brp{\sum_{y\in\ocso}\frac{1}{\metric{x}{y}}\psp(y)}
%      \\&= \argmax_{x\in\ocso}\brp{\frac{1}{9}\sum_{y\in\ocso}\frac{1}{\metric{x}{y}}\psp(y)9}
%      \\&= \argmax_{x\in\ocso}\sum_{y\in\ocso}\frac{9\psp(y)}{\metric{x}{y}}
%        && \text{because $\fphi(x)=\frac{1}{9}x$ is \prope{strictly isotone} and by \prefp{lem:argminmaxphi}}
%      %\\&= \argmin_{x\in\omsH}
%      %       \setn{\begin{array}{ccccc}
%      %         0                                + \frac{1}{\metric{0}{1}}{\psp(1)} + \frac{1}{\metric{0}{2}}{\psp(2)} + \frac{1}{\metric{0}{3}}{\psp(3)} + \frac{1}{\metric{0}{4}}{\psp(4)}\\
%      %         \frac{1}{\metric{1}{0}}{\psp(0)} + 0                                + \frac{1}{\metric{1}{2}}{\psp(2)} + \frac{1}{\metric{1}{3}}{\psp(3)} + \frac{1}{\metric{1}{4}}{\psp(4)}\\
%      %         \frac{1}{\metric{2}{0}}{\psp(0)} + \frac{1}{\metric{2}{1}}{\psp(1)} + 0                                + \frac{1}{\metric{2}{3}}{\psp(3)} + \frac{1}{\metric{2}{4}}{\psp(4)}\\
%      %         \frac{1}{\metric{3}{0}}{\psp(0)} + \frac{1}{\metric{3}{1}}{\psp(1)} + \frac{1}{\metric{3}{2}}{\psp(2)} + 0                                + \frac{1}{\metric{3}{4}}{\psp(4)}\\
%      %         \frac{1}{\metric{4}{0}}{\psp(0)} + \frac{1}{\metric{4}{1}}{\psp(1)} + \frac{1}{\metric{4}{2}}{\psp(2)} + \frac{1}{\metric{4}{3}}{\psp(3)} + 0                         
%      %       \end{array}}
%      %\\&= \argmin_{x\in\omsH}
%      %       \setn{\begin{array}{ccccc}
%      %         \brs{                            + \frac{1}{2}\cdot\frac{1}{9} + \frac{1}{1}\cdot\frac{1}{9} + \frac{1}{1}\cdot\frac{2}{9} + \frac{1}{2}\cdot\frac{3}{9}}^{-1}\\
%      %         \brs{\frac{1}{2}\cdot\frac{2}{9} + 0                           + \frac{1}{2}\cdot\frac{1}{9} + \frac{1}{1}\cdot\frac{2}{9} + \frac{1}{1}\cdot\frac{3}{9}}^{-1}\\
%      %         \brs{\frac{1}{1}\cdot\frac{2}{9} + \frac{1}{2}\cdot\frac{1}{9} + 0                           + \frac{1}{2}\cdot\frac{2}{9} + \frac{1}{1}\cdot\frac{3}{9}}^{-1}\\
%      %         \brs{\frac{1}{1}\cdot\frac{2}{9} + \frac{1}{1}\cdot\frac{1}{9} + \frac{1}{2}\cdot\frac{1}{9} + 0                           + \frac{1}{1}\cdot\frac{3}{9}}^{-1}\\
%      %         \brs{\frac{1}{2}\cdot\frac{2}{9} + \frac{1}{1}\cdot\frac{1}{9} + \frac{1}{1}\cdot\frac{1}{9} + \frac{1}{1}\cdot\frac{2}{9} + 0                          }^{-1}
%      %       \end{array}}
%      %\\&= \argmin_{x\in\omsH}9
%      %       \setn{\begin{array}{ccccc}
%      %         \brs{0           + \frac{1}{2} + \frac{1}{1} + \frac{2}{1} + \frac{3}{2}}^{-1}\\
%      %         \brs{\frac{2}{2} + 0           + \frac{1}{2} + \frac{2}{1} + \frac{3}{1}}^{-1}\\
%      %         \brs{\frac{2}{1} + \frac{1}{2} + 0           + \frac{2}{2} + \frac{3}{1}}^{-1}\\
%      %         \brs{\frac{2}{1} + \frac{1}{1} + \frac{1}{2} + 0           + \frac{3}{1}}^{-1}\\
%      %         \brs{\frac{2}{2} + \frac{1}{1} + \frac{1}{1} + \frac{2}{1} + 0          }^{-1}
%      %       \end{array}}
%      %\\&= \argmin_{x\in\omsH}9
%      %       \setn{\begin{array}{c}
%      %         \brs{\frac{10}{2}}^{-1}\\
%      %         \brs{\frac{13}{2}}^{-1}\\
%      %         \brs{\frac{13}{2}}^{-1}\\
%      %         \brs{\frac{13}{2}}^{-1}\\
%      %         \brs{\frac{11}{2}}^{-1}
%      %       \end{array}}
%      \\&=\mathrlap{\argmax_{x\in\omsH}
%             \setn{\begin{array}{*{4}{cc}c}
%               0           &+& \frac{1}{2} &+& \frac{1}{1} &+& \frac{2}{1} &+& \frac{3}{2}\\
%               \frac{2}{2} &+& 0           &+& \frac{1}{2} &+& \frac{2}{1} &+& \frac{3}{1}\\
%               \frac{2}{1} &+& \frac{1}{2} &+& 0           &+& \frac{2}{2} &+& \frac{3}{1}\\
%               \frac{2}{1} &+& \frac{1}{1} &+& \frac{1}{2} &+& 0           &+& \frac{3}{1}\\
%               \frac{2}{2} &+& \frac{1}{1} &+& \frac{1}{1} &+& \frac{2}{1} &+& 0          
%             \end{array}}
%      \quad= \argmax_{x\in\omsH}\frac{1}{2}
%             \setn{\begin{array}{c}
%               {10}\\
%               {13}\\
%               {13}\\
%               {13}\\
%               {11}
%             \end{array}}
%      \quad= \setn{1,2,3}}
    \end{align*}
\end{proof}
%
\begin{figure}[h]
  \gsize%
  \centering%
  %%============================================================================
% Daniel J. Greenhoe
% LaTeX file
% linear congruential (LCG) pseudo-random number generator (PRNG) mappings
% x_{n+1} = (7x_n+5)mod 9
% y_{n+1} = (y_n+2)mod 5
%============================================================================
\begin{pspicture}(-2.25,-2.25)(11.5,2.25)%
  %---------------------------------
  % options
  %---------------------------------
  \psset{%
    radius=1.25ex,
    labelsep=2.5mm,
    linecolor=blue,%
    }%
  %---------------------------------
  % LCG PRNG graph G
  % x_{n+1} = (7x_n+5)mod 9
  %   n  0   1   2   3   4   5   6   7   8  ;  9
  % x_n  1   3   8   7   0   5   4   6   2  ;  1
  %---------------------------------
  \rput(0,0){\psset{unit=1.5\psunit}%
    \rput{   0}(0,0){\rput(1,0){\Cnode[fillstyle=solid,fillcolor=snode](0,0){G1}}}%
    \rput{  40}(0,0){\rput(1,0){\Cnode[fillstyle=solid,fillcolor=snode](0,0){G3}}}%
    \rput{  80}(0,0){\rput(1,0){\Cnode[fillstyle=solid,fillcolor=snode](0,0){G8}}}%
    \rput{ 120}(0,0){\rput(1,0){\Cnode[fillstyle=solid,fillcolor=snode](0,0){G7}}}%
    \rput{ 160}(0,0){\rput(1,0){\Cnode[fillstyle=solid,fillcolor=snode](0,0){G0}}}%
    \rput{ 200}(0,0){\rput(1,0){\Cnode[fillstyle=solid,fillcolor=snode](0,0){G5}}}%
    \rput{ 240}(0,0){\rput(1,0){\Cnode[fillstyle=solid,fillcolor=snode](0,0){G4}}}%
    \rput{ 280}(0,0){\rput(1,0){\Cnode[fillstyle=solid,fillcolor=snode](0,0){G6}}}%
    \rput{ 320}(0,0){\rput(1,0){\Cnode[fillstyle=solid,fillcolor=snode](0,0){G2}}}%
    \rput(0,0){$\ocsG_9$}%
    }%
  \rput(G8){$8$}%
  \rput(G7){$7$}%
  \rput(G6){$6$}%
  \rput(G5){$5$}%
  \rput(G4){$4$}%
  \rput(G3){$3$}%
  \rput(G2){$2$}%
  \rput(G1){$1$}%
  \rput(G0){$0$}%
  %          
  \ncline{G1}{G3}%
  \ncline{G3}{G8}%
  \ncline{G8}{G7}%
  \ncline{G7}{G0}%
  \ncline{G0}{G5}%
  \ncline{G5}{G4}%
  \ncline{G4}{G6}%
  \ncline{G6}{G2}%
  \ncline{G2}{G1}%
  %
  \uput[  0](G1){$\frac{1}{9}$}
  \uput[ 40](G3){$\frac{1}{9}$}
  \uput[ 80](G8){$\frac{1}{9}$}
  \uput[120](G7){$\frac{1}{9}$}
  \uput[160](G0){$\frac{1}{9}$}
  \uput[200](G5){$\frac{1}{9}$}
  \uput[240](G4){$\frac{1}{9}$}
  \uput[280](G6){$\frac{1}{9}$}
  \uput[320](G2){$\frac{1}{9}$}
  %
  %
  %---------------------------------
  %---------------------------------
  %---------------------------------
  \rput(5,0){% shaped outcomes spaces
  %---------------------------------
  %---------------------------------
  %---------------------------------
  %
  %
  %---------------------------------
  % distribution shaping mapping y_n=s(x_n) to 5 element set
  %   n  0   1   2   3   4   5   6   7   8  ;  9
  % x_n  1   3   8   7   0   5   4   6   2  ;  1
  % y_n  1   3   0   2   4 ; 0   3   4   4  ;  1
  %---------------------------------
  \rput( 0,0){%\psset{unit=2\psunit}%
    \rput{288}(0,0){\rput(1,0){\Cnode(0,0){S4}}}%
    \rput{216}(0,0){\rput(1,0){\Cnode(0,0){S2}}}%
    \rput{144}(0,0){\rput(1,0){\Cnode(0,0){S0}}}%
    \rput{ 72}(0,0){\rput(1,0){\Cnode[fillstyle=solid,fillcolor=snode](0,0){S3}}}%
    \rput{  0}(0,0){\rput(1,0){\Cnode(0,0){S1}}}%
    \rput(0,0){$\ocsG$}%
    }
  \rput(S4){$4$}%
  \rput(S3){$3$}%
  \rput(S2){$2$}%
  \rput(S1){$1$}%
  \rput(S0){$0$}%
  %
  %\ncline{S4}{S0}
  \ncline{S4}{S1}\ncline{S3}{S4}%
  \ncline{S2}{S4}%
  \ncline{S0}{S2}%
  \ncline{S3}{S0}%
  \ncline{S1}{S3}%
  %
  \uput[288](S4){$\frac{3}{9}$}
  \uput[ 72](S3){$\frac{2}{9}$}
  \uput[216](S2){$\frac{1}{9}$}
  \uput[  0](S1){$\frac{1}{9}$}
  \uput[144](S0){$\frac{2}{9}$}
  %
  \ncarc[arcangle= 22,linewidth=0.75pt,linecolor=red]{->}{G8}{S0}%
  \ncarc[arcangle= 67,linewidth=0.75pt,linecolor=red]{->}{G7}{S2}%
  \ncarc[arcangle=-45,linewidth=0.75pt,linecolor=red]{->}{G6}{S4}%
  \ncarc[arcangle=-22,linewidth=0.75pt,linecolor=red]{->}{G5}{S0}%
  \ncarc[arcangle=-22,linewidth=0.75pt,linecolor=red]{->}{G4}{S3}%
  \ncarc[arcangle= 22,linewidth=0.75pt,linecolor=red]{->}{G3}{S3}%
  \ncarc[arcangle=-22,linewidth=0.75pt,linecolor=red]{->}{G2}{S4}%
  \ncarc[arcangle= 10,linewidth=0.75pt,linecolor=red]{->}{G1}{S1}%
  \ncarc[arcangle= 22,linewidth=0.75pt,linecolor=red]{->}{G0}{S4}%
  %%---------------------------------
  %% random variable mapping Y from S to integer line 
  %%---------------------------------
  %\rput(0,3){%\psset{unit=0.75\psunit}%
  %  \pnode(3,0){YB}%
  %  \Cnode(2,0){Y4}%
  %  \Cnode[fillstyle=solid,fillcolor=snode](1,0){Y3}%
  %  \Cnode[fillstyle=solid,fillcolor=snode](0,0){Y2}%
  %  \Cnode(-1,0){Y1}%
  %  \Cnode(-2,0){Y0}%
  %  \pnode(-3,0){YA}%
  %  }
  %\rput(Y4){$4$}%
  %\rput(Y3){$3$}%
  %\rput(Y2){$2$}%
  %\rput(Y1){$1$}%
  %\rput(Y0){$0$}%
  %%
  %\ncline[linestyle=dotted]{Y4}{YB}%
  %\ncline{Y3}{Y4}%
  %\ncline{Y2}{Y3}%
  %\ncline{Y1}{Y2}%
  %\ncline{Y0}{Y1}%
  %\ncline[linestyle=dotted]{Y0}{YA}%
  %%
  %\uput[ 0](YB){$\Z$}
  %\uput[90](Y4){$\frac{3}{9}$}
  %\uput[90](Y3){$\frac{2}{9}$}
  %\uput[90](Y2){$\frac{1}{9}$}
  %\uput[90](Y1){$\frac{1}{9}$}
  %\uput[90](Y0){$\frac{2}{9}$}
  %%
  %\ncarc[arcangle=-45,linewidth=0.75pt,linecolor=red]{->}{S4}{Y4}%
  %\ncarc[arcangle= 22,linewidth=0.75pt,linecolor=red]{->}{S3}{Y3}%
  %\ncarc[arcangle= 45,linewidth=0.75pt,linecolor=red]{->}{S2}{Y2}%
  %\ncarc[arcangle=-22,linewidth=0.75pt,linecolor=red]{->}{S1}{Y1}%
  %\ncarc[arcangle= 22,linewidth=0.75pt,linecolor=red]{->}{S0}{Y0}%
  %---------------------------------
  % random variable mapping Z from S to ring-like structure 
  %---------------------------------
  \rput( 5,0){%\psset{unit=2\psunit}%
    \rput{288}(0,0){\rput(1,0){\Cnode(0,0){Z4}}}%
    \rput{216}(0,0){\rput(1,0){\Cnode(0,0){Z2}}}%
    \rput{144}(0,0){\rput(1,0){\Cnode(0,0){Z0}}}%
    \rput{ 72}(0,0){\rput(1,0){\Cnode[fillstyle=solid,fillcolor=snode](0,0){Z3}}}%
    \rput{  0}(0,0){\rput(1,0){\Cnode(0,0){Z1}}}%
    \rput(0,0){$\omsH$}%
    }
  \rput(Z4){$4$}%
  \rput(Z3){$3$}%
  \rput(Z2){$2$}%
  \rput(Z1){$1$}%
  \rput(Z0){$0$}%
  %
  %\ncline{Z4}{Z0}
  \ncline{Z4}{Z1}\ncline{Z2}{Z4}\ncline{Z3}{Z4}%
  \ncline{Z2}{Z4}%
  \ncline{Z0}{Z2}%
  \ncline{Z3}{Z0}%
  \ncline{Z1}{Z3}%
  %
  \uput[288](Z4){$\frac{3}{9}$}%
  \uput[ 72](Z3){$\frac{2}{9}$}%
  \uput[216](Z2){$\frac{1}{9}$}%
  \uput[  0](Z1){$\frac{1}{9}$}%
  \uput[144](Z0){$\frac{2}{9}$}%
  %
  \ncarc[arcangle=-22,linewidth=0.75pt,linecolor=green]{->}{S4}{Z4}%
  \ncarc[arcangle= 22,linewidth=0.75pt,linecolor=green]{->}{S3}{Z3}%
  \ncarc[arcangle=-10,linewidth=0.75pt,linecolor=green]{->}{S2}{Z2}%
  \ncarc[arcangle= 10,linewidth=0.75pt,linecolor=green]{->}{S1}{Z1}%
  \ncarc[arcangle=-10,linewidth=0.75pt,linecolor=green]{->}{S0}{Z0}%
  %%---------------------------------
  %% random variable mapping X from S to real line
  %%---------------------------------
  %\rput(0,-3){%\psset{unit=0.5\psunit}%
  %  \psline{<->}(-2.5,0)(2.5,0)%
  %  \multirput(-2,0)(1,0){5}{\psline(0,-0.1)(0,0.1)}%
  %  \pnode(2.5,0){XB}%
  %  \pnode( 2,0){X4}%
  %  \pnode( 1,0){X3}%
  %  \pnode(0.4,0){X240}%
  %  \pnode(0.33,0){X233}%
  %  \pnode( 0,0){X2}%
  %  \pnode(-1,0){X1}%
  %  \pnode(-2,0){X0}%
  %  \pnode(-2.5,0){XA}%
  %  }
  %%
  %\rput(X240){\pscircle[fillstyle=solid,linecolor=snode,fillcolor=snode](0,0){1ex}}%
  %\rput(X233){\pscircle[fillstyle=none,linecolor=red,fillcolor=red](0,0){1ex}}%
  %%
  %\uput{1.5mm}[-90](X4){$\frac{3}{9}$}%
  %\uput{1.5mm}[-90](X3){$\frac{2}{9}$}%
  %\uput{1.5mm}[-90](X2){$\frac{1}{9}$}%
  %\uput{1.5mm}[-90](X1){$\frac{1}{9}$}%
  %\uput{1.5mm}[-90](X0){$\frac{2}{9}$}%
  %%
  %\uput[0](XB){$\R$}%
  %\uput{1.5mm}[90](X4){$4$}%
  %\uput{1.5mm}[90](X3){$3$}%
  %\uput{1.5mm}[90](X2){$2$}%
  %\uput{1.5mm}[90](X1){$1$}%
  %\uput{1.5mm}[90](X0){$0$}%
  %%
  %\ncarc[arcangle= 22,linewidth=0.75pt,linecolor=purple]{->}{S4}{X4}%
  %\ncarc[arcangle= 67,linewidth=0.75pt,linecolor=purple]{->}{S3}{X3}%
  %\ncarc[arcangle=-22,linewidth=0.75pt,linecolor=purple]{->}{S2}{X2}%
  %\ncarc[arcangle= 45,linewidth=0.75pt,linecolor=purple]{->}{S1}{X1}%
  %\ncarc[arcangle=-22,linewidth=0.75pt,linecolor=purple]{->}{S0}{X0}%
  %---------------------------------
  % labels
  %---------------------------------
  %\rput(0,1.9){$\rvY(\cdot)$}%
  %\rput(0,-1.9){$\rvX(\cdot)$}%
  \rput(2.75,0){$\rvZ(\cdot)$}%
  %\ncline[linestyle=dotted,nodesep=1pt]{->}{xzlabel}{xz}%
  %\ncline[linestyle=dotted,nodesep=1pt]{->}{ylabel}{y}%
  }%
  \rput(2.75,0){$\fs(x_n)$}%
\end{pspicture}%%
  \includegraphics{../common/math/graphics/pdfs/lcg7x1m9_seqorder.pdf}%
  \caption{sequentially ordered LCG mappings \xref{ex:lcg7x1m9_seqorder}\label{fig:lcg7x1m9_seqorder}}
\end{figure}
%---------------------------------------
%\begin{minipage}{\tw-65mm}%
\begin{example}[\exmd{LCG mappings, sequential ordering}]
\label{ex:lcg7x1m9_seqorder}\mbox{}\\
%---------------------------------------
In \prefpp{ex:lcg7x1m9_xyz}, the structures $\ocsG_9$, $\ocsG$, and $\omsH$ were ordered as a standard ring of 
integers ($0<1<2<\cdots<7<8<0$ for $\ocsG_9$).
In this current example, as illustrated in \prefpp{fig:lcg7x1m9_seqorder}, these structures are ordered as they appear in 
the sequences generated by $x_{n+1}=(7x_n+5)\mod9$ and $\fs$ (see \pref{ex:lcg7x1m9_xyz} for sequence description).
This yields the following statistics:
\\$\quad\begin{array}{>{\gsize}Mrcc rcccl}
  geometry of $\ocsG_9$:                                & \ocscen (\ocsG) &=& \mc{4}{l}{\setn{0,1,2,3,4,5,6,7,8}}  \\
  geometry of $\ocsG$:                                  & \ocscen (\ocsG) &=& \setn{3}  \\
  outcome subspace statistics on isomorphic structure:  & \ocsE   (\rvZ)  &=& \setn{3}   & \ocsVar(\rvZ;\ocsE) &=& \frac{10}{9} &\approx& 1.111
\end{array}$\\
Note that a change in ordering structure (from standard ring ordering to sequential ordering) yields a change in statistics
($\ocsE(\rvZ)=\setn{3}$ as opposed to $\ocsE(\rvZ)=\setn{4}$).
Intuitively, the sequential ordering of \pref{ex:lcg7x1m9_seqorder} should yield a better estimate
than that of \pref{ex:lcg7x1m9_xyz}, because it more closely matches the way the PRNG produces a sequence.
This intuition is also supported by the variance values 
($\ocsVar(\rvZ)=\sfrac{12}{9}$ for standard ring ordering, $\ocsVar(\rvZ)=\frac{10}{9}$ for sequential ordering).
However, counterintuitively, the sequential ordering no longer yields the maximally likely result of $\setn{4}$.
\end{example}
%\end{minipage}\hfill%
%\begin{tabular}{c}
%  \gsize%
%  %\psset{unit=5mm}%
%  {%============================================================================
% Daniel J. Greenhoe
% LaTeX file
% discrete metric real dice mapping to linearly ordered O6c
%============================================================================
{%\psset{unit=0.5\psunit}%
\begin{pspicture}(-3.25,-1.5)(3.25,1.5)%
  %---------------------------------
  % options
  %---------------------------------
  \psset{%
    radius=1.25ex,
    labelsep=2.5mm,
    linecolor=blue,%
    }%
  %---------------------------------
  % dice graph
  %---------------------------------
  \rput(-1.75,0){%\psset{unit=2\psunit}%
    \Cnode[fillstyle=solid,fillcolor=snode](-0.8660,-0.5){D4}%
    \Cnode(-0.8660,0.5){D5}%
    \Cnode(0,1){D6}%
    \Cnode(0.8660,0.5){D3}%
    \Cnode(0.8660,-0.5){D2}%
    \Cnode(0,-1){D1}%
    }
  \rput(D6){$\diceF$}%
  \rput(D5){$\diceE$}%
  \rput(D4){$\diceD$}%
  \rput(D3){$\diceC$}%
  \rput(D2){$\diceB$}%
  \rput(D1){$\diceA$}%
  %
  \ncline{D5}{D6}%
  \ncline{D4}{D5}\ncline{D4}{D6}%
  \ncline{D3}{D5}\ncline{D3}{D6}%
  \ncline{D2}{D3}\ncline{D2}{D4}\ncline{D2}{D6}%
  \ncline{D1}{D2}\ncline{D1}{D3}\ncline{D1}{D4}\ncline{D1}{D5}%
  %
  \uput[ 158](D6){$\frac{1}{30}$}
  \uput[ 150](D5){$\frac{1}{50}$}
  \uput[ 210](D4){$\frac{3}{5}$}
  \uput[  22](D3){$\frac{1}{30}$}
  \uput[ -45](D2){$\frac{1}{20}$}
  \uput[-158](D1){$\frac{1}{10}$}
  %---------------------------------
  % range graph
  %---------------------------------
  \rput(1.75,0){%\psset{unit=2\psunit}%
    \Cnode[fillstyle=solid,fillcolor=snode](-0.8660,-0.5){E4}%
    \Cnode(-0.8660,0.5){E5}%
    \Cnode(0,1){E6}%
    \Cnode(0.8660,0.5){E3}%
    \Cnode(0.8660,-0.5){E2}%
    \Cnode(0,-1){E1}%
    \Cnode(0,0){E0}%
    }
  \rput(E6){$6$}%
  \rput(E5){$5$}%
  \rput(E4){$4$}%
  \rput(E3){$3$}%
  \rput(E2){$2$}%
  \rput(E1){$1$}%
  \rput(E0){$0$}%
  %
  \ncline{E5}{E6}%
  \ncline{E4}{E5}\ncline{E4}{E6}%
  \ncline{E3}{E5}\ncline{E3}{E6}%
  \ncline{E2}{E3}\ncline{E2}{E4}\ncline{E2}{E6}%
  \ncline{E1}{E2}\ncline{E1}{E3}\ncline{E1}{E4}\ncline{E1}{E5}%
  \ncline{E0}{E1}\ncline{E0}{E2}\ncline{E0}{E3}\ncline{E0}{E4}\ncline{E0}{E5}\ncline{E0}{E6}%
  %
  \uput[ 158](E6){$\frac{1}{30}$}
  \uput[ 150](E5){$\frac{1}{50}$}
  \uput[ 210](E4){$\frac{3}{5}$}
  \uput[  22](E3){$\frac{1}{30}$}
  \uput[ -45](E2){$\frac{1}{20}$}
  \uput[-158](E1){$\frac{1}{10}$}
  \uput[  0](E0){$\sfrac{0}{6}$}
  %---------------------------------
  % mapping from die to O6c
  %---------------------------------
  \ncarc[arcangle= 22,linewidth=0.75pt,linecolor=red]{->}{D6}{E6}%
  \ncarc[arcangle= 22,linewidth=0.75pt,linecolor=red]{->}{D5}{E5}%
  \ncarc[arcangle= 22,linewidth=0.75pt,linecolor=red]{->}{D4}{E4}%
  \ncarc[arcangle=-22,linewidth=0.75pt,linecolor=red]{->}{D3}{E3}%
  \ncarc[arcangle=-22,linewidth=0.75pt,linecolor=red]{->}{D2}{E2}%
  \ncarc[arcangle=-22,linewidth=0.75pt,linecolor=red]{->}{D1}{E1}%
  %---------------------------------
  % labels
  %---------------------------------
  \rput(0,0){$\rvX(\cdot)$}%
  %\ncline[linestyle=dotted,nodesep=1pt]{->}{xzlabel}{xz}%
  %\ncline[linestyle=dotted,nodesep=1pt]{->}{ylabel}{y}%
\end{pspicture}
}%}%
%\end{tabular}
\begin{proof}
\begin{align*}
  \ocscen(\ocsG_9)
  \\&= \setn{0,1,2,\cdots,8}
    && \text{by \exme{LCG mappings standard ordering} example \xref{ex:lcg7x1m9_xyz}}
  \\
  \ocscen(\ocsG)
    &= \setn{3}
    &&\text{by \prefpp{ex:wring5short}}
  \\
  \ocsE(\rvZ)
    &= \rvZ\brs{\ocscen(\ocsG)}
    && \text{because $\ocsG$ and $\omsH$ are \prope{isomorphic} under $\rvZ$}
  \\&= \rvZ\brs{\setn{3}}
    && \text{by $\ocscen(\ocsG)$ result}
  \\&= \setn{3}
    && \text{by definition of $\rvZ$}
  \\
  \ocsVar(\rvZ;\ocsE)
    &= \ocsVaro(\ocsG)
    && \text{because $\ocsG$ and $\omsH$ are \prope{isomorphic} under $\rvZ$}\
  \\&= \frac{10}{9} \approx 1.111
    &&\text{by \prefpp{ex:wring5short}}
\end{align*}
\end{proof}

\begin{figure}[h]
  \gsize%
  \centering%
  %%============================================================================
% Daniel J. Greenhoe
% LaTeX file
% linear congruential (LCG) pseudo-random number generator (PRNG) mappings
% x_{n+1} = (7x_n+5)mod 9
% y_{n+1} = (y_n+2)mod 5
%============================================================================
\begin{pspicture}(-2.25,-2.3)(11.5,2.3)%
  %---------------------------------
  % options
  %---------------------------------
  \psset{%
    radius=1.25ex,
    labelsep=2.5mm,
    linecolor=blue,%
    }%
  %---------------------------------
  % LCG PRNG graph G
  % x_{n+1} = (7x_n+5)mod 9
  %   n  0   1   2   3   4   5   6   7   8  ;  9
  % x_n  1   3   8   7   0   5   4   6   2  ;  1
  %---------------------------------
  \rput(0,0){\psset{unit=1.5\psunit}%
    \rput{   0}(0,0){\rput(1,0){\Cnode[fillstyle=solid,fillcolor=snode](0,0){G1}}}%
    \rput{  40}(0,0){\rput(1,0){\Cnode[fillstyle=solid,fillcolor=snode](0,0){G3}}}%
    \rput{  80}(0,0){\rput(1,0){\Cnode[fillstyle=solid,fillcolor=snode](0,0){G8}}}%
    \rput{ 120}(0,0){\rput(1,0){\Cnode[fillstyle=solid,fillcolor=snode](0,0){G7}}}%
    \rput{ 160}(0,0){\rput(1,0){\Cnode[fillstyle=solid,fillcolor=snode](0,0){G0}}}%
    \rput{ 200}(0,0){\rput(1,0){\Cnode[fillstyle=solid,fillcolor=snode](0,0){G5}}}%
    \rput{ 240}(0,0){\rput(1,0){\Cnode[fillstyle=solid,fillcolor=snode](0,0){G4}}}%
    \rput{ 280}(0,0){\rput(1,0){\Cnode[fillstyle=solid,fillcolor=snode](0,0){G6}}}%
    \rput{ 320}(0,0){\rput(1,0){\Cnode[fillstyle=solid,fillcolor=snode](0,0){G2}}}%
    \rput(0,0){$\ocsG_9$}%
    }%
  \rput(G8){$8$}%
  \rput(G7){$7$}%
  \rput(G6){$6$}%
  \rput(G5){$5$}%
  \rput(G4){$4$}%
  \rput(G3){$3$}%
  \rput(G2){$2$}%
  \rput(G1){$1$}%
  \rput(G0){$0$}%
  %          
  \ncline{->}{G1}{G3}%
  \ncline{->}{G3}{G8}%
  \ncline{->}{G8}{G7}%
  \ncline{->}{G7}{G0}%
  \ncline{->}{G0}{G5}%
  \ncline{->}{G5}{G4}%
  \ncline{->}{G4}{G6}%
  \ncline{->}{G6}{G2}%
  \ncline{->}{G2}{G1}%
  %
  \uput[  0](G1){$\frac{1}{9}$}
  \uput[ 40](G3){$\frac{1}{9}$}
  \uput[ 80](G8){$\frac{1}{9}$}
  \uput[120](G7){$\frac{1}{9}$}
  \uput[160](G0){$\frac{1}{9}$}
  \uput[200](G5){$\frac{1}{9}$}
  \uput[240](G4){$\frac{1}{9}$}
  \uput[280](G6){$\frac{1}{9}$}
  \uput[320](G2){$\frac{1}{9}$}
  %
  %
  %---------------------------------
  %---------------------------------
  %---------------------------------
  \rput(5,0){% shaped outcomes spaces
  %---------------------------------
  %---------------------------------
  %---------------------------------
  %
  %
  %---------------------------------
  % distribution shaping mapping y_n=s(x_n) to 5 element set
  %   n  0   1   2   3   4   5   6   7   8  ;  9
  % x_n  1   3   8   7   0   5   4   6   2  ;  1
  % y_n  1   3   0   2   4 ; 0   3   4   4  ;  1
  %---------------------------------
  \rput( 0,0){%\psset{unit=2\psunit}%
    \rput{288}(0,0){\rput(1,0){\Cnode(0,0){S4}}}%
    \rput{216}(0,0){\rput(1,0){\Cnode(0,0){S2}}}%
    \rput{144}(0,0){\rput(1,0){\Cnode(0,0){S0}}}%
    \rput{ 72}(0,0){\rput(1,0){\Cnode[fillstyle=solid,fillcolor=snode](0,0){S3}}}%
    \rput{  0}(0,0){\rput(1,0){\Cnode(0,0){S1}}}%
    \rput(0,0){$\ocsG$}%
    }
  \rput(S4){$4$}%
  \rput(S3){$3$}%
  \rput(S2){$2$}%
  \rput(S1){$1$}%
  \rput(S0){$0$}%
  %
  %\ncline{S4}{S0}
  \ncline{->}{S4}{S1}\ncline{->}{S2}{S4}\ncline{->}{S3}{S4}%
  \ncline{->}{S2}{S4}%
  \ncline{->}{S0}{S2}%
  \ncline{->}{S3}{S0}%
  \ncline{->}{S1}{S3}%
  %
  \uput[288](S4){$\frac{3}{9}$}
  \uput[ 72](S3){$\frac{2}{9}$}
  \uput[216](S2){$\frac{1}{9}$}
  \uput[  0](S1){$\frac{1}{9}$}
  \uput[144](S0){$\frac{2}{9}$}
  %
  \ncarc[arcangle= 22,linewidth=0.75pt,linecolor=red]{->}{G8}{S0}%
  \ncarc[arcangle= 67,linewidth=0.75pt,linecolor=red]{->}{G7}{S2}%
  \ncarc[arcangle=-45,linewidth=0.75pt,linecolor=red]{->}{G6}{S4}%
  \ncarc[arcangle=-22,linewidth=0.75pt,linecolor=red]{->}{G5}{S0}%
  \ncarc[arcangle=-22,linewidth=0.75pt,linecolor=red]{->}{G4}{S3}%
  \ncarc[arcangle= 22,linewidth=0.75pt,linecolor=red]{->}{G3}{S3}%
  \ncarc[arcangle=-22,linewidth=0.75pt,linecolor=red]{->}{G2}{S4}%
  \ncarc[arcangle= 10,linewidth=0.75pt,linecolor=red]{->}{G1}{S1}%
  \ncarc[arcangle= 22,linewidth=0.75pt,linecolor=red]{->}{G0}{S4}%
  %%---------------------------------
  %% random variable mapping Y from S to integer line 
  %%---------------------------------
  %\rput(0,3){%\psset{unit=0.75\psunit}%
  %  \pnode(3,0){YB}%
  %  \Cnode(2,0){Y4}%
  %  \Cnode[fillstyle=solid,fillcolor=snode](1,0){Y3}%
  %  \Cnode[fillstyle=solid,fillcolor=snode](0,0){Y2}%
  %  \Cnode(-1,0){Y1}%
  %  \Cnode(-2,0){Y0}%
  %  \pnode(-3,0){YA}%
  %  }
  %\rput(Y4){$4$}%
  %\rput(Y3){$3$}%
  %\rput(Y2){$2$}%
  %\rput(Y1){$1$}%
  %\rput(Y0){$0$}%
  %%
  %\ncline[linestyle=dotted]{Y4}{YB}%
  %\ncline{Y3}{Y4}%
  %\ncline{Y2}{Y3}%
  %\ncline{Y1}{Y2}%
  %\ncline{Y0}{Y1}%
  %\ncline[linestyle=dotted]{Y0}{YA}%
  %%
  %\uput[ 0](YB){$\Z$}
  %\uput[90](Y4){$\frac{3}{9}$}
  %\uput[90](Y3){$\frac{2}{9}$}
  %\uput[90](Y2){$\frac{1}{9}$}
  %\uput[90](Y1){$\frac{1}{9}$}
  %\uput[90](Y0){$\frac{2}{9}$}
  %%
  %\ncarc[arcangle=-45,linewidth=0.75pt,linecolor=red]{->}{S4}{Y4}%
  %\ncarc[arcangle= 22,linewidth=0.75pt,linecolor=red]{->}{S3}{Y3}%
  %\ncarc[arcangle= 45,linewidth=0.75pt,linecolor=red]{->}{S2}{Y2}%
  %\ncarc[arcangle=-22,linewidth=0.75pt,linecolor=red]{->}{S1}{Y1}%
  %\ncarc[arcangle= 22,linewidth=0.75pt,linecolor=red]{->}{S0}{Y0}%
  %---------------------------------
  % random variable mapping Z from S to ring-like structure 
  %---------------------------------
  \rput( 5,0){%\psset{unit=2\psunit}%
    \rput{288}(0,0){\rput(1,0){\Cnode(0,0){Z4}}}%
    \rput{216}(0,0){\rput(1,0){\Cnode(0,0){Z2}}}%
    \rput{144}(0,0){\rput(1,0){\Cnode(0,0){Z0}}}%
    \rput{ 72}(0,0){\rput(1,0){\Cnode[fillstyle=solid,fillcolor=snode](0,0){Z3}}}%
    \rput{  0}(0,0){\rput(1,0){\Cnode(0,0){Z1}}}%
    \rput(0,0){$\omsH$}%
    }
  \rput(Z4){$4$}%
  \rput(Z3){$3$}%
  \rput(Z2){$2$}%
  \rput(Z1){$1$}%
  \rput(Z0){$0$}%
  %
  %\ncline{Z4}{Z0}
  \ncline{->}{Z4}{Z1}\ncline{->}{Z2}{Z4}\ncline{->}{Z3}{Z4}%
  \ncline{->}{Z2}{Z4}%
  \ncline{->}{Z0}{Z2}%
  \ncline{->}{Z3}{Z0}%
  \ncline{->}{Z1}{Z3}%
  %
  \uput[288](Z4){$\frac{3}{9}$}%
  \uput[ 72](Z3){$\frac{2}{9}$}%
  \uput[216](Z2){$\frac{1}{9}$}%
  \uput[  0](Z1){$\frac{1}{9}$}%
  \uput[144](Z0){$\frac{2}{9}$}%
  %
  \ncarc[arcangle=-22,linewidth=0.75pt,linecolor=green]{->}{S4}{Z4}%
  \ncarc[arcangle= 22,linewidth=0.75pt,linecolor=green]{->}{S3}{Z3}%
  \ncarc[arcangle=-10,linewidth=0.75pt,linecolor=green]{->}{S2}{Z2}%
  \ncarc[arcangle= 10,linewidth=0.75pt,linecolor=green]{->}{S1}{Z1}%
  \ncarc[arcangle=-10,linewidth=0.75pt,linecolor=green]{->}{S0}{Z0}%
  %%---------------------------------
  %% random variable mapping X from S to real line
  %%---------------------------------
  %\rput(0,-3){%\psset{unit=0.5\psunit}%
  %  \psline{<->}(-2.5,0)(2.5,0)%
  %  \multirput(-2,0)(1,0){5}{\psline(0,-0.1)(0,0.1)}%
  %  \pnode(2.5,0){XB}%
  %  \pnode( 2,0){X4}%
  %  \pnode( 1,0){X3}%
  %  \pnode(0.4,0){X240}%
  %  \pnode(0.33,0){X233}%
  %  \pnode( 0,0){X2}%
  %  \pnode(-1,0){X1}%
  %  \pnode(-2,0){X0}%
  %  \pnode(-2.5,0){XA}%
  %  }
  %%
  %\rput(X240){\pscircle[fillstyle=solid,linecolor=snode,fillcolor=snode](0,0){1ex}}%
  %\rput(X233){\pscircle[fillstyle=none,linecolor=red,fillcolor=red](0,0){1ex}}%
  %%
  %\uput{1.5mm}[-90](X4){$\frac{3}{9}$}%
  %\uput{1.5mm}[-90](X3){$\frac{2}{9}$}%
  %\uput{1.5mm}[-90](X2){$\frac{1}{9}$}%
  %\uput{1.5mm}[-90](X1){$\frac{1}{9}$}%
  %\uput{1.5mm}[-90](X0){$\frac{2}{9}$}%
  %%
  %\uput[0](XB){$\R$}%
  %\uput{1.5mm}[90](X4){$4$}%
  %\uput{1.5mm}[90](X3){$3$}%
  %\uput{1.5mm}[90](X2){$2$}%
  %\uput{1.5mm}[90](X1){$1$}%
  %\uput{1.5mm}[90](X0){$0$}%
  %%
  %\ncarc[arcangle= 22,linewidth=0.75pt,linecolor=purple]{->}{S4}{X4}%
  %\ncarc[arcangle= 67,linewidth=0.75pt,linecolor=purple]{->}{S3}{X3}%
  %\ncarc[arcangle=-22,linewidth=0.75pt,linecolor=purple]{->}{S2}{X2}%
  %\ncarc[arcangle= 45,linewidth=0.75pt,linecolor=purple]{->}{S1}{X1}%
  %\ncarc[arcangle=-22,linewidth=0.75pt,linecolor=purple]{->}{S0}{X0}%
  %---------------------------------
  % labels
  %---------------------------------
  %\rput(0,1.9){$\rvY(\cdot)$}%
  %\rput(0,-1.9){$\rvX(\cdot)$}%
  \rput(2.75,0){$\rvZ(\cdot)$}%
  %\ncline[linestyle=dotted,nodesep=1pt]{->}{xzlabel}{xz}%
  %\ncline[linestyle=dotted,nodesep=1pt]{->}{ylabel}{y}%
  }%
  \rput(2.75,0){$\fs(x_n)$}%
\end{pspicture}%%
  \includegraphics{../common/math/graphics/pdfs/lcg7x1m9_dgraph.pdf}%
  \caption{LCG mappings to \prope{linear} ($\rvX$), non-linear discrete ($\rvY$)
  and non-linear continuous ($\rvZ$) ordered metric spaces \xref{ex:lcg7x1m9_dgraph}\label{fig:lcg7x1m9_dgraph}}
\end{figure}
%---------------------------------------
%\begin{minipage}{\tw-65mm}%
\begin{example}[\exmd{LCG mappings, sequential directed graph}]
\label{ex:lcg7x1m9_dgraph}\mbox{}\\
%---------------------------------------
Let $\ocsG$, $\omsH$ and $\rvZ$ be a illustrated in \prefpp{fig:lcg7x1m9_dgraph}.
In \prefpp{ex:lcg7x1m9_seqorder}, the outcome values were ordered sequentially \emph{like} a PRNG,
but the metrics were \prope{commutative}, which is \emph{unlike} a PRNG.
In this example, the outcomes are assigned \fncte{quasi-metric}s \xxref{def:qmetric}{rem:qmetric} that are \prope{non-commutative}.
For example in the shaped sequence $\fs(x_n)=\seqn{\ldots,3, 4, 4, 1, 3, \ldots}$, 
the ``distance" from $3$ to $4$ is $\metric{3}{4}=1$,
but from $4$ to $3$ is $\metric{4}{3}=2$.
This yields the following statistics:
\\$\quad\begin{array}{>{\gsize}Mrcc rcccl}
  geometry of $\ocsG_9$:                                & \ocscen (\ocsG) &=& \mc{4}{l}{\setn{0,1,2,3,4,5,6,7,8}}  \\
  geometry of $\ocsG$:                                  & \ocscen (\ocsG) &=& \setn{3}  \\
  outcome subspace statistics on isomorphic structure:  & \ocsE   (\rvZ)  &=& \setn{3}   & \ocsVar(\rvZ;\ocsE) &=& \frac{4}{3} &\approx& 1.333
\end{array}$\\
Note that this technique yields the same estimate $\ocsE(\rvZ)=\setn{3}$ as \pref{ex:lcg7x1m9_seqorder}, 
but with a larger variance.
\end{example}
%\end{minipage}\hfill%
%\begin{tabular}{c}
%  \gsize%
%  %\psset{unit=5mm}%
%  {%============================================================================
% Daniel J. Greenhoe
% LaTeX file
% discrete metric real dice mapping to linearly ordered O6c
%============================================================================
{%\psset{unit=0.5\psunit}%
\begin{pspicture}(-3.25,-1.5)(3.25,1.5)%
  %---------------------------------
  % options
  %---------------------------------
  \psset{%
    radius=1.25ex,
    labelsep=2.5mm,
    linecolor=blue,%
    }%
  %---------------------------------
  % dice graph
  %---------------------------------
  \rput(-1.75,0){%\psset{unit=2\psunit}%
    \Cnode[fillstyle=solid,fillcolor=snode](-0.8660,-0.5){D4}%
    \Cnode(-0.8660,0.5){D5}%
    \Cnode(0,1){D6}%
    \Cnode(0.8660,0.5){D3}%
    \Cnode(0.8660,-0.5){D2}%
    \Cnode(0,-1){D1}%
    }
  \rput(D6){$\diceF$}%
  \rput(D5){$\diceE$}%
  \rput(D4){$\diceD$}%
  \rput(D3){$\diceC$}%
  \rput(D2){$\diceB$}%
  \rput(D1){$\diceA$}%
  %
  \ncline{D5}{D6}%
  \ncline{D4}{D5}\ncline{D4}{D6}%
  \ncline{D3}{D5}\ncline{D3}{D6}%
  \ncline{D2}{D3}\ncline{D2}{D4}\ncline{D2}{D6}%
  \ncline{D1}{D2}\ncline{D1}{D3}\ncline{D1}{D4}\ncline{D1}{D5}%
  %
  \uput[ 158](D6){$\frac{1}{30}$}
  \uput[ 150](D5){$\frac{1}{50}$}
  \uput[ 210](D4){$\frac{3}{5}$}
  \uput[  22](D3){$\frac{1}{30}$}
  \uput[ -45](D2){$\frac{1}{20}$}
  \uput[-158](D1){$\frac{1}{10}$}
  %---------------------------------
  % range graph
  %---------------------------------
  \rput(1.75,0){%\psset{unit=2\psunit}%
    \Cnode[fillstyle=solid,fillcolor=snode](-0.8660,-0.5){E4}%
    \Cnode(-0.8660,0.5){E5}%
    \Cnode(0,1){E6}%
    \Cnode(0.8660,0.5){E3}%
    \Cnode(0.8660,-0.5){E2}%
    \Cnode(0,-1){E1}%
    \Cnode(0,0){E0}%
    }
  \rput(E6){$6$}%
  \rput(E5){$5$}%
  \rput(E4){$4$}%
  \rput(E3){$3$}%
  \rput(E2){$2$}%
  \rput(E1){$1$}%
  \rput(E0){$0$}%
  %
  \ncline{E5}{E6}%
  \ncline{E4}{E5}\ncline{E4}{E6}%
  \ncline{E3}{E5}\ncline{E3}{E6}%
  \ncline{E2}{E3}\ncline{E2}{E4}\ncline{E2}{E6}%
  \ncline{E1}{E2}\ncline{E1}{E3}\ncline{E1}{E4}\ncline{E1}{E5}%
  \ncline{E0}{E1}\ncline{E0}{E2}\ncline{E0}{E3}\ncline{E0}{E4}\ncline{E0}{E5}\ncline{E0}{E6}%
  %
  \uput[ 158](E6){$\frac{1}{30}$}
  \uput[ 150](E5){$\frac{1}{50}$}
  \uput[ 210](E4){$\frac{3}{5}$}
  \uput[  22](E3){$\frac{1}{30}$}
  \uput[ -45](E2){$\frac{1}{20}$}
  \uput[-158](E1){$\frac{1}{10}$}
  \uput[  0](E0){$\sfrac{0}{6}$}
  %---------------------------------
  % mapping from die to O6c
  %---------------------------------
  \ncarc[arcangle= 22,linewidth=0.75pt,linecolor=red]{->}{D6}{E6}%
  \ncarc[arcangle= 22,linewidth=0.75pt,linecolor=red]{->}{D5}{E5}%
  \ncarc[arcangle= 22,linewidth=0.75pt,linecolor=red]{->}{D4}{E4}%
  \ncarc[arcangle=-22,linewidth=0.75pt,linecolor=red]{->}{D3}{E3}%
  \ncarc[arcangle=-22,linewidth=0.75pt,linecolor=red]{->}{D2}{E2}%
  \ncarc[arcangle=-22,linewidth=0.75pt,linecolor=red]{->}{D1}{E1}%
  %---------------------------------
  % labels
  %---------------------------------
  \rput(0,0){$\rvX(\cdot)$}%
  %\ncline[linestyle=dotted,nodesep=1pt]{->}{xzlabel}{xz}%
  %\ncline[linestyle=dotted,nodesep=1pt]{->}{ylabel}{y}%
\end{pspicture}
}%}%
%\end{tabular}
\begin{proof}
\begin{align*}
  \ocscen(\ocsG_9)
    &\eqd \argmin_{x\in\ocsG_9}\max_{y\in\ocsG_9}\metric{x}{y}\psp(y)
    &&\text{by definition of $\ocscen$ \xref{def:ocscen}}
  \\&= \argmin_{x\in\ocsG_9}\max_{y\in\ocsG_9}\metric{x}{y}\frac{1}{9}
    &&\text{by definition of $\ocsG_9$}
  \\&= \argmin_{x\in\ocsG_9}\max_{y\in\ocsG_9}\metric{x}{y}
    && \text{because $\fphi(x)=\frac{1}{9}x$ is \prope{strictly isotone} and by \prefp{lem:argminmaxphi}}
  \\&= \argmin_{x\in\ocsG_9}\setn{8,\,8,\,8,\,8,\,8,\,8,\,8,\,8,\,8}
    && \text{because the maximum distance in $\ocsG_9$ from any $x$ is $8$}
  \\&= \setn{0,1,2,\cdots,8}
    && \text{because the distances for values of $x$ in $\ocsG_9$ are the same}
  \\
  \ocscen(\ocsG)
    &= \setn{3}
    && \text{by \prefpp{ex:wring5shortd}}
  \\
  \ocsE(\rvZ)
    &= \rvZ\brs{\ocscen(\ocsG)}
    && \text{because $\ocsG$ and $\omsH$ are \prope{isomorphic} under $\rvZ$}
  \\&= \rvZ\brs{\setn{3}}
    && \text{by $\ocscen(\ocsG)$ result}
  \\&= \setn{3}
    && \text{by definition of $\rvZ$}
  \\
  \ocsVar(\rvZ;\ocsE)
    &= \ocsVaro(\ocscen(\ocsG))
    && \text{because $\ocsG$ and $\omsH$ are \prope{isomorphic} under $\rvZ$}
  \\&= \frac{4}{3} \approx 1.333
    && \text{by \prefpp{ex:wring5shortd}}
\end{align*}
\end{proof}


%=======================================
\subsubsection{Genomic signal processing (GSP) examples}
%=======================================
\begin{figure}[h]
  \gsize%
  \centering%
  %%============================================================================
% Daniel J. Greenhoe
% LaTeX file
% DNA to real line and integer line
%============================================================================
{%\psset{unit=0.5\psunit}%
\begin{pspicture}(-5.4,-1.75)(5.4,1.75)%
  %---------------------------------
  % options
  %---------------------------------
  \psset{%
    linecolor=blue,%
    radius=1.25ex,
    labelsep=2.5mm,
    }%
  %---------------------------------
  % DNA outcome space
  %---------------------------------
  \rput(0,0){%
    \Cnode[fillstyle=solid,fillcolor=snode](-1, 1){Da}%
    \Cnode[fillstyle=solid,fillcolor=snode]( 1, 1){Dt}%
    \Cnode[fillstyle=solid,fillcolor=snode](-1,-1){Dc}%
    \Cnode[fillstyle=solid,fillcolor=snode]( 1,-1){Dg}%
    \uput[-90](0,1){$\ocsG$}
    }%
  \ncline{Dc}{Dg}%
  \ncline{Dt}{Dc}\ncline{Dt}{Dg}%
  \ncline{Da}{Dt}\ncline{Da}{Dc}\ncline{Da}{Dg}%
  %
  \rput(Dg){$\symG$}%
  \rput(Dc){$\symC$}%
  \rput(Dt){$\symT$}%
  \rput(Da){$\symA$}%
  %
  \uput[ -45](Dg){$\frac{1}{4}$}%
  \uput[-135](Dc){$\frac{1}{4}$}%
  \uput[  45](Dt){$\frac{1}{4}$}%
  \uput[ 135](Da){$\frac{1}{4}$}%
  %---------------------------------
  % real line
  %---------------------------------
  \rput(-3.5,0){\psset{unit=0.5\psunit}%
    \pnode(0,2.5){RB}%
    \pnode(0,1.5){R3}%
    \pnode(0,0.5){R2}%
    \pnode(0,0){R12}%
    \pnode(0,-0.5){R1}%
    \pnode(0,-1.5){R0}%
    \pnode(0,-2.5){RA}%
    }%
  \ncline{<->}{RA}{RB}
  \rput(R3){\psline(-0.1,0)(0.1,0)}%
  \rput(R2){\psline(-0.1,0)(0.1,0)}%
  \rput(R1){\psline(-0.1,0)(0.1,0)}%
  \rput(R0){\psline(-0.1,0)(0.1,0)}%
  %
  \rput(R12){\pscircle[fillstyle=solid,linecolor=snode,fillcolor=snode](0,0){1ex}}%
  %\rput(X233){\pscircle[fillstyle=none,linecolor=red,fillcolor=red](0,0){1ex}}%
  %
  \uput[180](R3){$3$}%
  \uput[180](R2){$2$}%
  \uput[180](R1){$1$}%
  \uput[180](R0){$0$}%
  \uput[165](RB){$\omsR$}
  %
  \ncarc[arcangle=-22,linewidth=0.75pt,linecolor=red]{->}{Dg}{R1}%
  \ncarc[arcangle=-22,linewidth=0.75pt,linecolor=red]{->}{Dc}{R0}%
  \ncarc[arcangle=-22,linewidth=0.75pt,linecolor=red]{->}{Dt}{R3}%
  \ncarc[arcangle=-22,linewidth=0.75pt,linecolor=red]{->}{Da}{R2}%
  %---------------------------------
  % integer line
  %---------------------------------
  \rput(3.5,0){\psset{unit=0.5\psunit}%
    \pnode(0,2.5){ZB}%
    \Cnode(0,1.5){Z3}%
    \Cnode[fillstyle=solid,fillcolor=snode](0,0.5){Z2}%
    \Cnode[fillstyle=solid,fillcolor=snode](0,-0.5){Z1}%
    \Cnode(0,-1.5){Z0}%
    \pnode(0,-2.5){ZA}%
    }%
  \ncline[linestyle=dotted]{Z3}{ZB}%
  \ncline{Z2}{Z3}%
  \ncline{Z1}{Z2}%
  \ncline{Z0}{Z1}%
  \ncline[linestyle=dotted]{ZA}{Z0}%
  %
  \rput(Z3){$3$}%
  \rput(Z2){$2$}%
  \rput(Z1){$1$}%
  \rput(Z0){$0$}%
  \uput[15](ZB){$\omsZ$}
  %
  \ncarc[arcangle=-22,linewidth=0.75pt,linecolor=green]{->}{Dg}{Z1}%
  \ncarc[arcangle=-22,linewidth=0.75pt,linecolor=green]{->}{Dc}{Z0}%
  \ncarc[arcangle=-22,linewidth=0.75pt,linecolor=green]{->}{Dt}{Z3}%
  \ncarc[arcangle=-22,linewidth=0.75pt,linecolor=green]{->}{Da}{Z2}%
  %---------------------------------
  % labels
  %---------------------------------
  \rput(-2.5,0){$\rvX(\cdot)$}%
  \rput(2.5,0){$\rvY(\cdot)$}%
\end{pspicture}
}%%
  \includegraphics{../common/math/graphics/pdfs/gspRZ.pdf}%
  \caption{DNA random variable mappings to \structe{real line} and \structe{integer line} \xref{ex:gspRZ}\label{fig:gspRZ}}
\end{figure}
%---------------------------------------
\begin{example}[\exmd{DNA to linear structures}]
\label{ex:gspRZ}
%---------------------------------------
\structe{Genomic Signal Processing} (\structe{GSP}) analyzes biological sequences called \structe{genome}s.
These sequences are constructed over a set of 4 symbols that are commonly referred to as 
$\symA$, $\symT$, $\symC$, and $\symG$,
each of which corresponds to a nucleobase (adenine,  thymine, cytosine, and guanine, 
respectively).\footnote{
  \citePc{mendel1853e}{Mendel (1853): gene coding uses discrete symbols},
  \citePpc{watson1953}{737}{Watson and Crick (1953): gene coding symbols are adenine,  thymine, cytosine, and guanine},
  \citePp{watson1953may}{965},
  \citerpg{pommerville2013}{52}{1449647960}
  }
A typical genome sequence contains a large number of symbols 
(about 3 billion for humans, 29751 for the SARS virus).%
\footnote{%
  \citeWuc{genbank}{http://www.ncbi.nlm.nih.gov/genome/guide/human/}{Homo sapiens, NC\_000001--NC\_000022 (22 chromosome pairs), NC\_000023 (X chromosome), NC\_000024 (Y chromosome), NC\_012920 (mitochondria)},
  \citeWuc{genbank}{http://www.ncbi.nlm.nih.gov/nuccore/30271926}{SARS coronavirus, NC\_004718.3}
  \citePc{gregory2006}{homo sapien chromosome 1},
  \citePc{he2004}{SARS coronavirus}
  }
%\\[0.3ex]\begin{minipage}{\tw-45mm}%
Let $\ocsG\eqd\ocs{\setn{\symA,\symT,\symC,\symG}}{\metricn}{\emptyset}{\psp}$ 
be the \structe{outcome subspace} \xref{def:ocs} generated by a \structe{genome}
where $\metricn$ is the \fncte{discrete metric} \xref{def:dmetric},
$\orel=\emptyset$ indicates a completely unordered set \xref{def:order}, and
$\psp(\symA)=\psp(\symT)=\psp(\symC)=\psp(\symG)=\frac{1}{4}$ (uniformly distributed).
Let $\omsH\eqd\omsR$ be the \structe{real line} \xref{def:Rline}. 
This yields the following statistics:
\\$\indentx\begin{array}{>{\gsize}Mrcc rcccl}
  geometry of $\ocsG$:                                  & \ocscen (\ocsG) &=& \mc{4}{l}{\setn{\symA,\symT,\symC,\symG}}  \\
  traditional statistics on real line:                  & \pE  (\rvX)     &=& 1.5          & \ocsVar(\rvX;\pE)   &=& \frac{5}{4} &=& 1.25 \\
  outcome subspace statistics on real line:             & \ocsE (\rvX)    &=& 1.5          & \ocsVar(\rvX;\ocsE) &=& \frac{5}{4} &=& 1.25\\
  outcome subspace statistics on integer line:          & \ocsE (\rvY)    &=& \setn{1,2}   & \ocsVar(\rvY;\ocsE) &=& \frac{1}{2} &=& 0.5
\end{array}$\\
%\\
The symbols $\symA$, $\symT$, $\symC$ and $\symG$ in general again have 
an order structure and a \structe{metric geometry} \xref{rem:mgeo} 
that is fundamentally dissimilar from that mapped to by the random variables $\rvX$ and $\rvY$.
Therefore, statistical inferences made using these random variables will likely lead to results 
that arguably have little relationship with intuition or reality.
%(see figure to the right where genomic struture is has the discrete metric topology illustrated with an undirected graph).
\end{example}
\begin{proof}
    \begin{align*}
      \ocscen(\ocsG)
        &= \setn{\symA,\,\symT,\,\symC,\,\symG}
        && \text{by \prefp{ex:dna}}
      \\
      \pE(\rvX) 
        &\eqd \int_{\R} x\psp(x) \dx
        && \text{by definition of $\pE$ \xref{def:pE}}
      \\&= \sum_{x\in\Z} x\psp(x)
        && \text{by definition of $\psp$ and \prefp{prop:pE}}
      \\&= \frac{1}{4} \sum_{x\in\setn{0,1,2,3}} x
        && \text{by definitions of $\ocsG$, $\omsH$ and $\rvX$}
      \\&= \frac{1}{4}\brp{0+1+2+3}
         = \frac{6}{4}
         = \frac{3}{2}
         = 1.5
      \\
      \ocsVar(\rvX;\pE)
        &= \pVar(\rvX)
        && \text{by \prefp{thm:ocsVar}}
      \\&= \int_{\R} \brs{x-\pE(\rvX)}^2\psp(x)
        && \text{by definition of $\pVar$ \xref{def:pVar}}
      \\&= \sum_{x\in\Z} \brs{x-\pE(\rvX)}^2\psp(x)
        &&\text{by definition of $\psp$ and \prefp{prop:pE}}
      \\&=   \frac{1}{4}\sum_{x\in\omsH} \brs{x-\frac{3}{2}}^2
        &&\text{by $\pE(\rvX)$ result}
      \\&= \mathrlap{\frac{1}{4}\brs{\brp{0-\frac{3}{2}}^2 + \brp{1-\frac{3}{2}}^2 + \brp{2-\frac{3}{2}}^2 + \brp{3-\frac{3}{2}}^2}}
      \\&= \mathrlap{\frac{1}{4\cdot2^2}\brs{\brp{0-3}^2 + \brp{2-3}^2 + \brp{4-3}^2 + \brp{6-3}^2}
         = \frac{20}{16}= \frac{5}{4} = 1.25}
      \\
      \ocsE(\rvX)
        &= \pE(\rvX)
        && \text{because on \structe{real line}, $\psp$ is \prope{symmetric}, and by \prefp{thm:pEocsE}}
      \\&= \frac{3}{2}= 1.5
        && \text{by $\pE(\rvX)$ result}
      \\
      \ocsE(\rvX)
        &\eqd \argmin_{x\in\R}\max_{y\in\R}\metric{x}{y}\psp(y)
        &&\text{by definition of $\ocsE$ \xref{def:ocsE}.\qquad\textbf{(alternate proof)}}
      \\&=    \argmin_{x\in\R}\max_{y\in\R}\abs{x-y}\psp(y)
        &&\text{by definition usual metric on real line}
      \\&=    \argmin_{x\in\R}\max_{y\in\setn{0,1,2,3}}\abs{x-y}\frac{1}{4}
        &&\text{by definition of $\ocsG$}
      \\&=    \argmin_{x\in\R}\max_{y\in\setn{0,1,2,3}}\abs{x-y}
        && \text{because $\ff(x)=\frac{1}{4}x$ is \prope{strictly isotone} and by \prefpp{lem:argminmaxphi}}
      \\&= \mathrlap{%
           \argmin_{x\in\R}\brb{\begin{array}{lM}
             \abs{x-3} & for $x\le\frac{3}{2}$\\
             \abs{x-0} & otherwise
           \end{array}}
           %\qquad\text{\gsize\centering\psset{unit=6mm}{%============================================================================
% Daniel J. Greenhoe
% XeLaTeX file
%============================================================================
\begin{pspicture}(-0.75,-0.5)(7.25,2)%
  \psset{%
    labelsep=1pt,
    linewidth=1pt,
    yunit=0.5\psunit,
    }%
  \psaxes[linecolor=axis,yAxis=false,labels=none]{->}(0,0)(0,0)(3.5,4)% x axis
  \psaxes[linecolor=axis,xAxis=false,labels=none]{->}(0,0)(0,0)(3.5,4)% y axis
  \psline(0,3)(1.5,1.5)(3,3)%
  %
  \psline[linestyle=dotted,linecolor=red](1.5,1.5)(1.5,0)%
  \psline[linestyle=dotted,linecolor=red](0,1.5)(1.5,1.5)%
  %
  \uput[0]{0}(3.5,0){$x$}%
  \uput[-90]{0}(1.5,0){$\sfrac{3}{2}$}%
  \uput{5pt}[180]{0}(0,3){$3$}%
  \uput[180]{0}(0,1.5){$\sfrac{3}{2}$}%
  \rput[l](2.75,1.25){$\ds\ff(x)\eqd\max_{y\in\setn{0,1,2,3}}\abs{x-y}$}%
\end{pspicture}%
}}
           \qquad\begin{array}{l}\includegraphics{../common/math/graphics/pdfs/dna_Rmax.pdf}\end{array}%
           }
      \\&= \setn{\frac{3}{2}} = 1\frac{1}{2} = 1.5
        && \text{because expression is minimized at argument $x=\frac{3}{2}$}
      \\
      \ocsVar(\rvX;\ocsE)
        &= \ocsVar(\rvX;\pE)
        && \text{because $\ocsE(\rvX)=\pE(\rvX)$}
      \\&= \frac{5}{4}
        && \text{by $\ocsVar(\rvX;\pE)$ result}
      \\
      \ocsE(\rvY)
        &\eqd \argmin_{x\in\Z}\max_{y\in\Z}\metric{x}{y}\psp(y)
        &&\text{by definition of $\ocsE$ \xref{def:ocsE}}
      \\&= \argmin_{x\in\Z}\max_{y\in\Z}\abs{x-y}\psp(y)
        &&\text{by definition of \structe{integer line} \xref{def:Zline}}
      \\&= \argmin_{x\in\Z}\max_{y\in\setn{0,1,2,3}}\abs{x-y}\frac{1}{4}
      \\&= \argmin_{x\in\Z}\max_{y\in\setn{0,1,2,3}}\abs{x-y}
        && \text{because $\fphi(x)=\frac{1}{4}x$ is \prope{strictly isotone} and by \prefp{lem:argminmaxphi}}
      \\&=\mathrlap{\argmin_{x\in\setn{0,1,2,3}}\max_{y\in\setn{0,1,2,3}}
             \setn{\begin{array}{cccc}
                0 & 1 & 2 & 3\\
                1 & 0 & 1 & 2\\
                2 & 1 & 0 & 1\\
                3 & 2 & 1 & 0
             \end{array}}
      = \argmin_{x\in\setn{0,1,2,3}}
             \setn{\begin{array}{c}
                3\\
                2\\
                2\\
                3
             \end{array}}
      = \argmin_{x\in\setn{0,1,2,3}}
             \setn{\begin{array}{c}
                \mbox{}\\
                1\\
                2\\
                \mbox{}
             \end{array}}}
      \\
      \ocsVar(\rvY;\ocsE)
        &\eqd \sum_{x\in\Z}\metricsq{\ocsE(\rvY)}{x}\psp(x)
        && \text{by definition of $\ocsVar$ \xref{def:ocsVar}}
      \\&= \sum_{x\in\Z}\metricsq{\setn{1,2}}{x}\psp(x)
        && \text{by $\ocsE(\rvY)$ result}
      \\&= \mathrlap{
             \abs{0-1}^2\times\frac{1}{4} + 
             \abs{1-1}^2\times\frac{1}{4} + 
             \abs{2-2}^2\times\frac{1}{4} + 
             \abs{3-2}^2\times\frac{1}{4} 
         = \frac{1}{2} = 0.5}
    \end{align*}
\end{proof}



%---------------------------------------
\begin{example}[\exmd{GSP to complex plane}]
\label{ex:gsp_C}
%---------------------------------------
\mbox{}\\\begin{minipage}{\tw-46mm}%
A possible solution for the GSP problem \xref{ex:gspRZ} is to map $\setn{\symA, \symT, \symC, \symG}$
to the \structe{complex plane} \xref{ex:Cplane} 
rather than the \structe{real line} \xref{def:Rline} such that (see also illustration to the right)
\\\indentx$\begin{array}{rcrcl@{\qquad}rcrcl}
  a&\eqd&\rvX(\symA) &=& -1+i  & t&\eqd&\rvX(\symT) &=& 1+i\\
  c&\eqd&\rvX(\symC) &=& -1-i  & g&\eqd&\rvX(\symG) &=& 1-i .
\end{array}$
\end{minipage}%
\hfill%
{\begin{tabular}{c}%
  \gsize%
  \psset{unit=5mm}%
  %{%============================================================================
% Daniel J. Greenhoe
% LaTeX file
% genomic metric with discrete metric topology mapped to complex plain
%============================================================================
{%\psset{unit=0.5\psunit}%
\begin{pspicture}(-1,-2)(8,2)%
  %---------------------------------
  % options
  %---------------------------------
  \psset{%
    linecolor=blue,%
    radius=1.25ex,
    labelsep=2.5mm,
    }%
  %---------------------------------
  % genome graph
  %---------------------------------
  \rput(1.2,0){%
    \Cnode[fillstyle=solid,fillcolor=snode](-1, 1){A}%
    \Cnode[fillstyle=solid,fillcolor=snode]( 1, 1){T}%
    \Cnode[fillstyle=solid,fillcolor=snode](-1,-1){C}%
    \Cnode[fillstyle=solid,fillcolor=snode]( 1,-1){G}%
    }%
  \ncline{C}{G}%
  \ncline{T}{C}\ncline{T}{G}%
  \ncline{A}{T}\ncline{A}{C}\ncline{A}{G}%
  \rput(G){$\symG$}%
  \rput(C){$\symC$}%
  \rput(T){$\symT$}%
  \rput(A){$\symA$}%
  %---------------------------------
  % complex plain
  %---------------------------------
  \rput(6,0){%
    \pnode(-1, 1){a}%
    \pnode( 1, 1){t}%
    \pnode(-1,-1){c}%
    \pnode( 1,-1){g}%
    \psaxes[linecolor=axis,labels=none,linewidth=0.5pt]{<->}(0,0)(-1.5,-1.5)(1.5,1.5)%
    %\psline{<->}(-1.25,0)(1.25,0)% x axis
    %\psline{<->}(0,-1.25)(0,1.25)% y axis
    }%
  %\ncline{a}{t}%
  %\ncline{g}{a}%
  %\ncline{c}{g}%
  %
  \rput(g){$\bullet$}%
  \rput(c){$\bullet$}%
  \rput(t){$\bullet$}%
  \rput(a){$\bullet$}%
  %
  \uput[ -45](g){$g$}%
  \uput[-135](c){$c$}%
  \uput[  45](t){$t$}%
  \uput[ 135](a){$a$}%
  %---------------------------------
  % mapping from die to L6
  %---------------------------------
  \ncarc[arcangle=-22,linewidth=0.75pt,linecolor=red]{->}{G}{g}%
  \ncarc[arcangle= 22,linewidth=0.75pt,linecolor=red]{->}{C}{c}%
  \ncarc[arcangle=-22,linewidth=0.75pt,linecolor=red]{->}{T}{t}%
  \ncarc[arcangle= 22,linewidth=0.75pt,linecolor=red]{->}{A}{a}%
  %\ncline[linewidth=0.75pt,linecolor=red]{->}{b}{lb}%
  %---------------------------------
  % labels
  %---------------------------------
  \rput(3.5,-0.3){$\rvX(\cdot)$}%
  %\ncline[linestyle=dotted,nodesep=1pt]{->}{xzlabel}{xz}%
  %\ncline[linestyle=dotted,nodesep=1pt]{->}{ylabel}{y}%
\end{pspicture}
}%}%
  {\includegraphics{../common/math/graphics/pdfs/dmetricdnatoc.pdf}}%
\end{tabular}}
However, this solution also is arguably unsatisfactory for two reasons:
\begin{enumerate}
  \item The order structures are dissimilar. Note that 
        \\\indentx$c<a$, but $\symC$ and $\symA$ are \prope{incomparable} \xref{def:order}.
  \item The metric geometries are dissimilar. 
        Let $\metricn$ be the \fncte{discrete metric} and $\metrican$ the \fncte{usual metric} in $\C$.
        Note that 
        \\\indentx$\metric{\symA}{\symT} = \metric{\symA}{\symC} = \metric{\symA}{\symG} = 1$, but
        \\\indentx$\metrica{a}{t}=\abs{a-t}=2 \neq 2\sqrt{2}=\abs{a-g}=\metrica{a}{g}$.
\end{enumerate}
\end{example}



%---------------------------------------
\begin{minipage}{\tw-70mm}%
\begin{example}[\exmd{DNA mapping with extended range}]
\label{ex:dnaXO4c}\mbox{}\\
%---------------------------------------
\prefpp{ex:gspRZ} presented a mapping from a DNA structure to a linearly ordered lattices,
but the order and metric geometry was not preserved.
In this example, a different structure is used that does preserve both order and metric geometry
(see illustration to the right). This yields the following statistics:
\\\indentx$\ocsE(\rvX) = \setn{0} \qquad  \ocsVar (\rvX) = \frac{1}{4}$
\end{example}
\end{minipage}\hfill%
\begin{tabular}{c}
  \gsize%
  %\psset{unit=5mm}%
  %{%============================================================================
% Daniel J. Greenhoe
% LaTeX file
% DNA graph to O4 with center
%============================================================================
{%\psset{unit=0.5\psunit}%
\begin{pspicture}(-3.2,-1.75)(3.2,1.75)%
  %---------------------------------
  % options
  %---------------------------------
  \psset{%
    radius=1.25ex,
    labelsep=2.5mm,
    linecolor=blue,%
    }%
  %---------------------------------
  % dice graph
  %---------------------------------
  \rput(-1.75,0){%\psset{unit=2\psunit}%
    \Cnode[fillstyle=solid,fillcolor=snode]( 1,-1){DG}%
    \Cnode[fillstyle=solid,fillcolor=snode](-1,-1){DC}%
    \Cnode[fillstyle=solid,fillcolor=snode]( 1, 1){DT}%
    \Cnode[fillstyle=solid,fillcolor=snode](-1, 1){DA}%
    \uput[-90](0,1){$\ocsG$}%
    }
  \rput(DG){$\symG$}%
  \rput(DC){$\symC$}%
  \rput(DT){$\symT$}%
  \rput(DA){$\symA$}%
  %
  \ncline{DC}{DG}\naput[labelsep=0pt]        {${\scy\metric{\symC}{\symG}=}1$}%
  \ncline{DT}{DG}\naput[labelsep=0pt,nrot=:U]{${\scy\metric{\symT}{\symG}=}1$}%
  \ncline{DA}{DT}\naput[labelsep=0pt]        {${\scy\metric{\symA}{\symT}=}1$}%
  \ncline{DC}{DA}\naput[labelsep=0pt,nrot=:U]{${\scy\metric{\symA}{\symC}=}1$}%
  \ncline{DC}{DT}\naput[labelsep=0pt,nrot=:U,npos=0.25]{${\scy\metric{\symT}{\symC}=}1$}%
  \ncline{DA}{DG}\naput[labelsep=0pt,npos=0.5,nrot=:U]{${\scy\metric{\symA}{\symG}=}1$}%
  %
  \uput[ -90](DG){${\scy\psp(\symG)=}\frac{1}{4}$}
  \uput[ -90](DC){${\scy\psp(\symC)=}\frac{1}{4}$}
  \uput[  90](DT){${\scy\psp(\symT)=}\frac{1}{4}$}
  \uput[  90](DA){${\scy\psp(\symA)=}\frac{1}{4}$}
  %---------------------------------
  % range graph
  %---------------------------------
  \rput(1.75,0){%\psset{unit=2\psunit}%
    \Cnode( 1,-1){Dg}%
    \Cnode(-1,-1){Dc}%
    \Cnode( 1, 1){Dt}%
    \Cnode(-1, 1){Da}%
    \Cnode[fillstyle=solid,fillcolor=snode](0, 0){D0}%
    \rput(0,-0.75){$\omsH$}%
    }
  \rput(Dg){$g$}%
  \rput(Dc){$c$}%
  \rput(Dt){$t$}%
  \rput(Da){$a$}%
  \rput(D0){$0$}%
  %
  \ncline{Da}{D0}\naput[labelsep=0pt,nrot=:U]{${\scy\metric{0}{a}=}\sfrac{1}{2}$}
  \ncline{D0}{Dt}\naput[labelsep=0pt,nrot=:U]{${\scy\metric{0}{t}=}\sfrac{1}{2}$}
  \ncline{Dc}{D0}\naput[labelsep=0pt,nrot=:U]{${\scy\metric{0}{c}=}\sfrac{1}{2}$}
  \ncline{D0}{Dg}\naput[labelsep=0pt,nrot=:U]{${\scy\metric{0}{g}=}\sfrac{1}{2}$}%
  %\ncline{Dc}{Dg}\naput*{$1$}%
  %\ncline{Dt}{Dg}\naput*{$1$}%
  %\ncline{Da}{Dt}\naput*{$1$}
  %\ncline{Da}{Dc}\naput*{$1$}%
  %
  \uput[ -90](Dg){${\scy\psp(g)=}\frac{1}{4}$}
  \uput[ -90](Dc){${\scy\psp(c)=}\frac{1}{4}$}
  \uput[  90](Dt){${\scy\psp(t)=}\frac{1}{4}$}
  \uput[  90](Da){${\scy\psp(a)=}\frac{1}{4}$}
  \uput[20](D0){${\scy\psp(0)=}\frac{0}{4}$}
  %---------------------------------
  % mapping from die to O6c
  %---------------------------------
  \ncarc[arcangle=-22,linewidth=0.75pt,linecolor=red]{->}{DG}{Dg}%
  \ncarc[arcangle= 22,linewidth=0.75pt,linecolor=red]{->}{DC}{Dc}%
  \ncarc[arcangle= 22,linewidth=0.75pt,linecolor=red]{->}{DT}{Dt}%
  \ncarc[arcangle=-22,linewidth=0.75pt,linecolor=red]{->}{DA}{Da}%
  %---------------------------------
  % labels
  %---------------------------------
  \rput(0,0){$\rvX(\cdot)$}%
  %\ncline[linestyle=dotted,nodesep=1pt]{->}{xzlabel}{xz}%
  %\ncline[linestyle=dotted,nodesep=1pt]{->}{ylabel}{y}%
\end{pspicture}
}%}%
  {\includegraphics{../common/math/graphics/pdfs/dnaXO4c.pdf}}%
\end{tabular}
\begin{proof}
\begin{align*}
  \ocsE(\omsH)
    &\eqd \argmin_{x\in\omsH}\max_{y\in\omsH}\metric{x}{y}\psp(y)
    &&\text{by definition of $\ocsE$ \xref{def:ocsE}}
  \\&= \argmin_{x\in\omsH}\max_{y\in\omsH\setd\setn{0}}\metric{x}{y}\psp(y)
    &&\text{because $\psp(0)=0$}
  \\&= \argmin_{x\in\omsH\setd\setn{0}}\max_{y\in\omsH\setd\setn{0}}\metric{x}{y}\frac{1}{4}
    && \text{by definition of $\ocsG$}
  \\&= \argmin_{x\in\omsH\setd\setn{0}}\max_{y\in\omsH\setd\setn{0}}\metric{x}{y}
    && \text{because $\fphi(x)=\frac{1}{4}x$ is \prope{strictly isotone} and by \prefp{lem:argminmaxphi}}
  %\\&= \argmin_{x\in\omsH}\max_{y\in\omsH}
  %       \setn{\begin{array}{ccccccc}
  %         \metricn(1,1) &\metricn(1,2) &\metricn(1,3) &\metricn(1,4) &\metricn(1,0)\frac{0}{4}\\
  %         \metricn(2,1) &\metricn(2,2) &\metricn(2,3) &\metricn(2,4) &\metricn(2,0)\frac{0}{4}\\
  %         \metricn(3,1) &\metricn(3,2) &\metricn(3,3) &\metricn(3,4) &\metricn(3,0)\frac{0}{4}\\
  %         \metricn(4,1) &\metricn(4,2) &\metricn(4,3) &\metricn(4,4) &\metricn(4,0)\frac{0}{4}\\
  %         \metricn(0,1) &\metricn(0,2) &\metricn(0,3) &\metricn(0,4) &\metricn(0,0)\frac{0}{4}
  %       \end{array}}
  \\&=\mathrlap{\argmin_{x\in\omsH}\max_{y\in\omsH}
         \setn{\begin{array}{cccccc}
           \metricn(1,1) &\metricn(1,2) &\metricn(1,3) &\metricn(1,4) \\
           \metricn(2,1) &\metricn(2,2) &\metricn(2,3) &\metricn(2,4) \\
           \metricn(3,1) &\metricn(3,2) &\metricn(3,3) &\metricn(3,4) \\
           \metricn(4,1) &\metricn(4,2) &\metricn(4,3) &\metricn(4,4) \\
           \metricn(0,1) &\metricn(0,2) &\metricn(0,3) &\metricn(0,4) 
         \end{array}}
    = \argmin_{x\in\omsH}\max_{y\in\omsH}
         \setn{\begin{array}{cccc}
           {0}&{1}&{1}&{1}\\
           {1}&{0}&{1}&{1}\\
           {1}&{1}&{0}&{1}\\
           {1}&{1}&{1}&{0}\\
           \frac{1}{2}&\frac{1}{2}&\frac{1}{2}&\frac{1}{2}
         \end{array}}
    = \argmin_{x\in\omsH}
         \setn{\begin{array}{c}
           {1}\\
           {1}\\
           {1}\\
           {1}\\
           \frac{1}{2}
         \end{array}}}
  \\&= \setn{0}
    && \text{because expression is minimized at $x=\setn{0}$}
  %\\
  %\ocsEa(\rvX)
  %  &\eqd \argmin_{x\in\omsH}\sum_{y}\ocsmom(x,y)
  %      &&\text{by definition of $\ocsEa$ \xref{def:ocsEa}}
  %\\&\eqd \argmin_{x\in\omsH}\sum_{y}\metric{x}{y}\psp(y)
  %      &&\text{by definition of $\ocsmom$ \xref{def:ocsmom}}
  %\\&= \argmin_{x\in\omsH}\sum_{y\in\omsH\setd\setn{0}}\metric{x}{y}\psp(y)
  %     &&\text{because $\psp(0)=0$}
  %\\&= \argmin_{x\in\omsH}\sum_{y\in\omsH\setd\setn{0}}\metric{x}{y}\frac{1}{4}
  %\\&= \argmin_{x\in\omsH}\sum_{y\in\omsH\setd\setn{0}}\metric{x}{y}
%        && \text{because $\fphi(x)=\frac{1}{4}x$ is \prope{strictly isotone} and by \prefp{lem:argminmaxphi}}
  %\\&=\mathrlap{\argmin_{x\in\omsH}
  %       \setn{\begin{array}{*{7}{c}}
  %         0           &+& 1           &+& 1           &+& 1          \\
  %         1           &+& 0           &+& 1           &+& 1          \\
  %         1           &+& 1           &+& 0           &+& 1          \\
  %         1           &+& 1           &+& 1           &+& 0          \\
  %         \frac{1}{2} &+& \frac{1}{2} &+& \frac{1}{2} &+& \frac{1}{2}
  %       \end{array}}
  %\quad= \argmin_{x\in\omsH}
  %       \setn{\begin{array}{c}
  %         3\\
  %         3\\
  %         3\\
  %         3\\
  %         2
  %       \end{array}}
  %\quad= \setn{0}}
  \\
  \ocsVar(\rvX)
    &\eqd \sum_{x\in\omsH}\metricsq{\ocsE(\rvX)}{x}\psp(x)
    && \text{by definition of $\ocsVar$ \xref{def:ocsVar}}
  \\&= \sum_{x\in\omsH}\metricsq{\setn{0}}{x}\psp(x)
    && \text{by $\ocsE(\rvX)$ result}
  \\&= \mathrlap{
       \sum_{x\in\omsH\setd\setn{0}}\brp{\frac{1}{2}}^2\frac{1}{4}
     = \seto{\omsH\setd\setn{0}}\brp{\frac{1}{2}}^2\frac{1}{4}
     = 4\brp{\frac{1}{2}}^2\frac{1}{4}
     = \frac{1}{4}
     }
\end{align*}
\end{proof}

\begin{figure}[h]
  \gsize%
  \centering%
  %%============================================================================
% Daniel J. Greenhoe
% LaTeX file
% GSP with depth 2 Markov probability model
%============================================================================
\begin{pspicture}(-7.2,-7.2)(7.2,7.2)%
  %---------------------------------
  % options
  %---------------------------------
  \psset{%
    linecolor=blue,%
    radius=1em,
    nodesep=1.5pt,
    labelsep=2.5mm,
    arrowsize=5pt,
    }%
  \rput(0,0){\psset{unit=2\psunit}%
                     \Cnode(0, 3){TT}%
    \Cnode(-1, 2){TC}\Cnode(0, 2){TG}\Cnode(1, 2){TA}%
    %
                     \Cnode(-2, 1){CT}\Cnode( 2, 1){AT}%
    \Cnode(-3, 0){CC}\Cnode(-2, 0){CA}\Cnode( 2, 0){AC}\Cnode( 3, 0){AA}%
                     \Cnode(-2,-1){CG}\Cnode( 2,-1){AG}%
    %
    \Cnode(-1,-2){GC}\Cnode(0,-2){GT}\Cnode(1,-2){GA}%
                     \Cnode(0,-3){GG}%
    }
  %
  \rput(GA){$\symG\symA$}\rput(GT){$\symG\symT$}\rput(GC){$\symG\symC$}\rput(GG){$\symG\symG$}%
  \rput(CA){$\symC\symA$}\rput(CT){$\symC\symT$}\rput(CC){$\symC\symC$}\rput(CG){$\symC\symG$}%
  \rput(TA){$\symT\symA$}\rput(TT){$\symT\symT$}\rput(TC){$\symT\symC$}\rput(TG){$\symT\symG$}%
  \rput(AA){$\symA\symA$}\rput(AT){$\symA\symT$}\rput(AC){$\symA\symC$}\rput(AG){$\symA\symG$}%
  %
  \ncline{->}{AA}{AT}\ncline{->}{AA}{AC}\ncline{->}{AA}{AG}%
  \ncline{->}{GG}{GA}\ncline{->}{GG}{GT}\ncline{->}{GG}{GC}%
  \ncline{->}{CC}{CT}\ncline{->}{CC}{CA}\ncline{->}{CC}{CG}%
  \ncline{->}{TT}{TA}\ncline{->}{TT}{TC}\ncline{->}{TT}{TG}%
  %
  {\psset{doubleline=false,linecolor=red,arcangle=10}%
    \ncarc{->}{AT}{TA}\ncarc{->}{TA}{AT}%
    \ncarc{->}{AC}{CA}\ncarc{->}{CA}{AC}%
    \ncarc{->}{AG}{GA}\ncarc{->}{GA}{AG}%
    \ncarc{->}{CT}{TC}\ncarc{->}{TC}{CT}
    \ncarc{->}{CG}{GC}\ncarc{->}{GC}{CG}%
    \ncarc{->}{GT}{TG}\ncarc{->}{TG}{GT}%
    }%
  %
  \ncline{->}{AT}{TC}\ncline{->}{AT}{TG}\ncarc[arcangle=-45,linecolor=green]{->}{AT}{TT}%
  \ncline{->}{AC}{CT}\ncline{->}{AC}{CG}\ncarc[arcangle= 45,linecolor=green]{->}{AC}{CC}%
  \ncline{->}{AG}{GT}\ncline{->}{AG}{GC}\ncarc[arcangle= 45,linecolor=green]{->}{AG}{GG}%
  \ncline{->}{TA}{AC}\ncline{->}{TA}{AG}\ncarc[arcangle= 45,linecolor=green]{->}{TA}{AA}%
  \ncline{->}{TC}{CA}\ncline{->}{TC}{CG}\ncarc[arcangle=-45,linecolor=green]{->}{TC}{CC}%
  \ncline{->}{TG}{GA}\ncline{->}{TG}{GC}\ncarc[arcangle= 45,linecolor=green]{->}{TG}{GG}%
  \ncline{->}{CA}{AT}\ncline{->}{CA}{AG}\ncarc[arcangle= 45,linecolor=green]{->}{CA}{AA}%
  \ncline{->}{CT}{TA}\ncline{->}{CT}{TG}\ncarc[arcangle= 45,linecolor=green]{->}{CT}{TT}%
  \ncline{->}{CG}{GA}\ncline{->}{CG}{GT}\ncarc[arcangle=-45,linecolor=green]{->}{CG}{GG}%
  \ncline{->}{GA}{AT}\ncline{->}{GA}{AC}\ncarc[arcangle=-45,linecolor=green]{->}{GA}{AA}%
  \ncline{->}{GT}{TA}\ncline{->}{GT}{TC}\ncarc[arcangle= 45,linecolor=green]{->}{GT}{TT}%
  \ncline{->}{GC}{CA}\ncline{->}{GC}{CT}\ncarc[arcangle= 45,linecolor=green]{->}{GC}{CC}%
  %
  \nccircle[linecolor=red,angleA=-90]{->}{AA}{1.5em}%
  \nccircle[linecolor=red,angleA=0]{->}{TT}{1.5em}%
  \nccircle[linecolor=red,angleA=90]{->}{CC}{1.5em}%
  \nccircle[linecolor=red,angleA=180]{->}{GG}{1.5em}%
  %
  %\uput[ 158](D6){$\frac{1}{6}$}
  %\uput[ 150](D5){$\frac{1}{6}$}
  %\uput[ 210](D4){$\frac{1}{6}$}
  %\uput[  22](D3){$\frac{1}{6}$}
  %\uput[ -45](D2){$\frac{1}{6}$}
  %\uput[-158](D1){$\frac{1}{6}$}
\end{pspicture}
  \includegraphics{../common/math/graphics/pdfs/gsp_markov2.pdf}
  \caption{DNA with depth-2 Markov modelling \xref{ex:gsp_markov2}\label{fig:gsp_markov2}}
\end{figure}
%---------------------------------------
\begin{example}[\exmd{GSP with Markov model}]
\label{ex:gsp_markov2}
%---------------------------------------
Markov probability models have often been used in genomic signal processing (GSP).
A change in the statistics in the sequence may in some cases mean a change in function of the genomic sequence (DNA code).
Finding such a change in statistics then is very useful in identifying functions of segments of genomic sequences.
Let $\ocsG$ be an \structe{outcome subspace} \xref{def:ocs} representing a 
Markov model of depth 2 for a genomic sequence as illustrated in \prefpp{fig:gsp_markov2},
with joint and conditional probabilities computed over a finite window.
Let $\ocsH$ be an outcome subspace isomorphic to $\ocsG$, and 
$\rvX$ be a random variable mapping $\ocsG$ to $\omsH$.
A change in the value of the statistic $\ocsE(\rvX)$ over the window then may indicate a change in function
within the genomic sequence.
\end{example}



\end{tabstr}

%\fi

  %============================================================================
% Daniel J. Greenhoe
% XeLaTeX file
% stochastic systems
%============================================================================

%=======================================
\section{Operations on outcome subspaces}
%=======================================
\begin{tabstr}{0.75}
%=======================================
\subsection{Summation}
%=======================================
\begin{figure}[h]%
  \gsize%
  \centering%
  %$\mcom{\begin{array}{N}{%============================================================================
% Daniel J. Greenhoe
% LaTeX file
% discrete metric real dice mapping to linearly ordered O6c
%============================================================================
{%\psset{unit=0.5\psunit}%
\begin{pspicture}(-1.4,-1.4)(1.4,1.4)%
  %---------------------------------
  % options
  %---------------------------------
  \psset{%
    radius=1.25ex,
    labelsep=2.5mm,
    linecolor=blue,%
    }%
  %---------------------------------
  % dice graph
  %---------------------------------
  \rput(0,0){%\psset{unit=2\psunit}%
    \uput{1}[210](0,0){\Cnode[fillstyle=solid,fillcolor=snode](0,0){D4}}%
    \uput{1}[150](0,0){\Cnode[fillstyle=solid,fillcolor=snode](0,0){D5}}%
    \uput{1}[ 90](0,0){\Cnode[fillstyle=solid,fillcolor=snode](0,0){D6}}%
    \uput{1}[ 30](0,0){\Cnode[fillstyle=solid,fillcolor=snode](0,0){D3}}%
    \uput{1}[-30](0,0){\Cnode[fillstyle=solid,fillcolor=snode](0,0){D2}}%
    \uput{1}[-90](0,0){\Cnode[fillstyle=solid,fillcolor=snode](0,0){D1}}%
    }
  \rput(D6){$\diceF$}%
  \rput(D5){$\diceE$}%
  \rput(D4){$\diceD$}%
  \rput(D3){$\diceC$}%
  \rput(D2){$\diceB$}%
  \rput(D1){$\diceA$}%
  %
  \ncline{D5}{D6}%
  \ncline{D4}{D5}\ncline{D4}{D6}%
  \ncline{D3}{D5}\ncline{D3}{D6}%
  \ncline{D2}{D3}\ncline{D2}{D4}\ncline{D2}{D6}%
  \ncline{D1}{D2}\ncline{D1}{D3}\ncline{D1}{D4}\ncline{D1}{D5}%
  %
  \uput[158](D6){$\frac{1}{6}$}
  \uput[150](D5){$\frac{1}{6}$}
  \uput[210](D4){$\frac{1}{6}$}
  \uput[ 22](D3){$\frac{1}{6}$}
  \uput[-45](D2){$\frac{1}{6}$}
  \uput[-158](D1){$\frac{1}{6}$}
\end{pspicture}
}%}\end{array}}{\structe{real die} \xref{ex:realdie}}$
  $\mcom{\begin{array}{N}{\includegraphics{../common/math/graphics/pdfs/ocs_rdie.pdf}}\end{array}}{\structe{real die} \xref{ex:realdie}}$
  \begin{tabular}{c} \Huge$+$                                         \end{tabular}
  %$\mcom{\begin{array}{N}{%============================================================================
% Daniel J. Greenhoe
% LaTeX file
% discrete metric real dice mapping to linearly ordered O6c
%============================================================================
{%\psset{unit=0.5\psunit}%
\begin{pspicture}(-1.4,-1.4)(1.4,1.4)%
  %---------------------------------
  % options
  %---------------------------------
  \psset{%
    radius=1.25ex,
    labelsep=2.5mm,
    linecolor=blue,%
    }%
  %---------------------------------
  % dice graph
  %---------------------------------
  \rput(0,0){%\psset{unit=2\psunit}%
    \uput{1}[210](0,0){\Cnode[fillstyle=solid,fillcolor=snode](0,0){D4}}%
    \uput{1}[150](0,0){\Cnode[fillstyle=solid,fillcolor=snode](0,0){D5}}%
    \uput{1}[ 90](0,0){\Cnode[fillstyle=solid,fillcolor=snode](0,0){D6}}%
    \uput{1}[ 30](0,0){\Cnode[fillstyle=solid,fillcolor=snode](0,0){D3}}%
    \uput{1}[-30](0,0){\Cnode[fillstyle=solid,fillcolor=snode](0,0){D2}}%
    \uput{1}[-90](0,0){\Cnode[fillstyle=solid,fillcolor=snode](0,0){D1}}%
    }
  \rput(D6){$\diceF$}%
  \rput(D5){$\diceE$}%
  \rput(D4){$\diceD$}%
  \rput(D3){$\diceC$}%
  \rput(D2){$\diceB$}%
  \rput(D1){$\diceA$}%
  %
  \ncline{D5}{D6}%
  \ncline{D4}{D5}\ncline{D4}{D6}%
  \ncline{D3}{D5}\ncline{D3}{D6}%
  \ncline{D2}{D3}\ncline{D2}{D4}\ncline{D2}{D6}%
  \ncline{D1}{D2}\ncline{D1}{D3}\ncline{D1}{D4}\ncline{D1}{D5}%
  %
  \uput[158](D6){$\frac{1}{6}$}
  \uput[150](D5){$\frac{1}{6}$}
  \uput[210](D4){$\frac{1}{6}$}
  \uput[ 22](D3){$\frac{1}{6}$}
  \uput[-45](D2){$\frac{1}{6}$}
  \uput[-158](D1){$\frac{1}{6}$}
\end{pspicture}
}%}\end{array}}{\structe{real die} \xref{ex:realdie}}$
  $\mcom{\begin{array}{N}{\includegraphics{../common/math/graphics/pdfs/ocs_rdie.pdf}}\end{array}}{\structe{real die} \xref{ex:realdie}}$
  %\begin{tabular}{c}{%============================================================================
% Daniel J. Greenhoe
% LaTeX file
% discrete metric real dice mapping to linearly ordered O6c
%============================================================================
{%\psset{unit=0.5\psunit}%
\begin{pspicture}(-1.4,-1.4)(1.4,1.4)%
  %---------------------------------
  % options
  %---------------------------------
  \psset{%
    radius=1.25ex,
    labelsep=2.5mm,
    linecolor=blue,%
    }%
  %---------------------------------
  % dice graph
  %---------------------------------
  \rput(0,0){%\psset{unit=2\psunit}%
    \uput{1}[210](0,0){\Cnode[fillstyle=solid,fillcolor=snode](0,0){D4}}%
    \uput{1}[150](0,0){\Cnode[fillstyle=solid,fillcolor=snode](0,0){D5}}%
    \uput{1}[ 90](0,0){\Cnode[fillstyle=solid,fillcolor=snode](0,0){D6}}%
    \uput{1}[ 30](0,0){\Cnode[fillstyle=solid,fillcolor=snode](0,0){D3}}%
    \uput{1}[-30](0,0){\Cnode[fillstyle=solid,fillcolor=snode](0,0){D2}}%
    \uput{1}[-90](0,0){\Cnode[fillstyle=solid,fillcolor=snode](0,0){D1}}%
    }
  \rput(D6){$\diceF$}%
  \rput(D5){$\diceE$}%
  \rput(D4){$\diceD$}%
  \rput(D3){$\diceC$}%
  \rput(D2){$\diceB$}%
  \rput(D1){$\diceA$}%
  %
  \ncline{D5}{D6}%
  \ncline{D4}{D5}\ncline{D4}{D6}%
  \ncline{D3}{D5}\ncline{D3}{D6}%
  \ncline{D2}{D3}\ncline{D2}{D4}\ncline{D2}{D6}%
  \ncline{D1}{D2}\ncline{D1}{D3}\ncline{D1}{D4}\ncline{D1}{D5}%
  %
  \uput[158](D6){$\frac{1}{6}$}
  \uput[150](D5){$\frac{1}{6}$}
  \uput[210](D4){$\frac{1}{6}$}
  \uput[ 22](D3){$\frac{1}{6}$}
  \uput[-45](D2){$\frac{1}{6}$}
  \uput[-158](D1){$\frac{1}{6}$}
\end{pspicture}
}%} \end{tabular}
  \begin{tabular}{c} \Huge$=$                                         \end{tabular}
  %\begin{tabular}{c}{%============================================================================
% Daniel J. Greenhoe
% LaTeX file
% pair of dice
%============================================================================
{\psset{unit=1.5\psunit}%
\begin{pspicture}(-2.2,-1.4)(2.2,1.4)%
  %---------------------------------
  % options
  %---------------------------------
  \psset{%
    radius=1.25ex,
    linecolor=blue,%
    labelsep=2.5mm,
    }%
  %---------------------------------
  % node locations
  %---------------------------------
  \Cnode[fillstyle=solid,fillcolor=snode](0,0){D7}%
  \Cnode(-0.9239,0.3827){D4}\Cnode(-0.9239,-0.3827){D5}\Cnode(0.9239,-0.3827){D9}\Cnode(0.9239,0.3827){D10}%
  \Cnode(-0.3827,0.9239){D3}\Cnode(-0.3827,-0.9239){D6}\Cnode(0.3827,-0.9239){D8}\Cnode(0.3827,0.9239){D11}%
  \Cnode(-2,0){D2}\Cnode(2,0){D12}%
  %---------------------------------
  % edges
  %---------------------------------
  \ncline{D11}{D12}%
  \ncline{D10}{D11}\ncline{D10}{D12}%
  \ncline{D9}{D10}\ncline{D9}{D11}\ncline{D9}{D12}%
  \ncline{D8}{D9}\ncline{D8}{D10}\ncline{D8}{D11}\ncline{D8}{D12}%
  \ncline{D7}{D8}\ncline{D7}{D9}\ncline{D7}{D10}\ncline{D7}{D11}%
  \ncline{D6}{D7}\ncline{D6}{D8}\ncline{D6}{D9}\ncline{D6}{D10}%
  \ncline{D5}{D6}\ncline{D5}{D7}\ncline{D5}{D8}\ncline{D5}{D9}%
  \ncline{D4}{D5}\ncline{D4}{D6}\ncline{D4}{D7}\ncline{D4}{D8}%
  \ncline{D3}{D4}\ncline{D3}{D5}\ncline{D3}{D6}\ncline{D3}{D7}%
  \ncline{D2}{D3}\ncline{D2}{D4}\ncline{D2}{D5}\ncline{D2}{D6}%
  %---------------------------------
  % node labels
  %---------------------------------
  \rput(D12){$12$}%
  \rput(D11){$11$}%
  \rput(D10){$10$}%
  \rput(D9){$9$}%
  \rput(D8){$8$}%
  \rput(D7){$7$}%
  \rput(D6){$6$}%
  \rput(D5){$5$}%
  \rput(D4){$4$}%
  \rput(D3){$3$}%
  \rput(D2){$2$}%
  %---------------------------------
  % probability labels
  %---------------------------------
  \uput[-90](D2){$\frac{1}{36}$}\uput[-90](D12){$\frac{1}{36}$}%
  \uput[135](D3){$\frac{2}{36}$}\uput[45](D11){$\frac{2}{36}$}%
  \uput[  22](D4){$\frac{3}{36}$}\uput[168](D10){$\frac{3}{36}$}%
  \uput[ 135](D5){$\frac{4}{36}$}\uput[45](D9){$\frac{4}{36}$}%
  \uput[-135](D6){$\frac{5}{36}$}\uput[-45](D8){$\frac{5}{36}$}%
  \uput[90](D7){$\frac{6}{36}$}%
  %
  %---------------------------------
  % labels
  %---------------------------------
  %\rput(6.5,1.25){$\rvX(\cdot)$}%
  %\ncline[linestyle=dotted,nodesep=1pt]{->}{xzlabel}{xz}%
  %\ncline[linestyle=dotted,nodesep=1pt]{->}{ylabel}{y}%
\end{pspicture}
}%}\end{tabular}
  \begin{tabular}{c}{\includegraphics{../common/math/graphics/pdfs/pairdice.pdf}}\end{tabular}
    \begin{tabstr}{0.75}%
    $\begin{array}{|r|*{11}{c}|}%
       \hline
         & 2 & 3 & 4 & 5 & 6 & 7 & 8 & 9 & 10 & 11 & 12\\
       \hline
       2 & 0 & 1 & 1 & 1 & 1 & 2 & 2 & 2 & 2  & 3  & 4 \\
       3 & 1 & 0 & 1 & 1 & 1 & 1 & 2 & 2 & 2  & 2  & 3 \\
       4 & 1 & 1 & 0 & 1 & 1 & 1 & 1 & 2 & 2  & 2  & 2 \\
       5 & 1 & 1 & 1 & 0 & 1 & 1 & 1 & 1 & 2  & 2  & 2 \\
       6 & 1 & 1 & 1 & 1 & 0 & 1 & 1 & 1 & 1  & 2  & 2 \\
       7 & 2 & 1 & 1 & 1 & 1 & 0 & 1 & 1 & 1  & 1  & 2 \\
       8 & 2 & 2 & 1 & 1 & 1 & 1 & 0 & 1 & 1  & 1  & 1 \\
       9 & 2 & 2 & 2 & 1 & 1 & 1 & 1 & 0 & 1  & 1  & 1 \\
      10 & 2 & 2 & 2 & 2 & 1 & 1 & 1 & 1 & 0  & 1  & 1 \\
      11 & 3 & 2 & 2 & 2 & 2 & 1 & 1 & 1 & 1  & 0  & 1 \\
      12 & 4 & 3 & 2 & 2 & 2 & 2 & 1 & 1 & 1  & 1  & 0 \\
      \hline
    \end{array}$%
    \end{tabstr}%
  \caption{metrics based on number of die edges for a \structe{pair of real dice} \xref{ex:pairdice}\label{fig:pairdice}}
\end{figure}
\footnotetext{Many many thanks to Katie L. Greenhoe and Jonathan J. Greenhoe for help computing these values.}
%---------------------------------------
\begin{example}[\exmd{pair of dice outcome subspace}]
\label{ex:pairdice}
%---------------------------------------
A \structe{pair of real dice} has a structure as illustrated in \prefpp{fig:pairdice}.
The values represent the standard sum of die faces and thus range from $2$ to $12$.
The table in the figure provides the metric distances between summed values based on the number of edges
that must be transversed to move from the first value to the second value.
Alternatively, the distance is the number of times the dice must be rotated 90 degrees to move from the first 
value being face up to the second value being face up.
This structure is also illustrated in the undirected graph on the right in \pref{fig:pairdice},
along with each value's standard probability.
\end{example}
\begin{proof}
\begin{align*}
  \ocscen(\ocsG)
    &\eqd \argmin_{x\in\ocsG}\max_{y\in\ocsG}\metric{x}{y}\psp(y)
    &&\text{by definition of $\ocscen$ \xref{def:ocscen}}
   \\&=\mathrlap{\argmin_{x\in\ocsG}\max_{y\in\ocsG}
         \setn{\begin{array}{cccccc}
           \metricn(2,2)\psp(2) &\metricn(2,3)\psp(3)  &\metricn(2,4)\psp(4)  &\cdots & \metricn(2,12)\psp(12)\\
           \metricn(3,2)\psp(2) &\metricn(3,3)\psp(3)  &\metricn(3,4)\psp(4)  &\cdots & \metricn(3,12)\psp(12)\\
           \vdots               &\vdots                &\vdots                &\ddots & \vdots                \\
           \metricn(12,2)\psp(2)&\metricn(12,3)\psp(3) &\metricn(12,4)\psp(4) &\cdots & \metricn(12,12)\psp(12)
         \end{array}}}
      \\&=\mathrlap{\argmin_{x\in\ocsG}\max_{y\in\ocsG}\frac{1}{36}
         \setn{\begin{array}{*{12}{c}}
            %  2          3           4           5           6           7           8           9           10         11          12
           0\times1 & 1\times2 & 1\times3 & 1\times4 & 1\times5 & 2\times6 & 2\times5 & 2\times4 & 2\times3 & 3\times2 & 4\times1 \\
           1\times1 & 0\times2 & 1\times3 & 1\times4 & 1\times5 & 1\times6 & 2\times5 & 2\times4 & 2\times3 & 2\times2 & 3\times1 \\
           1\times1 & 1\times2 & 0\times3 & 1\times4 & 1\times5 & 1\times6 & 1\times5 & 2\times4 & 2\times3 & 2\times2 & 2\times1 \\
           1\times1 & 1\times2 & 1\times3 & 0\times4 & 1\times5 & 1\times6 & 1\times5 & 1\times4 & 2\times3 & 2\times2 & 2\times1 \\
           1\times1 & 1\times2 & 1\times3 & 1\times4 & 0\times5 & 1\times6 & 1\times5 & 1\times4 & 1\times3 & 2\times2 & 2\times1 \\
           2\times1 & 1\times2 & 1\times3 & 1\times4 & 1\times5 & 0\times6 & 1\times5 & 1\times4 & 1\times3 & 1\times2 & 2\times1 \\
           2\times1 & 2\times2 & 1\times3 & 1\times4 & 1\times5 & 1\times6 & 0\times5 & 1\times4 & 1\times3 & 1\times2 & 1\times1 \\
           2\times1 & 2\times2 & 2\times3 & 1\times4 & 1\times5 & 1\times6 & 1\times5 & 0\times4 & 1\times3 & 1\times2 & 1\times1 \\
           2\times1 & 2\times2 & 2\times3 & 2\times4 & 1\times5 & 1\times6 & 1\times5 & 1\times4 & 0\times3 & 1\times2 & 1\times1 \\
           3\times1 & 2\times2 & 2\times3 & 2\times4 & 2\times5 & 1\times6 & 1\times5 & 1\times4 & 1\times3 & 0\times2 & 1\times1 \\
           4\times1 & 3\times2 & 2\times3 & 2\times4 & 2\times5 & 2\times6 & 1\times5 & 1\times4 & 1\times3 & 1\times2 & 0\times1 \\
         \end{array}}}
     \\&=\mathrlap{\argmin_{x\in\ocsG}\max_{y\in\ocsG}\frac{1}{36}
         \setn{\begin{array}{*{12}{c}}
           0 & 2 & 3 & 4 & 5 &12 &10 & 8 & 6 & 6 & 4 \\
           1 & 0 & 3 & 4 & 5 & 6 &10 & 8 & 6 & 4 & 3 \\
           1 & 2 & 0 & 4 & 5 & 6 & 5 & 8 & 6 & 4 & 2 \\
           1 & 2 & 3 & 0 & 5 & 6 & 5 & 4 & 6 & 4 & 2 \\
           1 & 2 & 3 & 4 & 0 & 6 & 5 & 4 & 3 & 4 & 2 \\
           2 & 2 & 3 & 4 & 5 & 0 & 5 & 4 & 3 & 2 & 2 \\
           2 & 4 & 3 & 4 & 5 & 6 & 0 & 4 & 3 & 2 & 1 \\
           2 & 4 & 6 & 4 & 5 & 6 & 5 & 0 & 3 & 2 & 1 \\
           2 & 4 & 6 & 8 & 5 & 6 & 5 & 4 & 0 & 2 & 1 \\
           3 & 4 & 6 & 8 &10 & 6 & 5 & 4 & 3 & 0 & 1 \\
           4 & 6 & 6 & 8 &10 &12 & 5 & 4 & 3 & 2 & 0 \\
         \end{array}}
       = \argmin_{x\in\ocsG}\frac{1}{36}
         \setn{\begin{array}{c}
           12\\% 2
           10\\% 3
            8\\% 4
            6\\% 5
            6\\% 6
            5\\% 7
            6\\% 8
            6\\% 9
            8\\%10
           10\\%11
           12\\%12
         \end{array}}
  \quad= \setn{7}}
  \\
  \ocscena(\ocsG)
    &\eqd \argmin_{x\in\ocsG}\sum_{y\in\ocsG}\metric{x}{y}\psp(y)
    &&\text{by definition of $\ocscena$ \xref{def:ocscenx}}
     \\&=\mathrlap{\argmin_{x\in\ocsG}\frac{1}{36}
         \setn{\begin{array}{*{23}{@{\,}c}}
           0 &+& 2 &+& 3 &+& 4 &+& 5 &+&12 &+&10 &+& 8 &+& 6 &+& 6 &+& 4 \\%  2
           1 &+& 0 &+& 3 &+& 4 &+& 5 &+& 6 &+&10 &+& 8 &+& 6 &+& 4 &+& 3 \\%  3
           1 &+& 2 &+& 0 &+& 4 &+& 5 &+& 6 &+& 5 &+& 8 &+& 6 &+& 4 &+& 2 \\%  4
           1 &+& 2 &+& 3 &+& 0 &+& 5 &+& 6 &+& 5 &+& 4 &+& 6 &+& 4 &+& 2 \\%  5
           1 &+& 2 &+& 3 &+& 4 &+& 0 &+& 6 &+& 5 &+& 4 &+& 3 &+& 4 &+& 2 \\%  6
           2 &+& 2 &+& 3 &+& 4 &+& 5 &+& 0 &+& 5 &+& 4 &+& 3 &+& 2 &+& 2 \\%  7
           2 &+& 4 &+& 3 &+& 4 &+& 5 &+& 6 &+& 0 &+& 4 &+& 3 &+& 2 &+& 1 \\%  8
           2 &+& 4 &+& 6 &+& 4 &+& 5 &+& 6 &+& 5 &+& 0 &+& 3 &+& 2 &+& 1 \\%  9
           2 &+& 4 &+& 6 &+& 8 &+& 5 &+& 6 &+& 5 &+& 4 &+& 0 &+& 2 &+& 1 \\% 10
           3 &+& 4 &+& 6 &+& 8 &+&10 &+& 6 &+& 5 &+& 4 &+& 3 &+& 0 &+& 1 \\% 11
           4 &+& 6 &+& 6 &+& 8 &+&10 &+&12 &+& 5 &+& 4 &+& 3 &+& 2 &+& 0 \\% 12
         \end{array}}
       = \argmin_{x\in\ocsG}\frac{1}{36}
         \setn{\begin{array}{c}
            60\\%  2
            50\\%  3
            43\\%  4
            38\\%  5
            34\\%  6
            32\\%  7
            34\\%  8
            38\\%  9
            43\\% 10
            50\\% 11
            60\\% 12
         \end{array}}
       \quad= \setn{7}}
  \\
  \ocsVar(\ocsG)
    &\eqd \sum_{x\in\ocsG}\brs{\metric{\ocsE(\ocsG)}{x}}^2\psp(x)
    &&\text{by definition of $\ocsVar$ \xref{def:ocsVar}}
  \\&= \sum_{x\in\ocsG}\brs{\metric{7}{x}}^2\psp(x)
    &&\text{by definition of $\ocsVar$ \xref{def:ocsVar}}
  \\&=\mathrlap{\frac{1}{36}\brp{2^2\times1 + 1^2\times2 + 1^2\times3 + 1^2\times4 + 1^2\times5 + 0^2\times6 +
                        1^2\times5 + 1^2\times4 + 1^2\times3 + 1^2\times2 + 2^2\times1}}
  \\&= \frac{1}{36}\brp{4+2+3+4+5+0+5+4+3+2+4}
  \\&= \frac{36}{36}
  \\&= 1
\end{align*}
\end{proof}




The next two examples are examples of sums of \structe{outcome subspace}s \xref{def:ocs}:
\prefp{ex:dicepair_moments} (sum of dice pair) and \prefp{ex:pspinner_xy} (sum of spinner pair).

\begin{figure}[h]
  \gsize%
  \centering%
  %{%============================================================================
% Daniel J. Greenhoe
% LaTeX file
% pair of dice with isomorphic and linear mappings
%============================================================================
\begin{pspicture}(-5.2,-2.5)(11.3,2.7)%
  %---------------------------------
  % options
  %---------------------------------
  \psset{%
    radius=1.25ex,
    linecolor=blue,%
    labelsep=2.5mm,
    }%
  %---------------------------------
  % pair of dice outcome space
  %---------------------------------
  \rput(0,0){\psset{unit=1.5\psunit}%
    \Cnode[fillstyle=solid,fillcolor=snode](0,0){D7}%
    \Cnode(-0.9239,0.3827){D4}\Cnode(-0.9239,-0.3827){D5}\Cnode(0.9239,-0.3827){D9}\Cnode(0.9239,0.3827){D10}%
    \Cnode(-0.3827,0.9239){D3}\Cnode(-0.3827,-0.9239){D6}\Cnode(0.3827,-0.9239){D8}\Cnode(0.3827,0.9239){D11}%
    \Cnode(-2,0){D2}\Cnode(2,0){D12}%
    }%
  \ncline{D11}{D12}%
  \ncline{D10}{D11}\ncline{D10}{D12}%
  \ncline{D9}{D10}\ncline{D9}{D11}\ncline{D9}{D12}%
  \ncline{D8}{D9}\ncline{D8}{D10}\ncline{D8}{D11}\ncline{D8}{D12}%
  \ncline{D7}{D8}\ncline{D7}{D9}\ncline{D7}{D10}\ncline{D7}{D11}%
  \ncline{D6}{D7}\ncline{D6}{D8}\ncline{D6}{D9}\ncline{D6}{D10}%
  \ncline{D5}{D6}\ncline{D5}{D7}\ncline{D5}{D8}\ncline{D5}{D9}%
  \ncline{D4}{D5}\ncline{D4}{D6}\ncline{D4}{D7}\ncline{D4}{D8}%
  \ncline{D3}{D4}\ncline{D3}{D5}\ncline{D3}{D6}\ncline{D3}{D7}%
  \ncline{D2}{D3}\ncline{D2}{D4}\ncline{D2}{D5}\ncline{D2}{D6}%
  %
  \rput(D12){$12$}%
  \rput(D11){$11$}%
  \rput(D10){$10$}%
  \rput(D9){$9$}%
  \rput(D8){$8$}%
  \rput(D7){$7$}%
  \rput(D6){$6$}%
  \rput(D5){$5$}%
  \rput(D4){$4$}%
  \rput(D3){$3$}%
  \rput(D2){$2$}%
  %
  \uput[-90](D2){$\frac{1}{36}$}\uput[-90](D12){$\frac{1}{36}$}%
  \uput[135](D3){$\frac{2}{36}$}\uput[45](D11){$\frac{2}{36}$}%
  \uput[  22](D4){$\frac{3}{36}$}\uput[168](D10){$\frac{3}{36}$}%
  \uput[ 135](D5){$\frac{4}{36}$}\uput[45](D9){$\frac{4}{36}$}%
  \uput[-135](D6){$\frac{5}{36}$}\uput[-45](D8){$\frac{5}{36}$}%
  \uput[90](D7){$\frac{6}{36}$}%
  %
  %---------------------------------
  % random variable mapping X to real line
  %---------------------------------
  \rput(-4.5,-2.8){\psset{unit=0.4\psunit}%
    \pnode(0,13){XB}%
    \pnode(0,12){X12}%
    \pnode(0,11){X11}%
    \pnode(0,10){X10}%
    \pnode(0, 9){X9}%
    \pnode(0, 8){X8}%
    \pnode(0, 7){X7}%
    \pnode(0, 6){X6}%
    \pnode(0, 5){X5}%
    \pnode(0, 4){X4}%
    \pnode(0, 3){X3}%
    \pnode(0, 2){X2}%
    \pnode(0, 1){XA}%
    }
  %\rput(4,-3){\psset{unit=0.5\psunit}%
  %  \pnode( 6,0){XB}%
  %  \pnode( 5,0){X12}%
  %  \pnode( 4,0){X11}%
  %  \pnode( 3,0){X10}%
  %  \pnode( 2,0){X9}%
  %  \pnode( 1,0){X8}%
  %  \pnode( 0,0){X7}%
  %  \pnode(-1,0){X6}%
  %  \pnode(-2,0){X5}%
  %  \pnode(-3,0){X4}%
  %  \pnode(-4,0){X3}%
  %  \pnode(-5,0){X2}%
  %  \pnode(-6,0){XA}%
  %  }
  \ncline{->}{X12}{XB}
  \ncline{X2}{X12}
  \ncline{->}{X2}{XA}
  %
  \pscircle[fillstyle=solid,linecolor=snode,fillcolor=snode](X7){1ex}%
  \pscircle[fillstyle=none,linecolor=red,fillcolor=red](X7){1ex}%
  \rput(X12){\psline(-0.15,0)(0.15,0)}%
  \rput(X11){\psline(-0.15,0)(0.15,0)}%
  \rput(X10){\psline(-0.15,0)(0.15,0)}%
  \rput(X9) {\psline(-0.15,0)(0.15,0)}%
  \rput(X8) {\psline(-0.15,0)(0.15,0)}%
  \rput(X7) {\psline(-0.15,0)(0.15,0)}%
  \rput(X6) {\psline(-0.15,0)(0.15,0)}%
  \rput(X5) {\psline(-0.15,0)(0.15,0)}%
  \rput(X4) {\psline(-0.15,0)(0.15,0)}%
  \rput(X3) {\psline(-0.15,0)(0.15,0)}%
  \rput(X2) {\psline(-0.15,0)(0.15,0)}%
  %
  \uput[180](X12){$12$}%
  \uput[180](X11){$11$}%
  \uput[180](X10){$10$}%
  \uput[180](X9){$9$}%
  \uput[180](X8){$8$}%
  \uput[180](X7){$7$}%
  \uput[180](X6){$6$}%
  \uput[180](X5){$5$}%
  \uput[180](X4){$4$}%
  \uput[180](X3){$3$}%
  \uput[180](X2){$2$}%
  \uput[-22](XB){$\omsR$}%
  %
  %\uput[-90](X2){$\frac{1}{36}$}\uput[-90](X12){$\frac{1}{36}$}%
  %\uput[135](X3){$\frac{2}{36}$}\uput[45](X11){$\frac{2}{36}$}%
  %\uput[  22](X4){$\frac{3}{36}$}\uput[168](X10){$\frac{3}{36}$}%
  %\uput[ 135](X5){$\frac{4}{36}$}\uput[45](X9){$\frac{4}{36}$}%
  %\uput[-135](X6){$\frac{5}{36}$}\uput[-45](X8){$\frac{5}{36}$}%
  %\uput[90](X7){$\frac{6}{36}$}%
  %
  \ncarc[arcangle=22,linewidth=0.75pt,linecolor=red]{->}{D2}{X2}%
  \ncarc[arcangle=22,linewidth=0.75pt,linecolor=red]{->}{D3}{X3}%
  \ncarc[arcangle=22,linewidth=0.75pt,linecolor=red]{->}{D4}{X4}%
  \ncarc[arcangle=22,linewidth=0.75pt,linecolor=red]{->}{D5}{X5}%
  \ncarc[arcangle=22,linewidth=0.75pt,linecolor=red]{->}{D6}{X6}%
  \ncarc[arcangle=22,linewidth=0.75pt,linecolor=red]{->}{D7}{X7}%
  \ncarc[arcangle=45,linewidth=0.75pt,linecolor=red]{->}{D8}{X8}%
  \ncarc[arcangle=-45,linewidth=0.75pt,linecolor=red]{->}{D9}{X9}%
  \ncarc[arcangle=-11,linewidth=0.75pt,linecolor=red]{->}{D10}{X10}%
  \ncarc[arcangle=-22,linewidth=0.75pt,linecolor=red]{->}{D11}{X11}%
  \ncarc[arcangle=-45,linewidth=0.75pt,linecolor=red]{->}{D12}{X12}%
  %---------------------------------
  % random variable mapping Y to isomorphic outcome space
  %---------------------------------
  \rput(8,0){\psset{unit=1.5\psunit}%
    \Cnode[fillstyle=solid,fillcolor=snode](0,0){Y7}%
    \Cnode(-0.9239,0.3827){Y4}\Cnode(-0.9239,-0.3827){Y5}\Cnode(0.9239,-0.3827){Y9}\Cnode(0.9239,0.3827){Y10}%
    \Cnode(-0.3827,0.9239){Y3}\Cnode(-0.3827,-0.9239){Y6}\Cnode(0.3827,-0.9239){Y8}\Cnode(0.3827,0.9239){Y11}%
    \Cnode(-2,0){Y2}\Cnode(2,0){Y12}%
    }
  \ncline{Y11}{Y12}%
  \ncline{Y10}{Y11}\ncline{Y10}{Y12}%
  \ncline{Y9}{Y10}\ncline{Y9}{Y11}\ncline{Y9}{Y12}%
  \ncline{Y8}{Y9}\ncline{Y8}{Y10}\ncline{Y8}{Y11}\ncline{Y8}{Y12}%
  \ncline{Y7}{Y8}\ncline{Y7}{Y9}\ncline{Y7}{Y10}\ncline{Y7}{Y11}%
  \ncline{Y6}{Y7}\ncline{Y6}{Y8}\ncline{Y6}{Y9}\ncline{Y6}{Y10}%
  \ncline{Y5}{Y6}\ncline{Y5}{Y7}\ncline{Y5}{Y8}\ncline{Y5}{Y9}%
  \ncline{Y4}{Y5}\ncline{Y4}{Y6}\ncline{Y4}{Y7}\ncline{Y4}{Y8}%
  \ncline{Y3}{Y4}\ncline{Y3}{Y5}\ncline{Y3}{Y6}\ncline{Y3}{Y7}%
  \ncline{Y2}{Y3}\ncline{Y2}{Y4}\ncline{Y2}{Y5}\ncline{Y2}{Y6}%
  %
  \rput(Y12){$12$}%
  \rput(Y11){$11$}%
  \rput(Y10){$10$}%
  \rput(Y9){$9$}%
  \rput(Y8){$8$}%
  \rput(Y7){$7$}%
  \rput(Y6){$6$}%
  \rput(Y5){$5$}%
  \rput(Y4){$4$}%
  \rput(Y3){$3$}%
  \rput(Y2){$2$}%
  %
  %\uput[-90](Y2){$\frac{1}{36}$}\uput[-90](Y12){$\frac{1}{36}$}%
  %\uput[135](Y3){$\frac{2}{36}$}\uput[45](Y11){$\frac{2}{36}$}%
  %\uput[  22](Y4){$\frac{3}{36}$}\uput[168](Y10){$\frac{3}{36}$}%
  %\uput[ 135](Y5){$\frac{4}{36}$}\uput[45](Y9){$\frac{4}{36}$}%
  %\uput[-135](Y6){$\frac{5}{36}$}\uput[-45](Y8){$\frac{5}{36}$}%
  %\uput[90](Y7){$\frac{6}{36}$}%
  %
  \ncarc[arcangle=67,linewidth=0.75pt,linecolor=green]{->}{D2}{Y2}%
  \ncarc[arcangle=22,linewidth=0.75pt,linecolor=green]{->}{D3}{Y3}%
  \ncarc[arcangle=22,linewidth=0.75pt,linecolor=green]{->}{D4}{Y4}%
  \ncarc[arcangle=-22,linewidth=0.75pt,linecolor=green]{->}{D5}{Y5}%
  \ncarc[arcangle=-22,linewidth=0.75pt,linecolor=green]{->}{D6}{Y6}%
  \ncarc[arcangle=11,linewidth=0.75pt,linecolor=green]{->}{D7}{Y7}%
  \ncarc[arcangle=-22,linewidth=0.75pt,linecolor=green]{->}{D8}{Y8}%
  \ncarc[arcangle=-22,linewidth=0.75pt,linecolor=green]{->}{D9}{Y9}%
  \ncarc[arcangle=22,linewidth=0.75pt,linecolor=green]{->}{D10}{Y10}%
  \ncarc[arcangle=22,linewidth=0.75pt,linecolor=green]{->}{D11}{Y11}%
  \ncarc[arcangle=-67,linewidth=0.75pt,linecolor=green]{->}{D12}{Y12}%
  %---------------------------------
  % labels
  %---------------------------------
  \rput[t](0,2.3){$\ocsG$}%
  \rput[t](8,2.3){$\ocsH$}%
  \rput(-3.8,0){$\rvX(\cdot)$}%
  \rput( 4,0){$\rvY(\cdot)$}%
  %\ncline[linestyle=dotted,nodesep=1pt]{->}{xzlabel}{xz}%
  %\ncline[linestyle=dotted,nodesep=1pt]{->}{ylabel}{y}%
\end{pspicture}%}%
  {\includegraphics{../common/math/graphics/pdfs/pdice_xy.pdf}}%
  \caption{\structe{pair of dice} mappings \xref{ex:wdie_xy}\label{fig:pdice_xy}}
\end{figure}
%---------------------------------------
\begin{example}[\exmd{pair of dice and hypothesis testing}]
\label{ex:dicepair_moments}
%---------------------------------------
Let $\ocsG$ be the \structe{pair of dice outcome subspace} \xref{ex:pairdice},
    $\rvX\in\clOCSgr$ an \fncte{outcome random variable} mapping from $\ocsG$ to the \structe{real line} \xref{def:Rline},
and $\rvX\in\clOCSgh$ a mapping to a structure $\omsH$ that is \prope{isomorphic} to $\ocsG$,
as illustrated in \prefpp{fig:pdice_xy}.
This yields the following statistics:
\\$\begin{array}{>{\footnotesize}Mlcc lcccl}
  geometry of $\ocsG$:                                          & \ocscen(\ocsG)=\ocscena(\ocsG) &=& \setn{7}     & \ocsVaro(\ocsG) &=& 1\\
  traditional statistics on real line $\omsR$:                  & \pE  (\rvX)     &=& 7            & \ocsVar(\rvW;\pE)   &=& \frac{35}{6}&\approx& 5.833\\
  outcome subspace statistics on real line $\omsR$:             & \ocsE(\rvX)     &=& \setn{7}     & \ocsVar(\rvW;\ocsE) &=& \frac{35}{6}&\approx& 5.833\\
  outcome subspace statistics on  isomorphic structure $\ocsH$: & \ocsE(\rvY)     &=& \setn{7}     & \ocsVar(\rvY;\ocsE) &=& 1           &       &
\end{array}$  
\\
Although the expected values of both \structe{outcome subspace}s are the essentially the same ($7$ and $\setn{7}$), 
the isomorphic structure $\omsH$ yields a much smaller variance (a much smaller expected error).
This is significant in statistical applications such as hypothesis testing. 
Suppose for example we have two pair of \structe{real dice} \xref{ex:realdie}, 
one pair being made of two uniformly distributed die and one pair of weighted die. 
We want to know which pair is the uniform die.
So we roll each pair one time. Suppose the outcome of the first pair is $11$ and the 
the outcome of the second pair is $6$. Which pair is more likely to be the uniform pair?
Using traditional statistical analysis, the answer is the second pair, because it is closer to the 
expected value ($0.414$ standard deviations as opposed to $1.656$ standard deviations).
However, this result is deceptive, because as can be seen in \prefpp{fig:pairdice},
the distance from the expected value to the values $11$ and $6$ are the same ($\metric{7}{11}=\metric{7}{6}=1 = 1$ standard deviation).
So arguably the outcome of the single roll test would contribute nothing to a good decision algorithm.
\end{example}
\begin{proof}
    \begin{align*}
      \ocscen (\ocsG) &= \setn{7}  && \text{by \prefpp{ex:pairdice}}
      \\
      \ocsVar (\ocsG) &= 1         && \text{by \prefpp{ex:pairdice}}
      \\
      \pE(\rvX)
        &\eqd\int_{x\in\R} x\psp(x)\dx
        && \text{by definition of $\pE$ \xref{def:pE}}
      \\&\eqd\frac{1}{36}\sum_{x\in\Z} x36\psp(x)
        && \text{by definition of $\psp$}
      %\\&= \frac{1}{36}\sum_{x\in\R} x 36\psp(x)
      \\&= \mathrlap{\frac{1}{36}\brp{2\times1 + 3\times2 + 4\times3 +5\times4 + 6\times5 + 7\times6 + 
                            8\times5 + 9\times4 + 10\times3 + 11\times2 + 12\times1}}
      \\&= \frac{252}{36}
         = 7
      \\
      \ocsVar(\rvX;\pE)
        &= \pVar(\rvX)
        && \text{by \prefp{thm:ocsVar}}
      \\&= \int_{x\in\R} \brs{x-\pE(\rvX)}^2 \psp(x)\dx
        &&\text{by definition of $\pVar$ \xref{def:pVar}}
      \\&= \sum_{x\in\Z} \brs{x-\pE(\rvX)}^2 \psp(x)
        &&\text{by definition of $\psp$}
      \\&= \sum_{x\in\Z} \brp{x-7}^2 \frac{1}{36}
        &&\text{by $\pE(\rvX)$ result}
      \\&= \mathrlap{2\frac{25\times1 + 16\times2 + 9\times3 + 4\times4 + 1\times 5}{36}
         = \frac{35}{6} \approx 5.833}
      \\
      \ocsE(\rvX)
        &= \pE(\rvX)
        && \text{by \prefpp{thm:pEocsE}}
      \\&= \setn{7}
        && \text{by $\pE(\rvX)$ result}
      \\
      \ocsVar(\rvX;\ocsE)
        &= \ocsVar(\rvX;\pE)
        && \text{because $\ocsE(\rvX)\equiv\pE(\rvX)$}
      \\&= \frac{35}{6} \approx 5.833
        && \text{by $\ocsVar(\rvX;\pE)$ result}
      \\
      \ocsE(\rvY)
        &= \rvY\brs{\ocscen(\ocsG)}
        && \text{because $\ocsG$ and $\omsH$ are \prope{isometric} under $\rvY$}
      \\&= \rvY\brs{\setn{7}}
        && \text{by $\ocscen(\ocsG)$ result}
      \\&= \setn{7}
        && \text{by definition of $\rvY$}
      \\
      \ocsVar(\rvY;\ocsE)
        &= \ocsVaro(\ocsG)
        && \text{because $\ocsG$ and $\omsH$ are \prope{isometric} under $\rvY$}
      \\&= 1
        && \text{by $\ocsVaro(\ocsG)$ result}
  \end{align*}
\end{proof}


\begin{figure}[h]
  \gsize%
  \centering%
  %$\mcom{\begin{array}{N}{%============================================================================
% Daniel J. Greenhoe
% LaTeX file
% spinner 6 mapping to linearly ordered L6
%============================================================================
{%\psset{unit=0.5\psunit}%
\begin{pspicture}(-1.5,-1.5)(1.5,1.5)%
  %---------------------------------
  % options
  %---------------------------------
  \psset{%
    linecolor=blue,%
    radius=1.25ex,
    labelsep=2.5mm,
    }%
  %---------------------------------
  % spinner graph
  %---------------------------------
  \rput(0,0){%\psset{unit=2\psunit}%
    \uput{1}[210](0,0){\Cnode[fillstyle=solid,fillcolor=snode](0,0){D6}}%
    \uput{1}[150](0,0){\Cnode[fillstyle=solid,fillcolor=snode](0,0){D5}}%
    \uput{1}[ 90](0,0){\Cnode[fillstyle=solid,fillcolor=snode](0,0){D4}}%
    \uput{1}[ 30](0,0){\Cnode[fillstyle=solid,fillcolor=snode](0,0){D3}}%
    \uput{1}[-30](0,0){\Cnode[fillstyle=solid,fillcolor=snode](0,0){D2}}%
    \uput{1}[-90](0,0){\Cnode[fillstyle=solid,fillcolor=snode](0,0){D1}}%
    }
  \rput[-150](D6){$\circSix$}%
  \rput[ 150](D5){$\circFive$}%
  \rput[  90](D4){$\circFour$}%
  \rput[  30](D3){$\circThree$}%
  \rput[   0](D2){$\circTwo$}%
  \rput[ -90](D1){$\circOne$}%
  %
  \ncline{D6}{D1}%
  \ncline{D5}{D6}%
  \ncline{D4}{D5}%
  \ncline{D3}{D4}%
  \ncline{D2}{D3}%
  \ncline{D1}{D2}%
  %
  \uput[ 210](D6){$\frac{1}{6}$}
  \uput[ 150](D5){$\frac{1}{6}$}
  \uput[  22](D4){$\frac{1}{6}$}
  \uput[  30](D3){$\frac{1}{6}$}
  \uput[ -30](D2){$\frac{1}{6}$}
  \uput[ -22](D1){$\frac{1}{6}$}
  %
  %\uput[ 210](D6){${\scy\psp(\circSix)=}\frac{1}{6}$}
  %\uput[ 150](D5){${\scy\psp(\circFive)=}\frac{1}{6}$}
  %\uput[  22](D4){${\scy\psp(\circFour)=}\frac{1}{6}$}
  %\uput[  30](D3){${\scy\psp(\circThree)=}\frac{1}{6}$}
  %\uput[ -30](D2){${\scy\psp(\circTwo)=}\frac{1}{6}$}
  %\uput[-22](D1){${\scy\psp(\circOne)=}\frac{1}{6}$}
\end{pspicture}
}%}\end{array}}{\structe{spinner} \xref{ex:spinner}}$
  $\mcom{\begin{array}{N}{\includegraphics{../common/math/graphics/pdfs/ocs_spinner.pdf}}\end{array}}{\structe{spinner} \xref{ex:spinner}}$
  \begin{tabular}{c} \Huge$+$                                         \end{tabular}
  %$\mcom{\begin{array}{N}{%============================================================================
% Daniel J. Greenhoe
% LaTeX file
% spinner 6 mapping to linearly ordered L6
%============================================================================
{%\psset{unit=0.5\psunit}%
\begin{pspicture}(-1.5,-1.5)(1.5,1.5)%
  %---------------------------------
  % options
  %---------------------------------
  \psset{%
    linecolor=blue,%
    radius=1.25ex,
    labelsep=2.5mm,
    }%
  %---------------------------------
  % spinner graph
  %---------------------------------
  \rput(0,0){%\psset{unit=2\psunit}%
    \uput{1}[210](0,0){\Cnode[fillstyle=solid,fillcolor=snode](0,0){D6}}%
    \uput{1}[150](0,0){\Cnode[fillstyle=solid,fillcolor=snode](0,0){D5}}%
    \uput{1}[ 90](0,0){\Cnode[fillstyle=solid,fillcolor=snode](0,0){D4}}%
    \uput{1}[ 30](0,0){\Cnode[fillstyle=solid,fillcolor=snode](0,0){D3}}%
    \uput{1}[-30](0,0){\Cnode[fillstyle=solid,fillcolor=snode](0,0){D2}}%
    \uput{1}[-90](0,0){\Cnode[fillstyle=solid,fillcolor=snode](0,0){D1}}%
    }
  \rput[-150](D6){$\circSix$}%
  \rput[ 150](D5){$\circFive$}%
  \rput[  90](D4){$\circFour$}%
  \rput[  30](D3){$\circThree$}%
  \rput[   0](D2){$\circTwo$}%
  \rput[ -90](D1){$\circOne$}%
  %
  \ncline{D6}{D1}%
  \ncline{D5}{D6}%
  \ncline{D4}{D5}%
  \ncline{D3}{D4}%
  \ncline{D2}{D3}%
  \ncline{D1}{D2}%
  %
  \uput[ 210](D6){$\frac{1}{6}$}
  \uput[ 150](D5){$\frac{1}{6}$}
  \uput[  22](D4){$\frac{1}{6}$}
  \uput[  30](D3){$\frac{1}{6}$}
  \uput[ -30](D2){$\frac{1}{6}$}
  \uput[ -22](D1){$\frac{1}{6}$}
  %
  %\uput[ 210](D6){${\scy\psp(\circSix)=}\frac{1}{6}$}
  %\uput[ 150](D5){${\scy\psp(\circFive)=}\frac{1}{6}$}
  %\uput[  22](D4){${\scy\psp(\circFour)=}\frac{1}{6}$}
  %\uput[  30](D3){${\scy\psp(\circThree)=}\frac{1}{6}$}
  %\uput[ -30](D2){${\scy\psp(\circTwo)=}\frac{1}{6}$}
  %\uput[-22](D1){${\scy\psp(\circOne)=}\frac{1}{6}$}
\end{pspicture}
}%}\end{array}}{\structe{spinner} \xref{ex:spinner}}$
  $\mcom{\begin{array}{N}{\includegraphics{../common/math/graphics/pdfs/ocs_spinner.pdf}}\end{array}}{\structe{spinner} \xref{ex:spinner}}$
  %\begin{tabular}{c}{%============================================================================
% Daniel J. Greenhoe
% LaTeX file
% discrete metric real dice mapping to linearly ordered O6c
%============================================================================
{%\psset{unit=0.5\psunit}%
\begin{pspicture}(-1.4,-1.4)(1.4,1.4)%
  %---------------------------------
  % options
  %---------------------------------
  \psset{%
    radius=1.25ex,
    labelsep=2.5mm,
    linecolor=blue,%
    }%
  %---------------------------------
  % dice graph
  %---------------------------------
  \rput(0,0){%\psset{unit=2\psunit}%
    \uput{1}[210](0,0){\Cnode[fillstyle=solid,fillcolor=snode](0,0){D4}}%
    \uput{1}[150](0,0){\Cnode[fillstyle=solid,fillcolor=snode](0,0){D5}}%
    \uput{1}[ 90](0,0){\Cnode[fillstyle=solid,fillcolor=snode](0,0){D6}}%
    \uput{1}[ 30](0,0){\Cnode[fillstyle=solid,fillcolor=snode](0,0){D3}}%
    \uput{1}[-30](0,0){\Cnode[fillstyle=solid,fillcolor=snode](0,0){D2}}%
    \uput{1}[-90](0,0){\Cnode[fillstyle=solid,fillcolor=snode](0,0){D1}}%
    }
  \rput(D6){$\diceF$}%
  \rput(D5){$\diceE$}%
  \rput(D4){$\diceD$}%
  \rput(D3){$\diceC$}%
  \rput(D2){$\diceB$}%
  \rput(D1){$\diceA$}%
  %
  \ncline{D5}{D6}%
  \ncline{D4}{D5}\ncline{D4}{D6}%
  \ncline{D3}{D5}\ncline{D3}{D6}%
  \ncline{D2}{D3}\ncline{D2}{D4}\ncline{D2}{D6}%
  \ncline{D1}{D2}\ncline{D1}{D3}\ncline{D1}{D4}\ncline{D1}{D5}%
  %
  \uput[158](D6){$\frac{1}{6}$}
  \uput[150](D5){$\frac{1}{6}$}
  \uput[210](D4){$\frac{1}{6}$}
  \uput[ 22](D3){$\frac{1}{6}$}
  \uput[-45](D2){$\frac{1}{6}$}
  \uput[-158](D1){$\frac{1}{6}$}
\end{pspicture}
}%} \end{tabular}
  \begin{tabular}{c} \Huge$=$                                         \end{tabular}
  %\begin{tabular}{c}{%============================================================================
% Daniel J. Greenhoe
% LaTeX file
% pair of dice
%============================================================================
{\psset{unit=1.5\psunit}%
\begin{pspicture}(-1.3,-1.4)(1.3,1.4)%
  %---------------------------------
  % options
  %---------------------------------
  \psset{%
    radius=1.25ex,
    linecolor=blue,%
    labelsep=2.5mm,
    }%
  %---------------------------------
  % node locations
  %---------------------------------
  \rput( 0,0){%\psset{unit=2\psunit}%
    \rput{342}(0,0){\rput(1,0){\Cnode(0,0){S8}}}%
    \rput{306}(0,0){\rput(1,0){\Cnode(0,0){S9}}}%
    \rput{270}(0,0){\rput(1,0){\Cnode(0,0){S10}}}%
    \rput{234}(0,0){\rput(1,0){\Cnode(0,0){S11}}}%
    \rput{198}(0,0){\rput(1,0){\Cnode(0,0){S12}}}%
    \Cnode[fillstyle=solid,fillcolor=snode](0,0){S7}%
    \rput{162}(0,0){\rput(1,0){\Cnode(0,0){S6}}}%
    \rput{126}(0,0){\rput(1,0){\Cnode(0,0){S5}}}%
    \rput{ 90}(0,0){\rput(1,0){\Cnode(0,0){S4}}}%
    \rput{ 54}(0,0){\rput(1,0){\Cnode(0,0){S3}}}%
    \rput{ 18}(0,0){\rput(1,0){\Cnode(0,0){S2}}}%
    }
  %---------------------------------
  % edges
  %---------------------------------
  \ncline{S2}{S3}\ncline{S3}{S4}\ncline{S4}{S5}\ncline{S5}{S6}%
  \ncline{S8}{S9}\ncline{S9}{S10}\ncline{S10}{S11}\ncline{S11}{S12}%
  \ncline{S3}{S8}\ncline{S4}{S9}\ncline{S5}{S10}\ncline{S6}{S11}%
  \ncline{S7}{S2}\ncline{S7}{S6}\ncline{S7}{S8}\ncline{S7}{S12}%
  %---------------------------------
  % node labels
  %---------------------------------
  \rput(S12){$12$}%
  \rput(S11){$11$}%
  \rput(S10){$10$}%
  \rput(S9){$9$}%
  \rput(S8){$8$}%
  \rput(S7){$7$}%
  \rput(S6){$6$}%
  \rput(S5){$5$}%
  \rput(S4){$4$}%
  \rput(S3){$3$}%
  \rput(S2){$2$}%
  %---------------------------------
  % probability labels
  %---------------------------------
  \uput[198](S12){$\frac{1}{36}$}%
  \uput[234](S11){$\frac{2}{36}$}%
  \uput[270](S10){$\frac{3}{36}$}%
  \uput[306](S9) {$\frac{4}{36}$}%
  \uput[342](S8) {$\frac{5}{36}$}%
  \uput[-90](S7) {$\frac{6}{36}$}%
  \uput[162](S6) {$\frac{5}{36}$}%
  \uput[126](S5) {$\frac{4}{36}$}%
  \uput[ 90](S4) {$\frac{3}{36}$}%
  \uput[ 54](S3) {$\frac{2}{36}$}%
  \uput[ 18](S2) {$\frac{1}{36}$}%
  %
  %---------------------------------
  % labels
  %---------------------------------
  %\rput(6.5,1.25){$\rvX(\cdot)$}%
  %\ncline[linestyle=dotted,nodesep=1pt]{->}{xzlabel}{xz}%
  %\ncline[linestyle=dotted,nodesep=1pt]{->}{ylabel}{y}%
\end{pspicture}
}%}\end{tabular}
  \begin{tabular}{c}{\includegraphics{../common/math/graphics/pdfs/pspinner.pdf}}\end{tabular}
%  \\
  %\begin{tabular}{cc}
    \begin{tabstr}{0.75}%
    $\begin{array}{|r|*{11}{c}|}%
       \hline
         & 2 & 3 & 4 & 5 & 6 & 7 & 8 & 9 & 10 & 11 & 12\\
       \hline
       2 & 0 & 1 & 2 & 3 & 2 & 1 & 2 & 3 & 4  & 3  & 2 \\
       3 & 1 & 0 & 1 & 2 & 3 & 2 & 1 & 2 & 3  & 4  & 3 \\
       4 & 2 & 1 & 0 & 1 & 2 & 3 & 2 & 1 & 2  & 3  & 4 \\
       5 & 3 & 2 & 1 & 0 & 1 & 2 & 3 & 2 & 1  & 2  & 3 \\
       6 & 2 & 3 & 2 & 1 & 0 & 1 & 2 & 3 & 2  & 1  & 2 \\
       7 & 1 & 2 & 3 & 2 & 1 & 0 & 1 & 2 & 3  & 2  & 1 \\
       8 & 2 & 1 & 2 & 3 & 2 & 1 & 0 & 1 & 2  & 3  & 2 \\
       9 & 3 & 2 & 1 & 2 & 3 & 2 & 1 & 0 & 1  & 2  & 3 \\
      10 & 4 & 3 & 2 & 1 & 2 & 3 & 2 & 1 & 0  & 1  & 2 \\
      11 & 3 & 4 & 3 & 2 & 1 & 2 & 3 & 2 & 1  & 0  & 1 \\
      12 & 2 & 3 & 4 & 3 & 2 & 1 & 2 & 3 & 2  & 1  & 0 \\
      \hline
    \end{array}$%
    \end{tabstr}%
  %  &%
  %\\
  %  \begin{tabular}{c}%
  %    \gsize%
  %    %\psset{unit=10mm}%
  %  \end{tabular}
  %  \\%
  %  table form & graph form with standard probabilities and shaded center
  %\end{tabular}
  \caption{metrics based on a pair of spinners \xref{ex:pspinner}\label{fig:pspinner}}
\end{figure}

%---------------------------------------
\begin{example}[\exmd{pair of spinners}]
\label{ex:pspinner}
%---------------------------------------
A pair of \structe{spinner}s \xref{ex:spinner} has a structure as illustrated in \prefpp{fig:pspinner}.
The values represent the standard sum of spinner positions ($1, 2, \ldots, 6$) and thus range from $2$ to $12$.
The table in the figure provides the metric distances between summed values based on how many positions
one must traverse to get from one value to the next (in which ever direction is shortest).
This structure is also illustrated in the undirected graph in the upper right of \pref{fig:pspinner},
along with each value's standard probability.
\end{example}
\begin{proof}
\begin{align*}
  \ocscen(\ocsG)
    &\eqd \argmin_{x\in\ocsG}\max_{y\in\ocsG}\metric{x}{y}\psp(y)
    &&\text{by definition of $\ocscen$ \xref{def:ocscen}}
   \\&=\mathrlap{\argmin_{x\in\ocsG}\max_{y\in\ocsG}
         \setn{\begin{array}{cccccc}
           \metricn(2,2)\psp(2) &\metricn(2,3)\psp(3)  &\metricn(2,4)\psp(4)  &\cdots & \metricn(2,12)\psp(12)\\
           \metricn(3,2)\psp(2) &\metricn(3,3)\psp(3)  &\metricn(3,4)\psp(4)  &\cdots & \metricn(3,12)\psp(12)\\
           \vdots               &\vdots                &\vdots                &\ddots & \vdots                \\
           \metricn(12,2)\psp(2)&\metricn(12,3)\psp(3) &\metricn(12,4)\psp(4) &\cdots & \metricn(12,12)\psp(12)
         \end{array}}}
      \\&=\mathrlap{\argmin_{x\in\ocsG}\max_{y\in\ocsG}\frac{1}{36}
         \setn{\begin{array}{*{12}{c}}
            %  2          3           4           5           6           7           8           9           10         11          12
           0\times1 & 1\times2 & 2\times3 & 3\times4 & 2\times5 & 1\times6 & 2\times5 & 3\times4 & 4\times3  & 3\times2  & 2\times1 \\
           1\times1 & 0\times2 & 1\times3 & 2\times4 & 3\times5 & 2\times6 & 1\times5 & 2\times4 & 3\times3  & 4\times2  & 3\times1 \\
           2\times1 & 1\times2 & 0\times3 & 1\times4 & 2\times5 & 3\times6 & 2\times5 & 1\times4 & 2\times3  & 3\times2  & 4\times1 \\
           3\times1 & 2\times2 & 1\times3 & 0\times4 & 1\times5 & 2\times6 & 3\times5 & 2\times4 & 1\times3  & 2\times2  & 3\times1 \\
           2\times1 & 3\times2 & 2\times3 & 1\times4 & 0\times5 & 1\times6 & 2\times5 & 3\times4 & 2\times3  & 1\times2  & 2\times1 \\
           1\times1 & 2\times2 & 3\times3 & 2\times4 & 1\times5 & 0\times6 & 1\times5 & 2\times4 & 3\times3  & 2\times2  & 1\times1 \\
           2\times1 & 1\times2 & 2\times3 & 3\times4 & 2\times5 & 1\times6 & 0\times5 & 1\times4 & 2\times3  & 3\times2  & 2\times1 \\
           3\times1 & 2\times2 & 1\times3 & 2\times4 & 3\times5 & 2\times6 & 1\times5 & 0\times4 & 1\times3  & 2\times2  & 3\times1 \\
           4\times1 & 3\times2 & 2\times3 & 1\times4 & 2\times5 & 3\times6 & 2\times5 & 1\times4 & 0\times3  & 1\times2  & 2\times1 \\
           3\times1 & 4\times2 & 3\times3 & 2\times4 & 1\times5 & 2\times6 & 3\times5 & 2\times4 & 1\times3  & 0\times2  & 1\times1 \\
           2\times1 & 3\times2 & 4\times3 & 3\times4 & 2\times5 & 1\times6 & 2\times5 & 3\times4 & 2\times3  & 1\times2  & 0\times1   
         \end{array}}}
     \\&=\mathrlap{\argmin_{x\in\ocsG}\max_{y\in\ocsG}\frac{1}{36}
         \setn{\begin{array}{*{12}{c}}
           0 & 2 & 6 &12 &10 & 6 &10 &12 &12  & 6  & 2 \\
           1 & 0 & 3 & 8 &15 &12 & 5 & 8 & 9  & 8  & 3 \\
           2 & 2 & 0 & 4 &10 &18 &10 & 4 & 6  & 6  & 4 \\
           3 & 4 & 3 & 0 & 5 &12 &15 & 8 & 3  & 4  & 3 \\
           2 & 6 & 6 & 4 & 0 & 6 &10 &12 & 6  & 2  & 2 \\
           1 & 4 & 9 & 8 & 5 & 0 & 5 & 8 & 9  & 4  & 1 \\
           2 & 2 & 6 &12 &10 & 6 & 0 & 4 & 6  & 6  & 2 \\
           3 & 4 & 3 & 8 &15 &12 & 5 & 0 & 3  & 4  & 3 \\
           4 & 6 & 6 & 4 &10 &18 &10 & 4 & 0  & 2  & 2 \\
           3 & 8 & 9 & 8 & 5 &12 &15 & 8 & 3  & 0  & 1 \\
           2 & 6 &12 &12 &10 & 6 &10 &12 & 6  & 2  & 0 \\
         \end{array}}
       = \argmin_{x\in\ocsG}\frac{1}{36}
         \setn{\begin{array}{c}
           12\\% 2
           15\\% 3
           18\\% 4
           15\\% 5
           12\\% 6
            9\\% 7
           12\\% 8
           15\\% 9
           18\\%10
           15\\%11
           12%12
         \end{array}}
  \quad= \setn{7}}
  \\
  \ocscena(\ocsG)
    &\eqd \argmin_{x\in\ocsG}\sum_{y\in\ocsG}\metric{x}{y}\psp(y)
    &&\text{by definition of $\ocscena$ \xref{def:ocscenx}}
     \\&=\mathrlap{\argmin_{x\in\ocsG}\frac{1}{36}
         \setn{\begin{array}{*{23}{@{\,}c}}
           0 &+&  2 &+&  6 &+& 12 &+& 10 &+&  6 &+& 10 &+& 12 &+& 12  &+&  6  &+&  2 \\
           1 &+&  0 &+&  3 &+&  8 &+& 15 &+& 12 &+&  5 &+&  8 &+&  9  &+&  8  &+&  3 \\
           2 &+&  2 &+&  0 &+&  4 &+& 10 &+& 18 &+& 10 &+&  4 &+&  6  &+&  6  &+&  4 \\
           3 &+&  4 &+&  3 &+&  0 &+&  5 &+& 12 &+& 15 &+&  8 &+&  3  &+&  4  &+&  3 \\
           2 &+&  6 &+&  6 &+&  4 &+&  0 &+&  6 &+& 10 &+& 12 &+&  6  &+&  2  &+&  2 \\
           1 &+&  4 &+&  9 &+&  8 &+&  5 &+&  0 &+&  5 &+&  8 &+&  9  &+&  4  &+&  1 \\
           2 &+&  2 &+&  6 &+& 12 &+& 10 &+&  6 &+&  0 &+&  4 &+&  6  &+&  6  &+&  2 \\
           3 &+&  4 &+&  3 &+&  8 &+& 15 &+& 12 &+&  5 &+&  0 &+&  3  &+&  4  &+&  3 \\
           4 &+&  6 &+&  6 &+&  4 &+& 10 &+& 18 &+& 10 &+&  4 &+&  0  &+&  2  &+&  2 \\
           3 &+&  8 &+&  9 &+&  8 &+&  5 &+& 12 &+& 15 &+&  8 &+&  3  &+&  0  &+&  1 \\
           2 &+&  6 &+& 12 &+& 12 &+& 10 &+&  6 &+& 10 &+& 12 &+&  6  &+&  2  &+&  0 \\
         \end{array}}
       =\argmin_{x\in\ocsG}\frac{1}{36}
         \setn{\begin{array}{c}
           78\\%  2
           72\\%  3
           66\\%  4
           60\\%  5
           56\\%  6
           54\\%  7
           56\\%  8
           60\\%  9
           66\\% 10
           72\\% 11
           78%   12
         \end{array}}
       \quad= \setn{7}}
  \\
  \ocsVar(\ocsG)
    &\eqd \sum_{x\in\ocsG}\brs{\metric{\ocsE(\ocsG)}{x}}^2\psp(x)
    &&\text{by definition of $\ocsVar$ \xref{def:ocsVar}}
  \\&= \sum_{x\in\ocsG}\brs{\metric{7}{x}}^2\psp(x)
    &&\text{by definition of $\ocsVar$ \xref{def:ocsVar}}
  \\&=\mathrlap{\frac{1}{36}\brp{1^2\times1 + 2^2\times2 + 3^2\times3 + 2^2\times4 + 1^2\times5 + 0^2\times6 +
                        1^2\times5 + 2^2\times4 + 2^2\times3 + 2^2\times2 + 1^2\times1}}
  \\&= \frac{1}{36}\brp{2+8+27+16+5+0+5+16+12+8+1}
  \\&= \frac{100}{36} = \frac{25}{9} = 2\frac{7}{9} \approx 2.778
\end{align*}
\end{proof}




\begin{figure}[h]
  \gsize%
  \centering%
  %{%============================================================================
% Daniel J. Greenhoe
% LaTeX file
% pair of dice with isomorphic and linear mappings
%============================================================================
\begin{pspicture}(-6.2,-2.5)(9.8,2.5)%
  %---------------------------------
  % options
  %---------------------------------
  \psset{%
    radius=1.25ex,
    linecolor=blue,%
    labelsep=2.5mm,
    }%
  %---------------------------------
  % pair of spinners outcome space
  %---------------------------------
  \rput( 0,0){\psset{unit=1.5\psunit}%
    \rput{342}(0,0){\rput(1,0){\Cnode(0,0){S8}}}%
    \rput{306}(0,0){\rput(1,0){\Cnode(0,0){S9}}}%
    \rput{270}(0,0){\rput(1,0){\Cnode(0,0){S10}}}%
    \rput{234}(0,0){\rput(1,0){\Cnode(0,0){S11}}}%
    \rput{198}(0,0){\rput(1,0){\Cnode(0,0){S12}}}%
    \Cnode[fillstyle=solid,fillcolor=snode](0,0){S7}%
    \rput{162}(0,0){\rput(1,0){\Cnode(0,0){S6}}}%
    \rput{126}(0,0){\rput(1,0){\Cnode(0,0){S5}}}%
    \rput{ 90}(0,0){\rput(1,0){\Cnode(0,0){S4}}}%
    \rput{ 54}(0,0){\rput(1,0){\Cnode(0,0){S3}}}%
    \rput{ 18}(0,0){\rput(1,0){\Cnode(0,0){S2}}}%
    }
  \ncline{S2}{S3}\ncline{S3}{S4}\ncline{S4}{S5}\ncline{S5}{S6}%
  \ncline{S8}{S9}\ncline{S9}{S10}\ncline{S10}{S11}\ncline{S11}{S12}%
  \ncline{S3}{S8}\ncline{S4}{S9}\ncline{S5}{S10}\ncline{S6}{S11}%
  \ncline{S7}{S2}\ncline{S7}{S6}\ncline{S7}{S8}\ncline{S7}{S12}%
  %
  \rput(S12){$12$}%
  \rput(S11){$11$}%
  \rput(S10){$10$}%
  \rput(S9){$9$}%
  \rput(S8){$8$}%
  \rput(S7){$7$}%
  \rput(S6){$6$}%
  \rput(S5){$5$}%
  \rput(S4){$4$}%
  \rput(S3){$3$}%
  \rput(S2){$2$}%
  %
  \uput[198](S12){$\frac{1}{36}$}%
  \uput[234](S11){$\frac{2}{36}$}%
  \uput[270](S10){$\frac{3}{36}$}%
  \uput[306](S9) {$\frac{4}{36}$}%
  \uput[342](S8) {$\frac{5}{36}$}%
  \uput[-90](S7) {$\frac{6}{36}$}%
  \uput[162](S6) {$\frac{5}{36}$}%
  \uput[126](S5) {$\frac{4}{36}$}%
  \uput[ 90](S4) {$\frac{3}{36}$}%
  \uput[ 54](S3) {$\frac{2}{36}$}%
  \uput[ 18](S2) {$\frac{1}{36}$}%
  %
  %---------------------------------
  % random variable mapping X to real line
  %---------------------------------
  \rput(-5.5,0){\psset{unit=0.4\psunit}%
    \pnode(0, 6){XB}%
    \pnode(0, 5){X12}%
    \pnode(0, 4){X11}%
    \pnode(0, 3){X10}%
    \pnode(0, 2){X9}%
    \pnode(0, 1){X8}%
    \pnode(0, 0){X7}%
    \pnode(0,-1){X6}%
    \pnode(0,-2){X5}%
    \pnode(0,-3){X4}%
    \pnode(0,-4){X3}%
    \pnode(0,-5){X2}%
    \pnode(0,-6){XA}%
    }
  %\rput(4,-3){\psset{unit=0.5\psunit}%
  %  \pnode( 6,0){XB}%
  %  \pnode( 5,0){X12}%
  %  \pnode( 4,0){X11}%
  %  \pnode( 3,0){X10}%
  %  \pnode( 2,0){X9}%
  %  \pnode( 1,0){X8}%
  %  \pnode( 0,0){X7}%
  %  \pnode(-1,0){X6}%
  %  \pnode(-2,0){X5}%
  %  \pnode(-3,0){X4}%
  %  \pnode(-4,0){X3}%
  %  \pnode(-5,0){X2}%
  %  \pnode(-6,0){XA}%
  %  }
  \ncline{->}{X12}{XB}
  \ncline{X2}{X12}
  \ncline{->}{X2}{XA}
  %
  \pscircle[fillstyle=solid,linecolor=snode,fillcolor=snode](X7){1ex}%
  \pscircle[fillstyle=none,linecolor=red,fillcolor=red](X7){1ex}%
  %
  \rput(X12){\psline(-0.15,0)(0.15,0)}%
  \rput(X11){\psline(-0.15,0)(0.15,0)}%
  \rput(X10){\psline(-0.15,0)(0.15,0)}%
  \rput(X9) {\psline(-0.15,0)(0.15,0)}%
  \rput(X8) {\psline(-0.15,0)(0.15,0)}%
  \rput(X7) {\psline(-0.15,0)(0.15,0)}%
  \rput(X6) {\psline(-0.15,0)(0.15,0)}%
  \rput(X5) {\psline(-0.15,0)(0.15,0)}%
  \rput(X4) {\psline(-0.15,0)(0.15,0)}%
  \rput(X3) {\psline(-0.15,0)(0.15,0)}%
  \rput(X2) {\psline(-0.15,0)(0.15,0)}%
  %
  \uput[180](X12){$12$}%
  \uput[180](X11){$11$}%
  \uput[180](X10){$10$}%
  \uput[180](X9){$9$}%
  \uput[180](X8){$8$}%
  \uput[180](X7){$7$}%
  \uput[180](X6){$6$}%
  \uput[180](X5){$5$}%
  \uput[180](X4){$4$}%
  \uput[180](X3){$3$}%
  \uput[180](X2){$2$}%
  \uput[-22](XB){$\omsR$}%
  %
  %\uput[-90](X2){$\frac{1}{36}$}\uput[-90](X12){$\frac{1}{36}$}%
  %\uput[135](X3){$\frac{2}{36}$}\uput[45](X11){$\frac{2}{36}$}%
  %\uput[  22](X4){$\frac{3}{36}$}\uput[168](X10){$\frac{3}{36}$}%
  %\uput[ 135](X5){$\frac{4}{36}$}\uput[45](X9){$\frac{4}{36}$}%
  %\uput[-135](X6){$\frac{5}{36}$}\uput[-45](X8){$\frac{5}{36}$}%
  %\uput[90](X7){$\frac{6}{36}$}%
  %
  \ncarc[arcangle=22,linewidth=0.75pt,linecolor=red]{->}{S2}{X2}%
  \ncarc[arcangle=22,linewidth=0.75pt,linecolor=red]{->}{S3}{X3}%
  \ncarc[arcangle=22,linewidth=0.75pt,linecolor=red]{->}{S4}{X4}%
  \ncarc[arcangle=22,linewidth=0.75pt,linecolor=red]{->}{S5}{X5}%
  \ncarc[arcangle=22,linewidth=0.75pt,linecolor=red]{->}{S6}{X6}%
  \ncarc[arcangle=22,linewidth=0.75pt,linecolor=red]{->}{S7}{X7}%
  \ncarc[arcangle=45,linewidth=0.75pt,linecolor=red]{->}{S8}{X8}%
  \ncarc[arcangle=-45,linewidth=0.75pt,linecolor=red]{->}{S9}{X9}%
  \ncarc[arcangle=-11,linewidth=0.75pt,linecolor=red]{->}{S10}{X10}%
  \ncarc[arcangle=-22,linewidth=0.75pt,linecolor=red]{->}{S11}{X11}%
  \ncarc[arcangle=-45,linewidth=0.75pt,linecolor=red]{->}{S12}{X12}%
  %---------------------------------
  % random variable mapping Y to isomorphic outcome space
  %---------------------------------
  \rput( 8,0){\psset{unit=1.5\psunit}%
    \rput{342}(0,0){\rput(1,0){\Cnode(0,0){Y8}}}%
    \rput{306}(0,0){\rput(1,0){\Cnode(0,0){Y9}}}%
    \rput{270}(0,0){\rput(1,0){\Cnode(0,0){Y10}}}%
    \rput{234}(0,0){\rput(1,0){\Cnode(0,0){Y11}}}%
    \rput{198}(0,0){\rput(1,0){\Cnode(0,0){Y12}}}%
    \Cnode[fillstyle=solid,fillcolor=snode](0,0){Y7}%
    \rput{162}(0,0){\rput(1,0){\Cnode(0,0){Y6}}}%
    \rput{126}(0,0){\rput(1,0){\Cnode(0,0){Y5}}}%
    \rput{ 90}(0,0){\rput(1,0){\Cnode(0,0){Y4}}}%
    \rput{ 54}(0,0){\rput(1,0){\Cnode(0,0){Y3}}}%
    \rput{ 18}(0,0){\rput(1,0){\Cnode(0,0){Y2}}}%
    }
  \ncline{Y2}{Y3}\ncline{Y3}{Y4}\ncline{Y4}{Y5}\ncline{Y5}{Y6}%
  \ncline{Y8}{Y9}\ncline{Y9}{Y10}\ncline{Y10}{Y11}\ncline{Y11}{Y12}%
  \ncline{Y3}{Y8}\ncline{Y4}{Y9}\ncline{Y5}{Y10}\ncline{Y6}{Y11}%
  \ncline{Y7}{Y2}\ncline{Y7}{Y6}\ncline{Y7}{Y8}\ncline{Y7}{Y12}%
  %
  \rput(Y12){$12$}%
  \rput(Y11){$11$}%
  \rput(Y10){$10$}%
  \rput(Y9){$9$}%
  \rput(Y8){$8$}%
  \rput(Y7){$7$}%
  \rput(Y6){$6$}%
  \rput(Y5){$5$}%
  \rput(Y4){$4$}%
  \rput(Y3){$3$}%
  \rput(Y2){$2$}%
  %
  %\uput[-90](Y2){$\frac{1}{36}$}\uput[-90](Y12){$\frac{1}{36}$}%
  %\uput[135](Y3){$\frac{2}{36}$}\uput[45](Y11){$\frac{2}{36}$}%
  %\uput[  22](Y4){$\frac{3}{36}$}\uput[168](Y10){$\frac{3}{36}$}%
  %\uput[ 135](Y5){$\frac{4}{36}$}\uput[45](Y9){$\frac{4}{36}$}%
  %\uput[-135](Y6){$\frac{5}{36}$}\uput[-45](Y8){$\frac{5}{36}$}%
  %\uput[90](Y7){$\frac{6}{36}$}%
  %
  \ncarc[arcangle=67,linewidth=0.75pt,linecolor=green]{->}{S2}{Y2}%
  \ncarc[arcangle=22,linewidth=0.75pt,linecolor=green]{->}{S3}{Y3}%
  \ncarc[arcangle=22,linewidth=0.75pt,linecolor=green]{->}{S4}{Y4}%
  \ncarc[arcangle=-22,linewidth=0.75pt,linecolor=green]{->}{S5}{Y5}%
  \ncarc[arcangle=-22,linewidth=0.75pt,linecolor=green]{->}{S6}{Y6}%
  \ncarc[arcangle=11,linewidth=0.75pt,linecolor=green]{->}{S7}{Y7}%
  \ncarc[arcangle=-22,linewidth=0.75pt,linecolor=green]{->}{S8}{Y8}%
  \ncarc[arcangle=-22,linewidth=0.75pt,linecolor=green]{->}{S9}{Y9}%
  \ncarc[arcangle=22,linewidth=0.75pt,linecolor=green]{->}{S10}{Y10}%
  \ncarc[arcangle=22,linewidth=0.75pt,linecolor=green]{->}{S11}{Y11}%
  \ncarc[arcangle=-67,linewidth=0.75pt,linecolor=green]{->}{S12}{Y12}%
  %---------------------------------
  % labels
  %---------------------------------
  \rput[t](0.55,2.3){$\ocsG$}%
  \rput[t](8,2.3){$\omsH$}%
  \rput(-3.8,0){$\rvX(\cdot)$}%
  \rput( 4,0){$\rvY(\cdot)$}%
  %\ncline[linestyle=dotted,nodesep=1pt]{->}{xzlabel}{xz}%
  %\ncline[linestyle=dotted,nodesep=1pt]{->}{ylabel}{y}%
\end{pspicture}%}%
  {\includegraphics{../common/math/graphics/pdfs/pspinner_xy.pdf}}%
  \caption{\structe{pair of spinner} mappings \xref{ex:pspinner_xy}\label{fig:pspinner_xy}}
\end{figure}
%---------------------------------------
\begin{example}[\exmd{pair of spinner and hypothesis testing}]
\label{ex:pspinner_xy}
%---------------------------------------
Let $\ocsG$ be a \structe{pair of spinners} \xref{ex:pspinner},
    $\rvX$ a \fncte{random variable} mapping to the \structe{real line} \xref{def:Rline},
and $\rvY$ a \fncte{random variable} mapping to an \structe{ordered metric space} \xref{def:oms}
that is \prope{isomorphic} to $\ocsG$ under $\rvY$,
as illustrated in \prefpp{fig:pspinner_xy}.
This yields the following statistics:
\\$\begin{array}{>{\footnotesize}Mlcc lcccl}
  geometry of $\ocsG$:                                          & \ocscen (\ocsG)=\ocscena(\ocsG) &=& \setn{7}     & \ocsVaro(\ocsG) &=& \frac{25}{9} &\approx& 2.778\\
  traditional statistics on real line $\omsR$:                  & \pE  (\rvX)     &=& 7            & \ocsVar(\rvW;\pE)   &=& \frac{35}{6}&\approx& 5.833\\
  outcome subspace statistics on real line $\omsR$:             & \ocsE(\rvX)     &=& \setn{7}     & \ocsVar(\rvW;\ocsE) &=& \frac{35}{6}&\approx& 5.833\\
  outcome subspace statistics on  isomorphic structure $\ocsH$: & \ocsE(\rvY)     &=& \setn{7}     & \ocsVar(\rvY;\ocsE) &=& 1           &       &
\end{array}$  
\\
Although the expected value of both \structe{outcome subspace}s are the same ($\pE(\rvX)=\ocsE(\rvY)=7$), 
the isomorphic outcome subspace $\ocsH$ yields a much smaller variance (a much smaller expected error).
This is significant in statistical applications such as hypothesis testing. 
Suppose for example we have two pair of \structe{spinners} \xref{ex:spinner}, 
one pair being made of two uniformly distributed spinners, 
and one pair of weighted spinners. We want to estimate which is which.
So we spin each pair one time. Suppose the outcome of the first pair is $12$ and the 
the outcome of the second pair is $10$. Which pair is more likely to be the uniform pair?
Using traditional statistical analysis, the answer is the second pair, because it is closer to the 
expected value ($\metric{7}{10}=\abs{7-10}=3=1.8$ standard deviations as opposed to 
$\metric{7}{12}=\abs{7-12}=5=3$ standard deviations).
However, this result is deceptive, because as can be seen in the table in \prefpp{fig:pspinner},
$12$ is acually closer to the expected value in $\ocsG$ than is $10$
($\metric{7}{12}=1 < 3=\metric{7}{10}$).
So arguably the better choice, based on this one trial, is the first pair.
\end{example}
\begin{proof}
\begin{align*}
  \ocscen (\ocsG) &= \setn{7}                  && \text{by \prefpp{ex:pspinner}}\\
  \ocscena(\ocsG) &= \setn{7}                  && \text{by \prefpp{ex:pspinner}}\\
  \ocsVaro(\ocsG) &= \frac{25}{9}\approx 2.778 && \text{by \prefpp{ex:pspinner}}
  \\
  \pE  (\rvX) &= \frac{252}{36} = 7            && \text{by \prefpp{ex:dicepair_moments}}  \\
  \pVar(\rvX) &= \frac{35}{6}   \approx 5.833  && \text{by \prefpp{ex:dicepair_moments}}
  \\
  \ocsE(\rvX)
    &= \pE(\rvX)
    && \text{because on \structe{real line}, $\psp$ is \prope{symmetric}, and by \prefp{thm:pEocsE}}
  \\&= \setn{7}
    && \text{by $\pE(\rvX)$ result}
  %  &\eqd \argmin_{x\in\R}\max_{y\in\R} \ocsmom(x,y)
  %  &&\text{by definition of $\ocsE$ \xref{def:ocsE}}
  %\\&\eqd \argmin_{x\in\R}\max_{y\in\R} \metric{x}{y}\psp(y)
  %  &&\text{by definition of $\ocsmom$ \xref{def:ocsmom}}
  %\\&= \argmin_{x\in\R}\setn{30,25,20,16,12,9,12,16,20,25,30}
  %  &&\text{by definition of \structe{real line} $\omsR$ \xref{def:Rline}}
  %\\&= \setn{7}
  \\
  \ocsVar(\rvX;\ocsE)
    &= \ocsVar(\rvX;\pE)
    && \text{because $\ocsE(\rvX)=\pE(\rvX)$}
  \\&= \pVar(\rvX)
    && \text{by \prefp{thm:ocsVar}}
  \\&= \frac{35}{6} \approx 5.833
  \\
  \ocsE(\rvY)
    &= \rvY\brs{\ocscen(\ocsG)}
    &&\text{because $\ocsG$ and $\omsH$ are \prope{isomorphic} under $\rvY$}
  \\&= \rvY\brs{\setn{7}}
    &&\text{by \prefpp{ex:pspinner}}
  \\&= \setn{7}
    &&\text{by definition of $\rvX$}
  %\\
  %\ocsEa(\rvX)
  %  &\eqd \argmin_{x\in\ocsH}\sum_{y\in\ocsH}\ocsmom(x,y)
  %  &&\text{by definition of $\ocsEa$ \xref{def:ocsEa}}
  %\\&= \rvX\brs{\argmin_{x\in\ocsG}\sum_{y\in\ocsG}\ocsmom(x,y)}
  %  &&\text{because $\ocsG$ and $\ocsH$ are \prope{isomorphic}}
  %\\&= \rvX\brs{\ocscena(\ocsG)}
  %  &&\text{by definition of $\ocscena$ \xref{def:ocscena}}
  %\\&= \rvX\brs{\setn{7}}
  %  &&\text{by \prefpp{ex:pspinner}}
  %\\&= \setn{7}
  %  &&\text{by definition of $\rvX$}
  \\
  \ocsVar(\rvY;\ocsE)
  %  &\eqd \sum_{x,y\in\ocsH}\brs{\metric{x}{y}}^2\psp(y)
  %  &&\text{by definition of $\ocsVar$ \xref{def:ocsVar}}
  %\\&=    \sum_{y\in\ocsH}\brs{\metric{7}{y}}^2\psp(y)
  %\\&= \frac{1}{36}\brp{2^2\times1 + 1^2\times2 + 1^2\times3 + 1^2\times4 + 1^2\times5 + 0^2\times6 +
  %                      1^2\times5 + 1^2\times4 + 1^2\times3 + 1^2\times2 + 2^2\times1}
  %\\&= \frac{1}{36}\brp{4+2+3+4+5+0+5+4+3+2+4}
  %\\&= \frac{36}{36}
    &= \ocsVar(\ocsG)
    &&\text{because $\ocsG$ and $\omsH$ are \prope{isomorphic} under $\rvY$}
  \\&= 1
    && \text{by \prefpp{ex:pspinner}}
\end{align*}
\end{proof}




%\fi

\begin{figure}[h]
  \gsize%
  \centering%
  %{%============================================================================
% Daniel J. Greenhoe
% XeLaTeX file
%============================================================================
%\psset{unit=8mm}
\begin{pspicture}(-4.5,-0.5)(10.5,2.625)%
  \psset{%
    labelsep=1pt,
    linewidth=1pt,
    }%
  \psaxes[linecolor=axis,yAxis=false]{<->}(0,0)(-4,0)(10,2.5)% x axis
  \psaxes[linecolor=axis,xAxis=false]{ ->}(0,0)(-4,0)(2,2.5)% y axis
  \psline[linewidth=3pt,linecolor=yellow](-3,1.875)(-2,1.5)(2.667,0.333)(3,0.375)(9,2.625)%
  \psline(-3,0.375)(0,0)(9,1.125)% 1/8 |x|
  \psline(-3,0.5)(1,0)(9,1)%       1/8 |x-1|
  \psline(-3,1.875)(2,0)(9,2.625)% 3/8 |x-2|
  \psline(-3,0.75)(3,0)(9,0.75)%   1/8 |x-3|
  \psline(-3,1.75)(4,0)(9,1.25)%   2/8 |x-4|
  %
  \psline[linestyle=dotted,linecolor=red](-2,1.5)(-2,0)%
  \psline[linestyle=dotted,linecolor=red](-2,1.5)(0,1.5)%
  \psline[linestyle=dotted,linecolor=red](2.667,0.333)(2.667,0)%
  \psline[linestyle=dotted,linecolor=red](2.667,0.333)(0,0.333)%
  \psline[linestyle=dotted,linecolor=red](3,0.375)(3,0)%
  \psline[linestyle=dotted,linecolor=red](-2,1.5)(-2,0)%
  \uput[0]{0}(10,0){$x$}%
  \uput[-90]{0}(2.667,0){$\frac{8}{3}$}%
  \uput[180]{0}(0,0.333){$\frac{1}{3}$}%
  \uput[0]{0}(0,1.5){$1.5$}%
\end{pspicture}
}%
  {\includegraphics{../common/math/graphics/pdfs/rline_argminmax_11312.pdf}}%
  \caption{real line addition $\argmin_x\max_y$ calculation graph \xref{ex:rline_11312}\label{fig:rline_11312}}
\end{figure}
%---------------------------------------
\begin{example}[\exmd{linear addition}]
\label{ex:rline_11312}
%---------------------------------------
Let $\rvX$ be a \structe{random variable} \xref{def:ocsrv} mapping to 
a \structe{real line} \structe{ordered metric space} \xref{def:Rline} resulting in probability values of 
\\\indentx$\psp(0)=\psp(2)=\frac{1}{2}$, and $\psp(x)=0$ otherwise.\\
Let $\rvY$ be a \structe{random variable} mapping to 
a \structe{real line} \structe{ordered metric space} resulting in probability values of 
\\\indentx$\psp(0)=\psp(1)=\frac{1}{4}$, $\psp(2)=\frac{1}{2}$, and $\psp(x)=0$ otherwise.\\
Let $\rvZ\eqd\rvX+\rvY$ be the random variable mapping to the \structe{outcome subspace} \xref{def:ocs} induced by 
adding $\rvX$ and $\rvY$ resulting in probabilities 
\\\indentx$\begin{array}{r|ccccc}
  z      & 0 & 1 & 2 & 3 & 4 
  \\\hline
 \psp(z) & \frac{1}{8} & \frac{1}{8} & \frac{3}{8} & \frac{1}{8} & \frac{2}{8}  
\end{array}$, and $\psp(z)=0$ otherwise.
\\
Note that although the traditional expectation $\pE$ \xref{def:pE} \prope{distributes} over addition such that
\\\indentx$\pE(\rvX+\rvY) = \frac{18}{8} = 1 + \frac{5}{4} = \pE(\rvX) + \pE(\rvY)$,\\
the alternative expecation $\ocsE$ \xref{def:ocsE} does \emph{not}:
\\\indentx$\ocsE(\rvX+\rvY) = \frac{8}{3} \neq \frac{7}{3} =  1 + \frac{4}{3} = \ocsE(\rvX) + \ocsE(\rvY)$.
\end{example}
\begin{proof}
    \begin{align*}
      \pE(\rvX)
        &= \cramped{\int_{x\in\R} x\psp(x) \dx}
        && \text{by definition of $\pE$ \xref{def:pE}}
      \\&= \sum_{x\in\Z} x\psp(x) \dx
        && \text{by definition of $\psp$ and \prefp{prop:pE}}
      \\&= 0\times\frac{1}{2} + 2\times\frac{1}{2}
        && \text{by definition of $\psp$}
      \\&= 1
      \\
      \pE(\rvY)
        &= \cramped{\int_{y\in\R} y\psp(y) \dy}
        && \text{by definition of $\pE$ \xref{def:pE}}
      \\&= \sum_{y\in\Z} y\psp(y) \dy
        && \text{by definition of $\psp$ and \prefp{prop:pE}}
      \\&= 0\times\frac{1}{4} + 1\times\frac{1}{4} + 2\times\frac{1}{2}
        && \text{by definition of $\psp$}
      \\&= \frac{5}{4}
      \\
      \pE(\rvZ)
        &= \cramped{\int_{z\in\R} z\psp(z) \dz}
        && \text{by definition of $\pE$ \xref{def:pE}}
      \\&= \sum_{z\in\Z} z\psp(z) \dz
        && \text{by definition of $\psp$ and \prefp{prop:pE}}
      \\&= 0\times\frac{1}{8} + 1\times\frac{1}{8} + 2\times\frac{3}{8} + 3\times\frac{1}{8} + 4\times\frac{2}{8}
        && \text{by definition of $\psp$}
      \\&= \frac{9}{4}
      \\
      \ocsE(\rvX)
        &= \argmin_{x\in\R}\max_{y\in\R} \metric{x}{y}\psp(y)
        && \text{by definition of $\ocsE$ \xref{def:ocsE}}
      \\&= \argmin_{x\in\Z}\max_{y\in\Z} \metric{x}{y}\psp(y)
        && \text{by definition of $\psp(x)$}
      \\&= \argmin_{x\in\Z}\brb{\begin{array}{lM}
             \abs{2-x}\frac{1}{2} & for $x\orel 1$\\
             \abs{x}\frac{1}{2}   & otherwise
           \end{array}}
      \\&= \setn{1}
        && \text{because $\max(x)$ is minimized at $x=1$}
      \\
      \ocsE(\rvY)
        &= \argmin_{x\in\R}\max_{y\in\R} \metric{x}{y}\psp(y)
        && \text{by definition of $\ocsE$ \xref{def:ocsE}}
      \\&= \argmin_{x\in\Z}\max_{y\in\Z} \metric{x}{y}\psp(y)
        && \text{by definition of $\psp(x)$}
      \\&= \argmin_{x\in\Z}\brb{\begin{array}{lM}
             \abs{2-x}\frac{1}{2} & for $\frac{4}{3}\ge x \ge 4$\\
             \abs{x}\frac{1}{4} & otherwise
           \end{array}}
      \\&= \setn{\frac{4}{3}}
        && \text{because $\max(y)$ is minimized at $y=\frac{4}{3}$}
      \\
      \ocsE(\rvZ)
        &= \argmin_{x\in\R}\max_{y\in\R} \metric{x}{y}\psp(y)
        && \text{by definition of $\ocsE$ \xref{def:ocsE}}
      \\&= \argmin_{x\in\Z}\max_{y\in\Z} \metric{x}{y}\psp(y)
        && \text{by definition of $\psp(x)$}
      \\&= \argmin_{x\in\Z}\brb{\begin{array}{lM}
             \abs{x-2}\frac{3}{8} & for $-2 \ge x \ge 3$\\
             \abs{x-4}\frac{2}{8} & for $-2 \le x \le \frac{8}{3}$\\
             \abs{x}\frac{1}{8} &   for $\frac{8}{3} \le x \le 3$
            %\abs{x}\frac{1}{8} &   for $\frac{8}{3} \le x \le 3$ (otherwise)
           \end{array}}
      \\&= \setn{\frac{8}{3}}
        && \text{because $\max(z)$ is minimized at $z=\frac{8}{3}$ (see \prefp{fig:rline_11312})}
    \end{align*}
\end{proof}



%=======================================
\subsection{Multiplication}
%=======================================
\begin{figure}[h]
  \gsize%
  \centering%
  %{%============================================================================
% Daniel J. Greenhoe
% LaTeX file
% spinner non-linear mappings
%============================================================================
\begin{pspicture}(-4.75,-1.6)(4.75,1.6)%
  %---------------------------------
  % options
  %---------------------------------
  \psset{%
    radius=1.25ex,
    labelsep=2.5mm,
    linecolor=blue,%
    }%
  %---------------------------------
  % spinner graph
  %---------------------------------
  \rput(0,0){%\psset{unit=2\psunit}%
    \rput{ 210}(0,0){\rput(1,0){\Cnode[fillstyle=solid,fillcolor=snode](0,0){S5}}}%
    \rput{ 150}(0,0){\rput(1,0){\Cnode(0,0){S4}}}%
    \rput{  90}(0,0){\rput(1,0){\Cnode(0,0){S3}}}%
    \rput{  30}(0,0){\rput(1,0){\Cnode(0,0){S2}}}%
    \rput{ -30}(0,0){\rput(1,0){\Cnode(0,0){S1}}}%
    \rput{ -90}(0,0){\rput(1,0){\Cnode(0,0){S0}}}%
    \rput(0,0){$\ocsG$}%
    }
  \rput(S5){$5$}%
  \rput(S4){$4$}%
  \rput(S3){$3$}%
  \rput(S2){$2$}%
  \rput(S1){$1$}%
  \rput(S0){$0$}%
  %
  \ncline{S5}{S0}%
  \ncline{S4}{S5}%
  \ncline{S3}{S4}%
  \ncline{S2}{S3}%
  \ncline{S1}{S2}%
  \ncline{S0}{S1}%
  %
  \uput[210](S5){$\frac{5}{10}$}
  \uput[150](S4){$\frac{1}{10}$}
  \uput[ 90](S3){$\frac{1}{10}$}
  \uput[ 30](S2){$\frac{1}{10}$}
  \uput[-30](S1){$\frac{1}{10}$}
  \uput[-90](S0){$\frac{1}{10}$}
  %---------------------------------
  % X range graph
  %---------------------------------
  \rput(-3.5,0){%\psset{unit=2\psunit}%
    \rput{ 210}(0,0){\rput(1,0){\Cnode[fillstyle=solid,fillcolor=snode](0,0){X5}}}%
    \rput{ 150}(0,0){\rput(1,0){\Cnode(0,0){X4}}}%
    \rput{  90}(0,0){\rput(1,0){\Cnode(0,0){X3}}}%
    \rput{  30}(0,0){\rput(1,0){\Cnode(0,0){X2}}}%
    \rput{ -30}(0,0){\rput(1,0){\Cnode(0,0){X1}}}%
    \rput{ -90}(0,0){\rput(1,0){\Cnode(0,0){X0}}}%
    \rput(0,0){$\ocsH_1$}%
    }
  \rput(X5){$5$}%
  \rput(X4){$4$}%
  \rput(X3){$3$}%
  \rput(X2){$2$}%
  \rput(X1){$1$}%
  \rput(X0){$0$}%
  %
  \ncline{X5}{X0}%
  \ncline{X4}{X5}%
  \ncline{X3}{X4}%
  \ncline{X2}{X3}%
  \ncline{X1}{X2}%
  \ncline{X0}{X1}%
  %
  \uput[210](X5){$\frac{5}{10}$}
  \uput[150](X4){$\frac{1}{10}$}
  \uput[ 90](X3){$\frac{1}{10}$}
  \uput[ 30](X2){$\frac{1}{10}$}
  \uput[-30](X1){$\frac{1}{10}$}
  \uput[-90](X0){$\frac{1}{10}$}
  %
  \ncarc[arcangle=-22,linewidth=0.75pt,linecolor=red]{->}{S5}{X5}%
  \ncarc[arcangle= 22,linewidth=0.75pt,linecolor=red]{->}{S4}{X4}%
  \ncarc[arcangle=-22,linewidth=0.75pt,linecolor=red]{->}{S3}{X3}%
  \ncarc[arcangle=-22,linewidth=0.75pt,linecolor=red]{->}{S2}{X2}%
  \ncarc[arcangle=-22,linewidth=0.75pt,linecolor=red]{->}{S1}{X1}%
  \ncarc[arcangle= 22,linewidth=0.75pt,linecolor=red]{->}{S0}{X0}%
  %---------------------------------
  % Y=2X range graph
  %---------------------------------
  \rput(3.5,0){%\psset{unit=2\psunit}%
    \rput{ 150}(0,0){\rput(1,0){\Cnode[fillstyle=solid,fillcolor=snode](0,0){Y4}}}%
    \rput{  30}(0,0){\rput(1,0){\Cnode(0,0){Y2}}}%
    \rput{ -90}(0,0){\rput(1,0){\Cnode(0,0){Y0}}}%
    \rput(0,0){$\ocsH_2$}%
    }
  \rput(Y4){$4$}%
  \rput(Y2){$2$}%
  \rput(Y0){$0$}%
  %
  \ncline{Y4}{Y0}%
  \ncline{Y2}{Y4}%
  \ncline{Y0}{Y2}%
  %
  \uput[150](Y4){$\frac{6}{10}$}
  \uput[ 30](Y2){$\frac{2}{10}$}
  \uput[-90](Y0){$\frac{2}{10}$}
  %
  \ncarc[arcangle= 22,linewidth=0.75pt,linecolor=green]{->}{S5}{Y4}%
  \ncarc[arcangle= 22,linewidth=0.75pt,linecolor=green]{->}{S4}{Y2}%
  \ncarc[arcangle= 22,linewidth=0.75pt,linecolor=green]{->}{S3}{Y0}%
  \ncarc[arcangle= 22,linewidth=0.75pt,linecolor=green]{->}{S2}{Y4}%
  \ncarc[arcangle=-22,linewidth=0.75pt,linecolor=green]{->}{S1}{Y2}%
  \ncarc[arcangle=-22,linewidth=0.75pt,linecolor=green]{->}{S0}{Y0}%
  %---------------------------------
  % labels
  %---------------------------------
  \rput(-1.75,0){$\rvX(\cdot)$}%
  \rput(1.75,0){$2\rvX(\cdot)$}%
  %\ncline[linestyle=dotted,nodesep=1pt]{->}{xzlabel}{xz}%
  %\ncline[linestyle=dotted,nodesep=1pt]{->}{ylabel}{y}%
\end{pspicture}%}%
  {\includegraphics{../common/math/graphics/pdfs/spinner_x2x.pdf}}%
  \caption{\structe{pair of spinner} mappings \xref{ex:spinner_x2x}\label{fig:spinner_x2x}}
\end{figure}
%---------------------------------------
\begin{example}[\exmd{ring multiplication}]
\label{ex:spinner_x2x}
%---------------------------------------
Let $\rvX\in\clOCSgh$ be a random variable where $\ocsG$ is the \structe{weighted spinner}s 
illustrated in \prefpp{fig:spinner_x2x}.
Note that, in agreement with \prefpp{cor:ocsrv_Eax_R}, %{thm:EfX},
\\\indentx$\ocsE(2\rvX) = \setn{4} = \setn{2\times5\mod6} = 2\setn{5}\mod6 = 2\ocsE(\rvX)\mod6$ .
\end{example}
\begin{proof}
    \begin{align*}
      \ocscen(\ocsG)
        &\eqd \argmin_{x\in\ocsH}\max_{y\in\ocsH}\metric{x}{y}\psp(y)
        &&\text{by definition of $\ocscen$ \xref{def:ocscen}}
      \\&=\mathrlap{\argmin_{x\in\ocsH}\max_{y\in\ocsH}\frac{1}{10}
             \setn{\begin{array}{cccccc}
               {0}\times1 & {1}\times1 & {2}\times1 &{3}\times1 &{2}\times1 &{1}\times5\\
               {1}\times1 & {0}\times1 & {1}\times1 &{2}\times1 &{3}\times1 &{2}\times5\\
               {2}\times1 & {1}\times1 & {0}\times1 &{1}\times1 &{2}\times1 &{3}\times5\\
               {3}\times1 & {2}\times1 & {1}\times1 &{0}\times1 &{1}\times1 &{2}\times5\\
               {2}\times1 & {3}\times1 & {2}\times1 &{1}\times1 &{0}\times1 &{1}\times5\\
               {1}\times1 & {2}\times1 & {3}\times1 &{2}\times1 &{1}\times1 &{0}\times5
             \end{array}}
      \quad= \argmin_{x\in\ocsH}\frac{1}{10}
             \setn{\begin{array}{ccccccc}
                5\\
               10\\
               15\\
               10\\
                5\\
                3
             \end{array}}}
      \\&= \setn{5}
      \\
      \ocscena(\ocsG)
        &\eqd \argmin_{x\in\ocsH}\sum_{y\in\ocsH}\metric{x}{y}\psp(y)
        &&\text{by definition of $\ocscena$ \xref{def:ocscena}}
      \\&=\mathrlap{\argmin_{x\in\ocsH}\frac{1}{10}
             \setn{\begin{array}{*{11}{@{\,}c}}
               {0}\times1 &+& {1}\times1  &+&  {2}\times1  &+& {3}\times1  &+& {2}\times1  &+& {1}\times5\\
               {1}\times1 &+& {0}\times1  &+&  {1}\times1  &+& {2}\times1  &+& {3}\times1  &+& {2}\times5\\
               {2}\times1 &+& {1}\times1  &+&  {0}\times1  &+& {1}\times1  &+& {2}\times1  &+& {3}\times5\\
               {3}\times1 &+& {2}\times1  &+&  {1}\times1  &+& {0}\times1  &+& {1}\times1  &+& {2}\times5\\
               {2}\times1 &+& {3}\times1  &+&  {2}\times1  &+& {1}\times1  &+& {0}\times1  &+& {1}\times5\\
               {1}\times1 &+& {2}\times1  &+&  {3}\times1  &+& {2}\times1  &+& {1}\times1  &+& {0}\times5
             \end{array}}
         = \argmin_{x\in\ocsH}\frac{1}{10}
             \setn{\begin{array}{c}
               13\\
               17\\
               21\\
               17\\
               13\\
                9
             \end{array}}}
      \\&= \setn{5}
      \\
      %\\
      %\ocsVar(\rvY)
      %  &\eqd \sum_{x\in\ocsH}\ocsmom_2(\ocscen(\ocsG),x)
      %  &&\text{by definition of $\ocsVar$ \xref{def:ocsVar}}
      %\\&= \mathrlap{\ocsmom_2(0,0)+\ocsmom_2(0,1)+\ocsmom_2(0,2)+\ocsmom_2(0,3)+\ocsmom_2(6,4)+\ocsmom_2(6,5)+\ocsmom_2(6,6)}
      %\\&= \brs{\metric{0}{0}}^2\psp(0)+\brs{\metric{0}{1}}^2\psp(1)+\brs{\metric{0}{2}}^2\psp(2)+\brs{\metric{0}{3}}^2\psp(3)
      %  \\&\qquad+\brs{\metric{6}{4}}^2\psp(4)+\brs{\metric{6}{5}}^2\psp(5)+\brs{\metric{6}{6}}^2\psp(6)
      %\\&= \mathrlap{\brs{0}^2\frac{3}{10}+\brs{1}^2\frac{1}{10}+\brs{2}^2\frac{1}{10}+\brs{3}^2\frac{1}{10}+\brs{2}^2\frac{1}{10}+\brs{1}^2\frac{1}{10}+\brs{0}^2\frac{3}{10}}
      %\\&= \frac{1}{10}\brs{19}
      %\\&= 1.9
  %  \begin{align*}
  %    \pE  (\rvX) &= \frac{252}{36} &= 7            && \text{by \prefpp{ex:dicepair_moments}}  \\
  %    \pVar(\rvX) &= \frac{35}{6}   &\approx 5.833  && \text{by \prefpp{ex:dicepair_moments}}
  %  \end{align*}
  %
  %\item alternative statistics of random variable mapping $\rvX$:
  %  \begin{align*}
  %    \ocsE(\rvX)
  %      &\eqd \argmin_{x\in\R}\max_{y\in\R} \ocsmom(x,y)
  %      &&\text{by definition of $\ocsE$ \xref{def:ocsE}}
  %    \\&\eqd \argmin_{x\in\R}\max_{y\in\R} \metric{x}{y}\psp(y)
  %      &&\text{by definition of $\ocsmom$ \xref{def:ocsmom}}
  %    \\&= \argmin_{x\in\R}\setn{30,25,20,16,12,9,12,16,20,25,30}
  %      &&\text{by definition of \structe{real line} $\omsR$ \xref{def:Rline}}
  %    \\&= \setn{7}
  %  \end{align*}
      \ocsE(\rvX)
        &= \rvX\brs{\ocscen(\ocsG)}
        &&\text{because $\ocsG$ and $\ocsH_1$ are \prope{isomorphic}}
      \\&= \rvX\brs{\setn{5}}
        &&\text{by $\ocscen(\ocsG)$ result}
      \\&= \setn{5}
        &&\text{by definition of $\rvX$}
      %\\
      %\ocsEa(\rvX)
      %  &\eqd \argmin_{x\in\ocsH_1}\sum_{y\in\ocsH_1}\ocsmom(x,y)
      %  &&\text{by definition of $\ocsEa$ \xref{def:ocsEa}}
      %\\&= \rvX\brs{\argmin_{x\in\ocsG}\sum_{y\in\ocsG}\ocsmom(x,y)}
      %  &&\text{because $\ocsG$ and $\ocsH_1$ are \prope{isomorphic}}
      %\\&= \rvX\brs{\ocscena(\ocsG)}
      %  &&\text{by definition of $\ocscena$ \xref{def:ocscena}}
      %\\&= \rvX\brs{\setn{5}}
      %  &&\text{by \prefpp{item:pspinner_x2x_geo}}
      %\\&= \setn{5}
      %  &&\text{by definition of $\rvX$}
    %  \\
    %\ocsVar(\rvX)
    %  &\eqd \sum_{x\in\ocsH}\ocsmom_2(\ocsE(\ocsG),x)
    %  &&\text{by definition of $\ocsVar$ \xref{def:ocsVar}}
    %\\&\eqd \sum_{x,y\in\ocsH}\brs{\metric{x}{y}}^2\psp(y)
    %  &&\text{by definition of $\pVar$ \xref{def:ocsmom}}
    %\\&= \ocsVar(\ocsG)
    %\\&= 1
    %  && \text{by \prefpp{ex:pspinner}}
    \\
      \ocsE(2\rvX)
        &\eqd \argmin_{x\in\ocsH_2}\max_{y\in\ocsH_2}\metric{x}{y}\psp(y)
        &&\text{by definition of $\ocsE$ \xref{def:ocsE}}
      \\&=\mathrlap{\argmin_{x\in\ocsH_2}\max_{y\in\ocsH_2}\frac{1}{10}
             \setn{\begin{array}{ccc}
               {0}\times2 & {1}\times2 & {1}\times6\\
               {1}\times2 & {0}\times2 & {1}\times6\\
               {1}\times2 & {1}\times2 & {0}\times6
             \end{array}}
      \quad= \argmin_{x\in\ocsH_2}\frac{1}{10}
             \setn{\begin{array}{c}
                6\\
                6\\
                2\\
             \end{array}}}
      \\&= \setn{4}
      %\\
      %\ocsEa(2\rvX)
      %  &\eqd \argmin_{x\in\ocsH_2}\sum_{y\in\ocsH_2}\ocsmom(x,y)
      %  &&\text{by definition of $\ocscena$ \xref{def:ocscena}}
      %\\&\eqd \argmin_{x\in\ocsH_2}\sum_{y\in\ocsH_2}\metric{x}{y}\psp(y)
      %  &&\text{by definition of $\ocsmom$ \xref{def:ocsmom}}
      %\\&=\mathrlap{\argmin_{x\in\ocsH_2}\frac{1}{10}
      %       \setn{\begin{array}{*{5}{@{\,}c}}
      %         {0}\times2 &+& {1}\times2  &+&  {1}\times6\\
      %         {1}\times2 &+& {0}\times2  &+&  {1}\times6\\
      %         {1}\times2 &+& {1}\times2  &+&  {0}\times6\\
      %       \end{array}}
      %   = \argmin_{x\in\ocsH_2}\frac{1}{10}
      %       \setn{\begin{array}{c}
      %          8\\
      %          8\\
      %          4
      %       \end{array}}}
      %\\&= \setn{4}
      %\\
      %\ocsVar(\rvY)
      %  &\eqd \sum_{x\in\ocsH}\ocsmom_2(\ocscen(\ocsG),x)
      %  &&\text{by definition of $\ocsVar$ \xref{def:ocsVar}}
      %\\&= \mathrlap{\ocsmom_2(0,0)+\ocsmom_2(0,1)+\ocsmom_2(0,2)+\ocsmom_2(0,3)+\ocsmom_2(6,4)+\ocsmom_2(6,5)+\ocsmom_2(6,6)}
      %\\&= \brs{\metric{0}{0}}^2\psp(0)+\brs{\metric{0}{1}}^2\psp(1)+\brs{\metric{0}{2}}^2\psp(2)+\brs{\metric{0}{3}}^2\psp(3)
      %  \\&\qquad+\brs{\metric{6}{4}}^2\psp(4)+\brs{\metric{6}{5}}^2\psp(5)+\brs{\metric{6}{6}}^2\psp(6)
      %\\&= \mathrlap{\brs{0}^2\frac{3}{10}+\brs{1}^2\frac{1}{10}+\brs{2}^2\frac{1}{10}+\brs{3}^2\frac{1}{10}+\brs{2}^2\frac{1}{10}+\brs{1}^2\frac{1}{10}+\brs{0}^2\frac{3}{10}}
      %\\&= \frac{1}{10}\brs{19}
      %\\&= 1.9
    \end{align*}
\end{proof}


%=======================================
\subsection{Metric transformation}
%=======================================
It is possible to use a \fncte{metric transform} \xref{def:mpf} to transform the structure of an 
\structe{outcome subspace} \xref{def:ocs} into a completely different \structe{outcome subspace}. 
This is demonstrated in \prefpp{ex:ocsop_discrete}--\prefpp{ex:ocsop_x1x2}.
Naturally, by doing so one can sometimes even change the geometric \structe{center}s \xxref{def:ocscen}{def:ocscenx}
of the outcome subspaces, and hence also the statistics of random variables that map to/from them.
This is demonstrated in \prefpp{ex:ocsop_x1x2}--\prefpp{ex:rline_11312a}.

%---------------------------------------
\begin{theorem}
\label{thm:ocsop_mpf}
%---------------------------------------
Let $\fphi$ be a \fncte{metric preserving function} \xref{def:mpf}.
Let $\ocsG\eqd\ocsD$ and $\ocsH$ be \structe{outcome subspace}s \xref{def:ocs}.
\thmbox{
  \brb{\begin{array}{FMD}
    (1). & $\fphi(\ocsH)=\ocsG$ & and\\
    (2). & $\fphi$ is \prope{strictly isotone} & and \\
    (3). & $\psp$ is uniform
  \end{array}}
  \quad\implies\quad
  \ocscen(\ocsH)=\ocscen(\ocsG)
  }
\end{theorem}
\begin{proof}
\begin{align*}
  \ocscen(\ocsH)
    &= \fphi\brs{\ocscen(\ocsH)}
    && \text{by hypothesis (1)}
  \\&\eqd \argmin_{x\in\ocsG}\max_{y\in\ocsG}\fphi\brs{\metric{x}{y}}\psp(y)
    && \text{by definition of $\ocscen$ \xref{def:ocscen}}
  \\&= \argmin_{x\in\ocsG}\max_{y\in\ocsG}\fphi\brs{\metric{x}{y}}
    && \text{by hypothesis (3) and \prefp{lem:argminmaxphi}}
  \\&= \argmin_{x\in\ocsG}\max_{y\in\ocsG}\metric{x}{y}
    && \text{by hypothesis (2) and \prefp{lem:argminmaxphi}}
  \\&= \ocscen(\ocsG)
    && \text{by definition of $\ocscen$ \xref{def:ocscen}}
\end{align*}
\end{proof}

\begin{figure}[h]
  \gsize%
  \centering%
  \psset{unit=7.5mm}%
  $\ds
  \text{\Large$\fphi$}
  %\brp{\mcom{\begin{array}{c}{%============================================================================
% Daniel J. Greenhoe
% LaTeX file
% discrete metric real dice mapping to linearly ordered O6c
%============================================================================
{%\psset{unit=0.5\psunit}%
\begin{pspicture}(-1.4,-1.4)(1.4,1.4)%
  %---------------------------------
  % options
  %---------------------------------
  \psset{%
    radius=1.25ex,
    labelsep=2.5mm,
    linecolor=blue,%
    }%
  %---------------------------------
  % dice graph
  %---------------------------------
  \rput(0,0){%\psset{unit=2\psunit}%
    \uput{1}[210](0,0){\Cnode[fillstyle=solid,fillcolor=snode](0,0){D4}}%
    \uput{1}[150](0,0){\Cnode[fillstyle=solid,fillcolor=snode](0,0){D5}}%
    \uput{1}[ 90](0,0){\Cnode[fillstyle=solid,fillcolor=snode](0,0){D6}}%
    \uput{1}[ 30](0,0){\Cnode[fillstyle=solid,fillcolor=snode](0,0){D3}}%
    \uput{1}[-30](0,0){\Cnode[fillstyle=solid,fillcolor=snode](0,0){D2}}%
    \uput{1}[-90](0,0){\Cnode[fillstyle=solid,fillcolor=snode](0,0){D1}}%
    }
  \rput(D6){$\diceF$}%
  \rput(D5){$\diceE$}%
  \rput(D4){$\diceD$}%
  \rput(D3){$\diceC$}%
  \rput(D2){$\diceB$}%
  \rput(D1){$\diceA$}%
  %
  \ncline{D5}{D6}%
  \ncline{D4}{D5}\ncline{D4}{D6}%
  \ncline{D3}{D5}\ncline{D3}{D6}%
  \ncline{D2}{D3}\ncline{D2}{D4}\ncline{D2}{D6}%
  \ncline{D1}{D2}\ncline{D1}{D3}\ncline{D1}{D4}\ncline{D1}{D5}%
  %
  \uput[158](D6){$\frac{1}{6}$}
  \uput[150](D5){$\frac{1}{6}$}
  \uput[210](D4){$\frac{1}{6}$}
  \uput[ 22](D3){$\frac{1}{6}$}
  \uput[-45](D2){$\frac{1}{6}$}
  \uput[-158](D1){$\frac{1}{6}$}
\end{pspicture}
}%}\end{array}}{\structe{real die} \xref{ex:realdie}}}
  \brp{\mcom{\begin{array}{c}{\includegraphics{../common/math/graphics/pdfs/ocs_rdie.pdf}}\end{array}}{\structe{real die} \xref{ex:realdie}}}
  \qquad\text{\Large$=$}\qquad
  \text{\Large$\fphi\circ\fg$}
  %\brp{\mcom{\begin{array}{c}{%============================================================================
% Daniel J. Greenhoe
% LaTeX file
% spinner 6 mapping to linearly ordered L6
%============================================================================
{%\psset{unit=0.5\psunit}%
\begin{pspicture}(-1.5,-1.5)(1.5,1.5)%
  %---------------------------------
  % options
  %---------------------------------
  \psset{%
    linecolor=blue,%
    radius=1.25ex,
    labelsep=2.5mm,
    }%
  %---------------------------------
  % spinner graph
  %---------------------------------
  \rput(0,0){%\psset{unit=2\psunit}%
    \uput{1}[210](0,0){\Cnode[fillstyle=solid,fillcolor=snode](0,0){D6}}%
    \uput{1}[150](0,0){\Cnode[fillstyle=solid,fillcolor=snode](0,0){D5}}%
    \uput{1}[ 90](0,0){\Cnode[fillstyle=solid,fillcolor=snode](0,0){D4}}%
    \uput{1}[ 30](0,0){\Cnode[fillstyle=solid,fillcolor=snode](0,0){D3}}%
    \uput{1}[-30](0,0){\Cnode[fillstyle=solid,fillcolor=snode](0,0){D2}}%
    \uput{1}[-90](0,0){\Cnode[fillstyle=solid,fillcolor=snode](0,0){D1}}%
    }
  \rput[-150](D6){$\circSix$}%
  \rput[ 150](D5){$\circFive$}%
  \rput[  90](D4){$\circFour$}%
  \rput[  30](D3){$\circThree$}%
  \rput[   0](D2){$\circTwo$}%
  \rput[ -90](D1){$\circOne$}%
  %
  \ncline{D6}{D1}%
  \ncline{D5}{D6}%
  \ncline{D4}{D5}%
  \ncline{D3}{D4}%
  \ncline{D2}{D3}%
  \ncline{D1}{D2}%
  %
  \uput[ 210](D6){$\frac{1}{6}$}
  \uput[ 150](D5){$\frac{1}{6}$}
  \uput[  22](D4){$\frac{1}{6}$}
  \uput[  30](D3){$\frac{1}{6}$}
  \uput[ -30](D2){$\frac{1}{6}$}
  \uput[ -22](D1){$\frac{1}{6}$}
  %
  %\uput[ 210](D6){${\scy\psp(\circSix)=}\frac{1}{6}$}
  %\uput[ 150](D5){${\scy\psp(\circFive)=}\frac{1}{6}$}
  %\uput[  22](D4){${\scy\psp(\circFour)=}\frac{1}{6}$}
  %\uput[  30](D3){${\scy\psp(\circThree)=}\frac{1}{6}$}
  %\uput[ -30](D2){${\scy\psp(\circTwo)=}\frac{1}{6}$}
  %\uput[-22](D1){${\scy\psp(\circOne)=}\frac{1}{6}$}
\end{pspicture}
}%}\end{array}}{\structe{spinner} \xref{ex:spinner}}}
  \brp{\mcom{\begin{array}{c}{\includegraphics{../common/math/graphics/pdfs/ocs_spinner.pdf}}\end{array}}{\structe{spinner} \xref{ex:spinner}}}
  \qquad\text{\Large$=$}\qquad
  %\brp{\mcom{\begin{array}{c}{%============================================================================
% Daniel J. Greenhoe
% LaTeX file
% discrete metric real dice mapping to linearly ordered L6
%============================================================================
\begin{pspicture}(-1.4,-1.4)(1.4,1.4)%
  %---------------------------------
  % options
  %---------------------------------
  \psset{%
    linecolor=blue,%
    radius=1.25ex,
    labelsep=2.5mm,
    }%
  %---------------------------------
  % dice graph
  %---------------------------------
  \rput(0,0){%\psset{unit=2\psunit}%
    \uput{1}[210](0,0){\Cnode[fillstyle=solid,fillcolor=snode](0,0){D4}}%
    \uput{1}[150](0,0){\Cnode[fillstyle=solid,fillcolor=snode](0,0){D5}}%
    \uput{1}[ 90](0,0){\Cnode[fillstyle=solid,fillcolor=snode](0,0){D6}}%
    \uput{1}[ 30](0,0){\Cnode[fillstyle=solid,fillcolor=snode](0,0){D3}}%
    \uput{1}[-30](0,0){\Cnode[fillstyle=solid,fillcolor=snode](0,0){D2}}%
    \uput{1}[-90](0,0){\Cnode[fillstyle=solid,fillcolor=snode](0,0){D1}}%
    }%
  \rput(D6){$\diceF$}%
  \rput(D5){$\diceE$}%
  \rput(D4){$\diceD$}%
  \rput(D3){$\diceC$}%
  \rput(D2){$\diceB$}%
  \rput(D1){$\diceA$}%
  %
  \ncline{D5}{D6}%
  \ncline{D4}{D5}\ncline{D4}{D6}%
  \ncline{D3}{D5}\ncline{D3}{D6}%
  \ncline{D2}{D3}\ncline{D2}{D4}\ncline{D2}{D6}%
  \ncline{D1}{D2}\ncline{D1}{D3}\ncline{D1}{D4}\ncline{D1}{D5}%
  \ncline{D3}{D4}%
  \ncline{D2}{D5}%
  \ncline{D1}{D6}%
  %
  \uput[ 158](D6){$\frac{1}{6}$}
  \uput[ 150](D5){$\frac{1}{6}$}
  \uput[ 210](D4){$\frac{1}{6}$}
  \uput[  22](D3){$\frac{1}{6}$}
  \uput[ -45](D2){$\frac{1}{6}$}
  \uput[-158](D1){$\frac{1}{6}$}
\end{pspicture}}\end{array}}{\structe{fair die} \xref{ex:fairdie}}}
  \brp{\mcom{\begin{array}{c}{\includegraphics{../common/math/graphics/pdfs/ocs_fdie.pdf}}\end{array}}{\structe{fair die} \xref{ex:fairdie}}}
  $
  \caption{\fncte{discrete metric preserving function} $\fphi$ on outcome subspaces \xref{ex:ocsop_discrete}\label{fig:ocsop_mpf_discrete}}
\end{figure}
%---------------------------------------
\begin{example}[\exmd{discrete metric transform on outcome subspaces}]
\label{ex:ocsop_discrete}
%---------------------------------------
Let $\fg$ be a function (a \fncte{pullback function} \xrefnp{thm:pullback}) such that
$\fg(\circOne)=\diceA$,
$\fg(\circTwo)=\diceB$,
$\fg(\circThree)=\diceC$,
$\fg(\circFour)=\diceF$,
$\fg(\circFive)=\diceE$, and
$\fg(\circSix)=\diceD$.
Then under the \fncte{dicrete metric preserving function} $\fphi$ \xref{ex:mpf_discrete}
the \structe{real die outcome subspace} \xref{ex:realdie} becomes 
the \structe{fair die outcome subspace} \xref{ex:fairdie},
and under $\fphi\circ\fg$ 
the \structe{spinner outcome subspace} \xref{ex:spinner} also becomes 
the \structe{fair die outcome subspace},
as illustrated in \prefpp{fig:ocsop_mpf_discrete}.
This yields the following geometric statistics:
\\\indentx$\ocscen(\ocsG)=\ocscen(\ocsH)=\setn{\diceA,\diceB,\diceC,\diceD,\diceE,\diceF}$ .\\
%with the equalities being just as predicted by \prefpp{thm:ocsop_mpf}.
\end{example}





\begin{figure}[h]
  \gsize%
  \centering%
  \psset{unit=6mm}%
  $\ds
  \text{\Large$\fphi_{1}$}
  %\brp{\mcom{\begin{array}{c}{%============================================================================
% Daniel J. Greenhoe
% LaTeX file
% spinner 6 mapping to linearly ordered L6
%============================================================================
{%\psset{unit=0.5\psunit}%
\begin{pspicture}(-1.5,-1.5)(1.5,1.5)%
  %---------------------------------
  % options
  %---------------------------------
  \psset{%
    linecolor=blue,%
    radius=1.25ex,
    labelsep=2.5mm,
    }%
  %---------------------------------
  % spinner graph
  %---------------------------------
  \rput(0,0){%\psset{unit=2\psunit}%
    \uput{1}[210](0,0){\Cnode[fillstyle=solid,fillcolor=snode](0,0){D6}}%
    \uput{1}[150](0,0){\Cnode(0,0){D5}}%
    \uput{1}[ 90](0,0){\Cnode[fillstyle=solid,fillcolor=snode](0,0){D4}}%
    \uput{1}[ 30](0,0){\Cnode(0,0){D3}}%
    \uput{1}[-30](0,0){\Cnode(0,0){D2}}%
    \uput{1}[-90](0,0){\Cnode(0,0){D1}}%
    }
  \rput[-150](D6){$\circSix$}%
  \rput[ 150](D5){$\circFive$}%
  \rput[  90](D4){$\circFour$}%
  \rput[  30](D3){$\circThree$}%
  \rput[   0](D2){$\circTwo$}%
  \rput[ -90](D1){$\circOne$}%
  %
  \ncline{D6}{D1}%
  \ncline{D5}{D6}%
  \ncline{D4}{D5}%
  \ncline{D3}{D4}%
  \ncline{D2}{D3}%
  \ncline{D1}{D2}%
  %
  \uput[ 210](D6){$\frac{1}{10}$}
  \uput[ 150](D5){$\frac{4}{10}$}
  \uput[  22](D4){$\frac{1}{10}$}
  \uput[  30](D3){$\frac{1}{10}$}
  \uput[ -30](D2){$\frac{2}{10}$}
  \uput[ -22](D1){$\frac{1}{10}$}
  %
  %\uput[ 210](D6){${\scy\psp(\circSix)=}\frac{1}{6}$}
  %\uput[ 150](D5){${\scy\psp(\circFive)=}\frac{1}{6}$}
  %\uput[  22](D4){${\scy\psp(\circFour)=}\frac{1}{6}$}
  %\uput[  30](D3){${\scy\psp(\circThree)=}\frac{1}{6}$}
  %\uput[ -30](D2){${\scy\psp(\circTwo)=}\frac{1}{6}$}
  %\uput[-22](D1){${\scy\psp(\circOne)=}\frac{1}{6}$}
\end{pspicture}
}%}\end{array}}{\structe{spinner} \xref{ex:spinner}}}
  \brp{\mcom{\begin{array}{c}{\includegraphics{../common/math/graphics/pdfs/ocs_spinner_121141.pdf}}\end{array}}{\structe{spinner}}}
  \text{\Large$=$}
  %\brp{\mcom{\begin{array}{c}{%============================================================================
% Daniel J. Greenhoe
% LaTeX file
% wagon wheel outcome subspace 
%============================================================================
{%\psset{unit=0.5\psunit}%
\begin{pspicture}(-1.5,-1.5)(1.5,1.5)%
  %---------------------------------
  % options
  %---------------------------------
  \psset{%
    linecolor=blue,%
    radius=1.25ex,
    labelsep=2.5mm,
    }%
  %---------------------------------
  % spinner graph
  %---------------------------------
  \rput(0,0){%\psset{unit=2\psunit}%
    \uput{1}[210](0,0){\Cnode(0,0){D6}}%
    \uput{1}[150](0,0){\Cnode[fillstyle=solid,fillcolor=snode](0,0){D5}}%
    \uput{1}[ 90](0,0){\Cnode(0,0){D4}}%
    \uput{1}[ 30](0,0){\Cnode(0,0){D3}}%
    \uput{1}[-30](0,0){\Cnode(0,0){D2}}%
    \uput{1}[-90](0,0){\Cnode(0,0){D1}}%
    }
  \rput[-150](D6){$\circSix$}%
  \rput[ 150](D5){$\circFive$}%
  \rput[  90](D4){$\circFour$}%
  \rput[  30](D3){$\circThree$}%
  \rput[   0](D2){$\circTwo$}%
  \rput[ -90](D1){$\circOne$}%
  %
  \ncline{D6}{D1}%
  \ncline{D5}{D6}%
  \ncline{D4}{D5}%
  \ncline{D3}{D4}%
  \ncline{D2}{D3}%
  \ncline{D1}{D2}%
  %
  \ncline{D1}{D4}%
  \ncline{D2}{D5}%
  \ncline{D3}{D6}%
  %
  \uput[ 210](D6){$\frac{1}{10}$}
  \uput[ 150](D5){$\frac{4}{10}$}
  \uput[  22](D4){$\frac{1}{10}$}
  \uput[  30](D3){$\frac{1}{10}$}
  \uput[ -30](D2){$\frac{2}{10}$}
  \uput[ -22](D1){$\frac{1}{10}$}
  %
  %\uput[ 210](D6){${\scy\psp(\circSix)=}\frac{1}{6}$}
  %\uput[ 150](D5){${\scy\psp(\circFive)=}\frac{1}{6}$}
  %\uput[  22](D4){${\scy\psp(\circFour)=}\frac{1}{6}$}
  %\uput[  30](D3){${\scy\psp(\circThree)=}\frac{1}{6}$}
  %\uput[ -30](D2){${\scy\psp(\circTwo)=}\frac{1}{6}$}
  %\uput[-22](D1){${\scy\psp(\circOne)=}\frac{1}{6}$}
\end{pspicture}
}%}\end{array}}{\structe{wagon wheel}}}
  \brp{\mcom{\begin{array}{c}{\includegraphics{../common/math/graphics/pdfs/ocs_wagon_121141.pdf}}\end{array}}{\structe{wagon wheel}}}
  \text{\Large$.$}
  \quad%
  \text{\Large$\fphi_{2}\circ\fg$}
  %\brp{\mcom{\begin{array}{c}{%============================================================================
% Daniel J. Greenhoe
% LaTeX file
% spinner 6 mapping to linearly ordered L6
%============================================================================
{%\psset{unit=0.5\psunit}%
\begin{pspicture}(-1.5,-1.5)(1.5,1.5)%
  %---------------------------------
  % options
  %---------------------------------
  \psset{%
    linecolor=blue,%
    radius=1.25ex,
    labelsep=2.5mm,
    }%
  %---------------------------------
  % spinner graph
  %---------------------------------
  \rput(0,0){%\psset{unit=2\psunit}%
    \uput{1}[210](0,0){\Cnode[fillstyle=solid,fillcolor=snode](0,0){D6}}%
    \uput{1}[150](0,0){\Cnode(0,0){D5}}%
    \uput{1}[ 90](0,0){\Cnode[fillstyle=solid,fillcolor=snode](0,0){D4}}%
    \uput{1}[ 30](0,0){\Cnode(0,0){D3}}%
    \uput{1}[-30](0,0){\Cnode(0,0){D2}}%
    \uput{1}[-90](0,0){\Cnode(0,0){D1}}%
    }
  \rput[-150](D6){$\circSix$}%
  \rput[ 150](D5){$\circFive$}%
  \rput[  90](D4){$\circFour$}%
  \rput[  30](D3){$\circThree$}%
  \rput[   0](D2){$\circTwo$}%
  \rput[ -90](D1){$\circOne$}%
  %
  \ncline{D6}{D1}%
  \ncline{D5}{D6}%
  \ncline{D4}{D5}%
  \ncline{D3}{D4}%
  \ncline{D2}{D3}%
  \ncline{D1}{D2}%
  %
  \uput[ 210](D6){$\frac{1}{10}$}
  \uput[ 150](D5){$\frac{4}{10}$}
  \uput[  22](D4){$\frac{1}{10}$}
  \uput[  30](D3){$\frac{1}{10}$}
  \uput[ -30](D2){$\frac{2}{10}$}
  \uput[ -22](D1){$\frac{1}{10}$}
  %
  %\uput[ 210](D6){${\scy\psp(\circSix)=}\frac{1}{6}$}
  %\uput[ 150](D5){${\scy\psp(\circFive)=}\frac{1}{6}$}
  %\uput[  22](D4){${\scy\psp(\circFour)=}\frac{1}{6}$}
  %\uput[  30](D3){${\scy\psp(\circThree)=}\frac{1}{6}$}
  %\uput[ -30](D2){${\scy\psp(\circTwo)=}\frac{1}{6}$}
  %\uput[-22](D1){${\scy\psp(\circOne)=}\frac{1}{6}$}
\end{pspicture}
}%}\end{array}}{\structe{spinner} \xref{ex:spinner}}}
  \brp{\mcom{\begin{array}{c}{\includegraphics{../common/math/graphics/pdfs/ocs_spinner_121141.pdf}}\end{array}}{\structe{spinner}}}
  \text{\Large$=$}
  %\brp{\mcom{\begin{array}{c}{%============================================================================
% Daniel J. Greenhoe
% LaTeX file
% discrete metric real dice mapping to linearly ordered O6c
%============================================================================
{%\psset{unit=0.5\psunit}%
\begin{pspicture}(-1.4,-1.4)(1.4,1.4)%
  %---------------------------------
  % options
  %---------------------------------
  \psset{%
    radius=1.25ex,
    labelsep=2.5mm,
    linecolor=blue,%
    }%
  %---------------------------------
  % dice graph
  %---------------------------------
  \rput(0,0){%\psset{unit=2\psunit}%
    \uput{1}[210](0,0){\Cnode[fillstyle=solid,fillcolor=snode](0,0){D4}}%
    \uput{1}[150](0,0){\Cnode[fillstyle=solid,fillcolor=snode](0,0){D5}}%
    \uput{1}[ 90](0,0){\Cnode[fillstyle=solid,fillcolor=snode](0,0){D6}}%
    \uput{1}[ 30](0,0){\Cnode[fillstyle=solid,fillcolor=snode](0,0){D3}}%
    \uput{1}[-30](0,0){\Cnode(0,0){D2}}%
    \uput{1}[-90](0,0){\Cnode[fillstyle=solid,fillcolor=snode](0,0){D1}}%
    }
  \rput(D6){$\diceF$}%
  \rput(D5){$\diceE$}%
  \rput(D4){$\diceD$}%
  \rput(D3){$\diceC$}%
  \rput(D2){$\diceB$}%
  \rput(D1){$\diceA$}%
  %
  \ncline{D5}{D6}%
  \ncline{D4}{D5}\ncline{D4}{D6}%
  \ncline{D3}{D5}\ncline{D3}{D6}%
  \ncline{D2}{D3}\ncline{D2}{D4}\ncline{D2}{D6}%
  \ncline{D1}{D2}\ncline{D1}{D3}\ncline{D1}{D4}\ncline{D1}{D5}%
  %
  \uput[ 158](D6){$\frac{1}{10}$}
  \uput[ 150](D5){$\frac{4}{10}$}
  \uput[ 210](D4){$\frac{1}{10}$}
  \uput[  22](D3){$\frac{1}{10}$}
  \uput[ -45](D2){$\frac{2}{10}$}
  \uput[-158](D1){$\frac{1}{10}$}
\end{pspicture}
}%}\end{array}}{\structe{weighted die} \xref{ex:wdie}}}
  \brp{\mcom{\begin{array}{c}{\includegraphics{../common/math/graphics/pdfs/ocs_wdie_121141.pdf}}\end{array}}{\structe{weighted die}}}
  \text{\Large$.$}
  $
  \caption{\prefpp{ex:mpf_0121} \fncte{metric preserving function} $\fphi_{1}$  and 
           \prefpp{ex:mpf_x1x2} \fncte{metric preserving function} $\fphi_{2}$
           on \structe{spinner outcome subspace} 
           \xxref{ex:ocsop_0121}{ex:ocsop_x1x2}\label{fig:ocsop_x1x2_0121}}
\end{figure}
%---------------------------------------
\begin{example}
\label{ex:ocsop_0121}
%---------------------------------------
Let $\fphi_{1}$ be the \fncte{metric preserving function} defined in \prefpp{ex:mpf_0121}.
Then under $\fphi_{1}$, the \structe{spinner outcome subspace} \xref{ex:spinner} becomes what is here called the 
\structe{wagon wheel output subspace},
as illustrated on the left in \prefpp{fig:ocsop_x1x2_0121}.
Let $\ocsG$ be the \structe{spinner outcome subspace} and $\ocsH$    the \structe{wagon wheel outcome subspace}.
This yields the following geometric statistics:
\\\indentx
 $\ocscen(\ocsG)=\setn{\circFour,\circSix} \qquad \ocscen(\ocsH)=\setn{\circFive}$ .
\\
Note that the metric transform $\fphi_{1}$ also moves the \structe{outcome center} from one that is 
\emph{not} maximally likely, to one that \emph{is}.
\end{example}
\begin{proof}
\begin{align*}
  \ocscen(\ocsG)
    &\eqd \mathrlap{\argmin_{x\in\ocsG}\max_{y\in\ocsG}\metric{x}{y}\psp(y)
    \qquad\text{by definition of $\ocscen$ \xref{def:ocscen}}}
  \\&=\argmin_{x\in\ocsG}\max_{y\in\ocsG}\frac{1}{10}
         \setn{\begin{array}{cccccc}
           {0}\times1&{1}\times2 & {2}\times1 &{3}\times1 &{2}\times4 &{1}\times1\\
           {1}\times1&{0}\times2 & {1}\times1 &{2}\times1 &{3}\times4 &{2}\times1\\
           {2}\times1&{1}\times2 & {0}\times1 &{1}\times1 &{2}\times4 &{3}\times1\\
           {3}\times1&{2}\times2 & {1}\times1 &{0}\times1 &{1}\times4 &{2}\times1\\
           {2}\times1&{3}\times2 & {2}\times1 &{1}\times1 &{0}\times4 &{1}\times1\\
           {1}\times1&{2}\times2 & {3}\times1 &{2}\times1 &{1}\times4 &{0}\times1
         \end{array}}
     &&= \argmin_{x\in\ocsG}\frac{1}{10}
         \setn{\begin{array}{c}
            8\\
           12\\
            8\\
            4\\
            6\\
            4
         \end{array}}
     &&= \setn{\begin{array}{c}
           \mbox{ }\\
           \mbox{ }\\
           \mbox{ }\\
           \circFour\\
           \mbox{ }\\
           \circSix
         \end{array}}
  \\
  \ocscen(\ocsH)
    &\eqd \mathrlap{\argmin_{x\in\ocsG}\max_{y\in\ocsG}\metric{x}{y}\psp(y)
    \qquad\text{by definition of $\ocscen$ \xref{def:ocscen}}}
  \\&=\argmin_{x\in\ocsH}\max_{y\in\ocsH}\frac{1}{10}
         \setn{\begin{array}{cccccc}
           {0}\times1&{1}\times2 & {2}\times1 &{1}\times1 &{2}\times4 &{1}\times1\\
           {1}\times1&{0}\times2 & {1}\times1 &{2}\times1 &{1}\times4 &{2}\times1\\
           {2}\times1&{1}\times2 & {0}\times1 &{1}\times1 &{2}\times4 &{1}\times1\\
           {1}\times1&{2}\times2 & {1}\times1 &{0}\times1 &{1}\times4 &{2}\times1\\
           {2}\times1&{1}\times2 & {2}\times1 &{1}\times1 &{0}\times4 &{1}\times1\\
           {1}\times1&{2}\times2 & {1}\times1 &{2}\times1 &{1}\times4 &{0}\times1
         \end{array}}
    &&= \argmin_{x\in\ocsH}\frac{1}{10}
         \setn{\begin{array}{c}
            8\\
            4\\
            8\\
            4\\
            2\\
            4
         \end{array}}
    &&= \setn{\begin{array}{c}
           \mbox{ }\\
           \mbox{ }\\
           \mbox{ }\\
           \mbox{ }\\
           \circFive\\
           \mbox{ }
         \end{array}}
\end{align*}
\end{proof}

%---------------------------------------
\begin{example}
\label{ex:ocsop_x1x2}
%---------------------------------------
Let $\fphi_{2}$ be the \fncte{metric preserving function} defined in \prefpp{ex:mpf_x1x2}.
Let $\fg$ be the function defined in \prefpp{ex:ocsop_discrete}.
Then under under $\fphi_{2}\circ\fg$, 
the \structe{spinner outcome subspace} \xref{ex:spinner} becomes 
the \structe{weighted die outcome subspace} \xref{ex:wdie},
as illustrated on the right in \prefpp{fig:ocsop_x1x2_0121}.
Let $\ocsG$ be the \structe{spinner outcome subspace} and $\ocsH$ the \structe{weighted die outcome subspace}.
These structures have the following geometric statistics:
\\\indentx
 $\ocscen(\ocsG)=\setn{\circFour,\circSix} \qquad \ocscen(\ocsH)=\setn{\circOne,\circThree,\circFour,\circFive,\circSix}$ .
\\
Note that in \prefpp{ex:ocsop_0121}, the metric transform $\fphi_{1}$ results in a smaller 
(smaller \fncte{cardinality} \xrefnp{def:seto}) \structe{center} 
($\seto{\ocscen(\ocsG)}=2>1=\seto{\ocscen(\ocsH)}$).
But here, the metric transform $\fphi_{2}\circ\fg$ results in a larger \structe{center}.
($\seto{\ocscen(\ocsG)}=2<5=\seto{\ocscen(\ocsH)}$).
\end{example}
\begin{proof}
\begin{align*}
  \ocscen(\ocsG)
    &= \setn{\circFour,\circSix}
    && \text{by \prefpp{ex:ocsop_0121}}
    \\
  \ocscen(\ocsH)
    &\eqd \argmin_{x\in\ocsG}\max_{y\in\ocsG}\metric{x}{y}\psp(y)
    && \text{by definition of $\ocscen$ \xref{def:ocscen}}
  \\&= \mathrlap{\argmin_{x\in\ocsH}\max_{y\in\ocsH}\frac{1}{10}
         \setn{\begin{array}{cccccc}
           {0}\times1&{1}\times2 & {1}\times1 &{1}\times1 &{1}\times4 &{2}\times1\\
           {1}\times1&{0}\times2 & {1}\times1 &{1}\times1 &{2}\times4 &{1}\times1\\
           {1}\times1&{1}\times2 & {0}\times1 &{2}\times1 &{1}\times4 &{1}\times1\\
           {1}\times1&{1}\times2 & {2}\times1 &{0}\times1 &{1}\times4 &{1}\times1\\
           {1}\times1&{2}\times2 & {1}\times1 &{1}\times1 &{0}\times4 &{1}\times1\\
           {2}\times1&{1}\times2 & {1}\times1 &{1}\times1 &{1}\times4 &{0}\times1
         \end{array}}
      = \argmin_{x\in\ocsH}\frac{1}{10}
         \setn{\begin{array}{c}
            4\\
            8\\
            4\\
            4\\
            4\\
            4
         \end{array}}
      = \setn{\begin{array}{c}
           \circOne\\
           \mbox{ }\\
           \circThree\\
           \circFour\\
           \circFive\\
           \circSix\\
         \end{array}}}
\end{align*}
\end{proof}





\begin{figure}[h]
  \gsize%
  \centering%
  {\includegraphics{../common/math/graphics/pdfs/rline_argminmax_11312a.pdf}}%
  %{%============================================================================
% Daniel J. Greenhoe
% XeLaTeX file
%============================================================================
%\psset{unit=8mm}
\begin{pspicture}(-4.5,-0.5)(10.5,2.5)%
  \psset{%
    labelsep=1pt,
    linewidth=1pt,
    }%
  \psaxes[linecolor=axis,yAxis=false]{<->}(0,0)(-4,0)(10,2.5)% x axis
  \psaxes[linecolor=axis,xAxis=false]{ ->}(0,0)(-4,0)(2,2.5)% y axis
  %\psline[linewidth=3pt,linecolor=yellow](-3,1.875)(-2,1.5)(2.667,0.333)(3,0.375)(9,2.625)%
  %\psline(-3,0.375)(0,0)(9,1.125)% 1/8 |x|
  %\psline(-3,0.5)(1,0)(9,1)%       1/8 |x-1|
  %\psline(-3,1.875)(2,0)(9,2.625)% 3/8 |x-2|
  %\psline(-3,0.75)(3,0)(9,0.75)%   1/8 |x-3|
  %\psline(-3,1.75)(4,0)(9,1.25)%   2/8 |x-4|
  %
  \psplot[plotpoints=64,linewidth=3pt,linecolor=yellow]{ 1}{2.333}{x 4 sub abs 2.060 exp 2 mul 8 div}
  \psplot[plotpoints=64,linewidth=3pt,linecolor=yellow]{2.33}{4}{x abs 2.060 exp 8 div}
  %
  \psplot[plotpoints=64]{-3}{4}{x abs 2.060 exp 8 div}
  \psplot[plotpoints=64]{-3}{5}{x 1 sub abs 2.060 exp 8 div}
  \psplot[plotpoints=64]{-0.5}{4}{x 2 sub abs 2.060 exp 3 mul 8 div}
  \psplot[plotpoints=64]{-1}{6}{x 3 sub abs 2.060 exp 8 div}
  \psplot[plotpoints=64]{ 1}{6}{x 4 sub abs 2.060 exp 2 mul 8 div}
  \psline[linestyle=dotted,linecolor=red](2.333,0.716)(2.333,0)%
  \psline[linestyle=dotted,linecolor=red](2.333,0.716)(0,0.716)%
  %\psline[linestyle=dotted,linecolor=red](-2,1.5)(-2,0)%
  %\psline[linestyle=dotted,linecolor=red](-2,1.5)(0,1.5)%
  \psline[linestyle=dotted,linecolor=red](2.667,0.333)(2.667,0)%
  \psline[linestyle=dotted,linecolor=red](2.667,0.333)(0,0.333)%
  %\psline[linestyle=dotted,linecolor=red](3,0.375)(3,0)%
  %\psline[linestyle=dotted,linecolor=red](-2,1.5)(-2,0)%
  \uput[0]{0}(10,0){$x$}%
  \uput[-90]{0}(2.333,0){$\frac{7}{3}$}%
  \uput[-90]{0}(2.667,0){$\frac{8}{3}$}%
  \uput[180]{0}(0,0.333){$\frac{1}{3}$}%
  \uput[180]{0}(0,0.716){$\approx0.716$}%
  \uput[0]{0}(0,1.5){$1.5$}%
\end{pspicture}
}%
  \caption{real line addition $\argmin_x\max_y$ calculation graph \xref{ex:rline_11312a}\label{fig:rline_11312a}}
\end{figure}
%---------------------------------------
\begin{example}[\exmd{linear addition with metric transform}]
\label{ex:rline_11312a}
%---------------------------------------
\prefpp{ex:rline_11312} gave the result $\ocsE(\rvX+\rvY) = \frac{8}{3}$ rather than the perhaps more desirable result of 
$\frac{7}{3}$ (which equals $1 + \frac{4}{3} = \ocsE(\rvX) + \ocsE(\rvY)$).
Again, we adjust the geometric statistics of an outcome subspace by use of a \structe{metric preserving function} \xref{def:mpf}.
In particular, we use the \exme{power transform}/\exme{snowflake transform} \xref{ex:mpf_snowflake}
$\ff(x)= x^a$. If we let $a=\frac{\ln2}{\ln7-\ln5}\approx2.0600427$, then
$\ocsE(\rvX+\rvY) = \frac{7}{3}$, as illustrated in \prefpp{fig:rline_11312a}.
\end{example}



\end{tabstr}







