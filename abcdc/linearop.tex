%============================================================================
% NCTU - Hsinchu, Taiwan
% LaTeX File
% Daniel Greenhoe
%============================================================================

%======================================
\newapp{Linear Operators}
\label{app:linearop}
%======================================


%=======================================
\section{Operators on arbitrary spaces}
%=======================================
%---------------------------------------
\begin{definition}
%---------------------------------------
An {\bf operator} $\opA:\spX\to\spY$ is a mapping from
the space $\spX$ to the space $\spY$.
\end{definition}

%---------------------------------------
\begin{definition}
\citep{michel}{411}
%---------------------------------------
The {\bf identity operator} $\opI:\spX\to\spX$ is an operator which 
for all $x\in\opX$ satisfies
  \defbox{\opI x = x}
\end{definition}

%---------------------------------------
\begin{definition}
\citep{ab}{224}
\citep{michel}{407}
%---------------------------------------
Let $\spX$ and $\spY$ be vector spaces over a ground field $F$.\\
A {\bf linear operator} $\opL:\spX\to\spY$ is an operator which 
for all $x,y\in\spX$ and $\alpha\in F$ satisfies
  \defbox{\begin{array}{crcll}
    1. & \opL(x + y)    &=&  \opL x + \opL y \\
    2. & \opL(\alpha x) &=&  \alpha\opL x 
  \end{array}}
\end{definition}

%=======================================
\section{Operators on Banach spaces}
%=======================================
%---------------------------------------
\begin{definition}
\citep{ab}{225}
%---------------------------------------
Let $\opA:\spX\to\spY$ be an operator between normed spaces  
$\spX$ and $\spY$. \\
Then the {\bf norm} $\norm{\opA}$ of $\opA$ is defined as
\defbox{
  \norm{\opA} \eqd \sup_{x\in\spX}\set{\norm{\opA x}}{\norm{x}=1}
  }
A {\bf bounded} operator $\opA$ is one that satisfies the condition 
$\norm{\opA}<\infty$. \\
An {\bf unbounded} operator is one that is {\bf not} bounded.
\end{definition}


%---------------------------------------
\begin{theorem}
\citep{ab}{225}
%---------------------------------------
Let $\opA:\spX\to\spY$ be a linear operator between normed spaces
$\spX$ and $\spY$. \\
Then for all $x\in\spX$
\formbox{ \norm{\opA x} \le \norm{\opA}\:\norm{x} }
\end{theorem}
\begin{proof}
\[\begin{array}{rcll}
  \norm{\opA x}
    &=&    \ds \frac{\norm{x}}{\norm{x}} \norm{\opA x}
  \\&=&    \ds \norm{x} \norm{\frac{1}{\norm{x}}\opA x}
      &    \ds \mbox{by property of norms}
  \\&=&    \ds \norm{x} \norm{\opA \frac{x}{\norm{x}}}
      &    \ds \mbox{by property of linear operators}
  \\&\eqd& \ds \norm{x} \norm{\opA y}
      &    \ds \mbox{where $y\eqd\frac{x}{\norm{x}}$}
  \\&\le&  \ds \norm{x} \sup_y \norm{\opA y}
      &    \ds \mbox{by definition of supremum}
  \\&=&    \ds \norm{x} \sup_y\set{ \norm{\opA y}}{\norm{y}=1}
      &    \ds \mbox{because $\norm{y}=\norm{\frac{x}{\norm{x}}}=\frac{\norm{x}}{\norm{x}}=1$}
  \\&\eqd& \ds \norm{x} \norm{\opA}
      &    \ds \mbox{by definition of operator norm}
\end{array}\]
\end{proof}
%=======================================
\section{Operators on Hilbert spaces}
%=======================================
%---------------------------------------
\begin{definition}
\citep{keener}{14}
%\footnote{adjoint notation: \url{http://mathworld.wolfram.com/Adjoint.html}}
\index{adjoint}
%---------------------------------------
Let $\spH$ be a Hilbert space and
$\opA:\spH\to\spH$ be an operator over $\spH$.\\ 
The {\bf adjoint} $\opAa$ of $\opA$ is an operator that 
for all $x,y\in\spH$ satisfies
\defbox{
  \inprod{\opA x}{y} = \inprod{x}{\opAa y}
  }
An operator $\opA$ is {\bf self adjoint} if $\opA=\opAa$.
\end{definition}

%---------------------------------------
\begin{definition}
\citep{lax}{370}
%---------------------------------------
A {\bf normal} operator $\opN:\spH\to\spH$ with 
adjoint $\opNa:\spH\to\spH$ on a Hilbert space $\spH$ 
is one which satisfies
  \defbox{ \opNa\opN = \opN\opNa  }
\end{definition}

%---------------------------------------
\begin{definition}
\citep{michel}{431}
%---------------------------------------
Let $\opI:\spH\to\spH$ be the identity operator on a Hilbert space $\spH$.
An {\bf isometric operator} $\opA:\spH\to\spH$ with 
adjoint $\opAa:\spH\to\spH$ is one which satisfies
  \defbox{ \opAa\opA = \opI  }
\end{definition}

%---------------------------------------
\begin{definition}
\citep{michel}{431}
%---------------------------------------
Let $\opI:\spH\to\spH$ be the identity operator on a Hilbert space $\spH$.
An {\bf unitary operator} $\opU:\spH\to\spH$ with 
adjoint $\opUa:\spH\to\spH$ is one which satisfies
  \defbox{ \opUa\opU = \opU\opUa = \opI  }
\end{definition}


%---------------------------------------
\begin{theorem}
\citep{michel}{432}
%---------------------------------------
Let $\opA:\spH\to\spH$ be an operator on a Hilbert space $\spH$.
Then the following statments are equivalent:
\formbox{\begin{array}{ll}
  1. & \opA \mbox{ is isometric}                \\
  2. & \inprod{\opA x}{\opA y}=\inprod{x}{y}    \\
  3. & \norm{\opA x-\opA y}=\norm{x-y}
\end{array}}
\end{theorem}
\begin{proof}
\begin{enumerate}
\item Proof for $\opA$ is isometric $\implies$ $\inprod{\opA x}{\opA y}=\inprod{x}{y}$:
\[\begin{array}{rcll}
  \inprod{\opA x}{\opA y}
    &=& \inprod{x}{\opAa \opA y} & \mbox{by definition of {\em adjoint} $\opAa$}
  \\&=& \inprod{x}{\opI y}       & \mbox{by isometry hypothesis ($\opAa\opA=\opI$)}
  \\&=& \inprod{x}{y}            & \mbox{by definition of {\em identity} $\opI$}
\end{array}\]

\item Proof for 
      $\inprod{\opA x}{\opA y}=\inprod{x}{y} \implies \norm{\opA x-\opA y}=\norm{x-y}$:
\[\begin{array}{rcll}
  \norm{\opA x - \opA y}^2
    &=& \norm{\opA(x-y)}^2
  \\&=& \inprod{\opA(x-y)}{\opA(x-y)}  
  \\&=& \inprod{x-y}{x-y}              & \mbox{by hypothesis}
  \\&=& \norm{x-y}^2
\end{array}\]

\item Proof for $\norm{\opA x-\opA y}=\norm{x-y}$ $\implies$ $\opA$ is isometric: \attention
\[\begin{array}{rcll}
  \inprod{\opAa\opA x}{x}
    &=&    \inprod{\opA x}{\opA^{\ast\ast} x}
  \\&\eqq& \inprod{\opA x}{\opA x}
  \\&=&    \norm{\opA x}^2
  \\&=&    \norm{x}^2   & \mbox{by hypothesis with $y=0$}
  \\&=&    \inprod{x}{x}
\end{array}\]


\end{enumerate}
\end{proof}





%---------------------------------------
\begin{theorem}
\citep{michel}{432}
%---------------------------------------
Let $\opA:\spH\to\spH$ be an isometric operator. Then
\formbox{\begin{array}{ll}
  1. & \norm{\opA x} = \norm{x} \mbox{ for all } x\in\spH  \\
  2. & \norm{\opA} = 1
\end{array}}
\end{theorem}
\begin{proof}
\begin{enumerate}
\item Proof for $\norm{\opA x} = \norm{x}$:
\[\begin{array}{rcll}
  \norm{\opA x}
    &=& \norm{\opA x - \opA 0}
  \\&=& \norm{x - 0}
  \\&=& \norm{x}
\end{array}\]

\item Proof for $\norm{\opA} = 1$:
\[\begin{array}{rcll}
  \norm{\opA}
    &\eqd& \ds \sup_{\norm{x}=1}\set{\norm{\opA x}}{x\in\spH}
  \\&=&    \ds \sup_{\norm{x}=1}\set{\norm{x}}{x\in\spH}
  \\&=&    \ds \sup_{\norm{x}=1}\set{1}{x\in\spH}
  \\&=&    \ds 1
\end{array}\]
\end{enumerate}
\end{proof}
%---------------------------------------
\begin{theorem}
\citep{lax}{372}
%---------------------------------------
Let $\opU:\spH\to\spH$ be a unitary operator. Then
\formbox{\begin{array}{ll}
  1. & \norm{\opU x}=\norm{x}  \\
  2. & \opUa = \opUi:
\end{array}}
\end{theorem}
\begin{proof}
\begin{enumerate}
  \item Proof for $\norm{\opU x}=\norm{x}$: \\
        A unitary operator is always isometric, so $\norm{\opU x}=\norm{x}$.

  \item Proof for $\opUa = \opUi$:
  \[\begin{array}{rcll}
    \inprod{x}{\opUa\opU y}
      &=& \inprod{\opU x}{\opU y}  &\mbox{by definition of adjoint} 
    \\&=& \inprod{x}{y}            &\mbox{by (1)}
    \\&=& \inprod{x}{\opI y}       &\mbox{by definition of identity operator $\opI$}
  \end{array}\]
\end{enumerate}
\end{proof}

%---------------------------------------
\begin{definition}
%---------------------------------------
An {\bf integral opeator} $\opA$ is an operator of the form
\[ [\opA \ff](t) \eqd \int_u \kappa(t,u) \ff(u) \du \]
The function $\kappa(t,u)$ is called the {\bf kernel} of operator $\opA$.
\end{definition}

Examples of integral operators include
\[
\begin{array}{llrclrcl}
  1. & \mbox{Fourier Transform}
     & [\opFT \fx](f) &=& \int_t \fx(t) e^{-i2\pi ft}\dt 
     & \kappa(t,f)    &=& e^{-i2\pi ft}
\\
  2. & \mbox{Inverse Fourier Transform}
     & [\opFi \Fx](t) &=& \int_f \Fx(f) e^{i2\pi ft}\df 
     & \kappa(f,t)    &=& e^{i2\pi ft}
\\
  3. & \mbox{Laplace operator}
     & [\opL \fx](s) &=& \int_t \fx(t) e^{-st}\dt 
     & \kappa(t,s)   &=& e^{-st}
\end{array}
\]

%---------------------------------------
\begin{example}
%---------------------------------------
In matrix algebra, 
\begin{liste}
  \item the innerproduct operation $\inprod{\vx}{\vy}$ is represented by 
         $\vy^H \vx$ 
  \item the linear operator is represented as a matrix $\vA$ 
  \item the operation of $\vA$ on vector $\vx$ is represented as $\vA\vx$
\end{liste}

The adjoint of matrix $\vA$ is the Hermetian matrix $\vA^H$:
\begin{eqnarray*}
  \inprod{\vA\vx}{\vy}
    &\eqd& \vy^H \vA\vx
     =     [(\vA\vx)^H \vy ]^H
     =     [\vx^H\vA^H \vy ]^H
     =      (\vA^H \vy)^H \vx
     \eqd  \inprod{\vx}{\vA^H \vy}
\end{eqnarray*}
\end{example}

%---------------------------------------
\begin{example}
%---------------------------------------
Let a correlation operator $\opR$ be defined as
  \begin{eqnarray*}
    [\opR \ff](t) &\eqd& \int_u \Rxx(t,u) \ff(u) \du  \\
    \Rxx(t,u) &\eqd& \pE x(t)x^\ast(u) 
  \end{eqnarray*}
\begin{enumerate}
  \item The autocorrelation function $\Rxx(t,u)$ is {\bf hermitian symmetric}:
    \begin{eqnarray*}
      \Rxx(t,u)
        &\eqd& \pE  \fx(t) \fx^\ast(u)
         =     [\pE \fx^\ast(t) \fx(u)]^\ast
         =     [\pE \fx(u) \fx^\ast(t)]^\ast
         =     \Rxx^\ast(u,t)
    \end{eqnarray*}

  \item The operator $\opR$ is {\bf self-adjoint}:
    \begin{eqnarray*}
      \inprod{\opR \ff}{\fg}
        &=& \inprod{\int_u \Rxx(t,u) \ff(u) \du}{\fg(t)}
      \\&=& \int_u \ff(u) \inprod{\Rxx(t,u) }{\fg(t)} \du
      \\&=& \int_u \ff(u) \int_t \Rxx(t,u) \fg^\ast(t) \dt \du
      \\&=& \int_u \ff(u) \int_t \Rxx^\ast(u,t) \fg^\ast(t) \dt \du
      \\&=& \int_u \ff(u) \left[\int_t \Rxx(u,t) \fg(t) \dt\right]^\ast \du
      \\&=& \int_u \ff(u) \left[\opR \fg \right]^\ast \du
      \\&=& \inprod{ \ff}{\opR \fg}
    \end{eqnarray*}
\end{enumerate}
\end{example}

%---------------------------------------
\begin{example}
%---------------------------------------
Let the Fourier Transform operator $\opFT$ and 
inverse Fourier Transform operator $\opFi$ be defined as
  \begin{eqnarray*}
    [\opFT \fx](f) &\eqd& \int_t \fx(t) e^{-i2\pi ft} \dt \\
    \left[\opFi \Fx\right](t) &\eqd& \int_f \Fx(f) e^{i2\pi ft} \df.
  \end{eqnarray*}
Then $\opFT$ is unitary ($\opFa = \opFi$)
  \begin{eqnarray*}
    \inprod{\opFT \fx}{\Fy}
      &=& \inprod{\int_t \fx(t) e^{-i2\pi ft} \dt }{\Fy(f)}
    \\&=& \int_t \fx(t) \inprod{e^{-i2\pi ft} }{\Fy(f)} \dt
    \\&=& \int_t \fx(t) \int_f e^{-i2\pi ft} \Fy^\ast(f) \df \dt
    \\&=& \int_t \fx(t) \left[\int_f e^{i2\pi ft} \Fy(f) \df \right]^\ast \dt
    \\&=& \inprod{\fx(t)}{\int_f \Fy(f) e^{i2\pi ft} \df }
    \\&=& \inprod{\fx}{\opFi \Fy}
  \end{eqnarray*}
\end{example}

%=======================================
\section{Spectrum of an operator}
%=======================================
%---------------------------------------
\begin{definition}
\index{eigenvalue} \index{eigenvector}
%---------------------------------------
Let $\opA:\spH\to\spH$ be an operator over a Hilbert space $\spH$.
If for some $x\in\spH$ and $\lambda\in\C$
\[ \opA x = \lambda x, \]
then $\lambda$ is called an {\bf eigenvalue} of $\opA$,
$x$ is called an {\bf eigenvector} of $\opA$, and
$(\lambda,x)$ is called an {\bf eigenpair} of $\opA$.
\end{definition}

%---------------------------------------
\begin{theorem}
\index{self-adjoint}
\citepp{lax}{315}{316}
\citepp{keener}{114}{119}
%---------------------------------------
Let $\opA:\spH\to\spH$ be an operator over a Hilbert space $\spH$.\\
If $\opA$ is self adjoint, then
\begin{enume}
   \item The {\bf hermitian quadratic} form of $\opA$ is {\bf real} for all $x\in\spH$:
         \formbox{\forall x\in\spH, \hspace{4ex} \inprod{\opA x}{x}\in\R}
   \item {\bf eigenvalues} of $\opA$ are {\bf real}:
         \formbox{ \opA\psi_n=\lambda_n\psi_n \implies
            \lambda_n \in \R }
   \item {\bf eigenfunctions} associated with distinct eigenvalues are {\bf orthogonal}
         \formbox{ \lambda_n\not=\lambda_m \implies \inprod{\psi_n}{\psi_m}=0 }
\end{enume}

\end{theorem}

\begin{proof}\\
\begin{enumerate}
\item Prove the {\bf hermitian quadratic} form of $\opA$ is {\bf real}:
  \begin{marray}
    \inprod{x}{\opA x}
      &=& \inprod{\opA x}{x} &\mbox{by self-adjoint property of $\opA$} \\
      &=& \inprod{x}{\opA x}^\ast & \mbox{by skew symmetry property of $\inprod{\cdot}{\cdot}$}
  \end{marray}

\item Prove eigenvalues are real:
\begin{marray}
   \lambda_n\norm{\psi_n}^2
     &=& \lambda_n \inprod{\psi_n}{\psi_n}
      =  \inprod{\lambda_n \psi_n}{\psi_n}
      =  \inprod{\opA \psi_n}{\psi_n}
   \\&=& \inprod{\psi_n}{\opA \psi_n} 
         &\mbox{because $\opA$ is self-adjoint}
   \\&=& \inprod{\psi_n}{\lambda_n \psi_n}
      =  \lambda_n^\ast \inprod{\psi_n}{\psi_n}
      =  \lambda_n^\ast \norm{\psi_n}^2
\end{marray}

\item Prove eigenfunctions are orthogonal:
\begin{marray}
   \lambda_n\inprod{\psi_n}{\psi_m}
      &=&\inprod{\lambda_n\psi_n}{\psi_m}
      =  \inprod{\opA\psi_n}{\psi_m}
    \\&=&\inprod{\psi_n}{\opA\psi_m}
         &\mbox{because $\opA$ is self-adjoint}
    \\&=&\inprod{\psi_n}{\lambda_m\psi_m}
      =  \lambda_m^\ast\inprod{\psi_n}{\psi_m}
    \\&=&\lambda_m\inprod{\psi_n}{\psi_m}
         &\mbox{because $\lambda_m$ is real}
\end{marray}
This implies for $\lambda_n\not=\lambda_m\not=0$,
  $\inprod{\psi_n}{\psi_m}=0$.

\end{enumerate}
\end{proof}


%---------------------------------------
\begin{theorem}
\index{self-adjoint}
\index{spectral theorem}
\citepp{lax}{316}{318}
%---------------------------------------
Let $\opA:\spH\to\spH$ be an operator and $\spH$ a Hilbert space. \\
If $\opA$ is {\bf self-adjoint}, then the eigenvectors of $\opA$ form 
a basis for $\spH$.
\end{theorem}
\begin{proof}
No proof at this time.   \attention
\end{proof}


Note: Classifications of operator spectrum\citep{keener}{286}
(depends on $(\opA - \lambda\opI)^{-1}$): \attention
\begin{enume}
  \item resolvent set
  \item discrete spectrum
  \item continuous spectrum
  \item residual spectrum
\end{enume}









