%============================================================================
% Daniel J. Greenhoe
% LaTeX file
%============================================================================


%=======================================
\subsection{Lattices}
%=======================================
%=======================================
%\subsubsection{Definition}
%=======================================
The structure available in an \structe{ordered set} \xref{def:orel} 
%is a set\ifsxref{found}{def:set} together with the additional 
%structure of an ordering relation\ifsxref{order}{def:poset}.
%However, this amount of structure 
tends to be insufficient to ensure ``well-behaved" mathematical systems.
This situation is greatly remedied if every pair of elements in the ordered set %(partially or linearly ordered)
has both a \prope{least upper bound} and a \prope{greatest lower bound} \xref{def:glb} in the set;
in this case, that ordered set is a \structe{lattice} (next definition).\footnote{
Gian-Carlo Rota (1932--1999) has illustrated the advantage of lattices over simple ordered sets
by pointing out that the \structe{ordered set of partitions of an integer} 
``is fraught with pathological properties", while 
the \emph{lattice} of partitions of a set 
``remains to this day rich in pleasant surprises".
{
  \citePpc{rota1997}{1440}{(illustration)}, %{http://www.ams.org/notices/199711/comm-rota.pdf},
  \citePpc{rota1964}{498}{partitions of a set} %{http://www.jstor.org/pss/2312585}
  }}
%Further examples of lattices follow in \prefpp{sec:lat_examples}.
%---------------------------------------
\begin{definition}
\footnote{
  \citerpg{maclane1999}{473}{0821816462},
  \citerpg{birkhoff1948}{16}{3540120440},
  \citeP{ore1935}, %{???}, % cf rudeanu2001
  \citePp{birkhoff1933}{442},
  \citerp{maeda1970}{1}
  }
\label{def:lattice}
\label{def:latlin}
%---------------------------------------
\defboxt{%
  An algebraic structure $\latL\eqd\latticed$ is a \structd{lattice} if
  \\\indentx$\begin{array}{FlDD}
      1. & \opair{\setX}{\orel} \text{ is an \structe{ordered set}}   & \xref{def:poset} & and \\
      2. & x,y\in\setX \quad\implies\quad x\join y \in\setX           & \xref{def:lub}   & and \\
      3. & x,y\in\setX \quad\implies\quad x\meet y \in\setX           & \xref{def:glb}.  &
  \end{array}$
  \\
  %The algebraic structure $\latL^\ast\eqd\lattice{\setX}{\oreld}{\joina}{\meeta}$ 
  %is the \hid{dual} lattice of $\latL$, where $\joina$ and $\meeta$ are determined by $\oreld$.
  The \structe{lattice} $\latL$ is \propd{linear} if $\opair{\setX}{\orel}$ is a \structe{linearly ordered set} \xref{def:chain}.
  }
\end{definition}


%---------------------------------------
\begin{theorem}
\citetblp{
  \citerppgc{maclane1999}{473}{475}{0821816462}{{\scshape Lemma 1, Theorem 4}},
  \citerppg{burris1981}{4}{7}{0387905782},
  \citerpp{birkhoff1938}{795}{796},
  \citePpc{ore1935}{409}{($\alpha$)},
  \citePp{birkhoff1933}{442},
  \citePppc{dedekind1900}{371}{372}{(1)--(4)}
  %\citorpg{birkhoff1948}{18}{3540120440}
  %\cithrpp{burris2000}{5}{6}
  %\citerp{maclane1967}{485}  \\
  %\citerp{johnstone1982}{1}
  }
\label{thm:lattice}
%---------------------------------------
\thmboxt{%
  $\latticed$ is a \structe{lattice} \xref{def:lattice} \qquad$\iff$ %\textbf{if and only if}% $\forall x,y,z\in\setX$
  \\\indentx
  $\brb{\begin{array}{rcl | rcl | C  DD}
        x \join x &=& x
      & x \meet x &=& x
      & \forall x\in\setX
      & (\prope{idempotent})
      & and
      \\
        x \join y &=& y \join x
      & x \meet y &=& y \meet x
      & \forall x,y\in\setX
      & (\prope{commutative})
      & and
      \\
        (x \join y) \join z &=& x \join (y \join z)
      & (x \meet y) \meet z &=& x \meet (y \meet z)
      & \forall x,y,z\in\setX
      & (\prope{associative})
      & and
      \\
        x \join (x \meet y) &=& x
      & x \meet (x \join y) &=& x
      & \forall x,y\in\setX
      & (\prope{absorptive}).
      & 
      %& (\prope{absorptive}/\prope{contractive})
      %\\
      %  \mc{3}{l}{x \orel y \iff x \join y = y}
      %& \mc{3}{l}{x \orel y \iff x \meet y = x}
      %& \forall x,y\in\setX
      %& (\prope{consistent})
  \end{array}}$
  }
\end{theorem}
%%---------------------------------------
%\begin{lemma}
%\footnote{
%  \citer{holland1970} %{???}
%  }
%\label{lem:le_meet}
%%---------------------------------------
%Let $\latL\eqd\latticed$ be \structe{lattice} \xref{def:lattice}.
%\lembox{
%  x\le y \qquad\iff\qquad x = x\meet y \qquad\scy\forall x,y\in\latL
%  }
%\end{lemma}
%\begin{proofns}
%\begin{enumerate}
%  \item Proof for $\implies$ case: by left hypothesis and definition of $\meet$ \xref{def:meet}.
%    %\begin{align*}
%    %  x\meet y
%    %    &= x
%    %    && \text{by left hypothesis and definition of $\meet$ \xref{def:meet}}
%    %\end{align*}
%
%  \item Proof for $\impliedby$ case: by right hypothesis and definition of $\meet$ \xref{def:meet}.
%\end{enumerate}
%\end{proofns}
%
%
%
%
%
%
%%\paragraph{Monotony laws.}
%%---------------------------------------
%\begin{proposition}[\thmd{Monotony laws}]
%\footnote{
%  \citerpg{givant2009}{39}{0387402934},
%  \citePpp{doner1969}{97}{99},
%  \citergc{greenhoe2014sro}{0983801118}{\textsection4.2}
%  }
%\label{prop:latmono}
%%---------------------------------------
%Let $\latticed$ be a lattice.
%\propbox{
%  \brb{\begin{array}{rclD}
%    a &\orel& b & and \\
%    x &\orel& y
%  \end{array}}
%  \qquad\implies\qquad
%  \brb{\begin{array}{rclD}
%    a \meet x &\orel& b \meet y & and \\
%    a \join x &\orel& b \join y
%  \end{array}}
%  }
%\end{proposition}
\paragraph{Minimax inequality.}
Suppose we arrange a finite sequence of values into $m$ groups of $n$ elements per group.
This could be represented as an $m\cprod n$ matrix.
Suppose now we find the minimum value in each row,
and the maximum value in each column.
We can call the maximum of all the minimum row values the \hie{maximin},
and the minimum of all the maximum column values the \hie{minimax}.
Now, which is greater, the maximin or the minimax?
The \hie{minimax inequality} demonstrates that
the maximin is always less than or equal to the minimax.
The minimax inequality is illustrated below and stated formerly in
\prefpp{thm:minimax_ineq}.
  \[\mcom{
    \joinop_1^m\left\{\begin{array}{>{\ds}l llll >{\ds}l}
      \meetop_1^n \Big\{ & x_{11} & x_{12} & \cdots & x_{1n} & \Big\}\\\hline
      \meetop_1^n \Big\{ & x_{21} & x_{22} & \cdots & x_{2n} & \Big\}\\\hline
      \meetop_1^n \Big\{ & \vdots & \ddots & \ddots & \vdots & \Big\}\\\hline
      \meetop_1^n \Big\{ & x_{m1} & x_{m2} & \cdots & x_{mn} & \Big\}
    \end{array}\right\}
    }{maximin}
    \qquad\orel\qquad
    \mcom{
    \meetop_1^n\left\{\renewcommand{\arraystretch}{1.8}\begin{array}{>{\ds}c | >{\ds}c | >{\ds}c | >{\ds}c}
      \joinop_1^m & \joinop_1^m & \joinop_1^m & \joinop_1^m \\
      x_{11} & x_{12} & \cdots & x_{1n} \\
      x_{21} & x_{22} & \cdots & x_{2n} \\
      \vdots & \ddots & \ddots & \vdots \\
      x_{m1} & x_{m2} & \cdots & x_{mn}
    \end{array}\right\}
    }{minimax}
  \]
%---------------------------------------
\begin{theorem}[\thmd{minimax inequality}]
\footnote{
  \citerppg{birkhoff1948}{19}{20}{3540120440}
  }
\label{thm:minimax}
\label{thm:minimax_ineq}
\index{minimax inequality}
\index{theorems!minimax inequality}
\index{inequalities!minimax}
%---------------------------------------
Let $\latticed$ be a \structe{lattice} \xref{def:lattice}.
%with $X\eqd\seq{x_{ij}}{i=1,2,\ldots,n;\;j=1,2,\ldots,m}$.
\thmbox{
  \mcomr{\joinop_{i=1}^m   \meetop_{j=1}^n x_{ij}}{maxmini: largest of the smallest}
  \orel
  \mcoml{\meetop_{j=1}^n \joinop_{i=1}^m   x_{ij}}{minimax: smallest of the largest}
  \qquad
  \forall x_{ij}\in\setX
  }
\end{theorem}
\begin{proof}
\[\begin{array}{l >{\ds}rc>{\ds}l @{\qquad}l}
  &     \mcomr{\left( \meetop_{k=1}^n x_{ik} \right)}{smallest for any given $i$}
  &\orel& x_{ij}
   \orel  \mcoml{\left( \joinop_{k=1}^n x_{kj} \right)}{largest for any given $j$}
  &     \forall i,j
  \\
  \implies
  &     \mcomr{\joinop_{i=1}^m  \left( \meetop_{k=1}^n x_{ik} \right)}{largest amoung all $i$s of the smallest values}
  &\orel& \mcoml{\meetop_{j=1}^n\left( \joinop_{k=1}^m   x_{kj} \right)}{smallest amoung all $j$s of the largest values}
  &
  \\
  \implies
  &     \mcom{\joinop_{i=1}^m  \left( \meetop_{j=1}^n x_{ij} \right)}{maxmini}
  &\orel& \mcom{\meetop_{j=1}^n\left( \joinop_{i=1}^m x_{ij} \right)  }{minimax}
  & \text{(change of variables)}
\end{array}\]
\end{proof}


%\paragraph{Distributive inequalities.}
Special cases of the minimax inequality include three distributive \emph{inequalities}
(next theorem).
If for some lattice any \emph{one} of these inequalities is an \emph{equality}, 
then \emph{all three} are \emph{equalities}; %\ifsxref{latd}{thm:lat_dis};
and in this case, the lattice is a called a 
\propd{distributive} lattice. %\ifsxref{latd}{def:lat_distributive}.
%---------------------------------------
\begin{theorem}[\thmd{distributive inequalities}]
\footnote{
  \citerpg{davey2002}{85}{0521784514},
  \citerpg{gratzer2003}{38}{3764369965},
  %\citerpg{maclane1999}{474}{0821816462}\\
  %\citerppg{burris1981}{10}{11}{0387905782}\\
  \citePp{birkhoff1933}{444},
  \citePp{korselt1894}{157},
  %\citerp{maclane1967}{486} \\
  %\cithepp{burris2000}{12}{13}
  \citerpgc{mullerolm1997}{13}{3540634061}{terminology}
  }
\label{thm:lat_dis_<}
\index{inequalities!distributive}
%---------------------------------------
\thmboxt{
  $\latticed$ is a \structe{lattice} \xref{def:lattice} $\implies$
  \\$\ds\brb{\begin{array}{rclCDD}
  x \meet (y \join z) &\ge& (x \meet y) \join (x \meet z)
    & \forall x,y,z \in\setX
    & (\prope{join super-distributive})
    & and
  \\
  x \join (y \meet z) &\orel& (x \join y) \meet (x \join z)
    & \forall x,y,z \in\setX
    & (\prope{meet sub-distributive})
    & and
  \\
  (x\meet y)\join(x\meet z)\join(y\meet z) &\orel& (x\join y)\meet(x\join z)\meet(y\join z)
    & \forall x,y,z \in\setX
    & (\prope{median inequality}).
  \end{array}}$
  }
\end{theorem}
\begin{proof}
\begin{enumerate}
  \item Proof that $\meet$ sub-distributes over $\join$:
    \begin{align*}
      (x \meet y) \join (x \meet z)
        &\orel (x\join x) \meet (y\join z)
        && \text{by \thme{minimax inequality} \xref{thm:minimax_ineq}}
      \\&= x \meet (y\join z)
        && \text{by \prope{idempotent} property of lattices \xref{thm:lattice}}
    \end{align*}

  \[
    \joinop\left\{\begin{array}{l ll l}
      \meetop \Big\{ & x & y & \Big\}\\\hline
      \meetop \Big\{ & x & z & \Big\}
    \end{array}\right\}
    \qquad\orel\qquad
    \meetop\left\{\begin{array}{c|c}
      \joinop & \joinop  \\
      x & y  \\
      x & z
    \end{array}\right\}
  \]

  \item Proof that $\join$ super-distributes over $\meet$:
    \begin{align*}
      x \join (y \meet z)
        &=   (x \meet x) \join (y \meet z)
        && \text{by \prope{idempotent} property of lattices \xref{thm:lattice}}
      \\&\orel (x \join y) \meet (x \join z)
        && \text{by \thme{minimax inequality} \xref{thm:minimax_ineq}}
    \end{align*}

  \[
    \joinop\left\{\begin{array}{l ll l}
      \meetop \Big\{ & x & x & \Big\}\\\hline
      \meetop \Big\{ & y & z & \Big\}
    \end{array}\right\}
    \qquad\orel\qquad
    \meetop\left\{\begin{array}{c|c}
      \joinop & \joinop  \\
      x & x  \\
      y & z
    \end{array}\right\}
  \]

  \item Proof that of \ineq{median} inequality: 
        by \thme{minimax inequality} \xref{thm:minimax_ineq}

\end{enumerate}

\end{proof}



%\paragraph{Modular inequalities.}
Besides the distributive property, another consequence of the minimax inequality
is the \ineqe{modularity inequality} (next theorem).
A lattice in which this inequality becomes equality is said to be 
\propd{modular}. %\ifsxref{latm}{def:lat_mod}.
%---------------------------------------
\begin{theorem}[\thmd{Modular inequality}]
\footnote{
  \citerpg{birkhoff1948}{19}{3540120440},
  \citerpg{burris1981}{11}{0387905782},
  \citePp{dedekind1900}{374}
  %\citerpg{maclane1999}{474}{0821816462}
  %\cithrp{burris2000}{13}
  }
\label{thm:lat_mod}
\index{inequalities!modular}
\index{modular inequality}
%---------------------------------------
Let $\latticed$ be a \structe{lattice} \xref{def:lattice}.
\thmbox{
  x\orel y
  \qquad\implies\qquad
  x\join(y\meet z) \orel y\meet(x\join z)
  }
\end{theorem}
\begin{proof}
\begin{align*}
  x\join(y\meet z)
    &=   (x\meet x) \join (y\meet z)
    &&   \text{by \prope{absorptive} property \xref{thm:lattice}}
  \\&\orel (x\join y) \meet (x \join z)
    &&   \text{by the \thme{minimax inequality} \xref{thm:minimax_ineq}}
  \\&=   y \meet (x \join z)
    &&   \text{by left hypothesis}
\end{align*}

  \[
    \joinop\left\{\begin{array}{l ll l}
      \meetop \Big\{ & x & x & \Big\}\\\hline
      \meetop \Big\{ & y & z & \Big\}
    \end{array}\right\}
    \qquad\orel\qquad
    \meetop\left\{\begin{array}{c|c}
      \joinop & \joinop  \\
      x & x  \\
      y & z
    \end{array}\right\}
  \]
\end{proof}




%\prefpp{thm:lattice} gives 4 necessary and sufficient pairs of properties for a structure 
%$\latticeX$ to be a \structe{lattice}. 
%However, these 4 pairs are actually \emph{overly} sufficient
%(they are not \prope{independent}), as demonstrated next.
%%---------------------------------------
%\begin{theorem}
%\footnote{
%  \citerppg{padmanabhan2008}{7}{8}{9812834540},
%  \citerpg{beran1985}{5}{902771715X},
%  \citePp{mckenzie1970}{24},
%  \citePc{greenhoe2014flat}{Theorem 1.22},
%  \citergc{greenhoe2014sro}{0983801118}{\textsection4.4}
%  }
%\label{thm:lat_char_6e3v}
%%---------------------------------------
%\thmbox{\begin{array}{M}
%  $\latticed$ is a lattice \qquad$\iff$ %\textbf{if and only if}% $\forall x,y,z\in\setX$
%  \\
%  $\brb{\begin{array}{rcl | rcl | C  DD}
%        x \join y &=& y \join x
%      & x \meet y &=& y \meet x
%      & \forall x,y\in\setX
%      & (\prope{commutative})
%      & and
%      \\
%        (x \join y) \join z &=& x \join (y \join z)
%      & (x \meet y) \meet z &=& x \meet (y \meet z)
%      & \forall x,y,z\in\setX
%      & (\prope{associative})
%      & and
%      \\
%        x \join (x \meet y) &=& x
%      & x \meet (x \join y) &=& x
%      & \forall x,y\in\setX
%      & (\prope{absorptive})
%      & 
%  \end{array}}$
%\end{array}}
%\end{theorem}
%
%
%%=======================================
%\subsubsection{Bounded lattices}
%%=======================================
%Let $\latL\eqd\latticed$ be a lattice.
%By the definition of a \structe{lattice} \xref{def:lattice},
%the \structe{upper bound} ($x \join y$) and \structe{lower bound} ($x\meet y$) of any two elements in $\setX$
%is also in $\setX$.
%But what about the upper and lower bounds
%of the entire set $\setX$ ($\joinop\setX$ and $\meetop\setX$) \xxref{def:join}{def:meet}?
%If both of these are in $\setX$, then the lattice $\latL$ is said to be
%\prope{bounded} (next definition).
%All \prope{finite} lattices are bounded (next proposition).
%However, not all lattices are bounded---%
%for example, the lattice $\opair{\Z}{\le}$ (the lattice of integers 
%with the standard integer ordering relation) is \prope{unbounded}.
%%\prefpp{thm:latb_prop} gives two properties of bounded lattices.
%%Boundedness is one of the ``\hie{classic 10}" properties\ifsxref{boolean}{thm:boo_prop}
%%of \hie{Boolean algebras}\ifsxref{boolean}{def:booalg}.
%%Conversely, a bounded and complemented lattice that satisfies the conditions
%%$1'=0$ and \thme{Elkan's law} \emph{is} a \structe{Boolean algebra}\ifsxref{boolean}{prop:boo_char_elkan}.
%%---------------------------------------
%\begin{definition}
%\label{def:latb}
%%---------------------------------------
%Let $\latL\eqd\latticed$ be a lattice.
%Let $\joinop\setX$ be the least upper bound of $\opair{\setX}{\orel}$ and
%let $\meetop\setX$ be the greatest lower bound of $\opair{\setX}{\orel}$.\\
%\defboxt{
%  %\begin{tabular}{>{$\imark$\quad}ll}
%  \indentx\begin{tabular}{ll}
%    $\latL$ is \hid{upper bounded} & if $\brp{\joinop\setX}\in\setX.$ \\
%    $\latL$ is \hid{lower bounded} & if $\brp{\meetop\setX}\in\setX.$ \\
%    $\latL$ is \hid{bounded}       & if $\latL$ is both upper and lower bounded.
%  \end{tabular}
%  \\
%  A \prope{bounded} lattice is optionally denoted $\latbd$,
%  where $\bzero\eqd\meetop\setX$ and $\bid\eqd\joinop\setX$.
%  }
%\end{definition}
%
%%---------------------------------------
%\begin{proposition}
%\label{prop:latb_finite}
%%---------------------------------------
%Let $\latL\eqd\latticed$ be a lattice.
%\propbox{
%  \brb{\text{$\latL$ is \prope{finite}}}
%  \qquad\implies\qquad
%  \brb{\text{$\latL$ is \prope{bounded}}}
%  }
%\end{proposition}
%
%
%%---------------------------------------
%\begin{proposition}
%\footnote{
%  \citePc{greenhoe2014flat}{\textsection1.2.2},
%  \citergc{greenhoe2014sro}{0983801118}{\textsection4.5}
%  }
%\label{prop:latb_prop}
%%---------------------------------------
%Let $\latL\eqd\latticed$ be a lattice with $\joinop\setX\eqd\bid$ and
%$\meetop\setX\eqd\bzero$.
%\propbox{%
%  \brb{\text{$\latL$ is \prope{bounded}}}
%  \qquad\implies\qquad
%  \brb{\begin{array}{lclCDD}
%    x \join \bid   &=& \bid   & \forall x\in\setX & (\prop{upper bounded}) & and \\
%    x \meet \bzero &=& \bzero & \forall x\in\setX & (\prop{lower bounded}) & and \\
%    x \join \bzero &=& x      & \forall x\in\setX & (\prop{join-identity}) & and \\
%    x \meet \bid   &=& x      & \forall x\in\setX & (\prop{meet-identity}) & 
%  \end{array}}%
%  }
%\end{proposition}
%
%
%%---------------------------------------
%\begin{definition}
%\footnote{
%  \citerpg{birkhoff1967}{5}{0821810251}
%  }
%\label{def:height}
%%---------------------------------------
%Let $\latL\eqd\latbX$ be a \structe{bounded lattice} \xref{def:latb}.
%\defboxp{
%  The \fnctd{height} $\height(x)$ of a point $x\in\latL$ is the 
%  \vale{least upper bound} of the \fncte{length}s \xref{def:length} of all the \structe{chain}s 
%  that have $\lzero$ and in which $x$ is the \vale{least upper bound}.
%  The \fnctd{height} $\height(\latL)$ of the lattice $\latL$ is defined as
%  \\\indentx$\height(\latL)\eqd\height(\lid)$ .
%  }
%\end{definition}
%
%%---------------------------------------
%\begin{example}
%%---------------------------------------
%The \vale{height} of the lattice illustrated in \prefpp{fig:m3m3_antichain} is 3 because
%\begin{align*}
%  \height(\latL)
%    &\eqd \height(\lid)
%  \\&\eqd \joinop\set{\latlen(\latC)}{\text{$\latC$ is a \structe{chain} in $\latL$ containing both $\lzero$ and $\lid$}}
%  \\&= \joinop\big\{
%         \latlen\opair{\setn{0,a,p,1}}{\orel},\,
%         \latlen\opair{\setn{0,b,p,1}}{\orel},\,
%         \latlen\opair{\setn{0,c,p,1}}{\orel},\,
%         \latlen\opair{\setn{0,c,q,1}}{\orel},\\&\qquad\qquad
%         \latlen\opair{\setn{0,c,r,1}}{\orel},
%         \big\}
%  \\&=\joinop\setn{4-1,4-1,4-1,4-1,4-1}
%  \\&=\joinop\setn{3,3,3,3,3}
%  \\&= 3
%\end{align*}
%\end{example}