%============================================================================
% Daniel J. Greenhoe
% XeLaTeX File
%============================================================================
%=======================================
\chapter{Negation}
%=======================================
%=======================================
\section{Simple negation}
%=======================================
\ftbox{\zhtsdH{否}{定}{f/>ou}{d/`ing}{to negate}{to set}{to negate}}
\quad\begin{minipage}{\tw-100mm}%
  Negation in English is performed using the word \hie{not} and
  sometimes using prefixes such as \hie{non-} and \hie{un-}.
  In Chinese, negation is performed using the characters
  \zht{不} (b/`u) or \zht{沒} (m/'ei).
  The character \zht{沒} is always either used with \zht{有} (y\'ou) giving \zht{沒有} (not have)
  or used alone as a shortened form of \zht{沒有}.
\end{minipage}\quad%
\ftbox{\zhtssH{不}{b/`u}{not}\zhtssH{沒}{m/'ei}{not (have)}}
\\[0.5ex]
When you want to negate a verb, the trick is to determine when
to use the ``have" form (and thus use \zht{沒}) and when not to use the
``have" form (and thus use \zht{不})---good luck.

%---------------------------------------
\begin{example}[\exmd{simple negation}]
\mbox{}\\
%---------------------------------------
\begin{enumerate}
  \item \zhtss{我}{w/>o}{I}%
        \zhtss{不}{b/`u}{not}%
        \zhtsd{喜}{歡}{x/>i}{h/-uan}{happy}{cheerful}{like}%
        \zhtss{吃}{ch/-i}{eat}%
        \zhtsd{香}{蕉}{xi/-ang}{ji/-ao}{fragrant}{banana}{banana}%
        \zhtsP
        \engbox{I do not like to eat bananas.}

  \item \zhtss{我}{w/>o}{I}%
        \zhtss{沒}{m/'ei}{not (have)}%
        \zhtsd{吃}{過}{ch/-i}{g/`uo}{eat}{to pass}{eaten}%
        \zhtsd{香}{蕉}{xi/-ang}{ji/-ao}{fragrant}{banana}{banana}%
        \zhtsP
        \engbox{I have never eaten bananas.}

  \item \zhtss{你}{n/>i}{you}%
        \zhtss{要}{y/`ao}{want}% 
        \zhtss{不}{b/'u}{not}% 
        \zhtss{要}{y/`ao}{want}%
        \zhtsd{睡}{覺}{sh/`ui}{ji/`ao}{to sleep}{sleep}{sleep}%
        \zhtsQ
        \engbox{Do you want to sleep or not?}

  \item \zhtss{你}{n/>i}{you}%
        \zhtss{睡}{sh/`ui}{to sleep}%
        \zhtss{飽}{b/>ao}{full}%
        \zhtss{了}{le}{}%
        \zhtss{沒}{m/'ei}{not (have)}%
        \zhtsQ
        \engbox{Have you had enough sleep?}

  \item \zhtss{你}{n/>i}{you}
        \zhtss{要}{y/`ao}{want} 
        \zhtss{不}{b/`u}{not} 
        \zhtss{要}{y/`ao}{want}
        \zhtss{吃}{ch/-i}{eat}
        \zhtsQ
        \engbox{Do you want to eat?}

  \item \zhtss{你}{n/>i}{you}
        \zhtss{吃}{ch/-i}{eat}
        \zhtss{飽}{b/>ao}{full}
        \zhtss{了}{le}{}
        \zhtss{沒}{m/'ei}{not (have)}
        \zhtsQ
        \engbox{Have you eaten?}

  \item \zhtss{你}{n/>i}{you}
        \zhtsd{聽}{到}{ti\=ng}{d/`ao}{hear}{to reach}{hear}
        \zhtss{了}{le}{}
        \zhtss{沒}{m/'ei}{not (have)}
        \zhtsX%
        \zhtsQ
        \engbox{Do you hear me or not!?}
 
  \item \zhtss{你}{n/>i}{you}
        \zhtss{要}{y/`ao}{want} 
        \zhtss{不}{b/'u}{not} 
        \zhtss{要}{y/`ao}{want}
        \zhtss{看}{k/`an}{see}
        \zhtsQ
        \engbox{Do you want to take a look?}

  \item \zhtss{你}{n/>i}{you}
        \zhtsd{看}{到}{k/`an}{d/`ao}{see}{to reach}{hear}
        \zhtss{了}{le}{}
        \zhtss{沒}{m/'ei}{not (have)}
        \zhtsQ
        \engbox{Do you see it or not?}

  \item \zhtss{他}{t/-a}{he}
        \zhtss{不}{b/'u}{not}
        \zhtss{要}{y/`ao}{want}
        \zhtss{去}{q/`u}{go}
        \zhtsd{歐}{洲}{o/-u}{zh/-ou}{Europe}{continent}{Europe}
        \zhtsP
        \engbox{He does not want to go to Europe.}

  \item \zhtss{他}{t/-a}{he}
        \zhtss{沒}{m/'ei}{not (have)}
        \zhtsd{去}{過}{q/`u}{g/`uo}{go}{to pass}{went}
        \zhtsd{歐}{洲}{o/-u}{zh/-ou}{Europe}{continent}{Europe}
        \zhtsP
        \engbox{He has not been to Europe.}
\end{enumerate}
\end{example}

%\begin{tblss}{Negation}{sen:negation}
%%    I do not like to eat bananas.
%%    & {\raggedright
%%    \chars{我}{w/>o}{I}
%%    \chars{不}{b/`u}{not}
%%    \chard{喜}{歡}{x/>i}{h/-uan}{happy}{cheerful}{like}
%%    \chars{吃}{ch/-i}{eat}
%%    \chard{香}{蕉}{xi/-ang}{ji/-ao}{fragrant}{banana}{banana}}\charP
%%  &
%%    I have never eaten bananas.
%%    &
%%    \chars{我}{w/>o}{I}
%%    \chars{沒}{m/'ei}{not (have)}
%%    \chard{吃}{過}{ch/-i}{g/`uo}{eat}{to pass}{eaten}
%%    \chard{香}{蕉}{xi/-ang}{ji/-ao}{fragrant}{banana}{banana}\charP
%%  \tblh
%    Do you want \blank or not?
%    &
%    \chars{你}{n/>i}{you}
%    \chars{要}{y/`ao}{want} \chars{不}{b/`u}{not} \chars{要}{y/`ao}{want}\charQ
%  &
%    Do you have \blank or not?
%    &
%    \chars{你}{n/>i}{you}
%    \chars{有}{y/>ou}{have} \chars{沒}{m/'ei}{not (have)} \chars{有}{y/>ou}{have}\charQ
%  \tblh
%%    Do you want to eat?
%%    &
%%    \chars{你}{n/>i}{you}
%%    \chars{要}{y/`ao}{want} \chars{不}{b/`u}{not} \chars{要}{y/`ao}{want}
%%    \chars{吃}{ch/-i}{eat}\charQ
%%  &
%%    Have you eaten?
%%    &
%%    \chars{你}{n/>i}{you}
%%    \chars{吃}{ch/-i}{eat}
%%    \chars{飽}{b/>ao}{full}
%%    \charsp{了}{le}{}
%%    \chars{沒}{m/'ei}{not (have)}\charQ
%%  \tblh
%%    Do you want to sleep?
%%    &
%%    \chars{你}{n/>i}{you}
%%    \chars{要}{y/`ao}{want} \chars{不}{b/`u}{not} \chars{要}{y/`ao}{want}
%%    \chard{睡}{覺}{sh/`ui}{ji/`ao}{to sleep}{sleep}{sleep}\charQ
%%  &
%%    Have you had enough sleep?
%%    &
%%    \chars{你}{n/>i}{you}
%%    \chars{睡}{sh/`ui}{to sleep}
%%    \chars{飽}{b/>ao}{full}
%%    \charsp{了}{le}{}
%%    \chars{沒}{m/'ei}{not (have)}\charQ
%  \tblh
%    Do you want to hear \blank?
%    &
%    \chars{你}{n/>i}{you}
%    \chars{要}{y/`ao}{want} \chars{不}{b/`u}{not} \chars{要}{y/`ao}{want}
%    \chars{聽}{ti\=ng}{hear}\charQ
%  &
%%    Do you hear me or not!?
%%    &
%%    \chars{你}{n/>i}{you}
%%    \chard{聽}{到}{ti\=ng}{d/`ao}{hear}{to reach}{hear}
%%    \charsp{了}{le}{}
%%    \chars{沒}{m/'ei}{not (have)}\charX\charQ
%%  \tblh
%%    Do you want to take a look?
%%    &
%%    \chars{你}{n/>i}{you}
%%    \chars{要}{y/`ao}{want} \chars{不}{b/`u}{not} \chars{要}{y/`ao}{want}
%%    \chars{看}{k/`an}{see}\charQ
%%  &
%%    Do you see it or not?
%%    &%
%%    \chars{你}{n/>i}{you}%
%%    \chard{看}{到}{k/`an}{d/`ao}{see}{to reach}{hear}%
%%    \charsp{了}{le}{}%
%%    \chars{沒}{m/'ei}{not (have)}\charQ%
%%  \tblh
%%    He does not want to go to Europe.
%%    &
%%    \chars{他}{t/-a}{he}
%%    \chars{不}{b/'u}{not}
%%    \chars{要}{y/`ao}{want}
%%    \chars{去}{q/`u}{go}
%%    \chard{歐}{洲}{o/-u}{zh/-ou}{Europe}{continent}{Europe}\charP
%%  &
%%    He has not been to Europe.
%%    &
%%    \chars{他}{t/-a}{he}
%%    \chars{沒}{m/'ei}{not (have)}
%%    \chard{去}{過}{q/`u}{g/`uo}{go}{to pass}{went}
%%    \chard{歐}{洲}{o/-u}{zh/-ou}{Europe}{continent}{Europe}\charP
%\end{tblss}

%=======================================
\section{Success and failure in verbs.}
%=======================================
Nouns and verbs in Chinese are very often composed of pairs of characters
such as ``\zht{打到}" in ``\zht{打到求}" (d/>a d/`ao qi/'u)---``hit the ball".
A verb is by nature an ``action" word---and of course one intent on
performing an action may either succeed or fail at his or her attempt.
\\
\ftbox{\tcom{\charsh{\blank}{}{(initial verb)} \zhtss{不}{b/`u}{not}    \charsh{\blank}{}{(complimentary verb)}}
            {fail to perform the action}}%
\hfill\begin{minipage}{\tw-120mm}%
  Success or failure in performing the action of a two character
  verb can be conveniently expressed by inserting a \zht{不} indicating failure
  or a \zht{得} indicating success.
  \pref{ex:verb_bu_de} (next) illustrates these structures.
  %Such use of \zht{不} and \zht{得} are illustrated in \pref{ex:verb_bu_de} (next).
  %Examples follow next.
\end{minipage}\hfill%
\ftbox{\tcom{\charsh{\blank}{}{(initial verb)} \zhtss{得}{d/'e}{to get} \charsh{\blank}{}{(complimentary verb)}}
            {successfully perform the action}}

%---------------------------------------
\begin{example}[verb pairs with \zht{不} or \zht{得}]
\label{ex:verb_bu_de}
\mbox{}\\
%---------------------------------------
\begin{enumerate}
  \item \zhtss{她}{t/-a}{she}
        \tcom{\zhtss{打}{d/>a}{hit} 
              \zhtss{得}{d/'e}{to get} 
              \zhtss{到}{d/`ao}{to reach}
             }{successfully hit}
        \zhtss{求}{qi/'u}{ball}
        \zhtsP
        \engbox{She (successfully) hit the ball.}

  \item \zhtss{她}{t/-a}{she}
        \tcom{\zhtss{打}{d/>a}{hit} \zhtss{不}{b/'u}{not} \zhtss{到}{d/`ao}{to reach}}{failed to hit}
        \zhtss{求}{qi/'u}{ball}
        \zhtsP
        \engbox{She failed tried but failed to hit the ball.}

  \item \zhtss{貓}{m/-ao}{cat}
        \tcom{\zhtss{抓}{zh/-ua}{grabbed} \zhtss{得}{d/'e}{to get} \zhtss{到}{d/`ao}{to reach}}{successfully grabbed}
        \zhtsd{老}{鼠}{l/>ao}{sh/>u}{old}{mouse}{mouse}
        \zhtsP
        \engbox{The cat successfully caught the mouse.}

  \item \zhtss{貓}{m/-ao}{cat}
      \tcom{\zhtss{抓}{zh/-ua}{grabbed} \zhtss{不}{b/'u}{not} \zhtss{到}{d/`ao}{to reach}}
           {failed to grab}
      \zhtsd{老}{鼠}{l/>ao}{sh/>u}{old}{mouse}{mouse}\zhtsP
      \engbox{The cat tried but failed to catch the mouse.}

  \item \zhtsd{王}{子}{w/'ang}{z/>i}{king}{son}{prince}
      \zhtsd{根}{本}{g/-en}{b/>en}{roots}{stem}{ultimately}
      \zhtss{把}{b/>a}{}
      \zhtsd{城}{堡}{ch/'eng}{b/>ao}{a city wall}{fortress}{castle}
      \tcom{\zhtss{進}{j/`in}{to enter} \zhtss{不}{b/'u}{not} \zhtss{去}{q/`u}{to go}}
           {failed to enter}\zhtsP
      \citetbl{\citerp{prince_pauper}{18}}
      \engbox{The Prince ultimately\\could not enter\\the castle.}

  \item %媽媽真是拿他沒辦法。
    \zhtsd{媽}{媽}{m/-a}{m/-a}{mom}{mom}{Mama}%
    \zhtss{真}{zh/-en}{really}%
    \zhtss{是}{sh/`i}{is}%
    \zhtss{拿}{n/'a}{take}%
    \zhtss{他}{t/-a}{him}%
    \zhtss{沒}{m/'ei}{no}%
    \zhtsd{辦}{法}{b/`an}{f/>a}{to deal with}{method}{way}%
    \zhtsP\citetbl{\citerp{kiki2008}{3}}
    \engbox{Mama really couldn't\\control him.}
    
  \item %奇奇總喜歡和大家唱反調,說什麼他都偏不要。
    \zhtsd{奇}{奇}{q/'i}{q/'i}{unusual}{unusual}{Kiki}%
    \zhtss{總}{z/>ong}{always}%
    \zhtsd{喜}{歡}{x/>i}{hu/-an}{happy}{cheerful}{to like}%
    \zhtss{和}{h/`an}{with}
    \zhtsd{大}{家}{d/`a}{ji/-a}{big}{family}{everyone}%
    \zhtss{唱}{ch/`ang}{sing}%
    \zhtss{反}{f/>an}{opposite}%
    \zhtss{調}{di/`ao}{tune}%
    \zhtsC%
    \zhtss{說}{sh/-uo}{to say}%
    \zhtsd{什}{麼}{sh/'e}{me}{what}{(particle)}{what}%
    \zhtss{他}{t/-a}{he}%
    \zhtss{都}{d/-ou}{all}%
    \zhtss{偏}{pi/-an}{slightly}%
    \zhtss{不}{b/'u}{not}%
    \zhtss{要}{y/`ao}{want}%
    \zhtsP\citetbl{\citerp{kiki2008}{3}}
    \engbox{Kiki always liked to sing a different tune;\\no matter what anyone said,  he would\\not in the slightest degree agree.}
\end{enumerate}
\end{example}

%=======================================
\section{Double negation}
%=======================================
\ftbox{\zhtssH{非}{f/-ei}{not} {\Huge\ldots} \zhtsdH{不}{可}{b/`u}{k/>e}{not}{to permit}{not permitted}}
\quad\begin{minipage}{\tw-68mm}%
  Sentences in Chinese are sometimes stated as a double negative,
  with the first negation toward the beginning of the sentence and the
  second toward the end.
  This same construct exists in English as well, such as in
  ``If I don't make it to the top of the mountain,\ldots well that is simply unthinkable!"
  That is, you state that something absolutely must occur and
  for it not to occur is unfathomable.
  Examples follow in \prefpp{ex:negation_double}.
\end{minipage}

%---------------------------------------
\begin{example}{Double negation}
\mbox{}\\
\label{ex:negation_double}
%---------------------------------------
\begin{enumerate}
  \item %真面目
      \zhtss{這}{zh\`e}{this}
      \zhtsd{一}{次}{y/'i}{c/`i}{one}{order}{once}
      \zhtss{我}{w/>o}{I}
      \zhtss{非}{f/-ei}{not}
      \zhtss{要}{y/`ao}{will}
      \zhtsd{看}{到}{k/`an}{d/`ao}{see}{arrive}{see}
      \zhtsd{他}{的}{t/-a}{de}{he}{'s}{his}
      \zhtss{真}{zh/-en}{real}
      \zhtsd{面}{目}{mi/`an}{m/`u}{face}{eye}{face},
      \zhtsd{不}{可}{b/`u}{k/>e}{not}{to permit}{not permissible}\zhtsX
    \citetbl{\citerp{baobao_gorilla}{7}}
    \\\engbox{This time I don't see him truly face-to-face eye-to-eye,\ldots it just can't be!}
  
  \item \zhtss{我}{w/>o}{I}
        \zhtss{非}{f/-ei}{not}
        \zhtss{要}{y/`ao}{want}
        \zhtsd{找}{到}{zh/>ao}{d/`ao}{to look for}{to reach}{to find}
        \zhtss{他}{t/-a}{him}
        \zhtss{不}{b/`u}{not} 
        \zhtss{可}{k/>e}{to permit}
        \zhtsP\citetbl{\citerp{3musketeers}{6}}
        \engbox{There is no way\\that I don't want\\to find him.}
\end{enumerate}
\end{example}

\ftbox{\zhtsdH{才}{不}{c/'ai}{b/`u}{just}{not}{just don't}}
\begin{minipage}{\tw-75mm}
  An additional negation is constructed with \zht{才不}
\end{minipage}

%---------------------------------------
\begin{example}
%---------------------------------------
    \zhtsd{怪}{物}{g/`uai}{w/`u}{strange}{matter}{monster}%
    \zhtsd{才}{不}{c/'ai}{b/`u}{just}{not}{just don't}%
    \zhtss{吃}{ch/-i}{eat}%
    \zhtss{魚}{y/'u}{fish}%
    \zhtsd{蛋}{糕}{d/`an}{g/-ao}{egg}{pastry}{cake}%
    \zhtss{呢}{ne}{}%
    \zhtsX%
    \cittrp{stephens1999}{11}%
    \engbox{Monsters just don't\\eat fish cake!}
\end{example}

%=======================================
\section{Emphasizing negation}
%=======================================
%\begin{tblvv}{Emphasis and contrasts}{voc:emphasis}
%  \tblc entirely                & \zhtssp{並}{bi/`ng}{entirely}
%\end{tblvv}

\ftbox{\zhtssp{並}{bi/`ng}{entirely}}\qquad
\begin{minipage}{\tw-50mm}
The character \zht{卻} is used to support the contrast of ideas.
Adding the character \zht{並} in front of a negating verb adds emphasis to the verb.
The use of \zht{並} is completely optional.
\prefp{ex:emphasis} (next) gives an example.
\end{minipage}

%---------------------------------------
\begin{example}[\exmd{Emphasizing negation using \zht{並}}]
\mbox{}\\
\label{ex:emphasis}
%---------------------------------------
\begin{enumerate}
  \item \zhtsd{湯}{姆}{t/-ang}{m/>u}{soup}{governess}{Tom}%
        \zhtss{並}{b/`ing}{entirely}%
        \zhtsd{沒}{有}{m/'ei}{y/>ou}{not}{have}{not have}%
        \zhtss{像}{xi/`ang}{to resemble}%
        \zhtsd{王}{子}{w/'ang}{z/>i}{king}{son}{prince}%
        \zhtsd{那}{麼}{n/`a}{me}{that}{}{so}%
        \zhtsd{幸}{運}{x/`ing}{y/`un}{good fortune}{luck}{fortunate}\zhtsP%
        \citetbl{\citerp{prince_pauper}{2}}%
        \\\engbox{Tom's life was entirely without resemblance to the Prince's so fortunate life.}
\end{enumerate}
\end{example}

