%============================================================================
% Daniel J. Greenhoe
% XeLaTeX file
%============================================================================

%=======================================
\chapter{Graphemes of Mandarin Chinese}
%=======================================
%=======================================
\section{Language scripts}
%=======================================
Many spoken languages also have written forms, which express information
in terms of symbols, called
\href{http://en.wikipedia.org/wiki/Grapheme}{\hie{graphemes}}.
Linguists roughly classify written languages into two categories
based on the graphemes that they use:
\\\begin{tabular}{lll}
  \circOne & \href{http://en.wikipedia.org/wiki/Phonogram_(linguistics)}{\hie{phonograms}} (or \hie{phonographs})
           & symbols represent sounds
           \\
  \circTwo & \href{http://en.wikipedia.org/wiki/Logogram}{\hie{logogram}} (or \hie{logographs})
           & graphemes represent words
\end{tabular}\\

The advantage of a phonogram is that it is somewhat easier to know how a grapheme ``sounds",
but relatively more difficult to know what it means.
The advantage of a logogram is that it is somewhat easier to know what a grapheme means,
but relatively more difficult to know how it ``sounds".

Examples of Phonographic languages include all the written languages based on the
\href{http://en.wikipedia.org/wiki/Latin_alphabet}{Latin alphabet}
such as English, German, Spanish, French, etc.,
as well as Greek and other languages.
One of the most famous examples, howbeit not currently in common use, of a logograph is
\href{http://en.wikipedia.org/wiki/Egyptian_hieroglyphics}{Egyptian hieroglyphics},
a language that the entire world forgot how to read until the discovery of the
\href{http://en.wikipedia.org/wiki/Rosetta_Stone}{\hie{Rosetta Stone}} by
Napolean's troops and subsequent decipherment by
\href{http://en.wikipedia.org/wiki/Thomas_Young_(scientist)}{Thomas Young} and
\href{http://en.wikipedia.org/wiki/Jean-Fran\%C3\%A7ois_Champollion}{Jean-Francois Champollion}.


\begin{figure}
  \centering
  \includegraphics[width=75mm]{../common/graphics/pictographs/usnpsn_wkp_pdomain.jpg}
  \caption{Pictographs used by the U.S. National Park Service \label{fig:usnpsn}}
\end{figure}
%---------------------------------------
\begin{example}[\exmd{U.S. National Park Service}]
\label{ex:usnps}
%---------------------------------------
And the \hie{U.S. National Park Service} uses a kind of logographic language, as illustrated in 
\prefpp{fig:usnpsn}.\footnote{\url{http://en.wikipedia.org/wiki/File:National_Park_Service_sample_pictographs.svg}}
\end{example}

In any written language (a script), each basic written picture is called a \hib{grapheme}.
In English, a grapheme is usually called a ``letter".
In Chinese, a grapheme is usually called a ``character".
In general there are many ways to render (or draw) any given grapheme.
Each such rendering is called a \hib{glyph}.

%---------------------------------------
\begin{example}[\exmd{The ``a" grapheme}]
\label{ex:grapheme_a}
%---------------------------------------
\exboxt{
  In English, all these \hib{glyphs} (font typefaces) represent {the same} \hib{grapheme} (character):
  \\\indentx\Huge
  {\fntSimple a}, {\fntUMtypewriter a}, {\fntHeuristica a}, {\fntPrintDashed a}, {\fntLavi a}, {\fntZapf a}
  }
\end{example}

%---------------------------------------
\begin{example}[\exm{The \zht{山} grapheme}]
\footnote{
  FireFly font: {\fntFirefly 山}; 
  UKai font:    {\fntUKai 山};
  Han Wang CC02 font: {\fntHanWangCC 山};
  HanWangKanDaYan font: {\fntHanWangKanDaYan 山};\\
  \citerc{sears}{\url{http://hanziyuan.net/\#\%E5\%B1\%B1}: J21374, S06639, L06208, B14280}
  }
\label{ex:grapheme_shan}
%---------------------------------------
\exboxt{
  In Chinese, all of these \hib{glyphs} (font typefaces) represent {the same} \hib{grapheme} (character):
  \\\indentx\Huge
  {\fntFirefly 山}, %{\fntUMing 山}, 
  {\fntUKai 山}, %{\fntWangClear 山}
  {\fntHanWangCC 山}, 
  {\fntHanWangKanDaYan 山},
  \includegraphics[height=7.5mm]{../common/zht/characters/shan_hanziyuan_J21374.jpg},
  \includegraphics[height=7.5mm]{../common/zht/characters/shan_hanziyuan_S06639.jpg},
  \includegraphics[height=7.5mm]{../common/zht/characters/shan_hanziyuan_L06208.jpg},
  \includegraphics[height=7.5mm]{../common/zht/characters/shan_hanziyuan_B14280.jpg}
  }
\end{example}



%---------------------------------------
\begin{example}[\exmd{Egyptian Hieroglyphs}]
\citetbl{
  \citerpg{atiya2006}{52}{9771736345}
  }
\label{ex:ehier}
%---------------------------------------
The English \hib{script} is mostly \prope{phonetic}.
The Chinese \hib{script} is mostly \prope{ideographic}.
Egyptian Hieroglphics {combines}  %{結合} 
\\\indentx\begin{tabular}{cll}
  \imark& phonetic graphemes & (e.g. 
  {\fnthieroglyphs\symbol{"13216}}=``h" sound, 
  {\fnthieroglyphs\symbol{"13254}}=``n" sound, 
  {\fnthieroglyphs\symbol{"13158}}=``w" sound)
  \quad and 
  \\
  \imark& ideographic graphemes & (e.g.
  {\fnthieroglyphs\symbol{"133CA}}=liquid,
  {\fnthieroglyphs\symbol{"1300A}}=jubilation,
  {\fnthieroglyphs\symbol{"13015}}=man,
  {\fnthieroglyphs\symbol{"1302D}}=woman,
  |||=plural)
\end{tabular}
\\
Here are some examples:
\exboxt{\begin{tabular}{>{\scs}r>{\fnthieroglyphs\Huge}lclcl}
      1. & \symbol{"13216} \symbol{"13254} \symbol{"13158} \symbol{"133CA} &=& h+n+w+liquid     &=& beer pot
    \\2. & \symbol{"13216} \symbol{"13254} \symbol{"13158} \symbol{"1300A} &=& h+n+w+jubilation &=& rejoicing
    \\3. & \symbol{"13216} \symbol{"13254} \symbol{"13158} 
      \begin{tabstr}{0.25}\begin{tabular}{@{}c}\symbol{"13015}\symbol{"1302D}\\\normalsize$|\;\;|\;\;|$\end{tabular}\end{tabstr} 
          &=& \tbox{h+n+w+(man and woman\\over a plural symbol)} &=& neighbors
\end{tabular}}
\end{example}
%\\
%      \symbol{"13216} \symbol{"13254} \symbol{"13158} \symbol{"133CA}} = h+n+w+\zcom{liquid}{液體} = \zcom{beer}{啤酒} pot
%    \\\symbol{"13216} \symbol{"13254} \symbol{"13158} \symbol{"1300A}} = h+n+w+\zcom{jubilation}{歡騰} = \zcom{rejoicing}{欣喜}
%    \\\symbol{"13216} \symbol{"13254} \symbol{"13158} \begin{tabstr}{0.5}\begin{tabular}{c}\symbol{"13015}\symbol{"1302D}\\\normalsize$|\;\;|\;\;|$\end{tabular}\end{tabstr}} = h+n+w+(man and woman over a plural symbol) = neighbors
%{\fnthieroglyphs\Huge  \symbol{"13216} \symbol{"13254} \symbol{"13158} \begin{tabstr}{0.25}\begin{tabular}{c}\symbol{"13015}\symbol{"1302D}\\\symbol{"133FC}\end{tabular}\end{tabstr}} = h+n+w+(man and woman) = neighbors

%---------------------------------------
\begin{example}[\exmd{Assorted scripts}]
\label{ex:scripts}
%---------------------------------------
Here are some examples of scripts and graphemes belonging to those scripts:
%{Hints  Arabic, English, Egyptian Hieroglyphics, \zcomi{Greek}{希臘語}, \zcomi{Hebrew}{希伯來語}, Latin, Tagalog,}
\\\exboxt{\begin{tabular}{>{\scs}rll}
      %1. & {\fntagreek{ABGDEZHJIKLMNXOPRSTUFQYW abgdezhjiklmnxoprstufqyw}}
       1. & {\fntagreek{ABGDEZHJI$\cdots$TUFQYW abgdezhjiklmnxoprstufqyw}}
          & {Greek}
    \\2.  & {{ABCDEFGHI$\cdots$UVWXYZ abcdefghijklmnopqrstuvwxyz}}
          & {Latin}
    \\3.  & {\fnthebrew\symbol{"05D0}\symbol{"05D1}\symbol{"05D2}\symbol{"05D3}\symbol{"05D4}\symbol{"05D5}\symbol{"05D6}\symbol{"05D7}\symbol{"05D8}\symbol{"05D9}\symbol{"05DA}\symbol{"05DB}\symbol{"05DC}\symbol{"05DD}\symbol{"05DE}\symbol{"05DF}
                            \symbol{"05E0}\symbol{"05E1}\symbol{"05E2}\symbol{"05E3}\symbol{"05E4}\symbol{"05E5}\symbol{"05E6}\symbol{"05E7}\symbol{"05E8}\symbol{"05E9}\symbol{"05EA}\symbol{"05F0}}
          & {Hebrew}
    \\4.  & {\fntarabic\symbol{"0627}\symbol{"0628}\symbol{"0629}\symbol{"062A}\symbol{"062B}\symbol{"062C}\symbol{"062D}\symbol{"062E}\symbol{"062F}
                            \symbol{"0630}\symbol{"0631}\symbol{"0632}\symbol{"0633}\symbol{"0634}\symbol{"0635}\symbol{"0636}\symbol{"0637}\symbol{"0638}\symbol{"0639}\symbol{"063A}}
          & {Arabic}
    \\5.  & {\fnttagalog\symbol{"1700}\symbol{"1701}\symbol{"1702}\symbol{"1703}\symbol{"1704}\symbol{"1705}\symbol{"1706}\symbol{"1707}\symbol{"1708}\symbol{"1709}\symbol{"170A}\symbol{"170B}\symbol{"170C}\symbol{"170E}\symbol{"170F}
                            \symbol{"1710}\symbol{"1711}}
          & {Tagalog}
    \\6.  & {\fnthieroglyphs\Huge\symbol{"13000}\symbol{"1303C}\symbol{"13002}\symbol{"13003}\symbol{"130E4}\symbol{"1314A}\symbol{"1335D}\symbol{"133B9}\symbol{"133DE}
                            }
          & {Egyptian Hieroglyphs}
  \end{tabular}}
\end{example}


%=======================================
\section{Chinese glyphs}
%=======================================
\begin{minipage}{\tw-30mm}
By far the most commonly used logographic language in the world today is Chinese.
This logograph, as far as current common use is concerned, comes in two forms,
\hie{traditional Chinese} and \hie{simplified Chinese}.
In much of this text, typesetting is performed as illustrated to the right
with PinYin (see \prefp{sec:pinyin}) on top, the traditional character below that, 
the simplified character below that, and English meanings below that.
\end{minipage}\hfill%
\ftbox{\zhtssH{見}{ji/`an}{to see}}

\begin{minipage}{\tw-30mm}
Because Chinese is a \prope{logographic} language, sometimes one can make an educated guess
as to the meaning of a grapheme (called a \hie{character} for the Chinese written language)
by the way it ``looks". For example, the character \zht{山} means ``mountain",
and it kind of looks like a three peak mountain.
In some older forms, the resemblence is even closer (see \prefp{ex:grapheme_shan}).
\end{minipage}\hfill%
\ftbox{\zhtssH{山}{sh/-an}{mountain}}

But this only really works for some characters. 
Over the centuries since the publication in 121 A.D. of \zht{說文解字} (Sh/-uo W/'en Ji/>e Z/`i)
by \zht{許慎} (X/>u Sh/`en), characters are often organized into these six categories:%
  \citetbl{
    \citerppg{ross2006}{14}{15}{0415700108}\\
    \citer{harbaugh2009}\\
    \citer{xu121}
    }
\begin{tabular}[t]{|c|c|c|}
    \hline
     \circOne\quad   pictographs 
    &\circTwo\quad   ideographs
    &\circThree\quad associative compounds
    \\
     \zhtsds{象}{形}{xi/`ang}{x/'ing}{appearance}{shape}{pictograph}
    &\zhtsds{指}{事}{zh/>i}{sh/`i}{finger}{affair}{ideograph}
    &\zhtsds{會}{意}{h/`ui}{y/`i}{to meet}{meaning}{understanding}
    \\\hline
     \circFour\quad phonetic compounds     
    &\circFive\quad false barrowings
    &\circSix\quad  semantic derivations
    \\
     \zhtsds{形}{聲}{x/'ing}{sh/-eng}{shape}{sound}{phonetic compound}
    &\zhtsds{假}{借}{ji/>e}{ji/`a}{false}{to barrow}{to make use of}
    &\zhtsds{轉}{注}{zh/>uan}{zh/`u}{to turn}{to pour}{semantic derivation}
    \\\hline
\end{tabular}

%---------------------------------------
\begin{example}[\exmd{pictographs}]
%---------------------------------------
Here are some examples of \structe{pictographs}:%
\begin{longtable}[l]{|*{3}{>{\centering}c >{\raggedright}p{\tw/3-28mm}|}}
  \hline
  \tblx \zhtssu{水}{sh/>ui}{water}{http://zhongwen.com/d/164/x244.htm}   & streams flowing together
  \tblc \zhtssu{人}{r/'en}{person}{http://zhongwen.com/d/164/x72.htm}    & a person standing two legs
  \tblc \zhtssu{儿}{r/'en}{person}{http://zhongwen.com/d/164/x73.htm}    & the two legs of a person
  \tblh \zhtssu{入}{r/`u} {to enter}{http://zhongwen.com/d/164/x74.htm}  & roots descending into the ground
  \tblc \zhtssu{力}{l/`i}{force}{http://zhongwen.com/d/164/x79.htm}      & a tendon
  \tblc \zhtssu{刀}{d/-ao}{knife}{http://zhongwen.com/d/164/x77.htm}     & a knife
  \tblh \zhtssu{丁}{d/-ing}{nail}{http://zhongwen.com/d/164/x66.htm}     & a nail
  \tblc \zhtssu{几}{j/-i}{small table}{http://zhongwen.com/d/164/x76.htm}& a small table or stool
  \tblc \zhtssu{大}{d/`a}{big}{http://zhongwen.com/d/164/x106.htm}       & resembles a standing person with open arms
  \tblh \zhtssu{尸}{sh/-i}{corpse}{http://zhongwen.com/d/164/x114.htm}   & resembles a person lying down
  \tblc \zhtssu{弋}{y/`i}{stake}{http://zhongwen.com/d/164/x124.htm}     & a stake
  \tblc \zhtssu{丌}{j/-i}{stool}{http://zhongwen.com/d/201/x70.htm}      & a stool or workbench
  \tblh \zhtssu{凡}{f/'an}{all}{http://zhongwen.com/d/164/x90.htm}       & earlier form was an encompassing square
  \tblc \zhtssu{干}{g/-an}{shield}{http://zhongwen.com/d/164/x122.htm}   & a shield
  \tblc \zhtssu{工}{g/-ong}{work}{http://zhongwen.com/d/164/x117.htm}    & a carpenter's square
  \tblh \zhtssu{匚}{f/-ang}{}{http://zhongwen.com/d/201/x67.htm}         & a hollowed out piece of wood
  \tblc \zhtssu{巾}{j/-in}{cloth}{http://zhongwen.com/d/164/x121.htm}    & a hanging handkerchief
  \tblc \zhtssu{女}{n/>:u}{woman}{http://zhongwen.com/d/164/x107.htm}   & a kneeling woman
  \tblh \zhtssu{子}{\v{z}i}{son}{http://zhongwen.com/d/164/x108.htm}     & an infant with legs bundled together
  \tblc \zhtssu{也}{y/>e}{also}{http://zhongwen.com/d/164/x93.htm}       & an ancient funnel or wash basin
  \tblc \zhtss{見}{ji/`an}{to see}          & resembles an eye
  \tblh \zhtss{口}{k/>ou}{mouth}            & resembles an open mouth
  \tblc \zhtss{木}{m/`u}{wood/tree}         & resembles a tree
  \tblc \zhtss{哭}{k/-u}{to cry}            & resembles a person with two tear drops in the eyes
  \tblh \zhtss{去}{q/`u}{to go}             & resembles a walking person
  \tblc \zhtss{山}{sh/-an}{mountain}        & resembles a three peak mountain
  \tblc \zhtss{上}{sh/`ang}{above}          & the vertical and small strokes are \emph{above} the long horizontal stroke
  \tblh \zhtss{下}{xi/`a}{below}            & the diagonal and small strokes are \emph{below} the long horizontal stroke
  \tblc \zhtss{笑}{xi/`ao}{to smile}        & resembles a smiling person with a moustache
  \tblc \zhtss{爪}{zh/>ao}{claw/talon}      & resembles the claw of an animal (e.g. lion)
  \tblh
\end{longtable}
\end{example}

Moreover, complex characters are sometimes constructed from more primitive ones.
If you know the meaning of the primitive characters, it can help
suggest the meaning of some complex ones.

%---------------------------------------
\begin{example}[\exmd{Complex from primitive}]
\mbox{}\\
%---------------------------------------
\begin{longtable}[l]{*{2}{|>{\centering}p{2\tw/16} >{\raggedright}p{5\tw/16}|}}
  \hline
  \tblx \zhtss{林}{l/'in}{forest}           & resembles two trees (\zht{木}) as in \zht{林}=\zht{木}+\zht{木}.
  \tblc \zhtss{森}{s/-en}{trees}            & resembles three trees as in \zht{森}=\zht{木}+\zht{木}+\zht{木}
  \tblh \zhtss{仙}{xi/-an}{celestial being} & resembles a man (\zht{\symbol{"2E85}}, r/'en) of the mountains (\zht{山}, sh/'an).
  \tblc \zhtss{好}{h/>ao}{good}             & life is good for a man who has a wife (\zht{女}=woman, n/>:u) and son (\zht{子}, zi)
  \tblh \href{http://zhongwen.com/d/171/x79.htm}{\zhtss{保}{b/>ao}{to protect}} 
                                            & a man (\zht{\symbol{"2E85}}, r/'en) 
                                              with a baby (\zht{呆}, d/-ai)
  \tblc \href{http://zhongwen.com/d/178/x238.htm}{\zhtss{船}{chu/'an}{boat}}           
                                            & \zhtss{船}{chu/'an}{boat}=
                                              \zhtss{舟}{zhou}{boat}+
                                              \zhtss{八}{b/-a}{8}+
                                              \zhtss{口}{k/>ou}{mouth};\footnotemark
  \tblh
\end{longtable}
\footnotetext{
  \href{http://zhongwen.com/d/178/x238.htm}{\zht{船}} = \zht{舟}+\zht{八}+\zht{口}:
  Interestingly enough, this has similarity to the Biblical account of the flood 
  in \href{https://www.biblegateway.com/passage/?search=Genesis+6&version=NASB}{Genesis 6--9}.}
\end{example}

%=======================================
\section{Onomatopoeia characters}
%=======================================
Despite Chinese being a logographic language and therefore characters represent
meanings rather than sounds, there are some characters that,
although they have meanings, are not primarily used for those meanings,
but rather are primarily used just for the sound that they make.
That is, these characters represent
\href{http://en.wikipedia.org/wiki/Onomatopoeia}{\hie{onomatopoeias}}.
Most of these characters
contain the \zht{口} (k/>ou, mouth) character,
indicating that the character represents a sound (made with the mouth).
If you wish to write, say, stories for children, then such characters may come in very handy.
\pref{tbl:onomatopoeia} provides some examples of such characters,
while \prefpp{ex:onomatopoeia} illustrates their use with sentences.%
\citetbl{%
  \citer{sound_animal1}\\
  \citer{sound_animal2}\\
  \citer{sound_life1}\\
  \citer{sound_life2}\\
  \citer{sound_vocal}\\
  \citer{sound_car}%
  }

\begin{longtable}[l]{*{2}{|>{\centering}p{2\tw/16} >{\raggedright}p{5\tw/16}|}}
  \caption{Onomatopoeias characters \label{tbl:onomatopoeia}}
  \tblh \zhtssp{啊}{/-a}{(onomatopoeia)}        & the ``ah" sound made such as when scared
  \tblc \zhtssp{哎}{/-ai}{(onomatopoeia)}       & the expression ``hey" or ``ah"; can be used together with \zht{唷} as an exclamation as in \zht{哎唷﹗} (/-ai y/'o!)
  \tblh \zhtss {嗷}{a/'o}{cry of hunger}        & \zht{嗷}=\zht{口}+\zht{敖} (a/`o, proud)
  \tblc \zhtss {叭}{b/-a}{trumpet}              & the honking sound of a car; \zht{叭}=\zht{口}+\zht{八} (b/-a, 8)
  \tblh \zhtss {吧}{b/-a}{(a particle)}         &
  \tblc \zhtssp{嗶}{b/`i}{(onomatopoeia)}       & the ``beep" sound of a whistle
  \tblh \zhtss {嗤}{ch/-i}{to sneer}            & sound of ripping paper
  \tblc \zhtssp{噹}{d/-ang}{(onomatopoeia)}     & metallic crashing sound
  \tblh \zhtss {叮}{d/-ing}{to sting}           & used with \zht{咚} (d/-ong) for the ``ding-dong" sound of a doorbell
  \tblc \zhtssp{咚}{d/-ong}{(onomatopoeia)}     & sound of a drum, sound of falling object impacting the ground
  \tblh \zhtssp{嘟}{d/-u}{(onomatopoeia)}       & the ``toot" sound of vehicles such as trains; \zht{嘟}=\zht{口}+\zht{都} (d/-u, big city)
  \tblc \zhtssp{嘎}{g/-a}{(onomatopoeia)}       & the ``quack quack" sound of ducks as in ``\zht{嘎嘎}"
  \tblh \zhtss {咯}{g/'e}{a click}              & a click
  \tblc \zhtss {咕}{g/-u}{to murmur}            & the hooting sound of an owl, the crowing shound of a rooster
  \tblh \zhtssp{呱}{g/-ua}{(onomatopoeia)}      & the ``ribbet ribbet" croaking sound of frogs as in ``\zht{呱呱}"
  \tblc \zhtss {哈}{h/-a}{to breathe on}        & the ``ha ha" laughing sound as in ``\zht{哈哈}!", the ``achoo" sneezing sound as in \zht{哈啾} (h/-a ji/-u)
  \tblh \zhtss {嗨}{h/-ai}{``hi""}\footnotemark & the greeting "Hi"
  \tblc \zhtss {吼}{h/>ou}{to roar}           & the roaring sound of a lion
  \tblh \zhtss {呼}{h/-u}{to exhale}            & \zht{呼}=\zht{口}+\zht{乎} (h/-u, at)
  \tblc \zhtssp{嘩}{h/-ua}{(onomatopoeia)}      & sound of a wave; \zht{嘩}=\zht{口}+\zht{華} (h/'ua, China)
  \tblh \zhtssp{嚄}{h/`uo}{(onomatopoeia)}      & the ``oink" sound of pigs
  \tblc \zhtss {唧}{j/-i}{pump}                 & the chirping sound of crickets as in ``\zht{唧唧}"
  \tblh \zhtss {嘰}{j/-i}{to talk indistinctly} & the chirping sound of birds as in ``\zht{嘰嘰}"
  \tblc \zhtss {啾}{ji/-u}{chirping sound}      & the chirping sounds of birds or insects
  \tblh \zhtss {叩}{k/`ou}{to knock}            & the sound of knocking on a door
  \tblc \zhtssp{啦}{la}{(onomatopoeia)}         &
  \tblh \zhtss {喵}{mi/-ao}{``meow"}            & the sound of cats as in ``\zht{喵}\ldots\zht{喵}"
  \tblc \zhtssp{咩}{mi/-e}{(onomatopoeia)}      & the bleating sound (\zht{口}) of goats (\zht{山羊});
  \tblh \zhtssp{哞}{m/'ou}{(onomatopoeia)}      & the lowing sound (\zht{口}) of cows (\zht{牛})
  \tblc \zhtss {嚕}{l/-u}{verbose}              & used with \zht{咕} (g/-u, to murmur) to describe a hungry stomach \ldots as in ``\zht{咕嚕 咕嚕}!"
  \tblh \zhtss {囉}{l/-uo}{to chatter}          & used at end of sentences as in ``\ldots oh!"
  \tblc\zhtssp{喔}{/-o}{(onomatopoeia)}        & an exclamation or crowing sound of roosters;
  \tblh\zhtssp{啪}{p/-a}{(onomatopoeia)}       &
  \tblc\zhtssp{砰}{p/-eng}{(onomatopoeia)}     & sound of a bouncing ball
  \tblh\zhtssp{乒}{p/-ing}{(onomatopoeia)}     & sound of a piano
  \tblc\zhtssp{噗}{p/-u}{(onomatopoeia)}       & the ``putt putt putt" sound of a slow vehicle
  \tblh\zhtssp{嘶}{s/-i}{(onomatopoeia)}       & the neighing sound of a horse
  \tblc\zhtss {嗒}{t/`a}{dejected}             &
  \tblh\zhtssp{哇}{w/-a}{(a particle)}         & sound of crying
  \tblc\zhtssp{嗡}{w/-eng}{(onomatopoeia)}     & the buzzing sound of insects as in ``\zht{嗡嗡}".
  \tblh\zhtssp{嗚}{w/-u}{(onomatopoeia)}       & ``toot" or ``hoot" sound of vehicles
  \tblc\zhtss {咻}{xi/-u}{din}                 & the sound of airplanes as in ``\zht{咻 咻}"
  \tblh\zhtssp{唷}{y/-o}{}                     & can be used together with \zht{唷} as an exclamation as in \zht{哎唷﹗} (/-ai y/'o!)
  \tblc\zhtss {吱}{z/-i}{a squeak}             & a squeak; the sound that monkeys make as in ``\zht{吱吱}"
  \tabularnewline\hline
  %\tblh&
\end{longtable}
\citetblt{\citerp{AnnieSpider}{7}}

%\begin{tbls}{Onomatopoeias}{sen:onomatopoeia}
%  \tblx ``Cock-a-doodle-doo!" the rooster crowed, and the sky was bright again.
%      & \zhtssp{咕}{g/-u}{to murmur}\zhtssp{咕}{g/-u}{to murmur}\zhtssp{咕}{g/-u}{to murmur}\zhtsX
%        \zhtsd{公}{雞}{g/-ong}{j/-i}{male}{chicken}{rooster}%
%        \zhtss{叫}{ji/`ao}{to call}%
%        \zhtss{著}{zhe}{(adverbial)}\zhtsC
%        \zhtss{天}{ti/-an}{sky}%
%        \zhtss{亮}{li/`ang}{bright}%
%        \zhtss{了}{le}{(change of state)}\zhtsP%
%        \citep{dream}{27}%
%  \tblh The cow's stomach was so hungry that with a ``gurgle gurgle" it called out.
%     %& \zhtss{黃}{hu/'ang}{yellow}\zhtss{牛}{ni/'u}{cow}\zhtss{的}{de}{'s}%
%      & \zhtss{牛}{ni/'u}{cow}\zhtss{的}{de}{'s}%
%        \zhtsd{肚}{子}{d/`u}{zi}{stomach}{son}{stomach}%
%        \zhtss{餓}{/`e}{hungry}%
%        \zhtss{得}{de}{to the extent of}%
%        \zhtsd{咕}{嚕}{g/-u}{l/-u}{(a sound)}{(a sound)}{``gurgle"}%
%        \zhtsd{咕}{嚕}{g/-u}{l/-u}{(a sound)}{(a sound)}{``gurgle"}%
%        \zhtss{叫}{ji/`ao}{to call}\zhtsP%
%        \citep{baobao_cow}{22}%
%  \tblh ``Achoo! Achoo!" It seems Bin-Bin has caught a cold.
%      & \zhtsd{哈}{啾}{h/-a}{ji/-u}{to breathe}{chirps}{achoo}\zhtsX
%        \zhtsd{哈}{啾}{h/-a}{ji/-u}{to breathe}{chirps}{achoo}\zhtsX
%        \zhtsd{彬}{彬}{bi/-n}{bi/-n}{intelligent}{intelligent}{Bin-Bin}%
%        \zhtsd{好}{像}{h/>ao}{xi/`ang}{good}{to resemble}{to seem}%
%        \zhtsd{感}{冒}{g/>an}{m/`ao}{to feel}{to emit}{to catch a cold}%
%        \zhtss{了}{le}{(change of state)}%
%        \zhtsP%
%        \citep{achoo}{1}%
%  \tblh ``Knock knock knock!" There arose the sound of knocking at the door.
%      & \zhtss {叩}{k/`ou}{to knock}\zhtsX
%        \zhtss {叩}{k/`ou}{to knock}\zhtsX
%        \zhtss {叩}{k/`ou}{to knock}\zhtsX
%        \zhtss{這}{zh/`e}{this}%
%        \zhtss{時}{/'shi}{time}%
%        \zhtss{門}{m/'en}{door}%
%        \zhtss{外}{w/`ai}{outside}%
%        \zhtss{響}{xi/>ang}{sound}%
%        \zhtss{起}{\v{q}i}{to start}%
%        \zhtsd{敲}{門}{qi/-ao}{m/'en}{to knock}{door}{to knock at the door}%
%        \zhtss{的}{de}{('s)}%
%        \zhtsd{聲}{音}{sh/-eng}{yi/-n}{sound}{sound}{sound}%
%        \zhtsP%
%        \citep{cloud_pillow}{23}%
%  \tblh He leaned his head against the pillow, and in a brief moment, with a "ZZZZZ" just fell asleep.
%      & \zhtss{他}{t/-a}{he}%
%        \zhtssp{把}{b/>a}{}%
%        \zhtss{頭}{t/'ou}{head}%
%        \zhtss{靠}{k/`ao}{to lean}%
%        \zhtss{在}{z/`ai}{at}%
%        \zhtsd{枕}{頭}{zh/>en}{t/'ou}{pillow}{head}{pillow}%
%        \zhtss{上}{sh/`ang}{on top of}\zhtsC
%        \zhtsd{一}{下}{/'yi}{xi/`a}{one}{next}{in a short while}%
%        \zhtss{子}{zi}{son}\zhtsC
%        \zhtss{就}{ji/`u}{just}%
%        \zhtssp{呼}{h/-u}{to exhale}\zhtssp{嚕}{l/-u}{verbose}%
%        \zhtssp{呼}{h/-u}{to exhale}\zhtssp{嚕}{l/-u}{verbose}%
%        \zhtsd{睡}{著}{sh/`ui}{zh/'ao}{to sleep}{}{to sleep}%
%        \zhtss{了}{le}{(change of state)}\zhtsP%
%        \citep{cloud_pillow}{27}%
%  %\tblh The sun had come out, the sky was bright, and the little birds were singing ``chirp chirp chirp".
%  %    & 太陽出來天亮了小鳥知知zha zha
%  %      \citep{dream}{7}%
%  %\tblh Every morning, all the people in the village can hear the rooster's
%  %      ``Cock-a-doodle-doo! Cock-a-doodle-doo!" singing.
%  %    & \zhtsd{每}{天}{m/>ei}{ti/-an}{every}{sky}{everyday}%
%  %      \zhtsd{早}{上}{z/>ao}{sh/`ang}{morning}{above}{morning}%
%  %      \zhtsd{村}{子}{c/-un}{zi}{village}{son}{village}%
%  %      \zhtss{裡}{l/>i}{inside}%
%  %      \zhtss{的}{de}{('s)}%
%  %      \zhtss{人}{r/'en}{people}%
%  %      \zhtss{都}{d/-ou}{all}%
%  %      \zhtss{會}{h/`ui}{to be able}%
%  %      \zhtsd{聽}{到}{ti/-ng}{d/`ao}{to listen}{to reach}{to hear}%
%  %      %\zhtss{大}{d/`a}{big}%
%  %      \zhtsd{公}{雞}{g/-ong}{j/-i}{male}{chicken}{rooster}%
%  %      \zhtss{在}{z/`ai}{in the process of}%
%  %      \zhtssp{咕}{g/-u}{to murmur}%
%  %      \zhtssp{咕}{g/-u}{to murmur}%
%  %      \zhtssp{咕}{g/-u}{to murmur}%
%  %      \zhtsX
%  %      \zhtssp{咕}{g/-u}{to murmur}%
%  %      \zhtssp{咕}{g/-u}{to murmur}%
%  %      \zhtssp{咕}{g/-u}{to murmur}%
%  %      \zhtsX
%  %      \zhtss{的}{de}{('s)}%
%  %      \zhtsd{唱}{歌}{ch/`ang}{g/-e}{to sing}{song}{to sing}%
%  %      \zhtsP%
%  %      \citep{baobao_rooster}{2}%
%\end{tbls}

%---------------------------------------
\begin{example}[\exmd{Onomatopoeias}]
\mbox{}\\
\label{ex:onomatopoeia}
%---------------------------------------
\begin{enumerate}
  \item \zhtssp{咕}{g/-u}{to murmur}\zhtssp{咕}{g/-u}{to murmur}\zhtssp{咕}{g/-u}{to murmur}\zhtsX
        \zhtsd{公}{雞}{g/-ong}{j/-i}{male}{chicken}{rooster}%
        \zhtss{叫}{ji/`ao}{to call}%
        \zhtss{著}{zhe}{(adverbial)}\zhtsC
        \zhtss{天}{ti/-an}{sky}%
        \zhtss{亮}{li/`ang}{bright}%
        \zhtss{了}{le}{(change of state)}\zhtsP%
        \citetbl{\citerp{dream}{27}}\\
        \engbox{``Cock-a-doodle-doo!", the rooster crowed, and the sky became bright again.}

  \item \zhtss{黃}{hu/'ang}{yellow}\zhtss{牛}{ni/'u}{cow}\zhtss{的}{de}{'s}%
        \zhtsd{肚}{子}{d/`u}{zi}{stomach}{son}{stomach}%
        \zhtss{餓}{/`e}{hungry}%
        \zhtss{得}{de}{to the extent of}%
        \zhtsd{咕}{嚕}{g/-u}{l/-u}{(a sound)}{(a sound)}{``gurgle"}%
        \zhtsd{咕}{嚕}{g/-u}{l/-u}{(a sound)}{(a sound)}{``gurgle"}%
        \zhtss{叫}{ji/`ao}{to call}\zhtsP%
        \citetbl{\citerp{baobao_cow}{22}}
        \engbox{Yellow Cow's stomach was so hungry that with a ``gurgle gurgle" it called out.}

  \item \zhtsd{哈}{啾}{h/-a}{ji/-u}{to breathe}{chirps}{achoo}\zhtsX
        \zhtsd{哈}{啾}{h/-a}{ji/-u}{to breathe}{chirps}{achoo}\zhtsX
        \zhtsd{彬}{彬}{b/-in}{b/-in}{intelligent}{intelligent}{Bin-Bin}%
        \zhtsd{好}{像}{h/>ao}{xi/`ang}{good}{to resemble}{to seem}%
        \zhtsd{感}{冒}{g/>an}{m/`ao}{to feel}{to emit}{to catch a cold}%
        \zhtssp{了}{le}{(change of state)}%
        \zhtsP%
        \citetbl{\citerp{achoo}{1}}
        \\\engbox{``Achoo! Achoo!" It seems Bin-Bin has caught a cold.}
  
  \item \zhtss {叩}{k/`ou}{to knock}\zhtsX
        \zhtssp{叩}{k/`ou}{to knock}\zhtsX
        \zhtssp{叩}{k/`ou}{to knock}\zhtsX
        \zhtss{這}{zh/`e}{this}%
        \zhtss{時}{/'shi}{time}%
        \zhtss{門}{m/'en}{door}%
        \zhtss{外}{w/`ai}{outside}%
        \zhtss{響}{xi/>ang}{sound}%
        \zhtss{起}{\v{q}i}{to start}%
        \zhtsd{敲}{門}{qi/-ao}{m/'en}{to knock}{door}{to knock at the door}%
        \zhtss{的}{de}{('s)}%
        \zhtsd{聲}{音}{sh/-eng}{yi/-n}{sound}{sound}{sound}%
        \zhtsP%
        \citetbl{\citerp{cloud_pillow}{23}}
        %\begin{tabular}{l}``Knock knock knock!" There arose\\the sound of knocking at the door.\end{tabular}
        \engbox{``Knock knock knock!" There arose the sound of knocking at the door.}

  \item \zhtss{他}{t/-a}{he}%
        \zhtssp{把}{b/>a}{}%
        \zhtss{頭}{t/'ou}{head}%
        \zhtss{靠}{k/`ao}{to lean}%
        \zhtss{在}{z/`ai}{at}%
        \zhtsd{枕}{頭}{zh/>en}{t/'ou}{pillow}{head}{pillow}%
        \zhtss{上}{sh/`ang}{on top of}\zhtsC
        \tcom{\zhtsd{一}{下}{/'yi}{xi/`a}{one}{next}{a short while}%
              \zhtss{子}{zi}{son}}{a moment}
        \zhtsC
        \zhtss{就}{ji/`u}{just}%
        \zhtssp{呼}{h/-u}{to exhale}\zhtssp{嚕}{l/-u}{verbose}%
        \zhtssp{呼}{h/-u}{to exhale}\zhtssp{嚕}{l/-u}{verbose}%
        \zhtsd{睡}{著}{sh/`ui}{zh/'ao}{to sleep}{a move}{to fall asleep}%
        \zhtssp{了}{le}{(change of state)}\zhtsP%
        \citetbl{\citerp{cloud_pillow}{27}}
        \engbox{He leaned his head against the pillow,\\and in a brief moment,\\with a ``ZZZZZ" just fell asleep.}
        %\engbox{He leaned his head against the pillow, and in a brief moment,\\with a "ZZZZZ" just fell asleep.}
  %\tblh The sun had come out, the sky was bright, and the little birds were singing ``chirp chirp chirp".
  %    & 太陽出來天亮了小鳥知知zha zha
  %      \citep{dream}{7}%
  %\tblh Every morning, all the people in the village can hear the rooster's
  %      ``Cock-a-doodle-doo! Cock-a-doodle-doo!" singing.
  %    & \zhtsd{每}{天}{m/>ei}{ti/-an}{every}{sky}{everyday}%
  %      \zhtsd{早}{上}{z/>ao}{sh/`ang}{morning}{above}{morning}%
  %      \zhtsd{村}{子}{c/-un}{zi}{village}{son}{village}%
  %      \zhtss{裡}{l/>i}{inside}%
  %      \zhtss{的}{de}{('s)}%
  %      \zhtss{人}{r/'en}{people}%
  %      \zhtss{都}{d/-ou}{all}%
  %      \zhtss{會}{h/`ui}{to be able}%
  %      \zhtsd{聽}{到}{ti/-ng}{d/`ao}{to listen}{to reach}{to hear}%
  %      %\zhtss{大}{d/`a}{big}%
  %      \zhtsd{公}{雞}{g/-ong}{j/-i}{male}{chicken}{rooster}%
  %      \zhtss{在}{z/`ai}{in the process of}%
  %      \zhtssp{咕}{g/-u}{to murmur}%
  %      \zhtssp{咕}{g/-u}{to murmur}%
  %      \zhtssp{咕}{g/-u}{to murmur}%
  %      \zhtsX
  %      \zhtssp{咕}{g/-u}{to murmur}%
  %      \zhtssp{咕}{g/-u}{to murmur}%
  %      \zhtssp{咕}{g/-u}{to murmur}%
  %      \zhtsX
  %      \zhtss{的}{de}{('s)}%
  %      \zhtsd{唱}{歌}{ch/`ang}{g/-e}{to sing}{song}{to sing}%
  %      \zhtsP%
  %      \citep{baobao_rooster}{2}%
\end{enumerate}
\end{example}

%=======================================
\section{Radicals}
%=======================================
\begin{minipage}{\tw-60mm}
Many Chinese characters contain a sub-component called a \hie{radical}.
Identifying this radical can be helpful in recalling the meaning of the character or to
locate the character within a Chinese dictionary.
Probably the most well known set of radicals is referred to as the
\hie{Kangxi radicals}, and are presented in the following table.\footnotemark
\end{minipage}\hfill%
\ftbox{
  \zhtsd{康}{熙}{k/-ang}{x/-i}{health}{sunny}{Kangxi}
  \zhtsd{部}{首}{b/`u}{sh/>ou}{part}{head}{radical}
  }
\citetblt{%
  \citeru{unicode510}{http://www.unicode.org/charts/PDF/U2F00.pdf} \\
  \citer{mcnaughton1999}
  }

%\begin{longtable}{|>{\scriptsize}r>{\fntwqyzenhei}lll|@{\qquad}|>{\scriptsize}r>{\fntwqyzenhei}lll|}
%\begin{longtable}{|>{\scriptsize}r>{\fntFirefly}lll|@{\qquad}|>{\scriptsize}r>{\fntFirefly}lll|}
\begin{longtable}{|>{\scriptsize}r>{\fntzht}lll|@{\qquad}|>{\scriptsize}r>{\fntzht}lll|}
%\hline
%  \mc{8}{>{\rowcolor[rgb]{0,0,1}\bfseries}c}{\textcolor{white}{1 stroke}}
%  \mc{8}{|c|}{\cellcolor[rgb]{0,0,1}\bfseries\color{white}1 stroke}
  \mc{8}{|B|}{1 stroke}\\
\cnto & \symbol{"2F00} & /-yi     & \hie{one}
\cntc & \symbol{"2F01} & sh/`u    & \hie{line}   % g/>un
\cntn & \symbol{"2F02} & zh/>u    & \hie{dot}
\cntc & \symbol{"2F03} & pi/>e    & \hie{slash}
\cntn & \symbol{"2F04} & \v{y}i   & \hie{second}
\cntc & \symbol{"2F05} & j/'ue    & \hie{hook}
\\\mc{8}{|B|}{2 strokes}
\cntn & \symbol{"2F06} & /`er     & \hie{two}
\cntc & \symbol{"2F07} & t/'ou    & \hie{lid}
\cntn & \symbol{"2F08} & r/'en    & \hie{man}
\cntc & \symbol{"2F09} & /'er     & \hie{legs}
\cntn & \symbol{"2F0A} & r/`u     & \hie{enter}
\cntc & \symbol{"2F0B} & b/-a     & \hie{eight}
\cntn & \symbol{"2F0C} & ji/>ong  & \hie{down box}
\cntc & \symbol{"2F0D} & m/`i     & \hie{cover}
\cntn & \symbol{"2F0E} & b/-ing   & \hie{ice}
\cntc & \symbol{"2F0F} & j/-i     & \hie{table}
\cntn & \symbol{"2F10} & k/>an    & \hie{open box}
\cntc & \symbol{"2F11} & d/-ao    & \hie{knife}
\cntn & \symbol{"2F12} & l/`i     & \hie{power}
\cntc & \symbol{"2F13} & b/-ao    & \hie{wrap}
\cntn & \symbol{"2F14} & b/>i   & \hie{spoon}
\cntc & \symbol{"2F15} & f/-ang   & \hie{right open box}
\cntn & \symbol{"2F16} & x/`i     & \hie{hiding enclosure} % \v{x}i
\cntc & \symbol{"2F17} & sh/'i    & \hie{ten} 
\cntn & \symbol{"2F18} & b/>u     & \hie{divination}
\cntc & \symbol{"2F19} & ji/'e    & \hie{seal}
\cntn & \symbol{"2F1A} & h/>an    & \hie{cliff}
\cntc & \symbol{"2F1B} & s/-i     & \hie{private}
\cntn & \symbol{"2F1C} & y/`ou    & \hie{again}
    &&&&
\\\mc{8}{|B|}{3 strokes}
\cntn & \symbol{"2F1D} & k/>ou    & \hie{mouth}
\cntc & \symbol{"2F1E} & w/'ei    & \hie{enclosure}
\cntn & \symbol{"2F1F} & t/>u     & \hie{earth}
\cntc & \symbol{"2F20} & sh/`i    & \hie{scholar}
\cntn & \symbol{"2F21} & zh/>i    & \hie{go}
\cntc & \symbol{"2F22} & s/-ui    & \hie{go slowly}
\cntn & \symbol{"2F23} & x/-i     & \hie{evening}
\cntc & \symbol{"2F24} & d/`a     & \hie{big}
\cntn & \symbol{"2F25} & n/>:u    & \hie{woman}
\cntc & \symbol{"2F26} & z/>i     & \hie{child}
\cntn & \symbol{"2F27} & mi/'an   & \hie{roof}
\cntc & \symbol{"2F28} & c/`un    & \hie{inch}
\cntn & \symbol{"2F29} & xi/>ao   & \hie{small}
\cntc & \symbol{"2F2A} & w/-ang   & \hie{lame}
\cntn & \symbol{"2F2B} & sh/-i    & \hie{corpse}
\cntc & \symbol{"2F2C} & ch/`e    & \hie{sprout}
\cntn & \symbol{"2F2D} & sh/-an   & \hie{mountain}
\cntc & \symbol{"2F2E} & ch/-uan  & \hie{river}
\cntn & \symbol{"2F2F} & g/-ong   & \hie{work}
\cntc & \symbol{"2F30} & j/>i     & \hie{oneself}
\cntn & \symbol{"2F31} & ji/-n    & \hie{turban}
\cntc & \symbol{"2F32} & g/-an    & \hie{dry}
\cntn & \symbol{"2F33} & y/-ao    & \hie{short thread}
\cntc & \symbol{"2F34} & y/>an    & \hie{dotted cliff}
\cntn & \symbol{"2F35} & y/>in  & \hie{long stride}
\cntc & \symbol{"2F36} & g/>ong & \hie{two hands}
\cntn & \symbol{"2F37} & y/`i     & \hie{shoot}
\cntc & \symbol{"2F38} & g/-ong   & \hie{bow}
\cntn & \symbol{"2F39} & j/`i     & \hie{snout}
\cntc & \symbol{"2F3A} & sh/-an   & \hie{bristle}
\cntn & \symbol{"2F3B} & ch/`i    & \hie{step}
    &&&&
\\\mc{8}{|B|}{4 strokes}
\cntn & \symbol{"2F3C} & x/-in    & \hie{heart}
\cntc & \symbol{"2F3D} & g/-e     & \hie{halberd}
\cntn & \symbol{"2F3E} & h/`u     & \hie{door}
\cntc & \symbol{"2F3F} & sh/>ou & \hie{hand}
\cntn & \symbol{"2F40} & zh/-i    & \hie{branch}
\cntc & \symbol{"2F41} & p/>u   & \hie{wrap}
\cntn & \symbol{"2F42} & w/'en    & \hie{script}
\cntc & \symbol{"2F43} & d/>ou  & \hie{dipper}
\cntn & \symbol{"2F44} & j/-in    & \hie{axe}
\cntc & \symbol{"2F45} & f/-ang   & \hie{square}
\cntn & \symbol{"2F46} & w/'u     & \hie{not}
\cntc & \symbol{"2F47} & r/`i     & \hie{sun}
\cntn & \symbol{"2F48} & y/-ue    & \hie{say}
\cntc & \symbol{"2F49} & y/`ue    & \hie{moon}
\cntn & \symbol{"2F4A} & m/`u     & \hie{tree}
\cntc & \symbol{"2F4B} & qi/`an   & \hie{lack}
\cntn & \symbol{"2F4C} & zh/>i  & \hie{stop}
\cntc & \symbol{"2F4D} & d/>ai  & \hie{death}
\cntn & \symbol{"2F4E} & sh/-u    & \hie{weapon}
\cntc & \symbol{"2F4F} & w/'u     & \hie{do not}
\cntn & \symbol{"2F50} & b/>i   & \hie{compare}
\cntc & \symbol{"2F51} & m/'ao    & \hie{fur}
\cntn & \symbol{"2F52} & sh/`i    & \hie{clan}
\cntc & \symbol{"2F53} & q/`i     & \hie{steam}
\cntn & \symbol{"2F54} & sh/>ui & \hie{water}
\cntc & \symbol{"2F55} & h/>uo  & \hie{fire}
\cntn & \symbol{"2F56} & zh/>ao & \hie{claw}    % zh/>uazhao3
\cntc & \symbol{"2F57} & f/`u     & \hie{father}
\cntn & \symbol{"2F58} & y/'ao    & \hie{double x}
\cntc & \symbol{"2F59} & qi/-ang  & \hie{half tree trunk}
\cntn & \symbol{"2F5A} & pi/`an   & \hie{slice}
\cntc & \symbol{"2F5B} & y/'a     & \hie{fang}
\cntn & \symbol{"2F5C} & ni/'u    & \hie{cow}
\cntc & \symbol{"2F5D} & q/>uan & \hie{dog}
\\\mc{8}{|B|}{5 strokes}
\cntn & \symbol{"2F5E} & x/'uan   & \hie{profound}
\cntc & \symbol{"2F5F} & y/`u     & \hie{jade}
\cntn & \symbol{"2F60} & g/-ua    & \hie{melon}
\cntc & \symbol{"2F61} & w/>a     & \hie{tile}
\cntn & \symbol{"2F62} & g/-an    & \hie{sweet}
\cntc & \symbol{"2F63} & sh/-eng  & \hie{life}
\cntn & \symbol{"2F64} & y/`ong   & \hie{use}
\cntc & \symbol{"2F65} & ti/'an   & \hie{field}
\cntn & \symbol{"2F66} & p/>i     & \hie{bolt of cloth}
\cntc & \symbol{"2F67} & n/`i     & \hie{sickness}  %ch/'uang
\cntn & \symbol{"2F68} & b/`o     & \hie{dotted tent}
\cntc & \symbol{"2F69} & b/'ai    & \hie{white}
\cntn & \symbol{"2F56} & zh/>ao   & \hie{claw}    % zh/>ua
\cntc & \symbol{"2F6B} & m/>in    & \hie{dish}
\cntn & \symbol{"2F6C} & m/`u     & \hie{eye}
\cntc & \symbol{"2F6D} & m/'ao    & \hie{spear}
\cntn & \symbol{"2F6E} & sh/>i  & \hie{arrow}
\cntc & \symbol{"2F6F} & sh/'i    & \hie{stone}
\cntn & \symbol{"2F70} & sh/`i    & \hie{spirit}
\cntc & \symbol{"2F71} & r/>ou  & \hie{track}
\cntn & \symbol{"2F72} & h/'e     & \hie{grain}
\cntc & \symbol{"2F73} & x/'ue    & \hie{cave}
\cntn & \symbol{"2F74} & l/`i     & \hie{stand}
    &&&&
\\\mc{8}{|B|}{6 strokes}
\cntn & \symbol{"2F75} & zh/'u    & \hie{bamboo}
\cntc & \symbol{"2F76} & m/>i     & \hie{rice}
\cntn & \symbol{"2F77} & s/-i     & \hie{silk}  %/`mi
\cntc & \symbol{"2F78} & f/>ou    & \hie{jar}
\cntn & \symbol{"2F79} & w/>ang   & \hie{net}
\cntc & \symbol{"2F7A} & y/'ang   & \hie{sheep}
\cntn & \symbol{"2F7B} & y/>u     & \hie{feather}
\cntc & \symbol{"2F7C} & l/>ao    & \hie{old}
\cntn & \symbol{"2F7D} & /'er     & \hie{and}
\cntc & \symbol{"2F7E} & l/>ei    & \hie{plow}
\cntn & \symbol{"2F7F} & />er     & \hie{ear}
\cntc & \symbol{"2F80} & y/`u     & \hie{brush}
\cntn & \symbol{"2F81} & r/`ou    & \hie{meat}
\cntc & \symbol{"2F82} & ch/'en   & \hie{minister}
\cntn & \symbol{"2F83} & z/`i     & \hie{self}
\cntc & \symbol{"2F84} & zh/`i    & \hie{arrive}
\cntn & \symbol{"2F85} & ji/`u    & \hie{mortar}
\cntc & \symbol{"2F86} & sh/'e    & \hie{tongue}
\cntn & \symbol{"2F87} & ch/>uan& \hie{oppose}
\cntc & \symbol{"2F88} & zh/-ou   & \hie{boat}
\cntn & \symbol{"2F89} & g/`en    & \hie{stopping}
\cntc & \symbol{"2F8A} & s/`e     & \hie{color}
\cntn & \symbol{"2F8B} & c/>ao    & \hie{grass}
\cntc & \symbol{"2F8C} & h/-u     & \hie{tiger}
\cntn & \symbol{"2F8D} & ch/'ong  & \hie{insect}
\cntc & \symbol{"2F8E} & x/`ue    & \hie{blood}
\cntn & \symbol{"2F8F} & x/'ing   & \hie{walk enclosure}
\cntc & \symbol{"2F90} & y/-i     & \hie{clothes}
\cntn & \symbol{"2F91} & xi/`a    & \hie{west}
    &&&&
\\\mc{8}{|B|}{7 strokes}
\cntn & \symbol{"2F92} & ji/`an   & \hie{see}
\cntc & \symbol{"2F93} & ji/>ao   & \hie{horn}
\cntn & \symbol{"2F94} & y/'an    & \hie{speech}
\cntc & \symbol{"2F95} & g/>u     & \hie{valley}
\cntn & \symbol{"2F96} & d/`ou    & \hie{bean}
\cntc & \symbol{"2F97} & sh/>i    & \hie{pig}
\cntn & \symbol{"2F98} & zh/`i    & \hie{badger}
\cntc & \symbol{"2F99} & b/`ei    & \hie{shell}
\cntn & \symbol{"2F9A} & ch/`i    & \hie{red}
\cntc & \symbol{"2F9B} & z/>ou    & \hie{run}
\cntn & \symbol{"2F9C} & z/'u     & \hie{foot}
\cntc & \symbol{"2F9D} & sh/-en   & \hie{body}
\cntn & \symbol{"2F9E} & ch/-e    & \hie{cart}
\cntc & \symbol{"2F9F} & x/-in    & \hie{bitter}
\cntn & \symbol{"2FA0} & ch/'en   & \hie{morning}
\cntc & \symbol{"2FA1} & ch/`uo   & \hie{walk}
\cntn & \symbol{"2FA2} & y/`i     & \hie{city}
\cntc & \symbol{"2FA3} & y/>ou    & \hie{wine}
\cntn & \symbol{"2FA4} & bi/`an   & \hie{distinguish}
\cntc & \symbol{"2FA5} & l/>i     & \hie{village}
\\\mc{8}{|B|}{8 strokes}
\cntn & \symbol{"2FA6} & j/-in    & \hie{gold}
\cntc & \symbol{"2FA7} & ch/'ang  & \hie{long}
\cntn & \symbol{"2FA8} & m/'en    & \hie{gate}
\cntc & \symbol{"2FA9} & f/`u     & \hie{mound}
\cntn & \symbol{"2FAA} & d/`ai    & \hie{slave}
\cntc & \symbol{"2FAB} & zh/-ui   & \hie{short tailed bird}
\cntn & \symbol{"2FAC} & y/>u     & \hie{rain}
\cntc & \symbol{"2FAD} & q/-ing   & \hie{blue}
\cntn & \symbol{"2FAE} & f/-ei    & \hie{wrong}
    &&&&
\\\mc{8}{|B|}{9 strokes}
\cntn & \symbol{"2FAF} & mi/`an   & \hie{face}
\cntc & \symbol{"2FB0} & g/'e     & \hie{leather}
\cntn & \symbol{"2FB1} & w/'ei    & \hie{tanned leather}
\cntc & \symbol{"2FB2} & ji/>u    & \hie{leek}
\cntn & \symbol{"2FB3} & y/-in    & \hie{sound}
\cntc & \symbol{"2FB4} & y/`e     & \hie{leaf}
\cntn & \symbol{"2FB5} & f/-eng   & \hie{wind}
\cntc & \symbol{"2FB6} & f/-ei    & \hie{fly}
\cntn & \symbol{"2FB7} & sh/'i    & \hie{eat}
\cntc & \symbol{"2FB8} & sh/>ou   & \hie{head}
\cntn & \symbol{"2FB9} & xi/-ang  & \hie{fragrant}
    &&&&
\\\mc{8}{|B|}{10 strokes}
\cntn & \symbol{"2FBA} & m/>a     & \hie{horse}
\cntc & \symbol{"2FBB} & g/>u     & \hie{bone}
\cntn & \symbol{"2FBC} & g/-ao    & \hie{tall}
\cntc & \symbol{"2FBD} & bi/-ao   & \hie{hair}
\cntn & \symbol{"2FBE} & d/`ou    & \hie{fight}
\cntc & \symbol{"2FBF} & ch/`ang  & \hie{sacrificial wine}
\cntn & \symbol{"2FC0} & l/`i     & \hie{cauldron}
\cntc & \symbol{"2FC1} & g/>ui    & \hie{ghost}
\\\mc{8}{|B|}{11 strokes}
\cntn & \symbol{"2FC2} & y/'u     & \hie{fish}
\cntc & \symbol{"2FC3} & ni/>ao   & \hie{bird}
\cntn & \symbol{"2FC4} & l/>u     & \hie{salt}
\cntc & \symbol{"2FC5} & l/`u     & \hie{deer}
\cntn & \symbol{"2FC6} & m/`ai    & \hie{wheat}
\cntc & \symbol{"2FC7} & m/'a     & \hie{hemp}
\\\mc{8}{|B|}{12 strokes}
\cntn & \symbol{"2FC8} & hu/'ang  & \hie{yellow}
\cntc & \symbol{"2FC9} & sh/>u    & \hie{millet}
\cntn & \symbol{"2FCA} & h/-ei    & \hie{black}
\cntc & \symbol{"2FCB} & zh/>i    & \hie{embroidery}
\\\mc{8}{|B|}{13 strokes}
\cntn & \symbol{"2FCC} & m/>in    & \hie{frog}
\cntc & \symbol{"2FCD} & d/>ing   & \hie{tripod}
\cntn & \symbol{"2FCE} & g/>u     & \hie{drum}
\cntc & \symbol{"2FCF} & sh/>u    & \hie{rat}
\\\mc{8}{|B|}{14 strokes}
\cntn & \symbol{"2FD0} & b/'i     & \hie{nose}
\cntc & \symbol{"2FD1} & q/'i     & \hie{even}
\\\mc{8}{|B|}{15 strokes}
\cntn & \symbol{"2FD2} & ch/>i    & \hie{tooth}
    &&&&             
\\\mc{8}{|B|}{16 strokes}
\cntn & \symbol{"2FD3} & l/'ong   & \hie{dragon}
\cntc & \symbol{"2FD4} & g/-ui    & \hie{tortoise}
\\\mc{8}{|B|}{17 strokes}
\cntn & \symbol{"2FD5} & y/`ue    & \hie{flute}
    &&&&
\\\hline
\end{longtable}


%\newpage
%
%\begin{tabular}{llll}
%  \verb.\zht{}.    &=& \zht{王漢宗中明體繁} & ``HanWangMingMedium" \\
%  \verb.\zhtwqy{}. &=& \zht{文泉驛正黑}     & ``WenQuanYi Zen Hei"
%\end{tabular}
%
%\begin{footnotesize}
%\begin{verbatim}
%\begin{tabular}{r|ccc}
%  num & keyboard &   2F00-2FDF block   & 4E00+ block   \\
%  \hline
%   1. & \zht{一} & \zhtwqy{\zhts"2F00} & \zht{\zhts"4E00}  \\
%   2. &          & \zhtwqy{\zhts"2F01} & \zht{\zhts"4E28}  \\
%   3. &          & \zhtwqy{\zhts"2F02} & \zht{\zhts"4E36}  \\
%   4. &          & \zhtwqy{\zhts"2F03} & \zht{\zhts"4E3F}  \\
%   5. & \zht{乙} & \zhtwqy{\zhts"2F04} & \zht{\zhts"4E59}  \\
%   6. &          & \zhtwqy{\zhts"2F05} & \zht{\zhts"4E85}  \\
%   7. & \zht{二} & \zhtwqy{\zhts"2F06} & \zht{\zhts"4E8C}  \\
%   8. &          & \zhtwqy{\zhts"2F07} & \zht{\zhts"4EA0}  \\
%   9. & \zht{人} & \zhtwqy{\zhts"2F08} & \zht{\zhts"4EBA}  \\
%  10. & \zht{儿} & \zhtwqy{\zhts"2F09} & \zht{\zhts"513F}  \\
%  11. & \zht{入} & \zhtwqy{\zhts"2F0A} & \zht{\zhts"5165}  \\
%  12. & \zht{八} & \zhtwqy{\zhts"2F0B} & \zht{\zhts"516B}  \\
%  13. &          & \zhtwqy{\zhts"2F0C} & \zht{\zhts"5182}  \\
%  14. &          & \zhtwqy{\zhts"2F0D} & \zht{\zhts"5196}  \\
%  15. &          & \zhtwqy{\zhts"2F0E} & \zht{\zhts"51AB}  \\
%  16. & \zht{几} & \zhtwqy{\zhts"2F0F} & \zht{\zhts"51E0}
%\end{tabular}
%\end{verbatim}
%\end{footnotesize}
%
%\begin{tabular}{r|ccc}
%  num & keyboard &   2F00-2FDF block   & 4E00+ block   \\
%  \hline
%   1. & \zht{一} & \zhtwqy{\zhts"2F00} & \zht{\zhts"4E00}  \\
%   2. &          & \zhtwqy{\zhts"2F01} & \zht{\zhts"4E28}  \\
%   3. &          & \zhtwqy{\zhts"2F02} & \zht{\zhts"4E36}  \\
%   4. &          & \zhtwqy{\zhts"2F03} & \zht{\zhts"4E3F}  \\
%   5. & \zht{乙} & \zhtwqy{\zhts"2F04} & \zht{\zhts"4E59}  \\
%   6. &          & \zhtwqy{\zhts"2F05} & \zht{\zhts"4E85}  \\
%   7. & \zht{二} & \zhtwqy{\zhts"2F06} & \zht{\zhts"4E8C}  \\
%   8. &          & \zhtwqy{\zhts"2F07} & \zht{\zhts"4EA0}  \\
%   9. & \zht{人} & \zhtwqy{\zhts"2F08} & \zht{\zhts"4EBA}  \\
%  10. & \zht{儿} & \zhtwqy{\zhts"2F09} & \zht{\zhts"513F}  \\
%  11. & \zht{入} & \zhtwqy{\zhts"2F0A} & \zht{\zhts"5165}  \\
%  12. & \zht{八} & \zhtwqy{\zhts"2F0B} & \zht{\zhts"516B}  \\
%  13. &          & \zhtwqy{\zhts"2F0C} & \zht{\zhts"5182}  \\
%  14. &          & \zhtwqy{\zhts"2F0D} & \zht{\zhts"5196}  \\
%  15. &          & \zhtwqy{\zhts"2F0E} & \zht{\zhts"51AB}  \\
%  16. & \zht{几} & \zhtwqy{\zhts"2F0F} & \zht{\zhts"51E0}
%\end{tabular}
%

%\setcounter{cntr}{1}
%\begin{longtable}{r<{.}lll}
%    1 & \zhtwqy{\zhts"2F02} & zh/>u  & \hie{dot}
%\\  2 & \zht{一} & /-yi     & \hie{one}
%\\  3 &          & sh/`u    & \hie{down}
%\\  4 &          & pi/>e  & \hie{left}
%\\  5 &          &          & \hie{back-turned stroke}
%\\  6 &          &          & \hie{top of \zht{刁}}
%\\  7 & \zht{乙} & \v{y}i   & \hie{twist}
%\\  8 &          & bi/-ng   & \hie{ice}
%\\  9 &          & t/'ou    & \hie{lid}
%\\ 10 &
%\\ 11 & \zht{二} & e/`r     & \hie{two}
%\\ 12 & \zht{十} & /'shi    & \hie{ten} 
%\\ 13 & \zht{厂} & h/>an  & \hie{slope}
%\\ 14 &          & z/>uo  & \hie{left}
%\\ 15 & \zht{匚} & f/-ang   & \hie{basket}
%\\ 16 & \zht{卜} & b/>u   & \hie{divine}
%\\ 17 &          & d/-ao    & \hie{standing side knife}
%\\ 18 &          & /`mi     & \hie{crown}
%\\ 19 &          & ji/>ong& \hie{borders}
%\\ 20 &          &          & \hie{top of \zht{每}}
%\\ 21 &          & r/'en    & \hie{side-man}
%\\ 22 &          &          & \hie{top of \zht{盾}}
%\\ 23 & \zht{人} & r/'en    & \hie{man}
%\\ 24 & \zht{八} & b/-a     & \hie{eight}
%\\ 25 &          &          & \hie{bottom of yi}
%\\ 26 &          & b/-ao    & \hie{wrap}
%\\ 27 & \zht{刀} & d/-ao    & \hie{knife}
%\\ 28 & \zht{力} & l/`i     & \hie{strength}
%\\ 29 & \zht{儿} & r/'en    & \hie{legs}
%\\ 30 & \zht{几} & j/-i     & \hie{table}
%\\ 31 &          &          & \hie{top of \zht{子}}
%\\ 32 &          & ji/'e    & \hie{seal}
%\\ 33 &          & f/`u     & \hie{mound}
%\\ 34 &          & y/`i     & \hie{city}
%\\ 35 & \zht{又} & y/`ou    & \hie{right hand}
%\\ 36 &          & \v{y}in  & \hie{march}
%\\ 37 &          & s/-i     & \hie{cocoon}
%\\ 38 & \zht{凵} & k/>an  & \hie{bowl}
%\\ 39 & \zht{匕} & \v{y}i   & \hie{ladle}
%\\ 40 &          & sh/>ui & \hie{water}
%\\ 41 &          & xi/-n    & \hie{vertical heart side}
%\\ 42 & \zht{田 \zhts"2F65} & ti/'an   & \hie{field}
%\\ 43 & \zht{亡} & w/'ang   & \hie{die}
%\\ 44 &          & y/>an  & \hie{lean-to}
%\\ 45 &          & mi/'an   & \hie{roof}
%\\ 46 & \zht{門} & m/'en    & \hie{gate}
%\\ 47 &          & ch/`uo   & \hie{halt}
%\\ 48 & \zht{工} & g/-ong   & \hie{work}
%\\ 49 & \zht{土} & t/>u   & \hie{earth}
%\\ 50 &          & c/>ao  & \hie{grass}
%\\ 51 & \zht{廾} & g/>ong & \hie{clasp}
%\\ 52 & \zht{大} & d/`a     & \hie{big}
%\\ 53 & \zht{尢} & w/-ang   & \hie{lame}
%\\ 54 & \zht{寸} & c/`un    & \hie{thumb}
%\\ 55 &          & sh/>ou & \hie{side-hand}
%\\ 56 & \zht{弋} & y/`i     & \hie{dart}
%\\ 57 & \zht{巾} & ji/-n    & \hie{cloth}
%\\ 58 & \zht{口} & k/>ou  & \hie{mouth}
%\\ 59 & \zht{囗} & w/'ei    & \hie{surround}
%\\ 60 & \zht{山} & sh/-an   & \hie{mountain}
%\\ 61 & \zht{屮} & ch/`e    & \hie{sprout}
%\\ 62 & \zht{彳} & /`chi    & \hie{step}
%\\ 63 &          & s/-ui    & \hie{slow}
%\\ 64 &          & /-xi     & \hie{dusk}
%\\ 65 &          & \v{z}hi  & \hie{follow}
%\\ 66 & \zht{丸} & w/'an    & \hie{bullet}
%\\ 67 & \zht{尸} & h/-i    & \hie{corpse}
%\\ 68 &          & /'shi    & \hie{side-food}
%\\ 69 &          &          & \hie{side-dog}
%\\ 70 &          & j/`i     & \hie{pig's head}
%\\ 71 & \zht{弓} & g/-ong   & \hie{bow}
%\\ 72 & \zht{己} & j/>i   & \hie{self}
%\\ 73 & \zht{女} & n/>:u   & \hie{woman}
%\\ 74 & \zht{子} & \v{z}i   & \hie{child}
%\\ 75 & \zht{馬} & m/>a   & \hie{horse}
%\\ 76 &          & y/-ao    & \hie{coil}
%\\ 77 & \zht{糸} & s/-i     & \hie{silk}
%\\ 78 & \zht{川} & ch/-uan  & \hie{river}
%\\ 79 & \zht{小} & xi/>ao & \hie{small}
%\\ 80 &          &
%\\ 81 & \zht{心} & xi/-n    & \hie{heart}
%\\ 82 & \zht{斗} & d/>ou  & \hie{peck}
%\\ 83 & \zht{火} & h/>uo  & \hie{fire}
%\\ 84 & \zht{文} & w/'en    & \hie{pattern}
%\\ 85 & \zht{方} & f/-ang   & \hie{square}
%\\ 86 & \zht{戶} & h/`u     & \hie{door}
%\\ 87 &          & /`shi    & \hie{sign}
%\\ 88 & \zht{王} & w/'ang   & \hie{king}
%\\ 89 &                     & \hie{top of \zht{青}}
%\\ 90 & \zht{夭} & y/-ao    & \hie{gentle}
%\\ 91 & \zht{韋} & w/'ei    & \hie{walk off}
%\\ 92 &          &          & \hie{top of \zht{老}}
%\\ 93 & \zht{廿} & ni/`an   & \hie{twenty}
%\\ 94 & \zht{木} & m/`u     & \hie{tree}
%\\ 95 & \zht{不} & b/`u     & \hie{not}
%\\ 96 & \zht{犬} & q/>uan & \hie{dog}
%\\ 97 & \zht{歹} & d/>ai  & \hie{chip}
%\\ 98 & \zht{瓦} & w/>a   & \hie{tile}
%\\ 99 & \zht{牙} & y/'a     & \hie{tooth}
%\\100 & \zht{車} & ch/-e    & \hie{car}     % need "short form"
%\\101 & \zht{戈} & g/-e     & \hie{lance}
%\\102 & \zht{止} & \v{z}hi  & \hie{toe}
%\\103 & \zht{日} & /`ri     & \hie{sun}
%\\104 & \zht{曰} & y/-ue    & \hie{say}
%\\105 & \zht{中} & zh/-ong  & \hie{middle}
%\\106 & \zht{貝} & b/`ei    & \hie{cowrie}
%\\107 & \zht{見} & ji/`an   & \hie{see}
%\\108 & \zht{父} & f/`u     & \hie{father}
%\\109 &          & /`qi     & \hie{breath}
%\\110 & \zht{牛} & ni/'u    & \hie{cow}
%\\111 & \zht{手} & sh/>ou & \hie{hand}
%\\112 & \zht{毛} & m/'ao    & \hie{fur}
%\\113 &          & p/>u   & \hie{knock}
%\\114 & \zht{片} & pi/`an   & \hie{slice}
%\\115 & \zht{斤} & ji/-n    & \hie{axe}
%\\116 &          & zh/>ao & \hie{claws}
%\\117 & \zht{尺} & \v{c}hi  & \hie{foot}
%\\118 & \zht{月} & y/`ue    & \hie{moon}
%\\119 & \zht{殳} & sh/-u    & \hie{club}
%\\120 & \zht{欠} & qi/`an   & \hie{yawn}
%\\121 & \zht{風} & f/-eng   & \hie{wind}
%\\122 & \zht{式} & /`shi    & \hie{clan}
%\\123 & \zht{比} & b/>i   & \hie{compare}
%\\124 &          &          & \hie{top of \zht{聿}}
%\\125 & \zht{水} & sh/>ui & \hie{water}
%\\126 & \zht{立} & l/`i     & \hie{stand}
%\\127 &          & /`ni     & \hie{sick}
%\\128 &          & x/'ue    & \hie{cave}
%\\129 &          & /-yi     & \hie{gown}
%\\130 &          &          & \hie{top of \zht{春}}
%\\131 & \zht{玉} & y/`u     & \hie{jade}
%\\132 & \zht{示} & /`shi    & \hie{sign}
%\\133 & \zht{去} & q/`u     & \hie{go}
%\\134 &          &          & \hie{top of \zht{勞}} % ???
%\\135 & \zht{甘} & g/-an    & \hie{sweet}
%\\136 & \zht{石} & /'shi    & \hie{rock}
%\\137 & \zht{龍} & l/'ong   & \hie{dragon}  % short form
%\\138 & \zht{戊} & w/`u     & \hie{5th heavenly stem}
%\\139 &          &          & \hie{top of ...}
%\\141 & \zht{目} & m/`u     & \hie{eye}
%\\142 & \zht{田} & ti/'an   & \hie{field}
%\\143 & \zht{血} & x/`ue    & \hie{blood}
%\\144 & \zht{申} & sh/-en   & \hie{stretch}
%\\145 &          & w/>ang & \hie{net}
%\\146 & \zht{皿} & mi\v{n}  & \hie{dish}
%\\147 &          & ji/-n    & \hie{side-gold}
%\\148 & \zht{矢} & sh/>i  & \hie{arrow}
%\\149 & \zht{禾} & h/'e     & \hie{grain}
%\\150 & \zht{白} & b/'ai    & \hie{white}
%\\151 & \zht{瓜} & g/-ua    & \hie{melon}
%\\152 & \zht{鳥} & ni/>ao & \hie{bird}
%\\153 & \zht{皮} & /'pi     & \hie{skin}
%\\154 &          & b/`o     & \hie{back}
%\\155 & \zht{矛} & m/'ao    & \hie{spear}
%\\156 & \zht{疋} & \v{p}i   & \hie{bolt}
%\\157 & \zht{羊} & y/'ang   & \hie{sheep}
%\\158 &          & j/>uan & \hie{roll}
%\\159 & \zht{米} & \v{m}i   & \hie{rice}
%\\160 & \zht{齊} & /'qi     & \hie{line-up}
%\\161 & \zht{衣} & /-yi     & \hie{gown}
%\\162 &          &          & \hie{also}
%\\163 & \zht{耳} & />er   & \hie{ear}
%\\164 & \zht{臣} & ch/'en   & \hie{bureaucrat}
%\\165 &          &          & \hie{top of \zht{栽}}
%\\166 & \zht{襾} & xi/`a    & \hie{cover}
%\\167 & \zht{朿} & /`ci     & \hie{thorn}
%\\168 &          & y/-a     & \hie{inferior}
%\\169 & \zht{而} & e/'r     & \hie{beard}
%\\170 & \zht{頁} & y/`e     & \hie{head}
%\\171 & \zht{至} & /`zhi    & \hie{reach}
%\\172 & \zht{光} & gu/-ang  & \hie{light}
%\\173 &          & h/-u     & \hie{tiger}
%\\174 & \zht{虫} & ch/'ong  & \hie{bug}
%\\175 & \zht{缶} & f/>ou  & \hie{crock}
%\\176 & \zht{耒} & l/>i   & \hie{plow}
%\\177 & \zht{舌} & sh/'e    & \hie{tongue}
%\\178 & \zht{竹} & zh/'u    & \hie{bamboo}
%\\179 & \zht{臼} & ji/`u    & \hie{mortar}
%\\180 & \zht{自} & /`zi     & \hie{nose}
%\\181 & \zht{血} & x/`ue    & \hie{blood}
%\\182 & \zht{舟} & zh/-ou   & \hie{boat}
%\\183 & \zht{羽} & y/>u   & \hie{wings}
%\\184 & \zht{艮} & g/`en    & \hie{stubborn}
%\\185 & \zht{言} & y/'an    & \hie{words}
%\\186 & \zht{辛} & xi/-n    & \hie{bitter}
%\\187 & \zht{辰} & ch/'en   & \hie{early}
%\\188 & \zht{麥} & m/`ai    & \hie{wheat}
%\\189 & \zht{走} & z/>ou  & \hie{walk}
%\\190 & \zht{赤} & /`chi    & \hie{red}
%\\191 & \zht{豆} & d/`ou    & \hie{flask}
%\\192 & \zht{束} & sh/`u    & \hie{bundle}
%\\193 & \zht{酉} & y/>ou  & \hie{wine}
%\\194 & \zht{豕} & sh/>i  & \hie{pig}
%\\195 & \zht{里} & l/>i   & \hie{village}
%\\196 & \zht{足} & z/'u     & \hie{foot}
%\\197 & \zht{釆} & bi/`an   & \hie{sift}
%\\198 & \zht{豸} & /`zhi    & \hie{snake}
%\\199 & \zht{谷} & g/>u   & \hie{valley}
%\\200 & \zht{身} & sh/-en   & \hie{torso}
%\\201 & \zht{角} & ji/>ao & \hie{horn}
%\\202 & \zht{青} & qi/-ng   & \hie{green}
%\\203 &          &          & \hie{side of \zht{朝}}
%\\204 & \zht{雨} & y/>u   & \hie{rain}
%\\205 & \zht{非} & f/-ei    & \hie{wrong}
%\\206 & \zht{齒} & \v{c}hi  & \hie{teeth}
%\\207 & \zht{黽} & mi\v{n}  & \hie{toad}
%\\208 & \zht{隹} & zh/-ui   & \hie{dove}
%\\209 & \zht{金} & ji/-n    & \hie{gold}
%\\210 & \zht{魚} & y/'u     & \hie{fish}
%\\211 & \zht{音} & yi/-n    & \hie{tone}
%\\212 &          & g/'e     & \hie{hide}
%\\213 & \zht{是} & /`shi    &
%\\214 &          &          &
%\\215 & \zht{香} & xi/-ang  & \hie{scent}
%\\216 & \zht{鬼} & g/>ui  & \hie{ghost}
%\\217 & \zht{食} &
%\\218 & \zht{高} & g/-ao    & \hie{tall}
%\\219 & \zht{骨} & g/>u   & \hie{bone}
%\\220 & \zht{髟} & bi/-ao   & \hie{hair}
%\\221 & \zht{麻} & m/'a     & \hie{hemp}
%\\222 & \zht{鹿} & l/`u     & \hie{deer}
%\\223 & \zht{黑} & h/-ei    & \hie{black}
%\\224 & \zht{鼓} & g/>u   & \hie{drum}
%\\225 & \zht{鼠} & sh/>u  & \hie{mouse}
%\\226 & \zht{鼻} & b/'i     & \hie{big nose}
%\end{longtable}

