%============================================================================
% Daniel J. Greenhoe
% XeLaTeX file
%============================================================================

%=======================================
\chapter{Fast Wavelet Transform (FWT) }
\index{fast wavelet transform}
\index{FWT}
%=======================================
The Fast Wavelet Transform can be computed using simple 
discrete filter operations (as a conjugate mirror filter).

%--------------------------------------
\begin{definition}[Wavelet Transform]
\label{def:wt}
\index{wavelet transform}
%--------------------------------------
Let the wavelet transform 
$\opW:\{f:\R\to\C\}\to \{w:\Z^2\to\C\}$ be defined as \footnotemark
\defbox{\begin{array}{rcl}
   [\opW \ff](j,n) &\eqd& \inprod{\ff(x)}{\fpsi_{k,n}(x)} 
\end{array}}
\end{definition}
\footnotetext{
Notice that this definition is similar to the definition of transforms 
of other analysis systems:
\[
\begin{array}{cllllll}
   \imark                             &
   \mbox{Laplace Transform}            & 
   \mathcal{L}f(s)                     & 
   \eqd& \inprod{\ff(x)}{e^{sx}}       & 
   \eqd& \int_x \ff(x) e^{-sx} \dx 
   \\
   \imark                             &
   \mbox{Continuous Fourier Transform} & 
   \mathcal{F}f(\omega)                & 
   \eqd& \inprod{\ff(x)}{e^{i\omega x}} & 
   \eqd& \int_x \ff(x) e^{-i\omega x} \dx 
   \\
   \imark                             &
   \mbox{Fourier Series Transform}     & 
   \mathcal{F}_sf(k)                     & 
   \eqd& \inprod{\ff(x)}{e^{i\frac{2\pi}{T}kx}} & 
   \eqd& \int_x \ff(x) e^{-i\frac{2\pi}{T}kx} \dx 
   \\
   \imark                             &
   \mbox{Z-Transform}                  & 
   \mathcal{Z}f(z)                     & 
   \eqd& \inprod{\ff(x)}{z^n} & 
   \eqd& \sum_n \ff(x) z^{-n} 
   \\
   \imark                             &
   \mbox{Discrete Fourier Transform}                  & 
   \mathcal{F}_df(k)                     & 
   \eqd& \inprod{f(n)}{e^{i\frac{2\pi}{N}kn}} & 
   \eqd& \sum_n \ff(x) e^{-i\frac{2\pi}{N}kn} 
   \\
\end{array}
\]
}

%--------------------------------------
\begin{definition}
\index{scaling coefficients}
\index{wavelet coefficients}
\index{scaling filter coefficients}
\index{wavelet filter coefficients}
%--------------------------------------
The following relations are defined as described below:
\defbox{\begin{array}{lllrcl}
   \mbox{scaling coefficients} & v_j:\Z\to\C & \mbox{such that} & 
   v_j(n) &\eqd& \ds \inprod{f(x)}{\fphi_{j,n}(x)}
\\
   \mbox{wavelet coefficients} & w_j:\Z\to\C & \mbox{such that} & 
   w_j(n) &\eqd& \ds \inprod{f(x)}{\fpsi_{j,n}(x)}
\\
   \mbox{scaling filter coefficients} & \fhb:\Z\to\C & \mbox{such that} & 
   \fhb(n) &\eqd& \ds \fh(-n)
\\
   \mbox{wavelet filter coefficients} & \fgb:\Z\to\C & \mbox{such that} & 
   \fgb(n) &\eqd& \ds \fg(-n)
\end{array}}
\end{definition}

The scaling and wavelet filter coefficients at scale~$j$ are 
equal to the filtered and downsampled \xref{thm:downsample}
scaling filter coefficients at scale~$j+1$:%
\footnote{
  \citerp{mallat}{257},
  \citerp{burrus}{35}
  }

\begin{liste}
   \item The convolution \xref{def:convd} of $v_{j+1}(n)$ with $\fhb(n)$ 
         and then downsampling by 2 produces $v_j(n)$.
   \item The convolution of $v_{j+1}(n)$ with $\fgb(n)$ 
         and then downsampling by 2 produces $w_j(n)$.
\end{liste}
This is formally stated and proved in the next theorem.
%--------------------------------------
\begin{theorem}
\index{scaling filters}
\index{wavelet filters}
%--------------------------------------
\thmbox{
   \begin{array}{rcl}
      v_j(n) &=& [\fhb \conv v_{j+1}](2n) \\
      w_j(n) &=& [\fgb \conv v_{j+1}](2n)
   \end{array}
}
\end{theorem}
\begin{proof}
\begin{align*}
    v_j(n) 
      &= \inprod{\ff(x)}{\fphi_{j,n}(x)} 
    \\&= \inprod{\ff(x)}{\sqrt{2^j}\fphi\left(2^jx-n \right)}
    \\&= \inprod{\ff(x)}{\sqrt{2^j}\sqrt{2}\sum_m h(m)\fphi\left(2(2^jx-n)-m \right)}
    \\&= \inprod{\ff(x)}{\sum_m h(j)\sqrt{2^{j+1}}\fphi\left(2^{j+1}x-2n -m \right)}
    \\&= \sum_m h(m)\inprod{\ff(x)}{\sqrt{2^{j+1}}\fphi\left(2^{j+1}x-2n -m \right)}
    \\&= \sum_m h(m)\inprod{\ff(x)}{\fphi_{j+1,2n+m}(x)}
    \\&= \sum_m h(m)v_{j+1}(2n+m)
    \\&= \sum_p h(p-2n)v_{j+1}(p)
      && \text{let }p=2n+m \iff m=p-2n
    \\&= \sum_p \fhb(2n-p) v_{j+1}(p)
    \\&= [\fhb \conv v_{j+1}](2n)
\\ 
\\
    w_j(n)
      &= \inprod{\ff(x)}{\fpsi_{j,n}(x)} 
    \\&= \inprod{\ff(x)}{\sqrt{2^j}\fpsi\left(2^jx-n \right)}
    \\&= \inprod{\ff(x)}{\sqrt{2^j}\sqrt{2}\sum_m \fg(j)\fphi\left(2(2^jx-n)-m \right)}
    \\&= \inprod{\ff(x)}{\sum_m \fg(m)\sqrt{2^{j+1}}\fphi\left(2^{j+1}x-2n -m \right)}
    \\&= \sum_m \fg(m)\inprod{\ff(x)}{\sqrt{2^{j+1}}\fphi\left(2^{j+1}x-2n -m \right)}
    \\&= \sum_m \fg(m)\inprod{\ff(x)}{\fphi_{j+1,2n+m}(x)}
    \\&= \sum_m \fg(m)v_{j+1}(2n+m)
    \\&= \sum_p \fg(p-2n)v_{j+1}(p)
      && \text{let }p=2n+m \iff m=p-2n
    \\&= \sum_p \fgb(2n-p)v_{j+1}(p) 
    \\&= [\fgb \conv v_{j+1}](2n)
   \end{align*}
\end{proof}

\begin{figure}[h] %\color{figcolor}
\centering\gsize%============================================================================
% Daniel J. Greenhoe
% XeLaTeX file
%============================================================================
{\setlength{\unitlength}{9mm}
\color{blue}%
\footnotesize%
\begin{tabular}{c}
\begin{picture}(16,1.65)
\thicklines
\put( 5,   1.1){\makebox(2,1)[b]{$v_j(n)=\inprod{f(x)}{\fphi_{j,n}(x)}$ }}
\put( 6  , 1  ){\vector(0,-1){1  } }
\put( 6  , 0.5){\line  (1, 0){3  } }
\put( 9  , 0.5){\vector(0,-1){0.5} }
\end{picture}
\\
\begin{picture}(16,4)
\thinlines
\put( 5.5, 3  ){\framebox(1,1){$\fhb(n)$} }
\put( 8.5, 3  ){\framebox(1,1){$\fgb(n)$} }
\put( 6  , 3  ){\vector(0,-1){0.5} }
\put( 9  , 3  ){\vector(0,-1){0.5} }
\put( 5.5, 1.5){\framebox(1,1){$\downarrow 2$} }
\put( 8.5, 1.5){\framebox(1,1){$\downarrow 2$} }
\put( 6  , 1.5){\vector(0,-1){1.5} }
\put( 9  , 1.5){\line  (0,-1){0.5} }
\put( 9  , 1  ){\vector(1, 0){1  } }
\put( 0,   0  ){\makebox(5,2)[r]{$v_{k-1}(n) = \inprod{\ff(x)}{\fphi_{k-1,n}(x)}$ }}
\put(10.5, 0  ){\makebox(5,2)[l]{$w_{k-1}(n) = \inprod{\ff(x)}{\fpsi_{k-1,n}(x)}$ }}
\put( 6  , 0.5){\line  (1, 0){3  } }
\put( 9  , 0.5){\vector(0,-1){0.5} }
\end{picture}
\\
\begin{picture}(16,4)
\thinlines
\put( 5.5, 3  ){\framebox(1,1){$\fhb(n)$} }
\put( 8.5, 3  ){\framebox(1,1){$\fgb(n)$} }
\put( 6  , 3  ){\vector(0,-1){0.5} }
\put( 9  , 3  ){\vector(0,-1){0.5} }
\put( 5.5, 1.5){\framebox(1,1){$\downarrow 2$} }
\put( 8.5, 1.5){\framebox(1,1){$\downarrow 2$} }
\put( 6  , 1.5){\vector(0,-1){1.5} }
\put( 9  , 1.5){\line  (0,-1){0.5} }
\put( 9  , 1  ){\vector(1, 0){1  } }
\put( 0,   0  ){\makebox(5,2)[r]{$v_{k-2}(n) = \inprod{\ff(x)}{\fphi_{k-2,n}(x)}$ }}
\put(10.5, 0  ){\makebox(5,2)[l]{$w_{k-2}(n) = \inprod{\ff(x)}{\fpsi_{k-2,n}(x)}$ }}
\put( 6  , 0.5){\line  (1, 0){3  } }
\put( 9  , 0.5){\vector(0,-1){0.5} }
\end{picture}
\\
$\vdots$\hspace{3cm}$\vdots$\hspace{1cm}
\\
\begin{picture}(16,1.5)
\thinlines
\put( 6  , 1.5){\vector(0,-1){1.5} }
\put( 0,   0  ){\makebox(5,2)[r]{$v_{1}(n) = \inprod{\ff(x)}{\fphi_{1,n}(x)}$ }}
\put(10.5, 0  ){\makebox(5,2)[l]{$w_{1}(n) = \inprod{\ff(x)}{\fpsi_{1,n}(x)}$ }}
\put( 6  , 0.5){\line  (1, 0){3  } }
\put( 9  , 0.5){\vector(0,-1){0.5} }
\end{picture}
\\
\begin{picture}(16,4)
\thinlines
\put( 5.5  , 3  ){\framebox(1,1){$\fhb(n)$} }
\put( 8.5  , 3  ){\framebox(1,1){$\fgb(n)$} }
\put( 6  , 3  ){\vector(0,-1){0.5} }
\put( 9  , 3  ){\vector(0,-1){0.5} }
\put( 5.5, 1.5){\framebox(1,1){$\downarrow 2$} }
\put( 8.5, 1.5){\framebox(1,1){$\downarrow 2$} }
\put( 6  , 1.5){\vector(0,-1){0.5} }
\put( 9  , 1.5){\line  (0,-1){0.5} }
\put( 9  , 1  ){\vector(1, 0){1  } }
\put( 0,   0  ){\makebox(5,2)[r]{$v_0(n) = \inprod{\ff(x)}{\fphi(x-n)}$ }}
\put(10.5, 0  ){\makebox(5,2)[l]{$w_0(n) = \inprod{\ff(x)}{\fpsi(x-n)}$ }}
%\put( 6  , 0.5){\line  (1, 0){3  } }
%\put( 9  , 0.5){\vector(0,-1){0.5} }
\end{picture}
\end{tabular}}



\caption{
   $k$-Stage Fast Wavelet Transform
   \label{fig:fwt}
   }
\end{figure}
These filtering and downsampling operations are equivalent to the 
operations performed by a filter bank.  
Therefore, a filter bank can be 
used to implement a \ope{Fast Wavelet Transform} (\ope{FWT}), 
as illustrated in \prefpp{fig:fwt}.

%\begin{center}
%\begin{figure}[h] \color{figcolor}
%\begin{small}
%\begin{tabular}{c}
%   \ingr{\tw-5mm}{20mm}{../common/wavelets/pare.eps} \\
%      $\ff(x)$: truncated parabola                           \\
%   \ingr{\tw/2-10mm}{60mm}{../common/wavelets/d2_par_fwte.eps} \\
%      $[\opW \ff](k,n)$ using Haar                     \\
%   \ingr{\tw/2-10mm}{60mm}{../common/wavelets/d4_par_fwte.eps} \\
%      $[\opW \ff](k,n)$ using Daubechies-4            
%\end{tabular}
%\end{small}
%\caption{Fast Wavelet Transforms using Haar and Daubechies-4 basis}
%\end{figure}
%\end{center}


