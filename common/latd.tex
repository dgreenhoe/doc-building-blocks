%============================================================================
% Daniel J. Greenhoe
% XeLaTeX file
%============================================================================


%=======================================
\chapter{Distributive Lattices}
%=======================================

%=======================================
\section{Distributivity relation}
%=======================================
%---------------------------------------
\begin{definition}
\footnote{
  \citerpc{maeda1970}{15}{Definition 4.1},
  \citePp{foulis1962}{67},
  \citerpc{vonNeumann1960}{32}{Definition 5.1},
  \citePpc{davis1955}{314}{\scs\prope{disjunctive distributive} and \prope{conjunctive distributive} f.}
  }
\label{def:Drel}
\label{def:jdis}
\label{def:mdis}
\label{def:distributive}
%---------------------------------------
Let $\latL\eqd\latticed$ be a \structe{lattice} \xref{def:lattice}.
Let $\clRxxx$ be the set of all \structe{relation}s in $\setX^3$.
\defboxp{%
  The \relxd{distributivity}      relation $\distrib \in\clRxxx$ and
  the \relxd{dual distributivity} relation $\distribd\in\clRxxx$ are defined as
  \\\indentx$\begin{array}{lclD}
    \symxd{\distrib}  &\eqd& \set{\otriple{x}{y}{z}\in\setX^3}{x\meet\brp{y\join z} = \brp{x\meet y}\join\brp{x\meet z} } & (each $\otriple{x}{y}{z}$ is \prope{disjunctive distributive}) and \\
    \symxd{\distribd} &\eqd& \set{\otriple{x}{y}{z}\in\setX^3}{x\join\brp{y\meet z} = \brp{x\join y}\meet\brp{x\join z} } & (each $\otriple{x}{y}{z}$ is \prope{conjunctive distributive}).
  \end{array}$
  \\
  A triple $\otriple{x}{y}{z}\in\distrib$  is alternatively denoted as
  $\otriple{x}{y}{z}\distrib$,
  and is called a \propd{distributive} triple.
  %
  A triple $\otriple{x}{y}{z}\in\distribd$  is alternatively denoted as
  $\otriple{x}{y}{z}\distribd$,
  and is called a \propd{dual distributive} triple.
  %
  A set $\setn{x,y,z}\subseteq\setX$ is \propd{distributive} in $\latL$ if each of the possible $3!=6$ triples
  [$\otriple{x}{y}{z}$, $\otriple{z}{x}{y}$, \ldots]
  constructed from the set is \prope{distributive} in $\latL$.
  }
\end{definition}

%=======================================
\section{Distributive Lattices}
%=======================================
%=======================================
\subsection{Definition}
%=======================================
This section introduces \structe{distributive lattices}.
\prefpp{thm:lat_dis_<} demonstrates that \emph{all}
        lattices $\latticed$ satisfy
        the following \hie{distributive inequalities}:
  \[
    \begin{array}{rclCDD}
      x \meet (y \join z) &\oreld& (x \meet y) \join (x \meet z)
        & \forall x,y,z \in\setX
        & (\prope{join super-distributive})
        & and
      \\
      x \join (y \meet z) &\orel& (x \join y) \meet (x \join z)
        & \forall x,y,z \in\setX
        & (\prope{meet sub distributive}).
        & and
      \\
      (x\meet y)\join(x\meet z)\join(y\meet z) &\orel& (x\join y)\meet(x\join z)\meet(y\join z)
        & \forall x,y,z \in\setX
        & (\prope{median inequality}).
    \end{array}
  \]
\prefpp{thm:lat_dis} demonstrates that when \emph{one}
of these inequalities is equality, then \emph{all three} of them are equalities.
And in this case, the lattice is defined to be \prope{distributive} (next definition).
%---------------------------------------
\begin{definition}
\label{def:lat_distributive}
\footnote{
  \citerpg{burris1981}{10}{0387905782},
  \citerpg{birkhoff1948}{133}{3540120440},
  \citePpc{ore1935}{414}{\hie{arithmetic axiom}},
  \citePp{birkhoff1933}{453},
  \citerpgc{balbes1975}{48}{098380110X}{Definition II.5.1}
  %\citerpg{maclane1999}{483}{0821816462}
  %\cithrp{burris2000}{12}
  }
\label{def:latd}
\index{lattice!distributive}
%---------------------------------------
\defboxt{
  A lattice $\latticed$ is \propd{distributive} if
    \\\indentx$\ds \otriple{x}{y}{z}\in\distrib \quad \forall x,y,z\in\setX $
  }
\end{definition}

Are all lattices \prope{distributive}? The answer is ``no".
\prefpp{lem:lat_N5} and \prefpp{lem:lat_M3} demonstrate two
lattices that are \emph{not} distributive:
the N5 lattice \xref{def:lat_N5} and the M3 lattice \xref{def:lat_M3}.


%=======================================
\subsection{Characterizations}
\index{characterizations!distributive lattices}
\index{equational bases!distributive lattices}
%=======================================
This section describes some \hi{characterizations} (\hi{equational bases}) 
of distributive lattices both in terms of lattices (order characterizations)
and in terms of abstract algebraic structures (algebraic characterizations).
\begin{liste}
  \item Order characterizations (first assuming a structure is a lattice):
    \\\begin{tabular}{@{\qquad}Nllll}
        \imark & \thm{Median property} & 1894  & \pref{thm:lat_dis}          & \prefpo{thm:lat_dis}
      \\\imark & \thm{Birkhoff distributivity criterion} & 1934  & \pref{thm:latd_char_n5m3}   & \prefpo{thm:latd_char_n5m3}
      \\\imark & \thm{Cancellation property} &  1934 & \pref{thm:latd_char_cancel}  & \prefpo{thm:latd_char_cancel}
    \end{tabular}

  \item Algebraic characterizations (first assuming nothing):
    \\\begin{tabular}{@{\qquad}Nllll}
        \imark & Birkhoff  & 1946  & \pref{prop:lat_char_b1946}  & \prefpo{prop:lat_char_b1946}
      \\\imark & Birkhoff  & 1948  & \pref{prop:lat_char_b1948}  & \prefpo{prop:lat_char_b1948}
      \\\imark & Sholander & 1951  & \pref{thm:lat_char_s1951}  & \prefpo{thm:lat_char_s1951}
    \end{tabular}

\end{liste}

Alternatively, any of the sets of properties listed in this section could be
used as the definition of distributive lattices and the definition
would in turn become a theorem/proposition.

In addition, if a lattice is \prope{uniquely complemented} and satisfies any one
of a number of \prope{Huntington properties}, then
it is also \prope{distributive} \xref{thm:latuc_hp}, and hence also a
\prope{Boolean algebra}\ifsxref{boolean}{def:boolean}.



%=======================================
\subsubsection{Order characterizations}
%=======================================
By the definition given in \prefpp{def:latd},
a lattice is \prope{distributive} if the \op{meet} operation $\meet$ distributes
over the \op{join} operation $\join$.
And in view that the properties of lattices are self-dual, it is perhaps not
surprising that the dual of the identity of \pref{def:latd} is also true
for any distributive lattice---
that is, the \op{join} operation $\join$ distributes over
the \op{meet} operation $\meet$ (next theorem).
But besides these two identities that are duals of each other,
there is another identity that is not only equivalent to the first two,
but is a dual of itself.
This is called the \prope{median} \emph{property},%
\footnote{%
  \prope{median} \emph{property}: see also \prefp{lit:lat_median}
  }
and is given by (3) in \pref{thm:lat_dis} (next theorem).
%---------------------------------------
\begin{theorem}
\footnote{
  \citePp{dilworth1984}{237},
  \citerpg{burris1981}{10}{0387905782},
  %\citerp{burris2000}{12}
  \citePpc{ore1935}{416}{(7),(8), Theorem 3},
  \citePc{ore1940}{cf Gratzer 2003 page 159},
  \citorpc{schroder1890}{286}{cf Birkhoff(1948)p.133},
  \citePc{korselt1894}{cf Birkhoff(1948)p.133}
  }
\label{thm:lat_dis}
\index{distributive}
%---------------------------------------
Let $\latL\eqd\latticed$ be a \hie{lattice} \xref{def:lattice}.
\thmboxt{
  $\latL$ is \prope{distributive} \xref{def:latd}
  \\$\begin{array}{clCCD}
    \iff& x \meet (y \join z) = (x\meet y) \join (x\meet z)
        & \forall x,y,z\in\setX
        & \mc{2}{D}{(\prope{disjunctive distributive})}
        %& (1)
  \\\iff& x \join (y \meet z) = (x\join y) \meet (x\join z)
        & \forall x,y,z\in\setX
        & \mc{2}{D}{(\prope{conjunctive distributive})}
        %& (2)
  \\\iff& \mc{2}{l}{(x\join y)\meet(x\join z)\meet(y\join z)=(x\meet y)\join(x\meet z)\join(y\meet z)}
        & \forall x,y,z\in\setX
        & (\prope{median property})
        %& (3)
  \end{array}$
  }
\end{theorem}
\begin{proof}
Let the join operation $\join$ be represented by $+$,
    the meet operation $\meet$ be represented by juxtaposition,
and let meet take algebraic precedence over join (+).
\begin{enumerate}
  \item Proof that \prope{distributive} $\iff$ \prope{disjunctive distributive}: %(1):
    \begin{align*}
      \brb{\text{$\latL$ is \prope{distributive}}}
        &\iff \brb{x\meet\brp{y\join z} = \brp{x\meet y}\join\brp{x\meet z} \quad \forall x,y,z\in\setX}
        && \text{by \prefp{def:latd}}
      \\&\iff \brb{\otriple{x}{y}{z}\in\distrib \quad \forall x,y,z\in\setX}
        && \text{by \prefp{def:Drel}}
    \end{align*}

  \item Proof that \prope{disjunctive distributive} $\implies$ \prope{conjunctive distributive}: %$(1)\implies(2)$:
    \begin{align*}
      x + (yz)
        &= \mcom{\big[ x + (x y)\big]}{expand $x$ wrt $y$} + (yz)
        && \text{by \prope{absorptive} property of lattices \prefpo{thm:lattice}}
      \\&= x + \big[ (x y) + (yz) \big]
        && \text{by \prope{associative} property of lattices \prefpo{thm:lattice}}
      \\&= x + \big[ (y x) + (yz) \big]
        && \text{by \prope{commutative} property of lattices \prefpo{thm:lattice}}
      \\&= x + \big[ y(x + z) \big]
        && \text{by left hypothesis}
      \\&= \mcom{\big[ x(x + z)\big]}{expand $x$ wrt $z$}
          + \big[ y(x + z) \big]
        && \text{by \prope{absorptive} property of lattices \prefpo{thm:lattice}}
      \\&= \big[ (x + z)x\big] + \big[ (x + z)y\big]
        && \text{by \prope{commutative} property of lattices \prefpo{thm:lattice}}
      \\&= (x + z)(x + y)
        && \text{by left hypothesis}
      \\&= (x + y)(x + z)
        && \text{by \prope{commutative} property of lattices \prefpo{thm:lattice}}
    \end{align*}

  \item Proof that \prope{conjunctive distributive} $\implies$ \prope{disjunctive distributive}: %$(2)\implies(1)$:
    \begin{align*}
      x(y + z)
        &= \mcom{\big[ x(x+ y)\big]}{expand $x$ wrt $y$}(y + z)
        && \text{by \prope{absorptive} property of lattices \prefpo{thm:lattice}}
      \\&= x\big[ (x+ y)(y + z) \big]
        && \text{by \prope{associative} property of lattices \prefpo{thm:lattice}}
      \\&= x\big[ (y+ x)(y + z) \big]
        && \text{by \prope{commutative} property of lattices \prefpo{thm:lattice}}
      \\&= x\big[ y+(xz) \big]
        && \text{by right hypothesis}
      \\&= \mcom{\big[ x + (xz)\big]}{expand $x$ wrt $z$}
          \big[ y+(xz) \big]
        && \text{by \prope{absorptive} property of lattices \prefpo{thm:lattice}}
      \\&= \big[ (xz) + x\big]\big[ (xz) + y\big]
        && \text{by \prope{commutative} property of lattices \prefpo{thm:lattice}}
      \\&= (xz) + (xy)
        && \text{by left hypothesis}
      \\&= (xy) + (xz)
        && \text{by \prope{commutative} property of lattices \prefpo{thm:lattice}}
    \end{align*}

  \item Proof that \prope{disjunctive distributive} $\implies$ \prope{median property}: %$(1)\implies(3)$:
    \begin{align*}
      &(x+y)(x+z)(y+z)
      \\&= (x+y)\brs{(x+z)y+(x+z)z}
        && \text{by \prope{disjunctive distributive} hypothesis}
      \\&= (x+y)\brs{y(x+z)+z(x+z)}
        && \text{by \prope{commutative} property \xref{thm:lattice}}
      \\&= (x+y)\brp{yx+yz+zx+zz}
        && \text{by \prope{disjunctive distributive} hypothesis}
      \\&= (x+y)\brp{xy+xz+yz+z}
        && \text{by \prefp{thm:lattice}}
      \\&= (x+y)xy+(x+y)xz+(x+y)yz+(x+y)z
        && \text{by \prope{disjunctive distributive} hypothesis}
      \\&= xy(x+y)+xz(x+y)+yz(x+y)+z(x+y)
        && \text{by \prope{commutative} property \xref{thm:lattice}}
      \\&= xyx+xyy+xzx+xzy+yzx+yzy+zx+zy
        && \text{by \prope{disjunctive distributive} hypothesis}
      \\&= xy+xy+xz+xyz+xyz+yz+xz+yz
        && \text{by \prefp{thm:lattice}}
      \\&= xy+xyz+xz+yz
        && \text{by \prope{idempotent} property \xref{thm:lattice}}
      \\&= (xy)(xy)+xyz+xz+yz
        && \text{by \prope{idempotent} property \xref{thm:lattice}}
      \\&= (xy)(xy+z)+xz+yz
        && \text{by \prope{disjunctive distributive} hypothesis}
      \\&= xy+xz+yz
        && \text{by \prope{absorptive} property \xref{thm:lattice}}
    \end{align*}

  \item Proof that \prope{median property} $\implies$ \prope{disjunctive distributive}: %$(3)\implies(1)$:
    \begin{enumerate}
      \item Proof that $\latL$ is \prope{modular}:\label{item:lat_dis_mod}
        \begin{align*}
          y\orel x \implies x(y+z)
            &= x(x+z)(y+z)
            && \text{by \prope{absorptive} property \xref{thm:lattice}}
          \\&= (x+y)(x+z)(y+z)
            && \text{by $y\orel x$ hypothesis}
          \\&= xy + xz + yz
            && \text{by \prope{median property} hypothesis}
          \\&= y + xz + yz
            && \text{by $y\orel x$ hypothesis}
          \\&= y + xz
            && \text{by \prope{absorptive} property \xref{thm:lattice}}
          \\&\implies \text{$\latL$ is \prope{modular}}
        \end{align*}

      \item Proof that $a+ab=a$: \label{item:lat_dis_aab}
        \begin{align*}
          ab
            &\orel a
            &&     \text{by definition of $\meet$ \prefp{def:meet}}
          \\&\implies a+ab = a
            &&     \text{by definition of $\join$ \prefp{def:join}}
        \end{align*}

      \item Proof that \prope{median property} $\implies$ \prope{disjunctive distributive}: %(1) is true:
        \begin{align*}
          x(y+z)
            &= xx(y+z)
            && \text{by \prope{idempotent} property \xref{thm:lattice}}
          \\&= \mcom{x(x+y)}{$x$}\mcom{x(x+z)}{$x$}(y+z)
            && \text{by \prope{absorptive} property \xref{thm:lattice}}
          \\&= x\brs{(x+y)(x+z)(y+z)}
            && \text{by \prefp{thm:lattice}}
          \\&= x(xy + \mcom{xz + yz}{$z'$})
            && \text{by \prope{median property} hypothesis}
          \\&= x(xy) + x(xz + \mcom{yz}{$z''$})
            && \text{by \pref{item:lat_dis_mod} and by \prefp{thm:latm_id}}
          \\&= x(xy) + x(xz) + x(yz)
            && \text{by \pref{item:lat_dis_mod} and by \prefp{thm:latm_id}}
          \\&= xy + xz + xyz
            && \text{by \prefp{thm:lattice}}
          \\&= xy + xz
            && \text{by \pref{item:lat_dis_aab}}
        \end{align*}
    \end{enumerate}
\end{enumerate}
\end{proof}



%---------------------------------------
\begin{lemma}
\label{lem:lat_N5}
\citetbl{
  \citerpg{burris1981}{11}{0387905782}
  }
\index{lattice!N5}
\index{N5 lattice}
%---------------------------------------
\lembox{\text{The $N5$ lattice is \prope{non-distributive}}}
\end{lemma}
\begin{proof}
  \begin{align*}
    y \meet ( x\join z)
      &= y \meet b
      && \text{by \prefp{def:lub} (lub)}
    \\&= y
      && \text{by \prefp{def:glb} (glb)}
    \\&= y \join a
      && \text{by \prefp{def:lub} (lub)}
    \\&= y \join (y \meet z)
      && \text{by \prefp{def:glb} (glb)}
    \\&\ne  x \join (y \meet z)
      && \text{because $x\ne y$}
    \\&= (y \meet x) \join (y \meet z)
      && \text{by \prefp{def:glb} (glb)}
  \end{align*}
\end{proof}




\begin{minipage}{\tw-33mm}%
%---------------------------------------
\begin{definition}[M3 lattice/diamond]
\label{def:lat_M3}
\label{def:m3}
\index{lattice!M3}
\footnotemark
%---------------------------------------
\defboxt{
  The \hid{M3 lattice} is the ordered set $\opair{\setn{0,p,q,r,1}}{\orel}$
  with covering relation
  \\\indentx
    $\ds\coverrel = \setn{\opair{p}{1},\, \opair{q}{1},\, \opair{r}{1},\, \opair{0}{p},\, \opair{0}{q},\, \opair{0}{r} }.$
  \\
  The M3 lattice is also called the \structd{diamond},
  }
%The M3 lattice 
and is illustrated by the Hasse diagram to the right.
\end{definition}
\end{minipage}%
\citetblt{%
  \citerppg{beran1985}{12}{13}{902771715X},
  \citorpc{korselt1894}{157}{$p_1\equiv x,\,p_2\equiv y,\,p_3\equiv z,\,g\equiv\bid,\,0\equiv\bzero$}
  }%
\hfill\tbox{\includegraphics{graphics/lat5_m3_pqr.pdf}}\hfill\mbox{}\\


%---------------------------------------
\begin{remark}
%---------------------------------------
The M3 lattice is isomorphic to the lattices
\\\begin{tabular}{@{\qquad}MMl}
  \imark & \opair{\sssP{\setn{x,y,z}}}{\orel}\footnotemark 
          & where $\sssP{\setn{x,y,z}}$ is the set of \hie{partitions} on $\setn{x,y,z}$\\
         && and with $\orel$ defined as in \prefpp{prop:partition_orel}\\
  \imark & \opair{\sssR{\setn{x,y}}}{\sorel}   
          & where $\sssR{\setn{x,y}}$ is the set of \hie{rings of sets} on $\setn{x,y}$\\
  \imark & \opair{\sssA{\setn{x,y,z}}}{\sorel} 
          & where $\sssA{\setn{x,y,z}}$ is the set of \hie{algebras of sets} on $\setn{x,y,z}$.
\end{tabular}
\\
\citetblt{\citerpg{salii1988}{22}{0821845225}}
See \prefpp{ex:set_partitions}, \prefpp{ex:set_rings}, \prefpp{ex:set_asets}, and
\prefpp{fig:set_partition_xyz}. 
\end{remark}

%---------------------------------------
\begin{lemma}
\index{lattice!M3}
\index{M3 lattice}
\footnote{
  \citorpg{birkhoff1948}{6}{3540120440},
  \citerpg{burris1981}{11}{0387905782},
  \citorpc{korselt1894}{157}{cf Salii1988 p. 37}
  }%
\label{lem:lat_M3}
%---------------------------------------
\lembox{
  \brb{\begin{array}{M}
    $\latL$ is an \structb{M3 lattice}\\ 
    \xref{def:m3}
  \end{array}}
  \implies
  \brb{\begin{array}{FMDD}
    1. & $\latL$ is \emph{not} distributive  & \xref{def:latd} & and \\
    2. & $\latL$ \emph{is} modular           & \xref{def:latm} &
  \end{array}}
  }
\end{lemma}
\begin{proof}
\begin{enumerate}
  \item Proof that $M3$ is non-distributive:
    \begin{align*}
      x \meet ( a\join c)
        &= x \meet y
        && \text{by def. of l.u.b. \prefpo{def:lub}}
      \\&= x
        && \text{by def. of g.l.b. \prefpo{def:glb}}
      \\&\ne  b
      \\&= b \join b
        && \text{by \prefp{thm:lattice} (idempotent property)}
      \\&= \mcom{(x\meet a)}{$b$} \join \mcom{(x \meet c)}{$b$}
        && \text{by def. of g.l.b. \prefpo{def:glb}}
    \end{align*}

  \item Proof that $M3$ is modular: (proof by exhaustion)
  \begin{multicols}{3}
    \begin{align*}
      x \join ( y \meet a)
        &= x \join a
      \\&= y
      \\&= y \meet y
      \\&= y \meet ( x\join a)
      \\
      x \join ( y \meet b)
        &= x \join b
      \\&= x
      \\&= y \meet x
      \\&= y \meet ( x\join b)
      \\
      x \join ( y \meet c)
        &= x \join c
      \\&= y
      \\&= y \meet y
      \\&= y \meet ( x\join c)
      \\
      \\
      a \join ( y \meet x)
        &= a \join x
      \\&= y
      \\&= y \meet y
      \\&= y \meet ( a\join x)
      \\
      a \join (y \meet b)
        &= a \join b
      \\&= a
      \\&= y \meet a
      \\&= y \meet (a\join b)
      \\
      a \join ( y \meet c)
        &= a \join c
      \\&= y
      \\&= y \meet y
      \\&= y \meet (a\join c)
      \\
      \\
      c \join ( y \meet a)
        &= c \join a
      \\&= y
      \\&= y \meet y
      \\&= y \meet (c\join a)
      \\
      c \join ( y \meet x)
        &= c \join x
      \\&= y
      \\&= y \meet y
      \\&= y \meet ( c\join x)
      \\
      c \join (y \meet b)
        &= c \join b
      \\&= c
      \\&= y \meet c
      \\&= y \meet (c\join b)
      \\
      \\
      b \join ( y \meet a)
        &= b \join a
      \\&= a
      \\&= y \meet a
      \\&= y \meet (b\join a)
      \\
      b \join ( y \meet x)
        &= b \join x
      \\&= x
      \\&= y \meet x
      \\&= y \meet (b\join x)
      \\
      b \join (y \meet c)
        &= b \join c
      \\&= c
      \\&= y \meet c
      \\&= y \meet (b\join c)
      \\
      \\
      b \join (a \meet x)
        &= b \join b
      \\&= b
      \\&= a \meet x
      \\&= a \meet ( b\join x)
      \\
      b \join (a \meet y)
        &= b \join a
      \\&= a
      \\&= a \meet y
      \\&= a \meet ( b\join y)
      \\
      b \join (a \meet c)
        &= b \join b
      \\&= b
      \\&= a \meet c
      \\&= a \meet ( b\join c)
      \\
      \\
      b \join (x \meet a)
        &= b \join b
      \\&= b
      \\&= x \meet a
      \\&= x \meet ( b\join a)
      \\
      b \join (x \meet c)
        &= b \join b
      \\&= b
      \\&= x \meet c
      \\&= x \meet ( b\join c)
      \\
      b \join (x \meet y)
        &= b \join x
      \\&= x
      \\&= x \meet y
      \\&= x \meet ( b\join y)
      \\
      \\
      b \join (c \meet x)
        &= b \join b
      \\&= b
      \\&= c \meet x
      \\&= c \meet ( b\join x)
      \\
      b \join (c \meet y)
        &= b \join c
      \\&= c
      \\&= c \meet y
      \\&= c \meet ( b\join y)
      \\
      b \join (c \meet a)
        &= b \join b
      \\&= b
      \\&= c \meet a
      \\&= c \meet ( b\join a)
    \end{align*}
\end{multicols}
\end{enumerate}
\end{proof}



The \thme{Birkhoff distributivity criterion} (next)
demonstrates that a lattice is distributive \emph{if and only if}
it does not contain either the N5 or M3 lattices.
If a lattice does contain either of these,
it is \emph{not} distributive.
If a lattice is distributive, it does \emph{not} contain either the N5 or M3 lattices.
There was a similar theorem for \prope{modular} lattices and the N5 lattice
\xref{thm:latm_n5}.
%---------------------------------------
\begin{theorem}[\thmd{Birkhoff distributivity criterion}]
\label{thm:latd_char_n5m3}
\label{thm:bdc}
\citetbl{
  %\citerpg{davey2002}{89}{0521784514}\\
  \citerpg{burris1981}{12}{0387905782},
  %\citerpg{gratzer1971}{70}{0716704420}
  \citerpg{birkhoff1948}{134}{3540120440},
  \citeP{birkhoff1934}%{cf Stern 1999 page 9}
  %\cithrp{burris2000}{14}
  }
%---------------------------------------
Let $\latL\eqd\latticed$ be a \hie{lattice}.
\thmbox{
  \text{$\latL$ is \prope{distributive}}
  \quad\iff\quad
  \left\{\begin{tabular}{lll}
    $\latL$ does {\bf not} contain $N5$ as a sublattice &
    \setlength{\unitlength}{0.02mm}
    \begin{picture}(240,340)(-100,0)
      %\graphpaper[10](0,0)(600,200)
      \thinlines
      \color{latline}
        \qbezier(-100,200)(-50,250)(0,300)%
        \qbezier(100,150)(50,75)(0,0)%
        \qbezier(100,150)(50,225)(0,300)%
        \qbezier(-100,100)(-100,150)(-100,200)%
        \qbezier(0,0)(-50,50)(-100,100)%
        \qbezier(0,0)(50,75)(100,150)%
      \color{latdot}%
        \put(   0, 300){\circle*{60}}%
        \put(-100, 200){\circle*{60}}%
        \put( 100, 150){\circle*{60}}%
        \put(-100, 100){\circle*{60}}%
        \put(   0,   0){\circle*{60}}%
    \end{picture}
    & and
    \\
    $\latL$ does {\bf not} contain $M3$ as a sublattice &
    \setlength{\unitlength}{0.02mm}
    \begin{picture}(240,240)(-100,0)
      %\graphpaper[10](0,0)(600,200)
      \thinlines
      \color{latline}%
        \qbezier(0,200)(-50,150)(-100,100)%
        \qbezier(0,200)( 50,150)( 100,100)%
        \qbezier(0,0)( 50,50)( 100,100)%
        \qbezier(0,0)(  0,100)(   0,200)%
        \qbezier(0,0)(-50,50)(-100,100)%
      \color{latdot}%
        \put(   0, 200){\circle*{60}}%
        \put( 100, 100){\circle*{60}}%
        \put(-100, 100){\circle*{60}}%
        \put(   0, 100){\circle*{60}}%
        \put(   0,   0){\circle*{60}}%
    \end{picture}
  \end{tabular}\right.
}
\end{theorem}
\begin{proof}
\begin{enumerate}
  \item Proof that $\latL$ is distributive $\implies$ $\latL$ does \emph{not} contain $N5$:\\
    This follows directly from \prefpp{lem:lat_N5}.

  \item Proof that $\latL$ is distributive $\implies$ $\latL$ does \emph{not} contain $M3$:\\
    This follows directly from \prefpp{lem:lat_M3}.

  \item Proof that $\latL$ is distributive $\impliedby$ $N5\notin L$ and $M3\notin L$:



    \begin{enumerate}
      \item Proof that this statement is equivalent to
        {\begin{minipage}[t]{2\tw/16+1ex}%
            \fontsize{0.5mm}{0.5mm}\usefont{T1}{put}{b}{n}%
            \begin{Verbatim}[frame=single, gobble=8, label={\footnotesize\rmfamily email}]
              Consider the 3 statements:

              A:  L is nondistributive
              B:  M3 can be embedded in L
              C: N5 can be embedded in L

              Theorem 3.6 says:
                      A <--> (B or C)
              is true of lattices.

              This is equivalent to:
                  (A --> (B or C)) and ((B or C) --> A)
              which is equivalent to
                  ((A --> (B or C))  and ((B --> A) and (C --> A))
              The last two implications are your (1) and (2).

              So what remains after these two is to show
                  A --> (B or C)
              is true. But this is equivalent to
                  not A or B or  C
              which is equivalent to
                  not A or not (not C) or B
               which is equivalent  (by de Morgan's Laws) to
                  not (A and not C)) or B
              which is equivalent to
                  (A and not C) --> B
              which is your (3).

              Note: one could have used (A and not B) --> C, which is your (4),
              instead of (3). But no need to prove both. The above is just an
              application of propositional logic. You can also use truth tables
              to show that

                  A <--> (B or C)
              and
                  ((A and not C) --> B) and ((B --> A) and (C --> A))

              have the same the same truth values, no matter how you assign TRUE/FALSE
              to each of A,B,C. So they are equivalent.

              Stanley Burris
              Professor Emeritus
              Dept. of Pure Mathematics
              University of Waterloo
              Waterloo, Ontario
              Canada N2L 3G1
            \end{Verbatim}
          \end{minipage}
          \hfill
          \begin{minipage}[t]{13\tw/16-3ex}
            Many many thanks to University of Waterloo Professor Emeritus
            \href{http://www.thoralf.uwaterloo.ca}{Stanley Burris}
            for his brilliant help with the logical structure of this proof.
            (If you are viewing this text as a pdf file,
             zoom in on the figure to the left to see text from Professor Burris' 2007 October 9 email.)
          \end{minipage}
        }
          \[ (\text{$\latL$ is nondistributive}) \land (N5\notin L) \implies (M3\in L): \]
        Let $P\equiv Q$ denote that statement $P$ is equivalent to statement $Q$. Then \ldots
        \begin{align*}
          &\text{($\latL$ is distributive)} \impliedby (N5\notin L) \land (M3\notin L)
          \\&\equiv \text{($\latL$ is nondistributive)} \implies (N5\in L) \lor (M3\in L)
            && \text{contrapositive}
          \\&\equiv \lnot(\text{$\latL$ is nondistributive}) \lor \brs{(N5\in L) \lor (M3\in L)}
            && \text{by definition of $\implies$\ifsxref{logic}{def:iff}}
          \\&\equiv \brs{\lnot(\text{$\latL$ is nondistributive}) \lor (N5\in L)} \lor (M3\in L)
            && \text{by associative property\ifsxref{logic}{thm:logic}}
          \\&\equiv \lnot\lnot\brs{\lnot(\text{$\latL$ is nondistributive}) \lor \lnot(N5\notin L)} \lor (M3\in L)
            && \text{by involutary property\ifsxref{logic}{thm:logic}}
          \\&\equiv \lnot\brs{(\text{$\latL$ is nondistributive}) \land (N5\notin L)} \lor (M3\in L)
            && \text{by de Morgan's law\ifsxref{logic}{thm:logic}}
          \\&\equiv (\text{$\latL$ is nondistributive}) \land (N5\notin L) \implies (M3\in L)
            && \text{by definition of $\implies$\ifsxref{logic}{def:iff}}
        \end{align*}

      \item Proof that
            $\latL$ is \emph{not} distributive and $N5\notin L$ $\implies$ $M3\in L$:
        \begin{enumerate}
          \item Because $N5\notin L$ and by \prefpp{thm:latm_n5}, $\latL$ is modular
                (so we can use the modularity property of \prefp{def:lat_mod}).

          \item We will show that the five values defined below form an $M3$
                lattice:\label{item:lat_distrib_iff_N5M3_def}
            \\\begin{minipage}{8\tw/16}
            \begin{align*}
              b       &\eqd (x \join y) \meet (x \join z) \meet (y \join z) \\
              a       &\eqd (x \meet y) \join (x \meet z) \join (y \meet z) \\
              \dot{x} &\eqd (x \meet b) \join a  \\
              \dot{y} &\eqd (y \meet b) \join a  \\
              \dot{z} &\eqd (z \meet b) \join a
            \end{align*}
            \end{minipage}%
            \begin{minipage}{2\tw/16}
              \latmatl{3}{
                              & [name=1]\null                 \\
                [name=x]\null & [name=y]\null & [name=z]\null \\
                              & [name=0]\null
                }{
                \ncline{1}{x} \ncline{1}{y} \ncline{1}{z}
                \ncline{0}{x} \ncline{0}{y} \ncline{0}{z}
                }{
                \nput{ 90}{1}{$\bid$}
                \nput{180}{x}{$\dot{x}$} \nput{180}{y}{$\dot{y}$} \nput{0}{z}{$\dot{z}$}
                \nput{-90}{0}{$\bzero$}
                }            
            \end{minipage}

          \item Proof that $a\orel b$:   \label{item:lat_distrib_iff_N5M3_ab}
            \begin{align*}
              a
                &= (x \meet y) \join (x \meet z) \join (y \meet z)
                && \text{by definition of $a$ (\pref{item:lat_distrib_iff_N5M3_def})}
              \\&= (x \meet y \meet x) \join (x \meet z \meet z) \join (y \meet z \meet z)
                && \text{by \prope{idempotent property} of lattices (\prefpo{thm:lattice})}
              \\&\orel (x \join x \join y) \meet (y \join z \join z) \meet (x \join z \join z)
                && \text{by minimax inequality \prefp{thm:minimax_ineq}}
              \\&= (x \join y) \meet (y \join z) \meet (x \join z)
                && \text{by \prope{idempotent property} of lattices (\prefpo{thm:lattice})}
              \\&= (x \join y) \meet (x \join z) \meet (y \join z)
                && \text{by \prope{commutative property} of lattices (\prefpo{thm:lattice})}
              \\&= b
                && \text{by definition of $b$ (\pref{item:lat_distrib_iff_N5M3_def})}
            \end{align*}

            \[
              \joinop\left\{\begin{array}{l lll l}
                \meetop \Big\{ & x & y & x & \Big\}\\\hline
                \meetop \Big\{ & x & z & z & \Big\}\\\hline
                \meetop \Big\{ & y & z & z & \Big\}
              \end{array}\right\}
              \qquad\orel\qquad
              \meetop\left\{\begin{array}{c|c|c}
                \joinop & \joinop & \joinop  \\
                x & y & x \\
                x & z & z \\
                y & z & z
              \end{array}\right\}
            \]

          \item Proof that $a\orel\dot{x}\orel\dot{y}\orel\dot{z}\orel b$:
            \begin{enumerate}
              \item By \pref{item:lat_distrib_iff_N5M3_ab}, $a\orel b$.
              \item By definition of $\meet$, $(x\meet b)$ must be less than or equal to $b$.
              \item By definition of $\join$, $(x\meet b)\join a$ must be greater than or equal to $a$.
              \item By definition of $\dot{x}$ (\pref{item:lat_distrib_iff_N5M3_def}), $a\orel\dot{x}\orel b$.
              \item The proofs for $a\orel\dot{y}\orel b$ and $a\orel\dot{z}\orel b$ are essentially identical to that of
                    $a\orel\dot{x}\orel b$.
            \end{enumerate}

          \item Proof that $\dot{x}\meet\dot{y}=\dot{x}\meet\dot{z}=\dot{y}\meet\dot{z}=a$:\label{item:lat_distrib_iff_N5M3_xya}
            \begin{align*}
              \dot{x} \meet \dot{y}
                &= [\mcom{(x\meet b)\join a}{$\dot{x}$}] \meet \dot{y}
                && \text{by definition of $\dot{x}$ \pref{item:lat_distrib_iff_N5M3_def}}
              \\&= [(x\meet b) \muline{\meet \dot{y}}] \muline{\join a}
                && \text{by modularity \prefpo{def:lat_mod}}
              \\&= [(x\meet b) \meet (\mcom{(y\meet b)\join a}{$\dot{y}$})] \join a
                && \text{by definition of $\dot{y}$ \pref{item:lat_distrib_iff_N5M3_def}}
              \\&= [(x\meet b) \meet (y\muline{\join a}) \muline{\meet b}] \join a
                && \text{by modularity \prefpo{def:lat_mod}}
              \\&= [(x\meet b) \meet (y\join a) ] \join a
                && \text{by idempotent property \prefpo{thm:lattice}}
              \\&= \left[\brp{x\meet \mcom{\brs{(x \join y) \meet (x \join z) \meet (y \join z)}}{$b$}}\right.
                   \meet
                   \\&\quad
                   \left.\brp{y\join \mcom{\brs{(x \meet y) \join (x \meet z) \join (y \meet z)}}{$a$}}\right]
                   \join a
                && \text{by definitions of $a$ and $b$ \pref{item:lat_distrib_iff_N5M3_def}}
              \\&= \brs{\brp{x \meet (y \join z)} \meet \brp{y \join (x \meet z)} } \join a
                && \text{by absorption property \prefpo{thm:lattice}}
              \\&= \brs{x \meet \brp{\muline{y \join} \brp{\muline{(y \join z) \meet} (x \meet z)}} } \join a
                && \text{by modularity \prefpo{def:lat_mod}}
              \\&= \brs{x \meet \brp{y \join (x \meet z)} } \join a
                && \text{because $(x\meet z)\orel(y\join z)$}
              \\&= \brs{\muline{(x \meet z) \join} (\muline{x \meet} y) } \join a
                && \text{by modularity \prefpo{def:lat_mod}}
              \\&= \brs{(x \meet z) \join (x \meet y) } \join
                   \mcom{\brs{(x \meet y) \join (x \meet z) \join (y \meet z)}}{$a$}
                && \text{by definition of $a$ \pref{item:lat_distrib_iff_N5M3_def}}
              \\&= (x \meet y) \join (x \meet z) \join (y \meet z)
                && \text{by idempotent property \prefpo{thm:lattice}}
              \\&= a
                && \text{by definition of $a$ \pref{item:lat_distrib_iff_N5M3_def}}
            \end{align*}

          \item To prove that $\dot{x}\meet\dot{z}=a$,
                simply replace $\dot{y}$ with $\dot{z}$
                and $y$ with $z$ in \pref{item:lat_distrib_iff_N5M3_xya}.

          \item To prove that $\dot{y}\meet\dot{z}=a$,
                simply replace $\dot{x}$ with $\dot{z}$
                and $x$ with $z$ in \pref{item:lat_distrib_iff_N5M3_xya}.

          \item Proof that $\dot{x}\join\dot{y}=b$:\label{item:lat_distrib_iff_N5M3_xyb}
            \begin{align*}
              \dot{x} \join \dot{y}
                &= [\mcom{(x\meet b)\join a}{$\dot{x}$}] \join \dot{y}
                && \text{by definition of $\dot{x}$ \pref{item:lat_distrib_iff_N5M3_def}}
              \\&= [(x\muline{\join a})\muline{\meet b}] \join \dot{y}
                && \text{by modularity \prefpo{def:lat_mod}}
              \\&= [(x\join a)\muline{\join \dot{y}}] \muline{\meet b}
                && \text{by modularity \prefpo{def:lat_mod}}
              \\&= [(x\join a) \join \mcom{((y\meet b)\join a)}{$\dot{y}$}] \meet b
                && \text{by definition of $\dot{y}$ \pref{item:lat_distrib_iff_N5M3_def}}
              \\&= [(x\join a) \join (y\meet b)  ] \meet b
                && \text{by idempotent property \prefpo{thm:lattice}}
              \\&= \left[\brp{x\join \mcom{\brs{(x \meet y) \join (x \meet z) \join (y \meet z)}}{$a$}}\right.
                   \join
                   \\&\quad
                   \left.\brp{y\meet \mcom{\brs{(x \join y) \meet (x \join z) \meet (y \join z)}}{$b$}}\right]
                   \meet b
                && \text{by definitions of $a$ and $b$ \pref{item:lat_distrib_iff_N5M3_def}}
              \\&= \brs{\brp{x\join (y \meet z)} \join \brp{y \meet (x \join z)}} \meet b
                && \text{by absorption property \prefpo{thm:lattice}}
              \\&= \brs{x\join (y \meet z) \join \brp{y \meet (x \join z)}} \meet b
                && \text{by associative property \prefpo{thm:lattice}}
              \\&= \brs{x\join \brp{\muline{y \meet} \brs{ \muline{(y \meet z) \join} (x \join z)}}} \meet b
                && \text{by modularity \prefpo{def:lat_mod}}
              \\&= \brs{x\join \brp{y \meet (x \join z)}} \meet b
                && \text{by \pref{def:lub} and \pref{def:glb}}
                %&& \text{by definitions of $\join$ and $\meet$ \prefpo{def:lub} and \prefpo{def:glb}}
              \\&= \brs{\muline{(x \join z)\meet} \brp{\muline{x\join} y }} \meet b
                && \text{by modularity \prefpo{def:lat_mod}}
              \\&= \brs{(x \join z)\meet (x\join y)} \meet
                       [\mcom{(x \join z)\meet (x\join y) \meet (y\join z)}{$b$}]
                && \text{by definition of $b$ \pref{item:lat_distrib_iff_N5M3_def}}
              \\&= (x \join z)\meet (x\join y) \meet (y\join z)
                && \text{by idempotent property \prefpo{thm:lattice}}
              \\&= b
                && \text{by definition of $b$ \pref{item:lat_distrib_iff_N5M3_def}}
            \end{align*}

          \item To prove that $\dot{x}\join\dot{z}=b$,
                simply replace $\dot{y}$ with $\dot{z}$
                and $y$ with $z$ in \pref{item:lat_distrib_iff_N5M3_xyb}.

          \item To prove that $\dot{y}\join\dot{z}=b$,
                simply replace $\dot{x}$ with $\dot{z}$
                and $x$ with $z$ in \pref{item:lat_distrib_iff_N5M3_xyb}.
        \end{enumerate}
    \end{enumerate}

\end{enumerate}
\end{proof}

%---------------------------------------
\begin{theorem}[\thmd{cancellation criterion}]
\citetbl{
  \citerppg{blyth2005}{67}{68}{1852339055},
  \citeP{birkhoff1934}%{cf Blyth 2005 page 67}
  }
\label{thm:latd_char_cancel}
\label{thm:lat_distrib_a=b}
%---------------------------------------
Let $\latL\eqd\latticed$ be a \prope{lattice}.
\thmbox{
  \text{$\latL$ is \prope{distributive}}
  \quad\iff\quad
  \mcom{\brb{\brb{\begin{array}{rclCDD}
    x \join z &=& y \join z & \forall x,y,z\in\setX & and & (1)\\
    x \meet z &=& y \meet z & \forall x,y,z\in\setX &     & (2)
  \end{array}}\implies x=y}}{\prope{cancellation} property}
  }
\end{theorem}
\begin{proof}
\begin{enumerate}
  \item Proof that \prope{distributive} property $\implies$ \prope{cancellation} property:
    \begin{align*}
      x
        &= x(x+z)
        && \text{by \prope{absorbtive} property \xref{thm:lattice}}
      \\&= x(y+z)
        && \text{by (1)}
      \\&= xy+xz
        && \text{by \prope{distributive} hypothesis}
      \\&= xy+yz
        && \text{by (2)}
      \\&= yx+yz
        && \text{by \prope{commutative} property \xref{thm:lattice}}
      \\&= y(x+z)
        && \text{by \prope{distributive} hypothesis}
      \\&= y(y+z)
        && \text{by (1)}
      \\&= y
        && \text{by \prope{absorbtive} property \xref{thm:lattice}}
    \end{align*}

  \item Proof that \prope{distributive} property $\impliedby$ \prope{cancellation} property:
    \begin{enumerate}
      \item Define \label{item:cancel_abcd}
        \begin{align*}
          a &\eqd x(y+z) \\
          b &\eqd y(x+z) \\
          c &\eqd z(x+y) \\
          d &\eqd (x+y)(x+z)(y+z)
        \end{align*}

      \item Proof that $ab=xy$, $ac=xz$, and $bc=yz$: \label{item:cancel_ab}
        \begin{align*}
          ab
            &= \brs{x(y+z)}\brs{y(x+z)}
            && \text{by \pref{item:cancel_abcd}}
          \\&= \brs{x(x+z)}\brs{y(y+z)}
            && \text{by \prope{commutative} property \xref{thm:lattice}}
          \\&= xy
            && \text{by \prope{absorptive} property \xref{thm:lattice}}
          \\
          ac
            &= \brs{x(y+z)}\brs{z(x+y)}
            && \text{by \pref{item:cancel_abcd}}
          \\&= \brs{x(x+y)}\brs{z(z+y)}
            && \text{by \prope{commutative} property \xref{thm:lattice}}
          \\&= xz
            && \text{by \prope{absorptive} property \xref{thm:lattice}}
          \\
          bc
            &= \brs{y(x+z)}\brs{z(x+y)}
            && \text{by \pref{item:cancel_abcd}}
          \\&= \brs{y(y+x)}\brs{z(z+x)}
            && \text{by \prope{commutative} property \xref{thm:lattice}}
          \\&= yz
            && \text{by \prope{absorptive} property \xref{thm:lattice}}
        \end{align*}

      \item Proof of some inequalities: \label{item:cancel_ineq}
        \begin{align*}
          a
            &= x(y+z)
            && \text{by \pref{item:cancel_abcd}}
          \\&\orel (x+y)(y+z)
            && \text{by definition of $\join$}
          \\&\orel (x+y)\brs{\brp{x+y}+z}
            && \text{by definition of $\join$}
          \\&= x+y
            && \text{by \prope{absorptive} property \xref{thm:lattice}}
          \\
          a
            &= x(y+z)
            && \text{by \pref{item:cancel_abcd}}
          \\&= x(z+y)
            && \text{by \prope{commutative} property \xref{thm:lattice}}
          \\&\orel (x+z)(z+y)
            && \text{by definition of $\join$}
          \\&\orel (x+z)\brs{\brp{x+z}+y}
            && \text{by definition of $\join$}
          \\&= x+z
            && \text{by \prope{absorptive} property \xref{thm:lattice}}
          \\
          b
            &= y(x+z)
            && \text{by \pref{item:cancel_abcd}}
          \\&\orel (x+y)(x+z)
            && \text{by definition of $\join$}
          \\&\orel (x+y)\brs{\brp{x+y}+z}
            && \text{by definition of $\join$}
          \\&= x+y
            && \text{by \prope{absorptive} property \xref{thm:lattice}}
          \\
          c
            &= z(x+y)
            && \text{by \pref{item:cancel_abcd}}
          \\&\orel (x+z)(x+y)
            && \text{by definition of $\join$}
          \\&\orel (x+z)\brs{\brp{x+z}+y}
            && \text{by definition of $\join$}
          \\&= x+z
            && \text{by \prope{absorptive} property \xref{thm:lattice}}
        \end{align*}

      \item Proof that $\latL$ is \prope{modular}:\label{item:cancel_mod}
        \begin{enumerate}
          \item Consider the following $N5$ lattice: \tbox{\includegraphics{graphics/lat5_n5_xyz.pdf}}%
        
          \item For the $N5$ lattice, the \prope{cancellation} property does not hold because
            \\
            $\begin{array}{@{\qquad}cc lcl ccN}
              \bid   &=& x + z &=& y+z &=& \bid & and \\
              \bzero &=& xz    &=& yz  &=& \bzero,
            \end{array}$
            \\
            but yet $x\ne y$.
        
          \item Because $N5$ does \emph{not} support the \prope{cancellation} property
            and by the hypothesis that $\latL$ \emph{does} support the cancellation property,
            $\latL$ therefore does \emph{not} contain $N5$.

          \item Because $\latL$ does not contain $N5$ and by \prefpp{thm:latm_n5},
            $\latL$ is \prope{modular}.
        \end{enumerate}%

      \item Proof that $a+b=a+c=b+c=d$: \label{item:cancel_deq}
        \begin{align*}
          a+b
            &= a + y(x+z)
            && \text{by definition of $c$ \xref{item:cancel_abcd}}
          \\&= \brp{a + y}\brp{x+z}
            && \text{by \hie{modularity}: \pref{item:cancel_ineq} and \pref{item:cancel_mod}}
          \\&= [x\brp{y+z} + y]\brp{x+z}
            && \text{by definition of $a$ \xref{item:cancel_abcd}}
          \\&= [y+x\brp{y+z}]\brp{x+z}
            && \text{by \prope{commutative} property \xref{thm:lattice}}
          \\&= \brp{y+x}\brp{y+z}\brp{x+z}
            && \text{by \hie{modularity}: \pref{item:cancel_ineq} and \pref{item:cancel_mod}}
          \\&= \brp{x+y}\brp{x+z}\brp{y+z}
            && \text{by \prope{commutative} property \xref{thm:lattice}}
          \\&= d
            && \text{by definition of $d$ \xref{item:cancel_abcd}}
          \\
          a+c
            &= a + z(x+y)
            && \text{by definition of $c$ \xref{item:cancel_abcd}}
          \\&= \brp{a + z}\brp{x+y}
            && \text{by \hie{modularity}: \pref{item:cancel_ineq} and \pref{item:cancel_mod}}
          \\&= [x\brp{y+z} + z]\brp{x+y}
            && \text{by definition of $a$ \xref{item:cancel_abcd}}
          \\&= [z+x\brp{y+z}]\brp{x+y}
            && \text{by \prope{commutative} property \xref{thm:lattice}}
          \\&= \brp{z+x}\brp{y+z}\brp{x+y}
            && \text{by \hie{modularity}: \pref{item:cancel_ineq} and \pref{item:cancel_mod}}
          \\&= \brp{x+y}\brp{x+z}\brp{y+z}
            && \text{by \prope{commutative} property \xref{thm:lattice}}
          \\&= d
            && \text{by definition of $d$ \xref{item:cancel_abcd}}
          \\
          b+c
            &= b + z(x+y)
            && \text{by definition of $c$ \xref{item:cancel_abcd}}
          \\&= \brp{b + z}\brp{x+y}
            && \text{by \hie{modularity}: \pref{item:cancel_ineq} and \pref{item:cancel_mod}}
          \\&= [y\brp{x+z} + z]\brp{x+y}
            && \text{by definition of $a$ \xref{item:cancel_abcd}}
          \\&= [z+y\brp{x+z}]\brp{x+y}
            && \text{by \prope{commutative} property \xref{thm:lattice}}
          \\&= \brp{z+y}\brp{x+z}\brp{x+y}
            && \text{by \hie{modularity}: \pref{item:cancel_ineq} and \pref{item:cancel_mod}}
          \\&= \brp{x+y}\brp{x+z}\brp{y+z}
            && \text{by \prope{commutative} property \xref{thm:lattice}}
          \\&= d
            && \text{by definition of $d$ \xref{item:cancel_abcd}}
        \end{align*}

      \item Proof that $(a+yz)+c=(b+xz)+c$ and $(a+yz)c=(b+xz)c$: \label{item:cancel_ayzc}
        \begin{align*}
          (a + yz) + c
            &= (a + bc)+c
            && \text{by \pref{item:cancel_ab}}
          \\&= a + \brp{c + cb}
            && \text{by \prope{commutative} property \xref{thm:lattice}}
          \\&= a + c
            && \text{by \prope{absorptive} property \xref{thm:lattice}}
          \\&= d
            && \text{by \pref{item:cancel_deq}}
          \\&= b + c
            && \text{by \pref{item:cancel_deq}}
          \\&= b + \brp{c + ca}
            && \text{by \prope{absorptive} property \xref{thm:lattice}}
          \\&= (b + ac) + c
            && \text{by \prope{commutative} property \xref{thm:lattice}}
          \\&= (b + xz) + c
            && \text{by \pref{item:cancel_ab}}
          \\
          (a+yz)c
            &= c(a+yz)
            && \text{by \prope{commutative} property \xref{thm:lattice}}
          \\&= c(a+bc)
            && \text{by \pref{item:cancel_ab}}
          \\&= (bc+a)c
            && \text{by \prope{commutative} property \xref{thm:lattice}}
          \\&= bc+ac
            && \text{by \hie{modularity}: \pref{item:cancel_ineq} and \pref{item:cancel_mod}}
          \\&= ac+bc
            && \text{by \prope{commutative} property \xref{thm:lattice}}
          \\&= (ac+b)c
            && \text{by \hie{modularity}: \pref{item:cancel_ineq} and \pref{item:cancel_mod}}
          \\&= (b+ac)c
            && \text{by \prope{commutative} property \xref{thm:lattice}}
          \\&= (b+xz)c
            && \text{by \pref{item:cancel_ab}}
        \end{align*}

      \item Proof that $a+yz=b+xz$: \label{item:cancel_ayz}
        by \pref{item:cancel_ayzc} and \hie{cancellation hypothesis}.

      \item Proof that $a+yz=d$: \label{item:cancel_ayzd}
        \begin{align*}
          a+yz
            &= (a+yz)+(a+yz)
            && \text{by \prope{idempotent} property \xref{thm:lattice}}
          \\&= (a+yz)+(b+xz)
            && \text{by \pref{item:cancel_ayz}}
          \\&= (a+bc)+(b+ac)
            && \text{by \pref{item:cancel_ab}}
          \\&= (a+ac)+ (b+bc)
            && \text{by \prope{commutative} property \xref{thm:lattice}}
          \\&= a+b
            && \text{by \prope{absorptive} property \xref{thm:lattice}}
          \\&= d
            && \text{by \pref{item:cancel_deq}}
        \end{align*}

      \item Proof that $z(x+y)=zx + zy$ (\prope{distributivity}):
        \begin{align*}
          z(x+y)
            &= c
            && \text{by \pref{item:cancel_abcd}}
          \\&= c(c+a)
            && \text{by \prope{absorptive} property \xref{thm:lattice}}
          \\&= c(a+c)
            && \text{by \prope{commutative} property \xref{thm:lattice}}
          \\&= cd
            && \text{by \pref{item:cancel_deq}}
          \\&= c(a+yz)
            && \text{by \pref{item:cancel_ayzd}}
          \\&= c(a+bc)
            && \text{by \pref{item:cancel_ab}}
          \\&= (bc+a)c
            && \text{by \prope{commutative} property \xref{thm:lattice}}
          \\&= bc+ac
            && \text{by \hie{modularity}: \pref{item:cancel_ineq} and \pref{item:cancel_mod}}
          \\&= yz+xz
            && \text{by \pref{item:cancel_ab}}
          \\&= zx+zy
            && \text{by \prope{commutative} property \xref{thm:lattice}}
        \end{align*}
    \end{enumerate}
\end{enumerate}
\end{proof}





%=======================================
\subsubsection{Algebraic characterizations}
%=======================================
%---------------------------------------
\begin{proposition}
\citetbl{
  \citorppg{birkhoff1948}{135}{136}{3540120440},
  \citorc{birkhoff1946}{???}
  }
\label{prop:lat_char_b1946}
%---------------------------------------
Let $\latA\eqd\latticed$ be an \hie{algebraic structure}.
\propbox{%
  \brb{\begin{array}{M}
    $\latA$ is a\\ 
    \structb{distributive lattice}
  \end{array}}
  \iff
  \brb{\begin{array}{FlclCD}
    1. & x \meet x              &=& x                                  & \forall x\in\setX     & and \\
    2. & x \join \bid           &=& \bid \join x = \bid                & \forall x\in\setX     & and \\
    3. & x \meet \bid           &=& \bid \meet x = x                   & \forall x\in\setX     & and \\
    4. & x \meet \brp{y\join z} &=& \brp{x\meet y}\join \brp{x\meet z} & \forall x,y,z\in\setX & and\\
    5. & \brp{y\join z}\meet x  &=& \brp{y\meet x}\join \brp{z\meet x} & \forall x,y,z\in\setX
  \end{array}}
  }
\end{proposition}

%---------------------------------------
\begin{proposition}
\citetbl{
  \citerpg{padmanabhan2008}{58}{9812834540},
  \citorppgc{birkhoff1948}{134}{135}{3540120440}{Ex.6}
  }
\label{prop:lat_char_b1948}
%---------------------------------------
Let $\latA\eqd\latticed$ be an \hie{algebraic structure}.
\propbox{
  \brb{\begin{array}{M}
    $\latA$ is a\\ 
    \structb{distributive lattice}
  \end{array}}
  \iff
  \brb{\begin{array}{FlclCD}
    1. & x \meet x              &=& x                                  & \forall x\in\setX     & and \\
    2. & x \join y              &=& y \join x                          & \forall x,y\in\setX   & and \\
    3. & x \meet y              &=& y \meet x                          & \forall x,y\in\setX   & and \\
    4. & x \meet (y\meet z)     &=& (x\meet y)\meet z                  & \forall x,y,z\in\setX & and \\
    5. & x \meet (x\join y)     &=& x                                  & \forall x,y\in\setX   & and \\
    6. & x \meet \brp{y\join z} &=& \brp{x\meet y}\join \brp{x\meet z} & \forall x,y,z\in\setX.
  \end{array}}
  }
\end{proposition}

%---------------------------------------
\begin{theorem}
\citetbl{
  \citerpg{padmanabhan2008}{59}{9812834540},
  \citorpcu{sholander1951}{28}{P1, P2}{http://books.google.com/books?hl=en\&lr=\&id=dKDdYkMCfAIC\&pg=PA28}
  }
\label{thm:lat_char_s1951}
%---------------------------------------
Let $\latA\eqd\latticed$ be an \hie{algebraic structure}.
\thmbox{
  \brb{\begin{array}{M}
    $\latA$ is a\\ 
    \structb{distributive lattice}
  \end{array}}
  \iff
  \brb{\begin{array}{FlclCD}
    1. & x \meet (x\join y)  &=& x                             & \forall x,y\in\setX   & and \\
    2. & x \meet (y \join z) &=& (z \meet x) \join (y \meet x) & \forall x,y,z\in\setX
  \end{array}}
  }
\end{theorem}
\begin{proof}
\begin{enumerate}
  \item Proof that $xx=x$ (\prope{meet idempotent} property):\label{item:s1951_meeti}
    \begin{align*}
      %\intertext{Proof that $xx=x$ (\prope{meet idempotent} property):\tagp{item:s1951_meeti}}
      xx
        &= x\brs{x(x+x)}
        && \text{by 1}
      \\&= x\brp{xx+xx}
        && \text{by 2}
      \\&= xxx + xxx
        && \text{by 2}
      \\&= xxx(x+x) + xxx(x+x)
        && \text{by 1}
      \\&= xx(xx+xx) + xx(xx+xx)
        && \text{by 2}
      \\&= xx + xx
        && \text{by 1}
      \\&= x(x + x)
        && \text{by 2}
      \\&= x
        && \text{by 1}
    \end{align*}

  \item Proof that $x+x=x$ (\prope{join idempotent} property):\label{item:s1951_joini}
    \begin{align*}
      x+x
        &= xx + xx
        && \text{by \prope{meet idempotent} property \xref{item:s1951_meeti}}
      \\&= x(x+x)
        && \text{by 2}
      \\&= x
        && \text{by 1}
    \end{align*}

  \item Proof that $xy=yx$ (\prope{meet commutative} property):\label{item:s1951_meetc}
    \begin{align*}
      xy
        &= xy + xy
        && \text{by \prope{join idempotent} property \xref{item:s1951_joini}}
      \\&= y(x+x)
        && \text{by 2}
      \\&= yx
        && \text{by \prope{join idempotent} property \xref{item:s1951_joini}}
    \end{align*}

  \item Proof that $x(y+z)=xy+xz$ (\prope{conjunctive distributive} property):\label{item:s1951_meetd}
    \begin{align*}
      x(y+z)
        &= yx+zx
        && \text{by 2}
      \\&= xy+xz
        && \text{by \prope{meet commutative} property \xref{item:s1951_meetc}}
    \end{align*}

  \item Proof that $x+xy=x$ (\prope{join absorptive} property):\label{item:s1951_joinab}
    \begin{align*}
      x
        &= x(x+y)
        && \text{by 1}
      \\&= yx + xx
        && \text{by 2}
      \\&= yx + x
        && \text{by \prope{meet idempotent} property \xref{item:s1951_meeti}}
      \\&= (yx + x)(yx+x)
        && \text{by \prope{meet idempotent} property \xref{item:s1951_meeti}}
      \\&= x(yx + x)+yx(yx+x)
        && \text{by 2}
      \\&= x(yx + x)+yx
        && \text{by 1}
      \\&= \brs{xx + (yx)x} + yx
        && \text{by 2}
      \\&= x\brp{yx+x} + yx
        && \text{by 2}
      \\&= x\brp{yx+xx} + yx
        && \text{by \prope{meet idempotent} property \xref{item:s1951_meeti}}
      \\&= x\brs{x(x+y)} + yx
        && \text{by 2}
      \\&= xx + yx
        && \text{by 1}
      \\&= x + yx
        && \text{by \prope{meet idempotent} property \xref{item:s1951_meeti}}
      \\&= x + xy
        && \text{by \prope{meet commutative} property \xref{item:s1951_meetc}}
    \end{align*}

  \item Proof that $x+y=y+x$ (\prope{join commutative} property):\label{item:s1951_joinc}
    \begin{align*}
      x+y
        &= (x+y)(x+y)
        && \text{by \prope{meet idempotent} property \xref{item:s1951_joini}}
      \\&= y(x+y)+x(x+y)
        && \text{by 2}
      \\&= y(x+y)+x
        && \text{by 1}
      \\&= (yy+xy)+x
        && \text{by 2}
      \\&= (y+xy)+x
        && \text{by \prope{meet idempotent} property \xref{item:s1951_joini}}
      \\&= (y+yx)+x
        && \text{by \prope{meet commutative} property \xref{item:s1951_meetc}}
      \\&= y+x
        && \text{by \prope{join absorptive} property \xref{item:s1951_joinab}}
    \end{align*}

  \item Proof that $(x+y)+z=x+(y+z)$ (\prope{join associative} property):\label{item:s1951_joinas}
    \begin{enumerate}
      \item Let $P\eqd(x+y)+z$ and $Q\eqd x+(y+z)$ \label{item:s1951_joinas_PQ}
      \item Proof that $Px=x$, $Py=y$, and $Pz=z$: \label{item:s1951_joinas_Px}
        \begin{align*}
          Px
            &= \brs{(x+y)+z}x
            && \text{by definition of $P$ \xref{item:s1951_joinas_PQ}}
          \\&= x\brs{(x+y)+z}
            && \text{by \prope{meet commutative} property \xref{item:s1951_meetc}}
          \\&= x(x+y)+xz
            && \text{by \prope{conjunctive distributive} property \xref{item:s1951_meetd}}
          \\&= x+xz
            && \text{by 1}
          \\&= x
            && \text{by \prope{join absorptive} property \xref{item:s1951_joinab}}
          \\
          Py
            &= \brs{(x+y)+z}y
            && \text{by definition of $P$ \xref{item:s1951_joinas_PQ}}
          \\&= y\brs{(x+y)+z}
            && \text{by \prope{meet commutative} property \xref{item:s1951_meetc}}
          \\&= y(x+y)+yz
            && \text{by \prope{conjunctive distributive} property \xref{item:s1951_meetd}}
          \\&= y(y+x)+yz
            && \text{by \prope{join commutative} property \xref{item:s1951_joinc}}
          \\&= y+yz
            && \text{by 1}
          \\&= y
            && \text{by \prope{join absorptive} property \xref{item:s1951_joinab}}
          \\
          Pz
            &= \brs{(x+y)+z}z
            && \text{by definition of $P$ \xref{item:s1951_joinas_PQ}}
          \\&= z\brs{(x+y)+z}
            && \text{by \prope{meet commutative} property \xref{item:s1951_meetc}}
          \\&= z\brs{z+(x+y)}
            && \text{by \prope{join commutative} property \xref{item:s1951_joinc}}
          \\&= z
            && \text{by 1}
        \end{align*}

      \item Proof that $Qx=x$, $Qy=y$, and $Qz=z$: \label{item:s1951_joinas_Qx}
        \begin{align*}
          Qx
            &= \brs{x+(y+z)}x
            && \text{by definition of $Q$ \xref{item:s1951_joinas_PQ}}
          \\&= x\brs{x+(y+z)}
            && \text{by \prope{meet commutative} property \xref{item:s1951_meetc}}
          \\&= x
            && \text{by 1}
          \\
          Qy
            &= \brs{x+(y+z)}y
            && \text{by definition of $Q$ \xref{item:s1951_joinas_PQ}}
          \\&= y\brs{x+(y+z)}
            && \text{by \prope{meet commutative} property \xref{item:s1951_meetc}}
          \\&= yx + y(y+z)
            && \text{by \prope{conjunctive distributive} property \xref{item:s1951_meetd}}
          \\&= yx + y
            && \text{by 2}
          \\&= y+yx
            && \text{by \prope{join commutative} property \xref{item:s1951_joinc}}
          \\&= y
            && \text{by \prope{join absorptive} property \xref{item:s1951_joinab}}
          \\
          Qz
            &= \brs{x+(y+z)}z
            && \text{by definition of $Q$ \xref{item:s1951_joinas_PQ}}
          \\&= z\brs{x+(y+z)}
            && \text{by \prope{meet commutative} property \xref{item:s1951_meetc}}
          \\&= zx + z(y+z)
            && \text{by \prope{conjunctive distributive} property \xref{item:s1951_meetd}}
          \\&= z(z+y) + zx  
            && \text{by \prope{join commutative} property \xref{item:s1951_joinc}}
          \\&= z + zx  
            && \text{by 1}
          \\&= z + zx  
            && \text{by 1}
          \\&= z
            && \text{by \prope{join absorptive} property \xref{item:s1951_joinab}}
        \end{align*}

      \item Proof that $(x+y)+z=x+(y+z)$:
        \begin{align*}
          (x + y) +z
            &= Qx + (Qy+Qz)
            && \text{by \pref{item:s1951_joinas_Qx}}
          \\&= Qx + Q(y+z)
            && \text{by \prope{conjunctive distributive} property \xref{item:s1951_meetd}}
          \\&= Q\brs{x + (y+z)}
            && \text{by \prope{conjunctive distributive} property \xref{item:s1951_meetd}}
          \\&= QP
            && \text{by definition of $Q$ \xref{item:s1951_joinas_PQ}}
          \\&= PQ
            && \text{by \prope{meet commutative} property \xref{item:s1951_meetc}}
          \\&= PQ
            && \text{by \prope{meet commutative} property \xref{item:s1951_meetc}}
          \\&= P\brs{x + (y+z)}
            && \text{by definition of $Q$ \xref{item:s1951_joinas_PQ}}
          \\&= Px + P(y+z)
            && \text{by \prope{conjunctive distributive} property \xref{item:s1951_meetd}}
          \\&= Px + (Py+Pz)
            && \text{by \prope{conjunctive distributive} property \xref{item:s1951_meetd}}
          \\&= x + (y+z)
            && \text{by \pref{item:s1951_joinas_Px}}
        \end{align*}   
    \begin{align*}
    \end{align*}
  \end{enumerate}

  \item Proof that $x+yz=(x+y)(x+z)$ (\prope{disjunctive distributive} property):\label{item:s1951_joind}
    \begin{align*}
      (x+y)(x+z)
        &= (x+y)x + (x+y)z
        && \text{by \prope{conjunctive distributive} property \xref{item:s1951_meetd}}
      \\&= x(x+y) + z(x+y)
        && \text{by \prope{meet commutative} property \xref{item:s1951_meetc}}
      \\&= x + z(x+y)
        && \text{by 1}
      \\&= x + (zx+zy)
        && \text{by \prope{conjunctive distributive} property \xref{item:s1951_meetd}}
      \\&= x + (xz+yz)
        && \text{by \prope{meet commutative} property \xref{item:s1951_meetc}}
      \\&= (x + xz)+yz
        && \text{by \prope{join associatiave} property \xref{item:s1951_joinas}}
      \\&= x +yz
        && \text{by \prope{join absorptive} property \xref{item:s1951_joinab}}
    \end{align*}

  \item Proof that $(xy)z=x(yz)$ (\prope{meet associative} property):\label{item:s1951_meetas}
    \begin{enumerate}
      \item Let $P\eqd(xy)z$ and $Q\eqd x(yz)$ \label{item:s1951_meetas_PQ}
      \item Proof that $P+x=x$, $P+y=y$, and $P+z=z$: \label{item:s1951_meetas_Px}
        \begin{align*}
          P+x
            &= (xy)z+x
            && \text{by definition of $P$ \xref{item:s1951_meetas_PQ}}
          \\&= x + (xy)z
            && \text{by \prope{join commutative} property \xref{item:s1951_joinc}}
          \\&= \brs{x + (xy)}\brs{x+z}
            && \text{by \prope{disjunctive distributive} property \xref{item:s1951_joind}}
          \\&= x\brs{x+z}
            && \text{by 1}
          \\&= x
            && \text{by 1}
          \\
          P+y
            &= (xy)z+y
            && \text{by definition of $P$ \xref{item:s1951_meetas_PQ}}
          \\&= y + (xy)z
            && \text{by \prope{join commutative} property \xref{item:s1951_joinc}}
          \\&= y + (yx)z
            && \text{by \prope{meet commutative} property \xref{item:s1951_meetc}}
          \\&= \brs{y + (yx)}\brs{y+z}
            && \text{by \prope{disjunctive distributive} property \xref{item:s1951_joind}}
          \\&= y\brs{y+z}
            && \text{by 1}
          \\&= y
            && \text{by 1}
          \\
          P+z
            &= (xy)z+z
            && \text{by definition of $P$ \xref{item:s1951_meetas_PQ}}
          \\&= z + (xy)z
            && \text{by \prope{join commutative} property \xref{item:s1951_joinc}}
          \\&= z + z(yx)
            && \text{by \prope{meet commutative} property \xref{item:s1951_meetc}}
          \\&= z
            && \text{by 1}
        \end{align*}

      \item Proof that $Q+x=x$, $Q+y=y$, and $Q+z=z$: \label{item:s1951_meetas_Qx}
        \begin{align*}
          Q+x
            &= x(yz)+x
            && \text{by definition of $Q$ \xref{item:s1951_meetas_PQ}}
          \\&= x + x(yz)
            && \text{by \prope{join commutative} property \xref{item:s1951_joinc}}
          \\&= x
            && \text{by 1}
          \\
          Q+y
            &= x(yz)+y
            && \text{by definition of $Q$ \xref{item:s1951_meetas_PQ}}
          \\&= y + x(yz)
            && \text{by \prope{join commutative} property \xref{item:s1951_joinc}}
          \\&= \brp{y + x}\brp{y+yz}
            && \text{by \prope{disjunctive distributive} property \xref{item:s1951_joind}}
          \\&= (y+x)y
            && \text{by 1}
          \\&= y(y+x)
            && \text{by \prope{meet commutative} property \xref{item:s1951_meetc}}
          \\&= y
            && \text{by 1}
          \\
          Q+z
            &= x(yz)+z
            && \text{by definition of $Q$ \xref{item:s1951_meetas_PQ}}
          \\&= z+x(yz)
            && \text{by \prope{join commutative} property \xref{item:s1951_joinc}}
          \\&= \brp{z + x}\brp{z+yz}
            && \text{by \prope{disjunctive distributive} property \xref{item:s1951_joind}}
          \\&= \brp{z + x}\brp{z+zy}
            && \text{by \prope{meet commutative} property \xref{item:s1951_meetc}}
          \\&= (z+x)z
            && \text{by 1}
          \\&= z(z+x)
            && \text{by \prope{meet commutative} property \xref{item:s1951_meetc}}
          \\&= z
            && \text{by 1}
        \end{align*}

      \item Proof that $(xy)z=x(yz)$: 
        \begin{align*}
          (xy)z
            &= \brs{(Q+x)(Q+y)}(Q+z)
            && \text{by \pref{item:s1951_meetas_Qx}}
          \\&= (Q+xy)(Q+z)
            && \text{by \prope{disjunctive distributive} property \xref{item:s1951_joind}}
          \\&= Q+(xy)z
            && \text{by \prope{disjunctive distributive} property \xref{item:s1951_joind}}
          \\&= Q+P
            && \text{by definition of $P$ \xref{item:s1951_meetas_PQ}}
          \\&= P+Q
            && \text{by \prope{join commutative} property \xref{item:s1951_joinc}}
          \\&= P+x(yz)
            && \text{by definition of $Q$ \xref{item:s1951_meetas_PQ}}
          \\&= (P+x)(P+yz)
            && \text{by \prope{disjunctive distributive} property \xref{item:s1951_joind}}
          \\&= (P+x)\brs{(P+y)(P+z)}
            && \text{by \prope{disjunctive distributive} property \xref{item:s1951_joind}}
          \\&= x(yz)
            && \text{by \pref{item:s1951_meetas_Px}}
        \end{align*}
    \end{enumerate}


  \item Proof that $\latA$ is a \prope{distributive} lattice:
    \begin{enumerate}
      \item Proof that $\latA$ is a lattice:
        \begin{enumerate}
          \item $\latA$ is \prope{idempotent}  by \pref{item:s1951_meeti} and \pref{item:s1951_joini}.
          \item $\latA$ is \prope{commutative} by \pref{item:s1951_meetc} and \pref{item:s1951_joinc}.
          \item $\latA$ is \prope{associative} by \pref{item:s1951_meetas} and \pref{item:s1951_joinas}.
          \item $\latA$ is \prope{absorptive}  by 1 and \pref{item:s1951_joinab}.
          \item Because $\latA$ is 
                \prope{idempotent}, \prope{commutative}, \prope{associative}, and \prope{absorptive},
                then by \prefpp{thm:lattice}, $\latA$ is a \hie{lattice}.
        \end{enumerate}

      \item Proof that $\latA$ is \prope{distributive}: 
            by \pref{item:s1951_meetd} and \prefpp{def:latd}.
    \end{enumerate}
\end{enumerate}
\end{proof}




%=======================================
\subsection{Properties}
%=======================================
Distributive lattices are a special case of modular lattices.
That is, all distributive lattices are modular, but not all
modular lattices are distributive (next theorem).
An example is the M3 lattice---
it is modular, but yet it is not \prope{distributive} \xref{lem:lat_M3}.
%---------------------------------------
\begin{theorem}
\label{thm:lat_dis=>mod}
\citetbl{
  \citorpg{birkhoff1948}{134}{3540120440},
  \citerpg{burris1981}{11}{0387905782}
  %\cithrp{burris2000}{13}
  }
%---------------------------------------
Let $\latticed$ be a lattice.
\thmbox{
  \latticed \text{ is \prope{distributive} }
  \qquad\impliesnotimpliedby\qquad
  \latticed \text{ is \prope{modular}. }
  }
\end{theorem}
\begin{proofns}
\begin{enumerate}
  \item Proof that distributivity $\implies$ modularity:
    \begin{align*}
      x \join (y\meet z)
        &= (x\join y) \meet (x\join z)
        && \text{by distributive hypothesis}
      \\&= y \meet (x\join z)
        && \text{by $x\orel y$ hypothesis}
    \end{align*}

  \item Proof that distributivity $\notimpliedby$ modularity: \\
    By \prefp{lem:lat_M3}, the $M3$ lattice is modular, but yet it is \prope{non-distributive}.
\end{enumerate}
\end{proofns}

%Because the more general case of lattices support
%the \hie{principle of duality} \xref{thm:lat_duality},
%and because the characterizations of distributive lattices given in \prefpp{thm:lat_dis}
%are \prope{self-dual},
%so distributive lattices also support the principle of duality (next theorem).
%%---------------------------------------
%\begin{theorem}[\thm{Principle of duality}]
%\label{thm:latd_duality}
%\citetbl{
%  \citerpg{padmanabhan2008}{54}{9812834540}
%  }
%%---------------------------------------
%Let $\latD\eqd\latticed$ be a \prope{distributive} lattice.
%\thmboxt{
%  $\brb{\parbox{6\tw/16}{\raggedright
%    $\phi$ is an identity on $\latD$ in terms of the operations
%    $\join$ and $\meet$}}$
%  $\qquad\implies\qquad$
%  $\opT\phi$ is also an identity on $\latD$
%  \\[1ex]
%  where the operator $\opT$ performs the following mapping on the operations of $\phi$:
%  \\\indentx $\join\rightarrow\meet,\qquad \meet\rightarrow\join$
%  }
%\end{theorem}
%\begin{proof}
%\begin{enumerate}
%  \item The 4 pairs of identities characterizing lattices in \prefpp{thm:lattice}
%        are prope{self-dual}.
%  \item The pair of identities (1) and (2) in \prefpp{thm:lat_dis} characterizing distributivity are duals of each other;
%        and (3) is self-dual.
%  \item Since these 5 pairs of identities characterizing distributive lattices
%        are \prope{self-dual}, the \hie{principle of duality} applies to distributive lattices.
%\end{enumerate}
%\end{proof}

%%---------------------------------------
%\begin{theorem}
%\citetbl{
%  \citerppg{maclane1999}{484}{485}{0821816462}
%  }
%\label{thm:lat_distrib_a=b}
%%---------------------------------------
%Let $\latL\eqd\latticed$ be a lattice.
%\thmbox{
%  \brb{\begin{array}{FlD}
%    1. & \text{$\latL$ is \prope{distributive}} & and \\
%    2. & x \join a = x \join b                  & and \\
%    3. & x \meet a = x \meet b
%  \end{array}}
%  \qquad\implies\qquad
%  \brb{a=b}
%  \qquad
%  \forall x,a,b\in\setX
%  }
%\end{theorem}
%\begin{proof}
%\begin{align*}
%  a
%    &= a \meet ( x \join a )
%    && \text{by \prope{absorption} property of lattices \xref{thm:lattice}}
%  \\&= a \meet ( x \join b )
%    && \text{by left hypothesis (2)}
%  \\&= (a \meet x) \join (a \meet b)
%    && \text{by \prope{distributive} hypothesis (1)}
%  \\&= (b \meet x) \join (a \meet b)
%    && \text{by left hypothesis (3)}
%  \\&= (b \meet x) \join (b \meet a)
%    && \text{by \prope{associative} property of lattices \xref{thm:lattice}}
%  \\&= b\meet (x \join a)
%    && \text{by \prope{distributive} hypothesis (1)}
%  \\&= b \meet (x \join b)
%    && \text{by left hypothesis (2)}
%  \\&= b
%    && \text{by \prope{absorption} property of lattices \xref{thm:lattice}}
%\end{align*}
%\end{proof}


%---------------------------------------
\begin{theorem}[\thmd{Birkhoff's Theorem}]
\citetbl{
  \citerpg{salii1988}{24}{0821845225}
  }
\label{thm:latd_birkhoff}
%---------------------------------------
Let $\latL\eqd\latticed$ be a lattice.
Let $\psetx$ be the \hi{power set} of some set $\setX$.
\thmbox{
  \brb{\begin{array}{M}%
    $\latL$ is\\
    \prope{distributive}
  \end{array}}
  \qquad\implies\qquad
  \brb{\begin{array}{M}
    $\latL$ is isomorphic to a sublattice of\quad
    $\lattice{\psetx}{\subseteq}{\setu}{\seti}$\\
    for some set $\setX$.
  \end{array}}
  }
\end{theorem}



%---------------------------------------
\begin{theorem}
\index{distributive laws}
\label{thm:latd_seq}
%---------------------------------------
Let $\latL\eqd\latticed$ be a lattice.
\thmbox{
  \brb{\begin{array}{M}%
    $\latL$ is\\
    \prope{distributive}%
  \end{array}}
  \implies
  \brb{\begin{array}{>{\ds}c @{\qquad} >{\ds}c}
    \text{tautology} & \text{dual}
    \\\hline
    \left( \meetop_{n=1}^N x_n \right) \lor y = \meetop_{n=1}^N (x_n \lor y)
    &
    \left( \joinop_{n=1}^N x_n \right) \land y = \joinop_{n=1}^N (x_n \land y)
  \end{array}}
  }
\end{theorem}
\begin{proof}
\begin{enumerate}
  \item Proof that
  $\left( \meetop_{n=1}^N x_n \right) \join y
     = \joinop_{n=1}^N (x_n \join y)
  $ (by induction):
    \begin{align*}
    \intertext{Proof for $N=1$ case:}
      \left( \meetop_{n=1}^{N=1} x_n \right) \join y
        &= x_1 \join y
        && \text{by definition of $\meet$}
      \\&= \meetop_{n=1}^{N=1} (x_n \join y)
        && \text{by definition of $\meet$}
      \\
    \intertext{Proof for $N=2$ case:}
      \left( \meetop_{n=1}^{N=2} x_n \right) \join y
        &= (x_1 \join y) \meet (x_2 \join y)
        && \text{by \prefp{thm:lat_dis}}
      \\&= \meetop_{n=1}^{N=2} (x_n \join y)
        && \text{by definition of $\meet$}
      \\
    \intertext{Proof that ($N$ case) $\implies$ ($N+1$ case):}
      \left( \meetop_{n=1}^{N+1} x_n \right) \join y
        &= \left[ \left( \meetop_{n=1}^{N} x_n \right) \meet x_{N+1} \right] \join y
        && \text{by definition of $\meet$}
      \\&= \left[\left(\meetop_{n=1}^{N} x_n \right) \join y\right] \meet (x_{N+1} \join y)
        && \text{by \prefp{thm:lat_dis}}
      \\&= \left[ \meetop_{n=1}^{N}(x_n \join y) \right] \meet (x_{N+1} \join y)
        && \text{by left hypothesis}
      \\&= \meetop_{n=1}^{N+1} (x_n \join y)
        && \text{by definition of $\meet$}
    \end{align*}

  \item Proof that
    $\left( \joinop_{n=1}^N x_n \right) \meet y
       = \meetop_{n=1}^N (x_n \meet y)
    $: by \hie{principle of duality} \xref{thm:duality}. %, \prefpp{thm:latd_duality}.

\end{enumerate}
\end{proof}


%=======================================
%\subsection{Functions on distributive lattices}
%=======================================


%---------------------------------------
\begin{theorem}
\citetbl{
  \citerpg{maclane1999}{484}{0821816462}
  }
\label{thm:lat_lin_distrib}
%---------------------------------------
Let $\latticed$ be a lattice.
\thmbox{
  \mcom{\opair{\setX}{\orel}}{ordered set} \text{ is \prope{linearly ordered}}
  \implies
  \mcom{\latticed}{lattice} \text{ is \prope{distributive}}
  }
\end{theorem}
\begin{proof}
\begin{align*}
  x \orel y \orel z &\implies x \meet (y\join z) &&= x \meet z &&= x &&= x \join x &&= (x\meet y) \join (x \meet z) \\
  x \orel z \orel y &\implies x \meet (y\join z) &&= x \meet y &&= x &&= x \join x &&= (x\meet y) \join (x \meet z) \\
  z \orel x \orel y &\implies x \meet (y\join z) &&= x \meet y &&= x &&= x \join z &&= (x\meet y) \join (x \meet z) \\
  y \orel z \orel x &\implies x \meet (y\join z) &&= x \meet z &&= z &&= y \join z &&= (x\meet y) \join (x \meet z) \\
  y \orel x \orel z &\implies x \meet (y\join z) &&= x \meet z &&= x &&= y \join x &&= (x\meet y) \join (x \meet z) \\
  z \orel y \orel x &\implies x \meet (y\join z) &&= x \meet y &&= y &&= y \join z &&= (x\meet y) \join (x \meet z) \\
\end{align*}
\end{proof}


%---------------------------------------
\begin{theorem}
\citetbl{
  \citerpg{maclane1999}{484}{0821816462}
  }
\label{thm:lat_Y_YX}
%---------------------------------------
Let $\setY^\setX \eqd \setn{\ff:\setX\to\setY}$
(the set of all functions from the set $\setX$ to the set $\setY$).
\thmboxt{
  $\ds\lattice{\setY}{\orela}{\joina}{\meeta} \text{ is a distributive lattice}
  \implies
  \lattice{\setY^\setX}{\orel}{\join}{\meet} \text{ is a distributive lattice}$
  \\
  \hspace{3ex}where
    $\ds\ff \orel \fg \iff \ff(x) \orela \fg(x) \quad\forall x\in\setX$
  }
\end{theorem}
\begin{proof}
\begin{align*}
  \brs{\ff\meet\brp{\fg\join\fh}}(x)
    &= \ff(x)\meeta\brp{\fg(x)\joina\fh(x)}
  \\&= \brp{\ff(x)\meeta\fg(x)} \joina \brp{\ff(x)\meeta\fh(x)}
    && \text{because $\lattice{\setY}{\orela}{\joina}{\meeta}$ is distributive}
  \\&= \brs{\ff\meet\fg}(x) \join \brs{\ff\meet\fh}(x)
    && \text{because $\lattice{\setY}{\orela}{\joina}{\meeta}$ is distributive}
\end{align*}
\end{proof}




%=======================================
\subsection{Examples}
%=======================================
%---------------------------------------
\begin{minipage}{\tw-26mm}%
%---------------------------------------
\begin{example}
\label{ex:lat_235_distrib}
\footnotemark
%---------------------------------------
For any pair of natural numbers $n,m\in\Zp$,
let $n|m$ represent the relation ``$m$ divides $n$",
$\lcm(n,m)$ the \hi{least common multiple} of $n$ and $m$, and
$\gcd(n,m)$ the \hi{greatest common divisor} of $n$ and $m$.
\exbox{\text{$\ds\lattice{\Zp}{|}{\gcd}{\lcm}$ is a \prope{distributive} lattice.}}
\end{example}%
\end{minipage}%
\citetblt{
  \citerpg{maclane1999}{484}{0821816462},
  %\citerp{huntington1933}{278}
  \citorpc{sheffer1920}{310}{footnote 1}
  }%
\hfill\tbox{\includegraphics{graphics/lat8_2e3_set235.pdf}}\hfill\mbox{}\\%
\\
\begin{proof}
\begin{enumerate}
  \item For all $m\in\Zp$, $m$ can be analyzed as a product of prime factors such that
    \[ m = 2^{\fe(1)} 3^{\fe(2)} 5^{\fe(3)} 7^{\fe(4)} \cdots p_k^{\fe(k)} \]
    where $\fe(n)$ is a function $\fe:\Zp\to\Znn$ expressing the number of prime factors $p_n$
    in $m$.
    For example,
      \[ 84 = 2^2 3^1 7^1 \qquad\implies\qquad \fe(1)=2,\, \fe(2)=1,\, \fe(3)=0,\, \fe(4)=1,\, \fe(5)=0,\, \fe(6)=0,\ldots \]

  \item Because $\Znn$ is a chain and by \prefp{thm:lat_lin_distrib},
        $\lattice{\Znn}{\orel}{\join}{\meet}$ is a distributive lattice
        where $\orel$ is the standard ordering on $\Znn$ and $\join$ and $\meet$ are defined
        in terms of $\orel$.

  \item Let $\Znn^\Zp$ represent the set of all functions $\fe:\Zp\to\Znn$.
        By \prefp{thm:lat_Y_YX}, $\lattice{\Znn^\Zp}{\orela}{\joina}{\meeta}$
        is also a distributive lattice where $\orela$ is defined in terms of $\orel$ as
        \[ \fe\orela\ff \quad\iff\quad \fe(n)\orel\ff(n) \quad \forall n\in\Zp. \]

  \item Again by \prefp{thm:lat_Y_YX},
        $\lattice{\Zp}{|}{\gcd}{\lcm}$ is a distributive lattice
        because $m|k$ if $e_m(n) \orela e_k(n)$.
\end{enumerate}
\end{proof}




%---------------------------------------
\begin{proposition}
\footnote{
  $l_n$: \citeoeis{A006966} | % lattices
  $m_n$: \citeoeis{A006981} | % modular
  $d_n$: \citeoeis{A006982} | % distributive
  $l_n$: \citer{heitzig2002} | %"Counting Finite Lattices"
  $m_n$: \citer{thakare2002}? | %"A structure theorem for dismantlable lattices and enumeration"
  $d_n$: \citerp{erne2002}{17} %"On the number of distributive lattices"
  }
\label{prop:lat_num_ldm}
\index{number of lattices}
%---------------------------------------
Let $\setX_n$ be a finite set with order $n=\seto{\setX_n}$.
Let $l_n$ be the number of unlabeled lattices on $\setX_n$,
    $m_n$ the number of unlabeled modular lattices on $\setX_n$,
and $d_n$ the number of unlabeled distributive lattices on $\setX_n$.
\thmbox{\begin{array}{l|*{17}{|c}}
  n   & 0 & 1 & 2 & 3 &  4 &  5 &   6 &   7 &    8 &      9  & 10 & 11 & 12 & 13 & 14\\
  \hline
  l_n  & 1 & 1 & 1 & 1 & 2 & 5 & 15 & 53 & 222 & 1078 & 5994 & 37622 &&&\\
  m_n  & 1 & 1 & 1 & 1 & 2 & 4 & 8  & 16 & 34  & 72   & 157  & 343   &&&\\
  d_n  & 1 & 1 & 1 & 1 & 2 & 3 & 5  & 8  & 15  & 26   & 47   & 82 & 151 & 269 & 494 % & 891 & 1639  & 2978 & 5483 & 10006 & \\
\end{array}}
\end{proposition}



%---------------------------------------
\begin{example}
\citetbl{
  \citerpp{erne2002}{4}{5}
  }
\label{ex:lat_set5_distrib}
%---------------------------------------
There are a total of five unlabeled lattices on a five element set;
and of these five, three are distributive \xref{prop:lat_num_ldm}.
\prefpp{ex:lat_set5} illustrated all five of the unlabeled lattices,
\prefpp{ex:lat_set5_mod} illustrated the 4 modular lattices,
and the following table illustrates the 3 distributive lattices.
Note that none of these lattices are \prope{complemented} 
(none are \prope{Boolean}\ifsxref{boolean}{def:boolean}).
%The lattice below each Hasse diagram is the smallest power set lattice 
%containing the Hasse diagram (see \prefp{thm:latd_birkhoff}).
\exbox{\begin{tabular}{cc|ccc}
  %\hline
   \mc{2}{c|}{\prope{non-distributive}}&\mc{3}{c}{\prope{distributive}}%
  \\\hline%
   \includegraphics{graphics/lat5_m3.pdf}
  &\includegraphics{graphics/lat5_n5.pdf}
  &\includegraphics{graphics/lat5_l2onm2.pdf}
  &\includegraphics{graphics/lat5_m2onl2.pdf}
  &\includegraphics{graphics/lat5_l5.pdf}
  %\\\hline
\end{tabular}}
\end{example}


%---------------------------------------
\begin{example}
\citetbl{
  \citerpp{erne2002}{4}{5}
  }
\label{ex:lat_set6_distrib}
%---------------------------------------
There are a total of 15 unlabeled lattices on a six element set;
and of these 15, five are distributive \xref{prop:lat_num_ldm}.
\prefpp{ex:lat_set6} illustrated all 15 of the unlabeled lattices,
\prefpp{ex:lat_set6_mod} illustrated the 8 modular lattices,
and the following illustrates the 5 distributive lattices.\\
Note that none of these lattices are \prope{complemented} 
(none are \prope{Boolean}\ifsxref{boolean}{def:boolean}).
\exbox{\begin{tabular}{ccccc}%
   \mc{5}{c}{distributive lattices on 6 element sets}%
  \\\hline%
   \includegraphics{graphics/lat6_o6slash.pdf}%
  &\includegraphics{graphics/lat6_l3onm2.pdf}%
  &\includegraphics{graphics/lat6_l2onm2onl2.pdf}%
  &\includegraphics{graphics/lat6_m2onl3.pdf}%
  &\includegraphics{graphics/lat6_l6.pdf}%
\end{tabular}}
%\input{../common/latd6.inp}
\end{example}


%---------------------------------------
\begin{example}
\citetbl{
  \citerpp{erne2002}{4}{5}
  }
\label{ex:lat_set7_dstrb}
%---------------------------------------
There are a total of 53 unlabeled lattices on a seven element set;
and of these, 8 are \prope{distributive} \xref{prop:lat_num_ldm}.
\prefpp{ex:lat_set7} illustrated all 53 of the unlabeled lattices,
\prefpp{ex:lat_set7_mod} illustrated the 16 \prope{modular} lattices,
and the following illustrates the 8 distributive lattices.
Note that none of these lattices are \prope{complemented} 
(none are \prope{Boolean}\ifsxref{boolean}{def:boolean}).
\exbox{\begin{tabular}{cccccccc}%
   \mc{8}{c}{distributive lattices on 7 element sets}%
  \\\hline%
   \includegraphics{graphics/lat7_m2onm2.pdf}%
  &\includegraphics{graphics/lat7_l2ono6slash.pdf}%
  &\includegraphics{graphics/lat7_o6slashonl2.pdf}%
  &\includegraphics{graphics/lat7_l4onm2.pdf}%
  &\includegraphics{graphics/lat7_l3onm2onl2.pdf}%
  &\includegraphics{graphics/lat7_l2onm2onl3.pdf}%
  &\includegraphics{graphics/lat7_m2onl4.pdf}%
  &\includegraphics{graphics/lat7_l7.pdf}%
\end{tabular}}
\end{example}




