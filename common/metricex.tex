%============================================================================
% Daniel J. Greenhoe
% LaTeX file
%============================================================================

%=======================================
%\chapter{Examples of Metrics}
%=======================================





%---------------------------------------
% Taxi-cab metric
%---------------------------------------
%---------------------------------------
\begin{example}[Taxi-cab metric]
\label{ex:ms_taxi}
\citetbl{
  \citerp{deza2006}{240} \\
  \citerp{dieudonne1969}{29}
%\footnotetext{\cite[page 2]{norfolk}}
  }
\index{taxi-cab metric}
\index{metrics!taxi-cab}
\\
%---------------------------------------
%The unit ball in $(\R^n, \metricn)$ is illustrated in the figure to the left.
\begin{minipage}{3\tw/4-3mm}%
\exbox{\begin{tabular}{>{$\imark$}ll}
     & $\ds\metric{\vx}{\vy} \eqd \sum_{i=1}^n \abs{x_i-y_i}$ \emph{is a metric}.
  \\ & $\metricn$ \emph{is generated by a norm}.
  \\ & $\ball{0}{1}$ in $\opair{\R^n}{\metricn}$ \emph{is convex}.
  \\ & $\diam\ball{\vx}{r}= 2r$
  \end{tabular}}
\end{minipage}%
\hfill%
\begin{minipage}{\tw/4}%
  \begin{center}
  \begin{fsL}
  \setlength{\unitlength}{\tw/300}
  \begin{picture}(300,300)(-130,-130)
    \thicklines
    \color{axis}%
      \put(-130,   0){\line(1,0){260} }%
      \put(   0,-130){\line(0,1){260} }%
      \put(-100, -10){\line(0,1){20} }%
      \put( 100, -10){\line(0,1){20} }%
      \put( -10,-100){\line(1,0){20} }%
      \put( -10, 100){\line(1,0){20} }%
      \put( -10, 110){\makebox(0,0)[br]{$1$} }%
      \put( -10,-110){\makebox(0,0)[tr]{$-1$} }%
      \put(-110,  10){\makebox(0,0)[br]{$-1$} }%
      \put( 110,  10){\makebox(0,0)[bl]{$1$} }%
      %\put( 140,   0){\makebox(0,0)[l]{$x$}}%
      %\put(   0, 140){\makebox(0,0)[b]{$y$}}%
    \color{blue}%
      \put(-100,   0){\line( 1, 1){100} }%
      \put(-100,   0){\line( 1,-1){100} }%
      \put( 100,   0){\line(-1, 1){100} }%
      \put( 100,   0){\line(-1,-1){100} }%
  \end{picture}
  \end{fsL}
  \end{center}
\end{minipage}%
\end{example}

\begin{proof}
\begin{enumerate}
  \item Proof that $\metricn$ is a metric:
    \begin{enumerate}
      \item By \prefpp{ex:d_usual}, $\fp(x,y)=\abs{x-y}$ is a metric.
      \item By the definition of $\metricn$, 
            $\metric{\vx}{\vy} \eqd \sum_{i=1}^n \abs{x_i-y_i}$
      \item And so $\metricn$ is a \structe{Fr{\'e}chet product metric} and is a \structe{metric} by \prefpp{thm:met_d=sumpi}.
    \end{enumerate}

  \item Proof $\metricn$ is generated by a norm: 
    \begin{enumerate}
      \item $\metricn$ is generated by a norm if and only if
            $\norm{\vx}\eqd\sum_{i=1}^n \abs{x_i}$ is a norm.
      \item Proof that $\norm{\vx}\eqd\sum_{i=1}^n \abs{x_i}$ is a norm
            is given by \prefpp{ex:norms}.
    \end{enumerate}

  \item Proof that the ball is convex: \\
    By \prefpp{thm:norm_convex}, all metrics generated by a norm are convex.
\end{enumerate}
\end{proof}





%---------------------------------------
% Euclidean metric
%---------------------------------------
%---------------------------------------
\begin{example}[Euclidean metric]
\citetbl{
  %\citerp{norfolk}{2} \\
  \citerp{dieudonne1969}{29}
  }
\label{ex:ms_euclidean}
\index{Euclidean metric}
\index{metrics!Euclidean}
\\
%---------------------------------------
\begin{minipage}{3\tw/4-3mm}%
\exbox{\begin{tabular}{>{$\imark$}ll}
     & $\metric{\vx}{\vy} \eqd \sqrt{\sum_{i=1}^n |x_i-y_i|^2}$
       \emph{is a metric}.
  \\ & $\metricn$ \emph{is generated by a norm}.
  \\ & $\ball{0}{1}$ in $\brp{ \R^2,\, \metricn }$ \emph{is convex}.
  \\ & $\diam\ball{\vx}{r}= 2r$
  \end{tabular}}
\end{minipage}%
\hfill%
\begin{minipage}{\tw/4}%
  \begin{center}
  \begin{fsL}
  \setlength{\unitlength}{\tw/300}
  \begin{picture}(300,300)(-130,-130)
    \thicklines
    \color{axis}%
      \put(-130,   0){\line(1,0){260} }%
      \put(   0,-130){\line(0,1){260} }%
      \put(-100, -10){\line(0,1){20} }%
      \put( 100, -10){\line(0,1){20} }%
      \put( -10,-100){\line(1,0){20} }%
      \put( -10, 100){\line(1,0){20} }%
      \put( -10, 110){\makebox(0,0)[br]{$1$} }%
      \put( -10,-110){\makebox(0,0)[tr]{$-1$} }%
      \put(-110,  10){\makebox(0,0)[br]{$-1$} }%
      \put( 110,  10){\makebox(0,0)[bl]{$1$} }%
      %\put( 140,   0){\makebox(0,0)[l]{$x$}}%
      %\put(   0, 140){\makebox(0,0)[b]{$y$}}%
    \color{blue}%============================================================================
% NCTU - Hsinchu, Taiwan
% LaTeX File
% Daniel Greenhoe
%
% Unit circle with radius 100
%============================================================================

\qbezier( 100,   0)( 100, 41.421356)(+70.710678,+70.710678) % 0   -->1pi/4
\qbezier(   0, 100)( 41.421356, 100)(+70.710678,+70.710678) % pi/4-->2pi/4
\qbezier(   0, 100)(-41.421356, 100)(-70.710678,+70.710678) %2pi/4-->3pi/4
\qbezier(-100,   0)(-100, 41.421356)(-70.710678,+70.710678) %3pi/4--> pi 
\qbezier(-100,   0)(-100,-41.421356)(-70.710678,-70.710678) % pi  -->5pi/4
\qbezier(   0,-100)(-41.421356,-100)(-70.710678,-70.710678) %5pi/4-->6pi/4
\qbezier(   0,-100)( 41.421356,-100)( 70.710678,-70.710678) %6pi/4-->7pi/4
\qbezier( 100,   0)( 100,-41.421356)( 70.710678,-70.710678) %7pi/4-->2pi


%
  \end{picture}
  \end{fsL}
  \end{center}
\end{minipage}%
\end{example}

\begin{proof}
\begin{enumerate}
  \item Proof that $\metricn$ is a metric:
    \begin{enumerate}
      \item By \prefpp{ex:d_usual}, $\fp(x,y)=\abs{x-y}$ is a metric.
      \item By the definition of $\metricn$, 
            $\metric{\vx}{\vy} \eqd \sqrt{\sum_{i=1}^n |x_i-y_i|^2}$
      \item And so $\metricn$ is a \structe{Fr{\'e}chet product metric} and is a \structe{metric} by \prefpp{thm:met_d=sumpi}.

    \end{enumerate}

  \item Proof $\metricn$ is generated by a norm: 
    \begin{enumerate}
      \item $\metricn$ is generated by a norm if and only if
            $\norm{\vx}\eqd\sqrt{\sum_{i=1}^n |x_i|^2}$ is a norm.
      \item Proof that $\norm{\vx}=\sqrt{\sum_{i=1}^n |x_i|^2}$ is a norm
            is given by \prefpp{ex:norms}.
    \end{enumerate}

  \item Proof that the ball is convex: \\
    By \prefpp{thm:norm_convex}, all metrics generated by a norm are convex.
\end{enumerate}
\end{proof}





%---------------------------------------
% Sup metric
%---------------------------------------
%---------------------------------------
\begin{example}[Sup metric]
\label{ex:ms_sup}
%\footnotemark
\index{sup metric}
\index{metrics!sup}
\mbox{}\\
%---------------------------------------
\begin{minipage}{3\tw/4}%
\exbox{\begin{tabular}{>{$\imark$}ll}
     & $\ds\metric{\vx}{\vy} \eqd \max\set{|x_i-y_i|}{i=1,2,\dots,n}$
       \emph{is a metric}.
  \\ & $\metricn$ \emph{is generated by a norm}.
  \\ & $\ball{0}{1}$ in $\opair{\R^n}{\metricn}$ \emph{is convex}.
  \\ & $\diam\ball{\vx}{r}= 2\sqrt{2}r$
  \end{tabular}}
\end{minipage}%
%\footnotetext{\cite[page 2]{norfolk}}
\begin{minipage}{\tw/4}%
  \begin{center}
  \begin{fsL}
  \setlength{\unitlength}{\tw/300}
  \begin{picture}(300,300)(-130,-130)
    \thicklines
    \color{axis}%
      \put(-130,   0){\line(1,0){260} }%
      \put(   0,-130){\line(0,1){260} }%
      \put(-100, -10){\line(0,1){20} }%
      \put( 100, -10){\line(0,1){20} }%
      \put( -10,-100){\line(1,0){20} }%
      \put( -10, 100){\line(1,0){20} }%
      \put( -10, 110){\makebox(0,0)[br]{$1$} }%
      \put( -10,-110){\makebox(0,0)[tr]{$-1$} }%
      \put(-110,  10){\makebox(0,0)[br]{$-1$} }%
      \put( 110,  10){\makebox(0,0)[bl]{$1$} }%
      %\put( 140,   0){\makebox(0,0)[l]{$x$}}%
      %\put(   0, 140){\makebox(0,0)[b]{$y$}}%
    \color{blue}%
      \put(-100,-100){\line( 1, 0){200} }%
      \put(-100,-100){\line( 0, 1){200} }%
      \put( 100, 100){\line(-1, 0){200} }%
      \put( 100, 100){\line( 0,-1){200} }%
  \end{picture}
  \end{fsL}
  \end{center}
\end{minipage}%
\end{example}

\begin{proof}
\begin{enumerate}
  \item Proof that $\metricn$ is a metric:
    \begin{enumerate}
      \item By \prefpp{ex:d_usual}, $\fp(x,y)=\abs{x-y}$ is a metric.
      \item By the definition of $\metricn$, 
            $\metric{\vx}{\vy} \eqd \max\set{\fp(x_i,y_i)}{i=1,2,\ldots n}$
      \item And so $\metricn$ is a \structe{Fr{\'e}chet product metric} and is a \structe{metric} by \prefpp{thm:met_d=sumpi}.
    \end{enumerate}

  \item Proof $\metricn$ is generated by a norm: 
    \begin{enumerate}
      \item $\metricn$ is generated by a norm if and only if
            $\norm{\vx}\eqd\max\set{\abs{x_i}}{i=1,2,\ldots n}$ is a norm.
      \item Proof that $\norm{\vx}\eqd\max\set{\abs{x_i}}{i=1,2,\ldots n}$ is a norm
            is given by \prefpp{ex:norms}.
    \end{enumerate}

  \item Proof that the ball is convex: \\
    By \prefpp{thm:norm_convex}, all metrics generated by a norm are convex.
\end{enumerate}
\end{proof}




\ifexclude{wsd}{
%---------------------------------------
% Parabolic metric
%---------------------------------------
%---------------------------------------
\begin{example}[Parabolic metric]
\citetbl{
  \citerp{norfolk}{2} \\
  \url{http://groups.google.com/group/sci.math/msg/c0eb7e19631c31ea}
  }
\label{ex:ms_parabolic}
\index{metrics!parabolic}
\index{parabolic metric}
\\
%---------------------------------------
\begin{minipage}{3\tw/4-3mm}%
Let $\setX$ be a set and
$\vx\eqd\tuplexn{x_k\in\setX}$ and 
$\vy\eqd\tuplexn{y_k\in\setX}$
be tuples on $\setX$.
\exbox{\begin{tabular}{ll}
  1. & $\ds\metric{\vx}{\vy}\eqd \sum_{i=1}^n \sqrt{\abs{x_i-y_i}}$
       \emph{is a metric}.
  \\
  2. & $\metricn$ is \emph{not generated by a norm}.
  \\
  3. & $\ball{0}{1}$ in $\opair{\R^n}{\metricn}$ is \emph{not convex}.
  \end{tabular}}
\end{minipage}%
\hfill%
\begin{minipage}{\tw/4}%
  \begin{center}
  \begin{fsL}
  \setlength{\unitlength}{\tw/300}
  \begin{picture}(300,300)(-130,-130)%
    %{\color{graphpaper}\graphpaper[10](-150,-150)(300,300)}%
    \thicklines%
    \color{axis}%
      \put(-130,   0){\line(1,0){260} }%
      \put(   0,-130){\line(0,1){260} }%
      \put( 140,   0){\makebox(0,0)[l]{$x$}}%
      \put(   0, 140){\makebox(0,0)[b]{$y$}}%
      \put(-100, -10){\line(0,1){20} }%
      \put( 100, -10){\line(0,1){20} }%
      \put( -10,-100){\line(1,0){20} }%
      \put( -10, 100){\line(1,0){20} }%
      \put( -15, 100){\makebox(0,0)[r]{$+1$} }%
      \put( -15,-100){\makebox(0,0)[r]{$-1$} }%
      \put(-100, -15){\makebox(0,0)[t]{$-1$} }%
      \put( 100, -15){\makebox(0,0)[t]{$+1$} }%
    \color{blue}%
      \qbezier( 100,0)(0,0)(0, 100)%
      \qbezier( 100,0)(0,0)(0,-100)%
      \qbezier(-100,0)(0,0)(0,-100)%
      \qbezier(-100,0)(0,0)(0, 100)%
    \color{red}%
      \qbezier[20]( 75,0)(37.5,37.5)(0, 75)%
      \put( 10,90){\makebox(0,0)[bl]{$\scriptstyle \lambda\vv_\circ+(1-\lambda)\vw_\circ$} }%
      \put( 90,90){\vector(-1,-1){50} }%
    \color{black}%
      \put(-100,-100){\vector(1,1){125} }%
      \put(-120, -110){\makebox(0,0)[tl]{$\scriptstyle x^2+y^2-2xy-2x-2y+1=0$} }%
      \put(-120, -140){\makebox(0,0)[tl]{\scriptsize(parabolic equation)} }%
  \end{picture}
  \end{fsL}
  \end{center}
\end{minipage}%
\end{example}

\begin{proof}
\begin{enumerate}
  \item Proof that $\metricn$ is a metric:
    \begin{align*}
      \intertext{$\imark$ Proof that $\vx=\vy\implies\metric{\vx}{\vy}=0$:}
      \metric{\vx}{\vy}
        &= \metric{\vx}{\vx}
        && \text{by left hypothesis}
      \\&= \sum_{i=1}^n \sqrt{\abs{x_i-x_i}}
        && \text{by definition of $\metricn$}
      \\&= 0
      \\
      \intertext{$\imark$ Proof that $\vx=\vy\impliedby\metric{\vx}{\vy}=0$:}
      0
        &= \metric{\vx}{\vy}
        && \text{by right hypothesis}
      \\&= \sum_{i=1}^n \sqrt{\abs{x_i-y_i}}
        && \text{by definition of $\metricn$}
      \\\implies& x_1=x_2 \text{ and } y_1=y_2
        && \text{because $\absn$ is positive}
      \\\implies& \vx=\vy
        && \text{by definitions of $\vv$ and $\vw$}
      \\
      \intertext{$\imark$ Proof that $\metric{\vx}{\vy}\le\metric{\vx}{\vz}+\metric{\vz}{\vy}$:}
      \metric{\vx}{\vy}
        &= \sum_{i=1}^n \sqrt{\abs{x_i-y_i}}
        && \text{by definition of $\metricn$}
      \\&\le \sum_{i=1}^n \sqrt{\abs{x_i-z_i} + \abs{z_i-y_i}} 
        &&   \text{by triangle inequality property of usual metric $\absn$}
      \\&=   \sum_{i=1}^n \sqrt{2}\,\sqrt{\frac{1}{2}\abs{x_i-z_i} + \frac{1}{2}\abs{z_i-y_i}} 
      \\&=   \sum_{i=1}^n \sqrt{2}\brp{\frac{1}{2}\sqrt{\abs{x_i-z_i}} + \frac{1}{2}\sqrt{\abs{z_i-y_i}} }
        &&   \text{by Jensen's inequality \prefpo{thm:jensen_ineq}}
      \\&=   \frac{\sqrt{2}}{2} \sum_{i=1}^n \brp{\sqrt{\abs{x_i-z_i}} + \sqrt{\abs{z_i-y_i}} }
      \\&\le \sum_{i=1}^n \sqrt{\abs{z_i-x_i}} + \sum_{i=1}^n \sqrt{\abs{z_i-y_i}}
      \\&=   \metric{\vz}{\vx} + \metric{\vz}{\vy}
    \end{align*}


  \item Proof $\metricn$ is not generated by a norm:
    \begin{align*}
      \norm{\alpha(\vv-\vw)}
        &= \norm{\alpha\vv-\alpha\vw}
      \\&= \metric{\alpha\vv}{\alpha\vw}
        && \text{for some function $\normn$}
      \\&= \sqrt{\abs{\alpha x_1- \alpha x_2}} + \sqrt{\abs{\alpha y_1- \alpha y_2}}
        && \text{by definition of $\metricn$}
      \\&= \sqrt{\abs{\alpha}\abs{ x_1- x_2}} + \sqrt{\abs{\alpha}\abs{y_1- y_2}}
      \\&= \sqrt{\abs{\alpha}}\brp{\sqrt{\abs{ x_1- x_2}} + \sqrt{\abs{y_1- y_2}}}
      \\&= \sqrt{\abs{\alpha}} \metric{\vv}{\vw}
        && \text{by definition of $\metricn$}
      \\&= \sqrt{\abs{\alpha}} \norm{\vv-\vw}
        && \text{by definition of function $\normn$}
      \\&\ne  \abs{\alpha} \norm{\vv-\vw}
      \\\implies & \normn \text{ is not a norm.}
        && \text{by homogeneous property of norms \prefpo{def:norm}}
    \end{align*}

  \item Proof that the ball is not convex:
Let $\vv\eqd\brp{\frac{3}{4},\,0}$ and $\vw\eqd\brp{0,\,\frac{3}{4}}$.
\begin{align*}
  \metric{\vzero}{\frac{1}{2}\vv+(1-\frac{1}{2})\vw}
    &= \metric{\vzero}{\frac{1}{2}\vv+\frac{1}{2}\vw}
    && \text{let $\lambda=\frac{1}{2}$}
  \\&= \metric{\brp{0,\,0}}{\frac{1}{2}\brp{\frac{3}{4},\,0}+\frac{1}{2}\brp{0,\,\frac{3}{4}}}
    && \text{by definition of $\vv$ and $\vw$}
  \\&= \metric{\brp{0,\,0}}{\brp{\frac{3}{8},\,0}+\brp{0,\,\frac{3}{8}}}
  \\&= \metric{\brp{0,\,0}}{\brp{\frac{3}{8},\,\frac{3}{8}}}
  \\&= \sqrt{\abs{0-\frac{3}{8}}} + \sqrt{\abs{0-\frac{3}{8}}}
    && \text{by definition of $\metricn$}
  \\&= 2\sqrt{\frac{3}{8}}
  \\&= \frac{2}{2} \sqrt{\frac{3}{2}}
  \\&> 1
\end{align*}
\end{enumerate}
\end{proof}




%---------------------------------------
% Inverse tangent---check diagram below
%---------------------------------------
%---------------------------------------
\begin{example}[Inverse tangent metric]
\label{ex:ms_atan}
\citetbl{
  \citerp{copson1968}{25}\\
  \citerpg{khamsi2001}{14}{0471418250}
  }
\index{metrics!inverse tangent}
\index{inverse tangent metric}
\\
%---------------------------------------
\begin{minipage}{3\tw/4-3mm}%
Let $\setX$ be a set and
$\vx\eqd\tuplexn{x_k\in\setX}$ and 
$\vy\eqd\tuplexn{y_k\in\setX}$
be sequences on $\setX$.
\exbox{\text{
  $\ds\metric{\vx}{\vy}\eqd \sum_{i=1}^n \abs{\arctan x_i - \arctan y_i}$\quad
       \emph{is a \structe{metric}}.
  %\\
  %2. & $\metricn$ is \emph{not generated by a norm}.
  %\\
  %3. & $\ball{0}{1}$ in $\opair{\R^n}{\metricn}$ is \emph{not convex}.
  }}
\end{minipage}%
\hfill%
\begin{minipage}{\tw/4}%
  \begin{center}
  \begin{fsL}
  \setlength{\unitlength}{\tw/300}
  \begin{picture}(300,300)(-130,-130)%
    %{\color{graphpaper}\graphpaper[10](-150,-150)(300,300)}%
    \thicklines%
    \color{axis}%
      \put(-130,   0){\line(1,0){260} }%
      \put(   0,-130){\line(0,1){260} }%
      \put( 140,   0){\makebox(0,0)[l]{$x$}}%
      \put(   0, 140){\makebox(0,0)[b]{$y$}}%
      \put(-100, -10){\line(0,1){20} }%
      \put( 100, -10){\line(0,1){20} }%
      \put( -10,-100){\line(1,0){20} }%
      \put( -10, 100){\line(1,0){20} }%
      \put( -15, 100){\makebox(0,0)[r]{$+1$} }%
      \put( -15,-100){\makebox(0,0)[r]{$-1$} }%
      \put(-100, -15){\makebox(0,0)[t]{$-1$} }%
      \put( 100, -15){\makebox(0,0)[t]{$+1$} }%
    \color{blue}%
      \qbezier( 100,0)(0,0)(0, 100)%
      \qbezier( 100,0)(0,0)(0,-100)%
      \qbezier(-100,0)(0,0)(0,-100)%
      \qbezier(-100,0)(0,0)(0, 100)%
    \color{red}%
      \qbezier[20]( 75,0)(37.5,37.5)(0, 75)%
      \put( 10,90){\makebox(0,0)[bl]{$\scriptstyle \lambda\vv_\circ+(1-\lambda)\vw_\circ$} }%
      \put( 90,90){\vector(-1,-1){50} }%
    %\color{black}%
      %\put(-100,-100){\vector(1,1){125} }%
      %\put(-120, -110){\makebox(0,0)[tl]{$\scriptstyle x^2+y^2-2xy-2x-2y+1=0$} }%
      %\put(-120, -140){\makebox(0,0)[tl]{\scriptsize(parabolic equation)} }%
  \end{picture}
  \end{fsL}
  \end{center}
\end{minipage}%
\end{example}
\begin{proof}
\begin{enumerate}
  \item The function $\metric{x}{y}\eqd\abs{x-y}$ is a \structe{metric} (the \hie{usual metric}, \prefp{ex:d_usual}).
  \item The function  $\fg(x)\eqd\arctan(x)$ is \prope{injective} in $\clFrr$.
  \item Therefore, $\metricn$ is a \structe{Pullback metric} (or \structe{$\fg$-transform metric}), 
        and by \prefpp{thm:met_sumpf}, $\metricn$ is a \structe{metric}.
\end{enumerate}
\end{proof}

%---------------------------------------
% Exponential
%---------------------------------------
%---------------------------------------
\begin{example}[Exponential metric]
\label{ex:ms_32x}
\index{metrics!exponential}
\index{exponential metric}
\mbox{}\\
%---------------------------------------
\begin{minipage}{3\tw/4}%
Let $\setX$ be a set and
$\vx\eqd\tuplexn{x_k\in\setX}$ and 
$\vy\eqd\tuplexn{y_k\in\setX}$
be sequences on $\setX$.
\exbox{\begin{tabular}{ll}
  1. & $\ds\metric{\vx}{\vy}\eqd 2\sum_{i=1}^n  \abs{\brp{\frac{3}{2}}^{x_i} - \brp{\frac{3}{2}}^{y_i}}$
       \emph{is a metric}.
  \\
  2. & $\metricn$ is \emph{not generated by a norm}.
  \\
  3. & $\ball{\theta}{1}$ in $\opair{\R^n}{\metricn}$ is \emph{not convex}.
  \end{tabular}}
\end{minipage}%
\hfill%
\begin{minipage}{\tw/4}%
  \begin{center}
  \begin{fsL}
  \setlength{\unitlength}{\tw/400}
  \begin{picture}(400,400)(-200,-200)%
    %{\color{graphpaper}\graphpaper[10](-150,-150)(300,300)}%
    \thicklines%
    \color{axis}%
      \put(-200,   0){\line(1,0){400} }%
      \put(   0,-200){\line(0,1){400} }%
      %\put( 140,   0){\makebox(0,0)[l]{$x$}}%
      %\put(   0, 140){\makebox(0,0)[b]{$y$}}%
      \put( 100, -10){\line(0,1){20} }%
      \put(-100, -10){\line(0,1){20} }%
      \put( -10,-100){\line(1,0){20} }%
      \put(-171, -10){\line(0,1){20} }%
      \put( -10,-171){\line(1,0){20} }%
      \put( -10, 100){\line(1,0){20} }%
      \put( -15, 100){\makebox(0,0)[r]{$+1$} }%
      \put( -15,-100){\makebox(0,0)[r]{$-1$} }%
      \put(-181, -15){\makebox(0,0)[tr]{$\frac{-\ln2}{\ln3-\ln2}$} }%
      \put( -15,-181){\makebox(0,0)[tr]{$\frac{-\ln2}{\ln3-\ln2}$} }%
      \put(-100, -15){\makebox(0,0)[t]{$-1$} }%
      \put( 100, -15){\makebox(0,0)[t]{$+1$} }%
    \color{blue}%
      \qbezier( 100,0)(100,100)(0, 100)%
      \qbezier(-171,0)(-50,50)(0, 100)%
      \qbezier(-171,0)(-50,-50)(0,-171)%
      \qbezier( 100,0)(50,-50)(0,-171)%
    \color{red}%
      \put(180,200){\makebox(0,0)[rt]{$\frac{1}{\ln3-\ln2}\,\ln\brs{\frac{5}{2}-\brp{\frac{3}{2}}^x}$} }%
      \put(140, 140){\vector(-1,-1){60} }%
  \end{picture}
  \end{fsL}
  \end{center}
\end{minipage}%
\end{example}

\begin{proof}
\begin{enumerate}
  \item Proof that $\metricn$ is a metric:
    \begin{enumerate}
      \item By \prefpp{ex:d_usual}, $\fp(x,y)\eqd \abs{x-y}$ is a metric (the \hie{usual metric}). \index{metrics!usual}
      \item The function $\ff(x)\eqd 2\brp{\brp{\frac{3}{2}}^x - 1}$ is strictly increasing in $x$. Proof:
        \begin{align*}
          \deriv{}{x} \ff(x)
            &= \deriv{}{x}\, 2\brp{\brp{\frac{3}{2}}^x - 1}
          \\&= 2 \deriv{}{x} \brp{\frac{3}{2}}^x
          \\&= 2 \deriv{}{x} \brp{e^{\ln\frac{3}{2}}}^x
          \\&= 2 \deriv{}{x} e^{x\ln\frac{3}{2}}
          \\&= 2 \brp{\ln\frac{3}{2}}\,e^{x\ln\frac{3}{2}}
          \\&= 2 \brp{\ln\frac{3}{2}}\,\brp{e^{\ln\frac{3}{2}}}^x
          \\&= 2 \brp{\ln\frac{3}{2}}\,\brp{\frac{3}{2}}^x
          \\&> 0 \qquad \forall x\in\R
        \end{align*}
      \item Therefore, by \prefpp{thm:met_sumpf}, $\metricn$ is a metric.
    \end{enumerate}

  \item Proof that $\metricn$ is not generated by a norm:
    \begin{align*}
      \norm{\alpha(\vx-\vy)}
        &= \norm{\alpha\vx-\alpha\vy}
      \\&= \metric{\alpha\vx}{\alpha\vy}
        && \text{for some function $\normn$}
      \\&= 2\sum_{i=1}^n  \abs{\brp{\frac{3}{2}}^{\alpha x_i} - \brp{\frac{3}{2}}^{\alpha y_i}}
        && \text{by definition of $\metricn$}
      \\&\ne  2\sum_{i=1}^n  \abs{\alpha \brp{\frac{3}{2}}^{x_i} - \alpha \brp{\frac{3}{2}}^{y_i}}
      \\&= \abs{\alpha}2\sum_{i=1}^n  \abs{\brp{\frac{3}{2}}^{x_i} -  \brp{\frac{3}{2}}^{y_i}}
      \\&= \abs{\alpha} \metric{\vx}{\vy}
        && \text{by definition of $\metricn$}
      \\&= \abs{\alpha}\;\norm{\vx-\vy}
    \end{align*}

  \item Proof that the ball is not convex:
    \begin{enumerate}
      \item The function $\ds\fp(\theta,\vx)\eqd 2\sum_{i=1}^n  \abs{\brp{\frac{3}{2}}^{\theta_i} - \brp{\frac{3}{2}}^{x_i}}$
            is not in general convex. Proof:
        \begin{align*}
          \pderiv{^2}{x_i^2} \fp(\vzero,\vx)
            &= \pderiv{^2}{x_i^2} 2\sum_{i=1}^n  \abs{\brp{\frac{3}{2}}^{0} - \brp{\frac{3}{2}}^{x_i}}
          \\&= \pderiv{^2}{x_i^2} 2 \abs{1 - \brp{\frac{3}{2}}^{x_i}}
          \\&= 2\pderiv{^2}{x_i^2} \brp{1 - \brp{\frac{3}{2}}^{x_i}}
            && \text{for $x_i<0$}
          \\&= -2\pderiv{}{x_i} \brp{\ln\frac{3}{2}}\,\brp{\frac{3}{2}}^{x_i}
            && \text{for $x_i<0$}
          \\&= -2 \brp{\ln\frac{3}{2}}^2\,\brp{\frac{3}{2}}^{x_i}
            && \text{for $x_i<0$}
          \\&< 0
            && \text{for $x_i<0$}
          \\\implies&\text{$\metricn$ is not convex}
        \end{align*}

      \item Therefore by \prefpp{thm:vsm_convex}, the ball is not convex.
    \end{enumerate}

\end{enumerate}
\end{proof}


%---------------------------------------
% Tangential
%---------------------------------------
%---------------------------------------
\begin{example}[Tangential metric]
\label{ex:ms_tan}
\index{metrics!tangential}
\index{tangential metric}
\mbox{}\\
%---------------------------------------
\begin{minipage}{3\tw/4-3mm}%
Let $\setX=\set{x\in\R}{x\intoo{-1}{1}}$ be a set and
$\vx\eqd\tuplexn{x_i\in\setX}$ and 
$\vy\eqd\tuplexn{y_i\in\setX}$
be sequences on $\setX$.
\exbox{\begin{tabular}{ll}
  1. & $\ds\metric{\vx}{\vy}\eqd \sum_{i=1}^n  \abs{\tan\brp{\frac{\pi}{2}x_i} - \tan\brp{\frac{\pi}{2} y_i}}$
       \emph{is a metric}.
  \\
  2. & $\metricn$ is \emph{not generated by a norm}.
  \\
  3. & $\ball{\theta}{1}$ in $\opair{\R^n}{\metricn}$ is \emph{convex}.
  \end{tabular}}
\end{minipage}%
\hfill%
\begin{minipage}{\tw/4}%
  \begin{center}
  \begin{fsL}
  \setlength{\unitlength}{\tw/300}
  \begin{picture}(300,300)(-150,-150)%
    %{\color{graphpaper}\graphpaper[10](-150,-150)(300,300)}%
    \thicklines%
    \color{axis}%
      \put(-130,   0){\line(1,0){260} }%
      \put(   0,-130){\line(0,1){260} }%
      \put( 140,   0){\makebox(0,0)[l]{$x$}}%
      \put(   0, 140){\makebox(0,0)[b]{$y$}}%
      \put( 100, -10){\line(0,1){20} }%
      \put(-100, -10){\line(0,1){20} }%
      \put( -10,-100){\line(1,0){20} }%
      \put( -10, 100){\line(1,0){20} }%
      \put( -15, 100){\makebox(0,0)[r]{$\frac{+1}{2}$} }%
      \put( -15,-100){\makebox(0,0)[r]{$\frac{-1}{2}$} }%
      \put(-100, -15){\makebox(0,0)[t]{$\frac{-1}{2}$} }%
      \put( 100, -15){\makebox(0,0)[t]{$\frac{+1}{2}$} }%
    \color{blue}%
      \qbezier( 100,0)(70,70)(0, 100)%
      \qbezier(-100,0)(-70,70)(0, 100)%
      \qbezier(-100,0)(-70,-70)(0,-100)%
      \qbezier( 100,0)(70,-70)(0,-100)%
    \color{red}%
      \put(130,200){\makebox(0,0)[rt]{$y=\frac{2}{\pi}\atan\brp{\pm1\mp\abs{\tan\brp{\frac{\pi}{2}x}}}$} }%
      %\put(140, 140){\vector(-1,-1){74} }%
  \end{picture}
  \end{fsL}
  \end{center}
\end{minipage}%
\end{example}

\begin{proof}
\begin{enumerate}
  \item Proof that $\metricn$ is a metric:
    \begin{enumerate}
      \item By \prefpp{ex:d_usual}, $\fp(x,y)\eqd \abs{x-y}$ is a metric (the \hie{usual metric}). \index{metrics!usual}
      \item The function $\ff(x)\eqd \tan\brp{\frac{\pi}{2}x}$ is strictly increasing in $x$. Proof:
        \begin{align*}
          \deriv{}{x} \ff(x)
            &= \deriv{}{x}\, \tan\brp{\frac{\pi}{2}x}
          \\&= \frac{\pi}{2} \sec^2\brp{\frac{\pi}{2}x}
          \\&> 0 \qquad \forall x\intoo{-1}{1}
        \end{align*}
      \item Therefore, by \prefpp{thm:met_sumpf}, $\metricn$ is a metric.
    \end{enumerate}

  \item Proof that $\metricn$ is not generated by a norm:
    \begin{align*}
      \norm{\alpha(\vx-\vy)}
        &= \norm{\alpha\vx-\alpha\vy}
      \\&= \metric{\alpha\vx}{\alpha\vy}
        && \text{for some function $\normn$}
      \\&= \sum_{i=1}^n  \abs{\tan\brp{\frac{\pi}{2}\alpha x_i} - \tan\brp{\frac{\pi}{2} \alpha y_i}}
        && \text{by definition of $\metricn$}
      \\&\ne  \sum_{i=1}^n \abs{\alpha\tan\brp{\frac{\pi}{2}x_i} - \alpha\tan\brp{\frac{\pi}{2} y_i}}
      \\&= \abs{\alpha} \sum_{i=1}^n \abs{\tan\brp{\frac{\pi}{2}x_i} - \tan\brp{\frac{\pi}{2} y_i}}
      \\&= \abs{\alpha} \metric{\vx}{\vy}
        && \text{by definition of $\metricn$}
      \\&= \abs{\alpha}\;\norm{\vx-\vy}
    \end{align*}

  \item Proof that the ball is convex:
    \begin{enumerate}
      \item The function $\ds\fp(\theta,\vx)\eqd \sum_{i=1}^n \abs{\tan\brp{\frac{\pi}{2}\theta_i} - \tan\brp{\frac{\pi}{2} x_i}}$
            is convex. Proof: \attention
        \begin{align*}
          \pderiv{^2}{x_i^2} \metric{\vzero}{\vx}
            &= \pderiv{^2}{x_i^2} \sum_{i=1}^n  \abs{\tan(0) - \tan\brp{\frac{\pi}{2} x_i}}
          \\&= \pderiv{^2}{x_i^2} \abs{\tan(0) - \tan\brp{\frac{\pi}{2} x_i}}
          \\&= \left\{\begin{array}{ll}
                 \pderiv{^2}{x_i^2}   \tan\brp{\frac{\pi}{2} x_i} & \text{ for } x_i\ge 0\\
                 \pderiv{^2}{x_i^2} - \tan\brp{\frac{\pi}{2} x_i} & \text{ for } x_i<0
               \end{array}\right.
          \\&= \left\{\begin{array}{ll}
                 \pderiv{}{x_i}   \frac{\pi}{2}\sec^2\brp{\frac{\pi}{2} x_i} & \text{ for } x_i\ge 0\\
                 \pderiv{}{x_i} - \frac{\pi}{2}\sec^2\brp{\frac{\pi}{2} x_i} & \text{ for } x_i<0
               \end{array}\right.
          \\&= \left\{\begin{array}{ll}
                   \frac{\pi}{2}\frac{\pi}{2}2\sec^2\brp{\frac{\pi}{2} x_i}\tan\brp{\frac{\pi}{2} x_i} & \text{ for } x_i\ge 0\\
                 - \frac{\pi}{2}\frac{\pi}{2}2\sec^2\brp{\frac{\pi}{2} x_i}\tan\brp{\frac{\pi}{2} x_i} & \text{ for } x_i<0
               \end{array}\right.
          \\&\ge 0
        \end{align*}

      \item Therefore by \prefpp{thm:vsm_convex}, the ball is convex.
    \end{enumerate}

\end{enumerate}
\end{proof}



%
%%---------------------------------------
%\begin{example}
%\citep{giles1987}{34}
%\index{metrics!Fr\'echet}
%%---------------------------------------
%The \hid{Fr\'echet metric} is defined as 
%\exbox{\begin{array}{>{\ds}l}
%  \fd(\vx,\vy) \eqd \sum_{n=1}^\infty \frac{1}{2^n} \: \frac{\abs{y_n-x_n}}{1+\abs{y_n-x_n}}
%  \\
%  \text{where}\qquad
%  \vx \eqd \seq{x_n}{n\in\Zp} 
%  \qquad\text{and}\qquad
%  \vy \eqd \seq{y_n}{n\in\Zp}
%\end{array}}
%The ball generated by this metric is {\bf not convex}.
%\end{example}








%--------------------------------------
\begin{example}
\label{ex:d_balls_nonorm_convex}
\footnote{\url{http://groups.google.com/group/sci.math/browse_thread/thread/da44b8a80e97d40f/a977cecea243ad0a}}
%--------------------------------------
Let $\metric{x}{y}=\abs{x-y}^2$ where $\absn$ is the absolute value on $\R$.
\begin{liste}
  \item Balls in $(\R,\fd)$ are \hie{convex} because they are simple intervals.
  \item But yet $\fd$ is \emph{not generated by a norm} because 
     \[ \fd(ax,ay) = \abs{ax-ay}^2 = \abs{a(x-y)}^2 = \abs{a}^2\abs{x-y}^2 \ne  \abs{a}\abs{x-y}^2.\]
\end{liste}
\end{example}


%--------------------------------------
\begin{example}
\label{ex:d_postoffice_norm}
\index{metrics!post office}
\index{post office metric}
\citetbl{
  \citerpg{giles1987}{17}{0521359287}\\
  \url{http://groups.google.com/group/sci.math/msg/38bb848a9c6d5c29}
  }
%--------------------------------------
Let $\normn_2$ be the $l_2$ norm.
Consider the \hie{post office metric}
\[
  \fd(\vx,\vy) \eqd 
    \left\{\begin{array}{ll}
      \norm{\vx}_2 + \norm{\vy}_2 & \text{for } \vx\ne \vy  \\
      0                           & \text{for } \vx=\vy
    \end{array}\right.
\]
\begin{tabular}{ll}
  \circOne & The post office metric is \emph{not generated by a norm}. \\
  \circTwo & The ball generated by the post office metric is in general \emph{not convex}.
\end{tabular}
\end{example}
\begin{proof}
\begin{enumerate}
  \item Proof that $\fd$ is not a norm:
    \begin{align*}
      \norm{\vzero}
        &= \norm{\vx-\vx}
      \\&= \fd(\vx,\vx)
        && \text{by assumption that $\fd$ can be generated by a norm $\normn$}
      \\&= \norm{\vx}_2 + \norm{\vx}_2
        && \text{by definition of the post office metric}
      \\&= 2\norm{\vx}_2
      \\&\ge 0
        && \text{by positive property of $\normn$ \prefpo{def:norm}}
    \end{align*}
    This implies $\normn$ is not a norm 
    because it fails the \hie{non-degenerate} property of norms
    ($\norm{\vzero}=0$---see \prefp{def:norm})
    and therefore $\fd$ is not generated by a norm.

  \item Proof that the ball generated by $\fd$ is not convex:\\
    Consider the ball with radius $1$ and center $\frac{3}{4}$ generated by the post office metric.
    \begin{enumerate}
      \item $\frac{3}{4}$ is in the ball because $\fd\brp{\frac{3}{4},\,\frac{3}{4}}=0\le 0$  
      \item $\frac{1}{8}$ is in the ball because $\fd\brp{\frac{3}{4},\,\frac{1}{8}}=\frac{3}{4}+\frac{1}{8}=\frac{7}{8}\le 1$
      \item \emph{But} $\frac{1}{2}\cdot\frac{3}{4} + \frac{1}{2}\cdot\frac{1}{8}=\frac{7}{16}$ 
            which is \emph{not} in the ball because 
            $\fd\brp{\frac{7}{16},\,\frac{3}{4}}=\frac{7}{16}+\frac{3}{4}=\frac{19}{16}>1$.
    \end{enumerate}
  
\end{enumerate}
\end{proof}



%---------------------------------------
\begin{example}[The bounded metric]
\citetbl{
  \citerp{copson1968}{22}
  }
\label{ex:d_bounded}
\index{metrics!bounded}
\index{bounded metric}
%---------------------------------------
Let $\setX$ be a set and $\metricn:X^2\to\Rnn$.
\exbox{\begin{array}{>{$\imark$} rcl>{\ds}l}
  & \metric{x}{y} &\eqd& \frac{\metrica{x}{y}}{1+\metrica{x}{y}}
    \text{ is a metric.}
  \\
  & \mc{3}{l}{\text{$\metricn$ is \emph{not} generated by a norm.}}
  \\
  & \ball{0}{1} &=& \setX
  \\
  & \diam \ball{0}{1} &=& \diam\setX
\end{array}}
\end{example}
\begin{proof}
\begin{enumerate}
  \item Proof that $\metric{x}{y}$ is a metric (using \prefp{thm:metric_equiv}):
        \prefpp{prop:p==>d}.

  \item Proof that $\metricn$ is not generated by a norm:
    \begin{align*}
      \norm{\alpha x}
        &= \metric{\alpha x}{0}
        && \text{for some function $\normn$}
      \\&= \frac{\fp(\alpha x, 0)}{1 + \fp(\alpha x,0)}
      \\&= \frac{\abs{\alpha}\fp( x, 0)}{1 + \abs{\alpha}\fp(x,0)}
        && \text{assuming $\fp$ is homogeneous}
      \\&= \abs{\alpha}\brs{\frac{\fp( x, 0)}{1 + \abs{\alpha}\fp(x,0)}}
      \\&\ne  \abs{\alpha}\brs{\frac{\fp( x, 0)}{1 + \fp(x,0)}}
      \\&= \abs{\alpha} \metric{x}{0}
      \\&= \abs{\alpha} \norm{x}
    \end{align*}

  \item Proof that $\ball{0}{1}=\setn{0}$:
    \begin{align*}
      \ball{0}{1}
        &= \set{x\in\spX}{\metric{0}{x}<1}
        && \text{by definition of open ball $\balln$ \prefpo{def:ball}}
      \\&= \set{x\in\spX}{\frac{\fp(x,y)}{1+\fp(x,y)}<1}
        && \text{by definition $\metricn$}
      \\&= \setn{x\in\spX}
      \\&= \setX
    \end{align*}

  \item Proof that $\diam\ball{0}{1}=\diam\setX$:
    \begin{align*}
      \diam\ball{0}{1}
        &= \diam\setX
        && \text{by previous result}
    \end{align*}

  \end{enumerate}
\end{proof}
}




