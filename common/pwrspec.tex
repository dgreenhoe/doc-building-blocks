%============================================================================
% LaTeX File
% Daniel J. Greenhoe
%============================================================================

%======================================
\chapter{Power Spectrum Functions}
\label{app:pwrspec}
%======================================

%======================================
\section{Correlation}
%======================================
\pref{def:Rfg} and \pref{def:Szfg} define four quantities.
In this document, the quantities' notation and terminology are
similar to those used in the study of \structe{random processes}.
%The three spectral density functions defined in \pref{def:wav_S}
%have an infinite series relationship with the Fourier transforms of the scaling and
%and wavelet functions, as shown in \pref{lem:SSS} (next).

%--------------------------------------
\begin{definition}
\footnote{
  \citerpg{chui}{134}{0121745848},
  \citerppgc{papoulis}{294}{332}{0070484775}{(10-29), (10-169)}
  }
\label{def:Rfg}
%--------------------------------------
%Let $\opT$ be the \fncte{translation operator} \xref{def:opT}.
%Let $\ff$ and $\fg$ be functions in the space $\spLLR$.
%Let $\inprod{\ff}{\fg}\eqd\int_\F\ff(x)\fg^\ast(x)\dx$.
Let $\inprodn$ be the \fncte{standard inner product} in $\spLLR$ \xref{def:spLLR}.
\defbox{\begin{array}{>{\ds}rc>{\ds}llM}
    \Rfg(n)&\eqd& \inprod{\ff(x)}{\opT^n\fg(x)}, &n\in\Z;\quad\ff,\fg\in\spLLF,  &is the \fnctd{cross-correlation function} of $\,\ff$ and $\,\fg$.
    \\                                    
    \Rff(n)&\eqd& \inprod{\ff(x)}{\opT^n\ff(x)}, &n\in\Z;\quad\ff\in\spLLF,      &is the \fnctd{autocorrelation function} of $\,\ff$.
\end{array}}
\end{definition}

%--------------------------------------
\begin{definition}
\footnote{
  \citerpg{chui}{134}{0121745848},
  \citerpgc{papoulis}{334}{0070484775}{(10-178)}
  }
\label{def:Szfg}
\label{def:Szff}
%--------------------------------------
Let $\Rfg(n)$ and $\Rff(n)$ be the sequences defined in \prefp{def:Rfg}.
%be the \fncte{cross-correlation function} \xref{def:Rfg} of functions $\,\ff$ and $\,\fg$ 
%and let $\Rff$ be the \fncte{autocorrelation function} of a function $\,\ff\in\spLLR$.
Let $\opZ\seqn{x_n}$ be the \fncte{z-transform} \xref{def:opZ} of a sequence $\seqxZ{x_n}$.
\defbox{\begin{array}{>{\ds}rc>{\ds}lCM}
    \hxs{\Szfg(z)} &\eqd&  \opZ\brs{\Rfg(n)}, &\ff,\fg\in\spLLF, &is the \hid{complex cross-power spectrum} of $\,\ff$ and $\,\fg$.\\
    \hxs{\Szff(z)} &\eqd&  \opZ\brs{\Rff(n)}, &\ff,\fg\in\spLLF, &is the \hid{complex auto-power spectrum} of $\,\ff$.
\end{array}}
%\\The Laurent polynomial $\Sphi(\omega)$ is also called the \hid{Euler-Frobenius polynomial}.
\end{definition}





%======================================
\section {Power Spectrum}
%======================================
%%--------------------------------------
%\begin{theorem}[\thmd{Parseval's Identities}]
%\label{lem:ft_parseval}
%\footnote{
%  \citerpgc{bachman2002}{358}{9780387988993}{5.18.3},
%  \citerpgc{pinsky2002}{128}{0534376606}{Theorem 2.4.2}
%  }
%%--------------------------------------
%\thmbox{\begin{array}{FrclD}
%  1. & \inprod{\ff}{\fg} &=& \inprod{\Ff}{\Fg} & and \\
%  2. & \norm{\ff}^2      &=& \norm{\Ff}^2      &     
%  %\int_\R \abs{\ff(x)}^2 \dx = \int_\R \abs{\Ff(\omega)}^2 \dw
%\end{array}}
%\end{theorem}

%--------------------------------------
\begin{definition}
\footnote{
  \citerpg{chui}{134}{0121745848},
  \citerpgc{papoulis}{333}{0070484775}{(10-179)}
  }
\label{def:Swfg}
\label{def:Swff}
%--------------------------------------
Let $\Szfg(z)$ and $\Szff(z)$ be the functions defined in \prefp{def:Szfg}.
%be the \fncte{complex cross-power spectrum} \xref{def:Szfg} of functions $\ff$ and $\fg$ in $\spLLR$
%and let $\Szff$ be the \fncte{complex auto-power spectrum} of a function $\ff\in\spLLR$.
\defbox{\begin{array}{>{\ds}rc>{\ds}lCM}
    \hxs{\Swfg(\omega)} &\eqd&  \Szfg\brp{e^{i\omega}},&\forall\ff,\fg\in\spLLF, &is the \fnctd{cross-power spectrum} of $\ff$ and $\fg$.\\
    \hxs{\Swff(\omega)} &\eqd&  \Szff\brp{e^{i\omega}},&\forall\ff\in\spLLF,     &is the \fnctd{auto-power spectrum} of $\ff$.
\end{array}}
%\\The Laurent polynomial $\Sphi(\omega)$ is also called the \hid{Euler-Frobenius polynomial}.
\end{definition}



%--------------------------------------
\begin{theorem}
\footnote{
  \citerpg{chui}{135}{0121745848}
  }
\label{thm:Swfg}
\label{thm:Swff}
%--------------------------------------
Let $\Swfg(\omega)$ and $\Swff(\omega)$ be defined as in \prefpp{def:Swfg}.\\
Let $\Ff(\omega)$ be the \fncte{Fourier transform} \xrefP{def:ft} of a function $\ff(x)\in\spLLF$. % and 
%let $\Fg(\omega)$ be the \fncte{Fourier transform} of $\fg(x)$.
\thmbox{\begin{array}{rc>{\ds}lC}
  \Swfg(\omega) &=& {2\pi}\sum_{n\in\Z} \Ff(\omega+2\pi n) \Fg^\ast(\omega+2\pi n) & \forall \ff,\fg\in\spLLF\\
  \Swff(\omega) &=& {2\pi}\sum_{n\in\Z} \abs{\Ff(\omega+2\pi n)}^2                 & \forall \ff\in\spLLF
\end{array}}
\end{theorem}
\begin{proof}
Let $z\eqd e^{i\omega}$.
\begin{align*}
  \Swfg(\omega)
    &\eqd \Szfg(z)
    && \text{by definition of $\Swfg$}
    && \text{\xref{def:Swfg}}
  \\&= \sum_{n\in\Z} \Rfg (n) z^{-n} %\fkernea{n}{\omega}
    && \text{by definition of $\Szfg$}
    && \text{\xref{def:Szfg}}
  \\&= \sum_{n\in\Z} \inprod{\ff(x)}{\fg(x-n)} z^{-n} %\fkernea{n}{\omega}
    && \text{by definition of $\Swfg$}
    && \text{\xref{def:Swfg}}
  \\&= \sum_{n\in\Z} \inprod{\opFT\brs{\ff(x)}}{\opFT\brs{\fg(x-n)}} z^{-n} %\fkernea{n}{\omega}
    && \text{by \prope{unitary} property of $\opFT$}
    && \text{\xref{thm:ft_prop}}
  \\&= \sum_{n\in\Z} \inprod{\Ff(v)}{e^{-iv n}\Fg(v)} z^{-n} %\fkernea{n}{\omega}
    && \text{by \thme{shift relation}}
    && \text{\xref{thm:ft_shift}}
  \\&= \sum_{n\in\Z} \cftr\brs{\cft\int_\R \Ff(v)\Fg^\ast(v) e^{iv u}\dv}_{u=n} z^{-n} %\fkernea{n}{\omega}
    && \text{by definition of $\spLLR$}
    && \text{\xref{def:spLLR}}
  \\&= \cftr \sum_{n\in\Z} \brs{\opFTi\brp{ \sqrt{2\pi}\Ff(v) \Fg^\ast(v)}}_{u=n} e^{-i\omega n} %\fkernea{n}{\omega}
    && \text{by \prefp{thm:opFTi}}
  \\&= {2\pi}\sum_{n\in\Z} \Ff(\omega+2\pi n) \Fg^\ast(\omega+2\pi n)
    && \text{by \thme{IPSF} with $\tau=1$}
    && \text{\xref{thm:ipsf}}
  \\
  \\
  \Swff(\omega)
    &= \brlr{\Swfg(\omega)}_{\fg=\ff}
    && \text{by definition of $\Swfg(\omega)$}
    && \text{\xref{def:Swfg}}
  \\&= {2\pi}\brlr{
       \sum_{n\in\Z} \Ff(\omega+2\pi n) \Fg^\ast(\omega+2\pi n)
       }_{\fg\eqd\ff}
    && \text{by previous result}
  \\&= {2\pi}\sum_{n\in\Z} \Ff(\omega+2\pi n) \Ff^\ast(\omega+2\pi n)
  \\&= {2\pi}\sum_{n\in\Z} \abs{\Ff(\omega+2\pi n)}^2
    && \text{because $\abs{z}^2\eqd zz^\ast\quad\forall z\in\C$}
\end{align*}
\end{proof}


%--------------------------------------
\begin{proposition}
\label{prop:ps_nonnegative}
%--------------------------------------
Let $\Swff(\omega)$ be defined as in \prefpp{def:Swfg}.
\propbox{\begin{array}{rclM}
  \Swff(\omega) &\ge& 0 & (\prope{non-negative})
\end{array}}
\end{proposition}
\begin{proof}
\begin{align*}
  \Swff(\omega) 
    &= {2\pi}\sum_{n\in\Z} \abs{\Ff(\omega+2\pi n)}^2
    && \text{by \prefp{thm:Swfg}}
  \\&\ge 0
    && \text{because $\abs{z}\ge0\quad\forall z\in\C$}
\end{align*}
\end{proof}

%--------------------------------------
\begin{proposition}
\label{prop:ps_periodic}
%--------------------------------------
Let $\Swfg(\omega)$ and $\Swff(\omega)$ be defined as in \prefpp{def:Swfg}.
\propbox{\begin{array}{rclM}
  \Swfg(\omega+2\pi) &=& \Swfg(\omega) & (\prope{periodic} with period $2\pi$)\\
  \Swff(\omega+2\pi) &=& \Swff(\omega) & (\prope{periodic} with period $2\pi$)
\end{array}}
\end{proposition}
\begin{proof}
\begin{align*}
  \Swfg(\omega+2\pi) 
    &= {2\pi}\sum_{n\in\Z} \Ff(\omega+2\pi+2\pi n) \Fg^\ast(\omega+2\pi+2\pi n)
    && \text{by \prefp{thm:Swfg}}
  \\&= {2\pi}\sum_{n\in\Z} \Ff\brs{\omega+2\pi(n+1)} \Fg^\ast\brs{\omega+2\pi(n+1)}
  \\&= {2\pi}\sum_{m\in\Z} \Ff\brs{\omega+2\pi m} \Fg^\ast\brs{\omega+2\pi m}
    && \text{where $m\eqd n+1$}
  \\&= \Swfg(\omega)
    && \text{by \prefp{thm:Swfg}}
  \\
  \Swff(\omega+2\pi) 
    &= \brlr{\Swfg(\omega+2\pi)}_{\fg=\ff}
  \\&= \brlr{\Swfg(\omega)}_{\fg=\ff}
    && \text{by previous result}
  \\&= \Swff(\omega)
\end{align*}
\end{proof}

%--------------------------------------
\begin{proposition}
\label{prop:ps_symmetry}
%--------------------------------------
Let $\Swfg(\omega)$ and $\Swff(\omega)$ be defined as in \prefpp{def:Swfg}.
\propbox{\begin{array}{McrclM}
  $\ff,\fg$ are real &\implies& \Swfg(-\omega) &=& \Swgf(\omega) &\\ 
  $\ff$ is real      &\implies& \Swff(-\omega) &=& \Swff(\omega) & (\prope{symmetric} about $0$)\\
  $\ff,\fg$ are real &\implies& \Swfg(\pi-\omega) &=& \Swgf(\pi+\omega) &\\ 
  $\ff$ is real      &\implies& \Swff(\pi-\omega) &=& \Swff(\pi+\omega) & (\prope{symmetric} about $\pi$)
\end{array}}
\end{proposition}
\begin{proof}
\begin{align*}
  \Swfg(-\omega) 
    &= {2\pi}\sum_{n\in\Z} \Ff(-\omega+2\pi n) \Fg^\ast(-\omega+2\pi n)
    && \text{by \prefp{thm:Swfg}}
  \\&= {2\pi}\sum_{n\in\Z} \Ff^\ast(\omega-2\pi n) \Fg(\omega-2\pi n)
    && \text{by hypothesis and \prefp{thm:ft_neg}}
    %&& \text{by $\ff,\fg$ real hypothesis and \prefp{thm:ft_neg}}
  \\&= {2\pi}\sum_{m\in\Z} \Fg(\omega+2\pi m) \Ff^\ast(\omega+2\pi m)
    && \text{where $m\eqd -n$}
  \\&= \Swgf(\omega) 
    && \text{by \prefp{thm:Swfg}}
  \\
  \\
  \Swfg(\pi-\omega) 
    &= {2\pi}\sum_{n\in\Z} \Ff(\pi-\omega+2\pi n) \Fg^\ast(\pi-\omega+2\pi n)
    && \text{by \prefp{thm:Swfg}}
  \\&= {2\pi}\sum_{n\in\Z} \Ff^\ast(-\pi+\omega-2\pi n) \Fg(-\pi+\omega-2\pi n)
    && \text{by hypothesis and \prefp{thm:ft_neg}}
    %&& \text{by $\ff,\fg$ real hypothesis and \prefp{thm:ft_neg}}
  \\&= {2\pi}\sum_{n\in\Z} \Ff^\ast(\pi+\omega-2\pi-2\pi n) \Fg(\pi+\omega-2\pi-2\pi n)
  \\&= {2\pi}\sum_{n\in\Z} \Ff^\ast(\pi+\omega+2\pi(-n-1)) \Fg(\pi+\omega+2\pi(-n-1))
  \\&= {2\pi}\sum_{m\in\Z} \Fg(\pi+\omega+2\pi m) \Ff^\ast(\pi+\omega+2\pi m)
    && \text{where $m\eqd -n-1$}
  \\&= \Swgf(\pi+\omega) 
    && \text{by \prefp{thm:Swfg}}
  \\
  \\
  \Swff(-\omega) 
    &= \brlr{\Swfg(-\omega)}_{\fg\eqd\ff}
  \\&= \brlr{\Swgf(+\omega)}_{\fg\eqd\ff}
    && \text{by previous result}
  \\&= \Swff(+\omega)
    && \text{by definition of $\fg$ ($\fg\eqd\ff$)}
  \\
  \\
  \Swff(\pi-\omega) 
    &= \brlr{\Swfg(\pi-\omega)}_{\fg\eqd\ff}
  \\&= \brlr{\Swgf(\pi+\omega)}_{\fg\eqd\ff}
    && \text{by previous result}
  \\&= \Swff(\pi+\omega)
    && \text{by definition of $\fg$ ($\fg\eqd\ff$)}
\end{align*}
\end{proof}

%--------------------------------------
\begin{proposition}
\label{prop:ps_antisymmetric}
%--------------------------------------
Let $\Swff(\omega)$ be the \fncte{auto-power spectrum} \xref{def:Swfg} of a function $\ff(x)\in\spLLR$
and $\Swff'(\omega)\eqd\ddw\Swff(\omega)$ \xref{def:ddx}.
\propbox{\begin{array}{rcl}
  \brb{\begin{array}{FMD}
    (a). & $\ff$ is \prope{real}                               & and\\
    (b). & \mc{2}{M}{$\Swff(\omega)$ is \prope{continuous} at $\omega=0$}
  \end{array}}
  &\implies&
  \brb{\begin{array}{FlD}
    (1). & \Swff'(0)  = 0 & and\\
    (2). & \mc{2}{l}{\mcom{\Swff'(\omega) = -\Swff'(-\omega)\quad\scy\forall\omega\in\R}
                {\prope{anti-symmetric} about $0$}}
  \end{array}}
  \\
  \brb{\begin{array}{FMD}
    (c). & $\ff$ is \prope{real}                               & and\\
    (d). & \mc{2}{M}{$\Swff(\omega)$ is \prope{continuous} at $\omega=\pi$}
  \end{array}}
  &\implies&
  \brb{\begin{array}{FlD}
    (3). & \Swff'(\pi)  = 0 & and\\
    (4). & \mc{2}{l}{\mcom{\Swff'(\pi+\omega) = -\Swff'(\pi-\omega)\quad\scy\forall\omega\in\R}
                {\prope{anti-symmetric} about $\pi$}}
  \end{array}}
\end{array}}
\end{proposition}
\begin{proof}
This follows from \prefpp{prop:ps_symmetry} and \prefpp{prop:ddx_symmetry}.
\end{proof}

\pref{thm:rbasis_S} (next) is a major result and provides strong motivation for 
bothering with \fncte{power spectrum} functions in the first place.
In particular, the \fncte{auto-power spectrum} being \prope{bounded} provides a necessary and sufficient condition 
for a sequence of functions $\seqnZ{\fphi(x-n)}$ to be a \structe{Riesz basis} \xref{def:basis_riesz}
for the \structe{span} $\linspan\seqn{\fphi(x-n)}$ of the sequence.
%--------------------------------------
\begin{theorem}
\footnote{
  \citerppgc{wojtaszczyk1997}{22}{23}{0521578949}{Proposition 2.8},
  \citerpgc{igari1996}{219}{0821821040}{Lemma 9.6},
  \citerpgc{pinsky2002}{306}{0534376606}{Theorem 6.4.8}
  }
\label{thm:rbasis_S}
%--------------------------------------
Let $\Swff(\omega)$ be defined as in \prefpp{def:Swfg}.
%Let $\Sphi(\omega)$ be the \hie{spectral density function} \xref{def:Swfg} of a function $\fphi(x)\in\spLLR$.
Let $\normn$ be defined as in \prefpp{def:spLLR}.
Let $0<A<B$. 
%$\ds \norm{\ff(x)}^2 \eqd \int_\R\abs{\ff(x)}^2\dx$
\thmbox{
  \mcom{\brb{A\sum_{n\in\Zp}\abs{a_n}^2 \le \norm{\sum_{n\in\Z}a_n\fphi(x-n)}^2 \le B\sum_{n\in\Zp}\abs{\alpha_n}^2\quad\scy\forall\seqn{a_n}\in\spllF}}
       {$\seqn{\fphi(x-n)}$ is a \structe{Riesz basis} for $\linspan\seqn{\fphi(x-n)}$ \xref{thm:rieszAB}}
  \qquad\iff\qquad
  \brb{A \le \Sphi(\omega) \le B}
  }
\end{theorem}
\begin{proof}
\begin{enumerate}
  \item lemma: \label{ilem:rbasis_S_aw}
    \begin{align*}
      \norm{\sum_{n\in\Z} a_n\fphi(x-n)}^2
        &= \norm{\opFT\sum_{n\in\Z} a_n\fphi(x-n)}^2
        && \text{because $\opFT$ is \prope{unitary} \xref{thm:ft_unitary}}
      \\&= \norm{\Da(\omega)\Fphi(\omega)}^2
        && \text{by \prefp{prop:Fsum_af}}
      \\&= \int_\R \abs{\Da(\omega)\Fphi(\omega)}^2\dw
        && \text{by definition of $\normn$}
      \\&= \sum_{n\in\Z}\int_0^{2\pi} \abs{\Da(\omega+2\pi n)\Fphi(\omega+2\pi n)}^2\dw
      \\&= \int_0^{2\pi}\sum_{n\in\Z} \abs{\Da(\omega+2\pi n)}^2 \abs{\Fphi(\omega+2\pi n)}^2\dw
      \\&= \int_0^{2\pi}\sum_{n\in\Z} \abs{\Da(\omega)}^2 \abs{\Fphi(\omega+2\pi n)}^2\dw
        && \text{by \prefp{prop:dtft_2pi}}
      \\&= \int_0^{2\pi}\abs{\Da(\omega)}^2 \frac{1}{2\pi}2\pi\sum_{n\in\Z} \abs{\Fphi(\omega+2\pi n)}^2\dw
      \\&= \frac{1}{2\pi}\int_0^{2\pi}\abs{\Da(\omega)}^2 \Sphi(\omega)\dw
        && \text{by definition of $\Sphi(\omega)$ \xref{thm:Swff}}
    \end{align*}

  \item lemma: \label{ilem:rbasis_S_an}
    \begin{align*}
      \frac{1}{2\pi}\int_0^{2\pi}\abs{\Da(\omega)}^2\dw
        &= \frac{1}{2\pi}\int_0^{2\pi}\abs{\sum_{n\in\Z}a_n e^{-i\omega n}}^2\dw
        && \text{by def. of \ope{DTFT} \xref{def:dtft}}
      \\&= \frac{1}{2\pi}\int_0^{2\pi}\brs{\sum_{n\in\Z}a_n e^{-i\omega n}}\brs{\sum_{m\in\Z}a_m e^{-i\omega m}}^\ast\dw
      \\&= \frac{1}{2\pi}\int_0^{2\pi}\brs{\sum_{n\in\Z}a_n e^{-i\omega n}}\brs{\sum_{m\in\Z}a_m^\ast e^{i\omega m}}\dw
      \\&= \frac{1}{2\pi}\sum_{n\in\Z}\sum_{m\in\Z}a_n a_m^\ast \int_0^{2\pi} e^{-i\omega (n-m)} \dw
      \\&= \frac{1}{2\pi}\sum_{n\in\Z}\sum_{m\in\Z}a_n a_m^\ast 2\pi\kdelta_{nm}
      \\&= \sum_{n\in\Z}\abs{a_n}^2
        && \text{by definition of $\kdelta$ \xref{def:kdelta}}
    \end{align*}

  \item Proof for ($\impliedby$) case:
    \begin{align*}
      \boxed{A\sum_{n\in\Z}\abs{a_n}^2}
        &=   \frac{A}{2\pi}\int_0^{2\pi}\abs{\Da(\omega)}^2\dw
        &&   \text{by \prefp{ilem:rbasis_S_an}}
      \\&=   \frac{1}{2\pi}\int_0^{2\pi}\abs{\Da(\omega)}^2 A \dw
      \\&\boxed{\le} \frac{1}{2\pi}\int_0^{2\pi}\abs{\Da(\omega)}^2 \Sphi(\omega) \dw
        &&   \text{by right hypothesis}
      \\&=\boxed{\norm{\sum_{n\in\Z} a_n \fphi(x-n)}^2 }
        &&   \text{by \prefp{ilem:rbasis_S_aw}}
      \\&=   \frac{1}{2\pi}\int_0^{2\pi}\abs{\Da(\omega)}^2 \Sphi(\omega) \dw
        &&   \text{by \prefp{ilem:rbasis_S_aw}}
      \\&\boxed{\le} \frac{1}{2\pi}\int_0^{2\pi}\abs{\Da(\omega)}^2 B \dw
        &&   \text{by right hypothesis}
      \\&=   \frac{B}{2\pi}\int_0^{2\pi}\abs{\Da(\omega)}^2 \dw
      \\&=\boxed{B\sum_{n\in\Z}\abs{a_n}^2}
        &&   \text{by \prefp{ilem:rbasis_S_an}}
    \end{align*}

  \item Proof for ($\implies$) case:
    \begin{enumerate}
      \item Let $\setY\eqd\set{\omega\in\intcc{0}{2\pi}}{\Sphi(\omega)>\alpha}$ \label{item:rbasis_S_Ya}
          \\and $\setX\eqd\set{\omega\in\intcc{0}{2\pi}}{\Sphi(\omega)<\alpha}$ \label{item:rbasis_S_Xa}

      \item Let $\setindAx$ be the \fncte{set indicator} \xref{def:setind} of a set $\setA$. \\
            Let $\seqxZ{b_n}$ be the \ope{inverse DTFT} \xref{thm:idtft} of $\setind_\setY(\omega)$
            such that
            \\\indentx$\ds\setind_\setY(\omega)\eqd\sum_{n\in\Zp}b_n e^{-i\omega n} \eqd \Fb(\omega)$.\label{item:rbasis_S_bn}
            \\
            Let $\seqxZ{a_n}$ be the \ope{inverse DTFT} \xref{thm:idtft} of $\setind_\setX(\omega)$
            such that
            \\\indentx$\ds\setind_\setX(\omega)\eqd\sum_{n\in\Zp}a_n e^{-i\omega n} \eqd \Da(\omega)$.

      \item Proof that $\alpha\le B$: \label{item:rbasis_S_aB}\\
            Let $\msm(\setA)$ be the \fncte{measure} of a set $\setA$.
        \begin{align*}
          \boxed{B}\sum_{n\in\Z}\abs{b_n}^2
            &\ge \norm{\sum_{n\in\Z} b_n \fphi(x-n)}^2
            && \text{by left hypothesis}
          \\&= \frac{1}{2\pi}\int_0^{2\pi}\abs{\Fb(\omega)}^2\Sphi(\omega)\dw
            && \text{by \prefp{ilem:rbasis_S_aw}}
          \\&= \frac{1}{2\pi}\int_0^{2\pi}\abs{\setind_\setY(\omega)}^2\Sphi(\omega)\dw
            && \text{by definition of $\setind_\setY(\omega)$}
            && \text{\xref{item:rbasis_S_bn}}
          \\&= \frac{1}{2\pi}\int_\setY \abs{1}^2\Sphi(\omega)\dw
            && \text{by definition of $\setind_\setY(\omega)$}
            && \text{\xref{item:rbasis_S_bn}}
          \\&\boxed{\ge} \frac{\alpha}{2\pi} \msm(\setY)
            && \text{by definition of $\setY$}
            && \text{\xref{item:rbasis_S_Ya}}
          \\&= \int_0^{2\pi} \abs{\setind_\setY(\omega)}^2\dw
            && \text{by definition of $\setind_\setY(\omega)$}
            && \text{\xref{item:rbasis_S_bn}}
          \\&= \int_0^{2\pi} \abs{\sum_{n\in\Z}b_n e^{-i\omega n}}^2\dw
            && \text{by definition of $\seqn{b_n}$}
            && \text{\xref{item:rbasis_S_bn}}
          \\&= \int_0^{2\pi} \abs{\Fb(\omega)}^2\dw
            && \text{by definition of $\Fb(\omega)$}
            && \text{\xref{item:rbasis_S_bn}}
          \\&= \boxed{\alpha}\sum_{n\in\Z}\abs{b_n}^2
            && \text{by \prefp{ilem:rbasis_S_an}}
        \end{align*}

      \item Proof that $\Sphi(\omega)\le B$:
        \\\begin{tabular}{rll}
                (i).& $\Sphi(\omega)>\alpha$ whenever $\omega\in\setY$ & \xref{item:rbasis_S_Ya}.
            \\ (ii).& But even then, $\alpha\le B$                     & \xref{item:rbasis_S_aB}.
            \\(iii).& So, $\Sphi(\omega)\le B$.
        \end{tabular}

      \item Proof that $A\le\alpha$: \label{item:rbasis_S_Aa}\\
            Let $\msm(\setA)$ be the \fncte{measure} of a set $\setA$.
        \begin{align*}
          \boxed{A}\sum_{n\in\Z}\abs{a_n}^2
            &\le \norm{\sum_{n\in\Z} a_n \fphi(x-n)}^2
            && \text{by left hypothesis}
          \\&= \frac{1}{2\pi}\int_0^{2\pi}\abs{\Da(\omega)}^2\Sphi(\omega)\dw
            && \text{by \prefp{ilem:rbasis_S_aw}}
          \\&= \frac{1}{2\pi}\int_0^{2\pi}\abs{\setind_\setX(\omega)}^2\Sphi(\omega)\dw
            && \text{by definition of $\setind_\setX(\omega)$}
            && \text{\xref{def:setind}}
          \\&= \frac{1}{2\pi}\int_\setX \abs{1}^2\Sphi(\omega)\dw
            && \text{by definition of $\setind_\setX(\omega)$}
            && \text{\xref{def:setind}}
          \\&\boxed{\le} \frac{\alpha}{2\pi} \msm(\setX)
            && \text{by definition of $\setX$}
            && \text{\xref{item:rbasis_S_Ya}}
          \\&= \int_0^{2\pi} \abs{\setind_\setX(\omega)}^2\dw
            && \text{by definition of $\setind_\setX(\omega)$}
            && \text{\xref{def:setind}}
          \\&= \int_0^{2\pi} \abs{\sum_{n\in\Z}a_n e^{-i\omega n}}^2\dw
            && \text{by definition of $\seqn{a_n}$}
            && \text{\xref{ilem:rbasis_S_an}}
          \\&= \int_0^{2\pi} \abs{\Da(\omega)}^2\dw
            && \text{by definition of $\Da(\omega)$}
            && \text{\xref{ilem:rbasis_S_an}}
          \\&= \boxed{\alpha}\sum_{n\in\Z}\abs{a_n}^2
            && \text{by \prefp{ilem:rbasis_S_an}}
        \end{align*}

      \item Proof that $A\le \Sphi(\omega)$:
        \\\begin{tabular}{rll}
              (i).& $\Sphi(\omega)<\alpha$ whenever $\omega\in\setX$ & \xref{item:rbasis_S_Xa}.
          \\ (ii).& But even then, $A\le\alpha$                      & \xref{item:rbasis_S_Aa}.
          \\(iii).& So, $A\le\Sphi(\omega)$.                         & 
        \end{tabular}
  \end{enumerate}
\end{enumerate}
\end{proof}

%The \structe{spectral density relations}\ifsxrefs{dsp}{def:Swfg}for a \hie{general} wavelet system 
%have already been presented in \prefpp{thm:Swfg}.
In the case that $\ff$ and $\fg$ are \prope{orthonormal}, the spectral density relations simplify 
considerably (next).
%--------------------------------------
\begin{theorem}
\index{Harmonic shifted orthonormality requirement}
\footnote{
  \citerpgc{hernandez1996}{50}{0849382742}{\scshape Proposition 2.1.11},
  \citerpgc{wojtaszczyk1997}{23}{0521578949}{Corollary 2.9},
  \citerppgc{igari1996}{214}{215}{0821821040}{Lemma 9.2},
  \citerpgc{pinsky2002}{306}{0534376606}{Corollary 6.4.9}
  %\citerp{goswami}{108}
  }
%\label{lem:SSSo}
\label{thm:Swfgo}
%--------------------------------------
%Let $\wavsys$ be a wavelet system \xref{def:wavsys}.
Let $\Swff$ and $\Swfg$ be the \fncte{spectral density function}s \xref{def:Swfg}.
%\thmbox{\begin{array}{rcl @{\qquad}c@{\qquad} rcl@{\qquad}C}
\thmbox{\begin{array}{rcl Dc rcl C}
  \inprod{\ff(x)}{\ff(x-n)} &=& \kdelta_n 
    &($\seqn{\ff(x-n)}$ is \prope{orthonormal})  
    &\iff& 
    \Swff(\omega) &=& 1  
    & \forall \ff\in\spLLF
    \\
  \inprod{\ff(x)}{\fg(x-n)} &=& 0          
    &($\ff(x)$ is \prope{orthogonal} to $\seqn{\fg(x-n)}$)  
    &\iff& 
    \Swfg(\omega) &=& 0  
    & \forall \ff,\fg\in\spLLF
\end{array}}
\end{theorem}
\begin{proof}
\begin{enumerate}
  \item Proof that $\inprod{\ff(x)}{\ff(x-n)}=\kdelta_n  \iff \Swff(\omega)=1$: 
        This follows directly from \prefpp{thm:rbasis_S} with $A=B=1$ 
        (by \thme{Parseval's Identity} \prefp{thm:fst} since $\setn{\opTrn^n\ff}$ is \prope{orthonormal})

  \item Alternate proof that $\inprod{\ff(x)}{\ff(x-n)}=\kdelta_n  \implies \Swff(\omega)=1$:
    \begin{align*}
      \Swff(\omega)
        &= \sum_{n\in\Z} \Rff(n) e^{-i\omega n}
        && \text{by definition of $\Swfg$}
        && \text{\xref{def:Swfg}}
      \\&= \sum_{n\in\Z} \inprod{\ff(x)}{\ff(x-n)} e^{-i\omega n}
        && \text{by definition of $\Rff$}
        && \text{\xref{def:Rfg}}
      \\&= \sum_{n\in\Z} \kdelta_n  e^{-i\omega n}
        && \text{by left hypothesis}
      \\&= 1
        && \text{by definition of $\kdelta$}
        && \text{\xref{def:kdelta}}
    \end{align*}

  \item Alternate proof that $\inprod{\ff(x)}{\ff(x-n)}=\kdelta_n  \impliedby \Swff(\omega)=1$:
    \begin{align*}
      &\inprod{\ff(x)}{\ff(x-n)}
      \\&= \inprod{\opFT\ff(x)}{\opFT\ff(x-n)}
        && \text{by \prope{unitary} property of $\opFT$}
        && \text{\xref{thm:ft_prop}}
      \\&= \inprod{\Ff(\omega)}{e^{-i\omega n}\Ff(\omega)}
        && \text{by \prope{shift property} of $\opFT$}
        && \text{\xref{thm:ft_shift}}
      \\&= \int_\R \Ff(\omega) e^{i\omega n} \Ff^\ast(\omega) \dw
        && \text{by definition of $\inprodn$}
        && \text{\xref{def:spLLR}}
      \\&= \int_\R \abs{\Ff(\omega)}^2 e^{i\omega n} \dw
      \\&= \sum_{n\in\Z} \int_{2\pi n}^{2\pi(n+1)} \abs{\Ff(\omega)}^2 e^{i\omega n} \dw
      \\&= \sum_{n\in\Z} \int_{0}^{2\pi} \abs{\Ff(u+2\pi n)}^2 e^{i(u+2\pi n)n} \du
        &&\mathrlap{\text{where $u\eqd \omega-2\pi n$ $\implies$ $\omega=u+2\pi n$}}
      \\&= \frac{1}{2\pi}\int_{0}^{2\pi} \brs{2\pi\sum_{n\in\Z} \abs{\Ff(u+2\pi n)}^2} e^{iun} \cancelto{1}{e^{i2\pi nn}} \du
      \\&= \frac{1}{2\pi}\int_{0}^{2\pi} \Swff(\omega) e^{iun} \du
        && \text{by \prefp{thm:Swff}}
      \\&= \frac{1}{2\pi}\int_{0}^{2\pi} e^{iun} \du
        && \text{by right hypothesis}
      \\&= \kdelta_n
        && \text{by definition of $\kdelta$}
        && \text{\xref{def:kdelta}}
    \end{align*}

  \item Proof that $\inprod{\ff(x)}{\fg(x-n)}=0  \implies \Swfg(\omega)=0$:
    \begin{align*}
      \Swfg(\omega)
        &= \sum_{n\in\Z} \Rfg(n) e^{-i\omega n}
        && \text{by definition of $\Swfg$}
        && \text{\xref{def:Swfg}}
      \\&= \sum_{n\in\Z} \inprod{\ff(x)}{\fg(x-n)} e^{-i\omega n}
        && \text{by definition of $\Rfg$}
        && \text{\xref{def:Rfg}}
      \\&= \sum_{n\in\Z} 0 e^{-i\omega n}
        && \text{by left hypothesis}
      \\&= 0
    \end{align*}

  \item Proof that $\inprod{\ff(x)}{\fg(x-n)}=0  \impliedby \Swfg(\omega)=0$:
    \begin{align*}
      &\inprod{\ff(x)}{\fg(x-n)}
      \\&= \inprod{\opFT\ff(x)}{\opFT\fg(x-n)}
        && \text{by \prope{unitary} property of $\opFT$}
        && \text{\xref{thm:ft_prop}}
      \\&= \inprod{\Ff(\omega)}{e^{-i\omega n}\Fg(\omega)}
        && \text{by \prope{shift property} of $\opFT$}
        && \text{\xref{thm:ft_shift}}
      \\&= \int_\R \Ff(\omega) e^{i\omega n} \Fg^\ast(\omega) \dw
        && \text{by definition of $\inprodn$}
        && \text{\xref{def:spLLR}}
      \\&= \int_\R \Ff(\omega)\Fg^\ast(\omega) e^{i\omega n} \dw
      \\&= \sum_{n\in\Z} \int_{2\pi n}^{2\pi(n+1)} \Ff(\omega)\Fg^\ast(\omega) e^{i\omega n} \dw
      \\&= \sum_{n\in\Z} \int_{0}^{2\pi} \Ff(u+2\pi n) \Fg^\ast(u+2\pi n) e^{i(u+2\pi n)n} \du
        && \mathrlap{\text{where $u\eqd \omega-2\pi n$ $\implies$ $\omega=u+2\pi n$}}
      \\&= \frac{1}{2\pi}\int_{0}^{2\pi} \brs{2\pi\sum_{n\in\Z} \Ff(u+2\pi n) \Fg^\ast(u+2\pi n)} e^{iun} \cancelto{1}{e^{i2\pi nn}} \du
      \\&= \frac{1}{2\pi}\int_{0}^{2\pi} \Swfg(u) e^{iun} \du
        && \text{by \prefp{thm:Swfg}}
      \\&= \frac{1}{2\pi}\int_{0}^{2\pi} 0\cdot e^{iun} \du
        && \text{by right hypothesis}
      \\&= 0
    \end{align*}
\end{enumerate}
\end{proof}





