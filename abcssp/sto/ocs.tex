%============================================================================
% Daniel J. Greenhoe
% XeLaTeX file
% stochastic systems
%============================================================================

%=======================================
\section{Outcome subspaces}
\label{sec:ocs}
%=======================================
%=======================================
\subsection{Definitions}
%=======================================
Traditional probability theory is performed in a \structe{probability space} $\ps$. % \xref{def:ps}.
This section extends\footnote{
  \citerpgc{feldman2010}{4}{3642051588}{``The name ``random variable" is actually a misnomer, since it is not random and not a variable.\ldots the random variable simply maps each point (outcome) in the sample space to a number on the real line\ldots Technically, the space into which the random variable maps the sample space may be more general than the real line\ldots"}
  }
 the probability space structure to include what herein is called an
\structe{outcome subspace} (next definition).
%---------------------------------------
\begin{definition}
\label{def:ocs}
\label{def:ocsm}
%---------------------------------------
\defboxp{
  An \structd{extended probability space} is the tuple $\epsX$ where 
  $\ps$ is a \structe{probability space} \xref{def:ps}
  and $\otriple{\epso}{\metricn}{\orel}$ is an \structe{ordered quasi-metric space} \xref{def:oms}.
  The 4-tuple $\ocsD$ is an \structd{outcome subspace}
  of the \structe{extended probability space} $\epsX$.
  }
\end{definition}

%---------------------------------------
\begin{definition}
\label{def:ocsmom}
%---------------------------------------
%Let $\Zp\eqd\setn{1,2,3,\ldots}$ be the set of \structe{natural numbers}.
Let $\ocsG\eqd\ocsD$ be an \structe{outcome subspace} \xref{def:ocs}.
\\\defboxt{$\begin{array}{MrclC}
    The \structd{$n$th-moment} $\ocsmom_n(x,y)$ from $x$ to $y$ in $\ocsG$ is defined as & \ocsmom(x,y) &\eqd& \brs{\metric{x}{y}}^n\psp(y) & \forall x,y\in\ocso,\,n\in\Zp.\\
    The \structd{moment}       $\ocsmom(x,y)$   from $x$ to $y$ in $\ocsG$ is defined as & \ocsmom(x,y) &\eqd& \ocsmom_1(x,y)               & \forall x,y\in\ocso.
  \end{array}$}
\end{definition}

This paper introduces a quantity called the \structe{outcome center} of an \structe{outcome subspace} (next definition)
which is in essence the same as the \structe{center} of a \structe{graph} \xref{def:gcen}.
%---------------------------------------
\begin{definition}
\label{def:ocscen}
%---------------------------------------
Let $\ocsG\eqd\ocsD$ be an \structe{outcome subspace} \xref{def:ocsm}.
\\\defbox{
  \begin{array}{>{\ds}rc>{\ds}lMM}
      \ocscen (\ocsG) &\eqd& \argmin_{x\in\ocso}\max_{y\in\ocso}\mcom{\metric{x}{y}\psp(y)}{$\ocsmom(x,y)$} & is the \structd{outcome center}    &of $\ocsG$. 
  \end{array}
  }
\end{definition}

The following additional definitions are of interest due in part to 
\prefpp{cor:means} and the \ineqe{minimax inequality} \xref{thm:minimax}.
They are illustrated in several examples in this section.
However, %their usefulness is somewhat limited in the context of this paper and 
most of them are not used outside this section.
%---------------------------------------
\begin{definition}
\label{def:ocscena}
\label{def:ocsceng}
\label{def:ocscenh}
\label{def:ocscenm}
\label{def:ocscenM}
\label{def:ocscenx}
%---------------------------------------
Let $\ocsG\eqd\ocsD$ be an \structe{outcome subspace} \xref{def:ocsm}.
\\\defbox{
  \begin{array}{>{\ds}rc>{\ds}lMM}
      \ocscena(\ocsG) &\eqd& \argmin_{x\in\ocso}\sum_{y\in\ocso}\metric{x}{y}\psp(y)                                   & is the \structd{arithmetic center} &of $\ocsG$.
    \\\ocsceng(\ocsG) &\eqd& \argmin_{x\in\ocso}\prod_{y\in\ocso\setd\setn{x}}\brs{\metric{x}{y}^{\psp(y)}}            & is the \structd{geometric center}  &of $\ocsG$.
    \\\ocscenh(\ocsG) &\eqd& \argmin_{x\in\ocso}\brp{\sum_{y\in\ocso\setd\setn{x}}\frac{1}{\metric{x}{y}}\psp(y)}^{-1} & is the \structd{harmonic center}   &of $\ocsG$.
    \\\ocscenm(\ocsG) &\eqd& \argmin_{x\in\ocso}\min_{y\in\ocso\setd\setn{x}}\metric{x}{y}\psp(y)                      & is the \structd{minimal center}    &of $\ocsG$.
    \\\ocscenM(\ocsG) &\eqd& \argmax_{x\in\ocso}\min_{y\in\ocso\setd\setn{x}}\metric{x}{y}\psp(y)                      & is the \structd{maxmin center}    &of $\ocsG$.
  \end{array}
  }
\end{definition}

In a manner similar to the traditional \fncte{variance} function \xref{def:pVar},
the \fncte{outcome variance} (next) is a kind of measure of the quality of the outcome center as a representative estimate 
of all the values of the \structe{outcome subspace}.
Said another way, it is in essence the expected error of the center measure.
%---------------------------------------
\begin{definition}
\label{def:ocsVarG}
%---------------------------------------
Let $\ocsG\eqd\ocsD$ be an \structe{outcome subspace} \xref{def:ocsm}.
\\\defboxp{
  The \structd{outcome variance} $\ocsVaro(\ocsG;\ocscen_x)$ of $\ocsG$ with respect to $\ocscen_x$ is
  \indentx$\ds\ocsVaro(\ocsG;\ocscen_x)\eqd\sum_{x\in\ocso} \mcom{\metricsq{\ocscen_x(\ocsG)}{x}\psp(x)}{$\ocsmom_2(\ocscen(\ocsG),x)$}$.\\
  where $\ocscen_x$ is any of the operators defined in \pref{def:ocscen} or \prefpp{def:ocsceng}.\\
  Moreover, $\ocsVaro(\ocsG)\eqd\ocsVaro(\ocsG;\ocscen)$, where $\ocscen$ is the \structe{outcome center} \xref{def:ocscen}.
  }
\end{definition}

%=======================================
%\subsection{Properties}
%=======================================
%%---------------------------------------
%\begin{theorem}
%\label{thm:ocs_isotone}
%%---------------------------------------
%Let $\ocsG\eqd\ocs{\ocso}{\metricn} {\orel}{\psp}$
%and $\ocsH\eqd\ocs{\ocso}{\metrican}{\orel}{\psp}$ be \structe{outcome subspace}s \xref{def:ocs}.
%Let $\ff\in\clFrr$ be a function from $\R$ onto $\R$.
%\thmbox{
%  \brb{\begin{array}{FMD}
%    A. & $\ff$ is \prope{strictly isotone} \xref{def:smono}   & and\\
%    B. & $\metric{x}{y} = \ff\brs{\metrica{x}{y}}$ & $\forall x,y\in\ocso$
%  \end{array}}
%  \quad\implies\quad
%  \brb{\begin{array}{FlclD}
%    1. & \ocscen (\ocsG) &=& \ocscen (\ocsH) & and \\
%    2. & \ocscenm(\ocsG) &=& \ocscenm(\ocsH) & and \\
%    3. & \ocscenM(\ocsG) &=& \ocscenM(\ocsH) & 
%  \end{array}}
%  }
%\end{theorem}
%\begin{proof}
%\begin{align*}
%  \ocscen(\ocsG)
%    &\eqd \argmin_{x\in\ocso}\max_{y\in\ocso} \metric{x}{y}\psp(y)
%    && \text{by definition of $\ocscen$ \xref{def:ocscen} and $\ocsG$}
%  \\&= \argmin_{x\in\ocso}\max_{y\in\ocso} \metrica{x}{y} \psp(y)
%    && \text{by $\ff$ is \prope{strictly isotone} hypothesis and \prefpp{lem:argminmaxphi}}
%  \\&\eqd \ocscen(\ocsK)
%    && \text{by definition of $\ocscen$ \xref{def:ocscen} and $\ocsK$}
%  \\
%  \ocscenm(\ocsG)
%    &\eqd \argmin_{x\in\ocso}\min_{y\in\ocso} \metric{x}{y}\psp(y)
%    && \text{by definition of $\ocscen$ \xref{def:ocscen} and $\ocsG$}
%  \\&= \argmin_{x\in\ocso}\min_{y\in\ocso} \metrica{x}{y} \psp(y)
%    && \text{by $\ff$ is \prope{strictly isotone} hypothesis and \prefpp{lem:argminmaxphi}}
%  \\&\eqd \ocscenm(\ocsK)
%    && \text{by definition of $\ocscen$ \xref{def:ocscen} and $\ocsK$}
%  \\
%  \ocscenM(\ocsG)
%    &\eqd \argmax_{x\in\ocso}\min_{y\in\ocso} \metric{x}{y}\psp(y)
%    && \text{by definition of $\ocscen$ \xref{def:ocscen} and $\ocsG$}
%  \\&= \argmax_{x\in\ocso}\min_{y\in\ocso} \metrica{x}{y} \psp(y)
%    && \text{by $\ff$ is \prope{strictly isotone} hypothesis and \prefpp{lem:argminmaxphi}}
%  \\&\eqd \ocscenM(\ocsK)
%    && \text{by definition of $\ocscen$ \xref{def:ocscen} and $\ocsK$}
%\end{align*}
%\end{proof}
%---------------------------------------
\begin{remark}
%---------------------------------------
The quantity $\ds\metrican\brp{x,\ocso}\eqd \sum_{y\in\ocso}\metric{x}{y}\psp(y)$
in the \structe{arithmetic center} $\ocscena(\ocsG)$ \xref{def:ocscena}
is itself a \structe{metric} \xref{def:metric}.
Thus, $\ocscena(\ocsG)$ is the $x$ that produces the minimum of all the metrics with center $x$.
\end{remark}
\begin{proof}
This follows directly from \structe{power mean metrics} theorem with $r=1$ \xref{thm:met_power}.
\end{proof}

%%---------------------------------------
%\begin{proposition}
%%---------------------------------------
%Let $\ocsmom_n(x,y)$ be the \structe{$n$th-moment} \xref{def:ocsmom} 
%on the \structe{outcome subspace} $\ocsD$ \xref{def:ocs}.
%\propbox{
%  \ocsmom_n(x,x)=0 \qquad n=1,2,3,\ldots
%  }
%\end{proposition}
%\begin{proof}
%\begin{align*}
%  \ocsmom_n(x,x)
%    &= \brs{\metric{x}{x}}^n\psp(x)
%    && \text{by \prefp{def:ocsmom}}
%  \\&= \brs{0}^n\psp(x)
%    && \text{by \prope{nondegenerate} property of \fncte{metric}s \xref{def:metric}}
%  \\&= 0 
%    && \text{by field property of $0$}
%\end{align*}
%\end{proof}
%
%%---------------------------------------
%\begin{proposition}
%%---------------------------------------
%Let $\ocsG\eqd\ocsD$ be an \structe{outcome subspace} \xref{def:ocs}.
%Let $\ocsmom(x,y)$ be the \structe{moment} \xref{def:ocsmom} on $\ocsG$
%and $\ocscen(\ocsG)$ be the \structe{center} \xref{def:ocscen} of $\ocsG$.
%\propbox{
%  \ocscen(\ocsG) = \argmin_{x\in\ocso}\max_{y\in\ocso}\ocsmom(x,y)
%  }
%\end{proposition}
%\begin{proof}
%\begin{align*}
%  \ocscen(\ocsG)
%    &\eqd \argmin_{x\in\ocso}\max_{y\in\ocso}\metric{x}{y}\psp(y)
%    && \text{by definition of $\ocscen$ \xref{def:ocscen}}
%  \\&= \argmin_{x\in\ocso}\max_{y\in\ocso}\ocsmom(x,y)
%    && \text{by definition of $\ocsmom$ \xref{def:ocsmom}}
%\end{align*}
%\end{proof}

%\if 0
%=======================================
\subsection{Specific outcome subspaces}
%=======================================
\begin{figure}[h]
  \centering%
  \footnotesize%
  \begin{tabular}{|*{4}{c|}}
    \hline
     \includegraphics{sto/graphics/ocs_fdie.pdf}%
    &\includegraphics{sto/graphics/ocs_rdie.pdf}%
    &\includegraphics{sto/graphics/ocs_wrdie.pdf}%
    &\includegraphics{sto/graphics/ocs_wdie.pdf}%
    \\
      \structe{fair die} \xrefr{def:fdie}%
     &\structe{real die} \xrefr{def:rdie}%
     &\structe{weighted real die} \xrefr{def:wrdie}%
     &\structe{weighted die} \xrefr{def:wdie}%
    \\\hline
     \includegraphics{sto/graphics/ocs_spinner.pdf}%
    &\includegraphics{sto/graphics/ocs_dna.pdf}%
    &\includegraphics{sto/graphics/ocs_dnan.pdf}%
    &
    \\
     \structe{spinner} \xrefr{def:spinner}%
    &\structe{DNA} \xrefr{def:dna}%
    &\structe{scaffold DNA} \xrefr{def:dnan}%
    &
    \\\hline
  \end{tabular}%
  \caption{example \structe{outcome subspace}s \xref{def:ocs} 
     illustrated by \structe{weighted graph}s\label{fig:ocs}
     with shaded \ope{expected value}s.
     %For more details, see \citeP{greenhoe2015sto}%
     }
\end{figure}

%---------------------------------------
\begin{definition}
%\footnote{\citeP{greenhoe2015sto}}
\label{def:wdie}
%---------------------------------------
\defboxp{
  The structure
  $\ocsG\eqd\ocs{\setn{\diceA,\diceB,\diceC,\diceD,\diceE,\diceF}}{\ocsd}{\ocsr}{\ocsp}$
  is the \structd{weighted die outcome subspace} if $\ocsG$ is an \structe{outcome subspace}, %\xref{def:ocsm},
  $\ocsr=\emptyset$ (\prope{unordered} \xrefnp{def:unordered}),
  and $\ocsd$ is the \fncte{discrete metric} \xref{def:discretemetric}.
  }
\end{definition}

%---------------------------------------
\begin{definition}
%\footnote{\citeP{greenhoe2015sto}}
\label{def:fdie}
%---------------------------------------
\defboxp{
  The structure
  $\ocsG\eqd\ocs{\setn{\diceA,\diceB,\diceC,\diceD,\diceE,\diceF}}{\ocsd}{\ocsr}{\ocsp}$
  is the \structd{fair die outcome subspace} if $\ocsG$ is a \structe{weighted die outcome subspace} \xrefr{def:wdie},
  and 
  \\\indentx$\ocsp(\dieA)=\ocsp(\dieB)=\ocsp(\dieC)=\ocsp(\dieD)=\ocsp(\dieE)=\ocsp(\dieF)=\sfrac{1}{6}$.
  }
\end{definition}

%---------------------------------------
\begin{minipage}{\tw-53mm}%
\begin{definition}
%\footnotemark
\label{def:wrdie}
%---------------------------------------
The structure
$\ocsG\eqd\ocs{\setn{\diceA,\diceB,\diceC,\diceD,\diceE,\diceF}}{\ocsd}{\emptyset}{\ocsp}$
is the \structd{weighted real die outcome subspace} if $\ocsG$ is an \structe{outcome subspace}, % \xref{def:ocsm},
%$\ocsr=\emptyset$, % (\prope{unordered} \xrefnp{def:unordered}),
%\\\indentx$\ocsp(\dieA)=\ocsp(\dieB)=\ocsp(\dieC)=\ocsp(\dieD)=\ocsp(\dieE)=\ocsp(\dieF)=\sfrac{1}{6}$,\\
and \fncte{metric} $\ocsd$ is defined as in the table to the right.
\end{definition}
\end{minipage}%
%\citetblt{\citeP{greenhoe2015sto}}%
\hfill%
  $\begin{tabstr}{0.75}\begin{array}{|c|*{6}{@{\hspace{2pt}}c}|}
    \hline
    \ocsd(x,y) & \dieA &\dieB &\dieC &\dieD &\dieE &\dieF
    \\\hline
      \dieA &    0   &   1   &   1   &   1   &   1   &   2
    \\\dieB &    1   &   0   &   1   &   1   &   2   &   1
    \\\dieC &    1   &   1   &   0   &   2   &   1   &   1
    \\\dieD &    1   &   1   &   2   &   0   &   1   &   1
    \\\dieE &    1   &   2   &   1   &   1   &   0   &   1
    \\\dieF &    2   &   1   &   1   &   1   &   1   &   0
    \\\hline
  \end{array}\end{tabstr}$

%---------------------------------------
\begin{definition}
%\footnote{\citeP{greenhoe2015sto}}
\label{def:rdie}
%---------------------------------------
The structure
$\ocsG\eqd\ocs{\setn{\diceA,\diceB,\diceC,\diceD,\diceE,\diceF}}{\ocsd}{\emptyset}{\ocsp}$
is the \structd{real die outcome subspace} if $\ocsG$ is a
\structe{weighted real die outcome subspace} \xref{def:wrdie} with
\\\indentx$\ocsp(\dieA)=\ocsp(\dieB)=\ocsp(\dieC)=\ocsp(\dieD)=\ocsp(\dieE)=\ocsp(\dieF)=\sfrac{1}{6}$.
\end{definition}

%---------------------------------------
\begin{minipage}{\tw-53mm}%
\begin{definition} %[\structd{real die outcome subspace}]
%\footnote{\citeP{greenhoe2015sto}}
\label{def:spinner}%\mbox{}\\
%---------------------------------------
The structure
$\ocsG\eqd\ocs{\setn{\circOne,\circTwo,\circThree,\circFour,\circFive,\circSix}}{\ocsd}{\emptyset}{\ocsp}$
is the 
\\\structd{spinner outcome subspace} if $\ocsG$ is an \structe{outcome subspace}, % \xref{def:ocsm},
%\\$\ocsr=\emptyset$, % (\prope{unordered} \xrefnp{def:unordered}),
\\\indentx$\ocsp(\circOne)=\ocsp(\circTwo)=\ocsp(\circThree)=\ocsp(\circFour)=\ocsp(\circFive)=\ocsp(\circSix)=\sfrac{1}{6}$,
\\and \fncte{metric} $\ocsd$ is defined as in the table to the right.
\end{definition}
\end{minipage}%
\hfill%
  $\begin{tabstr}{0.75}\begin{array}{|c|*{6}{@{\hspace{5pt}}c}|}
    \hline
    \ocsd(x,y)    & \circOne & \circTwo & \circThree & \circFour & \circFive & \circSix
    \\\hline
      \circOne    &    0     &   1      &   2        &   3       &   2       &   1
    \\\circTwo    &    1     &   0      &   1        &   2       &   3       &   2
    \\\circThree  &    2     &   1      &   0        &   1       &   2       &   3
    \\\circFour   &    3     &   2      &   1        &   0       &   1       &   2
    \\\circFive   &    2     &   3      &   2        &   1       &   0       &   1
    \\\circSix    &    1     &   2      &   3        &   2       &   1       &   0
    \\\hline
  \end{array}\end{tabstr}$

%---------------------------------------
\begin{definition}
%\footnote{\citeP{greenhoe2015sto}}
\label{def:dna}
%---------------------------------------
The structure
$\ocsH\eqd\ocs{\setn{\symA,\symT,\symC,\symG}}{\ocsd}{\emptyset}{\ocsp}$
is the \structd{DNA outcome subspace}, or \structd{genome outcome subspace}, 
if $\ocsH$ is an \structe{outcome subspace}, % \xref{def:ocsm},
and $\ocsd$ is the \fncte{discrete metric} \xref{def:dmetric}.
\end{definition}

%---------------------------------------
\begin{minipage}{\tw-58mm}%
\begin{definition}
%\footnotemark
\label{def:dnan}
%---------------------------------------
The structure
$\ocsH\eqd\ocs{\setn{\symA,\symT,\symC,\symG,\symN}}{\ocsd}{\ocsr}{\ocsp}$\\
is the \structd{DNA scaffold outcome subspace}, or \structd{genome scaffold outcome subspace}, 
if $\ocsH$ is an \structe{outcome subspace}, % \xref{def:ocsm},
\\\indentx
  $\ocsr=\setn{\opair{\symN}{\symA},\,\opair{\symN}{\symT},\,\opair{\symN}{\symC},\,\opair{\symN}{\symG},\,}$ 
\\
($\symN<\symA$, $\symN<\symT$, $\symN<\symC$, and $\symN<\symG$, but otherwise \prope{unordered}),\\
$\ocsp$ is a \fncte{probability function},
and \fncte{metric} $\ocsd$ is defined as in the table to the right.
\end{definition}
\end{minipage}%
%\citetblt{\citeP{greenhoe2015sto}}%
\hfill%
  $\begin{tabstr}{0.75}\begin{array}{|c|*{5}{@{\hspace{5pt}}c}|}
    \hline
    \ocsd(x,y)    & \symN    & \symA    & \symT      & \symC     & \symG 
    \\\hline
      \symN       &    \ds  0                 & \ds\sfrac{\sqrt{2}}{2} & \ds\sfrac{\sqrt{2}}{2} & \ds\sfrac{\sqrt{2}}{2}  & \ds\sfrac{\sqrt{2}}{2} 
    \\\symA       &    \ds\sfrac{\sqrt{2}}{2} &   0                 &   1                 &   1                  &   1   
    \\\symT       &    \ds\sfrac{\sqrt{2}}{2} &   1                 &   0                 &   1                  &   1   
    \\\symC       &    \ds\sfrac{\sqrt{2}}{2} &   1                 &   1                 &   0                  &   1   
    \\\symG       &    \ds\sfrac{\sqrt{2}}{2} &   1                 &   1                 &   1                  &   0   
    \\\hline
  \end{array}\end{tabstr}$


%=======================================
\subsection{Example calculations}
%=======================================
\begin{tabstr}{0.75}
%---------------------------------------
\begin{example}%[\exmd{fair die outcome subspace} / \exmd{discrete outcome subspace}]%\mbox{}\\
\label{ex:fairdie}
%---------------------------------------
%\begin{minipage}{\tw-35mm}%
Let $\ocsG\eqd\ocs{\setn{\diceA,\diceB,\diceC,\diceD,\diceE,\diceF}}{\ocsd}{\ocsr}{\ocsp}$ 
be the \structe{fair die outcome subspace} \xref{def:fdie}.
This structure is illustrated by the \structe{weighted graph} \xref{def:wgraph} in \prefpp{fig:ocs} (A), %to the right, 
where each line segment represents a distance of $1$.
%
%be the \structe{outcome subspace} \xref{def:ocsm} generated by a \structe{fair die}
%where $\metricn$ is the \fncte{discrete metric} \xref{def:dmetric},
%$\orel\eqd\emptyset$ (completely unordered set), and 
%$\psp(\diceA)=\psp(\diceB)=\cdots=\psp(\diceF)=\frac{1}{6}$.
%This is illustrated by the \structe{weighted graph} \xref{def:wgraph} to the right, 
%where each line segment represents a distance of $1$, and elements in the \structe{outcome center} $\ocscen(\ocsG)$
%are \colorbox{snode}{shaded}.
This structure has the following geometric values:
  \\\indentx$\begin{array}{l}
      \ocscen(\ocsG)=\ocscena(\ocsG)=\ocsceng(\ocsG)=\ocscenh(\ocsG)=\ocscenm(\ocsG)=\ocscenM(\ocsG)  
         = \setn{\diceA,\diceB,\diceC,\diceD,\diceE,\diceF} %& \xxref{def:ocscen}{def:ocscenx}  & (shaded in illustration)
    \\\ocsVaro(\ocsG) = \ocsVaro(\ocsG;\ocscena) = \ocsVaro(\ocsG;\ocsceng) = \ocsVaro(\ocsG;\ocscenh) = \ocsVaro(\ocsG;\ocscenm) = \ocsVaro(\ocsG;\ocscenM)  =  0     
  \end{array}$
%\end{minipage}\hfill%
%\begin{tabular}{c}
%  \gsize%
%  \centering%
%  {\includegraphics{sto/graphics/ocs_fdie.pdf}}%
%\end{tabular}
\end{example}
\begin{proof}
\begin{align*}
  \ocscen(\ocsG)
    &\eqd \argmin_{x\in\ocsG}\max_{y\in\ocsG}\metric{x}{y}\psp(y)
    && \text{by definition of $\ocscen$ \xref{def:ocscen}}
  \\&= \argmin_{x\in\ocsG}\max_{y\in\ocsG}\metric{x}{y}\frac{1}{6}
    && \text{by definition of $\ocsG$}
  %\\&= \argmin_{x\in\ocsG}\max_{y\in\ocsG}\metric{x}{y}
  %  && \text{because $\ff(x)=\frac{1}{6}x$ is \prope{strictly isotone} and by \prefpp{lem:argminmaxphi}}
  \\&= \argmin_{x\in\ocsG}\frac{1}{6}\setn{1,\,1,\,1,\,1,\,1,\,1}
    && \text{by definition of \structe{discrete metric} \xref{def:dmetric}}
  \\&= \setn{\diceA,\diceB,\diceC,\diceD,\diceE,\diceF}
    && \text{by definition of $\ocsG$}
  \\
  \ocscena(\ocsG)
    &\eqd \argmin_{x\in\ocsG}\sum_{y\in\ocsG}\metric{x}{y}\psp(y)
    && \text{by definition of $\ocscena$ \xref{def:ocscenx}}
  \\&= \argmin_{x\in\ocsG}\sum_{y\in\ocsG}\metric{x}{y}\frac{1}{6}
    && \text{by definition of $\ocsG$}
  %\\&= \argmin_{x\in\ocsG}\sum_{y\in\ocsG}\metric{x}{y}
  %  && \text{because $\ff(x)=\frac{1}{6}x$ is \prope{strictly isotone} and by \prefpp{lem:argminmaxphi}}
  \\&= \argmin_{x\in\ocsG}\frac{1}{6}\setn{5,\,5,\,5,\,5,\,5,\,5}
    && \text{by definition of \structe{discrete metric} \xref{def:dmetric}}
  \\&= \setn{\diceA,\diceB,\diceC,\diceD,\diceE,\diceF}
    && \text{by definition of $\ocsG$}
  \\
  \ocsceng(\ocsG)
    &\eqd \argmin_{x\in\ocsG}\prod_{y\in\ocsG\setd\setn{x}}\metric{x}{y}^{\psp(y)}
    &&\text{by definition of $\ocsceng$ \xref{def:ocsceng}}
  \\&= \argmin_{x\in\ocsG}\prod_{y\in\ocsG\setd\setn{x}}1^{\frac{1}{6}}
    && \text{by definition of $\ocsG$ and \structe{discrete metric} \xref{def:dmetric}}
  \\&= \argmin_{x\in\ocsG}\setn{1,1,1,1,1,1}
  \\&= \setn{\diceA,\diceB,\diceC,\diceD,\diceE,\diceF}
    && \text{by definition of $\ocsG$}
  \\
  \ocscenh(\rvX)
    &\eqd \argmin_{x\in\ocsG}\brp{\sum_{y\in\ocsG\setd\setn{x}}\frac{1}{\metric{x}{y}}\psp(y)}^{-1}
    &&\text{by definition of $\ocscenh$ \xref{def:ocscenh}}
  \\&= \argmin_{x\in\ocsG}\brp{\sum_{y\in\ocsG\setd\setn{x}}1\times\frac{1}{6}}^{-1}
    &&\text{by definition of $\ocsG$}
  %\\&= \argmin_{x\in\ocsG}\brp{\sum_{y\in\ocsG\setd\setn{x}}1\times\frac{1}{6}}^{-1}
  \\&= \argmin_{x\in\ocsG}\setn{\frac{6}{5},\frac{6}{5},\frac{6}{5},\frac{6}{5},\frac{6}{5},\frac{6}{5}}
  \\&= \setn{\diceA,\diceB,\diceC,\diceD,\diceE,\diceF}
    && \text{by definition of $\ocsG$}
  \\
  \ocscenm(\ocsG)
    &\eqd \argmin_{x\in\ocsG}\min_{y\in\ocsG\setd\setn{x}}\metric{x}{y}\psp(y)
    &&\text{by definition of $\ocscenm$ \xref{def:ocscenm}}
  \\&= \argmin_{x\in\ocsG}\min_{y\in\ocsG\setd\setn{x}}1\times\frac{1}{6}
    &&\text{by definition of $\ocsG$ and \structe{discrete metric} \xref{def:dmetric}}
  \\&= \argmin_{x\in\ocsG}\frac{1}{6}\setn{1,1,1,1,1,1}
  \\&= \setn{\diceA,\diceB,\diceC,\diceD,\diceE,\diceF}
    && \text{by definition of $\ocsG$}
  \\
  \ocscenM(\ocsG)
    &\eqd \argmax_{x\in\ocsG}\min_{y\in\ocsG\setd\setn{x}}\metric{x}{y}\psp(y)
    &&\text{by definition of $\ocscenM$ \xref{def:ocscenM}}
  \\&= \argmax_{x\in\ocsG}\frac{1}{6}\setn{1,1,1,1,1,1}
    && \text{by $\ocscenm(\ocsG)$ result}
  \\&= \setn{\diceA,\diceB,\diceC,\diceD,\diceE,\diceF}
    && \text{by definition of $\ocsG$}
  \\
  \ocsVaro(\ocsG;\ocscena)
    &= \mathrlap{\ocsVaro(\ocsG;\ocsceng)= \ocsVaro(\ocsG;\ocscenh)= \ocsVaro(\ocsG;\ocscenm)= \ocsVaro(\ocsG;\ocscenM) = \ocsVaro(\ocsG)}
  \\&\eqd \sum_{x\in\ocsG}\brs{\metric{\ocscen(\ocsG)}{x}}^2\psp(x)
    && \text{by definition of $\ocsVaro$ \xref{def:ocsVarG}}
  \\&= \sum_{x\in\ocsG}\brp{0^2}\frac{1}{6}
    && \text{because $\ocscen(\ocsG)=\ocsG$}
  \\&= 0
    && \text{by field property of \vale{additive identity element} $0$}
\end{align*}
\end{proof}


%---------------------------------------
\begin{example} %[\exmd{real die outcome subspace}]
\label{ex:realdie}%\mbox{}\\
%---------------------------------------
%\begin{minipage}{\tw-35mm}%
Let $\ocsG\eqd\ocs{\setn{\diceA,\diceB,\diceC,\diceD,\diceE,\diceF}}{\emptyset}{\orel}{\psp}$ 
be the \structe{real die outcome subspace} \xref{def:rdie}.
%be the \structe{outcome subspace} \xref{def:ocsm} generated by a \structe{real die}.
%On a real die (as opposed to a \structe{fair die} \xrefnp{ex:fairdie}),
%some symbols (die faces) are physically closer (in real 3 dimensional space) than others.
%Instead of using the discrete metric, we define the distance $\metric{x}{y}$ from face $x$ to face $y$
%to be the number of physical die edges that must be crossed to go from $x$ to $y$.
%In this case, it is still true that $\metric{\diceA}{\diceA}=0$, and
%\\\indentx$\begin{array}{rclclclclM}
%  \metric{\diceA}{\diceB}&=&\metric{\diceA}{\diceC}&=&\metric{\diceA}{\diceD}&=&\metric{\diceA}{\diceE}&=&1,&but\\
%  \metric{\diceA}{\diceF}&=&\metric{\diceB}{\diceE}&=&\metric{\diceC}{\diceD}&=&2&\neq& 1.
%\end{array}$
%\end{minipage}\hfill%
%\begin{tabular}{c}
%  \gsize%
%  %\psset{unit=5mm}%
%  %{%============================================================================
% Daniel J. Greenhoe
% LaTeX file
% discrete metric real dice mapping to linearly ordered O6c
%============================================================================
{%\psset{unit=0.5\psunit}%
\begin{pspicture}(-1.4,-1.4)(1.4,1.4)%
  %---------------------------------
  % options
  %---------------------------------
  \psset{%
    radius=1.25ex,
    labelsep=2.5mm,
    linecolor=blue,%
    }%
  %---------------------------------
  % dice graph
  %---------------------------------
  \rput(0,0){%\psset{unit=2\psunit}%
    \uput{1}[210](0,0){\Cnode[fillstyle=solid,fillcolor=snode](0,0){D4}}%
    \uput{1}[150](0,0){\Cnode[fillstyle=solid,fillcolor=snode](0,0){D5}}%
    \uput{1}[ 90](0,0){\Cnode[fillstyle=solid,fillcolor=snode](0,0){D6}}%
    \uput{1}[ 30](0,0){\Cnode[fillstyle=solid,fillcolor=snode](0,0){D3}}%
    \uput{1}[-30](0,0){\Cnode[fillstyle=solid,fillcolor=snode](0,0){D2}}%
    \uput{1}[-90](0,0){\Cnode[fillstyle=solid,fillcolor=snode](0,0){D1}}%
    }
  \rput(D6){$\diceF$}%
  \rput(D5){$\diceE$}%
  \rput(D4){$\diceD$}%
  \rput(D3){$\diceC$}%
  \rput(D2){$\diceB$}%
  \rput(D1){$\diceA$}%
  %
  \ncline{D5}{D6}%
  \ncline{D4}{D5}\ncline{D4}{D6}%
  \ncline{D3}{D5}\ncline{D3}{D6}%
  \ncline{D2}{D3}\ncline{D2}{D4}\ncline{D2}{D6}%
  \ncline{D1}{D2}\ncline{D1}{D3}\ncline{D1}{D4}\ncline{D1}{D5}%
  %
  \uput[158](D6){$\frac{1}{6}$}
  \uput[150](D5){$\frac{1}{6}$}
  \uput[210](D4){$\frac{1}{6}$}
  \uput[ 22](D3){$\frac{1}{6}$}
  \uput[-45](D2){$\frac{1}{6}$}
  \uput[-158](D1){$\frac{1}{6}$}
\end{pspicture}
}%}%
%  {\includegraphics{sto/graphics/ocs_rdie.pdf}}%
%\end{tabular}
%\\
This structure is illustrated by the \structe{weighted graph} \xref{def:wgraph} in \prefpp{fig:ocs} (B), %to the right, 
where each line segment represents a distance of $1$.
The structure has the following geometric values:
\\\indentx$\begin{array}{rclDD}
  \ocscen(\ocsG)=\ocscena(\ocsG)=\ocsceng(\ocsG)=\ocscenh(\ocsG)=\ocscenm(\ocsG)=\ocscenM(\ocsG)  
    &=& \setn{\diceA,\diceB,\diceC,\diceD,\diceE,\diceF} &%\xxref{def:ocscen}{def:ocscenx}  & (shaded in illustration)
    \\
  \ocsVaro(\ocsG) = \ocsVaro(\ocsG;\ocscena) = \ocsVaro(\ocsG;\ocsceng) = \ocsVaro(\ocsG;\ocscenh) = \ocsVaro(\ocsG;\ocscenm) = \ocsVaro(\ocsG;\ocscenM) &=& 0
\end{array}$
\end{example}
\begin{proof}
\begin{align*}
  \ocscen(\ocsG)
    &\eqd \argmin_{x\in\ocsG}\max_{y\in\ocsG}\metric{x}{y}\psp(y)
    && \text{by definition of $\ocscen$ \xref{def:ocscen}}
  \\&= \argmin_{x\in\ocsG}\max_{y\in\ocsG}\metric{x}{y}\frac{1}{6}
    && \text{by definition of $\ocsG$}
  %\\&= \argmin_{x\in\ocsG}\max_{y\in\ocsG}\metric{x}{y}
  %  && \text{because $\ff(x)=\frac{1}{6}x$ is \prope{strictly isotone} and by \prefpp{lem:argminmaxphi}}
  \\&= \argmin_{x\in\ocsG}\frac{1}{6}\setn{2,\,2,\,2,\,2,\,2,\,2}
    && \text{because for each $x$, there is a $y$ such that $\metric{x}{y}=2$}
  \\&= \setn{\diceA,\diceB,\diceC,\diceD,\diceE,\diceF}
    && \text{by definition of $\ocsG$}
  \\
  \ocscena(\ocsG)
    &\eqd \argmin_{x\in\ocsG}\sum_{y\in\ocsG}\metric{x}{y}\psp(y)
    &&\text{by definition of $\ocscena$ \xref{def:ocscenx}}
  \\&= \argmin_{x\in\ocsG}\sum_{y\in\ocsG}\metric{x}{y}\frac{1}{6}
    &&\text{by definition of $\ocsG$}
  %\\&= \argmin_{x\in\ocsG}\sum_{y\in\ocsG}\metric{x}{y}
  %  &&\text{because $\ff(x)=\frac{1}{6}x$ is \prope{strictly isotone} and by \prefpp{lem:argminmaxphi}}
  \\&= \mathrlap{\argmin_{x\in\ocsG}\frac{1}{6}
         \setn{\begin{array}{*{11}{@{\hspace{2pt}}c}}
           0 &+& 1 &+& 1 &+& 1 &+& 1 &+& 2\\
           1 &+& 0 &+& 1 &+& 1 &+& 2 &+& 1\\
           1 &+& 1 &+& 0 &+& 2 &+& 1 &+& 1\\
           1 &+& 1 &+& 2 &+& 0 &+& 1 &+& 1\\
           1 &+& 2 &+& 1 &+& 1 &+& 0 &+& 1\\
           2 &+& 1 &+& 1 &+& 1 &+& 1 &+& 0
         \end{array}}
       = \argmin_{x\in\ocsG}\frac{1}{6}
         \setn{\begin{array}{c}
           6\\
           6\\
           6\\
           6\\
           6\\
           6
         \end{array}}
       = \setn{\begin{array}{c}
           \diceA\\
           \diceB\\
           \diceC\\
           \diceD\\
           \diceE\\
           \diceF
         \end{array}}}
    \\
  \ocsceng(\ocsG)
    &\eqd \argmin_{x\in\ocsG}\prod_{y\in\ocsG\setd\setn{x}}\metric{x}{y}^{\psp(y)}
    &&\text{by definition of $\ocsceng$ \xref{def:ocsceng}}
  \\&\eqd \argmin_{x\in\ocsG}\prod_{y\in\ocsG\setd\setn{x}}\metric{x}{y}^{\frac{1}{6}}
    &&\text{by definition of $\ocsG$}
  \\&= \argmin_{x\in\ocsG}\brp{\prod_{y\in\ocsG\setd\setn{x}}\metric{x}{y}}^{\frac{1}{6}}
  \\&= \argmin_{x\in\ocsG}\prod_{y\in\ocsG\setd\setn{x}}\metric{x}{y}
    &&\text{because $\ff(x)=x^\frac{1}{6}$ is \prope{strictly isotone} and by \prefpp{lem:argminphi}}
  \\&= \argmin_{x\in\ocsG}\setn{2,2,2,2,2,2}
  \\&= \setn{\diceA,\diceB,\diceC,\diceD,\diceE,\diceF}
  \\
  \ocscenh(\rvX)
    &\eqd \argmin_{x\in\ocsG}\brp{\sum_{y\in\ocsG\setd\setn{x}}\frac{1}{\metric{x}{y}}\psp(y)}^{-1}
    &&\text{by definition of $\ocscenh$ \xref{def:ocscenh}}
  \\&= \argmin_{x\in\ocsG}\brp{\sum_{y\in\ocsG\setd\setn{x}}\frac{1}{\metric{x}{y}}\frac{1}{6}}^{-1}
    && \text{by definition of $\ocsG$}
  \\&= \argmin_{x\in\ocsG}6\brp{\sum_{y\in\ocsG\setd\setn{x}}\frac{1}{\metric{x}{y}}}^{-1}
  \\&= \argmax_{x\in\ocsG}\frac{1}{2}6\sum_{y\in\ocsG\setd\setn{x}}\frac{2}{\metric{x}{y}}
    && \text{because $\fphi(x)\eqd x^{-1}$ is \prope{strictly antitone} and by \prefp{lem:minphia}}
  \\&= \mathrlap{\argmin_{x\in\ocsG}3
         \setn{\begin{array}{*{11}{c}}
           0 &+& 2 &+& 2 &+& 2 &+& 2 &+& 1\\
           2 &+& 0 &+& 2 &+& 2 &+& 1 &+& 2\\
           2 &+& 2 &+& 0 &+& 1 &+& 2 &+& 2\\
           2 &+& 2 &+& 1 &+& 0 &+& 2 &+& 2\\
           2 &+& 1 &+& 2 &+& 2 &+& 0 &+& 2\\
           1 &+& 2 &+& 2 &+& 2 &+& 2 &+& 0
         \end{array}}
       = \argmin_{x\in\ocsG}\frac{1}{6}
         \setn{\begin{array}{c}
           9\\
           9\\
           9\\
           9\\
           9\\
           9
         \end{array}}
       = \setn{\begin{array}{c}
           \diceA\\
           \diceB\\
           \diceC\\
           \diceD\\
           \diceE\\
           \diceF
         \end{array}}}
  \\
  \ocscenm(\ocsG)
    &\eqd \argmin_{x\in\ocsG}\min_{y\in\ocsG\setd\setn{x}}\metric{x}{y}\psp(y)
    &&\text{by definition of $\ocscenm$ \xref{def:ocscenm}}
  \\&= \argmin_{x\in\ocsG}\min_{y\in\ocsG\setd\setn{x}}1\times\frac{1}{6}
    &&\text{by definition of $\ocscenm$ \xref{def:ocscen}}
  \\&= \frac{1}{6}\setn{2,2,2,2,2,2}
  \\&= \setn{\diceA,\diceB,\diceC,\diceD,\diceE,\diceF}
    && \text{by definition of $\ocsG$}
  \\
  \ocscenM(\ocsG)
    &\eqd \argmax_{x\in\ocsG}\min_{y\in\ocsG\setd\setn{x}}\metric{x}{y}\psp(y)
    &&\text{by definition of $\ocscenM$ \xref{def:ocscenM}}
  \\&= \frac{1}{6}\setn{2,2,2,2,2,2}
    && \text{by $\ocscenm(\ocsG)$ result}
  \\&= \setn{\diceA,\diceB,\diceC,\diceD,\diceE,\diceF}
    && \text{by definition of $\ocsG$}
  \\
  \ocsVaro(\ocsG;\ocscena)
    &= \mathrlap{\ocsVaro(\ocsG;\ocsceng)= \ocsVaro(\ocsG;\ocscenh)= \ocsVaro(\ocsG;\ocscenm)= \ocsVaro(\ocsG;\ocscenM) = \ocsVaro(\ocsG)}
  \\&\eqd \sum_{x\in\ocsG}\brs{\metric{\ocscen(\ocsG)}{x}}^2\psp(x)
    && \text{by definition of $\ocsVaro$ \xref{def:ocsVarG}}
  \\&= \sum_{x\in\ocsG}(0)\frac{1}{6}
    && \text{because $\ocscen(\ocsG)=\ocsG$}
  \\&= 0
    && \text{by field property of \vale{additive identity element} $0$}
\end{align*}
\end{proof}


%---------------------------------------
\begin{remark}
\label{rem:dietop}
\label{rem:mgeo}
%---------------------------------------
Let $\ocsG\eqd\ocs{\ocso}{\metricn} {\orel}{\psp}$ be the \structe{fair die outcome subspace} \xref{ex:fairdie}.
Let $\ocsH\eqd\ocs{\ocso}{\metrican}{\orel}{\psp}$ be the \structe{real die outcome subspace} \xref{ex:realdie}.
These two subspaces are identical except for their metrics $\metricn$ and $\metrican$.
So we can say that $\ocsG$ and $\ocsH$ \emph{are} distinguished by their metrics.
However, note that they are \emph{indistinguishable} by the topologies induced by their metrics,
because they both induce the same topology---the \structe{discrete topology} $\pset{\ocso}$ \xref{def:pset}.
That is, the geometric distinction provided in metric spaces is in general lost in topological spaces.
Thus, topological spaces are arguably too general for the type of stochastic processing presented in this paper;
rather, the stochastic processing discussed in this paper calls for metric space structure.
And in this paper, this type of metric space structure is referred to as \structd{metric geometry}.
\end{remark}
\begin{proof}
\begin{enumerate}
  \item Every \structe{metric space} $\opair{\ocso}{\metricn}$ \xref{def:metric} 
        induces a \structd{topological space} $\opair{\ocso}{\topT}$.
  \item In particular, a \structe{metric} $\metricn$ induces an \structd{open ball} 
        $\ball{x}{r}\in\clF{\brp{\ocso\times\Rp}}{\brp{\pset{\ocso}}}$ centered at $x$ with radius $r$ such that 
        $\ball{x}{r}\eqd\set{y\in\ocso}{\metric{x}{y}<r}$.
  \item At each outcome $x$ in $\ocsG$, only two \structe{open ball}s are possible:
        $\ball{x}{r}=\brb{\begin{array}{lM}
                            \setn{x}  & for $0<r\le1$\\
                            \ocso     & for $r>1$
                          \end{array}}$.
  \item Let $x'$ represent the die face which, 
        when its numeric value is summed ``in the usual way" with the numeric value of the die face $x$,
        equals $7$.
        Then at each point $x$ in $\ocsH$, three \structe{open ball}s are possible:\\
        $\ball{x}{r}=\brb{\begin{array}{lM}
                             \setn{x}     & for $0<r\le1$\\
                             \setn{x,x'}  & for $1<r\le2$\\
                             \ocso        & for $r>2$
                           \end{array}}$.

  \item The \structe{open ball}s of $\opair{\ocso}{\metricn}$ or $\opair{\ocso}{\metrican}$ 
        in turn induce a \structd{base} for a \structd{topology} $\topT$, such that
        \\$\topT=\set{\setU\in\pset{\ocso}}{\text{$\setU$ is a union of \structe{open ball}s}}$.
        The topology induced by $\ocsG$ is the \structe{discrete topology} $\pset{\ocso}$ \xref{def:pset}.
        The topology induced by $\ocsH$ is also the \structe{discrete topology} $\pset{\ocso}$.

  \item So the metrics of $\ocsG$ and $\ocsH$ are different.
        And the balls induced by $\ocsG$ and those induced by $\ocsH$ are different.
        However, the topologies induced by $\ocsG$ and $\ocsH$ are the same.
        %they both induce the \structe{discrete topology} $\pset{\ocso}$.
\end{enumerate}
\end{proof}


%---------------------------------------
\begin{example} %[\exmd{weighted real die outcome subspace}]
\label{ex:wdie}
%---------------------------------------
%A \structe{weighted die} generates an \structe{outcome subspace} $\ocsG$ \xref{def:ocs}.
The \structe{weighted real die outcome subspace} \xref{def:wdie} illustrated in \prefp{fig:ocs} (C) 
has the following geometric characteristics:
\\ %\begin{minipage}{\tw-35mm}%
\indentx$\begin{array}{rcl @{\qquad} lclcl}
    \ocscen(\ocsG)=\ocscena(\ocsG) &=&\setn{\diceD}                      & \ocsVaro(\ocsG)          &=&\frac{101}{300} &\approx& 0.33\\
    \ocsceng(\ocsG)                &=&\setn{\diceB}                      & \ocsVaro(\ocsG;\ocsceng) &=&\frac{127}{150} &\approx& 0.847 \\ 
    \ocscenh(\ocsG)                &=&\setn{\diceE}                      & \ocsVaro(\ocsG;\ocscenh) &=&\frac{145}{150} &\approx& 0.967 \\
    \ocscenm(\ocsG)                &=&\setn{\diceA,\diceC,\diceD,\diceF} & \ocsVaro(\ocsG;\ocscenm) &=&\frac{ 11}{ 50} &=&       0.22 \\
    \ocscenM(\ocsG)                &=&\setn{\diceB,\diceE}               & \ocsVaro(\ocsG;\ocscenM) &=&\frac{ 11}{ 50} &\approx& 0.767
  \end{array}$.
%\end{minipage}\hfill%
%\begin{minipage}{30mm}
%  \gsize%
%  %\psset{unit=8mm}%
%  \centering%
%  %{%============================================================================
% Daniel J. Greenhoe
% LaTeX file
% discrete metric real dice mapping to linearly ordered O6c
%============================================================================
\begin{pspicture}(-1.5,-1.5)(1.5,1.5)%
  %---------------------------------
  % settings
  %---------------------------------
  \psset{%
    radius=1.25ex,
    labelsep=2.5mm,
    %linecolor=blue,%
    }%
  %---------------------------------
  % dice graph nodes
  %---------------------------------
  \rput(0,0){%
    \uput{1}[210](0,0){\Cnode[fillstyle=solid,fillcolor=snode](0,0){D4}}%
    \uput{1}[150](0,0){\Cnode(0,0){D5}}%
    \uput{1}[ 90](0,0){\Cnode(0,0){D6}}%
    \uput{1}[ 30](0,0){\Cnode(0,0){D3}}%
    \uput{1}[-30](0,0){\Cnode(0,0){D2}}%
    \uput{1}[-90](0,0){\Cnode(0,0){D1}}%
    }%
  %-------------------------------------
  % graph node labels
  %-------------------------------------
  \rput(D6){$\diceF$}%
  \rput(D5){$\diceE$}%
  \rput(D4){$\diceD$}%
  \rput(D3){$\diceC$}%
  \rput(D2){$\diceB$}%
  \rput(D1){$\diceA$}%
  %-------------------------------------
  % graph edges
  %-------------------------------------
  \ncline{D5}{D6}%
  \ncline{D4}{D5}\ncline{D4}{D6}%
  \ncline{D3}{D5}\ncline{D3}{D6}%
  \ncline{D2}{D3}\ncline{D2}{D4}\ncline{D2}{D6}%
  \ncline{D1}{D2}\ncline{D1}{D3}\ncline{D1}{D4}\ncline{D1}{D5}%
  %\ncline{D3}{D4}%
  %\ncline{D2}{D5}%
  %\ncline{D1}{D6}%
  %-------------------------------------
  % labels
  %-------------------------------------
  \uput[ 158](D6){$\frac{1}{30}$}
  \uput[ 150](D5){$\frac{1}{50}$}
  \uput[ 210](D4){$\frac{3}{5}$}
  \uput[  22](D3){$\frac{1}{30}$}
  \uput[ -45](D2){$\frac{1}{20}$}
  \uput[-158](D1){$\frac{1}{10}$}
\end{pspicture}%}%
%  {\includegraphics{sto/graphics/ocs_wdie.pdf}}%
%\end{minipage}
\\
Note that the \fncte{outcome center} $\ocscen(\ocsG)$ and \fncte{arithmetic center} $\ocscena(\ocsG)$ 
again yield identical results.
Also note that of the four center measures of cardinality $1$
($\seto{\ocscen(\ocsG)}=\seto{\ocscena(\ocsG)}=\seto{\ocsceng(\ocsG)}=\seto{\ocscenh(\ocsG)}=1$ \xrefnp{def:seto}),
$\ocscen$ and $\ocscena$ yield by far the lowest variance measures.
\end{example}
\begin{proof}
\begin{align*}
  \ocscen(\ocsG)
    &\eqd \argmin_{x\in\ocsG}\max_{y\in\ocsG}\metric{x}{y}\psp(y)
    &&\text{by definition of $\ocscen$ \xref{def:ocscen}}
  \\&\eqd \argmin_{x\in\ocsG}\max_{y\in\ocsG}\frac{1}{300}\metric{x}{y}\psp(y)300
  \\&\eqd \argmin_{x\in\ocsG}\max_{y\in\ocsG}\metric{x}{y}300\psp(y)
   %&&\text{because $\ff(x)=\frac{1}{300}x$ is \prope{strictly isotone}  and by \prefpp{lem:argminmaxphi}}
   %&&\text{because $\ff(x)=\frac{1}{300}x$ is \prope{strictly isotone}}
    &&\text{by \prefpp{lem:argminmaxphi}}
  \\&=\mathrlap{\argmin_{x\in\ocsG}\max_{y\in\ocsG}300
    \setn{\begin{array}{ccccc}
      \metricn(\diceA,\diceA)\psp(\diceA) &\metricn(\diceA,\diceB)\psp(\diceB) &\metricn(\diceA,\diceC)\psp(\diceC) & \cdots & \metricn(\diceA,\diceF)\psp(\diceF) \\
      \metricn(\diceB,\diceA)\psp(\diceA) &\metricn(\diceB,\diceB)\psp(\diceB) &\metricn(\diceB,\diceC)\psp(\diceC) & \cdots & \metricn(\diceB,\diceF)\psp(\diceF) \\
      \vdots                              &\vdots                              &\vdots                              & \ddots & \vdots                              \\
      \metricn(\diceF,\diceA)\psp(\diceA) &\metricn(\diceF,\diceB)\psp(\diceB) &\metricn(\diceF,\diceC)\psp(\diceC) & \cdots & \metricn(\diceF,\diceF)\psp(\diceF) 
    \end{array}}}
   \\&=\mathrlap{\argmin_{x\in\ocsG}\max_{y\in\ocsG}
         \setn{\begin{array}{cccccc}
           0\times30 & 1\times15 & 1\times10 & 1\times180 & 1\times6 & 2\times10 \\
           1\times30 & 0\times15 & 1\times10 & 1\times180 & 2\times6 & 1\times10 \\
           1\times30 & 1\times15 & 0\times10 & 2\times180 & 1\times6 & 1\times10 \\
           1\times30 & 1\times15 & 2\times10 & 0\times180 & 1\times6 & 1\times10 \\
           1\times30 & 2\times15 & 1\times10 & 1\times180 & 0\times6 & 1\times10 \\
           2\times30 & 1\times15 & 1\times10 & 1\times180 & 1\times6 & 0\times10 \\
         \end{array}}
     %\\&=\argmin_{x\in\ocsG}\max_{y\in\ocsG}
     %    \setn{\begin{array}{cccccc}
     %       0 & 15 & 10 & 180 &  6 & 20\\
     %      30 &  0 & 10 & 180 & 12 & 10\\
     %      30 & 15 &  0 & 360 &  6 & 10\\
     %      30 & 15 & 20 &   0 &  6 & 10\\
     %      30 & 30 & 10 & 180 &  0 & 10\\
     %      60 & 15 & 10 & 180 &  6 & 0
     %    \end{array}}
       = \argmin_{x\in\ocsG}
         \setn{\begin{array}{c}
           180 \\
           180 \\
           360 \\
            30 \\
           180 \\
           180 \\
         \end{array}}
       = \setn{\begin{array}{c}
           \mbox{ } \\
           \mbox{ } \\
           \mbox{ } \\
           \diceD \\
           \mbox{ } \\
           \mbox{ } \\
         \end{array}}}
  \\
  \ocscena(\ocsG)
    &\eqd \argmin_{x\in\ocsG}\sum_{y\in\ocsG}\metric{x}{y}\psp(y)
    &&\text{by definition of $\ocscena$ \xref{def:ocscena}}
  \\&= \argmin_{x\in\ocsG}\frac{1}{300}\sum_{y\in\ocsG}\metric{x}{y}\psp(y)300
    &&\text{by definition of $\ocsG$}
  \\&= \argmin_{x\in\ocsG}\sum_{y\in\ocsG}\metric{x}{y}\psp(y)300
    %&&\text{because $\ff(x)=\frac{1}{300}x$ is \prope{strictly isotone} and by \prefpp{lem:argminmaxphi}}
    %&&\text{because $\ff(x)=\frac{1}{300}x$ is \prope{strictly isotone}}
    &&\text{by \prefpp{lem:argminphi}}
  \\&=\mathrlap{\argmin_{x\in\ocsG}
         \setn{\begin{array}{*{11}{r}}
            0 &+& 15 &+& 10 &+& 180 &+&  6 &+& 20\\
           30 &+&  0 &+& 10 &+& 180 &+& 12 &+& 10\\
           30 &+& 15 &+&  0 &+& 360 &+&  6 &+& 10\\
           30 &+& 15 &+& 20 &+&   0 &+&  6 &+& 10\\
           30 &+& 30 &+& 10 &+& 180 &+&  0 &+& 10\\
           60 &+& 15 &+& 10 &+& 180 &+&  6 &+& 0     
         \end{array}}
       = \argmin_{x\in\ocsG}
         \setn{\begin{array}{r}
           231\\
           242\\
           421\\
            81\\
           260\\
           271
         \end{array}}
       = \setn{\begin{array}{c}
           \mbox{ } \\
           \mbox{ } \\
           \mbox{ } \\
           \diceD \\
           \mbox{ } \\
           \mbox{ } \\
         \end{array}}}
  \\
  \ocsceng(\ocsG)
    &\eqd \argmin_{x\in\ocsG}\prod_{y\in\ocsG\setd\setn{x}}\brs{\metric{x}{y}^{\psp(y)}}
    &&\text{by definition of $\ocsceng$ \xref{def:ocsceng}}
  \\&= \argmin_{x\in\ocsG}\prod_{y\in\ocsG\setd\setn{x}}\brs{\metric{x}{y}^{300\psp(y)\frac{1}{300}}}
  \\&= \argmin_{x\in\ocsG}\brp{\prod_{y\in\ocsG\setd\setn{x}}\brs{\metric{x}{y}^{300\psp(y)}}}^\frac{1}{300}
  \\&= \argmin_{x\in\ocsG}\prod_{y\in\ocsG\setd\setn{x}}\brs{\metric{x}{y}^{300\psp(y)}}
    %&&\text{because $\ff(x)=\frac{1}{300}x$ is \prope{strictly isotone} and by \prefpp{lem:argminmaxphi}}
    %&&\text{because $\ff(x)=x^\frac{1}{300}$ is \prope{strictly isotone}}
    &&\text{by \prefpp{lem:argminphi}}
  \\&=\mathrlap{\argmin_{x\in\ocsG}
         \setn{\begin{array}{*{11}{c}}
                  &\times& 1^{15} &\times& 1^{10} &\times& 1^{180} &\times& 1^{6} &\times& 2^{10} \\
           1^{30} &\times&        &\times& 1^{10} &\times& 1^{180} &\times& 2^{6} &\times& 1^{10} \\
           1^{30} &\times& 1^{15} &\times&        &\times& 2^{180} &\times& 1^{6} &\times& 1^{10} \\
           1^{30} &\times& 1^{15} &\times& 2^{10} &\times&         &\times& 1^{6} &\times& 1^{10} \\
           1^{30} &\times& 2^{15} &\times& 1^{10} &\times& 1^{180} &\times&       &\times& 1^{10} \\
           2^{30} &\times& 1^{15} &\times& 1^{10} &\times& 1^{180} &\times& 1^{6} &      &           
         \end{array}}
       = \argmin_{x\in\ocsG}
         \setn{\begin{array}{l}
           2^{10}\\
           2^{6 }\\
           2^{180}\\
           2^{10}\\
           2^{15}\\
           2^{30}
         \end{array}}
       = \setn{\begin{array}{c}
           \mbox{ } \\
           \diceB{ } \\
           \mbox{ } \\
           \mbox{ } \\
           \mbox{ } \\
           \mbox{ } \\
         \end{array}}}
  \\
  \ocscenh(\ocsG)
    &\eqd \argmin_{x\in\ocsG}\brp{\sum_{y\in\ocsG\setd\setn{x}}\frac{1}{\metric{x}{y}}\psp(y)}^{-1}
    &&\text{by definition of $\ocscenh$ \xref{def:ocscenh}}
  \\&= \argmax_{x\in\ocsG}\sum_{y\in\ocsG\setd\setn{x}}\frac{1}{\metric{x}{y}}\psp(y)
    && \text{by \prefp{lem:minphi}}
   %&& \text{because $\fphi(x)\eqd x^{-1}$ is \prope{strictly antitone} and by \prefp{lem:minphia}}
  \\&= \argmax_{x\in\ocsG}\frac{1}{300}\sum_{y\in\ocsG\setd\setn{x}}\frac{1}{\metric{x}{y}}300\psp(y)
  \\&= \argmax_{x\in\ocsG}\sum_{y\in\ocsG\setd\setn{x}}\frac{300\psp(y)}{\metric{x}{y}}
    %&&\text{because $\ff(x)=\frac{1}{300}x$ is \prope{strictly isotone} and by \prefpp{lem:argminmaxphi}}
    %&&\text{because $\ff(x)=\frac{1}{300}x$ is \prope{strictly isotone}}
    &&\text{by \prefpp{lem:argminphi}}
  \\&=\mathrlap{\argmax_{x\in\ocsG}
         %\setn\begin{array}{*{11}{@{\,}c}}
         \setn{\begin{array}{*{11}{c}}
                        &+& \frac{15}{1} &+& \frac{10}{1} &+& \frac{180}{1} &+& \frac{6}{1} &+& \frac{10}{2} \\
           \frac{30}{1} &+&              &+& \frac{10}{1} &+& \frac{180}{1} &+& \frac{6}{2} &+& \frac{10}{1} \\
           \frac{30}{1} &+& \frac{15}{1} &+&              &+& \frac{180}{2} &+& \frac{6}{1} &+& \frac{10}{1} \\
           \frac{30}{1} &+& \frac{15}{1} &+& \frac{10}{2} &+&               &+& \frac{6}{1} &+& \frac{10}{1} \\
           \frac{30}{1} &+& \frac{15}{2} &+& \frac{10}{1} &+& \frac{180}{1} &+&             &+& \frac{10}{1} \\
           \frac{30}{2} &+& \frac{15}{1} &+& \frac{10}{1} &+& \frac{180}{1} &+& \frac{6}{1} &+&                     
         \end{array}}
       = \argmax_{x\in\ocsG}
         \setn{\begin{array}{r}
           216.0\\
           233.0\\
           151.0\\
            66.0\\
           237.5\\
           226.0
         \end{array}}
       = \setn{\begin{array}{c}
           \mbox{ } \\
           \mbox{ } \\
           \mbox{ } \\
           \mbox{ } \\
           \diceE \\
           \mbox{ }
         \end{array}}}
  \\
  \ocscenm(\ocsG)
    &\eqd \argmin_{x\in\ocsG}\min_{y\in\ocsG}\metric{x}{y}\psp(y)
    &&\text{by definition of $\ocscenm$ \xref{def:ocscenm}}
     \\&=\mathrlap{\argmin_{x\in\ocsG}\min_{y\in\ocsG\setd\setn{x}}
         \setn{\begin{array}{rrrrrr}
            0 & 15 & 10 & 180 &  6 & 20\\
           30 &  0 & 10 & 180 & 12 & 10\\
           30 & 15 &  0 & 360 &  6 & 10\\
           30 & 15 & 20 &   0 &  6 & 10\\
           30 & 30 & 10 & 180 &  0 & 10\\
           60 & 15 & 10 & 180 &  6 & 0
         \end{array}}
       = \argmin_{x\in\ocsG}
         \setn{\begin{array}{r}
            6\\
           10\\
            6\\
            6\\
           10\\
            6\\
         \end{array}}
       = \setn{\begin{array}{r}
           \diceA\\
                 \\
           \diceC\\
           \diceD\\
                 \\
           \diceF  
         \end{array}}}
  \\
  \ocscenM(\ocsG)
    &\eqd \argmax_{x\in\ocsG}\min_{y\in\ocsG}\metric{x}{y}\psp(y)
    &&\text{by definition of $\ocscenM$ \xref{def:ocscenM}}
  \\&= \argmax_{x\in\ocsG}\setn{6,10,6,6,10,6}
    && \text{by $\ocscenm(\ocsG)$ result}
  \\&= \setn{\diceB,\diceE}
    && \text{by definition of $\ocsG$}
  \\
  \ocsVaro(\ocsG)
    &\eqd \sum_{x\in\ocsG}\metricsq{\ocscen(\ocsG)}{x} \psp(x)
    &&\text{by definition of $\ocsVaro$ \xref{def:ocsVarG}}
  \\&= \sum_{x\in\ocsG}\metricsq{\diceD}{x} \psp(x)
    &&\text{by $\ocscen(\ocsG)$ result}
  \\&= 1^2\times\frac{1}{10} + 1^2\times\frac{1}{20} + 2^2\times\frac{1}{30} + 
       0^2\times\frac{3}{5}  + 1^2\times\frac{1}{50} + 1^2\times\frac{1}{30}  
  \\&= \frac{1}{300}(30+15+40+0+6+10)
     = \frac{101}{300}
     \approx 0.337
  \\
  \ocsVaro(\ocsG;\ocsceng)
    &\eqd \sum_{x\in\ocsG}\metricsq{\ocsceng(\ocsG)}{x} \psp(x)
    &&\text{by definition of $\ocsVaro$ \xref{def:ocsVarG}}
  \\&= \sum_{x\in\ocsG}\metricsq{\diceB}{x} \psp(x)
    &&\text{by $\ocsceng(\ocsG)$ result}
  \\&= 1^2\times\frac{1}{10} + 0^2\times\frac{1}{20} + 1^2\times\frac{1}{30} + 
       1^2\times\frac{3}{5}  + 2^2\times\frac{1}{50} + 1^2\times\frac{1}{30}  
  \\&= \frac{1}{300}(30+0+10 + 180+24+10)
     = \frac{254}{300}
     = \frac{127}{150}
     \approx 0.847
  \\
  \ocsVaro(\ocsG;\ocscenh)
    &\eqd \sum_{x\in\ocsG}\metricsq{\ocscenh(\ocsG)}{x} \psp(x)
    &&\text{by definition of $\ocsVaro$ \xref{def:ocsVarG}}
  \\&= \sum_{x\in\ocsG}\metricsq{\diceE}{x} \psp(x)
    &&\text{by $\ocscenh(\ocsG)$ result}
  \\&= 1^2\times\frac{1}{10} + 2^2\times\frac{1}{20} + 1^2\times\frac{1}{30} + 
       1^2\times\frac{3}{5}  + 0^2\times\frac{1}{50} + 1^2\times\frac{1}{30}  
  \\&= \frac{1}{300}(30+60+10 + 180+0+10)
     = \frac{290}{300}
     = \frac{145}{150}
     \approx 0.967
  \\
  \ocsVaro(\ocsG;\ocscenm)
    &\eqd \sum_{x\in\ocsG}\metricsq{\ocscenm(\ocsG)}{x} \psp(x)
    &&\text{by definition of $\ocsVaro$ \xref{def:ocsVarG}}
  \\&= \sum_{x\in\ocsG}\metricsq{\diceA,\diceC,\diceD,\diceF}{x} \psp(x)
    &&\text{by $\ocscenm(\ocsG)$ result}
  \\&= 0^2\times\frac{1}{10} + 1^2\times\frac{1}{20} + 0^2\times\frac{1}{30} + 
       0^2\times\frac{3}{5}  + 1^2\times\frac{1}{50} + 0^2\times\frac{1}{30}  
  \\&= \frac{1}{300}( 0+60+ 0 +   0+6+ 0)
     = \frac{ 66}{300}
     = \frac{ 11}{ 50}
     =       0.22 
  \\
  \ocsVaro(\ocsG;\ocscenM)
    &\eqd \sum_{x\in\ocsG}\metricsq{\ocscenM(\ocsG)}{x} \psp(x)
    &&\text{by definition of $\ocsVaro$ \xref{def:ocsVarG}}
  \\&= \sum_{x\in\ocsG}\metricsq{\diceB,\diceE}{x} \psp(x)
    &&\text{by $\ocscenM(\ocsG)$ result}
  \\&= 1^2\times\frac{1}{10} + 0^2\times\frac{1}{20} + 1^2\times\frac{1}{30} + 
       1^2\times\frac{3}{5}  + 0^2\times\frac{1}{50} + 1^2\times\frac{1}{30}  
  \\&= \frac{1}{300}(30+ 0+10 + 180+0+10)
     = \frac{230}{300}
     = \frac{ 11}{ 50}
     \approx 0.767
\end{align*}
\end{proof}


%\if 0
%---------------------------------------
%\begin{minipage}{\tw-35mm}%
\begin{example}[\exmd{board game spinner outcome subspace}]
\label{ex:spinner} %\mbox{}\\
%---------------------------------------
The six value \structe{spinner outcome subspace} \xref{def:spinner} %illustrated to the right 
has the following geometric values:
\\$\begin{array}{rcl}
  \ocscen(\ocsG)=\ocscena(\ocsG)=\ocsceng(\ocsG)=\ocscenh(\ocsG)=\ocscenm(\ocsG)=\ocscenM(\ocsG) &=& \setn{1,2,3,4,5,6} \\
  \ocsVaro(\ocsG) = \ocsVaro(\ocsG;\ocscena) = \ocsVaro(\ocsG;\ocsceng) = \ocsVaro(\ocsG;\ocscenh) = \ocsVaro(\ocsG;\ocscenm) = \ocsVaro(\ocsG;\ocscenM)  &=&  0     
\end{array}$
\end{example}
%\end{minipage}\hfill%
%\begin{tabular}{c}
%  \gsize%
%  \psset{unit=8mm}%
%  \centering%
%  %{%============================================================================
% Daniel J. Greenhoe
% LaTeX file
% spinner 6 mapping to linearly ordered L6
%============================================================================
{%\psset{unit=0.5\psunit}%
\begin{pspicture}(-1.5,-1.5)(1.5,1.5)%
  %---------------------------------
  % options
  %---------------------------------
  \psset{%
    linecolor=blue,%
    radius=1.25ex,
    labelsep=2.5mm,
    }%
  %---------------------------------
  % spinner graph
  %---------------------------------
  \rput(0,0){%\psset{unit=2\psunit}%
    \uput{1}[210](0,0){\Cnode[fillstyle=solid,fillcolor=snode](0,0){D6}}%
    \uput{1}[150](0,0){\Cnode[fillstyle=solid,fillcolor=snode](0,0){D5}}%
    \uput{1}[ 90](0,0){\Cnode[fillstyle=solid,fillcolor=snode](0,0){D4}}%
    \uput{1}[ 30](0,0){\Cnode[fillstyle=solid,fillcolor=snode](0,0){D3}}%
    \uput{1}[-30](0,0){\Cnode[fillstyle=solid,fillcolor=snode](0,0){D2}}%
    \uput{1}[-90](0,0){\Cnode[fillstyle=solid,fillcolor=snode](0,0){D1}}%
    }
  \rput[-150](D6){$\circSix$}%
  \rput[ 150](D5){$\circFive$}%
  \rput[  90](D4){$\circFour$}%
  \rput[  30](D3){$\circThree$}%
  \rput[   0](D2){$\circTwo$}%
  \rput[ -90](D1){$\circOne$}%
  %
  \ncline{D6}{D1}%
  \ncline{D5}{D6}%
  \ncline{D4}{D5}%
  \ncline{D3}{D4}%
  \ncline{D2}{D3}%
  \ncline{D1}{D2}%
  %
  \uput[ 210](D6){$\frac{1}{6}$}
  \uput[ 150](D5){$\frac{1}{6}$}
  \uput[  22](D4){$\frac{1}{6}$}
  \uput[  30](D3){$\frac{1}{6}$}
  \uput[ -30](D2){$\frac{1}{6}$}
  \uput[ -22](D1){$\frac{1}{6}$}
  %
  %\uput[ 210](D6){${\scy\psp(\circSix)=}\frac{1}{6}$}
  %\uput[ 150](D5){${\scy\psp(\circFive)=}\frac{1}{6}$}
  %\uput[  22](D4){${\scy\psp(\circFour)=}\frac{1}{6}$}
  %\uput[  30](D3){${\scy\psp(\circThree)=}\frac{1}{6}$}
  %\uput[ -30](D2){${\scy\psp(\circTwo)=}\frac{1}{6}$}
  %\uput[-22](D1){${\scy\psp(\circOne)=}\frac{1}{6}$}
\end{pspicture}
}%}%
%  {\includegraphics{sto/graphics/ocs_spinner.pdf}}%
%\end{tabular}
\begin{proof}
\begin{align*}
  \ocscen(\ocsG)
    &\eqd \argmin_{x\in\ocsG}\max_{y\in\ocsG}\metric{x}{y}\psp(y)
    && \text{by definition of $\ocscen$ \xref{def:ocscen}}
  \\&= \argmin_{x\in\ocsG}\max_{y\in\ocsG}\metric{x}{y}\frac{1}{6}
    && \text{by definition of $\ocsG$}
  \\&= \argmin_{x\in\ocsG}\max_{y\in\ocsG}\metric{x}{y}
    && \text{because $\ff(x)=\frac{1}{6}x$ is \prope{strictly isotone} and by \prefpp{lem:argminmaxphi}}
   \\&=\mathrlap{\argmin_{x\in\ocsG}\max_{y\in\ocsG}
        %\setn{\begin{array}{cccc}
        %  \metricn(1,1)&\metricn(1,2)&\cdots & \metricn(1,6) \\
        %  \metricn(2,1)&\metricn(2,2)&\ddots & \metricn(2,6) \\
        %  \vdots       &\ddots       &\ddots & \vdots        \\
        %  \metricn(6,1)&\metricn(6,2)&\cdots & \metricn(6,6)
        %\end{array}}
        \setn{\begin{array}{cccc}
          \metricn(1,1)&\cdots & \metricn(1,6) \\
          \metricn(2,1)&\cdots & \metricn(2,6) \\
          \metricn(3,1)&\cdots & \metricn(3,6) \\
          \metricn(4,1)&\cdots & \metricn(4,6) \\
          \metricn(5,1)&\cdots & \metricn(5,6) \\
          \metricn(6,1)&\cdots & \metricn(6,6)
        \end{array}}
   \quad=\argmin_{x\in\ocsG}\max_{y\in\ocsG}
        \setn{\begin{array}{cccccc}
           0 & 1 & 2 & 3 & 2 & 1 \\
           1 & 0 & 1 & 2 & 3 & 2 \\
           2 & 1 & 0 & 1 & 2 & 3 \\
           3 & 2 & 1 & 0 & 1 & 2 \\
           2 & 3 & 2 & 1 & 0 & 1 \\
           1 & 2 & 3 & 2 & 1 & 0 
        \end{array}}
     \quad= \argmin_{x\in\ocsG}
        \setn{\begin{array}{c}
          {3}\\
          {3}\\
          {3}\\
          {3}\\
          {3}\\
          {3}
        \end{array}}}
  %\\&= \argmin_{x\in\ocsG}\setn{3,\,3,\,3,\,3,\,3,\,3}
  %  && \text{by definition of $\metricn$ \xref{def:dmetric}}
  \\&= \setn{\diceA,\diceB,\diceC,\diceD,\diceE,\diceF}
    && \text{by definition of $\ocsG$}
  \\
  \ocscena(\ocsG)
    &\eqd \argmin_{x\in\ocsG}\sum_{y\in\ocsG}\metric{x}{y}\psp(y)
    && \text{by definition of $\ocscena$ \xref{def:ocscenx}}
  \\&= \argmin_{x\in\ocsG}\sum_{y\in\ocsG}\metric{x}{y}\frac{1}{6}
    && \text{by definition of $\ocsG$}
  \\&= \argmin_{x\in\ocsG}\sum_{y\in\ocsG}\metric{x}{y}
    && \text{because $\ff(x)=\frac{1}{6}x$ is \prope{strictly isotone} and by \prefpp{lem:argminphi}}
  \\&=\mathrlap{\argmin_{x\in\ocsG}
         \setn{\begin{array}{*{11}{c}}
           0 &+& 1 &+& 2 &+& 3 &+& 2 &+& 1 \\
           1 &+& 0 &+& 1 &+& 2 &+& 3 &+& 2 \\
           2 &+& 1 &+& 0 &+& 1 &+& 2 &+& 3 \\
           3 &+& 2 &+& 1 &+& 0 &+& 1 &+& 2 \\
           2 &+& 3 &+& 2 &+& 1 &+& 0 &+& 1 \\
           1 &+& 2 &+& 3 &+& 2 &+& 1 &+& 0 
         \end{array}}
       = \argmin_{x\in\ocsG}
         \setn{\begin{array}{c}
           9\\
           9\\
           9\\
           9\\
           9\\
           9
         \end{array}}
       = \setn{\begin{array}{c}
           \diceA\\
           \diceB\\
           \diceC\\
           \diceD\\
           \diceE\\
           \diceF
         \end{array}}}
  \\
  \ocsceng(\ocsG)
    &\eqd \argmin_{x\in\ocsG}\prod_{y\in\ocsG\setd\setn{x}}\brs{\metric{x}{y}^{\psp(y)}}
    &&\text{by definition of $\ocsceng$ \xref{def:ocsceng}}
  \\&= \argmin_{x\in\ocsG}\prod_{y\in\ocsG\setd\setn{x}}\brs{\metric{x}{y}^\frac{1}{6}}
    &&\text{by definition of $\ocsG$}
  \\&= \argmin_{x\in\ocsG}\brp{\prod_{y\in\ocsG\setd\setn{x}}\metric{x}{y}}^\frac{1}{6}
  \\&= \argmin_{x\in\ocsG}\prod_{y\in\ocsG\setd\setn{x}}\metric{x}{y}
    %&&\text{because $\ff(x)=\frac{1}{300}x$ is \prope{strictly isotone} and by \prefpp{lem:argminmaxphi}}
    %&&\text{because $\ff(x)=x^\frac{1}{300}$ is \prope{strictly isotone}}
    &&\text{by \prefpp{lem:argminphi}}
  \\&=\mathrlap{\argmin_{x\in\ocsG}
         \setn{\begin{array}{*{11}{c}}
             &\times& 1 &\times& 2 &\times& 3 &\times& 2 &\times& 1 \\
           1 &\times&   &\times& 1 &\times& 2 &\times& 3 &\times& 2 \\
           2 &\times& 1 &\times&   &\times& 1 &\times& 2 &\times& 3 \\
           3 &\times& 2 &\times& 1 &\times&   &\times& 1 &\times& 2 \\
           2 &\times& 3 &\times& 2 &\times& 1 &\times&   &\times& 1 \\
           1 &\times& 2 &\times& 3 &\times& 2 &\times& 1 &\times&   
         \end{array}}
       = \argmin_{x\in\ocsG}
         \setn{\begin{array}{c}
           12\\
           12\\
           12\\
           12\\
           12\\
           12
         \end{array}}
       = \setn{\begin{array}{c}
           \diceA\\
           \diceB\\
           \diceC\\
           \diceD\\
           \diceE\\
           \diceF
         \end{array}}}
  \\
  \ocscenh(\ocsG)
    &\eqd \argmin_{x\in\ocsG}\brp{\sum_{y\in\ocsG\setd\setn{x}}\frac{1}{\metric{x}{y}}\psp(y)}^{-1}
    &&\text{by definition of $\ocscenh$ \xref{def:ocscenh}}
  \\&= \argmax_{x\in\ocsG}\sum_{y\in\ocsG\setd\setn{x}}\frac{1}{\metric{x}{y}}\psp(y)
    && \text{because $\fphi(x)\eqd x^{-1}$ is \prope{strictly antitone} and by \prefp{lem:minphia}}
  \\&= \argmax_{x\in\ocsG}\sum_{y\in\ocsG\setd\setn{x}}\frac{1}{\metric{x}{y}}\frac{1}{6}
  \\&= \argmax_{x\in\ocsG}\sum_{y\in\ocsG\setd\setn{x}}\frac{1}{\metric{x}{y}}
    %&&\text{because $\ff(x)=\frac{1}{300}x$ is \prope{strictly isotone} and by \prefpp{lem:argminmaxphi}}
    %&&\text{because $\ff(x)=\frac{1}{6}x$ is \prope{strictly isotone}}
    &&\text{by \prefpp{lem:argminphi}}
  \\&=\mathrlap{\argmax_{x\in\ocsG}
         %\setn\begin{array}{*{11}{@{\,}c}}
         \setn{\begin{array}{*{11}{c}}
                       &+& \frac{1}{1} &+& \frac{1}{2} &+& \frac{1}{3} &+& \frac{1}{2} &+& \frac{1}{1} \\
           \frac{1}{1} &+&             &+& \frac{1}{1} &+& \frac{1}{2} &+& \frac{1}{3} &+& \frac{1}{2} \\
           \frac{1}{2} &+& \frac{1}{1} &+&             &+& \frac{1}{1} &+& \frac{1}{2} &+& \frac{1}{3} \\
           \frac{1}{3} &+& \frac{1}{2} &+& \frac{1}{1} &+&             &+& \frac{1}{1} &+& \frac{1}{2} \\
           \frac{1}{2} &+& \frac{1}{3} &+& \frac{1}{2} &+& \frac{1}{1} &+&             &+& \frac{1}{1} \\
           \frac{1}{1} &+& \frac{1}{2} &+& \frac{1}{3} &+& \frac{1}{2} &+& \frac{1}{1} &+&            
         \end{array}}
       = \argmax_{x\in\ocsG}\frac{1}{6}
         \setn{\begin{array}{r}
           20\\
           20\\
           20\\
           20\\
           20\\
           20
         \end{array}}
       = \setn{\begin{array}{c}
           \diceA\\
           \diceB\\
           \diceC\\
           \diceD\\
           \diceE\\
           \diceF
         \end{array}}}
  \\
  \ocscenm(\ocsG)
    &\eqd \argmin_{x\in\ocsG}\min_{y\in\ocsG\setd\setn{x}}\metric{x}{y}\psp(y)
    &&\text{by definition of $\ocscenm$ \xref{def:ocscenm}}
  \\&= \argmin_{x\in\ocsG}\min_{y\in\ocsG\setd\setn{x}}\metric{x}{y}\frac{1}{6}
  \\&= \argmin_{x\in\ocsG}\min_{y\in\ocsG\setd\setn{x}}\metric{x}{y}
    %&& \text{because $\ff(x)=\frac{1}{6}x$ is \prope{strictly isotone}}
    && \text{by \prefpp{lem:argminmaxphi}}
   %&& \text{because $\ff(x)=\frac{1}{6}x$ is \prope{strictly isotone} and by \prefpp{lem:argminmaxphi}}
  \\&= \argmin_{x\in\ocsG}\setn{1,1,1,1,1,1}
  \\&= \setn{\diceA,\diceB,\diceC,\diceD,\diceE,\diceF}
    && \text{by definition of $\ocsG$}
  \\
  \\
  \ocscenM(\ocsG)
    &\eqd \argmax_{x\in\ocsG}\min_{y\in\ocsG\setd\setn{x}}\metric{x}{y}\psp(y)
    &&\text{by definition of $\ocscenM$ \xref{def:ocscenM}}
  \\&= \argmax_{x\in\ocsG}\setn{1,1,1,1,1,1}
    && \text{by $\ocscenm(\ocsG)$ result}
  \\&= \setn{\diceA,\diceB,\diceC,\diceD,\diceE,\diceF}
    && \text{by definition of $\ocsG$}
  \\
  \\
  \ocsVaro(\ocsG;\ocscena)
    &= \mathrlap{\ocsVaro(\ocsG;\ocsceng)= \ocsVaro(\ocsG;\ocscenh)= \ocsVaro(\ocsG;\ocscenm)= \ocsVaro(\ocsG;\ocscenM) = \ocsVaro(\ocsG)}
  \\&\eqd \sum_{x\in\ocsG}\brs{\metric{\ocscen(\ocsG)}{x}}^2\psp(x)
    && \text{by definition of $\ocsVaro$ \xref{def:ocsVarG}}
  \\&= \sum_{x\in\ocsG}\brp{0^2}\frac{1}{6}
    && \text{because $\ocscen(\ocsG)=\ocsG$}
  \\&= 0
    && \text{by field property of \vale{additive identity element} $0$}
\end{align*}
\end{proof}

%---------------------------------------
\begin{example}[\exmd{weighted spinner outcome subspace}]
\label{ex:wspinner}\mbox{}\\
%---------------------------------------
\begin{minipage}{\tw-35mm}%
The six value \structe{weighted spinner outcome subspace} $\ocsG$ \xref{def:ocs} 
illustrated to the right has the following geometric values:
\\\indentx$\begin{array}{rcl@{\qquad}lclcl}
  \ocscen(\ocsG)=\ocscena(\ocsG)=\ocsceng(\ocsG) &=& \setn{1,6}         &\ocsVaro(\ocsG) &=& \frac{5}{3} &\approx& 1.667\\
  \ocscenh(\ocsG)                                &=& \setn{2,5}         &\ocsVaro(\ocsG;\ocscenh) &=& \frac{4}{3} &\approx& 1.333\\
  \ocscenm(\ocsG)=\ocscenM(\ocsG)                &=& \setn{1,2,3,4,5,6} &\ocsVaro(\ocsG;\ocscenm) &=& 0           &=&       0
\end{array}$
\end{minipage}\hfill%
\begin{tabular}{c}
  \gsize%
  \psset{unit=8mm}%
  \centering%
  %{%============================================================================
% Daniel J. Greenhoe
% LaTeX file
% spinner 6 mapping to linearly ordered L6
%============================================================================
{%\psset{unit=0.5\psunit}%
\begin{pspicture}(-1.5,-1.5)(1.5,1.5)%
  %---------------------------------
  % options
  %---------------------------------
  \psset{%
    linecolor=blue,%
    radius=1.25ex,
    labelsep=2.5mm,
    }%
  %---------------------------------
  % spinner graph
  %---------------------------------
  \rput(0,0){%\psset{unit=2\psunit}%
    \Cnode[fillstyle=solid,fillcolor=snode](-0.8660,-0.5){D6}%
    \Cnode(-0.8660,0.5){D5}%
    \Cnode(0,1){D4}%
    \Cnode(0.8660,0.5){D3}%
    \Cnode(0.8660,-0.5){D2}%
    \Cnode[fillstyle=solid,fillcolor=snode](0,-1){D1}%
    }
  \rput[-150](D6){$\circSix$}%
  \rput[ 150](D5){$\circFive$}%
  \rput[  90](D4){$\circFour$}%
  \rput[  30](D3){$\circThree$}%
  \rput[   0](D2){$\circTwo$}%
  \rput[ -90](D1){$\circOne$}%
  %
  \ncline{D6}{D1}%
  \ncline{D5}{D6}%
  \ncline{D4}{D5}%
  \ncline{D3}{D4}%
  \ncline{D2}{D3}%
  \ncline{D1}{D2}%
  %
  \uput[ 210](D6){$\frac{3}{6}$}
  \uput[ 150](D5){$\frac{1}{6}$}
  \uput[  22](D4){$\frac{1}{6}$}
  \uput[  30](D3){$\frac{1}{6}$}
  \uput[ -30](D2){$\frac{1}{6}$}
  \uput[ -22](D1){$\frac{3}{6}$}
  %
  %\uput[ 210](D6){${\scy\psp(\circSix)=}\frac{1}{6}$}
  %\uput[ 150](D5){${\scy\psp(\circFive)=}\frac{1}{6}$}
  %\uput[  22](D4){${\scy\psp(\circFour)=}\frac{1}{6}$}
  %\uput[  30](D3){${\scy\psp(\circThree)=}\frac{1}{6}$}
  %\uput[ -30](D2){${\scy\psp(\circTwo)=}\frac{1}{6}$}
  %\uput[-22](D1){${\scy\psp(\circOne)=}\frac{1}{6}$}
\end{pspicture}
}%}%
  {\includegraphics{sto/graphics/wspinner.pdf}}%
\end{tabular}\\
The \structe{outcome center} result is used later in \prefpp{ex:wspinner_xyz}.
Note that, unlike the \structe{weighted real die outcome subspace} \xref{ex:wdie},
of the center measures of cardinality 2 or less, the \structe{harmonic center} $\ocscenh(\ocsG)$
yields the lowest \fncte{outcome variance} \xref{def:ocsVarG}.
This is surprising since it suggests that $\ocscenh(\ocsG)$ is superior to 
all the other \fncte{center measures} \xxref{def:ocscen}{def:ocscenx},
but yet unlike the other center measures, it yields center values that are not maximally likely.
\end{example}
\begin{proof}
  \begin{align*}
    \ocscen(\ocsG)
      &\eqd \argmin_{x\in\ocsG}\max_{y\in\ocsG}\metric{x}{y}\psp(y)
      &&\text{by definition of $\ocscen$ \xref{def:ocscen}}
    %\\&=\mathrlap{\argmin_{x\in\ocsG}\max_{y\in\ocsG}
    %       \setn{\begin{array}{cccccc}
    %         \metricn(1,1)\frac{3}{10}&\metricn(1,2)\frac{1}{10}&\metricn(1,3)\frac{1}{10}&\metricn(1,4)\frac{1}{10}&\metricn(1,5)\frac{1}{10}&\metricn(1,6)\frac{3}{10}\\
    %         \metricn(2,1)\frac{3}{10}&\metricn(2,2)\frac{1}{10}&\metricn(2,3)\frac{1}{10}&\metricn(2,4)\frac{1}{10}&\metricn(2,5)\frac{1}{10}&\metricn(2,6)\frac{3}{10}\\
    %         \metricn(3,1)\frac{3}{10}&\metricn(3,2)\frac{1}{10}&\metricn(3,3)\frac{1}{10}&\metricn(3,4)\frac{1}{10}&\metricn(3,5)\frac{1}{10}&\metricn(3,6)\frac{3}{10}\\
    %         \metricn(4,1)\frac{3}{10}&\metricn(4,2)\frac{1}{10}&\metricn(4,3)\frac{1}{10}&\metricn(4,4)\frac{1}{10}&\metricn(4,5)\frac{1}{10}&\metricn(4,6)\frac{3}{10}\\
    %         \metricn(5,1)\frac{3}{10}&\metricn(5,2)\frac{1}{10}&\metricn(5,3)\frac{1}{10}&\metricn(5,4)\frac{1}{10}&\metricn(5,5)\frac{1}{10}&\metricn(5,6)\frac{3}{10}\\
    %         \metricn(6,1)\frac{3}{10}&\metricn(6,2)\frac{1}{10}&\metricn(6,3)\frac{1}{10}&\metricn(6,4)\frac{1}{10}&\metricn(6,5)\frac{1}{10}&\metricn(6,6)\frac{3}{10}\\
    %       \end{array}}}
    \\&=\mathrlap{\argmin_{x\in\ocsG}\max_{y\in\ocsG}\frac{1}{10}
           \setn{\begin{array}{cccccc}
             {0}\times3&{1}\times1 & {2}\times1 &{3}\times1 &{2}\times1 &{1}\times3\\
             {1}\times3&{0}\times1 & {1}\times1 &{2}\times1 &{3}\times1 &{2}\times3\\
             {2}\times3&{1}\times1 & {0}\times1 &{1}\times1 &{2}\times1 &{3}\times3\\
             {3}\times3&{2}\times1 & {1}\times1 &{0}\times1 &{1}\times1 &{2}\times3\\
             {2}\times3&{3}\times1 & {2}\times1 &{1}\times1 &{0}\times1 &{1}\times3\\
             {1}\times3&{2}\times1 & {3}\times1 &{2}\times1 &{1}\times1 &{0}\times3
           \end{array}}
    \quad= \argmin_{x\in\ocsG}\frac{1}{10}
           \setn{\begin{array}{ccccccc}
             {3}\\
             {6}\\
             {9}\\
             {9}\\
             {6}\\
             {3}
           \end{array}}
    \quad= \setn{\begin{array}{c}
             1\\
             \mbox{ }\\
             \mbox{ }\\
             \mbox{ }\\
             \mbox{ }\\
             6
           \end{array}}}
    \\
    \ocscena(\ocsG)
      &\eqd \argmin_{x\in\ocsG}\sum_{y\in\ocsG}\metric{x}{y}\psp(y)
      &&\text{by definition of $\ocscena$ \xref{def:ocscena}}
    \\&=\mathrlap{\argmin_{x\in\ocsG}\frac{1}{10}
           \setn{\begin{array}{*{11}{c@{\,}}}
             {0}\times3 &+& {1}\times1  &+&  {2}\times1  &+& {3}\times1  &+& {2}\times1  &+& {1}\times3\\
             {1}\times3 &+& {0}\times1  &+&  {1}\times1  &+& {2}\times1  &+& {3}\times1  &+& {2}\times3\\
             {2}\times3 &+& {1}\times1  &+&  {0}\times1  &+& {1}\times1  &+& {2}\times1  &+& {3}\times3\\
             {3}\times3 &+& {2}\times1  &+&  {1}\times1  &+& {0}\times1  &+& {1}\times1  &+& {2}\times3\\
             {2}\times3 &+& {3}\times1  &+&  {2}\times1  &+& {1}\times1  &+& {0}\times1  &+& {1}\times3\\
             {1}\times3 &+& {2}\times1  &+&  {3}\times1  &+& {2}\times1  &+& {1}\times1  &+& {0}\times3
           \end{array}}
       = \argmin_{x\in\ocsG}\frac{1}{10}
           \setn{\begin{array}{c}
             11\\
             15\\
             19\\
             19\\
             15\\
             11
           \end{array}}
    = \setn{\begin{array}{c}
             1\\
             \mbox{ }\\
             \mbox{ }\\
             \mbox{ }\\
             \mbox{ }\\
             6
           \end{array}}}
  \\
  \ocsceng(\ocsG)
    &\eqd \argmin_{x\in\ocsG}\prod_{y\in\ocsG\setd\setn{x}}\brs{\metric{x}{y}^{\psp(y)}}
    &&\text{by definition of $\ocsceng$ \xref{def:ocsceng}}
  \\&= \argmin_{x\in\ocsG}\prod_{y\in\ocsG\setd\setn{x}}\brs{\metric{x}{y}^{6\psp(y)\frac{1}{6}}}
  \\&= \argmin_{x\in\ocsG}\brp{\prod_{y\in\ocsG\setd\setn{x}}\brs{\metric{x}{y}^{6\psp(y)}}}^\frac{1}{6}
  \\&= \argmin_{x\in\ocsG}\prod_{y\in\ocsG\setd\setn{x}}\brs{\metric{x}{y}^{6\psp(y)}}
    %&&\text{because $\ff(x)=\frac{1}{300}x$ is \prope{strictly isotone} and by \prefpp{lem:argminmaxphi}}
    %&&\text{because $\ff(x)=x^\frac{1}{6}$ is \prope{strictly isotone}}
    &&\text{by \prefpp{lem:argminphi}}
  \\&=\mathrlap{\argmin_{x\in\ocsG}
         %\setn\begin{array}{*{11}{@{\,}c}}
         \setn{\begin{array}{*{11}{c}}
                 &\times& 1^{1} &\times& 2^{1} &\times& 3^{1} &\times& 2^{1} &\times& 1^{3} \\
           1^{3} &\times&       &\times& 1^{1} &\times& 2^{1} &\times& 3^{1} &\times& 2^{3} \\
           2^{3} &\times& 1^{1} &\times&       &\times& 1^{1} &\times& 2^{1} &\times& 3^{3} \\
           3^{3} &\times& 2^{1} &\times& 1^{1} &\times&       &\times& 1^{1} &\times& 2^{3} \\
           2^{3} &\times& 3^{1} &\times& 2^{1} &\times& 1^{1} &\times&       &\times& 1^{3} \\
           1^{3} &\times& 2^{1} &\times& 3^{1} &\times& 2^{1} &\times& 1^{1} &      &           
         \end{array}}
       = \argmin_{x\in\ocsG}
         \setn{\begin{array}{r}
            2^2 \times 3^1\\
            2^4 \times 3^1\\
            2^4 \times 3^3\\
            2^4 \times 3^3\\
            2^4 \times 3^1\\
            2^2 \times 3^1
         \end{array}}
       %= \argmin_{x\in\ocsG}
       %  \setn{\begin{array}{r}
       %     12\\
       %     64\\
       %    432\\
       %    432\\
       %     48\\
       %     12
       %  \end{array}}
    = \setn{\begin{array}{c}
             1\\
             \mbox{ }\\
             \mbox{ }\\
             \mbox{ }\\
             \mbox{ }\\
             6
           \end{array}}}
  \\
  \ocscenh(\ocsG)
    &\eqd \argmin_{x\in\ocsG}\brp{\sum_{y\in\ocsG\setd\setn{x}}\frac{1}{\metric{x}{y}}\psp(y)}^{-1}
    &&\text{by definition of $\ocscenh$ \xref{def:ocscenh}}
  \\&= \argmax_{x\in\ocsG}\sum_{y\in\ocsG\setd\setn{x}}\frac{1}{\metric{x}{y}}\psp(y)
    && \text{because $\fphi(x)\eqd x^{-1}$ is \prope{strictly antitone} and by \prefp{lem:minphia}}
  \\&= \argmax_{x\in\ocsG}\sum_{y\in\ocsG\setd\setn{x}}\frac{1}{\metric{x}{y}}6\psp(y)\frac{1}{6}
  \\&= \argmax_{x\in\ocsG}\sum_{y\in\ocsG\setd\setn{x}}\frac{6\psp(y)}{\metric{x}{y}}
    %&&\text{because $\ff(x)=\frac{1}{300}x$ is \prope{strictly isotone} and by \prefpp{lem:argminmaxphi}}
    %&&\text{because $\ff(x)=\frac{1}{6}x$ is \prope{strictly isotone}}
    &&\text{by \prefpp{lem:argminphi}}
  \\&=\mathrlap{\argmax_{x\in\ocsG}
         %\setn\begin{array}{*{11}{@{\,}c}}
         \setn{\begin{array}{*{11}{c}}
                       &+& \frac{1}{1} &+& \frac{1}{2} &+& \frac{1}{3} &+& \frac{1}{2} &+& \frac{3}{1} \\
           \frac{3}{1} &+&             &+& \frac{1}{1} &+& \frac{1}{2} &+& \frac{1}{3} &+& \frac{3}{2} \\
           \frac{3}{2} &+& \frac{1}{1} &+&             &+& \frac{1}{1} &+& \frac{1}{2} &+& \frac{3}{3} \\
           \frac{3}{3} &+& \frac{1}{2} &+& \frac{1}{1} &+&             &+& \frac{1}{1} &+& \frac{3}{2} \\
           \frac{3}{2} &+& \frac{1}{3} &+& \frac{1}{2} &+& \frac{1}{1} &+&             &+& \frac{3}{1} \\
           \frac{3}{1} &+& \frac{1}{2} &+& \frac{1}{3} &+& \frac{1}{2} &+& \frac{1}{1} &+&               
         \end{array}}
       = \argmax_{x\in\ocsG}\frac{1}{6}
         \setn{\begin{array}{r}
           32\\
           38\\
           30\\
           30\\
           38\\
           32
         \end{array}}
       = \setn{\begin{array}{r}
             \\
            2\\
             \\
             \\
            5\\
           \mbox{}
         \end{array}}}
  \\
  \ocscenm(\ocsG)
    &\eqd \argmin_{x\in\ocsG}\min_{y\in\ocsG}\metric{x}{y}\psp(y)
    &&\text{by definition of $\ocscenm$ \xref{def:ocscenm}}
     \\&=\mathrlap{\argmin_{x\in\ocsG}\min_{y\in\ocsG\setd\setn{x}}
         \setn{\begin{array}{cccccc}
             & 1 & 2 & 3 & 2 & 1\\
           3 &   & 1 & 2 & 3 & 2\\
           6 & 1 &   & 1 & 2 & 3\\
           9 & 2 & 1 &   & 1 & 2\\
           6 & 3 & 2 & 1 &   & 1\\
           3 & 2 & 3 & 2 & 1 &  
         \end{array}}
       = \argmin_{x\in\ocsG}
         \setn{\begin{array}{c}
           1\\
           1\\
           1\\
           1\\
           1\\
           1\\
         \end{array}}
       = \setn{\begin{array}{c}
           1\\
           2\\
           3\\
           4\\
           5\\
           6\\
         \end{array}}}
  \\
  \ocscenM(\ocsG)
    &\eqd \argmax_{x\in\ocsG}\min_{y\in\ocsG}\metric{x}{y}\psp(y)
    &&\text{by definition of $\ocscenM$ \xref{def:ocscenM}}
  \\&= \argmax_{x\in\ocsG}\setn{1,1,1,1,1,1}
    && \text{by $\ocscenm(\ocsG)$ result}
  \\&= \setn{1,2,3,4,5,6}
  \\
    \ocsVaro(\ocsG;\ocscena) 
      &= \ocsVaro(\ocsG;\ocsceng) = \ocsVaro(\ocsG)
    \\&\eqd \sum_{x\in\ocsG}\metricsq{\ocscen(\ocsG)}{x}\psp(x)
      &&\text{by definition of $\ocsVaro$ \xref{def:ocsVarG}}
    \\&= \sum_{x\in\ocsG}\metricsq{\setn{1,6}}{x}\psp(x)
      &&\text{by $\ocscen(\ocsG)$ result}
    \\&= \mathrlap{(0)^2\frac{3}{6}+(1)^2\frac{1}{6}+(2)^2\frac{1}{6}+(2)^2\frac{1}{6}+(1)^2\frac{1}{6}+(0)^2\frac{3}{6}}
    \\&= \frac{10}{6}= \frac{5}{3}= 1\frac{2}{3}\approx 1.667
  \\
    \ocsVaro(\ocsG;\ocscenh)
      &\eqd \sum_{x\in\ocsG}\metricsq{\ocscenh(\ocsG)}{x}\psp(x)
      &&\text{by definition of $\ocsVaro$ \xref{def:ocsVarG}}
    \\&= \sum_{x\in\ocsG}\metricsq{\setn{2,5}}{x}\psp(x)
      &&\text{by $\ocscenh(\ocsG)$ result}
    \\&= \mathrlap{(1)^2\frac{3}{6}+(0)^2\frac{1}{6}+(1)^2\frac{1}{6}+(1)^2\frac{1}{6}+(0)^2\frac{1}{6}+(1)^2\frac{3}{6}}
    \\&= \frac{8}{6}= \frac{4}{3} \approx 1.333
  \\
    \ocsVaro(\ocsG;\ocscenM)
      &= \ocsVaro(\ocsG;\ocscenm)
    \\&\eqd \sum_{x\in\ocsG}\metricsq{\ocscenm(\ocsG)}{x}\psp(x)
      &&\text{by definition of $\ocsVaro$ \xref{def:ocsVarG}}
    \\&= \sum_{x\in\ocsG}\metricsq{\setn{1,2,3,4,5,6}}{x}\psp(x)
      &&\text{by $\ocscenh(\ocsG)$ result}
    \\&= \sum_{x\in\ocsG}0^2\psp(x)
       = 0
  \end{align*}
\end{proof}



%\begin{figure}
%  \gsize%
%  \centering%
%  %============================================================================
% Daniel J. Greenhoe
% LaTeX file
% linear congruential (LCG) pseudo-random number generator (PRNG) mappings
% x_{n+1} = (7x_n+5)mod 9
% y_{n+1} = (y_n+2)mod 5
%============================================================================
\begin{pspicture}(-1.3,-1.5)(1.5,1.5)%
  %---------------------------------
  % options
  %---------------------------------
  \psset{%
    radius=1.25ex,
    labelsep=2.5mm,
    linecolor=blue,%
    }%
  \rput{288}(0,0){\rput(1,0){\Cnode[fillstyle=solid,fillcolor=snode](0,0){S4}}}%
  \rput{216}(0,0){\rput(1,0){\Cnode(0,0){S3}}}%
  \rput{144}(0,0){\rput(1,0){\Cnode(0,0){S2}}}%
  \rput{ 72}(0,0){\rput(1,0){\Cnode(0,0){S1}}}%
  \rput{  0}(0,0){\rput(1,0){\Cnode(0,0){S0}}}%
  %
  \rput(S4){$4$}%
  \rput(S3){$3$}%
  \rput(S2){$2$}%
  \rput(S1){$1$}%
  \rput(S0){$0$}%
  %
  \ncline{S4}{S0}%
  \ncline{S3}{S4}%
  \ncline{S2}{S3}%
  \ncline{S1}{S2}%
  \ncline{S0}{S1}%
  %
  \uput[-30](S4){$\frac{3}{9}$}
  \uput[210](S3){$\frac{2}{9}$}
  \uput[144](S2){$\frac{1}{9}$}
  \uput[ 30](S1){$\frac{1}{9}$}
  \uput[  0](S0){$\frac{2}{9}$}
  \rput(0,0){$\ocsG$}%
\end{pspicture}%%
%  \caption{weighted ring \xref{ex:wring5}\label{fig:wring5}}
%\end{figure}
%---------------------------------------
\begin{example}[\exmd{weighted ring}]
\label{ex:wring5}
%---------------------------------------
The weighted five element ring illustrated to the right has the geometric values below.
The \structe{outcome center} result is used later in \prefpp{ex:lcg7x1m9_xyz}.
\\\begin{minipage}{\tw-65mm}%
\indentx$\begin{array}{rcl @{\qquad} lcccl}
    \ocscen (\ocsG) &=&\setn{4}      & \ocsVaro(\ocsG)          &=& \frac{11}{9} &\approx& 1.222 \\
    \ocscena(\ocsG) &=&\setn{3,4}    & \ocsVaro(\ocsG;\ocscena) &=& \frac{ 7}{9} &\approx& 0.778\\
    \ocsceng(\ocsG) &=&\setn{3}      & \ocsVaro(\ocsG;\ocsceng) &=& \frac{16}{9} &\approx& 1.778\\
    \ocscenh(\ocsG) &=&\setn{1,2,3}  & \ocsVaro(\ocsG;\ocscenh) &=& \frac{5}{9}  &\approx& 0.556\\
    \ocscenm(\ocsG) &=&\setn{0,3,4}  & \ocsVaro(\ocsG;\ocscenm) &=& \frac{2}{9}  &\approx& 0.222\\
    \ocscenM(\ocsG) &=&\setn{1,2}    & \ocsVaro(\ocsG;\ocscenM) &=& \frac{16}{9} &\approx& 1.778
  \end{array}$.
\end{minipage}\hfill%
\begin{tabular}{c}
  \gsize%
  %\psset{unit=5mm}%
  %{%============================================================================
% Daniel J. Greenhoe
% LaTeX file
% linear congruential (LCG) pseudo-random number generator (PRNG) mappings
% x_{n+1} = (7x_n+5)mod 9
% y_{n+1} = (y_n+2)mod 5
%============================================================================
\begin{pspicture}(-1.3,-1.5)(1.5,1.5)%
  %---------------------------------
  % options
  %---------------------------------
  \psset{%
    radius=1.25ex,
    labelsep=2.5mm,
    linecolor=blue,%
    }%
  \rput{288}(0,0){\rput(1,0){\Cnode[fillstyle=solid,fillcolor=snode](0,0){S4}}}%
  \rput{216}(0,0){\rput(1,0){\Cnode(0,0){S3}}}%
  \rput{144}(0,0){\rput(1,0){\Cnode(0,0){S2}}}%
  \rput{ 72}(0,0){\rput(1,0){\Cnode(0,0){S1}}}%
  \rput{  0}(0,0){\rput(1,0){\Cnode(0,0){S0}}}%
  %
  \rput(S4){$4$}%
  \rput(S3){$3$}%
  \rput(S2){$2$}%
  \rput(S1){$1$}%
  \rput(S0){$0$}%
  %
  \ncline{S4}{S0}%
  \ncline{S3}{S4}%
  \ncline{S2}{S3}%
  \ncline{S1}{S2}%
  \ncline{S0}{S1}%
  %
  \uput[-30](S4){$\frac{3}{9}$}
  \uput[210](S3){$\frac{2}{9}$}
  \uput[144](S2){$\frac{1}{9}$}
  \uput[ 30](S1){$\frac{1}{9}$}
  \uput[  0](S0){$\frac{2}{9}$}
  \rput(0,0){$\ocsG$}%
\end{pspicture}%}%
  {\includegraphics{sto/graphics/wring5.pdf}}%
\end{tabular}
\end{example}
\begin{proof}
    \begin{align*}
      \ocscen(\ocsG)
        &\eqd \argmin_{x\in\ocsG}\max_{y\in\ocsG}\metric{x}{y}\psp(y)
        &&\text{by definition of $\ocscen$ \xref{def:ocscen}}
      \\&= \argmin_{x\in\ocsG}\max_{y\in\ocsG}\frac{1}{9}\metric{x}{y}\psp(y)9
      \\&= \argmin_{x\in\ocsG}\max_{y\in\ocsG}\metric{x}{y}\psp(y)9
        &&\text{because $\ff(x)=\frac{1}{9}x$ is \prope{strictly isotone} and by \prefpp{lem:argminmaxphi}}
      \\&=\mathrlap{\argmin_{x\in\ocsG}\max_{y\in\ocsG}
             %\setn{\begin{array}{*{5}{@{\,\,}c@{\,\,}}}
             \setn{\begin{array}{*{5}{c}}
               \metricn(0,0)\psp(0)9&\metricn(0,1)\psp(1)9&\metricn(0,2)\psp(2)9&\metricn(0,3)\psp(3)9&\metricn(0,4)\psp(4)9\\
               \metricn(1,0)\psp(0)9&\metricn(1,1)\psp(1)9&\metricn(1,2)\psp(2)9&\metricn(1,3)\psp(3)9&\metricn(1,4)\psp(4)9\\
               \vdots              &\vdots              &\vdots              &\vdots              &\vdots              \\
              %\metricn(2,0)\psp(0)9&\metricn(2,1)\psp(1)9&\metricn(2,2)\psp(2)9&\metricn(2,3)\psp(3)9&\metricn(2,4)\psp(4)9\\
              %\metricn(3,0)\psp(0)9&\metricn(3,1)\psp(1)9&\metricn(3,2)\psp(2)9&\metricn(3,3)\psp(3)9&\metricn(3,4)\psp(4)9\\
               \metricn(4,0)\psp(0)9&\metricn(4,1)\psp(1)9&\metricn(4,2)\psp(2)9&\metricn(4,3)\psp(3)9&\metricn(4,4)\psp(4)9\\
             \end{array}}}
      \\&=\mathrlap{\argmin_{x\in\ocsG}\max_{y\in\ocsG}
             \setn{\begin{array}{*{5}{c}}
               {0}\times2 & {1}\times1 & {2}\times1 & {2}\times2 & {1}\times3\\
               {1}\times2 & {0}\times1 & {1}\times1 & {2}\times2 & {2}\times3\\
               {2}\times2 & {1}\times1 & {0}\times1 & {1}\times2 & {2}\times3\\
               {2}\times2 & {2}\times1 & {1}\times1 & {0}\times2 & {1}\times3\\
               {1}\times2 & {2}\times1 & {2}\times1 & {1}\times2 & {0}\times3\\
             \end{array}}
      %\quad= \argmin_{x\in\ocsG}\max_{y\in\ocsG}
      %       \setn{\begin{array}{*{5}{c}}
      %         0 & 2 & 1 & 2 & 6\\
      %         4 & 0 & 2 & 2 & 3\\
      %         2 & 2 & 0 & 4 & 3\\
      %         2 & 1 & 2 & 0 & 3\\
      %         4 & 1 & 1 & 2 & 0\\
      %       \end{array}}
      \quad= \argmin_{x\in\ocsG}
             \setn{\begin{array}{c}
                4\\
                6\\
                6\\
                4\\
                2
             \end{array}}
      \quad= \setn{\begin{array}{c}
                \mbox{ }\\
                \mbox{ }\\
                \mbox{ }\\
                \mbox{ }\\
                4
             \end{array}}}
      \\
      \ocscena(\ocsG)
        &\eqd \argmin_{x\in\ocsG}\sum_{y\in\ocsG}\metric{x}{y}\psp(y)
        &&\text{by definition of $\ocscena$ \xref{def:ocscena}}
      \\&= \argmin_{x\in\ocsG}\sum_{y\in\ocsG}\frac{1}{9}\metric{x}{y}\psp(y)9
      \\&= \argmin_{x\in\ocsG}\sum_{y\in\ocsG}\metric{x}{y}\psp(y)9
        &&\text{because $\ff(x)=\frac{1}{9}x$ is \prope{strictly isotone} and by \prefpp{lem:argminphi}}
      \\&=\mathrlap{\argmin_{x\in\ocsG}%\sum_{y\in\ocsG}%\frac{1}{9}
             \setn{\begin{array}{ccccccccc}
               {0}\times2&+&{2}\times1&+&{1}\times1&+&{1}\times2&+&{2}\times3\\
               {2}\times2&+&{0}\times1&+&{2}\times1&+&{1}\times2&+&{1}\times3\\
               {1}\times2&+&{2}\times1&+&{0}\times1&+&{2}\times2&+&{1}\times3\\
               {1}\times2&+&{1}\times1&+&{2}\times1&+&{0}\times2&+&{1}\times3\\
               {2}\times2&+&{1}\times1&+&{1}\times1&+&{1}\times2&+&{0}\times3\\
             \end{array}}
      \quad= \argmin_{x\in\ocsG}%\frac{1}{9}
             \setn{\begin{array}{ccccc}
                11\\
                11\\
                11\\
                 8\\
                 8
             \end{array}}
      \quad= \argmin_{x\in\ocsG}%\frac{1}{9}
             \setn{\begin{array}{c}
                \mbox{ }\\
                \mbox{ }\\
                \mbox{ }\\
                3\\
                4
             \end{array}}}
      \\
      \ocsceng(\ocsG)
        &\eqd \argmin_{x\in\ocsG}\prod_{y\in\ocsG\setd\setn{x}}{\metric{x}{y}^{\psp(y)}}
        &&\text{by definition of $\ocsceng$ \xref{def:ocsceng}}
      \\&= \argmin_{x\in\ocsG}\prod_{y\in\ocsG\setd\setn{x}}{\metric{x}{y}^{9\psp(y)\frac{1}{9}}}
      \\&= \argmin_{x\in\ocsG}\brs{\prod_{y\in\ocsG\setd\setn{x}}{\metric{x}{y}^{9\psp(y)}}}^\frac{1}{9}
      \\&= \argmin_{x\in\ocsG}\prod_{y\in\ocsG\setd\setn{x}}{\metric{x}{y}^{9\psp(y)}}
        &&\text{because $\ff(x)\eqd x^\frac{1}{9}$ is \prope{strictly isotone} and by \prefpp{lem:argminphi}}
      \\&=\mathrlap{\argmin_{x\in\ocsG}
             \setn{\begin{array}{ccccccccc}
                     &      & {2}^1 &\times& {1}^1 &\times& {1}^2 &\times& {2}^3\\
               {2}^2 &\times&       &      & {2}^1 &\times& {1}^2 &\times& {1}^3\\
               {1}^2 &\times& {2}^1 &\times&       &      & {2}^2 &\times& {1}^3\\
               {1}^2 &\times& {1}^1 &\times& {2}^1 &\times&       &      & {1}^3\\
               {2}^2 &\times& {1}^1 &\times& {1}^1 &\times& {1}^2 &      &
             \end{array}}
      \quad= \argmin_{x\in\ocsG}
             \setn{\begin{array}{c}
                16\\
                 8\\
                 8\\
                 2\\
                 4
             \end{array}}
      \quad= \argmin_{x\in\ocsG}%\frac{1}{9}
             \setn{\begin{array}{c}
                \mbox{ }\\
                \mbox{ }\\
                \mbox{ }\\
                3\\
                \mbox{}
             \end{array}}}
      \\
      \ocscenh(\ocsG)
        &\eqd \argmin_{x\in\ocso}\brp{\sum_{y\in\ocso}\frac{1}{\metric{x}{y}}\psp(y)}^{-1} 
        &&\text{by definition of $\ocscenh$ \xref{def:ocscenh}}
      \\&=\argmax_{x\in\ocso}\brp{\sum_{y\in\ocso}\frac{1}{\metric{x}{y}}\psp(y)}
        && \text{because $\fphi(x)\eqd x^{-1}$ is \prope{strictly antitone} and by \prefp{lem:minphia}}
      \\&= \argmax_{x\in\ocso}\brp{\frac{1}{9}\sum_{y\in\ocso}\frac{1}{\metric{x}{y}}\psp(y)9}
      \\&= \argmax_{x\in\ocso}\sum_{y\in\ocso}\frac{9\psp(y)}{\metric{x}{y}}
        &&\text{because $\ff(x)=\frac{1}{9}x$ is \prope{strictly isotone} and by \prefpp{lem:argminphi}}
      \\&=\mathrlap{\argmax_{x\in\ocsG}
             \setn{\begin{array}{*{4}{cc}c}
               0           &+& \frac{1}{2} &+& \frac{1}{1} &+& \frac{2}{1} &+& \frac{3}{2}\\
               \frac{2}{2} &+& 0           &+& \frac{1}{2} &+& \frac{2}{1} &+& \frac{3}{1}\\
               \frac{2}{1} &+& \frac{1}{2} &+& 0           &+& \frac{2}{2} &+& \frac{3}{1}\\
               \frac{2}{1} &+& \frac{1}{1} &+& \frac{1}{2} &+& 0           &+& \frac{3}{1}\\
               \frac{2}{2} &+& \frac{1}{1} &+& \frac{1}{1} &+& \frac{2}{1} &+& 0          
             \end{array}}
      \quad= \argmax_{x\in\ocsG}\frac{1}{2}
             \setn{\begin{array}{c}
               {10}\\
               {13}\\
               {13}\\
               {13}\\
               {11}
             \end{array}}
      \quad= \argmin_{x\in\ocsG}%\frac{1}{9}
             \setn{\begin{array}{c}
                \mbox{ }\\
                1\\
                2\\
                3\\
                \mbox{ }
             \end{array}}}
      \\
      \ocscenm(\ocsG)
        &\eqd \argmin_{x\in\ocsG}\min_{y\in\ocsG\setd\setn{x}}\metric{x}{y}\psp(y)
        &&\text{by definition of $\ocscenm$ \xref{def:ocscenm}}
      \\&= \argmin_{x\in\ocsG}\min_{y\in\ocsG\setd\setn{x}}\frac{1}{9}\metric{x}{y}\psp(y)9
      \\&= \argmin_{x\in\ocsG}\min_{y\in\ocsG\setd\setn{x}}\metric{x}{y}\psp(y)9
        &&\text{because $\ff(x)=\frac{1}{9}x$ is \prope{strictly isotone} and by \prefpp{lem:argminmaxphi}}
      \\&=\mathrlap{\argmin_{x\in\ocsG}\min_{y\in\ocsG\setd\setn{x}}%\frac{1}{9}
             \setn{\begin{array}{ccccccccc}
                         &{2}\times1&{1}\times1&{1}\times2&{2}\times3\\
               {2}\times2&          &{2}\times1&{1}\times2&{1}\times3\\
               {1}\times2&{2}\times1&          &{2}\times2&{1}\times3\\
               {1}\times2&{1}\times1&{2}\times1&          &{1}\times3\\
               {2}\times2&{1}\times1&{1}\times1&{1}\times2& 
             \end{array}}
      \quad= \argmin_{x\in\ocsG}
             \setn{\begin{array}{c}
                1\\
                2\\
                2\\
                1\\
                1
             \end{array}}
      \quad= \argmin_{x\in\ocsG}%\frac{1}{9}
             \setn{\begin{array}{c}
                0\\
                \mbox{ }\\
                \mbox{ }\\
                3\\
                4
             \end{array}}}
      \\
      \ocscenM(\ocsG)
        &\eqd \argmax_{x\in\ocsG}\min_{y\in\ocsG\setd\setn{x}}\metric{x}{y}\psp(y)
        &&\text{by definition of $\ocscenM$ \xref{def:ocscenM}}
      \\&= \argmax_{x\in\ocsG}\setn{1,2,2,1,1}
        && \text{by $\ocscenm(\ocsG)$ result}
      \\&= \setn{1,2}
      \\
      \ocsVaro(\ocsG)
        &\eqd \sum_{x\in\ocsG}\metricsq{\ocscen(\ocsG)}{x}\psp(x)
        && \text{by definition of $\ocsVaro$ \xref{def:ocsVarG}}
      \\&= \sum_{x\in\ocsG}\metricsq{\setn{4}}{x}\psp(x)
        && \text{by $\ocscen(\ocsG)$ result}
      \\&= \mathrlap{
           (1)^2\frac{2}{9}+  % x = 0
           (2)^2\frac{1}{9}+  % x = 1
           (2)^2\frac{1}{9}+  % x = 2
           (1)^2\frac{2}{9}+  % x = 3
           (0)^2\frac{3}{9}   % x = 4
         = \frac{11}{9}  \approx 1.222}
      \\
      \ocsVaro(\ocsG;\ocscena)
        &\eqd \sum_{x\in\ocsG}\metricsq{\ocscena(\ocsG)}{x}\psp(x)
        && \text{by definition of $\ocsVaro$ \xref{def:ocsVarG}}
      \\&= \sum_{x\in\ocsG}\metricsq{\setn{3,4}}{x}\psp(x)
        && \text{by $\ocscena(\ocsG)$ result}
      \\&= \mathrlap{
           (1)^2\frac{2}{9}+  % x = 0
           (2)^2\frac{1}{9}+  % x = 1
           (1)^2\frac{1}{9}+  % x = 2
           (0)^2\frac{2}{9}+  % x = 3
           (0)^2\frac{3}{9}   % x = 4
         = \frac{7}{9}  \approx 0.778}
      \\
      \ocsVaro(\ocsG;\ocsceng)
        &\eqd \sum_{x\in\ocsG}\metricsq{\ocsceng(\ocsG)}{x}\psp(x)
        && \text{by definition of $\ocsVaro$ \xref{def:ocsVarG}}
      \\&= \sum_{x\in\ocsG}\metricsq{\setn{3}}{x}\psp(x)
        && \text{by $\ocsceng(\ocsG)$ result}
      \\&= \mathrlap{
           (2)^2\frac{2}{9}+  % x = 0
           (2)^2\frac{1}{9}+  % x = 1
           (1)^2\frac{1}{9}+  % x = 2
           (0)^2\frac{2}{9}+  % x = 3
           (1)^2\frac{3}{9}   % x = 4
         = \frac{16}{9}  \approx 1.778}
      \\
      \ocsVaro(\ocsG;\ocscenh)
        &\eqd \sum_{x\in\ocsG}\metricsq{\ocscenh(\ocsG)}{x}\psp(x)
        && \text{by definition of $\ocsVaro$ \xref{def:ocsVarG}}
      \\&= \sum_{x\in\ocsG}\metricsq{\setn{1,2,3}}{x}\psp(x)
        && \text{by $\ocscenh(\ocsG)$ result}
      \\&= \mathrlap{
           (1)^2\frac{2}{9}+  % x = 0
           (0)^2\frac{1}{9}+  % x = 1
           (0)^2\frac{1}{9}+  % x = 2
           (0)^2\frac{2}{9}+  % x = 3
           (1)^2\frac{3}{9}   % x = 4
         = \frac{5}{9} \approx 0.556}
      \\
      \ocsVaro(\ocsG;\ocscenm)
        &\eqd \sum_{x\in\ocsG}\metricsq{\ocscenm(\ocsG)}{x}\psp(x)
        && \text{by definition of $\ocsVaro$ \xref{def:ocsVarG}}
      \\&= \sum_{x\in\ocsG}\metricsq{\setn{0,3,4}}{x}\psp(x)
        && \text{by $\ocscenm(\ocsG)$ result}
      \\&= \mathrlap{
           (0)^2\frac{2}{9}+  % x = 0
           (1)^2\frac{1}{9}+  % x = 1
           (1)^2\frac{1}{9}+  % x = 2
           (0)^2\frac{2}{9}+  % x = 3
           (0)^2\frac{3}{9}   % x = 4
         = \frac{2}{9} \approx 0.222}
      \\
      \ocsVaro(\ocsG;\ocscenM)
        &\eqd \sum_{x\in\ocsG}\metricsq{\ocscenM(\ocsG)}{x}\psp(x)
        && \text{by definition of $\ocsVaro$ \xref{def:ocsVarG}}
      \\&= \sum_{x\in\ocsG}\metricsq{\setn{1,2}}{x}\psp(x)
        && \text{by $\ocscenm(\ocsG)$ result}
      \\&= \mathrlap{
           (1)^2\frac{2}{9}+  % x = 0
           (0)^2\frac{1}{9}+  % x = 1
           (0)^2\frac{1}{9}+  % x = 2
           (1)^2\frac{2}{9}+  % x = 3
           (2)^2\frac{3}{9}   % x = 4
         = \frac{16}{9} \approx 1.778}
    \end{align*}
\end{proof}


%---------------------------------------
\begin{minipage}{\tw-65mm}%
\begin{example}%[\exmd{weigted ring}]
\label{ex:wring5short}
%---------------------------------------
The weighted five element structure illustrated to the right has the following geometric values:
\\\indentx$\begin{array}{rcl@{\qquad} lcccl}
    \ocscen(\ocsG)=\ocsceng(\ocsG) &=&\setn{3}      & \ocsVaro(\ocsG)          &=& \frac{10}{9} &\approx& 1.111 \\
    \ocscena(\ocsG)                &=&\setn{3,4}    & \ocsVaro(\ocsG;\ocscena) &=& \frac{ 4}{9} &\approx& 0.444  \\
    \ocscenh(\ocsG)                &=&\setn{1,2,3}  & \ocsVaro(\ocsG;\ocscenh) &=& \frac{ 5}{9} &\approx& 0.555  \\
    \ocscenm(\ocsG)                &=&\setn{0,3,4}  & \ocsVaro(\ocsG;\ocscenm) &=& \frac{ 2}{9} &\approx& 0.222  \\
    \ocscenM(\ocsG)                &=&\setn{1,2}    & \ocsVaro(\ocsG;\ocscenM) &=& \frac{7}{9}  &\approx& 0.778
  \end{array}$.
\end{example}
\end{minipage}\hfill%
\begin{tabular}{c}
  \gsize%
  %\psset{unit=5mm}%
  %{%============================================================================
% Daniel J. Greenhoe
% LaTeX file
% linear congruential (LCG) pseudo-random number generator (PRNG) mappings
% x_{n+1} = (7x_n+5)mod 9
% y_{n+1} = (y_n+2)mod 5
%============================================================================
\begin{pspicture}(-1.3,-1.5)(1.5,1.5)%
  %---------------------------------
  % options
  %---------------------------------
  \psset{%
    radius=1.25ex,
    labelsep=2.5mm,
    linecolor=blue,%
    }%
  \rput{288}(0,0){\rput(1,0){\Cnode(0,0){S4}}}%
  \rput{216}(0,0){\rput(1,0){\Cnode(0,0){S2}}}%
  \rput{144}(0,0){\rput(1,0){\Cnode(0,0){S0}}}%
  \rput{ 72}(0,0){\rput(1,0){\Cnode[fillstyle=solid,fillcolor=snode](0,0){S3}}}%
  \rput{  0}(0,0){\rput(1,0){\Cnode(0,0){S1}}}%
  %
  \rput(S4){$4$}%
  \rput(S3){$3$}%
  \rput(S2){$2$}%
  \rput(S1){$1$}%
  \rput(S0){$0$}%
  %
  \ncline{S3}{S0}\ncline{S3}{S4}%
  \ncline{S1}{S3}%
  \ncline{S4}{S1}%
  \ncline{S2}{S4}%
  \ncline{S0}{S2}%
  %
  \uput[-30](S4){$\frac{3}{9}$}
  \uput[ 30](S3){$\frac{2}{9}$}
  \uput[216](S2){$\frac{1}{9}$}
  \uput[  0](S1){$\frac{1}{9}$}
  \uput[144](S0){$\frac{2}{9}$}
  \rput(0,0){$\ocsG$}%
\end{pspicture}%}%
  {\includegraphics{sto/graphics/wring5short.pdf}}%
\end{tabular}
\\
The \structe{outcome center} result is used later in \prefpp{ex:lcg7x1m9_seqorder}.
Note that only the operators $\ocscen$ and $\ocsceng$ were able to successfully isolate a single center point
($\seto{\ocscen(\ocsG)}=\seto{\ocsceng(\ocsG)}=\seto{\setn{3}}=1$).
\\
\begin{proof}
  \begin{align*}
      \ocscen(\ocsG)
        &\eqd \argmin_{x\in\ocsG}\max_{y\in\ocsG}\metric{x}{y}\psp(y)
        &&\text{by definition of $\ocscen$ \xref{def:ocscen}}
      \\&= \argmin_{x\in\ocsG}\max_{y\in\ocsG}\frac{1}{9}\metric{x}{y}\psp(y)9
      \\&= \argmin_{x\in\ocsG}\max_{y\in\ocsG}\metric{x}{y}\psp(y)9
        &&\text{because $\ff(x)=\frac{1}{9}x$ is \prope{strictly isotone} and by \prefpp{lem:argminmaxphi}}
      \\&=\mathrlap{\argmin_{x\in\ocsG}\max_{y\in\ocsG}
             \setn{\begin{array}{*{5}{c}}
               \metricn(0,0)\psp(0)9&\metricn(0,1)\psp(1)9&\metricn(0,2)\psp(2)9&\metricn(0,3)\psp(3)9&\metricn(0,4)\psp(4)9\\
               \metricn(1,0)\psp(0)9&\metricn(1,1)\psp(1)9&\metricn(1,2)\psp(2)9&\metricn(1,3)\psp(3)9&\metricn(1,4)\psp(4)9\\
               \vdots              &\vdots              &\vdots              &\vdots              &\vdots              \\
              %\metricn(2,0)\psp(0)9&\metricn(2,1)\psp(1)9&\metricn(2,2)\psp(2)9&\metricn(2,3)\psp(3)9&\metricn(2,4)\psp(4)9\\
              %\metricn(3,0)\psp(0)9&\metricn(3,1)\psp(1)9&\metricn(3,2)\psp(2)9&\metricn(3,3)\psp(3)9&\metricn(3,4)\psp(4)9\\
               \metricn(4,0)\psp(0)9&\metricn(4,1)\psp(1)9&\metricn(4,2)\psp(2)9&\metricn(4,3)\psp(3)9&\metricn(4,4)\psp(4)9\\
             \end{array}}}
      %\\&= \argmin_{x\in\ocsG}\max_{y\in\ocsG}
      %       \setn{\begin{array}{*{5}{c}}
      %         \metricn(0,0)\frac{2}{9}&\metricn(0,1)\frac{1}{9}&\cdots&\metricn(0,4)\frac{3}{9}\\
      %         \metricn(1,0)\frac{2}{9}&\metricn(1,1)\frac{1}{9}&\cdots&\metricn(1,4)\frac{3}{9}\\
      %         \vdots                  &\ddots                  &\ddots&\vdots                  \\
      %         \metricn(4,0)\frac{2}{9}&\metricn(4,1)\frac{1}{9}&\cdots&\metricn(4,4)\frac{3}{9}\\
      %       \end{array}}
      \\&=\mathrlap{\argmin_{x\in\ocsG}\max_{y\in\ocsG}
             \setn{\begin{array}{*{5}{c}}
               {0}\times2 & {2}\times1 & {1}\times1 & {1}\times2 & {2}\times3\\
               {2}\times2 & {0}\times1 & {2}\times1 & {1}\times2 & {1}\times3\\
               {1}\times2 & {2}\times1 & {0}\times1 & {2}\times2 & {1}\times3\\
               {1}\times2 & {1}\times1 & {2}\times1 & {0}\times2 & {1}\times3\\
               {2}\times2 & {1}\times1 & {1}\times1 & {1}\times2 & {0}\times3\\
             \end{array}}
      %\quad= \argmin_{x\in\ocsG}\max_{y\in\ocsG}
      %       \setn{\begin{array}{*{5}{c}}
      %         0 & 2 & 1 & 2 & 6\\
      %         4 & 0 & 2 & 2 & 3\\
      %         2 & 2 & 0 & 4 & 3\\
      %         2 & 1 & 2 & 0 & 3\\
      %         4 & 1 & 1 & 2 & 0\\
      %       \end{array}}
      \quad= \argmin_{x\in\ocsG}
             \setn{\begin{array}{c}
                6\\
                4\\
                4\\
                3\\
                4
             \end{array}}
      \quad= \argmin_{x\in\ocsG}%\frac{1}{9}
             \setn{\begin{array}{c}
                \mbox{ }\\
                \mbox{ }\\
                \mbox{ }\\
                3\\
                \mbox{}
             \end{array}}}
      \\
      \ocscena(\ocsG)
        &\eqd \argmin_{x\in\ocsG}\sum_{y\in\ocsG}\metric{x}{y}\psp(y)
        &&\text{by definition of $\ocscena$ \xref{def:ocscena}}
      \\&=\mathrlap{\argmin_{x\in\ocsG}%\sum_{y\in\ocsG}%\frac{1}{9}
             \setn{\begin{array}{*{9}{c}}
               {0}\times2 &+& {2}\times1 &+& {1}\times1 &+& {1}\times2 &+& {2}\times3\\
               {2}\times2 &+& {0}\times1 &+& {2}\times1 &+& {1}\times2 &+& {1}\times3\\
               {1}\times2 &+& {2}\times1 &+& {0}\times1 &+& {2}\times2 &+& {1}\times3\\
               {1}\times2 &+& {1}\times1 &+& {2}\times1 &+& {0}\times2 &+& {1}\times3\\
               {2}\times2 &+& {1}\times1 &+& {1}\times1 &+& {1}\times2 &+& {0}\times3
             \end{array}}
      \quad= \argmin_{x\in\ocsG}%\frac{1}{9}
             \setn{\begin{array}{c}
                11\\
                11\\
                11\\
                 8\\
                 8
             \end{array}}
      \quad= \argmin_{x\in\ocsG}%\frac{1}{9}
             \setn{\begin{array}{c}
                \mbox{ }\\
                \mbox{ }\\
                \mbox{ }\\
                3\\
                4
             \end{array}}}
      \\
      \ocsceng(\ocsG)
        &\eqd \argmin_{x\in\ocsG}\prod_{y\in\ocsG\setd\setn{x}}{\metric{x}{y}^{\psp(y)}}
        &&\text{by definition of $\ocsceng$ \xref{def:ocsceng}}
      \\&= \argmin_{x\in\ocsG}\prod_{y\in\ocsG\setd\setn{x}}{\metric{x}{y}^{9\psp(y)\frac{1}{9}}}
      \\&= \argmin_{x\in\ocsG}\brs{\prod_{y\in\ocsG\setd\setn{x}}{\metric{x}{y}^{9\psp(y)}}}^\frac{1}{9}
      \\&= \argmin_{x\in\ocsG}\prod_{y\in\ocsG\setd\setn{x}}{\metric{x}{y}^{9\psp(y)}}
        &&\text{because $\ff(x)\eqd x^\frac{1}{9}$ is \prope{strictly isotone} and by \prefpp{lem:argminphi}}
      %\\&= \argmin_{x\in\ocsG}
      %       \setn{\begin{array}{ccccccccc}
      %                                 &      & \metric{0}{1}^{\psp(1)} &\times& \metric{0}{2}^{\psp(2)} &\times& \metric{0}{3}^{\psp(3)} &\times& \metric{0}{4}^{\psp(4)}\\
      %         \metric{1}{0}^{\psp(0)} &      &                         &\times& \metric{1}{2}^{\psp(2)} &\times& \metric{1}{3}^{\psp(3)} &\times& \metric{1}{4}^{\psp(4)}\\
      %         \metric{2}{0}^{\psp(0)} &\times& \metric{2}{1}^{\psp(1)} &      &                         &\times& \metric{2}{3}^{\psp(3)} &\times& \metric{2}{4}^{\psp(4)}\\
      %         \metric{3}{0}^{\psp(0)} &\times& \metric{3}{1}^{\psp(1)} &\times& \metric{3}{2}^{\psp(2)} &      &                         &\times& \metric{3}{4}^{\psp(4)}\\
      %         \metric{4}{0}^{\psp(0)} &\times& \metric{4}{1}^{\psp(1)} &\times& \metric{4}{2}^{\psp(2)} &\times& \metric{4}{3}^{\psp(3)} &      &
      %       \end{array}}
      %\\&= \argmin_{x\in\ocsG}
      %       \setn{\begin{array}{ccccccccc}
      %                                   &      & \metricn(0,1)^\frac{1}{9} &\times& \metricn(0,2)^\frac{1}{9} &\times& \metricn(0,3)^\frac{2}{9} &\times& \metricn(0,4)^\frac{3}{9}\\
      %         \metricn(1,0)^\frac{2}{9} &      &                           &\times& \metricn(1,2)^\frac{1}{9} &\times& \metricn(1,3)^\frac{2}{9} &\times& \metricn(1,4)^\frac{3}{9}\\
      %         \metricn(2,0)^\frac{2}{9} &\times& \metricn(2,1)^\frac{1}{9} &      &                           &\times& \metricn(2,3)^\frac{2}{9} &\times& \metricn(2,4)^\frac{3}{9}\\
      %         \metricn(3,0)^\frac{2}{9} &\times& \metricn(3,1)^\frac{1}{9} &\times& \metricn(3,2)^\frac{1}{9} &      &                           &\times& \metricn(3,4)^\frac{3}{9}\\
      %         \metricn(4,0)^\frac{2}{9} &\times& \metricn(4,1)^\frac{1}{9} &\times& \metricn(4,2)^\frac{1}{9} &\times& \metricn(4,3)^\frac{2}{9} &      &                          
      %       \end{array}}
      \\&=\mathrlap{\argmin_{x\in\ocsG}
             \setn{\begin{array}{ccccccccc}
                     &      & {2}^1 &\times& {1}^1 &\times& {1}^2 &\times& {2}^3\\
               {2}^2 &\times&       &      & {2}^1 &\times& {1}^2 &\times& {1}^3\\
               {1}^2 &\times& {2}^1 &\times&       &      & {2}^2 &\times& {1}^3\\
               {1}^2 &\times& {1}^1 &\times& {2}^1 &\times&       &      & {1}^3\\
               {2}^2 &\times& {1}^1 &\times& {1}^1 &\times& {1}^2 &      &
             \end{array}}
      \quad= \argmin_{x\in\ocsG}
             \setn{\begin{array}{c}
                 2^4\\
                 2^3\\
                 2^3\\
                 2^1\\
                 2^2
             \end{array}}
      \quad= \argmin_{x\in\ocsG}%\frac{1}{9}
             \setn{\begin{array}{c}
                \mbox{ }\\
                \mbox{ }\\
                \mbox{ }\\
                3\\
                \mbox{ }
             \end{array}}}
      \\
      \ocscenh(\ocsG)
        &\eqd \argmin_{x\in\ocso}\brp{\sum_{y\in\ocso}\frac{1}{\metric{x}{y}}\psp(y)}^{-1} 
        &&\text{by definition of $\ocscenh$ \xref{def:ocscenh}}
      \\&= \argmax_{x\in\ocso}\brp{\sum_{y\in\ocso}\frac{1}{\metric{x}{y}}\psp(y)}
        && \text{because $\fphi(x)\eqd x^{-1}$ is \prope{strictly antitone} and by \prefp{lem:minphia}}
      \\&= \argmax_{x\in\ocso}\brp{\frac{1}{9}\sum_{y\in\ocso}\frac{1}{\metric{x}{y}}\psp(y)9}
      \\&= \argmax_{x\in\ocso}\sum_{y\in\ocso}\frac{9\psp(y)}{\metric{x}{y}}
        &&\text{because $\ff(x)=\frac{1}{9}x$ is \prope{strictly isotone} and by \prefpp{lem:argminphi}}
      %\\&= \argmin_{x\in\ocsG}
      %       \setn{\begin{array}{ccccc}
      %         0                                + \frac{1}{\metric{0}{1}}{\psp(1)} + \frac{1}{\metric{0}{2}}{\psp(2)} + \frac{1}{\metric{0}{3}}{\psp(3)} + \frac{1}{\metric{0}{4}}{\psp(4)}\\
      %         \frac{1}{\metric{1}{0}}{\psp(0)} + 0                                + \frac{1}{\metric{1}{2}}{\psp(2)} + \frac{1}{\metric{1}{3}}{\psp(3)} + \frac{1}{\metric{1}{4}}{\psp(4)}\\
      %         \frac{1}{\metric{2}{0}}{\psp(0)} + \frac{1}{\metric{2}{1}}{\psp(1)} + 0                                + \frac{1}{\metric{2}{3}}{\psp(3)} + \frac{1}{\metric{2}{4}}{\psp(4)}\\
      %         \frac{1}{\metric{3}{0}}{\psp(0)} + \frac{1}{\metric{3}{1}}{\psp(1)} + \frac{1}{\metric{3}{2}}{\psp(2)} + 0                                + \frac{1}{\metric{3}{4}}{\psp(4)}\\
      %         \frac{1}{\metric{4}{0}}{\psp(0)} + \frac{1}{\metric{4}{1}}{\psp(1)} + \frac{1}{\metric{4}{2}}{\psp(2)} + \frac{1}{\metric{4}{3}}{\psp(3)} + 0                         
      %       \end{array}}
      %\\&= \argmin_{x\in\ocsG}
      %       \setn{\begin{array}{ccccc}
      %         \brs{                            + \frac{1}{2}\cdot\frac{1}{9} + \frac{1}{1}\cdot\frac{1}{9} + \frac{1}{1}\cdot\frac{2}{9} + \frac{1}{2}\cdot\frac{3}{9}}^{-1}\\
      %         \brs{\frac{1}{2}\cdot\frac{2}{9} + 0                           + \frac{1}{2}\cdot\frac{1}{9} + \frac{1}{1}\cdot\frac{2}{9} + \frac{1}{1}\cdot\frac{3}{9}}^{-1}\\
      %         \brs{\frac{1}{1}\cdot\frac{2}{9} + \frac{1}{2}\cdot\frac{1}{9} + 0                           + \frac{1}{2}\cdot\frac{2}{9} + \frac{1}{1}\cdot\frac{3}{9}}^{-1}\\
      %         \brs{\frac{1}{1}\cdot\frac{2}{9} + \frac{1}{1}\cdot\frac{1}{9} + \frac{1}{2}\cdot\frac{1}{9} + 0                           + \frac{1}{1}\cdot\frac{3}{9}}^{-1}\\
      %         \brs{\frac{1}{2}\cdot\frac{2}{9} + \frac{1}{1}\cdot\frac{1}{9} + \frac{1}{1}\cdot\frac{1}{9} + \frac{1}{1}\cdot\frac{2}{9} + 0                          }^{-1}
      %       \end{array}}
      %\\&= \argmin_{x\in\ocsG}9
      %       \setn{\begin{array}{ccccc}
      %         \brs{0           + \frac{1}{2} + \frac{1}{1} + \frac{2}{1} + \frac{3}{2}}^{-1}\\
      %         \brs{\frac{2}{2} + 0           + \frac{1}{2} + \frac{2}{1} + \frac{3}{1}}^{-1}\\
      %         \brs{\frac{2}{1} + \frac{1}{2} + 0           + \frac{2}{2} + \frac{3}{1}}^{-1}\\
      %         \brs{\frac{2}{1} + \frac{1}{1} + \frac{1}{2} + 0           + \frac{3}{1}}^{-1}\\
      %         \brs{\frac{2}{2} + \frac{1}{1} + \frac{1}{1} + \frac{2}{1} + 0          }^{-1}
      %       \end{array}}
      %\\&= \argmin_{x\in\ocsG}9
      %       \setn{\begin{array}{c}
      %         \brs{\frac{10}{2}}^{-1}\\
      %         \brs{\frac{13}{2}}^{-1}\\
      %         \brs{\frac{13}{2}}^{-1}\\
      %         \brs{\frac{13}{2}}^{-1}\\
      %         \brs{\frac{11}{2}}^{-1}
      %       \end{array}}
      \\&=\mathrlap{\argmax_{x\in\ocsG}
             \setn{\begin{array}{*{4}{cc}c}
               0           &+& \frac{1}{2} &+& \frac{1}{1} &+& \frac{2}{1} &+& \frac{3}{2}\\
               \frac{2}{2} &+& 0           &+& \frac{1}{2} &+& \frac{2}{1} &+& \frac{3}{1}\\
               \frac{2}{1} &+& \frac{1}{2} &+& 0           &+& \frac{2}{2} &+& \frac{3}{1}\\
               \frac{2}{1} &+& \frac{1}{1} &+& \frac{1}{2} &+& 0           &+& \frac{3}{1}\\
               \frac{2}{2} &+& \frac{1}{1} &+& \frac{1}{1} &+& \frac{2}{1} &+& 0          
             \end{array}}
      \quad= \argmax_{x\in\ocsG}\frac{1}{2}
             \setn{\begin{array}{c}
               {10}\\
               {13}\\
               {13}\\
               {13}\\
               {10}
             \end{array}}
      \quad= \setn{\begin{array}{c}
                \\
               1\\
               2\\
               3\\
               \mbox{}
             \end{array}}}
      \\
      \ocscenm(\ocsG)
        &\eqd \argmin_{x\in\ocsG}\min_{y\in\ocsG\setd\setn{x}}\metric{x}{y}\psp(y)
        &&\text{by definition of $\ocscenm$ \xref{def:ocscenm}}
      \\&= \argmin_{x\in\ocsG}\min_{y\in\ocsG\setd\setn{x}}\frac{1}{9}\metric{x}{y}\psp(y)9
      \\&= \argmin_{x\in\ocsG}\min_{y\in\ocsG\setd\setn{x}}\metric{x}{y}\psp(y)9
        &&\text{because $\ff(x)=\frac{1}{9}x$ is \prope{strictly isotone} and by \prefpp{lem:argminmaxphi}}
      \\&=\mathrlap{\argmin_{x\in\ocsG}\min_{y\in\ocsG\setd\setn{x}}%\frac{1}{9}
             \setn{\begin{array}{ccccccccc}
                         &{2}\times1&{1}\times1&{1}\times2&{2}\times3\\
               {2}\times2&          &{2}\times1&{1}\times2&{1}\times3\\
               {1}\times2&{2}\times1&          &{2}\times2&{1}\times3\\
               {1}\times2&{1}\times1&{2}\times1&          &{1}\times3\\
               {2}\times2&{1}\times1&{1}\times1&{1}\times2& 
             \end{array}}
      \quad= \argmin_{x\in\ocsG}
             \setn{\begin{array}{c}
                1\\
                2\\
                2\\
                1\\
                1
             \end{array}}
      \quad= \setn{\begin{array}{c}
                0\\
                 \\
                 \\
                3\\
                4
             \end{array}}}
      \\
      \ocscenM(\ocsG)
        &\eqd \argmax_{x\in\ocsG}\min_{y\in\ocsG\setd\setn{x}}\metric{x}{y}\psp(y)
        &&\text{by definition of $\ocscenM$ \xref{def:ocscenM}}
      \\&= \argmax_{x\in\ocsG}\setn{1,2,2,1,1}
        &&\text{by $\ocscenm(\ocsG)$ result}
      \\&= \setn{1,2}
      \\
      \ocsVaro(\ocsG)
        &= \ocsVaro(\ocsG;\ocsceng)
        && \text{by $\ocscen(\ocsG)$ and $\ocsceng(\ocsG)$ results}
      \\&\eqd \sum_{x\in\ocsG}\metricsq{\ocsceng(\ocsG)}{x}\psp(x)
        && \text{by definition of $\ocsVaro$ \xref{def:ocsVarG}}
      \\&= \sum_{x\in\ocsG}\metricsq{\setn{3}}{x}\psp(x)
        && \text{by $\ocsceng(\ocsG)$ result}
      \\&= \mathrlap{
           (1)^2\frac{2}{9}+ % x=0
           (1)^2\frac{1}{9}+ % x=1
           (2)^2\frac{1}{9}+ % x=2
           (0)^2\frac{2}{9}+ % x=3
           (1)^2\frac{3}{9}  % x=4
         = \frac{10}{9} \approx 1.111}
      \\
      \ocsVaro(\ocsG;\ocscena)
        &\eqd \sum_{x\in\ocsG}\metricsq{\ocscena(\ocsG)}{x}\psp(x)
        && \text{by definition of $\ocsVaro$ \xref{def:ocsVarG}}
      \\&= \sum_{x\in\ocsG}\metricsq{\setn{3,4}}{x}\psp(x)
        && \text{by $\ocscena(\ocsG)$ result}
      \\&= \mathrlap{
           (1)^2\frac{2}{9}+ % x=0
           (1)^2\frac{1}{9}+ % x=1
           (1)^2\frac{1}{9}+ % x=2
           (0)^2\frac{2}{9}+ % x=3
           (0)^2\frac{3}{9}  % x=4
         = \frac{4}{9} \approx 0.444}
      \\
      \ocsVaro(\ocsG;\ocscenh)
        &\eqd \sum_{x\in\ocsG}\metricsq{\ocscenh(\ocsG)}{x}\psp(x)
        && \text{by definition of $\ocsVaro$ \xref{def:ocsVarG}}
      \\&= \sum_{x\in\ocsG}\metricsq{\setn{1,2,3}}{x}\psp(x)
        && \text{by $\ocscenh(\ocsG)$ result}
      \\&= \mathrlap{
           (1)^2\frac{2}{9}+ % x=0
           (0)^2\frac{1}{9}+ % x=1
           (0)^2\frac{1}{9}+ % x=2
           (0)^2\frac{2}{9}+ % x=3
           (1)^2\frac{3}{9}  % x=4
         = \frac{5}{9} \approx 0.555}
      \\
      \ocsVaro(\ocsG;\ocscenm)
        &\eqd \sum_{x\in\ocsG}\metricsq{\ocscenm(\ocsG)}{x}\psp(x)
        && \text{by definition of $\ocsVaro$ \xref{def:ocsVarG}}
      \\&= \sum_{x\in\ocsG}\metricsq{\setn{0,3,4}}{x}\psp(x)
        && \text{by $\ocscenm(\ocsG)$ result}
      \\&= \mathrlap{
           (0)^2\frac{2}{9}+ % x=0
           (1)^2\frac{1}{9}+ % x=1
           (1)^2\frac{1}{9}+ % x=2
           (0)^2\frac{2}{9}+ % x=3
           (0)^2\frac{3}{9}  % x=4
         = \frac{2}{9} \approx 0.222}
      \\
      \ocsVaro(\ocsG;\ocscenM)
        &\eqd \sum_{x\in\ocsG}\metricsq{\ocscenM(\ocsG)}{x}\psp(x)
        && \text{by definition of $\ocsVaro$ \xref{def:ocsVarG}}
      \\&= \sum_{x\in\ocsG}\metricsq{\setn{1,2}}{x}\psp(x)
        && \text{by $\ocscenM(\ocsG)$ result}
      \\&= \mathrlap{
           (1)^2\frac{2}{9}+ % x=0
           (0)^2\frac{1}{9}+ % x=1
           (0)^2\frac{1}{9}+ % x=2
           (1)^2\frac{2}{9}+ % x=3
           (1)^2\frac{3}{9}  % x=4
         = \frac{7}{9} \approx 0.778}
    \end{align*}
\end{proof}


%---------------------------------------
\begin{minipage}{\tw-65mm}%
\begin{example}%[\exmd{weigted ring}]
\label{ex:wring5shortd}
%---------------------------------------
The \structe{outcome subspace} \xref{def:ocs} illustrated to the right, with a \fncte{quasi-metric} \xref{def:qmetric}
has the following geometric values:
\\\indentx$\begin{array}{rcl@{\qquad} lcccl}
  \ocscen(\ocsG)=\ocscena(\ocsG)=\ocsceng(\ocsG)=\ocscenh(\ocsG) &=& \setn{3}    & \ocsVaro(\ocsG)          &=& \frac{12}{9} &\approx& 1.333\\
  \ocscenm(\ocsG)                                                &=& \setn{0,4}  & \ocsVaro(\ocsG;\ocscenm) &=& \frac{10}{9} &\approx& 1.111\\
  \ocscenM(\ocsG)                                                &=& \setn{1,2,3}& \ocsVaro(\ocsG;\ocscenM) &=& \frac{ 5}{9} &\approx& 0.555
\end{array}$
\end{example}
\end{minipage}\hfill%
\begin{tabular}{c}
  \gsize%
  %\psset{unit=5mm}%
  %{%============================================================================
% Daniel J. Greenhoe
% LaTeX file
% linear congruential (LCG) pseudo-random number generator (PRNG) mappings
% x_{n+1} = (7x_n+5)mod 9
% y_{n+1} = (y_n+2)mod 5
%============================================================================
\begin{pspicture}(-1.3,-1.5)(1.5,1.5)%
  %---------------------------------
  % options
  %---------------------------------
  \psset{%
    radius=1.25ex,
    labelsep=2.5mm,
    linecolor=blue,%
    }%
  \rput{288}(0,0){\rput(1,0){\Cnode(0,0){S4}}}%
  \rput{216}(0,0){\rput(1,0){\Cnode(0,0){S2}}}%
  \rput{144}(0,0){\rput(1,0){\Cnode(0,0){S0}}}%
  \rput{ 72}(0,0){\rput(1,0){\Cnode[fillstyle=solid,fillcolor=snode](0,0){S3}}}%
  \rput{  0}(0,0){\rput(1,0){\Cnode(0,0){S1}}}%
  %
  \rput(S4){$4$}%
  \rput(S3){$3$}%
  \rput(S2){$2$}%
  \rput(S1){$1$}%
  \rput(S0){$0$}%
  %
  \ncline{->}{S4}{S1}\ncline{->}{S2}{S4}\ncline{->}{S3}{S4}%
  \ncline{->}{S2}{S4}%
  \ncline{->}{S0}{S2}%
  \ncline{->}{S3}{S0}%
  \ncline{->}{S1}{S3}%
  %
  \uput[-30](S4){$\frac{3}{9}$}
  \uput[ 30](S3){$\frac{2}{9}$}
  \uput[216](S2){$\frac{1}{9}$}
  \uput[  0](S1){$\frac{1}{9}$}
  \uput[144](S0){$\frac{2}{9}$}
  \rput(0,0){$\ocsG$}%
\end{pspicture}%}%
  {\includegraphics{sto/graphics/wring5shortd.pdf}}%
\end{tabular}
\\
This is the first example in this section to use a \structe{directed graph} 
(rather than an \structe{undirected graph} \xrefnp{def:dgraph})
and to require the use of a \fncte{quasi-metric} \xref{def:qmetric} that is not a \fncte{metric}.
Unlike \prefpp{ex:wring5short}, which had neither of these restrictions,
twice as many center operators (4 rather than 2) were able to successfully isolate a single center point.
The \structe{outcome center} result is used later in \prefpp{ex:lcg7x1m9_dgraph}.
\\
\begin{proof}
    \begin{align*}
      \ocscen(\ocsG)
        &\eqd \argmin_{x\in\ocsG}\max_{y\in\ocsG}\metric{x}{y}\psp(y)
        &&\text{by definition of $\ocscen$ \xref{def:ocscen}}
      \\&\eqd \argmin_{x\in\ocsG}\max_{y\in\ocsG}\frac{1}{9}\metric{x}{y}\psp(y)9
      \\&\eqd \argmin_{x\in\ocsG}\max_{y\in\ocsG}\metric{x}{y}\psp(y)9
        &&\text{because $\ff(x)=\frac{1}{9}x$ is \prope{strictly isotone} and by \prefpp{lem:argminmaxphi}}
      \\&=\mathrlap{\argmin_{x\in\ocsG}\max_{y\in\ocsG}
             \setn{\begin{array}{*{5}{c}}
               \metricn(0,0)\psp(0)9&\metricn(0,1)\psp(1)9&\metricn(0,2)\psp(2)9&\metricn(0,3)\psp(3)9&\metricn(0,4)\psp(4)9\\
               \metricn(1,0)\psp(0)9&\metricn(1,1)\psp(1)9&\metricn(1,2)\psp(2)9&\metricn(1,3)\psp(3)9&\metricn(1,4)\psp(4)9\\
               \vdots              &\vdots              &\vdots              &\vdots              &\vdots              \\
              %\metricn(2,0)\psp(0)9&\metricn(2,1)\psp(1)9&\metricn(2,2)\psp(2)9&\metricn(2,3)\psp(3)9&\metricn(2,4)\psp(4)9\\
              %\metricn(3,0)\psp(0)9&\metricn(3,1)\psp(1)9&\metricn(3,2)\psp(2)9&\metricn(3,3)\psp(3)9&\metricn(3,4)\psp(4)9\\
               \metricn(4,0)\psp(0)9&\metricn(4,1)\psp(1)9&\metricn(4,2)\psp(2)9&\metricn(4,3)\psp(3)9&\metricn(4,4)\psp(4)9\\
             \end{array}}}
      %\\&= \argmin_{x\in\ocsG}\max_{y\in\ocsG}
      %       \setn{\begin{array}{*{5}{c}}
      %         \metricn(0,0)\frac{2}{9}&\metricn(0,1)\frac{1}{9}&\cdots&\metricn(0,4)\frac{3}{9}\\
      %         \metricn(1,0)\frac{2}{9}&\metricn(1,1)\frac{1}{9}&\cdots&\metricn(1,4)\frac{3}{9}\\
      %         \vdots                  &\ddots                  &\ddots&\vdots                  \\
      %         \metricn(4,0)\frac{2}{9}&\metricn(4,1)\frac{1}{9}&\cdots&\metricn(4,4)\frac{3}{9}\\
      %       \end{array}}
      \\&=\mathrlap{\argmin_{x\in\ocsG}\max_{y\in\ocsG}
             %\setn{\begin{array}{*{5}{@{\hspace{1pt}}c@{\hspace{1pt}}}}
             \setn{\begin{array}{*{5}{c}}
               {0}\times2 & {3}\times1 & {1}\times1 & {4}\times2 & {2}\times3\\
               {2}\times2 & {0}\times1 & {3}\times1 & {1}\times2 & {2}\times3\\
               {4}\times2 & {2}\times1 & {0}\times1 & {3}\times2 & {1}\times3\\
               {1}\times2 & {2}\times1 & {2}\times1 & {0}\times2 & {1}\times3\\
               {3}\times2 & {1}\times1 & {4}\times1 & {2}\times2 & {0}\times3\\
             \end{array}}
      \quad= \argmin_{x\in\ocsG}
             \setn{\begin{array}{c}
                8\\
                6\\
                8\\
                3\\
                6
             \end{array}}
      \quad= \setn{\begin{array}{c}
                \mbox{ }\\
                \mbox{ }\\
                \mbox{ }\\
                3\\
                \mbox{ }
             \end{array}}}
      \\\\
      \ocscena(\ocsG)
        &\eqd \argmin_{x\in\ocsG}\sum_{y\in\ocsG}\metric{x}{y}\psp(y)
        &&\text{by definition of $\ocscena$ \xref{def:ocscena}}
      \\&= \argmin_{x\in\ocsG}\sum_{y\in\ocsG}\frac{1}{9}\metric{x}{y}\psp(y)9
      \\&= \argmin_{x\in\ocsG}\sum_{y\in\ocsG}\metric{x}{y}\psp(y)9
        &&\text{because $\ff(x)=\frac{1}{9}x$ is \prope{strictly isotone} and by \prefpp{lem:argminphi}}
      %\\&= \argmin_{x\in\ocsG}\max_{y\in\ocsG}
      %       \setn{\begin{array}{ccccccccc}
      %         \metricn(0,0)\frac{2}{9}&+&\metricn(0,1)\frac{1}{9}&+&\metricn(0,2)\frac{1}{9}&+&\metricn(0,3)\frac{2}{9}&+&\metricn(0,4)\frac{3}{9}\\
      %         \metricn(1,0)\frac{2}{9}&+&\metricn(1,1)\frac{1}{9}&+&\metricn(1,2)\frac{1}{9}&+&\metricn(1,3)\frac{2}{9}&+&\metricn(1,4)\frac{3}{9}\\
      %         \metricn(2,0)\frac{2}{9}&+&\metricn(2,1)\frac{1}{9}&+&\metricn(2,2)\frac{1}{9}&+&\metricn(2,3)\frac{2}{9}&+&\metricn(2,4)\frac{3}{9}\\
      %         \metricn(3,0)\frac{2}{9}&+&\metricn(3,1)\frac{1}{9}&+&\metricn(3,2)\frac{1}{9}&+&\metricn(3,3)\frac{2}{9}&+&\metricn(3,4)\frac{3}{9}\\
      %         \metricn(4,0)\frac{2}{9}&+&\metricn(4,1)\frac{1}{9}&+&\metricn(4,2)\frac{1}{9}&+&\metricn(4,3)\frac{2}{9}&+&\metricn(4,4)\frac{3}{9}\\
      %       \end{array}}
      \\&=\mathrlap{\argmin_{x\in\ocsG}\max_{y\in\ocsG}%\frac{1}{9}
             \setn{\begin{array}{ccccccccc}
               {0}\times2 &+& {3}\times1 &+& {1}\times1 &+& {4}\times2 &+& {2}\times3\\
               {2}\times2 &+& {0}\times1 &+& {3}\times1 &+& {1}\times2 &+& {2}\times3\\
               {4}\times2 &+& {2}\times1 &+& {0}\times1 &+& {3}\times2 &+& {1}\times3\\
               {1}\times2 &+& {2}\times1 &+& {2}\times1 &+& {0}\times2 &+& {1}\times3\\
               {3}\times2 &+& {1}\times1 &+& {4}\times1 &+& {2}\times2 &+& {0}\times3\\
             \end{array}}
      \quad= \argmin_{x\in\ocsG}%\frac{1}{9}
             \setn{\begin{array}{ccccc}
                18\\
                15\\
                19\\
                 9\\
                15
             \end{array}}
      \quad= \setn{\begin{array}{c}
                \mbox{ }\\
                \mbox{ }\\
                \mbox{ }\\
                3\\
                \mbox{ }
             \end{array}}}
      \\\\
      \ocsceng(\ocsG)
        &\eqd \argmin_{x\in\ocsG}\prod_{y\in\ocsG\setd\setn{x}}{\metric{x}{y}^{\psp(y)}}
        &&\text{by definition of $\ocsceng$ \xref{def:ocsceng}}
      \\&= \argmin_{x\in\ocsG}\prod_{y\in\ocsG\setd\setn{x}}{\metric{x}{y}^{9\psp(y)\frac{1}{9}}}
      \\&= \argmin_{x\in\ocsG}\brs{\prod_{y\in\ocsG\setd\setn{x}}{\metric{x}{y}^{9\psp(y)}}}^\frac{1}{9}
      \\&= \argmin_{x\in\ocsG}\prod_{y\in\ocsG\setd\setn{x}}{\metric{x}{y}^{9\psp(y)}}
        &&\text{because $\ff(x)\eqd x^\frac{1}{9}$ is \prope{strictly isotone} and by \prefpp{lem:argminphi}}
      \\&=\mathrlap{\argmin_{x\in\ocsG}
             \setn{\begin{array}{ccccccccc}
                     &      & {3}^1 &\times& {1}^1 &\times& {4}^2 &\times& {2}^3\\
               {2}^2 &\times&       &      & {3}^1 &\times& {1}^2 &\times& {2}^3\\
               {4}^2 &\times& {2}^1 &\times&       &      & {3}^2 &\times& {1}^3\\
               {1}^2 &\times& {2}^1 &\times& {2}^1 &\times&       &      & {1}^3\\
               {3}^2 &\times& {1}^1 &\times& {4}^1 &\times& {2}^2 &      &
             \end{array}}
      \quad= \argmin_{x\in\ocsG}
             \setn{\begin{array}{c}
               384\\
               192\\
               432\\
                24\\
               144
             \end{array}}
      \quad= \setn{\begin{array}{c}
                \mbox{ }\\
                \mbox{ }\\
                \mbox{ }\\
                3\\
                \mbox{ }
             \end{array}}}
      \\\\
      \ocscenh(\ocsG)
        &\eqd \argmin_{x\in\ocso}\brp{\sum_{y\in\ocso}\frac{1}{\metric{x}{y}}\psp(y)}^{-1} 
        &&\text{by definition of $\ocscenh$ \xref{def:ocscenh}}
      \\&= \argmax_{x\in\ocso}\brp{\sum_{y\in\ocso}\frac{1}{\metric{x}{y}}\psp(y)}
        && \text{because $\fphi(x)\eqd x^{-1}$ is \prope{strictly antitone} and by \prefp{lem:minphia}}
      \\&= \argmax_{x\in\ocso}\brp{\frac{1}{9}\sum_{y\in\ocso}\frac{1}{\metric{x}{y}}\psp(y)9}
      \\&= \argmax_{x\in\ocso}\sum_{y\in\ocso}\frac{9\psp(y)}{\metric{x}{y}}
        &&\text{because $\ff(x)=\frac{1}{9}x$ is \prope{strictly isotone} and by \prefpp{lem:argminphi}}
      \\&=\mathrlap{\argmax_{x\in\ocsG}
             \setn{\begin{array}{*{4}{cc}c}
               0           &+& \frac{1}{3} &+& \frac{1}{1} &+& \frac{2}{4} &+& \frac{3}{2}\\
               \frac{2}{2} &+& 0           &+& \frac{1}{3} &+& \frac{2}{1} &+& \frac{3}{2}\\
               \frac{2}{4} &+& \frac{1}{2} &+& 0           &+& \frac{2}{3} &+& \frac{3}{1}\\
               \frac{2}{1} &+& \frac{1}{2} &+& \frac{1}{2} &+& 0           &+& \frac{3}{1}\\
               \frac{2}{3} &+& \frac{1}{1} &+& \frac{1}{4} &+& \frac{2}{2} &+& 0          
             \end{array}}
      \quad= \argmax_{x\in\ocsG}\frac{1}{2}
             \setn{\begin{array}{c}
               {20}\\
               {27}\\
               {28}\\
               {36}\\
               {35}
             \end{array}}
      \quad= \setn{\begin{array}{c}
                \mbox{ }\\
                \mbox{ }\\
                \mbox{ }\\
                3\\
                \mbox{ }
             \end{array}}}
      \\
      \ocscenm(\ocsG)
      \\&\eqd \argmin_{x\in\ocsG}\min_{y\in\ocsG\setd\setn{x}}\metric{x}{y}\psp(y)
        &&\text{by definition of $\ocscenm$ \xref{def:ocscenm}}
      \\&= \argmin_{x\in\ocsG}\min_{y\in\ocsG\setd\setn{x}}\frac{1}{9}\metric{x}{y}\psp(y)9
      \\&= \argmin_{x\in\ocsG}\min_{y\in\ocsG\setd\setn{x}}\metric{x}{y}\psp(y)9
        &&\text{because $\ff(x)=\frac{1}{9}x$ is \prope{strictly isotone} and by \prefpp{lem:argminmaxphi}}
      \\&=\mathrlap{\argmin_{x\in\ocsG}\min_{y\in\ocsG\setd\setn{x}}%\frac{1}{9}
             \setn{\begin{array}{ccccccccc}
                          & {3}\times1 & {1}\times1 & {4}\times2 & {2}\times3\\
               {2}\times2 &            & {3}\times1 & {1}\times2 & {2}\times3\\
               {4}\times2 & {2}\times1 &            & {3}\times2 & {1}\times3\\
               {1}\times2 & {2}\times1 & {2}\times1 &            & {1}\times3\\
               {3}\times2 & {1}\times1 & {4}\times1 & {2}\times2 &             
             \end{array}}
      \quad= \argmin_{x\in\ocsG}
             \setn{\begin{array}{c}
                1\\
                2\\
                2\\
                2\\
                1
             \end{array}}
      \quad= \setn{\begin{array}{c}
                0\\
                \mbox{ }\\
                \mbox{ }\\
                \mbox{ }\\
                4
             \end{array}}}
    \\
      \ocscenM(\ocsG)
        &\eqd \argmax_{x\in\ocsG}\min_{y\in\ocsG\setd\setn{x}}\metric{x}{y}\psp(y)
        &&\text{by definition of $\ocscenM$ \xref{def:ocscenM}}
      \\&\eqd \argmax_{x\in\ocsG}\setn{1,2,2,2,1}
        &&\text{by $\ocscenm(\ocsG)$ result}
      \\&= \setn{1,2,3}
      \\
      \ocsVaro(\ocsG)
        &= \mathrlap{\ocsVaro(\ocsG;\ocscena)= \ocsVaro(\ocsG;\ocsceng)= \ocsVaro(\ocsG;\ocscenh)}
        && \text{by $\ocscen(\ocsG)$, $\ocscena(\ocsG)$, $\ocsceng(\ocsG)$, and  $\ocscenh(\ocsG)$ results}
      \\&\eqd \sum_{x\in\ocsG}\metricsq{\ocscenh(\ocsG)}{x}\psp(x)
        && \text{by definition of $\ocsVaro$ \xref{def:ocsVarG}}
      \\&= \sum_{x\in\ocsG}\metricsq{\setn{3}}{x}\psp(x)
        && \text{by $\ocscenh(\ocsG)$ result}
      \\&= \mathrlap{
           (1)^2\frac{2}{9}+  % x=0
           (2)^2\frac{1}{9}+  % x=1
           (2)^2\frac{1}{9}+  % x=2
           (0)^2\frac{2}{9}+  % x=3
           (1)^2\frac{3}{9}   % x=4
         = \frac{12}{9} = \frac{4}{3} \approx 1.333}
      \\
      \ocsVaro(\ocsG;\ocscenm)
        &\eqd \sum_{x\in\ocsG}\metricsq{\ocscenm(\ocsG)}{x}\psp(x)
        && \text{by definition of $\ocsVaro$ \xref{def:ocsVarG}}
      \\&= \sum_{x\in\ocsG}\metricsq{\setn{0,4}}{x}\psp(x)
        && \text{by $\ocscenm(\ocsG)$ result}
      \\&= \mathrlap{
           (0)^2\frac{2}{9}+  % x=0
           (1)^2\frac{1}{9}+  % x=1
           (1)^2\frac{1}{9}+  % x=2
           (2)^2\frac{2}{9}+  % x=3
           (0)^2\frac{3}{9}   % x=4
         = \frac{10}{9} \approx 1.111}
      \\
      \ocsVaro(\ocsG;\ocscenM)
        &\eqd \sum_{x\in\ocsG}\metricsq{\ocscenM(\ocsG)}{x}\psp(x)
        && \text{by definition of $\ocsVaro$ \xref{def:ocsVarG}}
      \\&= \sum_{x\in\ocsG}\metricsq{\setn{1,2,3}}{x}\psp(x)
        && \text{by $\ocscenm(\ocsG)$ result}
      \\&= \mathrlap{
           (1)^2\frac{2}{9}+  % x=0
           (0)^2\frac{1}{9}+  % x=1
           (0)^2\frac{1}{9}+  % x=2
           (0)^2\frac{2}{9}+  % x=3
           (1)^2\frac{3}{9}   % x=4
         = \frac{5}{9} \approx 0.555}
    \end{align*}
\end{proof}



%\begin{table}%
%  \gsize%
%  \centering%
%  \begin{tabular}{|r|*{11}{c}|c||c|}%
%     \hline
%         & \mc{11}{c|}{moments $\ocsmom(x,y)/36$ {\scs($\max$ values shaded)}}         &  sum   &             \\
%     x/y & 2 & 3 & 4 & 5 & 6 & 7 & 8 & 9 & 10 & 11 & 12 &\scs ($\min$ value shaded) & $\ocsmom_2\brs{\ocsE(\rvX),x}/36$\\
%     \hline
%     2 & 0 & 2 & 3 & 3 & 5 &\scell{12} &10 & 6 & 6  & 6  & 4 & 57 & 4  \\
%     3 & 1 & 0 & 3 & 3 & 5 & 6 &\scell{10} & 6 & 6  & 4  & 3 & 47 & 2  \\
%     4 & 1 & 2 & 0 & 3 & 5 & \scell{6} & 5 & \scell{6} & \scell{6}  & 4  & 2 & 40 & 3  \\
%     5 & 1 & 2 & 3 & 0 & 5 & \scell{6} & 5 & 3 & \scell{6}  & 4  & 2 & 37 & 3  \\
%     6 & 1 & 2 & 3 & 3 & 0 & \scell{6} & 5 & 3 & 3  & 4  & 2 & 32 & 5  \\
%    {7} & 2 & 2 & 3 & 3 & \scell{5} & 0 & \scell{5} & 3 & 3  & 2  & 2 & \scell{30} & 0  \\
%     8 & 2 & 4 & 3 & 3 & 5 & \scell{6} & 0 & 3 & 3  & 2  & 1 & 32 & 5 \\
%     9 & 2 & 4 & \scell{6} & 3 & 5 & \scell{6} & 5 & 0 & 3  & 2  & 1 & 37 & 3 \\
%    10 & 2 & 4 & \scell{6} & \scell{6} & 5 & \scell{6} & 5 & 3 & 0  & 2  & 1 & 40 & 3 \\
%    11 & 3 & 4 & 6 & 6 & \scell{10}& 6 & 5 & 3 & 3  & 0  & 1 & 47 & 2 \\
%    12 & 4 & 6 & 6 & 6 & 10&\scell{12} & 5 & 3 & 3  & 2  & 0 & 57 & 4 \\
%    \hline
%       &   &   &   &   &   &   &   &   &    &    &   &    &$\pVar(\rvX)=\sfrac{34}{36}$\\
%    \hline
%  \end{tabular}%
%  \caption{moments of a pair of fair dice \xref{ex:dicepair_moments}\label{tbl:pairdice_moments}}
%\end{table}


%The physical geometry inducing a stochastic process and the \structe{metric geometry} \xref{rem:mgeo} 
%of the stochastic process itself 
%may be very different, as illustrated in the next two examples.\\
%{\begin{tabular}{c}%
%  \gsize%
%  \centering%
%  %\psset{unit=3mm}%
%  %{%============================================================================
% Daniel J. Greenhoe
% LaTeX file
% ocs archery
%============================================================================
{%\psset{unit=0.5\psunit}%
\begin{pspicture}(-6,-6)(6,6)%
  %---------------------------------
  % options
  %---------------------------------
  \psset{%
    linecolor=blue,%
    %radius=1.25ex,
    %labelsep=2.5mm,
    }%
  \pscircle(0,0){1}%
  \pscircle(0,0){2}%
  \pscircle(0,0){3}%
  \pscircle(0,0){4}%
  \pscircle(0,0){5}%
  \psframe(-6,-6)(6,6)%
  %
  %
  \rput(0,0){$5$}%
  \rput(0,-1.5){$4$}%
  \rput(0,-2.5){$3$}%
  \rput(0,-3.5){$2$}%
  \rput(0,-4.5){$1$}%
  \rput(0,-5.5){$0$}%
\end{pspicture}
}%}%
%  {\includegraphics{sto/graphics/archerytarget.pdf}}%
%\end{tabular}}\hfill%
%%---------------------------------------
%\begin{minipage}{\tw-62mm}
%\begin{example}[\exmd{archery}]
%\label{ex:archery}
%%---------------------------------------
%Consider the archery target illustrated to the left.
%It consists of several concentric circles, each with a different point value.
%However, its' \structe{outcome subspace} structure, has a very different geometry, as illustrated to the right.
%Assuming uniform distribution, the graph center is shaded in the illustration to right.
%\end{example}
%\end{minipage}\hfill%
%{\begin{tabular}{c}%
%  \gsize%
%  \centering%
%  %\psset{unit=5mm}%
%  %{%============================================================================
% Daniel J. Greenhoe
% LaTeX file
% ocs archery
%============================================================================
\begin{pspicture}(-0.5,-0.5)(0.5,5.5)%
  %---------------------------------
  % options
  %---------------------------------
  \psset{%
    linecolor=blue,%
    radius=1.25ex,
    labelsep=2.5mm,
    }%
  \Cnode(0,5){L5}%
  \Cnode(0,4){L4}%
  \Cnode[fillstyle=solid,fillcolor=snode](0,3){L3}%
  \Cnode[fillstyle=solid,fillcolor=snode](0,2){L2}%
  \Cnode(0,1){L1}%
  \Cnode(0,0){L0}%
  %
  \ncline{L4}{L5}%
  \ncline{L3}{L4}%
  \ncline{L2}{L3}%
  \ncline{L1}{L2}%
  \ncline{L0}{L1}%
  %
  \rput(L5){$5$}%
  \rput(L4){$4$}%
  \rput(L3){$3$}%
  \rput(L2){$2$}%
  \rput(L1){$1$}%
  \rput(L0){$0$}%
\end{pspicture}%}%
%  {\includegraphics{sto/graphics/ocsarchery.pdf}}%
%\end{tabular}}
%
%{\begin{tabular}{c}%
%  \gsize%
%  \centering%
%  %\psset{unit=3mm}%
%  %{%============================================================================
% Daniel J. Greenhoe
% LaTeX file
% simplified darts target
%============================================================================
{%\psset{unit=0.5\psunit}%
\begin{pspicture}(-6,-6)(6,6)%
  %---------------------------------
  % options
  %---------------------------------
  \psset{%
    linecolor=blue,%
    %radius=1.25ex,
    %labelsep=2.5mm,
    }%
  \rput[b]{0}{\psline(0,1)(0,5)}%
  \rput[b]{-120}{\psline(0,1)(0,5)}%
  \rput[b]{120}{\psline(0,1)(0,5)}%
  %
  \pscircle(0,0){1}%
  \pscircle(0,0){3}%
  \pscircle(0,0){5}%
  \psframe(-6,-6)(6,6)%
  %
  %
  \rput(-3.464,2){$3$}%
  \rput(3.464,2){$1$}%
  \rput(-1.732,1){$6$}%
  \rput(1.732,1){$4$}%
  \rput(0,0){$7$}%
  \rput(0,-2){$5$}%
  \rput(0,-4){$2$}%
  \rput(5.196,-3){$0$}%
\end{pspicture}
}%}%
%  {\includegraphics{sto/graphics/dartstarget.pdf}}%
%\end{tabular}}\hfill%
%%---------------------------------------
%\begin{minipage}{\tw-70mm}
%\begin{example}[\exmd{darts}]
%\label{ex:darts}
%%---------------------------------------
%Consider the simplified dart board illustrated to the left
%and \structe{outcome subspace} illustrated to the right.
%%It consists of several concentric circles, each with a different point value.
%Unlike the archery \structe{outcome subspace} \xref{ex:archery}, the outcome subspace is \prope{non-linear}.
%%Its' \structe{outcome subspace} strurture, is has a very different geometry, as illustrated to the right.
%Assuming uniform distribution, the graph center is shaded in the illustration.
%\end{example}
%\end{minipage}\hfill%
%{\begin{tabular}{c}%
%  \gsize%
%  \psset{unit=7mm}%
%  \centering%
%  %{%============================================================================
% Daniel J. Greenhoe
% LaTeX file
% ocs darts
%============================================================================
{%\psset{unit=0.5\psunit}%
\begin{pspicture}(-1.3,-0.3)(1.3,5.3)%
  %---------------------------------
  % options
  %---------------------------------
  \psset{%
    linecolor=blue,%
    radius=1.25ex,
    labelsep=2.5mm,
    }%
  \Cnode(0,5){D7}%
  \Cnode[fillstyle=solid,fillcolor=snode](-1,4){D6}%
  \Cnode[fillstyle=solid,fillcolor=snode]( 1,4){D4}%
  \Cnode[fillstyle=solid,fillcolor=snode]( 0,3){D5}%
  \Cnode[fillstyle=solid,fillcolor=snode](-1,2){D3}%
  \Cnode[fillstyle=solid,fillcolor=snode]( 0,1){D2}%
  \Cnode[fillstyle=solid,fillcolor=snode]( 1,2){D1}%
  \Cnode(0,0){D0}%
  %
  \ncline{D4}{D5}\ncline{D5}{D6}\ncline{D6}{D4}%
  \ncline{D1}{D2}\ncline{D2}{D3}\ncline{D3}{D1}%
  \ncline{D7}{D4}\ncline{D7}{D5}\ncline{D7}{D6}%
  \ncline{D3}{D6}\ncline{D2}{D5}\ncline{D1}{D4}%
  \ncline{D0}{D1}\ncline{D0}{D2}\ncline{D0}{D3}%
  %
  \rput(D7){$7$}%
  \rput(D6){$6$}%
  \rput(D5){$5$}%
  \rput(D4){$4$}%
  \rput(D3){$3$}%
  \rput(D2){$2$}%
  \rput(D1){$1$}%
  \rput(D0){$0$}%
\end{pspicture}
}%}%
%  {\includegraphics{sto/graphics/ocsdarts.pdf}}%
%\end{tabular}}
%
%
%---------------------------------------
\begin{example}[\exmd{DNA}]
\label{ex:dna}
%---------------------------------------
\structe{Genomic Signal Processing} (\structe{GSP}) analyzes biological sequences called \structe{genome}s.
These sequences are constructed over a set of 4 symbols that are commonly referred to as 
$\symA$, $\symT$, $\symC$, and $\symG$,
each of which corresponds to a nucleobase (adenine,  thymine, cytosine, and guanine, 
respectively).\footnote{
  \citePc{mendel1853e}{Mendel (1853): gene coding uses discrete symbols},
  \citePpc{watson1953}{737}{Watson and Crick (1953): gene coding symbols are adenine,  thymine, cytosine, and guanine},
  \citePp{watson1953may}{965},
  \citerpg{pommerville2013}{52}{1449647960}
  }
A typical genome sequence contains a large number of symbols 
(about 3 billion for humans, 29751 for the SARS virus).%
\footnote{%
  \citeWuc{genbank}{http://www.ncbi.nlm.nih.gov/genome/guide/human/}{Homo sapiens, NC\_000001--NC\_000022 (22 chromosome pairs), NC\_000023 (X chromosome), NC\_000024 (Y chromosome), NC\_012920 (mitochondria)},
  \citeWuc{genbank}{http://www.ncbi.nlm.nih.gov/nuccore/30271926}{SARS coronavirus, NC\_004718.3}
  \citePc{gregory2006}{homo sapien chromosome 1},
  \citePc{he2004}{SARS coronavirus}
  }
\\[0.3ex]\begin{minipage}{\tw-37mm}%
Let $\ocsG\eqd\ocs{\setn{\symA,\symT,\symC,\symG}}{\metricn}{\orel}{\psp}$ 
be the \structe{outcome subspace} \xref{def:ocsm} generated by a \structe{genome}
where $\metricn$ is the \fncte{discrete metric} \xref{def:dmetric},
$\orel\eqd\emptyset$ (completely unordered set), and 
$\psp(\symA)=\psp(\symT)=\psp(\symC)=\psp(\symG)=\frac{1}{4}$.
This space is illustrated by the \structe{graph} \xref{def:graph} to the right
with shaded \structe{center} \xref{def:ocscen}.
\end{minipage}%
\hfill%
{\begin{tabular}{c}%
  \gsize%
  \psset{unit=8mm}%
  \centering%
  %{%============================================================================
% Daniel J. Greenhoe
% LaTeX file
%============================================================================
{%\psset{unit=0.5\psunit}%
\begin{pspicture}(-1.4,-1.4)(1.4,1.4)%
  %---------------------------------
  % options
  %---------------------------------
  \psset{%
    linecolor=blue,%
    radius=1.25ex,
    labelsep=2.5mm,
    }%
  %---------------------------------
  % DNA graph
  %---------------------------------
  \rput(0,0){%
    \uput{1}[135](0,0){\Cnode[fillstyle=solid,fillcolor=snode](0,0){Da}}%
    \uput{1}[ 45](0,0){\Cnode[fillstyle=solid,fillcolor=snode](0,0){Dt}}%
    \uput{1}[225](0,0){\Cnode[fillstyle=solid,fillcolor=snode](0,0){Dc}}%
    \uput{1}[-45](0,0){\Cnode[fillstyle=solid,fillcolor=snode](0,0){Dg}}%
    %\Cnode[fillstyle=solid,fillcolor=snode](-1, 1){Da}%
    %\Cnode[fillstyle=solid,fillcolor=snode]( 1, 1){Dt}%
    %\Cnode[fillstyle=solid,fillcolor=snode](-1,-1){Dc}%
    %\Cnode[fillstyle=solid,fillcolor=snode]( 1,-1){Dg}%
    %\Cnode[fillstyle=solid,fillcolor=snode](-0.707, 0.707){Da}%
    %\Cnode[fillstyle=solid,fillcolor=snode]( 0.707, 0.707){Dt}%
    %\Cnode[fillstyle=solid,fillcolor=snode](-0.707,-0.707){Dc}%
    %\Cnode[fillstyle=solid,fillcolor=snode]( 0.707,-0.707){Dg}%
    }%
  \ncline{Dc}{Dg}%
  \ncline{Dt}{Dc}\ncline{Dt}{Dg}%
  \ncline{Da}{Dt}\ncline{Da}{Dc}\ncline{Da}{Dg}%
  \rput(Dg){$\symG$}%
  \rput(Dc){$\symC$}%
  \rput(Dt){$\symT$}%
  \rput(Da){$\symA$}%
  %
  \uput[-45](Dg){$\frac{1}{4}$}
  \uput[-135](Dc){$\frac{1}{4}$}
  \uput[45](Dt){$\frac{1}{4}$}
  \uput[135](Da){$\frac{1}{4}$}
\end{pspicture}
}%}%
  {\includegraphics{sto/graphics/dna.pdf}}%
\end{tabular}}
\\
The graph has the following geometric values:
  \\\indentx$\begin{array}{rclDD}
    \ocscen(\ocsG)  &=& \setn{\symA,\symT,\symC,\symG} & \xref{def:ocscen}  & (shaded in illustration)\\
    \ocscena(\ocsG) &=& \setn{\symA,\symT,\symC,\symG} & \xref{def:ocscenx} & (shaded in illustration)\\
    \ocsVaro(\ocsG)  &=& 0                              & \xref{def:ocsVarG}     
  \end{array}$
\end{example}
\begin{proof}
\begin{align*}
  \ocscen(\ocsG)
    &\eqd \argmin_{x\in\ocsG}\max_{y\in\ocsG}\metric{x}{y}\psp(y)
    && \text{by definition of $\ocscen$ \xref{def:ocscen}}
  \\&= \argmin_{x\in\ocsG}\max_{y\in\ocsG}\metric{x}{y}\frac{1}{4}
    && \text{by definition of $\ocsG$}
  \\&= \argmin_{x\in\ocsG}\max_{y\in\ocsG}\metric{x}{y}
    && \text{because $\ff(x)=\frac{1}{4}x$ is \prope{strictly isotone} and by \prefpp{lem:argminmaxphi}}
  \\&= \argmin_{x\in\ocsG}\setn{1,\,1,\,1,\,1}
    && \text{because for the \fncte{discrete metric} \xref{def:dmetric}, $\max\metricn=1$}
  \\&= \setn{\symA,\symT,\symC,\symG}
    && \text{by definition of $\ocsG$}
  \\
  \ocscena(\ocsG)
    &\eqd \argmin_{x\in\ocsG}\sum_{y\in\ocsG}\metric{x}{y}\psp(y)
    &&\text{by definition of $\ocscena$ \xref{def:ocscenx}}
  \\&= \argmin_{x\in\ocsG}\sum_{y\in\ocsG}\metric{x}{y}\frac{1}{6}
    &&\text{by definition of $\ocsG$}
  \\&= \argmin_{x\in\ocsG}\sum_{y\in\ocsG}\metric{x}{y}
    &&\text{because $\ff(x)=\frac{1}{4}x$ is \prope{strictly isotone} and by \prefpp{lem:argminmaxphi}}
  \\&= \mathrlap{\argmin_{x\in\ocsG}
         \setn{\begin{array}{*{11}{@{\hspace{2pt}}c}}
           0 &+& 1 &+& 1 &+& 1\\
           1 &+& 0 &+& 1 &+& 1\\
           1 &+& 1 &+& 0 &+& 1\\
           1 &+& 1 &+& 1 &+& 0
         \end{array}}
       = \argmin_{x\in\ocsG}
         \setn{\begin{array}{c}
           3\\
           3\\
           3\\
           3  
         \end{array}}
     = \setn{\symA,\symT,\symC,\symG}}
    \\
  \ocsVaro(\ocsG)
    &\eqd \sum_{x\in\ocsG}\brs{\metric{\ocscen(\ocsG)}{x}}^2\psp(x)
    && \text{by definition of $\ocsVaro$ \xref{def:ocsVarG}}
  \\&= \sum_{x\in\ocsG}(0)\frac{1}{6}
    && \text{because $\ocscen(\ocsG)=\ocsG$}
  \\&= 0
    && \text{by field property of \vale{additive identity element} $0$}
\end{align*}
\end{proof}


%
%%---------------------------------------
%\begin{example}[\exmd{DQPSK}]
%\footnote{
%  \citer{wesolowski2009}
%  }
%\label{ex:dqpsk}
%%---------------------------------------
%In digital communications, there are several modulation techniques available.
%Most of these manipulate (``modulate") the parameters a sinusoidal signal (called the \structe{carrier})
%at the transmitter to ``carry" information (such as a person's voice) to a receiver
%where under certain reasonable conditions the information can be recovered 
%with an acceptably low error rate (e.g. $\le0.0001$\%).
%Parameters of the sinusoid that may be manipulated include the sinusoid's amplitude
%(\hie{Amplitude Shift Keying} or \hie{ASK}), frequency (\hie{Frequency Shift Keying} or \hie{FSK}),
%or phase (\hie{Phase Shift Keying} or \hie{PSK}).
%The information to be carried is first encoded into a sequence of ``symbols", 
%with each symbol carrying $\xN$ bits (typically $\xN=$ 1, 2, or 3).
%All of modulation techniques generate a code space with $2^\xN$ code points.
%In each modulation techique, the receiver must somehow have a reference by which it can recover the 
%information from the carrier.
%A receiver may easily generate an amplitude reference for ASK modulation by using a 
%simple low pass filter to find the 0 Hertz component of the received signal.
%A receiver may generate a frequency reference for FSK modulation by using an oscillator circuit
%that oscillates at the same frequency as the unmodulated carrier.
%For PSK, generating a reference cannot be done without assistance from the transmitter. 
%The transmitter may provide this assistance by also transmitting a reference sinusoid (one that does not 
%change it's phase), or by encoding the reference signal into the information sequence itself.
%\\[0.3ex]\begin{minipage}{\tw-37mm}%
%One way to do the latter using $\xN=2$ bit encoding is a modulation technique called 
%\hie{Differential Quadrature Phase Shift Keying} or \hie{DQPSK}.
%In DQPSK, consecutive code points at the transmitter cannot change by more than 1 bit.
%In the illustration, this means that they cannot ``jump" across the square to the opposite corner. 
%That is, each symbol is partly a function of the previous symbol.
%By doing so, the phase information can be encoded into the sequence.
%\end{minipage}%
%\hfill%
%{\begin{tabular}{c}%
%  \gsize%
%  \psset{unit=10mm}%
%  \centering%
%  %{%============================================================================
% D11niel J. Greenhoe
% LaTeX file
% lattice M2 on M2
%============================================================================
{%\psset{unit=0.5\psunit}%
\begin{pspicture}(-1.5,-1.5)(1.5,1.5)%
  %---------------------------------
  % options
  %---------------------------------
  \psset{%
    linecolor=blue,%
    radius=1.35ex,
    labelsep=2.5mm,
    }%
  %---------------------------------
  % DNA graph
  %---------------------------------
  \rput(0,0){%
    %\psaxes[linecolor=axis,linewidth=0.5pt]{<->}(0,0)(-1.5,1.5)(-1.5,1.5)% x axis
    \Cnode(-1, 1){D11}%
    \Cnode( 1, 1){D01}%
    \Cnode(-1,-1){D10}%
    \Cnode( 1,-1){D00}%

    }%
  \ncline{D10}{D00}\nbput[labelsep=0pt]{${\scy\metric{00}{10}=}1$}%%
  \ncline{D10}{D01}\naput[labelsep=0pt,nrot=:U,npos=0.25]{${\scy\metric{01}{10}=}2$}%
  \ncline{D00}{D01}\nbput[labelsep=0pt,nrot=:U]{${\scy\metric{00}{01}=}1$}%
  \ncline{D11}{D01}\naput[labelsep=0pt]{${\scy\metric{01}{11}=}1$}%
  \ncline{D10}{D11}\naput[labelsep=0pt,nrot=:U]{${\scy\metric{10}{11}=}1$}%
  \ncline{D11}{D00}\naput[labelsep=0pt,nrot=:U]{${\scy\metric{00}{11}=}2$}%
  %
  \rput(D00){$00$}%
  \rput(D10){$10$}%
  \rput(D01){$01$}%
  \rput(D11){$11$}%
  %
  %\uput[-45](D00){$\frac{1}{4}$}
  %\uput[-135](D10){$\frac{1}{4}$}
  %\uput[45](D01){$\frac{1}{4}$}
  %\uput[135](D11){$\frac{1}{4}$}
\end{pspicture}
}%}%
%  {\includegraphics{sto/graphics/dqpsk.pdf}}%
%\end{tabular}}
%\end{example}
%






%\begin{figure}
%  \gsize%
%  \centering%
%  \psset{unit=10mm}%
%  {%============================================================================
% Daniel J. Greenhoe
% LaTeX file
% spinner 6 mapping to linearly ordered L6
%============================================================================
{\psset{yunit=1.25\psunit}%
\begin{pspicture}(-6.5,-0.5)(6.5,3.5)%
  %---------------------------------
  % options
  %---------------------------------
  \psset{%
    linecolor=blue,%
    cornersize=relative,
    framearc=0.25,
    subgriddiv=1,
    gridlabels=4pt,
    gridwidth=0.2pt,
    }%
  %---------------------------------
  % states
  %---------------------------------
     \begin{tabstr}{0.75}
     \rput( 5,1.5){\rnode{Sdeposit} {\psframebox{\begin{tabular}{c}deposit coins\\in vault\end{tabular}}}}%
     \rput(-5,1.5){\rnode{Sreturn}  {\psframebox{\begin{tabular}{c}return coins          \end{tabular}}}}%
     \rput( 0,3)  {\rnode{Seject}   {\psframebox{\begin{tabular}{c}deliver product       \end{tabular}}}}%
     \rput( 0,2)  {\rnode{Scheck}   {\psframebox{\begin{tabular}{c}check availability    \end{tabular}}}}%
     \rput( 0,1)  {\rnode{Swait}    {\psframebox{\begin{tabular}{c}wait for selection    \end{tabular}}}}%
     \rput( 0,0)  {\rnode{Sidle}    {\psframebox{\begin{tabular}{c}idle                  \end{tabular}}}}%
     \end{tabstr}
  %\Cnode(3,1.5) {Sdeposit}%
  %\Cnode(-3,1.5){Sreturn}%
  %\Cnode(0,3)   {Seject}%
  %\Cnode(0,2)   {Scheck}%
  %\Cnode(0,1)   {Swait}%
  %\Cnode(0,0)   {Sidle}%
  %
  %\uput[  90](Seject)  {deliver product}%
  %\uput[   0](Sdeposit){deposit coins in vault}%
  %\uput[ -90](Sreturn) {return coins}%
  %\uput[ 180](Scheck)  {check availability}%
  %\uput[ 180](Swait)   {wait for selection}%
  %\uput[ -90](Sidle)   {idle}%
  %---------------------------------
  % edges
  %---------------------------------
  {\psset{labelsep=1pt}%
  \ncline{->}{Sreturn}{Sidle}\nbput[nrot=:U]{coins returned}%
  \ncline{<-}{Sreturn}{Seject}\naput[nrot=:U]{jammed}%
  \ncline{->}{Seject}{Sdeposit}\naput[nrot=:U]{delivery successful}%
  \ncline{<-}{Sidle}{Sdeposit}\nbput[nrot=:U]{deposit successful}%
  \ncline{<-}{Sreturn}{Scheck}\nbput[nrot=:U]{not available}%
  \ncline{->}{Scheck}{Seject}\ncput{available}%
  \ncline{->}{Swait}{Scheck}\ncput{selection made}%
  \ncline{->}{Sidle}{Swait}\ncput{coins inserted}%
  }%
\end{pspicture}
}%}%
%  \caption{state machine for vending machine \xref{ex:vending}\label{fig:vending}}
%\end{figure}
%%---------------------------------------
%\begin{example}[\exmd{state machine}]
%\label{ex:vending}
%%---------------------------------------
%A simple state machine for a vending machine is illustrated with a directed graph in \prefpp{fig:vending}.
%At any given time, the state machine is in exactly one of the 6 states. 
%The sequence of states it is in at sampled time intervals is a random process.
%Again, the order and topology of this structure is very dissimilar to that of the real line mapped to by 
%the traditional random variable.
%\end{example}
%
%%\begin{figure}
%%  \gsize%
%%  \centering%
%%  \psset{unit=5mm}%
%%  {%============================================================================
% Daniel J. Greenhoe
% LaTeX file
% "Fishing in Sierpinski's Sea" illustration
%============================================================================
{\psset{xunit=1.5\psunit}%
\begin{pspicture}(-1.1,-0.5)(1.1,4.5)%
  %---------------------------------
  % options
  %---------------------------------
  \psset{%
    linecolor=blue,%
    %cornersize=relative,
    %framearc=0.25,
    %subgriddiv=1,
    %gridlabels=4pt,
    %gridwidth=0.2pt,
    }%
  %---------------------------------
  % states
  %---------------------------------
  \Cnode(1,2.66){H}
  \Cnode(1,1.33){G}
  \Cnode(-1,2)  {F}
  \Cnode(0,4)   {B}
  \Cnode(0,3)   {E}
  \Cnode(0,2)   {D}
  \Cnode(0,1)   {C}
  \Cnode(0,0)   {A}
  %
  \uput[ 0](B){$B$}
  \uput[ 0](A){$A$}
  %---------------------------------
  % edges
  %---------------------------------
  \ncline{G}{D}%
  \ncline{H}{D}\ncline{H}{C}\ncline{H}{G}%
  \ncline{E}{B}\ncline{B}{F}\ncline{B}{H}%
  \ncline{D}{E}%
  \ncline{C}{D}%
  \ncline{A}{C}\ncline{A}{F}\ncline{A}{G}%
\end{pspicture}
}%}%
%%  \caption{state machine for vending machine \xref{ex:network}\label{fig:network}}
%%\end{figure}
%\begin{minipage}{\tw-25mm}%
%%---------------------------------------
%\begin{example}[\exmd{network}]
%\label{ex:vending}
%%---------------------------------------
%A simple network is illustrated with an undirected graph to the right. %in \prefpp{fig:network}.
%Suppose a packet is sent from $A$ to $B$. 
%%At any given time, the packet is in exactly one of the 8 servers. 
%The sequence of servers the packet transverses through is a random process.
%Again, the order and topology of this structure is very dissimilar to that of the real line mapped to by 
%the traditional random variable.
%\end{example}
%\end{minipage}\hfill%
%\begin{tabular}{c}
%  \gsize%
%  \centering%
%  \psset{unit=5mm}%
%  {%============================================================================
% Daniel J. Greenhoe
% LaTeX file
% "Fishing in Sierpinski's Sea" illustration
%============================================================================
{\psset{xunit=1.5\psunit}%
\begin{pspicture}(-1.1,-0.5)(1.1,4.5)%
  %---------------------------------
  % options
  %---------------------------------
  \psset{%
    linecolor=blue,%
    %cornersize=relative,
    %framearc=0.25,
    %subgriddiv=1,
    %gridlabels=4pt,
    %gridwidth=0.2pt,
    }%
  %---------------------------------
  % states
  %---------------------------------
  \Cnode(1,2.66){H}
  \Cnode(1,1.33){G}
  \Cnode(-1,2)  {F}
  \Cnode(0,4)   {B}
  \Cnode(0,3)   {E}
  \Cnode(0,2)   {D}
  \Cnode(0,1)   {C}
  \Cnode(0,0)   {A}
  %
  \uput[ 0](B){$B$}
  \uput[ 0](A){$A$}
  %---------------------------------
  % edges
  %---------------------------------
  \ncline{G}{D}%
  \ncline{H}{D}\ncline{H}{C}\ncline{H}{G}%
  \ncline{E}{B}\ncline{B}{F}\ncline{B}{H}%
  \ncline{D}{E}%
  \ncline{C}{D}%
  \ncline{A}{C}\ncline{A}{F}\ncline{A}{G}%
\end{pspicture}
}%}%
%\end{tabular}
%
%%---------------------------------------
%\begin{example}[\exmd{Fishing in Sierpi/'nski's Sea}]
%\label{ex:ssea}
%%---------------------------------------
%Suppose there is a sea, called ``Sierpi/'nski's Sea", with only two fish---one blue fish and one red fish.
%This is a very special sea with very special fish.
%You throw a net into Sierpi/'nski's Sea.
%In this sea, there are only three possibilities of what you may find when you pull the net back in:
%\\\indentx$\begin{array}{clM}
%  \circOne   & \emptyset:  & the net is empty\\
%  \circTwo   & \setn{b}:   & the blue fish is in the net\\
%  \circThree & \setn{b,r}: & both the blue and red fish are in the net
%\end{array}$\\
%However, in this very special sea of very special fish, the blue fish is a very loyal friend to the 
%red fish and so he will always rescue the red fish from your net, or be caught together with the red fish attempting 
%to do so. Thus, it is not possible that only the red fish is in the net.
%The set of three possible sets $\topT\eqd\setn{\emptyset,\,\setn{b},\,\setn{b,r}}$ form a topology on $\setX\eqd\setn{b,r}$
%that is \prope{non-metrizable}. 
%The \structe{topological space} $\opair{\setX}{\topT}$ is called \structe{Sierpi/'nski's space}.
%\\\begin{minipage}{\tw-25mm}%
%Suppose the probabilities of the three outcomes are 
%$\psp(\emptyset)=\frac{5}{8}$, $\psp(\setn{b})=\frac{1}{8}$, and $\psp(\setn{b,r})=\frac{1}{4}$.
%The structure $\ocs{\setX}{\topT}{\subseteq}{\psp}$ is an \structe{outcome subspace} \xref{def:ocs}.
%The set $\setX$ is ordered using the subset relation $\subseteq$, producing a 
%\structe{directed graph} \xref{def:graph} that is also a \structe{linearly ordered lattice} \xref{def:chain},
%as illustrated to the right.
%\end{minipage}\hfill%
%\begin{tabular}{c}
%  \gsize%
%  \centering%
%  \psset{unit=5mm}%
%  {%============================================================================
% Daniel J. Greenhoe
% LaTeX file
% "Fishing in Sierpinski's Sea" illustration
%============================================================================
\begin{pspicture}(-2,-0.5)(2,2.5)%
  %---------------------------------
  % options
  %---------------------------------
  \psset{%
    linecolor=blue,%
    %cornersize=relative,
    %framearc=0.25,
    %subgriddiv=1,
    %gridlabels=4pt,
    %gridwidth=0.2pt,
    }%
  %---------------------------------
  % states
  %---------------------------------
  \Cnode(0,2)   {BR}
  \Cnode(0,1)   {B}
  \Cnode(0,0)   {E}
  %
  \uput[180](BR){$\setn{b,r}$}
  \uput[180](B){$\setn{b}$}
  \uput[180](E){$\emptyset$}
  %
  \uput[0](BR){${\scy\psp(\setn{b,r})=\frac{1}{4}}$}
  \uput[0](B) {${\scy\psp(\setn{b})=\frac{1}{8}}$}
  \uput[0](E) {${\scy\psp(\emptyset)=\frac{5}{8}}$}
  %---------------------------------
  % edges
  %---------------------------------
  \ncline{B}{BR}%
  \ncline{E}{B}%
\end{pspicture}%}%
%\end{tabular}
%\end{example}
%
%
%%---------------------------------------
%\begin{example}[\exmd{program termination}]
%\label{ex:csob}
%%---------------------------------------
%For a certain computer program, define the following:\footnote{
%  \citeI{smyth1992},
%  \citePp{schroder2006}{605}
%  }
%\\\indentx$\begin{array}{cM}
%  \setn{\downtack}:         & the program terminates after a finite time\\
%  \setn{\uptack}:           & the program does \emph{not} terminate in a finite time\\
%  \setn{\downtack,\uptack}: & the program has been executed\\
%  \emptyset:                & the program has not been executed
%\end{array}$\\
%That is, $\setn{\uptack}$ indicates a program is ``stuck" in an ``infinite loop".
%The topology $\topT\eqd\setn{\emptyset,\setn{\downtack},\setn{\downtack,\uptack}}$ 
%on the set $\setX\eqd\setn{\downtack,\uptack}$ is a \structe{Sierpi/'nsk Space} \xref{ex:ssea}.
%The \structe{open sets} \xref{def:openset} of $\topT$ are all \prope{observable}.
%However, $\setn{\uptack}$ is \emph{not} observable.
%\end{example}

%\fi

\end{tabstr}
