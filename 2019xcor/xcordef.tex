%============================================================================
% LaTeX File
% Daniel J. Greenhoe
%============================================================================
%======================================
\section{Definitions}
%======================================
%In the literature, there are varying and in general incompatible definitions of
%\fncte{cross-correlation} functions $\Rxy(n,m)$ and $\Rxy(m)$ \xxref{def:Rxynm}{def:Rxym}.
%The choice of definitions has consequences for results involving $\Swxy$ \xref{def:Swxy}.
%\begin{definition}
Here is a very limited overview of the definitions of $\Rxy(m)$:
\\\defbox{\begin{array}{lM>{\ds}rc>{\ds}l}
  \mc{5}{M}{References that put conjugate $\conj$ on $\rvy$:}
  \\&\citerpg{papoulis1984}{263}{0070484686} & R_{xy}(m) &\eqd& E\brb{\rvx(m)\rvy^\ast(0)}
  \\&\citerpg{cadzow}{341}{0023180102}       & r_{xy}(m) &\eqd& E\brs{\rvx(m)\rvy^\ast(0)}
  \\&\citeP{matlab_xcorr}                    & R_{xy}(m) &\eqd& E\setn{x_{n+m}y_n^\ast}
  \\&\citeP{matlab_cpsd}                     & R_{xy}(m) &\eqd& E\setn{x_{n+m}y_n^\ast}
  \\
  \mc{5}{M}{References that put conjugate $\conj$ on $\rvx$:}
  \\&\citerpg{kay1988}{52}{8131733564}                    & r_{xy}[m] &\eqd& \mathcal{E}\brb{x^\ast[0]y[m]}
  \\&\citerpg{weisstein2002}{594}{1420035223}\footnotemark& f\star g  &\eqd& \int_{-\infty}^{\infty}\bar{f}(\tau)g(t+\tau)\dtau
  \\&\citePpc{leuridan1986}{2}{(7)}                       & GXY_1     &\eqd& \sum X_1^\ast Y
  \\
  \mc{5}{M}{References that use no conjugate $\conj$:}
  \\&\citerpg{bendat2010}{111}{1118210824}          & R_{xy}(m)            &\eqd& E\brs{\rvx(0)\rvy(m)}
  \\&\citerpg{helstrom1991}{369}{0023535717}        & \Rxy(t_1,t_2)        &\eqd& E[\rvx(t_1)\rvy(t_2)]
  \\&\citerpg{proakis1996}{A4}{0133737624}          & \gamma_{xy}(t_1,t_2) &\eqd& E(X_{t_1}Y_{t_2})
  \\&\citerpg{shin2008}{280}{0470725648}            & R_{xy}(\tau)         &\eqd& E[x(t)y(t+\tau)]
  \\&\citerpg{bracewell1978}{46}{007007013X}\footnotemark & g^\ast\star h  &\eqd& \int_{-\infty}^{\infty}g^\ast(u)h(u+x)\du   %{Pentagram notation for cross correlation}
\end{array}}
\addtocounter{footnote}{-1}
\footnotetext{
  Bracewell and Weisstein here use the \ope{integral operator} $\int_{\R}\!\dx$ rather than the 
  \ope{expectation operator} $\pE$. 
  That is, they use a \ope{time average} rather than an \ope{ensemble average}. 
  But in essence, the two types of operators are ``the same" because both types represent
  \ope{inner product}s. 
  That is, $\int_{x\in\R}\ff(x)\fg^\ast(x)\dx\eqd\inprod{\ff(x)}{\fg(x)}_1$ and
  $\pE\brs{\rvx(t)\rvy^\ast(t)}\eqd\inprod{\rvx(t)}{\rvy(t)}_2$
  (both are inner products, but operate in perpendicular orientations across the ensemble plane).
  }
\stepcounter{footnote}
\footnotetext{
  Note that Bracewell's ``\ope{Pentagram notation for cross correlation}"
  $g^\ast\star h =\int_{-\infty}^{\infty}g^\ast(u)h(u+x)\du$ 
  implies
  $g\star h =\int_{-\infty}^{\infty}g(u)h(u+x)\du$ 
  (and hence in the ``References that use no conjugate" category).
  }
%\end{definition}

%=======================================
\section{Results}
%=======================================
There are a total of eight different ways of defining the cross-correlation $\Rxy(m)$ of
\prope{complex-valued} \prope{wide-sense stationary} sequences $\rvx(n)$ and $\rvy(n)$.
There are eight because each sequence may be defined with or without the conjugate operator $\ast$,
and $\rvx$ may lead or lag $\rvy$
($2\times2\times2=8$).
Depending on which one of the eight definitions of $\Rxy(m)$ we select:
\\\indentx$\begin{array}{FMM}
  \imark & We get four different results for $\Szxy(z)$ & \xref{prop:RxySzxy}\\
  \imark & We get four different results for $\Swxy(\omega)$ & \xref{prop:RxySwxy}\\
  \imark & If $\rvx$ and $\rvy$ are \prope{real-valued}, we get two different results for $\Swxy(\omega)$ & \xref{prop:RxySwxy_real}
\end{array}$

%---------------------------------------
\begin{proposition}
\label{prop:RxySzxy}
%---------------------------------------
\propbox{\begin{array}{FM   >{\ds}r  c     l *{3}{@{\hspace{0pt}}l}          c         >{\ds}r   c l@{\hspace{0pt}}l@{\hspace{0pt}}l}
    (1).&Papoulis:        & \Rxy(m) &\eqd& \pE[\rvx     &(m)\rvy^\ast&(0&)] &\implies& \Szxy(z) &=&\ZH^\ast&\brp{\frac{1}{z^\ast}} &\Szxx^\ast\brp{\frac{1}{z^\ast}}
  \\(2).&Kay:             & \Rxy(m) &\eqd& \pE[\rvx^\ast&(0)\rvy     &(m&)] &\implies& \Szxy(z) &=&\ZH     &(z)                    &\Szxx     (z)
  \\(3).&y-star-m:        & \Rxy(m) &\eqd& \pE[\rvx     &(0)\rvy^\ast&(m&)] &\implies& \Szxy(z) &=&\ZH^\ast&\brp{z^\ast}           &\Szxx^\ast\brp{z^\ast}
  \\(4).&x-star-m:        & \Rxy(m) &\eqd& \pE[\rvx^\ast&(m)\rvy     &(0&)] &\implies& \Szxy(z) &=&\ZH     &\brp{\frac{1}{z}}      &\Szxx(z)
  \\(5).&Bendat:          & \Rxy(m) &\eqd& \pE[\rvx     &(0)\rvy     &(m&)] &\implies& \Szxy(z) &=&\ZH     &(z)                    &\Szxx(z)
  \\(6).&alt-Bendat:      & \Rxy(m) &\eqd& \pE[\rvx     &(m)\rvy     &(0&)] &\implies& \Szxy(z) &=&\ZH     &\brp{\frac{1}{z}}      &\Szxx^\ast\brp{z^\ast}
  \\(7).&Bendat-star      & \Rxy(m) &\eqd& \pE[\rvx^\ast&(0)\rvy^\ast&(m&)] &\implies& \Szxy(z) &=&\ZH^\ast&\brp{z^\ast}           &\Szxx(z)
  \\(8).&alt-Bendat-star: & \Rxy(m) &\eqd& \pE[\rvx^\ast&(m)\rvy^\ast&(0&)] &\implies& \Szxy(z) &=&\ZH     &\brp{\frac{1}{z}}      &\Szxx(z)
\end{array}}
\end{proposition}
\begin{proof}
\begin{align*}
\intertext{1. If we follow Papoulis $\brp{\Rxy(m)\eqd\pE\brs{\rvx(m)\rvy^\ast(0)}}$, then\ldots}
    \Szxy(z)
      &\eqd \opZ\pE\Rxy(m)
      && \text{by definition of $\Szxy(\omega)$}
      && \text{\xref{def:Szxy}}
    \\&\eqd \opZ\pE\brs{\rvx(m)\rvy^\ast(0)}
      && \text{by Papoulis' definition of $\Rxy(m)$}
    \\&=    \opZ\pE\brs{\rvx(m)\brp{\sum_{k\in\Z} \fh(k)\rvx(0-k)}^\ast}
      && \text{by \prope{LTI} property}
    \\&=    \opZ\pE\brs{\rvx(m) \sum_{k\in\Z} \fh^\ast(k)      \rvx^\ast(0-k)}
      && \text{by \prope{distributive} property of \structd{$\invo$-algebra}s}
      && \text{\xref{def:staralg}}
    \\&=    \opZ        \sum_{k\in\Z} \fh^\ast(k) \pE\brs{\rvx(m)\rvx^\ast(0-k)}
      && \text{by \prope{linearity} of $\opE$}
      && \text{\xref{thm:pE_linop}}
    \\&=    \opZ        \sum_{k\in\Z} \fh^\ast(k) \pE\brs{\rvx(0)\rvx^\ast(-m-k)}
      &&    \text{by \prope{wide sense stationary} property}
    \\&=    \opZ        \sum_{k\in\Z} \fh^\ast(k) \brp{\pE\brs{\rvx(-m-k)\rvx^\ast(0)}}^\ast
      && \text{by \prope{distributive} property of \structd{$\invo$-algebra}s}
      && \text{\xref{def:staralg}}
    \\&\eqd \opZ        \sum_{k\in\Z} \fh^\ast(k) \Rxx^\ast(-m-k)
      && \text{by this definition of $\Rxy(m)$}
    \\&\eqd \opZ\brs{\fh^\ast(-m) \convd \Rxx^\ast(-m)}
      && \text{by definition of \ope{convolution}}
      && \text{\xref{def:convd}}
    \\&=    \brs{\opZ\fh^\ast(-m)} \brs{\opZ\Rxx^\ast(-m)}
      && \text{by \thme{convolution theorem}}
      && \text{\xref{thm:conv}}
    \\&= \boxed{\ZH^\ast\brp{\frac{1}{z^\ast}} \Szxx^\ast\brp{\frac{1}{z^\ast}}}
      && \text{by property of the \ope{Z-transform}}
      && \text{by \xref{thm:opZ}}
\end{align*}

\begin{align*}
    2.\quad\Szxy(z)
      &\eqd \opZ\Rxy(m)
     &&\eqd \opZ\pE\brs{\rvy(m)\rvx^\ast(0)}
     &&=    \opZ\pE\brs{\brp{\fh(m)\convd\rvx(m)}\rvx^\ast(0)}
    \\&=    \opZ\pE\brs{\brp{\sum_{k\in\Z} \ZH(k)\rvx(m-k)}\rvx^\ast(0)}
     &&=    \opZ\brp{\sum_{k\in\Z} \ZH(k)\pE\brs{\rvx(m-k)\rvx^\ast(0)}}
     &&\eqd \opZ\brp{\sum_{k\in\Z} \ZH(k)\Rxx(m-k)}
    \\&\eqd \opZ\brp{\fh(m)\convd\Rxx(m)}
     &&=    \brs{\opZ\fh(m)}\brs{\opZ\Rxx(m)}
     &&\eqd \boxed{\ZH(z) \Szxx(z)}
    \\
    3.\quad\Szxy(z)
      &\eqd \opZ\pE\Rxy(m)
     &&\eqd \opZ\pE\brs{\rvx(0)\rvy^\ast(m)}
       \mathrlap{\qquad=    \opZ\pE\brs{\rvx(0)\brp{\sum_{k\in\Z} \fh(k)\rvx(m-k)}^\ast}}
    \\&=    \opZ\pE\brs{\rvx(0) \sum_{k\in\Z} \fh^\ast(k)      \rvx^\ast(m-k)}
     &&=    \opZ        \sum_{k\in\Z} \fh^\ast(k) \pE\brs{\rvx(0)\rvx^\ast(m-k)}
     &&\eqd \opZ        \sum_{k\in\Z} \fh^\ast(k) \Rxx^\ast(m-k)
    \\&\eqd \opZ\brs{\fh^\ast(m) \convd \Rxx^\ast(m)}
     &&=    \brs{\opZ\fh^\ast(m)} \brs{\opZ\Rxx^\ast(m)}
     &&= \boxed{\ZH^\ast\brp{z^\ast} \Szxx^\ast\brp{z^\ast}}
\\
    4.\quad\Szxy(z)
      &\eqd \opZ\pE\Rxy(m)
     &&\eqd \opZ\pE\brs{\rvx^\ast(m)\rvy(0)}
     &&=    \opZ\pE\brs{\rvx^\ast(m)\sum_{k\in\Z} \fh(k)           \rvx(0-k)}
    \\&=    \opZ\sum_{k\in\Z} \fh(k)\pE\brs{\rvx^\ast(m)\rvx(0-k)}
     &&=    \opZ\sum_{k\in\Z} \fh(k)\pE\brs{\rvx^\ast(m+k)\rvx(0)}
     &&=    \opZ\sum_{k\in\Z} \fh(k)\Rxx(m+k)
    \\&=    \opZ\sum_{k'\in\Z} \fh(-k')\Rxx(m-k')
      && \text{where $k'\eqd k$}
    \\&\eqd \opZ\brs{\fh(-m) \convd \Rxx(m)}
     &&= \brs{\opZ\fh(-m)} \brs{\opZ\Rxx(m)}
     &&= \boxed{\ZH\brp{\frac{1}{z}} \Szxx(z)}
\\
    5.\quad\Szxy(z)
      &\eqd \opZ\pE\Rxy(m)
     &&\eqd \opZ\pE\brs{\rvx(0)\rvy(m)}
    \\&=    \opZ\pE\brs{\rvx(0)\brp{\sum_{k\in\Z} \fh(k) \rvx(m-k)}}
     &&=    \opZ                    \sum_{k\in\Z} \fh(k) \pE\brs{\rvx(0)\rvx(m-k)}
     &&\eqd \opZ                    \sum_{k\in\Z} \fh(k) \Rxx(m-k)
    \\&\eqd \opZ\brs{\fh(m) \convd \Rxx(m)}
      &&
      &&= \boxed{\ZH(z) \Szxx(z)}
\\
   6.\quad\Szxy(z)
      &\eqd \opZ\pE\Rxy(m)
     &&\eqd \opZ\pE\brs{\rvx(m)\rvy(0)}
       \mathrlap{\qquad= \opZ\pE\brs{\rvx(m)\brp{\sum_{k\in\Z} \fh(k) \rvx(0-k)}}}
    \\&=    \opZ\sum_{k\in\Z} \fh(k) \pE\brs{\rvx(m)\rvx(-k)}
     &&=    \opZ\sum_{k\in\Z} \fh(k) \pE\brs{\rvx(0)\rvx(-m-k)}
     &&=    \opZ\sum_{k\in\Z} \fh(k) \pE\brs{\rvx(-m-k)\rvx(0)}
    \\&\eqd \opZ                    \sum_{k\in\Z} \fh(k) \Rxx(-m-k)
     &&\eqd \opZ\brs{\fh(-m) \convd \Rxx(-m)}
     &&= \opZ\brs{\fh(-m) \convd \Rxx^\ast(m)}
    \\&= \opZ\brs{\fh(-m)} \brs{\opZ\Rxx^\ast(m)}
  && &&= \boxed{\ZH\brp{\frac{1}{z}} \Szxx^\ast\brp{z^\ast}}
\\
    7.\quad\Szxy(z)
      &\eqd \opZ\pE\Rxy(m)
     &&\eqd \opZ\pE\brs{\rvx^\ast(0)\rvy^\ast(m)}
     &&=    \opZ\pE\brs{\rvx^\ast(0)\rvy^\ast(m)}
    \\&=    \opZ\pE\brs{\rvx^\ast(0)\brp{\sum_{k\in\Z} \fh(k) \rvx(m-k)}^\ast}
     &&=    \opZ                    \sum_{k\in\Z} \fh^\ast(k) \pE\brs{\rvx^\ast(0)\rvx^\ast(m-k)}
     &&\eqd \opZ                    \sum_{k\in\Z} \fh^\ast(k) \Rxx(m-k)
    \\&\eqd \opZ\brs{\fh^\ast(m) \convd \Rxx(m)}
     &&= \brs{\opZ\fh^\ast(m)} \brs{\opZ\Rxx(m)}
     &&= \boxed{\ZH^\ast\brp{z^\ast} \Szxx(z)}
\\
    8.\quad\Szxy(z)
      &\eqd \opZ\pE\Rxy(m)
     &&\eqd \opZ\pE\brs{\rvx^\ast(m)\rvy^\ast(0)}
       \mathrlap{\qquad=    \opZ\pE\brs{\rvx^\ast(m)\brp{\sum_{k\in\Z} \fh^\ast(k) \rvx^\ast(0-k)}}}
    \\&=    \opZ\sum_{k\in\Z} \fh^\ast(k) \pE\brs{\rvx^\ast(m)\rvx^\ast(-k)}
     &&=    \opZ\sum_{k\in\Z} \fh^\ast(k) \pE\brs{\rvx^\ast(0)\rvx^\ast(-m-k)}
    \\&=    \opZ\sum_{k\in\Z} \fh^\ast(k) \pE\brs{\rvx^\ast(-m-k)\rvx^\ast(0)}
     &&\eqd \opZ                    \sum_{k\in\Z} \fh^\ast(k) \Rxx^\ast(-m-k)
     &&\eqd \opZ\brs{\fh(-m) \convd \Rxx^\ast(-m)}
    \\&= \opZ\brs{\fh(-m) \convd \Rxx(m)}
      && \text{by property of $\Rxx(m)$}
      && \text{\xref{cor:Rxxm}}
    \\&= \brs{\opZ\fh(-m)} \brs{\opZ\Rxx(m)}
     &&&&= \boxed{\ZH\brp{\frac{1}{z}} \Szxx(z)}
  \end{align*}
\end{proof}

%---------------------------------------
\begin{proposition}
\label{prop:Rxym}
\label{prop:RxySwxy}
%---------------------------------------
\propbox{\begin{array}{FM >{\ds}rc l*{3}{@{\hspace{0pt}}l} c >{\ds}rc l*{2}{@{\hspace{0pt}}l}}
    (1).&Papoulis:        & \Rxy(m) &\eqd& \pE[\rvx     &(m)\rvy^\ast&(0&)] &\implies& \Swxy(\omega) &=&\FH^\ast&( &\omega)\Swxx(\omega)
  \\(2).&Kay:             & \Rxy(m) &\eqd& \pE[\rvx^\ast&(0)\rvy     &(m&)] &\implies& \Swxy(\omega) &=&\FH     &( &\omega)\Swxx(\omega)
  \\(3).&y-star-m:        & \Rxy(m) &\eqd& \pE[\rvx     &(0)\rvy^\ast&(m&)] &\implies& \Swxy(\omega) &=&\FH^\ast&(-&\omega)\Swxx(\omega)
  \\(4).&x-star-m:        & \Rxy(m) &\eqd& \pE[\rvx^\ast&(m)\rvy     &(0&)] &\implies& \Swxy(\omega) &=&\FH     &(-&\omega)\Swxx(\omega)
  \\(5).&Bendat:          & \Rxy(m) &\eqd& \pE[\rvx     &(0)\rvy     &(m&)] &\implies& \Swxy(\omega) &=&\FH     &( &\omega)\Swxx(\omega)
  \\(6).&alt-Bendat:      & \Rxy(m) &\eqd& \pE[\rvx     &(m)\rvy     &(0&)] &\implies& \Swxy(\omega) &=&\FH     &(-&\omega)\Swxx(\omega)
  \\(7).&Bendat-star      & \Rxy(m) &\eqd& \pE[\rvx^\ast&(0)\rvy^\ast&(m&)] &\implies& \Swxy(\omega) &=&\FH^\ast&(-&\omega)\Swxx(\omega)
  \\(8).&alt-Bendat-star: & \Rxy(m) &\eqd& \pE[\rvx^\ast&(m)\rvy^\ast&(0&)] &\implies& \Swxy(\omega) &=&\FH     &(-&\omega)\Swxx(\omega)
\end{array}}
\end{proposition}
\begin{proof}
\begin{align*}
\intertext{1. If we follow Papoulis $\brp{\Rxy(m)\eqd\pE\brs{\rvx(m)\rvy^\ast(0)}}$, then\ldots}
    \Swxy(\omega)
      &\eqd \brlr{\Szxy(z)}_{z=e^{i\omega}}
    \\&=    \brlr{\ZH^\ast\brp{\frac{1}{z^\ast}} \Szxx^\ast\brp{\frac{1}{z^\ast}}}_{z=e^{i\omega}}
      &&    \text{by $\Szxy(z)$ result}
      &&    \text{\xref{prop:RxySzxy}}
    \\&=    \ZH^\ast\brp{e^{i\omega}} \Szxx^\ast\brp{e^{i\omega}}
    \\&=    \FH^\ast\brp{\omega} \Swxx^\ast\brp{\omega}
    \\&=    \boxed{\FH^\ast\brp{\omega} \Swxx\brp{\omega}}
      && \text{because $\Swxx(\omega)$ is \prope{real-valued}}
      && \text{\xref{cor:Swxy_sym}}
\end{align*}
\begin{align*}
    2.\quad\Swxy(\omega)
      &\eqd \brlr{\Szxy(z)}_{z=e^{i\omega}}
     &&=    \brlr{\ZH(z) \Szxx(z)}_{z=e^{i\omega}}
     &&=    \ZH\brp{e^{i\omega}} \Szxx\brp{e^{i\omega}}
     &&\eqd \boxed{\FH\brp{\omega} \Swxx\brp{\omega}}
\\
    3.\quad\Swxy(\omega)
      &\eqd \brlr{\Szxy(z)}_{z=e^{i\omega}}
     &&=    \brlr{\ZH^\ast\brp{z^\ast} \Swxx^\ast\brp{z^\ast}}_{z=e^{i\omega}}
     &&=    \ZH^\ast\brp{e^{-i\omega}} \Szxx^\ast\brp{e^{-i\omega}}
     &&\eqd \boxed{\FH^\ast\brp{-\omega} \Swxx^\ast\brp{-\omega}}
\\
    4.\quad\Swxy(\omega)
      &\eqd \brlr{\Szxy(z)}_{z=e^{i\omega}}
     &&=    \brlr{\ZH\brp{\frac{1}{z}} \Szxx(z)}_{z=e^{i\omega}}
     &&=    \ZH\brp{e^{-i\omega}} \Szxx\brp{e^{i\omega}}
     &&\eqd \boxed{\FH(-\omega) \Swxx\brp(\omega)}
\\
    5.\quad\Swxy(\omega)
      &\eqd \brlr{\Szxy(z)}_{z=e^{i\omega}}
     &&=    \brlr{\ZH(z) \Szxx(z)}_{z=e^{i\omega}}
     &&=    \ZH\brp{e^{i\omega}} \Szxx\brp{e^{i\omega}}
     &&\eqd \boxed{\FH(\omega) \Swxx\brp(\omega)}
\\
    6.\quad\Swxy(\omega)
      &\eqd \brlr{\Szxy(z)}_{z=e^{i\omega}}
     &&=    \brlr{\ZH\brp{\frac{1}{z}} \Szxx^\ast\brp{z^\ast}}_{z=e^{i\omega}}
     &&=    \ZH\brp{e^{-i\omega}} \Szxx\brp{e^{-i\omega}}
     &&\eqd \boxed{\FH(-\omega) \Swxx\brp(-\omega)}
\\
    7.\quad\Swxy(\omega)
      &\eqd \brlr{\Szxy(z)}_{z=e^{i\omega}}
     &&=    \brlr{\ZH^\ast\brp{z^\ast} \Szxx(z)}_{z=e^{i\omega}}
     &&=    \ZH^\ast\brp{e^{-i\omega}} \Szxx\brp{e^{i\omega}}
     &&\eqd \boxed{\FH^\ast(-\omega) \Swxx\brp(\omega)}
\\
    8.\quad\Swxy(\omega)
      &\eqd \brlr{\Szxy(z)}_{z=e^{i\omega}}
     &&=    \brlr{\ZH\brp{\frac{1}{z}} \Szxx(z)}_{z=e^{i\omega}}
     &&=    \ZH^\ast\brp{e^{-i\omega}} \Szxx\brp{e^{i\omega}}
     &&\eqd \boxed{\FH(-\omega) \Swxx\brp(\omega)}
  \end{align*}
\end{proof}

Note the following:
\begin{enumerate}
\item The \ope{expectation} operator $\pE\brp{\rvX\rvY^\ast}$ is an \fncte{inner product}.
As such, it would seem the most natural to follow the convention of other inner product definitions
and thus put the conjugate $\conj$ on $\rvy$ (i.e. follow Papoulis):
\\\indentx$\begin{array}{c>{\ds}rc>{\ds}l}
    \imark & \inprod{\fx(t)}{\fy(t)} &\eqd& \int_{t\in\R} \fx(t)\fy^\conj(t) \dt
  \\\imark & \inprod{\fx(n)}{\fy(n)} &\eqd& \sum_{n\in\Z} \fx(n)\fy^\conj(n)
  \\\imark & \inprod{\rvX}{\rvY}     &\eqd& \pE\brp{\rvX\rvY^\conj}
\end{array}$

\item If we view $\Rxy(m)$ as an \ope{analysis} of $\rvy$ in terms of $\rvx$ 
      (or as a \ope{projection} of $\rvy$ onto $\rvx$),
      then it would seem more natural to put the conjugate on $\rvx$ (i.e. follow Kay).
      This is what is done in Fourier analysis when projecting a function $\ff(t)$ onto the 
      set of basis functions $\set{e^{i\omega n}}{\omega\in\R}$, as in 
      \\\begin{align*}
        \opDTFT\brs{\rvy(n)}(\omega) 
          &\eqd \inprod{\rvy(n)}{e^{i\omega n}} 
          && \text{(\ope{project} $\rvy(n)$ onto $e^{i\omega n}$ for some $\omega\in\R$)}
        \\&\eqd \sum_{n\in\Z} \rvy(n) \brs{e^{+i\omega n}}^\ast
        \\&\eqd \sum_{n\in\Z} \rvy(n) e^{-i\omega n}
      \end{align*}
      But arguably, a ``projection of $\rvy$ onto $\rvx$" would better be served by the use of $\Ryx(m)$ rather than $\Rxy(m)$.

\item If we follow Kay, then there is the advantage that you also end up with Bendat's result for $\Swxy(\omega)$.
\end{enumerate}

%---------------------------------------
\begin{proposition}
\label{prop:RxySwxy_real}
%---------------------------------------
In the special case where $\rvx(n)$ and $\rvy(n)$ are \propb{real-valued}\ldots
\propbox{\begin{array}{FM >{\ds}rc l*{3}{@{\hspace{0pt}}l} c >{\ds}rc l@{\hspace{0pt}}l}
    (1s).&Papoulis:        & \Rxy(m) &\eqd& \pE[\rvx     &(m)\rvy^\ast&(0&)] &\implies& \Swxy(\omega) &=&\FH^\ast&(\omega)\Swxx(\omega)
  \\(2s).&Kay:             & \Rxy(m) &\eqd& \pE[\rvx^\ast&(0)\rvy     &(m&)] &\implies& \Swxy(\omega) &=&\FH     &(\omega)\Swxx(\omega)
  \\(3s).&y-star-m:        & \Rxy(m) &\eqd& \pE[\rvx     &(0)\rvy^\ast&(m&)] &\implies& \Swxy(\omega) &=&\FH     &(\omega)\Swxx(\omega)
  \\(4s).&x-star-m:        & \Rxy(m) &\eqd& \pE[\rvx^\ast&(m)\rvy     &(0&)] &\implies& \Swxy(\omega) &=&\FH^\ast&(\omega)\Swxx(\omega)
  \\(5s).&Bendat:          & \Rxy(m) &\eqd& \pE[\rvx     &(0)\rvy     &(m&)] &\implies& \Swxy(\omega) &=&\FH     &(\omega)\Swxx(\omega)
  \\(6s).&alt-Bendat:      & \Rxy(m) &\eqd& \pE[\rvx     &(m)\rvy     &(0&)] &\implies& \Swxy(\omega) &=&\FH^\ast&(\omega)\Swxx(\omega)
  \\(7s).&Bendat-star      & \Rxy(m) &\eqd& \pE[\rvx^\ast&(0)\rvy^\ast&(m&)] &\implies& \Swxy(\omega) &=&\FH     &(\omega)\Swxx(\omega)
  \\(8s).&alt-Bendat-star: & \Rxy(m) &\eqd& \pE[\rvx^\ast&(m)\rvy^\ast&(0&)] &\implies& \Swxy(\omega) &=&\FH^\ast&(\omega)\Swxx(\omega)
\end{array}}
\end{proposition}
\begin{proof}
These results follow directly from \pref{prop:RxySwxy} and \prefpp{thm:dtft_conjneg_real}.
\end{proof}
