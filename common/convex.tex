%============================================================================
% Daniel J. Greenhoe
% LaTeX file
%============================================================================




%=======================================
\chapter{Intervals and Convexity}
\label{chp:convex}
%=======================================


%%=======================================
%\section{Convexity spaces}
%%=======================================
%%---------------------------------------
%\begin{definition}
%\label{def:convex}
%\footnote{
%  \citerpg{vel1993}{3}{0444815058}\\
%  \citerpg{khamsi2001}{35}{0471418250}
%  }
%\index{space!convex}
%\index[xsym]{$\ssetC$}
%%---------------------------------------
%\defboxt{
%  The family of sets $\ssetC$ is a \propd{convexity} on a set $\setX$ if
%  \\\indentx
%  $\begin{array}{F >{\ds}l C DD}
%      1. & \emptyset \in \ssetC
%         &
%         & (\prop{union identity element})
%         & and
%    \\2. & \setX \in \ssetC
%         &
%         & (\prop{intersection identity element})
%         & and
%    \\3. & \brb{\begin{array}{ll}
%             \ssetD \ne  \emptyset \\
%             \ssetD \subseteq \ssetC
%           \end{array}}
%           \implies
%           \setopi \ssetD \in \ssetC
%         & 
%         & (\prop{stable for intersections}) 
%         & and
%    \\4. & \brb{\begin{array}{ll}
%             \ssetD \ne  \emptyset \\
%             \ssetD \subseteq \ssetC \\
%             \opair{\ssetD}{\subseteq} \text{ is totally ordered}
%           \end{array}}
%           \implies
%           \setopu \ssetD \in \ssetC
%         & 
%         & (\prop{stable for nested unions}) 
%         & 
%  \end{array}$
%  \\
%  A set that is a member of the convexity $\ssetC$ is called a \structd{convex set};\\
%  A set that is not a member is called a \structd{concave set}.\\
%  The pair $\opair{\setX}{\ssetC}$ is called a \structd{convexity space}, 
%  an \structd{aligned space}, or a \structd{convexity structure}.
%  }
%\end{definition}

%=======================================
\section{Intervals}
%=======================================
%=======================================
%\subsection{Intervals}
%=======================================
In the real number system, for $a\le b$, the \structe{interval} $\intcc{a}{b}$
is the set $a$ and $b$ and all the numbers inbetween, as in
$\intcc{a}{b}\eqd\set{x\in\R}{a\le x\le b}$.
This concept can be easily generalized:
\begin{listi}
  \item In an \structb{ordered set}\ifsxref{order}{def:order}, \label{item:intcc_order}
        if two elements $x$ and $y$ are \prope{comparable} and $x\le y$,
        then we say that $x$ and $y$ and all the elements inbetween,
        as determined by the ordering relation $\le$, are the interval $\intcc{a}{b}$\ifsxref{order}{def:intcc_order}.

  \item In a \structb{lattice}\ifsxref{lattice}{def:lattice},
        the concept of the \structe{interval} can be generalized even further.
        In an arbitrary ordered set, the interval $\intcc{x}{y}$ of \pref{item:intcc_order} is restricted to the case in 
        which $x$ and $y$ are \prope{comparable}\ifsxref{order}{def:order}.
        This restriction can be lifted \xref{def:intcc_lat} with the additional structure of 
        upper and lower bounds provided by lattices.

  \item A \structb{metric space}\ifsxref{metric}{def:metric} in general has no \rele{order relation} $\orel$\ifsxref{order}{def:order}. 
        But intervals can still be defined \xref{def:intcc_metric} in a metric space in terms of the \prope{triangle inequality}.

  \item A \structb{linear space}\ifsxref{vector}{def:vspace} over a real or complex field 
       in general has no \rele{order relation} that compares \structe{vectors} in the space,
       but the standard order relation $\orel$ for real numbers $\R$ can still be used \xref{def:intcc_linspace}
       to define an interval in a linear space.
\end{listi}

%---------------------------------------
\begin{definition}
\footnote{
  \citerpg{apostol1975}{4}{0201002884},
  \citorp{ore1935}{409}
  }
\label{def:intcc}
\label{def:intcc_order}
%---------------------------------------
Let $\opair{\setX}{\le}$ be  an \structe{ordered set}\ifsxref{order}{def:order}.
\defbox{\indxs{\intccn}\indxs{\intoon}\indxs{\intocn}\indxs{\intcon}
  \begin{array}{MlclMD}
    The set & \intcc{x}{y} &\eqd& \set{z\in\setX}{x\le z\le y} & is called a  \structd{closed interval}     & and \\
    The set & \intoc{x}{y} &\eqd& \set{z\in\setX}{x<   z\le y} & is called a  \structd{half-open interval}  & and \\
    The set & \intco{x}{y} &\eqd& \set{z\in\setX}{x\le z<   y} & is called a  \structd{half-open interval}  & and \\
    The set & \intoo{x}{y} &\eqd& \set{z\in\setX}{x<   z<   y} & is called an \structd{open interval}.      & 
  \end{array}
  }
\end{definition}

%=======================================
%\section{Intervals}
%\label{sec:lat_intervals}
%=======================================
%---------------------------------------
\begin{definition}[intervals on lattices]
\footnote{
  \citorp{duthie1942}{2}, %{$a$=\hie{lower extremity}, $b$=\hie{upper extremity}}\\
  \citorpc{ore1935}{425}{\hie{quotient structures}}
  }
\label{def:lat_intervals}
\label{def:intcc_lat}
%---------------------------------------
Let $\latticeX$ be  a \structe{lattice}\ifsxref{lattice}{def:lattice}.
\defbox{\indxs{\intccn}\indxs{\intoon}\indxs{\intocn}\indxs{\intcon}
  \begin{array}{Mcl@{\qquad}M}
    The set $\intcc{x}{y}$ &\eqd& \set{z\in\setX}{x\meet y \le z\le x\join y} & is called a \hid{closed interval}.    \\
    The set $\intoc{x}{y}$ &\eqd& \set{z\in\setX}{x\meet y <   z\le x\join y} & is called a \hid{half-open interval}. \\
    The set $\intco{x}{y}$ &\eqd& \set{z\in\setX}{x\meet y \le z<   x\join y} & is called a \hid{half-open interval}. \\
    The set $\intoo{x}{y}$ &\eqd& \set{z\in\setX}{x\meet y <   z<   x\join y} & is called an \hid{open interval}.
  \end{array}
  }
\end{definition}
When $x$ and $y$ \emph{are} comparable and $x\orel y$, 
then \pref{def:intcc_lat} (previous) simplifies to \prefpp{item:intcc_order}.

%---------------------------------------
\begin{definition}
\footnote{
  \citerp{maeda1970}{1}
  }
%---------------------------------------
Let $\latL\eqd\latticed$ be a \hie{lattice} with dual $\latLd$.
Let $\intcc{x}{y}$ be a \structe{closed interval} \xref{def:intcc_lat} on set $\setX$.
The sublattices $\latL\intcc{x}{y}$ and $\latLd\intcc{x}{y}$ are defined as follows:
\defbox{\begin{array}{lcl@{\qquad}C}
  \latL \intcc{x}{y} &\eqd& \set{z\in\latL }{z\in\intcc{x}{y}} & \forall x,y\in\setX \\
  \latLd\intcc{x}{y} &\eqd& \set{z\in\latLd}{z\in\intcc{x}{y}} & \forall x,y\in\setX 
\end{array}}
\end{definition}


%--------------------------------------
\begin{definition}
\label{def:intcc_metric}
\footnote{
  %\citerpg{khamsi2001}{35}{0471418250}\\
  \citerpg{vel1993}{8}{0444815058}
  %\citer{hormander1994}\\
  %\citorp{menger1928}{81?}
  }
%--------------------------------------
%Let $\metspaceX$ be a \structe{metric space} \xref{def:metric}.
\defboxt{
  In a \structe{metric space} $\metspaceX$\ifsxref{metric}{def:metric},\\
  the set $\intcc{a}{b}$ is the \structd{closed interval} from $x$ to $y$ and is defined as
    \\\indentx$\intcc{x}{y} \eqd \set{z\in\setX}{\metric{x}{z} + \metric{z}{y} = \metric{x}{y}}$.\\
  An element $z\in\setX$ is \propd{geodesically between} $x$ and $y$ if $z\in\intcc{x}{y}$.
  }
\end{definition}

%---------------------------------------
\begin{definition}
\label{def:intcc_linspace}
\footnote{
  \citerpg{barvinok2002}{2}{0821872311}
  }
%---------------------------------------
\defboxt{\indxs{\intccn}\indxs{\intoon}\indxs{\intocn}\indxs{\intcon}
  In a \structe{linear space} $\linearspaceX$\ifsxref{vector}{def:vspace},
  \\\indentx$\begin{array}{lclMD}
    \intcc{\vx}{\vy} &\eqd& \set{\lambda\vx + (1-\lambda)\vy=\vz}{0\leq\lambda\leq 1} & is called a  \structd{closed interval}    & and\\
    \intoc{\vx}{\vy} &\eqd& \set{\lambda\vx + (1-\lambda)\vy=\vz}{0 <  \lambda\leq 1} & is called a  \structd{half-open interval} & and\\
    \intco{\vx}{\vy} &\eqd& \set{\lambda\vx + (1-\lambda)\vy=\vz}{0\leq\lambda <   1} & is called a  \structd{half-open interval} & and\\
    \intoo{\vx}{\vy} &\eqd& \set{\lambda\vx + (1-\lambda)\vy=\vz}{0 <  \lambda <   1} & is called an \structd{open interval}.     & 
  \end{array}$%
  }
%\defboxt{\indxs{\intccn}\indxs{\intoon}\indxs{\intocn}\indxs{\intcon}
%  In a \structe{linear space} $\linearspaceX$ \xref{def:vspace},
%  \\$\begin{array}{lclMD}
%    \intcc{\vx}{\vy} &\eqd& \set{\vz\in\setX}{\exists\lambda \st \begin{tabstr}{1}\begin{array}{l}0\leq\lambda\leq 1\,\text{\scs and}\\\lambda\vx + (1-\lambda)\vy=\vz\end{array}\end{tabstr}} & is a  \structd{closed interval}    & and\\\\
%    \intoc{\vx}{\vy} &\eqd& \set{\vz\in\setX}{\exists\lambda \st \begin{tabstr}{1}\begin{array}{l}0 <  \lambda\leq 1\,\text{\scs and}\\\lambda\vx + (1-\lambda)\vy=\vz\end{array}\end{tabstr}} & is a  \structd{half-open interval} & and\\\\
%    \intco{\vx}{\vy} &\eqd& \set{\vz\in\setX}{\exists\lambda \st \begin{tabstr}{1}\begin{array}{l}0\leq\lambda <   1\,\text{\scs and}\\\lambda\vx + (1-\lambda)\vy=\vz\end{array}\end{tabstr}} & is a  \structd{half-open interval} & and\\\\
%    \intoo{\vx}{\vy} &\eqd& \set{\vz\in\setX}{\exists\lambda \st \begin{tabstr}{1}\begin{array}{l}0 <  \lambda <   1\,\text{\scs and}\\\lambda\vx + (1-\lambda)\vy=\vz\end{array}\end{tabstr}} & is an \structd{open interval}.     & 
%  \end{array}$%
%  \renewcommand{\arraystretch}{1}%
%  }
\end{definition}

%=======================================
\section{Convex sets}
%=======================================
Using the concept of the \structe{interval} (previous section),
we can define the \structe{convex set} (next definition).
%---------------------------------------
\begin{definition}
\footnote{
  \citerpg{barvinok2002}{5}{0821872311}
%  \citerpg{beran1985}{4}{902771715X}\\
%  \citerpc{alfsen1963}{432}{(5.1)}\\
%  \citerpg{vel1993}{6}{0444815058}\\
%  \citerpg{stern1999}{284}{0521461057}\\
%  \citerpg{denkowski2003}{116}{0306474565}
  }
\label{def:convex_set}
%---------------------------------------
Let $\setX$ be a \structe{set} in an \structe{ordered set} $\osetX$, a \structe{lattice} $\latticeX$, 
  a \structe{metric space} $\metspaceX$, or a \structe{linear space} $\linearspaceX$.
\defboxt{
  A subset $\setD\subseteq\setX$ is a \structd{convex set} in $\setX$ if
  \\\indentx$x,y\in\setD \qquad\implies\qquad \intcc{x}{y}\subseteq\setD$.\\
  A set that is {\bf not} convex is \propd{concave}.
  }
\end{definition}


%%--------------------------------------
%\begin{definition}
%\footnote{
%  \citerpg{mitrinovic2010}{1}{9048142253}\\
%  \citerppg{vel1993}{5}{6}{0444815058}\\
%  \citerp{bollobas1999}{2}
%  }
%\label{def:convex_set}
%\label{def:convex}
%\index{standard convexity}
%\index{convexity!standard}
%%--------------------------------------
%Let $\spO\eqd\linearspaceX$ be a \structe{linear space} \xref{def:vspace}.
%\defboxt{
%  A set $\setD\subseteq\setX$ is a \structd{convex set} in $\spO$ if
%  \\\indentx$\begin{array}{lCDC}
%    \lambda \vx + (1-\lambda)\vy \in\setD  & \forall\vx,\vy\in\setD &and& \forall\lambda\in\opair{0}{1}
%  \end{array}$\\
%  A set that is {\bf not} convex is \hid{concave}.
%  }
%\end{definition}


%--------------------------------------
\begin{example}
%--------------------------------------
Consider the \structe{Euclidean space} $\R^2$ (a special case of a \structe{linear space}).\\
\begin{minipage}{\tw/3}
  \begin{center}
  \color{figcolor}
  \begin{fsL}
  \setlength{\unitlength}{\tw/500}
  \begin{picture}(300,300)(-130,-130)
    %\graphpaper[10](0,0)(600,600)
    \thicklines
    \put(-130,   0){\line(1,0){260} }
    \put(   0,-130){\line(0,1){260} }
    \qbezier(-100,0)(0, 160)(100,0)
    \qbezier(-100,0)(0,-160)(100,0)
    {\color{red}
      \put(0,0){\vector(1, 1){55}}
      \put(0,0){\vector(1,-1){55}}
      \put(  60, 60){\makebox(0,0)[bl]{$\vx$} }
      \put(  60,-60){\makebox(0,0)[tl]{$\vy$} }
      \put(55,-55){\line(0,1){110}}
      \put(-100,100){\makebox(0,0)[bl]{$\lambda \vx + (1-\lambda)\vy$} }
      \put(-50,90){\vector(3,-2){105}}
    }
  \end{picture}
  \end{fsL}
  \end{center}
\end{minipage}%
%
\begin{minipage}{\tw/3}
    \[
      \Leftarrow
      \left\{
        \parbox[c][][c]{\tw-6ex}
        {The figure to the left is a \prope{convex} set in $\R^2$.}
      \right.
    \]
    \[
      \left.
        \parbox[c][][c]{\tw-6ex}
        {The figure to the right is a \prope{concave} set in $\R^2$.}
      \right\}
      \Rightarrow
    \]
\end{minipage}%
%
\begin{minipage}{\tw/3}
  \begin{center}
  \color{figcolor}
  \begin{fsL}
  \setlength{\unitlength}{\tw/500}
  \begin{picture}(300,300)(-130,-130)
    %\graphpaper[10](-130,-130)(300,300)
    \thicklines
    \put(-130,   0){\line(1,0){260} }
    \put(   0,-130){\line(0,1){260} }
    \qbezier(-100,0)(0, 100)(100, 100)
    \qbezier(-100,0)(0,-100)(100,-100)
    \qbezier(100,100)(-50,0)(100,-100)
    {\color{red}
      %\put(0,0){\vector(1, 1){55}}
      %\put(0,0){\vector(1,-1){55}}
      \put(  75, 60){\makebox(0,0)[tl]{$\vx$} }
      \put(  75,-60){\makebox(0,0)[bl]{$\vy$} }
      \put(60,-68){\line(0,1){136}}
      \put(-100,100){\makebox(0,0)[bl]{$\lambda \vx + (1-\lambda)\vy$} }
      \put(-50,90){\vector(3,-2){105}}
    }
  \end{picture}
  \end{fsL}
  \end{center}
\end{minipage}
\end{example}

%--------------------------------------
\begin{example}
%--------------------------------------
In a \structe{metric space}\ifsxref{metric}{def:metric},
examples of \structe{convex set}s are convex \structe{ball}s.
Examples include those balls generated by the following metrics:\\
\indentx\begin{tabular}{>{$\imark$}llll}
  & Taxi-cab metric   \ifdochas{metric}{& \pref{ex:ms_taxi}       & \prefpo{ex:ms_taxi}} \\
  & Euclidean metric  \ifdochas{metric}{& \pref{ex:ms_euclidean} & \prefpo{ex:ms_euclidean}} \\
  & Sup metric        \ifdochas{metric}{& \pref{ex:ms_sup}       & \prefpo{ex:ms_sup}} \\
  & Tangential metric \ifdochas{metric}{& \pref{ex:ms_tan}       & \prefpo{ex:ms_tan}}
\end{tabular}

Examples of metrics generating balls which are \emph{not} convex include the following:\\
\indentx\begin{tabular}{>{$\imark$}llll}
  & Parabolic metric    \ifdochas{metric}{& \pref{ex:ms_parabolic} & \prefpo{ex:ms_parabolic}} \\
  & Exponential metric  \ifdochas{metric}{& \pref{ex:ms_32x}       & \prefpo{ex:ms_32x}} \\
\end{tabular}
\end{example}

%%--------------------------------------
%\begin{definition}
%\label{def:convex_metric}
%\footnote{
%  \citerpg{khamsi2001}{35}{0471418250}\\
%  \citerpg{vel1993}{8}{0444815058}\\
%  \citer{hormander1994}\\
%  \citorp{menger1928}{81?}
%  }
%\index{geodesic convexity}
%\index{convexity!geodesic}
%\index{metrically convex}
%%--------------------------------------
%\defboxt{
%  A set $\setD\subseteq\setX$ is \propd{metrically convex} in the \hie{metric space} $\opair{\setX}{\metricn}$ if
%  \\\indentx$\ds
%    \brbr{\begin{array}{lD}
%      x,y\in\setD           & and \\
%      z\in\setX             & and \\
%      \mcoml{\metric{x}{z} + \metric{z}{y} = \metric{x}{y}}{$z$ is \hie{geodesically between} $x$ and $y$}
%    \end{array}}
%    \qquad\implies\qquad
%    z\in\setD
%  $
%  \\
%  A set $\setD$ that is metrically convex is also said to be \propd{geodesically convex}.
%  }
%\end{definition}



%=======================================
\section{Convex functions}
%=======================================
%--------------------------------------
\begin{definition}
\label{def:convexf}
\footnote{
  \citerpg{simon2011}{2}{1107007313},
  \citerpg{barvinok2002}{2}{0821872311},
  \citerp{bollobas1999}{3},
  \citerp{jensen1906}{176}
  }
\index{convex!functional}
\index{convex!strictly}
\index{strictly convex}
%--------------------------------------
Let $\linearspaceX$ be a \structe{linear space}\ifsxref{vector}{def:vspace} 
and $\setD$ a \structe{convex set} \xref{def:convex_set} in $\setX$.
\defbox{\begin{array}{rcl@{\qquad}l}
  \mc{4}{M}{A function $\ff\in\clF{\setD}{F}$ is \hid{convex} if}\\
  \indentx\ff\big(\lambda \vx + [1-\lambda]\vy\big) 
    &\le& \lambda \ff(\vx) + (1-\lambda)\:\ff(\vy)  
    &     \forall \vx,\vy\in\setD \text{ and } \forall\lambda\in(0,1)
  \\ 
  \mc{4}{M}{A function $\fg\in\clF{\setD}{F}$ is \hid{strictly convex} if }\\
  \indentx\fg\big(\lambda \vx + [1-\lambda]\vy\big) 
    &=&   \lambda \fg(\vx) + (1-\lambda)\:\fg(\vy)  
    &     \forall \vx,\vy\in\setD,\;\vx\ne \vy,\; \text{and } \forall\lambda\in(0,1)
  \\ 
  \mc{4}{M}{A function $\ff\in\clF{\setD}{F}$ is \propd{concave} if $-\ff$ is \prope{convex}.}
  \\ 
  \mc{4}{M}{A function $\ff\in\clF{\setD}{F}$ is \propd{affine} if $\ff$ is \prope{convex} and \prope{concave}.}\\
  \end{array}}
\end{definition}

%---------------------------------------
%---------------------------------------
\begin{minipage}{\tw/2-2mm}%
  \begin{example}
  %In analysis, one of the most useful properties that a function can possess is \prope{convexity}.
  The function $\ff(x)=2^x$ is a \structe{convex function} \xref{def:convexf}, 
  as illustrated to the right.
  \end{example}
\end{minipage}%
\hfill%
\begin{minipage}{\tw/2-2mm}%
  \small%
  \psset{unit=10mm}%
  %============================================================================
% Daniel J. Greenhoe
% LaTeX file
%============================================================================
%\psset{unit=10mm}%
\begin{pspicture}(-2,-0.6)(3.5,3.5)%
  \psset{linewidth=1pt,labelsep=1pt}%
  \psaxes[linecolor=axis,yAxis=false,labels=none]{<->}(0,0)(-2,0)(3,3)%
  \psaxes[linecolor=axis,xAxis=false,labels=none]{ ->}(0,0)(-2,0)(3,3)%
  \uput[0](3,0){$x$}%
  \psplot[plotpoints=64,linecolor=blue]{-2}{1.5}{2 x exp}
  \psline[linecolor=red](-1,0.5)(1,2)%
  \psline[linestyle=dotted](-1,0.5)(-1,0)%
  \psline[linestyle=dotted](1,2)(1,0)%
  \psdot(-1,0.5)%
  \psdot(1,2)%
  \uput[135] (-1,0.5){$\ff(x)$}%
  \uput[-45] (1,2){$\ff(y)$}%
  \uput[45] (1.5,2.82){$\ff(x)$}%
  \uput{0.5ex}[-90](-1,0){$x$}%
  \uput{0.5ex}[-90](1,0){$y$}%
  \pnode(0.25,1.189){f14}%
  \pnode(1.25,0.75){f14label}%
  \uput[-90](f14label){$\ff\big(\lambda x + [1-\lambda]y\big)$}%
  \pnode(-0.5,0.88){line}%
  \pnode(-1.5,2.2){linelabel}%
  \uput[45](linelabel){$\lambda\ff(x) + (1-\lambda)\ff(y)$}%
  \ncline{->}{f14label}{f14}%
  \ncline{->}{linelabel}{line}%
\end{pspicture}%
%
\end{minipage}

%--------------------------------------
\begin{definition}
\footnote{
  \citerpgc{beer1993}{13}{0792325311}{\textsection 1.3},
  \citerpg{aubin2009}{222}{0817648488},
  \citerpg{aubin2011}{223}{1118030974}
  }
%--------------------------------------
Let $\linearspaceXR$ be a \structe{linear space}\ifsxref{vector}{def:vspace}.
\defboxt{
  The \structd{epigraph} $\hxs{\epi(\ff)}$ and \structd{hypograph} $\hxs{\hyp(\ff)}$ of a functional $\ff\in\clFxr$ are defined as
  \\\indentx$\begin{array}{rcl}
    \epi(\ff) &\eqd& \set{\opair{x}{y}\in\setX\cprod\R}{y\ge\ff(x)}\\
    \hyp(\ff) &\eqd& \set{\opair{x}{y}\in\setX\cprod\R}{y\le\ff(x)}
  \end{array}$
  }
\end{definition}

%--------------------------------------
\begin{example}
\mbox{}\\
%--------------------------------------
\exbox{%
  \begin{tabular}{m{50mm}cm{50mm}}%
  \psset{unit=10mm}%
  \small%
  %============================================================================
% Daniel J. Greenhoe
% XeLaTeX file
%============================================================================
%\psset{unit=8mm}
\begin{pspicture}(-2.5,-0.75)(2.5,2.5)%
  \psset{%
    labelsep=1pt,
    }%
  \pscustom[linestyle=none,fillstyle=solid,fillcolor=vennshade]{%
    \psplot[plotpoints=64,linecolor=blue]{-1.3}{1.3}{x 2 exp}%
    }%
  \psaxes[linecolor=axis,yAxis=false]{<->}(0,0)(-2,0)(2,2.5)% x axis
  \psaxes[linecolor=axis,xAxis=false]{ ->}(0,0)(-2,0)(2,2.5)% y axis
  \psplot[plotpoints=64,linecolor=blue]{-1.2}{1.2}{x 2 exp}%
  \psplot[plotpoints=64,linestyle=dotted,linecolor=blue]{1.2}{1.4}{x 2 exp}%
  \psplot[plotpoints=64,linestyle=dotted,linecolor=blue]{-1.2}{-1.4}{x 2 exp}%
  \uput[0]{0}(2,0){$x$}%
  \uput[-45]{0}(1,1){$x^2$}%
  \uput[75]{0}(0.5,2){$\epi(x^2)$}%
  \psline{->}(0.5,2)(0.4,1.1)%
\end{pspicture}
%
  &
  %\begin{tabular}{clc}
  %  $\leftarrow$&$x^2$   and $\epi(x^2)$.  &\\
  %              &$1-x^2$ and $\epi(1-x^2)$.&$\rightarrow$
  %\end{tabular}
  &
  \psset{unit=10mm}%
  \small%
  %============================================================================
% Daniel J. Greenhoe
% XeLaTeX file
%============================================================================
%\psset{unit=8mm}
\begin{pspicture}(-2.5,-0.75)(2.5,2.5)%
  \psset{%
    labelsep=1pt,
    }%
  \pscustom[linestyle=none,fillstyle=solid,fillcolor=vennshade]{%
    \psplot[plotpoints=64,linecolor=blue]{-1.2}{1.2}{1 x 2 exp sub}%
    \psline(1.2,1.69)(-1.2,1.69)%
    }%
  \psaxes[linecolor=axis,yAxis=false]{<->}(0,0)(-2,0)(2,2.5)% x axis
  \psaxes[linecolor=axis,xAxis=false]{ ->}(0,0)(-2,0)(2,2.5)% y axis
  \psplot[plotpoints=64,linecolor=blue]{-1}{1}{1 x 2 exp sub}%
  \psplot[plotpoints=64,linestyle=dotted,linecolor=blue]{1}{1.2}{1 x 2 exp sub}%
  \psplot[plotpoints=64,linestyle=dotted,linecolor=blue]{-1}{-1.2}{1 x 2 exp sub}%
  \uput[0]{0}(2,0){$x$}%
  \uput[225]{0}(0.5,0.7){$1-x^2$}%
  \uput[55]{0}(0.5,2){$\epi(1-x^2)$}%
  \psline{->}(0.5,2)(0.4,1.2)%
\end{pspicture}
%
  \end{tabular}
  }
\end{example}

%--------------------------------------
\begin{proposition}
\footnote{
  \citerpg{udriste1994}{63}{0792330021}, %{2.6 Theorem}\\
  \citerpgc{kurdila2005}{178}{3764321989}{Proposition 6.1.1},
  \citerpgc{rockafellar1970}{23}{0691015864}{Section 4 Convex Functions},
  \citerpgc{cinlar2013}{86}{1461452570}{5.4 Theorem}
  }
%--------------------------------------
Let $\linearspaceXR$ be a \structe{linear space}\ifsxref{vector}{def:vspace}.
Let $\ff$ be a \structe{functional} in $\clFxr$.
\propbox{
  \brb{\begin{array}{M}
    $\ff$ is a\\
    \structe{convex function}
  \end{array}}
  \quad\iff\quad
  \brb{\begin{array}{M}
    $\epi(\ff)$ is a\\
    \structe{convex set}
  \end{array}}
  }
\end{proposition}

%--------------------------------------
\begin{theorem}[\thmd{Jensen's Inequality}]
\footnote{
  \citerpg{mitrinovic2010}{6}{9048142253},
  \citerpg{bollobas1999}{3}{0521655773},
  \citerpg{lay1982}{7}{0471095842},
  \citorpp{jensen1906}{179}{180}
  }
\label{thm:jensen_ineq}
\label{thm:jensenineq}
\index{inequalities!Jensen's}
%--------------------------------------
Let $\linearspaceX$ be a \structe{linear space}\ifsxref{vector}{def:vspace},
$\setD$ a subset of $\setX$, and 
$\ff$ a functional in $\clF{\setD}{\F}$.
Let $\sum$ be the \ope{summation operator}\ifsxref{sums}{def:sum}.
\thmbox{
  \brb{\begin{array}{FM@{\quad}D}
    1. & $\setD$ is \prope{convex}       & and \\
    2. & $\ff$ is \prope{convex}         & and \\
    3. & $\ds\sum_{n=1}^\xN \lambda_n=1$ & (\structe{weights})
  \end{array}}  
  \quad\implies\quad
  \ff\brp{\sum_{n=1}^\xN \lambda_n \: \vx_n} 
  \le \sum_{n=1}^\xN \lambda_n\:\ff\brp{\vx_n}
  \qquad
  \forall \vx_n \in \setD,\;\xN\in\Zp
  }
\end{theorem}
\begin{proof}
Proof is by induction:
\begin{align*}
  \intertext{1. Proof that statement is true for $\xN=1$:} 
    \ff\left(\sum_{n=1}^{\xN=1} \lambda_n \: \vx_n\right) 
      &=   \ff(\lambda_1 \vx_1) 
    \\&\le \ff(\lambda_1 \vx_1) 
    \\&=   \sum_{n=1}^{\xN=1} \lambda_n\:\ff(\vx_n)
    \\
  \intertext{2. Proof that statement is true for $\xN=2$:} 
    \ff\left(\sum_{n=1}^{\xN=2} \lambda_n \: \vx_n\right) 
      &=   \ff(\lambda_1 \vx_1 + \lambda_2\vx_2) 
    \\&\le \lambda_1 \ff(\vx_1) + \lambda_2 \ff(\vx_2)
      &&   \text{by convexity hypothesis}
    \\&=   \sum_{n=1}^{\xN=2} \lambda_n\:\ff(\vx_n)
    \\
  \intertext{3. Proof that if the statement is true for $\xN$, then it is also true for $\xN+1$:} 
    \ff\left(\sum_{n=1}^{\xN+1} \lambda_n \: \vx_n\right) 
      &=   \ff\left(\sum_{n=1}^{\xN} \lambda_n \: \vx_n + \lambda_{\xN+1}\vx_{\xN+1}\right) 
    \\&=   \ff\left([1-\lambda_{\xN+1}]\sum_{n=1}^{\xN} \frac{\lambda_n}{1-\lambda_{\xN+1}} \: \vx_n + \lambda_{\xN+1}\vx_{\xN+1}\right) 
    \\&\le [1-\lambda_{\xN+1}]\ff\left(\sum_{n=1}^{\xN} \frac{\lambda_n}{1-\lambda_{\xN+1}} \: \vx_n \right) 
         + \lambda_{\xN+1}\ff(\vx_{\xN+1}) 
      &&   \text{by convexity hypothesis}
    \\&\le [1-\lambda_{\xN+1}]\sum_{n=1}^{\xN} \frac{\lambda_n}{1-\lambda_{\xN+1}}\ff(\vx_n) 
         + \lambda_{\xN+1}\ff(\vx_{\xN+1}) 
      &&   \text{by ``true for $\xN$" hypothesis}
    \\&=   \sum_{n=1}^{\xN} \lambda_n \ff(\vx_n) + \lambda_{\xN+1}\ff(\vx_{\xN+1}) 
    \\&=   \sum_{n=1}^{\xN+1} \lambda_n \ff(\vx_n)
    \\
  \intertext{4. Since the statement is true for $\xN=1$, $\xN=2$, and 
                true for $\xN$ $\implies$ true for $\xN+1$,
                then it is true for $\xN=1,2,3,4,\ldots$}
\end{align*}
\end{proof}



\begin{figure}[ht]
\color{figcolor}
\setlength{\unitlength}{0.3mm}
\begin{center}
\begin{picture}(200,130)(-50,-30)
  \put(  0,  0){\line(1, 0){120}}
  \put(  0,  0){\line(0, 1){120}}
  \put( 20, 80){\circle*{5}}
  \put(100, 40){\circle*{5}}
  \put( 20, 80){\line(2,-1){80}}
  \put(125,  0){\makebox(0,0)[l]{$x$}}
  \put( -5, 80){\makebox(0,0)[r]{$\ff(b)$}}
  \put( -5, 40){\makebox(0,0)[r]{$\ff(a)$}}
  \put( 20, -5){\makebox(0,0)[t]{$a$}}
  \put(100, -5){\makebox(0,0)[t]{$b$}}
  \qbezier(20,80)(70,120)(100,40)
  \qbezier(20,80)(30,0)(100,40)
  \qbezier[28](20,0)(20,40)(20,80)
  \qbezier[14](100,0)(100,20)(100,40)
  \qbezier[7](0,80)(10,80)(20,80)
  \qbezier[32](0,40)(50,40)(100,40)
  \put(120,90){\vector(-1,0){50}}
  \put(120,60){\vector(-1,0){50}}
  \put(120,30){\vector(-1,0){30}}
  \put(125,90){\makebox(0,0)[l]{concave}}
  \put(125,60){\makebox(0,0)[l]{chord $\fy(x)$}}
  \put(125,30){\makebox(0,0)[l]{convex}}
\end{picture}
\end{center}
\caption{
  Convex and concave functions
  \label{fig:convex}
  }
\end{figure}

Often a function can be proven to be \hie{convex} or \hie{concave}.
\hie{Convex} and \hie{concave} functions are
defined in \prefpp{def:convex} (next) and
illustrated in \prefpp{fig:convex}.
%---------------------------------------
\begin{definition}
\label{def:convex}
%---------------------------------------
Let
  \[ \fy(x) = \frac{\ff(b)-\ff(a)}{b-a} (x-a) + \ff(a). \]
A function $\ff:\R\to\R$ is

\defbox{
  \begin{array}{llrcll}
    \mbox{\bf convex}           \mbox{ in } (a,b) & \mbox{if} & \ff(x) &\le& \fy(x) & \mbox{for } x\in(a,b) \\
    \mbox{\bf concave}          \mbox{ in } (a,b) & \mbox{if} & \ff(x) &\ge& \fy(x) & \mbox{for } x\in(a,b) \\
    \mbox{\bf strictly convex}  \mbox{ in } (a,b) & \mbox{if} & \ff(x) &<  & \fy(x) & \mbox{for } x\in(a,b) \\
    \mbox{\bf strictly concave} \mbox{ in } (a,b) & \mbox{if} & \ff(x) &>  & \fy(x) & \mbox{for } x\in(a,b)
  \end{array}
  }
\end{definition}

The next theorem gives another form of convex functions that is a little
less intuitive but provides powerful analytic results.
%---------------------------------------
\begin{theorem}
\label{thm:convex_lambda}
%---------------------------------------
Let $\ff:\R\to\R$.
For every $x_1,x_2\in(a,b)$ and $\lambda\in[0,1]$
\formbox{
  \ff \mbox{ is convex in $(a,b)$ } \iff
        \ff(\lambda x_1+(1-\lambda)x_2) \le \lambda\ff(x_1) + (1-\lambda)\ff(x_2)
  }
\end{theorem}
\begin{proof}
\begin{enumerate}
  \item prove $\ff$ is convex $\implies$
        $\ff(\lambda x_1+(1-\lambda)x_2) \le \lambda\ff(x_1) + (1-\lambda)\ff(x_2)$:
  \begin{eqnarray*}
    \ff(\lambda x_1+(1-\lambda)x_2)
      &\le& \frac{\ff(b)-\ff(a)}{b-a}[\lambda x_1+(1-\lambda)x_2-a] + \ff(a)
    \\&=&   \frac{\ff(x_2)-\ff(x_1)}{x_2-x_1}[\lambda x_1+(1-\lambda)x_2-x_1] + \ff(x_1)
    \\&=&   \frac{\ff(x_2)-\ff(x_1)}{x_2-x_1}[(x_2-x_1)(1-\lambda)] + \ff(x_1)
    \\&=&   (1-\lambda)\ff(x_2)-(1-\lambda)\ff(x_1) + \ff(x_1)
    \\&=&   \lambda\ff(x_1) + (1-\lambda)\ff(x_2)
  \end{eqnarray*}
  \item prove $\ff$ is convex $\impliedby$
        $f(\lambda x_1+(1-\lambda)x_2) \le \lambda\ff(x_1) + (1-\lambda)\ff(x_2)$:

  Let $x=\lambda(b-a)+a$ Notice that as $\lambda$ varies from $0$ to $1$,
      $x$ varies from $b$ to $a$.
      So free variable $\lambda$ works as a change of variable for
      free variable $x$.
  \begin{eqnarray*}
    \lambda &=& \frac{x-a}{b-a} \\
\\
    \ff(x)
      &=&   \ff(\lambda(b-a)+a)
    \\&\le& \lambda\ff(b) + (1-\lambda)\ff(a)
    \\&=&   \lambda[\ff(b)-\ff(a)] + \ff(a)
    \\&=&   \frac{\ff(b)-\ff(a)}{b-a}(x-a) + \ff(a)
  \end{eqnarray*}
\end{enumerate}
\end{proof}

Taking the second derivative of a function provides a convenient test
for whether that function is convex.
%---------------------------------------
\begin{theorem}
\label{thm:convex_d2}
\citepp{cover}{24}{25}
%---------------------------------------
\formbox{
  \ff''(x)>0 \implies \ff \mbox{ is convex}
  }
\end{theorem}
\begin{proof}
\begin{eqnarray*}
  \ff(x)
    &=&   \ff(x_0) + \ff'(x_0)(x-x_0) + \ff''(c)(x-x_0)^2
  \\&\ge& \ff(x_0) + \ff'(x_0)(x-x_0)
  \\&=&   \ff(x_0) + \ff'(x_0)(x-\lambda x_1 - (1-\lambda)x_2)
\\
\\
  \ff(x_1)
    &\ge& \ff(x_0) + \ff'(x_0)(x_1-\lambda x_1 - (1-\lambda)x_2)
  \\&=&   \ff(x_0) + \ff'(x_0)(1-\lambda)(x_1-x_2)
  \\&=&   \ff(x_0) - \ff'(x_0)(1-\lambda)(x_2-x_1)
\\
\\
  \ff(x_2)
    &\ge& \ff(x_0) + \ff'(x_0)(x_2-\lambda x_1 - (1-\lambda)x_2)
  \\&=&   \ff(x_0) + \ff'(x_0)\lambda (x_2-x_1)
\\
\\
  \lambda\ff(x_1) + (1-\lambda)\ff(x_2)
    &\ge& \lambda\left[\ff(x_0) - \ff'(x_0)(1-\lambda)(x_2-x_1)\right] +
          (1-\lambda)\left[\ff(x_0) + \ff'(x_0)\lambda (x_2-x_1)\right]
  \\&=&   \lambda\left[\ff(x_0) - \ff'(x_0)(1-\lambda)(x_2-x_1)\right]
          +\left[\ff(x_0) + \ff'(x_0)\lambda (x_2-x_1)\right]
          -\lambda\left[\ff(x_0) + \ff'(x_0)\lambda (x_2-x_1)\right]
  \\&=&  \ff(x_0)
  \\&=&  \ff(\lambda x_1 + (1-\lambda)x_2)
\end{eqnarray*}

By \prefpp{thm:convex_lambda}, $\ff(x)$ is convex.
\end{proof}




%=======================================
\section{Literature}
%=======================================
\begin{survey}
\begin{enumerate}
  \item Abstract convexity:
    \\\citer{edelman1985}
    \\\citer{vel1993} 
    \\\citer{hormander1994}

  \item Order convexity (lattice theory):
    \\\citer{edelman1986}

  \item Metric convexity:
    \\\citer{menger1928}
    \\\citerpgc{blumenthal1970}{41}{0828402426}{?}
    \\\citerppg{khamsi2001}{35}{38}{0471418250}

  \item Strictly convex:
    \\\citor{clarkson1936}

\end{enumerate}
\end{survey}




