%============================================================================
% Daniel J. Greenhoe
% XeLaTeX file
%============================================================================

%=======================================
\chapter{Complemented Lattices}
\label{chp:latc}
%=======================================
%=======================================
\section{Definitions}
%=======================================
%\pref{def:latc} (next) defines complements and \prope{complemented} lattices.
%Note that even though the definition defines complemented lattices in terms of
%the least upper bound and greatest lower bound of the lattice,
%complemented lattice are not required to be \prope{bounded};
%that is, the bounds of a complemented lattice do not need to be contained in the lattice,
%they only need to exist.
%---------------------------------------
\begin{definition}
\citetbl{
  \citerpg{stern1999}{9}{0521461057},
  \citorpg{birkhoff1948}{23}{3540120440}
  }
\label{def:complement}
\label{def:ucomp}
\label{def:latc}
\index{complement!lattice}
\index{lattice!complemented}
\index{lattice!relatively complemented}
%---------------------------------------
Let $\latL\eqd\latbd$ be a \structe{bounded lattice} \xref{def:latb}.
\defboxp{
  An element $x'\in\setX$ is a \fnctd{complement} of an element $x$ in $\latL$ if
  \\\indentx$\begin{array}{F rclDD}
    1. & x\meet x' &=& \bzero  & (\prope{non-contradiction})  & and \\
    2. & x\join x' &=& \bid    & (\prope{excluded middle}). &
  \end{array}$
  \\
  An element $x'$ in $\latL$ is the \fncte{unique complement} of $x$ in $\latL$ if 
  $x'$ is a \fncte{complement} of $x$ and $y'$ is a \fncte{complement} of $x$ $\implies$ $x'=y'$.
  $\latL$ is \propd{complemented} if every element in $\setX$ has a complement in $\setX$.
  $\latL$ is \propd{uniquely complemented} if every element in $\setX$ has a unique complement in $\setX$.
  A complemented lattice that is \emph{not} uniquely complemented is \propd{multiply complemented}.
  A \structd{complemented lattice} is optionally denoted $\latcd$.
  }
\end{definition}

\pref{def:latc} (previous) introduced the concept of a \hie{complement} of a lattice.
\pref{def:lat_relcompl} (next) introduces the concept of a \hie{relative complement}
in an \hie{interval} (\prefp{def:lat_intervals}).
%---------------------------------------
\begin{definition}
\label{def:lat_relcompl}
\citetbl{
  \citorpg{birkhoff1948}{23}{3540120440}
  }
\index{lattice!relatively complemented}
%---------------------------------------
Let $\latL\eqd\latticed$ be a lattice.
\defboxp{
  An element $y\in\setX$ is a \hid{relative complement} of $x$ in $\opaira{a}{b}$ with respect to $\latL$ if
  \\\indentx$\begin{array}{Frcl@{\qquad}D}
    1. & x\join y &=& b  & and \\
    2. & x\meet y &=& a.
  \end{array}$
  \\
  A lattice $\latL$ is \propd{relatively complemented} if
  every element in every closed interval $\opaira{a}{b}$ in $\latL$
  has a complement in $\opaira{a}{b}$.
  }
\end{definition}

%=======================================
\section{Examples}
%=======================================
%---------------------------------------
\begin{example}
\label{ex:lat_P(X)}
\citetbl{
  \citerpg{svozil}{72}{981020809X}
  }
\index{field of sets}
\index{complement!set}
%---------------------------------------
The lattice $\lattice{\pset{\setn{x,y,z}}}{\subseteq}{\setu}{\seti}$
of \prefp{ex:lat_xyz} is a complemented lattice.
The ``\prop{lattice complement}" of each element $\setA$ is simply the ``\prop{set complement}"
$\cmpA\eqd \pset{\setn{x,y,z}}\setd\setA$:
\exbox{\begin{array}{l@{\;}c@{\;}l | l@{\;}c@{\;}l@{\;}c@{\;}l | l@{\;}c@{\;}l@{\;}c@{\;}l}
    \mc{3}{c|}{\cmpA}
  & \mc{5}{c|}{\setA\setu\cmpA}
  & \mc{5}{c}{\setA\seti\cmpA}
  \\
  \hline
  \setopc\emptyset&=&     \setn{x,y,z}
  & \emptyset     &\setu& \setn{x,y,z} &=& \setn{x,y,z}
  & \emptyset     &\seti& \setn{x,y,z} &=& \emptyset
  \\
  \setopc\setn{x} &=&     \setn{y,z}
  & \setn{x}      &\setu& \setn{y,z}   &=& \setn{x,y,z}
  & \setn{x}      &\seti& \setn{y,z}   &=& \emptyset
  \\
  \setopc\setn{y} &=&     \setn{x,z}
  & \setn{y}      &\setu& \setn{x,z}   &=& \setn{x,y,z}
  & \setn{y}      &\seti& \setn{x,z}   &=& \emptyset
  \\
  \setopc\setn{x,y}&=&     \setn{z}
  & \setn{x,y}    &\setu& \setn{z}     &=& \setn{x,y,z}
  & \setn{x,y}    &\seti& \setn{z}     &=& \emptyset
  \\
  \setopc\setn{z} &=&     \setn{x,y}
  & \setn{z}      &\setu& \setn{x,y}   &=& \setn{x,y,z}
  & \setn{z}      &\seti& \setn{x,y}   &=& \emptyset
  \\
  \setopc\setn{x,z}&=&     \setn{y}
  & \setn{x,z}     &\setu& \setn{y}    &=& \setn{x,y,z}
  & \setn{x,z}     &\seti& \setn{y}    &=& \emptyset
  \\
  \setopc\setn{y,z}&=&     \setn{x}
  & \setn{y,z}     &\setu& \setn{x}    &=& \setn{x,y,z}
  & \setn{y,z}     &\seti& \setn{x}    &=& \emptyset
  \\
  \setopc\setn{x,y,z}&=&     \emptyset
  & \setn{x,y,z}     &\setu& \emptyset &=& \setn{x,y,z}
  & \setn{x,y,z}     &\seti& \emptyset &=& \emptyset
\end{array}}
\end{example}





%---------------------------------------
\begin{example}[\exmd{factors of 12}]
\citetbl{
  \citerpg{durbin2000}{271}{0471321478},
  \citerppg{salii1988}{26}{27}{0821845225}
  }%
\label{ex:lat_fact12}
%---------------------------------------
The lattice $\latL\eqd\lattice{\setn{1,2,3,4,6,12}}{|}{\lcm}{\gcd}$
(illustrated to the right) is \prope{non-complemented}.
In particular,
the elements $2$ and $6$ have no complements in $\latL$:
\\\begin{minipage}[c]{\tw-33mm}
$\begin{array}{@{\qquad}lcrcr @{\qquad} lcrcr}
  \lcm(2,3) &=&  6 &\ne & 12  & \gcd(2,3) &=& 1 &     &    \\
  \lcm(2,4) &=&  4 &\ne & 12  & \gcd(2,4) &=& 2 &\ne & 1  \\
  \lcm(2,6) &=&  6 &\ne & 12  & \gcd(2,2) &=& 2 &\ne & 1  \\
  \lcm(6,3) &=&  6 &\ne & 12  & \gcd(6,3) &=& 3 &\ne & 1  \\
  \lcm(6,4) &=& 12 &     &     & \gcd(6,4) &=& 2 &\ne & 1  \\
\end{array}$
\end{minipage}%
\hfill\tbox{\includegraphics{../common/math/graphics/pdfs/lat6_o6slash_123.pdf}}
\end{example}




\begin{minipage}[c]{\tw-26mm}
%---------------------------------------
\begin{example}
\footnotemark
%---------------------------------------
The lattice illustrated in the figure to the right is \hie{complemented}.
In this complemented lattice, complements are {\em not unique}.
For example, the complement of $x$ is both $y$ and $z$,
the complement of $y$ is both $x$ and $z$,
and the complement of $z$ is both $x$ and $y$.
\end{example}
\end{minipage}%
\hfill\tbox{\includegraphics{../common/math/graphics/pdfs/lat5_m3_xyz.pdf}}
\citetblt{\citerpg{durbin2000}{271}{0471321478}}

%---------------------------------------
\begin{example}
\label{ex:latc}
%---------------------------------------
Here are some more examples:\\
{\psset{unit=\latunit}%
\begin{tabular}{|*{9}{c}|}
  \mc{5}{|G|}{\prope{non-complemented} lattices}&\mc{4}{G|}{\prope{uniquely complemented} lattices}\\
   \includegraphics{../common/math/graphics/pdfs/lat4_l4.pdf}%
  &\includegraphics{../common/math/graphics/pdfs/lat5_m2onl2.pdf}%
  &\includegraphics{../common/math/graphics/pdfs/lat7_m2onm2.pdf}%
  &\includegraphics{../common/math/graphics/pdfs/lat6_o6slash.pdf}%
  &\includegraphics{../common/math/graphics/pdfs/lat7_o7slash.pdf}%
  &\vline&\includegraphics{../common/math/graphics/pdfs/lat2_l2.pdf}%
  &\includegraphics{../common/math/graphics/pdfs/lat4_m2.pdf}%
  &\includegraphics{../common/math/graphics/pdfs/lat8_l2e3.pdf}%
  %&%
  %&%
  %&%
  %&%
  %&%
  \\
  \mc{9}{|G|}{\prope{multiply complemented} lattices}\\
   \includegraphics{../common/math/graphics/pdfs/lat5_m3.pdf}%
  &\includegraphics{../common/math/graphics/pdfs/lat6_m4.pdf}%
  &\includegraphics{../common/math/graphics/pdfs/lat6_l4inm2.pdf}%
  &\includegraphics{../common/math/graphics/pdfs/lat5_n5.pdf}%
  &\includegraphics{../common/math/graphics/pdfs/lat6_o6.pdf}%
  &%
  &{\includegraphics{../common/math/graphics/pdfs/lat6_n6.pdf}}%
  &{\includegraphics{../common/math/graphics/pdfs/lat6_p3.pdf}}%
  &{\includegraphics{../common/math/graphics/pdfs/lat6_p3_dual.pdf}}%
  \\\hline
\end{tabular}}
\end{example}


%---------------------------------------
\begin{example}
%---------------------------------------
\exboxt{
  Of the 53 unlabeled lattices on a 7 element set \xref{ex:lat_set7},
  \\\indentx\begin{tabular}{rl}
     0 & are complemented with unique complements, \\
    17 & are complemented with multiple complements, and  \\
    36 & are non-complemented.
  \end{tabular}%
  }
\end{example}


%=======================================
\section{Properties}
%=======================================
\pref{thm:latc_dilworth1945} (next) is a landmark theorem in mathematics.
%\ifdochas{latd}{It's history is discussed in \prefpp{sec:latucd}.}
%---------------------------------------
\begin{theorem} %[Dilworth (1945)]
\citetbl{
  \citePp{dilworth1945}{123},
  \citerpg{salii1988}{51}{0821845225},
  \citerpgc{gratzer2003}{378}{3764369965}{Corollary 3.8}
  }
\label{thm:latc_dilworth1945}
%---------------------------------------
\thmboxt{
  For every lattice $\latL$, there exists a lattice $\latU$ such that
  \\\indentx$\begin{array}{FMD}
    1. & $\latL \subseteq \latU$  ($\latL$ is a sublattice of $\latU$) & and \\
    2. & $\latU$ is \prope{uniquely complemented}.
  \end{array}$
  }
\end{theorem}






%%=======================================
%\section{Pseudo-complemented lattices}
%%=======================================
%%---------------------------------------
%\begin{definition}
%\label{def:latpc}
%\citetbl{
%  \citerp{renedo2003}{71}
%  }
%\index{complement!lattice}
%\index{lattice!pseudo-complemented}
%%---------------------------------------
%Let $\latL\eqd\booalgd$ be a bounded lattice.
%\defboxp{
%  An element $x'\in\setX$ is a \hid{pseudo-complement} of an element $x$ with respect to $\latL$ if
%  \\$\begin{array}{@{\qquad}l rcl@{\qquad}C}
%    1. & 0' &=& 1  & \\
%    2. & x\orel y &\implies& y'\orel x' & \forall x,y\in\setX\\
%    3. & \brp{x'}' &=& x & \forall x\in\setX
%  \end{array}$
%  }
%\end{definition}
%
%%---------------------------------------
%\begin{definition}
%\label{def:latpcdm}
%\citetbl{
%  \citerp{renedo2003}{71}
%  }
%\index{lattice!de Morgan algebra}
%%---------------------------------------
%Let $\latL\eqd\booalgd$ be a bounded lattice.
%\defboxp{
%  $\latL$ is a \hid{de Morgan algebra} if
%  \\$\begin{array}{@{\qquad}l rcl@{\qquad} rcl@{\qquad}C}
%    1. & \mc{7}{E}{$\latL$ is \hie{pseudo-complemented}}
%       \\
%    2. & \brp{x\join y}' &=& x'\meet y'
%       & \brp{x\meet y}' &=& x'\join y'
%       & \forall x,y\in\setX
%       \\
%    3. & x\meet\brp{y\join z} &=& \brp{x\meet y} \join\brp{x\meet z}
%       & x\join\brp{y\meet z} &=& \brp{x\join y} \meet\brp{x\join z}
%       & \forall x,y,z\in\setX
%  \end{array}$
%  }
%\end{definition}
%
%%---------------------------------------
%\begin{theorem}
%\citetbl{
%  \citerp{renedo2003}{72}
%  }
%%---------------------------------------
%Let $\latL\eqd\booalgd$ be a bounded lattice.
%\thmbox{
%  \brbr{\begin{array}{ll}
%    1. & \text{$\latL$ is a \hie{de Morgan algebra}} \\
%    2. & \brp{x\meet y'}' = y \join x'\meet y' \quad\sst\forall x,y\in\setX
%  \end{array}}
%  \qquad\implies\qquad
%  \text{$\latL$ is a \hie{Boolean algebra}}
%  }
%\end{theorem}
%
%%---------------------------------------
%\begin{definition}
%\label{def:lat_omod}
%\citetbl{
%  \citerp{renedo2003}{71}
%  }
%\index{lattice!orthomodular}
%%---------------------------------------
%Let $\latL\eqd\booalgd$ be a bounded lattice.
%\defboxp{
%  $\latL$ is \hid{orthomodular} if
%  \\$\begin{array}{@{\qquad}l rcl@{\qquad} rcl@{\qquad}C}
%    1. & \mc{7}{E}{$\latL$ is \hie{pseudo-complemented}}
%       \\
%    2. & x\join x' &=& \bid
%       & x\meet x' &=& \bzero
%       & \forall x\in\setX
%       \\
%    3. & x\orel y &\implies&
%       & y &=& x\join \brp{x'\meet y}
%       & \forall x,y\in\setX
%  \end{array}$
%  }
%\end{definition}
%
%%---------------------------------------
%\begin{theorem}
%\citetbl{
%  \citerp{renedo2003}{72}
%  }
%%---------------------------------------
%Let $\latL\eqd\booalgd$ be a bounded lattice.
%\thmbox{
%  \brbr{\begin{array}{ll}
%    1. & \text{$\latL$ is an \hie{orthomodular} lattice} \\
%    2. & \brp{x\meet y'}' = y \join x'\meet y' \quad\sst\forall x,y\in\setX
%  \end{array}}
%  \qquad\implies\qquad
%  \text{$\latL$ is a \hie{Boolean algebra}}
%  }
%\end{theorem}



%=======================================
%\subsection{Uniquely complemented lattices and distributivity}
%\label{sec:latucd}
%=======================================
``I therefore propose the following problem\ldots".
With these words, Edward Huntington in a 1904 paper introduced one of the most
famous problems in mathematical history;%
\footnote{For more discussion, see \prefp{lit:lat_hprob}}
a question that took some 40 years to answer,
and that in the end had a very surprising solution.
Huntington's problem was essentially this:
\emph{Are all uniquely complemented lattices also distributive?}\citetbl{\citePp{huntington1904}{305}}
This question is significant because if a lattice is both complemented and distributive,
then it is \prope{uniquely complemented} (\pref{cor:latcd_uniquecomp}---next)
and, more importantly, is a \prope{Boolean algebra}\ifsxref{boolean}{def:boolean}.
Being a Boolean algebra is very significant in that it implies the lattice
has several powerful properties including that
it satisfies \prope{de Morgan's laws} \xref{thm:lattice}
and that it is isomorphic to an \hie{algebra of sets} \xref{thm:lat_algebra}.

A uniquely complemented lattice that satifies any one of a number of other conditions
is distributive \xxref{thm:latuc_hp}{lit:latuc}.
So there was ample evidence that the answer to Huntington's question is ``yes".
But the final answer to Huntington's problem is actually ``no"---%
an answer that took the mathematical community 40 years to find.
The resulting effort had a profound impact on lattice theory in general.
In fact, George Gr/:atzer, in a 2007 paper, identified uniquely complemented lattices
as one of the ``two problems that shaped a century of lattice theory".\citetbl{\citePp{gratzer2007}{696}}

This final solution to Huntington's problem was found by Robert Dilworth and published
in a 1945 paper.\citetbl{\citePp{dilworth1945}{123}}
And the answer is this:
\emph{Every lattice is a sublattice of a uniquely complemented lattice}
\xref{thm:latc_dilworth1945}.
To understand why this answers the question, consider either the
\structe{M3 lattice} \xref{def:lat_M3} or the \structe{N5 lattice} \xref{def:lat_N5}.
Neither of these lattices are \prope{distributive} \xref{thm:latd_char_n5m3},
but yet either of them can be a sublattice in a uniquely complemented lattice
(by \thme{Dilworth's theorem}).
That is, it is therefore possible to have a lattice that is
both \prope{uniquely complemented} and \prope{non-distributive}.

%---------------------------------------
\begin{corollary}
\citetbl{
  \citerpg{maclane1999}{488}{0821816462},
  \citerpgc{salii1988}{30}{0821845225}{Theorem 10}
  }
\label{cor:latcd_uniquecomp}
%---------------------------------------
Let $\latL\eqd\latticed$ be a lattice.
\corbox{
  \brb{\begin{array}{FlD}
    1. & \text{$\latL$ is \prope{distributive}} & and \\
    2. & \text{$\latL$ is \prope{complemented}} &
  \end{array}}
  \qquad\implies\qquad
  \brb{\text{$\latL$ is \prope{uniquely complemented}}}
  }
\end{corollary}
\begin{proof}
\begin{align*}
  &\text{$\latL$ is complemented}
  \\&\quad\iff     \forall x\in L\exists a,b \st \text{$a,b$ are complements of $x$ in $\latL$}
    &\quad&        \text{by definition of complement \prefpo{def:latc}}
  \\&\quad\iff     x\join a=1,\; x\join b=1,\; x\meet a=0,\;x\meet b=0
    &\quad&        \text{by definition of complement \prefpo{def:latc}}
  \\&\quad\implies a=b
    &\quad&        \text{by \prefp{thm:lat_distrib_a=b}}
  \\&\quad\implies \text{$\latL$ is uniquely complemented}
\end{align*}
\end{proof}

%---------------------------------------
\begin{theorem}[\thmd{Huntington properties}]
\footnote{
  \citerpg{roman2008}{103}{0387789006},
  \citerpg{adams1990}{79}{0817634347},
  \citerpg{salii1988}{40}{0821845225},
  \citePp{dilworth1945}{123},
  \citerp{gratzer2007}{698}
  }
\label{thm:latuc_hp}
%---------------------------------------
Let $\latL$ be a lattice.
\thmbox{
  \brb{\begin{array}{M}%
    $\latL$ is\\ 
    \emph{uniquely}\\
    \emph{complemented}%
  \end{array}}
  \text{ and }
  \mcom{\brb{\begin{array}{MD}
    %\mc{1}{N}{(\prope{Huntington properties})} & \\
    $\latL$ is        \prope{modular}                 & or \\
    $\latL$ is        \prope{atomic}                  & or \\
    $\latL$ is        \prope{ortho-complemented}      & or \\
    $\latL$ has       \prope{finite width}            & or \\
    $\latL$ has       \prope{de Morgan} properties &
  \end{array}}}{\prope{Huntington properties}}
  \implies
  \brb{\begin{array}{M}%
    $\latL$ is\\ 
    \prope{distributive}%
  \end{array}}
  }
\end{theorem}


%---------------------------------------
\begin{theorem}[\thm{Peirce's Theorem}]
\footnote{
  \citerppgc{salii1988}{38}{39}{0821845225}{``Peirce's Theorem"},
  \citePc{peirce_ln}{1902 January 31 entry},
  \citePc{peirce1903h}{letter to Huntington},
  \citePc{peirce1904h}{letter to Huntington},
  \citeP{huntington1904}%{reproduction of Peirce's 1904 letter and 1903 proof}
  }
\label{thm:peirce1902}
%---------------------------------------
Let $\latL\eqd\booalgd$ be a bounded lattice.\\
Let $\complement y\eqd\set{y'\in\setX}{\text{$y'$ is a complement of $y$}}$.
\thmbox{
  \brb{\forall y'\in\complement y,\, x\orelnot y' \implies x\meet y\ne \bzero}
  \implies
  \brb{\begin{array}{FMD}
    1. & $\latL$ is \prope{uniquely complemented} & and \\
    2. & $\latL$ is \prope{distributive}
  \end{array}}
  }
\end{theorem}


%=======================================
\section{Literature}
%=======================================

\begin{survey}
\begin{enumerate}
  \item General treatment of lattice varieties:
    \\\citer{jipsen1992}

  \item Distributive lattices:
    \\\citer{gratzer1971}
    \\\citer{balbes1975}
    \\\citer{dilworth1984}

  \item Uniquely complemented lattices: \label{lit:latuc}
    \\\citePc{dilworth1945}{``Every lattice is a sublattice of a lattice with unique complements."}
    \\\citerg{salii1988}{0821845225}
    \\\citerppg{adams1990}{79}{84}{0817634347}
    \\\citer{gratzer2007}
    \\\citerpg{roman2008}{103}{0387789006}
    \\\citerc{bergman1929}{uniquely complemented + \prope{modular} = distributive}
    \\\citerc{birkhoff1940}{uniquely complemented + \prope{ortho-complemented} = distributive}
    \\\citerc{birkhoff1939}{uniquely complemented + \prope{atomic} = distr.}
    %\\\citerc{birkhoff1939}{uniquely complemented + \prope{atomic} = distributive}
    \\\citerc{birkhoff1939r}{uniquely complemented + \prope{atomic} = distributive}

  \item Projective distributive lattices:
    \\\citer{balbes1967}
    \\\citer{balbes1970}

  \item Median property: \label{lit:lat_median}
    \\\citor{birkhoff1947}
    \\\citor{birkhoff1947r}
    \\\citor{grau1947}
    \\\citer{evans1977}
    \\\citer{isbell1980}
    \\\citer{bandelt1983}
    \\\citerppg{birkhoff1987}{1}{8}{0817631143}
    \\\citerpgc{artamonov2000}{554}{044450396X}{median algebras}
    \\\citerpg{gratzer2008}{356}{0387774866}

  \item Properties of lattices
    \begin{enumerate}
      %\item Peirce (1880) credits Jevons for the \prope{idempotent} property.%
      %  \citetbl{
      %    \citorp{jevons1864fl} \\
      %    \citerp{peirce1880ajm}{33}
      %    }
      \item The fact that lattices are not in general \prope{distributive} was 
            not always universally accepted. 
            In a famous 1880 paper, Charles S. Peirce\citep{peirce1880ajm}{33} 
            presents distributivity as a property of all lattices but says that 
            ``the proof is too tedius to give".
    \end{enumerate} 

  \item Note about \hie{Huntington's problem} concerning uniquely complemented lattices: \label{lit:lat_hprob}
    \begin{enumerate}
      \item Salii\citetbl{\citerppgc{salii1988}{38}{39}{0821845225}{``Peirce's Theorem"}} 
            suggests that Huntington's problem is actually motivated by and a simple extension of
            \thme{Peirce's Theorem} \xref{thm:peirce1902}.
            That is, Huntington's problem is equivalent to asking if
            the uniquely complemented property is equivalent to the left hypothesis
            in \thm{Peirce's Theorem}.

      \item George Gr{/:a}tzer in a 2007 paper seems to indicate that 
            Huntington's 1904 paper\citetbl{\citePp{huntington1904}{305}}
            is \emph{not} the original source of ``\hi{Huntington's problem}".
            In particular, Gr/:atzer says
            "\ldots Neither gives
            any references as to the origin of the problem.
            G. Birkhoff and M. Ward, 1933, reference E. V.
            Huntington, 1904, for the lattice axioms, which
            Huntington stated as being due to E. Schr/:oder, but
            not for the problem. If the reader is surprised, I
            suggest he try to read the original paper of E. V.
            Huntington, and there he may find the clue. In my
            earlier papers on the subject, I reference only R. P.
            Dilworth, 1945, but in my lattice books (e.g., [7]) I
            give the correct reference. But I have no recollection
            of reading E. V. Huntington, 1904, until the
            preparation for this article."
            (\citerp{gratzer2007}{699})
            The reference [7] is \citer{gratzer2003}.
            In this reference, Dilworth's 1945 theorem is presented on 
            \href{http://books.google.com/books?vid=ISBN3764369965\&pg=PA378}{page 378},
            and its historical background is discussed on 
            \href{http://books.google.com/books?vid=ISBN3764369965\&pg=PA392}{page 392}.
            However, this discussion does not seem to give credit for Huntington's problem
            to anyone other than Huntington (1904).
            Perhaps it is Peirce that Gr/:atzer has in mind with these comments---but so far 
            the person referred to by Gr/:atzer is unclear (to me).
            See also \url{http://groups.google.com/group/sci.math/browse_thread/thread/b7790be1efe8946e#}
    \end{enumerate}
%\end{enumerate}
%\end{survey}





%=======================================
%\section{Literature}
%=======================================
%\begin{survey}
%\begin{enumerate}
  \item General treatment of lattice varieties:
    \\\citer{jipsen1992}

  \item Atomic lattices:
    \\\citerpc{birkhoff1938}{800}{see footnote \ddag}

\end{enumerate}
\end{survey}


%\fi


