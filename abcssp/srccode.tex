%============================================================================
% LaTeX File
% Daniel J. Greenhoe
%============================================================================

%======================================
\chapter{C++ source code support}
\label{app:src_code}
\index{source code}
%======================================
This document seeks to conform to the principles of \hie{Reproducible Research} as detailed at
\\\indentx\url{http://reproducibleresearch.net/}\\
C++ source code, along with a CMake build script, can be found at
\\\indentx\url{https://github.com/dgreenhoe/symbolic-sequence-processing}
%This section contains a partial C++ source code listing for \lstinline{ssp.exe}, written by the author of this paper,
%which produced the {\TeX} files
%used to generate the 128 or so data plot files presented in \pref{chp:ssp}.
%The complete and downloadable source code for \lstinline{ssp.exe} is to accompany any online version of this document.

There are some who might hesitate to use computer simulation to demonstrate mathematical concepts.
And arguably their concern is not unfounded.
There is in fact evidence to suggest that while the invention of the printing press has greatly
assisted the progress of mathematical discovery, the introduction of computers has harmed it.
%But it is not that applications are no longer important to mathematics,
%but rather applications drive mathematics forward.
Evidence of this hypothesis is given in
\prefpp{fig:intro_timeline}.\footnote{Data for \prefpp{fig:intro_timeline} extracted from \\
  \url{http://www-history.mcs.st-andrews.ac.uk/Timelines/WhoWasThere.html}}
This graph shows the number of ``notable" mathematicians alive during the last 3000 years;
Here a ``notable mathematician" is defined as one whose name appears
in \hie{Saint Andrew's University's} \hie{Who Was There}
website.\footnote{\url{http://www-history.mcs.st-andrews.ac.uk/Timelines/WhoWasThere.html}}
Note the following:
\begin{liste}

\item The number of mathematicians starts to exponentially increase at
about the time of Gutenberg's invention of the printing press---
that is, when information of discoveries and results could be widely and economically
circulated.\footnote{This point is also made by Resnikoff and Wells in \citerp{resnikoff}{9}.}

\item There is another increase after the invention of the slide rule in the early 1600s---
that is, when computational power increased.

\item There are huge increases around the time of the first and second industrial revolutions---
that is, when there were many {\bf applications} that called for mathematical solutions.

\item After the invention of the pocket scientific calculator in 1972
and home IBM PC in 1981---machines that could
often make hard-core mathematical analysis unnecessary in real-world applications---
there was a huge drop in the number of mathematicians.

\end{liste}




%One further point here is, that however little some mathematicians may think of
%real-world applications,
%historically it would seem that applications are critical to
%how well mathematics thrives and develops.
%When mathematics is needed for the development of applications,
%mathematics prospers.
%But when mathematics is not viewed as critical to those applications
%(such as after the introduction of the personal computer),
%mathematics withers.

\begin{figure}
  \centering%
  \includegraphics[width=\tw]{../common/math/graphics/pdfs/notablemath.pdf}
  \caption{
    Number of notable mathematicians alive over time
    \label{fig:intro_timeline}
    }
\end{figure}

%see GitHub% %--------------------------------------
%see GitHub% %\section{$R^2$ normed space}
%see GitHub% %  \lstinputlisting{../2015ssp/c/sequence.h}
%see GitHub% %  \lstinputlisting{../2015ssp/c/sequence.cpp}
%see GitHub% %\section{$R^2$ normed space}
%see GitHub% %  \lstinputlisting{../2015ssp/c/r2nspace.h}
%see GitHub% %  \lstinputlisting{../2015ssp/c/r2nspace.cpp}
%see GitHub% %\section{$R^3$ normed space}
%see GitHub% %  \lstinputlisting{../2015ssp/c/r3nspace.h}
%see GitHub% %  \lstinputlisting{../2015ssp/c/r3nspace.cpp}
%see GitHub% %\section{Operators on $R^2$ linear space}
%see GitHub% %  \lstinputlisting{../2015ssp/c/r2op.h}
%see GitHub% %  \lstinputlisting{../2015ssp/c/r2op.cpp}
%see GitHub% %\section{Scaled Euclidean metric routines}
%see GitHub% %  \lstinputlisting{../2015ssp/c/euclid.cpp}
%see GitHub% %\section{Legrange arc metric routines}
%see GitHub% %  \lstinputlisting{../2015ssp/c/larc.cpp}
%see GitHub% %\section{Elliptic arc metric routines}
%see GitHub% %  \label{sec:earc}
%see GitHub% %  \lstinputlisting{../2015ssp/c/elliptic.cpp}
%see GitHub% \section{Symbolic sequence routines}
%see GitHub%   \label{sec:src_sym}
%see GitHub%   \lstinputlisting{../2015ssp/c/symseq.h}
%see GitHub%   \lstinputlisting{../2015ssp/c/symseq.cpp}
%see GitHub% \section{Die routines}
%see GitHub%   \label{sec:src_die}
%see GitHub%   \lstinputlisting{../2015ssp/c/die.h}
%see GitHub%   \lstinputlisting{../2015ssp/c/die.cpp}
%see GitHub% \section{Real die routines}
%see GitHub%   \label{sec:src_rdie}
%see GitHub%   \lstinputlisting{../2015ssp/c/realdie.h}
%see GitHub%   \lstinputlisting{../2015ssp/c/realdie.cpp}
%see GitHub% \section{Spinner routines}
%see GitHub%   \label{sec:src_spinner}
%see GitHub%   \lstinputlisting{../2015ssp/c/spinner.h}
%see GitHub%   \lstinputlisting{../2015ssp/c/spinner.cpp}
%see GitHub% \section{DNA routines}
%see GitHub%   \label{sec:src_dna}
%see GitHub%   \lstinputlisting{../2015ssp/c/dna.h}
%see GitHub%   \lstinputlisting{../2015ssp/c/dna.cpp}
%see GitHub% \section{Legrange arc distance routines}
%see GitHub%   \label{sec:src_larc}
%see GitHub%   \lstinputlisting{../2015ssp/c/larc.h}
%see GitHub%   \lstinputlisting{../2015ssp/c/larc.cpp}
